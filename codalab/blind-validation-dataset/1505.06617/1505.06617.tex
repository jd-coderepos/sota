\documentclass{llncs}
\usepackage[T1]{fontenc}
\usepackage[latin9]{inputenc}
\setlength{\parskip}{\smallskipamount}
\setlength{\parindent}{0pt}
\usepackage{amssymb}
\usepackage{amsmath}
\usepackage{graphicx}

\usepackage[english]{babel}
\makeatletter
\newtheorem{thm}{\protect\theoremname}
  \newtheorem{prop}[thm]{\protect\propositionname}
  \newtheorem{defn}[thm]{\protect\definitionname}
  \newtheorem{rem}[thm]{\protect\remarkname}

\newcommand{\coeff}{\mathsf{coeff}}
\renewcommand{\mp}{\mathsf{F}}
\newcommand{\MSOL}{\mathrm{MSOL}}
\newcommand{\FPT}{\mathrm{FPT}}
\newcommand{\FPPT}{\mathrm{FPPT}}
\newcommand{\NP}{\mathrm{NP}}
\newcommand{\Wone}{\mathrm{W}}
\newcommand{\sharpP}{\mathrm{\sharp P}}
\newcommand{\qr}{\mathrm{qr}}
\newcommand{\N}{\mathbb{N}}
\newcommand{\Op}{\mathsf{Op}}

\pagestyle{plain}


\makeatother

\usepackage{babel}
  \providecommand{\definitionname}{Definition}
  \providecommand{\examplename}{Example}
  \providecommand{\propositionname}{Proposition}
  \providecommand{\remarkname}{Remark}
\providecommand{\theoremname}{Theorem}

\begin{document}

\title{Efficient computation of generalized Ising polynomials on graphs
with fixed clique-width\thanks{Tomer Kotek was supported by the Austrian
National Research Network S11403-N23 (RiSE) of the Austrian Science
Fund (FWF) and by the Vienna Science and Technology Fund (WWTF)
grant PROSEED. }}
\author{Tomer Kotek\inst{1} \and Johann A. Makowsky\inst{2}}
\institute{TU Vienna \\ Vienna, Austria \\ 
\email{tkotek@tuwien.ac.at} \and
Technion --- Israel Institute of Technology \\ Haifa, Israel \\ \email{janos@cs.technion.ac.il}}

\maketitle
\begin{abstract}
Graph polynomials which are definable in 
Monadic Second
Order Logic () on the vocabulary
of graphs are Fixed-Parameter Tractable () with respect to clique-width.
In contrast, graph polynomials which are definable in  on the vocabulary of hypergraphs are 
fixed-parameter tractable
with respect to tree-width, but not necessarily
with respect to clique-width. No algorithmic meta-theorem is known
for the computation of graph polynomials definable in  on
the vocabulary of hypergraphs with respect to clique-width. We define an infinite class of such graph
polynomials extending the class of graph polynomials definable in
 on the vocabulary of graphs and prove that they are Fixed-Parameter 
Polynomial Time () computable, i.e. that they can be computed
in time , where  is the number of vertices and 
is the clique-width. 
\end{abstract}

\section{Introduction}

In recent years there has been growing interst in graph polynomials,
functions from graphs to polynomial rings which are invariant under
isomorphism. Graph polynomials encode information about the graphs
in a compact way in their evaulations, coeffcients, degree and roots.
Therefore, efficient computation of graph polynomials has received
considerable attention in the literature. Since most graph polynomials
which naturally arise are -hard to compute (see e.g. \cite{valiant1979complexity,ar:JVW90,pr:BH08}),
a natural perspective under which to study the complexity of graph polynomials
is that of {\em parameterized complexity}. 

Parameterized complexity is a successful approach to tackling -hard
problems \cite{bk:DF99,bk:FG06}, by measuring complexity with respect
to an additional {\em parameter} of the input; we will be interested in the parameters
tree-width and clique-width. A computational problem is {\em\bf fixed-parameter tractable}
() with respect to a parameter  if it can be solved in
time , where  is a computable function of ,
 is the size of the input, and  is a polynomial in .
Many -hard problems are fixed parameter tractable for an appropriate
choice of parameter, see \cite{bk:FG06} for many examples. Every
problem in the infinite class of decision problems definable in {\em Monadic Second Order Logic}
() is fixed-parameter tractable with respect to tree-width
by Courcelle's Theorem \cite{ar:Courcelle90,ar:ArnborgEtAl,bk:CourcelleEngelfriet12}
(though the result originally was not phrased in terms of parameterized
complexity). 

The computation problem we consider for a graph polynomial 
is the following:

~~~~~~~~~~\begin{minipage}[t]{1\columnwidth}

: A graph 

 Compute the coefficients 
of the monomials . \end{minipage}

For graph polynomials, a parameterized complexity theory with respect
to tree-width has been developed. Here, the goal is to compute, given
an input graph, the table of coefficients of the graph polynomial.
The Tutte polynomial has been shown to be fixed-parameter tractable
\cite{ar:noble98,ar:andrzejak98}. \cite{makowsky2005coloured} used
a logical method to study the parameterized complexity of an infinite
class of graph polynomials, including the Tutte polynomial, the matching
polynomial, the independence polynomial and the Ising polynomial.
\cite{makowsky2005coloured} showed that the class of graph polynomials
definable in  in the vocabulary of hypergraphs\footnote{In \cite{bk:CourcelleEngelfriet12},  in the vocabulary of hypergraphs is denoted , while  in the vocabulary of graphs is denoted .}
is fixed-parameter tractable. This class contains the vast majority
of graph polynomials which are of interest in the literature. 

Going beyond tree-width to clique-width the situation becomes more
complicated. \cite{ar:CMR2000} studied the class of graph polynomials
definable in  {\em in the vocabulary of graphs}. They proved
that every graph polynomial in this class is fixed-parameter tractable
with respect to clique-width. However, this class of graph polynomials
does not contain important examples such as the chromatic polynomial,
the Tutte polynomial and the matching polynomial. In fact, \cite{ar:FominGolovachLokshtanov10}
proved that the chromatic polynomial and the Tutte polynomial are
not fixed-parameter tractable with respect to clique-width (under
the widely believed complexity-theoretic assumption that ).
\cite{makowsky2006computing} proved that the chromatic polynomial
and the matching polynomial are {\em\bf fixed-parameter polynomial time}
computable with respect to clique-width, meaning that they can be
computed in time , where  is the size of the graph,
 is the clique-width of the graph and  is a computable
function. \cite{kotek2012complexity} proved an analogous result for
the Ising polynomial. The main result of this paper is a meta-theorem
generalizing the fixed-parameter polynomial time computability of
the chromatic polynomial, the matching polynomial and the Ising polynomial
to an infinite family of graph polynomials definable in  analogous
to \cite{ar:CMR2000}. 
\begin{thm}
Let  be an -Ising polynomial.  is fixed-parameter polynomial
time computable with respect to clique-width. 
\end{thm}
The class of -Ising polynomials is defined in Section \ref{se:MSOLISing}. 


\section{Preliminaries}

Let . Let  be the vocabulary
of graphs  consisting
of a single binary relation symbol . A -graph is a structure  which consists of a simple graph  together with a partition  of .  Let
 denote the vocabulary of -graphs 
extending  with unary relation symbols . 

The class  of -graphs of clique-width at most  is defined inductively. Singletons belong to , and  is closed under disjoint union  and 
two other operations,  and , to be defined next. 
For any ,  is obtained by relabeling any vertex with label  to label . 
For any ,  is obtained by adding all possible edges  between members of  
and members of . The clique-width of a graph  is the minimal  such that there exists a labeling  for 
which  belongs to . We denote the clique-width of  by .
The clique-width operations  and  are well-defined for -graphs.
The definitions of these operations extend naturally to structures  which expand -graphs with . 

A -expression is a term  which consists of singletons, disjoint unions , relabeling  and edge creations , which witnesses that the graph  obtained by performing the operations on the singletons is of clique-width at most . Every graph of tree-width at most  is of clique-width at most , cf. \cite{ar:CourcelleOlariu2000}. While computing the clique-width of a graph is -hard, S. Oum and P. Seymour showed that given a graph of clique-width , finding a -expression is fixed parameter tractable with clique-width as parameter, cf. \cite{ar:Oum2005,ar:SeymourOum2006}. 

For a formula , let  denote the quantifier
rank of . For every  and vocabulary ,
we denote by  the set of -formulas on the
vocabulary  which have quantifier rank at most . For two
-structures  and , we write 
to denote that  and  agree on all the
sentences of quantifier rank . 

\begin{definition}[Smooth operation]
 An -ary operation  on -structures is called {\em smooth} 
if for all , 
whenever   for all , 
we have  
\end{definition}

Smoothness
of the clique-width operations is an important technical tool for
us:
\begin{thm}
[Smoothness, cf. \cite{makowsky2004algorithmic}] \label{th:smooth}~
\begin{enumerate}
\item For every vocabulary , the disjoint union  of two
-structures is smooth. 
\item For every ,  and 
are smooth. 
\end{enumerate}
\end{thm}
It is convenient to reformulate Theorem \ref{th:smooth} in terms
of {\em Hintikka sentences} (see \cite{bk:EF2005}):
\begin{prop}
[Hintikka sentences] \label{prop:Hin}  Let  be a vocabulary.
For every  there is a finite set

 of -sentences such that
\begin{enumerate}
\item Every  has a finite model.
\item The conjunction  of any two distinct 
is unsatisfiable.
\item Every -sentence  is equivalent to exactly
one finite disjunction of sentences in .
\item Every -structure  satisfies a unique member 
of .
\end{enumerate}
\end{prop}
In order to simplify notation we omit the subscript  in  when  is clear from the context.


Let  the be the vocabulary consisting of the binary relation
symbol  and the unary relation symbol .
Let  extend  with the unary relation
symbols . 
From Theorem \ref{th:smooth} and Proposition \ref{prop:Hin} we get:
\begin{thm}
For every :
\begin{enumerate}
\item There is 
such that, for every  and , 
\newline
. 
\item For every unary operation ,
there is 
such that, for every , . 
\end{enumerate}
\end{thm}

\subsection{-Ising polynomials \label{se:MSOLISing}}

For every , let ,
where  are new unary relation symbols.
\begin{defn}
[-Ising polynomials] \label{def:MSOLIsing} For every ,
 and  we define 
as follows: 

 is the sum over partitions  of
 such that  satisfies 
of the monomials obtained as the product of  for
all  and 
for all . \end{defn}

 
\begin{example}
[Ising polynomial] The trivariate Ising polynomial 
is a partition function of the Ising model from statistical mechanics
used to study phase transitions in physical systems in the case of
constant energies and external field.  is given by 

where  denotes the set of edges between  and ,
and  denotes the set of edges inside .  was
the focus of study in terms of hardness of approximation in \cite{ar:GJP03}
and in terms of hardness of computation under the exponential time
hypothesis was studied in \cite{kotek2012complexity}. \cite{kotek2012complexity}
also showed that  is fixed-parameter polynomial time
computable.

 generalizes a bivariate Ising polynomial,
which was studied for its combinatorial properties in \cite{ar:AndrenMarkstrom2009}.
\cite{ar:AndrenMarkstrom2009} showed that  contains
the matching polynomial, the van der Waerden polynomial, the cut polynomial,
and, on regular graphs, the independence polynomial and clique polynomial. 

The evaluation of  
at 
, , ,  and 
gives  and therefore  is an -Ising polynomial.
\end{example}

\begin{example}
[Independence-Ising polynomial] The independence-Ising polynomial
 is given by

 contains the independence polynomial as the evaluation
. See the survey \cite{levit2005independence}
for a bibliography on the independence polynomial. The evaluation
 is , where 
is the number of isolated vertices in .  is an
evaluation of an -Ising polynomial: 

where . 
\end{example}

\begin{example}
[Dominating-Ising polynomial] The Dominating-Ising polynomial is
given by  

where  denotes the set of edges between  and .
 contains the domination polynomial . 
is the generating function of its dominating sets and we have .
The domination polynomial first studied in \cite{arocha2000mean}
and it and its variations have received considerable attention in
the literature in the last few years, see e.g. \cite{alikhani2009dominating,akbari2010zeros,akbari2010characterization,alaeiyan2011cyclically,DBLP:journals/combinatorics/KotekPSTT12,DBLP:journals/gc/KotekPT14,DBLP:journals/arscom/AkbariO14,alikhani2014introduction,brown2014roots,kahat2014dominating,DBLP:journals/dmgt/DodKPT15}.
Previous research focused on combinatorial properties such as recurrence
relations and location of roots. Hardness of computation was addressed
in \cite{kotek2013domination}.  encodes the degrees
of the vertices of : the number of vertices with degree  is
the coefficient of  in . 
is an -Ising polynomial given by ,
where 

\end{example}


\subsection{-Ising polynomials vs -polynomials}

Two classes of graph polynomials which have received attention in the literature are:
\begin{enumerate}
 \item -polynomials on the vocabulary of graphs, and
 \item -polynomials on the vocabulary of hypergraphs. 
\end{enumerate}
See e.g. \cite{ar:KM14connection} for the exact definitions. The former class contains graph polynomials such as the independence polynomial
and the domination polynomial. The latter class contains graph polynomials such as the Tutte polynomial and the matching polynomial. 
Every graph polynomial which is -definable on the vocabulary of graphs is also -definable on the vocabulary of hypergraphs. 

The class of -Ising polynomials strictly contains the -polynomials on graphs, see Fig. \ref{fig}. The containment is by definition. 
For the strictness, we use the fact that by definition the maximal degree of any indeterminate in an -polynomial on graphs grows at most linearly in the number of vertices, while 
the maximal degree of  in the Ising polynomial 
of the complete bipartite graph  equals .

Every -Ising polynomial  is an -polynomial on the
vocabulary of hypergraphs, given e.g. by 

where the summation over  is exactly as in Definition \ref{def:MSOLIsing},
and the summation over  is over tuples 
of subsets of the edge set of  satisfying ,
where 

We use the fact that  is a partition of the set of vertices is definable in .  

\begin{figure}
  \caption{\label{fig} Containments of classes of graph polynomials definable in . }
  \ \\
  \centering
    \includegraphics{Venn-diag}
\end{figure}



\section{Main result}

We are now ready to state the main theorem and prove a representative
case of it. 
\begin{thm}
[Main theorem]\label{th:main} For every -Ising polynomial
 there is a function  such that 
is computable on graphs  of size  and of clique-width at most
 in running time . 
\end{thm}


We prove
the theorem for graph polynomials of the form 

 for every . The summation in 
is over subsets  of the vertex set of . The graph polynomials
 are a notational variation of  with ,  and : 
for every , ,
where  is obtained from  by substituting 
with  and  with . The proof for the general case is 
in similar spirit. 

For every  there is a finite set  of -Ising polynomials
 such that, for every formula ,
 is a sum of members of  (see below).  
The algorithm computes the values of the members of  on  
by dynamic programming over the parse term of , and using those values, the value of  on .   

More precisely, for every ,
let 

and let 

Every  also belongs to , and hence there
exists by Proposition \ref{prop:Hin} a set 
such that 

Hence,

 setting  and  for all .

For tuples ,
let  be the coefficient
of 

in .

\subsection*{Algorithm.}

Given a -graph , the algorithm first computes a parse tree
 as in \cite{ar:Oum2005,ar:SeymourOum2006}. The algorithm
then computes  for all 
by induction over :
\begin{enumerate}
\item If  is a graph of size , then  is computed
directly. 
\item Let  be the disjoint union of  and . We compute
 for every 
and  as follows: 

 
\item Let . We compute 
for every  and 
as follows: 

where the inner summation is over  such that 

and 


\item Let  with . Let  be the number of vertices in . 
We compute 
for every  and 
as follows: 

where the summation is over  such that  and


\end{enumerate}

Finally, the algorithm computes 
as the sum from Eq. (\ref{eq:sumQA}).
\subsection{Runtime}

The main observations for the runtime analysis are:
\begin{itemize}
\item The size of the set  of Hintikka
sentences is a function of  but does not depend on . Let . 
\item By definition of , for a monomial 
to have a non-zero coefficient, it must hold that  and
, since  and 
are sizes of sets of vertices and sets of edges, respectively.
\item The coefficient of any monomial of  is at most . 
\item The parse tree guaranteed in \cite{ar:Oum2005,ar:SeymourOum2006}
is of size  for suitable  and . 
\end{itemize}
The algorithm performs a single operation for every node of the parse
tree. 

\textbf{Singletons}: the coefficients of every  for a singleton -graph
can be computed in time , which can be bounded by . 

\textbf{Disjoint union},\textbf{ recoloring} \textbf{and} \textbf{edge
additions}: the algorithm sums over (1) 
or pairs 
and (2) over  or pairs
,
then (3) performs a fixed number of arithmetic operations on numbers
which can be written in  space. 

Each node in the parse tree requires time at most .
Since the size of the parse tree is , the algorithm
runs in fixed-parameter polynomial time. 


\section{Conclusion}

We have defined a new class of graph polynomials, the -Ising
polynomials, extending the -polynomials on the vocabulary
of graphs and have shown that every -Ising polynomial can
be computed in fixed-parameter polynomial time. This result raises
the question of which graph polynomials are -Ising polynomials.
In previous work \cite{pr:GKM08,pr:Makowsky09icla,ar:KM14connection}
we have developed a method based on connection matrices to show that graph
polynomials are not definable in  over either the vocabulary
of graphs or hypergraphs. 
\begin{problem}
How can connection matrices be used to show that graph polynomials
are not -Ising polynomials?
\end{problem}
The Tutte polynomial does not seem to be an -Ising polynomial.
\cite{gimenez2006computing} proved that the Tutte polynomial can
be computed in subexponential time for graphs of bounded clique-width. 
More precisely, the time bound in \cite{gimenez2006computing}
is of the form
, where  for all . 
\begin{problem}
Is there a natural infinite class of graph polynomials definable in
 which includes the Tutte polynomial such that membership
in this class implies {\em fixed parameter subexponential time} computability
with respect to clique-width
(i.e., that the graph polynomial is computable in 
time for some function  satisfiying  for all )?
\end{problem}


\subsection*{Acknowledgement}
We are grateful to Nadia Labai for her comments and suggestions. 
\bibliographystyle{plain}
\begin{thebibliography}{10}

\bibitem{akbari2010zeros}
Saeed Akbari, Saeid Alikhani, Mohammad~Reza Oboudi, and Yee-Hock Peng.
\newblock On the zeros of domination polynomial of a graph.
\newblock {\em Combinatorics and Graphs}, 531:109--115, 2010.

\bibitem{akbari2010characterization}
Saeed Akbari, Saeid Alikhani, and Yee-hock Peng.
\newblock Characterization of graphs using domination polynomials.
\newblock {\em European Journal of Combinatorics}, 31(7):1714--1724, 2010.

\bibitem{DBLP:journals/arscom/AkbariO14}
Saieed Akbari and Mohammad~Reza Oboudi.
\newblock Cycles are determined by their domination polynomials.
\newblock {\em Ars Comb.}, 116:353--358, 2014.

\bibitem{alaeiyan2011cyclically}
Mehdi Alaeiyan, Amir Bahrami, and Mohammad~Reza Farahani.
\newblock Cyclically domination polynomial of molecular graph of some
  nanotubes.
\newblock {\em Digest Journal of Nanomaterials and Biostructures},
  6(1):143--147, 2011.

\bibitem{alikhani2009dominating}
Saeid Alikhani and Yee-Hock Peng.
\newblock Dominating sets and domination polynomials of paths.
\newblock {\em International journal of Mathematics and Mathematical sciences},
  2009.

\bibitem{alikhani2014introduction}
Saeid Alikhani and Yee-hock Peng.
\newblock Introduction to domination polynomial of a graph.
\newblock {\em Ars Combinatoria}, 114:257--266, 2014.

\bibitem{ar:AndrenMarkstrom2009}
Daniel Andr{\'e}n and Klas Markstr{\"o}m.
\newblock The bivariate ising polynomial of a graph.
\newblock {\em Discrete Applied Mathematics}, 157(11):2515--2524, 2009.

\bibitem{ar:andrzejak98}
Artur Andrzejak.
\newblock An algorithm for the {T}utte polynomials of graphs of bounded
  treewidth.
\newblock {\em DMATH: Discrete Mathematics}, 190, 1998.

\bibitem{ar:ArnborgEtAl}
Stefan Arnborg, Jens Lagergren, and Detlef Seese.
\newblock Easy problems for tree-decomposable graphs.
\newblock {\em Journal of Algorithms}, 12(2):308--340, 1991.

\bibitem{arocha2000mean}
Jorge~L. Arocha and Bernardo Llano.
\newblock Mean value for the matching and dominating polynomial.
\newblock {\em Discussiones Mathematicae Graph Theory}, 20(1):57--69, 2000.

\bibitem{pr:BH08}
Markus Bl{\"a}ser and Christian Hoffmann.
\newblock {On the Complexity of the Interlace Polynomial}.
\newblock In {\em {STACS 2008}}, pages 97--108. {IBFI Schloss Dagstuhl}, 2008.

\bibitem{brown2014roots}
Jason~I. Brown and Julia Tufts.
\newblock On the roots of domination polynomials.
\newblock {\em Graphs and Combinatorics}, 30(3):527--547, 2014.

\bibitem{ar:Courcelle90}
Bruno Courcelle.
\newblock The monadic second-order logic of graphs. i. recognizable sets of
  finite graphs.
\newblock {\em Information and computation}, 85(1):12--75, 1990.

\bibitem{bk:CourcelleEngelfriet12}
Bruno Courcelle and Joost Engelfriet.
\newblock {\em Graph structure and monadic second-order logic: a
  language-theoretic approach}, volume 138.
\newblock Cambridge University Press, 2012.

\bibitem{ar:CMR2000}
Bruno Courcelle, Johann~A Makowsky, and Udi Rotics.
\newblock Linear time solvable optimization problems on graphs of bounded
  clique-width.
\newblock {\em Theory of Computing Systems}, 33(2):125--150, 2000.

\bibitem{ar:CourcelleOlariu2000}
Bruno Courcelle and Stephan Olariu.
\newblock Upper bounds to the clique width of graphs.
\newblock {\em Discrete Applied Mathematics}, 101(1):77--114, 2000.

\bibitem{DBLP:journals/dmgt/DodKPT15}
Markus Dod, Tomer Kotek, James Preen, and Peter Tittmann.
\newblock Bipartition polynomials, the ising model and domination in graphs.
\newblock {\em Discussiones Mathematicae Graph Theory}, 35(2):335--353, 2015.

\bibitem{bk:DF99}
Rod~G. Downey and Michael~Ralph Fellows.
\newblock {\em Parameterized complexity}, volume~3.
\newblock springer Heidelberg, 1999.

\bibitem{bk:EF2005}
Heinz-Dieter Ebbinghaus and J{\"o}rg Flum.
\newblock {\em Finite model theory}.
\newblock Springer Science \& Business Media, 2005.

\bibitem{bk:FG06}
J{\"o}rg Flum and Martin Grohe.
\newblock Parameterized complexity theory, volume xiv of texts in theoretical
  computer science. an eatcs series, 2006.

\bibitem{ar:FominGolovachLokshtanov10}
Fedor~V. Fomin, Petr~A. Golovach, Daniel Lokshtanov, and Saket Saurabh.
\newblock Algorithmic lower bounds for problems parameterized with
  clique-width.
\newblock In {\em Proceedings of the Twenty-First Annual {ACM-SIAM} Symposium
  on Discrete Algorithms, {SODA} 2010, Austin, Texas, USA, January 17-19,
  2010}, pages 493--502, 2010.

\bibitem{gimenez2006computing}
Omer Gim{\'e}nez, Petr Hlinen{\`y}, and Marc Noy.
\newblock Computing the {T}utte polynomial on graphs of bounded clique-width.
\newblock {\em SIAM Journal on Discrete Mathematics}, 20(4):932--946, 2006.

\bibitem{pr:GKM08}
Benny Godlin, Tomer Kotek, and Johann~A. Makowsky.
\newblock Evaluations of graph polynomials.
\newblock In {\em Graph-Theoretic Concepts in Computer Science, 34th
  International Workshop, {WG} 2008, Durham, UK, June 30 - July 2, 2008.
  Revised Papers}, pages 183--194, 2008.

\bibitem{ar:GJP03}
Leslie~Ann Goldberg, Mark Jerrum, and Mike Paterson.
\newblock The computational complexity of two-state spin systems.
\newblock {\em Random Structures \& Algorithms}, 23(2):133--154, 2003.

\bibitem{ar:Oum2005}
Sang il~Oum.
\newblock Approximating rank-width and clique-width quickly.
\newblock In Dieter Kratsch, editor, {\em WG}, volume 3787 of {\em Lecture
  Notes in Computer Science}, pages 49--58. Springer, 2005.

\bibitem{ar:JVW90}
Fran{\c{c}}ois Jaeger, Dirk~L. Vertigan, and Dominic J.~A. Welsh.
\newblock On the computational complexity of the jones and tutte polynomials.
\newblock {\em Mathematical Proceedings of the Cambridge Philosophical
  Society}, 108(01):35--53, 1990.

\bibitem{kahat2014dominating}
Sahib~Sh Kahat, Abdul Jalil~M Khalaf, and Roslam Roslan.
\newblock Dominating sets and domination polynomial of wheels.
\newblock {\em Asian Journal of Applied Sciences}, 2(3), 2014.

\bibitem{kotek2012complexity}
Tomer Kotek.
\newblock Complexity of ising polynomials.
\newblock {\em Combinatorics, Probability and Computing}, 21(05):743--772,
  2012.

\bibitem{ar:KM14connection}
Tomer Kotek and Johann~A. Makowsky.
\newblock Connection matrices and the definability of graph parameters.
\newblock {\em Logical Methods in Computer Science}, 10(4), 2014.

\bibitem{DBLP:journals/combinatorics/KotekPSTT12}
Tomer Kotek, James Preen, Frank Simon, Peter Tittmann, and Martin Trinks.
\newblock Recurrence relations and splitting formulas for the domination
  polynomial.
\newblock {\em Electr. J. Comb.}, 19(3):P47, 2012.

\bibitem{kotek2013domination}
Tomer Kotek, James Preen, and Peter Tittmann.
\newblock Domination polynomials of graph products.
\newblock {\em arXiv preprint arXiv:1305.1475}, 2013.

\bibitem{DBLP:journals/gc/KotekPT14}
Tomer Kotek, James Preen, and Peter Tittmann.
\newblock Subset-sum representations of domination polynomials.
\newblock {\em Graphs and Combinatorics}, 30(3):647--660, 2014.

\bibitem{levit2005independence}
Vadim~E. Levit and Eugen Mandrescu.
\newblock The independence polynomial of a graph-a survey.
\newblock In {\em Proceedings of the 1st International Conference on Algebraic
  Informatics}, volume 233254, 2005.

\bibitem{makowsky2004algorithmic}
Johann~A. Makowsky.
\newblock Algorithmic uses of the feferman--vaught theorem.
\newblock {\em Annals of Pure and Applied Logic}, 126(1):159--213, 2004.

\bibitem{makowsky2005coloured}
Johann~A Makowsky.
\newblock Coloured tutte polynomials and kauffman brackets for graphs of
  bounded tree width.
\newblock {\em Discrete Applied Mathematics}, 145(2):276--290, 2005.

\bibitem{pr:Makowsky09icla}
Johann~A. Makowsky.
\newblock Connection matrices for {MSOL}-definable structural invariants.
\newblock In {\em Logic and Its Applications, Third Indian Conference, {ICLA}
  2009, Chennai, India, January 7-11, 2009. Proceedings}, pages 51--64, 2009.

\bibitem{makowsky2006computing}
Johann~A. Makowsky, Udi Rotics, Ilya Averbouch, and Benny Godlin.
\newblock Computing graph polynomials on graphs of bounded clique-width.
\newblock In {\em Graph-theoretic concepts in computer science}, pages
  191--204. Springer, 2006.

\bibitem{ar:noble98}
Steven~D. Noble.
\newblock Evaluating the {T}utte polynomial for graphs of bounded tree-width.
\newblock In {\em Combinatorics, Probability and Computing, Cambridge
  University Press}, volume~7. 1998.

\bibitem{ar:SeymourOum2006}
Oum and Seymour.
\newblock Approximating clique-width and branch-width.
\newblock {\em JCTB: Journal of Combinatorial Theory, Series B}, 96, 2006.

\bibitem{valiant1979complexity}
Leslie~G Valiant.
\newblock The complexity of enumeration and reliability problems.
\newblock {\em SIAM Journal on Computing}, 8(3):410--421, 1979.

\end{thebibliography}

\end{document}
