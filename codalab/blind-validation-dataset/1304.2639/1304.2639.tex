\documentclass[11pt]{amsart}
\usepackage{geometry}                \geometry{letterpaper}                   \usepackage{graphicx}
\usepackage{amssymb}
\usepackage{epstopdf}
\usepackage{enumerate}
\DeclareGraphicsRule{.tif}{png}{.png}{`convert #1 `dirname #1`/`basename #1 .tif`.png}

\newcommand{\Q}{\mathbb{Q}}
\newcommand{\N}{\mathbb{N}}
\newcommand{\Z}{\mathbb{Z}}

\newcommand{\sgn}[1]{\mathrm{sgn}(#1)}

\newcommand{\tab}{\;\;\;\;\;}

\newtheorem{lemma}{Lemma}
\newtheorem{theorem}{Theorem}
\newtheorem{corollary}{Corollary}
\newtheorem{algorithm}{Algorithm}

\theoremstyle{definition}
\newtheorem*{definition}{Definition}

\theoremstyle{remark}
\newtheorem*{remark}{Remark}

\title{The reachability problem for affine functions on the integers}
\author{Daniel Fremont}
\thanks{The author wishes to thank Henry Cohn, without whose copious and very helpful advice this work would not have been possible. Thanks are also due to the many people who gave feedback and suggestions, in particular Christoph Haase.}
\address{Department of Mathematics \\
Massachusetts Institute of Technology \\
Cambridge, MA 02139}
\email{dfremont@mit.edu}
\date{December 15, 2012}
\keywords{reachability problems; affine functions; regular expressions}

\begin{document}

\begin{abstract}
We consider the problem of determining, given  and a finite set  of affine functions on , whether  is reachable from  by applying the functions . We also consider the analogous problem over . These problems are known to be undecidable  for . We give \textsf{2-EXPTIME} algorithms for both problems in the remaining case . The exact complexities remain open, although we show a simple \textsf{NP} lower bound.
\end{abstract}

\maketitle

\section{Introduction}

Many dynamical systems with simple evolution rules nevertheless exhibit unpredictable long-term behavior. Multidimensional systems in particular can easily be so complex that questions like state reachability are undecidable. For instance, this is the case for states in  evolving under a piecewise-linear function \cite{koiran}. There are even simpler nondeterministic examples, such as states in  under a finite set of affine functions \cite{bell-potapov}. For both systems, having states with two coordinates whose evolution is not independent is essential to the undecidability proofs. Generally, while it is often not difficult to prove undecidability for systems with sufficiently high dimension, determining if and when the transition to decidability occurs at lower dimensions is harder. In particular, it is not known whether the reachability problem is decidable for nondeterministic affine evolution on .

In this paper, we consider the simpler problem of reachability under nondeterministic affine evolution on : given  and a finite set  of affine functions  with , determine whether  is reachable from  by applying functions in . The generalizations to  are undecidable for all , as implicitly shown in \cite{paterson} (and a little more clearly in \cite[Section 4.9]{gaubert-katz}). We prove that the remaining case, , is decidable, giving a \textsf{2-EXPTIME} algorithm for it in Section \ref{reach-z}. We also consider the version of this problem with evolution over : in this problem, the functions  can still have negative coefficients, but may not be applied if they would yield a negative result. In Section \ref{reach-n} we give a \textsf{2-EXPTIME} algorithm for this problem by modifying our algorithm for the case over . Finally, in Section \ref{lower-bound} we show that the problems over  and  are both \textsf{NP}-hard.

\section{Affine reachability over } \label{reach-z}

We begin by defining some notation that we will use throughout this paper.
\begin{definition}
If  is a set of affine functions on , we will call any function of the form  for some  with each  an -\emph{composition}. We will want to discuss the individual functions  which appear in an -composition, so  is formally the tuple , but we will often view -compositions as functions without further comment. The \emph{orbit} of  applied to the argument  is the set of values . We write  \emph{via}  to indicate that  is an -composition such that , and  to assert the existence of such a .
\end{definition}
In this notation, the affine reachability problem over  is to determine, given  and a finite set  of affine functions  with , whether . The problem takes several qualitatively different forms depending on the values of the linear coefficients . The simplest nontrivial case is when they all satisfy . 

\begin{lemma} \label{expanding-reachable}
There is an \textsf{EXPTIME} algorithm to decide, given any  and a finite set  of functions  with  and satisfying , whether .
\end{lemma}
\begin{proof}
Outside of some finite interval, for instance  with , each function  strictly increases absolute value. Putting , for any -composition  and  with  we have  and thus . This means that all preimages of  under -compositions must lie in the finite interval . Create a directed graph  with a vertex for each integer , and add edges from  to  for each  satisfying . Since every preimage of  under an -composition lies in , we have  if and only if  and there is a path in  from  to . We can determine whether such a path exists in exponential time using graph search, since  has exponentially-many vertices (at most linearly-many in the values of  and ).
\end{proof}
\begin{remark}
As will be important later, with a small modification of this algorithm we can handle the presence of one  with , so that . The only change necessary is to broaden  to the interval . Then  maps  onto itself, so all preimages of  under -compositions are in , and the argument goes through as above.
\end{remark}

However, when a function in  is of the form  this method breaks down, because the preimages of  under -compositions are no longer bounded. This also happens if there are two functions of the form , since then their composition is of the form . Fortunately, functions like  can contribute to an -composition in basically only one way.
\begin{lemma}
\label{extraction-lemma}
For any set  of functions  with  for  and function , put . Then for any -composition , we have:
\begin{enumerate}[\tab (a)]
\item If , then  where  and . \label{single-extraction}
\item  for some -composition  and . If  for every function  appearing in , then . \label{total-extraction}
\end{enumerate}
\end{lemma}
\begin{proof}
\begin{enumerate}[(a)]
\item We have , so . Repeating this  more times gives  with  and .
\item Apply the previous result once for each instance of  in , giving  for some  which are products of the coefficients . Then  with  the -composition  and . If  for each  appearing in , then  and so  (we could have  if there were no instances of  in ). \qedhere
\end{enumerate}
\end{proof}
We now have several cases, based on which -compositions  satisfy  via .
\begin{lemma} \label{lemma-cases}
For any , finite set  of functions  with  and , and function  with  and , put  and . Then the following are true:
\begin{enumerate}[\tab (A)]
\item If , then . \label{case-none}
\item If some  satisfies , then . \label{case-opposite-sign}
\item If some  with  has  for some , then . \label{case-negative-coeff}
\item If none of the above cases hold, then . \label{case-same-sign}
\end{enumerate}
\end{lemma}
\begin{proof}
\begin{enumerate}[(A)]
\item By Lemma \ref{extraction-lemma}\ref{total-extraction}, any -composition can be written as an -composition plus a multiple of . If , then no -composition can reach  from , and therefore neither can any -composition.
\item If some  satisfies , then either  or . If the first of these is true, then  and so  via . Otherwise, there is some  such that , so . Putting , we have  via .
\item If some  with  has  for some , take the smallest such . Defining , by Lemma \ref{extraction-lemma}\ref{single-extraction} we have  with , and because  for all  by our choice of , we have . We may assume that case \ref{case-opposite-sign} does not hold, since otherwise we have  immediately as shown above. Then we have , and so . Therefore with  sufficiently large we have . Then as in case \ref{case-opposite-sign}, we have  via  for some .
\item Suppose that  via some . Then by Lemma \ref{extraction-lemma}\ref{total-extraction} we have  for some -composition  and . Now , so . Since case \ref{case-opposite-sign} does not hold, we have . Since case \ref{case-negative-coeff} does not hold, we have , again by Lemma \ref{extraction-lemma}\ref{total-extraction}. If , then , so  and case \ref{case-opposite-sign} holds, contrary to our assumption. So , and thus . But this is impossible, since . So we cannot have . \qedhere
\end{enumerate}
\end{proof}

To test cases (\ref{case-none}), (\ref{case-opposite-sign}), and (\ref{case-negative-coeff}), we use the following algorithm.

\begin{lemma} \label{largest-mod-reachable}
Given any  with  and a set  of functions  with  and  for , put . There is a \textsf{2-EXPTIME} algorithm such that:
\begin{enumerate}
\item If , the algorithm returns \textsc{Empty}.
\item If there is some  with  and  for some , the algorithm returns \textsc{Negative}.
\item Otherwise, the algorithm returns .
\end{enumerate}
\end{lemma}

The flags \textsc{Empty} and \textsc{Negative} are enough for us to detect cases \ref{case-none} and \ref{case-negative-coeff} of Lemma \ref{lemma-cases}. If , the value  allows us to recognize case \ref{case-opposite-sign}, because this case holds if and only if . If  we need the value  instead, since then case \ref{case-opposite-sign} holds if and only if . The modifications to the algorithm of Lemma \ref{largest-mod-reachable} required to make it compute  instead of  are simple and obvious (just exchanging ``increases'' with ``decreases'' in several places, etc.), so we omit them. Now we prove Lemma \ref{largest-mod-reachable}, assuming a couple of auxiliary lemmas (Lemmas \ref{I-lemma} and \ref{I-algorithm}) which we will return to afterwards.

\begin{proof}[Proof of Lemma \ref{largest-mod-reachable}]
First we check if there is \emph{any} -composition mapping  to . Create a directed graph  with a vertex for each congruence class mod . Add edges indicating which classes are mapped to which under each . Then there is a path in  from the congruence class of  to the congruence class of  if and only if . Use graph search to determine if there is such a path, and return \textsc{Empty} if not. Since  has  vertices, this search takes exponential time.

If there are paths from  to , we need to analyze all of them to see which ones yield the largest final value. We can conveniently describe the paths using regular expressions. Consider  to be a deterministic finite automaton, where an input symbol  causes the edge corresponding to applying  to be followed. Let the initial state be , and the only accepting state be . If  is a sequence of input symbols, we write  (note the order!), and then  accepts  if and only if .

Now we convert  into a regular expression  with the same language . We write concatenation multiplicatively, use  to denote union/alternation, and use  and  as the symbols for the empty string and the empty language respectively. Because  has exponentially-many vertices (and a linearly-sized input alphabet),  has at most doubly-exponential size , and the conversion from  to  can be done in time at most polynomial in  (see \cite{mcnaughton-yamada, dfa-to-re-upper}). We store  as a tree, with literals, , or  at the leaf nodes and operators at the other nodes. The ``length''  in this representation is just the total number of nodes.

Next, reduce  to not include the symbol  by repeatedly passing through  applying the identities , , and  for any expression . Each pass takes time linear in the length of  and strictly decreases its length, so there can be at most  passes and the total time taken is . Afterwards, if the symbol  appears in  it must not be operated on by any operator, since otherwise one of the identities above would apply. Therefore  can appear in  only if , but since  is nonempty (because we returned \textsc{Empty} above if so) this is not the case. So  does not contain the symbol .

Now if  contains a literal corresponding to a function  with  (which can obviously be determined in  time), then  via an -composition which includes , so we return \textsc{Negative}. Otherwise, we convert  into disjunctive normal form  where each  has no union operations, by iteratively applying the identities , , and . Each identity either decreases the number of unions or moves one closer to the topmost level, so this process will also take time polynomial in .

We say that a regular expression  is \emph{reduced} if it contains only literals appearing in  and has no  symbols or union operations. The reductions we have done above ensure that any expression produced by concatenating subexpressions of the clauses  is reduced. Since every literal in  corresponds to a function  with  (because we would have returned \textsc{Negative} otherwise), and the composition of two linear polynomials with positive linear coefficients has a positive linear coefficient, for any reduced expression  and any ,  has a positive linear coefficient --- this will be important in a moment. We will want to refer to those -compositions which are generated by reduced expressions, and to decrease the proliferation of symbols, we will say that an -composition  \emph{matches} the expression  if there is some  such that . Then  consists precisely of those -compositions which match .

Now, given some  and a reduced expression , define  to mean that . In words,  is true if and only if there is some -composition matching  which increases . If  is finite, computing  is only a matter of testing various cases --- the difficulty is handling expressions with stars. Fortunately, we can reduce  to its values on expressions with fewer stars using the identity , which we will prove in Lemma \ref{I-lemma}. This then allows us to compute  recursively in polynomial time, as we will show in Lemma \ref{I-algorithm}.

Now we are ready to return to the main problem. For each clause , we want to find the supremum  of the possible values  is mapped to by any -composition matching . To do this, we keep track of the supremum of the values  is mapped to by -compositions which match progressively-longer prefixes of . Write  by flattening out concatenations, so that each  is either a literal or a starred subexpression. Let  for . Clearly , and we put  (since the largest possible value reachable after applying no functions is the starting value ). For , we calculate  in terms of  as follows. If  is a literal, then any -composition matching  must be of the form  where  is an -composition matching . Since by definition the largest possible value of  is , and  has a positive linear coefficient, the largest possible value of  is . Thus . If  is a starred subexpression instead, , we compute . If this is true, then some -composition  matching  increases , and because  has a positive linear coefficient it must increase all values larger than . So we can increase  as much as we want by repeatedly applying , and thus . If  is false, then no -composition matching  increases , so  (since  is matched by , which leaves  fixed). Thus we can iteratively compute , beginning with  and proceeding through . There are  intermediate values  which need to be computed, and each one requires at most one call to  on an expression of size at most . Thus we can compute each  in time polynomial in , and all the values  together in time polynomial in .

Now put . Because the union of the languages of each clause  is the language of ,  is the largest value reachable using -compositions matching  (or  if arbitrarily large values are reachable). Therefore , the desired value, and we return it.

As mentioned above, the first stage of this algorithm takes exponential time, and all subsequent stages take time at most polynomial in . Since  is at most doubly-exponential in the input length, the entire algorithm takes at most doubly-exponential time.
\end{proof}

Lemma \ref{largest-mod-reachable} depends on our ability to efficiently calculate . As was mentioned above, the key to doing this is to reduce  to values of  on smaller subexpressions, as made possible by the following lemma.

\begin{lemma} \label{I-lemma}
If ,  and  are reduced regular expressions, and  is a sequence of literals, the following are true:
\begin{enumerate}
\item  \label{I-lemma-literal-star-prefix}
\item  \label{I-lemma-literal-star}
\item  \label{I-lemma-star-prefix}
\item  \label{I-lemma-star}
\end{enumerate}
\end{lemma}
\begin{proof}
\begin{enumerate}
\item  implies  because  matches . If  holds, then there is an -composition  matching  which increases . Because  has a positive linear coefficient as observed above (since  is reduced),  must increase anything greater than , and thus repeated applications of  can increase  as much as desired. Therefore for any particular -composition  matching , there is some  such that  increases  (since  is reduced and so  also has a positive linear coefficient). Since  matches ,  holds.

Suppose neither  nor  hold. Any -composition matching  is of the form  where  matches  and  is a composition of -compositions matching . Since by our assumption no polynomials matching  increase , and these have positive linear coefficients, they cannot increase anything smaller than . Thus no composition of -compositions matching  increases , and so  does not increase . Now, since  does not increase  by assumption (since it matches ), and  has a positive linear coefficient,  does not increase  either. Thus no -composition matching  increases , and  does not hold.
\item Put  in (\ref{I-lemma-literal-star-prefix}) and use  and .
\item Put  in (\ref{I-lemma-literal-star-prefix}) and use  and .
\item Put  in (\ref{I-lemma-star-prefix}), use , and note that  is clearly false. \qedhere
\end{enumerate}
\end{proof}

These relationships allow us to give a straightforward recursive algorithm to compute .

\begin{lemma} \label{I-algorithm}
There is a \textsf{P} algorithm to compute  for any  and reduced regular expression .
\end{lemma}
\begin{proof}
First, note that if an expression  is a sequence of literals, then only  matches it, and we may determine  directly by evaluating . Now we compute  recursively, breaking into cases based on the topmost operator or symbol of :
\begin{itemize}
\item :  is clearly false.
\item  is a literal: As noted above,  may be directly calculated.
\item : By part \ref{I-lemma-star} of Lemma \ref{I-lemma}, .
\item : By flattening as necessary, we may write  for some  with  and where none of the subexpressions  are concatenations. If any of the subexpressions  are , we simply drop them and renumber appropriately --- this obviously leaves  fixed. Thus we may assume that each subexpression  is either a starred subexpression or a literal. If each one is a literal, then  is a sequence of literals and as noted above  can be computed directly. Otherwise, find the smallest  such that  is a starred subexpression. There are several cases:
\begin{itemize}
\item : Then  where , so by part \ref{I-lemma-star-prefix} of Lemma \ref{I-lemma} we have .
\item : Then  where  is a sequence of literals, so by part \ref{I-lemma-literal-star} of Lemma \ref{I-lemma} we have .
\item : Then  where  is a sequence of literals and , so by part \ref{I-lemma-literal-star-prefix} of Lemma \ref{I-lemma} we have .
\end{itemize}
\end{itemize}

In each case  is either directly computable, equivalent to  for some  and  a proper subexpression of , or equivalent to  for some ,  a proper subexpression or concatenation of disjoint subexpressions of , and  a subexpression of  disjoint from . Because  is always reduced to values of  on strictly shorter expressions, the tree of recursive calls has at most  levels. Since  is always reduced to values of  on disjoint expressions made up of subexpressions of , each level of the tree can have at most  calls. Thus the entire tree has at most  calls with polynomial computation each, so  may be computed in polynomial time.
\end{proof}

We can now give an algorithm to solve the affine reachability problem over  in full generality.

\begin{theorem} \label{main-algorithm}
There is a \textsf{2-EXPTIME} algorithm to decide, given any  and a finite set  of functions  with , whether .
\end{theorem}
\begin{proof}
There are several cases:
\begin{enumerate}
\item For some , : Clearly,  if and only if either  or ; recursively determine each of these and return true if and only if at least one is true. \label{ma-case-constant}
\item For some ,  and : Clearly  if and only if  (since  is the identity), so determine this recursively and return the result. \label{ma-case-identity}
\item For some ,  and : Assume for now that  (we will handle  momentarily). Run the algorithm of Lemma \ref{largest-mod-reachable} on  with . If it returns \textsc{Empty}, then case \ref{case-none} of Lemma \ref{lemma-cases} holds, so return false. If it returns \textsc{Negative}, then case \ref{case-negative-coeff} holds, so return true. Otherwise, the algorithm returns . Since now either case \ref{case-opposite-sign} or case \ref{case-same-sign} of Lemma \ref{lemma-cases} holds,  if and only if . So we return true if and only if . If  was in fact positive, we use the variant of the algorithm of Lemma \ref{largest-mod-reachable} which computes . By Lemma \ref{lemma-cases} again we have  if and only if , so we return true if and only if . \label{ma-case-shift}
\item For some , , and  for all : Use the algorithm of Lemma \ref{expanding-reachable}, modified as in the remark to handle . \label{ma-case-involution}
\item For some  with , : Define . Clearly  if and only if . But , and since  (since  and  are distinct functions),  for some  with . Recursively solve  using case (\ref{ma-case-shift}) and return the result. \label{ma-case-two-involutions}
\item Otherwise,  for all : Use the algorithm of Lemma \ref{expanding-reachable}. \label{ma-case-expanding}
\end{enumerate}
Cases (\ref{ma-case-involution}) and (\ref{ma-case-expanding}) invoke the algorithm of Lemma \ref{expanding-reachable} and take exponential time. Case (\ref{ma-case-shift}) invokes the algorithm of Lemma \ref{largest-mod-reachable} and takes doubly-exponential time. Case (\ref{ma-case-two-involutions}) makes a recursive call which always uses case (\ref{ma-case-shift}), so it also takes doubly-exponential time. Finally, cases (\ref{ma-case-identity}) and (\ref{ma-case-constant}) make one and two recursive calls respectively, each with one less affine function. Thus in total the algorithm will make at most exponentially-many recursive calls (this can easily be reduced to linearly-many by improving the handling of case (\ref{ma-case-constant}), but this does not decrease the worst-case runtime), with at most a doubly-exponential amount of computation each. Therefore the algorithm runs in at most doubly-exponential time.
\end{proof}

\section{Affine reachability over } \label{reach-n}

We can also consider the affine reachability problem over . Much of the analysis is the same, so we will only write out in full detail the considerations which are new. The main difference from the version over  is that now we cannot apply any functions which would yield a negative result.
\begin{definition}
An -composition is \emph{valid with respect to its argument} if every integer in its orbit for the given argument is nonnegative. Often the argument of the composition will be clear from context, in which case we will simply say that the composition is \emph{valid}. We write  to indicate that there is a valid -composition  such that .
\end{definition}
With this definition, the affine reachability problem over  is to determine, given  and a finite set  of functions  with , whether . As before, there are various cases depending on the values of the linear coefficients . The case where they all satisfy  is still simple.
\begin{lemma} \label{n-expanding-reachable}
There is an \textsf{EXPTIME} algorithm to decide, given any  and a finite set  of functions  with  and satisfying , whether .
\end{lemma}
\begin{proof}
Use the algorithm of Lemma \ref{expanding-reachable}, but with the interval  instead of . The argument in the proof of Lemma \ref{expanding-reachable} goes through as before, since all preimages of  under \emph{valid} -compositions must lie in .
\end{proof}
\begin{remark}
This algorithm also works with any number of functions of the form  with , since these strictly increase absolute value on , and so preimages of  under valid compositions including them must lie in .
\end{remark}

The algorithm of Lemma \ref{n-expanding-reachable} cannot handle functions of the form  with . Fortunately, as before we can reduce problems with this type of function to ``modular'' problems without them, using the following (much simpler) analog of Lemma \ref{lemma-cases}. We assume for now that all , and show how to handle other cases later.

\begin{lemma} \label{n-lemma-cases}
For any , a set  of functions  with  and  for , and function  with , put . Then  if and only if  for some .
\end{lemma}
\begin{proof}
Suppose  via . By Lemma \ref{extraction-lemma}\ref{total-extraction},  with  and  being  with all instances of  removed. Since , we have . Now note that  always decreases its argument, and since every  is positive, each  maps larger inputs to larger outputs. Therefore removing an instance of  from  can only increase the values of integers in its orbit. Since  is valid this means  must be as well, and since  this means . Therefore  with .

Conversely, suppose  via  with . Then  for some . Putting ,  is valid because  is, and so we have  via .
\end{proof}

The algorithm of Lemma \ref{largest-mod-reachable} is almost exactly what we need to test the condition in Lemma \ref{n-lemma-cases}, since it computes the largest  reachable by -compositions. However, we must consider only those reachable by \emph{valid} -compositions, and so need to modify the algorithm. This is not hard to do.

\begin{lemma} \label{n-largest-mod-reachable}
Given any ,  with , and a set  of functions  with  and  for , put . There is a \textsf{2-EXPTIME} algorithm which returns \textsc{Empty} if , and otherwise returns .
\end{lemma}
\begin{proof}
As in the algorithm of Lemma \ref{largest-mod-reachable}, construct the graph  and search it to determine if  via any -composition, not necessarily a valid one. If not, return \textsc{Empty}. Otherwise, consider  to be a finite automaton as before, and convert it into a reduced regular expression  in disjunctive normal form .

Given some  and  a sequence of literals which appear in , we define  to mean that  is a valid -composition with respect to . Now for any  and reduced expression , define  to mean . In words, this means that there is some -composition valid with respect to  which matches  and increases  (this is just the analog of  from Lemma \ref{largest-mod-reachable}, but restricted to only valid compositions). By an extension of Lemma \ref{I-lemma} which will prove momentarily, Lemma \ref{n-I-lemma}, we have that . Using this we may compute  in polynomial time with the analog of the algorithm of Lemma \ref{I-algorithm}, described shortly in Lemma \ref{n-I-algorithm}.

Now we continue as in the algorithm of Lemma \ref{largest-mod-reachable}, writing  and defining . We calculate the values  in the same way as before, except using  in place of  when dealing with starred subexpressions . This ensures that only -compositions which are valid with respect to  are used to compute  when  is a starred subexpression. When  is a literal, we put  as usual, but also check if . If so, then  is not valid with respect to , and thus no -composition matching  can be valid with respect to . So we discard  and move on. Otherwise again  can be obtained from  using a valid -composition. Then if we compute  without discarding , the value  can be obtained from  using a valid -composition, and so  is the supremum of possible values  is mapped to by any valid -composition matching .

If we discarded every , then no valid -compositions match , so  and we return \textsc{Empty}. Otherwise, if  is the largest of the values  (at least one of these is defined since we did not discard every ) then  and we return it. As in Lemma \ref{largest-mod-reachable}, this algorithm takes time polynomial in , and thus takes at most doubly-exponential time.
\end{proof}

Now we prove the analog of Lemma \ref{I-lemma} for .

\begin{lemma} \label{n-I-lemma}
If ,  and  are reduced regular expressions, and  is a sequence of literals, the following are true:
\begin{enumerate}
\item  \label{n-I-lemma-literal-star-prefix}
\item  \label{n-I-lemma-literal-star}
\item  \label{n-I-lemma-star-prefix}
\item  \label{n-I-lemma-star}
\end{enumerate}
\end{lemma}
\begin{proof}
\begin{enumerate}
\item  implies  because  matches . If  holds, then there is an -composition  matching  which is valid with respect to and increases . Because  has a positive linear coefficient (since we assumed all ),  must increase anything greater than , and thus repeated applications of  can increase  as much as desired. Therefore for any particular -composition  matching , there is some  such that  is valid with respect to and increases , and  increases . If we also have , then  is valid with respect to , and then  is valid with respect to and increases . Then, since  matches ,  holds.

Suppose  does not hold. Then no -composition  matching  can be valid with respect to , since the first part of  must be , which is not valid with respect to  and thus will cause some integers in the orbit of  to be negative. So  does not hold.

Suppose that neither  nor  hold. Any valid -composition matching  is of the form  where  matches  and  is a composition of -compositions matching . Since by our assumption valid -compositions matching  do not increase , and these have positive linear coefficients, they do not increase anything smaller than . Thus no valid composition of -compositions matching  increases , and so  does not increase . Therefore because  is valid with respect to ,  is valid with respect to , and since  does not hold,  does not increase . Because  has a positive linear coefficient,  does not increase  either. Thus no valid -composition matching  increases , and  does not hold.
\item Put  in (\ref{n-I-lemma-literal-star-prefix}) and use  and .
\item Put  in (\ref{n-I-lemma-literal-star-prefix}) and use  and .
\item Put  in (\ref{n-I-lemma-star-prefix}), use , and note that  is clearly false. \qedhere
\end{enumerate}
\end{proof}
As before, these relationships yield a recursive algorithm for computing :
\begin{lemma} \label{n-I-algorithm}
There is a \textsf{P} algorithm to compute  for any  and reduced regular expression .
\end{lemma}
\begin{proof}
The algorithm is identical to that of Lemma \ref{I-algorithm}, except that in addition to reducing  to values of  on shorter expressions, the identities in Lemma \ref{n-I-lemma} also require evaluations of . Clearly  can be computed in polynomial time, by simply calculating the entire orbit of  and checking that every integer in it is nonnegative. Adding a single computation of  at each recursive call in the algorithm of Lemma \ref{I-algorithm} multiplies its runtime by a polynomial factor, so the overall algorithm still runs in polynomial time.
\end{proof}

Now we can give a general algorithm for the affine reachability problem over .

\begin{theorem}
There is a \textsf{2-EXPTIME} algorithm to decide, given any  and a finite set  of functions  with , whether .
\end{theorem}
\begin{proof}
There are several cases:
\begin{enumerate}
\item For some , : As in Theorem \ref{main-algorithm}, recursively determine  and  and return true if and only if at least one holds.
\item For some , : There are only a finite number of  which  can be applied to without giving a negative result: for example, they are all in . Create a directed graph  with a vertex for each of these, as well as vertices for  and  if not already present. Add edges indicating how  maps these values to each other. Use this algorithm recursively on  to add edges corresponding to mappings by all possible valid -compositions. Then there is a path in  from  to  if and only if . Use graph search to test if such a path exists, and return the result. \label{n-ma-negative}
\item For some ,  and : As in Theorem \ref{main-algorithm}, recursively solve  and return the result.
\item For some ,  and : Run the algorithm in Lemma \ref{n-largest-mod-reachable} on  with . If it returns \textsc{Empty}, then by Lemma \ref{n-lemma-cases} we cannot have  and we return false. Otherwise the algorithm returns , and again by Lemma \ref{n-lemma-cases} we have  if and only if . So return true if and only if .
\item Otherwise, for all  we have , and if  then : Use the algorithm of Lemma \ref{n-expanding-reachable} (which works in this case as noted in the remark), and return the result.
\end{enumerate}
The analysis of the runtime is similar to that given in Theorem \ref{main-algorithm}, except for case \ref{n-ma-negative}. The graph  created in that case has exponentially-many vertices (linear in the value of ), so each invocation of the algorithm makes at most exponentially-many recursive calls. There are at most a linear number of levels, since each recursive call has one less affine function than its parent. Thus there are exponentially-many recursive calls in total. The work done in each call takes at most doubly-exponential time (since the algorithm in Lemma \ref{n-largest-mod-reachable} can take this long), so the overall running time of the algorithm is at most doubly-exponential.
\end{proof}

\section{A Lower Bound} \label{lower-bound}

While it may be hoped that there are vastly more efficient algorithms for the affine reachability problems than the \textsf{2-EXPTIME} methods we have given here, the following theorem shows that polynomial-time algorithms are unlikely.

\begin{theorem}
The affine reachability problems over  and  are \textsf{NP}-hard.
\end{theorem}
\begin{proof}
We give a reduction from the Integer Knapsack Problem (IKP), which is to determine, given , whether there are  such that . This problem is known to be \textsf{NP}-complete \cite{lueker}. For a given instance of the IKP, , let the set  consist of the affine functions  for . If there exist  such that , then , so . Since the functions  all commute, if  then  for some . Thus  if and only if the IKP instance is solvable. Computing  given an IKP instance can obviously be done in polynomial time, so this gives a polynomial-time many-one reduction from IKP to the affine reachability problem over , showing that the latter is \textsf{NP}-hard. The reduction to affine reachability over  is exactly the same, since all -compositions are valid.
\end{proof}

\section{Conclusion}

We gave \textsf{2-EXPTIME} algorithms for the affine reachability problems over  and , and showed that they are \textsf{NP}-hard. Beyond improving these upper and lower bounds, a natural generalization that would be interesting to consider is if integer or integer-valued polynomials are allowed instead of just affine functions. Also, the original problem which this paper treated a special case of, namely reachability for affine evolution over , remains open. This provides another clear direction for future work.

\bibliographystyle{elsart-num}
\bibliography{main}

\end{document}