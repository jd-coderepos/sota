\documentclass[
final
]{dmtcs-episciences}        \usepackage{graphics, amsthm, amsmath, amssymb, algorithm, algorithmic}
\newtheorem{theorem}{Theorem}
\newtheorem{observation}{Observation}
\newtheorem{definition}{Definition}
\newtheorem{lemma}{Lemma}
\newtheorem{corollary}{Corollary}
\newtheorem{statement}{Statement}
\newtheorem{problem}{Problem}
\newtheorem{property}{Property}






\author{Jessica Enright\affiliationmark{1}
	\and Lorna Stewart\affiliationmark{2}}
\title{Equivalence of the filament and overlap graphs of subtrees of limited trees}

\affiliation{
University of Stirling, Stirling, Scotland, UK\\
	University of Alberta, Edmonton, Alberta, Canada}
	
	\keywords{graph algorithms, intersection graphs, filament graphs}
\received{2015-8-7}
	
	\revised{2016-7-27, 2017-6-2}
	
	\accepted{2017-6-2}
	
	
	\begin{document}
	\publicationdetails{19}{2017}{1}{20}{1274}

\maketitle

\begin{abstract}
The overlap graphs of subtrees of a tree are equivalent to subtree filament graphs, the overlap graphs of subtrees of a star are cocomparability graphs, and the overlap graphs of subtrees of a caterpillar are interval filament graphs. In this paper, we show the equivalence of many more classes of subtree overlap and subtree filament graphs, and equate them to classes of complements of cochordal-mixed graphs. 
Our results generalise the previously known results mentioned above.
\end{abstract}

\section{Introduction}\label{intro}
The class of subtree overlap graphs is equivalent to the class of subtree filament graphs, which means that those graphs have both overlap and filament intersection representations on trees \cite{Jess}.
The class contains many graph classes that have extensive structural properties, algorithms, and complexity results, such as interval graphs, permutation graphs, cocomparability graphs, chordal graphs, circle graphs, circular-arc graphs, polygon-circle graphs, and interval filament graphs.  
Some of these graphs have been characterised in terms of subtree overlap representations on restricted host trees. In particular, cocomparability graphs are the overlap graphs of subtrees of a star (follows from \cite{GolSch}), circle graphs are the overlap graphs of subtrees of a path \cite{gavril1973}, and interval filament graphs are the overlap graphs of subtrees of a caterpillar (this fact was presented at a workshop but not published \cite{CGO}).
Thus, we have the equivalence of general subtree overlap graphs and subtree filament graphs, and we know that some subtree overlap graphs that admit representations on restricted host trees are equivalent to well-known graph classes, one of which is a natural class of subtree filament graphs.

In this paper, we identify new equivalences between subtree overlap and subtree filament graph classes based on host trees of their representations.
We first introduce the notion of a covering subtree of a tree representation. We show that the host tree of any subtree overlap representation can be modified so that it consists of just a covering subtree plus some additional leaves, without altering the represented graph.
In addition, we prove that a graph has a subtree overlap representation with a given covering subtree if and only if it is the complement of a restricted type of cochordal-mixed graph. 
Finally, we show that for a set  of trees that is closed under edge subdivision, a graph has a subtree filament representation with host tree in  if and only if it has a subtree overlap representation with covering subtree in . 

Our first theorem generalises the characterisation of cocomparability graphs as the overlap graphs of subtrees of a star by equating the overlap graphs of subtrees of a star with subtree overlap graphs that have a representation with a single-vertex covering subtree. Theorem 1 generalises this correspondence for any given covering subtree.

Our second theorem bridges the gap between the previously known equivalence of general subtree overlap graphs and subtree filament graphs, and the characterisation of interval filament graphs as the overlap graphs of subtrees of a caterpillar. Both of those previously known results and new equivalences between subtree overlap and subtree filament graph classes are given in Theorem 2. The theorem suggests a division of subtree overlap graphs into subclasses,
each of which contains all interval filament graphs, and the union of which
is the class of subtree overlap graphs. 
This view of subtree overlap graphs may give insight into their structure and the algorithmic complexity of problems in that domain. 

We consider finite, simple graphs. Let  be a graph.
The {\em neighbourhood} of a vertex  in  is .
 denotes the complete graph on  vertices.

Two sets  and  {\em intersect} if , and
{\em overlap}, denoted , if , , and . Sets  and  are {\em disjoint}, denoted , if .
Let  be four nonempty sets.  We say that  and  are \emph{similarly related}, denoted  if  if and only if  and  if and only if . 

Let  be a multiset of nonempty sets. We use the term multiset rather than set to allow for the possibility that  for some  where . The {\em intersection graph} 
(respectively, {\em overlap graph}, {\em disjointness graph}, {\em containment graph}) of  is the graph  where  and, for all ,  if and only if  and  intersect (respectively, overlap, are disjoint, are contained one in the other). 
If  is the intersection,
overlap, disjointness, or containment graph of  then  is
called an intersection, overlap, disjointness, or containment representation
of . Every graph has both an intersection and a disjointness
representation \cite{Marcz} as well as an overlap representation (obtained
by adding a unique new element to each set of
an intersection representation). 
Note
that, for  and 
 where
 for all ,
the intersection (respectively,
overlap, disjointness, containment) graphs of  and  are
isomorphic.

In this paper, we are concerned with the intersection, overlap, disjointness, and containment graphs of subtrees of a tree. For a given tree , we will assume that a collection of  subtrees of  is given as a multiset  of subsets of the vertices of , each of which induces a subtree of .

{\em Interval graphs} are the intersection graphs of intervals
on a line or, equivalently, the intersection graphs of subtrees of a path.
{\em Cointerval graphs} are the complements of interval graphs, that is, the disjointness graphs of subtrees of a path.
{\em Circle graphs} are the 
intersection graphs of chords in a circle or, equivalently, the overlap graphs of subtrees of a path.  {\em Chordal graphs} are graphs in which
every cycle of length greater than three has a chord
or, equivalently, the intersection graphs of subtrees in
a tree. 
{\em Cochordal graphs} are the complements of chordal graphs.
{\em Comparability graphs} are graphs whose edges
can be transitively oriented. Equivalently, comparability
graphs are the containment graphs of subtrees of a
tree, the containment graphs of subtrees of a star, and
the set of all containment graphs \cite{GolSch}. 
{\em Cocomparability} graphs are the complements of comparability graphs.
{\em Subtree overlap graphs} are the overlap graphs of subtrees in a tree.
If a graph  is the overlap (respectively, intersection, containment, or disjointness) graph of 
subtrees  of a tree , then  is a \emph{subtree overlap (respectively, intersection, containment, or disjointness) representation} of .  is termed the {\em host tree} of the representation. For convenience, we will use the notation that vertex  corresponds to subtree . 
A {\em caterpillar} is a tree such that the removal of its leaves results in a path.
All of the graph classes defined in this section are {\em hereditary}, that is, every induced subgraph of a graph in the class is also in the class.
For more information about graph classes, see \cite{BLS}.

Interval filament graphs, subtree filament graphs, and -mixed graphs were introduced by Gavril \cite{gavril2000}. We give the definitions of those graph classes and related concepts next.

Let  be a multiset of (closed) intervals on a line  and let  be a plane containing . We will refer to one of the half-planes into which  divides  as being above .
Then  is a set of interval filaments on the intervals of  if
each , , is a curve in , on and above , connecting the endpoints of , such that if two intervals are disjoint then their curves do not intersect.
Thus, pairs of filaments corresponding to disjoint intervals do not intersect, pairs of filaments corresponding to overlapping intervals intersect, and pairs of filaments corresponding to intervals where one is contained in the other may or may not intersect.
The intersection graph of  is called an {\em interval filament graph}.

Subtree filaments and subtree filament graphs are defined analogously.
Let  be a tree and let  be a multiset of subtrees of . Suppose that  is embedded in a plane  and let  be a surface perpendicular to  whose intersection with  is . 
(One can imagine forming the part of  that is above  by drawing  upwards from .)
Then  is a set of subtree filaments on the subtrees of  if
each , , is a curve in , in and above , connecting the leaves of , such that 
(i) if two subtrees are disjoint then their curves do not intersect, and
(ii) if two subtrees overlap then their curves intersect.
Thus, pairs of filaments corresponding to disjoint subtrees do not intersect, pairs of filaments corresponding to overlapping subtrees intersect, and 
pairs of filaments corresponding to intervals where one is contained in the other may or may not intersect.
If a graph  is the intersection graph of a collection of filaments on subtrees of a tree , then  is a {\em subtree filament graph} and the collection of filaments is a {\em subtree filament representation} of . The tree  is called the {\em host tree} of the representation. 

Let  a hereditary graph class. A graph  is said to be {\em -mixed} if there is a partition of its edges into  and  such that:
\begin{itemize}
 \item  is in  and 
 \item there is a transitive orientation 
 
 of the graph  such that
 for every three distinct vertices , if
       and , then .
 \end{itemize}
Such a partition is called a {\em -mixed partition} of the edges of .
Subtree filament graphs are exactly the complements of cochordal-mixed graphs, and interval filament graphs are exactly the complements of cointerval-mixed graphs \cite{gavril2000}.
 
Let  be a multiset of subtrees of a tree .
A subtree  of  is called
a {\em covering subtree} of 
 
if it intersects every member of . 
Note that the intersection of each element of  with a covering subtree  is a subtree of .
Let  be a tree and let  be a subtree of .
A vertex  of  is called {\em bushy} 
(with respect to  in )
if every neighbour of  that is not in  is a leaf of ; the entire subtree  is called {\em bushy} 
(in ) 
if every vertex of  is bushy
(with respect to  in ).  An example of a tree that contains both bushy and non-bushy vertices is given in Figure \ref{fig:bushinessExample}.
\begin{figure}
\begin{center}
\scalebox{1}{\includegraphics{bushinessExample}}
\caption{A tree  and a subtree , induced by the darker grey vertices.  The vertex  is not bushy with respect to  in  because it has a neighbour (above it in the diagram) that is not in  and is not a leaf of .  Vertex  is bushy with respect to  in  because all of its neighbours that are not in  are leaves of . }
\label{fig:bushinessExample}
\end{center}
\end{figure}
Now we can define the graph classes that will be examined in Section \ref{equiv}.
Let  be a graph and  be a set of trees. 
\begin{itemize}
\item
 is an {\em -covered subtree overlap graph} if it has a subtree overlap representation that has a covering subtree isomorphic to a tree in .
Such a representation is an {\em -covered subtree overlap representation} of . 
\item
 is a {\em bushy -covered subtree overlap graph} if it has a subtree overlap representation with a bushy covering subtree isomorphic to a tree in .
Such a representation is a {\em bushy -covered subtree overlap representation} of . 
\item
 is an {\em -subtree-filament graph} if there is a subtree filament representation of  such that the host tree is isomorphic to a member of . Such a representation is an {\em -subtree-filament representation} of .
\item
 is an {\em -cochordal graph} if it 
has a subtree disjointness representation such that the host tree is isomorphic to a member of .
Such a representation is an {\em -cochordal representation} of . 
\end{itemize}
When  has just one element, say , we sometimes write  instead of  in the above notation.


\section{Subtree representations}


In this section, we give methods for transforming a given multiset of subtrees of a tree into another representation of the same type 
(i.e., intersection, overlap, disjointness, containment) 
for the same graph.
Let  be a graph and let . For any given , 
the {\em subdivision of edge  (with vertex )} is the operation of removing the edge  from  and adding the vertex  and the edges  and . The vertex  is called a {\em subdivision vertex}. A graph  is a \emph{subdivision} of  if  can be obtained from  by zero or more edge subdivisions.  

We first mention some simple alterations that can be made to a given tree  and multiset  of subtrees of  without changing the relationships among the elements of .
First, if a new leaf is added to  but to no element of , then the subtrees are unchanged and therefore the relationships among them remain the same. Second, if an edge is subdivided in  and in every element of  that contains the edge, then the relationships among the subtrees remain unchanged.
Therefore, if  and  are trees such that  can be obtained from  by a sequence of leaf additions and edge subdivisions, then 
any graph that is the intersection, overlap, containment, or disjointness graph of subtrees of  is also the 
intersection, overlap, containment, or disjointness (respectively) graph of subtrees of .
Furthermore,
the -subtree-filament graphs form a subset of the -subtree filament graphs, and the -covered subtree overlap graphs form a subset of the -covered subtree overlap graphs.


Recalling that we denote subtrees as subsets of vertices of a tree, each of which induces a subtree:
\begin{lemma} \label{lem:subdiv}
Let  be a tree and  be a multiset of subtrees of . 
Let  be a vertex of  and let  be a neighbour of  in .
Let  be the tree obtained from  by 
subdividing the edge  with a vertex .
Let  be a (possibly empty) subset of , where each element of 
 contains .  
For all , let

Then
 is a multiset of subtrees of  and,
for all , .
\end{lemma}

\begin{proof}
The elements of  induce connected subgraphs of  by the construction.
Note that, for all , 
 only if , and
 if and only if .
Let  where .
We will prove that  by showing that
both of the following hold:
 if and only if ;  ( or ) if and only if ( or ). There are three cases to be considered, based on whether  is in neither, one, or both of  and .

If  is in neither  nor  then  and  and the result clearly follows.

If  is in both  and  then  is in both  and , and therefore  and  are both nonempty.
Furthermore,  and, similarly, . So the result follows.

If  is in just one of  and , 
then suppose without loss of generality that  is in  and not in . 
Then , , and . 
So 
.
Furthermore,
,
and .
If  then  and we have 
( or ) if and only if ( or ),
so the result follows.
In the remainder of the proof, we handle the case where .
Since  we also have  and, from before, .
Therefore, to complete the proof, we must show that .
Suppose for contradiction that , that is, , which implies that  since .
Since  is in , one of the following must hold: 
(i) ,
(ii) , 
or 
(iii) there is an element  such that .
But then, since , one of the following must also hold (respectively):
(i) , 
(ii)  is an element of  such that , 
or
(iii)  is an element of  such that .
In any case, this implies that , a contradiction.
Therefore  as required.

In each case,  and therefore the proof is complete.
\end{proof}

In order to define a subtree filament representation on subtrees  of a tree , it is convenient to make some assumptions about the elements of . Since we are concerned with the host trees of representations, we need to consider the effect on a host tree of enforcing those assumptions. This is the subject of the next definition and lemma, which will be used in the proof of Theorem \ref{bigThm}.

\begin{property}\label{A}
Subtrees
 of tree  are said to satisfy the {\em nontrivial intersection distinct leaf property} if:
\begin{itemize}
\item
each element of  is nontrivial, 
\item
every pair of elements of  are either disjoint or share two or more vertices, and 
\item
no vertex of  is a leaf of two distinct members of .
\end{itemize}
\end{property}


Note that the third requirement in the above property guarantees that the elements of  are distinct.

The next lemma shows that every multiset of subtrees of a tree can be transformed into subtrees of a tree that satisfy Property \ref{A}, without altering the relationships among the subtrees.
Because we are concerned with specific host trees, we consider the effect of the transformation on the host tree.


\begin{lemma} \label{lem:bushyGivesNice}
Let  be a nontrivial tree and  be a multiset of subtrees of .
There exists a tree  and set  of subtrees of  such that
\begin{itemize}
\item
for all , ,
\item
 is a subdivision of ,
and
\item
 satisfies Property \ref{A}.
\end{itemize}
\end{lemma}

\begin{proof}
Let  be a nontrivial tree and  be a multiset of subtrees of .
We may assume that no leaf of  is contained in any element of . Otherwise, for each leaf  of  that is contained in an element of , we could add a new leaf to  adjacent to . Then  would be isomorphic to a subdivision of the original tree and would satisfy the assumption. 
We show how to transform  and  into  and , respectively,  such that the conditions of the lemma are satisfied.


First, 
we perform  applications of the transformation of Lemma \ref{lem:subdiv}, as follows.
The initial application is performed on  and  
with  and  being the endpoints of an edge of  and  being the set of all elements of  that contain . By Lemma \ref{lem:subdiv}, this transformation
results in a tree and a multiset of subtrees of the tree. Each subsequent application is performed on the tree and subtrees resulting from the previous step, and again produces a tree and a multiset of subtrees. In the following description, we refer to the current tree and subtrees as  and  respectively.
Overall,
for each edge  of the initial tree , we apply the transformation of Lemma \ref{lem:subdiv} twice, once with , , and  being all elements of  that contain ;
then with ,  where  is the subdivision vertex from the previous step, and
 being all elements of  that contain .
By Lemma \ref{lem:subdiv}, this process finally results in a tree  and
subtrees  of , with
 for all .
Clearly,  is a subdivision of .



Let  be a trivial element of  and let  be the single vertex of . Since  is nontrivial,  is incident with an edge of , and the new vertex introduced in the subdivision of that edge with  is added to  in the construction of .
Therefore each element of  is nontrivial.

Suppose that distinct subtrees  intersect in just one vertex, . 
Since  is nontrivial,  is incident with an edge of . The new vertex that subdivides that edge is in both  and . Therefore every pair of elements of  are either disjoint or share an edge.

Since no leaf of  is contained in an element of ,
the leaves of members of 
are all vertices of  that are not in .
Therefore, every vertex that is a leaf of a subtree of 
 has degree two in .
Furthermore, if two elements of  share a leaf, say , then they both contain the neighbour of  that played the role of  during the subdivision when  was introduced, and not the other neighbour as that would contradict  being a leaf of both subtrees.

To complete the proof, we show how to 
reduce the number of vertices of  that are leaves of two or more distinct elements of . Applied iteratively, this leads to a representation that satisfies all conditions of the lemma.
Let  be a node in 
and let  be the elements of  that contain  as a leaf.
Let  and suppose that .
By the observation of the preceding paragraph,  has degree two in 
and every element of  also contains one of 's neighbours and not the other.
Let  and  be the neighbours of , such that every element of  contains  and not .
Let the elements of  be
sorted by nondecreasing size so that each element of  has a position from  to  in the sorted list.

We now use  new vertices, , which are not vertices of , as subdivision vertices in  applications of Lemma \ref{lem:subdiv}.
The first application is performed on  and , and produces a tree and subtrees of a tree. Each subsequent application is performed on the tree and subtrees resulting from the previous step. We will refer to the current tree and subtrees as  and , respectively, and the current subtrees corresponding to those of  as .
First, apply the transformation of Lemma \ref{lem:subdiv} to  and  with , , and  consisting of the single element of  having position  in the sorted list, using the vertex  as the subdivision vertex.
Then, for each  from  down to 1, apply the transformation of Lemma \ref{lem:subdiv} to  and  with  and  (the subdivision vertex from the previous step), with  being the set of all elements of  with corresponding elements in the sorted list of elements of  having positions greater than or equal to , and using subdivision vertex .
By Lemma \ref{lem:subdiv}, this process terminates with a tree  and subtrees  of .

By repeated application of Lemma \ref{lem:subdiv},
 is a multiset of subtrees of  and
 for all .
In addition,   is a subdivision of  and therefore of . Since  for all , each  has at least two vertices and each pair of intersecting subtrees of  shares an edge. No vertex has had its degree increased, and only degree two vertices have been added; therefore, every vertex of  that is a leaf of any subtree of  has degree two in . 
By the construction of , each element  of  contains all or none of  and , and therefore
 either contains all of the 's and , or none of the 's, and therefore does not contain any of the 's as a leaf.
Each  such that 
has exactly one of the 's as a leaf, namely,  where  is the position of  in the sorted list of the elements of . 
Therefore,  is not a leaf of any element of  and each new vertex  of  is a leaf of just one element of , namely, the subtree corresponding to the element of  in position  of the sorted list. 
Thus the number of vertices of  that are leaves of two or more distinct elements of  is less than than the number of vertices of  that are leaves of two or more distinct elements of . 
Applied iteratively, this method eventually produces subtrees  of a tree  that satisfy the lemma.
\end{proof}


\section{Equivalence of -covered subtree overlap graphs and -subtree-filament graphs}\label{equiv}


In this section, we show that for any tree , 
every -covered subtree overlap graph has a representation in which the host tree is just  with some additional leaves, and that -covered subtree overlap graphs are equivalent to the complements of -cochordal-mixed graphs. This equivalence does not extend to -subtree filament graphs since, for example, 
, the cycle on four vertices, is a -covered subtree overlap graph but not a -subtree-filament graph.
However, the equivalence does extend to subtree filament graphs when edge subdivision is allowed. In Theorem \ref{bigThm}
we show the equivalence of 
-covered subtree overlap graphs,
-subtree-filament graphs, and
complements of -cochordal-mixed graphs, 
when  is a nontrivial set of trees that is closed under edge subdivision.

\begin{theorem} \label{th:3parts}
Let  be a tree and let  be a graph. The following statements are equivalent:
\begin{enumerate}
\item  is an -covered subtree overlap graph. \label{3covered}
\item  is the complement of an -cochordal-mixed graph. \label{3mixed}
\item  is a bushy -covered subtree overlap graph. \label{3bushy}
\end{enumerate}
\end{theorem}
\begin{proof}

\textbf{\ref{3covered}  \ref{3mixed}}: 
Let  be a subtree overlap representation for  in tree  with covering subtree , and suppose that the elements of  are indexed such that  implies .
Then the sets  and 
 define a cochordal-mixed partition of the edges of .
 Since  is a covering subtree of , for all ,  is a subtree of  and  if and only if .
Therefore, subtrees  of  form an -cochordal representation of the graph .
 
\textbf{\ref{3mixed}  \ref{3bushy}}: 
This part of the proof combines elements of the proof that complements of cointerval-mixed graphs are interval filament graphs \cite{gavril2000} and the proof that subtree filament graphs are subtree overlap graphs \cite{Jess}. 
Let  be the complement of an -cochordal-mixed graph.
Let  and  be a partition of the edges of  
such that  is an -cochordal graph
and  is a transitive orientation of  such that
for all , if
 and  then . 
Let subtrees  of tree  be an -cochordal representation of , that is, for all ,  if and only if .
      
Suppose that 
. Then . 
Furthermore, if , then
replacing  with  produces another -cochordal representation of , as justified by the following argument from \cite{gavril2000}.
Suppose that .
By the definition of , for all , if  then . Equivalently, every  that intersects  also intersects  (as well as  since , , and  are all subtrees of a tree).
Therefore, replacing  with  produces another -cochordal representation of .
Applied repeatedly, this transformation results in an -cochordal representation of  such that for all ,  implies . 
      
Let  be the tree  with  additional nodes:  where, for each ,  is adjacent in  to exactly one arbitrary node of .
Then, for , let  and let . 
Each element of  induces a connected subgraph and therefore a subtree of .
Suppose there are two elements of ,  and  such that  and . Then  is in  and  is in  which implies that  and   are both in . But this contradicts transitivity since 
 is a simple graph. Therefore,
the elements of  are distinct.

For each , ,  is in exactly one of , , or .
Since  is a cochordal representation of ,  if and only if .
Of the nodes , only  (respectively ) and those corresponding to subtrees contained in or equal to  (respectively ) are in  (respectively ). Furthermore,  and . Therefore  if and only if   and so  if and only if .
If  then suppose without loss of generality that . Then, by our earlier argument, . In addition, every vertex of  is also in  by transitivity of  , and .
Therefore, 
if  then , and
if  then  or .
Finally, if  then,
since  is a cochordal representation of ,  and therefore 
. In addition,  and .
Thus,   implies . We conclude that the subtrees  of tree  form a subtree overlap representation of .

Furthermore, since for each ,  and each  is adjacent to a vertex of , subtrees  of tree  form an -covered subtree overlap representation of  in which  is bushy.

\textbf{\ref{3bushy}  \ref{3covered}}: Obvious.
\end{proof}

The classes of -covered subtree overlap graphs and bushy -covered subtree overlap graphs are equivalent to the class of cocomparability graphs. 
This follows from \cite{GolSch} combined with the observation that subtrees of a tree that all have a vertex in common overlap if and only if neither is contained in the other.
Thus, Theorem \ref{th:3parts}
generalises characterisations of cocomparability graphs as the overlap graphs of subtrees of a tree where all subtrees have a vertex in common, the complements of cochordal-mixed graphs where all edges are in the  block of the partition, and the overlap graphs of subtrees of a star \cite{eowynStewart, gavril2000, GolSch}.

\begin{theorem}\label{bigThm}
Let  be a graph and  be a nonempty set of trees that is closed under edge subdivision. The following statements are equivalent:
\begin{enumerate}
\item  is an -covered subtree overlap graph. \label{covered}
\item  is the complement of an -cochordal-mixed graph. \label{mixed}
\item  is a bushy -covered subtree overlap graph. \label{bushy}
\item  is an -subtree-filament graph. \label{filament}
\end{enumerate}
\end{theorem}
\begin{proof}

\textbf{\ref{covered}  \ref{mixed}  \ref{bushy}}: by Theorem \ref{th:3parts}.

\textbf{\ref{filament}  \ref{mixed}}: 
By Theorem 4 of \cite{gavril2000}, a graph is a subtree filament graph if and only if it is the complement of a cochordal-mixed graph. 
In the proof of that theorem, a subtree-filament representation of a graph  is transformed to a cochordal representation of the graph  on the same host tree, where  and  is a cochordal-mixed partition of the edges of . Thus \ref{filament} implies \ref{mixed}. 

For the other direction, suppose that  is the complement of an -cochordal-mixed graph where , and let  and  be a cochordal-mixed partition of the edges of . By Lemma \ref{lem:bushyGivesNice}, there is a -cochordal representation of  that satisfies Property \ref{A}, such that  is a subdivision of . The proof of Theorem 4 of \cite{gavril2000} transforms a -cochordal representation of  and cochordal-mixed partition  and  of the edges of  to a -subtree-filament representation of , provided that the 
-cochordal representation of  satisfies Property \ref{A}.
Therefore, since , we have \ref{mixed} implies \ref{filament}.
\end{proof}

Theorem \ref{bigThm} does not hold for  since only complete graphs are 
-subtree-filament graphs while, as previously noted, 
the classes of -covered subtree overlap graphs, 
-cochordal-mixed graphs, and
bushy -covered subtree overlap graphs
are all equivalent to cocomparability graphs.


When  is the set of subdivisions of , Theorem \ref{bigThm} becomes the following characterisation of interval filament graphs, which includes results of \cite{CGO} and \cite{gavril2000}.

\begin{corollary}
The following statements are equivalent for a graph :
 is a path-covered subtree overlap graph;
 is the complement of a cointerval-mixed graph;
 is the overlap graph of subtrees of a caterpillar;
 is an interval filament graph.
\end{corollary}


\section{Conclusion}

We have presented two main results: 

\begin{enumerate}
\item
The following graph classes are equivalent for any tree : -covered subtree overlap graphs, the complements of -cochordal-mixed graphs, and bushy -covered subtree overlap graphs.
\item
The following graph classes are equivalent for any nonempty set of trees  that is closed under edge subdivision:
-covered subtree overlap graphs,
the complements of -cochordal-mixed graphs,
bushy -covered subtree overlap graphs,
and -subtree-filament graphs.
\end{enumerate}

The first result is a generalization of characterisations of cocomparability graphs, as can be seen in the simplest case of Theorem \ref{th:3parts}, when . 
The second result generalises characterisations of interval filament graphs. The simplest case of Theorem \ref{bigThm}, when  is the set of subdivisions of , states that the following graph classes are equivalent: path-covered subtree overlap graphs, the complements of cointerval-mixed graphs, the overlap graphs of subtrees of caterpillars, and interval filament graphs.
The second result suggests that the -covered subtree overlap graphs, for sets  of trees closed under edge subdivision, might be a useful way of breaking down the class of subtree overlap graphs. We propose three avenues based on that idea for future study.


While some subclasses of subtree overlap graphs can be recognised in polynomial time (including interval, permutation, cocomparability, chordal, circular arc, and circle graphs), for others the recognition problem is NP-complete (including interval filament graphs \cite{Perm},  overlap graphs of subtrees of a tree with a bounded number of leaves, the overlap graphs of subtrees of subdivisions of a fixed tree with at least three leaves, and the overlap graphs of paths in a tree with bounded maximum degree \cite{JessThesis,PergelThesis}). The complexity of the recognition problem for subtree overlap graphs is open.  
An efficient recognition algorithm that could output subtree overlap representations for yes instances would have significant algorithmic implications since several optimisation problems that are NP-complete in general can be solved efficiently for subtree overlap graphs when a subtree overlap representation is given \cite{eowynStewart, gavril2007, gavril2009, gavril2011, Keil}. Does the recognition problem on -covered subtree overlap graphs give insight into the recognition problem on subtree overlap graphs as a whole?

Several optimisation problems remain NP-hard on subtree overlap graphs by virtue of hardness results on the subclasses. It would be interesting to explore the possible P vs. NP-complete boundaries for various optimisation problems within the 
-covered subtree overlap graphs
over sets  of trees that are closed under edge subdivision.

Parameters of chordal graphs based on their subtree intersection representations include leafage, that is,
the minimum number of leaves in the host tree of a representation \cite{LinMcKeeWest}, and vertex leafage, that is,
the minimum maximum number of leaves of a subtree in a representation \cite{ChaplickStacho}.
How do analogous and other parameters of subtree overlap graphs relate to the -covered subtree overlap graph classes of this paper?


\section{Acknowledgements}
The authors gratefully acknowledge support from an NSERC Discovery grant, an iCORE ICT Graduate Student Scholarship and a University of Alberta Dissertation Fellowship.

\bibliographystyle{plain}

\begin{thebibliography}{99}

\bibitem{BLS}
Andreas Brandst\"adt, Van Bang Le, Jeremy P. Spinrad,
Graph Classes: A Survey, SIAM Monographs on Discrete Mathematics and Applications, Volume 3, SIAM, 1999.

\bibitem{eowynStewart}
Eowyn \v{C}enek, Lorna Stewart, 
Maximum independent set and maximum clique
algorithms for overlap graphs, 
Discrete Applied Mathematics 131 (2003) 77-91.

\bibitem{CGO}
J\'er\'emie Chalopin, Daniel Gon\c{c}alves, and Pascal Ochem,
On graph classes defined by overlap and intersection models, Paper presented at
Sixth Czech-Slovak International Symposium on Combinatorics, Graph Theory, Algorithms and Applications, Prague, July 10-15, 2006.

\bibitem{ChaplickStacho}
Steven Chaplick, Juraj Stacho,
The vertex leafage of chordal graphs,
Discrete Applied Mathematics 168 (2014) 14-25.

\bibitem{JessThesis}
Jessica Enright, 
Results on Set Representations of Graphs, University of Alberta PhD Thesis, 2011.

\bibitem{Jess}
Jessica Enright, Lorna Stewart, 
Subtree filament graphs are subtree overlap graphs, 
Information Processing Letters 104 (2007) 228-232.

\bibitem{gavril1973}
Fanica Gavril, 
Algorithms for a maximum clique and a maximum independent set of a circle graph, 
Networks 3 (1973) 261-273.

\bibitem{gavril2000}
Fanica Gavril, 
Maximum weight independent sets and cliques in
intersection graphs of filaments, 
Information Processing Letters 73 (2000) 181-188.

\bibitem{gavril2007}
Fanica Gavril, 
3D-interval-filament graphs,
Discrete Applied Mathematics 155(18) (2007) 2625-2636.

\bibitem{gavril2009}
Fanica Gavril, 
Algorithms on subtree filament graphs,
in Marina Lipshteyn, Vadim Efimovich Levit, Ross M. McConnell (Eds.): Graph theory, computational intelligence and thought: essays dedicated to Martin Charles Golumbic on the occasion of his 60th Birthday, LNCS 5420 (2009) 27-35.

\bibitem{gavril2011}
Fanica Gavril, 
Algorithms for induced biclique optimization problems, 
Information Processing Letters 111(10) (2011) 469-473.

\bibitem{GolSch}
Martin C. Golumbic, Edward R. Scheinerman, Containment graphs, posets,
and related classes of graphs, in Combinatorial Mathematics:
Proceedings of the Third International Conference, 
Annals of the New York Academy of Sciences 555 (1989) 192-204.

\bibitem{Keil}
J. Mark Keil, Lorna Stewart, Approximating the minimum clique cover
and other hard problems in subtree filament graphs, Discrete Applied
Mathematics 154 (2006) 1983-1995.

\bibitem{LinMcKeeWest}
In-Jen Lin, Terry A. McKee, Douglas B. West,
The leafage of a chordal graph,
Discussiones Mathematicae Graph Theory 18 (1998) 23-48.

\bibitem{Marcz}
Edward Marczewski,
Sur deux propri\'et\'es des classes d'ensembles, Fundamenta Mathematicae 33 (1945) 303-307.

\bibitem{Perm}
Martin Pergel, Recognition of polygon-circle graphs and graphs of interval filaments is NP-complete, in Andreas Brandst\"adt, Dieter Kratsch, and Haiko M\"uller (Eds.): WG 2007, LNCS 4769 (2007) 238-247.

\bibitem{PergelThesis}
Martin Pergel, 
Special Graph Classes and Algorithms on Them, Charles University in Prague PhD Thesis, 2008.

\end{thebibliography}

\end{document}