





\documentclass[9pt, preprint]{sigplanconf}





\usepackage{amsmath,proof}

\usepackage{etex}
\usepackage{amssymb}
\usepackage{amsmath}
\usepackage{amsthm}
\usepackage{mathtools}
\usepackage{stmaryrd}
\usepackage{color}
\usepackage{comment}



\theoremstyle{theorem}
\newtheorem{theorem}{Theorem}[section]
\newtheorem{proposition}[theorem]{Proposition}
\newtheorem{lemma}[theorem]{Lemma}
\newtheorem{corollary}[theorem]{Corollary}
\theoremstyle{definition}
\newtheorem{definition}[theorem]{Definition}
\newtheorem{example}[theorem]{Example}
\newtheorem{problem}[theorem]{Problem}
\newtheorem{remark}[theorem]{Remark}
\newtheorem{note}[theorem]{Note}
\newtheorem{notation}[theorem]{Notation}

\newcommand{\memo}[1]{\textcolor{red}{(#1)}}
\newcommand{\cont}{\memo{cont.}}
\newcommand{\stable}{\textcolor{blue}{(stable)\ }}
\newcommand{\notyet}{\textcolor{red}{(not yet)\ }}


\newcommand{\bbN}{\mathbb{N}}
\newcommand{\bbZ}{\mathbb{Z}}
\newcommand{\bbQ}{\mathbb{Q}}
\newcommand{\bbR}{\mathbb{R}}
\newcommand{\bbC}{\mathbb{C}}
\newcommand{\bbS}{\mathbb{S}}
\newcommand{\bbone}{\mathbf{1}}

\newcommand{\bfC}{\mathbf{C}}
\newcommand{\bfD}{\mathbf{D}}
\newcommand{\bfE}{\mathbf{E}}


\newcommand{\calA}{\mathcal{A}}
\newcommand{\calC}{\mathcal{C}}
\newcommand{\calD}{\mathcal{D}}
\newcommand{\calF}{\mathcal{F}}
\newcommand{\calL}{\mathcal{L}}
\newcommand{\calO}{\mathcal{O}}
\newcommand{\calS}{\mathcal{S}}
\newcommand{\calR}{\mathcal{R}}

\newcommand{\cat}[1]{\mathcal{#1}}
\newcommand{\yoneda}{\mathbf{y}}

\renewcommand{\restriction}{\!\upharpoonright\!}
\renewcommand{\vec}{\overrightarrow}
\renewcommand{\bar}{\overbar}

\newcommand{\closure}{\overbar}


\newcommand{\overbar}[1]{\mkern 1.5mu\overline{\mkern-1.5mu#1\mkern-1.5mu}\mkern 1.5mu}

\renewcommand{\phi}{\varphi}

\newcommand{\xxtos}[1]{\stackrel{#1}{\to}}
\newcommand{\xxto}{\xrightarrow}
\newcommand{\xxot}{\xleftarrow}

\newcommand{\xxinto}{\xhookedrightarrow}
\newcommand{\xxonto}{\xtwoheadrightarrow}

\newcommand{\tto}{\rightrightarrows}
\newcommand{\ot}{\leftarrow}
\newcommand{\longot}{\longleftarrow}
\newcommand{\into}{\rightarrowtail}
\newcommand{\onto}{\twoheadrightarrow}

\newcommand{\lang}[1]{\mathcal{L}(#1)}

\newcommand{\pow}{\mathcal{P}}
\newcommand{\dist}{\mathcal{D}_{=1}}
\newcommand{\sdist}{\mathcal{D}_{\le 1}}

\DeclareMathOperator{\Inf}{Inf}
\DeclareMathOperator{\cod}{cod}
\DeclareMathOperator{\dom}{dom}

\newcommand{\colim}{\varinjlim}

\newcommand{\set}[2]{\left\{\, #1 \mathrel{}\middle|\mathrel{} #2 \,\right\}}
\newcommand{\sett}[1]{\left\{ #1 \right\}}

\newcommand{\place}{{-}}

\newcommand{\op}{\mathrm{op}}
\newcommand{\ev}{\mathrm{ev}}
\newcommand{\id}{\mathrm{id}}

\newcommand{\uintv}{[0, 1]}

\newcommand{\Set}{\mathbf{Set}}
\newcommand{\Sets}{\Set}
\newcommand{\Pos}{\mathbf{Pos}}
\newcommand{\Rel}{\mathbf{Rel}}
\newcommand{\Top}{\mathbf{Top}}
\newcommand{\CL}{\mathbf{CL}}
\newcommand{\GEMod}{\mathbf{GEMod}}
\newcommand{\DcEMod}{\mathbf{DcEMod}}
\newcommand{\Conv}{\mathbf{Conv}}

\newcommand{\EM}{\mathcal{E}{\kern-.5ex}\mathcal{M}}
\newcommand{\Kl}{\mathcal{K}{\kern-.2ex}\ell}
\newcommand{\Mod}{\mathbf{Mod}}

\newcommand{\biglor}{\bigvee}
\newcommand{\bigland}{\bigwedge}

\DeclarePairedDelimiter\bra{\langle}{\rvert}
\DeclarePairedDelimiter\ket{\lvert}{\rangle}
\DeclarePairedDelimiter\abs{\lvert}{\rvert}
\DeclarePairedDelimiter\p{(}{)}
\DeclarePairedDelimiter\pair{\langle}{\rangle}



\newcommand{\OmegaC}{\Omega_{\cat{C}}}
\newcommand{\OmegaD}{\Omega_{\cat{D}}}

\newcommand{\pho}{\rho}

\newcommand{\EMD}{\EM_{\!\cat{D}}}


\newcommand{\setcompr}{\set}
\newcommand{\List}{\mathsf{List}}



\newcommand{\Upx}{\mathsf{Up}}

\newcommand{\lt}{<}
\newcommand{\gt}{>}

\renewcommand{\subset}{\subseteq}

\newcommand{\StrCL}{\mathrm{\textbf{StrCL}}_{\bigland,\mathrm{pos}}} 

\newif\ifignore \ignorefalse
\newcommand{\auxproof}[1]{
\ifignore\mbox{}\newline
\textbf{BEGIN: AUX-PROOF} \dotfill\newline
{#1}\mbox{}\newline
\textbf{END: AUX-PROOF}\dotfill\newline
\fi}
\def\comment#1{\ifignore \marginpar[\renewcommand{\baselinestretch}{0.9}\raggedleft\sloppy{}#1]{\renewcommand{\baselinestretch}{0.9}\raggedright\sloppy{}#1}\fi}





\usepackage[all,dvips]{xy}


\xyoption{v2}
\xyoption{curve}
\xyoption{2cell}
\SelectTips{cm}{}  \UseAllTwocells
\SilentMatrices
\def\labelstyle{\scriptstyle}
\def\twocellstyle{\scriptstyle}
\newdir{ >}{{}*!/-8pt/@{>}}
\newdir{|>}{!/1.6pt/@{|}*:(1,-.2)@^{>}*:(1,+.2)@_{>}}
\newdir{pb}{:(1,-1)@^{|-}}
\def\pb#1{\save[]+<20 pt,0 pt>:a(#1)\ar@{pb{}}[]\restore}
\newcommand{\shifted}[3]{\save[]!<#1,#2>*{#3}\restore}
\usepackage{wrapfig}
\setlength{\intextsep}{.1\intextsep}
\setlength{\columnsep}{.7\columnsep}
\newenvironment{myproof}{\begin{proof}}{\end{proof}\ignorespacesafterend}
\def\myqedsymbol{\squareforqed}
\def\myqedbox{\squareforqed}
\def\myqed{\qed}

\newcommand{\after}{\mathrel{\circ}}
\newcommand{\co}{\mathrel{\circ}}
\newcommand{\vafter}{\mathrel{\bullet}}
\newcommand{\vco}{\mathbin{\bullet}}
\newcommand{\Id}{\mathrm{Id}}
\newcommand{\Vect}{\mathbf{Vect}}
\newcommand{\CAT}{\mathbf{CAT}}
\newcommand{\Alg}{\mathbf{Alg}}
\newcommand{\Coalg}[1]{\mathbf{Coalg}({#1})}
\newcommand{\Mon}{\mathbf{Mon}}
\newcommand{\Preord}{\mathbf{Preord}}
\newcommand{\onecat}{\mathbf{1}}
\newcommand{\Kleisli}[1]{\mathcal{K}{\kern-.2ex}\ell(#1)}
\newcommand{\twoheadlongrightarrow}{\relbar\joinrel\twoheadrightarrow}
\newcommand{\Scottint}[1]{\llbracket #1 \rrbracket}
\newcommand{\sem}[1]{\llbracket #1 \rrbracket}
\newcommand{\csem}[1]{[#1]}
\newcommand{\insem}[1]{[#1]}
\newcommand{\ssp}[1]{\llparenthesis #1\rrparenthesis} \newcommand{\A}{\mathbb{A}}
\newcommand{\B}{\mathbb{B}}
\newcommand{\C}{\mathbb{C}}
\newcommand{\E}{\mathbb{E}}
\newcommand{\bbP}{\mathbb{P}}
\newcommand{\F}{\mathbb{F}}
\newcommand{\I}{\mathbb{I}}
\newcommand{\J}{\mathbb{J}}
\newcommand{\D}{\mathbb{D}}
\newcommand{\K}{\mathbb{K}}
\newcommand{\N}{\mathbf{N}}
\newcommand{\Q}{\mathcal{Q}}
\newcommand{\Fin}{\mathbf{F}}
\newcommand{\Inj}{\mathbf{I}}
\newcommand{\Lt}{\mathbb{L}}
\newcommand{\LawTh}{\mathbb{L}}
\newcommand{\hLt}{\widehat{\mathbb{L}}}
\newcommand{\f}{\mathsf{f}}
\newcommand{\g}{\mathsf{g}}
\newcommand{\m}{\mathsf{m}}
\newcommand{\monoidalCat}{\C}
\newcommand{\monoidObj}{X}
\newcommand{\Nat}{\mathbf{Nat}}
\newcommand{\nat}{\mathbb{N}}
\newcommand{\epi}{\twoheadrightarrow}
\newcommand{\inj}{\hookrightarrow}
\newcommand{\tuple}[1]{\langle#1\rangle}
\newcommand{\Lobj}{\mathbb{L}\text{-}}
\newcommand{\LCAT}{\mathbb{L}\text{-}}
\DeclareMathOperator{\tr}{tr}
\newcommand{\fpow}{\mathcal{P}_{\omega}}
\newcommand{\multi}{\mathcal M}
\newcommand{\lift}{\mathcal{L}}
\newcommand{\sbullet}{\mbox{}}  \newcommand{\iso}{\mathrel{\stackrel{
\raisebox{.5ex}{}}{
\raisebox{0ex}[0ex][0ex]{}}}}
\newcommand{\isotwocl}{\mathrel{\stackrel{
\raisebox{.5ex}{}}{
\raisebox{0ex}[0ex][0ex]{}}}}
\newcommand{\botimes}{\mathbin{\pmb{\otimes}}}
\newcommand{\btimes}{\mathbin{\pmb{\times}}}
\newcommand{\twocl}{\Rightarrow}
\newcommand{\To}{\Rightarrow}
\newcommand{\longto}{\longrightarrow}
\newcommand{\oF}{\overline{F}}
\newcommand{\innerMult}{\mu}
\newcommand{\innerUnit}{\eta}
\newcommand{\LBicat}{\Lt_{\mathrm{C}}}
\newcommand{\LPremon}{\Lt_{\mathrm{PMC}}}
\newcommand{\LMon}{\Lt_{\mathrm{MC}}}
\DeclareMathOperator{\sync}{\mathsf{sync}}
\DeclareMathOperator{\beh}{\mathsf{beh}}
\DeclareMathOperator{\st}{\mathsf{st}}
\DeclareMathOperator{\dst}{\mathsf{dst}}
\newcommand{\textlb}{
\def\labelstyle{\textstyle}
\def\twocellstyle{\textstyle}}
\newcommand{\scriptlb}{
\def\labelstyle{\scriptstyle}
\def\twocellstyle{\scriptstyle}}
\def\compsign{\mathrel>\kern-2pt\joinrel>\kern-2pt\joinrel>}
\newcommand{\comp}{\ensuremath{\mathbin{\compsign}}\xspace}
\newcommand{\arr}{\ensuremath{\mathsf{arr}}\xspace}
\newcommand{\first}{\ensuremath{\mathsf{first}}\xspace}
\newcommand{\second}{\ensuremath{\mathsf{second}}\xspace}
\newcommand{\seq}[2]{{#1}_{1},\dotsc,{#1}_{#2}}
\newcommand{\bmd}{\mathbb{B}}
\newcommand{\md}{\mathbb{M}}
\newcommand{\Colim}{\mathop{\mathrm{Colim}}}
\newcommand{\Lim}{\mathop{\mathrm{Lim}}}
\newcommand{\Lan}{\mathop{\mathrm{Lan}}}
\newcommand{\ind}[1]{\mathsf{i}_{#1}}
\newcommand{\no}[1]{|#1|}
\newcommand{\x}{\mathtt{x}}
\newcommand{\y}{\mathtt{y}}
\newcommand{\Fp}{F_{\bullet}}
\newcommand{\Coalgp}[1]{\mathbf{Coalg}_{\bullet}({#1})}
\newcommand{\zetap}{\zeta_{\bullet}}
\newcommand{\kstar}{(\place)^{*}}
\newcommand{\defiff}{\stackrel{\text{def.}}{\Longleftrightarrow}}
\newcommand{\fib}[3]{\raisebox{.00in}
{\mbox{\large }}}
\newcommand{\vcoalg}[3]{\raisebox{.00in}
{\mbox{\large }}}
\newcommand{\valg}[3]{\raisebox{.00in}
{\mbox{\large }}}
\newcommand{\coalg}[3]{#1\stackrel{#2}{\to}#3}
\newcommand{\Sub}{\mathrm{Sub}}
\newcommand{\RegSub}{\mathrm{RegSub}}
\newcommand{\fibp}{\fib{\bbP}{p}{\C}}
\newcommand{\subfib}[1]{\fib{\Sub(#1)}{}{#1}}
\newcommand{\Fam}{\mathrm{Fam}}
\newcommand{\famfib}[1]{\fib{\Fam(#1)}{}{\Sets}}
\newcommand{\Predfib}{\fib{\Pred}{}{\Sets}}
\newcommand{\Relfib}{\fib{\Rel}{}{\Sets}}
\newcommand{\presheaf}[1]{\widehat{#1}}
\newcommand{\commacat}[2]{(#1\!\downarrow\! #2)}
\newcommand{\Posets}{\mathbf{Poset}}
\newcommand{\FP}{\mathrm{FP}}
\newcommand{\Fml}{\mathsf{Fml}}
\newcommand{\AtProp}{\mathsf{AP}}


\newcommand{\seqComp}{\mbox{}}
\newcommand{\embFunc}[1]{\lceil{#1}\rceil}

\newcommand{\rrep}{(!)}
\newcommand{\rseqR}{(;\textsc{R})}
\newcommand{\rseqL}{(;\textsc{L})}
\newcommand{\rkstar}{(\kstar)}
\newcommand{\rparL}{(\mbox{}\textsc{L})}
\newcommand{\rparR}{(\mbox{}\textsc{R})}
\newcommand{\rparSync}{(\mbox{}\textsc{Sync})}
\newcommand{\npar}{*+[F-]{\parallel}}
\newcommand{\nrep}{*+[F-]{!}}
\newcommand{\nseq}{*+[F-]{;}}
\newcommand{\nx}{{x}}
\newcommand{\nxone}{{x_{1}}}
\newcommand{\nxtwo}{{x_{2}}}
\newcommand{\nkstar}{*+[F-]{\kstar}}
\newcommand{\dar}{\ar@{..>}}
\newcommand{\lar}{\ar@{-}}
\newcommand{\Cstar}{\mathrm{C}^{*}}
\newcommand{\Int}{\mathrm{Int}}
\newcommand{\Ass}{\mathrm{Ass}} \newcommand{\PER}{\mathbf{PER}}
\newcommand{\Kco}{\mathbin{\odot}} \newcommand{\cmbt}[1]{\mathsf{#1}} \newcommand{\limp}{\multimap} \newcommand{\bang}{\mathop{!}\nolimits} \newcommand{\relto}{\mathrel{\ooalign{\hfil\raisebox{.3pt}{}\hfil\crcr}}}
\newcommand{\longrelto}{\mathrel{\ooalign{\hfil\raisebox{.3pt}{}\hfil\crcr}}}
\newcommand{\dirsum}{\prod} \newcommand{\bc}{\mathrm{bc}}
\newcommand{\punit}{\mathrm{I}} \newcommand{\ptensor}{\boxtimes} \newcommand{\qll}{\mathbf{q}\lambda_{\ell}}
\newcommand{\synt}[1]{\mathtt{#1}} \newcommand{\qbit}{\synt{qbit}}
\newcommand{\nqbit}{n\text{-}\synt{qbit}}
\newcommand{\npoqbit}{(n+1)\text{-}\synt{qbit}}
\newcommand{\mqbit}{m\text{-}\synt{qbit}}
\newcommand{\kqbit}{k\text{-}\synt{qbit}}
\newcommand{\Nqbit}{N\text{-}\synt{qbit}}
\newcommand{\zeroqbit}{0\text{-}\synt{qbit}}
\newcommand{\bit}{\synt{bit}}
\newcommand{\letcl}[2]{\synt{let}\,{#1}\,\synt{in}\,{#2}}
\newcommand{\meas}{\synt{meas}}
\newcommand{\new}{\synt{new}}
\newcommand{\injl}{\synt{inj}_{\ell}}
\newcommand{\injr}{\synt{inj}_{r}}
\newcommand{\matchcl}[2]{\synt{match}\,{#1}\,\synt{with}\,{#2}}
\newcommand{\letreccl}[2]{\synt{letrec}\,{#1}\,\synt{in}\,{#2}}
\newcommand{\letrecclCtr}[3]{\synt{letrec}^{#1}\,{#2}\,\synt{in}\,{#3}}
\newcommand{\cmp}{\synt{cmp}}
\newcommand{\subtp}{\mathrel{<:}}
\newcommand{\ttrue}{\mathtt{t{\kern-1.5pt}t}}
\newcommand{\ffalse}{\mathtt{f{\kern-1.5pt}f}}
\newcommand{\der}{\mathrm{der}}
\newcommand{\weak}{\mathrm{weak}}
\newcommand{\con}{\mathrm{con}}
\newcommand{\mult}{\mathrm{mult}}
\newcommand{\str}{\mathrm{str}}
\newcommand{\dtimes}{\mathbin{\dot{\mbox{}}}}
\newcommand{\fix}{\mathrm{fix}}
\newcommand{\Val}{\mathrm{Val}}
\newcommand{\Norm}{\mathcal{N}}
\newcommand{\den}{\curlyveedownarrow}
\newcommand{\test}{\mathrm{test}}
\newcommand{\prb}{\mathrm{prob}}
\newcommand{\abort}{\synt{abort}}
\newcommand{\tree}{\mathrm{tree}}
\newcommand{\UpP}{\mathsf{Up}\pow}
\newcommand{\ttwo}[1]{2^{2^{#1}}}
\newcommand{\ssub}{\mathrel{\subset{\kern-1.6ex}\subset}}
\newcommand{\rel}{\mathrm{Rel}}
\newcommand{\Attr}{\mathrm{Attr}}
\newcommand{\Cppo}{\mathbf{Cppo}}
\newcommand{\wpre}{\mathop{\mathrm{wp}}\nolimits}
\newcommand{\spost}{\mathop{\mathrm{spost}}}
\newcommand{\EMod}{\mathbf{EMod}}
\newcommand{\Spec}{\mathbb{S}}
\newcommand{\Logic}{\mathsf{L}}
\newcommand{\tauTotal}{\tau_{\mathrm{total}}}
\newcommand{\tauPartial}{\tau_{\mathrm{partial}}}
\newcommand{\tauD}{\tau_{\Diamond}}
\newcommand{\tauB}{\tau_{\Box}}
\newcommand{\PredKl}[1]{\bbP^{\mathcal{K}{\kern-.2ex}\ell}(#1)}
\newcommand{\PredEM}[1]{\bbP^{\mathcal{E}{\kern-.5ex}\mathcal{M}}(#1)}
\newcommand{\Upcl}{\mathsf{Up}}
\newcommand{\Dw}{\mathsf{Dw}}
\newcommand{\Cv}{\mathsf{Cv}}
\newcommand{\Player}{\mathsf{P}}
\newcommand{\Opponent}{\mathsf{O}}
\newcommand{\powP}{\pow_{\Player}}
\newcommand{\powO}{\pow_{\Opponent}}
\newcommand{\upcl}{\mathop{\uparrow}\nolimits}
\newcommand{\dwcl}{\mathop{\downarrow}\nolimits}
\newcommand{\UP}{\mathcal{U{\kern-.3ex}P}}
\newcommand{\CD}{\mathcal{C{\kern-.3ex}D}}
\newcommand{\R}{R}
\newcommand{\rhoP}{\rho_{\Player}}
\newcommand{\rhoO}{\rho_{\Opponent}}
\newcommand{\Meas}{\mathbf{Meas}}
\newcommand{\V}{\mathcal{V}}


\newcommand{\Pred}{\mathbf{Pred}}
\newcommand{\omegaSets}{\mathbf{\omega}\text{-}\mathbf{Sets}}
\newcommand{\Eff}{\mathbf{Eff}}
\newcommand{\Rnu}{\mathsf{R}\nu}
\newcommand{\RG}{\mathsf{R}_{\mathsf{G}}}
\newcommand{\RC}{\mathsf{Cv}}
\newcommand{\rhoInf}{\rho_{\mathrm{inf}}}
\newcommand{\rhoSup}{\rho_{\mathrm{sup}}}
\newcommand{\dimg}[1]{\mathop{\amalg}\nolimits_{#1}}

\DeclareMathOperator{\Image}{Im}
\DeclareMathOperator{\Quot}{Quot}
 
\begin{document}

\special{papersize=8.5in,11in}
\setlength{\pdfpageheight}{\paperheight}
\setlength{\pdfpagewidth}{\paperwidth}

\conferenceinfo{CONF 'yy}{Month d--d, 20yy, City, ST, Country}
\copyrightyear{20yy}
\copyrightdata{978-1-nnnn-nnnn-n/yy/mm}
\doi{nnnnnnn.nnnnnnn}







\titlebanner{}        \preprintfooter{}   

\title{Healthiness from Duality}


\authorinfo{Wataru Hino \and Hiroki Kobayashi \and\\ Ichiro Hasuo}
           {University of Tokyo, Japan}
           {\{wataru, hkoba7de, ichiro\}@is.s.u-tokyo.ac.jp}
\authorinfo{Bart Jacobs}
           {Radboud University Nijmegen, the Netherlands}
           {bart@cs.ru.nl}

\maketitle

\begin{abstract}
  \emph{Healthiness} is a good old question in program logics that
  dates back to Dijkstra. It asks for an intrinsic characterization of
  those predicate transformers which arise as the (backward)
  interpretation of a certain class of programs. There are several
  results known for healthiness conditions: for deterministic
  programs, nondeterministic ones, probabilistic ones, etc.  Building
  upon our previous works on so-called \emph{state-and-effect
    triangles}, we contribute a unified categorical framework for
  investigating healthiness conditions. This framework is based on a
  \emph{dual adjunction} induced by a dualizing object and on our
  notion of \emph{relative Eilenberg-Moore algebra}.  The latter
  notion seems interesting in its own right in the context of monads,
  Lawvere theories and enriched categories.
\end{abstract}

\category{F.3.2}{Semantics of Programming Languages}{Algebraic Approaches to Semantics}




\keywords
program logic, category theory, duality




\section{Introduction}\label{sec:intro}


\paragraph{Predicate Transformer Semantics of Computation}
\emph{Program logics} are formal systems for reasoning about
programs. They come in different styles:
in the \emph{Floyd-Hoare logic}~\cite{Hoare69} one derives 
 triples
of a
precondition, a program and a postcondition;  \emph{dynamic
logics}~\cite{HarelTK00}
are  logics that have programs as modal operators;
type-theoretic presentations would have predicates as \emph{refinement}
(or \emph{dependent})
\emph{types}, allowing smooth extension to higher-order programs; and many
 program verification tools for imperative programs have
programs represented as \emph{control flow graphs}, where predicates
are labels to the edges. Whatever presentation style is taken, the
basic idea that underlies these variations of program logics is that of
\emph{weakest precondition}, dating back to
Dijkstra~\cite{Dijkstra76}. It asks: \emph{in order to guarantee a given
postcondition after the execution of a given program, what
precondition does it suffice to assume, before the execution?}

Through weakest preconditions a program gives rise to a \emph{(backward)
predicate transformer} that carries a given postcondition to the
corresponding weakest precondition. This way of interpreting
programs---sometimes called \emph{axiomatics
semantics}~\cite{Winskel93}---is in contrast to \emph{(forward) state
transformer semantics} where programs are understood as functions
(possibly with branching or side effects) that carry input states/values
to output ones.

\paragraph{Predicate Transformer Semantics and Quantum Mechanics}
The topic of weakest precondition and predicate transformer semantics
is  classic in computer science, in decades of foundational and
practical studies. Recently, fresh light has been shed on their
\emph{structural} aspects: the same kind of interplay between
\emph{dynamics} and \emph{observations} for \emph{quantum mechanics}
and \emph{quantum logic} appears in predicate transformer semantics,
as noted by one of the current authors---together with his
colleagues~\cite{Jacobs14CMCS,jacobs2015dijkstra,Jacobs15LMCS}.  This
enabled them to single out a simple categorical scheme---called
\emph{state-and-effect triangles}---that is shared by program
semantics and quantum mechanics.

On the program semantics side,
the scheme of state-and-effect triangles
allows
the informal ``duality'' between  state and predicate transformer
semantics to be formalized as a
categorical duality. Interestingly, the quantum counterpart of this
duality is the one between the \emph{Schr\"{o}dinger} and
\emph{Heisenberg} pictures of quantum mechanics. In this sense the idea
of weakest precondition dates back before Dijkstra, and before the
notion of program.

 State-and-effect triangles will be
elaborated on in Section~\ref{sub:stateAndEffectTriangles}; we note at this
stage that the term ``effect'' in the name refers to a notion in quantum
mechanics and should be read as \emph{predicate} in the programming
context. In particular, it has little to do with \emph{computational
effect}.


\paragraph{In Search of Healthiness}
The question of \emph{healthiness conditions} is one that is as old as
the idea of weakest precondition~\cite{Dijkstra76}: it asks for an intrinsic characterization of those predicate
  transformers which arise as the (backward) interpretation of
  programs.
One basic healthiness result is for \emph{nondeterministic}
programs.  The result is stated, in elementary terms, as follows.
\begin{theorem}[healthiness under the ``may''-nondeterminism]
\label{thm:healthiness-nondet-elementary}
 \begin{enumerate}
  \item Let  be a binary relation; it is thought
	of as a nondeterministic computation from   to . This
	 induces a predicate transformer ( for
	``weakest precondition'')

	for each  (thought of as a \emph{predicate} and
	more specifically as a \emph{postcondition}) and each .
  \item (Healthiness) Let  be a function. The following
	are equivalent.
	\begin{enumerate}
	 \item The function  arises  in
	       the way prescribed above. That is, there exists
	        such that
	       .
	 \item The map  is \emph{join-preserving}, where  and
	       
	       are equipped with (the pointwise extensions of) the order
	        in .
	\end{enumerate}
\end{enumerate}
\end{theorem}
\noindent
Here we
interpret  as false and  as true, a convention we adopt throughout the paper.

There are many different instances of healthiness results.  For example,
the works~\cite{Kozen81,Jones90PhD} study \emph{probabilistic}
computations in place of nondeterministic ones; the (alternating)
combination of nondeterministic and probabilistic branching is studied
in~\cite{MorganMS96}; and Dijkstra's original work~\cite{Dijkstra76}
deals with the (alternating) combination of nondeterminism and
divergence. In fact it is implicit in our notation 
that there is a possible ``must'' variant of
Theorem~\ref{thm:healthiness-nondet-elementary}. In this
variant, another  predicate transformer  is defined by

requiring that every possible poststate must satisfy the postcondition
.  The corresponding healthiness result has it that the resulting predicate
transformers are characterized by \emph{meet-preservation}.


The goal of the current work is to identify a structural and categorical
principle behind  healthiness, and hence to provide a common
ground for the  existing body of healthiness results, also providing a methodology that possibly aids finding
new  results.

 As a concrete instance of this goal, we wish to answer why
join-preservation should characterize ``may''-nondeterministic predicate
transformers  in
Theorem~\ref{thm:healthiness-nondet-elementary}. A first observation
would be that the powerset monad ---that occurs in the alternative
description  of a binary relation ---has
complete join-semilattices as its Eilenberg-Moore algebras.  This alone
should not be enough though---the framework needs to account for
different modalities, such as  (``may'') vs.\ 
(``must'') for nondeterminism. (In fact it turns out that this ``first
observation'' is merely a coincidence. See
Section~\ref{sub:diamond-modality} later.)


\paragraph{Our Contributions}
We shall answer to the above question of ``categorical healthiness
condition'' by unifying two constructions---or \emph{recipes}---of
state-and-effect triangles.
\begin{itemize}
 \item One recipe~\cite{Hasuo14,Hasuo15TCS} is called the \emph{modality} one,
       whose modeling of situations like in
       Theorem~\ref{thm:healthiness-nondet-elementary} is centered
       around the notion of \emph{monad}. Firstly,
       the relevant class of
       computations (nondeterministic, diverging,
       probabilistic, etc.) is determined by a monad , and
       a computation is then a function of the type . Secondly,
              the set  of truth values (such as 
       in Theorem~\ref{thm:healthiness-nondet-elementary})
carries a -algebra
       ; it represents a modality such
       as  and .
 \item The other recipe~\cite{Jacobs15CALCO} is referred to as the
       \emph{dual adjunction} one. It takes a dual adjunction
       \quad
       as an ingredient; and uses two \emph{comparison functors}---from
       a Kleisli category and to an Eilenberg-Moore category---to form a
       state-and-effect triangle, additionally exploiting 's
       completeness assumption. One notable feature is that the
       resulting state-and-effect triangle is automatically
       ``healthy''---this is because comparison functors are full and
       faithful.
\end{itemize}
Combining the two recipes we take advantages of both: the former
provides a concrete presentation of predicate transformers by a
modality; and the latter establishes healthiness.  We demonstrate that
many known healthiness results are instances of this framework.

The key to combining the two recipes is to interpret a
monad  on  in a category  that is other than
. For this purpose---assuming that the dual adjunction in the
second recipe is given with a dualizing
object---we introduce the notion of \emph{-relative -algebra}
and develop its basic theory. Notably the structure map of a -relative -algebra  is given by a \emph{monad map} from  to a suitable
continuation-like monad (that arises from the aforementioned dual
adjunction). This notion seems to be more than a tiny side-product of the
current venture: we expect it to play an important role in the
\emph{categorical model theory} (see e.g.~\cite{AdamekR94,LackP09,MakkaiP89}) where the
equivalence between (finitary) monads and \emph{Lawvere theories} is
fundamental. See below for further discussions.







\paragraph{Related and Future Work}
We believe the current results allow rather straightforward
generalization (from ordinary, -based category theory) to
enriched category theory~\cite{Kelly82}. For example, the use of
the -fold product  can be replaced by the \emph{cotensor}
. Doing so, and identification of this generalization's
relevance in program logics, is left as future work.

The current theoretical developments are heavily influenced by
\emph{Lawvere theories}, another categorical formalization of algebraic
structures that is (if finitary) equivalent to monads.
In particular, our notion of relative algebra is aimed to be a (partial)
answer to the oft-heard question: \emph{A Lawvere theory can be interpreted
in different categories. Why not a monad?} We intend to
establish
formal relationships in  future work, possibly in an enriched
setting.
 There the line of works on enriched Lawvere theories will be
relevant~\cite{LackP09,hyland2007category}.
The first observation in this direction is that: a monad  on 
gives rise to a (possibly large) ``Lawvere theory'' ; and
then its ``algebra'' in a category  (with enough products)
is a product-preserving functor . 

What is definitely lacking in the current work (and in our previous work~\cite{Jacobs15CALCO,Hasuo15TCS}) is  syntax for
programs/computations and program logics. In this direction the
work~\cite{GoncharovS13b} presents a generic  set of inference
rules---that is sound and relatively
complete---for a certain class of monadic computations.

We are grateful to a referee who brought our attention to
recent~\cite{HofmannN15}. Motivated by the modal logic question of
equivalences between Kripke frames and modal algebras---possibly
equipped with suitable topological structures---they are led to a
framework that is close to ours.  Their aim is a dual equivalence
between a Kleisli category  and a category of algebras
, and our goal of healthiness (i.e.\ a full and faithful
functor ) comes short of such only by failure of
iso-denseness. Some notable differences are as follows. Firstly,
in~\cite{HofmannN15} principal examples of a monad  is for
nondeterminism, so that a Kleisli arrow is a relation, whereas we have
probability and alternation as other leading examples.  Secondly, in
place of relative algebra (that is our novelty), in~\cite{HofmannN15}
they use the notion of algebra that is syntactically presented with
operations. Unifying the results as well as the motivations of the two
papers is an exciting direction of future research. See also
Remark~\ref{rem:CABA}.

Another closely related work~\cite{Keimel15}
studies healthiness from a domain-theoretic point of
view. While it is based on syntactic presentations of algebras
(differently from our monadic presentations), notable similarity
is found in its emphasis on continuation monads. Its domain-theoretic
setting---every construct is -enriched---will be relevant
when we wish to accommodate recursion in our current results, too.


\paragraph{Organization of the Paper}
We exhibited our leading example in
Theorem~\ref{thm:healthiness-nondet-elementary}. In
Section~\ref{sec:nondet} we describe its proof---in a categorical
language---and this will motivate our general framework. After
recalling the scheme of state-and-effect triangles
in
Section~\ref{sec:general-healthiness}, in
Section~\ref{sec:relativeAlgebraRecipe} we unify  two known recipes for them to present a
new \emph{relative algebra} recipe. The basic theory of relative
algebras is developed there, too. Section~\ref{sec:prob-examples} is devoted
to probabilistic  instances of our framework. Finally in Section~\ref{sec:two-player-setting}
we
further extend the generic framework to accommodate \emph{alternating}
branching that involve two players typically with conflicting interests.

Some missing proofs are found in the appendix.




\paragraph{Preliminaries and Notations}
We assume familiarity with basic category theory, from references
like~\cite{MacLane71,BarrW85}. We list some categories that we will use, mostly for fixing notations:
 the category   of sets and functions;
 the category   of sets and binary relations;
 and the categories
  and
  of complete join- and meet-semilattices, and
 join- and meet-preserving maps between them,
 respectively.\footnote{Here a \emph{complete join-semilattice} is a
 poset with arbitrary joins . It is well-known that in this case
 arbitrary meets  exist, too; we say ``join-'' to indicate the notion of
homomorphism we are interested in.
 } Given a monad ,
 its \emph{Eilenberg-Moore} and \emph{Kleisli} categories are denoted
 by  and , respectively. Their definitions are found
 e.g.\ in~\cite{MacLane71,BarrW85}.

Let  be monads on . The standard notion of
 \emph{monad map} from  to  is defined by
a natural transformation  that is compatible with
the monad structures. For its explicit requirements see
Appendix~\ref{appendix:monadMap}.




We shall be using various ``hom-like'' entities such as
homsets, exponentials, cotensors and so on; they are denoted by
, , , etc. For those entities we abuse
the notations  and  and use them uniformly for the
\emph{precomposition} and \emph{postcomposition} morphisms, such as:

for .
Another generic notation  we will use for those hom-like entities is
  for  correspondences like

An example of such is via the universality of products:

 where ,  and  is the -fold
 product of .




We shall use a somewhat unconventional notation of writing  for an
(Eilenberg-Moore) -algebra . In our arguments the
monad  is mostly obvious from the context, and this notional
convention turns out to be succinct and informative.








\auxproof{
\begin{definition}
  Let  be a binary relation.
  We then define mappings
  
  and
  
  as follows:
  

  The first operator  takes the \emph{image} of a given set
  by ; the  is the dual of  (that is probably used less
  commonly).  The operators  and  are well-known in the
  context of modal logic. They  are called \emph{diamond modality} and
  \emph{box modality}.



\end{definition}

Notice that  and  work covariantly (with respect
to ) whereas  and  work contravariantly. In fact it holds that  and .
We have two pairs of adjunctions 
and .

\begin{remark}
  If  is functional, that is, there is a function 
  such that    if and only if ,
then we have 
  and the situation degenerates to the familiar adjunction
  .
\end{remark}
}

\section{Leading Example: Nondeterministic Computation and Join- (or
 Meet-) Preservation}
\label{sec:nondet}

In this section, as a leading example, we revisit the well-known
healthiness result in Theorem~\ref{thm:healthiness-nondet-elementary}
together with its ``must''
variant. We shall
 prove the results in an abstract categorical language, paving the way to the
 general and axiomatic modeling in
 Section~\ref{sec:general-healthiness}.

\subsection{``May''-Nondeterminism}
\label{sub:diamond-modality}
In Section~\ref{sec:intro}, regarding
Theorem~\ref{thm:healthiness-nondet-elementary}, we noted
the coincidence between the healthiness condition (join-preservation)
and Eilenberg-Moore -algebras (complete join-semilattices).
This turns out to be a deceptive coincidence---the essence lies rather
in a \emph{factorization} of the powerset monad  by a dual
adjunction, as we shall describe.

We have a dual adjunction between  and the category

of complete join-semilattices and join-preserving maps.

It is given by a \emph{dualizing object} , in the ``homming-in''
manner:

here  is the poset ,
the poset  is the  -fold product of ,
and  is the set of
join-preserving maps.
This adjunction yields a monad
 on ;
the unit  of the monad  is defined by 
and the multiplication  is
.





The following  is the first key observation.
\begin{lemma}
  \label{lem:monad-isom-jslat}
  The monad   is isomorphic to the powerset
 monad , with an isomorphism
  given by
  .
  \qed
\end{lemma}
\noindent
 The isomorphism in Lemma~\ref{lem:monad-isom-jslat} put us in the
 following situation.

Here  is
the
 functor induced by the isomorphism  in
 Lemma~\ref{lem:monad-isom-jslat}; and  is the \emph{comparison
 functor}
from the Kleisli adjunction as the
 ``initial'' factorization of a monad. See
 e.g.~\cite{MacLane71,BarrW85}.

The second key observation is that
 the top composite
---its action on arrows, precisely---coincides
 with the predicate transformer  in
 Theorem~\ref{thm:healthiness-nondet-elementary}.
 Indeed, identifying a binary relation  with
 a function  and hence with
 a morphism  in , the action of  can
 be concretely described as follows. The arrows on the second line are all
 in .

Unfolding the construction of the comparison functor , the function
 in the end is presented as follows.
Given ,

This is nothing but the predicate  as
defined in Theorem~\ref{thm:healthiness-nondet-elementary}. Thus we have
established

for each  and .

The last key observation is that
 a comparison functor is full and faithful
in general.   The action  is therefore bijective;
 hence so is .
This proves Theorem~\ref{thm:healthiness-nondet-elementary}.


In the arguments above the key observations have been: 1) factorization of
a monad via a \emph{dual adjunction} (Lemma~\ref{lem:monad-isom-jslat});
2) a \emph{monad map}  giving rise to a predicate transformer
; and 3) the role of a \emph{comparison
functor} ---in particular that its fullness entails healthiness.
Our general framework will be centered around these three notions (dual
adjunction, monad map and comparison), with our notion of \emph{relative
algebra} bonding them together.


\begin{remark}\label{rem:CABA}
In the above (and in Theorem~\ref{thm:healthiness-nondet-elementary}) we
 established a full and faithful functor .  Cutting down its codomain, together with a
 well-known isomorphism between  and the category  of
 sets and relations, gives us a dual equivalence . Here  is the
 category of complete atomic Boolean algebras and join-preserving maps
 between them. The last dual equivalence is a well-known one, found
 e.g.\ in~\cite[Section~II.9]{Halmos06} and~\cite{JonssonT51}.

 Our principal interest in this paper---motivated by healthiness in
 program logics---is in a full and faithful functor. A dual equivalence,
 in contrast, is pursued typically in the context of modal logic
 (specifically for correspondences between modal algebras and relational
 frames); see e.g.~\cite{HofmannN15}. The relevance of such equivalences
 in program logics would lie in identification of (not only programs
 but) appropriate \emph{state spaces} that realize desired predicate
 transformers. Further investigation is future work.
\end{remark}

\begin{remark}\label{rem:pitfalls}
  For a join-semilattice  there is a poset isomorphism
  .
\auxproof{ Here  is the poset obtained
   from  by reversing the order; and
the isomorphism
is  given concretely  by
\ldotp y \not\leq x)0\in 21\in 2\pow[2^{(\place)}, 2]_{\biglor}L^{\op} \cong [L, 2]_{\biglor}\sigma\sigma_X\sigma''_X \colon \pow X \to [2^X, 2]_{\biglor}S \in \pow Xf \colon X \to 2ff^{\sharp} \colon \pow X \to 22\pow XX\sigma''_X(S)(f) = f^{\sharp}(S)\sigma=\sigma''\wpre_{\Box}\wpre_{\Diamond}2^{(\place)} \colon \Set \to (\CL_{\bigland})^{\op};  X \mapsto
2^X[\place, 2]_{\bigland} \colon (\CL_{\bigland})^{\op} \to \Set;\;  L \mapsto [L, 2]_{\bigland}\pow\sigma' \colon \pow \to [2^{(\place)}, 2]_{\bigwedge}\sigma'_X(S) = \lambda f \ldotp \bigland_{x \in S}  f(x)\Kl(\pow)\xrightarrow{\sigma'}
\Kl\bigl([2^{(\place)}, 2]_{\bigwedge}\bigr)
\xrightarrow{K'}
(\CL_{\bigland})^{\op}K'\varphi\colon 2^{Y}\to 2^{X}R\subseteq X\times Y\varphi=\wpre_{\Box}(R)\wpre_{\Box}\varphi\sum_{i\in
       I}c_{i}\ket{\varphi_{i}}\bra{\varphi_{i}}[0,1]\wpre_{\Diamond}\cong LK\wpre_{\Diamond}K\Kl{(\pow)}\EM(\pow)K\colon \Kl(\pow)\to\EM(\pow)K\colon \Kl(\pow)\to
(\CL_{\bigvee})^{\op}\Kl(\pow)K\Kl(\pow)\SetsU\in \pow X\wpre_{\Diamond}K\wpre_{\Diamond}\wpre_{\Diamond}T\cat{C}KRR\cat{D}\cat{D}^{\op}T=GF\cat{D}LRK\cong KKKKT\cong GFT=GF\Pos\Sets\Sets\PosT\tau\colon T\Omega\to \Omega\OmegaX\to TYT\bbP^{\tau}\tau \colon T {\Omega} \to \OmegaT\bbP^{\tau} \colon \Kl(T) \to \Set^{\op}  \bbP^{\tau} X = \Omega^X\Kl(T)f^{*}f\tau^{\sharp}h \colon Y \to \OmegaT\tau^{\sharp}(h) \colon TY \to \Omega\Sets\EM(T)\bbP^{\tau}(f)f\colon X\to TYh\colon Y\to \Omega\bbP^{\tau}(f)(h)\colon X\to \Omega\bbP^{\tau}\tauX\Omega^XXf \colon X \to TY\bbP^{\tau}f \colon {\Omega}^Y \to {\Omega}^X\bbP^{\tau}fT\OmegaT2=\{0,1\}1\pow20<1\tau\colon T\Omega\to \Omega\tau\colon T\Omega\to \Omega\bbP^{\tau}\bbP^{\tau}\cong [\place, \Omega_{\tau}]_{T}\co K\bbP^{\tau}K\Omega_{\tau}\tau\colon
 T\Omega\to \Omega\Omega_{\tau}[\place, \Omega_{\tau}]_{T}=\EM(T)(\place, \Omega_{\tau}) \Omega_{\tau}^XXXT\Omega_{\tau}\id^{\sharp}\id \colon \Omega^X \to \Omega^XT_{\Omega^{X},\Omega}TT\Sets\cat{D}\cat{D}T\Sets\cat{D}\VT\V\V\cat{D}\cat{D}A\in\cat{D}A\Sets\cat{D}A^{X}|X|A\in\cat{D}\cat{D}(A^{(\place)}, A)\cat{D}TT\Sets\cat{D}\cat{D}TA\in\cat{D}\alphaT\cat{D}(A^{(\place)}, A)\cat{D}T(A,\alpha)(B,\beta)f \colon A \to B\cat{D}X \in
 \Set\alpha^{\sharp}_{X}\alpha_{X}\colon TX\to \cat{D}(A^{X}, A)\cat{D}T\EM(T; \cat{D})\cat{D}T\cat{D}U_{\cat{D}}\cat{C}T\cat{C}A\in \cat{C}T\hat{\alpha} \colon T A \to AA\alpha \colon T \to A^{\cat{C}(\place, A)}f \colon A \to BT(A,\hat{\alpha})(B,\hat{\beta})a^{\sharp}_{X}\Sets(A^{X},A)\cong A^{\Sets(X,A)}\SetsT\Sets\EM(T;\Sets)\cong\EM(T)\hat{\alpha} \colon
 TA\to A\alpha_{X}\colon TX\to A^{\cat{C}(X,A)}t\in TXXV\colon X\to A\alpha_{X}(t)(V)\in
 AtV\hat{\alpha}\cat{D}\cat{D}TT\Sets\cat{D}TT\cat{L}_{T}TTT\pow\List\Sets\cat{D} = \Top\EM(\List; \Top)\cat{D} = \Pos\EM(\List; \Pos)T\cat{D}\cat{D}\cat{D'}H \colon \cat{D} \to \cat{D'}\cat{D}\cat{D'}H \colon \cat{D} \to \cat{D'}H\bar{H} \colon \EM(T; \cat{D}) \to \EM(T; \cat{D'})\bar{H}H\bar{H}\cat{D}T\tau \colon T \to \cat{D}(A^{(\place)}, A)\alpha^{\sharp} \colon A^{X} \to A^{TX}\Set\EM(T)T\Sets\cat{D}\cat{D}TT\OmegaD\in\cat{D}\cat{D}\cat{D}T\bbP^{\tau} \cong [-, \bar{\Omega}]_{T} \co KK[\place, \bar{\Omega}]_{\cat{D}}D \in \cat{D}[D, \bar{\Omega}]_{\cat{D}}\cat{D}(D, \OmegaD)T\zeta_D\id^{\sharp}\colon D\to
       \OmegaD^{\cat{D}(D, \OmegaD)}\cat{D}k \colon D \to ETk^* = [k, \bar{\Omega}]_{\cat{D}}
  \colon [E, \bar{\Omega}]_{\cat{D}} \to [D, \bar{\Omega}]_{\cat{D}}k^* \colon \cat{D}(E, \OmegaD) \to \cat{D}(D, \OmegaD)[\place, \bar{\Omega}]_{T}TA_a = (A, a \colon TA \to A)\cat{D}[A_a, \bar{\Omega}]_{T}a^*, \tau^{\sharp}_A \colon \OmegaD^A \tto \OmegaD^{TA}f \colon A_a \to B_bT\cat{D}f^* = [f, \bar{\Omega}]_{T} \colon [B_b, \bar{\Omega}]_{T} \to [A_a, \bar{\Omega}]_{T}\bbP^{\tau}\bbP^{\tau} \colon \Kl(T) \to \cat{D}^{\op}\bbP^{\tau}(X) = \OmegaD^X\Kl(T)[A_a, \bar{\Omega}]_{T}[D,
  \bar{\Omega}]_{\cat{D}}\bar{\Omega}T\cat{D}\cat{D}TTT\cat{D}\SetsT=GF\tau\colon T\to
\cat{D}(\OmegaD^{(\place)}, \OmegaD)X,Y\in\Sets\tau_{Y}\colon TY\to
  \cat{D}(\OmegaD^{Y}, \OmegaD)\bbP^{\tau}\tau_{Y}   \bbP^{\tau}_{XY} \tau\colon T\to \cat{D}(\OmegaD^{(\place)},
 \OmegaD)\bbP^{\tau}\wpre_{\Diamond}\Diamond\cat{D}V \colon \cat{D} \to \SetV\EM(T)T\Sets\Top\PosV \colon \cat{D} \to \Set\bar{V} \colon \EM(T; \cat{D}) \to \EM(T)U_{\cat{D}}\EM(T; \cat{D})\cat{D}TA, A_{\cat{D}}, A_{\hat{\alpha}},\bar{A}AA_{\cat{D}}\cat{D}V(A_{\cat{D}})=AA_{\hat{\alpha}}T\hat{\alpha}\colon TA\to AU(A_{\hat{\alpha}})=A\bar{A}\in \EM(T; \cat{D})TV U_{\cat{D}}\bar{A}=U\bar{V}\bar{A}=AAA_{\cat{D}}A_{\hat{a}}\bar{A}\cat{D}T\cat{D}T\cat{D}A_{\hat{\alpha}} = \p[\big]{A, \hat{\alpha} \colon TA \to A}TA_{\cat{D}}\in\cat{D}V A_{\cat{D}} = A\cat{D}T\bar{A}\bar{V} (\bar{A}) = A_{\hat{\alpha}}U_{\cat{D}} \bar{A} = A_\cat{D}\alpha \colon T \to \Set(A^{(\place)}, A)\hat{\alpha}V \colon \cat{D}(A_{\cat{D}}^{(\place)}, A_{\cat{D}}) \to
	  \Set(A^{(\place)}, A)X \in \Sett \in TX(\alpha_X)(t) \colon A^X \to A\cat{D}(\bar{\alpha}_X)(t) \colon A_{\cat{D}}^X \to A_{\cat{D}}A_{\cat{D}}A_{\hat{\alpha}}A\in\Sets\cat{D}TT\cat{D}\cat{D} = \PosTTA_{\hat{\alpha}}A_{\cat{D}}nXT(\Sigma,E)\sigma \in \Sigma(\Omega, \star, e)\OmegaD \in \cat{D}\star \colon \Omega \times \Omega \to \Omegae \colon 1 \to \Omega\cat{D}\cat{D}V \colon \cat{D} \to \Set\cat{D}DT[D, \bar{\Omega}]_{\cat{D}}\cat{D}(D, \bar{\Omega})T\Omega_{\tau}\Omega_{\tau}^{VD}[A_a, \bar{\Omega}]_{T}\cat{D}\EM(T)(A_a, \Omega_{\tau})\cat{D}U \after [D, \bar{\Omega}]_{\cat{D}} = \cat{D}(D,
 \bar{\Omega})V \colon \cat{D} \to \Set\bar{V}(\bar{\Omega}) = \Omega_{\tau}[D, \bar{\Omega}]_{\cat{D}}T\Omega_{\tau}^{VD}\tauV \co [\place,
 \bar{\Omega}]_{T} \cong \EM(T)(\place, \Omega_{\tau})\cat{D}V\colon \cat{D}\to\Sets\bar{\Omega} = (\OmegaD, \bar{\tau}) \in \EM(T;\cat{D})\bar{V}\bar{\Omega} = \Omega_{\tau} = (\Omega, \tau \colon T{\Omega} \to \Omega)T\bbP^{\bar{\tau}} \colon \Kl(T) \to \cat{D}^{\op}\bar{\Omega}\bbP^{\tau} \colon \Kl(T) \to \cat{\Set}^{\op}T\Omega_{\tau}V\colon \cat{D}\to\Sets\bbP^{\tau} \cong V \after \bbP^{\bar{\tau}}V\colon \cat{D}\to \SetsV \co [\place,
 \bar{\Omega}]_{T} \cong \EM(T)(\place, \Omega_{\tau})VV\colon \cat{D}\to \Sets\cat{D}T\bar{\Omega} = (\OmegaD, \bar{\tau})\Omega_{\hat{\tau}} = \bar{V} \bar{\Omega}T\hat{\tau}\colon
T\Omega\to \Omega\tauXt \in TXT(\tau_X)(t) \colon \Omega^{X} \to \Omegas^* \colon \Omega^{X} \to \Omega^{X'}s \colon X' \into XX\phi' \colon \Omega^{X'} \to \Omega(\tau_X)(t) = \phi' \co s^*\phi \colon \Omega^Y \to \Omega^Xx \in Xs \colon Y' \into
 Y\pi_{x}\co \varphis^{*}Tf \colon X \to Y\Kl(T)\bbP^{\tau} (f) \colon \Omega^Y \to \Omega^XT\bar{\Omega} =
    \p[\big]{\OmegaD, \bar{\tau} \colon T \to \cat{D}(\OmegaD^{(\place)}, \OmegaD)}\cat{D}TT\bar{\tau}_XX\phi \colon \Omega^Y \to \Omega^Xf \colon X \to Y\Kl(T)\bbP^{\tau}(f) = \phi\phi\cat{D}\bar{\phi} \colon \OmegaD^Y \to \OmegaD^X\bar{\phi} = \phiT\bbP^{\tau}(f) \colon \Omega^Y \to \Omega^Xf\colon X\to TY\Omega\phi \colon \Omega^Y \to \Omega^X\phi\Omega^Y\Omega^X\pow\dist,\sdist\dist\Sets\dist X=\{p:X\to [0,1]\mid \sum_{x\in X}p(x)= 1, \text{and 
 for all but finitely many }\}\dist f(p)(y)= \sum_{x\in f^{-1}(y)}p(x)\eta^{\dist}_{X}(x)(y)= 1y=x0\mu^{\dist}_{X}(\Phi)(x)=\sum_{p\in \dist X}\Phi(p)\cdot p(x)\sdist\sdist X=\{p \text{ w/ finite supp.}
\mid \sum_{x\in X}p(x)\le 1\}\dist\bbR^n\distx \oplus_p y = (1-p)x + py\dist\bbR\sett{x, y}(1-p)x + pyxp = 0y\dist \to \fpow\fpow\fpow\sdistx\colon \sdist X\to Xx(0)\in X0\sdistX\distXXf \colon X \to Y\sdist\sdist\dist\sdiste \colon \sdist X \to X\dist \subset \sdiste\distXe(0)\diste' \colon \dist X \to Xx \in Xe \colon \sdist X \to Xe(\mu) = e'(\mu + (1 - w(\mu)) \ket{x})e\sdiste(\ket{x}) = x\diste'\sdist\dist\sdist\Omega=[0,1][0,1]\dist,\sdist\dist\tau\colon \dist[0,1]\to [0,1][0,1]\sdistr\in\uintv\tau_r \colon \sdist{\uintv} \to \uintv\tau_r(p) = \sum_{x \in \uintv} x p(x) + r (1 - \sum_x p(x))\tau_{\mathrm{total}} = \tau_0\tau_{\mathrm{partial}} = \tau_1\sdist1 - \sum_x p(x)\tau_{\mathrm{total}}\tau_{\mathrm{partial}}\Diamond\Box\sdist\tau_{\mathrm{total}}\bbP^{\tau_{\mathrm{total}}}\colon \Kl(\sdist)\to
\Sets^{\op}X\to \sdist YXY\bar{\Omega}\cat{D}\GEModM\ovee0 \in M(x\ovee y) \ovee z \simeq x \ovee (y \ovee z)x \ovee 0 \simeq xx \ovee y \simeq y \ovee x\simeq(M, \ovee, 0)x \ovee y = 0 \Rightarrow x = y = 0x \ovee y = x \ovee z \Rightarrow y = zM{\cdot} \colon [0, 1] \times M \to M(r \ovee s) \cdot x \simeq (r \cdot x) \ovee (s \cdot x)r \cdot (x \ovee y) \simeq (r \cdot x) \ovee (r \cdot y)1 \cdot x = xr \cdot (s \cdot x) = (r \cdot s) \cdot xr, s \in [0, 1]r \ovee s = r + sr + s \leq 1\GEMod\sdist
XXp\ovee q\in\sdist X(p\ovee q)(x)=p(x)+q(x)\sum_{x\in X}p(x)+q(x)\le 1\sdist X\uintv\uintv^{X}\sdist\tauTotal\tauTotal\tauTotal\colon \sdist\to \Sets(\uintv^{(\place)},
 \uintv)(\tauTotal)_X(p)(f)
  = \sum_{x \in X} f(x) p(x)\tauTotal\colon \sdist\to \GEMod(\uintv^{(\place)},
 \uintv)(\tauTotal)_X(p)\colon [0,1]^{X}\to [0,1] Xp\in
 \sdist X0\ovee(\tauTotal)_{Y}\colon \sdist Y\to \GEMod(\uintv^{Y},
 \uintv)Y\mu_X \colon \GEMod(\uintv^X, \uintv) \to \sdist X\phi \colon \uintv^X \to \uintv\GEMod\mu_X(\phi)X\mu_X(\phi) = \sum_{x \in X} \phi(\delta_x) \ket{x}\delta_x(x) = 1\delta_x(y) = 1y \neq xX\sum_{x \in X} \phi(\delta_x) \leq 1\sum_{x \in X} \delta_x \le 1\sum_{x \in X} \phi(\delta_x) = \phi \p*{\sum_{x \in X} \delta_x} \le 1\sigma_X\mu_X\sdist\sdist\tauTotal\varphi \colon \uintv^Y \to \uintv^Xf\colon X\to \sdist Y\varphi=\bbP^{\tauTotal}(f)\phi0\ovee\varphi\Set\DcEMod\EMod\sdist\tauPartial\GEMod\cat{D}\tauPartial\sdist\tauPartial\varphi \colon \uintv^Y \to \uintv^Xf\colon X\to \sdist Y\varphi = \bbP^{\tauPartial}(f)\phi[0, 1]1\owedgex \owedge y = x + y - 1[0, 1]\cor \co x = r \cdot x + (1 - r)\dist\tau\EMod\GEMod\EModM1\leMx\le yy=x\ovee
 zz\in M0,1,\ovee\EMod\dist\tau\varphi \colon \uintv^Y \to \uintv^Xf\colon X\to \dist Y\varphi = \bbP^{\tau}(f)\phi01\ovee\varphi\sigma_X \colon \dist X \to \EMod(\uintv^X, \uintv)X\phi \colon \uintv^Y \to \uintv^XP \colon X \to \sdist Y\phiT\SetsR\EM(T)R\star TT\SetR\EM(T)R \star TF \dashv U \colon \EM(T) \to \SetRT=\pow\SetsR=\Upx\EM(\pow)\cong \CL_{\biglor}\calST=\dist\SetsR=\RC\EM(\dist)\cong \ConvSx_{1},\dotsc,x_{n}\in S\lambda_{1}+\cdots+\lambda_{n}=1\textstyle\sum_{i}\lambda_{i}x_{i}\in SR\SetsTK \colon \Kl(R \star T) \to \Kl(R)L \colon \EM(R) \to \EM(R \star T)\cat{D}(\OmegaD^{(\place)},\OmegaD)\EM(T)\Sets[[\place , \bar{\Omega}]_{T} ,
\bar{\Omega}]_{\cat{D}}\EM(T)\tau \colon T \to \cat{D}(\OmegaD^{(\place)}, \OmegaD)\rho \colon R \to [[\place , \bar{\Omega}]_{T} ,
\bar{\Omega}]_{\cat{D}}\bbP^{\rho}\bbP^{(\tau, \rho)}T\Set\cat{D}\bar{\Omega} = (\OmegaD, \tau)\cat{D}TR\EM(T)\rho \colon R \to [[\place , \bar{\Omega}]_{T} , \bar{\Omega}]_{\cat{D}}[[\place , \bar{\Omega}]_{T} , \bar{\Omega}]_{\cat{D}}\bar{\Omega}\bbP^{\rho} \colon \Kl(R) \to \cat{D}^{\op}\bbP^{\rho} A_a = A_a\Kl(R)K \colon \Kl(R \star T) \to \Kl(R)\bbP^{(\tau, \rho)}X\to (R\star T)YXYTR\cat{D}XY\bbP^{(\tau, \rho)}_{XY} \colon \Kl(R \star T)(X, (R \star T) Y) \to
    \cat{D}(\Omega^Y, \Omega^X)\bbP^{(\tau, \rho)}\cat{D}\bar{\tau} \colon T \to \cat{D}(\OmegaD^{(\place)}, \OmegaD)T\hat{\tau} \colon T{\Omega} \to \Omega\bar{\rho} \colon R \to [[\place , \bar{\Omega}]_{T} , \bar{\Omega}]_{\cat{D}}R\hat{\rho} \colon R{\Omega_{\tau}} \to {\Omega_{\tau}}TA_a[[A_a , \bar{\Omega}]_{T} , \bar{\Omega}]_{\cat{D}}T\Omega_{\tau}^{V [A_a , \bar{\Omega}]_{T}}
  \cong \Omega_{\tau}^{\EM(T)(A_a, \Omega_{\tau})}T\Set\cat{D}V \colon \cat{D}
\to \Set\cat{D}T\bar{\Omega} = (\OmegaD, \bar{\tau})R\EM(T)R\hat{\rho} \colon R{\Omega_{\tau}} \to {\Omega_{\tau}}{\Omega_{\tau}}\rho \colon R \to {\Omega_{\tau}}^{\EM(T)(\place, \Omega_{\tau})}\hat{\rho}\bar{\rho} \colon R \to [[\place ,
	  \bar{\Omega}]_{T}, \bar{\Omega}]_{\cat{D}}\cat{D}TA_a\rho_{A_a}^{\sharp} \colon \EM(T)(A_a, \Omega_{\tau}) \to \EM(T)(R A_a, \Omega_{\tau})Tf \colon A_a \to \Omega_{\tau}\hat{\rho} \co Rf \colon R A_a \to \Omega_{\tau}V\cat{D}\bar{\rho}^{\sharp} \colon [A_a, \bar{\Omega}]_{T} \to [R A_a, \bar{\Omega}]_{T}\bar{\rho}^{\sharp}V \bar{\rho}^{\sharp} = \rho^{\sharp}A_ax \in U R A_a({\rho}^{\sharp})_x = \pi_x \after \rho^{\sharp} \colon \EM(A_a, \Omega_{\tau}) \to \Omega\cat{D}(\bar{\rho}^{\sharp})_x \colon [A_a, \bar{\Omega}] \to \OmegaD\pi_x\Set\ev_xh \colon U R A_a \to \OmegaxV [A_a, \bar{\Omega}]_{T} \cong \EM(T)(A_a, \bar{\Omega})X\to (\pow_{+}\star\lift) Y\lift X=1+X\Sets\pow_{+}\EM(\lift)\cong\Sets_{*}\pow_{+}(X,x)=\bigl(\{S\subseteq
X\mid S\neq \emptyset\}, \{x\}\bigr)\tau\rho\cat{D}0\varphi \colon 2^Y \to 2^Xf\colon X \to (\pow_+ \star\lift) Y\varphi = \bbP^{(\rho, \tau)}(f)\phi0X\to (\Upx\star\pow) Y\Upx\EM(\pow)\cong \CL_{\biglor}\tau\rho\cat{D}\varphi \colon 2^Y \to 2^Xf \colon X \to (\Upx\star\pow) Y\varphi = \bbP^{(\rho, \tau)}(f)\phiX\to
(\RC\star \dist)Y\RC\EM(\dist)\cong\Conv[0,1]\tau\rhox \perp yf(x) \perp f(y)f(x) \ovee f(y) \leq f(x \ovee y)f(\lambda
x) = \lambda f(x)f(x \ovee \lambda 1) =
f(x) \ovee \lambda 1x \perp \lambda 1=YY\varphi \colon [0, 1]^Y \to [0, 1]^Xf \colon X \to (\RC\star \dist) Y\varphi = \bbP^{(\rho, \tau)}(f)\phif \colon X \to \pow_+(Y+1)\pow_+(X)XT = \calLR = \pow_+ \colon \Sets_* \to \Sets_*\calL\calL = (\place) + 1\Sets\Sets_*f \colon X \to \pow_+(Y + 1)\wpre(f) \colon 2^{Y + 1} \to 2^{X + 1}p \colon 2^{Y + 1} \to 2^{X + 1}p(\mathbf 0) = \mathbf 0\mathbf 0Ip(\bigland_{i \in I}\pi_i) = \bigland_{i\in I}p(\pi_i)\StrCL\calL(X, x) \mapsto \pow_+(X \setminus \{x\}) + 1\Sets_*\pow_+(\place) + 1\alpha \colon \pow_+ \star \calL = \pow_+(\place + 1) \To \pow_+(\place) + 1p \colon 2^Y \to 2^Xp\StrCLD_f = \set{x \in X}{f(x) = 1}f \colon X \to 2X\chi_f = \lambda x \in X \ldotp (x \in S)S \subset X\UP\Sets\Upx\CL_{\biglor}\calS\calS\UpxL \le L'x \in Ly \in L'x \le yN \colon X \to \UP Y\phi \colon 2^Y \to 2^X\phi(N) = \lambda x \ldotp \biglor_{S \in N(x)} \p*{ \bigland_{y \in S} f(y)}
    = \lambda x \ldotp \p*{D_f \in N(x)}\UP\SetF \dashv U \colon \EM(\pow) \cong \CL_{\biglor} \to \Set\Upx\CL_{\biglor}\UP = U \co \Upx \co F[\place, 2]_{\le} \dashv [\place, 2]_{\biglor} \colon \CL_{\biglor} \to \Pos^{\op}[[\place, 2]_{\biglor}, 2]_{\leq}\Upx\phi \colon 2^Y \to 2^XN \colon X \to \UP Y\RC\EM(D) \cong \ConvX\RC XX\sum_{i} \lambda_i C_i = \set{\sum_{i} \lambda_i x_i}{x_i \in C_i}f \colon X \to Y\RC f\RC f (C) = f[C]f[C]Cf\eta \colon X \to \RC X\mu \colon \RC (\RC X) \to \RC X\eta_X (x) = \sett{x}\mu_X (\calS) = \bigcup \calSf \colon M \to Nx \perp yf(x) \perp f(y)f(x) \ovee f(y) \leq f(x \ovee y)f(cx) = c f(x)f(x \ovee \lambda 1) = f(x) \ovee \lambda 1x \perp \lambda 1\EMod_{\leq}\EMod\oplus_p \colon \uintv^2 \to \uintv\Conv\EMod_{\le}[\place, \uintv] \colon \Conv \to \EMod_{\le}[\place, \uintv] \colon \EMod_{\le} \to \Conv\RC\hat{\rho} \colon \RC \uintv \to \uintv\uintv\hat{\rho}(C) = \inf C\RC\hat{\rho} \colon \RC \uintv \to \uintv\EMod_{\leq}X\rho^{\sharp}_X \colon \Conv(X, \uintv) \to \Conv(\RC X, \uintv)C \subset \dist XC(\rho_X^{\sharp})_C(\rho_X^{\sharp})_C (f) = \inf_{x \in C} f(x)\tau \colon \RC \to [[\place , \uintv], \uintv]\dist n \cong \Delta^nn\tau_{\Delta^n}\phi \colon \uintv^n \to \uintvC \subset \Delta^nC = \set{x \in \Delta^n}
        {\forall f \in \uintv^n \ldotp \phi(f) \leq f^{\sharp}(x)}C\phi(f) = \inf_{x \in C} f^{\sharp}(x)\phi(f) \leq \inf_{x \in C} f^{\sharp}(x)C\phi(f) \lt \inf_{x \in C} f^{\sharp}(x)f\phi(f) \lt f^{\sharp}(x)x \in CC\p*{\bigcap_{g} F_g} \cap F'F_g = \set{x \in \Delta^n}{ \phi(g) \leq g^{\sharp}(x) }g \in \uintv^nF' = \set{x \in \Delta^n}{ \phi(f) \geq f^{\sharp}(x) }F_gF'\Delta^ng_0, \ldots, g_k\p*{\bigcap_{i \le k} F_{g_i}} \cap F'c_1, \ldots, c_k, d, \lambda, \lambda'\phiY\phi \colon \uintv^Y \to \uintv^X\phi(p) = \lambda x \ldotp \inf_{\mu \in f(x)} \int p \muf \colon X \to \RC (\dist Y)\phiS,T\cat{C}ST\alpha\colon S\to T\eta^{(\place)}\mu^{(\place)}[2^{(\place)}, 2]_{\biglor}T\sigma_XXS\subseteq X\bigvee_{x \in S} \p[\big]{\bigvee_{f \in \calF} f(x)}
  = \bigvee_{f \in \calF} \p[\big]{\bigvee_{x \in S} f(x)}\calF \subset 2^X\sigma\calS \in \pow \pow X\sigma_{X}\delta_{x}\colon X\to 2\delta_x(x) = 1\delta_x(y) = 0x \neq yX[KX, \Omega_{\tau}]_T \cong \Omega^XKXTXX \in \Kl(T)f \colon X \to TYH\theta \colon H(A^X) \to (HA)^X\cat{D'}\theta\psi^A\psi^A_X = (\theta^{-1})^* \co H\psi\bar{H}Hf\cat{D'}T\bar{H}\bar{H}\bar{H}H\OmegaD^X\cat{D}(M, \OmegaD)X^*M^*[\place, \bar{\Omega}]_{T}[\place, \bar{\Omega}]_{\cat{D}}TA_aM \in \cat{D}f \colon M \to A^*\cat{D}\bbP^{\tau}K\Kl(\tau)\Kl(T)\SetsK\Kl(\tau) \colon \Kl(T)(X, Y) \to \Kl(\cat{D}(\OmegaD^{(\place)}, \OmegaD))(X, Y)\Kl(\tau)\tau\Kl(T)(X, Y) = \Set(X, TY)\Kl(\cat{D}(\OmegaD^{(\place)}, \OmegaD))(X, Y) = \Set(X, \cat{D}(\OmegaD^{Y}, \OmegaD))\tau\tau_{*}\tau\tau_{*}Tt \colon 1 \to TXs \colon X' \into Xt = Ts \circ t'X = 1Y = \emptysetY \neq \emptyset\phi\phi\phi = \phi' \after s^*s \colon Y' \into Y\phi' \colon \Omega^{Y'} \to \OmegaY'r \colon Y \onto Y'\phi' = \phi \after r^*\phi'\cat{D}\bar{\phi'}\sigma_{Y'}t' \in T{Y'}\sigma_{Y'}(t') = f't = Ts(t')\sigma_X(t) = \phi\Omega^X\OmegaXC \subset \Omega^{X}(s^{*})^{-1}(S)s \colon X' \into XS\Omega^{X'}X = 1\phiz \in \Omega\phi^{-1}(z)(\iota_z^*)^{-1}(S_z)\iota_z \colon Y_z \into YS_z \subset \Omega^{Y_z}\iota \colon Y' = \bigcup_{z \in \Omega} Y_z \into Y\phi\iota^*\phis \colon n \to Y\phi' \colon \Omega^n \to \Omega\phi' \circ s^* = \phis^*\phi'\Omega^n\phi$ is continuous.
\end{proof}






\end{document}
