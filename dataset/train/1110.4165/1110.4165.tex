\begin{figure}[t]
\begin{align*}
      \tau & \grmeq \unitk \grmor \clockk(\alpha)
      & & \textit{Types}
    \end{align*}
 \caption{Syntax of types}
  \label{fig:syn-types}
\end{figure}

 \begin{figure}[t]
 \begin{gather*}
    \tag{\Twfc, \Twfu}
    \mathcal R, \alpha \vdash \clockT \alpha
    \qquad
    \mathcal R \vdash \unitT
\\
\tag{\Tvar, \TclockR, \Tunit}
   \frac{
     \mathcal R \vdash \tau
   }{
     \Gamma,x\colon\tau; \mathcal R
     \vdash x \colon \tau 
   }
   \qquad
\Gamma,c\colon\clockT \alpha; \mathcal R, \alpha
     \vdash c \colon \clockT \alpha \qquad
   \Gamma; \mathcal R
   \vdash \unitV \colon \unitT
   \\
   \tag{\TclockSeq}
   \frac{
     \Gamma; \mathcal R \vdash v_1 \colon \clockT {\alpha_1} 
     \quad
     \cdots
     \quad
     \Gamma; \mathcal R \vdash v_n \colon \clockT {\alpha_n} 
     \qquad 
     \alpha_i \neq \alpha_j, \text{ if } i \neq j 
     \qquad
     \alpha_i \text{ not in } \Gamma
   }{
     \Gamma; \mathcal R \vdash v_1 \dots v_n \colon \{\alpha_1
     , \dots, \alpha_n\}
   }
 \end{gather*}
 \caption{Typing rules for values and for well-formed types}
 \label{fig:typing-values}
\end{figure}

 \begin{figure}[t]
 \begin{gather*}
    \tag{\Tvalue, \Tmake}
    \frac{
      \Gamma; \mathcal R \vdash v \colon \tau
    }{
      \Gamma; \mathcal R; \mathcal Q
      \vdash v \colon (\tau, \mathcal R, \mathcal Q) 
    }
    \qquad
    \frac{
      \alpha \text{ is fresh}
    }{
      \Gamma; \mathcal R; \mathcal Q
      \vdash \newClock \colon (\clockT{\alpha}, \mathcal R \cup
      \{\alpha\}, \mathcal Q)
    }
    \\
   \tag{\Tresume, \Tdrop}
    \frac{
      \Gamma; \mathcal R
      \vdash v \colon \clockT \alpha
      \qquad
      \alpha \notin \mathcal Q
    }{
      \Gamma; \mathcal R; \mathcal Q
      \vdash \resume v \colon (\unitk, \mathcal R; \mathcal Q \cup \{\alpha\})
    }
    \qquad
    \frac{
      \Gamma; \mathcal R
      \vdash v \colon \clockT \alpha
    }{
      \Gamma; \mathcal R; \mathcal Q
      \vdash \drop v \colon (\unitk, \mathcal R \setminus \{\alpha\}, \mathcal Q \setminus \{\alpha\})
    }
    \\
   \tag{\Tasync,\Tnext}
   \frac{
     \Gamma; \mathcal R
     \vdash \vec v \colon \mathcal R'
\qquad
     \Gamma; \mathcal R'; \mathcal {Q} \cap \mathcal R'
     \vdash e \colon (\_, \emptyset, \emptyset)
   }{
     \Gamma; \mathcal R; \mathcal Q
     \vdash \async {\vec v} e \colon (\unitk, \mathcal R; \mathcal Q)
   }
   \qquad
   \Gamma; \mathcal R; \mathcal R
    \vdash \nextk \colon (\unitk, \mathcal R; \emptyset)
\\
   \tag{\Tfinish}
   \frac{
     \Gamma; \emptyset; \emptyset \vdash e \colon (\tau,
     \emptyset, \emptyset)
   }{
     \Gamma; \mathcal R; \mathcal Q
     \vdash \finish e \colon (\tau, \mathcal R; \mathcal Q)
   }
  \\
   \tag{\Tlet}
   \frac{
     \Gamma; \mathcal R; \mathcal Q \vdash e_1 \colon (\tau,
     \mathcal{R'}, \mathcal{Q'})
     \qquad
     \Gamma, x \colon \tau; \mathcal{R'}; \mathcal{Q'} \vdash
     e_2 \colon (\tau', \mathcal{R''}, \mathcal{Q''})
   }{
     \Gamma; \mathcal R; \mathcal Q
     \vdash \lets x {e_1} {e_2} \colon (\tau', \mathcal {R''},
     \mathcal {Q''})
   }
 \end{gather*}
 \caption{Typing rules for expressions}
 \label{fig:typing-expressions}
\end{figure}

 
\section{Type System}
\label{sec:type-system}

This section presents a type system that uses \emph{singleton types}
to track clock usage throughout a program.



For types we rely on an additional base set of singleton types ranged
over by $\alpha$.
The syntax of types, depicted in Figure~\ref{fig:syn-types},
introduces the type \unitT{} of unit values, and the type $\clockT
\alpha$ of a \emph{particular} clock. We assign a different type to
each clock in order to ensure the correct usage of the clock
constructs within a program.

The type system for \xtenclocks{} programs is defined in
Figures~\ref{fig:typing-values} and \ref{fig:typing-expressions}.
A typing~$\Gamma$ is a map from variables (or activity labels) and
clocks to types.
We write $\dom \Gamma$ for the domain of $\Gamma$. When $x \not \in
\dom \Gamma$ we write $\Gamma, x \colon \tau$ for the typing $\Gamma'$
such that $\dom \Gamma' = \dom \Gamma \cup \{x\}$, $\Gamma'(x) =
\tau$, and $\Gamma'(y) = \Gamma(y)$ for $y \neq x$.
The type system also uses sets of singleton types, ranged over by
$\mathcal R$, for \textbf registered clocks, and~$\mathcal Q$, for
\textbf quiescent clocks.



The typing rules for values and for well formed types
(Figure~\ref{fig:typing-values}) are simple to follow. Well-formed\-ness
for clock types (rule~\Twfc) ensures that activities only make use of
clocks they are registered with.
Rule~\TclockSeq{} ensures that different clocks (as those in the heap)
have distinct singleton clock types, a property that is crucial for
establishing type safety.
For typing expressions we use a type system
(Figure~\ref{fig:typing-expressions}) that records the changes made to
the set of registered clocks, either by creating or dropping clocks,
and to the set of quiescent clocks (using $\resumek$ and $\nextk$) of
an expression.
Typing judgements are of the form $\Gamma; \mathcal R; \mathcal Q
\vdash e \colon (\tau, \mathcal R', \mathcal Q')$ meaning that
expression $e$ is well typed assuming the types for the free
identifiers in $\Gamma$, the registered clocks in~$\mathcal R$, and
the quiescent clocks in~$\mathcal Q$.
The type of an expression is a triple recording the type $\tau$ of its
value, as well as the registered $\mathcal R'$ and the
quiescent~$\mathcal {Q'}$ sets after execution of the expression.

Most typing rules are straightforward.
When creating a clock (rule~\Tmake) we associate a new singleton type
$\alpha$ with the clock and include it in set of clocks registered by
the activity ($\mathcal R \cup \{\alpha\}$).
Rule~$\Tresume$, which asserts that ``an activity may invoke
\texttt{resume()} only on a clock it is registered with, and has not
yet dropped'' (page 208, \textit{vide} rule~\Tvar), marks clock~$\alpha$
as quiescent.
Notice that a clock cannot be resumed more than once for the same
phase~($\alpha \not \in \mathcal R$), contrary to the language
reference that reads, ``Nothing happens if the activity has already
invoked a resume on this clock in the current phase'' (page 208).
A~$\drop v$ expression removes clock~$v$ from both the sets~$\mathcal
R$ and~$\mathcal Q$, thus the clock cannot be passed to new
activities, be the target of a~$\resumek$ expression, or be dropped
again.

For expression~$\async {\vec c} e$, the language reference reads
``Starts a new activity, initially registered with clocks [$\vec v$],
and running [$e$]. The activity running this code must be registered
on those clocks'' (page 207, w.r.t.\@ rule~$\Tasync$).
Rule~$\Tasync$ asserts that when an activity spawns another
activity registered on a sequence of clocks, the quiescent property of
the clocks is preserved by propagating the information about the
quiescent clock $\alpha$ ($\mathcal {Q} \cap \mathcal R$).
Moreover, the new activity must have dropped all its clocks upon
termination, contrary to the language reference that reads, ``All
activities are automatically deregistered from all clocks they are
registered with on termination (normal or abrupt)'' (page 207).
An activity cannot share a clock it does not hold, as noted in the
language reference, ``lacking that registration, cannot register a
sub-activity on it [a clock] with async'' (page 208).
Expression $\nextk$ marks the end of a phase; it checks
that all clocks have been resumed and clears
the quiescent clocks for the new phase (rules~$\Tnext$).


The $\finishk$ construct may interfere with clocks and cause
programs to deadlock.
In order to avoid such situations we prevent the body of
a $\finish e$ expression ($e$) from accessing any clock already
defined, thus eliminating (nested) dependencies between clocks
and $\finishk$.
Rule~\Tfinish{} also forces $e$ to unregister from all clocks it has
created, and therefore $\finish e$ has no effect on registered and
quiescent clocks.
This follows the semantics of the current version of X10 that reads,
``Inside of \texttt{finish\{S\}}, all \texttt{clocked} asyncs must be
in the scope an unclocked async'' (page 210).
Refer to the examples below for further discussion on the deadlock 
problem.
When typing a $\letk$ expression (rule~\Tlet), its continuation~$e_2$
is typed taking into consideration the effects produced by
expression~$e_1$.
The type of the $\letk$ is that of~$e_2$, as usual.

We have deliberately deviated from the standard X10 semantics in three
cases: $\nextk$, $\dropk$, and $\resumek$. The reasons for such
deviation are: (a) to illustrate the power of singleton types in
keeping track of clocks, (b) to simplify the (operational and static)
semantics, (c) to enforce a programming discipline that may avoid
potential bugs, and (d) because the compiler has enough information to
suggest, or automatically insert, code fixes (\eg by enumerating the
clocks that need to be dropped before a \nextk; see examples below).

Below we discuss a few \xtenclocks programs and the semantic
guarantees our type system enforces.
We decorate the examples with the typing assumptions ($\Gamma,
\mathcal R, \mathcal Q$) holding for each expression.

\paragraph{Example 1: Aliasing}

Our first example concerns clock aliasing, only introduced in X10 in
version 2.01. The example may read a bit trivial but illustrates more
sophisticated aliasing situations, derived for example from procedure
calls. 
Clearly a type system with linear control like the one we are going to
present allows to relieve such a restriction.
\begin{lstlisting}[numbers=right]
  //activity a1
  let x = makeClock in (                  // {x:clock(alpha)},{alpha},emptyset
    async x ( //activity a2               // {x:lock(alpha)},{alpha},emptyset
       let y = x in (                     // {x:clock(alpha),y:clock(alpha)},{alpha},emptyset
         resume x;                        // {x:clock(alpha),y:clock(alpha)},{alpha},{alpha} 
         drop y)                          // {x:clock(alpha),y:clock(alpha)},emptyset,emptyset
    );                                    // {x:clock(alpha)},{alpha},emptyset 
    drop x)                               // emptyset,emptyset,emptyset
\end{lstlisting}
In our case the code is typable, assigning the same singleton type
$\clockT\alpha$ to both $x$ and $y$. Upon introducing variable
$x$~(line 2), it gets assigned type $\clockT \alpha$ (\textit{vide}
rules~\Tlet{} and~\Tmake). Variable~$y$~(line~4) gets assigned type
$\clockT \alpha$, the type of $x$~(again using rules~\Tlet{} and~\Tmake).

\paragraph{Example 2: Resume state inheritance}

Our second example deals with a restriction the language reference
disallowed up until version 2.04, ``A potentially live execution path
from a \texttt{resume} statement on a clock~\texttt{c} to an async
spawn statement'' (page 153, version 2.04).
Our operational semantics allows the forked activity to inherit the
\emph{resume state} (resumed/not resumed) of the parent activity
(\textit{vide} rule~$\Rasync$) and therefore preserves the quiescence
property of clocks and avoids a race condition on the clock. Notice
that the forked activity~$a_2$ inherits the quiescent state of clock
$\alpha$~(line~4), resumed at line~3.
\begin{lstlisting}[numbers=right]
  //activity a1
  let x = makeClock in (                  // {x:clock(alpha)},{alpha},emptyset
    resume x;                             // {x:clock(alpha)},{alpha},{alpha}
    async x ( //activity a2               // {x:clock(alpha)},{alpha},{alpha}
      next;                               // {x:clock(alpha)},{alpha},emptyset
      drop x                              // {x:clock(alpha)},emptyset,emptyset
     );                                   // {x:clock(alpha)},{alpha},{alpha}
    drop x)                               // {x:clock(alpha)},emptyset,emptyset
\end{lstlisting}

\paragraph{Example 3: Race condition generated by not inheriting the
  resume state}

Describes a race condition triggered by clock synchronisation.
Activity~$a_1$ creates a clock~$x$, starts a second activity $a_2$
registered with clock~$x$ that, in turn, resumes on~$x$ and starts a
third activity~$a_3$ also registered with~$x$.
\begin{lstlisting}[numbers=right]
  //activity a1
  let x = makeClock in (                  // {x:clock(alpha)},{alpha},emptyset 
    async x ( //activity a2               // {x:clock(alpha)},{alpha},emptyset
      resume x;                           // {x:clock(alpha)},{alpha},{alpha}
      async x ( //activity a3             // {x:clock(alpha)},{alpha},{alpha}
        next;                             // {x:clock(alpha)},{alpha},emptyset
        drop x                            // {x:clock(alpha)},emptyset,emptyset
       );                                 // {x:clock(alpha)},{alpha},{alpha}
       next;                              // {x:clock(alpha)},{alpha},emptyset
       drop x                             // {x:clock(alpha)},emptyset,emptyset
    );                                    // {x:clock(alpha)},{alpha},emptyset
    resume x;                             // {x:clock(alpha)},{alpha},{alpha}
    next;                                 // {x:clock(alpha)},{alpha},emptyset
    drop x)                               // {x:clock(alpha)},emptyset,emptyset
\end{lstlisting}
The race condition might occur because after $a_2$ resumes on $x$
(line~4), either activity~$a_1$ may advance clock~$x$ phase by executing
$\nextk$ (line~13) or $a_2$ may register a new activity~$a_3$
with~$x$ (line~5), blocking~$a_1$ until activity~$a_3$ executes
its $\nextk$ instruction (line~6).
By inheriting the resume status of clock~$x$, activity~$a_3$ does not
block activity~$a_1$ and the race condition disappears (\textit{vide}
rule~\Rasync{} in Figure~\ref{fig:reduction-rules-activity} and
rule~\Tasync{} in Figure~\ref{fig:typing-expressions}). 

\paragraph{Example 4: Resume after resume}

The next example deals with resuming after resuming, a pattern
accepted in X10. Rule~$\Tresume$ rejects the program below, since it
is able to determine that clock~$x$ is resumed twice.
\begin{lstlisting}[numbers=right]
  //activity a1
  let x = makeClock in (                  // {x:clock(alpha)},{alpha},emptyset
    resume x;                             // {x:clock(alpha)},{alpha},{alpha}
    resume x)                             // error:clock x already quiescent for activity a1
\end{lstlisting} 


\paragraph{Example 5: Explicit drops}

Unlike X10, we have decided to explicitly deregister activities from
clocks upon activity termination.
Our type system keeps track of the clocks an activity is registered
with, and rejects programs with activities that finish before
deregistering from all its clocks.
Clocks without registered activities can be safely garbage collected
(\textit{vide} rule~\Rdrop).
The following example fails to type check, since the launched
activity does not drop clock $x$.
The compiler may easily suggest an appropriate fix: adding a
\lstinline|drop x| after the \lstinline|next| instruction on line 5,
or even automatically introduce such an instruction.
\begin{lstlisting}[numbers=right]
  //activity a1
  let x = makeClock in (                  // {x:clock(alpha)},{alpha},emptyset
    async x (        //activity a2        // {x:clock(alpha)},{alpha},emptyset
      resume x;                           // {x:clock(alpha)},{alpha},{alpha}
      next                                // {x:clock(alpha)},{alpha},emptyset
    );                                    // error: activity a2 did not drop x 
    drop x)
\end{lstlisting}


\paragraph{Example 6: Finish/async deadlock}

Finally, we discuss the interplay among\finishk\!, $\asynck$, and
clocks, which may cause programs to deadlock. 
The following program deadlocks because activity $a_2$ is waiting on
$\nextk$ (line~6) for activity~$a_1$ to advance on $x$, which is
planned to occur at line~10, but $a_1$ is waiting on
$\finishk$ (line~3) for activity $a_2$ to terminate, so $a_1$ never
reaches line~10 and the program deadlocks.
\begin{lstlisting}[numbers=right]
  //activity a1
  let x = makeClock in (                  // {x:clock(alpha)},{alpha},emptyset
    finish                                // {x:clock(alpha)},emptyset,emptyset
      async x (        //activity a2      // error: clock x is not in scope of finish
        resume x; 
        next; 
        drop x
      ); 
    resume x;
    next;
    drop x)
\end{lstlisting}
The cause for deadlock is that activity $a_2$ is registered with a clock
that is defined outside the enclosing \finishk\!: clock $x$ is defined
in line~2, whereas the\finishk expression extends from line~3 to line~8.
Our type system rejects this program, because when typing a$\finish e$ expression we type check $e$ in an environment 
with no registered clocks (\textit{vide} rule~\Rfinish). 


\paragraph{Example 7: Clocked finish/clocked async}

Version 2.10 of the X10 language introduces keyword \lstinline{clocked} to
prefix async and finish.
\begin{quote}
  In the most common case of a single clock coordinating a few
  behaviors, X10 allows coding with an implicit clock.
[...]
A \lstinline{clocked finish} introduces a new clock.
It executes its body in the usual way that a finish does---except
  that, when its body completes, the activity executing the clocked
  finish drops the clock, while it waits for asynchronous spawned
  asyncs to terminate.
A \lstinline{clocked async} registers its async with the implicit clock
  of the surrounding clocked \lstinline{finish}.
[...]
Clocked finishes may be nested. The inner
  \lstinline{clocked finish} operates in a single phase of the outer one.
\end{quote}

This featured introduced in the language is purely ``syntactic
sugar,'' which makes the common practice of creating a single clock and
sharing it among activities simpler.
The following two code listings present a program with and without the
syntactic extension side-by-side.
On the left column we show an activity written on a language that
resembles X10~\cite{saraswat:x10-report}. Remember that, in X10,
expression~$\nextk$ implicitly issues a~$\resumek$ and also that
activities implicitly drop all clocks on exit.
On the right column we find an equivalent activity written in our language.
Activity $a_1$ spawns three activities $a_2$, $a_3$, and $a_4$.
Activities~$a_2$ and $a_3$ share the same clock, say~$c_1$,
whereas activity~$a_4$ holds a different clock, say~$c_2$, but
it does not hold~$c_1$.
In the first phase of clock~$c_1$ the system executes
concurrently~$e_1$, $e_3$, and activity~$a_4$.
In the second phase of clock~$c_1$, activity~$a_4$ has
terminated, and expressions~$e_2$ and $e_4$ execute concurrently.

\begin{tabular}{l|r}
\!\!\!\!\!
\begin{lstlisting}[boxpos=t]
//activity a1
clocked finish (

 clocked async ( //activity a2
  e1
  next;
  e2

 );
 clocked async ( //activity a3
  e3
  next;
  e4

 )
 clocked finish (


  clocked async ( //activity a4
   e5
   next;
   e6

  )

 )

)
\end{lstlisting}
\!\!
&
\begin{lstlisting}[boxpos=t]
//activity a1
finish                          //emptyset,emptyset,emptyset
 let c1 = makeClock in (        //{c1:clock(alpha1)},{alpha1},emptyset
  async c1 ( //activity a2      //{c1:clock(alpha1)},{alpha1},emptyset
   e1
   resume c1; next;             //{c1:clock(alpha1)},{alpha1},emptyset
   e2         
   drop c1;                     //{c1:clock(alpha1)},emptyset,emptyset
  );                            //{c1:clock(alpha1)},{alpha1},emptyset
  async c1 ( //activity a3      //{c1:clock(alpha1)},{alpha1},emptyset
   e3
   resume c1; next;             //{c1:clock(alpha1)},{alpha1},emptyset
   e4
   drop c1;                     //{c1:clock(alpha1)},emptyset,emptyset
  );                            //{c1:clock(alpha1)},{alpha1},emptyset
  finish                        //{c1:clock(alpha1)},emptyset,emptyset
   let c2 = makeClock in (      //{c1:clock(alpha1),
                                // c2:clock(alpha2)},{alpha2},emptyset
    async c2 (//activity a4     //{c1:clock(alpha1),
     e5                         // c2:clock(alpha2)},{alpha2},emptyset
     resume c2; next;           //{c1:clock(alpha1),
     e6                         // c2:clock(alpha2)},{alpha2},emptyset
     drop c2           //{c1:clock(alpha1),c2:clock(alpha2)},emptyset,emptyset
    );             //{c1:clock(alpha1),c2:clock(alpha2)},{alpha2}, emptyset
    drop c2            //{c1:clock(alpha1),c2:clock(alpha2)},emptyset,emptyset
   );                           //{c1:clock(alpha1)},{alpha1},emptyset
  drop c1                       //{c1:clock(alpha1)},emptyset,emptyset
 )                              //emptyset,emptyset,emptyset
\end{lstlisting}
\end{tabular}
~\\
The activity (on the right) obtained by expanding
\lstinline[morekeywords=clocked]{clocked finish} and
\lstinline[morekeywords=clocked]{clocked async} is still typable with
the type system we propose.

