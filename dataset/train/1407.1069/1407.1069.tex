\documentclass[twocolumn,english,journal]{IEEEtran}
\usepackage[T1]{fontenc}
\usepackage{babel}
\usepackage{amsmath}
\usepackage{amssymb}
\usepackage[unicode=true,
 bookmarks=true,bookmarksnumbered=true,bookmarksopen=true,bookmarksopenlevel=1,
 breaklinks=false,pdfborder={0 0 0},backref=false,colorlinks=false]
 {hyperref}
\hypersetup{pdftitle={Your Title},
 pdfauthor={Your Name},
 pdfpagelayout=OneColumn, pdfnewwindow=true, pdfstartview=XYZ, plainpages=false}

\makeatletter
\usepackage[caption=false,font=footnotesize]{subfig}

\newtheorem{assumption}{Assumption}
\newtheorem{theorem}{Theorem}
\newtheorem{remark}{Remark}
\newtheorem{definition}{Definition}

\makeatother

\begin{document}

\title{Control of nonlinear systems:\\
a model inversion approach}


\author{C. Novara, M. Milanese\thanks{Carlo Novara is with Dipartimento di Automatica e Informatica, Politecnico
di Torino, Italy, e-mail: \protect\href{http://carlo.novara@polito.it}{carlo.novara@polito.it}.
Mario Milanese is with Modelway srl, Italy, e-mail: \protect\href{http://mario.milanese@modelway.it}{mario.milanese@modelway.it}.}}
\maketitle
\begin{abstract}
A novel control design approach for general nonlinear systems is presented
in this paper. The approach is based on the identification of a polynomial
model of the system to control and on the on-line inversion of this
model. An efficient technique is developed to perform the inversion,
which allows an effective control implementation on real-time processors.
\end{abstract}

\section{Introduction}

\label{sec:ibc_approach}

Consider a nonlinear discrete-time SISO system in regression form:


where  is the input, 
is the output, is a disturbance
including both process and measurement noises, and  is the system
order.  and  are compact sets. In particular, 
accounts for input saturation.

Suppose that the system (\ref{ss_sys}) is unknown, but a set of measurements
is available:

where  and  are bounded for all .
The tilde is used to indicate the input and output samples of the
data set \eqref{eq:data}.

Let  be a set of initial
conditions of interest for the system (\ref{ss_sys}) and, for a given
initial condition , let 
be a set of output sequences of interest.

The aim is to control the system (\ref{ss_sys}) in such a way that,
starting from any initial condition ,
the system output sequence 
tracks any reference sequence .
The set of all solutions of interest is defined as .
The set of all possible disturbance sequences is defined as .

After a brief section regarding the notation used in the paper, an
approach called NIC (Nonlinear Inversion Control) is proposed for
the design of a controller , allowing the accomplishment
of the above task.


\section{Notation}

A column vector  is denoted as .
A row vector  is denoted as ,
where  indicates the transpose.

A discrete-time signal (i.e. a sequence of vectors) is denoted with
the bold style: , where 
and  indicates the discrete time;  is the
th component of the signal  at time .

A regressor, i.e. a vector that, at time , contains  present
and past values of a variable, is indicated with the bold style and
the time index: .

The  norms of a vector 
are defined as

The  norm is also used to denote the absolute value
of a scalar: 
for . 

The  norms of a signal 
are defined as

where  is the th component of the signal 
at time . These norms give rise to the well-known 
Banach spaces.


\section{NIC control design}

\label{sub:nl_des}

The proposed approach relies on identifying from the data \eqref{eq:data}
a model of the form

where  and  are the system input and output, and 
is the model output. For simplicity, the model is supposed of the
same order as the system but this choice is not necessary: all the
results presented in the paper hold also when the model and system
orders are different. Indications on the choice of the model order
are given in Section \ref{sec:design}.

A parametric structure is taken for the function :

where  are basis functions and  are parameters
to be identified. The basis function choice is in general a crucial
step, \cite{LLju1,HsNoAUT06,Novara2011711}. In the present NIC approach,
polynomial functions are used. The motivations are mainly two: (1)
polynomials have been proved to be effective approximators in a huge
number of problems; (2) as we will see later, they allow a ``fast''
controller evaluation. The identification of the parameter vector
 can be performed by
means of convex optimization, as shown in Section \ref{sec:design}.

Once a model of the form \eqref{eq:model} has been identified, the
controller  is obtained by its inversion:

Suppose that, at a time , the reference value for the time 
is  and the current regressor is .
Inversion consists in finding a command input  such that
the model output at time  is ``close'' to : 

The latter equality may be not exact for two reasons: (1) no 
may exist for which  is exactly equal to ;
(2) values of  with a limited  norm may be
of interest, in order to have a not too high command activity. This
kind of inversion is called (approximate) right-inversion and can
be performed also when  is not bijective with respect to 
(e.g., for some  and , more than one
value of  may exist such that \eqref{eq:inv1} holds).

The command input  yielding \eqref{eq:inv1} can be computed
according to the following optimality criterion: 

The objective function is given by

where 
and 
are normalization constants computed from the data set \eqref{eq:data},
and  is a design parameter, allowing us to determine the
trade-off between tracking precision and command activity. Indications
on the choice of  are given in Section \ref{sec:design}.

Note that the objective function \eqref{eq:objf2} is in general non-convex.
Moreover, the optimization problem \eqref{eq:opt2} has to be solved
on-line, and this may require a long time compared to the sampling
time used in the application of interest. In order to overcome these
two relevant problems, a technique is now proposed, allowing a very
efficient computation of the optimal command input .

Since a polynomial basis function expansion has been considered for
, it follows that the objective function 
is a polynomial in . The minima of 
can thus be found considering the roots of its derivative: Define
the set

where  denotes the set of all
real roots of , and  and  are
the boundaries of . The optimal command input is given by

where it has been considered that  depends on the reference
 and regressor . 

The nonlinear controller  is fully defined by the control
law \eqref{eq:opt3}. \medskip{}


\begin{remark}\label{rem:comp}The derivative 
can be computed analytically. Moreover,  is composed by a
``small'' number of elements: 

where card is the set cardinality and deg indicates the polynomial
degree. The evaluation of  through \eqref{eq:opt3} is
thus extremely fast, since it just requires to find the real roots
of a polynomial whose analytical expression is known and to compute
the objective function for a ``small'' number of values. This fact
allows a very efficient controller implementation on real-time devices.\end{remark}\medskip{}


\begin{remark}The system \eqref{ss_sys} is not required to be stable
and in general no preliminary stabilizing controllers are needed.
The only guideline is to generate the data using input signals for
which the system output does not diverge. This can be easily done
for many nonlinear systems like the single-corner model considered
below. Indeed, many systems are characterized by trajectories that
are unstable but bounded (a typical feature of chaotic systems). In
the presence of unbounded trajectories, for which a suitable input
signal can hardly be found, a preliminary stabilizing controller may
be required. The preliminary controller can also be a human operator,
who is able to drive the system within a bounded domain, see \cite{NoFaMiAUT13,movie_dfk}.\end{remark}


\section{Model identification algorithm}

\label{sec:design}

In this section, the algorithm for identifying the model \eqref{eq:model},
required by the nonlinear controller , is proposed. Systematic
criteria for the choice of the involved identification/design parameters
are also provided.

Choose a set of polynomial basis functions . For example,
these functions can be generated as products of univariate polynomials,
where the independent variables are scaled to range in the interval
. In most cases, no large polynomial degrees are required:
we observed in several simulated and real-world applications that
a degree  is sufficient to guarantee satisfactory model
accuracy and control performance.

Define 

where , , and 
and  are the input-output measurements of the data
set \eqref{eq:data}. Consider the set ,
defined as

where  and 
is the set of indexes given by 

and  is the minimum value for which every set 
contains at least two elements.  is defined by a set of linear
inequalities in  and is thus convex in .

The parameter vector 
of the model defined by \eqref{eq:model} and \eqref{eq:bfe} can
be identified by means of the following algorithm, completely based
on convex optimization. Note that the algorithm is ``self-tuning'',
in the sense that all the required parameters are chosen by the algorithm
itself, without requiring extensive trial and error procedures.\medskip{}


\textbf{Algorithm 1}\medskip{}


Initialization: choose a ``low'' model order (e.g. ).\medskip{}

\begin{enumerate}
\item \label{est_delta}Construct the vector  and
matrix  as indicated above.\medskip{}

\item \label{enu:eta}Compute


\item \label{st_1}Solve the optimization problem

where  denotes the th row of the matrix ,
 is the minimum value for which the constraints 
and  are feasible, and  is a real number slightly larger
than .\medskip{}

\item \label{enu:mod_ord}Repeat steps \ref{est_delta}-\ref{st_1} for
increasing model order. Stop when no significant reductions of 
are observed.\medskip{}

\item \label{enu:rho}Repeat steps \ref{st_1}-\ref{enu:mod_ord} for increasing
. Stop when .\medskip{}

\end{enumerate}
The algorithm allows the achievement of three important features:\medskip{}

\begin{enumerate}
\item \textbf{\emph{Closed-loop stability}}. As proven in \cite{NoFaMiAUT13},
under reasonable conditions, constraint  ensures that the function
 has a Lipschitz constant wrt 
non larger than , as . On the other
hand, a theoretical analysis shows that having this constant smaller
than  is a key condition for closed-loop stability. Another required
condition is that , where 
is the input-output gain of the system formed by the cascade connection
of the controller and the model (this latter working in prediction).
Since  can be imposed arbitrarily (see the discussion
below), it can be concluded that \emph{Algorithm 1 is able to ensure
closed-loop stability when the number of data becomes large}.\medskip{}

\item \textbf{\emph{``Small'' tracking error}}. Constraint  is aimed
at providing a model with a ``small'' prediction error (this error,
evaluated on the design data set, is given by ).
According to the mentioned theoretical analysis, reducing this error
allows us to obtain a ``small'' tracking error. Note that there is
a trade-off between stability and tracking performance: In step \ref{enu:rho},
 is increased until the stability condition is met. However,
increasing  causes an increase of the prediction error and,
consequently, of the tracking error.\medskip{}

\item \textbf{\emph{Model sparsity}}. In step \ref{st_1}, the 
norm of the coefficient vector  is minimized, leading to a
sparse coefficient vector , i.e. a vector with a ``small\textquotedblright{}
number of non-zero elements, \cite{Fuchs05,Donoho06_2,Candes06_1,Tropp06}.
Sparsity is important to ensure a low complexity and a high regularity
of the model, limiting at the same time well known issues such as
over-fitting and the curse of dimensionality. Sparsity allows also
an efficient implementation of the model/controller on real-time processors,
which may have limited memory and computation capacities.
\end{enumerate}
\hfill{}\medskip{}


Once a model has been identified, the nonlinear controller 
is obtained by its inversion, as explained in Section \ref{sub:nl_des}.
Only one design parameter needs to be chosen for this inversion: the
weight  in \eqref{eq:objf2}. If no particular requirements
on the activity of the command input  have to be satisfied,
the simplest choice is . Otherwise, if the input activity
has to be reduced, a value  can be chosen, where
 is the maximum value for which the stability condition
 holds.  is the input-output
gain of the system formed by the cascade connection of the controller
and the model. This condition can be checked (approximately) by deriving
an estimate  of  from the data \eqref{eq:data}.
Let

where 

 and  are the input-output measurements
of the data set \eqref{eq:data}, and . The estimate 
can be obtained applying the validation method of \cite{MiNoAUT04}
to the data set \eqref{eq:data_gamma} (the method is summarized in
the Appendix). Observing that 
and thus ,
 must be chosen in such a way that .\medskip{}


\begin{remark}The stability conditions  and 
can give indications on the choice of the control system sampling
time : As discussed in \cite{Goodwin2010}, a too small 
leads to models where . These kinds of
models have a strong dependence on past outputs and a weak dependence
on the input, resulting in large values of  and .
It is thus expected that  and  can be reduced
by increasing . Clearly, to capture the relevant dynamics
of the system and allow a prompt control action,  must be
not ``too large''.\end{remark}


\section{Appendix: Nonlinear Set Membership validation procedure}

\label{appendix1}

In this appendix, the validation method of \cite{MiNoAUT04} is summarized,
suitably adapted for the present setting. This method is useful within
the NIC approach for estimating the constant  appearing
in Sections \ref{sec:design} and \ref{sec:ibc_approach}. 

Suppose that  is the Lipschitz constant of an unknown
function . Let a set of data ,
 be available, described by

where  is a suitable set of indexes and  is
a noise. This noise may also include errors due to the fact that 
is not Lipschitz continuous (e.g. in the case where 
is the sum of a Lipschitz continuous function plus a discontinuous
but bounded function). Assume that , where
 is the  ball with radius ,
and that , where
 is the set of Lipschitz continuous
functions on the domain of  with constant .
Under this assumption, we have that , where
 is the Feasible Function Set.\medskip{}


\begin{definition} \emph{Feasible Function Set}:

 \end{definition}\medskip{}


According to this definition,  is the set of all functions consistent
with prior assumptions and data. As typical in any identification/estimation
theory, the problem of checking the validity of prior assumptions
arises. The only thing that can be actually done is to check if prior
assumptions are invalidated by the data, evaluating if no function
exists consistent with data and assumptions, i.e. if  is empty.\medskip{}


\begin{definition} \label{Def:validation}Prior assumptions are validated
if . \end{definition}\medskip{}


The following result provides necessary and sufficient conditions
for prior assumption validation. Define the function \medskip{}


\begin{theorem} \label{fals}A sufficient condition for prior assumptions
\\
to be validated is: 

 \end{theorem}\medskip{}


\textbf{Proof.} See Theorem 1 in \cite{MiNoAUT04}.\medskip{}


The validation Theorem \ref{fals} can be used for assessing the value
of the Lipschitz constant  so that the sufficient condition
holds. Suppose that  has been chosen by means of any
criterion (e.g. based on some prior knowledge on the noises, or by
means of standard filtering/smoothing techniques, or also using the
dispersion function defined in \cite{MiNoCDC05}). The constant

represents the minimum Lipschitz constant for which the prior assumptions
are validated. A reasonable estimate of  is thus a value
slightly larger than . Note that the evaluation
of  is quite simple, as shown by the examples
in \cite{MiNoAUT04} and \cite{MiNoTAC05}.

\bibliographystyle{IEEEtran}
\bibliography{lettnos_journals,lettnos_conferences,lettaltr,sparsification}

\end{document}
