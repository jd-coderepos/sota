\documentclass[lettersize,journal]{IEEEtran}
\usepackage{amsmath,amsfonts}
\usepackage{algorithmic}
\usepackage[ruled,linesnumbered]{algorithm2e}
\usepackage{array}
\usepackage[caption=false,font=normalsize,labelfont=sf,textfont=sf]{subfig}
\usepackage{textcomp}
\usepackage{stfloats}
\usepackage{url}
\usepackage{verbatim}
\usepackage{graphicx}
\usepackage{cite}

\usepackage{multirow}
\usepackage{hyperref}
\usepackage[capitalize]{cleveref} \crefname{section}{Sec.}{Secs.}
\Crefname{section}{Section}{Sections}
\Crefname{table}{Table}{Tables}
\crefname{table}{Tab.}{Tabs.}
\newcommand{\myPara}[1]{\vspace{.05in}\noindent\textbf{#1}}
\newcommand{\etal}[0]{\emph{et al.}}
\usepackage{tikz}
\newcommand*{\circled}[1]{\lower.7ex\hbox{\tikz\draw (0pt, 0pt)circle (.5em) node {\makebox[1em][c]{\small #1}};}}

\hyphenation{op-tical net-works semi-conduc-tor IEEE-Xplore}


\begin{document}

\title{MotionTrack: Learning Motion Predictor for Multiple Object Tracking}

\author{Changcheng~Xiao, Qiong~Cao, Yujie~Zhong, Long~Lan, Xiang Zhang, Huayue~Cai, Zhigang~Luo, Dacheng Tao,~\IEEEmembership{Fellow,~IEEE.}
\thanks{Changecheng Xiao, Huayue Cai and Zhigang Luo are with College of Computer Science, National University of Defense Technology, Changsha 410073, China (email: xiaochangcheng16@nudt.edu.cn; huayue\_cai@163.com; zgluo@nudt.edu.cn).}\thanks{Long Lan and Xiang Zhang are with Institute for Quantum Information \& State Key Laboratory of High Performance Computing (HPCL), National University of Defense Technology, Changsha 410073, China (email: long.lan@nudt.edu.cn; zhangxiang08@nudt.edu.cn).}
\thanks{Qiong Cao and Dacheng Tao are with JD Explore Academy, Beijing 102628, China (email: mathqiong2012@gmail.com; dacheng.tao@gmail.com).}
\thanks{Yujie Zhong is with Meituan Inc., Beijing 100000, China (email: jaszhong@hotmail.com).}
\thanks{\textit{Corresponding author: Qiong Cao; Long Lan.}}
}

\markboth{Journal of \LaTeX\ Class Files,~Vol.~14, No.~8, August~2021}{Shell \MakeLowercase{\textit{et al.}}: A Sample Article Using IEEEtran.cls for IEEE Journals}

\IEEEpubid{0000--0000/00\n_{past} \textbf{T} = \{\textbf{b}_{t_1}, \textbf{b}_{t_2,}, \dots \} \textbf{b}_t = \{ x, y, w, h\} t \textbf{D}_t tn_{past}\textbf{X} \boldsymbol{\hat{O}}_t\textbf{X} = ( \dots, \textbf{x}_{t-2}, \textbf{x}_{t-1}) \in \mathbb{R}^{n \times 9}t-1\textbf{x}_{t-1}(c_x, c_y)wha\delta _{c_x}\delta _{c_y}\delta_w\delta_h\textbf{X} \Bar{\textbf{X}} = \textbf{W}_x\textbf{X} \in \mathbb{R}^{n \times d_m} \textbf{E} \in \mathbb{R}^{n \times d_m}\textbf{E}QKVd_khhead_0, head_1,\dots, head_i, \dots, head_hQ_iK_iV_ii_{th}QKVh \textbf{E} \in \mathbb{R}^{n \times d_m}e_i \in \mathbb{R}^{d_m}d_m\Delta = \{ \delta_i \}_{i=1}^{d_m}\textbf{W} \in \mathbb{R}^{d_m \times d_m} \textbf{b} \in \mathbb{R}^{d_m}\hat{e}^I_i\hat{e}^T\hat{e}^I\odot\omega^{\{T,I\}} \in \mathbb{R}^{d_m}\dot{x} \in \mathbb{R}^{d_m}\hat{e}^T\hat{e}^IW^{\{I,T\}} \in \mathbb{R}^{d_m \times d_m}softmax(\cdot)llL\textit{p}_i\textit{n}_{past} \boldsymbol{\hat{O}} = \{ \delta _{c_x}, \delta _{c_y}, \delta_w, \delta_h \} \boldsymbol{O} \hat{\textbf{O}}_t\hat{\textbf{D}}_t\hat{\textbf{D}}_t\textbf{D}_t\textit{D}= \{\textbf{b}_t^i| 1 \leq t \leq M, 1 \leq i \leq N_t\}\texttt{MP}t_{max}\textit{T}\textit{T} \leftarrow \emptyset t \leftarrow 1:M\textbf{D}_t \leftarrow [\textbf{b}_t^1, \cdots, \textbf{b}_t^{N_t}]\hat{\textbf{D}}_t \leftarrow [\hat{\textbf{b}}_t^1, \cdots, \hat{\textbf{b}}_t^{|\textit{T}|}]\textit{T}\mathbf{C}_t \leftarrow C_{\rm IoU} (\hat{\textbf{D}}_t, \textbf{D}_t)\mathcal{M},\mathcal{T}_u,\mathcal{D}_u \gets \rm assignment(\mathbf{C}_t)\textit{T} \leftarrow \{T_i(\textbf{b}_t^j), \forall (i,j) \in \textit{M}\}\textit{T} \leftarrow \{T_i.age += 1, \forall (i) \in \textit{T}_u\}\textit{T} \leftarrow \{T_i(D_j), \forall {j} \in \textit{D}_u, i=|\textit{T}| + 1 \}\geq t_{max}T\textit{T}TL=6d_m\textit{p}_i = 0.1\textit{n}_{past}\beta_1 = 0.9\beta_2 = 0.98\epsilon = 10^{-8}^{\text{*}}^{\text{*}}^{\text{*}}\textit{n}_{past}\xrightarrow{}\xrightarrow{}\xrightarrow{}p_ip_ip_ip_ip_ip_in_{past}n_{past}$ & HOTA↑  & DetA↑  & AssA↑  & MOTA↑  & IDF1↑  \\
\hline
3   & 51.6 & 78.2  & 34.2 & \textbf{89.2} & 51.2  \\
5   & 52.3 & 78.2  & 35.2 & \textbf{89.2} & 52.8 \\
10  & \textbf{54.6}  & \textbf{78.6}  & \textbf{38.1}  & \textbf{89.2}  & \textbf{54.6}  \\
13  & 54.2 & 78.3  & 37.8  & 89.1  & 54.1  \\
15  & 53.3 & 78.4 & 36.4 & 89.1 & 53.3 \\
\hline
\end{tabular}
\label{table:window_size.}
\end{table}




\section{Limitations}
While our approach acquires simplicity and generality by relying solely on motion modeling, we have also identified certain challenges associated with this paradigm. We observe that our model struggles to handle video sequences taken from the first view, especially when superimposing fast camera movements and long-term object loss. We present two typical examples of failure in \cref{fig:failure_cases}. 




\section{Conclusion}
In this paper, we propose a novel online tracker that utilizes a learnable motion predictor. The proposed approach incorporates two distinct modules to capture information at varying levels of granularity, thus enabling efficient modeling of an object's temporal dynamics. Specifically, we employ Transformer networks for motion prediction, which are capable of capturing complex motion patterns at the token level. We further introduce DyMLP, a MLP-like architecture, to extract semantic information distributed across different channels. By integrating DyMLP with the self-attention module in Transformer, our approach leverages the dual-granularity information to achieve superior performance. The results of our experiments have illustrated that the proposed methodology surpasses current state-of-the-art techniques when applied to datasets that feature intricate motion scenarios. Additionally, the performance of the proposed method has exhibited its resilience in the face of complex motion patterns.

\begin{figure*}[ht]
  \centering
  \includegraphics[width=1.0\linewidth]{figures/sportsval.pdf}
   \caption{Qualitative results of our method on SportsMOT. Different colored bounding boxes indicate different identity. Best viewed in color and zoom in.}
  \label{fig:sportsval} 
\end{figure*}



\small
\bibliographystyle{IEEEtran}
\bibliography{./ref}


 
\vspace{11pt}





















\vfill

\vspace{11pt}


\end{document}
