\documentclass[
final
]{dmtcs-episciences}






\newtheorem{theorem}{Theorem}{\bfseries}{\itshape}
\newtheorem{lemma}{Lemma}{\bfseries}{\itshape}
\newtheorem{proposition}{Proposition}{\bfseries}{\itshape}
\newtheorem{remark}{Remark}{\bfseries}{\itshape}
\newtheorem{corollary}{Corollary}{\bfseries}{\itshape}
\newtheorem{claimN}{Claim}{\bfseries}{\itshape}

\newcommand{\yes}{YES }
\newcommand{\no}{NO }

\newcommand{\probl}[3]{
\begin{flushleft}
\fbox{
\begin{minipage}{11.8cm}
\noindent {\sc #1}\\
          {\bf Input:} #2\\
          {\bf Output:} #3
\end{minipage}}
\medskip
\end{flushleft}
}
\newcommand{\paraprobl}[4]
{
  \begin{flushleft}
    \fbox{
      \begin{minipage}{14.25cm}
        \noindent {\textsc {#1}}\\
        {\bf Input:} #2\\
        {\bf Parameter:} #4\\
        {\bf Question:} #3
      \end{minipage}
    }
  \end{flushleft}
}



\usepackage[utf8]{inputenc}
\usepackage{subfigure,cite,amsmath,dsfont}





\author{Luerbio Faria\affiliationmark{1}
  \and Sulamita Klein\affiliationmark{2}
  \and Ignasi Sau\affiliationmark{3,4}
  \and Rubens Sucupira\affiliationmark{1}}
\title[Improved Kernels for \textsc{Signed Max Cut ATLB} on -graphs]{Improved Kernels for Signed Max Cut Parameterized Above Lower Bound on -graphs\thanks{This work was partially supported by CNPq, CAPES, FAPERJ, and COFECUB.}}
\affiliation{
Instituto de Matem\'atica e Estat\'{i}stica - UERJ and  COPPE/Sistemas - UFRJ, Rio de Janeiro, Brazil\\
  Instituto de Matem\'atica and COPPE/Sistemas - UFRJ, Rio de Janeiro, Brazil\\
  CNRS, LIRMM, Universit\'e de Montpellier, Montpellier, France\\
  Departamento de Matem\'atica, Universidade Federal do Cear\'a, Fortaleza, Brazil}
\keywords{max cut, -graphs,  split graphs, parameterized complexity, parameterization above lower bound, polynomial kernels}


\received{2016-7-20}
\accepted{2017-5-2}
\revised{2017-3-5}

\begin{document}
\publicationdetails{19}{2017}{1}{14}{1540}
\maketitle
\begin{abstract}
  A graph  is {\it signed} if each edge is assigned ``'' or ``''.
A signed graph is {\it balanced} if
there is a bipartition of its vertex set such that
an edge has sign ``'' if and only if its endpoints are in different parts.
The Edwards-Erd\H{o}s bound states that
every signed graph with  vertices and  edges has a balanced subgraph with at least
 edges. In the {\sc Signed Max Cut Above Tight Lower Bound}
({\sc Signed Max Cut ATLB}) problem, given a signed graph  and a  parameter
, the question is whether  has a balanced subgraph with at least
 edges. This problem generalizes \textsc{Max Cut Above Tight Lower Bound}, for which a kernel with  vertices was given by Crowston et al.~[ICALP 2012, Algorithmica 2015]. Crowston et al. [TCS 2013] improved this result by providing a kernel with  vertices for the more general {\sc Signed Max Cut ATLB} problem. In this article we are interested in improving the size of the kernels for {\sc Signed Max Cut ATLB} on restricted graph classes for which the problem remains hard.  For two integers , a graph  is an \emph{-graph} if  can be partitioned into  independent
 sets and  cliques. Building on the techniques of Crowston et al. [TCS 2013], for any  we provide a kernel with  vertices on -graphs, and a simple linear kernel on subclasses of split graphs for which we prove that the problem is still {\sf{NP}}-hard.
\end{abstract}


\section{Introduction}
\label{sec:intro}

A graph  is a {\it{signed graph}} if each edge is assigned {\it{positive}}  or {\it{negative}} . In this paper all graphs will be signed graphs. The labels ``'' or ``'' are the {\it{signs}} of the corresponding edges. If  is a bipartition of the vertex set of a graph , we say that  is {\it{-balanced}} if the edges with both endpoints in  or both endpoints in  are positive and the edges with endpoints in distinct parts are negative. A graph  is {\it{balanced}} if there exists a partition  of  such that  is -balanced. (Note that we can use an algorithm for {\sc 2-sat} to determine whether a signed graph is balanced.)
	
The problem of finding a balanced subgraph of a signed graph with maximum number of edges is called {\sc{Signed Max Cut}} and is {\sf{NP}}-hard, as it is a generalization of {\sc{Max Cut}}, which is well known to be {\sf{NP}}-hard~\cite{garey1976some} (indeed, {\sc{Max Cut}} is exactly {\sc{Signed Max Cut}} when all edges of  are negative). We refer to the work of Crowston et al.~\cite{crowston2013maximum} and the references therein for some applications of the {\sc{Signed Max Cut}} problem.


In this article we study {\sc{Signed Max Cut}} from the perspective of Parameterized Complexity; we refer to the monographs~\cite{FG06,Nie06,DF13,CyganFKLMPPS15} for an introduction to the field. Namely, we consider a so-called \emph{parameterization above lower bound} of the problem, and we are particularly interested in obtaining small polynomial kernels on restricted graph classes. Let us first discuss \textsc{Max Cut}. Edwards~\cite{edwards, edwards2} showed that any connected
graph with  vertices and  edges\footnote{We assume that all input graphs of the problems under consideration have  vertices and  edges.}  has a cut of size , and that this value is \emph{tight} (that is, that there exist infinitely many graphs for which the size of a maximum cut equals this bound). This bound is commonly known as the {\it{Edwards-Erd{\H{o}}s}} bound, and justifies the following parameterization of \textsc{Max Cut} recently considered by Crowston et al.~\cite{crowston2012max}: given a graph  and a parameter , decide whether  has a cut of size at least . They provided {\sf{FPT}}-algorithms for this parameterization of the problem and a kernel of size  on general graphs.




	
Coming back to the {\sc{Signed Max Cut}} problem, Poljak and Turz\'{\i}k \cite{poljak1986polynomial} proved that  is also a tight lower bound on the number of edges in a balanced subgraph of any signed graph  with  connected components; we denote this lower bound by .
Motivated by the above result for \textsc{Max Cut}, Crowston et al. \cite{crowston2013maximum} considered the following parameterization of {\sc{Signed Max Cut}}, which is the same we consider in this article:



\vspace{.15cm}
\paraprobl
{\textsc{Signed Max Cut Above Tight Lower Bound}}
{ A connected signed graph   and a positive integer .}
{Does  contain a balanced subgraph with at least  edges?}
{.}
\vspace{.15cm}	






We call the above problem {\sc{Signed Max Cut ATLB}} for short. Crowston et al.~\cite{crowston2013maximum} proved that {\sc{Signed Max Cut ATLB}} is also {\sf{FPT}}  and provided a kernel of size
  on general graphs, therefore improving the kernel of size
  for the particular case of {\sc{Max Cut ATLB}} given by Crowston et al.~\cite{crowston2012max}.

\vspace{.15cm}\noindent
\textbf{Our results}. In this article we are interested in improving the size of the cubic kernel for {\sc{Signed Max Cut ATLB}} by Crowston et al.~\cite{crowston2013maximum} on particular graph classes for which the problem remains {\sf{NP}}-hard. In particular, we focus on -graphs. For two integers , a graph  is an \emph{-graph} if  can be partitioned into  independent sets and  cliques. These graph classes contain, for example, split graphs or bipartite graphs, and have been extensively studied in the literature~\cite{FoHa77,Golumbic04,book-graph-classes,Bra96,FHKM03,BFKS15,KoPa15}. In particular,
it is known that the recognition of -graphs is {\sf NP}-complete if and only if ~\cite{Bra96,FHKM03}.


Note that for this parameterization of  \textsc{Signed Max Cut} to make sense on -graphs, the bound  should be also {\sl tight} on these graph classes. Fortunately, this is indeed the case: let  be a clique with odd number of vertices, which is clearly an -graph as long as  (in particular, a split graph), and such that all edges are negative. It is clear that the largest balanced subgraph of  is given by the crossing edges of any bipartition  of  with  and . The number of edges of such a balanced subgraph, namely  , equals  .


From a parameterized complexity perspective, split graphs have received considerable attention. For instance, Raman and Saurabh~\cite{RamanS08} proved that \textsc{Dominating Set} is {\sf{W[2]}}-hard on connected split graphs, and Ghosh~et al.~\cite{GhoshK0MPRR15} provided  {\sf{FPT}}-algorithms for the problem of deleting at most  vertices to obtain a split graph. Concerning kernelization, Heggernes et al.~\cite{HeggernesHLS15} studied the \textsc{Disjoint Paths} problem on split graphs parameterized by the number of pairs of terminals, and provided a kernel of size  (resp. ) for the vertex-disjoint (resp. edge-disjoint) version of the problem. As for general -graphs from a parameterized point of view, recently a dichotomy on the parameterized complexity for the problem of deleting at most  vertices to obtain an -graph has been independently obtained by Baste et al.~\cite{BFKS15} and by Kolay and Panolan~\cite{KoPa15}. To the best of our knowledge, this is the first article that focuses on obtaining polynomial kernels on -graphs for arbitrary values of  and .



Our main result (Theorem~\ref{thm:quadratic-general} in Section~\ref{sec:quadratic-kernel}) is that {\sc{Signed Max Cut ATLB}} admits a quadratic kernel on -graphs for every fixed . More precisely, the kernel has  vertices. Our techniques in order to prove Theorem~\ref{thm:quadratic-general} are strongly based on the ones used by Crowston et al.~\cite{crowston2013maximum}, and for improving their cubic bound we further exploit the structure of -graphs in the analysis of the algorithm. In fact, our kernelization algorithm consists in applying exhaustively the reduction rules of Crowston et al.~\cite{crowston2013maximum}, only the analysis changes. As when   the recognition of -graphs is {\sf NP}-complete~\cite{Bra96,FHKM03}, we stress here that, given an instance , we do {\sl not} need to obtain any partition of  into  independent sets and  cliques; we only use the {\sl existence} of such partition in the analysis. Since the analysis of the quadratic kernel for the particular case of split graphs is simpler, we prove it separately in Subsection~\ref{ap:quadratic-split}.

In Section~\ref{sec:linear-kernel} we present a linear kernel for {\sc{Signed Max Cut ATLB}} on subclasses of split graphs. Namely, these subclasses are what we call \emph{-split graphs} for every integer  (see Section~\ref{sec:linear-kernel} for the definition). We first prove  that even \textsc{Max Cut} is {\sf{NP}}-hard on -split graphs for every integer , and in Theorem~\ref{thm:linear-d-split} we provide the linear kernel. As discussed later, this kernelization algorithm is the simplest possible algorithm that one could imagine, as it does {\sl nothing} to the input graph; its interest lies on the analysis of the kernel size, which is non-trivial. In particular, the analysis uses a new reduction rule that we introduce in Section~\ref{s2}.




\section{Preliminaries}
\label{s2}
	
We use standard graph-theoretic notation; see for instance Diestel's book~\cite{Diestel05}. All the graphs we consider are undirected and contain neither loops nor multiple edges. If , we define  and . A graph  is a {\it{split graph}} if there is a partition of  into an independent set  and a clique . Split graphs can be generalized as follows. Let  be two positive integers. A graph  is an {\it{-graph}} if  can be partitioned into at most  independent sets   and at most  cliques .






Given a signed graph , a cycle  in  is called {\it{positive}} if the number of negative edges in  is even. Otherwise  is called {\it{negative}}.
Harary~\cite{harary1953}  proved that the absence of {\it{negative}} cycles characterizes balanced graphs.
	
\begin{theorem}[Harary~\cite{harary1953}]\label{theorem2.1}
A signed graph  is balanced if and only if every cycle in  is positive.	
\end{theorem}
	

The following lemma by Crowston et al.~\cite{crowston2013maximum} is very useful for our purposes, as it gives a lower bound on the maximum size of a balanced signed subgraph of a graph. Let  be a signed graph and let . We denote by  the subset of  formed by the edges that have one endpoint in  and the other in . We also let   denote the maximum number of edges in a balanced subgraph of .

	\begin{lemma}[Crowston et al.~\cite{crowston2013maximum}]
 Let  be a connected signed graph and let  such that , , and . Then . In addition, if  has  components,  has  components, , and , then .  	 \label{beta}
	\end{lemma}
	


	We say that a problem is {\it{fixed-parameter tractable}} ({\sf{FPT}})~\cite{FG06,Nie06,DF13,CyganFKLMPPS15} with respect to parameter 
if there exists an algorithm that solves the problem  in time , where  is a computable function of  which is independent
of .
{\it{Kernelization}} is an important
technique used to shrink the size of a given problem instance by means of
polynomial-time data reduction rules until the size of this instance is bounded by a function of
the parameter . The reduced instance is called a problem {\it{kernel}}. Once a problem
kernel is obtained, we know that the problem is fixed-parameter tractable, since
the running time of any brute force algorithm depends on the parameter  only.
The converse is also true: whenever a parameterized problem is {\sf{FPT}}, then it admits a kernel~\cite{FG06,Nie06,DF13,CyganFKLMPPS15}. The natural question to be asked about kernels is whether a parameterized problem admits a {\em polynomial} kernel or not, that is, a kernel of size .

	In their article, Crowston et al.~\cite{crowston2013maximum} proved that {\sc{Signed Max Cut ATLB}} is {\sf{FPT}} on general graphs by designing an algorithm  running in  time . The algorithm applies some reduction rules to the input  that either answer that  is a {\sc{Yes}}-instance,  or produce a set  of at most  vertices such that  is a forest of cliques, or equivalently \emph{clique-forest}, which is a graph such that every 2-connected component (that is, each block) is a clique without positive edges. (Note that this definition differs from the usage of this term in the context of chordal graphs.) We state this property formally as follows.

\begin{proposition}[Crowston et al.~\cite{crowston2013maximum}]
\label{coro4.1}\label{coroker}
Let  be an instance of {\sc{Signed Max Cut ATLB}}. In polynomial time we can conclude that  is a {\sc{Yes}}-instance or we can find a set  of at most  vertices such that  is a clique-forest without positive edges.
\end{proposition}



	There are two kinds of reduction rules applied by Crowston et al.~\cite{crowston2013maximum} to an instance  in order to obtain the kernel of size  for {\sc{Signed Max Cut ATLB}}: one-way reduction rules and two-way reduction rules. In a \emph{two-way reduction rule}, the instance  produced by the reduction rule is {\sl equivalent} to  (that is,  is a {\sc{Yes}}-instance if and only if
 is a {\sc{Yes}}-instance), so these rules can be safely applied to any instance in order to obtain {\sf{FPT}}-algorithms or kernels, as long as the parameter  does not increase, which is always the case in the rules defined by Crowston et al.~\cite{crowston2013maximum}. A two-way reduction rule is {\it{valid}} if it transforms {\sc{Yes}}-instances into {\sc{Yes}}-instances and {\sc{No}}-instances into {\sc{No}}-instances, that is, if it works as it should.


 On the other hand, in a \emph{one-way reduction rule}, the instance  produced by the reduction rule does not need to be equivalent to , but only needs to satisfy that if  is a {\sc{Yes}}-instance, then  is a {\sc{Yes}}-instance as well. The usefulness of such rules relies on the fact that if after the application of some two-way or one-way reduction rules we obtain an instance  with , we can safely conclude that the original instance  is a {\sc{Yes}}-instance. This fact will be used in the linear kernel provided in Section~\ref{sec:linear-kernel}. A one-way reduction rule is {\it{safe}} if it does not transform a {\sc{No}}-instance into a {\sc{Yes}}-instance.











\section{A quadratic kernel on -graphs}
\label{sec:quadratic-kernel}

In this section we show that {\sc{Signed Max Cut ATLB}} admits a quadratic kernel on -graphs for any fixed integers . For this we strongly use the results and the techniques provided by Crowston et al.~\cite{crowston2013maximum}. The kernelization algorithm applies to the instance  the algorithm given by Proposition~\ref{coro4.1}. This algorithm uses the two-way reduction rules Rule~8, Rule~9, Rule~10, and Rule~11 defined by Crowston et al.~\cite{crowston2013maximum}. Namely, when these rules cannot be applied anymore, either one can directly conclude that  is a  \textsc{Yes}-instance or, using that the graph cannot be reduced anymore, one can prove that its size is bounded by a function of  only. It is at this point when we exploit the structure of -graphs in order to improve the cubic kernel on general graphs and obtain a quadratic kernel on -graphs. We would like to stress that we do {\sl not} need to obtain algorithmically any particular partition of  into  independent sets and  cliques, as such a partition will only be used for the {\sl analysis}, and not by the kernelization algorithm.



This section is organized as follows. For completeness, in Subsection~\ref{sec:rules} we state all the reduction rules and the results from Crowston et al.~\cite{crowston2013maximum} that we need for our purposes. For the sake of providing more intuition on the ideas behind our kernel and because it is simpler, we first present in Subsection~\ref{ap:quadratic-split} the quadratic kernel on split graphs (that is, for ), and we describe in Subsection~\ref{subsec:quadratic-general} its generalization to arbitrary values of  and . The main difference is that, in the case of split graphs, it can be seen that we do not have to worry about path vertices in .

\subsection{Reduction rules and known results}
\label{sec:rules}


Before stating the two-way reduction rules used in the kernelization algorithm, we first need to introduce some definitions from~\cite{crowston2013maximum}. Let  be a set of vertices as in Proposition~\ref{coro4.1}. For a block  in , let   be the {\it{interior}} of , and let  be the {\it{exterior}} of .
If a block  has only two vertices and these vertices belong to , then  is called a {\it{path block}}. A vertex that belongs only to path blocks is called a {\it{path vertex}}. A block  in  is called a {\it{leaf block}} if . We denote by  (resp. ) the set of neighbors (of a vertex or of a vertex set) adjacent via a positive (resp. negative) edge.



In the following rules, the main idea is that, if some simple local conditions are satisfied, then certain vertices or edges can be deleted from the graph without changing the answer to the problem. Most of these conditions concern blocks or connected components of  and their neighborhoods in .

\vspace{.3cm}

\noindent	{\bf{Rule 8.}}{ \it{Let  be a block in . If there exists  such that ,  and  for all , then delete two arbitrary vertices  and set .}}

\vspace{.3cm}

\noindent	{\bf{Rule 9.}}{ \it{Let  be a block in . If  is even and there exists  such that  and , then delete a vertex , and set .}}

\vspace{.3cm}
	
\noindent	{\bf{Rule 10.}}{ \it{Let  be a block in  with vertex set , such that . If the edge  is a bridge in , delete , add a new vertex , positive edges , negative edges , and set . Otherwise, delete  and the edge  and set .}}

\vspace{.3cm}

An illustration of the application of Rule~10 can be found in Fig.~\ref{fig:Rule10}.

\begin{figure}[h!]
\begin{center} \includegraphics[width=.45\textwidth]{pontexy.pdf}\\
\vspace{.4cm}
  \includegraphics[width=.45\textwidth]{removeCponte.pdf}\hspace{1cm}
\includegraphics[width=.45\textwidth]{regra10cc.pdf}
\caption{Illustration of the application of Rule 10: a block  satisfying the conditions (up), and the two possible resulting graphs after applying the rule (down).} \label{fig:Rule10} \end{center}
\end{figure}

\vspace{.3cm}
The \textsc{Max Cut with Weighted Vertices} problem is defined as follows. We are given a graph  with weight functions  and , and an integer , and the question is whether there exists an assignment  such that .
\vspace{.3cm}

	
\noindent	{\bf{Rule 11.}}{ \it{Let  be a connected component of  only adjacent to a vertex . Form a \textsc{Max Cut with Weighted Vertices} instance on  by defining  if  ( otherwise) and  if  ( otherwise). Let . Then delete  and set .}}

\vspace{.3cm}



By~\cite[Lemma 9]{crowston2012max}, the value of  in Rule~11 can be found in polynomial time.

\vspace{.3cm}

Crowston et al.~\cite{crowston2013maximum} proved that the two-way reduction Rules~8-11 are all valid. As mentioned in~\cite{crowston2013maximum}, since it can be easily verified  that none of these  reduction rules increases the number of positive edges, our proof also implies  a kernel of size  for {\sc{Signed Max Cut ATLB}} on -graphs.



We now state the results from Crowston et al.~\cite{crowston2013maximum} that we will use in the proofs of Theorem~\ref{thm:quadratic-split} and Theorem~\ref{thm:quadratic-general}.


\begin{lemma}[Lemma 9 in Crowston et al.~\cite{crowston2013maximum}]\label{Crow-Lemma-9}
Let  be a connected component of . Then for every leaf block  of , . Furthermore, if , then  consists of a single vertex.
\end{lemma}


\begin{lemma}[Corollary 5 in Crowston et al.~\cite{crowston2013maximum}]\label{Crow-Coro-5}
If , the instance is a \textsc{Yes}-instance. Otherwise, .
\end{lemma}


\begin{lemma}[Corollary 6 in Crowston et al.~\cite{crowston2013maximum}]\label{Crow-Coro-6}
  contains at most  non-path blocks and 
path vertices.
\end{lemma}

\begin{lemma}[Corollary 7 in Crowston et al.~\cite{crowston2013maximum}]\label{Crow-Coro-7}
 contains at most  vertices in the exteriors of
non-path blocks.
\end{lemma}

\begin{lemma}[Lemma 14 in Crowston et al.~\cite{crowston2013maximum}]\label{Crow-Lemma-14}
For a block , if ,
then  is a \textsc{Yes}-instance. Otherwise, .
\end{lemma}

\subsection{A quadratic kernel on split graphs}
\label{ap:quadratic-split}

In this subsection we present a quadratic kernel for {\sc{Signed Max Cut ATLB}} on split graphs, which contains the main ideas of the kernel described in Theorem~\ref{thm:quadratic-general} for arbitrary -graphs. The main simplification is that in the case of split graphs, we do not have to worry about path vertices in .

The proof is based on exploiting the structure given by the set  of Proposition~\ref{coro4.1}. We will strongly use the fact that, as  the input graph is a split graph and this property is hereditary, both  and  are split graphs as well.

     \begin{theorem}\label{thm:quadratic-split}
	{\sc{Signed Max Cut ATLB}} on split graphs admits a kernel with  vertices.
	\end{theorem}
     \begin{proof}
     Let  be the given instance, where  is a split graph. Applying the polynomial-time algorithm of Proposition~\ref{coro4.1}, we can either conclude that  is a \textsc{Yes}-instance, or we obtain a set  such that  is a clique-forest without positive edges. Let  and  be such that  is a clique and  is an independent set. We partition  into three subsets:  containing only isolated vertices in ,  with all vertices of degree one in , and  containing all vertices  such that ; see Fig.~\ref{fig:quadratic-split} for an illustration. Note that  can contain at most one vertex. Indeed, if , then the set  is empty by definition. Otherwise, if , if  contains two distinct vertices  and , then  is not a clique-forest, contradicting Proposition~\ref{coro4.1}.

\begin{figure}[h]
\begin{center} \includegraphics[width=.45\textwidth]{theorem5.pdf}
\caption{Illustration of the proof of Theorem~\ref{thm:quadratic-split}.} \label{fig:quadratic-split} \end{center}
\end{figure}


     These sets indeed define a partition of , as  if there exists a vertex  in  such that , we could find a block in  that is not a clique. On the other hand, due to the structure of the clique-forest, two maximal cliques can intersect only in one vertex.



     Observe that  has at most one path block. Indeed, it is easily seen that the only possible case in which a path block exists in  is when  (the complete graph with two vertices), , and each vertex of  has at least one adjacent vertex in . Hence,  has no path vertices since in  there is only one possible path block (namely, ) adjacent to at least two leaf blocks; see Fig.~\ref{fig:path-block} for an illustration.


     \begin{figure}[h!]
\begin{center} \includegraphics[width=.35\textwidth]{blococaminho1.pdf}
\caption{Illustration of the unique possible path block in .} \label{fig:path-block} \end{center}
\end{figure}

\newpage 

     For each vertex  in the set  there is a leaf block  with  and . So, Rule 9 could be successively applied to each vertex in  with no neighbor in , deleting all these vertices and reducing the parameter  accordingly.	We could also delete all the vertices of  with no neighbor in , because they are isolated vertices and make no contribution to the cut. So in this case we set .

     We note that Rule 10 could be applied only in two cases. In the first case  and  is a triangle, and in the second case, , , and .

	Using Lemma~\ref{Crow-Lemma-9}, it can be easily seen that after applying Rule 9, the remaining vertices of  are adjacent to some vertex in . Indeed, otherwise  Rule~9 could be applied again, deleting the vertices  of  with , contradicting the hypothesis that the graph  is reduced under all two-way reduction rules; see Fig.~\ref{fig:blocosfolha} for an illustration.


     \begin{figure}[h!]
\begin{center} \includegraphics[width=.38\textwidth]{blocosfolha.pdf}
\caption{The set , if it were non-empty, would be deleted by Rule 9.} \label{fig:blocosfolha} \end{center}
\end{figure}

	Let  be the set of non-path blocks of . As discussed before,  in a split graph   all blocks of , but possibly one, are non-path blocks. Since if there is some path block, we have that , we may assume in what follows that all blocks of  are in .



By Lemma~\ref{Crow-Coro-5}, either we can identify that  is a {\sc{Yes}}-instance or it follows that . Using Lemma~\ref{Crow-Lemma-14} it follows that for all blocks  in , we have

	Then  since we may assume that are no path blocks in . As  is a split graph,  is also a split graph and our previous analysis shows that  is composed by one big block  defined by , leaf blocks that are edges which have at least one neighbor in , and possibly isolated vertices. Let  be the set of leaf blocks in . We can rewrite the previous inequality as


\newpage
In particular, for the big block  we have . And for all leaf blocks   we have .
	Then

as a consequence of Lemma~\ref{Crow-Coro-5}, since every block  has a vertex  such that .
	Finally, the number of vertices in  can be bounded as follows:

where the last inequality is due to Lemma~\ref{Crow-Coro-7}.
\end{proof}



\subsection{Generalization to arbitrary -graphs}
\label{subsec:quadratic-general}

In this subsection we show how the ideas of the previous subsection can be generalized to arbitrary -graphs. We follow the same strategy as in the proof of Theorem~\ref{thm:quadratic-split}, but we need a number of extra arguments, mostly to deal with the path vertices in .

Namely, we will again exploit the structure given by the set  of Proposition~\ref{coro4.1}, and the fact that, as  the input graph is an -graph and this property is hereditary, both  and  are -graphs as well. In order to bound the size of the set of path vertices in , which we call  in the proof, we will need three technical claims.


\begin{theorem}\label{thm:quadratic-general}
For every two integers , \textsc{Signed Max Cut ATLB} on -graphs admits a kernel with  vertices.
\end{theorem}
\begin{proof}
Let  be an -graph for two integers . Denote by  the cliques and  the independent sets of .  We stress again that we just use the partition of  for the analysis, but we do not need to know exactly this partition.  By Proposition~\ref{coro4.1}, we can find a set  with at most  vertices such that the subgraph  is a forest of cliques without positive edges. For an -graph, we have  and  with  and  for  and . In contrast to split graphs (see  Subsection~\ref{ap:quadratic-split}), we may have path vertices in .

In order to bound the size of , we bound separately the total size of non-path blocks and the number of path vertices (that is, the only vertices that do not belong to non-path blocks) in . Denote again by  the set of non-path blocks, by  the set of non-path blocks containing at least two vertices  from the set , and .
	The structure of clique-forests implies that , and by Lemma~\ref{Crow-Lemma-14}, for every block  either we can conclude that  is a \textsc{Yes}-instance or we have that 

Then we can write

where in the second inequality we have used twice that , and the last inequality follows from Lemma~\ref{Crow-Coro-7}.


For each block  we have , as  contains only vertices of the  independent sets  plus possibly one vertex of some of the cliques. Indeed, otherwise if   then there are two adjacent vertices in the same independent set, which is impossible. Therefore, by~Lemma~\ref{Crow-Coro-6}, we can bound  the sum of the number of vertices in  as .

Therefore, by combining the bounds from the two above paragraphs we conclude that


It just remains to bound the number of path vertices in the blocks of . Note that we cannot use directly the bound on the number of path vertices in  given by~Lemma~\ref{Crow-Coro-6}, as this bound is , which is too much for our purposes.

Let  be the set of path vertices of . In order to bound the size of , we argue separately about the components in  having at most two vertices (Claim~\ref{lemma4.8}) and the remaining components (Claim~\ref{claim:2-3}), where  denotes the subgraph of  induced by .

We call an {\it{isolated vertex}} (resp. \emph{isolated edge}) a connected component of  of size 1 (resp. 2).
We show in Claim~\ref{lemma4.8}  that the number of isolated vertices and isolated edges in  are both bounded by the number of non-path blocks, which is at most  by~Lemma~\ref{Crow-Coro-6}.

\begin{claimN}\label{lemma4.8}
The number of isolated vertices in  and the number of isolated edges in  are upper-bounded by the number of non-path blocks.	
	\end{claimN}
	\begin{proof}
	Let us first deal with the number of isolated vertices in , which we denote by . We show by induction on  that this number is at most the number of non-path blocks minus one. Recall that a vertex is isolated if and only if it is exclusively adjacent to non-path vertices. We contract each non-path block into a  black vertex and represent each path vertex by a white vertex. So the forest of cliques in  is now represented by a simple forest colored by two colors: black and white.
If , then there are at least two black vertices adjacent to the single white vertex of . So the property is valid and the basis of the induction is proved. Suppose that the property is valid for , and consider  such that the number of isolated vertices is . Choose arbitrarily one of these isolated vertices. Without loss of generality, we can suppose that  is connected. Otherwise we can deal with the connected components of  separately to achieve the same result. The removal of this white vertex disconnects the tree, increasing the number of connected components. For all connected components the property holds by induction hypothesis, that is, the number of isolated vertices in each connected component is at most the number of non-path blocks in this component minus one. So, the number of isolated vertices in the original graph is at most the total number of non-path blocks minus one, and the first part of the claim follows.

As for the number of isolated edges in , similarly to above,  we contract each non-path block into a black vertex and represent each path vertex as a white vertex. An isolated edge is represented by an edge that has white endpoints. We identify all these edges and contract each of them into a red vertex. As these edges are isolated,  the red vertices have only black neighbors. Then, proceeding by induction as we did for the isolated vertices, we can show that the number of isolated edges is upper-bounded by the number of non-path blocks.
	\end{proof}



Let  be the subset of  formed by the non-isolated vertices. Note that the minimum degree of the vertices in , which we denote by , is such that , so clearly the number of edges of  is at least ; this value is attained when  consists of disjoint edges. We need to improve this bound of  for being able to state Equation~(\ref{eq:P1}) below, and we can easily do it by forbidding this latter case.



	\begin{claimN}\label{claim:2-3}
If each connected component of  has at least three vertices, then .
	\end{claimN}
	\begin{proof} Without loss of generality, we can suppose that  is connected by the same reason mentioned in the proof of Claim~\ref{lemma4.8}. Let . As  is a tree, we have . If  then , that is,  and  as desired. \end{proof}

The induced subgraph  is a forest, so by Theorem~\ref{theorem2.1}  is balanced. Let  denote the number of edges in . Then   and  if we suppose  to be connected, where . Claim~\ref{claim:2-3} implies that


\begin{claimN}\label{claim:number-comp}
Let  be an instance of {\sc{Signed Max Cut ATLB}}. If  is the set of path vertices in , then the number of connected components of  is at most . If , then the number of connected components of  is  at most .
	\end{claimN}
\begin{proof}
 The same ideas used in the proof of Claim~\ref{lemma4.8} easily imply that the number of connected components of  is at most . Since  is the complement of  in , each vertex in  belongs to a non-path block. And as the size of  is at most , by~Lemma~\ref{Crow-Coro-6} the number of connected components of  is  at most . \end{proof}

We are now ready to piece everything together and conclude the proof of the theorem.

\vspace{.3cm}



 Due to Claim~\ref{lemma4.8}, by incurring an additive term of  to the size of the kernel we may assume that each component of  has at least three vertices, so the hypothesis of Claim~\ref{claim:2-3} holds. We apply Lemma~\ref{beta} to the graph  with . We have  connected components of ,  connected components of , and , where  and . It holds of course that . By Equation~(\ref{eq:P1}) we have  for  and, by Claim~\ref{claim:number-comp}, it holds that . Then . Therefore, if , then   is a {\sc{Yes}}-instance. Otherwise, , that is, we may assume that . 



Therefore, combining Equation~(\ref{eq:non-path-blocks}) with the above discussion we can conclude that

Finally, note that in the whole proof, when we use some result of Crowston et al.~\cite{crowston2013maximum} in which one of the possible outputs is that  is a \textsc{Yes}-instance (like~Lemma 14 or to check whether  is at most  or not), the condition to be checked concerns just the blocks of , which we can obtain in polynomial time by Proposition~\ref{coro4.1}. Thus, as we claimed, we do {\sl not} need to compute any partition of  into cliques   and independent sets . This concludes the proof of the theorem.\end{proof}



\section{A linear kernel on subclasses of split graphs}
\label{sec:linear-kernel}





With the objective of improving the quadratic kernel on -graphs presented in Section~\ref{sec:quadratic-kernel}, in Subsection~\ref{subsec:linear-kernel} we provide a linear kernel on a smaller graph class, namely on the subclass of -split graphs for every integer , which is defined as follows. A graph  is a \emph{-split graph} if  can be partitioned into a clique  and an independent set  such that every vertex in  has at least one neighbor in , and every vertex in  has degree at most .




We first prove in Subsection~\ref{ap:NP-hard} that the problem remains {\sf{NP}}-hard restricted to this class of graphs for every . In fact, we prove that even {\sc{Max Cut}} is {\sf{NP}}-hard, and this result is tight in terms of . Indeed, for , a -split graph is an independent set and therefore {\sc{Max Cut}} is trivial. For , the claim is given in the following simple observation.

\begin{remark}{\sc{Max Cut}} on -split graphs can be solved in polynomial time.
\end{remark}
\begin{proof} Note that by the definition of  -split graphs, such a graph has a very precise structure. Namely, it consists of an arbitrarily large clique  such that every vertex  in  has a private neighbor  in  (that is, a vertex of degree one). Then, an optimal solution of {\sc{Max Cut}} consists of a balanced partition of the clique, and for each , we place its private neighbor  in the opposite part of .
\end{proof}


\subsection{Max Cut is NP-hard on -split graphs}
\label{ap:NP-hard}

In this subsection we prove that \textsc{Max Cut}, which is a particular case of {\sc{Signed Max Cut}}, is {\sf{NP}}-hard on -split graphs for every fixed .
We first prove easily that {\sc{Max Cut}} is {\sf{NP}}-hard on
graphs without universal vertices (a \emph{universal} vertex is a vertex adjacent to every other vertex in the graph). Then, using a proof  from Bodlaender and Jansen~\cite{bodlaender} on the complexity of \textsc{Max Cut}, we prove that this problem is still {\sf{NP}}-hard on -split graphs and, consequently, {\sf{NP}}-hard on -split graphs for every fixed .






\begin{lemma}\label{lemma4.3}
{\sc{Max Cut}} is {\sf{NP}}-hard on graphs without universal vertices.		
\end{lemma}
\begin{proof}
We use the well known fact that {\sc{Max Cut}} is {\sf{NP}}-hard on general graphs~\cite{garey1976some} to show that this problem is still {\sf{NP}}-hard on graphs without universal vertices. For this, we consider a graph  as instance of  {\sc{Max Cut}}, and construct a new graph  made of two disjoint copies of . Obviously  has no universal vertices because it is disconnected.
Let  be the size of a maximum cut for  and  be the size of a maximum cut for . We claim that . Indeed,  because if  is a cut in  of maximum size, then  is the size of the cut ; see Fig.~\ref{fig:no-universal-vertices} for an illustration. On the other hand, if , then as  is made of two disjoint copies of , necessarily  contains a cut of size greater than , a contradiction.
\end{proof}

\begin{figure}[h]
\vspace{-1.2cm}
\begin{center}
 \includegraphics[width=.82\textwidth]{lemma10.pdf}
\vspace{-.35cm}\caption{Reduction to show that \textsc{Max Cut} is {\sf{NP}}-hard on graphs without universal vertices.} \label{fig:no-universal-vertices}\vspace{-.2cm}
\end{center}
\end{figure}

The construction given in the proof of the next theorem is exactly the same as the one given by  Bodlaender and Jansen in~\cite[Theorem 3.1]{bodlaender}, but we reproduce it here because we prove that if we start with a graph without universal vertices (we can do so thanks to Lemma~\ref{lemma4.3}), then the graph constructed in the reduction is a -split graph.




\begin{theorem}\label{theorem4.4}
{\sc{Max Cut}} is {\sf{NP}}-hard on -split graphs.		
\end{theorem}
\begin{proof}
Let a graph  be given and let  be the complement of . Let , where   is an endpoint of edge . Then  forms a clique,  forms an independent set in , and every edge-representing vertex  is connected to the vertices that represent its endpoints. Therefore  is a split graph in which all the vertices in the independent set have  degree exactly two. We claim that  allows a partition with at least  cut edges if and only if  allows a partition with at least  cut edges.
		
		Suppose first we have a partition  of  with at least  cut edges. We partition the vertices of  as follows: partition  as in the partition of ; for every , if both endpoints of  belong to , then put  in , otherwise put  in . It is easy to see that this partition gives the desired number of cut edges.
		
		Now suppose we have a partition  of  with at least  cut edges. Partition the vertices of  into two subsets:  and . This partition gives the desired number of cut edges. This can be noted as follows: for every edge , we have one cut edge in  if  is a cut edge in , otherwise we have no cut edge. For every edge , we have that out of the three edges , exactly two will be cut edges. Hence, the total number of cut edges in  equals the number of cut edges in  plus . Note that  can clearly be constructed from  in polynomial time.

By Lemma~\ref{lemma4.3}, we can assume that the graph   has no universal vertices. Then the graph  constructed above has no isolated vertices, and therefore every vertex  has at least one neighbor in . Since every vertex in  has degree two, the graph  constructed in the above reduction is indeed a -split graph, and the theorem follows.
\end{proof}
		


Since for every , the class of  -split graphs contains the class of -split graphs, the following corollary is a direct consequence of Theorem~\ref{theorem4.4}.
		
\begin{corollary}\label{cor:NP-hard}
{\sc{Max Cut}} is {\sf{NP}}-hard on -split graphs for every fixed .				 
\end{corollary}











\subsection{The kernelization algorithm}
\label{subsec:linear-kernel}

As we will see in the proof of Theorem~\ref{thm:linear-d-split} below, our kernelization algorithm is the simplest possible algorithm that one could imagine, as it does {\sl nothing} to the input graph; its interest lies on the analysis of the kernel size,  which uses the following two new reduction rules.

The first one  is a one-way reduction rule that is a generalization of the one-way Rule~6 given by Crowston et al.~\cite{crowston2013maximum}.

\vspace{.3cm}

\noindent {\bf{Rule .}} {\it{Let  be a connected graph. If  and  are pairwise non-adjacent neighbors of  such that  and  is connected, then delete  and set .}}

\vspace{.3cm}

Note that Rule~6 of Crowston et al.~\cite{crowston2013maximum} corresponds exactly to the case  of Rule~ above. 



	
\begin{lemma}\label{lem:Rule6+-safe}\label{lem:6+-safe}
Rule~ is safe.		
\end{lemma}
		\begin{proof}
Let  and  be as in the description of Rule~ and let . Note that  and since  is a tree (namely, a star),  is a balanced graph by Theorem~\ref{theorem2.1} whatever the signs of its edges. Therefore, . Let  be the graph obtained from  by the deletion of , so  is connected by hypothesis.
Suppose that , where . Then, by Lemma~\ref{beta}, , and therefore  is a {\sc{Yes}}-instance. 		
		\end{proof}

The second new rule is a simple two-way reduction rule, which just eliminates vertices of degree 1.
\vspace{.25cm}

\noindent {\bf{Rule A}.} {\it{Let  be a connected graph, and let  be a vertex of degree 1 in . Then delete vertex  and set  .}}





\begin{lemma}\label{lem:RuleA}
Rule A is valid.
\end{lemma}
\begin{proof} Let  be a connected graph, let  be a vertex of degree 1 in , and let  be the graph obtained from  by applying Rule A on vertex . Since the edge containing  in belongs to every optimal balanced subgraph of  (regardless of its sign), it holds that . On the other hand, we have that  and , and therefore  if and only if . Thus,  is a \textsc{Yes}-instance of \textsc{Signed Max Cut ATLB} if and only if  is.
\end{proof}

We are now ready to prove the main result of this subsection.

\begin{theorem}
\label{thm:linear-d-split} For any fixed integer , {\sc{Signed Max Cut ATLB}} on -split graphs admits a kernel with at most  vertices.
\end{theorem}
\begin{proof} Given a signed -split graph  and an integer , let  be an arbitrary partition of  into a clique  and an independent set  such that every vertex in  has degree at most  and every vertex in  has at least one neighbor in . We note that we do not even need to compute this partition  of , as we will just use it for the analysis.

Let  be the set of vertices in  that have at least two neighbors in , let  (so by hypothesis, every vertex in  has exactly one neighbor in ), let , and let . Note that since we assume that  is connected, it holds that . We now state two claims that will allow us to certificate in some cases that  is a \textsc{Yes}-instance.









\begin{claimN}\label{claim:clique}
If , then  is a \textsc{Yes}-instance.
\end{claimN}

\begin{proof} We apply to the input  the following procedure, assuming that  is connected (we stress that this algorithm is only used for the analysis; as mentioned before we do not modify our input graph at all):

\begin{enumerate}
\item If the current graph contains a vertex  in the clique with  neighbors in the independent set, let  be this set of neighbors, and do the following:
    \begin{itemize}
\item[] Apply Rule~ to , thus removing  from the current graph and setting . Note that the resulting graph is clearly connected, so Rule~ can indeed be applied to .
     \item[] Go back to Step~1.
    \end{itemize}
\item Apply exhaustively Rule A to the current graph, removing all vertices of degree 1. Clearly this operation preserves connectivity, and by Lemma~\ref{lem:RuleA} it produces an equivalent instance.
\item If the current graph contains a vertex  in the clique with exactly 1 neighbor  in the independent set, do the following:
    \begin{itemize}
    \item[] Let  be a vertex in the clique that is non-adjacent to ; since the degree of  is at most , such a vertex  is guaranteed to exist as long as the size of the current clique is at least .
    \item[] Note that the removal of the vertices  does not disconnect the current graph. Indeed, assume for contradiction that there exists a vertex  in the independent set that gets disconnected after the removal of . Since Rule A cannot be applied anymore to the current graph,  cannot have degree 1, hence necessarily  is adjacent to both  and . But then  has at least 2 neighbors  and  in the independent set, and this contradicts the fact that we are in Step~3, since Step~1 could be applied to  together with its neighborhood in the independent set.
     \item[] Therefore, we can apply Rule~ to  , thus removing them from the current graph and setting .
    \item[] Go back to Step~2.
    \end{itemize}
 \item Stop the procedure.
\end{enumerate}



\noindent We now claim that if , then the instance  output by the above procedure satisfies that , and since  is obtained from  by applying the one-way reduction Rule~ and the two-way reduction Rule A, Lemma~\ref{lem:6+-safe} and Lemma~\ref{lem:RuleA} imply that   is also a \textsc{Yes}-instance, as we wanted to prove. Indeed, suppose that Step~1 has been applied  times, that Rule~A in Step~2 has been applied  times, and that Step~3 has been applied  times. Since in Step~1 it holds that , it follows that , so it suffices to prove that .

With slight abuse of notation, let us denote by  the set of vertices in the current clique that have some neighbor in the independent set. Note that as long as , Step 1, 2, or 3 can be applied to the current graph. For each application of Step~1, the size of  decreases by 1. For each application of Rule~A in Step~2, the size of  decreases by at most 1. Finally, for each application of Step~3, the size of  decreases by at most  (at most  neighbors of vertex  together with vertex  may not belong to  anymore). Since by assumption  is a -split graph, initially we have that , and therefore it follows that , concluding the proof.\end{proof}

\begin{claimN}\label{claim:indep-set}
If , then  is a \textsc{Yes}-instance.
\end{claimN}
\begin{proof}
We prove that if , then we could iteratively apply Rule~ to  until obtaining an instance  with , which implies that  is a  \textsc{Yes}-instance.

We apply to  the following procedure (we stress again that this algorithm is only used for the analysis; as mentioned before we do not modify our input graph at all). We set , , ,  , and . Note that  is connected by hypothesis. For the sake of readability, we want to stress that the sets   and  defined below will {\sl not} correspond to the intersection of  with  and , respectively.  Proceed as follows:

\begin{enumerate}
\item Let  be an arbitrary vertex in , let , and let . Note that  and that   is connected.
\item Apply Rule~ to , and let .
\item Let  be the set of vertices in  having at most one neighbor in , and let ; see Fig.~\ref{fig-linear-kernel} for an example of these sets for . Since every vertex in  has at least two neighbors in  and each vertex in  has degree at most , it follows that . On the other hand, by definition it holds that .
\item Let ,  , and . Note that , and that by construction, each vertex in  has at least two neighbors in .
\item If , update  and go back to Step 1. Otherwise, stop the procedure.
\end{enumerate}

\begin{figure}[h]
\begin{center}\includegraphics[width=.5\textwidth]{theorem13.pdf}\vspace{-.6cm}
\caption{Example for  of the sets defined in the proof of Theorem~\ref{thm:linear-d-split}.} \label{fig-linear-kernel}
\end{center}
\end{figure}

Assume that the above procedure has been applied  times, and let . Note that . Since  is the parameter obtained from  by iteratively applying Rule~ in  to the sets described in Step~2 above, our objective is to prove that .

By the condition in Step~5 above, necessarily . Since for every  we have that, as discussed in Step~4 above, , using the hypothesis that  it follows that

and therefore it follows that . As for every  it holds that  (see Step~1 above) and the function  is greater than or equal to the function  for , we have that . Finally,

and therefore we can conclude that   is a \textsc{Yes}-instance, as claimed.
\end{proof}

Assume now that the hypothesis of Claim~\ref{claim:clique} and Claim~\ref{claim:indep-set} are both false, that is, that  and . Then we have that


Thus, if the above inequality is not true, that is, if , then necessarily at least one of the hypothesis of Claim~\ref{claim:clique} or Claim~\ref{claim:indep-set} is true, and in any case we can safely conclude that  is a \textsc{Yes}-instance. Therefore, our linear kernel is extremely simple: if , we report that  is a \textsc{Yes}-instance, and otherwise we have that , as desired.\end{proof}


\newpage

\section{Conclusions}
\label{sec:conclusions}



In this article we presented a quadratic kernel for  {\sc{Signed Max Cut ATLB}} when the input graph is an -graph, for any fixed positive value of  and . Etscheid and Mnich~\cite{EtscheidM16} recently presented, among other results, a {\sl linear} kernel for this problem on {\sl general} graphs, improving the results of the current paper, as well as those of Crowston et al.~\cite{crowston2012max}. Like us, the linear kernel of Etscheid and Mnich~\cite{EtscheidM16} also relies on a number of one-way reduction rules and results given by Crowston et al.~\cite{crowston2012max}. The key idea is to use two new two-way reduction rules (similar to those of Crowston et al.~\cite{crowston2012max}) that, roughly speaking, maintain the connectivity among the blocks of . The analysis of the kernel size in~\cite{EtscheidM16} is considerably more involved that ours and entails, in particular, an elaborated ``discharging'' argument on the number of vertices removed by the rules.


Concerning {\sf{FPT}}-algorithms, it may be possible that {\sc{Signed Max Cut ATLB}} can be solved in {\sl subexponential} time on -graphs or, at least, on split graphs. We leave this question as an open problem.



\acknowledgements
\label{sec:ack}
We would like to thank the anonymous referees for helpful remarks that improved the presentation of the manuscript.


\bibliographystyle{abbrv}	
\bibliography{maxbalancsg2-Ignasi}
\label{sec:biblio}

\end{document}
