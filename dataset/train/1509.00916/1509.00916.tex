\documentclass{scrartcl}

\usepackage{amsmath,amssymb,amsthm}
\usepackage{hyperref}

\newtheorem{theorem}{Theorem}
\newtheorem{lemma}[theorem]{Lemma}
\newtheorem{claim}[theorem]{Claim}

\newcommand{\cost}{\mathsf{cost}}

\begin{document}

\title{Improved and Simplified Inapproximability for -means}
\author{Euiwoong Lee\thanks{Computer Science Department, Carnegie Mellon University, Pittsburgh, PA 15213}~\thanks{Supported by the Samsung Scholarship and NSF CCF-1115525.} \and Melanie Schmidt\footnotemark[1]~\thanks{Supported by the German Academic Exchange Service (DAAD).} \and John Wright\footnotemark[1]~\thanks{Supported by a Simons Fellowship in Theoretical Computer Science.}}




\maketitle
\begin{abstract}
The -means problem consists of finding  centers in  that minimize the sum of the squared distances of all points in an input set  from  to their closest respective center. 
Awasthi et.\ al.\ recently showed that there exists a constant  such that it is NP-hard to approximate the -means objective within a factor of . We establish that the constant  is at least . 
\end{abstract}

For a given set of points , the \emph{k-means problem} consists of finding a partition of  into  clusters  with corresponding centers  that minimize the sum of the squared distances of all points in  to their corresponding center, i.e.\ the quantity

where  denotes the Euclidean distance.
The -means problem has been well-known since the fifties, when Lloyd~\cite{L57} developed the famous local search heuristic also known as the -means algorithm. Various exact, approximate, and heuristic algorithms have been developed since then. For a constant number of clusters  and a constant dimension , the problem can be solved by enumerating weighted Voronoi diagrams~\cite{IKI94}. If the dimension is arbitrary but the number of centers is constant, many polynomial-time approximation schemes are known. For example,~\cite{FL11} gives an algorithm with running time . In the general case, only constant-factor approximation algorithms are known~\cite{JV01,KMNPSW04}, but no algorithm with an approximation ratio smaller than 9 has yet been found.

Surprisingly, no hardness results for the -means problem were known even as recently as ten years ago. Today, it is known that the -means problem is NP-hard, even for constant  and arbitrary dimension ~\cite{ADHP09,D08} and also for arbitrary  and constant ~\cite{MNV09}. Early this year, Awasthi et.\ al.~\cite{ACKS15} showed that there exists a constant  such that it is NP-hard to approximate the -means objective within a factor of . They use a reduction from the Vertex Cover problem on triangle-free graphs.  Here, one is given a graph  that does not contain a triangle, and the goal is to compute a minimal set of vertices  which \emph{covers} all the edges, meaning that for any , it holds that  or .
To decide if  vertices suffice to cover a given , they construct a -means instance in the following way. Let  be the th vector in the standard basis of . For an edge , set . The instance consists of the parameter  and the point set . Note that the number of points is  and their dimension is . 

A relatively simple analysis shows that this reduction is approximation-preserving. 
A vertex cover  of size  corresponds to a solution for -means where we have centers at  and each point  is assigned to a center in  (which is nonempty because  is a vertex cover).  In addition, it can also be shown that a good solution for -means reveals a small vertex cover of  when  is triangle-free. 

Unfortunately, this reduction transforms -hardness for Vertex Cover on triangle-free graphs to -hardness for -means where  and  is the maximum degree of .
Awasthi et.\ al.~\cite{ACKS15} proved hardness of Vertex Cover on triangle-free graphs via a reduction from general Vertex Cover, where the best hardness result of Dinur and Safra~\cite{DS05} has an unspecified large constant . 
Furthermore, the reduction uses a sophisticated spectral analysis to bound the size of the minimum vertex cover of a suitably chosen graph product.

Our result is based on the observation that hardness results for Vertex Cover on small-degree graphs lead to hardness of Vertex Cover on triangle-free graphs with the same degree in an extremely simple way. 
Combined with the result of Chleb\'{i}k and Chleb\'{i}kov\'{a}~\cite{CC06} that proves hardness of approximating Vertex Cover on -regular graphs within , 
this observation gives hardness of Vertex Cover on triangle-free, degree- graphs without relying on the spectral analysis. 
The same reduction from Vertex Cover on triangle-free graphs to -means then proves APX-hardness of -means, with an improved ratio due to the small degree of . 

\section{Main Result}
Our main result is the following theorem.
\begin{theorem}
It is NP-hard to approximate -means within a factor . 
\end{theorem}

We prove hardness of -means by a reduction from Vertex Cover on -regular graphs, 
for which we have the following hardness result of Chleb\'{i}k and Chleb\'{i}kov\'{a}~\cite{CC06}.
\begin{theorem}[\cite{CC06}, see also~\ref{appendix:thm2}] 
Given a -regular graph , it is NP-hard to distinguish to distinguish the following cases.
\begin{itemize}
\item  has a vertex cover with at most  vertices. 
\item Every vertex cover of  has at least  vertices. 
\end{itemize}
Here,  and  with . In particular, it is NP-hard to approximate Vertex Cover on degree- graphs within a factor of . 
\end{theorem}

Given a -regular graph  for Vertex Cover with  vertices and  edges, 
we first partition  into  and  such that  and such that the subgraph  is bipartite. 
Such a partition always exists: every graph has a cut containing at least half of the edges (well-known; see, e.~g.,~\cite{MU05}). Choose  of these cut edges for , let  be the remaining edges.
We define  by {\em splitting} each edge in  into three edges.
Formally,  is given by  

Notice that  has  vertices and  edges. 
It is also easy to see that the maximum degree of  is 4, and that  does not have any triangle, since any triangle of  contains at least one edge of  (because  is bipartite) and each edge of  is split into three. 

Given  as an instance of Vertex Cover on triangle-free graphs, the reduction to the -means problem is the same as before. 
Let  be the th vector in the standard basis of . For an edge , set . The instance consists of the parameter  and the point set . Notice that the number of points is now  and their dimension is . 

We now analyze the reduction. 
Note that for -means, once a cluster is fixed as a set of points, the optimal center and the cost of the cluster are determined\footnote{For , the optimal solution to the -means problem is the \emph{centroid} of the point set. This is due to a well-known fact, see, e.\ g., Lemma 2.1 in \cite{KMNPSW04}.}. 
Let  be the cost of a cluster . 
We abuse notation and use  for the set of edges  as well. 
For an integer , define an \emph{-star} to be a set of  distinct edges incident to a common vertex. 
The following lemma is proven by Awasthi et.\ al.\ and shows that if  is cost-efficient, then two vertices are sufficient to cover many edges in . Furthermore, an {\em optimal}  is either a star or a triangle. 

\begin{lemma}[\cite{ACKS15}, Proposition 9 and Lemma 11]\label{lem:cost}
Let  be a cluster. 
Then , and there exist two vertices that cover at least  edges in .
Furthermore,  if and only if  is either an -star or a triangle, and otherwise, .
\end{lemma}

\subsection{Completeness} \begin{lemma}
If  has a vertex cover of size at most , the instance of -means produced by the reduction admits a solution of cost at most .
\end{lemma}
\begin{proof}
Suppose  has a vertex cover  with at most  vertices.
For each edge , let  if , and  otherwise. 
Let . Since  is a vertex cover of , for every edge ,  and  cover all three edges of  corresponding to . 
Therefore,  is a vertex cover of , and since , it has at most  vertices. 

For the -means solution, let each cluster correspond to a vertex in , and assign each edge  to the cluster corresponding to a vertex incident to  (choose an arbitrary one if there are two). 
Each edge is assigned to a cluster since  is a vertex cover, and each cluster is a star by construction.
Since there are  points and , the total cost of the solution is, by Lemma~\ref{lem:cost},

\end{proof}

\subsection{Soundness} \begin{lemma}
If every vertex cover of  has size of at least , then any solution of the -means instance produced by the reduction costs at least .
\end{lemma}
\begin{proof}
Suppose every vertex cover of  has at least  vertices. We claim that every vertex cover of  also has to be large. 
\begin{claim}
Every vertex cover of  has at least  vertices. 
\end{claim}
\begin{proof}
Let  be a vertex cover of . If  contains both  and  for any , 
then  is a vertex cover with the same or smaller size.
Therefore, we can without loss of generality assume that for each ,  contains exactly one vertex in . 
Set , thus  has cardinality . 
Each  is covered by  by definition. If an  is not covered by , at least one of the three edges of  corresponding to  is not covered by . 
Thus, every edge  is covered by , so  is a vertex cover of . Since , . 
\end{proof}



Fix  clusters . 
Without loss of generality, let  be clusters that correspond to a star, and  be clusters that do not correspond to a star for any . 
For , let  be the vertex covering all edges in , and for , let  be two vertices covering at least  edges in  by Lemma~\ref{lem:cost}. 
Let  be the set of edges not covered by any  or . 
The cardinality of  is at most 


Adding one vertex for each edge of  to the set  yields a vertex cover of  of size at most 

Every vertex cover of  has size of at least , so we have 

Now, either  or .
In the former case, since  for  by Lemma~\ref{lem:cost}, the total cost is 

In the latter case, the total cost can be split to obtain that  

Therefore, in any case, the total cost is at least 

\end{proof}

The above completeness and soundness analyses show that it is NP-hard to distinguish the following cases.
\begin{itemize}
\item There exists a solution of cost at most . 
\item Every solution has cost at least .
\end{itemize}
Therefore, it is NP-hard to approximate -means within a factor of


\bibliography{references-kmeanskt}
\bibliographystyle{amsalpha}

\appendix

\section{Remark on Theorem 2}\label{appendix:thm2}
To obtain Theorem 2, note that the proof of Theorem 17 in~\cite{CC06} states that it is NP-hard to distinguish whether the vertex cover has at most
\footnotesize

\normalsize
vertices. By the assumption in the first sentence of the proof and because ,  and  can be replaced by  as defined in Definition 6 in~\cite{CC06}. By Theorem 16 in~\cite{CC06}, .

\end{document}