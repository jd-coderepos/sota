\section{Experimental Setup}
\label{setup}
We evaluate our techniques on two different software markets: the \ChromeMarket{}
and the \AndroidMarket{}. We describe our experimental  setup for each of 
these markets in detail below.  

\paragraphbe{Extensions from \ChromeMarket{}.}
We collected a set of  \Chrome{} extensions covering all the extensions that were published in the \ChromeMarket{} during early 2014. For each extension, we extracted
its developer-provided category, the list of related extensions, and the list of requested 
permissions from the \ChromeMarket{}. To extract the set of requested permissions, 
we first parse declared permissions from the extension manifests, removed any 
invalid permissions that the developers may have added by mistake. As specific host 
permissions are usually rarely repeated across applications, we also filtered out such 
permissions except . 



\paragraphbe{\Android{} applications from \AndroidMarket{}.}
We gathered more than a million \Android{} applications covering all applications present in the \AndroidMarket{} during a specific day in the last six months. We extracted  the developer-provided category, the list of requested permissions 
from the manifest, and the application binary for each application. The application binaries were disassembled 
using the smali disassembler and analyzed to enumerate all API calls as well as the permissions required 
to successfully make those API calls. To estimate the permissions from API calls, we used the mapping of 
API calls to permissions provided by Au et al.\cite{au2012pscout}. For the rest of the paper, we refer to 
these sets of permissions as ``estimated permissions'' of an application. Note that neither the estimated 
permissions nor the requested permissions exactly represent the permissions used by an 
application as an application binary often contains code that do not get executed and developers 
often request permissions that are never used by their applications. Therefore, to get a better approximation 
of the actual used permissions, we also computed the intersection of requested and estimated permissions
for each application.

\paragraphbe{Implementation.} We implemented our techniques using two Map-Reduce tasks: one for computing 
relative frequency of privileges for each peer group and the other for computing unexpectedness values for 
each application and detecting over-privileged applications. We implemented Algorithm~\ref{alg:score} as part of 
the second MapReduce task. For all our tests, we set  to  for the privileges that we deemed security 
sensitive and to  for the rest.  



\section{Evaluation} 
In this section, we first use our \Chrome{} extension and \Android{} application datasets to evaluate 
how the actual methods for peer group estimation and the values of different settings for peer group analysis affect 
the estimated unexpectedness scores. Next, we explore how useful the unexpectedness scores are for 
different purposes (e.g., helping developers adhere to the principle of least privilege, helping users avoid 
over-privileged applications) using the same datasets. 

\subsection{Effects of peer group parameters}
To estimate the effects of different settings of peer group analysis on its effectiveness, 
for each setting, we measure how many applications in our datasets have no unexpected 
security privileges relative to their respective peer groups. As these applications 
are deemed to be `normal' in their peer groups, we expect them to be the majority in our datasets. 
A low number of such applications will indicate that the peer groups are not well formed i.e. they contain 
applications with completely different functionality. 

\paragraphbe{Picking relative\_frequency\_threshold.} One of the main configuration parameters 
for our unexpectedness estimation algorithm (Algorithm~\ref{alg:score} in Section~\ref{compute_unexpected}) 
is . This parameter decides the minimum proportion of applications in a peer 
group that has to use a privilege in order to label that privilege as ``expected'' for that peer group. 
Figure~\ref{fig:percentage_different_thresholds} shows the variation of the percentage of applications 
that have at least one unexpected privilege with different values of the . 
For all of our tests, we set the relative frequency threshold to  for \Chrome{} extensions 
and  for \Android{} applications. 

\begin{figure}[!t]
\begin{center}
\includegraphics[scale=0.40]{different_thresholds_real_labels.png}
\caption{Variability of the percentage of applications with at least one unexpected privileges, for different  choices.}
\label{fig:percentage_different_thresholds}
\end{center}
\end{figure}

\begin{figure}[!htbp]
\centering
\includegraphics[scale=0.40]{different_peer_group_sizes.png}
\caption{Variability of the percentage of outlying extensions with unexpected privileges, for different peer group sizes.}
\label{fig:percentage_unexpected_peer_group_size}
\end{figure}

\paragraphbe{Peer group sizes.} Figure~\ref{fig:percentage_unexpected_peer_group_size} 
shows how the percentage of  \Chrome{} extensions with no unexpected privileges varies 
with different peer group sizes. For peer groups that contain only  peers, there are only 
around  of such extensions even for a low relative frequency threshold of .  
This indicates that such small peer groups may not be very effective in estimating unexpectedness 
as they might mark a large number of applications as unexpected. However, with peer groups of 
size , the percentage of extensions with no unexpected privileges rises to around  
for the same relative frequency threshold. Nonetheless, for peer groups of sizes larger than 
, their percentage falls to around .

\paragraphbe{Different types of privilege estimation.} Figure~\ref{fig:percentage_unexpected_different_privileges} 
shows how the percentage of less-privileged applications vary with different types privilege estimation using the 
\Android{} dataset. We used four different ways for privilege estimation: requested privileges by the 
developers, method calls statically extracted from application binary,  statically estimated permissions from 
application binary, and intersection of statically estimated permissions and developer-requested permissions. 
We found that the statically estimated permissions and the intersection of the requested and estimated permissions 
yield the best results. 



\begin{figure}[!t]
\centering
\includegraphics[scale=0.40]{different_privileges.png}
\caption{Variability of the percentage of applications with no unexpected privileges, for different privilege types.}
\label{fig:percentage_unexpected_different_privileges}
\end{figure}




\subsection{\mbox{Effectiveness of peer group analysis}}
\label{pga_effectiveness}
\paragraphbe{Provide incentives to developers for creating less-privileged applications.} 
Creating applications with lower privileges require significant amount of work and the existing system 
provides the developers no incentives to do it. Therefore, developers often tend to create over-privileged 
applications. However, we found that different developers tend to create differently over-privileged 
applications even when the applications have similar functionality. This is primarily caused 
by the differences in how the developers implement a particular functionality. We found wide 
variations in the unexpectedness scores of the applications from the same peer group.   
For example, Table~\ref{tbl:unexpected_chart} shows the unexpectedness scores and privileges of 
applications from two different peer groups in \ChromeMarket{}: one with five PDF readers and the other 
with eight tab managers. As we can see, different applications in the same peer group have widely different unexpectedness 
scores. For example, `docs-pdfpowerpoint-viewer' has a score of  while  `pdf-viewer' has a score 
of . Also, it can be seen form Table~\ref{tbl:unexpected_chart} that over-privileged applications 
in the same peer group like `pdf-viewer' and `\Chrome{}-office-viewer-beta' are over-privileged with 
very different privileges (pdf-viewer uses `WebNavigation' , `webRequest', and `webRequestBlocking' and
\Chrome{}-office-viewer-beta uses `clipboardWrite', `fileBrowserHandler', `fileSystem').

\begin{table*}[!htbp]
        \begin{center}
        \caption{Variation of unexpectedness across different peer groups.  
                 \label{tbl:unexpected_chart}
                }
                \begin{scriptsize}
                \begin{tabular}{|l|l|c|l|}
                        \hline
                        Peer group & App name & Score & Unexpected privileges \\ \hline
                        \multirow{5}{*}{pdf reader} & docs-pdfpowerpoint-viewer & 0 & - \\ \cline{2-4}
                                             & pdfescape-free-pdf-editor & 0 & - \\ \cline{2-4}
                                             & beeline-reader & 1 & webNavigation \\ \cline{2-4}
                                             & pdf-viewer & 3 & WebNavigation , webRequest, webRequestBlocking \\ \cline{2-4}
                                             & \Chrome{}-office-viewer-beta & 3 & clipboardWrite, fileBrowserHandler, fileSystem \\     
                                             \hline \hline                                             
                       \multirow{8}{*}{tab manager} & Tab Manager & 0 & - \\ \cline{2-4}
                                             & Project Tab Manager & 1 & bookmarks \\ \cline{2-4}
                                             & Awesome Tab Manager & 2 & bookmarks, unlimitedStorage \\ \cline{2-4}
                                             & TooManyTabs for \Chrome{} & 2 & bookmarks, unlimitedStorage \\ \cline{2-4}
                                             & Tabs Outliner & 3 & idle, notifications, storage \\ \cline{2-4}
                                             & Tabman Tabs Manager & 3 & history, topsites, webNavigation \\ \cline{2-4}
                                             & Fruumo Tab Manager & 3 & bookmarks, history, unlimitedStorage \\ \cline{2-4}
                                             & Session box - Tabs manager & 4 & clipboardWrite, cookies, management, unlimitedStorage \\ \hline
                \end{tabular}
                \end{scriptsize}
        \end{center}
\end{table*}

Such wide variation in the unexpectedness scores inside the same peer group also indicate 
that there is a large difference among the applications in terms of their adherence to 
the least privilege principle even when they are providing the same functionality. 
The unexpectedness score provides a metric to market owners for separating 
less-privileged applications from over-privileged ones. Market owners can leverage 
such information in different ways to encourage developers in creating less-privileged 
applications as mentioned below.
 
\paragraphbe{Rewarding applications with less privileges.} Applications with low unexpectedness
scores indicate that their developers have put more effort in following the principle of least privilege 
than other developers. To encourage such behavior, market owners can provide positive reinforcements like ranking such applications 
higher in the search results than their peers or providing the developers of such applications 
with special badges/points. Such positive reinforcements may encourage other developers 
to do a better job at creating less-privileged applications.

\paragraphbe{Penalizing callous developers.} An over-privileged application often 
indicates that its developer did not invest the required effort to lower its privileges. 
Among the over-privileged applications detected by peer group analysis, we found numerous 
such cases. For example, we found that the `pdf-viewer' uses the webRequest privilege to serve 
advertisements even when there are less-privileged ways of achieving the same goal.
As another example, consider the case of a Facebook game named `facebook mafia wars' that 
uses the privilege for installing or removing other extensions to scan and automatically 
remove any conflicting extensions. A less-privileged way of performing the same task would  
have been to simply ask the user to disable the conflicting extension. We also found that several 
developers repeatedly violate least-privilege in most of their applications irrespective of the 
functionality. Such developers can be identified easily with the unexpectedness scores of their 
applications and be warned or banned by market owners. 



\paragraphbe{Sharing information about peer groups with developers.}  In order to help developers 
create less-privileged applications, market owners can inform developers about the 
functionality and privileges used by the applications with low unexpectedness scores 
from different peer groups . Such information can be useful to the developers in several 
ways. First, a developer creating a new application can check the minimal of privileges 
used by applications in the corresponding peer group and treat these privileges 
as the baseline for the new application. Next, existing application developers 
may be more motivated to use less privileges if they know that a given functionality can be 
implemented using less privileges than they are using in their application. Consider three 
different Facebook notification extensions from \ChromeMarket{}: Facebook Chat Notification, Facebook notification icon, and 
Facebook Notifier; all of them provide similar functionality of notifying the user 
about some Facebook event (e.g., new chat messages, wall posts). However, the first 
one only requires the privilege to access the browser history, the second one requires access to 
the user's data from \url{facebook.com}, and the last one requires access to the user's data 
from all domains. If the developers of the second and third extensions 
know about the existence of the first one, they might be willing to emulate 
its implementation by redesigning their extensions to remain competitive.  

\begin{figure}[!htbp]
\centering
\includegraphics[scale=0.40]{crx_10_score_dist_ratio_bar10.png}
\caption{Distribution of the percentages of suspended extensions with unexpectedness score in \ChromeMarket{}.}
\label{fig:banned_percentage_unexpectedness}
\end{figure}


\paragraphbe{Helping users to avoid over-privileged applications.} Over-privileged applications can 
cause significantly more damage to the user when compromised than their less-privileged counterparts. 
Moreover, we find  that over-privileged applications are more likely to be suspended by market owners 
due to policy violation, spamming, and other suspicious activities. Therefore, it is in the best interests 
of the users to try to avoid the over-privileged applications. 

To evaluate the hypothesis that the over-privileged applications get suspended more often than th less 
privileged ones, we created a new dataset by picking  randomly chosen extensions from 
our existing \ChromeMarket{} dataset and augmenting them with   suspended extensions. These 
extensions have been suspended by the market owner due to a variety of reasons including malicious 
activity, spamming, violating market policies etc.  We compute the unexpectedness values for all extensions 
in the dataset including both the suspended and live extensions. For each 
application with a unexpectedness score greater than zero, we compute the percentage of suspended applications 
for different scores. Figures~\ref{fig:banned_percentage_unexpectedness} and ~\ref{fig:bubble_plot_banned} 
show the results  from the experiments using the category-based peer groups and the peer groups based on 
related items respectively. Both these figures clearly demonstrate that, irrespective of the
technique used for creating peer groups, applications with high unexpectedness scores are very 
likely to be suspended. 


Unexpectedness score provides the users with a simple way to compare an application's privileges 
with respect to its peers (i.e., other applications providing similar functionality). A security-conscious 
user should pick applications with low unexpecteness scores from the peer group corresponding to the 
desired functionality. For example, if the user is looking for a PDF reader, she should probably pick 
the {\it pdf-viewer} application from the PDF reader peer group shown in Table~\ref{tbl:unexpected_chart}
(assuming, of course, it provides all the desired features).

The market owners can facilitate this process by designing user interfaces that help users to 
easily find less privileged applications for a given functionality. For example, the search results 
returned to the user for a particular query to the application store can contain the unexpectedness 
scores of all applications present in the search results. The users should be able to 
sort the applications based on their unexpectedness score. Also, instead of showing 
the numeric unexpectedness scores to the user, the scores can be color coded 
according to their score (e.g., high, medium, and low scores can be represented by red, 
medium scores by red, yellow, green respectively as shown in Figure~\ref{fig:mockup_ui}).    

\begin{figure*}[!tbp]
\centering
\includegraphics[scale=0.30]{ui_mockup.jpg}
\caption{A sample interface for displaying color-coded unexpectedness scores of each application 
present in the search results.} 
\label{fig:mockup_ui}
\end{figure*}

\begin{figure}[!htbp]
\centering
\includegraphics[scale=0.40]{10_bubble_plot2.png}
\caption{Bubble plot showing variation of the percentage of suspended extensions in \ChromeMarket{} 
with unexpectedness score and  number of related items i.e. peer group size. The sizes of the 
triangles are proportional to the percentage of suspended extensions with a particular unexpectedness 
value and a particular peer group size.}
\label{fig:bubble_plot_banned}
\end{figure}





























