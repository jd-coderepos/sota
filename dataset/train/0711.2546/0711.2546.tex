\documentclass{LMCS}

\def\Rlap#1{\rlap{}}
\def\Llap#1{\llap{}}

\usepackage{ifthen}
\newboolean{long}
\setboolean{long}{false}
\usepackage{url}

\pagestyle{empty}

\usepackage{color}
\usepackage{amssymb}

\usepackage{amsmath}
\usepackage{amsfonts}
\usepackage{enumerate}
\usepackage{proof}
\usepackage{611}
\usepackage{hyperref}

\newtheorem{definition}[thm]{Definition}
\newtheorem{problem}[thm]{Problem}
\newtheorem{corollary}[thm]{Corollary}
\newtheorem{conjecture}[thm]{Conjecture}
\newtheorem{lemma}[thm]{Lemma}
\newtheorem{hyp}[thm]{Hypothesis}
\newtheorem{example}[thm]{Example}
\newcommand\compare{\operatorname{compare}}
\newcommand\accept{\operatorname{accept}}
\newcommand\start{\operatorname{start}}
\newcommand\dcl{\operatorname{dcl}}
\newcommand\map{\operatorname{map}}
\newcommand\fst{\operatorname{fst}}
\newcommand\acl{\operatorname{acl}}
\newcommand\plus{\operatorname{plus}}
\newcommand\iffl{\ensuremath{\leftrightarrow}}
\renewcommand\iff{\ensuremath{\Leftrightarrow}}
\renewcommand\int{\mathbf{Z}}
\newcommand\rimpl{\ensuremath{\Rightarrow}}
\newcommand\limpl{\ensuremath{\Leftarrow}}
\newcommand{\ignore}[1]{}
\newcommand{\todo}[1]{}
\newcommand{\ov}[1]{\ensuremath{\overline{#1}}}
\newcommand{\nat}{\ensuremath{\mathbb{N}}}
\newcommand{\q}{\ensuremath{\mathbb{Q}}}
\newcommand{\lp}{\ensuremath{\lambda P_1\ }}
\newcommand\real{\ensuremath{\Vdash}}
\newcommand\reals{\ensuremath{\Vdash}}
\newcommand{\p}{\proves}
\newcommand{\x}{\ov{x}}
\newcommand{\g}{\Gamma}
\newcommand{\piz}{\p_{IZF_R^-}}
\newcommand{\gp}{\Gamma \proves}
\newcommand{\ga}{\Gamma^{\ov{a}}}
\newcommand{\gap}{\ga \p}
\newcommand{\rg}{rg(\Gamma)}
\newcommand{\rgp}{\rg \proves}
\newcommand{\og}{\ov{\g}}
\newcommand{\rrho}{\reals_\rho}
\newcommand{\oga}{\og^{\ov{a}}}
\newcommand{\ogp}{\og \proves}
\newcommand{\gx}{\Gamma, x : \tau}
\newcommand{\gxp}{\gx \proves}
\newcommand{\lst}{\lambda^{\to}}
\newcommand{\pl}[1]{\ensuremath{\mathrm{#1}}}
\newcommand{\FST}{\pl{fst}}
\newcommand{\SND}{\pl{snd}}
\newcommand{\LET}{\pl{let}}
\newcommand{\CASE}{\pl{case}}\newcommand{\DOM}{\pl{dom}}
\newcommand{\MAGIC}{\pl{magic}}
\newcommand{\INL}{\pl{inl}}
\newcommand{\INR}{\pl{inr}}
\newcommand{\IN}{\pl{in}}
\newcommand{\IND}{\pl{ind}}
\newcommand{\sr}[1]{\ensuremath{\SB{#1}_\rho}}
\newcommand{\izfc}{IZF}
\newcommand{\izfr}{IZF}
\newcommand{\iizf}{IZF}
\newcommand{\iizfr}{IZF}
\newcommand{\li}{\lambda Z}
\newcommand{\la}{\lambda Z}
\newcommand{\lz}{\lambda Z}
\newcommand{\lrk}{\ensuremath{\lambda rk}}
\def\squareforqed{\hbox{\rlap{}}}


\def\doi{4 (2:1) 2008}
\lmcsheading {\doi}
{1--29}
{}
{}
{Jan.~\phantom{0}2, 2007}
{Apr.~08, 2008}
{}   

\begin{document}

\title{Normalization of IZF with Replacement}

\author{Wojciech Moczyd\l owski}
\address{Department of Computer Science, Cornell University, Ithaca, NY\ 14853, USA}
\email{wojtek@cs.cornell.edu}
\thanks{Partly supported by NSF grants DUE-0333526 and 0430161.}

\keywords{Intuitionistic set theory, Curry-Howard isomorphism, normalization, realizability}
\subjclass{F.4.1}
\titlecomment{}

\begin{abstract}
\noindent
IZF is a well investigated impredicative constructive version of
Zermelo-Fraen\-kel set theory. Using set terms, we axiomatize IZF with
Replacement, which we call \izfr, along with its intensional
counterpart \iizfr. We define a typed lambda calculus 
corresponding to proofs in \iizfr\ according to the Curry-Howard
isomorphism principle. Using realizability for \iizfr, we show weak
normalization of . We use normalization to prove the disjunction,
numerical existence and term existence properties.  An inner
extensional model is used to show these properties, along with the set
existence property, for full, extensional \izfr.
\end{abstract}

\maketitle

\section{Introduction}

Four salient properties of constructive set theories are:
\begin{enumerate}[]
\item Numerical Existence Property (NEP): From a proof of a statement
``there exists a natural number  such that {\ldots}'' a witness  can be extracted. 
\item Disjunction Property (DP): If  is provable, then
either  or  is provable.
\item Term Existence Property (TEP): If  is
provable, then  is provable for some term .
\item Set Existence Property (SEP): If  is
provable, then there is a formula  such that  is provable, where both  and  are term-free. 
\end{enumerate}

How to prove these properties for a given theory? There is a variety of
methods applicable to constructive theories. Cut-elimination, proof
normalization, realizability, Kripke models{\ldots}.
Normalization proofs, based on the Curry-Howard isomorphism principle, have the advantage of
providing an explicit method of witness and program extraction from
proofs. They also provide information about the behaviour of the proof
system. 

We are interested in intuitionistic set theory IZF. It is essentially what 
remains of ZF set theory after excluded middle is carefully taken away. An important 
decision to make on the way is whether to use Replacement or Collection axiom schema. 
We will call the version with Collection \izfc\  and the version with Replacement \izfr. In the literature,
IZF usually denotes \izfc. Both theories extended with excluded middle are
equivalent to ZF \cite{friedmancons}. They are not equivalent \cite{frsce3}.
While the proof-theoretic power of \izfc\ is equivalent to that of ZF, the exact
power of \izfr\ is unknown. Arguably \izfc\ is less constructive, as
Collection, similarly to Choice, asserts the existence of a set without
defining it. 

Both versions have been investigated thoroughly. Results up to 1985 are
presented in \cite{beesonbook,scedrov85}. Later research was concentrated on
weaker subsystems \cite{ar,ikp}. A predicative constructive set theory CZF has attracted
particular interest. \cite{ar} describes the set-theoretic apparatus
available in CZF and provides further references.

We axiomatize \izfr, along with its intensional version \iizfr, using set
terms. We define a typed lambda calculus  corresponding to proofs in
\iizfr. We also define realizability for \iizfr, in the spirit of
\cite{mccarty}, and use it to show that  weakly normalizes. Strong normalization of  does not hold; moreover, we show
that in non-well-founded IZF even weak normalization fails.

With normalization in hand, the properties NEP, DP and TEP
easily follow. To show these properties for full, extensional \izfr, we define an inner
model  of \izfr, consisting of what we call transitively L-stable sets.
We show that a formula is true in \izfr\ iff its relativization to  is true
in \iizfr. Therefore \izfr\ is interpretable in \iizfr. This allows us to use 
the properties proven for \iizfr. In \izfr, SEP easily follows from TEP. 

The importance of these properties in the context of computer science stems
from the fact that they make it possible to extract programs from
constructive proofs. For example, suppose \izfr\ . From this proof a program can be extracted --- 
take a natural number , construct a proof \izfr\ . Combine the proofs to get \izfr\  and apply 
NEP to get a number  such that \izfr\ . A
detailed account of program extraction from \izfr\ proofs can be found in \cite{chol}.

There are many provers with the program extraction capability. However, 
they are usually based on variants of type theory, which is a foundational basis very different from set
theory. This makes the process of formalizing program specification more
difficult, as an unfamiliar new language and logic have to be learned from
scratch. \cite{lamport99} strongly argues \emph{against} using type theory
for the specification purposes, instead promoting standard set theory. 

\izfr\ provides therefore the best of both worlds. It is a set theory,
with familiar language and axioms. At the same time, programs can be
extracted from proofs. Our  calculus and the normalization
theorem make the task of constructing the prover based on \izfr\ not very
difficult.

This paper is mostly self-contained. We assume some familiarity with
set theory, proof theory and programming languages terminology, found for example 
in \cite{kunen,urzy,pierce}. The paper is organized as follows. We start by presenting in details
intuitionistic first-order logic in section \ref{ifol}. In section \ref{izf} we
define \izfr\ along with its intensional version \iizfr. In section \ref{lz} we define a
lambda calculus  corresponding to \iizfr\ proofs. Realizability for
\iizfr\ is defined in section \ref{izfreal}. We use it to prove normalization of  in
section \ref{sectionnorm}, where we also show that non-well-founded IZF does
not normalize. We prove the properties in section \ref{secapp}, and show how to derive them 
for full, extensional \izfr\ in section \ref{lei}. Comparison with other results can be found in section \ref{others}.

\section{Intuitionistic first-order logic}\label{ifol}

Due to the syntactic character of our results, we present the intuitionistic first-order logic
(IFOL) in details. We use a natural deduction style of proof rules. The terms will be denoted by
letters . The variables will be denoted by letters . The notation  stands for a finite sequence, treated as a set when
convenient. The -th element of a sequence is denoted by . We consider -equivalent
formulas equal. The capture-avoiding substitution is defined as usual; the
result of substituting  for  in a term  is denoted by . We
write  to denote the result
of substituting simultaneously  for . Contexts, denoted by , are sets of formulas. 
The set of free variables of a formula , denoted by , are
defined as usual. The free variables of a context , denoted by , are
the free variables of all formulas in . The notation  means
that all free variables of  are among . The proof rules are as follows:






Negation in IFOL is an abbreviation: . So is the symbol : . Note that IFOL does not contain equality. The excluded middle rule added to IFOL makes it equivalent
to the classical first-order logic without equality. We adopt the
``dot''-convention --- a formula  should be parsed as
. In other words\footnote{Borrowed from \cite{urzy}.}, the
dot represents a left parenthesis whose scope extends as far to the right as
possible. 



\begin{lemma}\label{formsubst}
For any formula , , for . 
\end{lemma}
\proof Straightforward structural induction on .\qed



\section{\izfr}\label{izf}

Intuitionistic set theory \izfr\ is a first-order theory, equivalent to ZF
when extended with excluded middle. It is a definitional extension of
term-free versions presented in \cite{myhill72,beesonbook,frsce3}.
The signature consists of one binary relational symbol  and function symbols used in the axioms below.
The set of all \izfr\ terms will be denoted by . The notation  is an abbreviation for . Function symbols  and  are abbreviations for
 and . Bounded quantifiers and the quantifier  (there exists exactly one ) are also abbreviations 
defined in the standard way. The axioms are as follows:

\begin{enumerate}[]
\item (EMPTY) 
\item (PAIR) 
\item (INF) 
\item (SEP)

\item (UNION) 
\item (POWER) 
\item (REPL) 
\item (IND) 
\item (L) 
\end{enumerate}

Axioms SEP, REPL, IND and L are axiom schemas, and
so are the corresponding function symbols ---
there is one function symbol for each formula . Formally, we define formulas and terms by mutual induction:


Our presentation is not minimal; for example, the empty set axiom can be
derived as usual using Separation and Infinity. However, we aim for a
\emph{natural} axiomatization of \izfr, not necessarily the most optimal one. 

The Leibniz axiom schema L is usually not present among the axioms of set
theories, as it is assumed that logic contains equality and the axiom is 
a proof rule. We include L among the axioms of \izfr, because
there is no obvious way to add it to intuitionistic logic in the Curry-Howard isomorphism context,
as its computational content is unclear. Our axiom of Replacement is
equivalent to the usual formulations, see \cite{jatrinac2006} for details.

\iizfr\ will denote \izfr\  without the Leibniz axiom schema L. \iizfr\ is
an intensional version of \izfr\  --- even though extensional equality is used
in the axioms, it does not behave as the ``real'' equality. 

The terms  and  can be displayed
as  and .

The axioms (EMPTY), (PAIR), (INF), (SEP), (UNION), (POWER) and (REPL)
all assert the existence of certain classes and have the same form: , where  is a 
function symbol and  a corresponding formula
for the axiom A. For example, for (POWER),  is  and
 is . We reserve the notation  and  to denote the term and
the corresponding formula for the axiom A.

\begin{lemma}\label{tdef0}
Every term  of \izfr\ is definable. In
other words, there is a term-free formula  such that \izfr 
.
\end{lemma}
\proof Straightforward induction on the size of .
We first show the claim for , then for the rest of the terms. For , the defining
formula\footnote{Strictly speaking, it is not term-free, but eliminating
terms used in  is straightforward.} is:

Indeed,  holds. Suppose  for some
, we need to show that . To do this, we prove by
-induction . Take any  and 
suppose . Then  or there is  such that .
In the former case , in the latter , so by the
induction hypothesis  and hence . The other
direction is symmetric. 

Consider now arbitrary . Let
 denote , so .
By the induction hypothesis there are formulas  defining . Consider the formula:

We will now show that  defines . Take any  and take 
. We have
 and
by the axiom (A) corresponding to , we get . Furthermore, suppose  for
some . Then there are  such that  and . Since  define ,
 and thus also .
To show that , it suffices to show that , which follows easily.

It remains to consider the situation when  contains some terms,
which can happen if  is the Separation or Replacement axiom. However, by the 
induction hypothesis all these terms are definable as well, so there is also a term-free formula  equivalent to .
\ignore{

Let the formulas
 define .
defines . To see that  holds, take .
By the induction hypothesis we have  and 
by the axiom (A) defining , we get . Furthermore, suppose  for
some . Then there are  such that  and . Since  define ,
 and thus also . 
To show that , it suffices to show that , which follows easily.}\qed


\begin{corollary}\label{tdef}
For any closed term  there is a term-free formula  such that \izfr .
\end{corollary}

\section{The  calculus}\label{lz}

We now present a lambda calculus  for \iizfr, based on the Curry-Howard isomorphism
principle. The first-order part of  is essentially  from
\cite{urzy}. The lambda terms in the calculus correspond to proofs in \iizfr.
The correspondence is captured formally by Lemma \ref{ch}. 

The lambda terms in  will be denoted by letters . 
There are two kinds of lambda abstractions, one used for proofs of implications, the other for proofs of
universal quantifications. We use separate sets of variables for these abstractions and call them
proof and first-order variables, respectively. We use letters  for proof variables and  for first-order variables. 
Letters  are reserved for \izfr\ terms. The types in the system are
\izfr\ formulas. The lambda terms are generated by an abstract grammar. The
first group of terms is standard and used for IFOL proofs:

The rest of the terms correspond to the axioms of \iizfr:








The \pl{ind} term corresponds to the -induction axiom schema (IND),
and \pl{Prop} and \pl{Rep} terms correspond to the respective axioms.
The exact nature of the correspondence will become clear in the next
section. Briefly and informally, the \pl{Rep} terms are
\emph{representatives} of the fact that a  is a member of a term
 and the \pl{Prop} terms provide the defining \emph{property} of
. To avoid listing all of them every
time, we adopt a convention of using \pl{axRep} and \pl{axProp} terms to tacitly
mean all \pl{Rep} and \pl{Prop} terms, for \pl{ax} being one of \pl{empty}, \pl{pair}, \pl{union},
\pl{sep}, \pl{power}, \pl{inf} and \pl{repl}. With this convention in mind, we can summarize the
definition of the \pl{Prop} and \pl{Rep} terms as:

where the number of terms in the sequence  depends on the particular
axiom. 

The free variables of a lambda term are defined as usual, taking into
account that variables in , \pl{case} and \pl{let} terms bind respective
terms. The relation of  -equivalence is defined taking this information into account. We consider -equivalent terms equal.
We denote the set of all free variables of a term  by  and the set
of the free first-order variables of a term by . The free (first-order) variables of a context 
are denoted by  () and defined in a natural way. The
notation  stands for a term  with  substituted for . 
The set of all  lambda terms will be denoted by . 

\subsection{Reduction rules}\label{rr}

The deterministic reduction relation  arises by lazily evaluating the
following base reduction rules:






The laziness is specified formally by the following evaluation contexts:


In other words, the (small-step) reduction relation arises from the base
reduction rules and the following inductive definition:





\begin{definition}
We write  if the reduction sequence starting from 
terminates. We write  if we want to state that 
is the term at which this reduction sequence terminates. We write  if
 reduces to  in some number of steps. \todo{Better formulation?}
\end{definition}

We distinguish certain  terms as values. The values are generated
by the following abstract grammar, where  is an arbitrary term. Clearly, there are
no reductions possible from values. 


\subsection{Types}\label{lambdaa}

The type system for  is constructed according to the principle
of Curry-Howard isomorphism for \iizfr. Types are \izfr\ formulas.
Contexts, denoted by , are finite sets of pairs , written
as . The \emph{domain} of a context
 is the set  and it is denoted by .  The \emph{range} of a context  is the corresponding first-order logic context that contains
only formulas and is denoted by . The first group of rules
corresponds to the rules of IFOL:







The rest of the rules correspond to \iizfr\ axioms:



\subsection{Properties of }

We now prove a standard sequence of lemmas for . 

\begin{lemma}[Canonical Forms]
Suppose  is a value and . Then: 
\begin{enumerate}[]
\item  iff  and .
\item  iff  ( and ) or ( and ).
\item  iff ,  and . 
\item  iff  and . 
\item  iff  and .
\item  iff  and .
\item  never happens.
\end{enumerate}
\end{lemma}
\proof Immediate from the typing rules and the definition of values.\qed


\begin{lemma}[Weakening]
If  and  are fresh with respect to the proof tree , then .
\end{lemma}
\proof Straightforward induction on . The freshness assumption is
used in the treatment of the proof rules having side-conditions, such as 
introduction of the universal quantifier.\qed


There are two substitution lemmas, one for the propositional part, the other
for the first-order part of the calculus. Since the rules and terms of 
corresponding to \izfr\ axioms do not interact with substitutions in a
significant way, the proofs are routine. 

\begin{lemma}\label{lamsl}
If  and  , then
.
\end{lemma}
\proof By induction on . We show two interesting cases.
\begin{enumerate}[]
\item , . Using -conversion 
we can choose  to be new, so that . The
proof tree must end with:

By the induction hypothesis, , so . By the choice of , . 
\item . The proof tree ends with:

Choose  and  to be fresh. By the induction hypothesis,  and . Thus . By  and  fresh,  which is what we want.\qed 
\end{enumerate}


\begin{lemma}\label{logsl}
If , then .
\end{lemma}
\proof By induction on . Most of the rules do not interact with
first-order substitution, so we show the proof just for the four of them which
do. 
\begin{enumerate}[]
\item , . The proof tree ends with:

Without loss of generality we can assume that . By the induction hypothesis, . Therefore  and by the choice of , . 
\item ,  for some term . The proof tree ends with:

Choosing  to be fresh, by the induction hypothesis we get , so . By Lemma \ref{formsubst} and , we
get .
\item 

Choosing  to be fresh, by the induction hypothesis we get . 
By Lemma \ref{formsubst} and , we get . Therefore , so also .
\item

We choose  so that . By the induction hypothesis  and . By our choice of  and , we also
have . Thus also .\qed
\end{enumerate}


With the lemmas at hand, Progress and Preservation easily follow:

\begin{lemma}[Subject Reduction, Preservation]
If  and , then .
\end{lemma}
\proof By induction on the definition of . We show several cases. Case  of:
\begin{enumerate}[]
\item . The term  has
the form  and the proof 
proof tree  ends with:

By Lemma \ref{lamsl}, .
\item .
The term  has the form 
and the proof tree  ends with:

Choose  to be fresh. Thus  and . By the side-condition of the last
typing rule, , so . By Lemma
\ref{logsl} we get ,
so also . By Lemma \ref{lamsl}, we
get .
\item . In this
case the term
 is has the form   and the proof tree ends with:

The claim follows immediately.
\item . The term  has the form  and the proof tree ends with:

We choose  to be fresh. By applying -conversion we can also obtain a proof
tree of , where . Then
by Weakening we get , so also . Let the proof tree  be defined as:

Then the following proof tree shows the claim:
\qed
\end{enumerate}

\begin{lemma}[Progress]
If , then either  is a value or there is  such that .
\end{lemma}
\proof Straightforward induction on the length of . We show the cases for the
terms corresponding to \izfr\ axioms.
\begin{enumerate}[]
\item If , then  is a value.
\item If , then we have the following proof tree:

By the induction hypothesis, either  is a value or there is  such
that . In the former case, by Canonical Forms,  and . In the latter, by the evaluation rules .
\item The  terms always reduce.\qed
\end{enumerate}

\begin{corollary}\label{corlz}
If  and , then  and  is a value.
\end{corollary}

\begin{corollary}\label{corbot}
If , then  does not normalize.
\end{corollary}
\proof If  normalized, then by Corollary \ref{corlz} we would have a value of
type , which by Canonical Forms is impossible.\qed


Finally, we state the formal correspondence between  and \iizfr:

\begin{lemma}[Curry-Howard Isomorphism]\label{ch}
If  then \iizfr , where . If \iizfr , then there exists a term  such that , where .
\end{lemma}
\proof Both parts follow by easy induction on the proof. The first part is
straightforward, to get the claim simply erase the lambda terms from the
proof tree. For the second part, we show terms and trees corresponding to \iizfr\ axioms:
\begin{enumerate}[]
\item Let  be one of the \iizfr\ axioms apart from -Induction.
Then  for the axiom (A). Recall that  is an
abbreviation for . Let 
and let . 
Let  be the following proof tree:

And let  be the following proof tree:

Then the following proof tree shows the claim:

\item Let  be the -induction axiom. Let .
The following proof tree shows the claim:
\qed
\end{enumerate}

Note that all proofs in this section are constructive and quite weak from
the proof-theoretic point of view --- Heyting Arithmetic should be
sufficient to formalize the arguments. However, by the Curry-Howard isomorphism
and Corollary \ref{corbot}, normalization of  entails consistency of \iizfr,
which easily interprets Heyting Arithmetic. Therefore a normalization
proof must utilize much stronger means, which we introduce in the following
section. 

\newcommand\vl{V^{\lambda}}
\newcommand\vla{\vl_\alpha}
\newcommand\vlb{\vl_\beta}

\section{Realizability for \iizfr}\label{izfreal}

In this section we work in ZF. It is likely that \izfc\ would be sufficient, as excluded middle is not used explicitly; however, arguments using ordinals
and ranks would need to be done very carefully, as the notion of an ordinal
in constructive set theories is problematic \cite{powell, taylor96}.

Our definition of realizability is inspired by McCarty's presentation
in his Ph. D. thesis \cite{mccarty}. However, while he used it mainly to
prove independence results for \izfc\ and to carry out recursive mathematics,
we use it to prove normalization of .

The realizability relation  relates \emph{realizers} with \izfr\
formulas over an extended signature. The realizers are terms of ; the
signature is extended with class-many constants we call -names. We
proceed with the formal definitions.

\begin{definition}
The set of all values in  is denoted by . 
\end{definition}

\begin{definition}
A set  is a -name iff  is a set of pairs  such that
 and  is a -name.
\end{definition}

In other words, -names are sets hereditarily labelled by  values.

\begin{definition}
The class of -names is denoted by .
\end{definition}

Formally,  is generated by the following transfinite inductive
definition on ordinals:


The \emph{-rank} of a -name , denoted by , is the
smallest  such that .

\begin{definition}
For any ,  denotes . 
\end{definition}

\begin{definition}
An \emph{environment} is a finite partial function from first-order
variables to . 
\end{definition}
We will use the letter  to denote \emph{environments}.

The environments are used to store elements of . In order to smoothen the
presentation and make the account closer to the standard accounts of realizability for constructive set theories
\cite{mccarty,rathjendp,rathjenizf}, we make it possible for the formulas to mention constants 
from  as well. Strictly speaking this is unnecessary and we could give
the account of the realizability relation and the normalization theorem using
only environments; the cost to pay would be some loss of clarity.

Formally, we extend the first-order language of \izfr\ in the following way:

\begin{definition}
A (class-sized) first-order language  arises by enriching the \izfr\ signature
with constants for all -names.
\end{definition}

From now on until the end of this section, the letters  range over -names. 

\begin{definition}
For any formula  of , any term  of  and  defined on all free variables of
 and , we define by metalevel mutual induction a realizability relation  in an environment  and a meaning of a term  
 in an environment :
\begin{enumerate}[(1)]
\item 
\item 
\item \label{omegadef} , where  is defined by the means
of inductive definition:  is the smallest set such that:
\begin{enumerate}[]
\item  if ,  and . 
\item If , then  if , , , , . 
\end{enumerate}
Note that if , then there is a finite ordinal 
such that .
\item \label{termdef} 
\item 
\item 
\item 
\item 
\item 
\item 
\item 
\end{enumerate}
\end{definition}

Note that  iff .

The definition of the ordinal  in item \ref{termdef} 
depends on . This ordinal is close to the rank of the set denoted
by  and is chosen so that Lemma \ref{realsterms} can be proven.
Let .
Case  of:
\begin{enumerate}[]
\item  --- . 
\item  --- . 
\item  --- .
\item  --- . 
\item  --- .
\item . This case is more complicated.
The names are chosen to match the corresponding clause in the proof of Lemma \ref{realsterms}. 
Let , where
. Then for all  there is  and  such that  and . Use Collection to collect these 's in one set , so that for
all  there is  such that the property holds. Apply Replacement
to  to get the set of -ranks of sets in . Then  is
an ordinal and for any , . Therefore for all  there is  and  such that  and  holds. Set .
\end{enumerate}

\begin{lemma}
The definition of realizability is well-founded. 
\end{lemma}
\proof We define a measure function  which takes a clause in the
definition and returns a triple of natural numbers:
\begin{enumerate}[]
\item  = (``number of constants  in '',
``number of function symbols in '', ``structural complexity of '')
\item  = (``number of constants  in '', ``number of function symbols in '', 0)
\end{enumerate}
With lexicographical order in , it is trivial to check that the measure
of the definiendum is always greater than the measure of the definiens ---
the number of terms does not increase in the clauses for realizability and
the formula complexity goes down, in the clause for ,  disappears and in the rest of clauses for terms,
the topmost  disappears. Since  with lexicographical order is
well-founded, the claim follows.\qed


Since the definition is well-founded, (metalevel) inductive proofs on the
definition of realizability are justified, such as the proof of the following lemma:

\begin{lemma}\label{realsubst}
 and  iff  iff .
\end{lemma}
\proof Straightforward induction on the definition of realizability. We show representative
cases. Case  of:
\begin{enumerate}[]
\item  --- then . 
\item  --- then ,  and also .
\item . Then . By the induction
hypothesis, this set is equal to 
and also to 
 and thus to 
. 
\end{enumerate}
Case  of:
\begin{enumerate}[]
\item . We have  iff  iff  and . By
the induction hypothesis, this is equivalent to  and to
, so
also to  and to . This shows the claim.
\item . We have  iff
(choosing  to be fresh)  iff  and . By the choice of , this is equivalent to
. By the induction hypothesis, this is equivalent to

and to , from which we easily recover the claim. \qed
\end{enumerate}

\begin{lemma}\label{realnorm}
If  then .
\end{lemma}
\proof Straightforward from the definition of realizability. For ,
the claim trivially follows and in every other case the definition starts with a clause assuring normalization of .\qed

\begin{lemma}\label{realredclosed}
If  then  iff .
\end{lemma}
\proof Whether  or not depends only on the value of , which does not
change with reduction or expansion.\qed


\begin{lemma}\label{afvreal}
If  agrees with  on , then  iff . In particular, if , then  iff . 
\end{lemma}
\proof Straightforward induction on the definition of realizability --- the environment
is used only to provide the meaning of the free variables of terms in a
formula.\qed


\begin{lemma}\label{realimpl}
If  and , then . 
\end{lemma}
\proof Suppose . Then 
and for all , . Now, . Lemma \ref{realredclosed} gives us the claim.\qed


We now prove a sequence of lemmas which culminates in Lemma
\ref{realsterms}, the keystone in the normalization proof. 

\begin{lemma}\label{ineqrank}
If  then there is  such that
for all , if , then . Also, if , then .
\end{lemma}
\proof Take . Then there is  such that . Take any . If , then  and , so .

For the second part, suppose . 
This means that , so  and for all , for all , , so ,  and . Thus, for all  and for
all , . Take any element . Then , so . Thus by the first part, . 
Therefore , so , so .\qed



The following two lemmas will be used for the treatment of  in Lemma
\ref{realsterms}.

\begin{lemma}\label{realunorderedpair}
If , then .
\end{lemma}
\proof Take any . By the definition of , any such  is in , so .\qed


\begin{lemma}\label{union}
If  and , then .
\end{lemma}
\proof By the definition of , if  then
, so .\qed


\begin{lemma}\label{realsucc}
If  and , then .
\end{lemma}
\proof  means .
By Lemma \ref{ineqrank}, it suffices to show that . Applying Lemma \ref{realunorderedpair}
twice, we find that . By
Lemma \ref{union}, if , then , which
shows the claim.\qed


The following lemma states the crucial property of the realizability relation.
\begin{lemma}\label{realsterms}
 iff  and . 
\end{lemma}
\proof For all terms apart from , the left-to-right part is immediate. For the
right-to-left part, suppose  and . To show that , we need to show that .  The proof proceeds by case analysis
on . Let . Case  of:
\begin{enumerate}[]
\item . If  then anything holds, in particular . 
\item . Suppose that . Then either  or . By Lemma \ref{ineqrank}, in the former case , in the latter , so . 
\item . Suppose that . Then  and for all ,
 and .
Take any . Then . So .
By Lemma \ref{ineqrank} any such  is in , so .
\item . Suppose . Then  and there is  such that ,  and . Two
applications of Lemma \ref{ineqrank} provide the claim. 
\item . Suppose . Then  and . Lemma \ref{ineqrank} shows the claim. 
\item . Suppose . Then  and . Thus ,  and there is  such that  and . We also
have , so  and for all ,  and for all , . So taking , 
and , there is  such that ,
,  and
. Therefore  from the definition of , so 
there is  such that , ,  and . So  and . Therefore, ,  (since we can take again  and ) and . By Lemma \ref{ineqrank}, .
\end{enumerate}

Now we tackle . For the left-to-right direction, obviously . For the claim about , we proceed by induction on the
definition of :
\begin{enumerate}[]
\item The base case. Then  and , so . 
\item The inductive step. Then , ,
, , .
Therefore, there is  (namely ) such that  and . Thus , so . 
\end{enumerate}
For the right-to-left direction, suppose . Then either  or . In the former case, , so by Lemma \ref{ineqrank} . In the latter, . Thus  and there is  such that . So ,  and . This is exactly the inductive step of the
definition of , so it remains to show that . Since , there is a finite ordinal
 such that . By Lemma \ref{realsucc}, , so also  and we get the claim.\qed


\section{Normalization}\label{sectionnorm}

In this section, environments  are finite partial functions mapping 
proof variables to terms of  and first-order variables to pairs , where  and . Therefore, , where  denotes the set of proof variables
and  denotes the set of first-order variables. For any ,
 denotes the restriction of  to the mapping from first-order
variables into terms: .  
Note that any  can be used as a realizability environment by considering
only the mapping of first-order variables to . 

We first define a reduction-preserving forgetting map  on the terms of
. The map changes all first-order arguments of  and
 terms to . It is induced inductively in a natural way by the cases:

So for example,  and so on. The reduction-preserving
character of the map is captured by the following lemmas:

\begin{lemma}\label{en1}
If  then . 
\end{lemma}
\proof Straightforward. The first-order terms mapped to  do not play a
role in reductions.\qed


\begin{lemma}\label{erasurenorm}
If  normalizes, then so does . 
\end{lemma}
\proof By Lemma \ref{en1}, an infinite reduction sequence starting from  would
induce an infinite reduction sequence starting from .\qed


\begin{definition}
For a sequent ,  means that  is
defined on  and for all , .
\end{definition}

Note that if , then for any term  in ,
 is defined and so is the realizability relation .

\begin{definition}
For a sequent , if  then 
is , where  and .
Similarly, if  is defined on the free variables 
of , then  denotes . 
\end{definition}

\begin{lemma}\label{rhosubst}
If  is not defined on , then . Also
if  is not defined on , then . 
\end{lemma}
\proof Straightforward structural induction on .\qed


\begin{thm}[Normalization]\label{norm}
If  then for all , .
\end{thm}
\proof For any  term ,  in the proof denotes .
We proceed by metalevel induction on . Case  of:
\begin{enumerate}[]
\item 

Then  and the claim follows.
\item 

By the induction hypothesis,  and . Lemma
\ref{realimpl} gives the claim.
\item

Take any  and fresh . We need to show that for any , . Take any such . Let
. Then , so by the
induction hypothesis . Since  is
fresh,  is undefined on , so by Lemma \ref{rhosubst} . Therefore . Since  agrees with  on logic variables, by Lemma \ref{afvreal} we get .
\item 

By the induction hypothesis, , which is not the case, so
anything holds, in particular .
\item

By the induction hypothesis, , so  and
. Therefore .
Lemma \ref{realredclosed} gives the claim. 
\item 

Symmetric to the previous case. 
\item 

All we need to show is  and , which we
get from the induction hypothesis.
\item

We need to show that , which we get from the induction hypothesis.
\item

Symmetric to the previous case.
\item 

By the induction hypothesis, . Take  fresh, so
that  is undefined on . Therefore either  and  or  and
. We only treat the former case, the latter is symmetric.
Since , by the
induction hypothesis we get . We also
have . By Lemma
\ref{rhosubst}, , so Lemma \ref{realredclosed} gives us the claim.
\item

By the induction hypothesis, for all , . We need to show that for all , . Take any such .  Using -conversion we can assure that  is not defined on
, so it suffices to show that ,
which is equivalent to . Take any  and . By Lemma \ref{realsubst} it suffices to
show that . 
Since , by the induction hypothesis we get 
. By Lemma \ref{rhosubst}
, which shows the claim. 
\item

By the induction hypothesis, , so 
and . In particular . By Lemma \ref{realsubst}, . Since , Lemma \ref{realredclosed} gives us  the claim.
\item 

By the induction hypothesis, , so by Lemma
\ref{realsubst}, . Thus, there is a -name , namely , such that . Thus,
, which is what we want.
\item

Let . Choose  so that  is undefined on these variables. 
We need to
show .
By the induction hypothesis, , so  and
for some , . By the induction hypothesis again, for any  we have . Take
. Since , by Lemma
\ref{afvreal} . Now, .
Lemma \ref{realredclosed} gives us the claim.
\item 

By the induction hypothesis, . By Lemma \ref{realsubst} 
this is equivalent to .
By Lemma \ref{realsterms}, , so . 
\item

By the induction hypothesis, . This means that 
 and . 
By Lemma \ref{realsterms},  and .
By Lemma \ref{realsubst}, .
Moreover, 

 Lemma \ref{realredclosed} gives us the claim.
\item

Since  reduces to , by Lemma \ref{realredclosed} it suffices to show that for all ,
. We proceed by induction on -rank of . Take any . 
By the induction hypothesis, , so  and . By Lemma
\ref{realredclosed}, , so 
by Lemma \ref{realimpl}, it suffices to
show that .
Take any , , we need to show that
. As , it suffices
to show that , which, by Lemma \ref{realredclosed}, is equivalent to . 
As , the -rank of  is less than the
-rank of  and we get the claim by the induction hypothesis.\qed
\end{enumerate}

\begin{corollary}[Normalization]\label{cornorm}
If , then . 
\end{corollary}
\proof Take  mapping all free proof variables of  to themselves
and all free first-order variables  of  to . 
Then . By Theorem \ref{norm}, 
normalizes. By the definition of , . By Lemma
\ref{erasurenorm},  normalizes.\qed


Recall that in non-deterministic reduction systems, strong normalization means
that for any term , all reduction paths starting from  terminate, while weak normalization means
that for any term  there is a terminating reduction path starting from
. Our reduction system for  can be viewed as selecting a call-by-need reduction
strategy in a non-deterministic reduction system, where a reduction can be
applied anywhere inside of the term. In this view, our results show only
weak normalization of the calculus. Strong normalization then, surprisingly, does not hold. One reason, trivial, are
 terms. However, even without them, the system would not strongly
normalize, as the following counterexample, invented by M. Crabb\'e and adapted to our framework shows:

\begin{thm}[Crabb\'e's counterexample]
There is a formula  and a term  such that  and 
does not strongly normalize.
\end{thm}
\proof Let . Consider the terms:

We first show that these terms can be typed. Let  denote the following proof tree, showing that 
:

By Weakening, we can also obtain a tree  showing that . The following proof tree shows that :

We now exhibit an infinite reduction sequence starting from :
\qed
Note that the counterexample also shows that the weak
normalization of  is really weak --- although  entails weak 
normalization of ,  does not, as there is a context  such that 
 and  does not normalize. 

Moreover, a slight (from a semantic point of view) modification to \iizfr,
namely making it non-well-founded, results in a system which is not even
weakly normalizing. A very small fragment is sufficient for this effect to
arise. Let  be an intuitionistic set theory consisting of 2 axioms:

\begin{enumerate}[]
\item (C) 
\item (D) .
\end{enumerate}

The constant  denotes a non-well-founded set. The existence of  can
be derived from the Separation axiom: . The lambda calculus corresponding to  is defined just as for \iizfr.

\begin{lemma}\label{dc}

\end{lemma}
\proof It suffices to show that . Take any , then . On the
other hand, suppose . Since obviously , we
also get .\proof


\begin{thm}\label{notweakly}
There is a formula  and a term  such that  and 
does not weakly normalize.
\end{thm}
\proof Let  be the lambda term corresponding to the proof of Lemma \ref{dc} along
with the proof tree . Take . Consider the terms:

Again, we first show that these terms are typable. Let  be the following proof
tree, showing that :

Then the following proof tree shows that  is typable:

Finally, we exhibit the only reduction sequence starting from :
\qed


These counterexamples to normalization properties can also be presented in a
cleaner way in the framework of higher-order rewriting \cite{jawst2006}. 

\section{Applications}\label{secapp}

The normalization theorem immediately provides several results. 

\begin{corollary}[Disjunction Property]
If \iizfr , then \iizfr  or \iizfr . 
\end{corollary}
\proof Suppose \iizfr . By the Curry-Howard isomorphism, there is a
 term  such that . By Corollary
\ref{corlz},  and . By
Canonical Forms, either  and  or 
and . By applying the other direction of the Curry-Howard isomorphism
we get the claim.\qed


\begin{corollary}[Term Existence Property]
If \iizfr , then there is a closed term  such that \iizfr . 
\end{corollary}
\proof By the Curry-Howard isomorphism, there is a -term  such that . By normalizing  and applying
Canonical Forms, we get  such that  and thus by
the Curry-Howard isomorphism \iizfr . If  is not closed already, 
then let . We have \iizfr ,
so also .\qed


To show NEP, we first define an extraction function  
which takes a proof  and returns a natural number .
 works as follows:

It normalizes  to . By Canonical Forms, .  then normalizes  to
either  or . In the former case,  returns . In the
latter, . Normalizing  it
gets , where . Normalizing
 it obtains  such that . Then  returns . 

To show that  terminates for all its arguments, consider the
sequence  of
terms  obtained throughout the execution of .
We have \iizfr , \iizfr , \iizfr 
and so on. The length of the sequence is therefore exactly the natural
number denoted by . 

\begin{corollary}[Numerical Existence Property]
If \iizfr , then there is a natural number
 and term  such that \iizfr . 
\end{corollary}
\proof As before, use the Curry-Howard isomorphism to get a value  such that . Thus , so  and .
Take . By patching together
the proofs \iizfr , \iizfr , {\ldots}
,\iizfr  obtained throughout the execution of , we get \iizfr .\qed


This version of NEP differs from the one usually found in the literature,
where in the end  is derived. However, \iizfr\ does not have the
Leibniz axiom for the final step. We conjecture that it is the only version
which holds in non-extensional set theories. More specifically, we
conjecture that there is a term  and formula  such that \iizfr  and \iizfr\ does not prove . 

\section{Extensional \izfr}\label{lei}

We will show that we can extend our results to full \izfr. We work in \iizfr.

\begin{lemma}
Equality is an equivalence relation.
\end{lemma}
\proof Straightforward.\qed


\begin{definition}
A set  is \emph{L-stable}, if  and  implies . 
\end{definition}

Thus, L-stable sets are well-behaved as far as the atomic version of the
Leibniz axiom () is concerned. 

\begin{definition}
A set  is \emph{transitively L-stable} (we say that TLS(C) holds) if it is L-stable and every
element of  is transitively L-stable. 
\end{definition}

This definition is formalized in a standard way, using transitive closure, available
in \iizfr, as shown e.g. in \cite{ar}. We denote the class of transitively L-stable sets
by . The statement  stands for . The class  in
\iizfr\ plays a similar role to the class of well-founded sets in ZF without
Foundation. 

\begin{lemma}
\izfr . 
\end{lemma}
\proof Straightforward -induction.\qed


The restriction of a formula  to , denoted by , is defined
as usual, taking into account the following translation of terms:


The notation  means that  holds. 
\begin{lemma}
 is transitive. 
\end{lemma}
\proof Take any  in  and suppose . Then by the definition of ,  as well.\qed


\begin{lemma}\label{t1}
If  and , then . 
\end{lemma}
\proof This is \emph{not} obvious, as there is no Leibniz axiom in the logic. 
Suppose  and . Since , . Since  is L-stable, , so also . Thus  is L-stable. 

If , then . Since  and  is transitive, . Thus  is transitively L-stable.\qed


\begin{lemma}
Equality is absolute for .
\end{lemma}
\proof Take any . Suppose . This means that for all ,
. As  is transitive, this is equivalent to for all
, , so also  in the real world. On the
other hand, if , then obviously also
.\qed


The following three lemmas are essentially used to show that  is closed
under the axioms of \izfr. 

\begin{lemma}\label{omegat}
. If , then . 
\end{lemma}
\proof That  is obvious. Take any . To show that , suppose  and . If , then by  we have  and . If , then , so
also  and by Lemma \ref{t1} . In both cases  which shows the claim.\qed


The following two lemmas are proved together by mutual induction on the
definition of terms and formulas. 

\begin{lemma}\label{trieq}
For any term , .
\end{lemma}
\proof Case  of:
\begin{enumerate}[]
\item , . The claim is trivial.
\item . It suffices to show that . We show by
-induction on  that . Take any . 
Then either  or there is  such that . Take
any  such that . In the former case , so  and by Lemmas
\ref{t1} and \ref{omegat} we get . In the latter case, take this . We have ,
so . By , , so by the induction hypothesis
, thus by Lemma \ref{omegat} we also get . 
\item . By the induction hypothesis, 
 and .
In order to show that , take any . Then
either  or , so either
 or , in both cases . The other direction is symmetric and
we get . 

Furthermore, by the induction hypothesis,  and
. Thus in both cases by Lemma \ref{t1}, . Suppose .
Then either , or . In both cases . Thus
we have shown that . 
\item . Take any . By the induction hypothesis, . Thus there is  such that
. Thus also , so . The other direction is symmetric and we get 
. 

Furthermore, by the induction hypothesis, , 
so by transitivity of ,  and also . Finally, 
suppose that . Then since , , so . This shows the claim. 
\item . By the induction hypothesis, . Suppose . Then  and . Thus also , so . The other direction is symmetric and we get . 

Suppose . Since , by Lemma \ref{t1} . 
It is easy to see that also , so .
\item . Suppose . Then
.
By the induction hypothesis,  and
. Thus, by transitivity of , .
Moreover, by the induction hypothesis,  and
. Therefore . By Lemma \ref{tril} we get . This shows
that . The
other direction is symmetric and we get . 

Suppose . By Lemma \ref{t1}, . Since , . By
Lemma \ref{tril},  holds. Thus . 
\item . Suppose  and . This means that:
\begin{enumerate}[]
\item .
Take any . By the induction hypothesis, . Thus there is  such that  and . We will now show that there is exactly one 
such that . Take . 
By the induction hypothesis, . 
By Lemma \ref{tril}, .
Take any  and assume . By Lemma
\ref{tril}, , so . Thus we have
shown that . 
\item . Take this .
By Lemma \ref{t1}, , so by Lemma \ref{tril}, .  
Moreover, by Lemma \ref{tril}, . 
Thus there is  such that .
\end{enumerate}
Altogether, this shows that . The other direction is symmetric and we get
. We have
also shown that , so the proof is complete.\qed
\end{enumerate}


\begin{lemma}\label{tril}
. In other words, . 
\end{lemma}
\proof We show representative cases. Case  of:
\begin{enumerate}[]
\item  for some terms . We need to show
that if ,  and , then . By Lemma \ref{trieq},  and .
Therefore , which entails .
\item . Take any , assume ,   and . By the induction
hypothesis for ,  . Using the assumption we
obtain . By the induction hypothesis for  we get 
. 
\item . Take any ,
assume  and . Then there is
a set  such that  holds. By the induction
hypothesis, merging  with , we get , so also .\qed
\end{enumerate}


\begin{thm}\label{tlsmodel}
\izfr. In other words,  is an inner model of \izfr. 
\end{thm}
\proof We proceed axiom by axiom. 
\begin{enumerate}[]
\item (EMPTY) Straightforward. 
\item (PAIR) Take any . That  satisfies the (PAIR) axiom in T 
follows by absoluteness of equality.
\item (UNION) Take any . Suppose . Then there is
some  such that . Since  is transitive, . On the
other hand, if there is  such that , then obviously . 
\item (INF) Suppose . Then either  or there is  such that . We need to show that either  or there
is  such that  and . If , the claim
is trivial. Otherwise, suppose there is  such that .
Then , so by transitivity of , . 
We also know that  and . The claim follows. 

On the other hand, suppose  or there is  such that  and . In both cases,  is trivially in . 
\item (POWER) Take any . Suppose . Then , so also for all , . On the other hand,
suppose that for all , .  
To show that , we need to show that  and for all , . We already have the former. To show the latter, note that by
transitivity of , any  is also in , so by the assumption in
. This shows the claim.
\item (SEP) Take any  and suppose . Then  and , which is what we need. 
On the other hand, if  and , then
also .
\item (REPL) Take any  such that
. This is equivalent to 
.
Since  and  is closed under equality, it is also equivalent to , which is what we want. 
\item (IND) Take  and suppose that . We have to
show that .
We proceed by -induction on . Take any . By the assumption
instantiated with , . We have to show that . It suffices to show 
that . Take any . By the induction hypothesis for , we get  and
the claim.
\item (L) Follows by Lemma \ref{tril}.\qed
\end{enumerate}


\begin{lemma}\label{tt}
For any term  and any formula , \izfr .
\end{lemma}
\proof By induction on the generation of terms and formulas. 
Case  of:
\begin{enumerate}[]
\item . The proof is obvious. 
\item . By the induction hypothesis,  and
. So if , then  or , so 
. The other direction is symmetric.
\item . By the induction hypothesis, . If , then there is
 such that , so  and .
The other direction is symmetric.
\item . By the induction hypothesis, . If , 
then , so also  and consequently . 
On the other hand, if , then by  we also get , so . 
\item . By the induction hypothesis,
. Suppose . Then , so . Since 
 and we work in \izfr, . By the
induction hypothesis, , so . The other direction is symmetric. 
\item . By the induction hypothesis,  and . Suppose . Then:
\begin{enumerate}[]
\item For all  there is exactly one  such that . By the induction hypothesis and , we also have for all 
there is exactly one  such that .  
\item There is  such that  and . Then also there is  such that .
\end{enumerate}
Altogether, . The other direction is
similar. 
\end{enumerate}
For the formulas, we show representative cases. Case  of:
\begin{enumerate}[]
\item . By the induction hypothesis,  and , so by the Leibniz axiom  is equivalent to .
\item . Suppose , then since  we
have . By the induction hypothesis, . The other direction is similar.\qed
\end{enumerate}


\begin{lemma}\label{liff}
\izfr  iff \iizf . 
\end{lemma}
\proof The left-to-right direction follows by Theorem \ref{tlsmodel}. For the
right-to-left direction, if \iizf , then also \izfr 
and Lemma \ref{tt} shows the claim.\proof


\begin{corollary}\label{dpnep}
\izfr\ satisfies DP, NEP and  TEP. 
\end{corollary}
\proof For DP, suppose \izfr . By Lemma \ref{liff},
\iizf . By DP for \iizf, either \iizf  or
\iizf . Using Lemma \ref{liff} again we get either \izfr  or
\izfr .

For NEP, suppose \izfr . By Lemma
\ref{liff}, \iizf , so \iizf . Since , using NEP for \iizf\ we get
a natural number  such that \iizf , thus also \iizf . By Lemma \ref{liff} and , we get \izfr . By the Leibniz axiom, \izfr .

For TEP, suppose \izfr . By Lemma \ref{liff}, \iizf
. By TEP for \iizf, there is a term  such
that \iizf . This implies \izfr . By Lemma
\ref{tt}, , so by the Leibniz axiom in \izfr\ we get \izfr . 
Since , by Lemma \ref{tt} we get \izfr .\qed


\begin{corollary}[Set Existence Property]\label{sep}
If \izfr  and  is term-free, then
there is a term-free formula  such that \izfr .
\end{corollary}
\proof Take the closed  from Term Existence Property, so that \izfr . By Corollary \ref{tdef} there is a term-free formula  defining , so that
\izfr . Then \izfr  can be easily derived.\qed


\ignore{

We cannot establish TEP and SEP as easily, since it is not
the case that  for all terms . The problem lies in terms
corresponding to the Power Set, Separation and Replacement axioms. 
However, a simple modification to the axiomatization of \izfr\ yields these results too.
It suffices to incorporate the restriction to  into troublesome terms.
Since in the extensional universe  holds, the modification is harmless.

More formally, we modify three axioms of \izfr\ and add one new, 
axiomatizing transitive closure. Let  be the formula `` and  is transitive''. The axioms are:
\begin{enumerate}[]
\item (SEP') 
\item (POWER') 
\item (REPL') 
\item (TC) . 
\end{enumerate}

In the modified axioms, the definition of  is written using  and
relativization of formulas to  this time leaves terms intact, we set  for all terms . 

It is not difficult to see that this axiomatization is equivalent to the old
one and is still a definitional extension of term-free versions of \cite{myhill72}, \cite{beesonbook} and \cite{frsce3}.We can therefore adopt it as the official axiomatization of \izfr.
All the developments in sections 4-8 can be done for the new axiomatization in the similar way. 
In the end we get:
\begin{corollary}\label{dpneptepsep}
\izfr\  satisfies DP, NEP, TEP and SEP. 
\end{corollary}
}

A different technique to tackle the problem of Leibniz axiom, used by Friedman in \cite{friedmancons}, 
is to define new membership () and equality () relations in an intensional universe from scratch, so that  interprets his intuitionistic set theory along with 
Leibniz axiom. Our , on the other hand, utilizes existing  relations.
We present an alternative normalization proof, where the method to
tackle Leibniz axiom is closer to Friedman's ideas, in \cite{jatrinac2006}.

\section{Related work}\label{others}

Several normalization results for impredicative constructive set theories
much weaker than \izfr\ exist. Bailin
\cite{bailin88} proved  strong normalization of a constructive set theory
without the induction and replacement axioms. Miquel 
interpreted a theory of similar strength in a Pure Type System
\cite{miquelpts}. In \cite{miquel} he also defined a strongly normalizing
lambda calculus with types based on ,
capable of interpreting \izfc\ without the -induction axiom. This result was
later extended --- Dowek and Miquel \cite{dowek} interpreted a version of constructive
Zermelo set theory in a strongly normalizing deduction-modulo system.

Krivine \cite{krivine} defined realizability using lambda calculus for classical set theory conservative
over ZF. The types for the calculus were defined. However, it seems that the types
correspond more to the truth in the realizability model than to provable
statements in the theory. Moreover, the calculus does not even weakly normalize.

The standard metamathematical properties of theories related to \izfr\ are well investigated.
Myhill \cite{myhill72} showed DP, NEP, SEP and TEP for IZF with Replacement and
non-recursive list of set terms. Friedman and \^S\^cedrov \cite{frsce1} showed SEP and
TEP for an extension of that theory with countable choice
axioms. Recently DP and NEP were shown for \izfc\ extended with various choice principles by Rathjen \cite{rathjenizf}.
However, the technique does not seem to be strong enough to provide TEP and SEP.

In \cite{jatrinac2006}, we show normalization of \izfr\ extended with
-many inaccessible sets.

\section*{Acknowledgments}

I would like to thank my advisor, Bob Constable, for support and for
giving me the idea for  and this research, Richard Shore for
helpful discussions, David Martin for commenting on my ideas, Daria
Walukiewicz-Chrz{\fontencoding{T1}\selectfont\k a}szcz for the
higher-order rewriting counterexample, thanks to which I could prove
Theorem \ref{notweakly} and anonymous referees for helpful comments.
\bibliographystyle{alpha} \bibliography{latex8}

\end{document}
