\subsection{Approximating the Interval Assignment LP}
\label{sec:interval}
Before we describe the general approximation , we consider two special
cases for insight.

\paragraph{Example: single edge.}
Suppose the tree consists of a single unit-length edge ,
all requests reside at , and all terminals at . In this case,  for all pairs  and  so
the second set of constraints in \eqref{lp:interval} is simply
 where we
write  for . A minimum solution satisfies these
constraints with equality. Summing over , we get
that in this case \eqref{lp:interval} is equivalent to


This is exactly the linear relaxation for the hitting set\footnote{A
  subset  of a universe  is a \emph{hitting set} for
   if  for all .} problem where the sets we want to hit are
intervals. While the general hitting set problem is hard, it turns out
that this special case can be solved exactly in polynomial time and
the relaxation has no integrality gap\footnote{One way to see this is
  that the columns of the constraint matrix has consecutive ones, and
  thus the constraint matrix is totally unimodular.}. Thus, we get an
optimal solution via a reduction to the minimum interval hitting set
problem: compute a minimum hitting set  for the set of intervals
, and then add  to  for .

\paragraph{Example: two edges.}
Suppose the tree is a line graph consisting of three vertices ,
 and  with unit-length edges  and  (Figure~\ref{fig:examples}(a)). Requests reside at  and , and all terminals at
. For each  and  residing at , we have . For each  and  residing at  we have . Thus feasible solutions to
\eqref{lp:interval} satisfy the constraints

The constraints suggest that the vector  is a
fractional hitting set for the collection of intervals
, and  for
.
In light of the single-edge special case, a naive approach is to first
compute minimum hitting sets  and  for
 and , respectively. Then we add
 to  for , and  to  for . However, the resulting edge sets may not be connected.
Instead, we make use of the following crucial facts:
\begin{enumerate}
\item[(1)] We should include  in  only if , and,
\item[(2)] Minimal hitting sets are at most twice minimum fractional
  hitting sets (see Lemma \ref{lem:disjoint-intervals}).
\end{enumerate}
These facts suggest that we should first compute a minimal hitting set
 for  and then compute a minimal hitting set
 for  with the constraint that . This is a valid solution to \eqref{lp:interval}
since . We proceed as usual
to compute . The resulting  is connected by (1) and
 by (2).

\begin{figure}
\begin{center}
\begin{tabular}{ccc}
\includegraphics[height=2.0 in]{two-edges.pdf}
& \hspace{1 in}  & 
\includegraphics[height=2.0 in]{greedy-extension.pdf}\\
(a) & \hspace{1 in}  &   (b)\\
\end{tabular}
\end{center}
\caption{(a) The two-edge example; (b)  is the path from  to , for . Note that arc  precedes  and no arc precedes .}   
\label{fig:examples}
\end{figure}

\paragraph{General case.} 


Motivated by the two-edge example, at a high level, our approach for
the general case is as follows:
\begin{enumerate}
\item We construct interval hitting set instances over each edge.
\item We solve these instances starting from the edges nearest to the
  paths  first.
\item We iteratively ``extend'' solutions for the instances nearer the
  paths to get minimal hitting sets for the instances further from
  the paths.
\end{enumerate}
We then use Lemma \ref{lem:disjoint-intervals} to argue a -approximation
on an edge-by-edge basis.









Figure~\ref{fig:examples}(b) gives an example of an instance of
interval request cover. Note that whether an edge is closer to some
 along a path  for some  depends on which direction we
are considering the edge in. We therefore modify \eqref{lp:interval}
to include directionality of edges, replacing each edge  with
bidirected arcs and directing the paths  from  to .




For every edge  and window , there is a single orientation of
edge  that belongs to  for some . So there is a 1-1
correspondence between the variables  in \eqref{lp:interval}
and the variables  in \eqref{lp:d-interval}, and the two LPs are
equivalent. Henceforth we focus on \eqref{lp:d-interval}.



Before presenting our final approximation we need some more notation.

\begin{definition}
  For each request , we define  to be the directed path from
   to . For each arc , we
  define 
and the set of intervals . We
  say that  is a \emph{cut arc} if .

  We say that an arc  \emph{precedes} arc , written , if there exists a directed path in the tree containing both the
  arcs and  appears after  in the path.
\end{definition}

\begin{lemma}
  \label{lem:fractional-feasible}
  Feasible solutions  of \eqref{lp:d-interval} satisfy the
  following set of constraints for all arcs :
\end{lemma}

\begin{proof}
  Let  be a feasible solution of \eqref{lp:d-interval}. Fix an
  arc  and . For each , we have 
  since  is a path from  to a connected subgraph containing
  . By feasibility, we have . Summing over
  , we get  where the last inequality
  follows from feasibility. 
\end{proof}





We are now ready to describe the algorithm. At a high level, Algorithm
\ref{alg:greedy-ext} does the following: initially, it finds a cut arc
 with no cut arc preceding it and computes a minimal hitting set
 for ; iteratively, it finds a cut arc  whose
preceding cut arcs have been processed previously, and minimally
``extends'' the hitting sets  computed previously for the
preceding arcs  to form a minimal hitting set .



\begin{algorithm}
\caption{Greedy extension}
  \begin{algorithmic}[1]
   \label{alg:greedy-ext}
    \STATE  
    \STATE  for all 
    \STATE  for all arcs 
    \WHILE {}
\STATE Let  be any arc in 
    \WHILE {there exists  in }
    \STATE 
    \ENDWHILE
    \STATE Let 
\STATE 
    \STATE Set  to be a minimal hitting set
    for the intervals 
    \STATE  for all \STATE 
    \ENDWHILE
\STATE jA_iP_i for all 
    \STATE  for all 
    \RETURN , 
  \end{algorithmic}
\end{algorithm}

We prove that Algorithm \ref{alg:greedy-ext} actually manages to
process all cut arcs  and that  is a hitting set for
. First, we make the following observation.

\begin{lemma}
  \label{lem:invariants}
  For each iteration, the following holds.
  \begin{enumerate}
  \item If , the inner `while' loop finds an arc.
  \item  is
    a hitting set for the intervals .
  \end{enumerate}
\end{lemma}

\begin{proof}
  Since we have a bidirected tree and an arc does not precede its
  reverse arc, the inner `while' loop does not repeat arcs and hence it stops with
  some arc. This proves the first statement.


  We prove the second statement by induction on the algorithm's
  iterations. In the first iteration, the set  consists of cut arcs
  so  for all cut arcs . Therefore, for all
  ,  is the arc on  closest to
   and . This proves the base case. Now we prove the
  inductive case.
Fix an interval . If  is the arc
  on  closest to  and , then . If not, then there exists a neighboring arc  closer to . We have that
   and . Since the
  algorithm has processed all cut arcs preceding , by the inductive
  hypothesis we have . This
  implies that  is a hitting set for  and
  so .
  Hence,  is a hitting set for .  \end{proof}



Let  be the set of edges whose corresponding arcs are in 
and , i.e. the undirected version of
.
\begin{lemma}
  \label{lem:connected}
   is an interval request cover.
\end{lemma}

\begin{proof}
The connectivity of  follows from the fact that the
  algorithm starts with , and in each iteration an
  arc  is added to  only if  or  is
  incident to some edge previously added to .

  Now it remains to show that  for all
  requests , i.e. that there exists  such that
   is incident to  or . If , then
  . On the other hand if
  , then let  be the arc
  incident to . Since the algorithm processes all cut arcs, we have  and thus  is incident to
  . In both cases, we have
  .  \end{proof}

Next, we analyze the cost of the algorithm. Let  be the number
of disjoint intervals in .
\begin{lemma}
  \label{lem:disjoint-intervals}
   for all arcs .
\end{lemma}

\begin{proof}
  Let  be the elements of . For each
  , there exists an interval  such that , because otherwise  would still be
  a hitting set, contradicting the minimality of . We observe
  that the intervals  and  are
  disjoint since  contains  and
   contains  but neither contains
  . Therefore, the set of  intervals
  1 \leq l \leq |M(a)|l
  is disjoint.  
\end{proof}







\begin{lemma}
.
\end{lemma}

\begin{proof}
  Fix an arc . From Lemmas \ref{lem:fractional-feasible} and
  \ref{lem:disjoint-intervals}, we get
  
  Since , we have
   
\end{proof}






Together with Lemmas \ref{lem:2-strict} and
\ref{lem:fract-interval-LP}, we have that  is a
-strict request cover of length at most . Lemmas \ref{lem:rcoveropt} and \ref{lem:construct}
imply that  travels at most  and
uses a buffer of capacity . This gives us the following theorem.

\begin{theorem}
  \label{thm:trees}
  There exists an offline -bicriteria
  approximation for RBP when the underlying metric is a weighted tree.
\end{theorem}
Using tree embeddings of \cite{FRT}, we get
\begin{theorem}
  \label{thm:general}
  There exists an offline -bicriteria approximation for RBP over
  general metrics.
\end{theorem}