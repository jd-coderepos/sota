\documentclass{article} 


\usepackage{hyperref}
\usepackage{url}
\usepackage{booktabs}       \usepackage{amsfonts}       \usepackage{nicefrac}       \usepackage{microtype}      

\usepackage{amsmath,amsfonts,amsthm}       \usepackage{subcaption}
\usepackage{bm}
\usepackage{bbm}
\usepackage{multirow}
\usepackage[inline]{enumitem}
\usepackage{diagbox}
\usepackage[capitalise]{cleveref}  \usepackage{comment}
\usepackage{etoolbox}
\usepackage{graphicx}
\usepackage{wrapfig}
\usepackage[font=small]{caption}
\usepackage{algorithm,algorithmicx,algpseudocode}
\usepackage{subcaption}
\usepackage{tablefootnote}


\usepackage{pifont}\newcommand{\cmark}{\ding{51}}\newcommand{\xmark}{\ding{55}}


\newtheorem{theorem}{Theorem}
\newtheorem{lemma}{Lemma}[section]
\newtheorem{corollary}[lemma]{Corollary}
\newtheorem{observation}[lemma]{Observation}
\newtheorem{proposition}[theorem]{Proposition}
\newtheorem{definition}{Definition}
\newtheorem{remark}[lemma]{Remark}
\newtheorem{claim}{Claim}

\newcommand{\dt}{\Delta}
\newcommand{\dd}{\mathop{}\!d}
\DeclareMathOperator{\hippo}{\mathsf{hippo}}
\DeclareMathOperator*{\diag}{diag}

\newcommand{\methodabbrv}{S4}

\usepackage[square,numbers,sort]{natbib}
\setlength{\textwidth}{6.5in}
\setlength{\textheight}{9in}
\setlength{\oddsidemargin}{0in}
\setlength{\evensidemargin}{0in}
\setlength{\topmargin}{-0.5in}
\newlength{\defbaselineskip}
\setlength{\defbaselineskip}{\baselineskip}
\setlength{\marginparwidth}{0.8in}
\setlength{\parskip}{5pt}\setlength{\parindent}{0pt}\usepackage[dvipsnames]{xcolor}         


\title{Efficiently Modeling Long Sequences with Structured State Spaces}
\usepackage{authblk}
\author[]{Albert Gu}
\author[]{Karan Goel}
\author[]{Christopher R{\'e}}
\affil[]{Department of Computer Science, Stanford University}
\affil[]{{\texttt{\{albertgu,krng\}@stanford.edu}, \texttt{chrismre@cs.stanford.edu}}}
\date{}

\bibliographystyle{plainnat}


\begin{document}


\maketitle

\begin{abstract}
  A central goal of sequence modeling is designing a single principled model that can address sequence data across a range of modalities and tasks, particularly on long-range dependencies.
  Although conventional models including RNNs, CNNs, and Transformers have specialized variants for capturing long dependencies, they still struggle to scale to very long sequences of  or more steps.
  A promising recent approach proposed modeling sequences by simulating the fundamental state space model (SSM) , and showed that for appropriate choices of the state matrix , this system could handle long-range dependencies mathematically and empirically.
  However, this method has prohibitive computation and memory requirements, rendering it infeasible as a general sequence modeling solution.
  We propose the Structured State Space sequence model (\methodabbrv{}) based on a new parameterization for the SSM, and show that it can be computed much more efficiently than prior approaches while preserving their theoretical strengths.
  Our technique involves conditioning  with a low-rank correction, allowing it to be diagonalized stably and reducing the SSM to the well-studied computation of a Cauchy kernel.
  \methodabbrv{} achieves strong empirical results across a diverse range of established benchmarks, including (i) 91\% accuracy on sequential CIFAR-10 with no data augmentation or auxiliary losses, on par with a larger 2-D ResNet, (ii) substantially closing the gap to Transformers on image and language modeling tasks, while performing generation  faster (iii) SoTA on every task from the Long Range Arena benchmark, including solving the challenging Path-X task of length 16k that all prior work fails on, while being as efficient as all competitors.\footnote{Code is publicly available at \url{https://github.com/HazyResearch/state-spaces}.}
\end{abstract}



\section{Introduction}
\label{sec:intro}


A central problem in sequence modeling is efficiently handling data that contains long-range dependencies (LRDs).
Real-world time-series data often requires reasoning over tens of thousands of time steps, while few sequence models address even thousands of time steps.
For instance, results from the long-range arena (LRA) benchmark~\citep{tay2021long} highlight that sequence models today perform poorly on LRD tasks,
including one (Path-X) where no model performs better than random guessing.










\begin{figure}[!t]
    \centering
    \includegraphics[width=\linewidth]{figs/properties.pdf}
    \caption{
      (\textbf{Left}) State Space Models (SSM) parameterized by matrices  map an input signal  to output  through a latent state .
      (\textbf{Center}) Recent theory on continuous-time memorization derives special  matrices that allow SSMs to capture LRDs mathematically and empirically.
      (\textbf{Right}) SSMs can be computed either as a recurrence (left) or convolution (right).
      However, materializing these conceptual views requires utilizing different representations of its parameters (\textcolor{BrickRed}{red}, \textcolor{RoyalBlue}{blue}, \textcolor{ForestGreen}{green}) which are very expensive to compute.
      \methodabbrv{} introduces a novel parameterization that efficiently swaps between these representations, allowing it to handle a wide range of tasks, be efficient at both training and inference, and excel at long sequences.
    }
\label{fig:properties}
\end{figure}


Since LRDs are perhaps {the} foremost challenge for sequence models, all standard model families such as continuous-time models (CTMs), RNNs, CNNs, and Transformers include many specialized variants designed to address them.
Modern examples include orthogonal and Lipschitz RNNs \citep{arjovsky2016unitary,erichson2021lipschitz} to combat vanishing gradients, dilated convolutions to increase context size \citep{bai2018empirical,oord2016wavenet}, and an increasingly vast family of efficient Transformers that reduce the quadratic dependence on sequence length \citep{katharopoulos2020transformers,choromanski2020rethinking}.
Despite being designed for LRDs, these solutions still perform poorly on challenging benchmarks such as LRA~\citep{tay2021long} or raw audio classification~\citep{gu2021lssl}.


An alternative approach to LRDs was recently introduced based on the \textbf{state space model (SSM)} (\cref{fig:properties}).
SSMs are a foundational scientific model used in fields such as control theory, computational neuroscience, and many more, but have not been applicable to deep learning for concrete theoretical reasons.
In particular, \citet{gu2021lssl} showed that deep SSMs actually struggle even on simple tasks,
but can perform exceptionally well when equipped with special state matrices  recently derived to solve a problem of continuous-time memorization \citep{voelker2019legendre,gu2020hippo}.
Their Linear State Space Layer (LSSL) conceptually unifies the strengths of CTM, RNN and CNN models, and provides a proof of concept that deep SSMs can address LRDs in principle.

Unfortunately, the LSSL is infeasible to use in practice because of prohibitive computation and memory requirements induced by the state representation.
For state dimension  and sequence length , computing the latent state requires  operations and  space -- compared to a  lower bound for both.
Thus for reasonably sized models (e.g.  in \citet{gu2021lssl}), the LSSL uses orders of magnitude more memory than comparably-sized RNNs or CNNs.
Although theoretically efficient algorithms for the LSSL were proposed, we show that these are numerically unstable.
In particular, the special  matrix is highly non-normal in the linear algebraic sense, which prevents the application of conventional algorithmic techniques. Consequently, although the LSSL showed that SSMs have strong performance, they are currently computationally impractical as a general sequence modeling solution.

In this work, we introduce the \textbf{Structured State Space (\methodabbrv)} sequence model based on the SSM that solves the critical computational bottleneck in previous work.
Technically, \methodabbrv{} reparameterizes the structured state matrices  appearing in \citet{voelker2019legendre,gu2020hippo} by decomposing them as the sum of a low-rank and normal term.
Additionally, instead of expanding the standard SSM in coefficient space, we compute its truncated generating function in frequency space, which can be simplified into a multipole-like evaluation.
Combining these two ideas, we show that the low-rank term can be corrected by the Woodbury identity while the normal term can be diagonalized stably,
ultimately reducing to a well-studied and theoretically stable Cauchy kernel \citep{pan2001structured,pan2017fast}.
This results in  computation and  memory usage, which is essentially tight for sequence models.
Compared to the LSSL, \methodabbrv{} is up to  faster with  less memory usage, while exceeding the LSSL's performance empirically.






Empirically, \methodabbrv{} significantly advances the state-of-the-art for LRD.
On the LRA benchmark for efficient sequence models, \methodabbrv{} is as fast as all baselines while outperforming them by  points on average.
\methodabbrv{} is the first model to solve the difficult LRA Path-X task (length-), achieving \textbf{88\% accuracy compared to 50\% random guessing} for all prior work.
On speech classification with length- sequences, \methodabbrv{} halves the test error () of specialized Speech CNNs --
by contrast, all RNN and Transformer baselines fail to learn ( error).



\paragraph{Towards a general-purpose sequence model.}
Beyond LRD, a broad goal of machine learning is to develop a single model that can be used across a wide range of problems.
Models today are typically specialized to solve problems from a particular domain (e.g. images, audio, text, time-series), and enable a narrow range of capabilities (e.g. efficient training, fast generation, handling irregularly sampled data).
This specialization is typically expressed via domain-specific preprocessing, inductive biases, and architectures. Sequence models provide a general framework for solving many of these problems with reduced specialization
-- e.g. Vision Transformers for image classification with less 2D information \citep{dosovitskiy2020image}. However, most models such as Transformers generally still require substantial specialization per task to achieve high performance.





Deep SSMs in particular have conceptual strengths that suggest they may be promising as a general sequence modeling solution.
These strengths include a principled approach to handling LRDs, as well as the ability to move between continuous-time, convolutional, and recurrent model representations, each with distinct capabilities (\cref{fig:properties}). Our technical contributions enable SSMs to be applied successfully to a varied set of benchmarks with minimal modification: \begin{itemize}[leftmargin=*]
    \item {\it Large-scale generative modeling.}
        On CIFAR-10 density estimation, \methodabbrv{} is competitive with the best autoregressive models ( bits per dim). On WikiText-103 language modeling, \methodabbrv{} substantially closes the gap to Transformers (within  perplexity), setting SoTA for attention-free models.
    \item {\it Fast autoregressive generation.}
        Like RNNs, \methodabbrv{} can use its latent state to perform  faster pixel/token generation than standard autoregressive models on CIFAR-10 and WikiText-103.
    \item {\it Sampling resolution change.}
        Like specialized CTMs, \methodabbrv{} can adapt to changes in time-series sampling frequency without retraining, e.g.\ at  frequency on speech classification.
    \item {\it Learning with weaker inductive biases.}
        With no architectural changes, \methodabbrv{} surpasses Speech CNNs on speech classification, outperforms the specialized Informer model on time-series forecasting problems,
        and matches a 2-D ResNet on sequential CIFAR with over  accuracy.



\end{itemize}









 

\section{Background: State Spaces}
\label{sec:background}

\cref{sec:ss-continuous,sec:ss-memory,sec:ss-convolution,sec:ss-recurrent} describe the four properties of SSMs in \cref{fig:properties}:
the classic continuous-time representation, addressing LRDs with the HiPPO framework, the discrete-time recurrent representation, and the parallelizable convolution representation.
In particular, \cref{sec:ss-convolution} introduces the SSM convolution kernel ,
which is
the focus of our theoretical contributions in \cref{sec:s4}.

\subsection{State Space Models: A Continuous-time Latent State Model}
\label{sec:ss-continuous}

The state space model is defined by the simple equation \eqref{eq:1}.
It maps a 1-D input signal  to an -D latent state 
before projecting to a 1-D output signal .

SSMs are broadly used in many scientific disciplines
and related to latent state models such as Hidden Markov Models (HMM).
Our goal is to simply use the SSM as a black-box representation in a deep sequence model,
where  are parameters learned by gradient descent.
For the remainder of this paper, we will omit the parameter  for exposition (or equivalently, assume  because the term  can be viewed as a skip connection and is easy to compute.

\subsection{Addressing Long-Range Dependencies with HiPPO}
\label{sec:ss-memory}



Prior work found that the basic SSM \eqref{eq:1} actually performs very poorly in practice.
Intuitively, one explanation is that linear first-order ODEs solve to an exponential function,
and thus may suffer from gradients scaling exponentially in the sequence length (i.e., the vanishing/exploding gradients problem~\citep{pascanu2013difficulty}).
To address this problem, the LSSL leveraged the HiPPO theory of continuous-time memorization \citep{gu2020hippo}.
HiPPO specifies a class of certain matrices  that when incorporated into \eqref{eq:1}, allows the state  to memorize the history of the input .
The most important matrix in this class is defined by equation \eqref{eq:hippo},
which we will call the HiPPO matrix. For example, the LSSL found that simply modifying an SSM from a random matrix  to equation \eqref{eq:hippo} improved its performance on the sequential MNIST benchmark from  to .


\subsection{Discrete-time SSM: The Recurrent Representation}
\label{sec:ss-recurrent}

To be applied on a discrete input sequence  instead of continuous function ,
\eqref{eq:1} must be discretized by a \textbf{step size}  that represents the resolution of the input.
Conceptually, the inputs  can be viewed as sampling an implicit underlying continuous signal , where .




To discretize the continuous-time SSM, we follow prior work in using the bilinear method~\citep{tustin1947method}, which converts the state matrix  into an approximation  .
The discrete SSM is

Equation \eqref{eq:2} is now a \emph{sequence-to-sequence} map  instead of function-to-function.
Moreover the state equation is now a recurrence in ,
allowing the discrete SSM to be computed like an RNN.
Concretely,  can be viewed as a \emph{hidden state} with transition matrix .

Notationally, throughout this paper we use  to denote discretized SSM matrices defined by \eqref{eq:2}. Note that these matrices are a function of both  as well as a step size ; we suppress this dependence for notational convenience when it is clear.


\subsection{Training SSMs: The Convolutional Representation}
\label{sec:ss-convolution}

The recurrent SSM \eqref{eq:2} is not practical for training on modern hardware due to its sequentiality.
Instead,
there is a well-known connection between linear time-invariant (LTI) SSMs such as \eqref{eq:1} and continuous convolutions.
Correspondingly, \eqref{eq:2} can actually be written as a discrete convolution.

For simplicity let the initial state be .
Then unrolling \eqref{eq:2} explicitly yields 
This can be vectorized into a convolution \eqref{eq:convolution} with an explicit formula for the convolution kernel \eqref{eq:krylov}.


\normalsize
In other words, equation \eqref{eq:convolution} is a single (non-circular) convolution and can be computed very efficiently with FFTs, \emph{provided} that  is known. However, computing  in \eqref{eq:krylov} is non-trivial and is the focus of our technical contributions in \cref{sec:s4}.
We call  the \textbf{SSM convolution kernel} or filter.
 


\section{Method: Structured State Spaces (\methodabbrv)}
\label{sec:s4}

Our technical results focus on developing the \methodabbrv{} parameterization and showing how to efficiently compute all views of the SSM (\cref{sec:background}):
the continuous representation  \eqref{eq:1},
the recurrent representation  \eqref{eq:2},
and the convolutional representation  \eqref{eq:convolution}.


\cref{sec:s4-motivation} motivates our approach, which is based on the linear algebraic concepts of conjugation and diagonalization, and discusses why the naive application of this approach does not work.
\cref{sec:s4-overview} gives an overview of the key technical components of our approach and formally defines the \methodabbrv{} parameterization.
\cref{sec:s4-efficiency} sketches the main results, showing that \methodabbrv{} is asymptotically efficient (up to log factors) for sequence models.
Proofs are in \cref{sec:lssl-instability,sec:s4-details}.






















\subsection{Motivation: Diagonalization}
\label{sec:s4-motivation}

The fundamental bottleneck in computing the discrete-time SSM \eqref{eq:2} is that it involves repeated matrix multiplication by .
For example, computing \eqref{eq:krylov} naively as in the LSSL involves  successive multiplications by , requiring  operations and  space.

To overcome this bottleneck, we use a structural result that allows us to simplify SSMs.
\begin{lemma}\label{lmm:conjugation}
  Conjugation is an equivalence relation on SSMs \small .
\end{lemma}
\begin{proof}Write out the two SSMs with state denoted by  and  respectively:
  
  After multiplying the right side SSM by , the two SSMs become identical with .
  Therefore these compute the exact same operator , but with a change of basis by  in the state .
\end{proof}

\cref{lmm:conjugation} motivates putting  into a canonical form by conjugation\footnote{Note that although we ultimately require , conjugation commutes with discretization so we refer to .}, which is ideally more structured and allows faster computation.
For example, if  were diagonal, the resulting computations become much more tractable.
In particular, the desired  (equation \eqref{eq:convolution}) would be a \textbf{Vandermonde product} which theoretically only needs  arithmetic operations \citep{pan2001structured}.

Unfortunately, the naive application of diagonalization does not work due to numerical issues.
Werive the explicit diagonalization for the HiPPO matrix \eqref{eq:hippo} and show it has entries exponentially large in the state size , rendering the diagonalization numerically infeasible (e.g.  in \cref{lmm:conjugation} would not be computable).
We note that \citet{gu2021lssl} proposed a different (unimplemented) algorithm to compute  faster than the naive algorithm.
In \cref{sec:lssl-instability}, we prove that it is also numerically unstable for related reasons.
\begin{lemma}\label{lmm:hippo-diagonalization}
  The HiPPO matrix  in equation \eqref{eq:hippo} is diagonalized by the matrix .
  In particular, .
  Therefore  has entries of magnitude up to .
\end{lemma}

\subsection{The \methodabbrv{} Parameterization: Normal Plus Low-Rank}
\label{sec:s4-overview}

The previous discussion implies that we should only conjugate by well-conditioned matrices .
The ideal scenario is when the matrix  is diagonalizable by a perfectly conditioned (i.e., unitary) matrix.
By the Spectral Theorem of linear algebra, this is exactly the class of \textbf{normal matrices}.
However, this class of matrices is restrictive; in particular, it does not contain the HiPPO matrix \eqref{eq:hippo}.


\begin{algorithm}[t]
  \caption{\textsc{\methodabbrv{} Convolution Kernel (Sketch)}}\label{alg:s4-convolution}
  \begin{algorithmic}[1]
    \renewcommand{\algorithmicrequire}{\textbf{Input:}}
    \Require{\methodabbrv{} parameters  and step size }
    \renewcommand{\algorithmicensure}{\textbf{Output:}}
    \Ensure{SSM convolution kernel  for } (equation \eqref{eq:krylov})
    \State 
    \Comment{Truncate SSM generating function (SSMGF) to length }
    \State
    
    \label{step:cauchy}
    \Comment{Black-box Cauchy kernel}
    \State
    
    \Comment{Woodbury Identity} \State 
    \Comment{Evaluate SSMGF at all roots of unity }
    \State 
    \Comment{Inverse Fourier Transform}
  \end{algorithmic}
\end{algorithm}


We make the observation that although the HiPPO matrix is not normal, it can be decomposed as the \emph{sum of a normal and low-rank matrix}.
However, this is still not useful by itself:
unlike a diagonal matrix, powering up this sum (in \eqref{eq:krylov}) is still slow and not easily optimized.
We overcome this bottleneck by simultaneously applying three new techniques.
\begin{itemize}[leftmargin=*]\item Instead of computing  directly,
    we compute its spectrum by evaluating its \textbf{truncated generating function}  at the roots of unity .
     can then be found by applying an inverse FFT.
  \item This generating function is closely related to the matrix resolvent, and now involves a matrix \emph{inverse} instead of \emph{power}.
    The low-rank term can now be corrected by applying the \textbf{Woodbury identity} which reduces  in terms of , truly reducing to the diagonal case.
  \item Finally, we show that the diagonal matrix case is equivalent to the computation of a \textbf{Cauchy kernel} , a well-studied problem with stable near-linear algorithms \citep{pan2015transformations,pan2017fast}.
\end{itemize}


Our techniques apply to any matrix that can be decomposed as \emph{\textbf{Normal Plus Low-Rank (NPLR)}}.
\begin{theorem}\label{thm:hippo-nplr}
  All HiPPO matrices from \citep{gu2020hippo} have a NPLR representation
  
  for unitary , diagonal , and low-rank factorization .
  These matrices HiPPO- LegS, LegT, LagT all satisfy  or .
  In particular, equation \eqref{eq:hippo} is NPLR with .
\end{theorem}

\subsection{\methodabbrv{} Algorithms and Computational Complexity}
\label{sec:s4-efficiency}

By equation \eqref{eq:nplr}, note that NPLR matrices can be conjugated into \emph{diagonal plus low-rank} (DPLR) form
(now over  instead of .
\cref{thm:s4-recurrence,thm:s4-convolution} describe the complexities of SSMs where  is in DPLR form.
\methodabbrv{} is optimal or near-optimal for both recurrent and convolutional representations.


\begin{theorem}[\methodabbrv{} Recurrence]\label{thm:s4-recurrence}
  Given any step size , computing one step of the recurrence \eqref{eq:2} can be done in  operations where  is the state size.
\end{theorem}

\cref{thm:s4-recurrence} follows from the fact that the inverse of a DPLR matrix is also DPLR (e.g. also by the Woodbury identity).
This implies that the discretized matrix  is the product of two DPLR matrices and thus has  matrix-vector multiplication.
\cref{sec:s4-recurrence-proof} computes  in closed DPLR form.
\begin{theorem}[\methodabbrv{} Convolution]\label{thm:s4-convolution}
  Given any step size , computing the SSM convolution filter  can be reduced to 4 Cauchy multiplies,
  requiring only
   operations and  space.
\end{theorem}


\cref{sec:s4-details}, \cref{def:cauchy} formally defines Cauchy matrices, which are related to rational interpolation problems.
Computing with Cauchy matrices is an extremely well-studied problem in numerical analysis,
with both fast arithmetic and numerical algorithms based on the famous Fast Multipole Method (FMM)
\citep{pan2001structured,pan2015transformations,pan2017fast}.
The computational complexities of these algorithms under various settings are described in \cref{sec:s4-details}, \cref{prop:cauchy}.

We reiterate that \cref{thm:s4-convolution} is our core technical contribution,
and its algorithm is the very motivation of the NPLR \methodabbrv{} parameterization.
This algorithm is formally sketched in \cref{alg:s4-convolution}.


\subsection{Architecture Details of the Deep \methodabbrv{} Layer}
\label{sec:s4-architecture}

Concretely, an \methodabbrv{} layer is parameterized as follows.
First initialize a SSM with  set to the HiPPO matrix \eqref{eq:hippo}.
By \cref{lmm:conjugation} and \cref{thm:hippo-nplr},
this SSM is unitarily equivalent to some  for some diagonal  and vectors .
These comprise \methodabbrv's  trainable parameters.


The overall deep neural network (DNN) architecture of \methodabbrv{} is similar to prior work.
As defined above,
\methodabbrv{} defines a map from , i.e. a 1-D sequence map.
Typically, DNNs operate on feature maps of size  instead of .
\methodabbrv{} handles multiple features by simply defining  independent copies of itself, and then mixing the  features with a position-wise linear layer for a total of  parameters per layer.
Nonlinear activation functions are also inserted between these layers.
Overall, \methodabbrv{} defines a sequence-to-sequence map of shape (batch size, sequence length, hidden dimension), exactly the same as related sequence models such as Transformers, RNNs, and CNNs.


\begin{table}
  \caption{
    Complexity of various sequence models in terms of sequence length (), batch size (), and hidden dimension ();
    tildes denote log factors.
    Metrics are parameter count, training computation, training space requirement, training parallelizability, and inference computation (for 1 sample and time-step).
    For simplicity, the state size  of \methodabbrv{} is tied to .
    Bold denotes model is theoretically best for that metric.
    Convolutions are efficient for training while recurrence is efficient for inference, while SSMs combine the strengths of both.
  }
  \small
  \centering
  \begin{tabular}{@{}lllll@{}}
    \toprule
               & Convolution\tablefootnote{Refers to global (in the sequence length) and depthwise-separable convolutions, similar to the convolution version of S4.} & Recurrence     & Attention            & \methodabbrv                                     \\
    \midrule
    Parameters &                                                                                                                                              &  &        &                                    \\
    Training   &                                                                                                                          &     &  &  \\
    Space      &                                                                                                                                        &  &     &                                    \\
    Parallel   & \textbf{Yes}                                                                                                                                         & No             & \textbf{Yes}         & \textbf{Yes}                                     \\
    Inference  &                                                                                                                                            &  &     &                                    \\
    \bottomrule
  \end{tabular}
  \label{tab:complexity}
\end{table}

Note that the core S4 module is a linear transformation, but the addition of non-linear transformations through the depth of the network makes the overall deep SSM non-linear.
This is analogous to a vanilla CNN, since convolutional layers are also linear.
The broadcasting across  hidden features described in this section is also analogous to depthwise-separable convolutions.
Thus, the overall deep S4 model is closely related to a depthwise-separable CNN but with \emph{global} convolution kernels.

Finally, we note that follow-up work found that this version of S4 can sometimes suffer from numerical instabilities when the  matrix has eigenvalues on the right half-plane \citep{goel2022sashimi}.
It introduced a slight change to the NPLR parameterization for S4 from  to  that corrects this potential problem.

\cref{tab:complexity} compares the complexities of the most common deep sequence modeling mechanisms.
 
\section{Experiments}
\label{sec:experiments}

\cref{sec:experiments-benchmark} benchmarks \methodabbrv{} against the LSSL and efficient Transformer models.
\cref{sec:experiments-lrd} validates \methodabbrv{} on LRDs: the LRA benchmark and raw speech classification.
\cref{sec:experiments-general} investigates whether \methodabbrv{} can be used as a general sequence model to perform effectively and efficiently in a wide variety of settings including image classification, image and text generation, and time series forecasting.

\subsection{\methodabbrv{} Efficiency Benchmarks}
\label{sec:experiments-benchmark}
We benchmark that \methodabbrv{} can be trained quickly and efficiently, both compared to the LSSL, as well as efficient Transformer variants designed for long-range sequence modeling.
As outlined in \cref{sec:s4}, \methodabbrv{} is theoretically much more efficient than the LSSL, and \cref{tab:ssm-benchmark} confirms that the \methodabbrv{} is orders of magnitude more speed- and memory-efficient for practical layer sizes.
In fact, \methodabbrv's speed and memory use is competitive with the most efficient Transformer variants benchmarked by \citet{tay2021long}---Linear Transformer~\citep{katharopoulos2020transformers} and Performer~\citep{choromanski2020rethinking}---in a parameter-matched setting (\cref{tab:lra-benchmark}, following the protocol of \citet{tay2021long}).


\begin{figure}[t!]
  \begin{minipage}[t]{0.54\linewidth}
    \small
    \centering
    \captionsetup{type=table}
    \caption{Deep SSMs: The \methodabbrv{} parameterization with \cref{alg:s4-convolution} is asymptotically more efficient than the LSSL.}
    \resizebox{\linewidth}{!}{
      \begin{tabular}{@{}llllllllll@{}}
        \toprule
        & \multicolumn{3}{c}{\textsc{Training Step (ms)}} & \multicolumn{3}{c}{\textsc{Memory Alloc. (MB)}} \\
        \cmidrule{2-4} \cmidrule{5-7}
        Dim.        & 128                                         & 256                                        & 512                                        & 128   & 256  & 512    \\
        \midrule
        LSSL        & 9.32                                        & 20.6                                       & 140.7                                      & 222.1 & 1685 & 13140  \\
        \textbf{\methodabbrv} & 4.77                                        & 3.07                                       & 4.75                                       & 5.3   & 12.6 & 33.5   \\
        \midrule
        Ratio &  &  &  &  &  &  \\
        \bottomrule
      \end{tabular}}
    \label{tab:ssm-benchmark}
  \end{minipage}
  \hfill
  \begin{minipage}[t]{0.45\linewidth}
    \small
    \captionsetup{type=table}
    \centering
    \caption{Benchmarks vs. efficient Transformers}
    \resizebox{\linewidth}{!}{
      \begin{tabular}{@{}lllll@{}}
        \toprule
                              & \multicolumn{2}{c}{\textsc{Length 1024}} & \multicolumn{2}{c}{\textsc{Length 4096}} \\
        \cmidrule{2-3} \cmidrule{4-5}
                              & Speed                                    & Mem.                                      & Speed                        & Mem.                          \\
        \midrule
        Transformer           & 1                            & 1                             & 1                & 1                 \\
        \midrule
        Performer             & 1.23                         & \underline{0.43}              & 3.79             & \underline{0.086} \\
        Linear Trans.         & \textbf{1.58}                & \textbf{0.37}                 & \textbf{5.35}    & \textbf{0.067}    \\
        \midrule
        \textbf{\methodabbrv} & \textbf{1.58}                & \underline{0.43}              & \underline{5.19} & 0.091             \\
        \bottomrule
      \end{tabular}}
    \label{tab:lra-benchmark}
  \end{minipage}
\end{figure}

\subsection{Learning Long Range Dependencies}
\label{sec:experiments-lrd}
As described in \cref{sec:ss-memory,sec:s4-motivation}, \methodabbrv{} uses a principled approach to address LRDs based on the HiPPO theory of continuous-time memorization.
Our goal in this section is to validate that \methodabbrv{} achieves high performance on difficult tasks that require long-range reasoning.
We focus here on two problems:
(i) the Long-Range Arena, a well-known benchmark designed to test efficient sequence models on LRDs, and
(ii) a speech classification problem as a real-world test of LRDs.


\begin{figure}[t!]
  \centering
    \begin{subfigure}{0.3\linewidth}
        \centering
        \includegraphics[width=\linewidth]{figs/pathx_positive.png}
    \end{subfigure}\quad
    \begin{subfigure}{0.58\linewidth}
        \centering
        \lineskip=0pt
        \includegraphics[width=\linewidth]{figs/pathfinder_4_filters_layer_0.png}\\
        \includegraphics[width=\linewidth]{figs/pathfinder_4_filters_layer_5.png}
    \end{subfigure}
      \caption{
        Visualizations of a trained \methodabbrv{} model on LRA Path-X. SSM convolution kernels  are reshaped into a   image. (\textit{Left}) Example from the Path-X task, which involves deducing if the markers are connected by a path (\textit{Top}) Filters from the first layer (\textit{Bottom}) Filters from the last layer.
    }
  \label{fig:pathx-filters}
\end{figure}


\begin{table}[t!]
  \small
  \caption{
    (\textbf{Long Range Arena})
    (\textit{Top}) Original Transformer variants in LRA. Full results in \cref{sec:experiment-details-lrd}.
    (\textit{Bottom}) Other models reported in the literature.
    \emph{Please read \cref{sec:reproduction} before citing this table.}
  }
    \centering
    \begin{tabular}{@{}llllllll@{}}
        \toprule
        \textsc{Model}        & \textsc{ListOps}  & \textsc{Text}     & \textsc{Retrieval} & \textsc{Image}    & \textsc{Pathfinder} & \textsc{Path-X} & \textsc{Avg}      \\
        \midrule
        Transformer           & 36.37             & 64.27             & 57.46              & 42.44             & 71.40               & \xmark          & 53.66             \\
        Reformer              & \underline{37.27} & 56.10             & 53.40              & 38.07             & 68.50               & \xmark          & 50.56             \\
        BigBird               & 36.05             & 64.02             & 59.29              & 40.83             & 74.87               & \xmark          & 54.17             \\
        Linear Trans.         & 16.13             & \underline{65.90} & 53.09              & 42.34             & 75.30               & \xmark          & 50.46             \\
        Performer             & 18.01             & 65.40             & 53.82              & 42.77             & 77.05               & \xmark          & 51.18             \\
        \midrule
        FNet                  & 35.33             & 65.11             & 59.61              & 38.67             & \underline{77.80}   & \xmark          & 54.42             \\
        Nystr{\"o}mformer     & 37.15             & 65.52             & \underline{79.56}  & 41.58             & 70.94               & \xmark          & 57.46             \\
        Luna-256              & 37.25             & 64.57             & 79.29              & \underline{47.38} & 77.72               & \xmark          & \underline{59.37} \\
        \textbf{\methodabbrv} & \textbf{59.60}    & \textbf{86.82}    & \textbf{90.90}     & \textbf{88.65}    & \textbf{94.20}      & \textbf{96.35}  & \textbf{86.09}    \\
        \bottomrule
    \end{tabular}
    \label{tab:lra}
\end{table}


\textbf{Long Range Arena (LRA).}
LRA~\citep{tay2021long} contains  tasks with lengths 1K-16K steps, encompassing modalities and objectives that require similarity, structural, and visuospatial reasoning.
\cref{tab:lra} compares \methodabbrv{} against the 11 Transformer variants from \citet{tay2021long} as well as follow-up work.
\methodabbrv{} substantially advances the SoTA, outperforming all baselines on all tasks and averaging  compared to less than  for every baseline.
Notably, \methodabbrv{} solves the Path-X task, an extremely challenging task
that involves reasoning about LRDs over sequences of length .
All previous models have failed (i.e.\ random guessing) due to memory or computation bottlenecks, or simply being unable to learn such long dependencies.

We analyze \methodabbrv's performance on Path-X by visualizing its learned representations,
in particular 1-D convolution kernels 
which are the focus of our technical results in \cref{sec:s4}.
\cref{fig:pathx-filters} shows that \methodabbrv{} learns a variety of filters that display spatially consistent structure
and demonstrate awareness of the 2-D nature of the data.
In particular, the lower layers learn simple kernels that extract features from just a few rows of local context while ignoring the rest of the image.
On the other hand, higher layers aggregate information globally across full columns of the image at varying spatial frequencies.
Filters in these higher layers span the entire context ( pixels), confirming \methodabbrv's ability to learn LRDs.



\begin{figure}[b!]
\begin{minipage}[t]{0.47\linewidth}
  \small
  \centering
  \captionsetup{type=table}
  \caption{
    (\textbf{SC10 classification})
    Transformer, CTM, RNN, CNN, and SSM models.
    (\textit{MFCC}) Standard pre-processed MFCC features (length 161).
    (\textit{Raw}) Unprocessed signals (length 16000).
    ( Frequency change at test time.
    \xmark{} denotes not applicable or computationally infeasible on single GPU.
    \emph{Please read \cref{sec:reproduction} before citing this table.}
  }
  \vspace*{-5pt}
  \begin{tabular}{@{}llllllllll@{}}
    \toprule
                 & \textsc{MFCC}     & \textsc{Raw}      &    \\
    \midrule
    Transformer  & 90.75             & \xmark            & \xmark            \\
    Performer    & 80.85             & 30.77             & 30.68             \\
    \midrule
    ODE-RNN      & 65.9              & \xmark            & \xmark            \\
    NRDE         & 89.8              & 16.49             & 15.12             \\
    \midrule
    ExpRNN       & 82.13             & 11.6              & 10.8              \\
    LipschitzRNN & 88.38             & \xmark            & \xmark            \\
    \midrule
    CKConv       & \textbf{95.3}     & 71.66             & \underline{65.96} \\
    WaveGAN-D    & \xmark            & \underline{96.25} & \xmark            \\
    \midrule
    LSSL         & 93.58             & \xmark            & \xmark            \\
    \textbf{\methodabbrv}  & \underline{93.96} & \textbf{98.32}    & \textbf{96.30}    \\
    \bottomrule
  \end{tabular}
  \label{tab:sc}
\end{minipage}
\hfill
\begin{minipage}[t]{0.47\linewidth}
  \small
  \centering
  \captionsetup{type=table}
  \caption{
    (\textbf{Pixel-level 1-D image classification})
    Comparison against reported test accuracies from prior works (Transformer, RNN, CNN, and SSM models).
    Extended results and citations in \cref{sec:experiment-details}.
  }
  \begin{tabular}{@{}llll@{}}
    \toprule
                     & \textsc{sMNIST} & \textsc{pMNIST}   & \textsc{sCIFAR}   \\
    \midrule
    Transformer      & 98.9            & 97.9              & 62.2              \\
    \midrule
    LSTM             & 98.9            & 95.11             & 63.01             \\
    r-LSTM           & 98.4            & 95.2              & 72.2              \\
    UR-LSTM          & 99.28           & 96.96             & 71.00             \\
    UR-GRU           & 99.27           & 96.51             & 74.4              \\
    HiPPO-RNN        & 98.9            & 98.3              & 61.1              \\
    LMU-FFT          & -               & 98.49             & -                 \\
    LipschitzRNN     & 99.4            & 96.3              & 64.2              \\
    \midrule
    TCN              & 99.0            & 97.2              & -                 \\
    TrellisNet       & 99.20           & 98.13             & 73.42             \\
    CKConv           & 99.32           & 98.54             & 63.74             \\
    \midrule
    LSSL & \underline{99.53}           & \textbf{98.76}    & \underline{84.65} \\
    \textbf{\methodabbrv}      & \textbf{99.63}  & \underline{98.70} & \textbf{91.13}    \\
    \bottomrule
  \end{tabular}
  \label{tab:image}
\end{minipage}
\end{figure}

\textbf{Raw Speech Classification.}
Speech is a typical real-world time series domain, involving signals sampled from an underlying physical process at high frequency.
We perform speech classification using the SC10 subset of the \emph{Speech Commands} dataset~\citep{Warden2018SpeechCA} (see \cref{sec:reproduction}).
While most sequence models for speech rely on extensive preprocessing (e.g. to MFCC features), we classify raw speech (length-) following~\citet{romero2021ckconv}.
\methodabbrv{} achieves  accuracy, higher than all baselines that use the  shorter MFCC features, and validates that a powerful LRD model is able to extract more information from the raw data and outperform hand-crafted pre-processing.
Additionally, we include a baseline CNN specifically designed for raw speech, the discriminator from the WaveGAN model~\citep{Donahue2019AdversarialAS},
which performs worse than \methodabbrv{} while having  more parameters and incorporating many more
architectural heuristics (\cref{sec:experiment-details-lrd}).

\subsection{\methodabbrv{} as a General Sequence Model}
\label{sec:experiments-general}

A key goal of sequence modeling research is to develop a single model that can be applied in many domains (e.g. images, audio, text, time-series) with a broad range of capabilities (e.g. efficient training, fast generation, handling irregularly sampled data).
As a fundamental scientific model, SSMs are a promising candidate that come with a range of capabilities, and \methodabbrv's strong results on LRD benchmarks spanning images, text, and speech are evidence of \methodabbrv's potential as a general sequence model.
In this section, we focus on understanding this question in more depth by highlighting key strengths of \methodabbrv{} in settings that usually require specialized models.
The tasks we focus on (generative modeling, image classification, time-series forecasting)
are considered as LRD tasks in the literature,
and serve as additional validation that \methodabbrv{} handles LRDs efficiently.


\textbf{Large-scale generative modeling.}
We investigate two well-studied image and text benchmarks to validate the scalability, flexibility, and efficiency of \methodabbrv.
These tasks require much larger models than our previous tasks -- up to M parameters.

First, CIFAR density estimation is a popular benchmark for autoregressive models, where images are flattened into a sequence of  RGB subpixels that are predicted one by one.
\cref{tab:cifar-generation} shows that \emph{with no 2D inductive bias}, \methodabbrv{} is competitive with the best models designed for this task.

Second, WikiText-103 is an established benchmark for language modeling, an important task for large-scale sequence models where tokens are predicted sequentially based on past context.
Although RNNs were the model of choice for many years, Transformers are now the dominant model in such applications that contain data that is inherently discrete.
We show that alternative models to Transformers can still be competitive in these settings.
By simply taking a strong Transformer baseline \citep{baevski2018adaptive} and replacing the self-attention layers,
\methodabbrv{} substantially closes the gap to Transformers (within  ppl), setting SoTA for attention-free models by over  ppl.

\textbf{Fast autoregressive inference.}
A prominent limitation of autoregressive models is inference speed (e.g. generation),
since they require a pass over the full context for every new sample.
Several methods have been specifically crafted to overcome this limitation such as the Linear Transformer, a hybrid Transformer/RNN that switches to a stateful, recurrent view at inference time for speed.

As a stateful model, SSMs automatically have this ability (\cref{fig:properties}).
By switching to its recurrent representation (\cref{sec:ss-recurrent}),
\methodabbrv{} requires \emph{constant memory and computation} per time step -- in contrast to standard autoregressive models which scale in the context length.
On both CIFAR-10 and WikiText-103,
we report the throughput of various models at generation time, with \methodabbrv{} around  faster than a vanilla Transformer on both tasks (details in \cref{sec:experiment-details-general-speed}).


\textbf{Sampling resolution change.}
As a continuous-time model, \methodabbrv{} automatically adapts to data sampled at different rates,
a challenging setting for time series with a dedicated line of work \citep{rubanova2019latent,de2019gru,romero2021ckconv}.
Without re-training, \methodabbrv{} achieves  accuracy at  the frequency on Speech Commands 10 (\cref{tab:sc}), simply by changing its internal step size  (\cref{sec:ss-recurrent}).


\begin{figure}
\begin{minipage}[t]{0.525\linewidth}
  \small
  \centering
  \captionsetup{type=table}
  \caption{
    (\textbf{CIFAR-10 density estimation})
    As a generic \mbox{sequence} model, \methodabbrv{} is competitive with previous autoregressive models (in bits per dim.) while incorporating no 2D inductive bias,
    and has fast generation through its recurrence mode.
  }
  \vspace*{-4pt}
  \resizebox{\linewidth}{!}{
    \begin{tabular}{@{}lllll@{}}
      \toprule
      Model               & bpd              & 2D bias          & Images / sec                           \\
      \midrule
      Transformer         & 3.47             & \textbf{None}    & 0.32 (                   \\
      Linear Transf.       & 3.40             & \textbf{None}    & 17.85 (     \\
      PixelCNN           & 3.14             & 2D conv.         & -                                      \\
      Row PixelRNN        & 3.00             & 2D BiLSTM        & -                                      \\
      PixelCNN++          & 2.92             & 2D conv.         & \underline{19.19} (              \\
      Image Transf.       & 2.90             & 2D local attn.   & 0.54 (1.7)                     \\
      PixelSNAIL          & \underline{2.85} & 2D conv. + attn. & 0.13 (0.4)                     \\
      Sparse Transf.      & \textbf{2.80}    & 2D sparse attn.  & -                                      \\
      \midrule
      \textbf{\methodabbrv} (base)  & 2.92             & \textbf{None}    & \textbf{20.84} ( \\
      \textbf{\methodabbrv} (large) & \underline{2.85} & \textbf{None}    & 3.36 (                \\
      \bottomrule
    \end{tabular}
  }
  \label{tab:cifar-generation}
\end{minipage}
\hfill
\begin{minipage}[t]{0.47\linewidth}
  \small
  \centering
  \captionsetup{type=table}
  \caption{
    (\textbf{WikiText-103 language modeling})
    \methodabbrv{} approaches the performance of Transformers with much faster generation.
    (\textit{Top}) Transformer baseline which our implementation is based on, with attention replaced by \methodabbrv.
    (\textit{Bottom}) Attention-free models (RNNs and CNNs).
  }
  \resizebox{\linewidth}{!}{
    \begin{tabular}{@{}llll@{}}
      \toprule
      Model         & Params & Test ppl.      & Tokens / sec                       \\
      \midrule
      Transformer   & 247M   & \textbf{20.51} & 0.8K (               \\
      \midrule
      GLU CNN       & 229M   & 37.2           & -                                  \\
      AWD-QRNN      & 151M   & 33.0           & -                                  \\
      LSTM + Hebb.  & -      & 29.2           & -                                  \\
      TrellisNet    & 180M   & 29.19          & -                                  \\
      Dynamic Conv. & 255M   & 25.0           & -                                  \\
      TaLK Conv.    & 240M   & 23.3           & -                                  \\
      \textbf{\methodabbrv}   & 249M   & \textbf{20.95} & \textbf{48K} ( \\
      \bottomrule
    \end{tabular}
  }
  \label{tab:wt103}
\end{minipage}
\end{figure}


\begin{table*}[!b]
\caption{Univariate long sequence time-series forecasting results. Full results in \cref{sec:experiment-details-general-informer}.}
\centering
\resizebox{\linewidth}{!}{
  \begin{tabular}{@{}llllllllllll@{}}
    \toprule
                 & \textbf{\methodabbrv}          & {Informer}        & {LogTrans}   & {Reformer}   & {LSTMa}      & {DeepAR}     & {ARIMA}      & {Prophet}    \\
    \midrule
                 & MSE~~MAE                       & MSE~~MAE          & MSE~~MAE     & MSE~~MAE     & MSE~~MAE     & MSE~~MAE     & MSE~~MAE     & MSE~~MAE     \\
    \midrule
    {{ETTh}} & \textbf{0.116}~~\textbf{0.271} & 0.269~~0.435      & 0.273~~0.463 & 2.112~~1.436 & 0.683~~0.768 & 0.658~~0.707 & 0.659~~0.766 & 2.735~~3.253 \\
    {{ETTh}} & \textbf{0.187}~~\textbf{0.358} & {0.277}~~{0.431} & 0.303~~0.493 & 2.030~~1.721 & 0.640~~0.681 & 0.429~~0.580 & 2.878~~1.044 & 3.355~~4.664 \\
    {{ETTm}} & \textbf{0.292}~~\textbf{0.466} & {0.512}~~{0.644}  & 0.598~~0.702 & 1.793~~1.528 & 1.064~~0.873 & 2.437~~1.352 & 0.639~~0.697 & 2.747~~1.174 \\
    {{Weather}}  & \textbf{0.245}~~\textbf{0.375} & {0.359}~~{0.466}  & 0.388~~0.499 & 2.087~~1.534 & 0.866~~0.809 & 0.499~~0.596 & 1.062~~0.943 & 3.859~~1.144 \\
    {{ECL}}      & \textbf{0.432}~~\textbf{0.497} & {0.582}~~{0.608}  & 0.624~~0.645 & 7.019~~5.105 & 1.545~~1.006 & 0.657~~0.683 & 1.370~~0.982 & 6.901~~4.264 \\
    \bottomrule

\end{tabular}}
\label{tab:informer-s-long}
\end{table*}

\textbf{Learning with weaker inductive bias.}
Beyond our results on speech (\cref{sec:experiments-lrd}), we further validate that \methodabbrv{} can be applied with minimal modifications
on two domains that typically require specialized domain-specific preprocessing and architectures.
First, we compare \methodabbrv{} to the Informer~\citep{haoyietal-informer-2021},
a new Transformer architecture that uses a complex encoder-decoder designed for time-series forecasting problems.
A simple application of \methodabbrv{} that treats forecasting as a masked sequence-to-sequence transformation (\cref{fig:s4-architecture})
outperforms the Informer and other baselines on  settings across  forecasting tasks.
Notably, \methodabbrv{} is better on the longest setting in each task, e.g.\ reducing MSE by  when forecasting  days of weather data
(\cref{tab:informer-s-long}).

Finally,
we evaluate \methodabbrv{} on pixel-level sequential image classification tasks (\cref{tab:image}),
popular benchmarks which were originally LRD tests for RNNs~\citep{arjovsky2016unitary}.
Beyond LRDs, these benchmarks point to a recent effort of the ML community to solve vision
problems with reduced domain knowledge,
in the spirit of models such as Vision Transformers \citep{dosovitskiy2020image} and MLP-Mixer \citep{tolstikhin2021mlp} which involve patch-based models that without 2-D inductive bias.
Sequential CIFAR is a particularly challenging dataset where outside of SSMs, all sequence models have a gap of over  to a simple 2-D CNN.
By contrast, \methodabbrv{} is competitive with a larger ResNet18 (7.9M vs. 11.0M parameters), both with (\textbf{93.16\%} vs. 95.62\%) or without (\textbf{91.12\%} vs. 89.46\%) data augmentation.
Moreover, it is much more robust to other architectural choices (e.g.\ \textbf{90.46\%} vs. 79.52\% when swapping BatchNorm for LayerNorm).

\subsection{SSM Ablations: the Importance of HiPPO}
A critical motivation of S4 was the use of the HiPPO matrices to initialize an SSM.
We consider several simplifications of S4 to ablate the importance of each of these components, including:
\begin{enumerate*}[label=(\roman*)]\item how important is the HiPPO initialization?
  \item how important is training the SSM on top of HiPPO?
  \item are the benefits of S4 captured by the NPLR parameterization without HiPPO?
\end{enumerate*}

As a simple testbed, all experiments in this section were performed on the sequential CIFAR-10 task,
whicih we found transferred well to other settings.
Models were constrained to at most 100K trainable parameters and trained with a simple plateau learning rate scheduler and no regularization.

\paragraph{Unconstrained SSMs.}
We first investigate generic SSMs with various initializations.
We consider a random Gaussian initialization (with variance scaled down until it did not NaN),  and the HiPPO initialization.
We also consider a random diagonal Gaussian matrix as a potential structured method; parameterizing  as a diagonal matrix would allow substantial speedups without going through the complexity of S4's NPLR parameterization.
We consider both freezing the  matrix and training it.

\cref{fig:ssm-ablation-real} shows both training and validation curves, from which we can make several observations.
First, training the SSM improved all methods, particularly the randomly initialized ones.
For all methods, training the SSM led to improvements in both training and validation curves.

Second, a large generalization gap exists between the initializations.
In particular, note that when  is trained, all initializations are able to reach perfect training accuracy.
However, their validation accuracies are separated by over .

\begin{figure}[!ht]
\begin{subfigure}{.5\linewidth}\centering
    \includegraphics[width=\linewidth]{figs/ssm_ablation_real_train.png}
\end{subfigure}
\begin{subfigure}{.5\linewidth}\centering
    \includegraphics[width=\linewidth]{figs/ssm_ablation_real_validation.png}
\end{subfigure}
\caption{CIFAR-10 classification with unconstrained, real-valued SSMs with various initializations. (\emph{Left}) Train accuracy. (\emph{Right}) Validation accuracy.}
\label{fig:ssm-ablation-real}
\end{figure}

\paragraph{NPLR SSMs.}
The previous experiment validates the importance of HiPPO in SSMs.
This was the main motivation of the NPLR algorithm in S4,
which utilizes structure of the HiPPO matrix \eqref{eq:hippo} to make SSMs computationally feasible.
\cref{fig:ssm-ablation-nplr} shows that random NPLR matrices still do not perform well,
which validates that S4's effectiveness primarily comes from the HiPPO initialization, not the NPLR parameterization.

\begin{figure}[!t]
\begin{subfigure}{.5\linewidth}\centering
    \includegraphics[width=\linewidth]{figs/ssm_ablation_nplr_validation.png}
    \caption{}
    \label{fig:ssm-ablation-nplr}
\end{subfigure}
\begin{subfigure}{.5\linewidth}\centering
    \includegraphics[width=\linewidth]{figs/ssm_ablation_reg.png}
    \caption{}
    \label{fig:ssm-ablation-reg}
\end{subfigure}
\caption{
  CIFAR-10 validation accuracy of SSMs with different initializations and parameterizations.
  (\emph{Left}) NPLR parameterization with random versus HiPPO initialization.
  (\emph{Right}) All methods considered in this section, including minor Dropout regularization. S4 achieves SotA accuracy on sequential CIFAR-10 with just 100K parameters.
}
\end{figure}

Finally, \cref{fig:ssm-ablation-reg} considers the main ablations considered in this section (with trainable SSMs) and adds minor regularization.
With 0.1 Dropout, the same trends still hold, and the HiPPO initialization---in other words, the full S4 method---achieves  test accuracy with just 100K parameters.
 

\section{Conclusion}
\label{sec:conclusion}

We introduce \methodabbrv, a sequence model that uses a new parameterization for the state space model's continuous-time, recurrent, and convolutional views to efficiently model LRDs in a principled manner.
Results across established benchmarks evaluating a diverse range of data modalities and model capabilities
suggest that \methodabbrv{} has the potential to be an effective general sequence modeling solution.
 

\subsubsection*{Acknowledgments}
We thank Aditya Grover and Chris Cundy for helpful discussions about earlier versions of the method.
We thank Simran Arora, Sabri Eyuboglu, Bibek Paudel, and Nimit Sohoni for valuable feedback on earlier drafts of this work.
This work was done with the support of Google Cloud credits under HAI proposals 540994170283 and 578192719349.
We gratefully acknowledge the support of NIH under No. U54EB020405 (Mobilize), NSF under Nos. CCF1763315 (Beyond Sparsity), CCF1563078 (Volume to Velocity), and 1937301 (RTML); ONR under No. N000141712266 (Unifying Weak Supervision); ONR N00014-20-1-2480: Understanding and Applying Non-Euclidean Geometry in Machine Learning; N000142012275 (NEPTUNE); the Moore Foundation, NXP, Xilinx, LETI-CEA, Intel, IBM, Microsoft, NEC, Toshiba, TSMC, ARM, Hitachi, BASF, Accenture, Ericsson, Qualcomm, Analog Devices, the Okawa Foundation, American Family Insurance, Google Cloud, Salesforce, Total, the HAI-AWS Cloud Credits for Research program, the Stanford Data Science Initiative (SDSI), and members of the Stanford DAWN project: Facebook, Google, and VMWare. The Mobilize Center is a Biomedical Technology Resource Center, funded by the NIH National Institute of Biomedical Imaging and Bioengineering through Grant P41EB027060. The U.S. Government is authorized to reproduce and distribute reprints for Governmental purposes notwithstanding any copyright notation thereon. Any opinions, findings, and conclusions or recommendations expressed in this material are those of the authors and do not necessarily reflect the views, policies, or endorsements, either expressed or implied, of NIH, ONR, or the U.S. Government.


\bibliography{biblio}

\newpage

\appendix


\section{Discussion}
\label{sec:discussion}


\paragraph{Related Work.}
Our work is most closely related to a line of work originally motivated by a particular biologically-inspired SSM, which led to mathematical models for addressing LRDs. \citet{voelker2019dynamical,voelker2019legendre} derived a non-trainable SSM motivated from approximating a neuromorphic spiking model, and \citet{chilkuri2021parallelizing} showed that it could be sped up at train time with a convolutional view.
\citet{gu2020hippo} extended this special case to a general continuous-time function approximation framework with several more special cases of  matrices designed for long-range dependencies.
However, instead of using a true SSM, all of these works fixed a choice of  and built RNNs around it.
Most recently, \citet{gu2021lssl} used the full \eqref{eq:1} explicitly as a deep SSM model, exploring new conceptual views of SSMs, as well as allowing   to be trained.
As mentioned in \cref{sec:intro}, their method used a naive instantiation of SSMs that suffered from an additional factor of  in memory and  in computation.

Beyond this work, our technical contributions (\cref{sec:s4}) on the \methodabbrv{} parameterization and algorithms are applicable to a broader family of SSMs including these investigated in prior works,
and our techniques for working with these models may be of independent interest.


\paragraph{Implementation.}

The computational core of \methodabbrv's training algorithm is the Cauchy kernel discussed in \cref{sec:s4-overview,sec:s4-efficiency,sec:s4-convolution-proof}.
As described in \cref{sec:s4-convolution-proof} \cref{prop:cauchy},
there are many algorithms for it with differing computational complexities and sophistication.
Our current implementation of \methodabbrv{} actually uses the naive  algorithm
which is easily parallelized on GPUs and has more easily accessible libraries allowing it to be implemented;
we leverage the \texttt{pykeops} library for memory-efficient kernel operations.
However, this library is a much more general library that may not be optimized for the Cauchy kernels used here,
and we believe that a dedicated CUDA implementation can be more efficient.
Additionally, as discussed in this work, there are asymptotically faster and numerically stable algorithms for the Cauchy kernel (\cref{prop:cauchy}).
However, these algorithms are currently not implemented for GPUs due to a lack of previous applications that require them.
We believe that more efficient implementations of these self-contained computational kernels are possible,
and that \methodabbrv{} (and SSMs at large) may have significant room for further improvements in efficiency.


\paragraph{Limitations and Future Directions.}
In this work, we show that \methodabbrv{} can address a wide variety of data effectively.
However, it may not necessarily be the most suitable model for all types of data.
For example, \cref{tab:wt103} still found a gap compared to Transformers for language modeling.
An interesting future direction is exploring combinations of S4 with other sequence models to complement their strengths.
We are excited about other directions, including continuing to explore the benefits of S4 on audio data (e.g. pre-training or generation settings),
and generalizing HiPPO and S4 to higher-dimensional data for image and video applications.
 \section{Numerical Instability of LSSL}
\label{sec:lssl-instability}


This section proves the claims made in \cref{sec:s4-motivation} about prior work.
We first derive the explicit diagonalization of the HiPPO matrix, confirming its instability because of exponentially large entries.
We then discuss the proposed theoretically fast algorithm from \citep{gu2021lssl} (Theorem 2) and show that it also involves exponentially large terms and thus cannot be implemented.

\subsection{HiPPO Diagonalization}

\begin{proof}[Proof of \cref{lmm:hippo-diagonalization}]The HiPPO matrix \eqref{eq:hippo} is equal, up to sign and conjugation by a diagonal matrix, to
  
  Our goal is to show that this  is diagonalized by the matrix
  
  or in other words that columns of this matrix are eigenvectors of .

  Concretely, we will show that the -th column of this matrix  with elements
  
  is an eigenvector with eigenvalue .
  In other words we must show that for all indices ,
  

  If , then for all  inside the sum, either  or .
  In the first case  and in the second case ,
  so both sides of equation \eqref{eq:diagonalization-proof} are equal to .

  It remains to show the case , which proceeds by induction on .
  Expanding equation \eqref{eq:diagonalization-proof} using the formula for  yields
  

  In the base case , the sum disappears and we are left with , as desired.

  Otherwise, the sum for  is the same as the sum for  but with sign reversed and a few edge terms.
  The result follows from applying the inductive hypothesis and algebraic simplification:
  

\end{proof}

\subsection{Fast but Unstable LSSL Algorithm}

Instead of diagonalization,
\citet[Theorem 2]{gu2021lssl} proposed a sophisticated fast algorithm to compute

This algorithm runs in  operations and  space.
However, we now show that this algorithm is also numerically unstable.

There are several reasons for the instability of this algorithm, but most directly we can pinpoint a particular intermediate quantity that they use.
\begin{definition}The fast LSSL algorithm computes coefficients of , the characteristic polynomial of , as an intermediate computation.
  Additionally, it computes the coefficients of its inverse, .
\end{definition}

We now claim that this quantity is numerically unfeasible.
We narrow down to the case when  is the identity matrix.
Note that this case is actually in some sense the most typical case:
when discretizing the continuous-time SSM to discrete-time by a step-size ,
the discretized transition matrix  is brought closer to the identity.
For example, with the Euler discretization ,
we have  as the step size .

\begin{lemma}When , the fast LSSL algorithm requires computing terms exponentially large in .
\end{lemma}
\begin{proof}The characteristic polynomial of  is
  
  These coefficients have size up to .

  The inverse of  has even larger coefficients.
  It can be calculated in closed form by the generalized binomial formula:
  
  Taking this , the largest coefficient is
  
  When  this is
  
  already larger than the coefficients of , and only increases as  grows.
\end{proof}

\section{\methodabbrv{} Algorithm Details}
\label{sec:s4-details}

This section proves the results of \cref{sec:s4-efficiency}, providing complete details of our efficient algorithms for \methodabbrv{}.

\cref{sec:s4-nplr-proof,sec:s4-recurrence-proof,sec:s4-convolution-proof}
prove \cref{thm:hippo-nplr,thm:s4-recurrence,thm:s4-convolution}
respectively.

\subsection{NPLR Representations of HiPPO Matrices}
\label{sec:s4-nplr-proof}

We first prove \cref{thm:hippo-nplr},
showing that all HiPPO matrices for continuous-time memory fall under the \methodabbrv{} normal plus low-rank (NPLR) representation.

\begin{proof}[Proof of \cref{thm:hippo-nplr}]We consider each of the three cases HiPPO-LagT, HiPPO-LegT, and HiPPO-LegS separately.
  Note that the primary HiPPO matrix defined in this work (equation \eqref{eq:hippo}) is the HiPPO-LegT matrix.

  \textbf{HiPPO-LagT.}
  The HiPPO-LagT matrix is simply
  
  Adding the matrix of all , which is rank 1, yields
  
  This matrix is now skew-symmetric.
  Skew-symmetric matrices are a particular case of normal matrices
  with pure-imaginary eigenvalues.

  \citet{gu2020hippo} also consider a case of HiPPO corresponding to the generalized Laguerre polynomials that generalizes
  the above HiPPO-LagT case.
  In this case, the matrix  (up to conjugation by a diagonal matrix) ends up being close to the above matrix,
  but with a different element on the diagonal.
  After adding the rank-1 correction, it becomes the above skew-symmetric matrix plus a multiple of the identity.
  Thus after diagonalization by the same matrix as in the LagT case, it is still reduced to diagonal plus low-rank (DPLR) form,
  where the diagonal is now pure imaginary plus a real constant.

  \textbf{HiPPO-LegS.}
  We restate the formula from equation \eqref{eq:hippo} for convenience.
  
  Adding  to the whole matrix gives
  

  Note that this matrix is not skew-symmetric,
  but is  where  is a skew-symmetric matrix.
  This is diagonalizable by the same unitary matrix that diagonalizes .

  \textbf{HiPPO-LegT.}

  Up to the diagonal scaling,
  the LegT matrix is
  
  By adding  to this matrix and then the matrix
  
  the matrix becomes
  
  which is skew-symmetric.
  In fact, this matrix is the inverse of the Chebyshev Jacobi.

  An alternative way to see this is as follows.
  The LegT matrix is the inverse of the matrix
  
  This can obviously be converted to a skew-symmetric matrix by adding a rank 2 term.
  The inverses of these matrices are also rank-2 differences from each other by the Woodbury identity.

  A final form is
  
  This has the advantage that the rank-2 correction is symmetric (like the others),
  but the normal skew-symmetric matrix is now -quasiseparable instead of -quasiseparable.

\end{proof}

\subsection{Computing the \methodabbrv{} Recurrent View}
\label{sec:s4-recurrence-proof}

We prove \cref{thm:s4-recurrence} showing the efficiency of the \methodabbrv{} parameterization for computing one step of the recurrent representation (\cref{sec:ss-recurrent}).

Recall that without loss of generality, we can assume that the state matrix  is diagonal plus low-rank (DPLR), potentially over .
Our goal in this section is to explicitly write out a closed form for the discretized matrix .

Recall from equation \eqref{eq:2} that




We first simplify both terms in the definition of  independently.

\textbf{Forward discretization.}
The first term is essentially the Euler discretization motivated in \cref{sec:ss-recurrent}.

where  is defined as the term in the final brackets.

\textbf{Backward discretization.}
The second term is known as the Backward Euler's method.
Although this inverse term is normally difficult to deal with,
in the DPLR case we can simplify it using Woodbury's Identity (\cref{prop:woodbury}).

where 
and  is defined as the term in the final brackets.
Note that

is actually a scalar in the case when the low-rank term has rank .


\textbf{\methodabbrv{} Recurrence.}
Finally, the full bilinear discretization can be rewritten in terms of these matrices as

The discrete-time SSM \eqref{eq:2} becomes

Note that  are accessed only through matrix-vector multiplications.
Since they are both DPLR, they have  matrix-vector multiplication,
showing \cref{thm:s4-recurrence}.


\subsection{Computing the Convolutional View}
\label{sec:s4-convolution-proof}

The most involved part of using SSMs efficiently is computing .
This algorithm was sketched in \cref{sec:s4-overview} and is the main motivation for the \methodabbrv{} parameterization.
In this section, we define the necessary intermediate quantities and prove the main technical result. 


The algorithm for \cref{thm:s4-convolution} falls in roughly three stages, leading to \cref{alg:s4-convolution}.
Assuming  has been conjugated into diagonal plus low-rank form, we successively simplify the problem of computing 
by applying the techniques outlined in \cref{sec:s4-overview}.

\begin{remark}
  \textbf{We note that for the remainder of this section}, we transpose  to be a column vector of shape  or  instead of matrix or row vector  as in \eqref{eq:1}.
  In other words the SSM is
  
  This convention is made so that  has the same shape as  and simplifies the implementation of S4.
\end{remark}

\paragraph{Reduction 0: Diagonalization}
By \cref{lmm:conjugation}, we can switch the representation by conjugating with any unitary matrix.
For the remainder of this section, we can assume that  is (complex) diagonal plus low-rank (DPLR).

Note that unlike diagonal matrices, a DPLR matrix does not lend itself to efficient computation of .
The reason is that  computes terms  which involve powers of the matrix .
These are trivially computable when  is diagonal, but is no longer possible for even simple modifications to diagonal matrices such as DPLR.

\paragraph{Reduction 1: SSM Generating Function}

To address the problem of computing powers of , we introduce another technique.
Instead of computing the SSM convolution filter  directly,
we introduce a generating function on its coefficients and compute evaluations of it.

\begin{definition}[SSM Generating Function]\label{def:generating-function}
  We define the following quantities:
  \begin{itemize}\item The \emph{SSM convolution function} is 
      and the (truncated) SSM filter of length 
      
    \item The \emph{SSM generating function} at node  is
      
      and the \emph{truncated SSM generating function} at node  is
      
    \item The truncated SSM generating function at nodes  is
      

  \end{itemize}
\end{definition}

Intuitively, the generating function essentially converts the SSM convolution filter from the time domain to frequency domain.
Importantly, it preserves the same information, and the desired SSM convolution filter can be recovered from evaluations of its generating function.
\begin{lemma}The SSM function  can be computed from the SSM generating function 
  at the roots of unity 
  stably in  operations.
\end{lemma}
\begin{proof}For convenience define
  
  Note that
  
  Note that this is exactly the same as the Discrete Fourier Transform (DFT):
  
  Therefore  can be recovered from  with a single inverse DFT,
  which requires  operations with the Fast Fourier Transform (FFT) algorithm.
\end{proof}


\paragraph{Reduction 2: Woodbury Correction}

The primary motivation of \cref{def:generating-function} is that it turns \emph{powers} of  into a single \emph{inverse} of  (equation \eqref{eq:generating-function}).
While DPLR matrices cannot be powered efficiently due to the low-rank term, they can be inverted efficiently by the well-known Woodbury identity.

\begin{proposition}[Binomial Inverse Theorem or Woodbury matrix identity~\cite{woodbury1950,golub2013matrix}]
  \label{prop:woodbury}
  Over a commutative ring , let  and . Suppose  and  are invertible. Then  is invertible and
  
\end{proposition}

With this identity, we can convert the SSM generating function on a DPLR matrix  into one on just its diagonal component.

\begin{lemma}\label{lmm:resolvent-woodbury}
  Let  be a diagonal plus low-rank representation.
  Then for any root of unity , the truncated generating function satisfies
  
\end{lemma}
\begin{proof}Directly expanding \cref{def:generating-function} yields
  
  where .

  We can now explicitly expand the discretized SSM matrices  and  back in terms of the original SSM parameters  and .
  \cref{lmm:bilinear-resolvent} provides an explicit formula, which allows further simplifying
  
  The last line applies the Woodbury Identity (\cref{prop:woodbury}) where .
\end{proof}


The previous proof used the following self-contained result to back out the original SSM matrices from the discretization.
\begin{lemma}\label{lmm:bilinear-resolvent}
  Let  be the SSM matrices  discretized by the bilinear discretization with step size . Then
  
\end{lemma}
\begin{proof}Recall that the bilinear discretization that we use (equation \eqref{eq:2}) is
  
  The result is proved algebraic manipulations.
  
\end{proof}

Note that in the \methodabbrv{} parameterization, instead of constantly computing ,
we can simply reparameterize our parameters to learn  directly instead of ,
saving a minor computation cost and simplifying the algorithm.

\paragraph{Reduction 3: Cauchy Kernel}
We have reduced the original problem of computing  to the problem of computing the SSM generating function 
in the case that  is a diagonal matrix.
We show that this is exactly the same as a Cauchy kernel, which is a well-studied problem with fast and stable numerical algorithms.

\begin{definition}\label{def:cauchy}
  A \textbf{Cauchy matrix} or kernel on nodes  and  is
  
  The computation time of a Cauchy matrix-vector product of size  is denoted by .
\end{definition}

Computing with Cauchy matrices is an extremely well-studied problem in numerical analysis,
with both fast arithmetic algorithms and fast numerical algorithms based on the famous Fast Multipole Method (FMM)
\citep{pan2001structured,pan2015transformations,pan2017fast}.
\begin{proposition}[Cauchy]\label{prop:cauchy}
  A Cauchy kernel requires  space, and operation count
   \varepsilon 
\end{proposition}

\begin{corollary}Evaluating  (defined in \cref{lmm:resolvent-woodbury}) for any set of nodes , diagonal matrix , and vectors  can be computed in  operations and  space, where  is the cost of a Cauchy matrix-vector multiplication.
\end{corollary}
\begin{proof}For any fixed , we want to compute . Computing this over all  is therefore exactly a Cauchy matrix-vector multiplication.
\end{proof}

This completes the proof of \cref{thm:s4-convolution}.
In \cref{alg:s4-convolution},
note that the work is dominated by Step \ref{step:cauchy},
which has a constant number of calls to a black-box Cauchy kernel, with complexity given by \cref{prop:cauchy}.

 
\section{Experiment Details and Full Results}
\label{sec:experiment-details}

This section contains full experimental procedures and extended results and citations for our experimental evaluation in \cref{sec:experiments}.
\cref{sec:experiment-details-benchmarking} corresponds to benchmarking results in \cref{sec:experiments-benchmark},
\cref{sec:experiment-details-lrd} corresponds to LRD experiments (LRA and Speech Commands) in \cref{sec:experiments-lrd},
and \cref{sec:experiment-details-general} corresponds to the general sequence modeling experiments (generation, image classification, forecasting) in \cref{sec:experiments-general}.

\subsection{Benchmarking}
\label{sec:experiment-details-benchmarking}

Benchmarking results from \cref{tab:ssm-benchmark} and \cref{tab:lra-benchmark} were tested on a single A100 GPU.

\paragraph{Benchmarks against LSSL}

For a given dimension , a single LSSL or \methodabbrv{} layer was constructed with  hidden features.
For LSSL, the state size  was set to  as done in \citep{gu2021lssl}.
For \methodabbrv{}, the state size  was set to parameter-match the LSSL, which was a state size of  due to differences in the parameterization.
\cref{tab:ssm-benchmark} benchmarks a single forward+backward pass of a single layer.

\paragraph{Benchmarks against Efficient Transformers}
Following \citep{tay2021long}, the Transformer models had 4 layers, hidden dimension  with  heads, query/key/value projection dimension , and batch size , for a total of roughly  parameters.
The \methodabbrv{} model was parameter tied while keeping the depth and hidden dimension constant (leading to a state size of .

We note that the relative orderings of these methods can vary depending on the exact hyperparameter settings.

\subsection{Long-Range Dependencies}
\label{sec:experiment-details-lrd}
This section includes information for reproducing our experiments on the Long-Range Arena and Speech Commands long-range dependency tasks.

\paragraph{Long Range Arena}

\cref{tab:lra-full} contains extended results table with all 11 methods considered in \citep{tay2021long}.

\begin{table}[t]
  \small
  \caption{Full results for the Long Range Arena (LRA) benchmark for long-range dependencies in sequence models. (Top): Original Transformer variants in LRA. (Bottom): Other models reported in the literature.}
    \centering
    \begin{tabular}{@{}llllllll@{}}
        \toprule
        Model                 & \textsc{ListOps}  & \textsc{Text}     & \textsc{Retrieval} & \textsc{Image}    & \textsc{Pathfinder} & \textsc{Path-X} & \textsc{Avg}      \\
        \midrule
        Random                & 10.00             & 50.00             & 50.00              & 10.00             & 50.00               & 50.00           & 36.67             \\
        \midrule
        Transformer           & 36.37             & 64.27             & 57.46              & 42.44             & 71.40               & \xmark          & 53.66             \\
        Local Attention       & 15.82             & 52.98             & 53.39              & 41.46             & 66.63               & \xmark          & 46.71             \\
        Sparse Trans.         & 17.07             & 63.58             & 59.59              & 44.24             & 71.71               & \xmark          & 51.03             \\
        Longformer            & 35.63             & 62.85             & 56.89              & 42.22             & 69.71               & \xmark          & 52.88             \\
        Linformer             & 35.70             & 53.94             & 52.27              & 38.56             & 76.34               & \xmark          & 51.14             \\
        Reformer              & \underline{37.27} & 56.10             & 53.40              & 38.07             & 68.50               & \xmark          & 50.56             \\
        Sinkhorn Trans.       & 33.67             & 61.20             & 53.83              & 41.23             & 67.45               & \xmark          & 51.23             \\
        Synthesizer           & 36.99             & 61.68             & 54.67              & 41.61             & 69.45               & \xmark          & 52.40             \\
        BigBird               & 36.05             & 64.02             & 59.29              & 40.83             & 74.87               & \xmark          & 54.17             \\
        Linear Trans.         & 16.13             & \underline{65.90} & 53.09              & 42.34             & 75.30               & \xmark          & 50.46             \\
        Performer             & 18.01             & 65.40             & 53.82              & 42.77             & 77.05               & \xmark          & 51.18             \\
        \midrule
        FNet                  & 35.33             & 65.11             & 59.61              & 38.67             & \underline{77.80}   & \xmark          & 54.42             \\
        Nystr{\"o}mformer     & 37.15             & 65.52             & \underline{79.56}  & 41.58             & 70.94               & \xmark          & 57.46             \\
        Luna-256              & 37.25             & 64.57             & 79.29              & \underline{47.38} & 77.72               & \xmark          & \underline{59.37} \\
        \textbf{\methodabbrv} (original) & 58.35   & 76.02 & 87.09     & 87.26 & 86.05      & 88.10  & 80.48 \\
        \textbf{\methodabbrv} (updated)  & \textbf{59.60}   & \textbf{86.82} & \textbf{90.90}     & \textbf{88.65} & \textbf{94.20}      & \textbf{96.35}  & \textbf{86.09} \\
        \bottomrule
    \end{tabular}
    \label{tab:lra-full}
\end{table}

For the \methodabbrv{} model, hyperparameters for all datasets are reported in \cref{tab::best-hyperparameters}.
For all datasets, we used the AdamW optimizer with a constant learning rate schedule with decay on validation plateau.
However, the learning rate on HiPPO parameters (in particular  were reduced to a maximum starting LR of , which improves stability since the HiPPO equation is crucial to performance.

The \methodabbrv{} state size was always fixed to .

As \methodabbrv{} is a sequence-to-sequence model with output shape (batch, length, dimension) and LRA tasks are classification,
mean pooling along the length dimension was applied after the last layer.

We note that most of these results were trained for far longer than what was necessary to achieve SotA results (e.g., the \texttt{Image} task reaches SotA in 1 epoch).
Results often keep improving with longer training times.

\textbf{Updated results.}
The above hyperparameters describe the results reported in the original paper, shown in \cref{tab:lra-full}, which have since been improved.
See \cref{sec:reproduction}.

\textbf{Hardware.}
All models were run on single GPU.
Some tasks used an A100 GPU (notably, the Path-X experiments), which has a larger max memory of 40Gb.
To reproduce these on smaller GPUs, the batch size can be reduced or gradients can be accumulated for two batches.

\begin{table*}[!t]
  \caption{
    The values of the best hyperparameters found for classification datasets; LRA (Top) and images/speech (Bottom).
    LR is learning rate and WD is weight decay. BN and LN refer to Batch Normalization and Layer Normalization.
  }
  \label{tab::best-hyperparameters}
  \centering
  \resizebox{\textwidth}{!}{\begin{tabular}{@{}llllllllllll@{}}
      \toprule
                                      & \textbf{Depth} & \textbf{Features } & \textbf{Norm} & \textbf{Pre-norm} & {\bf Dropout} & {\bf LR} & {\bf Batch Size} & {\bf Epochs} & \textbf{WD} & \textbf{Patience} \\
      \midrule
      \textbf{ListOps}                & 6              & 128                       & BN            & False             & 0             & 0.01     & 100              & 50           & 0.01        & 5                 \\
      \textbf{Text}                   & 4              & 64                        & BN            & True              & 0             & 0.001    & 50               & 20           & 0           & 5                 \\
      \textbf{Retrieval}              & 6              & 256                       & BN            & True              & 0             & 0.002    & 64               & 20           & 0           & 20                \\
      \textbf{Image}                  & 6              & 512                       & LN            & False             & 0.2           & 0.004    & 50               & 200          & 0.01        & 20                \\
      \textbf{Pathfinder}             & 6              & 256                       & BN            & True              & 0.1           & 0.004    & 100              & 200          & 0           & 10                \\
      \textbf{Path-X}                 & 6              & 256                       & BN            & True              & 0.0           & 0.0005   & 32               & 100          & 0           & 20                \\
      \midrule
      \textbf{CIFAR-10}               & 6              & 1024                      & LN            & False             & 0.25          & 0.01     & 50               & 200          & 0.01        & 20                \\
      \midrule
      \textbf{Speech Commands (MFCC)} & 4              & 256                       & LN            & False             & 0.2           & 0.01     & 100              & 50           & 0           & 5                 \\
      \textbf{Speech Commands (Raw)}  & 6              & 128                       & BN            & True              & 0.1           & 0.01     & 20               & 150          & 0           & 10                \\
      \bottomrule
    \end{tabular}}
\end{table*}


\paragraph{Speech Commands}
We provide details of sweeps run for baseline methods run by us---numbers for all others method are taken from \citet{gu2021lssl}. The best hyperparameters used for \methodabbrv{} are included in Table~\ref{tab::best-hyperparameters}.

\textit{Transformer~\citep{vaswani2017attention}} For MFCC, we swept the number of model layers , dropout  and learning rates . We used  attention heads, model dimension , prenorm, positional encodings, and trained for  epochs with a batch size of . For Raw, the Transformer model's memory usage made training impossible.

\textit{Performer~\citep{choromanski2020rethinking}} For MFCC, we swept the number of model layers , dropout  and learning rates . We used  attention heads, model dimension , prenorm, positional encodings, and trained for  epochs with a batch size of . For Raw, we used a model dimension of ,  attention heads, prenorm, and a batch size of . We reduced the number of model layers to , so the model would fit on the single GPU. We trained for  epochs with a learning rate of  and no dropout.

\textit{ExpRNN~\citep{lezcano2019cheap}} For MFCC, we swept hidden sizes  and learning rates . Training was run for  epochs, with a single layer model using a batch size of . For Raw, we swept hidden sizes  and learning rates  (however, ExpRNN failed to learn).

\textit{LipschitzRNN~\citep{erichson2021lipschitz}} For MFCC, we swept hidden sizes  and learning rates . Training was run for  epochs, with a single layer model using a batch size of . For Raw, we found that LipschitzRNN was too slow to train on a single GPU (requiring a full day for  epoch of training alone).

\textit{WaveGAN Discriminator~\citep{Donahue2019AdversarialAS}}
The WaveGAN-D in \cref{tab:sc} is actually our improved version of the discriminator network from the recent WaveGAN model for speech~\citep{Donahue2019AdversarialAS}.
This CNN actually did not work well out-of-the-box, and we added several features to help it perform better.
The final model is highly specialized compared to our model, and includes:
\begin{itemize}\item Downsampling or pooling between layers, induced by strided convolutions, that decrease the sequence length between layers.
  \item A global fully-connected output layer; thus the model only works for one input sequence length and does not work on MFCC features or the frequency-shift setting in \cref{tab:sc}.
  \item Batch Normalization is essential, whereas \methodabbrv{} works equally well with either Batch Normalization or Layer Normalization.
  \item Almost  as many parameters as the \methodabbrv{} model (M vs. M).
\end{itemize}

\subsection{General Sequence Modeling}
\label{sec:experiment-details-general}

This subsection corresponds to the experiments in \cref{sec:experiments-general}.
Because of the number of experiments in this section,
we use subsubsection dividers for different tasks to make it easier to follow:
CIFAR-10 density estimation (\cref{sec:experiment-details-general-cifargen}),
WikiText-103 language modeling (\cref{sec:experiment-details-general-wt103}),
autoregressive generation (\cref{sec:experiment-details-general-speed}),
sequential image classification (\cref{sec:experiment-details-general-image}),
and time-series forecasting (\cref{sec:experiment-details-general-informer}).

\subsubsection{CIFAR Density Estimation}
\label{sec:experiment-details-general-cifargen}

This task used a different backbone than the rest of our experiments.
We used blocks of alternating \methodabbrv{} layers and position-wise feed-forward layers (in the style of Transformer blocks).
Each feed-forward intermediate dimension was set to  the hidden size of the incoming \methodabbrv{} layer.
Similar to \citet{salimans2017pixelcnn++}, we used a UNet-style backbone consisting of  identical blocks followed by a downsampling layer.
The downsampling rates were  (the 3 chosen because the sequence consists of RGB pixels).
The base model had  with starting hidden dimension 128,
while the large model had  with starting hidden dimension 192.

We experimented with both the mixture of logistics from \citep{salimans2017pixelcnn++} as well as a simpler 256-way categorical loss.
We found they were pretty close and ended up using the simpler softmax loss along with using input embeddings.

We used the LAMB optimizer with learning rate 0.005.
The base model had no dropout, while the large model had dropout 0.1 before the linear layers inside the \methodabbrv{} and FF blocks.


\subsubsection{WikiText-103 Language Modeling}
\label{sec:experiment-details-general-wt103}

The RNN baselines included in \cref{tab:wt103} are the
AWD-QRNN~\citep{merity2018scalable}, an efficient linear gated RNN,
and the LSTM + Cache + Hebbian + MbPA \citep{rae2018fast}, the best performing pure RNN in the literature.
The CNN baselines are
the CNN with GLU activations~\citep{dauphin2017language},
the TrellisNet~\citep{trellisnet},
Dynamic Convolutions~\citep{wu2019pay},
and TaLK Convolutions~\citep{lioutas2020time}.

The Transformer baseline is \citep{baevski2018adaptive},
which uses Adaptive Inputs with a tied Adaptive Softmax.
This model is a standard high-performing Transformer baseline on this benchmark,
used for example by \citet{lioutas2020time} and many more.

Our \methodabbrv{} model uses the same Transformer backbone as in \citep{baevski2018adaptive}.
The model consists of 16 blocks of \methodabbrv{} layers alternated with position-wise feedforward layers, with a feature dimension of 1024.
Because our \methodabbrv{} layer has around 1/4 the number of parameters as a self-attention layer with the same dimension, we made two modifications to match the parameter count better:
(i) we used a GLU activation after the \methodabbrv{} linear layer (\cref{sec:s4-architecture})
(ii) we used two \methodabbrv{} layers per block.
Blocks use Layer Normalization in the pre-norm position.
The embedding and softmax layers were the Adaptive Embedding from \citep{baevski2018adaptive} with standard cutoffs 20000, 40000, 200000.

Evaluation was performed similarly to the basic setting in \citep{baevski2018adaptive}, Table 5,
which uses sliding non-overlapping windows.
Other settings are reported in \citep{baevski2018adaptive} that include more context at training and evaluation time and improves the score.
Because such evaluation protocols are orthogonal to the basic model, we do not consider them and report the base score from \citep{baevski2018adaptive} Table 5.

Instead of SGD+Momentum with multiple cosine learning rate annealing cycles,
our \methodabbrv{} model was trained with the simpler AdamW optimizer with a single cosine learning rate cycle with a maximum of 800000 steps.
The initial learning rate was set to 0.0005.
We used 8 A100 GPUs with a batch size of 1 per gpu and context size 8192.
We used no gradient clipping and a weight decay of 0.1.
Unlike \citep{baevski2018adaptive} which specified different dropout rates for different parameters,
we used a constant dropout rate of 0.25 throughout the network, including before every linear layer and on the residual branches.


\subsubsection{Autoregressive Generation Speed}
\label{sec:experiment-details-general-speed}

\paragraph{Protocol.}
To account for different model sizes and memory requirements for each method,
we benchmark generation speed by throughput,
measured in images per second (\cref{tab:cifar-generation}) or tokens per second (\cref{tab:wt103}).
Each model generates images on a single  GPU,
maximizing batch size to fit in memory.
(For CIFAR-10 generation we limited memory to 16Gb, to be more comparable to the Transformer and Linear Transformer results reported from \citep{katharopoulos2020transformers}.)

\paragraph{Baselines.}
The Transformer and Linear Transformer baselines reported in \cref{tab:cifar-generation} are the results reported directly from \citet{katharopoulos2020transformers}.
Note that the Transformer number is the one in their Appendix, which implements the optimized cached implementation of self-attention.

For all other baseline models, we used open source implementations of the models to benchmark generation speed.
For the PixelCNN++, we used the fast cached version by \citet{ramachandran2017fast},
which sped up generation by orders of magnitude from the naive implementation.
This code was only available in TensorFlow, which may have slight differences compared to the rest of the baselines which were implemented in PyTorch.

We were unable to run the Sparse Transformer~\citep{child2019generating} model due to issues with their custom CUDA implementation of the sparse attention kernel, which we were unable to resolve.

The Transformer baseline from \cref{tab:wt103} was run using a modified GPT-2 backbone from the HuggingFace repository, configured to recreate the architecture reported in \citep{baevski2018adaptive}.
These numbers are actually slightly favorable to the baseline, as we did not include the timing of the embedding or softmax layers, whereas the number reported for \methodabbrv{} is the full model.

\subsubsection{Pixel-Level Sequential Image Classification}
\label{sec:experiment-details-general-image}

Our models were trained with the AdamW optimizer for up to 200 epochs.
Hyperparameters for the CIFAR-10 model is reported in \cref{tab::best-hyperparameters}.

For our comparisons against ResNet-18, the main differences between the base models are that \methodabbrv{} uses LayerNorm by default while ResNet uses BatchNorm.
The last ablation in \cref{sec:experiments-general} swaps the normalization type,
using BatchNorm for \methodabbrv{} and LayerNorm for ResNet,
to ablate this architectural difference.
The experiments with augmentation take the base model and train with mild data augmentation: horizontal flips and random crops (with symmetric padding).

\begin{table}[t]
  \small
  \centering
  \captionsetup{type=table}
  \caption{
    (\textbf{Pixel-level image classification.})
    Citations refer to the original model; additional citation indicates work from which this baseline is reported.
  }
  \begin{tabular}{@{}llll@{}}
    \toprule
    Model                                                      & \textsc{sMNIST}   & \textsc{pMNIST}   & \textsc{sCIFAR}   \\
    \midrule
    Transformer~\citep{vaswani2017attention,trinh2018learning} & 98.9              & 97.9              & 62.2              \\
    \midrule
    CKConv~\citep{romero2021ckconv}                            & 99.32             & \underline{98.54} & 63.74             \\
    TrellisNet~\citep{trellisnet}                              & 99.20             & 98.13             & 73.42             \\
    TCN~\citep{bai2018empirical}                               & 99.0              & 97.2              & -                 \\
    \midrule
    LSTM~\citep{lstm,gu2020improving}                          & 98.9              & 95.11             & 63.01             \\
    r-LSTM ~\citep{trinh2018learning}                          & 98.4              & 95.2              & 72.2              \\
    Dilated GRU~\citep{chang2017dilated}                       & 99.0              & 94.6              & -                 \\
    Dilated RNN~\citep{chang2017dilated}                       & 98.0              & 96.1              & -                 \\
    IndRNN~\citep{indrnn}                                      & 99.0              & 96.0              & -                 \\
    expRNN~\citep{lezcano2019cheap}                            & 98.7              & 96.6              & -                 \\
    UR-LSTM                                                    & 99.28             & 96.96             & 71.00             \\
    UR-GRU~\citep{gu2020improving}                             & 99.27             & 96.51             & \underline{74.4}  \\
    LMU~\citep{voelker2019legendre}                            & -                 & 97.15             & -                 \\
    HiPPO-RNN~\citep{gu2020hippo}                              & 98.9              & 98.3              & 61.1              \\
    UNIcoRNN~\citep{rusch2021unicornn}                         & -                 & 98.4              & -                 \\
    LMUFFT~\citep{chilkuri2021parallelizing}                   & -                 & 98.49             & -                 \\
    LipschitzRNN~\citep{erichson2021lipschitz}                 & \underline{99.4}  & 96.3              & 64.2              \\
    \midrule
    \textbf{\methodabbrv}                                                & \textbf{99.63}    & \textbf{98.70} & \textbf{91.13}    \\
    \bottomrule
  \end{tabular}
  \label{tab:image-full}
\end{table}


\subsubsection{Time Series Forecasting compared to Informer}
\label{sec:experiment-details-general-informer}

We include a simple figure (\cref{fig:s4-architecture}) contrasting the architecture of \methodabbrv{} against that of the Informer \citep{haoyietal-informer-2021}.

In \cref{fig:s4-architecture},
the goal is to forecast a contiguous range of future predictions (Green, length  )
given a range of past context (Blue, length  ).
We simply concatenate the entire context with a sequence of masks set to the length of the forecast window.
This input is a single sequence of length  that is run through the same simple deep \methodabbrv{} model used throughout this work,
which maps to an output of length  .
We then use just the last  outputs as the forecasted predictions.


\begin{figure}[t]
    \centering
    \begin{subfigure}{\linewidth}\centering
      \includegraphics[width=\linewidth]{figs/s4_forecasting.pdf}
    \end{subfigure}
    \caption{Comparison of \methodabbrv{} and specialized time-series models for forecasting tasks. (\textit{Top Left}) The forecasting task involves predicting future values of a time-series given past context. (\textit{Bottom Left}) We perform simple forecasting using a sequence model such as \methodabbrv{} as a black box. (\textit{Right}) Informer uses an encoder-decoder architecture designed specifically for forecasting problems involving a customized attention module (figure taken from~\citet{haoyietal-informer-2021}).}
    \label{fig:s4-architecture}
\end{figure}

\cref{tab:informer-s,tab:informer-m} contain full results on all 50 settings considered by \citet{haoyietal-informer-2021}.
\methodabbrv{} sets the best results on 40 out of 50 of these settings.

\begin{table*}[t]
\centering
\fontsize{9pt}{9pt}\selectfont
\centering
\resizebox{\linewidth}{!}{
\begin{tabular}{c|c|c|c|c|c|c|c|c|c|c|c}
\toprule[1.0pt]
\multicolumn{2}{c|}{Methods}              & \textbf{\methodabbrv} & {Informer}                     & {Informer}            & {LogTrans}       & {Reformer}   & {LSTMa}      & {DeepAR}     & {ARIMA}                 & {Prophet}    \\
\midrule[0.5pt]
\multicolumn{2}{c|}{Metric}               & MSE~~MAE              & MSE~~MAE                       & MSE~~MAE                       & MSE~~MAE         & MSE~~MAE     & MSE~~MAE     & MSE~~MAE     & MSE~~MAE                & MSE~~MAE     \\
\midrule[1.0pt]
\multirow{5}{*}{\rotatebox{90}{ETTh}} & 24                    & \textbf{0.061}~~\textbf{0.191} & 0.098~~0.247                   & {0.092}~~{0.246} & 0.103~~0.259 & 0.222~~0.389 & 0.114~~0.272 & 0.107~~0.280            & 0.108~~0.284  & 0.115~~0.275 \\
                                          & 48                    & \textbf{0.079}~~\textbf{0.220} & {0.158}~~{0.319}               & 0.161~~0.322     & 0.167~~0.328 & 0.284~~0.445 & 0.193~~0.358 & 0.162~~0.327            & 0.175~~0.424  & 0.168~~0.330 \\
                                          & 168                   & \textbf{0.104}~~\textbf{0.258} & {0.183}~~{0.346}               & 0.187~~0.355     & 0.207~~0.375 & 1.522~~1.191 & 0.236~~0.392 & 0.239~~0.422            & 0.396~~0.504  & 1.224~~0.763 \\
                                          & 336                   & \textbf{0.080}~~\textbf{0.229} & 0.222~~0.387                   & {0.215}~~{0.369} & 0.230~~0.398 & 1.860~~1.124 & 0.590~~0.698 & 0.445~~0.552            & 0.468~~0.593  & 1.549~~1.820 \\
                                          & 720                   & \textbf{0.116}~~\textbf{0.271} & 0.269~~0.435                   & {0.257}~~{0.421} & 0.273~~0.463 & 2.112~~1.436 & 0.683~~0.768 & 0.658~~0.707            & 0.659~~0.766  & 2.735~~3.253 \\
\midrule[0.5pt]
\multirow{5}{*}{\rotatebox{90}{ETTh}} & 24                    & 0.095~~0.234                   & \textbf{0.093}~~\textbf{0.240} & 0.099~~0.241     & 0.102~~0.255 & 0.263~~0.437 & 0.155~~0.307 & 0.098~~0.263            & 3.554~~0.445  & 0.199~~0.381 \\
                                          & 48                    & 0.191~~0.346                   & \textbf{0.155}~~\textbf{0.314} & 0.159~~0.317     & 0.169~~0.348 & 0.458~~0.545 & 0.190~~0.348 & 0.163~~0.341            & 3.190~~0.474  & 0.304~~0.462 \\
                                          & 168                   & \textbf{0.167}~~\textbf{0.333} & {0.232}~~{0.389}               & 0.235~~0.390     & 0.246~~0.422 & 1.029~~0.879 & 0.385~~0.514 & 0.255~~0.414            & 2.800~~0.595  & 2.145~~1.068 \\
                                          & 336                   & \textbf{0.189}~~\textbf{0.361} & 0.263~~{0.417}                 & {0.258}~~0.423   & 0.267~~0.437 & 1.668~~1.228 & 0.558~~0.606 & 0.604~~0.607            & 2.753~~0.738  & 2.096~~2.543 \\
                                          & 720                   & \textbf{0.187}~~\textbf{0.358} & {0.277}~~{ 0.431}              & 0.285~~0.442     & 0.303~~0.493 & 2.030~~1.721 & 0.640~~0.681 & 0.429~~0.580            & 2.878~~1.044  & 3.355~~4.664 \\
\midrule[0.5pt]
\multirow{5}{*}{\rotatebox{90}{ETTm}} & 24                    & \textbf{0.024}~~\textbf{0.117} & {0.030}~~{0.137}               & 0.034~~0.160     & 0.065~~0.202 & 0.095~~0.228 & 0.121~~0.233 & 0.091~~0.243            & 0.090~~0.206  & 0.120~~0.290 \\
                                          & 48                    & \textbf{0.051}~~\textbf{0.174} & 0.069~~0.203                   & {0.066}~~{0.194} & 0.078~~0.220 & 0.249~~0.390 & 0.305~~0.411 & 0.219~~0.362            & 0.179~~0.306  & 0.133~~0.305 \\
                                          & 96                    & \textbf{0.086}~~\textbf{0.229} & 0.194~~{0.372}                 & {0.187}~~0.384   & 0.199~~0.386 & 0.920~~0.767 & 0.287~~0.420 & 0.364~~0.496            & 0.272~~0.399  & 0.194~~0.396 \\
                                          & 288                   & \textbf{0.160}~~\textbf{0.327} & {0.401}~~0.554                 & 0.409~~{0.548}   & 0.411~~0.572 & 1.108~~1.245 & 0.524~~0.584 & 0.948~~0.795            & 0.462~~0.558  & 0.452~~0.574 \\
                                          & 672                   & \textbf{0.292}~~\textbf{0.466} & {0.512}~~{0.644}               & 0.519~~0.665     & 0.598~~0.702 & 1.793~~1.528 & 1.064~~0.873 & 2.437~~1.352            & 0.639~~0.697  & 2.747~~1.174 \\
\midrule[0.5pt]
\multirow{5}{*}{\rotatebox{90}{Weather}}  & 24                    & 0.125~~0.254                   & \textbf{0.117}~~\textbf{0.251} & 0.119~~0.256     & 0.136~~0.279 & 0.231~~0.401 & 0.131~~0.254 & 0.128~~0.274            & 0.219~~0.355  & 0.302~~0.433 \\
                                          & 48                    & 0.181~~\textbf{0.305}          & \textbf{0.178}~~0.318          & 0.185~~0.316     & 0.206~~0.356 & 0.328~~0.423 & 0.190~~0.334 & 0.203~~0.353            & 0.273~~0.409  & 0.445~~0.536 \\
                                          & 168                   & \textbf{0.198}~~\textbf{0.333} & {0.266}~~{0.398}               & 0.269~~0.404     & 0.309~~0.439 & 0.654~~0.634 & 0.341~~0.448 & 0.293~~0.451            & 0.503~~0.599  & 2.441~~1.142 \\
                                          & 336                   & 0.300~~0.417                   & \textbf{0.297}~~\textbf{0.416} & 0.310~~0.422     & 0.359~~0.484 & 1.792~~1.093 & 0.456~~0.554 & 0.585~~0.644            & 0.728~~0.730  & 1.987~~2.468 \\
                                          & 720                   & \textbf{0.245}~~\textbf{0.375} & {0.359}~~{0.466}               & 0.361~~0.471     & 0.388~~0.499 & 2.087~~1.534 & 0.866~~0.809 & 0.499~~0.596            & 1.062~~0.943  & 3.859~~1.144 \\
\midrule[0.5pt]
\multirow{5}{*}{\rotatebox{90}{ECL}}      & 48                    & 0.222~~\textbf{0.350}          & 0.239~~0.359                   & 0.238~~0.368     & 0.280~~0.429 & 0.971~~0.884 & 0.493~~0.539 & \textbf{0.204}~~{0.357} & 0.879~~0.764  & 0.524~~0.595 \\
                                          & 168                   & 0.331~~\textbf{0.421}          & 0.447~~0.503                   & 0.442~~0.514     & 0.454~~0.529 & 1.671~~1.587 & 0.723~~0.655 & \textbf{0.315}~~{0.436} & 1.032~~0.833  & 2.725~~1.273 \\
                                          & 336                   & \textbf{0.328}~~\textbf{0.422} & 0.489~~0.528                   & 0.501~~0.552     & 0.514~~0.563 & 3.528~~2.196 & 1.212~~0.898 & {0.414}~~{0.519}        & 1.136~~0.876  & 2.246~~3.077 \\
                                          & 720                   & \textbf{0.428}~~\textbf{0.494} & {0.540}~~{0.571}               & 0.543~~0.578     & 0.558~~0.609 & 4.891~~4.047 & 1.511~~0.966 & 0.563~~0.595            & 1.251~~0.933  & 4.243~~1.415 \\
                                          & 960                   & \textbf{0.432}~~\textbf{0.497} & {0.582}~~{0.608}               & 0.594~~0.638     & 0.624~~0.645 & 7.019~~5.105 & 1.545~~1.006 & 0.657~~0.683            & 1.370~~0.982  & 6.901~~4.264 \\

\midrule[1.0pt]
\multicolumn{2}{c|}{Count}                & {22}                  & {5}                            & {0}                            & {0}              & {0}          & {0}          & {2}          & {0}                     & {0}          \\
\bottomrule[1.0pt]

\end{tabular}}
\caption{Univariate long sequence time-series forecasting results on four datasets (five cases).}
\label{tab:informer-s}
\end{table*}


\begin{table*}[t]
\centering
\fontsize{9pt}{9pt}\selectfont
\resizebox{\linewidth}{!}{
\begin{tabular}{c|c|cc|cc|cc|cc|cc|cc|cc}
\toprule[1.0pt]
\multicolumn{2}{c}{Methods}               & \multicolumn{2}{|c}{\textbf{\methodabbrv}} & \multicolumn{2}{|c}{Informer} & \multicolumn{2}{|c}{Informer} & \multicolumn{2}{|c}{LogTrans} & \multicolumn{2}{|c}{Reformer} & \multicolumn{2}{|c}{LSTMa} & \multicolumn{2}{|c}{LSTnet} \\
\midrule[0.5pt]
\multicolumn{2}{c|}{Metric}               & MSE                                        & MAE                           & MSE                                    & MAE                           & MSE                           & MAE                        & MSE                          & MAE     & MSE   & MAE   & MSE   & MAE   & MSE   & MAE     \\
\midrule[1.0pt]
\multirow{5}{*}{\rotatebox{90}{ETTh}} & 24                                         & \textbf{0.525}                & \textbf{0.542}                         & {0.577}                       & {0.549}                       & 0.620                      & 0.577                        & 0.686   & 0.604 & 0.991 & 0.754 & 0.650 & 0.624 & 1.293    & 0.901 \\
                                          & 48                                         & \textbf{0.641}                & \textbf{0.615}                         & {0.685}                       & {0.625}                       & 0.692                      & 0.671                        & 0.766   & 0.757 & 1.313 & 0.906 & 0.702 & 0.675 & 1.456    & 0.960 \\
                                          & 168                                        & 0.980                         & 0.779                                  & \textbf{0.931}                & \textbf{0.752}                & 0.947                      & 0.797                        & 1.002   & 0.846 & 1.824 & 1.138 & 1.212 & 0.867 & 1.997    & 1.214 \\
                                          & 336                                        & 1.407                         & 0.910                                  & 1.128                         & 0.873                         & \textbf{1.094}             & \textbf{0.813}               & 1.362   & 0.952 & 2.117 & 1.280 & 1.424 & 0.994 & 2.655    & 1.369 \\
                                          & 720                                        & \textbf{1.162}                & \textbf{0.842}                         & {1.215}                       & {0.896}                       & 1.241                      & 0.917                        & 1.397   & 1.291 & 2.415 & 1.520 & 1.960 & 1.322 & 2.143    & 1.380 \\
\midrule[0.5pt]
\multirow{5}{*}{\rotatebox{90}{ETTh}} & 24                                         & 0.871                         & 0.736                                  & \textbf{0.720}                & \textbf{0.665}                & 0.753                      & 0.727                        & 0.828   & 0.750 & 1.531 & 1.613 & 1.143 & 0.813 & 2.742    & 1.457 \\
                                          & 48                                         & \textbf{1.240}                & \textbf{0.867}                         & {1.457}                       & {1.001}                       & 1.461                      & 1.077                        & 1.806   & 1.034 & 1.871 & 1.735 & 1.671 & 1.221 & 3.567    & 1.687 \\
                                          & 168                                        & \textbf{2.580}                & \textbf{1.255}                         & 3.489                         & {1.515}                       & 3.485                      & 1.612                        & 4.070   & 1.681 & 4.660 & 1.846 & 4.117 & 1.674 & {3.242}  & 2.513 \\
                                          & 336                                        & \textbf{1.980}                & \textbf{1.128}                         & 2.723                         & 1.340                         & 2.626                      & {1.285}                      & 3.875   & 1.763 & 4.028 & 1.688 & 3.434 & 1.549 & {2.544}  & 2.591 \\
                                          & 720                                        & \textbf{2.650}                & \textbf{1.340}                         & {3.467}                       & {1.473}                       & 3.548                      & 1.495                        & 3.913   & 1.552 & 5.381 & 2.015 & 3.963 & 1.788 & 4.625    & 3.709 \\
\midrule[0.5pt]
\multirow{5}{*}{\rotatebox{90}{ETTm}} & 24                                         & 0.426                         & 0.487                                  & 0.323                         & \textbf{0.369}                & \textbf{0.306}             & 0.371                        & 0.419   & 0.412 & 0.724 & 0.607 & 0.621 & 0.629 & 1.968    & 1.170 \\
                                          & 48                                         & 0.580                         & 0.565                                  & 0.494                         & 0.503                         & \textbf{0.465}             & \textbf{0.470}               & 0.507   & 0.583 & 1.098 & 0.777 & 1.392 & 0.939 & 1.999    & 1.215 \\
                                          & 96                                         & 0.699                         & 0.649                                  & \textbf{0.678}                & 0.614                         & 0.681                      & \textbf{0.612}               & 0.768   & 0.792 & 1.433 & 0.945 & 1.339 & 0.913 & 2.762    & 1.542 \\
                                          & 288                                        & \textbf{0.824}                & \textbf{0.674}                         & {1.056}                       & {0.786}                       & 1.162                      & 0.879                        & 1.462   & 1.320 & 1.820 & 1.094 & 1.740 & 1.124 & 1.257    & 2.076 \\
                                          & 672                                        & \textbf{0.846}                & \textbf{0.709}                         & {1.192}                       & {0.926}                       & 1.231                      & 1.103                        & 1.669   & 1.461 & 2.187 & 1.232 & 2.736 & 1.555 & 1.917    & 2.941 \\
\midrule[0.5pt]
\multirow{5}{*}{\rotatebox{90}{Weather}}  & 24                                         & \textbf{0.334}                & 0.385                                  & {0.335}                       & \textbf{0.381}                & 0.349                      & 0.397                        & 0.435   & 0.477 & 0.655 & 0.583 & 0.546 & 0.570 & 0.615    & 0.545 \\
                                          & 48                                         & 0.406                         & 0.444                                  & 0.395                         & 0.459                         & \textbf{0.386}             & \textbf{0.433}               & 0.426   & 0.495 & 0.729 & 0.666 & 0.829 & 0.677 & 0.660    & 0.589 \\
                                          & 168                                        & \textbf{0.525}                & \textbf{0.527}                         & {0.608}                       & {0.567}                       & 0.613                      & 0.582                        & 0.727   & 0.671 & 1.318 & 0.855 & 1.038 & 0.835 & 0.748    & 0.647 \\
                                          & 336                                        & \textbf{0.531}                & \textbf{0.539}                         & {0.702}                       & {0.620}                       & 0.707                      & 0.634                        & 0.754   & 0.670 & 1.930 & 1.167 & 1.657 & 1.059 & 0.782    & 0.683 \\
                                          & 720                                        & \textbf{0.578}                & \textbf{0.578}                         & {0.831}                       & {0.731}                       & 0.834                      & 0.741                        & 0.885   & 0.773 & 2.726 & 1.575 & 1.536 & 1.109 & 0.851    & 0.757 \\
\midrule[0.5pt]
\multirow{5}{*}{\rotatebox{90}{ECL}}      & 48                                         & \textbf{0.255}                & \textbf{0.352}                         & 0.344                         & {0.393}                       & 0.334                      & 0.399                        & 0.355   & 0.418 & 1.404 & 0.999 & 0.486 & 0.572 & 0.369    & 0.445 \\
                                          & 168                                        & \textbf{0.283}                & \textbf{0.373}                         & 0.368                         & 0.424                         & {0.353}                    & {0.420}                      & 0.368   & 0.432 & 1.515 & 1.069 & 0.574 & 0.602 & 0.394    & 0.476 \\
                                          & 336                                        & \textbf{0.292}                & \textbf{0.382}                         & 0.381                         & {0.431}                       & 0.381                      & 0.439                        & {0.373} & 0.439 & 1.601 & 1.104 & 0.886 & 0.795 & 0.419    & 0.477 \\
                                          & 720                                        & \textbf{0.289}                & \textbf{0.377}                         & 0.406                         & 0.443                         & {0.391}                    & {0.438}                      & 0.409   & 0.454 & 2.009 & 1.170 & 1.676 & 1.095 & 0.556    & 0.565 \\
                                          & 960                                        & \textbf{0.299}                & \textbf{0.387}                         & {0.460}                       & {0.548}                       & 0.492                      & 0.550                        & 0.477   & 0.589 & 2.141 & 1.387 & 1.591 & 1.128 & 0.605    & 0.599 \\
\midrule[1.0pt]
\multicolumn{2}{c}{Count}                 & \multicolumn{2}{|c}{18}                    & \multicolumn{2}{|c}{5}        & \multicolumn{2}{|c}{6}                 & \multicolumn{2}{|c}{0}        & \multicolumn{2}{|c}{0}        & \multicolumn{2}{|c}{0}     & \multicolumn{2}{|c}{0}      \\
\bottomrule[1.0pt]

\end{tabular}}
\caption{Multivariate long sequence time-series forecasting results on four datasets (five cases).}
\label{tab:informer-m}
\end{table*}


\subsection{Visualizations}
We visualize the convolutional filter  learned by \methodabbrv{} for the Pathfinder and CIFAR-10 tasks in \cref{fig:pathfinder-all-conv-filters}.

\begin{figure}
    \centering
    \begin{subfigure}{\linewidth}
        \includegraphics[width=\linewidth]{figs/pathfinder_filters_layer_0_trunc.png}
    \end{subfigure}
    \begin{subfigure}{\linewidth}
        \includegraphics[width=\linewidth]{figs/pathfinder_filters_layer_5_trunc.png}
    \end{subfigure}
    \label{fig:pathfinder-all-conv-filters}
    \caption{({\bf Convolutional filters on Pathfinder}) A random selection of filters learned by \methodabbrv{} in the first layer (top 2 rows) and last layer (bottom 2 rows) of the best model.}
\end{figure}

\subsection{Reproduction}
\label{sec:reproduction}

Since the first version of this paper, several experiments have been updated. Please read the corresponding paragraph below before citing LRA or SC results.

\paragraph{Long Range Arena}

Follow-ups to this paper expanded the theoretical understanding of S4 while improving some results.
The results reported in \cref{tab:lra} have been updated to results from the papers \citep{gu2022s4d,gu2022hippo}.
More specifically, the method S4-LegS in those works refers to the \emph{same model} presented in this paper, with the ``-LegS'' suffix referring to the initialization defined in equation \eqref{eq:hippo}. As such, results from the original \cref{tab:lra} have been directly updated.

The updated results have only minor hyperparameter changes compared to the original results. The original results and hyperparameters are shown in \cref{tab:lra-full} (\cref{sec:experiment-details-lrd}).
Appendix B of \citep{gu2022s4d} describes the changes in hyperparameters, which are also documented from the experiment configuration files in the publically available code at \url{https://github.com/HazyResearch/state-spaces}.

\paragraph{Speech Commands}

The Speech Commands (SC) dataset~\citep{Warden2018SpeechCA} is originally a 35-class dataset of spoken English words.
However, this paper was part of a line of work starting with \citet{kidger2020neural} that has used a smaller 10-class subset of SC \citep{kidger2020neural,romero2021ckconv,gu2021lssl,romero2022flexconv}.
\emph{In an effort to avoid dataset fragmentation in the literature, we have since moved to the original dataset.}
We are now calling this 10-class subset \textbf{SC10} to distinguish it from the full 35-class \textbf{SC} dataset.
To cite S4 as a baseline for Speech Commands, please use Table 11 from \citep{gu2022s4d} instead of \cref{tab:sc} from this paper.
In addition to using the full SC dataset, it also provides a number of much stronger baselines than the ones used in this work.


\paragraph{WikiText-103}

The original version of this paper used an S4 model with batch size , context size  which achieved a validation perplexity of 20.88 and test perplexity of 21.28.
It was later retrained with a batch size of  and context size  which achieved a validation perplexity of 19.69 and test perplexity of 20.95, and a model checkpoint is available in the public repository.
The rest of the model is essentially identical, so the results from the original table have been updated.
 
\end{document}
