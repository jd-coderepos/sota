
\documentclass[twocolumn,10pt]{article}
\usepackage[cmex10]{amsmath}
\interdisplaylinepenalty=2500
\usepackage{amsthm}
\usepackage{amsfonts}
\usepackage{geometry}
\usepackage{graphicx}
\usepackage{xcolor}
\usepackage{url}
\usepackage{mdwlist}
\usepackage[noend]{algorithmic}
\usepackage[normalem]{ulem}


\geometry{letterpaper,hmargin=.75in,tmargin=1in,bmargin=1in,columnsep=0.25in}

\title{A Secure Submission System for Online Whistleblowing Platforms}

\author{Volker Roth, Benjamin G\"uldenring, Eleanor Rieffel, Sven
  Dietrich, Lars Ries\
  \psi_s &: \Za \times \Zm \leftrightarrow \Zn \\
  \psi_s(a;b) &\mapsto (1+N)^{a}\cdot b^{N^s} \imod N^{s+1}

  \begin{array}[t]{l}
    \EncData(m, r_0) = \\
    \quad r_1, r_2 \leftarrow R \\
    \quad \text{chk} \leftarrow \text{if}~m,r_0=0 ~\text{then}~ 0 \\
    \qquad \text{else}~ H(m, r_0) \\
    \quad c \leftarrow \psi(m; h^{\text{chk}}\cdot g^{r_1})) \\
    \quad t \leftarrow \psi(r_0 || r_1;g^{r_2}) \\
    \quad \text{return} ~ c, t
  \end{array}

  \begin{array}[t]{l}
    \DecVrfy(c,t) = \\
    \quad (m; k), (r_0 || r_1; \cdot) \leftarrow \psi^{-1}(c),
    \psi^{-1}(t) \\
    \quad \text{chk} \leftarrow \text{if}~m,r_0=0 ~\text{then}~ 0 \\
    \qquad \text{else}~ H(m, r_0) \\
    \quad \text{if}~ h^{\text{chk}}\cdot g^{r_1} = k ~\text{then} \\
    \qquad \text{return}~m, r_0\\
    \quad \text{return} ~ \bot
  \end{array}

  c\cdot c'
  &=
  \psi(m+m';h^{H(m,r_0)+H(m',r_0')}\cdot g^{r_1+r_1'}) \\
  t\cdot t'
  &=
  \psi((r_0 || r_1)+(r_0' || r_1');g^{r_2+r_2'}) \\
  &=
  \psi((r_0+r_0') || (r_1+r_1');g^{r_2+r_2'})
\label{eq:H}
  H(m,r_0)+H(m',r_0') &\equiv H(m+m',r_0+r_0')

  H(m,r_0)+0 &\equiv H(m+0,r_0+0)

  \Pr[i=1] &= (1-\frac{1}{m})^{k-1}  \\
  \Pr[i=1] &= 1-(1-(1-\frac{1}{m/t})^{k-1})^t

  E(X) &= 1 + \frac{1}{2}\cdot \sum_{i=1}^{n} 2^{i} \cdot
  (1-(1-p)^{2^{n+1-i}})

\begin{proof}
  The root of the tree must always be decrypted, hence the expectation is
  always at least .  Since the algorithm only decrypts left children and
  not right children, we need to count only half of the remaining nodes.
  Recall that a right child is calculated by subtracting its sibling from
  its parent.  At level  from the root, starting under the root node, we
  have  nodes.  We have to decrypt a left child at level  if its
  parent is not zero.  The probability that the parent is not zero is one
  minus the probability that its  leaves are all zeroes.
\end{proof}
If we divide the expected number of decryptions by  decryptions (the
na\"\i{}ve approach) and plot the normalized results for several values of
 then we obtain the graphs in Figure~\ref{fig:cryptcost}.  The graphs tell
us, for example, that we expect to save  of the decryptions if
AdLeaks operates at its limit, that is, a recovery probability of  and
 gray or black ciphertexts.  The
lighter the load is the more we save.

\begin{figure}
  \includegraphics[width=\columnwidth]{arxiv-crypto}
  \caption{Shows the normalized savings of the tree decryption algorithm for
    various tree sizes and load characteristics.  For  the graph looks
    identical to the graph for .}
  \label{fig:cryptcost}
\end{figure}



\subsection{Number of Whistleblowers}
\label{sec:numwb}

We characterize next how many concurrent whistleblowers a link capacity of
 data chunks per second can support.  Since no data set exists
that we could use to estimate the arrival times of data chunks, we model
them as a Poisson process that we approximate by a normally distributed
process as before.  To be on the safe side we seek a safe average sending
rate  so that the actual sending rate does not exceed 
\Umaxdata{} in 97.725\% of all cases, that is, the second quantile.  The
safe average rate can be found easily by solving 
for , which yields \Kavgdata{} \Uavgdata.  At \Kads{} transmissions per
day and whistleblower this means that AdLeaks can serve \Kmaxwb{} concurrent
whistleblowers at any time with a \Kcapiii{} \Ucapiii{} uplink for the
decryptor.


\subsection{Decryptors}

The performance of the decryptor is bound by the cost of two operations: the
time it takes to test whether a tree node is white, and the time it takes to
perform a full decryption and verification.  We measured \Ktestmean{} and
\Kdecmean{} seconds for these operations on a single core (2.66GHz Intel
Xeon) with negligible standard deviation.  If we assume that the decryptor
has \Kcores{} cores available for decryption then we estimate that our
running example requires \Kunitsiii{} decryptors.  Since the necessary
resources for decryption scale linearly with the number of whistleblowers
this means that AdLeaks can serve about  concurrent whistleblowers
with just one unit, for example, a dual 6-core Mac Pro.


\subsection{Client Measurements}

Our JavaScript implementations of the DJ scheme leverage several
optimizations~\cite{Jurik2003} that improve efficiency over unoptimized DJ
by a factor of 8 to 32 in our measurements. As a side effect, the bit
lengths of two parameters of our inner encryption scheme, namely 
(see Section~\ref{sec:crypto}), bound the time it takes to encrypt in the
browser.  Reasonable values range from 512 bits (probably sufficient) to
2044 bits (paranoid).  We measured these times on an Intel i5-2500K CPU at
3.30~GHz with Chromium Version 20.0.1132.47.  For 512 bits it took
\Kodjlmean{} seconds (, speedup ), for 1024 bits
it took \Kodjmmean{} seconds (, speedup ) and for
2044 bits it took \Kodjhmean{} seconds (, speedup ).  Since Java is still significantly faster than JavaScript we assume
that these times will become smaller still in the future.


\section{Financial Viability}
\label{sec:financial}

Given the server resources AdLeaks requires it is prudent to ask what are
the costs the AdLeaks organization has to bear.  We found that dual
quad-core servers with unmetered 1Gb interfaces are available for less then
\Kunitmonth~\Uunitmonth.  At this price, the infrastructure for the guards
and aggregators necessary to serve \Kusers~\Uusers{} users would cost
\Kdailyusd~\Udailyusd{}.  While this seems high it is instructive to look at
the revenue side.  Each ad loading corresponds to what is called an
``impression'' in the online advertising business.  The prices for
impressions are typically quoted as \emph{costs per mille,} or CPM.  Top
ranking websites can command prices over 100~USD while run-off-the-mill
websites receive in the order of \Kcpm~\Ucpm.  \emph{Cost per click} models
or \emph{cost per conversion} models generate additional revenue, which we
ignore for the sake of conservatism.  If AdLeaks had \Kusers~\Uusers{} users
who see \Klowads{} AdLeaks ads per day on average and if AdLeaks paid out
\Kcpm~\Ucpm{} per mille impressions and if its markup was \Kbreakeven{}
\Ubreakeven{} or better then AdLeaks would break even on its infrastructure
costs.  For comparison, ValueClick Inc.\@ reported a gross profit of over 96
million USD on a revenue of about 161 million USD in its second quarter of
2012, which translates to about 59 percent profitability, and reported 25
cents of net income per common share.  The numbers suggest that, if AdLeaks
was run as a not-for-profit operation, it could gain market share by
offering very competitive pricing while earning a decent plus.  This is in
sharp contrast to contemporary whistleblowing platforms who depend entirely
on donations.


\section{Implementation}
\label{sec:impl}

We developed fully-functional multi-threaded aggregation and decryption
servers with tree decryption support as well as a Fountain Code encoder and
decoder.  Decryptors write recovered data to disk and the decoder recovers
the original file.  We also developed a fake guard server which is capable
of generating and sending chunks according to a configurable ratio of white
and gray ciphertexts.  All servers connect to each other through SSH tunnels
via port forwarding.  The entire implementation consists of 101 C, header
and CMake files with 7493 lines of code overall.  This includes our
optimized DJ implementation~\cite{Jurik2003}, which is based on a library by
Andreas Steffen, a SHA-256 implementation by Olivier Gay, and several
benchmarking tools.  Our ads implement the DJ scheme based on the
\emph{JSBN.js} library and use \emph{Web Workers} to isolate the code from
the rest of the browser.  The entire ad currently measures less than 81~KB.
The size can be reduced further by eliminating unused library code and by
compressing it.  The ad submits ciphertexts via \emph{XmlHttpRequests.}  We
instrumented the Firefox browser for our prototype and patched the source
code in two locations.  First, we hook the compilation of \emph{Web Worker}
scripts and tag every script as an AdLeaks script if it is labeled as one in
lieu of carrying a valid signature.  We placed a second hook where Firefox
implements the \emph{XmlHttpRequest.}  Whenever the calling script is an
AdLeaks script running within a \emph{Web Worker,} we replace the zero chunk
in its request with a data chunk.


\section{Related Work}



Closely related to our work, there is early work on DC Nets
\cite{Chaum1988,WaidnerP1989} which aims to provide a cryptographic means to
hide who sends messages, the use of Raptor codes \cite{CastiglioneDFP2012}
to implement an asynchronous unidirectional one-to-one and one-to-many
covert channel using spam messages, and anonymous data
aggregation~\cite{PuttaswamyBP2010} for distributed sensing and
diagnostics. In preserving the privacy of web-based
email~\cite{ButlerEPTM2006} one can take advantage of a spread-spectrum
approach for hiding the existence of a message, but it is not secure against
a global attacker. Membership-concealment~\cite{VassermanJTHY2009} can also
be used to hide the real-world identities of participants in an overlay
network, but this doesn't suffice for AdLeaks.

In censorship resistance, there is Publius~\cite{WaldmanRC2000} which is an
anonymous publication system but does not offer any sort of connection-based
anonymity. Collage~\cite{BurnettFV2010} steganographically embeds content in
cover traffic such as photo-sharing sites and implements a rendezvous
mechanism to allow parties to publish and retrieve messages in this cover
traffic, but Alice and Bob must exchange a key a priori. More recent
work~\cite{InvernizziKV2012} explores an approach that assumes the ability
of being able to globally check and retrieve all blog posts in real time and
determining and extracting all the embedded content.

Another related area is secure data aggregation in wireless sensor
networks~\cite{Alzaid2011} or WSNs. One can try to securely aggregate
encrypted data~\cite{Castelluccia2009}, which identifies the key stream in
the header and requires removing a stream for each ciphertext received. The
latter wouldn't scale for AdLeaks because it requires millions of keystream
removals per aggregate. One approach~\cite{ViejoWD2012} for aggregating in
multicast communication uses the Okamoto-Uchiyama encryption scheme for
secure aggregation, which resembles Pederson's commitment scheme very
closely. The difference is really that AdLeaks is not used in the aggregate
but instead deals with collisions. Other work~\cite{Wagner2004,ChanPPS2007}
targets the security of statistical computations on the inputs from various
sensor nodes. The two key differences in the approaches found in WSNs are:
(i) WSNs want correctly aggregated data that allows for unencrypted sending,
whereas our approach seeks the opposite of that, and (ii) the attacker wants
to have a tainted input accepted, whereas in our context he wants to learn
the content of the input. In WSNs both event-driven and query-based
processing is of interest, with most approaches focusing on query-based
solutions, whereas AdLeaks is event-driven. That also means that we don't
know \emph{a priori} which client sends what in what round. It is difficult
for us to exchange keys beforehand and our approach remains unidirectional,
i.e. we cannot distribute keys. Lastly, in WSNs clients are not trusted
initially and are later vetted, whereas in our approach clients are never
trusted.


\section{Conclusions}

AdLeaks leverages the ubiquity of online advertising to provide anonymity
and unobservability to whistleblowers making a disclosure online. The system
introduces a large amount of cover traffic in which to hide whistleblower
submissions, and aggregation protocols that enable the system to manage the
huge amount of traffic involved, enabling a small number of trusted nodes
with access to the decryption keys to recover whistleblowers' submissions
with high probability. We analyzed the performance characteristics of our
system extensively. Our research prototype demonstrates the feasibility of
such a system. We expect many aspects of the system can be improved and
optimized, providing ample opportunity for further research.


\subsection*{Acknowledgements}

The first, second and last author are supported by an endowment of
\emph{Bundesdruckerei GmbH.}  An abridged version of this paper has been
accepted for publication in the proceedings of \emph{Financial Cryptography
  and Data Security 2013}~\cite{RothGRDR2013}.  Copies of the abridged
version will eventually become available at
\url{http://www.springer.de/comp/lncs/}.

\flushleft
\begin{thebibliography}{10}

\bibitem{Alexa.com}
{Alexa}.
\newblock Online at \url{http://www.alexa.com}, Apr. 2012.

\bibitem{Alzaid2011}
H.~Alzaid.
\newblock {\em Secure Data Aggregation in Wireless Sensor Networks}.
\newblock PhD thesis, Queensland University of Technology, March 2011.

\bibitem{Banisar2009}
D.~Banisar.
\newblock {\em Whistleblowing --- International Standards and Developments}.
\newblock Transparency International, Feb. 2009.

\bibitem{BertholdBK2009}
S.~Berthold, R.~B\"ohme, and S.~K\"opsell.
\newblock Data retention and anonymity services.
\newblock In {\em The Future of Identity in the Information Society}, volume
  298 of {\em IFIP Advances in Information and Communication Technology}, pages
  92--106. Springer, 2009.

\bibitem{BowenHKS2009}
B.~M. Bowen, S.~Hershkop, A.~D. Keromytis, and S.~J. Stolfo.
\newblock Baiting inside attackers using decoy documents.
\newblock In {\em Proc. SecureComm}, pages 51--70. Springer, 2009.

\bibitem{BurnettFV2010}
S.~Burnett, N.~Feamster, and S.~Vempala.
\newblock Chipping away at censorship firewalls with user-generated content.
\newblock In {\em Proc. USENIX Security}, pages 29--29, 2010.

\bibitem{ButlerEPTM2006}
K.~Butler, W.~Enck, J.~Plasterr, P.~Traynor, and P.~McDaniel.
\newblock Privacy preserving web-based email.
\newblock In {\em Proc. International Conference on Information Systems
  Security}, ICISS, pages 116--131. Springer, 2006.

\bibitem{CanettiKN2003}
R.~Canetti, H.~Krawczyk, and J.~Nielsen.
\newblock Relaxing chosen-ciphertext security.
\newblock In {\em Proc. CRYPTO}, volume 2729 of {\em LNCS}, pages 565--582.
  Springer, 2003.

\bibitem{Castelluccia2009}
C.~Castelluccia, A.~C.-F. Chan, E.~Mykletun, and G.~Tsudik.
\newblock Efficient and provably secure aggregation of encrypted data in
  wireless sensor networks.
\newblock {\em ACM Trans. Sen. Netw.}, 5(3):20:1--20:36, June 2009.

\bibitem{CastiglioneDFP2012}
A.~Castiglione, A.~De~Santis, U.~Fiore, and F.~Palmieri.
\newblock An asynchronous covert channel using spam.
\newblock {\em Comput. Math. Appl.}, 63(2):437--447, Jan. 2012.

\bibitem{ERC2012}
E.~R. Center.
\newblock {2011 National Business Ethics Survey}.
\newblock Online at \url{http://www.ethics.org/nbes}, 2345 Crystal Drive, Suite
  201, Arlington, VA 22202, USA, 2012.

\bibitem{ChakravartySK2010}
S.~Chakravarty, A.~Stavrou, and A.~D. Keromytis.
\newblock Traffic analysis against low-latency anonymity networks using
  available bandwidth estimation.
\newblock In {\em Proc. ESORICS}, pages 249--267. Springer, 2010.

\bibitem{ChanPPS2007}
H.~Chan, A.~Perrig, B.~Przydatek, and D.~Song.
\newblock Sia: Secure information aggregation in sensor networks.
\newblock {\em J. Computer Security}, 15(1):69--102, Jan. 2007.

\bibitem{Chaum1988}
D.~Chaum.
\newblock The dining cryptographers problem: Unconditional sender and recipient
  untraceability.
\newblock {\em Journal of Cryptology}, 1:65--75, 1988.
\newblock 10.1007/BF00206326.

\bibitem{webcryptoapi}
D.~Dahl and R.~Sleevi.
\newblock Web cryptography {API}.
\newblock W3C Draft, October 2012.

\bibitem{DamgardJN2010}
I.~Damg\aa{}rd, M.~Jurik, and J.~Nielsen.
\newblock A generalization of {Paillier's} public-key system with applications
  to electronic voting.
\newblock {\em International Journal of Information Security}, 9:371--385,
  2010.

\bibitem{DingledineMS2004}
R.~Dingledine, N.~Mathewson, and P.~Syverson.
\newblock Tor: the second-generation onion router.
\newblock In {\em Proc. USENIX Security Symposium}, pages 303--320, 2004.

\bibitem{DyerCRS2012}
K.~P. Dyer, S.~E. Coull, T.~Ristenpart, and T.~Shrimpton.
\newblock Peek-a-boo, i still see you: Why efficient traffic analysis
  countermeasures fail.
\newblock In {\em IEEE Symposium on Security and Privacy}, pages 332--346,
  2012.

\bibitem{rfc6101}
A.~Freier, P.~Karlton, and P.~Kocher.
\newblock The secure sockets layer ({SSL}) protocol version 3.0.
\newblock RFC 6101 (Historic), Aug. 2011.

\bibitem{Gustin2010}
S.~Gustin.
\newblock Columbia university reverses anti-{WikiLeaks} guidance.
\newblock
  \url{http://www.wired.com/threatlevel/2010/12/columbia-wikileaks-policy/},
  Dec. 2010.

\bibitem{HoumansadrNCB2011}
A.~Houmansadr, G.~T.~K. Nguyen, M.~Caesar, and N.~Borisov.
\newblock Cirripede: Circumvention infrastructure using router redirection with
  plausible deniability.
\newblock In {\em Proc. ACM CCS}, 2011.

\bibitem{InvernizziKV2012}
L.~Invernizzi, C.~Kruegel, and G.~Vigna.
\newblock {Message In A Bottle: Sailing Past Censorship}.
\newblock In {\em Hot Topics in Privacy Enhancing Technologies}, Vigo, Spain,
  2012.

\bibitem{Jurik2003}
M.~J. Jurik.
\newblock {\em Extensions to the Pailler Cryptosystem with Applications to
  Cryptological Protocols}.
\newblock Dissertation, BRICS, Department for Computer Science, University of
  Aarhus, Aug. 2003.

\bibitem{KarlinEJJLMS2011}
J.~Karlin, D.~Ellard, A.~W. Jackson, C.~E. Jones, G.~Lauer, D.~P. Mankins, and
  W.~T. Strayer.
\newblock Decoy routing: Toward unblockable internet communication.
\newblock In {\em Proc. Workshop on Free and Open Communications on the
  Internet}. USENIX, 2011.

\bibitem{Klein2008}
A.~Klein.
\newblock Temporary user tracking in major browsers and cross-domain
  information leakage and attacks.
\newblock Technical report, Trusteer, 2008.
\newblock Online at \url{http://www.trusteer.com}.

\bibitem{Lennane1993}
K.~J. Lennane.
\newblock ``{Whisteblowing}'': a health issue.
\newblock {\em British Medical Journal}, 307:667--670, Sept. 1993.

\bibitem{MacKay2005}
D.~MacKay.
\newblock Fountain codes.
\newblock {\em {IEE} {Proceedings}}, 152(6), Dec. 2005.

\bibitem{OsterhausF2009}
A.~Osterhaus and C.~Fagan.
\newblock {\em Alternative to Silence --- Whistleblower Protection in 10
  European Countries}.
\newblock Transparency International, 2009.

\bibitem{Pederson1991}
T.~Pedersen.
\newblock Non-interactive and information-theoretic secure verifiable secret
  sharing.
\newblock In {\em Proc. CRYPTO}, volume 576, pages 129--140. Springer, 1992.

\bibitem{PuttaswamyBP2010}
K.~P.~N. Puttaswamy, R.~Bhagwan, and V.~N. Padmanabhan.
\newblock Anonygator: privacy and integrity preserving data aggregation.
\newblock In {\em Proc. ACM/IFIP/USENIX 11th International Conference on
  Middleware}, Middleware, pages 85--106. Springer, 2010.

\bibitem{Google2010}
S.~Ramachandran.
\newblock Web metrics: Size and number of resources.
\newblock Online at
  \url{https://developers.google.com/speed/articles/web-metrics}, May 2010.

\bibitem{RogawayBB2003}
P.~Rogaway, M.~Bellare, and J.~Black.
\newblock {OCB}: A block-cipher mode of operation for efficient authenticated
  encryption.
\newblock {\em ACM Trans. Inf. Syst. Secur.}, 6:365--403, Aug. 2003.

\bibitem{RothGRDR2013}
V.~Roth, B.~G\"uldenring, E.~Rieffel, S.~Dietrich, and L.~Ries.
\newblock A secure submission system for online whistleblowing platforms.
\newblock In {\em Proc. Financial Cryptography and Data Security}, LNCS.
  Springer Verlag, 2013.
\newblock To appear.

\bibitem{StarkHB09}
E.~Stark, M.~Hamburg, and D.~Boneh.
\newblock Symmetric cryptography in javascript.
\newblock In {\em Proc. ACSAC}, pages 373--381. IEEE Computer Society, 2009.

\bibitem{Stats2012a}
{Persons using the {Internet} in and outside the home}.
\newblock
  \url{http://www.ntia.doc.gov/files/ntia/data/CPS2010Tables/t11_1.txt}, Jan.
  2011.

\bibitem{VassermanJTHY2009}
E.~Vasserman, R.~Jansen, J.~Tyra, N.~Hopper, and Y.~Kim.
\newblock Membership-concealing overlay networks.
\newblock In {\em Proc. ACM CCS}, pages 390--399, 2009.

\bibitem{ViejoWD2012}
A.~Viejo, Q.~Wu, and J.~Domingo-Ferrer.
\newblock Asymmetric homomorphisms for secure aggregation in heterogeneous
  scenarios.
\newblock {\em Inf. Fusion}, 13(4):285--295, Oct. 2012.

\bibitem{Wagner2004}
D.~Wagner.
\newblock Resilient aggregation in sensor networks.
\newblock In {\em Proc. ACM Workshop on Security of Ad Hoc and Sensor
  Networks}, SASN, pages 78--87. ACM, 2004.

\bibitem{WaidnerP1989}
M.~Waidner and B.~Pfitzmann.
\newblock The dining cryptographers in the disco: Unconditional sender and
  recipient untraceability with computationally secure serviceability.
\newblock In {\em EUROCRYPT}, volume 434 of {\em LNCS}, pages 690--690.
  Springer, 1990.

\bibitem{WaldmanRC2000}
M.~Waldman, A.~Rubin, and L.~Cranor.
\newblock Publius: {A} robust, tamper-evident, censorship-resistant and
  source-anonymous web publishing system.
\newblock In {\em Proc. USENIX Security Symposium}, pages 59--72, Aug. 2000.

\bibitem{WhiteLSRGNHBJ2002}
B.~White, J.~Lepreau, L.~Stoller, R.~Ricci, S.~Guruprasad, M.~Newbold,
  M.~Hibler, C.~Barb, and A.~Joglekar.
\newblock An integrated experimental environment for distributed systems and
  networks.
\newblock In {\em Proc. OSDI}, pages 255--270, Dec. 2002.

\bibitem{WustrowWGH2011}
E.~Wustrow, S.~Wolchok, I.~Goldberg, and J.~A. Halderman.
\newblock Telex: Anticensorship in the network infrastructure.
\newblock In {\em Proc. {USENIX} Security Symposium}, Aug. 2011.

\bibitem{ZhangZ2011}
H.~Zhang and S.~Zhao.
\newblock Measuring web page revisitation in tabbed browsing.
\newblock In {\em Proc. CHI}, pages 1831--1834, 2011.

\end{thebibliography}


\end{document}
