\documentclass[runningheads,envcountsame,envcountsect,oribibl]{llncs}






\usepackage[latin1]{inputenc}
\usepackage{epsfig}
\usepackage{psfrag}
\usepackage{amsmath}
\usepackage{amssymb}
\usepackage[english]{babel}





\makeatletter
\long\def\@doifundefined#1#2{\def\reserved@jhh{\relax}\expandafter\ifx\csname#1\endcsname\relax
    \def\reserved@jhh{#2}\fi
 \reserved@jhh}
\newcommand{\provideenvironment}[2]
  {\@doifundefined{#1}{\newenvironment{#1}#2}}
\newcommand{\providetheorem}[2]
  {\@doifundefined{#1}{\newtheorem{#1}#2}}
\makeatother

\newcommand{\bibselect}[2]{#1}

\def\rcs{#1}
\newcommand{\version}[1]{\thanks{\rcs#1}}

\newcommand{\call}{a}
\newcommand{\term}[1]{\emph{#1}}		\newcommand{\etal}{\textit{et al.}~}	\newcommand{\ie}{{i.e.},\ }		\newcommand{\eg}{{e.g.},\ }		\newcommand{\Prob}[1]{\mathord{\textbf{Pr}}\left[#1\right]} \newcommand{\CProb}[2]{\Prob{\left.#1\right| #2}}
\newcommand{\assign}{\mathrel{:=}}	\newcommand{\keyw}[1]{\ensuremath{\mathbf{#1}}} \newcommand{\xor}{\oplus}		\newcommand{\Z}{\mathbb{Z}}		\newcommand{\card}[1]{\lvert{#1}\rvert}		

\newcommand{\rem}[1]{\mbox{}\marginpar{\raggedleft\hspace{0pt}\footnotesize{#1}}\ignorespaces}

\usepackage{float}
\floatstyle{plain}
\newfloat{protocol}{t}{lop}[section]
\floatname{protocol}{Protocol}

\newcommand{\IF}{\keyw{if}~}
\newcommand{\THEN}{\keyw{then}~}
\newcommand{\ELSE}{\keyw{else}~}
\newcommand{\FOR}{\keyw{for}~}
\newcommand{\DO}{\keyw{do}~}

\makeatletter
\newcommand{\@Setstar}[1]{\left\{{#1}\right\}}
\newcommand{\@Set}[2]{\@Setstar{{#1},\ldots,{#2}}}
\newcommand{\Set}{\@ifstar{\@Setstar}{\@Set}}	
\makeatother




\renewcommand{\topfraction}{.95}
\renewcommand{\textfraction}{.05}

\newcommand{\eke}{KE}
\newcommand{\concat}{\parallel}

\makeatletter
\newcommand{\xRightarrow}[2][]{\ext@arrow 0359\Rightarrowfill@{#1}{#2}}
\newcommand{\xLeftarrow}[2][]{\ext@arrow 3095\Leftarrowfill@{#1}{#2}}
\makeatother



\newcommand{\sendright}[1]{}
\newcommand{\sendleft}[1]{}
\newcommand{\broadcastright}[1]{}
\newcommand{\broadcastleft}[1]{}





\newcommand{\phase}[1]{
\multicolumn{3}{l}{\raisebox{-2mm}{\fbox{\textit{#1}}}\hrulefill} \
  \adv_{\adverse}^P = 2 \, \Prob{S_{\adverse}^P} - 1~,

   \CProb{h_2(X) = a}{h_1(X) = b} = \Prob{h_2(X)=a} = 2^{-\cap}~.

O(1-e^{-q_{\text{send}}/2^{\sps}}) + O(2^{-\spl})~.
 
1-(1-2^{-\cap})^{q_{\text{send}}} \approx 1-e^{-2^{-\cap}q_{\text{send}}}
 
   O(q_{\text{send}} / 2^\sps) + O(2^{-\spl})~,

where, loosely speaking,  is the number of times the adversary
tries to guess the password. The difference is caused by the fact that in the
EKE setting, multiple instances of the protocol use the same 
password\footnote{
  This could be overcome by allowing only the first  connections to use the
  password alone, and using parts of the previously established shared secrets
  to generate new, longer, passwords. Then the bound on the advantage of the
  adversary essentially becomes equal to ours.
}.



\section{Applications}
\label{sec-appl}

Beyond those mentioned in the introduction, there are many other situations
that involve ephemeral pairing. 
\begin{itemize}
\item Connecting two  laptops over an infrared connection, while in a business
  meeting.  
\item Buying tickets wirelessly at a box office, or verifying them at the
  entrance. 
\item Unlocking doors using a wireless token, making sure the right door is
  unlocked. 
\end{itemize}
For all these applications it is very important that the burden of correctly
establishing the right pairing should not solely rest on the user. The user may
make mistakes, and frequent wrong pairings will decrease the trust in the
system. This is especially important for applications that involve financial
transactions. On the other hand, some user intervention will obviously always
be required. The trick is to make the user actions easy and intuitive given the
context of the pairing.

In the next section, we describe how a low bandwidth authentic or
private channel can be implemented in quite practical settings. These are of
course merely suggestions. There are probably many more and much better ways to
achieve the same effect. The point here is, however, to merely show that such
channels can be built in principle.


\subsection{Implementing the low bandwidth authentic or private channel}

To implement an authentic or private channel in practice,
several solution stra\-te\-gies are applicable. 
\begin{itemize}
\item Establishing physical contact, either by a wire, through a connector, or
  using proximity techniques.
\item Using physical properties of the wireless communication link, that may
  allow `aiming' your device to the one you wish to connect to.
\item Using fixed visible identities, either using explicitly shown unique
  names on devices, or using the unique appearance of each device.
\item Using small displays that can either be read by the operator of the other
  device or read directly by the other device.
\end{itemize}
Which strategy to select depends very much on the specific application
requiring ephemeral pairing.
We will discuss each of these strategies briefly.

\subsubsection{Physical contact}

The easiest way to solve ephemeral pairing is to connect both nodes
(temporarily) physically, either by a wire, or by making them touch each others
conductive pad. The resulting physical connection can be used as the private or
authentic channel in the previous protocol. Or it can be used to
exchange the shared secret directly, of course

\subsubsection{Fixed visible identities}

Here one could use for example numbers, or the physical appearance. Each node
holds a unique private key, and the physical identity is bound to the
corresponding public key using a certificate generated by the certification
authority (CA) managing the application.

The main drawback is that these solutions require a central Certification
Authority. Moreover, the a priori distribution of secrets is contrary to the
spirit of the \eke{} problem.

A variant (described in~\cite{MeT01}) uses the fixed identity of nodes in the
following way. Any node wishing to connect can do so. Each connection is
assigned a unique and small connection number, which is shown on a display. 
The user mentions the number to the merchant, who then initiates a payment
over the indicated connection. The problem with this setup is that it is
vulnerable to man-in-the-middle attacks.


\subsubsection{Physical link properties}

Depending on the properties of the physical link, one could reliable aim a
device at another, or safely rule out connections to/from other far away
devices. 

\subsubsection{Operator read displays}

In this scheme, each node has a small display and a way to select several
images or strings from the display (through function keys or using a
touch pad). An authentic channel can be implemented as follows. To send a 
bit string, it is converted to a simple pattern that is shown on the display
of the sender. The receiver enters the pattern on its device, which converts it
back to the  bits.


\section{Conclusions}
\label{sec-concl}

We have formulated the ephemeral pairing problem, and have presented several
ephemeral key exchange protocols showing that this problem can be solved using
small capacity, and mostly bidirectional, point to point channels and a
broadcast network with identities to separate communication streams.

More work needs to be done to develop \eke{} protocols using only
unidirectional channels, or on truly anonymous broadcast networks.

It would be interesting to develop protocols that are correct under less 
strong assumptions, \ie ones that do not require to assume either the 
random oracle model or hardness of the Decisional Diffie Helman problem (or
both). The same holds for the assumption on the authentic channel that
adversary cannot modify or inject messages of his choice at all. More research
is needed to investigate the effects on the advantage of the adversary
if he can modify or inject messages on the authentic channel with 
low success probability. 

\section{Acknowledgements}

This work was inspired by the work of Yan Yijun on a payments architecture for
mobile systems, under the supervision of Leonard Franken of the ABN AMRO bank
and myself. I would like to thank Yan and Leonard for fruitful initial
discussions on this topic.
 

\bibliography{pairing}


\end{document}
