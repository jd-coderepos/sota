\pdfoutput=1



\documentclass{sig-alternate}

\usepackage[utf8]{inputenc}
\usepackage[T1]{fontenc}

\usepackage{multirow}
\usepackage{graphicx} 
\DeclareGraphicsExtensions{.pdf,.eps,.png,.jpg,.mps}
\usepackage{adjustbox}
\usepackage{listings}


\lstset{numbers=left, numberstyle=\small, numbersep=8pt, frame=single, framexleftmargin=15pt, framextopmargin=0pt, columns=fullflexible, basicstyle=\ttfamily, breaklines=true, tabsize=2, captionpos=b}

\begin{document}
\conferenceinfo{COP '15}{July 05 2015, Prague, Czech Republic \\
Copyright is held by the owner/author(s). Publication rights licensed to ACM.}
\copyrightetc{ACM \the\acmcopyr}
\CopyrightYear{2015} \crdata{978-1-4503-3654-3/15/07 \ ...\contextcontext$__" + ((ASTFunction) child).getContexts().get(0);
  }
  ...
}
\end{lstlisting}

\subsubsection{From function call to the decision system}

In order to abstract and decouple the decision-maker from the rest of the code, we opted for an event-oriented system, as in Figure \ref{figure:congoloinvoke}. But to be able to use we first needed to add the support for contextual function invocation to Golo so that we could trick it as we wanted afterwards. We implemented this in the compilation process of Golo, whereas the decision-maker is designed as a Java API.

\subsubsection{Custom boostrap method}
First to be able to call the contextual function by their original given name, we changed the function invocation of contextual functions. Golo is based on the invokedynamic opcode to bind at the runtime the callsite to the correct target, as defined by the language. For this process it provides several utility classes in the \lstinline|fr.insalyon.citi.golo.runtime| package that acts accordingly to their name; FunctionCallSupport, MethodInvocationSupport, etc. We thus extended this package with the new class \lstinline|ContextualFunction CallSupport|. The Golo compiler replaces function calls, in the most generic terms of that, by invokedynamic instruction that are bound to specific bootstrap utility class according to their type (method, function, closure). In order to be able to use our newly designed bootstrap class we thus need to identify which of the function calls correspond to contextual function calls. We make this identification in the \lstinline|LocalReferenceAssignmentAndVerificationVisitor| class. We already know which functions are contextual so we build a list containing those so that we can latter check whether the name of the invoked function matches the name of a contextual function and marking it as such. Then in the bytecode generation process we simply use our custom bootstrap class for those function calls.

The contextual bootstrap method then wraps everything into a message that will be sent to the decision maker. Since the call needs to be synchronized (the bootstrap method should return a method handle that will later be used at the original callsite, we then block the current thread waiting for the answer from the decision-maker. The bootstrap class actually reacts on a specific type of message that triggers a function that will unlock the thread and finally configure the method handle according to the decision made by the decision-maker. Once the bootstrap method is made the call site is bound to the correct target as for the decision-maker.

\begin{center}
\begin{figure}
\centering
\includegraphics[width=0.9\columnwidth, bb=0 0 658 408]{images/contextual_invocation}
\caption{Congolo invocation process}
\label{figure:congoloinvoke}
\end{figure}
\end{center}

\subsubsection{Event messaging}
The messaging system used is a simple hierarchical- namespace messaging system developped especially for Golo \cite{flemouel_gololang-messaging_2013}. These events are used to make the link between the function call and the decision-maker. The implementation is multi-threaded and the hierarchical design allows for interesting behaviors for a COP languages. It will be indeed easy to reach multiple decision-makers, or implement various scopes for the contexts.

The framework either uses a callback mechanism where you can register a specific handle to be invoked upon the reception of the message, or an interface where you can register an instance of a class implementing the \lstinline|gololang.concurrent.messaging.MessagingFunction|.

\subsubsection{The decision maker}
The decision-making part is designed as a Java API that defines a common \lstinline|DecisionMaker| interface. We also provided a \lstinline|DefaultDecisionMaker| class that implements some basic needed functionality allowing to quickly test our solution. The interface requires the implementation functions such as \lstinline|train| or \lstinline|init| but which are not used in our implementation because of its simplicity. They would be used in more advance systems such as a neural network. 

\subsubsection{Context management}
In order to be able to retrieve the adequate context regarding the function call, we register context objects with the help of a context manager that keeps track of which contexts were declared in which module as it is for now the only level we support for context declaration. When the decision-maker wants to retrieve the context information it queries the context manager with the module name and gets the list of the context objects declared for this module.

 
\section{Preliminary results}
\label{results}

The evaluation of programming languages can be difficult. Benchmarks are a common way of doing this, but there usually are pitfalls because of the numerous options of compilation or runtime, the type of hardware, the versions of software and so on. In COP languages, a common benchmark is to compare calls to layered functions versus calls to standard functions and see the overhead such as described in \cite{appeltauer_comparison_2009} where multiple languages were tested to create a benchmark. To do so they compare 10 methods layered to plain implementation. 

\subsection{Micro-benchmark performance}
Benchmarking is not an easy task especially for runtime-evaluated language. In addition they often do no reflect the future use or condition of the tested language. However it is still important to have an idea of how it performs relatively to others. In this perspective we designed a simple micro-benchmark to test specifically the predominant contribution of this paper which is the separation of the desision-making process with the rest of the code.

For that purpose we used the Java JMH tool developped by OpenJDK. This framework provides a harness for writing test to avoid common pitfalls \cite{jponge_benchmarking_2014}. It is organized as a Maven project where thanks to the JMH each test is run in a separate VM and is first run for warmups and afterwards for the measurement, so that the measures are consistent. The test were done on Windows 7 64 bits SP1 laptop Core 2 Duo @ 2.53GHz with 4 GB of RAM, with the build 1.8.0\_05-b13 of the Java Runtime Environment and the build 25.5-b02 of Hotspot 64 bits.

First we tested a call to a standard Golo module-level defined function versus a contextual one. Afterwards we tested the imbrication of 10 layered methods as proposed before.

The results shown in Table \ref{table:benchmarking} show a major loss in performance from Golo to Congolo. We did not indicate the exact error (less than 10 percents in each case) because the results were significant enough. This can be explained quite simply by the implementation of the solution. 

\begin{table}
\begin{adjustbox}{center}
\begin{tabular}{l p{2cm} | c | c | c | }
        \hline
                Test & Language & Score (nops/ms) \\
                
        \hline
        \multirow{2}{*}{Single invocation} 
                & Congolo & 29 \\
                & Golo & 6135 \\

        \hline
    \multirow{2}{*}{Layered invocation (10)} 
                & Congolo & 1.9 \\
                & Golo & 3998\\                 
\end{tabular}
\end{adjustbox}
        \caption{Benchmark results}
        \label{table:benchmarking}
\end{table}

First the invokedynamic instruction can be bound to different types of callsite: static, mutable and volatile. In the static case, once the bootstrap method is made there is no possible change of target thus the method can be directly called and the the link is thus optimized at runtime. In the other cases, and especially for mutable callsites (volatile is not use in Golo nor Congolo), in the case we want to check if we should change it we can guard the callsite with a test. This test will be checked before invoking the target and if it fails the bootstrapping will occur again. For instance, in the case of closures in Golo, each callsite has a cache which allows to check whether the target has changed or not and thus improving the overall performances. In our case, the test would need to check whether the context has changed or not since last time, test that is not done at this time and instead return false so that the bootstrap occurs all over again. Thus for each call, the system need to make the decision based on the context which explain in part the results.

Another point comes from the method use to communicate between the bootstrap method and the decision-maker. We opted for events so allow for more separation and because of its flexibility. However this was not done considering performances. The messaging environment is a single-thread (it does support multiple threads for a general purpose use, since it is basically and extension of the Java \lstinline|Executor|, however in our case we do not support multiple thread and concurrency in the design of Congolo, and in heavy stress, as it is the case in the benchmarking, it can lead to errors (deadlocks), which can be a bottle-neck. We check the difference between the messaging environment and direct calls to the functions. The results are shown in Table \ref{table:benchmarking_direct}. We can see that if the difference for Golo with previous results are in the range of the score error, it is not the case for Congolo where the results are significantly different and show an improvement with the previous implementation.

On overall Congolo offers quite poor performances compared to Golo. However it is to be noted that achieving high performance was not the main axis of development, and achieving from 2 or 3 function calls - for highly layered calls - to 30 function calls per millisecond appears to be enough for Internet of Things environments - as context changes are usually linked to quite stable physical phenomena \cite{RELEASeD-2010-857551}.

\begin{table}
\begin{adjustbox}{center}
\begin{tabular}{l p{2cm} | c | c | c |}
        \hline
                Test & Language & Score (nops/ms) \\
                
        \hline
        \multirow{2}{*}{Single invocation} 
                & Congolo & 55 \\
                & Golo & 5939 \\

        \hline
    \multirow{2}{*}{Layered invocation (10)} 
                & Congolo & 2.7 \\
                & Golo & 3840 \\                
\end{tabular}
\end{adjustbox}
        \caption{Benchmark results with direct calls to the Decision-maker}
        \label{table:benchmarking_direct}
\end{table}
 
\section{Conclusion and Future Works}

We presented Congolo \cite{bmaingret_congolo_2014}, a context-oriented language based on the Golo language, a dynamic JVM-based language. The main aspect and originality was to separate the reasoning part from the application code which was achieved by the use of an event-based messaging system and the invokedynamic instruction. Even if achieving 2 to 30 function calls per milliseconds is enough for Internet of Things environments, major performance improvements are required. The event-based messaging performances are encouraging since only slowing a function call from few milliseconds and can still be improved by multi-thread dispatching. The decision-maker however is slowing 100 times a function call, and branching decision and caching are part of future works. Achieving high-level context composition is also part of future work \cite{vallejos2010}, and will probably require specific optimizations regarding the performances. 
\bibliographystyle{abbrv}
\bibliography{cop-ecoop-congolo15}

\end{document}
