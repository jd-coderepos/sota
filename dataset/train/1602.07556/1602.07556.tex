\documentclass[a4paper,USenglish]{lipics}

\usepackage{epic,gastex}
\usepackage{array}
\usepackage{cite}
\usepackage{enumerate}

\usepackage{tikz}
\usetikzlibrary{arrows,automata, positioning}
\definecolor{light-gray}{gray}{0.8}

\DeclareSymbolFont{rsfscript}{OMS}{rsfs}{m}{n}
\DeclareSymbolFontAlphabet{\mathrsfs}{rsfscript}

\DeclareMathOperator{\dir}{dir}
\DeclareMathOperator{\rt}{rt}
\DeclareMathOperator{\car}{car}

\theoremstyle{definition}
\newtheorem{problem}[theorem]{Problem}

\newcommand{\NZRC}[0]{}
\newcommand{\NZRCS}[0]{}


 
\usepackage{microtype}

\bibliographystyle{plain}
\title{Primitive sets of nonnegative matrices and synchronizing automata}

\author[1]{Bal\'azs Gerencs\'er}
\author[2,3]{Vladimir V. Gusev}
\author[2]{Rapha\"el M. Jungers}
\affil[1]{Alfréd Rényi Institute of Mathematics, Hungarian Academy of Sciences\\
  Budapest, Hungary\\
  \texttt{gerencser.balazs@renyi.mta.hu}}
\affil[2]{ICTEAM Institute, Universit{\'e} catholique de Louvain\\
  Louvain-la-Neuve, Belgium\\
  \texttt{\{vladimir.gusev, raphael.jungers\}@uclouvain.be}
  }
\affil[3]{Institute of Mathematics and Computer Science, Ural Federal University\\
  Ekaterinburg, Russia\\
  }

\authorrunning{B.\,Gerencs\'er, V.\,V.\,Gusev, R.\,Jungers} 
\Copyright{Bal\'azs Gerencs\'er, Vladimir V.\,Gusev, Raphael Jungers}
\subjclass{F.1.1 Models of Computation, G.2.1 Combinatorics}
\keywords{Nonnegative matrices, primitive sets of matrices, the exponent of a matrix set, carefully synchronizing automata, the \v{C}ern\'{y} conjecture}

\serieslogo{}\volumeinfo  {Billy Editor and Bill Editors}  {2}  {Conference title on which this volume is based on}  {1}  {1}  {1}\EventShortName{}
\DOI{10.4230/LIPIcs.xxx.yyy.p}
\begin{document}

\maketitle

\begin{abstract}

A set of nonnegative matrices  is called \emph{primitive} if there exist indices  such that  is positive (i.e. has all its entries ). The length of the shortest such product is called the \emph{exponent} of .
The concept of primitive sets of matrices comes up in a number of problems within control theory, non-homogeneous Markov chains, automata theory etc. Recently, connections between synchronizing automata and primitive sets of matrices were established. In the present paper, we significantly strengthen these links by providing equivalence results, both in terms of combinatorial characterization, and computational aspects.

We study the maximal exponent among all primitive sets of  matrices, which we denote by . 
We prove that , 
and moreover, we establish that this bound leads to a resolution of the \v{C}ern\'{y} problem for carefully synchronizing automata.
We also study the set of matrices with no zero rows and columns, denoted by , due to its intriguing connections to the \v{C}ern\'{y} conjecture and the recent generalization of Perron-Frobenius theory for this class.
We characterize computational complexity of different problems related to the exponent of  matrix sets,
and present a quadratic bound on the exponents of sets belonging to a special subclass. Namely, we show that the exponent of a set of matrices having total support is bounded by .
\end{abstract}

\section{Introduction}
A nonnegative matrix  of size  is called \emph{primitive} if  is positive (i.e. has all its entries larger than zero) for a positive integer . This notion was introduced by Frobenius in 1912 during the development of so-called \emph{Perron-Frobenius theory}. This theory has found numerous applications since then: in the theory of Markov chains, economics, population modelling, centrality measures in networks, see~\cite[Chapter 8]{Meyer2000} for an introduction to the topic.
Motivated by various applications Protasov and Voynov introduced the following generalization of this notion to sets of matrices~\cite{PrVo2012}: 
a finite set of (entrywise) nonnegative matrices  is called \emph{primitive} if  is (entrywise) positive for some indices . The length of the shortest such product is called the exponent  of . We will denote the value of the largest exponent among all sets of  matrices by . For example, the matrix set  in Fig~\ref{fig:primitive} is primitive, since the product  is entrywise positive, and its exponent is equal to . Since the actual values of positive entries of matrices in  do not influence the exponent, in the rest of our paper we will implicitly assume that entries of all matrices are equal to 0 or 1. Moreover, we assume that the product  of two matrices of size  is also a -matrix that is defined as follows\footnote{Formally speaking, we consider the matrices over the Boolean semiring.}:  if , and  otherwise.

Primitive sets of matrices received a lot of attention for different reasons. We refer the reader to the introduction in~\cite{BJO15} for the account of applications of primitive matrix sets to stochastic control theory and to the consensus problem. The connections to contractive matrix families and scrambling matrices are given in detail in~\cite[Section 5]{PrVo2012}. Primitive sets of matrices further arise in the study of time-inhomogeneous Markov chains~\cite{Hart02}, and are of importance in mathematical ecology~\cite{Log10}.
Furthermore, primitive sets of matrices are tightly related to boolean networks, which are widely used in biology to model gene regulatory networks. A special class of boolean networks -- disjunctive networks, can be seen as a set matrices over the Boolean semiring.
While researchers in theoretical biology are mostly interested in the attractors and the limit cycles for different types of update schedules, see for example~\cite{GoNo12}, we are mainly interested whether it is possible and how fast one can achieve the all-one state. 
The subfamily of nonnegative matrices that have no zero rows and columns, denoted by , will be of major interest to us for the following reasons. A matrix  is called \emph{irreducible} if for every  there exists a positive integer  such that . A set of  matrices  is \emph{irreducible} if the matrix  is irreducible. As usual, we will denote by  the th vector of the \emph{canonical basis} in , the th entry of  is one, all the others are zeros. We say that a matrix  acts as a permutation on a partition  of the vectors of the canonical basis if there exists a permutation  such that for all ,  belongs to the subspace spanned by . A classical theorem of Perron-Frobenius theory states that an irreducible matrix  is primitive if and only if there is no partition  of the canonical basis vectors for  such that  acts as a permutation on . Protasov and Voynov generalized this theorem to sets of matrices belonging to ~\cite{PrVo2012}: an irreducible set of matrices belonging to  is primitive if and only if there is no partition  for  such that every  acts as a permutation on . Thus, the class of primitive matrices belonging to  can be viewed as the right class for Perron-Frobenius-type theory of matrix sets. This characterization also leads to an efficient algorithm that decides whether a set of matrices belonging to  is primitive.

\subsection{Synchronizing automata}
A deterministic finite state automaton  is a triple\footnote{The classical definition also involves an initial and a set of final states. Since they don't play any role in our considerations, we will omit them.} , where  is a finite set of states,  is a finite set of input symbols called \emph{the alphabet}, and  is a transition function . The image of a state  under the action of a word  is denoted by . An automaton  is called \emph{synchronizing} if there exist a word  and a state  such that for every state  we have . Any such word is called a \emph{synchronizing} or \emph{reset} word. The length of the shortest such word is called \emph{the reset threshold}  of . Synchronizing automata naturally appear in different areas of research. For example, they were used to model sensorless parts orienting problems: given a part and a set of available actions that can change its spatial orientation, find a sequence of actions that would bring the part to a desired orientation independently of the initial position~\cite{Na1986}. Clearly, if we consider an automaton  with the set of spatial orientations as the set of states, and the available actions as letters, then the ``orienting sequence'' corresponds to a synchronizing word of . We refer the reader to~\cite{Volkov2008Survey} for the survey of main results and other applications. A recent account of applications of synchronizing automata in group theory can be found in~\cite{ArCaSt15}. Persisting interest of the research community to the topic is also driven by one of the most famous open problems in automata theory. Namely, \emph{the \v{C}ern\'{y} conjecture} states that the reset threshold of an -state automaton is at most ~\cite{Cerny1964,CernyPirickaRosenauerova1971}. This bound is reached by the -state \v{C}ern\'{y} automaton , see~\cite[p. 18]{Volkov2008Survey}, but despite intensive efforts of researchers, the best upper bound  was obtained more than 30 years ago in~\cite{Pin1983OnTwoCombinatorialProblems, Fr1982} and independently in~\cite{KRS1987}.

The notion of a synchronizing automaton can be generalized in three different ways to nondeterministic automata~\cite{ImSt99}. We will focus our attention on the most relevant for us. An automaton  is a \emph{partial} automaton if the transition function  is partial, i.e. there might be undefined transitions for some pairs of states and letters. A partial automaton is \emph{carefully synchronizing} if there exist a word  and a state  such that  is defined and equal to  for every state . Any such word is called a \emph{carefully synchronizing} word. The length of the shortest such word we will denote by .
We will denote by  the maximum of  among all -state partial automata. Essentially, carefully synchronizing automata model the problem of bringing a simple finite-state device to a known state with a single input sequence, while avoiding undefined transitions, which are undesirable or can break the device. In matrix terms, it amounts to consider a set of matrices with \emph{at most} one 1-entry per row, and to ask for a product with one (entrywise) positive column.


\subsection{Our contributions}
Our results can be informally arranged into three different groups. The contributions of the first group significantly improve the understanding of the relationships between primitive sets of matrices and synchronizing automata. The work within this framework started in~\cite{AGV2013}, where well-known examples of primitive matrices with large exponent were used to construct series of automata with relatively large reset thresholds, so-called ``slowly synchronizing automata''. In~\cite{BJO15} it was shown that a  bound on the reset threshold of -state automata implies a  bound on the exponent of  matrix sets. We significantly improve these results. We show that the growth rate of  is equal to . Thus, in a certain sense, the study of the exponents of sets of matrices is equivalent to the study of carefully synchronizing automata. We also formulate an analogous result for primitive  matrix sets. Namely, we introduce a special class of automata  such that the growth rate of the reset thresholds of automata in this class is equivalent to the growth rate of the exponents of  matrix sets. We propose and formalize a new open question of whether a quadratic bound on  leads to a breakthrough on the \v{C}ern\'{y} conjecture.

The contributions of the second group are of combinatorial nature. Our main result states that , and equivalently, . From the automata theory point of view our contribution can be seen as the resolution of the \v{C}ern\'{y}-like problem for the carefully synchronizing automata. From the point view of matrix theory, our result is a generalization of the classical theorem by Wielandt that the exponent of a single matrix is at most , see for example~\cite[Corollary 8.5.9]{HoJo}. It also answers the question of establishing the growth rate of  posed in~\cite{BJO15}. Another contribution in this group is a partial result for  matrix sets. Recall that a matrix  has \emph{total support} if every non-zero element  of  lies on a positive diagonal, i.e. for every  such that  there exists a permutation  with the following properties:  and for every  we have . We prove that the exponent of a set of matrices having total support is bounded by . In the proof we  utilize the well-known theorem by Kari that the reset threshold of an Eulerian automaton is bounded by . This result suggests that the bounds for other classes of synchronizing automata might be used to obtain upper bounds on the exponent in the special classes of  matrix sets.

The contributions of the last group are related to the computational complexity of finding the exponent of an  matrix set. Given a set of two matrices belonging to  and possibly an integer  encoded in binary, we establish the exact computational complexity 	of the following problems:
\begin{enumerate}
\item the problem of deciding whether  is -complete;
\item the problem of deciding whether  is -complete;
\item the problem of computing  is -complete.
\end{enumerate} 
Furthermore, we show that unless , for every positive  there is no polynomial-time algorithm that computes the exponent of an  matrix set with the approximation ratio , even in the case of only three matrices in the set. These results are based on a single relatively simple reduction from automata with a sink state to sets of matrices belonging to .

The paper is organized as follows. Section 2 deals with the primitive sets of matrices in the general case. We show that  and prove that . Section 3 is devoted to the  matrix sets. In subsection 3.1 we introduce the class  such that . We also present a quadratic bound on the exponent of a set of matrices having total support. In subsection 3.2 we deal with the complexity issues related to the computation of the exponent of  matrix sets.

\section{The general case}

Recall that we denote the value of the largest exponent among all  matrices by . The growth rate of  is one of the most basic questions one can ask about the sets of primitive matrices. Furthermore, an upper bound on  gives a bound on the running time of the straightforward algorithm that decides whether a given set of matrices is primitive: we iterate through all the possible products of length up to  and check, whether  they contain a positive matrix. Since the problem is -hard~\cite[Theorem 6]{BJO15}, such a simple algorithm might be the best we can hope for. The best known bounds on  were presented in~\cite[Theorem 10]{BJO15}:
\begin{theorem}
\label{th:expOld}
If  consists of  matrices of size  then . Moreover, if , then for all  there exists a sequence of positive integers  tending to infinity such that .
\end{theorem}

Recall that we denote the maximum of  among all -state automata  by .
In the upcoming theorem we are going to show that  grows asymptotically as . Thus, we can utilize the known bounds on  to infer the bounds on . Furthermore, in the next subsection we will be able to significantly improve the known upper bound on , and equivalently, on . Before stating the theorem we require one last definition. Given a (partial or complete) automaton , an \emph{adjacency matrix}  of a letter  is defined as follows:  if , and  otherwise. In Fig.~\ref{fig:primitive} the matrix  is an adjacency matrix of the letter .

\begin{figure}
\begin{center}
\scalebox{0.8}{

}
\scalebox{0.9}{
\begin{tikzpicture}[->,>=stealth',shorten >=1pt,auto,node distance=2.5cm,
                    semithick, baseline={([yshift=-1ex]current bounding box.center)}]
  \tikzstyle{every state}=[fill=light-gray,draw=none,text=black, scale=1,minimum size=0.5cm]

  \node[state] 		   (A) {};
  \node[state]         (B) [below right of=A] {};
  \node[state]         (C) [below left of=A] {};

  \path (A) edge [bend left=10] node {} (B)
			edge [loop above]  	node {} (A)
        (B) edge [bend left=10]	node {} (A)
            edge node {} (C)
            edge [loop right]  	node {} (B)
        (C) edge  node {} (A);
\end{tikzpicture}}
\end{center}
\caption{The matrix set  and the corresponding non-deterministic automaton .}
\label{fig:primitive}
\end{figure}

\begin{theorem}
\label{th:expcar}
Let  be the maximum value of the exponent among all sets of  matrices. Let  be the maximum value of  among all -state partial automata , then .
\end{theorem}
\begin{proof}
The proof of the first part of the theorem is inspired by~\cite[Theorem 16]{BJO15}. 
While, in~\cite{BJO15}, the result was restricted to  matrices, we extend it here to all primitive sets of matrices.  
Furthermore, we make it deterministic, which will be crucial for Theorem~\ref{th:nzrcclass}.
Let us consider an arbitrary primitive set of matrices . We are going to show now that , which implies . 
We will achieve this by presenting products  of matrices in  with the following properties: 
\begin{enumerate}
\item the th column of  is positive for some  and the length of  is at most ;
\item the th row of  is positive for some  and the length of  is at most ;
\item  and the length of  is at most .
\end{enumerate}
These properties clearly imply that  is positive and the length of  is at most . Thus, .


We will construct the product  by utilizing a partial automaton  defined as follows: a partial function  is a letter of  if and only if there is a matrix  with the properties: for all , if  is defined, , otherwise the th row has no positive entries. First, we are going to show that  is carefully synchronizing, then we will use the shortest carefully synchronizing word of  to obtain the matrix product .

We construct a carefully synchronizing word of  with the help of an auxiliary non-deterministic automaton  defined in the following manner:  the set of states of  is equal to ; for each matrix  we add a letter  such that  is the adjacency matrix of , see Fig.~\ref{fig:primitive}. It is straightforward to verify that for every  we have  if and only if there is a path from  to  in  labelled by the word . Since  is primitive, there exists a positive product of matrices in . Therefore, there exists a word  such that for every pair of states of  there is a path between them labelled by . 

It remains to show that the word  can be transformed to a carefully synchronizing word of . Let us fix a state . There are paths  in  labelled by  and for every  the path  goes from the state  to the state . Furthermore, we can impose an additional property on these paths. Namely, if at a step  paths  and  are in the same state, then their continuations coincide. Indeed, let  and . Then we can substitute the path  with the path , which still goes from  to  and it is labelled by .
Observe now, that the paths  can be easily treated as paths leading to the state  in the partial automaton : by construction for each letter  of  and states  with a property  for , there exists a letter  of  such that  for each ; due to this fact and the unique continuation property of the paths, we conclude that there exists a word  over the alphabet of  that labels the paths from every state to the state  in . Thus,  is carefully synchronizing.


Let  be the shortest carefully synchronizing word of . It is easy to see that a product  contains a column of ones, where  is the adjacency matrix of  for . Since for every  there is matrix  such that  we obtain a product  with the properties:  has a column of ones and its length is bounded by .

The product  is constructed in the same manner by applying the reasoning of the previous paragraphs to a matrix set . The resulting product  has a column of ones and the length at most . The existence of the product  easily follows from the fact that  is strongly connected (otherwise the set of matrices  is not primitive). Thus, for every pair of states  there exists a path of length at most  that bring  to .


Now, given a carefully synchronizing -state automaton  with the reset threshold equal to  we will construct a primitive set of matrices  such that . It will imply .
Let  be a row vector of 1's, and  be a row vector with the only non-zero entry equal to 1 at position . Let . The set of matrices  is defined as a union , where  is a set of the adjacency matrices of letters of the partial automaton . Since  is carefully synchronizing, there is a product  of matrices in  such that the th column is positive for some . If we multiply  by the matrix  on the right, we obtain a positive matrix product. Thus,  is primitive.

It remains to show that .
Let  be the shortest positive product of matrices in . Note, that  contains at least one matrix from , since every product of matrices in  contains at most one 1 in each row. Let , where the product  doesn't contain matrices from  and . Observe that  contains a positive column. Otherwise,  will have a zero row due to the presence of a zero row in . Therefore, the length of the product  is at least  and we obtain the desired inequality.
\end{proof}

\begin{corollary}
\label{corr:simpleBounds}
The growth rate of  is  and .
\end{corollary}
\begin{proof}
The first part of the claim follows from the result of Zs. Gazdag et al.~\cite[Theorem 3]{Ivan2009}: 
. Thus, .
The second part follows from the result of Martyugin~\cite{Mart2010}. He constructed a series of carefully synchronizing automata with the length of the shortest carefully synchronizing word equal to . Thus, .
\end{proof}


\subsection{Improving the upper bound on the exponent}
\label{sec:imprupper}

The goal of this section is to significantly improve the bound on  and, equivalently, on . 
We will present a new upper bound on the length of the shortest carefully synchronizing word by modifying constructions from~\cite{Ivan2009}. 


Recall that a \emph{partition} of a set  is a collection  of pairwise disjoint non-empty sets whose union is equal to . Given a partition  of , a set  is called a {\it transversal} of  with respect to the partition  if for each  there is a unique  such that .
A set  is a {\it partial transversal} with respect to  if for each  there is at most one  such that .

\begin{example}
For a partition  of  the sets  and  are transversals and  and  are partial transversals. The set  is neither transversal, nor partial transversal.
\end{example}
Let  be an -element set and  be an arbitrary partition. We will denote by  the number of different transversals with respect to  and by  the number of different partial transversals of size .
Let  be the largest value of  among all partitions  of  into  parts. Similarly, let  be the largest value of  among all partitions  of  into  parts. If the value of  is clear from the context, then we will often write  and  to simplify notation.
We will make use of the following bounds on  and :

\begin{lemma}
\label{lemma:trans}
\begin{enumerate}
\item  for .
\item  for .
\item  for .
\item  for .
\end{enumerate}
\end{lemma}
\begin{proof}
\begin{enumerate}
\item It is the statement of Proposition 5 in~\cite{Ivan2009}.
\item Let  be a partition of  into  parts such that , where . If  is the size of the th part of , then it is easy to see that . Observe that for any  we have . Otherwise, by moving an element from the th part to the th part of the partition , we will increase the number of transversals: . Therefore, for a given range of values , every  is equal to  or . Let  be the number of 's equal to , then  is the number of 's equal to . Since , we derive that , and the desired bound follows.
\item Let  be a partition of  into  parts such that , where . If  is the size of the th part of , then by the inequality of arithmetic and geometric means we have  Let us bound the right hand side. Note that . For , we have . Therefore, the largest value of the function  is achieved at .
Thus, .
\item Let  be a partition of  into  parts such that  and let  be the size of the th part of .


\end{enumerate}
\end{proof}

\begin{theorem}
\label{th:newBound}
Let  be the maximum value of the exponent among all sets of  matrices. Let  be the maximum value of  among all -state partial automata , then , and equivalently .
\end{theorem}
\begin{proof}
We will show that  is at most  for any  once  for some threshold . Since  is  by~\cite{Mart2010}, the statement  will clearly follow. Due to Theorem~\ref{th:expcar} we will have the same statement for .






Let  be a carefully synchronizing -state partial automaton with the set of states .
We will construct a carefully synchronizing word  of  via the following iterative procedure:
\begin{enumerate}[(a)]
\item Let  be a letter that is defined on every state  and satisfies , where  denotes the cardinality of a set. Since  is carefully synchronizing, there exists at least one such letter.
\item Choose a positive integer . Let  be a word of the form 

where the words  are defined iteratively for :  is the shortest word such that  is defined on every state and . 
\end{enumerate}
Note, that the word  is well-defined, since at every step the set of possible words for  contains carefully synchronizing words of . Our procedure further ensures that  for every . Thus,  is indeed a carefully synchronizing word. Our goal now is to bound the length of .
The bound on  presented in~\cite{Ivan2009} was obtained using the presented procedure with the parameter  ultimately fixed to 1. By choosing for every  a sufficiently large  satisfying certain conditions, we get a significant improvement. We proceed by bounding the length of intermediate words . To simplify the presentation and without loss of generality, we will further assume that  is divisible by .
\begin{enumerate}
\item . For these values of  we uniformly put . The proof of this case is presented in~\cite[Proposition 7]{Ivan2009}, which we repeat here for convenience.
We are going to show that  by induction. Note, that . Let .
The word  gives rise to a partition  of  into  parts as follows: a pair of states  belongs to the same part of  if . Observe that if , where  are letters, then the set  is a transversal with respect to  for every . Indeed, if it is not the case for some , then the word  satisfy the conditions  of our procedure and it is shorter than , which is impossible. Therefore, the length of  is bounded by . By lemma~\ref{lemma:trans} we conclude . Therefore, , which completes the induction. Observe, that for  we have .

\item . For each  we will choose the value of  at the end of the proof, independent of  and . As before, the word  gives rise to a partition  of  into  parts as follows: a pair of states  belongs to the same part of  if . 
Let us fix  and let , where  are letters. By construction, for every  the cardinality of the set  is equal to . Furthermore,  is a partial transversal with respect to . 
Indeed, if it is not the case, then  and the length of  is not minimal. 
Therefore, the length of  is bounded by the number of partial transversals . 
Using this bound for all  and part 4 of lemma~\ref{lemma:trans} we obtain

for some polynomial  of degree .
Note, that part 3 of lemma~\ref{lemma:trans} can be rewritten as  for . Applying this inequality and the inequality for   times we derive:

By choosing  large enough, such that it satisfies , we ensure that 
every term, starting from the second one, is majorated by the last term. Thus, 
for another polynomial .

\item . 
Since , for every  we can choose  such that . In the same way as before, we derive

where  is a polynomial. The last inequality holds for large enough .
\end{enumerate}
\end{proof}







\section{Sets with no zero rows nor zero columns}

\subsection{Bounds on the exponent}
A quadratic lower bound on the exponents of sets of matrices belonging to  was obtained in~\cite[Corollary 20]{BJO15}.
A first cubic upper bound  was given in~\cite[Theorem 1]{Voy13}\footnote{At the discussion of the connections with the \v{C}ern\'{y} conjecture the author refers to a wrong bound, see~\cite{GoJuTr15} for a discussion.}. The proof relies on standard linear algebraic techniques. This bound was improved in~\cite[Corollary 18]{BJO15} to . The proof is based on the following fact:
a bound  for the reset thresholds of synchronizing automata implies a bound  for the exponents of  matrix sets~\cite[Theorem 17]{BJO15}. In this subsection we will extend this result and present a quadratic bound for a special class of  matrix sets.

We denote by  the maximal exponent among all primitive matrix sets belonging to .
In order to state a theorem for , analogous to theorem~\ref{th:expcar}, we will introduce a new class of automata  defined as follows. An automaton  with the set of states  over an alphabet  belongs to  if there exists a partition of  into  such that for every  we have:
\begin{enumerate}
\item for each state  there exists a state  and a letter  such that ;
\item for every choice of states  such that  for some , there exists a letter  with the property  for all .
\end{enumerate}
In other words, for each , every state is reachable from somewhere by a letter in , and given a list of transformations of states by letters in , we can find a letter in  that performs all the transformations at once. 
\begin{example}
The \v{C}ern\'{y} automaton  equipped with an identity letter  belongs to . The required partition is  and . Clearly, the reset threshold is not changed with the addition of the letter .
\end{example}
\begin{theorem}
\label{th:nzrcclass}
Let   be the maximum value of the exponent among all sets of  matrices of size . Let  be the largest reset threshold among -state automata in , then .
\end{theorem}
\begin{proof}
In order to show that  is  we will reuse the reduction from primitive sets of matrices to partial automata presented in the first part of theorem~\ref{th:expcar}. Let  be a primitive set of  matrices. It can be reduced to partial automata  and  such that . Since  is an  matrix set, we conclude that the automata  and  are complete. Thus, their carefully synchronizing words are ordinary synchronizing words. Furthermore, the automata  and  belong to . Indeed, every letter of these automata was obtained from a matrix in  or . In order to obtain the desired partition, we group a pair of letters together if and only if they were derived from the same matrix. It is a straightforward check that both conditions on the partition are satisfied.

For the other direction, given an automaton  we will construct a set of  matrices  such that . It will imply that . Let  be the partition of the letters of . A set of matrices  consists of matrices , where  is the adjacency matrix of the letter . Clearly, each  is an  matrix due to the first property of the partition and the fact that  is complete. The second property ensures that , since every primitive word of  can be transformed into a synchronizing word of  as in the proof of theorem~\ref{th:expcar}.
\end{proof}

\begin{problem}
\label{prob:rtnzrc}
Improve the bounds  and  on the growth rate of  and, equivalently, . In particular, is there a constant  such that ?
\end{problem}
This problem can be settled in different ways. On one hand, one can show that problem~\ref{prob:rtnzrc} is as hard as the \v{C}ern\'{y} conjecture. There are not many natural problems equivalent to it and the problem of bounding  is a good candidate for this purpose. On the other hand, a quadratic bound for  is clearly of interest by itself.

In the remainder of this subsection we will present an upper bound on the exponent of a set of matrices from a special class. A matrix  has \emph{total support} if every non-zero element  of  lies on a positive diagonal, i.e. for every  such that  there exists a permutation  with the following properties:  and for every  we have . The class of matrices with total support received a lot of attention in the past. For example, it appears in the necessary and sufficient condition for the convergence of the classical Sinkhorn-Knopp method for matrix scaling, see~\cite{SK67}. Another characterization is related to a class of doubly stochastic matrices.
A square matrix  is called \emph{doubly stochastic} if the entries are nonnegative and the sum of elements in each row and column is equal to 1.
A matrix  is said to have a \emph{doubly stochastic pattern} if there exists a doubly stochastic matrix  such that for all  it holds  if and only if . A famous result of Perfect and Mirsky~\cite{PM65} states that a matrix  has total support if and only if  has a doubly stochastic pattern, see~\cite[Theorem 9.2.1]{Bru06} for a modern and much more general treatment of the problem. Now we are ready to state our result.

\begin{theorem}
If each matrix of a primitive set  has total support, then the exponent of  is at most , where  is the size of matrices in .
\end{theorem}
\begin{proof}
We will modify the reduction presented in the first part of theorem~\ref{th:expcar} from a primitive set of matrices  to partial automata  in order to prove the statement. By the aforementioned result of Perfect and Mirsky we conclude that for every matrix  there exists a doubly stochastic matrix  such that for every  we have  if and only if . It is not hard to see that we can further assume that  has only rational entries. Therefore, there exists  and a matrix  with nonnegative integer elements such that the sum of entries in each row and column is equal to , and  if and only if  for all .  We will define the automaton  as follows. For each matrix  we add a function  as a letter of  multiple times. Namely, we treat  as  different letters of the automaton . Let  be an automaton obtained from a matrix set  in the same manner. Similarly to the proofs of theorems~\ref{th:expcar} and~\ref{th:nzrcclass} we can conclude that the automata  and  are complete and synchronizing, moreover, .

Now we are going to show that the automaton  is Eulerian, i.e. the in-degree of each state is equal to its out-degree. Let  be the set of letters generated from a matrix  and  be the row (and column) sum of . It is not hard to see that by the definition of  the size of  is . Furthermore, the number of letters from  that move a state  to a state  is equal to . Thus, the number of incoming edges to  labelled by a letter from  is equal to , which is equal to the size of . Since the alphabet of  is  and for every  the number of incoming and outgoing edges labelled by  coincide, we conclude that the automaton  is Eulerian. The same reasoning allow us to conclude that the automaton  is also Eulerian.
By the famous result of Kari about the reset thresholds of Eulerian automata~\cite{Kari2003Eulerian} we have . Thus, the exponent of the matrix set  is at most .
\end{proof}

\subsection{Computation and approximation of the exponent}
In this subsection we will focus on the problem of computing the exponent of a set of  matrices. Our results rely on the following lemma:
\begin{lemma}
\label{lemma:zeronzrc}
For every synchronizing automaton  with a sink state there exists a set of  matrices  constructible in polynomial time such that .
\end{lemma}
\begin{proof}
If the number of states of  is equal to 1, then  can be an arbitrary set of matrices with the exponent equal to 2. For example,
.
In all the other cases we construct the matrix set  as follows. Since  is synchronizing, it has a unique sink state, which we denote by . Let  be the set of adjacency matrices of letters of . The matrix set  consists of matrices in  modified in such a way that the th row of every matrix is positive, i.e. , where  stands for a row vector of 1's, and  stands for a row vector with the only non-zero entry equal to 1 at position . Since the automaton  is complete, we conclude that every matrix  has no zero rows. Furthermore, each column of  has a positive element at position . Thus, the set of matrices  belongs to . Now we will demonstrate that the set  is primitive. Since the automaton  is synchronizing there exists a product  of matrices from  with a positive column. Due to the fact that for every letter  of  one has , we conclude that the only positive entry in the th row of  is located at position . Thus, the th column of  is positive. Altering each matrix  of the product  to a matrix  with the property  we obtain a product  of matrices from  with the positive th column. 
Now multiply by any matrix  to get a positive product . Thus, the set  is primitive and .

It remains to show that . Let  be the shortest positive product of matrices in . Let  be the corresponding product of matrices in , where  and  differ only in the th row. For each row  there exists the largest  such that the th row of  contains a unique positive entry. 
Due to the fact that the th row of  has several positive entries and the structure of matrices in  we conclude that the unique positive entry occupies the position  in the th row of . 
Therefore, the only positive entry in the th row of  is located at position . Since for every row the value of  is strictly less than , we conclude that the product  has a positive column . It implies that . Rearranging, we obtain the desired inequality .
\end{proof}

The computational complexities of problems related to computation of reset thresholds and synchronizing words were extensively studied. Now we will leverage lemma~\ref{lemma:zeronzrc} to easily obtain a large body of results on the computation and approximation of the exponents of  matrix sets. We assume that the reader is familiar with computational complexity theory. All the missing definitions can be found in the book by Arora and Barak~\cite{Aro09}. Recall that  stands for a class of all languages of the form  with . Note, that  is a superclass of both  and . Thus, the hardness result that we are going to obtain applies for both classes. The class  is contained in  -- the class of problems solvable by a deterministic polynomial-time Turing machine that can use logarithmic number of queries to an oracle for an -complete problem. It is generally believed that the inclusion is proper. We will denote by  the functional analogue of . For a function  we say that an algorithm approximates the exponent 
within a factor  if for every set  of matrices of size  the value  returned by the algorithm satisfies .

\begin{theorem}
\label{th:nzrcComplexity}
Given a set  of three  matrices belonging to  and possibly a positive integer  encoded in binary.
\begin{enumerate}
\item The problem of deciding whether  is -complete.
\item The problem of deciding whether  is -complete.
\item The problem of computing  is -complete.
\item For every constant  it is -hard to approximate  within a factor .
\end{enumerate}
\end{theorem}
\begin{proof}
The hardness results for these problems follow from the corresponding statements about the reset thresholds of synchronizing automata with a sink state and lemma~\ref{lemma:zeronzrc}. Eppstein~\cite{Ep1990} proved that it is -hard to decide whether , where  is given in binary. Olschewski and Ummels~\cite[Theorem 1]{OM2010} shown that it is -hard to decide whether . The same authors~\cite[Theorem 4]{OM2010} also proved that the problem of computing  is -hard. A recent breakthrough by Gawrychowski and Straszak~\cite[Theorem 16]{GawrychowskiStraszak2015StrongInapproximabilityOfTheShortestResetWord} states that for every constant  it is -hard to approximate  within a factor . Moreover, only automata with a sink state were utilized in all reductions presented by the aforementioned authors. In the proof presented by Gawrychowski and Straszak~\cite[Theorem 16]{GawrychowskiStraszak2015StrongInapproximabilityOfTheShortestResetWord} the number of letters  is equal to three, while in all the other reductions automata with only two-letters were utilized.

It remains to show that the stated problems belong to the corresponding classes:
\begin{enumerate}
\item Since  is bounded by a low-degree polynomial , e.g.  by~\cite[Corollary 18]{BJO15}, it suffices to guess a positive product of matrices from  of length . Verification for every such product can be done in polynomial time. Thus, the problem belongs to .
\item It is easy to see that  if and only if  and  does not hold. Since the problem of deciding whether  belongs to , we conclude that the problem of deciding whether  belongs to .
\item The algorithm from  that computes  is a simple binary search algorithm that uses logarithmic number of queries to the oracle for the -complete problem . Recall that the exponent of  is bounded by a polynomial . Thus, in  iterations we can establish precise value of .
\end{enumerate}
\end{proof}

\section{Conclusion}
The goal of our work was to emphasize, and leverage, the fact that several problems about primitive sets of matrices are in some sense equivalent to problems about synchronizing automata. More precisely, we related the bounds on the exponents and the lengths of the shortest carefully synchronizing words in the general case, and the exponents and reset thresholds in the case of matrices without zero rows and columns. Furthermore, we utilized these connections to easily establish the exact complexity classes of different problems concerning the computation of the exponent of a set of matrices belonging to . Thus, we believe that the joint research effort on both topics at the same time can lead to substantial progress on some of the most desperate problems in both fields at the same time. We left a quadratic upper bound on the exponent of an  matrix set as an open problem, and whether its existence brings any implication for the \v{C}ern\'{y} conjecture. Our future work includes the search for matrix counterparts of special classes of automata, which have quadratically bounded reset threshold. These statements will be analogous to our result on primitive sets of matrices having total support.

\subparagraph*{Acknowledgements}
This work was supported by Interuniversity Attraction Poles (IAP) Programme, and by the ARC grant 13/18-054 (Communaut\'e fran\c{c}aise de Belgique). B.~Gerencs\'er was also supported by the Hungarian Academy of Sciences. V.~Gusev benefited from the Russian foundation for basic research (grant 16-01-00795), Ministry of Education and Science of the Russian Federation (project no. 1.1999.2014/K), and the Competitiveness Program of Ural Federal University. R.~Jungers is a F.R.S./F.N.R.S. research associate.

\bibliography{synchronization}

\end{document}