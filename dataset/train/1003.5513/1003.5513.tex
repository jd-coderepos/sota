\documentclass[copyright]{eptcs}
\usepackage{breakurl}        \usepackage{amsmath,amssymb,amsthm}
\usepackage{prooftree}
\usepackage{display-figure}
\usepackage{uniqueness-macros}
\usepackage{stmaryrd}

\renewcommand{\subtype}{\ensuremath{\mathrel{<:}}}
\renewcommand{\struct}{\ensuremath{\mathrel{\Yleft}}} \newcommand{\rel}[1]{\ensuremath{\mathcal{#1}}}

\title{Uniqueness Typing for Resource Management in Message-Passing Concurrency}
\author{
  Edsko de Vries\thanks{This research was supported by SFI project SFI 06 IN.1 1898.}
  \institute{Trinity College Dublin, Ireland}
  \email{Edsko.de.Vries@cs.tcd.ie}
  \and
  Adrian Francalanza
  \institute{University of Malta}
  \email{Adrian.Francalanza@um.edu.mt}
  \and
  Matthew Hennessy\footnotemark[1]
  \institute{Trinity College Dublin, Ireland}
  \email{Matthew.Hennessy@cs.tcd.ie}
}

\def\titlerunning{Uniqueness Typing for Resource Management in Message Passing Concurrency}
\def\authorrunning{E. de Vries, A. Francalanza and M. Hennessy}

\begin{document}
\maketitle

\begin{abstract}
We view channels as the main form of resources in a message-passing programming
paradigm.  These channels need to be carefully managed in settings where
resources are scarce.  To study this problem, we extend the \pic with
primitives for channel allocation and deallocation and allow channels to be
reused to communicate values of different types.  Inevitably, the added
expressiveness increases the possibilities for runtime errors. We define a
substructural type system which combines uniqueness typing and affine typing to
reject these ill-behaved programs.
\end{abstract}

\section{Introduction}

Message-passing concurrency is a programming paradigm whereby shared memory is
prohibited and process interaction is limited to explicit message
communication. This concurrency paradigm forms the basis for a number of
process calculi such as the \pic \cite{Milner:Pi} and has been adopted by
programming languages such as the actor based language Erlang
\cite{Armstrong:Erlang}.   

Message-passing concurrency often abstracts away from resource management and
programs written at this abstraction level exhibit poor resource awareness. In
this paper we study ways of improving this shortcoming. Specifically, we
develop a statically typed extension of the \pic in which resources, i.e.
channels, can be reused at varying types and unwanted resources can be safely
deallocated. 

Idiomatic \pic processes are often characterized by wasteful
use-once-throw-away channels \cite{Milner:Pi,KobayashiPT:linearity}.  Consider
the following two \pic process definitions
 
\ptit{timeSrv} defines a server that repeatedly waits on a channel named
\textit{getTime} to dynamically receive a channel name, represented by
the bound variable , and then replies with the current time on .
\ptit{dateSrv} is a similar service which returns the current date.  An
idiomatic \pic client definition is

 uses two distinct channels  and
 as return channels to query the time and date servers, and
then continues as process  with the values obtained. These return
channels are scoped (private), to preclude interference from other clients
concurrently querying the servers.  

From a resource management perspective it makes pragmatic sense to try and
reduce the number of channels used and use \emph{one} channel to communicate
with both the time server and the date server. 

From a typing perspective, this reuse of the same channel entails \emph{strong
update} on the channel: that is, reuse of a channel to communicate values of
different types. Strong update must be carefully controlled; for instance, an
attempt to use one channel to communicate with both servers in parallel is
unsafe:

Standard \pic type systems accept only  and rule out both
 and . However,  is
safe because the communication with the date server happens \emph{strictly after}  the
communication with the time server, and also because the time server will only use the
return channel \emph{once}. 

Adequate resource management also requires precise descriptions of when
resources are allocated and existing ones are disposed.  The characteristic
scoping construct  is  unfit for this purpose and should be used
only as a bookkeeping construct delineating name scoping (which evolves during
computation through scope extrusion) . One reason against such an
interpretation  is the structural rule 

whose symmetric nature would entail implicit garbage collection of channels,
but also the possibility of unfettered spurious channel allocations. Another
reason against this interpretation is the fact that the \pic semantics does not
specify whether, in a process such as , channel
 is allocated before or after the input on .
This problem becomes more acute when scoping occurs inside recursive
definitions.  We address this shortcoming by introducing an explicit allocation
construct .  When the allocation is executed,  a new channel 
is created at runtime and the  changes to
.  Dually, we also extend the calculus with an
explicit \emph{deallocation} operator . We can then rewrite the
client as: 

Inevitably, the added expressiveness of this extended \pic increases the
possibilities for runtime errors like value mismatch during communication and
usage of channels which have been deallocated.  

We define a type system which rejects processes that are unsafe; the type
system combines uniqueness typing \cite{barendsen:functional} and affine typing
\cite{KobayashiPT:linearity}, while permitting value coercion across these
modes through subtyping. Uniqueness typing gives us \emph{global guarantees},
simplifying local reasoning when typing both strong updates and safe
deallocations. Uniqueness typing can be seen as dual to affine typing
\cite{harrington:uniquenesslogic}, and we make essential use of this duality to
allow uniqueness to be temporarily violated.

\section{The \picR}

\figref{fig:syntax-ext} shows the syntax for the \picr. The language is the
standard \pic extended with primitives for explicit channel allocation and
deallocation; moreover, channel scoping records whether a channel is allocated
() or deallocated ().  The syntax assumes two separate denumerable
sets of channel names  and variables , and lets
identifiers  range over both.  The input and channel
allocation constructs are binders for variables  and  \resp,
whereas scoping is a binder for names (\ie ).  The syntax also assumes a
denumerable set of process variables  which are bound by the
recursion construct. 

\begin{display}{Polyadic \picr syntax}{fig:syntax-ext}

\end{display}

Channels are stateful (allocated, \salloc, or deallocated, \sdalloc) and
process semantics is defined over configurations,  where
 describes the state of
the free channels in , and stateful scoping \restT{c}{s}{P} describes the
state of scoped channels.  A tuple  is a configuration
whenever  and is denoted  as \confSP. We say that a
configuration \confSP\ is closed whenever  does not have any free variables.
\figref{fig:pir-reduction} defines contexts over configuration where
 denotes the application of a context  to a
configuration \confSP.  In the case where a context scopes a name  the
definition extracts the state relating to  from \st{} and associates it with
. For example,


The reduction relation is defined as the least contextual relation over closed
configurations satisfying the rules in \figref{fig:pir-reduction}, using a
standard \pic structural equivalence relation .  Communication
(\rtit{rCom}) requires the communicating channel to be allocated. Allocation
(\rtit{rAll}) creates a private allocated channel and substitutes it for the
bound variable of the allocation construct in the continuation; the condition
 ensures that  is fresh in . Deallocation (\rtit{rFree})
is the only construct that changes the visible state of a configuration, \st. 

\begin{display}{Contexts, Reduction Rules and Error Predicates}{fig:pir-reduction}
\textbf{Contexts}
5pt]

\begin{array}{rll}
  \ctxtEmp{\confSP} & \deftxt \confSP \\
  \ctxtPar{\ctxtGenN{\context}{\confSP}}{Q} & \deftxt \conf{\st'}{(P'\paral Q)} & \text{if } \ctxtGenN{\context}{\confSP}=\conf{\st'}{P'}\\
  \ctxtPar{Q}{\ctxtGenN{\context}{\confSP}} & \deftxt \conf{\st'}{(Q\paral P')} & \text{if } \ctxtGenN{\context}{\confSP}=\conf{\st'}{P'}\\
  \ctxtR{c}{\ctxtGenN{\context}{\confSP}} & \deftxt \conf{\st'}{\restT{c}{s}{P'}} & \text{if } \ctxtGenN{\context}{\confSP}=\conf{\st',\envmap{c}{s}}{P'}
\end{array}
2em]
\begin{prooftree}
\strut
\justifiedBy{\rtit{rThen}}
\confS{\match{c}{c}{P}{Q}}\reduc \confS{P}
\end{prooftree}
&
\begin{prooftree}
c \neq d
\justifiedBy{\rtit{rElse}}
\confS{\match{c}{d}{P}{Q}}\reduc \confS{Q}
\end{prooftree}
\2em]
\multicolumn{2}{c}{
\begin{prooftree}
    P \steq P' \qquad \confS{P'} \reduc  \confS{Q'} \qquad Q'\steq Q
    \justifiedBy{\rtit{rStr}}
    \confSP\reduc \confS{Q} 
  \end{prooftree}
}\
\begin{prooftree}
|\vec{d}| \neq |\vec{x}|
\justifiedBy{eAty}
\confS{\piout{c}{\vec{d}}{P} \paralS \piin{c}{\vec{x}}{Q} \reducl{\text{err}}}
\end{prooftree}
\quad
\begin{prooftree}
\st(c)=\sdalloc
\justifiedBy{eOut}
\confS{\piout{c}{\vec{d}}{P}} \reducl{\text{err}}
\end{prooftree}
\quad
\begin{prooftree}
\st(c)=\sdalloc
\justifiedBy{eIn}
\confS{\piin{c}{\vec{x}}{Q}} \reducl{\text{err}}
\end{prooftree}
\quad 
\begin{prooftree}
P\steq Q \quad\confS{Q} \reducl{\text{err}}
\justifiedBy{eStr}
\confSP \reducl{\text{err}}
\end{prooftree}
\tcnf{\env}{\confSP}
\begin{array}{lrll}
\tV & \bnfdef & \chantyp{\lst{\tV}}{\aV} & \text{(channel type)} \\
    & \bnfsep & \proctyp                 & \text{(process)} 
\end{array}\qquad 
\begin{array}{lrll}
\aV & \bnfdef & \affine       & \text{(affine)} \\
    & \bnfsep & \unrestricted & \text{(unrestricted)} \\
    & \bnfsep & \unique{i}    & \text{(unique after  steps, )}  
\end{array}
2em]
&
\begin{prooftree}
u, v \in \Gamma \qquad
\tprocE{P} \qquad
\tprocE{Q}
\justifiedBy{tIf}
\tprocE{\match{u}{v}{P}{Q}}
\end{prooftree} 
 &
\begin{prooftree}
\tprocP{\env^\unrestricted, \envmap{X}{\proctyp}} 
\justifiedBy{tRec}
\tproc{\env^\unrestricted}{\recX{P}} 
\end{prooftree} 
 &
\begin{prooftree}
\strut
\justifiedBy{tVar}
\tproc{\envmap{X}{\proctyp}}{X}
\end{prooftree}
\2em]
&
\begin{prooftree}
\tprocP{\env,\envmap{c}{\tV}}
\justifiedBy{tRst1}
\tprocE{\restT{c}{\salloc}{P}}
\end{prooftree} 
 &
\begin{prooftree}
\tprocEP
\justifiedBy{tRst2}
\tprocE{\restT{c}{\sdalloc}{P}}
\end{prooftree} 
 \1em]
\textbf{Structural rules}



\textbf{Typing configurations}\
\begin{prooftree}
\forall c \in\dom{\env}.\, \st(c) = \salloc \qquad 
\tprocEP \qquad
\Gamma \text{ is a partial map}
\justifiedBy{tConf}
\tcnf{\env}{\confSP}
\end{prooftree}  

  \env, \envmap{c}{\chantyp{\lst{\tV}}{\aV-1}} & \deftxt
  \begin{cases}
    \env & \text{if }\,\aV=\affine\\
    \env, \envmap{c}{\chantyp{\vec{\tV}}{\unrestricted}} & \text{if }\,\aV=\unrestricted\\
    \env, \envmap{c}{\chantyp{\vec{\tV}}{\unique{i}}} &  \text{if }\,\aV=\unique{i + 1}
  \end{cases}

\begin{prooftree}
\justifiedBy{pUnr}
\chantyp{\tVlst}{\unrestricted} = \chantyp{\tVlst}{\unrestricted} \circ \chantyp{\tVlst}{\unrestricted}
\end{prooftree} \qquad
\begin{prooftree}
\justifiedBy{pProc}
\proctyp = \proctyp \circ \proctyp
\end{prooftree} \qquad 
\begin{prooftree}
\justifiedBy{pUnq}
\chantyp{\tVlst}{\unique{i}} = \chantyp{\tVlst}{\affine} \circ \chantyp{\tVlst}{\unique{i+1}}
\end{prooftree}

\begin{prooftree}
\phantom{\aV_1 \subtype \aV_2}
\justifiedBy{sIndx}
\unique{i} \subtype \unique{i+1}
\end{prooftree} \qquad 
\begin{prooftree}
\phantom{\aV_1 \subtype \aV_2}
\justifiedBy{sUnq}
\unique{i+1} \subtype \unrestricted
\end{prooftree} \qquad 
\begin{prooftree}
\phantom{\aV_1 \subtype \aV_2}
\justifiedBy{sAff}
\unrestricted \subtype \affine 
\end{prooftree} \qquad 
\begin{prooftree}
\aV_1 \subtype \aV_2
\justifiedBy{sTyp}
\chantyp{\tVlst}{\aV_1} \subtype \chantyp{\tVlst}{\aV_2}
\end{prooftree} 

\ptit{infHeap} \deftri \recX{\alloc{x}{\piout{\mathit{heap}}{x}{X}}}
c :
\chantyp{}{\unique{1}}, c : \chantyp{}{\affine}
\begin{prooftree}

    \justifiedBy{tOut}
    \envmap{a}{\chantyp{\chantyp{}{\affine}}{\unrestricted}}, \envmap{u}{\chantyp{}{\affine}} \vdash \pioutA{a}{u} 
  
    
    \justifiedBy{tIn}
    \envmap{a}{\chantyp{\chantyp{}{\affine}}{\unrestricted}}, \envmap{u}{\chantyp{}{\unique{1}}} \vdash \piin{a}{x}{\piinBB{u}{()}{\freeA{u} \paral \pioutA{a}{x}}}
  
\justifiedBy{tCon \textnormal{(twice)}}
\envmap{a}{\chantyp{\chantyp{}{\affine}}{\unrestricted}}, \envmap{u}{\chantyp{}{\uniqueNow}} \vdash \pioutA{a}{u} \paral \piin{a}{x}{\piinBB{u}{()}{\freeA{u} \paral \pioutA{a}{x}}}
\end{prooftree}

\begin{prooftree}

\justifiedBy{tCon}
\envmap{u}{\chantyp{}{\uniqueNow}} \vdash \piinBB{u}{()}{\freeA{u} \paral \pioutA{a}{u}}
\end{prooftree}
\env, \envmap{c}{\chantyp{\vec{\tV}}{a}},
\envmap{c}{\chantyp{\vec{\tV}}{a'}}\env,
\envmap{c}{\chantyp{\vec{\tV}}{a-1}}, \envmap{c}{\chantyp{\vec{\tV}}{a'}}
  \ptit{client}_2 \deftri \alloc{x}{\pioutB{\textit{getTime}}{x}{\piinB{x}{y_\mathit{hr}, y_\mathit{min}}{\pioutB{\textit{getDate}}{x}{\piinB{x}{z_\mathit{year}, z_\mathit{mon}, z_\mathit{day}}{\free{x}{P}}}}}}

\ptit{client}_3 \deftri \piin{\textit{heap}}{x}{\pioutB{\textit{getTime}}{x}{\piinB{x}{y_\mathit{hr}, y_\mathit{min}}{\pioutB{\textit{getDate}}{x}{\piinB{x}{z_\mathit{year}, z_\mathit{mon}, z_\mathit{day}}{\piout{\textit{heap}}{x}{P}}}}}}
\label{eq:2}
   &\ptit{client}_3 \paral \ptit{client}_3 \paral \ptit{client}_3 \paral \ptit{timeSrv} \paral \ptit{dateSrv} \paral \rest{c}{\pioutAB{\textit{heap}}{c}} 

\ptit{client}_4 \deftri \piin{\textit{heap}}{x}{\recX{(\pioutB{\textit{getTime}}{x}{\piinB{x}{y_\mathit{hr}, y_\mathit{min}}{P \paral X}})}}

\alloc{x}{\bigl(\;\piout{c}{d_1}{\pioutB{d_1}{}{\inert}} \quad\paral\quad \piout{c}{d_2}{\inert} \quad \paral\quad \piin{c}{y}{\piinB{y}{}{\freeA{x}}}\;\bigr)} 

Statically, it is hard to determine whether the third parallel process will
eventually execute the  operation.  This is due to the fact that it
can non-deterministically react with either the first or second parallel
process and, should it react with the second process, it will block at
\piinB{d_2}{}{\freeA{x}}.  In order to reject this process as ill-typed, the
type-system needs to detect potential deadlocks.  This can be done
\cite{kobayashi:2006}, but requires a type system that is considerably more
complicated than ours. We leave the responsibility to deallocate to the user,
but guarantee that resources once deallocated will no longer be used. 

The simplicity of our type-system makes it easily extensible.  For instance,
one useful extension would be that of input/output modalities, which blend
easily with the affine/unique duality.  Presently,   when a server process
splits a channel  into one channel of type
 and two channels of type 
to be given to two clients, the clients can potentially use this channel to
communicate amongst themselves instead of the server.  Modalities are a natural
mechanism  to preclude this from happening.

We are currently investigating ways how uniqueness types can be used to refine
existing equational theories, so as to be able to equate processes such as
 and .   This will probably require us to
establish a correspondence between uniqueness at the type level and restriction
at the term level.

\begin{thebibliography}{10}
\providecommand{\bibitemstart}[1]{\bibitem{#1}}
\providecommand{\bibitemend}{}
\providecommand{\bibliographystart}{}
\providecommand{\bibliographyend}{}
\providecommand{\url}[1]{\texttt{#1}}
\providecommand{\urlprefix}{Available at }
\providecommand{\bibinfo}[2]{#2}
\bibliographystart

\bibitemstart{achten:cleanio}
\bibinfo{author}{P.~M. Achten} \& \bibinfo{author}{M.~J. Plasmeijer}
  (\bibinfo{year}{1995}): \emph{\bibinfo{title}{The ins and outs of {Clean}
  {I/O}}}.
\newblock {\sl \bibinfo{journal}{Journal of Functional Programming}}
  \bibinfo{volume}{5}(\bibinfo{number}{1}), pp. \bibinfo{pages}{81--110}.
\bibitemend

\bibitemstart{ahmed:stepindexed}
\bibinfo{author}{Amal Ahmed}, \bibinfo{author}{Matthew Fluet} \&
  \bibinfo{author}{Greg Morrisett} (\bibinfo{year}{2005}):
  \emph{\bibinfo{title}{A Step-Indexed Model of Substructural State}}.
\newblock In: {\sl \bibinfo{booktitle}{Proceedings of the 10th ACM SIGPLAN
  International Conference on Functional Programming (ICFP)}},
  \bibinfo{publisher}{ACM}, pp. \bibinfo{pages}{78--91}.
\bibitemend

\bibitemstart{Armstrong:Erlang}
\bibinfo{author}{Joe Armstrong} (\bibinfo{year}{2007}):
  \emph{\bibinfo{title}{Programming Erlang: Software for a Concurrent World}}.
\newblock \bibinfo{publisher}{Pragmatic Bookshelf}.
\bibitemend

\bibitemstart{barendsen:functional}
\bibinfo{author}{Erik Barendsen} \& \bibinfo{author}{Sjaak Smetsers}
  (\bibinfo{year}{1996}): \emph{\bibinfo{title}{Uniqueness Typing for
  Functional Languages with Graph Rewriting Semantics}}.
\newblock {\sl \bibinfo{journal}{Mathematical Structures in Computer Science}}
  \bibinfo{volume}{6}, pp. \bibinfo{pages}{579--612}.
\bibitemend

\bibitemstart{Bornat:05Separation}
\bibinfo{author}{Richard Bornat}, \bibinfo{author}{Cristiano Calcagno},
  \bibinfo{author}{Peter O'Hearn} \& \bibinfo{author}{Matthew Parkinson}
  (\bibinfo{year}{2005}): \emph{\bibinfo{title}{Permission accounting in
  separation logic}}.
\newblock {\sl \bibinfo{journal}{SIGPLAN Not.}}
  \bibinfo{volume}{40}(\bibinfo{number}{1}), pp. \bibinfo{pages}{259--270}.
\bibitemend

\bibitemstart{boyland:03fractions}
\bibinfo{author}{John Boyland} (\bibinfo{year}{2003}):
  \emph{\bibinfo{title}{Checking Interference with Fractional Permissions}}.
\newblock In: \bibinfo{editor}{R.~Cousot}, editor: {\sl
  \bibinfo{booktitle}{Static Analysis: 10th International Symposium}}, {\sl
  \bibinfo{series}{Lecture Notes in Computer Science}} \bibinfo{volume}{2694},
  \bibinfo{publisher}{Springer}, pp. \bibinfo{pages}{55--72}.
\bibitemend

\bibitemstart{clarke:external}
\bibinfo{author}{Dave Clarke} \& \bibinfo{author}{Tobias Wrigstad}
  (\bibinfo{year}{2003}): \emph{\bibinfo{title}{External Uniqueness Is Unique
  Enough}}.
\newblock In: {\sl \bibinfo{booktitle}{{ECOOP} 2003 -- Object-Oriented
  Programming}}, {\sl \bibinfo{series}{Lecture Notes in Computer Science}}
  \bibinfo{volume}{2743}, \bibinfo{publisher}{Springer Berlin / Heidelberg},
  pp. \bibinfo{pages}{59--67}.
\bibitemend

\bibitemstart{fahndrich:2002}
\bibinfo{author}{Manuel F\"{a}hndrich} \& \bibinfo{author}{Robert DeLine}
  (\bibinfo{year}{2002}): \emph{\bibinfo{title}{Adoption and focus: practical
  linear types for imperative programming}}.
\newblock In: {\sl \bibinfo{booktitle}{PLDI '02: Proceedings of the ACM SIGPLAN
  2002 Conference on Programming language design and implementation}},
  \bibinfo{publisher}{ACM}, pp. \bibinfo{pages}{13--24}.
\bibitemend

\bibitemstart{foster:2002}
\bibinfo{author}{Jeffrey~S. Foster}, \bibinfo{author}{Tachio Terauchi} \&
  \bibinfo{author}{Alex Aiken} (\bibinfo{year}{2002}):
  \emph{\bibinfo{title}{Flow-sensitive type qualifiers}}.
\newblock In: {\sl \bibinfo{booktitle}{PLDI '02: Proceedings of the ACM SIGPLAN
  2002 Conference on Programming language design and implementation}},
  \bibinfo{publisher}{ACM}, pp. \bibinfo{pages}{1--12}.
\bibitemend

\bibitemstart{harrington:uniquenesslogic}
\bibinfo{author}{Dana Harrington} (\bibinfo{year}{2006}):
  \emph{\bibinfo{title}{Uniqueness logic}}.
\newblock {\sl \bibinfo{journal}{Theoretical Computer Science}}
  \bibinfo{volume}{354}(\bibinfo{number}{1}), pp. \bibinfo{pages}{24--41}.
\bibitemend

\bibitemstart{Honda:SessionTypes}
\bibinfo{author}{Kohei Honda}, \bibinfo{author}{Vasco~Thudichum Vasconcelos} \&
  \bibinfo{author}{Makoto Kubo} (\bibinfo{year}{1998}):
  \emph{\bibinfo{title}{Language Primitives and Type Discipline for Structured
  Communication-Based Programming}}.
\newblock In: {\sl \bibinfo{booktitle}{ESOP '98: Proceedings of the 7th
  European Symposium on Programming}}, \bibinfo{publisher}{Springer-Verlag},
  pp. \bibinfo{pages}{122--138}.
\bibitemend

\bibitemstart{kobayashi:typesystems}
\bibinfo{author}{Naoki Kobayashi} (\bibinfo{year}{2003}):
  \emph{\bibinfo{title}{Type Systems for Concurrent Programs}}.
\newblock In: {\sl \bibinfo{booktitle}{Formal Methods at the Crossroads: From
  Panacea to Foundational Support}}, {\sl \bibinfo{series}{LNCS}}
  \bibinfo{volume}{2757}, \bibinfo{publisher}{Springer Berlin / Heidelberg},
  pp. \bibinfo{pages}{439--453}.
\bibitemend

\bibitemstart{kobayashi:2006}
\bibinfo{author}{Naoki Kobayashi} (\bibinfo{year}{2006}):
  \emph{\bibinfo{title}{A New Type System for Deadlock-Free Processes}}.
\newblock In: {\sl \bibinfo{booktitle}{{CONCUR} 2006 – Concurrency Theory}},
  {\sl \bibinfo{series}{Lecture Notes in Computer Science}}
  \bibinfo{volume}{4137}, \bibinfo{publisher}{Springer Berlin / Heidelberg},
  pp. \bibinfo{pages}{233--247}.
\bibitemend

\bibitemstart{KobayashiPT:linearity}
\bibinfo{author}{Naoki Kobayashi}, \bibinfo{author}{Benjamin~C. Pierce} \&
  \bibinfo{author}{David~N. Turner} (\bibinfo{year}{1999}):
  \emph{\bibinfo{title}{Linearity and the pi-calculus}}.
\newblock {\sl \bibinfo{journal}{ACM Trans. Program. Lang. Syst.}}
  \bibinfo{volume}{21}(\bibinfo{number}{5}), pp. \bibinfo{pages}{914--947}.
\bibitemend

\bibitemstart{Milner:Pi}
\bibinfo{author}{Robin Milner}, \bibinfo{author}{Joachim Parrow} \&
  \bibinfo{author}{David Walker} (\bibinfo{year}{1992}):
  \emph{\bibinfo{title}{A calculus of mobile processes, parts I and II}}.
\newblock {\sl \bibinfo{journal}{Inf. Comput.}}
  \bibinfo{volume}{100}(\bibinfo{number}{1}), pp. \bibinfo{pages}{1--40}.
\bibitemend

\bibitemstart{odersky:observers}
\bibinfo{author}{Martin Odersky} (\bibinfo{year}{1992}):
  \emph{\bibinfo{title}{Observers for Linear Types}}.
\newblock In: \bibinfo{editor}{Bernd Krieg-Br\"uckner}, editor: {\sl
  \bibinfo{booktitle}{Proceedings of the 4th European Symposium on Programming
  (ESOP)}}, {\sl \bibinfo{series}{Lecture Notes in Computer Science}}
  \bibinfo{volume}{582}, \bibinfo{publisher}{Springer-Verlag}, pp.
  \bibinfo{pages}{390--407}.
\bibitemend

\bibitemstart{pottier:2007}
\bibinfo{author}{Fran\c{c}ois Pottier} (\bibinfo{year}{2007}).
\newblock \emph{\bibinfo{title}{Wandering through linear types, capabilities,
  and regions}}.
\newblock \bibinfo{note}{Survey talk given at INRIA, Rocquencourt, France}.
\bibitemend

\bibitemstart{smith:2000}
\bibinfo{author}{Frederick Smith}, \bibinfo{author}{David Walker} \&
  \bibinfo{author}{J.~Gregory Morrisett} (\bibinfo{year}{2000}):
  \emph{\bibinfo{title}{Alias Types}}.
\newblock In: {\sl \bibinfo{booktitle}{ESOP '00: Proceedings of the 9th
  European Symposium on Programming Languages and Systems}},
  \bibinfo{publisher}{Springer-Verlag}, pp. \bibinfo{pages}{366--381}.
\bibitemend

\bibitemstart{teller:resourcespi}
\bibinfo{author}{David Teller} (\bibinfo{year}{2004}):
  \emph{\bibinfo{title}{Recollecting resources in the pi-calculus}}.
\newblock In: {\sl \bibinfo{booktitle}{Proceedings of IFIP TCS 2004}},
  \bibinfo{publisher}{Kluwer Academic Publishing}, pp.
  \bibinfo{pages}{605--618}.
\bibitemend

\bibitemstart{edsko:thesis}
\bibinfo{author}{Edsko de~Vries} (\bibinfo{year}{2008}):
  \emph{\bibinfo{title}{Making Uniqueness Typing Less Unique}}.
\newblock \bibinfo{type}{Ph.D. thesis}, \bibinfo{school}{Trinity College
  Dublin, Ireland}.
\bibitemend

\bibitemstart{devries:linearity-appendix}
\bibinfo{author}{Edsko de~Vries}, \bibinfo{author}{Adrian Francalanza} \&
  \bibinfo{author}{Matthew Hennessy} (\bibinfo{year}{2009}):
  \emph{\bibinfo{title}{Uniqueness Typing for Resource Management in
  Message-Passing Concurrency---Technical Appendix}}.
\newblock \bibinfo{type}{Technical Report}, \bibinfo{institution}{Trinity
  College Dublin, Ireland}.
\newblock \bibinfo{note}{Available from
  \texttt{http://www.cs.tcd.ie/Edsko.de.Vries/pub}}.
\bibitemend

\bibitemstart{wadler:change}
\bibinfo{author}{Philip Wadler} (\bibinfo{year}{1990}):
  \emph{\bibinfo{title}{Linear types can change the world!}}
\newblock In: \bibinfo{editor}{M.~Broy} \& \bibinfo{editor}{C.~B. Jones},
  editors: {\sl \bibinfo{booktitle}{Proceedings of the IFIP TC2 WG 2.2/2.3
  Working Conference on Programming Concepts and Methods}},
  \bibinfo{publisher}{North-Holland}, pp. \bibinfo{pages}{561--581}.
\bibitemend

\bibitemstart{wadler:use}
\bibinfo{author}{Philip Wadler} (\bibinfo{year}{1991}):
  \emph{\bibinfo{title}{Is there a use for linear logic?}}
\newblock In: {\sl \bibinfo{booktitle}{Proceedings of the 2nd ACM SIGPLAN
  symposium on Partial Evaluation and semantics-based program manipulation
  (PEPM)}}, \bibinfo{publisher}{ACM}, pp. \bibinfo{pages}{255--273}.
\bibitemend

\bibitemstart{walker:2001}
\bibinfo{author}{David Walker} \& \bibinfo{author}{J.~Gregory Morrisett}
  (\bibinfo{year}{2001}): \emph{\bibinfo{title}{Alias Types for Recursive Data
  Structures}}.
\newblock In: {\sl \bibinfo{booktitle}{TIC '00: Third International Workshop on
  Types in Compilation}}, \bibinfo{publisher}{Springer-Verlag}, pp.
  \bibinfo{pages}{177--206}.
\bibitemend

\bibliographyend
\end{thebibliography}

\end{document}
