\documentclass[11pt]{article}
\usepackage{latexsym}
\usepackage{tabularx,booktabs,multirow,delarray,array}
\usepackage{graphicx,amssymb,amsmath,amssymb}
\usepackage[vlined, ruled]{algorithm2e}


\aboverulesep=0pt
\belowrulesep=0pt

\oddsidemargin=0.0in\evensidemargin=0.0in\headheight=0.0in
\topmargin=-0.40in \textheight=9.0in \textwidth=6.45in   







\newcommand{\TT}{\mbox{}}
\newcommand{\SF}{\mbox{}}
\newcommand{\RG}{\mbox{}}
\newcommand{\TR}{\mbox{}}
\newcommand{\BR}{\mbox{}}
\newcommand{\SR}{\mbox{}}
\newcommand{\BS}{\mbox{}}
\newcommand{\CBS}{\mbox{}}
\newcommand{\IP}{\mbox{}}
\newcommand{\IN}{\mbox{}}
\newcommand{\DP}{\mbox{}}
\newcommand{\BP}{\mbox{}}
\newcommand{\MST}{\mbox{}}
\newcommand{\SP}{\mbox{}}
\newcommand{\SPL}{\mbox{}}
\newcommand{\Vor}{\mbox{}}
\newcommand{\diff}{\mbox{}}

\renewcommand{\thefootnote}{\fnsymbol{footnote}}


\def\calD{\mathcal{D}}
\def\calVd{\mathcal{G}^d_\mathcal{V}}
\def\pra{PRA}
\def\sangle{\angle^{\uparrow}}

\def\lemmaspace{\vspace*{0in}}
\def\sectionspace{\vspace*{0in}}
\def\subsectionspace{\vspace*{0in}}



\begin{document}

\baselineskip=14.0pt

\title{
\vspace*{-0.55in} An Improved Algorithm for Reconstructing a Simple
Polygon from the Visibility Angles\thanks{This research was
supported in part by NSF under Grant CCF-0916606.}}

\author{
Danny Z. Chen\thanks{Department of Computer Science and Engineering,
University of Notre Dame, Notre Dame, IN 46556, USA.
E-mail: {\tt \{dchen, hwang6\}@nd.edu}.
}
\hspace*{0.2in} Haitao Wang\footnotemark[2] \thanks{Corresponding
author.
}}


\date{}

\maketitle

\thispagestyle{empty}

\newtheorem{lemma}{Lemma}
\newtheorem{theorem}{Theorem}
\newtheorem{corollary}{Corollary}
\newtheorem{fact}{Fact}
\newtheorem{definition}{Definition}
\newtheorem{observation}{Observation}
\newtheorem{condition}{Condition}
\newtheorem{property}{Property}
\newtheorem{claim}{Claim}
\newenvironment{proof}{\noindent {\textbf{Proof:}}\rm}{\hfill 
\rm}


\pagestyle{plain}
\pagenumbering{arabic}
\setcounter{page}{1}

\vspace*{-0.2in}
\begin{abstract}
In this paper, we study the following problem of reconstructing a
simple polygon: Given a cyclically 
ordered vertex sequence of an unknown simple polygon  of  vertices and, for each vertex
 of , the sequence of angles defined by all the visible vertices of  in ,
reconstruct the polygon  (up to
similarity). An 
time algorithm has been proposed for this problem. 
We present an improved algorithm 
with running time ,
based on new observations on the geometric structures of the problem.
Since the input size (i.e., the total number of input visibility angles) is
 in the worst case, our algorithm is worst-case optimal. 
\end{abstract}



\sectionspace
\section{Introduction}
\label{sec:intro}
In this paper, we study the problem of reconstructing a simple polygon 
from the visibility angles measured at the vertices of  and
from the cyclically ordered vertices of  along its boundary. 
Precisely, for an unknown simple polygon
 of  vertices, suppose we are given (1) the vertices ordered 
counterclockwise (CCW) along the boundary 
of , and (2) for each vertex  of , the angles
between any two adjacent rays emanating from  to the vertices of
 that are visible to  such that these angles
are in the CCW order as seen around , beginning
at the CCW neighboring vertex of  on the boundary of  (e.g., see
Fig.~\ref{fig:angles}). 
A vertex  of  is {\em visible} to a vertex  of  if the
line segment connecting  and  lies entirely in .
The objective of the problem is to reconstruct the simple polygon  (up
to similarity) that fits all the
given angles. We call this problem the {\em polygon
reconstruction problem from angles}, or {\em PRA} for short. 
Figure~\ref{fig:example} gives an example. 




\begin{figure}[t]
\begin{minipage}[t]{\linewidth}
\begin{center}
\includegraphics[totalheight=1.2in]{angles.eps}
\caption{\footnotesize Illustrating the angle measurement at a
vertex : The angle sequence 
is given. 
}\label{fig:angles}
\end{center}
\end{minipage}
\end{figure}

The \pra\ problem has been studied by Disser, Mihal\'ak, and Widmayer 
\cite{ref:DisserRe10}, who showed that the solution polygon 
for the input is unique (up to similarity) and gave an  
time algorithm for reconstructing such a polygon. Using the input, 
their algorithm first constructs the visibility graph  of  and
subsequently reconstructs the polygon . As shown in 
\cite{ref:DisserRe10}, once  is known, the polygon  can be 
obtained efficiently (e.g., in  time) with the help of the angle data 
and the CCW ordered vertex sequence of . 

Given a visibility
graph , the problem of determining whether there is a polygon 
that has  as its visibility graph is called the visibility graph
{\em recognition} problem, and the problem of actually constructing
such a polygon  is called the visibility graph {\em reconstruction} problem. 
Note that the general visibility graph recognition and reconstruction
problems are long-standing open problems with only partial results known
(e.g., see \cite{ref:AsanoVi00} for a short survey). Everett
\cite{ref:EverettVi90} showed that the visibility graph reconstruction
problem is in PSPACE, but no better upper bound on the complexity of
either problem is known. In our problem setting, we have the
angle data information and the ordered vertex list of ; thus
 can be constructed efficiently after knowing . 


Hence, the major part of the algorithm in \cite{ref:DisserRe10} is
dedicated to constructing the visibility graph  of .
As indicated in \cite{ref:DisserRe10}, the key difficulty 
is that the vertices in this
problem setting have no recognizable labels, e.g., the angle
measurement at a vertex  gives angles between visible vertices to
 but does not identify these visible vertices globally. 
The authors in \cite{ref:DisserRe10} also showed that some natural greedy
approaches do not seem to work. An  time algorithm for
constructing  is given in \cite{ref:DisserRe10}. The algorithm, 
called the {\em triangle witness algorithm}, is based
on the following observation: Suppose we wish to determine whether a
vertex  is visible to another vertex ; then  is
visible to  if and only if there is a vertex  on the portion of
the boundary of  from  to  in the CCW order such
that  is visible to both  and  and the triangle formed
by the three vertices , and  does not intersect the
boundary of 
except at these three vertices (such a vertex  is called a {\em triangle
witness vertex}). 

In this paper, based on the triangle witness algorithm
\cite{ref:DisserRe10}, by exploiting some new geometric properties, we
give an improved algorithm with a running time of . The 
improvement is due to two key observations. First, in the triangle
witness algorithm \cite{ref:DisserRe10}, to determine whether a vertex
 is visible to another vertex , the algorithm needs to
determine whether there exists a triangle witness vertex along the
boundary of  from  to  in the CCW order; to this
end, the algorithm checks every vertex in that boundary portion of 
. We observe that it suffices to check only one particular vertex in that
boundary portion. This removes an  factor from the running time of
the triangle witness algorithm \cite{ref:DisserRe10}. Second, some
basic operations in the triangle witness algorithm
\cite{ref:DisserRe10} take  time each; by utilizing certain
different  
data structures, our new algorithm can handle each of those
basic operations in constant time. 
This removes another  factor from the running time. Note that
since the input size is  in the worst case (e.g., the
total number of all visibility angles), our algorithm is worst-case optimal.  

As shown in \cite{ref:DisserRe10}, if only the angle measurements are
given, i.e., the ordered vertices along the
boundary of  are unknown, then the information is not sufficient for 
reconstructing . In other words, it may be possible to compute several simple
polygons that are not similar but all fit the given measured angles (see
\cite{ref:DisserRe10} for an example). 

\begin{figure}[t]
\begin{minipage}[t]{\linewidth}
\begin{center}
\includegraphics[totalheight=1.2in]{example.eps}
\caption{\footnotesize The given measured angles are ordered CCW along
the polygon boundary (left); reconstructing a simple polygon that fits these angles (right).
}\label{fig:example}
\end{center}
\end{minipage}
\end{figure}


\subsection{Related Work}

The problems of reconstructing geometric objects based on measurement
data have been studied extensively (e.g.,
\cite{ref:BiedlRe09,ref:BiloRe09,ref:JacksonOr02,ref:SidleskyPo06,ref:SnoeyinkCr99}).
As discussed above, the general visibility graph recognition and
reconstruction problems are in PSPACE \cite{ref:EverettVi90} and no better
complexity upper bound is known so far (e.g., see \cite{ref:EverettNe95}). 
Yet, some results have been given for certain special polygons. 
For example, Everett and
Corneil \cite{ref:ErerettRe90} characterized the visibility graphs of 
spiral polygons and
gave a linear time reconstruction algorithm. Choi, Shin, and Chwa
\cite{ref:ChoiCh95}, and Colley, Lubiw, and Spinrad
\cite{ref:ColleyVi97} characterized and
recognized the visibility graphs of funnel-shaped polygons. 

By adding extra information, some versions of the problems become more tractable. 
O'Rourke and Streinu \cite{ref:ORourkeTh98} considered the {\em
vertex-edge} visibility graph that includes edge-to-edge visibility
information. Wismath \cite{ref:WismathPo00} 
introduced the {\em stab graphs} which are
also an extended visibility structure and showed how parallel line
segments can be efficiently reconstructed from it. 
Snoeyink \cite{ref:SnoeyinkCr99} proved that a unique simple
polygon (up to similarity) can be determined by the interior angles at
its vertices and the cross-ratios of the diagonals of any given
triangulation, where the cross-ratio of a diagonal is the product of
the ratios of edge lengths for the two adjacent triangles. 
Jackson and Wismath \cite{ref:JacksonOr02} studied the reconstruction
of orthogonal polygons from horizontal and vertical visibility
information and gave an  time reconstruction algorithm. 
Biedl, Durocher, and Snoeyink \cite{ref:BiedlRe09} considered the problem of
reconstructing the two-dimensional floor plan of a polygonal room
using different types of scanned data, and proposed several problem models.
Sidlesky, Barequet, and Gotsman
\cite{ref:SidleskyPo06} studied the problem of reconstructing a planar
polygon from its intersections with a collection of
arbitrarily-oriented ``cutting" lines. 


Reconstructing a simple polygon from angle data was first
considered by Bil\`o {\em et al}.~\cite{ref:BiloRe09}, who aimed to understand what kinds of
sensorial capabilities are sufficient for a robot moving inside an unknown
polygon to reconstruct the visibility graph of the polygon. 
It was shown in \cite{ref:BiloRe09} that if the robot is equipped with a compass
to measure the angle between any two vertices that are
currently visible to the robot
and also has the ability to know where it came from when
moving from vertex to vertex, 
then the visibility graph of the polygon
can be uniquely reconstructed. Reconstruction and exploration of environments 
by robots in other problem settings have also been studied (e.g.,
see \cite{ref:DudekRo91,ref:FlocchiniTh99,ref:SuriSi08}). 

The rest of this paper is organized as follows. In Section \ref{sec:pre}, we 
give the problem definitions in detail and
introduce some notations and basic observations. 
To be self-contained, in Section \ref{sec:review}, we briefly review the
triangle witness algorithm given in \cite{ref:DisserRe10}. 
We then present our improved algorithm in Section \ref{sec:improve}. 

\sectionspace
\section{Preliminaries}
\label{sec:pre}
In this section, we define the \pra\ problem in detail and introduce
the needed notations and terminology. For ease of discussion and comparison, some of our
notations follow those in \cite{ref:DisserRe10}. 

Let  be a simple polygon of  vertices  in the CCW
order along 's boundary. Denote by  the visibility graph
of , where  consists of all vertices of  and for any two distinct
vertices  and ,  contains an edge  connecting  and
 if and only if  is visible to  inside . In this
paper, the indices of all 's are taken as congruent modulo ,
i.e., if , then  is the same vertex as , where
 (or ); similarly, if , then  is the same
vertex as , where . For each ,
denote by  its degree in the visibility graph , and denote by
 
the sequence of vertices in  visible to  from  to
 ordered CCW around . We refer to  as 's
{\em visibility angle sequence}. 
Note that since both  and  are visible to , 

and . For any two vertices  in
, let  denote the sequence
 of the vertices ordered CCW along the
boundary of  from  to . We refer to
 as a {\em chain}. Let  denote the
number of vertices of  in the chain . 


For any two distinct vertices , let  be the ray emanating from
 and going towards . For any three vertices , denote by
 the CCW angle defined by rotating
 around  to  ( or  need not be
visible to ). Note that the values
of all angles we use in this paper are in . For any
vertex  and , let  be
. 

The \pra\ problem can then be
re-stated as follows: Given a sequence of all vertices
 of an unknown simple polygon  in the CCW order along
's boundary, and the angles  for each vertex  of  with
, we seek to reconstruct  (up to
similarity) to fit all the given angles. 
Without loss of generality, we assume that no three distinct vertices of  are collinear.

It is easy to see that after  time preprocessing, for any  and any
, the angle  can be obtained
in constant time. In the following discussion, we assume that this
preprocessing has already been done. Sometimes we (loosely) say that these
angles are given as input.

The algorithm given in \cite{ref:DisserRe10} does not construct  directly.
Instead, the algorithm first computes its visibility graph . 
As mentioned earlier, after knowing ,  can be reconstructed efficiently with
the help of the angle data and the CCW ordered vertex sequence of . 
The algorithm for constructing  in \cite{ref:DisserRe10} is
called the {\em triangle witness algorithm}, which will be briefly reviewed
in Section \ref{sec:review}.
Since  consists of all vertices of , the
problem of constructing  is equivalent to constructing its edge set
, i.e., for any two distinct vertices , 
determine whether there is an edge  in  connecting 
and  (in other words,
determining whether  is visible to  inside ).


To discuss the involved algorithms, we need one more definition.
For any two vertices  with
, 
suppose a vertex  
is visible to both  and ; then we let  be the
{\em first} visible vertex to  on the chain  and let
 be the {\em last} visible vertex to  on the chain 
(e.g., see Fig.~\ref{fig:sangle}). Intuitively, imagine that we
rotate a ray from  around 
counterclockwise; then the first vertex on the chain
 hit by the rotating ray  is .
Similarly, if we rotate a ray from
 around  clockwise,
then the first vertex on the chain  hit by
the rotating ray is . 
Note that if  is visible to , then
 is  and  is .  We denote by 
 the angle  and denote
by  the angle . It should be 
pointed out that for ease of understanding this paper, the above statement of defining  
is different from that in \cite{ref:DisserRe10} but they refer to the
same angles in the algorithm. The motivation for
defining  will be clear after discussing the
following lemma, which has been proved in \cite{ref:DisserRe10}. 



\begin{figure}[t]
\begin{minipage}[t]{\linewidth}
\begin{center}
\includegraphics[totalheight=1.5in]{sangle.eps}
\caption{\footnotesize Illustrating the definitions of  and
, and  and
 which are the angles  and
, respectively. 
}\label{fig:sangle}
\end{center}
\end{minipage}
\end{figure}

\lemmaspace
\begin{lemma}\label{lem:10}\cite{ref:DisserRe10}
For any two vertices  with , 
 is visible
to  if and only if there exists a vertex  on
 such that  is visible to both  and
 and
. 
\end{lemma}
\lemmaspace

Since the above lemma is also critical to our improved algorithm in 
Section \ref{sec:improve}, we sketch the proof of the lemma below. 

For any two
vertices  with , if  is visible to
, then it is not difficult to see that 
there must exist a vertex  that is 
visible to both  and . Since the three vertices
, and  are mutually visible to each other, it is clear
that the triangle formed by these three vertices does not intersect the
boundary of  except at the three vertices, implying 
that . 
Such a vertex  is called a {\em triangle witness} of the edge
 in . 

Suppose  is not visible to ; then it is possible that there does not
exist a vertex  which is visible
to both  and . In fact, if there 
exists no vertex  that is visible
to both  and , then  cannot be visible to . Hence in
the following, we assume that such a vertex  exists, i.e., a
vertex  is visible to both  and  (yet  is not visible to ).
Note that  is not collinear with  and .
If  (i.e., the chain  is part of the
boundary of  that blocks
the visibility between  and ), then the lemma obviously holds.  Otherwise,
, and in this case,
the visibility between  and  is blocked
by the chain .
Since  is not visible to , for any choice of such a vertex , 
the angle 
is not given by 's visibility angle sequence . 
The ``closest approximation" for  in this case is determined by
a vertex  on the chain  such that  becomes  if and
only if  is visible to .
As in the definition of
, the vertex  is such a vertex , i.e., the
first visible vertex to  on the chain .
Similarly, the
vertex  in  is ``replaced" in the definition
of  by , i.e.,
the last visible vertex to  on .
Clearly, when  is not visible to , it must hold that 
 and
. Therefore,  
.
Lemma \ref{lem:10} thus follows. 

\sectionspace
\section{The Triangle Witness Algorithm}
\label{sec:review}
In this section, we briefly review the triangle witness algorithm in \cite{ref:DisserRe10} that
constructs the visibility graph  of the unknown simple polygon . 

The triangle witness algorithm is based on Lemma \ref{lem:10}. The
algorithm has  iterations. In the -th iteration
(),
the algorithm checks, for each , whether  is
visible to . After all iterations, the edge set  
 can be obtained. To this end, the algorithm maintains two maps 
and :  if  is identified as 
the -th visible vertex to  
in the CCW order, i.e., ; the definition of  is
the same as . During the algorithm,  will be filled in the CCW
order and  will be filled in the clockwise (CW) order. When the algorithm 
finishes, for each ,  will have all visible vertices to  on
the chain  while  will have
all visible vertices to  on the chain . Thus,  and  together contain all visible
vertices of  to . 
For ease of description, we also treat  and
 as sets, e.g.,  means that there is an entry
 and  means the number of entries in the
current .  

Initially, when , since every
vertex is visible to its two neighbors along the boundary of , we have 
and  for each . In the -th
iteration, we determine for each , whether  is visible to
. Below, we let . 
Note that  is the index of the first visible
vertex to  in the CCW order that is not yet identified;
similarly,  is the index of the first
visible vertex to  in the CW order that is not yet
identified. If  is visible to , then we know that
 is the -th visible vertex to  and 
is the -th visible vertex to ,
and thus we set  and
. If  is not visible to
, then we do nothing. 

It remains to discuss how to determine whether  is visible to
. According to Lemma \ref{lem:10}, we need to determine
whether there exists a triangle witness vertex  in the chain 
, i.e.,  is visible to both  and  and 
. 
To this end, the algorithm checks every vertex  in
. For each , the algorithm first determines 
whether  is visible to both
 and , by checking whether there is an entry for  in 
and in . The algorithm utilizes balanced binary search trees
to represent  and , and thus checking whether there is an entry
for  in  and in  can be done in  time. If  is
visible to both  and , then the next step is to determine whether
. 
It is easy to know that  is 
, which can be found readily from the input.
To obtain , we claim that it is the angle
. Indeed, observe that all visible
vertices in the chain  are in the current
. As explained above,  is the index of the first visible  
vertex to  in the CCW order that has not yet been identified, which is
the first visible vertex to  in the chain , i.e, the
vertex . Thus,  is
, which is .
Similarly,  is the angle
. Note that both the angles
 and

are known. Algorithm \ref{algo:10} in the Appendix summarizes the whole 
algorithm \cite{ref:DisserRe10}.

To analyze the running time of the above triangle witness algorithm,
note that it has  iterations. In each iteration, the
algorithm checks whether  is visible to  for
each . For each , the algorithm checks every 
for , i.e, in the chain . 
For each such , the algorithm takes  time as it 
uses balanced binary search trees to represent the two maps  and . 
In summary, the overall running time of the triangle witness algorithm in
\cite{ref:DisserRe10} is . 

\sectionspace
\section{An Improved Triangle Witness Algorithm}
\label{sec:improve}

In this section, we present an improved solution over the triangle
witness algorithm in \cite{ref:DisserRe10} 
sketched in Section \ref{sec:review}. Our improved
algorithm runs in  time. Since the input size (e.g., the total
number of visibility angles) is  in the worst case, our improved
algorithm is worst-case optimal. Our new algorithm follows the high-level 
scheme of the triangle witness algorithm in \cite{ref:DisserRe10}, and thus
we call it the {\em improved triangle witness algorithm}. 

As in \cite{ref:DisserRe10}, the new algorithm also has 
iterations, and in each iteration, we determine 
whether  is visible to  for each .
For every pair of vertices  and , let . To
determine whether 
is visible to , the triangle witness algorithm \cite{ref:DisserRe10}
checks each vertex  in the chain  to 
see whether there exists a triangle witness vertex. 
In our new algorithm, instead, we claim that we 
need to check only one particular vertex, , i.e., the
last visible vertex to  in the chain  in the
CCW order, as stated in the following lemma. 


\begin{figure}[t]
\begin{minipage}[t]{\linewidth}
\begin{center}
\includegraphics[totalheight=1.5in]{visible.eps}
\caption{\footnotesize Illustrating Lemma \ref{lem:20}:  is
, the
last visible vertex to  in ;  is
visible to  if and only if  is a triangle witness vertex.
}\label{fig:visible}
\end{center}
\end{minipage}
\end{figure}

\begin{figure}[t]
\begin{minipage}[t]{\linewidth}
\begin{center}
\includegraphics[totalheight=1.5in]{visPolygon.eps}
\caption{\footnotesize Illustrating the visibility polygon  
for the example in Fig.~\ref{fig:visible}: The gray (pink) area is
, which is a star-shaped polygon with the vertex  as a kernel point. 
}\label{fig:visPolygon}
\end{center}
\end{minipage}
\end{figure}

\begin{figure}[t]
\begin{minipage}[t]{\linewidth}
\begin{center}
\includegraphics[totalheight=1.5in]{boundary.eps}
\caption{\footnotesize Illustrating  for the example in
Fig.~\ref{fig:visible}: The thick line segments form . 
}\label{fig:boundary}
\end{center}
\end{minipage}
\end{figure}


\lemmaspace
\begin{lemma}\label{lem:20}
The vertex  is visible to  if and only if the vertex
 is a triangle witness vertex for  and . 
\end{lemma}
\lemmaspace
\begin{proof}
Let  denote the vertex , which is the last visible vertex
to  in the chain  in the CCW order.
Recall that  being a triangle witness vertex for  and 
is equivalent to saying that  is visible to both  and  and 
. 


If  is visible to , then we prove below that  is a triangle
witness vertex for  and , i.e., we prove that  is visible to both  and
 and
. 
Refer to Fig.~\ref{fig:visible} for an example. 
Let  denote the subpolygon of  that is visible to
the vertex . Usually,  is called the {\em visibility
polygon} of  and it is well-known that  is
a star-shaped polygon with  as a kernel point
(e.g., see \cite{ref:AsanoVi00}). Figure~\ref{fig:visPolygon}
illustrates .
Since  is the last visible vertex to  in
 and  is visible to , we claim that
 is also visible to . Indeed, since both  and  are visible to , 
and  are both on the boundary of the visibility polygon .  
Let  denote the portion of the boundary of  from  to
 counterclockwise (see Fig.~\ref{fig:boundary}). 
We prove below that  does not contain
any vertex of  except  and .
First,  cannot contain any vertex in the chain
 (otherwise,  would not be visible to
). Let the index of the vertex  be , i.e.,
. Similarly,  cannot contain any vertex in the chain
 (otherwise,  would not be visible to
). Finally,  cannot contain any vertex in the chain
, since otherwise  would not be the last
visible vertex to  in . Thus,  does
not contain any vertex of  except  and . Therefore, the
region bounded by , the line segment connecting  and
, and the line segment connecting  and  must be convex (in fact,
it is always a triangle) and this region is entirely contained in .  This
implies that  is visible to . 
Hence, the three vertices , and  are mutually visible to each
other, and we have 
. 

On the other hand, if  is a triangle witness vertex 
for  and , then by Lemma
\ref{lem:10}, the vertex  is visible to . The lemma thus
follows.  
\end{proof}

By Lemma \ref{lem:20}, to determine whether  is visible to
, instead of checking every vertex in , we
need to consider only the vertex  in the current set . 
Hence, Lemma \ref{lem:20} immediately reduces the running time of the triangle
witness algorithm by an  factor. The other  factor
improvement is due to a new way of defining and representing the maps  and
, as elaborated below. In the following discussion.
let  be the vertex . 

In our new algorithm, to determine whether  is visible
to , we check whether  is a triangle witness vertex for  and .
To this end, we already
know that  is visible to , but we still need to check whether 
is visible to . In the previous triangle witness algorithm
\cite{ref:DisserRe10}, this
step is performed in  time by representing  and  using
balanced binary search trees. In our new algorithm, we handle this
step in  time, by redefining  and  and using a new way to
represent them. 

We redefine  as follows: 
if  is the -th visible vertex to  in the CCW order; if
 is not visible to  or  has not yet been identified, 
then . For convenience, we let
. 
Thus, in our new definition, the size of
 is fixed throughout the algorithm, i.e.,  is always
. In
addition, for each , the new algorithm maintains two variables
 and  for , where  is the number of non-zero
entries in the current , which is also the number of visible
vertices to  that have been identified 
(i.e., the number of visible vertices to  
in the chain ) up to the -th iteration, 
and  is the index  
of the last non-zero entry in the current , i.e.,  is the
last visible vertex to  in the chain  in
the CCW order. 
Similarly, we redefine  in the same way as , i.e., for
each ,  and 
if  is the -th visible vertex to  in the CCW order.  
Further, for each , we also maintain 
two variables  and  for , where  is the
number of non-zero entries in the current , which is also the
number of visible vertices to  in the chain
 (up to the -th iteration), and  is the
index of the first non-zero entry in the current , i.e.,  
is the first visible vertex to  in the chain
 in the CCW order. During the
algorithm, the array  will be filled in the CCW order, i.e., 
from the first entry 
 to the end while the array  will be filled
in the CW order, i.e, from the last entry  to the
beginning. When the algorithm finishes,  will 
contain all the visible vertices to  in the chain
, and thus only the entries of the first half
of  are possibly filled with non-zero values.
Similarly, only the entries of the second half of  are possibly
filled with non-zero values. Below, we discuss the implementation details of
our new algorithm, which is summarized in Algorithm \ref{algo:20}. 


\begin{algorithm}[t]
\caption{The Improved Triangle Witness Algorithm}
\label{algo:20}
\KwIn{, , , and the angles
 for each .}
\KwOut{The edge set  of the visibility graph  for the target polygon .}
\BlankLine
\tcc{All indices below are understood modulo .}
\;
\For{ \KwTo }{
\;
\;
}
\For{ \KwTo }{
\;
\;
\;
\;
}

\For{ \KwTo }{
\For{ \KwTo }{
\;
\;
\If{}{
\;
\;
\;
}
\If{}{
\;
\;
\;
, , ,
\;
}
}
}
\end{algorithm}

Initially, when , for each , we set  and
, and set all other entries
of  and  to zero. In addition, we set
,  and , .  In the -th
iteration, with , for each , 
we check whether  is visible to ,
with . If  is not visible to , then we do nothing.
Otherwise, we set  and increase  by one;
similarly, we set  and increase  by one.
Further, we set  and . 

It remains to show how to check whether  is visible to . 
By Lemma \ref{lem:20}, we need to determine whether
 is a triangle witness vertex for  and . Since  is the last visible
vertex to  in the chain  in the CCW order, 
based on our definition,  is the vertex . After knowing
, we then check whether  is visible to , which can be
done by checking whether  is zero, in constant time. 
If  is
zero, then  is not visible to  and  is not a triangle
witness vertex; otherwise,  is visible to
. (Note that we can also check whether  is zero.)
In the following, we assume that  is visible to . The next step is
to determine whether  
.
To this end, we must know the involved three angles. Similar
to the discussion in Section \ref{sec:review}, we have 
, 
, and
. 
Thus, all these three angles can be
obtained in constant time. Hence, the step of checking whether 
is visible to  can be performed in constant time, which reduces
another  factor from the running time of the previous
triangle witness algorithm in \cite{ref:DisserRe10}.  
Algorithm \ref{algo:20} summarizes the whole algorithm. Clearly, the running
time of our new algorithm is bounded by . 

\lemmaspace
\begin{theorem}\label{theo:10}
Given the visibility angles and an ordered vertex sequence of a simple
polygon , the improved triangle witness algorithm can reconstruct
 (up to similarity) in  time. 
\end{theorem}
\lemmaspace

As discussed in \cite{ref:DisserRe10}, the above algorithm can also be used
to determine whether the input angle data are consistent. Namely, if
there is no polygon that can fit the input angle data, then 
the algorithm in Theorem \ref{theo:10} can be used to report this
case as well.



\begin{thebibliography}{10}

\bibitem{ref:AsanoVi00}
T.~Asano, S.K. Ghosh, and T.~Shermer.
\newblock chapter 19: Visibility in the plane, in {\em Handbook of
  Computational Geometry}, J. Sack and J. Urrutia (eds.), pages 829--876.
\newblock Elsevier, Amsterdam, The Netherlands, 2000.

\bibitem{ref:BiedlRe09}
T.~Biedl, S.~Durocher, and J.~Snoeyink.
\newblock Reconstructing polygons from scanner data.
\newblock In {\em Proc. of the 20th International Symposium on Algorithms and
  Computation}, volume 5878 of {\em Lecture Notes in Computer Science}, pages
  862--871. Springer-Verlag, 2009.

\bibitem{ref:BiloRe09}
D.~Bil\`o, Y.~Disser, M.~Mihal\'ak, S.~Suri, E.~Vicari, and P.~Widmayer.
\newblock Reconstructing visibility graphs with simple robots.
\newblock In {\em Proc. of the 16th International Colloquium on Structural
  Information and Communication Complexity}, volume 5869 of {\em Lecture Notes
  in Computer Science}, pages 87--99. Springer-Verlag, 2009.

\bibitem{ref:ChoiCh95}
S.-H. Choi, S.Y. Shin, and K.-Y. Chwa.
\newblock Characterizing and recognizing the visibility graph of a
  funnel-shaped polygon.
\newblock {\em Algorithmica}, 14(1):27--51, 1995.

\bibitem{ref:ColleyVi97}
P.~Colley, A.~Lubiw, and J.~Spinrad.
\newblock Visibility graphs of towers.
\newblock {\em Computational Geometry: Theory and Applications}, 7(3):161--172,
  1997.

\bibitem{ref:DisserRe10}
Y.~Disser, M.~Mihal\'ak, and P.~Widmayer.
\newblock Reconstructing a simple polygon from its angles.
\newblock In {\em Proc. of the 12th Scandinavian Symposium and Workshops on
  Algorithm Theory (SWAT)}, volume 6139 of {\em Lecture Notes in Computer
  Science}, pages 13--24. Springer, 2010.

\bibitem{ref:DudekRo91}
G.~Dudek, M.~Jenkins, E.~Milios, and D.~Wilkes.
\newblock Robotic exploration as graph construction.
\newblock {\em IEEE Transactions on Robotics and Automation}, 7(6):859--865,
  1991.

\bibitem{ref:EverettVi90}
H.~Everett.
\newblock {\em {Visibility Graph Recognition}}.
\newblock PhD thesis, University of Toronto, Toronto, 1990.

\bibitem{ref:ErerettRe90}
H.~Everett and D.~Corneil.
\newblock Recognizing visibility graphs of spiral polygons.
\newblock {\em Journal of Algorithms}, 11(1):1--26, 1990.

\bibitem{ref:EverettNe95}
H.~Everett and D.~Corneil.
\newblock Negative results on characterizing visibility graphs.
\newblock {\em Computational Geometry: Theory and Applications}, 5(2):51--63,
  1995.

\bibitem{ref:FlocchiniTh99}
P.~Flocchini, G.~Prencipe, N.~Santoro, and P.~Widmayer.
\newblock Hard tasks for weak robots: {The} role of common knowledge in pattern
  formation by autonomous mobile robots.
\newblock In {\em Proc. of the 10th International Symposium on Algorithms and
  Computation}, pages 93--102, 1999.

\bibitem{ref:JacksonOr02}
L.~Jackson and S.~Wismath.
\newblock Orthogonal polygon reconstruction from stabbing information.
\newblock {\em Computational Geometry: Theory and Applications}, 23(1):69--83,
  2002.

\bibitem{ref:ORourkeTh98}
J.~O'Rourke and I.~Streinu.
\newblock The vertex-edge visibility graph of a polygon.
\newblock {\em Computational Geometry: Theory and Applications},
  10(2):105--120, 1998.

\bibitem{ref:SidleskyPo06}
A.~Sidlesky, G.~Barequet, and C.~Gotsman.
\newblock Polygon reconstruction from line cross-sections.
\newblock In {\em Proc of the 18th Canadian Conference on Computational
  Geometry}, pages 81--84, 2006.

\bibitem{ref:SnoeyinkCr99}
J.~Snoeyink.
\newblock Cross-ratios and angles determine a polygon.
\newblock {\em Discrete and Computational Geometry}, 22:619--631, 1999.

\bibitem{ref:SuriSi08}
S.~Suri, E.~Vicari, and P.~Widmayer.
\newblock Simple robots with minimal sensing: {From} local visibility to global
  geometry.
\newblock {\em International Journal of Robotics Research}, 27(9):1055--1067,
  2008.

\bibitem{ref:WismathPo00}
S.~Wismath.
\newblock Point and line segment reconstruction from visibility information.
\newblock {\em International Journal of Computational Geometry and
  Applications}, 10(2):189--200, 2000.

\end{thebibliography}



\newpage
\normalsize
\appendix
\section*{Appendix}

\begin{algorithm}
\caption{The Triangle Witness Algorithm \cite{ref:DisserRe10}}
\label{algo:10}
\KwIn{, , , and the angles
 for each .}
\KwOut{The edge set  of the visibility graph  for the target polygon .}
\BlankLine
\tcc{All indices below are understood modulo .}
\; 
\;
\;
\For{ \KwTo }{
\;
\;
\;
}
\For{ \KwTo }{
\For{ \KwTo }{
\;
\For{ \KwTo }{
\If{ and }{
\;
\;
\;
}
\If{}{
\;
\;
\;
abort the innermost loop\;
}
}
}
}
\end{algorithm}









\end{document}
