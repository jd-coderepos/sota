

\chapter{Overall Description}

In this chapter we provide an overall insight of the general factors
that affect the SFS system and its requirements.

\section{Product Perspective}

The SFS system is intended to operate in a distributed environment:
clients machines, application server, database server, and LDAP
server. SFS is accessed via secure connections we intend to provide
in this work. The system's user can be either a normal user or an
administrator. A normal user has the ability to upload, download,
delete and view files. An administrator is able to: Upload,
download, delete and view files; add, delete and modify users;
generate user certificates along with all required information;
generate an ACL to each and every user; manage groups; perform
various maintenance actions such as: check log files, delete files,
etc.

\subsection{System interfaces}

The various parts of the SFS application will be installed on client
machines and different servers. The client that uses the system has
to have the certificate installed on his machine to provide client
authentication. The servers are responsible of one of the following
functions: provide the database, provide the LDAP server
functionalities, and provide the application server for different
clients.

\subsection{User Interfaces}

The user interfaces consist of web-based graphical components that
allow the user to interact with the SFS system. The user will use a
web browser to send and receive data. If the user is the
administrator, s/he will have more options to add, delete, etc.
users, generate users certificates, generate ACL for each user, etc.

\subsection{Hardware Interfaces}

The hardware interfaces will be achieved through the abstraction
layer of the Java Virtual Machine (JVM). The keyboard and the mouse
are examples of such hardware interfaces that allow users to
interact with the SFS system.

\subsection{Software Interfaces}

Among the most important software interfaces used in this project,
we have:
\begin{itemize}

\item The SFS  system is OS-independent due to the cross platform Java
implementation. It will support web browsers such as Internet
Explorer, Mozilla Fireworks, etc.

\item Access to databases will be provided by JDBC 3 on both Windows
and Linux environment.

\item JXplorer? will be used to provide a graphical access to LDAP server.

\item OpenLDAP? is used to host users' certificates.

\item Java 1.5 JDK from Sun.

\item JRE 1.5 from Sun.

\item Servlets for client and administrator interfaces.

\item Apache Tomcat5 server as the web server used in this project.

\item PostegreSQL? is the database used to host users information, files,
etc.

\item OpenSSL toolkit to generate the certificates for users.

\end{itemize}

Hereafter, we provide the software and documentation's locations
related to these interfaces:

\begin{itemize}

\item OpenLDAP? software and documentation found at: \emph{http://www.openldap.org/}

\item PostgreSQL database and documentation found at: \emph{http://www.postgresql.org/}

\item  Java development kit 1.5 available at: \emph{http://java.sun.com/j2se/}

\item JSP documentation found at: \emph{http://java.sun.com/products/jsp/}

\item OpenSSL  Toolkit found at: \emph{http://www.openssl.org/}

\item Apache Tomcat 5.0 web server found at: \emph{http://tomcat.apache.org/}

\end{itemize}

\section{Product Functions}

The SFS system will implement the following functionalities:

\begin{itemize}

\item \emph{Server authentication:} This use case allow
the user to authenticate the web server is connecting to.

\item \emph{Client authentication:} This is use case allow the server
to authenticate the user he is trying to connect to.

\item \emph{Secure communication:} Between users over the network.

\item \emph{Files handling:} Such as downloading, uploading, and
deletion files.

\item Users management: Administrator has the ability to add and delete users, add and delete groups, and assign users to groups.

\end{itemize}



\chapter{Specific Requirements}

In this section we describe the software requirements to design the
SFS system. The system design should satisfy the following
requirements.

\section{Functional Requirements} Hereafter, we express the
expectations in terms of system functions and constraints. This
includes the domain model and the most important use case diagrams
of the SFS system.

\subsection{Domain Model}

The SFS system domain model consists of many packages.

\begin{itemize}


\item \emph{Client authentication module:} used to provide users the
ability to authenticate the server.

\item \emph{Server authentication module:} used to provide the server the
ability to authenticate the clients.

\item \emph{LDAP connection module:} used to provide connection to the
LDAP server in order to check clients' credentials.

\item \emph{Database connection module:} used to provide users the
ability to connect to the database server in a secure mode.

\end{itemize}

\begin{figure}[htbp]
\begin{center}
  \fbox{
      \scalebox{0.5}{  \includegraphics{images/packages}
      }
    }
    \caption{\label{fig1} SFS system packages}
\end{center}
\end{figure}

\subsection{Use Case Model}

The SFS system consists of a set of use cases manageable by the
users of the application. There are two types of users of the
system: normal users and administrator. Normal can view, delete,
download, and upload files. For the administrator, s/he can: add,
delete, etc. users; generate users' certificates; generate ACL for
each user, etc.

The diagram in Figure \ref{useuse} shows the capabilities of a
normal user.

\begin{figure}[htbp]
\begin{center}
  \fbox{
      \scalebox{0.4}{  \includegraphics{images/usecase1}
      }
    }
    \caption{\label{useuse} Normal user use-case}
\end{center}
\end{figure}

\newpage
The diagram in Figure \ref{servuse} shows the capabilities of the
administrator user.

\begin{figure}[htbp]
\begin{center}
  \fbox{
      \scalebox{0.45}{  \includegraphics{images/usecaseadmin}
      }
    }
    \caption{\label{servuse} Administrator use-case}
\end{center}
\end{figure}



\section{Software System Attributes}

There are a number of software attributes that can serve as
requirements.  It is important that required attributes by specified
so that their achievement can be objectively verified. The following
items are some of the most important ones: security, reliability,
availability, maintainability, and portability.

\subsection{Security}

Security is the most important attribute of the SFS design and
implementation. The mutual authentication between the server and
clients is crucial for the system use. The system should be able to
authenticate users and differentiate among them, either are normal
users or administrator. In order to achieve the security feature
expected from the SFS system, the following tasks have to be
realized:

\begin{itemize}

\item Utilize cryptographic techniques

\item Check users' credentials before using the system and accessing the database

\item Provide secure communications between different parts of the
system

\end{itemize}

\subsection{Reliability}

The basis for the definition of reliability is the probability that
a system will fail during a given period. The reliability of the
whole system depends on the reliability of its components and on the
reliability of the communication between its components. The SFS
system is based mainly on some standard components such as OpenLDAP,
OpenSSL, JDBC, PostgreSQL, etc. The reliability of these service
components is already proved. This fact improves the reliability of
the system and restricts the proof work on assuring only of the
reliability of the added components and the communications between
the different components. In addition, the system must ensure the
security of the communications which is the most important issue of
the SFS system.


\subsection{Availability}

The SFS system must be able to work continuously in order to provide
users with an access to different server's parts of the system.
However, since this system depends on distributed information
systems and databases, many constraints should be taken into account
such as:

\begin{itemize}

\item The connection to the web server that provides access to the
system

\item The interconnection between different parts of the system should
always be available; otherwise, the users cannot complete their
tasks using the system.

\item The database should be available in the database server side

\item The LDAP server should be always available in order to check
users' credentials

\item The web server should be also available in order to allow users
connecting to the system.

\end{itemize}

\subsection{Maintainability}

Maintainability is defined as the capacity to undergo repairs and
modifications. The main goal in designing SFS system is to keep it
easy to be modified and extended.

\subsection{Portability}

The portability is one of the main specifications of Java. Since SFS
is implemented using the Java programming language, the portability
is automatically satisfied and the system is able to run on any
machine or operating system which supports the execution of a Java
virtual machine.

\section{Logical Database Requirements}

The rationale behind SFS system is to provide secure connections for
users accessing databases to view, delete, upload, and download
files through a web server and LDAP server. After analyzing the
requirements we propose using a relational database model to meet
our requirements. This database is required to store information
about users, files, groups of users, etc.  The database is expected
to work on 24 hours and 7 days in order to provide nonstop access to
the users. Therefore, backup of the database should be taken
periodically (daily or weekly. The relational database itself
guarantees the flexibility, simplicity and elimination of redundancy
once designed carefully. The entity relationship model will be
elaborated in detail in the database design of the design part, in
this document.

