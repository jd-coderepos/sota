\let\accentvec\vec
\documentclass[a4paper]{llncs}
\let\spvec\vec
\let\vec\accentvec
\usepackage{amsmath,amssymb,wasysym,bbm,url,ifthen,calc,hyperref}




\let\phi\varphi
\let\epsilon\varepsilon
\let\theta\vartheta
\let\mathbb\mathbbm

\providecommand{\abs}[1]{\lvert#1\rvert}
\providecommand{\norm}[1]{\lVert#1\rVert}

\newcommand{\NN}{\mathbb{N}}
\newcommand{\ZZ}{\mathbb{Z}}
\newcommand{\QQ}{\mathbb{Q}}
\newcommand{\RR}{\mathbb{R}}
\newcommand{\CC}{\mathbb{C}}

\renewcommand{\ker}{\operatorname{Ker}}
\newcommand{\im}{\operatorname{Im}}
\newcommand{\orth}[1]{#1^\perp}
\newcommand{\biorth}[1]{#1^{\perp\perp}}

\newcommand{\diag}{\operatorname{diag}}
\newcommand{\mat}[3]{#1^{#2}_{#3}}
\newcommand{\trp}[1]{#1^\intercal}
\newcommand{\trpst}[1]{#1^{\intercal*}}
\newcommand{\finset}{\smash{\NN^\circledast}}
\newcommand{\pow}{\mathcal{P}}
\newcommand{\Ev}{\operatorname{Ev}}

\newcommand{\der}{D}
\newcommand{\cum}{{\textstyle \varint}}
\newcommand{\fri}[1]{#1^\Diamond}
\newcommand{\trc}{\operatorname{trc}}

\newcommand{\galg}{\mathcal{F}}
\newcommand{\ogalg}{\mathcal{G}}
\newcommand{\oogalg}{\mathcal{H}}
\newcommand{\func}[2]{\galg^{#1}_{#2}}
\newcommand{\bspc}{\mathcal{B}}
\newcommand{\bvp}[2]{\boxed{\begin{array}{l}#1\\#2\end{array}}}

\spnewtheorem{myexample}[theorem]{Example}{\bfseries}{\upshape}
\spnewtheorem{mydefinition}[theorem]{Definition}{\bfseries}{\upshape}
\spnewtheorem{mylemma}[theorem]{Lemma}{\bfseries}{\itshape}
\spnewtheorem{myproposition}[theorem]{Proposition}{\bfseries}{\itshape}
\newenvironment{myproof}{\begin{proof}}{\qed\end{proof}}




\title{A Symbolic Approach to Boundary Problems for Linear Partial Differential
  Equations} \subtitle{Applications to the Completely Reducible Case of the
  Cauchy Problem with Constant Coefficients} \titlerunning{Symbolic Theory of
  Boundary Problems with Applications} \author{Markus Rosenkranz \and Nalina
  Phisanbut\thanks{The authors acknowledge support from the EPSRC First Grant
    EP/I037474/1.}\\
  \small\email{\{M.Rosenkranz,N.Phisanbut\}@kent.ac.uk}} \institute{University
  of Kent, Canterbury, Kent CT2 7NF, United Kingdom}



\begin{document}
\maketitle
\thispagestyle{plain}
\pagestyle{plain}

\begin{abstract}
  We introduce a general algebraic setting for describing linear boundary
  problems in a symbolic computation context, with emphasis on the case of
  partial differential equations. The general setting is then applied to the
  Cauchy problem for completely reducible partial differential equations with
  constant coefficients. While we concentrate on the theoretical features in
  this paper, the underlying operator ring is implemented and provides a
  sufficient basis for all methods presented here.
\end{abstract}

\section{Introduction}
\label{sec:intro}

A symbolic framework for boundary problems was built up
in~\cite{Rosenkranz2005,RosenkranzRegensburger2008a} for linear ordinary
differential equations (LODEs); see
also~\cite{RosenkranzRegensburgerTecBuchberger2012,KorporalRegensburgerRosenkranz2011,KorporalRegensburgerRosenkranz2012}
for more recent developments. One of our long-term goals is to extend this to
boundary problems for linear partial differential equations (LPDEs). Since this
is a daunting task in full generality, we want to tackle it in stages of
increasing generality. In the first instance, we restrict ourselves to
\emph{constant coefficients}, where the theory is quite
well-developed~\cite{Hormander1976}. Within this class we distinguish the
following three stages:
\begin{enumerate}
\item The simplest is the Cauchy problem for \emph{completely reducible}
  operators.
\item The next stage will be the \emph{Cauchy problem} for general hyperbolic
LPDEs.
\item After that we plan to study \emph{boundary problems} for
  elliptic/parabolic LPDEs.
\end{enumerate}

In this paper we treat the first case (Section~\ref{sec:cauchy-analytic}). But
before that we build up a \emph{general algebraic framework}
(Sections~\ref{sec:boundary-data} and~\ref{sec:greens-operators}) that allows a
symbolic description for all boundary problems (LPDEs/LODEs, scalar/system,
homogeneous/inhomogeneous, elliptic/hyperbolic/parabolic). Using these concepts
and tools we develop a general solution strategy for the Cauchy problem in the
case~(1). See the Conclusion for some thoughts about the next two steps.

\noindent The passage from LODEs to LPDEs was addressed at two earlier occasions:
\begin{itemize}
\item An \emph{abstract theory of boundary problems} was developed
  in~\cite{RegensburgerRosenkranz2009}, including LODEs and LPDEs as well as
  linear systems of these. The concepts and results of
  Sections~\ref{sec:boundary-data} and~\ref{sec:greens-operators} are built on
  this foundation, adding crucial concepts whose full scope is only appreciated
  in the LPDE setting: boundary data, semi-homogeneous problem, state operator.
\item An algebraic language for \emph{multivariate differential and integral
    operators} was introduced in Section~4
  of~\cite{RosenkranzRegensburgerTecBuchberger2009}, with a prototype
  implementation described in Section~5 of the same paper. This language is
  generalized in the PIDOS algebra of Section~\ref{sec:cauchy-analytic}, and it
  is also implemented in a Mathematica package.
\end{itemize}

In this paper we will not describe the current state of the
\emph{implementation} (mainly because of space limitations). Let us thus say a
few words about this here. A complete reimplementation of the PIDOS package
described in~\cite{RosenkranzRegensburgerTecBuchberger2009} is under way. The
new package is called OPIDO (Ordinary and Partial Integro-Differential
Operators), and it is implemented as a standalone Mathematica package unlike its
predecessor, which was incorporated into the THEOREMA system. In fact, our
reimplementation reflects several important design principles of THEOREMA,
emphasizing the use of functors and a strong support for modern two-dimensional
(user-controllable) parsing rules. We have called this programming paradigm
FUNPRO, first presented at the Mathematica symposium~\cite{Rosenkranz2012}. The
last current stable version of the (prototype) package can be found
at~\url{http://www.kent.ac.uk/smsas/personal/mgr/index.html}.

At the time of writing, the ring of \emph{ordinary integro-differential
  operators} is completed and the ring of \emph{partial integro-differential
  operators} is close to completion (for two independent variables). Compared
to~\cite{RosenkranzRegensburgerTecBuchberger2009}, the new PIDOS ring contains
several crucial new rewrite rules (instances of the substitution rule for
resolving multiple integrals). Our conjecture is that the new rewrite system is
noetherian and confluent but this issue will be analyzed at another occasion.

\emph{Notation}. The algebra of~ matrices over a field~ is
written as~, where~ or~ is omitted. Thus we
identify~ with the space of column vectors
and~ with the space of row vectors. More generally, we
have~.


\section{An Algebraic Language for Boundary Data}
\label{sec:boundary-data}


As mentioned in the Introduction, we follow the \emph{abstract setting}
developed in~\cite{RegensburgerRosenkranz2009}. We will motivate and
recapitulate some key concepts here, but for a fuller treatment of these issues
we must refer the reader to~\cite{RegensburgerRosenkranz2009} and its references.

Let us recall the notion of boundary problem. Fix vector
spaces~ and~ over a common ground field~
of characteristic zero (for avoiding trivialities one may
assume~ and~ to be
infinite-dimensional). Then a \emph{boundary problem}~
consists of an epimorphism~ and a
subspace~ that is orthogonally closed in
the sense defined below. We call~ the differential operator
and~ the \emph{boundary space}.

Similar to the correspondence of ideals/varieties in algebraic geometry, we make
use of the following Galois
connection~\cite[A.11]{RegensburgerRosenkranz2009}. If~ is any
subspace of the space~, its \emph{orthogonal}  is defined as~a
  \in \mathcal{A}. Dually, for a subspace~ of the dual
space~, the orthogonal~ is defined by~\beta \in \mathcal{B}. If we
think of~ as ``functions'' and of~ as ``boundary conditions'',
then~ is the space of valid conditions (the boundary
conditions satisfied by the given functions) while~ is the
space of admissible functions (the functions satisfying the given conditions).

Naturally, a subspace of either~ of~ or~ is called
\emph{orthogonally closed} if . But while
any subspace of~ itself is always orthogonally closed, this is far from
being the case of the subspaces of the dual~. Hence the condition on
boundary spaces~ to be orthogonally closed is in general not
trivial. However, if~ is finite-dimensional as in boundary problems
for LODEs (as in Example~\ref{ex:heat-conduction} below), then it is
automatically orthogonally closed. For LPDEs, the condition of orthogonal
closure is important; see Example~\ref{ex:wave-inhom-medium} for an intuitive
explanation.

In~\cite{Rosenkranz2005,RosenkranzRegensburger2008a} and also in the abstract
setting of~\cite{RegensburgerRosenkranz2009} we have only considered what is
sometimes called the \emph{semi-inhomgeneous boundary
  problem}~\cite{Stakgold1979}, more precisely the semi-inhomogeneous
incarnation of~; see Definition~\ref{def:transfer-operator}
for the full picture. This means we are given a \emph{forcing function}  and we search for a solution~ with

In other words,  satisfies the inhomogeneous ``differential equation''  and the homogeneous ``boundary conditions''  given
in~.

A boundary problem which admits a unique solution~ for every
forcing function~ is called \emph{regular}. In terms of the
spaces, this condition can be expressed equivalently by requiring that~; see~\cite{RegensburgerRosenkranz2009} for
further details. In this paper we shall deal exclusively with regular boundary
problems. For singular boundary problems we refer the reader
to~\cite{KorporalRegensburgerRosenkranz2011} and~\cite{Korporal2012}.

For a regular boundary problem, one has a linear operator~ sending~ to~ is known as the \emph{Green's operator} of the
boundary problem~. From the above we see that~ is
characterized by~ and~.

\begin{myexample}
  \label{ex:heat-conduction}
  A classical example of this notion is the two-point boundary problem. As a
  typical case, consider the simplified model of \emph{stationary heat
    conduction} described by

Here we can choose~ for the function space
  such that the differential operator is given by  and the boundary space by the two-dimensional subspace
  of~ spanned by the linear functionals~ and~ for evaluation on the left and right
  endpoint. In the sequel we shall write~, employing an
  important generalization for LPDEs, described in the next eamples. We can
  express its Green's operator in the language of integro-differential operators
  as explained in~\cite{RosenkranzRegensburger2008a}.
\end{myexample}

\begin{myexample}
  \label{ex:wave-inhom-medium}
  As a typical counterpart in the world of LPDEs, consider the
  \emph{equation for waves in an inhomogeneous medium}, described by

in one space dimension. In this case we choose~; again one could choose much larger spaces of functions (or
  distributions) in analysis and in the applications. Here the differential
  operator is~ while the boundary space~ is the orthogonal
  closure of the linear span of the families of functionals~ and~ defined
  by~ and~. Using the notation  for denoting the
  orthogonal closure of the linear span, we can thus write~ for the
  boundary space under consideration.

  The point of the \emph{orthogonal closure} is that the given conditions imply
  other conditions not in their span, for example~
  or~. Rather than being linear
  consequences, these two examples are differential and integral
  consequences. (Of course the full boundary space also contains many
  functionals without a natural analytic interpretation.)
\end{myexample}

In the problems above, the differential equation is inhomogeneous while the
boundary conditions are homogeneous. A \emph{semi-homogeneous boundary problem}
is the opposite, combining a homogeneous differential equation with
inhomogeneous boundary conditions. While this is a simple task for LODEs (as
always we assume that the fundamental system is available to us in some form!),
it is usually a nontrivial problem for LPDEs (even when they have constant
coefficients). We will give the formal definition of a semi-inhomogeneous
boundary problem in the next section
(Definition~\ref{def:transfer-operator}). Here it suffices to consider an
example for developing the necessary auxiliary notions.

\begin{myexample}
  \label{ex:cauchy-wave-eq}
  The \emph{Cauchy problem for the wave equation} in one dimension is

Being a hyperbolic problem, we could use rather general function spaces for
  the ``boundary data'' . For reasons of uniformity we will nevertheless
  restrict ourselves here to the analytic setting, so assume~. Note that the association of~ to~ is again a linear
  operator mapping (two univariate) functions to a (bivariate) function; we will
  come back to this point in Definition~\ref{def:transfer-operator}.
\end{myexample}

Going back to the abstract setting, one is tempted to define the notion of
boundary data as some kind of functions depending on ``fewer'' variables. But
the problem with this approach is that---abstractly speaking---we are not
dealing with any functions depending on any number of variables (but see
below). Moreover, the inhomogeneous boundary conditions in the
form~ are basis-dependent while the whole point of
the abstract theory is to provide a \emph{basis-free description} (which leads
to an elegant setting for describing composition and factorization of abstract
boundary problems); see Proposition~\ref{prop:bp-product}. We shall therefore
develop a basis-independent notion of boundary data (we can go back to the
traditional description by choosing a basis).

We define first the \emph{trace map}~ as
sending~ to the functional~. In
Example~\ref{ex:cauchy-wave-eq} this would map the function~ to its
position and velocity values on~. We call~ the trace
of~ and write it as~. Moreover, we denote the image of the map~
by~ and refer to its elements as \emph{boundary data}. Note
that~ is usually much smaller than the full dual~ since a
continuous function (let alone an analytic one) cannot assume arbitrary
values. (This situation is vaguely reminiscent of the algebraic and continuous
dual of a topological vector space.)

Since by definition the trace map is surjective from~ to~, it has
some right inverse~. We refer
to~ as an \emph{interpolator} for~ since it constructs a
``function''~ from given boundary values~ such that~. Of course, the choice of~ is usually
far from being unique. Apart from its use for describing boundary data (see at
the end of this section), the notion of interpolator will turn out to be useful
for solving the semi-homogeneous boundary problem (see
Proposition~\ref{prop:proj-transfer}).

Let us now describe how to relate these abstract notions to the usual
setting of initial and boundary values problems as they actually in
analysis: essentially by \emph{choosing a basis}. However, we have to
be a bit careful since we must deal with the orthogonal closure.

\begin{mydefinition}
  \label{def:boundary-basis}
  If~ is any orthogonally closed subspace, we call a
  family~ a \emph{boundary basis} if~, meaning~ is the orthogonal closure of the span of
  the~.
\end{mydefinition}

Note that a boundary basis is typically smaller than a~-linear
basis of~. All traditional boundary problems are given in terms
of such a boundary basis. In Example~\ref{ex:cauchy-wave-eq}, the
boundary basis could be spelled out by using~
with~ and~. Relative to a boundary basis~, we
call~ the \emph{boundary values}
of~. As we can see from the next proposition, we
may think of the trace as a basis-free description of boundary
values. Conversely, one can always extract from any given boundary
data~ the boundary values~ as its
coordinates relative to the boundary basis~.

\begin{mylemma}
  \label{lem:trc-coord}
  Let~ be a boundary space with boundary
  basis~. If for any~
  one has~ then
  also~. In particular, for any~,
  the trace~ depends only on the boundary values~.
\end{mylemma}
\begin{myproof}
  Since~ is the image under the trace map, we have~
  and~ for some~. So
  assume~ for all~ and
  thus~ for all~ by the definition of
  orthogonal closure. Then we have~ and thus~.
\end{myproof}

The \emph{analytic interpretation} of this proposition is clear in
concrete cases like Example~\ref{ex:wave-inhom-medium}: Once the
values~ and~ are fixed, all
differential and integral consequences, as in the above
examples~ or~,
are likewise fixed. It is therefore natural that an interpolator need
only consider the boundary values rather than the full trace
information. This is the contents of the next lemma.

\begin{mylemma}
  \label{lem:int-coord}
  Let~ be a boundary space with boundary
  basis~ and write~ for the boundary values of any~
  and~ for the -subspace of~ generated by all
  boundary values~. Then any linear map~ with~ induces a unique
  interpolator~ defined by
  .
\end{mylemma}
\begin{myproof}
  We must show that~ is a right inverse
  of~. So for arbitrary~ we must
  show~. By the definition of~ we can
  write~ for some~. Since~, we
  are left to prove~. Using Lemma~\ref{lem:trc-coord}, it
  suffices to prove that~, which is true by hypothesis.
\end{myproof}

As noted above, we can always extract the boundary values~ of some boundary data~ relative to fixed
basis~ of~. However, since one normally has got \emph{only}
the boundary values (coming from some function), where does the corresponding~ come from? By definition, it has to assign values to all~, not only to the~ making up the boundary basis. As suggested by
the above lemmata, for actual computations those additional values will be
irrelevant. Nevertheless, it gives a feeling of confidence to provide these
values: If~ is any \emph{interpolator}, we have~. This follows immediately from the fact
that~ is a right inverse of the trace map and that it depends only
on the boundary values~ by Lemma~\ref{lem:int-coord}. In the
analysis setting this means we interpolate the given boundary value and then do
with the resulting function whatever is desired (like derivatives and integrals
in Example~\ref{ex:wave-inhom-medium}).


\section{Green's Operators for Signals and States}
\label{sec:greens-operators}


Using the notion of boundary data developed in the previous section, we can now
give the formal definition of the \emph{semi-homogeneous boundary problem}. In
fact, we can distinguish three different incarnations of a ``boundary problem''
(as we assume regularity, the \emph{fully homogeneous problem} is of course
trivial).

\begin{mydefinition}
  \label{def:transfer-operator}
  Let~ be a regular boundary problem with~
  and boundary space~. Then we distinguish the
  following problems: \medskip

  \setlength{\tabcolsep}{0pt}\noindent \begin{tabular}{ccc}
    \parbox{0.4\textwidth}{\small
      Given~,\\
      find~ with\1ex]
      }\kern1.1em&
    \parbox{0.29\textwidth}{\small
      Given~,\\
      find~ with\
    (T, \bspc) (\tilde{T}, \tilde{\bspc}) = (T \tilde{T}, \bspc \tilde{T} +
    \tilde{\bspc}).
  
    \beta(u) = \beta(\fri{\bspc} \! B) - \beta((1-P) \fri{\bspc} \! B) =
    B(\beta) - 0
  
  A_n(u) = \int_0^{x_n} u(\dots, \xi, \dots) \, d\xi,

    \galg_\alpha = \{ f \in \galg \mid \der_i f = 0 \text{ for all } \},

    \label{eq:cauchy-problem}
    \left.
    \begin{aligned}
      &Tu = 0\\
      &D_t^{i-1}u(0,x_1, \dots, x_n) = f_{i}(x_1, \dots, x_n) \text{ for }
    \end{aligned}
    \quad\right\}
  
  \bspc = [L_{0,\xi} D_t^i \mid i = 0, \dots, m-1 \;\text{and}\; \xi \in \RR^m],

  \label{eq:fund-intp}
  \fri{\bspc}(f_1, \dots, f_m) = f_1(x) + t \, f_2(x) + \cdots +
  \tfrac{t^{m-1}}{(m-1)!} \, f_m(x),

    \label{eq:fundsys-const-coeff}
    &u(t, x_1, \dots, x_n) = \sum_{i=1}^m c_i(\bar{x}_1, \dots, \bar{x}_n) \,
    \tfrac{t^{i-1} e^{-a t/a_0}}{(i-1)!},\\
    \label{eq:fundsys-varchange}
    & \bar{x}_i = t + x_1 + \cdots + x_{i-1} - (a_0 + a_1 + \cdots +
    a_{i-1}) \, x_i/a_i,
  
    &\fri{T} = a_0^{-1} \, e^{at/a_0} \, \tilde{Z}^* \, A_t \, e^{-at/a_0} \, Z^*.
  
    (T_1, [L_{0, \xi} \mid \xi \in \RR]) \, (T_2, [L_{0, \xi} \mid \xi \in \RR]) 
    = (T_1 T_2, [L_{0, \xi}, L_{0, \xi} D_t \mid \xi \in \RR]
  
    (T_2 u)(0,\xi) = a \, u(0,\xi) + a_0 \, u_t(0, \xi) + a_1 \,
    \tfrac{\partial}{\partial x_1}(0, \xi) + \cdots +
    \tfrac{\partial}{\partial x_n}(0, \xi),
  
where the first term and the~-derivatives vanish since~. Using~, this implies that the two systems are indeed
  equivalent.
\end{myproof}

This settles the completely reducible case: Using
Proposition~\ref{prop:ivp-product} we can break down the \emph{general Cauchy
  problem}~\eqref{eq:cauchy-problem} into first-order factors with single
initial conditions. For each of these we compute the state and signal operator
via Lemma~\ref{lem:transport-equation}, hence the state and signal operator
of~\eqref{eq:cauchy-problem} by Proposition~\ref{prop:bp-product}.


\section{Conclusion}
\label{sec:conclusion}


As explained in the Introduction, we see the framework developed in this paper
as the first stage of a more ambitious endeavor aimed at boundary problems for
general constant-coefficient (and other) LPDEs. Following the enumeration of the
Introduction, the next steps are as follows:
\begin{enumerate}
\item Stage (1) was presented in this paper, but the \emph{detailed
    implementation} for some of the methods explained here is still ongoing. The
  crucial feature of this stage is that it allows us to stay within the (rather
  narrow) confines of the PIDOS algebra. In particular, no Fourier
  transformations are needed in this case, so the analytic setting is entirely
  sufficient.
\item As we enter Stage (2), it appears to be necessary to employ stronger
  tools. The most popular choice is certainly the \emph{framework of Fourier
    transforms} (and the related Laplace transforms). While this can be
  algebraized in a manner completely analogous to the PIDOS algebra, the issue
  of choosing the right function space becomes more pressing: Clearly one has to
  leave the holomorphic setting for more analysis-flavoured spaces like the
  Schwartz class or functions with compact support. (As of now we stop short of
  using distributions since that would necessitate a more radical departure,
  forcing us to give up rings in favor of modules.)
\item For the treatment of genuine boundary problems in Stage~(3) our plan is to
  use a powerful generalization of the Fourier transformation---the
  \emph{Ehrenpreis-Palamodov integral representation}~\cite{Hansen1983}, also
  applicable to systems of LPDEs.
\end{enumerate}

Much of this is still far away. But the \emph{general algebraic framework} for
boundary problems from Sections~\ref{sec:boundary-data}
and~\ref{sec:greens-operators} is applicable, so the main work ahead of us is to
identify reasonable classes of LPDEs and boundary problems that admit a symbolic
treatment of one sort or another.


\begin{thebibliography}{10}

\bibitem{Bhamra2010}
K.S. Bhamra.
\newblock {\em Partial Differential Equations}.
\newblock PHI Learning, New Delhi, 2010.

\bibitem{Hansen1983}
S{\"o}nke Hansen.
\newblock On the ``fundamental principle'' of {L}. {E}hrenpreis.
\newblock In {\em Partial differential equations ({W}arsaw, 1978)}, volume~10
  of {\em Banach Center Publ.}, pages 185--201. PWN, Warsaw, 1983.

\bibitem{Hormander1976}
Lars H{\"o}rmander.
\newblock {\em Linear partial differential operators}.
\newblock Springer, Berlin, 1976.

\bibitem{Knapp2005}
Anthony~W. Knapp.
\newblock {\em Advanced real analysis}.
\newblock Cornerstones. Birkh\"auser Boston Inc., Boston, MA, 2005.
\newblock Along with a companion volume {{\i}t Basic real analysis}.

\bibitem{Korporal2012}
Anja Korporal.
\newblock {\em Symbolic Methods for Generalized Green's Operators and Boundary
  Problems}.
\newblock PhD thesis, Johannes Kepler University, Linz, Austria, November 2012.

\bibitem{KorporalRegensburgerRosenkranz2011}
Anja Korporal, Georg Regensburger, and Markus Rosenkranz.
\newblock Regular and singular boundary problems in {MAPLE}.
\newblock In {\em Proceedings of the 13th International Workshop on Computer
  Algebra in Scientific Computing, CASC'2011 (Kassel, Germany, September 5-9,
  2011)}, volume 6885 of {\em Lecture Notes in Computer Science}. Springer,
  2011.

\bibitem{KorporalRegensburgerRosenkranz2012}
Anja Korporal, Georg Regensburger, and Markus Rosenkranz.
\newblock Symbolic computation for ordinary boundary problems in maple.
\newblock In {\em Proceedings of the 37th International Symposium on Symbolic
  and Algebraic Computation (ISSAC'12)}, 2012.
\newblock Software presentation.

\bibitem{OberstPauer2001}
Ulrich Oberst and Franz Pauer.
\newblock The constructive solution of linear systems of partial difference and
  differential equations with constant coefficients.
\newblock {\em Multidimens. Systems Signal Process.}, 12(3-4):253--308, 2001.
\newblock Special issue: Applications of Gr{\"o}bner bases to multidimensional
  systems and signal processing.

\bibitem{RegensburgerRosenkranz2009}
Georg Regensburger and Markus Rosenkranz.
\newblock An algebraic foundation for factoring linear boundary problems.
\newblock {\em Ann. Mat. Pura Appl. (4)}, 188(1):123--151, 2009.
\newblock DOI:10.1007/s10231-008-0068-3.

\bibitem{RenardyRogers2004}
Michael Renardy and Robert~C. Rogers.
\newblock {\em An introduction to partial differential equations}, volume~13 of
  {\em Texts in Applied Mathematics}.
\newblock Springer-Verlag, New York, second edition, 2004.

\bibitem{Rosenkranz2005}
Markus Rosenkranz.
\newblock A new symbolic method for solving linear two-point boundary value
  problems on the level of operators.
\newblock {\em {J}. {S}ymbolic {C}omput.}, 39(2):171--199, 2005.

\bibitem{Rosenkranz2012}
Markus Rosenkranz.
\newblock Functorial programming \& integro-differential operators.
\newblock Talk at the International Mathematica Symposium (IMS'12), London,
  United Kingdom., 13 June 2012.

\bibitem{RosenkranzRegensburger2008a}
Markus Rosenkranz and Georg Regensburger.
\newblock Solving and factoring boundary problems for linear ordinary
  differential equations in differential algebras.
\newblock {\em Journal of Symbolic Computation}, 43(8):515--544, 2008.

\bibitem{RosenkranzRegensburgerTecBuchberger2009}
Markus Rosenkranz, Georg Regensburger, Loredana Tec, and Bruno Buchberger.
\newblock A symbolic framework for operations on linear boundary problems.
\newblock In Vladimir~P. Gerdt, Ernst~W. Mayr, and Evgenii~H. Vorozhtsov,
  editors, {\em Computer Algebra in Scientific Computing. Proceedings of the
  11th International Workshop (CASC 2009)}, volume 5743 of {\em LNCS}, pages
  269--283, Berlin, 2009. Springer.

\bibitem{RosenkranzRegensburgerTecBuchberger2012}
Markus Rosenkranz, Georg Regensburger, Loredana Tec, and Bruno Buchberger.
\newblock Symbolic analysis of boundary problems: {F}rom rewriting to
  parametrized {G}r{\"o}bner bases.
\newblock In Ulrich Langer and Peter Paule, editors, {\em Numerical and
  Symbolic Scientific Computing: Progress and Prospects}, pages 273--331.
  Springer, 2012.

\bibitem{Stakgold1979}
Ivar Stakgold.
\newblock {\em Green's functions and boundary value problems}.
\newblock John Wiley \& Sons, New York, 1979.

\end{thebibliography}

\end{document}
