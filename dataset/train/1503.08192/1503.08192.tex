\documentclass[11pt]{article}
\usepackage{amsmath, amsthm, amsfonts, amssymb, fullpage, indentfirst, graphicx, subfigure, cite}
\linespread{1.21}

\theoremstyle{plain}
\newtheorem{proposition}{Proposition}
\newtheorem{lemma}{Lemma}
\newtheorem{theorem}{Theorem}
\newtheorem{corollary}{Corollary}
\theoremstyle{definition}
\newtheorem{definition}{Definition}
\newtheorem{problem}{Problem}
\newtheorem{assumption}{Assumption}
\newtheorem{algorithm}{Algorithm}
\newtheorem{scenario}{Scenario}
\theoremstyle{remark}
\newtheorem{remark}{Remark}
\newtheorem{example}{Example}

\newenvironment{algorithmstep}{\ \begin{list}{\labelenumi}{\topsep0in\itemsep0in\parsep0in\labelwidth1in\usecounter{enumi}}}{\hfill\end{list}}

\begin{document}

\title{\LARGE\bf Distributed Estimation of Graph Spectrum\footnote{This work was supported by the National Science Foundation under grant CMMI-0900806.}}
\author{Mu Yang and Choon Yik Tang\\ School of Electrical and Computer Engineering\\ University of Oklahoma, Norman, OK 73019, USA\\ {\sf\{muyangwz,cytang\}@ou.edu}}
\date{\today}
\maketitle

\begin{abstract}
In this paper, we develop a two-stage distributed algorithm that enables nodes in a graph to cooperatively estimate the spectrum of a matrix  associated with the graph, which includes the adjacency and Laplacian matrices as special cases. In the first stage, the algorithm uses a discrete-time linear iteration and the Cayley-Hamilton theorem to convert the problem into one of solving a set of linear equations, where each equation is known to a node. In the second stage, if the nodes happen to know that  is cyclic, the algorithm uses a Lyapunov approach to asymptotically solve the equations with an exponential rate of convergence. If they do not know whether  is cyclic, the algorithm uses a random perturbation approach and a structural controllability result to approximately solve the equations with an error that can be made small. Finally, we provide simulation results that illustrate the algorithm.
\end{abstract}

\section{Introduction}\label{sec:intr}

The spectrum of a graph, defined as the set of eigenvalues of either its adjacency or Laplacian matrix, provides a useful characterization of the properties of the graph. For instance, the distribution of such eigenvalues offers insights into the shapes and sizes of communities in a complex network \cite{Newman10}. As another example, the largest and smallest of such eigenvalues provides bounds on the maximum, minimum, and average node degrees \cite{Chung97}. The spectrum has also been used, for example, in chemistry, where it is associated with the stability of molecules \cite{Chung97}, and in quantum mechanics, where it is related to the energy of Hamiltonian systems \cite{Chung97}.

With the continued advances in technology that enable humans to build increasingly complex networks, it is becoming desirable that nodes in a network have the ability to analyze the network themselves, such as decentralizedly computing the spectrum of the network, so that valuable understanding about, say, the network structure may be gained. Motivated by this, a number of distributed algorithms have been proposed in the literature, including \cite{Sahai12, Franceschelli13, TranTMD14} that consider estimation of the entire spectrum of the Laplacian matrix, and \cite{YangP10, Aragues12, LiC13} that focus on estimation of its second smallest eigenvalue (i.e., the algebraic connectivity).

In this paper, we add to the literature by developing a two-stage distributed algorithm, which enables nodes in a graph to cooperatively estimate the spectrum of a matrix  associated with the graph. Unlike in \cite{Sahai12, Franceschelli13, TranTMD14, YangP10, Aragues12, LiC13}, the matrix  can be the adjacency or Laplacian matrix of the graph, a weighted version of these matrices, or any other matrix induced by the graph (see Section~\ref{sec:probform}). To construct the algorithm, we first use a discrete-time linear iteration and the Cayley-Hamilton theorem to convert the original problem into an equivalent problem of solving a set of linear equations of the form , where every row of  and  is known to a particular node (Section~\ref{sec:formlineequa}). We then show that the matrix  can be made almost surely nonsingular if the nodes happen to know that  is cyclic, but not necessarily so if they do not (Section~\ref{sec:formlineequa}). In the case of the former, we use a Lyapunov approach to asymptotically solve the equations with an exponential rate of convergence (Section~\ref{sec:solvlineequascenknow}). In the case of the latter, we use a random perturbation approach and a structural controllability result to approximately solve the equations with an error that can be made small (Section~\ref{sec:solvlineequascennotknow}). Finally, we provide simulation results that illustrate our distributed algorithm (Section~\ref{sec:simuresu}) and conclude the paper with a word on future research directions (Section~\ref{sec:conc}).

\section{Problem Formulation}\label{sec:probform}

Consider a network modeled as an undirected, connected graph , where  denotes the set of  nodes and  denotes the set of edges. Any two nodes  are neighbors and can communicate if and only if . The set of neighbors of each node  is denoted as , and the communications are assumed to be delay- and error-free, with no quantization.

Suppose associated with the graph  is a square matrix  satisfying the following assumption:

\begin{assumption}\label{asm:W}
The matrix  is such that for each  with , if , then .
\end{assumption}

Note that Assumption~\ref{asm:W} allows   to be arbitrary. It also allows  and   to be arbitrary and different. Thus,  can be the adjacency or Laplacian matrix of graph , a weighted version of these matrices, or any other matrix associated with  as long as Assumption~\ref{asm:W} holds.

Suppose each node  knows only , , and  , which it prefers to not share with any of its neighbors due perhaps to security and privacy reasons. Yet, despite having only such local information about the graph  and matrix , suppose every node  wants to determine the spectrum of , i.e., all the  eigenvalues of , denoted as

where complex eigenvalues must be in the form of conjugate pairs. Finally, suppose each node  knows the value of , which is not an unreasonable assumption since each of them wants to determine the values of  objects.

Given the above, the goal of this paper is to devise a distributed algorithm that enables every node  to estimate the spectrum \eqref{eq:lambda} of  with a guaranteed accuracy.

\section{Forming a Set of Linear Equations}\label{sec:formlineequa}

In this section, we show that by having the nodes execute a discrete-time linear iteration  times, the problem of finding the spectrum \eqref{eq:lambda} of  may be converted into one of solving a set of linear equations with appealing properties.

Observe that although none of the nodes has complete information about  and , each node  knows the entire row  of  (since it knows  and  , and since   by Assumption~\ref{asm:W}). This makes the nodes well-suited to carry out the discrete-time linear iteration

which in matrix form may be written as

where ,  is maintained in node 's local memory, and

Indeed, \eqref{eq:s1} or \eqref{eq:s2} can be implemented by having each node  repeatedly send its  to every neighbor .

Since \eqref{eq:s2} is a discrete-time linear system, we can write

so that

By the Cayley-Hamilton theorem,  in \eqref{eq:s5} may be expressed as

where  is the identity matrix and the scalars  are the  coefficients of the characteristic polynomial of , i.e.,

Substituting \eqref{eq:s6} into \eqref{eq:s5} and using \eqref{eq:s4}, we obtain

By using \eqref{eq:s3}, we can rewrite \eqref{eq:s8} as

where, for later convenience, we denote the matrix on the left-hand side of \eqref{eq:s9} as , the vector of characteristic polynomial coefficients as , and the vector on the right-hand side of \eqref{eq:s9} as .

The matrix equation \eqref{eq:s9} suggests the following approach for finding the spectrum \eqref{eq:lambda} of : suppose each node  selects an initial condition . Upon selecting the 's, suppose the nodes execute the discrete-time linear iteration \eqref{eq:s1} or equivalently \eqref{eq:s2}  times for . During the execution, suppose each node  stores the resulting  numbers  in its local memory. Then, \eqref{eq:s9} is a set of  linear equations in which each node  knows the entire row  of  and , and in which the vector  of  characteristic polynomial coefficients  of  are the  unknowns. It follows that if  is nonsingular, and if the nodes are able to cooperatively solve \eqref{eq:s9} for the unique , then each of them could determine on its own the  eigenvalues  of  using \eqref{eq:s7} and a polynomial root-finding algorithm.

To realize the above approach, it is necessary that  in \eqref{eq:s9} is nonsingular. To see whether this can be ensured, observe from \eqref{eq:s3}, \eqref{eq:s4}, and \eqref{eq:s9} that  may be expressed as

In the form \eqref{eq:s10},  is, interestingly, the controllability matrix of a fictitious discrete-time single-input linear system

where  is its state,  is its input,  is its state matrix, and  is its input matrix. Hence:

\begin{proposition}\label{pro:AWy0}
The matrix  in \eqref{eq:s9} or \eqref{eq:s10} is nonsingular if and only if the pair  of the system \eqref{eq:s11} is controllable.
\end{proposition}

Since  is given by the problem but  may be freely selected by the nodes, it may be possible to select  so that the pair  is controllable. The following definition and lemmas examine this possibility:

\begin{definition}[\!\!\cite{ZhouK96}]\label{def:cyc}
A square matrix with real entries is said to be {\em cyclic} if each of its distinct eigenvalues has a geometric multiplicity of .
\end{definition}

\begin{lemma}\label{lem:Wnotcyc}
If  is not cyclic, then for every , the pair  is not controllable.
\end{lemma}

\begin{proof}
Suppose  is not cyclic and let  be given. Then, by Definition~\ref{def:cyc},  has an eigenvalue  whose geometric multiplicity exceeds , i.e., . Since  is a column vector, . Therefore, by statements~(i) and~(iv) of Theorem~3.1 in \cite{ZhouK96}, the pair  is not controllable.
\end{proof}

\begin{lemma}\label{lem:Wcyc}
If  is cyclic, then for almost every , the pair  is controllable.
\end{lemma}

\begin{proof}
According to Lemma~3.12 in \cite{ZhouK96}, if  is cyclic and  is such that the pair  is controllable, then for almost every , the pair  is controllable. Applying this lemma with , , and , and using the fact that the pair  is controllable, we conclude that so is the pair .
\end{proof}

Proposition~\ref{pro:AWy0} and Lemma~\ref{lem:Wnotcyc} imply that  being cyclic is necessary for  in \eqref{eq:s9} or \eqref{eq:s10} to be nonsingular. Lemma~\ref{lem:Wcyc}, on the other hand, implies that  being cyclic is essentially sufficient because almost every  would work. This latter result is especially useful in a decentralized network because the result allows each node  to select its  independently from other nodes and randomly from any continuous probability distribution before executing \eqref{eq:s1} or \eqref{eq:s2}, and be almost sure that the resulting  would be nonsingular.

Motivated by the above analysis, in the rest of this paper we consider separately the following two scenarios:

\begin{scenario}\label{sce:know}
The nodes know that  is cyclic.
\end{scenario}

\begin{scenario}\label{sce:notknow}
The nodes do not know whether  is cyclic, or know that  is not cyclic.
\end{scenario}

We consider Scenarios~\ref{sce:know} and~\ref{sce:notknow} separately because Scenario~\ref{sce:know} is easier to deal with (in Section~\ref{sec:solvlineequascenknow}) and its treatment helps us deal with Scenario~\ref{sce:notknow} (in Section~\ref{sec:solvlineequascennotknow}). We note that both of these scenarios arise in applications. For instance, if the graph  represents a sensor network and the entries   and   of  represent random sensor measurements with continuous probability distributions, then Scenario~\ref{sce:know} takes place as the nodes could say with near certainty that  is cyclic because almost every -by- matrix has  distinct eigenvalues and, thus, is cyclic. In contrast, if  represents the adjacency or Laplacian matrix of , then Scenario~\ref{sce:notknow} takes place as  would be cyclic if  is, say, a path graph \cite{Chung97} and would not be cyclic if  is, say, a complete or cycle graph \cite{Chung97}, which the nodes could not tell because they only have local information about .

To summarize, in this section we have transformed the problem of finding the spectrum \eqref{eq:lambda} of  into one of solving the set of linear equations \eqref{eq:s9}, in which each node  knows the entire row  of  and , and in which  can be made almost surely nonsingular in Scenario~\ref{sce:know}, but not necessarily so in Scenario~\ref{sce:notknow}.

\section{Solving the Set of Linear Equations in Scenario~\ref{sce:know}}\label{sec:solvlineequascenknow}

In this section, we focus on Scenario~\ref{sce:know} and develop a continuous-time distributed algorithm that enables the nodes to asymptotically solve the set of linear equations \eqref{eq:s9} with an exponential rate of convergence.

To facilitate the development, we assume that the nodes have executed \eqref{eq:s1} or \eqref{eq:s2} to arrive at \eqref{eq:s9}. Moreover, since  in \eqref{eq:s9} can be made almost surely nonsingular in this Scenario~\ref{sce:know}, we assume that it {\em is} nonsingular throughout the section. With these assumptions, for each  let  and , so that \eqref{eq:s9} may be stated as

where  and  are known to node  because \eqref{eq:s1} or \eqref{eq:s2} has been executed. In addition to knowing  and , suppose each node  maintains in its local memory an estimate  of the unknown, unique solution , where here  denotes continuous-time (unlike in Section~\ref{sec:formlineequa} where  denotes discrete-time). Furthermore, let  and  be vectors obtained by stacking the  estimates 's and  copies of the solution .

To come up with a distributed algorithm that gradually drives  to , consider a quadratic Lyapunov function candidate , defined as

where   and   are parameters. Notice that each term in the first summation in \eqref{eq:V} is a measure of how far away from the hyperplane  the estimate  is. Moreover, because  is nonsingular and because of \eqref{eq:Ax*=b}, the  hyperplanes   have a unique intersection at . Furthermore, the second summation in \eqref{eq:V} is a measure of the disagreement among the estimates 's. Hence, both the first and second summations in \eqref{eq:V} are only positive semidefinite functions of . However, as the following proposition shows, adding them up makes  a legitimate Lyapunov function candidate:

\begin{proposition}\label{pro:AVpd}
If  in \eqref{eq:Ax*=b} is nonsingular, then the function  in \eqref{eq:V} is positive definite with respect to .
\end{proposition}

\begin{proof}
Clearly,  is a positive semidefinite function of . To show that it is positive definite with respect to , we show that  if and only if . Suppose . Then,   according to \eqref{eq:Ax*=b}. In addition, the second summation in \eqref{eq:V} drops out. Therefore, . Next, suppose . Then,

Since  is connected, \eqref{eq:xi=xj} implies that there exists  such that  . Substituting this into \eqref{eq:aixi=bi}, we get   or, equivalently, . Since  is nonsingular, we have , so that .
\end{proof}

\begin{remark}\label{rem:V}
Notice that  in \eqref{eq:V} can also be written as

where  is positive definite and given by

where  denotes the Kronecker product and  is a weighted Laplacian matrix of  with ,  if , and  if  and .\qed
\end{remark}

With Proposition~\ref{pro:AVpd} in hand, we next take the time derivative of  along the state trajectory  to obtain

Examining \eqref{eq:Vdot}, we see that  can be made negative semidefinite---at the very least---by letting each  be the negative of the expression within the brackets in \eqref{eq:Vdot}, i.e.,

The following theorem asserts that the continuous-time system \eqref{eq:xidot} possesses an excellent property:

\begin{theorem}\label{thm:uniqeqptges}
If  in \eqref{eq:Ax*=b} is nonsingular, then the system \eqref{eq:xidot} has a unique equilibrium point at  that is globally exponentially stable, so that , , i.e.,  .
\end{theorem}

\begin{proof}
For each , setting  in \eqref{eq:xidot} to zero yields

Summing both sides of \eqref{eq:t1} over  gives

Due to \eqref{eq:Ax*=b} and to  being nonsingular, the vectors  in \eqref{eq:t2} are linearly independent in . Thus,

Substituting \eqref{eq:t3} back into \eqref{eq:t1} results in

which is equivalent to

where  and  have been defined in Remark~\ref{rem:V}. Since  is connected, \eqref{eq:t4} implies that   for some . Plugging this into \eqref{eq:t3} yields  . Since  is nonsingular, we have , i.e., . Hence, the system \eqref{eq:xidot} has a unique equilibrium point at . Since for each  the right-hand side of \eqref{eq:xidot} is the negative of the expression within the brackets in \eqref{eq:Vdot},  is negative definite with respect to . Therefore, the equilibrium point  is globally exponentially stable.
\end{proof}

Having established Theorem~\ref{thm:uniqeqptges}, we now relate it back to the original problem of finding the spectrum \eqref{eq:lambda} of . To this end, suppose each node  maintains in its local memory an estimate  of the unknown, th eigenvalue  of  for . Also suppose at each time , node  lets its  estimates 's be the roots of an th-order polynomial formed by the estimate  that is also stored in its local memory, i.e.,

which can be implemented using a polynomial root-finding algorithm embedded in node . Then, because  in \eqref{eq:s7} is a continuous function of , and  in \eqref{eq:rootfind} is the {\em same} continuous function of , Theorem~\ref{thm:uniqeqptges} implies that

Equation \eqref{eq:limlambda}, in turn, implies that the system \eqref{eq:xidot} is a continuous-time distributed algorithm that enables the nodes to asymptotically learn the spectrum \eqref{eq:lambda} of .

Putting together the development in Sections~\ref{sec:formlineequa} and~\ref{sec:solvlineequascenknow}, we obtain the following two-stage distributed algorithm, which is applicable to this Scenario~\ref{sce:know}:

\begin{algorithm}[For Scenario~\ref{sce:know}]\label{alg:know}
\begin{algorithmstep}
\item Each node  selects its  independently from other nodes and randomly from any continuous probability distribution.
\item Upon completion, the nodes execute \eqref{eq:s1} or \eqref{eq:s2}  times for , so that each node  gradually learns the entire row  of  and  in \eqref{eq:s9}.
\item Upon completion, the nodes execute \eqref{eq:xidot} and \eqref{eq:rootfind} for , so that each node  is able to continuously update its  and 's.
\end{algorithmstep}
\end{algorithm}

\begin{remark}\label{rem:lite}
The current literature offers a few distributed algorithms \cite{Nedic10, MouS13} that may be used to solve linear equations \eqref{eq:s9}. These algorithms are different from \eqref{eq:xidot} in that they force the state of each node to stay in an affine set, whereas \eqref{eq:xidot} allows the state to freely roam the state space.\qed
\end{remark}

\section{Solving the Set of Linear Equations in Scenario~\ref{sce:notknow}}\label{sec:solvlineequascennotknow}

In this section, we focus on Scenario~\ref{sce:notknow} and provide a slightly different algorithm that enables the nodes to approximately solve \eqref{eq:s9} with an error that can be made small.

Recall that Scenario~\ref{sce:notknow} represents a situation where the nodes either do not know whether  is cyclic, or somehow know that  is not cyclic. Consequently, they either do not know whether  in \eqref{eq:s9} is nonsingular, or know that  is singular. Although the nodes could still apply Algorithm~\ref{alg:know}, there is no guarantee that their estimates 's would converge to . One way to address this issue is to have the nodes randomly perturb the matrix  and vector , so that the resulting  in \eqref{eq:s10} is hopefully nonsingular. Of course, such a random perturbation approach no longer allows them to asymptotically determine the exact spectrum of . However, getting an estimate of the spectrum of  may be sufficient in some applications. Thus, we will adopt this random perturbation approach in this Scenario~\ref{sce:notknow}.

For notational simplicity, let the matrix associated with the graph  be denoted as  instead of , and let  instead denote a perturbed version of . In addition, let 's and 's denote, respectively, the characteristic polynomial coefficients and eigenvalues of  that the nodes wish to determine, and let 's and 's denote those of  as before. Moreover, let the perturbed matrix  be obtained from  in a decentralized manner as follows: prior to executing \eqref{eq:s1} or \eqref{eq:s2}, each node  lets

where the 's and 's are independent, uniformly distributed random variables in the interval , so that  represents the perturbation magnitude. Notice that since    by Assumption~\ref{asm:W},

as well. Also note that because the nodes are slated to select their 's independently and randomly from a continuous probability distribution, there is no need to further randomly perturb these 's.

The following lemma uses a structural controllability result to show that the aforementioned approach is effective:

\begin{lemma}\label{lem:randpert}
If  is as defined in \eqref{eq:wii}--\eqref{eq:wij0} and  is as defined in Step~1 of Algorithm~\ref{alg:know}, then  in \eqref{eq:s10} is almost surely nonsingular.
\end{lemma}

\begin{proof}
Reconsider the graph  from Section~\ref{sec:probform}. Let  and . In addition, let  and  be the all-one vector. Then,  according to the definition of . Moreover,  because the controllability matrix formed by  is a Vandermonde matrix that is nonsingular. These two properties of , along with the definition of structural controllability \cite{LinCT74}, imply that every  is structurally controllable. Next, let  and  be given. Then, by Proposition~1 of \cite{LinCT74}, there exists  such that  and . Hence,  is a dense subset of . Lastly, note that  due to Assumption~\ref{asm:W}, \eqref{eq:wii}--\eqref{eq:wij0}, and Step~1 of Algorithm~\ref{alg:know}. Since  is a dense subset of ,  is almost surely in . Therefore, by Proposition~\ref{pro:AWy0},  in \eqref{eq:s10} is almost surely nonsingular.
\end{proof}

As it follows from Lemma~\ref{lem:randpert}, by having the nodes perform the extra step described in \eqref{eq:wii}--\eqref{eq:wij0}, the results developed in Sections~\ref{sec:formlineequa} and~\ref{sec:solvlineequascenknow} become applicable to this Scenario~\ref{sce:notknow}. Furthermore, because both the characteristic polynomial coefficients and eigenvalues of a matrix are continuous functions of its entries, by having the nodes decrease the perturbation magnitude  toward zero, the differences between the 's and 's of  and the 's and 's of  can be made arbitrarily small, at least in principle. Note, however, that numerical issues may arise when  is too small, or when the resulting  is ill-conditioned. At present, we do not have answers to these numerical issues, and we believe they are important future research directions.

Based on the above, we obtain the following two-stage distributed algorithm for this Scenario~\ref{sce:notknow}:

\begin{algorithm}[For Scenario~\ref{sce:notknow}]\label{alg:notknow}
\begin{algorithmstep}
\item Each node  executes \eqref{eq:wii}--\eqref{eq:wij0} to obtain a perturbed matrix .
\item The remaining steps are identical to those of Algorithm~\ref{alg:know}.
\end{algorithmstep}
\end{algorithm}

\section{Simulation Results}\label{sec:simuresu}

In this section, we present two sets of simulation results that demonstrate the effectiveness of Algorithm~\ref{alg:know} for Scenario~\ref{sce:know} and Algorithm~\ref{alg:notknow} for Scenario~\ref{sce:notknow}.

\subsection{Simulation of Algorithm~\ref{alg:know} for Scenario~\ref{sce:know}}\label{ssec:simualgoscenknow}

\begin{figure}[tb]
\centering\subfigure[A -node graph.]{\includegraphics[width=0.48\linewidth]{sce1a.eps}}\quad\subfigure[Data points  for  and  that form the set of linear equations \eqref{eq:s9}.]{\includegraphics[width=0.48\linewidth]{sce1b.eps}}\\ \subfigure[Node 's estimate  of the th characteristic polynomial coefficient  for .]{\includegraphics[width=0.48\linewidth]{sce1c.eps}}\quad\subfigure[Node 's estimate  of the st characteristic polynomial coefficient  for .]{\includegraphics[width=0.48\linewidth]{sce1d.eps}}
\caption{Performance of Algorithm~\ref{alg:know} for Scenario~\ref{sce:know}.}
\label{fig:sce1}
\end{figure}

Consider a sensor network with  nodes, modeled as an undirected, connected graph , whose topology is shown in Figure~\ref{fig:sce1}(a). Suppose associated with the graph  is a -by- matrix , whose entries satisfy Assumption~\ref{asm:W} and represent random sensor measurements given by

Assuming that such measurements are realizations of continuously distributed random variables, the nodes are almost certain that  is cyclic, so that Scenario~\ref{sce:know} takes place. Thus, to determine all the eigenvalues 's of , which are given by , , , and , the nodes may apply Algorithm~\ref{alg:know}.

Figures~\ref{fig:sce1}(b)--\ref{fig:sce1}(d) display the result of simulating Algorithm~\ref{alg:know} with   and  . Specifically, Figure~\ref{fig:sce1}(b) shows the data points  for  and  that are used to form the set of linear equations \eqref{eq:s9}. Figure~\ref{fig:sce1}(c) shows, as a function of time , node 's estimate  of the th characteristic polynomial coefficient  of  for . Likewise, Figure~\ref{fig:sce1}(d) shows node 's estimate  of the st coefficient  for . (Due to space limitation, we are unable to include plots of  for all  and .) Observe that despite having only local information about  and , the nodes are able to utilize Algorithm~\ref{alg:know} to asymptotically determine all the characteristic polynomial coefficients 's of  and, hence, all its eigenvalues 's.

\subsection{Simulation of Algorithm~\ref{alg:notknow} for Scenario~\ref{sce:notknow}}\label{ssec:simualgoscennotknow}

\begin{figure}[tb]
\centering\subfigure[A -node graph.]{\includegraphics[width=0.48\linewidth]{sce2a.eps}}\quad\subfigure[Data points  for  and  that form the set of linear equations \eqref{eq:s9}.]{\includegraphics[width=0.48\linewidth]{sce2b.eps}}\\ \subfigure[Node 's estimate  of the th perturbed and true characteristic polynomial coefficients  and  for .]{\includegraphics[width=0.48\linewidth]{sce2c.eps}}\quad\subfigure[Node 's estimate  of the nd perturbed and true characteristic polynomial coefficients  and  for .]{\includegraphics[width=0.48\linewidth]{sce2d.eps}}
\caption{Performance of Algorithm~\ref{alg:notknow} for Scenario~\ref{sce:notknow}.}
\label{fig:sce2}
\end{figure}

Consider next an undirected, connected graph  with  nodes, whose topology is shown in Figure~\ref{fig:sce2}(a). Let  represent the adjacency matrix of  and suppose the nodes wish to determine all the eigenvalues 's of , which are given by , , , , , and . Because they only have local information about , the nodes do not know whether  is cyclic, so that Scenario~\ref{sce:notknow} takes place. (In fact,  in this particular example is not cyclic because it is symmetric and has repeated eigenvalues, at .) Therefore, the nodes have to apply Algorithm~\ref{alg:notknow}. In doing so, they let the perturbation magnitude be  and obtain from \eqref{eq:wii}--\eqref{eq:wij0} a perturbed matrix  given by

whose eigenvalues 's are , , , , , and , which are all distinct and slightly different from the eigenvalues 's of .

Figures~\ref{fig:sce2}(b)--\ref{fig:sce2}(d) display the result of simulating Algorithm~\ref{alg:notknow} with   and  , using a format similar to that of Figures~\ref{fig:sce1}(b)--\ref{fig:sce1}(d). The only difference is that Figures~\ref{fig:sce2}(c) and~\ref{fig:sce2}(d) show not only the characteristic polynomial coefficients 's of the ``perturbed'' , but also the characteristic polynomial coefficients 's of the ``true'' . Observe that with Algorithm~\ref{alg:notknow}, the nodes are able to asymptotically determine the 's and 's. In other words, they are able to approximately calculate the 's and 's with small errors.

\section{Conclusion}\label{sec:conc}

In this paper, we have designed and analyzed a two-stage distributed algorithm that enables nodes in a graph to cooperatively estimate the graph spectrum. We have shown that asymptotically accurate estimation can be achieved if the nodes know that the associated matrix is cyclic, and estimation with small errors can be achieved if they do not. As for future research, we believe that making the algorithm numerically more robust, so that it can cope with poorly conditioned  and , is an important next step.

\bibliographystyle{IEEEtran}
\bibliography{paper}

\end{document}