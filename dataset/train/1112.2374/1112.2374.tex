
\documentclass[onecolumn,letterpaper,11pt,draftclsnofoot]{IEEEtran}

\usepackage{times,amsmath,epsfig}
\usepackage{cite}
\usepackage{graphicx}
\usepackage{psfrag}
\usepackage{subfigure}
\usepackage{url}
\usepackage{stfloats}
\usepackage{amsmath}
\usepackage{amssymb}
\interdisplaylinepenalty=2500
\usepackage{array}
\usepackage{array}


\begin{document}

\title{Performance Analysis of Bidirectional Relay Selection with Imperfect Channel State Information}

\author{\IEEEauthorblockN{Hongyu Cui, Rongqing Zhang, Lingyang Song, and Bingli
Jiao}\\
\IEEEauthorblockA{School of Electronics Engineering and Computer
Science\\ Peking University, Beijing, China, \\
}}

\maketitle
\begin{abstract}
In this paper, we investigate the performance of bidirectional relay
selection using amplify-and-forward protocol with imperfect channel state
information, i.e., delay effect and channel estimation error. The
asymptotic expression of end-to-end SER in high SNR regime is
derived in a closed form, which indicates that the delay effect
causes the loss of both coding gain and diversity order, while the
channel estimation error merely affects the coding gain. Finally,
analytical results are verified by Monte-Carlo simulations.
\end{abstract}
\begin{keywords}
bidirectional relay selection, analog network coding, imperfect
channel state information
\end{keywords}

\section{Introduction}

Bidirectional relay communications, in which two sources exchange
information through intermediate relays, have gained a lot of
interest by now, and different transmission schemes have been
proposed \cite{Katti2008}. In \cite{Popovski2007,Louie2010}, an
amplify-and-forward~(AF) based network coding scheme, named as
analog network coding~(ANC), was introduced. With ANC, the data
transmission process can be divided into two phases, and the
spectral efficiency, which is restricted by half-duplex antennas,
can get improved. Recently, relay selection~(RS) for bidirectional
relay networks has been intensively researched to achieve full
spatial diversity and better system performance, which requires
fewer orthogonal resources in comparison of all-participate relay
approaches\cite{Bletsas2006,Ibrahim2008}. Performing RS, the
``best'' relay is firstly selected before data transmission by the
predefined criterion
\cite{Zhang2009,Kyu2009,Song2010,Song2011,Jing2009,Nguyen2010}. In
\cite{Zhang2009,Kyu2009}, the authors proposed the max-min sum rate
selection criterion for AF bidirectional relay. In
\cite{Song2010,Song2011,Jing2009,Nguyen2010}, selection criterions
in minimizing the symbol error rate~(SER) were introduced and
analyzed.

To the authors' best knowledge, most works about RS in bidirectional
relay only consider perfect channel state information~(CSI).
However, imperfect CSI, i.e., delay effect and channel estimation
error~(CEE), has great impact on the performance of bidirectional
relay selection. Specifically, the time delay between relay
selection and data transmission causes that the selected relay may
not be optimal for data
transmission\cite{Torabi2010,Suraweera2010,Michalopoulos2010}. And
similarly, channel estimation errors can not be ignored either
\cite{Seyfi2010,Seung2009,Gedik2009,Ding2011}. In~\cite{Ding2011},
the authors analyzed the performance loss of bidirectional relay
selection using decode-and-forward protocol with CEE, but the impact
of imperfect CSI on a general bidirectional AF relay selection was
not provided.

In light of the aforementioned researches, we analyze the impact of
imperfect CSI, including delay effect and CEE, for bidirectional AF
relay selection in this paper, which has not been studied
previously. The asymptotic expression of end-to-end SER is derived
in a closed form, and verified by computer simulations. Analytical
and simulated results reveal that delay effect reduces both the
diversity order and the coding gain, while channel estimation error
merely causes the coding gain loss. The main contribution of this paper can be summarized as
follows:
\begin{enumerate}
  \item The asymptotic SER expression for bidirectional relay selection is provided in a closed form, which matches the simulated results in high SNR regime;
  \item Imperfect CSI, i.e., delay and channel estimation errors, is taken into account to derive the analytical results, and its therein impact is investigated.
\end{enumerate}

The remainder of this paper is organized as follows: In Section
\uppercase\expandafter{\romannumeral2}, the system model of
bidirectional AF relay selection, and the imperfect CSI model are
described in detail. Section \uppercase\expandafter{\romannumeral3}
provides the analytical expression of bidirectional relay selection
with imperfect CSI. Simulation results and performance analysis are
presented in Section \uppercase\expandafter{\romannumeral4}.
Finally, section \uppercase\expandafter{\romannumeral5} concludes
this paper.

\emph{Notation:} and  represent the conjugate and the absolute value, respectively.  is used for the expectation and  represents the probability. The probability density function and the cumulative probability function of variable  are denoted by  and , respectively.


\section{System Model}

The system investigated in this paper is a general bidirectional AF
relay network with two sources ,  exchanging information
through the intermediate  relays , . The
direct link between  and  does not exist, and each node is
equipped with a single half-duplex antenna. The transmit power of
the sources is assumed to be the same, denoted by , and all the
relays have the individual power constraint, denoted by . The
channel coefficients between sources and relays are reciprocal, and
these coefficients are constant over the duration of one data block.

The whole procedure of bidirectional AF relay selection is divided
into two parts periodically: \emph{relay selection process} and
\emph{data transmission process}, which will be described concretely
in the next section. Let  and  represent
the actual and the estimated channel coefficients between  and
 during the relay selection process, respectively; let
 and  represent the actual and the
estimated channel coefficients between  and  during the
data transmission process, respectively. All the actual channel
coefficients are independent identically distributed~(i.i.d.)
Rayleigh flat-fading with zero mean and unit variance, i.e.,
, and thus,
 and 
are both exponentially distributed with unit mean. Both the sources
can know the global channel coefficients by estimating the training
symbols, while each relay only has its local channel information.

\subsection{Model of Delay Effect}
Due to the time delay between relay selection process and data
transmission process,  is not the same as ,
which means the CSI is \emph{outdated}. Their relationship can be
modeled by the first-order autoregressive
model\cite{Michalopoulos2010}:

where  is a zero mean complex-Gaussian RV with variance of
;  and  are i.i.d.
random variable~(RVs) with zero mean and variance of  and , respectively. In this
paper, we assume .

The correlation coefficient  ( , where  represents no delay effect, in other words,
the CSI is not outdated) between  and relays is defined by
Jakes' autocorrelation model \cite{Michalopoulos2010}:

where  stands for the zeroth order Bessel
function\cite{Abramowitz},  is the Doppler frequency, and
 is the time delay between the relay selection process and the
data transmission process. In this paper, two variables
,  are used to represent the correlation
coefficients between  and the relays, respectively, for
 and  may be different.

\subsection{Model of Channel Estimation Error}
Let  denote the actual channel coefficient and 
represent the estimated channel coefficient, and then their relationship
can be modeled as follows\cite{Seyfi2010}:

and

where  and CEE  are independent complex-Gaussian RVs with zero
mean and variances of , , respectively.
 and CEE  are also independent complex-Gaussian RVs with
zero mean and variances of , ,
respectively. The correlation coefficient  ~(, where 
means no CEE)~is determined by the concrete channel estimation
method. In addition, ~can be modeled as an increasing
function of the training symbols' power , i.e., 
when  approaches infinity\cite{Yoo2006,Feifei2009}. In this
paper, we assume \cite{Ramya2009}.

According to the above relationship, the variances of CEE are given
by~:

and


Assuming  in this paper, we have  and  according
to~(\ref{Eq:vare})~and~(\ref{Eq:vard}).


\subsection{Relationship between~~and~~}
For the bidirectional relay selection communications,  is used for relay selection, and  is used
for data detection. According to the model of imperfect CSI, we
have~:

\textbf{\emph{Lemma 1: }} and  can be
related as :

where  and  are i.i.d. RVs, and


When the CSI is not outdated, i.e., ,  and   have the same
distribution.

When the CSI is outdated, i.e., , the probability
density function~(PDF) of  conditioned by  can be expressed as :

where  stands for the zeroth order modified
Bessel function of the first kind\cite{Abramowitz}, and .

\textbf{\emph{Proof:}} The proof of Lemma 1 can be found in Appendix
A.

\section{Performance Analysis of Bidirectional Relay Selection with Imperfect CSI}

\subsection{Instantaneous Received SNR at the Sources}

As mentioned above, the whole procedure of bidirectional relay
selection is divided into relay selection process and data
transmission process.

In the relay selection process, the central unit~(CU), i.e., 
or , estimates all the channel coefficients .
Then, based on the predefined selection criterion, CU selects the
``best" relay from all the available relays for the subsequent data
transmission and other relays keep idle until the next relay
selection instant comes. There are several selection criterions for
bidirectional relay
\cite{Zhang2009,Kyu2009,Song2010,Song2011,Jing2009,Nguyen2010}. In
this paper, we adopt the \emph{Best-Worse-Channel} method for relay
selection which has the best performance in minimizing the average
SER and is tractable for analysis\cite{Jing2009,Nguyen2010}.
According to this criterion, the index  of the selected relay
satisfies~:

and thus,


The subsequent data transmission process can be divided into two
phases. During the first phase, the sources simultaneously send
their respective information to the intermediate relays where only
the selected relay  is active. The superimposed signal at 
is , where
 denotes the modulated symbols transmitted by  with the
average power normalized, , and  is additive white
Gaussian noise~(AWGN) at , which is a zero mean
complex-Gaussian RV with two-sided power spectral density of 
per dimension. During the second phase,  amplifies the received
signal and forwards it back to the sources. Let  be the signal
generated by , then we have ,
where  is the amplification factor. In this paper, we
analyze the \emph{variable-gain} AF relay\cite{Seung2009}, then
 is decided by the
estimated instantaneous channel coefficients.

The received signals by  and  are similar due to the
symmetry of the network topology, and thus, we take  as an
example for analysis. The signal  received by  can be
written as , where  is AWGN at
;  and  are i.i.d. RVs. According to (\ref{Eq:hd}),
 and  can be rewritten as  and , where  and  are independent RVs due to the
independence of  and . Therefore, 
can be expanded as :

where (\ref{Eq:signal}) represents the useful information from
; (\ref{Eq:interI}) represents the inter-interference from
 caused by CEE; (\ref{Eq:selfI1}) and (\ref{Eq:selfI2})
represent the self-interference from  itself which can be
subtracted totally by self-canceling if CEE does not
exist\cite{Song2011}. However, with CEE,  can only reconstruct

at the receiver. Thus, only (\ref{Eq:selfI1}) can be subtracted
totally, whereas the self-interference of (\ref{Eq:selfI2}) is
residual; (\ref{Eq:noise}) includes the amplified noise from 
and the noise at .

After self-canceling  from , and then multiplied by
 to compensate the phase rotation,
the processed signal  at  is :


The transmitted information  can be recovered by maximum
likelihood detection:

where  represents the Euclid-distance,
 is the alphabet of modulation symbols, and  is the recovered signal.

According to (\ref{Eq:y1}), the instantaneous received SNR
 at  can be written as :

where , ,  is
the CEE coefficient, and the CEE variance .

In high SNR regime,  and
, then the item  in the denominator of
(\ref{Eq:gamma1}) approaches 1, which can also be ignored when SNR
approaches infinity\cite{Song2011}.

Therefore,  in high SNR regime can be simplified into~:

 where


 in (\ref{Eq:gamma1high}) is greater than that in
(\ref{Eq:gamma1}), whereas they match tightly in high SNR regime.
Therefore, we use  in (\ref{Eq:gamma1high}) for asymptotic
analysis in the followings.

\subsection{Distribution Function of the Received SNR}

The distribution of  in (\ref{Eq:gamma1high}) is decided
by  and , which are
determined by 
and  according
to Lemma 1. Furthermore, the distribution of 
and  can be obtained by the above selection
criterion. After some manipulations, we have


\textbf{\emph{Theorem 1:}} With the definition that~:
\label{Eq:cdf}
F_{\gamma _1 } \left(z\right) = 1 - N^2\sum\limits_{m = 0}^{N - 1}
{\sum\limits_{n = 0}^{N - 1} \binom{N-1}{m}} \binom{N-1}{n}
\frac{\left(-1\right)^m}{{2m + 1}}\frac{\left(-1\right)^n}{{2n + 1}}
\left(f_1 + f_2 +f_3 + f_4 \right) ~

f_1  = 2\sqrt {ab} z\exp \left( - \left(a + b\right)z\right)K_1
\left(2z\sqrt {ab} \right)~,

f_2 &= \sqrt {\frac{{16n^2 ab}}{{2\left(n + 1\right)\left[\left(2n +
1\right)\left(1 - \rho _1 \right) + 1\right]}}} z\exp \left( -
\left(\frac{{2\left(n + 1\right)a}}{{\left(2n + 1\right)\left(1 -
\rho _1 \right) + 1}}+b\right)z\right)\\\notag &\times K_1
\left(2z\sqrt {\frac{{2\left(n + 1\right)ab}}{{\left(2n +
1\right)\left(1 - \rho _1 \right) + 1}}}\right)~,

f_3  &= \sqrt {\frac{{8m^2 ab}}{{\left(m + 1\right)\left[\left(2m +
1\right)\left(1 - \rho _2 \right) + 1\right]}}} z\exp \left( -
\left(a + \frac{{2\left(m + 1\right)b}}{{\left(2m + 1\right)\left(1
- \rho _2 \right) + 1}}\right)z\right)\\\notag &\times K_1
\left(2z\sqrt {\frac{{2ab\left(m + 1\right)}}{{\left(2m +
1\right)\left(1 - \rho _2 \right) + 1}}}\right)~,

f_4  &= \sqrt {\frac{{16m^2 n^2 ab}}{{\left(m + 1\right)\left(n
+1\right)\left[\left(2n + 1\right)\left(1 -\rho _1 \right) +
1\right]\left[\left(2m + 1\right)\left(1 - \rho _2 \right)+
1\right]}}} z\\\notag &\times \exp \left( - \left(\frac{{2\left(n +
1\right)a}}{{\left(2n + 1\right)\left(1 -\rho _1 \right) + 1}} +
\frac{{2\left(m + 1\right)b}}{{\left(2m + 1\right)\left(1 - \rho _2 \right) + 1}}\right)z\right)\notag\\
&\times K_1 \left( 4z\sqrt {\frac{{ab\left(m + 1\right)\left(n +
1\right)}}{{\left[\left(2m +1\right)\left(1 - \rho _2 \right) +
1\right]\left[\left(2n+ 1\right)\left(1 - \rho _1 \right) +
1\right]}}} \right)~.

\overline {SER}  = \alpha \mathbb{E}  \left[Q\left(\sqrt {\beta
\gamma } \right)\right]= \frac{\alpha }{{\sqrt {2\pi }
}}\int\limits_0^\infty {F_\gamma \left(\frac{{t^2 }}{\beta
}\right)e^{ - \frac{{t^2 }}{2}} } dt~,\label{Eq:ps2}

\int_0^\infty {x^{\mu  - 1} } e^{ - \alpha x} K_\nu  \left(\beta
x\right)dx = \frac{{\sqrt \pi \left(2\beta \right)^\nu
}}{{\left(\alpha  + \beta \right)^{\mu + \nu } }}\frac{{\Gamma
\left(\mu  + \nu \right)\Gamma \left(\mu  + \nu \right)}}{{\Gamma
\left(\mu + 1/2\right)}}F\left(\mu  + \nu ,\nu + \frac{1}{2};\mu +
\frac{1}{2};\frac{{\alpha  - \beta }}{{\alpha + \beta }}\right)~,
\label{Eq:SER11}
\overline {SER}_1^\infty   = \frac{\alpha }{{4\beta ^N
}}\frac{{\left(2N\right)!}}{{N!}}\left(\left(\frac{{1 + \psi _r \rho
_e^2 \sigma _D^2 }}{{\psi _r \rho _e^3 }}\right)^N  +
\left(\frac{{5\psi _r \psi _s \rho _e^2 \sigma _D^2  + \psi _r \rho
_e^2 + \psi _s }}{{\psi _r \psi _s \rho _e^3 }}\right)^N \right)~,
\label{Eq:SER12}
\overline {SER}_1^\infty = &\frac{\alpha }{{2\beta }}\left(\frac{{1
+ \psi _r \rho _e^2 \sigma _D^2 }}{{\psi _r \rho _e^3
}}\right)N\sum\limits_{n = 0}^{N - 1} {\left( - 1\right)^n }
\binom{N-1}{n}\frac{{2 - \rho _1 }}{{\left(2n + 1\right)\left(1 -
\rho _1 \right) + 1}} \\\notag+ &\frac{\alpha }{{2\beta
}}\left(\frac{{5\psi _r \psi _s \rho _e^2 \sigma _D^2  + \psi _r
\rho _e^2 + \psi _s }}{{\psi _r \psi _s \rho _e^3
}}\right)N\sum\limits_{m = 0}^{N - 1} {\left( - 1\right)^m }
\binom{N-1}{m}\frac{{2 -\rho _2 }}{{\left(2m+ 1\right)\left(1 - \rho
_2 \right) +1}}~,

d =
\begin{cases}
  N, ~\text{if the CSI is not outdated, i.e.,~} \rho_{f_1}=\rho_{f_2}=1; \\
  1, ~\text{if the CSI is outdated, i.e.,~} \rho_{f_1}<1 \text{~or~}\rho_{f_2}<1.
\end{cases}

\hat h_{t,ji}= h_{t,ji}+e=\rho_{f_j}
h_{s,ji}+\sqrt{1-\rho_{f_j}^2}\varepsilon_j+e=\rho_{f_j}\rho_e \hat
h_{s,ji}+\rho_{f_j}d+\sqrt{1-\rho_{f_j}^2}\varepsilon_j+e~,

f_{\hat \gamma_{t,ji} ,\hat \gamma_{s,ji} } \left( {y,x} \right) =
\frac{1}{{\left( {1 - \rho _j^2 } \right)\sigma _{\hat h}^4 }}\exp
\left( { - \frac{{x + y}}{{\left( {1 - \rho _j^2 } \right)\sigma
_{\hat h}^2 }}} \right)I_0 \left( {\frac{{2\sqrt {\rho _j^2 xy}
}}{{\left( {1 - \rho _j^2 } \right)\sigma _{\hat h}^2 }}} \right)~.

 f_{\hat \gamma_{t,ji}|\hat
\gamma_{s,ji}} \left( {y|x} \right)=\frac{f_{\hat \gamma_{t,ji}
,\hat \gamma_{s,ji}} \left( {y,x} \right)}{f_{\hat \gamma_{s,ji} }
\left( x \right)}~,
\label{Eq:T10}
F_{\hat \gamma _{s,1k} } \left( x \right) & \buildrel \left(a\right)
\over= N\Pr \left\{ {\hat \gamma _{s,1i}  < x,k = i} \right\}
\\\notag
& \buildrel \left(b\right) \over = N\int\limits_0^x {f_{\hat \gamma
_{s,1i} } \left(y\right)} \Pr \left\{ {\hat \gamma _{s,1i}  \le \hat
\gamma _{s,2i} |\hat \gamma _{s,1i}  = y} \right\}\Pr \left\{ {k =
i|\hat \gamma _{s,1i}  \le \hat \gamma _{s,2i} ,\hat \gamma _{s,1i}
= y} \right\}dy
\\\notag
&+ N\int\limits_0^x {f_{\hat \gamma _{s,1i} } (y)} \Pr \left\{ {\hat
\gamma _{s,1i}  > \hat \gamma _{s,2i} |\hat \gamma _{s,1i}  = y}
\right\}\Pr \left\{ {k = i|\hat \gamma _{s,1i}  > \hat \gamma
_{s,2i} ,\hat \gamma _{s,1i}  = y} \right\}dy~
\label{Eq:T11}
\Pr \left\{ {k = i|\hat \gamma _{s,1i}  \le \hat \gamma _{s,2i}
,\hat \gamma _{s,1i}  = y} \right\} = \prod\limits_{p \ne i} {\Pr }
\left\{ {\min \left( {\hat \gamma _{s,1i} ,\hat \gamma _{s,2i} }
\right) \le y} \right\} = \left( {1 - \exp \left( { -
\frac{2y}{\sigma _{\hat h}^2 } }\right)} \right)^{N - 1}~.

F_{\hat \gamma _{s,1k} } \left( x \right) &=~N\int_0^x
{\frac{1}{{\sigma _{\hat h}^2 }}\exp \left( { - \frac{y}{{\sigma
_{\hat h}^2 }}} \right)} \left( {\int_0^y {\frac{1}{{\sigma _{\hat
h}^2 }}\exp \left( { - \frac{z}{{\sigma _{\hat h}^2 }}} \right)}
\left( {1 - \exp \left( { - \frac{{2z}}{{\sigma _{\hat h}^2 }}}
\right)} \right)^{N - 1} dz} \right)dy
\\\notag
&+~N\int_0^x {\frac{1}{{\sigma _{\hat h}^2 }}\exp \left( { -
\frac{y}{{\sigma _{\hat h}^2 }}} \right)} \left( {\int_y^\infty
{\frac{1}{{\sigma _{\hat h}^2 }}\exp \left( { - \frac{z}{{\sigma
_{\hat h}^2 }}} \right)} dz} \right)\left( {1 - \exp \left( { -
\frac{{2y}}{{\sigma _{\hat h}^2 }}} \right)} \right)^{N - 1} dy~.
\label{Eq:Fs1k}
F_{\hat \gamma _{s,1k} } \left( x \right) = 1 - &N\sum\limits_{n =
0}^{N - 1} \binom{N-1}{n} \frac{{\left( { - 1} \right)^n }}{{2n +
1}}\left[\exp \left( { - \frac{x}{{\sigma _{\hat h}^2 }}} \right) +
\frac{n}{{n + 1}}\exp \left( { - \frac{{2\left( {n + 1}
\right)x}}{{\sigma _{\hat h}^2 }}} \right)\right]~,

f_{\hat \gamma _{t,1k} } \left( x \right) &= N\sum\limits_{n = 0}^{N
- 1} {\left( { - 1} \right)^n \binom{N-1}{n}} \frac{1}{{2n +
1}}\left[ \frac{{\exp \left( { - x/\sigma _{\hat h}^2 }
\right)}}{{\sigma _{\hat h}^2 }}
\right.\\\notag&\left.+\frac{{2n/\sigma _{\hat h}^2 }}{{\left( {2n +
1} \right)\left( {1 - \rho_1^2 } \right) + 1}}\exp \left( { -
\frac{{2 \left( {n + 1} \right)x/\sigma _{\hat h}^2 }}{{\left( {2n +
1} \right)\left( {1 - \rho_1^2 } \right) + 1}}} \right) \right]~.
\label{Eq:FF}
F_{\gamma _1 } \left( z \right) &= \Pr \left\{ {\frac{{\Omega _1
\Omega _2 }}{{\Omega _1  + \Omega _2 }} < z} \right\}
\\\notag
& = \Pr \left\{ {\left( {\Omega _2  - z} \right)\Omega _1  < z\Omega
_2 ,\Omega _2  > z} \right\} + \Pr \left\{ {\left( {\Omega _2  - z}
\right)\Omega _1  < z\Omega _2 ,\Omega _2  \le z} \right\}
\\\notag
&= \int_z^\infty  {F_{\Omega _1 } \left( {\frac{{zx}}{{x - z}}}
\right)} f_{\Omega _2 } \left(x\right)dx + \int_0^z {\left[ {1 -
F_{\Omega _1 } \left( {\frac{{zx}}{{x - z}}} \right)} \right]}
f_{\Omega _2 } \left( x \right)dx
\\\notag
&= 1 - \int\limits_0^\infty  {f_{\Omega _2 } \left( {x + z}
\right)\left[ {1 - F_{\Omega _1 } \left( {z + \frac{{z^2 }}{x}}
\right)} \right]} dx
\label{Eq:hignFgamma}
F_{\gamma _1 }  \left( z \right) &\approx 1 - N^2 \sum\limits_{m =
0}^{N - 1} {\sum\limits_{n = 0}^{N - 1} {\left( { - 1} \right)^{m +
n} \binom{N-1}{m}} } \binom{N-1}{n}\frac{1}{{2m + 1}}\frac{1}{{2n +
1}}
\\\notag
&\times \bigg[\exp \left( { - \left( {a + b} \right)z} \right) +
\frac{n}{{n + 1}}\exp \left( { - \left( {\frac{{2\left( {n + 1}
\right)a}}{{\left( {2n + 1} \right)\left( {1 - \rho _1 } \right) +
1}} + b} \right)z} \right) \\\notag
 &+ \frac{m}{{m + 1}}\exp
\left( { - \left( {a + \frac{{2\left( {m + 1} \right)b}}{{\left( {2m
+ 1} \right)\left( {1 - \rho _2 } \right) + 1}}} \right)z}
\right)\\\notag
 &+ \frac{m}{{m + 1}}\frac{n}{{n + 1}}\exp
\left( { - \left( {\frac{{2\left( {n + 1} \right)a}}{{\left( {2n +
1} \right)\left( {1 - \rho _1 } \right) + 1}} + \frac{{2\left( {m +
1} \right)b}}{{\left( {2m + 1} \right)\left( {1 - \rho _2 } \right)
+ 1}}} \right)z} \right)\bigg]~.

F_{\gamma _1 } \left( z \right) &= 1 - N\sum\limits_{n = 0}^{N - 1}
{\left( { - 1} \right)^n } \binom{N-1}{n}\frac{1}{{2n + 1}}\left[
{\exp \left( { - az} \right) + \frac{n}{{n + 1}}\exp \left( { -
2\left( {n + 1} \right)az} \right)} \right]\\\notag &\times
N\sum\limits_{m = 0}^{N - 1} {\left( { - 1} \right)^m }
\binom{N-1}{m}\frac{1}{{2m + 1}}\left[ {\exp \left( { - bz} \right)
+ \frac{m}{{m + 1}}\exp \left( { - 2\left( {m + 1} \right)bz}
\right)} \right]~.
 \label{Eq:expansion}
&N\sum\limits_{n = 0}^{N - 1} {\left( { - 1} \right)^n }
\binom{N-1}{n}\frac{1}{{2n + 1}}\left[ {\exp \left( { - az} \right)
+ \frac{n}{{n + 1}}\exp \left( { - 2\left( {n + 1} \right)az}
\right)} \right]\\\notag =~&N\sum\limits_{n = 0}^{N - 1} {\left( { -
1} \right)^n } \binom{N-1}{n}\frac{1}{{n + 1}} {\exp \left( { -
2\left( {n + 1} \right)az} \right)} \\\notag +~&N\sum\limits_{n =
0}^{N - 1} {\left( { - 1} \right)^n } \binom{N-1}{n}\frac{1}{{2n +
1}} \left[\exp \left( { - az} \right) - \exp \left( { - 2\left( {n +
1} \right)az} \right)\right]
\\\notag
\buildrel \left(a\right) \over=~&1 - \left[ {1 - \exp \left( { -
2az} \right)} \right]^N  - \exp \left( { - az} \right)\left[
{N\sum\limits_{n = 0}^{N - 1} {\left( { - 1} \right)^n }
\binom{N-1}{n}\frac{1}{{2n + 1}}\left( {1 - \exp \left( { - \left(
{2n + 1} \right)az} \right)} \right)} \right]
\\\notag
\buildrel \left(b\right) \over =~&1 - \left[ { - \sum\limits_{p =
1}^\infty  {\frac{{\left( { - 2az} \right)^p }}{{p!}}} } \right]^N -
\sum\limits_{p = 0}^\infty  {\frac{{\left( { - az} \right)^p
}}{{p!}}} \left[N\sum\limits_{n = 0}^{N - 1} {\left( { - 1}
\right)^n } \binom{N-1}{n}\sum\limits_{p = 1}^\infty  {\left(2n +
1\right)^{p - 1} \frac{{\left( { - az} \right)^p }}{{p!}}}\right]
\\\notag
\buildrel \left(c\right) \over = ~&1 - \frac{1}{2}\left(2az\right)^N
+ o((az)^N )~,
 F_{\gamma
_1} \left(z\right) \approx [\frac{1}{2}\left(2a\right)^N  +
\frac{1}{2}\left(2b\right)^N ]z^N~.\label{Eq:what}

F_{\gamma _1}  \left( z \right) \approx  &azN\sum\limits_{n = 0}^{N
- 1} {\left( { - 1} \right)^n } \binom{N-1}{n}\frac{{2 - \rho _1
}}{{\left( {2n + 1} \right)\left( {1 - \rho _1 } \right) + 1}}
\\\notag
+ &bzN\sum\limits_{m = 0}^{N - 1} {\left( { - 1} \right)^m }
\binom{N-1}{m}\frac{{2 - \rho _2 }}{{\left( {2m + 1} \right)\left(
{1 - \rho _2 } \right) + 1}}~.

Then, (\ref{Eq:SER12}) in Theorem 2 is proved by (\ref{Eq:a}) and
(\ref{Eq:ps2}).

\begin{thebibliography}{1}

\bibitem{Katti2008}S. Katti, H. Rahul, W. Hu, D. Katabi, M. Medard,~and
J. Crowcroft, ``Xors in the air: Practical wireless network coding,"
\emph{IEEE/ACM Transactions on Networking}, vol. 16, no. 3, pp.
497--510, Jun. 2008.

\bibitem{Popovski2007}P. Popovski~and H. Yomo, ``Wireless network coding by
amplify-and-forward for bi-directional traffic flows," \emph{IEEE
Communication Letters}, vol. 11, no. 1, pp. 16--18, Jan. 2007.

\bibitem{Louie2010} R. H. Y. Louie, Y. Li,~ and B. Vucetic, ``Practical physical layer network
coding for two-way relay channels: Performance analysis and
comparison," \emph{IEEE Transactions on Wireless Communications},
vol. 9, no. 2, pp. 764--777, Feb. 2010.

\bibitem{Bletsas2006}A. Bletsas, A. Khisti, D. P. Reed, and A. Lippman, ``A simple
Cooperative diversity method based on network path selection,"
\emph{IEEE Journal on Selected Areas in Communications}, vol. 24,
no. 3, pp. 659--672, Mar. 2006.

\bibitem{Ibrahim2008}A. S. Ibrahim, A. K. Sadek, W. Su, and K. J. R. Liu, ``Cooperative communications with relay-selection: When to cooperate
and whom to cooperate with?," \emph{IEEE Transactions on Wireless
Communications}, vol. 7, no. 7, pp. 2814--2827, Jul. 2008.

\bibitem{Zhang2009} X. Zhang and Y. Gong, ``Adaptive power allocation in two-way amplify-and-forward relay
networks," in \emph{Proceedings of IEEE International Conference on
Communication}, Jun. 2009.

\bibitem{Kyu2009}K. Hwang, Y. Ko, and M.-S. Alouini, ``Performance
bounds for two-Way amplify-and-forward relaying based on relay path
selection," in \emph{Proceedings of Vehicular Technology
Conference}, Apr. 2009.

\bibitem{Song2010} L. Song, G. Hong, B. Jiao, and M. Debbah, ``Joint relay selection and analog network coding using differential modulation in two-way relay channels,"
\emph{IEEE Transactions on Vehicular Technology}, vol. 59, no. 6,
pp. 2932--2939, Jul. 2010.

\bibitem{Song2011} L. Song, ``Relay selection for two-way relaying with
amplify-and-forward protocols," \emph{IEEE Transactions on Vehicular
Technology}, vol. 60, no. 4, pp. 1954--1959, May 2011.

\bibitem{Jing2009} Y. Jing, ``A relay selection scheme for two-way amplify-and-forward relay
networks," in \emph{Proceedings of International Conference of
Wireless Communication and Signal Process}, Nov. 2009.

\bibitem{Nguyen2010}H. X. Nguyen, H. H. Nguyen, and T. Le-Ngoc, ``Diversity analysis of
relay selection schemes for two-way wireless relay networks," in
\emph{Wireless  Personal Communication}, Jan.  2010.

\bibitem{Torabi2010}M. Torabi and D. Haccoun, ``Capacity analysis of opportunistic
relaying in cooperative systems with outdated channel information,"
\emph{IEEE Communications Letters}, vol. PP, no. 99, pp. 1--3, Nov.
2010.

\bibitem{Suraweera2010}H. A. Suraweera, M. Soysa, C. Tellambura, and H. K. Garg, ``Performance analysis of partial relay selection with feedback delay,"
\emph{IEEE Signal Processing Letters }, vol. 17, no. 6, pp.
531--534, Jun. 2010.

\bibitem{Michalopoulos2010}D. S. Michalopoulos, H. A. Suraweera, G. K. Karagiannidis , and R. Schober, ``Amplify-and-forward relay selection with outdated
channel state information," in \emph{Proceedings of IEEE Global
Telecommunications Conference}, Dec. 2010.

\bibitem{Seyfi2010}M. Seyfi, S. Muhaidat, and J. Liang, ``Amplify-and-forward
selection cooperation with channel estimation error," in
\emph{Proceedings of IEEE Global Telecommunications Conference},
Dec. 2010.

\bibitem{Seung2009}S. Han, S. Ahn, E. Oh, and D. Hong, ``Effect
of channel estimation error on BER performance in cooperative
transmission," \emph{IEEE Transactions on Vehicular Technology},
vol. 58, no. 4, pp. 2083--2088, May 2009.

\bibitem{Gedik2009}B. Gedik and M. Uysal, ``Impact of imperfect channel estimation
on the performance of amplify-and-forward relaying," \emph{IEEE
Transactions on Wireless Communications}, vol. 8, no. 3, pp.
1468--1479, Mar. 2009.

\bibitem{Ding2011}Z. Ding and K. K. Leung, ``Impact of imperfect channel state
information on bi-directional communications with relay selection,"
\emph{IEEE Transactions on Signal Processing}, vol. 59, no. 11, pp.
5657--5662, Nov. 2011.

\bibitem{Yoo2006}T. Yoo and A. Goldsmith, ``Capacity and power allocation for
fading MIMO channels with channel estimation error," \emph{IEEE
Transactions on Information Theory}, vol. 52, no. 5, pp. 2203--2214,
May 2006.

\bibitem{Feifei2009}F. Gao, R. Zhang, and Y. Liang, ``Optimal channel
estimation and training design for two-way relay networks,"
\emph{IEEE Transactions on Communications}, vol. 57, no. 10, pp.
3024--3033, Oct. 2009.

\bibitem{Ramya2009}T. R. Ramya and S. Bhashyam, ``Using delayed feedback for antenna
selection in MIMO systems,'' \emph{IEEE Transactions on Wireless
Communications}, vol. 8, no. 12, pp. 6059--6067, Dec. 2009.

\bibitem{Zheng2003}Z. Wang and G. B. Giannakis, ``A simple and general parameterization quantifying performance in fading
channels,'' \emph{IEEE Transactions on Communications}, vol. 51, no.
8, pp. 1389--1398, Aug. 2003.


\bibitem{Abramowitz} M. Abramowitz and I. A. Stegun, \emph{Handbook of mathematical functions with formulas, graphs, and mathematical tables}, 9th ed.
    NewYork: Dover, 1970.

\bibitem{Gradshteyn94} I. S. Gradshteyn and I. M. Ryzhik, \emph{Table of integals, series, and products}, 5th Edition, Academic Press, 1994.

\bibitem{Simon}M. K. Simon and M.-S. Alouini, \emph{Digital Communication over Fading
Channels}. John Wiley  Sons, Inc., 2000.

\bibitem{Paoulis} A. Paoulis, S. U. Pialli, \emph{Probability, Random Variables and Stochastic
Processes}. 4th Edition, McGraw-Hill, 2002.

\bibitem{David1970} H. A. David, \emph{Order Statistics}. Jonh Wiley  Sons, Inc., 1970.

\end{thebibliography}


\newpage

\begin{figure}[h!]
\centering
\includegraphics[height=3.8in,width=4.5in]{perfect.eps}
\caption{Analytical and simulated SER with perfect CSI, with
different ,  and .}
\label{fig:4}
\end{figure}

\begin{figure}[h!]
\centering
\includegraphics[height=3.8in,width=4.5in]{pf.eps}
\caption{Analytical and simulated SER of  with delay effect,
with different  and , .} \label{fig:1}
\end{figure}

\begin{figure}[h!]
\centering
\includegraphics[height=3.8in,width=4.5in]{pe.eps}
\caption{Analytical and simulated SER of  with estimation
error, with different  and ,
.} \label{fig:2}
\end{figure}

\begin{figure}[h!]
\centering
\includegraphics[height=3.8in,width=4.5in]{pepf.eps}
\caption{Analytical and simulated SER of  with delay effect and
estimation error, with different  and , .}
\label{fig:3}
\end{figure}

\end{document}
