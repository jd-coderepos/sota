\documentclass[10pt,a4paper]{article}
\usepackage{amsmath}
\usepackage{amsfonts}
\usepackage{amssymb}
\usepackage{graphicx}
\usepackage[ruled]{algorithm2e}
\usepackage{booktabs}
\usepackage{tabularx}
\usepackage{lscape}


\addtolength{\textwidth}{5cm}
\addtolength{\hoffset}{-2.5cm}
\addtolength{\textheight}{4cm}
\addtolength{\voffset}{-2cm}

\renewcommand{\thefootnote}{\fnsymbol{footnote}}

\newtheorem{te}{Theorem}[section]
\newtheorem{lemma}{Lemma}[section]
\newtheorem{remark}{Remark}[section]
\newtheorem{pro}{Proposition}[section]

\newenvironment{proof} {\par \noindent \textbf{Proof: }}{\QED\par \bigskip \par}
\newcommand{\QED}{\hfill}
\newcommand{\rz}{\vspace{0.1cm}}

\renewcommand{\thefootnote}{\fnsymbol{footnote}}

\title { \bigskip
    \bf {Constructions of hamiltonian graphs with
    bounded degree and diameter }
    \thanks{Supported by the Research Program P1-0285 of Slovenian Agency for Research and the Grant 144007 of Serbian Ministry of Science and Technological Development.}
}


\author
{
{\large \sc Aleksandar Ili\' c\footnotemark[3]} \\
{\em \normalsize Faculty of Sciences and Mathematics, University of Ni\v s, Serbia} \\
{\normalsize e-mail: { \tt aleksandari@gmail.com }} \and
{\large \sc Dragan Stevanovi\' c} \\
{\em \normalsize University of Primorska---FAMNIT, Glagolja\v ska 8, 6000 Koper, Slovenia,} \\
{\em \normalsize Mathematical Institute, Serbian Academy of Science
and Arts,}
{\em \normalsize Knez Mihajlova 36, 11000 Belgrade, Serbia }\\
{\normalsize e-mail: { \tt dragance106@yahoo.com}} }


\begin{document}

\maketitle

\vspace{-0.5cm}


\begin{abstract}
    Token ring topology has been frequently used in the design of
    distributed loop computer networks and one measure
    of its performance is the diameter. We
    propose an algorithm for constructing hamiltonian graphs with 
    vertices and maximum degree  and diameter ,
    where  is an arbitrary number. The number of edges is asymptotically bounded by
    .
    In particular, we construct a family of hamiltonian
    graphs with diameter at most ,
    maximum degree  and at most  edges.
\end{abstract}

{\bf {Keywords:}} hamiltonian cycle, token ring, diameter, binary
tree, graph algorithm \rz

        \footnotetext[3]
        {
        Corresponding author. If possible, send your correspondence via e-mail.
        Otherwise, postal address is:
        Department of Mathematics and Informatics, Faculty of Sciences and Mathematics,
        Vi\v segradska 33, 18000 Ni\v s, Serbia
        }


\section{Introduction}



An undirected graph  can be used as a mathematical model
for computer networks, where  is the set of vertices and  is
the set of edges. The number of edges adjacent to a vertex  is
called the degree of the vertex~. A graph is regular if all
vertices have equal degrees. The distance  between two
vertices  and  is the number of edges on a shortest path
between  and . The diameter  of the graph is the maximum
distance between any pair of vertices: . A cycle is a sequence of three or more vertices such that two
consecutive vertices are adjacent and with no repeated vertices
other than the start and end vertex. A \emph{hamiltonian cycle} is a
cycle that visits each vertex of a graph exactly once. A graph 
is -hamiltonian if, after removing an arbitrary vertex or an
edge, it still remains hamiltonian. A -hamiltonian graph G is
optimal if it contains the least number of edges among all
-hamiltonian graphs with the same number of vertices as . \rz



Networks with at least one ring structure (hamiltonian cycle) are
called loop networks. Distributed loop networks are extensions of
ring networks and are widely used in the design and implementation
of local area networks and parallel processing architectures. There
are many mutually conflicting requirements when designing the
topology of a computer network. For example, no pair of processors
should be too far apart in order to support efficient parallel
computation demands. The hamiltonian property is one of the major
requirements. The \emph{token passing} is a channel access method
where data is transmitted sequentially from one ring station to the
next with a control token circulating around the ring controlling
access. \rz

An open problem considered in a survey  on
distributed loop networks is following: Find hamiltonian networks,
-regular on  vertices with a diameter of order . This problem is related to the famous  problem
in which we want to construct a graph of  vertices with maximum
degree  such that the diameter  is minimized, but
hamiltonicity is not an issue. The lower bound on the diameter 
is called the \emph{Moore bound},





Harary and Hayes  presented a family of optimal
-hamiltonian planar graphs on  vertices. Wang, Hung and Hsu
 presented another family of optimal -hamiltonian
graphs, each of which is planar, hamiltonian, cubic, and of diameter
. In the literature three other families of cubic,
planar and optimal -hamiltonian graphs with diameter 
are described. These constructions are possible only for special
choices of , as shown in Table \ref{optimalGraphs}. \rz

\begin{table}[ht]
\centering \small
    \begin{tabular}{ l l l l l }
    \toprule

    Reference & Name &  &  & Comment \\log_2 n - 10 \leqslant D (G) \leqslant \log_2 n + \log_2
    \log_2 n + 10D \leqslant 2 \cdot \lfloor \log_{\Delta - 1} n \rfloor.D (G) \leqslant 2 \cdot \lfloor \log_{\Delta - 1} n \rfloor.L (n) = \left \{
    \begin{array}{ll}
        n - \lfloor \frac {n}{\Delta - 1} \rfloor , & \quad
        \mbox{ if } n \equiv 1 \pmod{ (\Delta - 1) } \\
        n - 1 - \lfloor \frac {n}{\Delta - 1} \rfloor , & \quad
        \mbox{ if } n \not \equiv 1 \pmod{ (\Delta - 1) } \\
    \end{array} \right.L (n) = L (n - \Delta + 1) + \Delta - 2.
    |E| < \frac{n \Delta}{2} - \frac {\Delta - 3}{2} \left
    ( n - \left \lfloor \frac{n}{ \Delta - 1} \right \rfloor - 1 \right
    ) \leqslant \frac{3n}{2} + \frac {\Delta - 3}{2} \left (\frac
    {n}{\Delta - 1} + 1 \right ) = \left(2 - \frac{1}{\Delta -
    1} \right) n + \frac{\Delta - 3}{2}.
f (k) = 2 (k - 1) + 2 \cdot \sum_{i = 0}^{k - 2} f (i).f (k) = \left \{
\begin{array}{ll}
         \frac {2}{3} \left ( 2^k - 1 \right ), & \quad \mbox{if  is even} \\
         \frac {2}{3} \left ( 2^k - 2 \right ), & \quad \mbox{if  is odd} \\
\end{array} \right.
f (2k) &= 2 (2k - 1) + 2 \cdot \sum_{i = 0}^{k - 1} f (2i) + 2 \cdot
\sum_{i = 1}^{k - 1} f (2i - 1)
= 4k - 2 + \frac{4}{3} \cdot \sum_{i = 0}^{2k - 2} 2^i -
\frac{4}{3} \cdot (3k - 2) = \frac{2}{3} \cdot (2^{2k} - 1). \\
f (2k + 1) &= 2 (2k + 1 - 1) + 2 \cdot \sum_{i = 0}^{k - 1} f (2i) +
2 \cdot \sum_{i = 1}^{k} f (2i - 1)
= 4k + \frac{4}{3} \cdot \sum_{i = 0}^{2k - 1} 2^i - \frac{4}{3}
\cdot 3k = \frac{2}{3} \cdot (2^{2k + 1} - 2).
f (0) + f (1) + \ldots + f (k - 1) + k = \frac {f (k + 1)}{2} =
\left \lfloor \frac{n}{3} \right \rfloor.
\left( 2 - \frac{1}{\Delta - 1} - \frac{(\Delta - 2)^2}{(\Delta -
1)^3} \right) n + \frac{\Delta - 3}{2} + 2 (\Delta - 2)

edges. The same approach may be applied to complete binary subtrees
of greater heights to obtain a slightly finer bound for the total
number of edges.


\section{Concluding remarks}

The proposed algorithm may be modified to construct a planar
hamiltonian graph.

\begin{te}
    By appropriately choosing descendant and unvisited
    leaves in the algorithm, one can assure that the constructed graph is planar.
\end{te}

\begin{proof}
    In order to prove the theorem, we will construct a hamiltonian path that
    starts at the leftmost leaf and ends in the nearest leaf (neighboring
    leaf or leaf that is at distance three from it). For small values
    of , this can be easily verified. In the general case
    we first go upwards to the root and then to the rightmost leaf. Now,
    the -ary tree is partitioned into smaller trees, which will be
    traversed by induction from left to right. These binary trees
    do not have vertices in common, and we can independently add
    necessary edges which do not intersect the existing edges.
    We reduce our problem to the previous case by going to the leftmost
    leaf. Now we have disjoint trees and we traverse them starting
    from the leaves. Finally, we add the last edge without intersection problems
    (as in Figure~\ref{example}).
\end{proof}



For the case , in our implementation we always choose a
leaf that is farthest from the current leaf. This heuristic is done
by the breadth first search. We use only three arrays of length ,
so memory requirements are linear in . Time complexity is , because the number of edges is . The diameters of the examples of cubic graphs constructed by
this algorithm are shown in Figure \ref{graphics}, where the
-axis carries  and  is the number of nodes. It is obvious
from this figure that the constant~ from our bound is not the
best possible.
\newpage


\begin{figure}[h]
  \center
  \includegraphics [width = 10.5cm]{graphics.eps}
  \caption { \textit{The size of diameter for  to  for } }
  \label{graphics}
\end{figure}

Instead of choosing the leaves at random, we can do it more
sophisticatedly and further reduce the diameter of the graph. One
possible way is to add a matching which connects only the leaves on
the last level in the root's left and right subtree. This way we can
still apply the algorithm, but if we have to choose a random leaf to
continue---we first check whether the paired leaf in the matching is
marked. This way we decrease the diameter by a constant, which is at
least one. We leave for future study to see whether this approach
may be used to construct -edge hamiltonian graphs. \vspace{0.2cm}

{\bf Acknowledgement: } The authors are grateful to the reviewers
for their valuable comments and suggestions.



\begin{thebibliography}{99}

\bibitem{alon00}
  N. Alon, A. Gy\'{a}rf\'{a}s, M. Ruszink\'{o}: \textit{Decreasing the diameter of bounded degree graphs},
  Journal of Graph Theory 35 (2000), 161--172

\bibitem{bermond95}
  J. C. Bermond, F. Comellas, D. F. Hsu: \textit{Distributed Loop Computer Networks: A Survey},
  Journal of Parallel and Distributed Computing 24 (1995), 2--10


\bibitem{bollobas88}
  B. Bollob\' as, F. R. K. Chung: \textit{The diameter of a cycle plus
  a random matching}, SIAM Journal on Discrete Mathematics 1 (1988), 328--333

\bibitem{capablo03}
  M. Capalbo: \textit{An explicit construction of lower diameter cubic graphs}, SIAM Journal  on Discrete Mathematics 16
  (2003), 630--634

\bibitem{cormen01}
  T. H. Cormen, C. E. Leiserson, R.L. Rivest, C. Stein: \textit{Introduction to Algorithms}, Second Edition, MIT
  Press, 2001.


\bibitem{harary96}
  F. Harary, J. P. Hayes: \textit{Node fault tolerance in graphs},
  Networks 27 (1996), 19--23

\bibitem{hung99}
  Chun-Nan Hung, Lih-Hsing Hsu, Ting-Yi Sung: \textit{Christmas tree: A versatile 1-fault-tolerant design for token rings},
  Information Processing Letters 72 (1999), 55--63


\bibitem{kao03}
  Shin-Shin Kao, Lih-Hsing Hsu: \textit{Brother trees: A family of optimal -hamiltonian and -egde hamiltonian graphs},
  Information Processing Letters 86 (2003), 263--269

\bibitem{leighton92}
  F. T. Leighton: \textit{Introduction to Parallel Algorithms and
  Architectures: Arrays, Trees and Hypercubes}, Morgan Kaufmann, San
  Mateo, CA, 1992.

\bibitem{robinson92}
  R. W. Robinson, N. C. Wormald: \textit{Almost all regular graphs are hamiltonian},
  Random Structures and Algorithm, Volume 5 (2), (1992), 363--374

\bibitem{wang98}
  Jeng-Jung Wang, Ting-Yi Sung, Lih-Hsing Hsu, Men-Yang Lin:
  \textit{A New Family of Optimal 1-hamiltonian Graphs with Small Diameter},
  Proceedings of the 4th Annual International Conference on Computing and Combinatorics
  (1998), 269--278

\bibitem{wang98a}
  Jeng-Jung Wang, C.N. Hung, L. H. Hsu:
  \textit{Optimal -hamiltonian graphs}, Information Processing Letters 65
  (1998), 157--161

\end{thebibliography}


\end{document}
