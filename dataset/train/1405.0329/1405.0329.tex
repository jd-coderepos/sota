\documentclass[10pt]{article}
\setlength{\textwidth}{17cm} \setlength{\textheight}{23cm}
\oddsidemargin=-0.3cm \topmargin=-1.5cm

\usepackage{amsthm,amsmath,amssymb}
\usepackage{microtype}
\usepackage{paralist}
\usepackage{charter,eulervm}\usepackage{subfig}
\usepackage[pdftex,breaklinks,colorlinks,
linkcolor=blue,
citecolor=blue,
urlcolor=blue]{hyperref}
\usepackage{tkz-graph}
\usetikzlibrary{arrows,shapes,decorations.pathmorphing}
\graphicspath{{./figs/}}

\newtheorem{theorem}{Theorem}[section]
\newtheorem{lemma}[theorem]{Lemma}
\newtheorem{corollary}[theorem]{Corollary}
\newtheorem{proposition}[theorem]{Proposition}
\newtheorem{observation}{Observation}
\newtheorem{remark}[theorem]{Remark}
\newtheorem{reduction}{Reduction}
\newtheorem{claim}{Claim}
\newtheorem{definition}{Definition}
\def\boxit#1{\vbox{\hrule\hbox{\vrule\kern4pt
  \vbox{\kern1pt#1\kern1pt}
\kern2pt\vrule}\hrule}}
\newcommand{\keywords}[1]{\bigskip \par\noindent
{\small{\em Keywords\/}: #1}}

\newcommand{\badgraph}{minimal forbidden induced subgraph}
\newcommand{\hv}[1]{\ensuremath{N_H[#1]}}
\newcommand{\nhcag}{normal Helly circular-arc graph}
\newcommand{\ce}[1]{\ensuremath{{\mathtt{cp}(#1)}}}
\newcommand{\cce}[1]{\ensuremath{{\mathtt{ccp}(#1)}}}
\newcommand{\lp}[1]{\ensuremath{{\mathtt{lp}(#1)}}}
\newcommand{\rp}[1]{\ensuremath{{\mathtt{rp}(#1)}}}
\newcommand{\head}[1]{\ensuremath{{\mathtt{last}(#1)}}}
\newcommand{\tail}[1]{\ensuremath{{\mathtt{first}(#1)}}}
\newcommand{\stpath}[2]{(, )-path}
\newcommand{\stsep}[2]{(, )-separator}
\newcommand{\cT}{\ensuremath{{\cal T}}}
\newcommand{\ec}{\ensuremath{E_{\text{c}}}}
\newcommand{\ecc}{\ensuremath{E_{\text{cc}}}}
\newcommand{\oc}{\ensuremath{T_{\text{c}}}}
\newcommand{\oo}{\ensuremath{T}}
\newcommand{\occ}{\ensuremath{T_{\text{cc}}}}
\newcommand{\og}[1]{\ensuremath{\phi(#1)}}
\newcommand{\comment}[1]{\hfill  {\em #1}}
\newcommand{\yixin}[1]{\marginpar{!!!}\textcolor{blue}{#1}}

\title{Forbidden Induced Subgraphs of Normal Helly \\Circular-Arc
  Graphs: Characterization and Detection\thanks{Preliminary results of
    this paper appeared in the proceedings of SBPO 2012
    \cite{grippo-12-cag-without-dominating-triple} and FAW 2014
    \cite{cao-14-recognizing-nhcag}.}}

\author{Yixin Cao\thanks{Institute for Computer Science and Control,
    Hungarian Academy of Sciences.  Email:
    \href{mailto:yixin@sztaki.hu}{\tt yixin@sztaki.hu}.  Supported by
    the European Research Council (ERC) under the grant 280152 and the
    Hungarian Scientific Research Fund (OTKA) under the grant
    NK105645. }  \and Luciano N. Grippo\thanks{Instituto de Ciencias,
    Universidad Nacional de General Sarmiento, Los Polvorines, Buenos
    Aires, Argentina.
    Email:\href{mailto:lgrippo@ungs.edu.ar}{lgrippo@ungs.edu.ar},
    \href{mailto:msafe@ungs.edu.ar}{msafe@ungs.edu.ar}.  Partially
    supported by CONICET PIP 11220120100450CO and ANPCyT PICT
    2012-1324 grants.}
  \addtocounter{footnote}{-1}
  \and Mart\'in D. Safe\footnotemark
 }

\date{\today}

\begin{document}
\maketitle
\tikzstyle{corner}  = [fill=blue,inner sep=2.5pt]
\tikzstyle{special} = [fill=black,circle,inner sep=2pt]
\tikzstyle{vertex}  = [fill=black,circle,inner sep=2pt]
\tikzstyle{original edge} = [thick,-,blue,dashed]
\tikzstyle{forbidden edge} = [dashed,-,red]
\tikzstyle{edge}    = [draw,thick,-]
\tikzstyle{at edge} = [draw,ultra thick,-,red]

\begin{abstract}
  A normal Helly circular-arc graph is the intersection graph of arcs
  on a circle of which no three or less arcs cover the whole circle.
  Lin, Soulignac, and Szwarcfiter [Discrete Appl.\ Math.\ 2013]
  characterized circular-arc graphs that are not normal Helly
  circular-arc graphs, and used it to develop the first recognition
  algorithm for this graph class.  As open problems, they ask for the
  forbidden induced subgraph characterization and a direct recognition
  algorithm for normal Helly circular-arc graphs, both of which are
  resolved by the current paper.  Moreover, when the input is not a
  normal Helly circular-arc graph, our recognition algorithm finds in
  linear time a minimal forbidden induced subgraph as certificate.
 \end{abstract}
 \keywords{certifying algorithms, holes, interval models, (minimal)
   forbidden induced subgraphs, (normal, Helly) circular-arc models.}

\section{Introduction} 
This paper will be only concerned with undirected and simple graphs.
A graph is a \emph{circular-arc graph} if its vertices can be assigned
to arcs on a circle such that two vertices are adjacent if and only if
their corresponding arcs intersect.  Such a set of arcs is called a
\emph{circular-arc model} of this graph.  If there is some point on
the circle that is not covered by any arc in the model, then the graph
is an \emph{interval graph}, and it can also be represented by a set
of intervals on the real line, which is called an \emph{interval
  model}.  Circular-arc graphs and interval graphs are two of the most
famous intersection graph classes, and both have been studied
intensively for decades.  However, in contrast to interval graphs, our
understanding of circular-arc graphs is far limited, and to date some
fundamental problem remains unsolved.

\begin{figure*}[h]
  \centering\footnotesize
  \subfloat[long claw]{\label{fig:long-claw}
    \includegraphics{long-claw.pdf} 
  }
  
  \subfloat[whipping top]{\label{fig:whipping-top}
    \includegraphics{whipping-top.pdf} 
  }
  
  \subfloat[\dag]{\label{fig:net}
    \includegraphics{dag.pdf} 
  }
  
  \subfloat[\ddag]{\label{fig:tent}
    \includegraphics{ddag.pdf} 
  }
  \caption{Chordal minimal forbidden induced graphs.}
  \label{fig:at}
\end{figure*}

One fundamental combinatorial problem on a graph class is its
characterization by forbidden induced subgraphs.  For example,
Lekkerkerker and Boland \cite{lekkerkerker-62-interval-graphs} showed
in 1962 that a graph is an interval graph if and only if it contains
neither a hole (i.e., a induced cycle of length at least four) nor any
graph in Fig.~\ref{fig:at} as an induced subgraph.  Recall that holes
are the forbidden induced subgraphs of \emph{chordal graphs}, which
are the intersection of subtrees of a tree.  In contrast, since it was
first asked by Hadwiger et
al.~\cite{hadwiger-64-combinatorial-geometry} in 1964, all efforts
attempting to characterize circular-arc graphs by forbidden induced
subgraphs have succeeded only partially.  Tucker made the most
significant contribution to the study of the class of circular-arc
graphs and its subclasses, which includes the forbidden induced
subgraph characterizations of both unit circular-arc graphs (i.e., a
graph with a circular-arc model where every arc has the same length)
and proper circular-arc graphs (i.e., a graph with a circular-arc
model where no arc properly contains another)
\cite{tucker-74-structures-cag}; we will see more later.  There is a
similar line of research for other proper subclasses of circular-arc
graphs, which aims at determining their forbidden induced subgraphs or
some other kinds of obstructions; for this we refer to the surveys of
Lin and Soulignac \cite{lin-09-cag-and-subclasses} and Dur\'an et
al.~\cite{duran-14-survey} and references therein.

One fundamental algorithmic problem on a graph class is its
recognition, i.e., to efficiently decide whether a given graph belongs
to this class or not.  For intersection graph classes, all
{recognition algorithms} known to the authors provide an intersection
model when the membership is asserted.  Most of them, on the other
hand, simply return ``NO'' for non-membership, while one might also
want some verifiable \emph{certificate} for some reason
\cite{mcconnell-11-survey-certifying-algorithms}.  A recognition
algorithm is \emph{certifying} if it provides both positive and
negative certificates.  There are different forms of negative
certificates, while a minimal forbidden induced subgraph is arguably
the simplest and most preferable of them
\cite{heggernes-07-certifying-fis}.  Kratsch et
al.~\cite{kratsch-06-certifying-interval-and-permutation} reported a
certifying recognition algorithm for interval graphs, which in linear
time returns either an interval model of an interval graph or a
forbidden induced subgraph for a non-interval graph.  Although its
returned forbidden induced subgraph is not necessarily minimal, a
minimal one can be easily retrieved from it (see also
\cite{lindzey-13-find-forbidden-subgraphs} for another approach).
Likewise, a hole can be detected from a non-chordal graph in linear
time \cite{tarjan-84-chordal-recognition}.  On the other hand,
although a circular-arc model of a circular-arc graph can be produced
in linear time \cite{mcconnell-03-recognition-cag}, it remains a
challenging open problem to find a negative certificate for a
non-circular-arc graph.

The complication of circular-arc graphs may be attributed to two
special intersection patterns of circular-arc models that are not
possible in interval models.  The first is two arcs intersecting in
both ends, and a {circular-arc model} is called \emph{normal} if no
such pair exists.  The second is a set of arcs intersecting pairwise
but containing no common point, and a {circular-arc model} is called
\emph{Helly} if no such set exists.  Normal and Helly circular-arc
models are precisely those without three or less arcs covering the
whole circle
\cite{mckee-03-restricted-cag,lin-13-nhcag-and-subclasses}.  A graph
that admits such a model is called a \emph{normal Helly circular-arc
  graph}.  In particular, all interval graphs are normal Helly
circular-arc graphs.  

\begin{figure}[h!]
  \centering
  \subfloat[A circular-arc graph ]{\label{fig:example-graph}
    \includegraphics{sun.pdf} 
  }
  \qquad
  \subfloat[{A normal model of }]{\label{fig:normal-model}
    \includegraphics{normal-model.pdf} 
  }
  \qquad
  \subfloat[{A Helly model of }]{\label{fig:helly-model}
    \includegraphics{helly-model.pdf} 
}
  \caption{a non-\nhcag\ and its circular-arc models}
  \label{fig:normal-and-helly}
\end{figure}

A word of caution is worth on the definition of \nhcag s.  One graph
might admit both a normal circular-arc model and a Helly circular-arc
model but not a normal and Helly circular-arc model.  For example, see
the graph and its models in Fig.~\ref{fig:normal-and-helly}.  The fact
that the model of Fig.~\ref{fig:normal-model} (resp.,
Fig.~\ref{fig:helly-model}) is not Helly (resp., normal) can be
evidenced by arc set  (resp., ).  One may want to
verify that arranging a normal and Helly circular-arc model for this
graph is out of the question.  This example convinces us that the set
of \nhcag s is {\em not} equivalent to the intersection of normal
circular-arc graphs and Helly circular-arc graphs, but a proper subset
of it.

Let us mention some previous work related to normal Helly circular-arc
graphs.  Tucker \cite{tucker-75-coloring-cag} gave an algorithm that
outputs a proper coloring of any given normal Helly circular-arc graph
using at most  colors, where  denotes the size of a
maximum clique.  Note that by the Helly property,  is
equivalent to the maximum number of arcs covering a single point on
the circle.  This is tight as any odd hole, which has  and
needs at least three colors, is a \nhcag.  In the study of convergence
of circular-arc graphs under the clique operator, Lin et
al.~\cite{lin--10-clique-operator-cag} observed that normal Helly
circular-arc graphs arose naturally.  They then
\cite{lin-13-nhcag-and-subclasses} undertook a systematic study of
normal Helly circular-arc graphs as well as its subclass.  Their
results include a partial characterization of \nhcag s by forbidden
induced subgraph (more specifically, those restricted to Helly
circular-arc graphs), and a linear-time recognition algorithm (by
calling a recognition algorithm for circular-arc graphs).  As open
problems, they ask for determining the remaining minimal forbidden
induced subgraphs, and designing a direct recognition algorithm, both
of which are resolved by the current paper.

The first main result of this paper is a complete characterization of
\nhcag s by forbidden induced subgraphs.  A wheel (resp., )
comprises a hole and another vertex completely adjacent (resp.,
nonadjacent) to it.

\begin{theorem}\label{thm:characterization}
A graph is a normal Helly circular-arc graph if and only if it
contains no , wheel, or any graph depicted in Figs.~\ref{fig:at}
and \ref{fig:normal-helly}.
\end{theorem}
\begin{figure*}[t]
  \centering
  \subfloat[]{\label{fig:long-claw-1}
    \includegraphics{1-1.pdf} 
  }
  
  \subfloat[twin-]{\label{fig:g-2}
    \includegraphics{1-2.pdf} 
  }
  
  \subfloat[domino]{\label{fig:domino}
    \includegraphics{1-3.pdf} 
  }
    
  \subfloat[]{\label{fig:complement-c-6}
    \includegraphics{1-4.pdf} 
  }
    
  \subfloat[FIS-1]{\label{fig:fis-1}
    \includegraphics{1-5.pdf} 
  }
  
  \subfloat[FIS-2]{\label{fig:fis-2}
    \includegraphics{1-6.pdf} 
  }
  \caption{Non-chordal and finite minimal forbidden induced graphs.}
  \label{fig:normal-helly}
\end{figure*}

It is easy to use the definition to verify that a normal Helly
circular-arc graph is chordal if and only if it is an interval graph.
An interval model is always a normal and Helly circular-arc model, but
an interval graph might have circular-arc model that is neither normal
nor Helly, e.g., consider .  For non-chordal graphs we have:
\begin{proposition}[\cite{mckee-03-restricted-cag,lin-13-nhcag-and-subclasses}]
  \label{thm:always-noraml-and-helly}
  If a normal Helly circular-arc graph  is not chordal, then every
  circular-arc model of  is normal and Helly.
\end{proposition}
These observations inspire us to recognize \nhcag s as follows.  If
the input graph is chordal, it suffices to check whether it is an
interval graph.  Otherwise, we try to build a circular-arc model of
it, and if we succeed, verify whether the model is normal and Helly.  Lin
et al.~\cite{lin-13-nhcag-and-subclasses} showed that this approach
can be implemented in linear time.  Moreover, if there exists a set of
at most three arcs covering the circle, then their algorithm returns
it as a certificate.
This algorithm, albeit conceptually simple, suffers from twofold
weakness.  First, it needs to call some recognition algorithm for
circular-arc graphs, while all known algorithms are extremely
complicated.  Second, it is very unlikely to deliver a negative
certificate in general.  

The second main result of this paper is the following direct
certifying algorithm for recognizing \nhcag s, which would be
desirable for both efficiency and the detection of negative
certificates.  From now on, unless otherwise stated, whenever we refer
to a ``minimal forbidden induced subgraph'' it should be understood a
minimal forbidden induced subgraph for the class of normal Helly
circular-arc graphs.  We use  and 
throughout.
\begin{theorem}\label{thm:certifying-algorithm}
  There is an -time algorithm that given a graph , either
  constructs a normal and Helly circular-arc model of , or finds a
  minimal forbidden induced subgraph of .
\end{theorem}

It is clear that each graph specified in
Theorem~\ref{thm:characterization} is a minimal forbidden induced
subgraph.  First, every graph in Fig.~\ref{fig:at} is chordal but
non-interval graph, and thus cannot be a \nhcag.  Second, a  is
not a circular-arc graph, while a wheel cannot be arranged without
three or less arcs covering the circle.  Third, every graph in
Fig.~\ref{fig:normal-helly} has only a small number of vertices and
can be easily checked.  Therefore, to prove
Theorem~\ref{thm:characterization}, it suffices to show that a graph
containing none of them is a \nhcag.  That fact was actually proved in
\cite{grippo-12-cag-without-dominating-triple}, but the resulting
proof of Theorem 1.1 given there does not provide a linear-time
procedure to find the corresponding forbidden induced subgraphs when
the graph is not a normal Helly circular-arc graph.  Since the
algorithm we use to prove Theorem~\ref{thm:certifying-algorithm}
always finds such a subgraph in this case,
Theorem~\ref{thm:characterization} follows from the correctness proof
of our algorithm as a corollary.

Let us briefly discuss the basic idea behind the way we deal with a
non-chordal graph .  If  is a \nhcag, then for any vertex  of
, both  and its complement induce nonempty interval
subgraphs.  The main technical difficulty is how to combine interval
models for them to make a circular-arc model of .  For this purpose
we build an auxiliary graph  by taking two identical copies
of  and appending them to the two ends of 
respectively.  {The shape of symbol  is a good hint for
  understanding the structure of the auxiliary graph.} We show that
 is an interval graph and more importantly, a circular-arc
model of  can be produced from an interval model of .  On
the other hand, if  is not a \nhcag, then  cannot be an
interval graph.  In this case we use the following procedure to obtain
a minimal forbidden induced subgraph of .
\begin{theorem}\label{thm:negative-certificate}
  Given a minimal non-interval induced subgraph of , we can
  in  time find a \badgraph\ of .
\end{theorem}
The crucial idea behind our certifying algorithm is a novel
correlation between normal Helly circular-arc graphs and interval
graphs, which can be efficiently used for algorithmic purpose.  This
was originally proposed in the detection of small forbidden induced
subgraph of interval graphs \cite{cao-14-almost-interval-recognition},
i.e., the opposite direction of the current paper.  In particular, in
\cite{cao-14-almost-interval-recognition} we have used a similar
definition of the auxiliary graph and pertinent observations.
However, the main structures and the procedures for the detection of
forbidden induced subgraphs divert completely.  For example, the most
common forbidden induced subgraphs in
\cite{cao-14-almost-interval-recognition} are - and -holes,
which, however, are allowed in normal Helly circular-arc graphs.  This
means that the interaction between  and  are far more
subtle, and thus the detection of \badgraph s in the current paper is
significantly more complicated than that of
\cite{cao-14-almost-interval-recognition}.

\section{The recognition algorithm}\label{sec:recognition}
All graphs are stored as adjacency lists.  We use the customary
notation  to mean , and  to mean .  The \emph{degree} of a vertex  is defined by , where , called the \emph{neighborhood} of ,
comprises all vertices  such that .  The \emph{closed
  neighborhood} of  is defined by .  For a
vertex set , its closed neighborhood and neighborhood are defined
by  and ,
respectively.  Exclusively concerned with induced subgraphs, we use
 to denote both a subgraph and its vertex set.

Consider a circular-arc model .  If every point of the circle
is contained in some arc in , then we can find an
inclusion-wise minimal set  of arcs that cover the entire circle.
If  is normal and Helly, then  consists of at least four
vertices and thus corresponds to a hole.  Therefore, a \nhcag\  is
chordal if and only if it is an interval graph, for which it suffices
to call the algorithms of
\cite{kratsch-06-certifying-interval-and-permutation,lindzey-13-find-forbidden-subgraphs}.
We are hence focused on graphs that are not chordal.  We call the
algorithm of Tarjan and Yannakakis
\cite{tarjan-84-chordal-recognition} to detect a hole .
\begin{proposition}\label{lem:fundamental}
  Let  be a hole of a circular-arc graph .  In any circular-arc
  model of , the union of arcs for  covers the whole circle.
  In other words, .
\end{proposition}

The indices of vertices in the hole  should be understood as modulo , e.g., .  By Proposition~\ref{lem:fundamental}, every vertex
should have neighbors in .  We use  as a shorthand for
, regardless of whether  or not.  We start from
characterizing  for every vertex : we specify some
forbidden structures not allowed to appear in a \nhcag, and more
importantly, we show how to find a minimal forbidden induced subgraph
if one of these structures exists.  The fact that they are forbidden
can be easily seen from the definition of normal and Helly and
Proposition~\ref{lem:fundamental}, and hence the proofs given below
will focus on the detection of \badgraph s.
\begin{lemma}\label{lem:non-consecutive}
  For every vertex , we can in  time find either a proper
  sub-path of  induced by , or a \badgraph.
\end{lemma}
\begin{proof}
  We pre-allocate a list {\sf IND} of  slots, initially all
  empty.  For each neighbor of , if it is , then add  into
  the next empty slot of {\sf IND}.  After all neighbors of  have
  been checked, we shorten {\sf IND} by removing empty slots from the
  end, which leaves  slots.  If  is  or ,
  then we return  and  as a  or wheel.  In the remaining
  case,  is a nonempty and proper subset of .  We radix
  sort {\sf IND}; let  and  be its first and last elements
  respectively.

  Starting from the first element, we traverse {\sf IND} to the end
  for the first  such that .  If
  no such  exists, then we return () as the path.
  In the remaining cases, we may assume that we have found the ;
  let  and .  We continue
  to traverse from  to the end of {\sf IND} for the first 
  such that .  This step has three
  possible outcomes:
  \begin{inparaenum}[(1)]
  \item if  is found, then  and ;
  \item if no such  is found, and at least one of  and
     holds, then  and ; and
  \item otherwise (, , and  is not found).
  \end{inparaenum}
  In the third case, we return () as the path induced by \hv{v}.  In the first two cases,
   and  are defined, and .  In other words, we
  have two nontrivial sub-paths, ()
  and (), of  such that  is
  adjacent to their ends but none of their inner vertices.

  If , then we return () and  as a .  Likewise, if 
  for some  with , then we return
  () and  as a ;
  note this must hold true when  is adjacent to both 
  and .  Hence we may assume , and
  without loss of generality, .  

  If , then we return
  \begin{inparaenum}[(\itshape 1\upshape)]
  \item  as a  when ;
  \item  as a twin- when  and ;
  \item  as an FIS-1 when  and ; or
  \item  as a domino
    when .
  \end{inparaenum}
  Otherwise, , and we return
  \begin{inparaenum}[(\itshape 1\upshape)]
  \item  as a twin- when ;
  \item  as an FIS-2 when  and ;
  \item () and  as a 
    when  and ; or
  \item () and  as a  when
    .
  \end{inparaenum}

  The construction of {\sf IND} takes  time.  In the same
  time we can traverse it to find indices .  The rest
  uses constant time.  This concludes the time analysis and completes
  the proof.
\end{proof}

We designate the ordering  of traversing  as
\emph{clockwise}, and the other \emph{counterclockwise}.  In other
words, edges  and  are clockwise and
counterclockwise from , respectively.  Now let  be the path
induced by .  We can assign a direction to  in accordance
to the direction of , and then we have clockwise and
counterclockwise ends of .  For technical reasons, we assign
canonical indices to the ends of the path  as follows.
\begin{definition}
  For each vertex , we denote by \tail{v} and \head{v} the
  indices of the counterclockwise and clockwise, respectively, ends of
  the path induced by \hv{v} in  satisfying
  \begin{itemize}
  \item  {if } ; or
  \item , {otherwise}.
  \end{itemize}
\end{definition}
It is possible that , when .  In
general, , and  or  for each  with .  The indices
 and  can be easily retrieved from
Lemma~\ref{lem:non-consecutive}, with which we can check the adjacency
between  and any vertex  in constant time.  Now consider
the neighbors of more than one vertices in .
\begin{lemma}\label{lem:non-consecutive-2}
  Given a pair of adjacent vertices  such that  and
   are disjoint, then in  time we can find a
  \badgraph.
\end{lemma}
\begin{proof}
  Clearly, neither of  and  can be in .  We may assume both
   and  induce proper sub-paths; otherwise we can call
  Lemma~\ref{lem:non-consecutive}.  They partition  into four
  sub-paths, two of which are induced by  and .
  Denote by  and  the other two sub-paths; their ends are
  adjacent to  and  respectively, while their inner vertices, if
  any, are adjacent to neither  nor ..

  Assume first that both  and  are of length , then .  If  is adjacent to a single vertex 
  in , (noting that ,) then we return  as a .  A symmetric argument
  applies when .  If , then we
  return  as a .  It must be in some
  case above if , and henceforth we assume .  If 
  is adjacent to only  and  in , (noting that
  ,) then we return  as an FIS-1.  A symmetric argument applies when .  Now that both  and  are at least , we
  return  as a domino.

  Assume now that, without loss of generality,  is nontrivial.
  We can return () (when both  and ) or () and  (when
  the length of  is longer than ) as a .  A symmetric
  argument applies when .  In the remaining
  cases, we assume without loss of generality, , and
  both paths  and  contain at most  vertices.
  Consequently, .

  If  is also nontrivial, then .  We
  return  as a
  \dag\ when .
  Now that , then , and we return
  \begin{inparaenum}[(\itshape 1\upshape)]
  \item  as a long claw when ;
  \item   as a twin- when ; or
  \item  as a FIS-2 when .
  \end{inparaenum}
  In the final case,  is trivial but  is nontrivial, which
  means that neither  nor  is adjacent to .  If
  , then we return  as
  \begin{inparaenum}[(\itshape 1\upshape)]
  \item an FIS-1 when ; or
  \item a twin- when .
  \end{inparaenum}
  If , then we return
  \begin{inparaenum}[(\itshape 1\upshape)]
  \item  as a domino when ; or
  \item () and  as a  when .
  \end{inparaenum}
  This procedure enters only one case, which is decided only by
   and .  Therefore, it can be done in  time.
\end{proof}

\begin{lemma}\label{lem:non-helly}
  Given a set  of two or three pairwise adjacent vertices such that
  \begin{enumerate}[1)]
  \item ; and 
  \item for every , each end of  is adjacent to at
    least two vertices in ,
  \end{enumerate}
  then we can in  time find a \badgraph.
\end{lemma}
\begin{proof}
  Consider first that  contains only two vertices  and .
  The \stpath{h_{\tail{v_1}}}{h_{\head{v_1}}} whose inner vertices are
  nonadjacent to  makes a hole with .  This hole is
  completely adjacent to , and thus we return a wheel.

  Consider then .  We may assume that no two
  vertices of  satisfy the condition of the lemma, as otherwise we
  are in the previous case.  Without loss of generality, assume that
  , and then .
  The \stpath{h_{\tail{v_1}}}{h_{\head{v_2}}} whose inner vertices are
  adjacent to neither  or  makes a hole with  and
  .  By assumption,  is adjacent to every vertex in the
  hole, and thus we return a wheel.
\end{proof}

Let  and .  As we
have alluded to earlier, we want to duplicate  and append them to
different sides of .  Each edge between  and
 will be carried by only one copy of , and
this is determined by its direction specified as follows.  We may
assume that none of the Lemmas.~\ref{lem:non-consecutive},
\ref{lem:non-consecutive-2}, and \ref{lem:non-helly} applies to 
or/and , as otherwise we can terminate the algorithm by returning
the forbidden induced subgraph found by them.  As a result,  is
adjacent to either  or  but not both.  The edge  is said to be clockwise
from  if  for , and
counterclockwise otherwise.  Let \ec\ (resp., \ecc) denote the set of
clockwise (resp., counterclockwise) edges from , and let \oc\
(resp., \occ) denote the subsets of vertices of  that are
incident to edges in \ec\ (resp., \ecc).  Note that 
partitions edges between  and , but a vertex in
 might belong to both \occ\ and \oc, or neither of them.  We have
now all the details for the definition of the auxiliary graph
.

\begin{definition}
  The vertex set of  consists of , where  and  are distinct copies of , i.e.,
  for each , there are a vertex  in  and another
  vertex  in , and  is a new vertex distinct from .
  For each edge , we add to the edge set of 
  \begin{itemize}
  \item an edge  if neither  nor  is in ;
  \item two edges  and  if both  and  are in
    ; or
  \item an edge  or  if  or 
    respectively ( and ).
  \end{itemize}
  Finally, we add an edge  for every .
\end{definition} 

\begin{lemma}\label{lem:construct-mho}
  The numbers of vertices and edges of  are upper bounded by
   and  respectively.  Moreover, an adjacency list
  representation of  can be constructed in  time.
\end{lemma}
\begin{figure}[h!]
  \vspace*{-5mm}
  \setbox4=\vbox{\hsize28pc \noindent\strut
  \begin{quote}
  \vspace*{-5mm} \footnotesize

  {\sc input}: a graph  and a hole .
  \\
  {\sc output}: the auxiliary graph  or a forbidden induced
  subgraph of .
  \\boxit{\box4}
    \label{eq:arcs}A_v := 
    \begin{cases}
      [\lp{v^r}, \rp{v^l} ] & \text{if } v\in\occ,
      \\
      [\lp{v^l}, \rp{v^l} ] & \text{otherwise}.
    \end{cases}
  \boxit{\box4}1ex]
  0 \hspace*{2ex} ; ; ;
  \\
  1 \hspace*{2ex} {\bf for each}  {\bf do}
  \\
  1.1 \hspace*{4ex} compute \tail{v} and \head{v} in ;
  \\
  1.2 \hspace*{4ex} {\bf if} \big( and \big) or \big( and \big) {\bf then}
  \\
  \hspace*{10ex} ; ; ;
  \comment{.}
  \\
  2 \hspace*{2ex} {\bf return} ()
  where  is the new .
\end{quote} \vspace*{-3mm} \strut} 
\vspace*{-7mm}
\caption{Procedure for finding the hole for Lemma~\ref{lem:hole-conditions}.}
\label{fig:compress-hole}
\end{figure}
\begin{proof}
  We apply the procedure given in Fig.~\ref{fig:compress-hole}.
  Step~1 greedily searches for an inclusion-wise maximal 
  satisfying  and .  Initially,
   and .  Each iteration of step~1
  checks an unexplored vertex  in .  If either
  condition of step 1.2 is satisfied, then  properly contains
  , and  and  are updated to be
  \tail{v} and \head{v} respectively.  Note that the values of  and
   are non-increasing and nondecreasing respectively.  Thus, no
  previously explored vertex is adjacent to all of
  .  After step 1, all vertices have been
  explored, and the hole ()
  satisfies the claimed condition.

  What dominates the procedure is finding \tail{v} and \head{v} for
  all vertices (step~1.1).  It takes  time for each vertex
   and  time in total.
\end{proof}
This linear-time procedure can be called before step 2 of algorithm
{nhcag}, and it does not impact the asymptotic time complexity of
the algorithm, which remains linear.  Henceforth we may assume that
 satisfies the condition of Lemma~\ref{lem:hole-conditions}.  During
the construction of , we have checked  for every
vertex , and Lemma~\ref{lem:non-consecutive} was called if it
applies.  Therefore, for the proof of
Theorem~\ref{thm:negative-certificate} in this section, we may assume
that  always induces a proper sub-path of .

Each vertex  of  different from  is uniquely defined by
a vertex of , which is denoted by .  We say that  is
\emph{derived from} .   For example, 
for .  By abuse of notation, we will use the same letter for
a vertex  of  and the unique vertex of
 derived from ; its meaning is always clear from the
context.  Therefore,  for , and in
particular,  for .  We can mark
 for each vertex of  during its construction.  The
function  is also generalized to a set  of vertices that does
not contain , i.e., .  We point out
that possibly .

By construction of , if a pair of vertices  and 
(different from ) is adjacent in , then  and
 must be adjacent in  as well.  The converse is not
necessarily true, e.g., for any vertex  and edge , we have , and for any pair of adjacent vertices
, we have  and .  We say
that a pair of vertices  of  is a \emph{bad pair} if
 in  but  in .  By
definition,  does not participate in any bad pair, and at least one
vertex of a bad pair is in .  Note that any induced path of
length  between a bad pair  with  or  can be
extended to a \stpath{v^l}{v^r} with length .

We have seen that if  is a \nhcag, then for any , the
distance between  and  is at least .  We now see what
happens when this necessary condition is not satisfied by .
By definition of , there is no edge between  and ; for
any , there is no vertex adjacent to both  and .
In other words, for every , the distance between  and
 is at least .  The following observation can be derived from
Lemmas.~\ref{lem:non-consecutive} and \ref{lem:non-consecutive-2}.
\begin{lemma}\label{lem:3-cover}
  Given a \stpath{v^l}{v^r}  of length  for some , we
  can in  time find a \badgraph.
\end{lemma}
\begin{proof}
  Let .  Note that  must be a shortest
  \stpath{v^l}{v^r}, and .  The inner vertices  and 
  cannot be both in ; without loss of generality, let .  Assume first that  as well,
  i.e.,  and .  By definition, , and then  is adjacent to both  and .
  If follows from Lemma~\ref{lem:hole-conditions} that , and then  and .  If
  , i.e., , then we call
  Lemma~\ref{lem:non-consecutive-2} with  and .  If , then we call Lemma~\ref{lem:non-consecutive} with  and
  .  In the remaining case, , and
  () is a hole of ; this hole is
  completely adjacent to , and thus we find a wheel.

  Now assume that, without loss of generality, .  If
  , then we call
  Lemma~\ref{lem:non-consecutive-2} with  and .  Otherwise, () is a hole of ; this hole
  is completely adjacent to , and thus we find a wheel.
\end{proof}

If  is a \nhcag, then in a circular-arc model of , all arcs for
\occ\ and \oc\ contain  and  respectively.  Thus,
both \occ\ and \oc\ induce cliques.  This observation is complemented
by the following lemma.
\begin{lemma}\label{lem:O}
  Given a pair of nonadjacent vertices  (or \oc), we can
  in  time find a \badgraph\ of .
\end{lemma}
\begin{proof}
  By definition, we can find edges .  We have three
  (possibly intersecting) induced paths , , and
  .  If both  and  are adjacent to , then we
  return () as a wheel.  Hence we may assume
  .

  If , then by Lemma~\ref{lem:hole-conditions}, .  We consider the subgraph induced by the set
  of distinct vertices .  If  is
  adjacent to  or , then we can call
  Lemma~\ref{lem:non-helly} with  and .  By assumption, , and  make a triangle;  is adjacent to neither  nor
  ; and  is adjacent to neither  nor .  Therefore,
  the only uncertain adjacencies in this subgraph are between ,
  and .  The subgraph is thus isomorphic to
  \begin{inparaitem}
  \item[(1)] FIS-1 if there are two edges among , and ;
  \item[(2)]  if , and  are pairwise
    adjacent; or
  \item[(3)] net if , and  are pairwise nonadjacent.
  \end{inparaitem}
  In the remaining cases there is precisely one edge among , and
  , then we can return a , e.g., () and 
  when the edge is .

  Assume now that , and  are pairwise nonadjacent.  We
  consider the subgraph induced by ,
  where the only uncertain relations are between , and .
  The subgraph is thus isomorphic to
  \begin{inparaenum}[(1)]
  \item  if all of them are identical; or
  \item twin- if two of them are identical, and adjacent to the
    other.
  \end{inparaenum}
  If two of them are identical, and nonadjacent to the other, then the
  subgraph contains a , e.g., () and  when
  .  In the remaining cases, all of , and  are
  distinct, and then the subgraph
  \begin{inparaenum}[(1)]
  \item  is isomorphic to long claw if they are pairwise nonadjacent;
  \item contains net  if they are pairwise
    adjacent; or
  \item  is isomorphic to FIS-2 if there are two edges among them.
  \end{inparaenum}
  If there is one edge among them, then the subgraph contains a ,
  e.g., () and  when the edge is .

  A symmetrical argument applies to \oc.  Edges  and  can be
  found in  time, and only a small constant number of
  adjacencies are checked; it thus takes  time in total.
\end{proof}

It can be checked in linear time whether \occ\ and \oc\ induce
cliques.  When it is not, a pair of nonadjacent vertices can be found
in the same time.  By Lemma~\ref{lem:O}, we may assume hereafter that
\occ\ and \oc\ induce cliques.  We say that a vertex  is {\em
  simplicial} if  induces a clique.  Recall that ; as a result,  is simplicial and participates in no holes.
\begin{proposition}\label{lem:non-bypass}
  Given an \stpath{h^l_0}{h^r_0}  that is nonadjacent to  for
  some , we can in  time find a \badgraph.
\end{proposition}
\begin{proof}
  Inside  there must be a sub-path  whose ends  are in 
  and  respectively, and whose inner vertices are all in
  .  Let  and  be the neighbors of  and 
  in  respectively; note that they are both in .  If
   and  are disjoint, then we call
  Lemma~\ref{lem:non-consecutive-2}.  Otherwise, by assumption, we
  have ; likewise, we may assume
  .  Starting from , we traverse
   till the first pair of consecutive vertices  in  such
  that  and  are disjoint: note that such a pair must
  exist because no vertex between  and  is adjacent to 
  or .  Then we call Lemma~\ref{lem:non-consecutive-2}.
\end{proof}

We are now ready to prove Theorem~\ref{thm:negative-certificate}, which
is separated into three statements, the first of which considers the
case when  is not chordal.
\begin{lemma}\label{lem:hole}
  Given a hole  of , we can in  time find a
  \badgraph.
\end{lemma}
\begin{proof}
  Let us first take care of some trivial cases.  If  is contained
  in  or  or , then by construction, \og{C} is a
  hole of .  This hole is either nonadjacent or completely adjacent
  to  in , whereupon we can return \og{C} and  as a 
  or wheel respectively.  Since  and  are nonadjacent, one of
  the cases above must hold if  is disjoint from .
  Henceforth we may assume that  intersects  and,
  without loss of generality, ; it might intersect  as well, but
  this fact is irrelevant in the following argument.  Then we can find
  an edge  of  such that  and , i.e., .

  Let .  Assume first that ; then
  we must have .  Let  and  be the next two vertices
  of .  Note that , i.e., ;
  otherwise , which is impossible.  If  (or  when ), then
   induces a hole of , and
  we can return it and  as a wheel.  Note that  as they are non-consecutive vertices of the hole .  We now
  argue that .  Suppose for contradiction,
  .  We can extend the \stpath{x_3}{x_1}  in
   that avoids  to a \stpath{h^l_0}{h^r_0} avoiding the
  neighborhood of , which allows us to call
  Proposition~\ref{lem:non-bypass}.  We can call
  Lemma~\ref{lem:non-consecutive-2} with  and  if
  .  In the remaining case, .  Let  be the first vertex in  that is adjacent to
   (or  if ); its existence is clear as
   satisfies this condition.  Then  induces a hole of , and we can return it and
   as a wheel.

  Assume now that  is not in .  Denote by  the
  \stpath{x_2}{x_1} obtained from  by deleting the edge .
  Let  be the first neighbor of  in , and let  be
  either the first neighbor of  in the \stpath{x}{x_1} or the
  other neighbor of  in .  It is easy to verify that
   induces a hole of ,
  which is completely adjacent to , i.e., we have a wheel.
\end{proof}

In the rest  will be chordal, and thus we have a chordal
minimally non-interval subgraph  of .  This subgraph is
isomorphic to some graph in Fig.~\ref{fig:at}, on which we use the
following notation.  It is immediate from Fig.~\ref{fig:at} that each
of them contains precisely three simplicial vertices (squared
vertices), which are called \emph{terminals}, and others (round
vertices) are \emph{non-terminal vertices}.  In a long claw or \dag,
for each , terminal  has a unique neighbor, denoted by
.

Since the diameter and maximum clique of  is at most four, all bad
pairs in it can be found easily.
\begin{proposition}\label{lem:find-bad-pair}
  Given a subgraph  of  in Fig.~\ref{fig:at}, we can in
   time find either all bad pairs in  or a \badgraph.
\end{proposition}
\begin{proof}
  A long claw or whipping top has only seven vertices, and thus will
  not concern us.  A bad pair is either between  and , or
  between  and .  Let  and ; both induce cliques.  Since a clique of 
  contains at most  vertices, it contains at most  vertices of
   and .  Bad pairs intersecting them
  can thus be found in linear time.  We now consider other bad pairs,
  which must be between  and .  By construction, there is no edge between
   and ; there is no
  edge between  and .
  Therefore, the shortest distance between 
  and  is at least .  There exists only
  one pair of distance  vertices in a , and no such a pair in
  a .
\end{proof}

\begin{lemma}\label{lem:at-no-w}
  Given a subgraph  of  in Fig.~\ref{fig:at} that does not
  contain , we can in  time find a \badgraph.
\end{lemma}
\begin{figure}[h]
  \centering
  \footnotesize
  \subfloat[labeled long claw]{\label{fig:long-claw}
    \includegraphics{3-1.pdf} 
  }
  
  \subfloat[]{\label{fig:long-claw-5}\includegraphics{3-2.pdf} 
  }
  
  \subfloat[]{\label{fig:long-claw-6}\includegraphics{3-3.pdf} 
  }

  \subfloat[]{\label{fig:long-claw-1}\includegraphics{3-4.pdf} 
  }
  
  \subfloat[]{\label{fig:long-claw-2}\includegraphics{3-5.pdf} 
  }
  
  \subfloat[]{\label{fig:long-claw-3}\includegraphics{3-6.pdf} 
  }
  
  \subfloat[]{\label{fig:long-claw-4}\includegraphics{3-7.pdf} 
  }
  \caption{Illustrations for Lemma~\ref{lem:at-no-w} (blue dashed
    edges are in  only).}
\label{fig:negative-certificate-1}
\end{figure}
\begin{proof}
  We first call Proposition~\ref{lem:find-bad-pair} to find all bad
  pairs in .  If  has no bad pair, then we return the subgraph
  of  induced by \og{F}, which is isomorphic to .  Let  be
  a bad pair with the minimum distance in ; we may assume that it
  is  or , as otherwise we can call Lemma~\ref{lem:3-cover}.
  Noting that the distance between a pair of non-terminal vertices is
  at most , we may assume that without loss of generality,  is a
  terminal of .  We break the argument based on the type of .

  {\em Long claw.}  We may assume that  and ; other situations are symmetrical.  Let  be the
  unique \stpath{x}{y} in .  If \og{t_3} is nonadjacent to \og{P},
  then we return \og{P} and \og{t_3} as a ; we are thus focused
  on the adjacency between \og{t_3} and \og{P}
  (Fig.~\ref{fig:negative-certificate-1}).  If , then by the
  selection of  (they have the minimum distance among all bad
  pairs),  can be only adjacent to  and/or
  .  We return either  as an FIS-2
  (Fig.~\ref{fig:long-claw-5}), or 
  as a net (Fig.~\ref{fig:long-claw-6}).  In the remaining cases, , and  can only be adjacent to \og{u_1}, \og{u_2},
  and/or \og{t_1}.  We point out that possibly ,
  which is irrelevant as  will not be used below.  If
   is adjacent to both  and  in ,
  then we get a  (Fig.~\ref{fig:long-claw-1}).  Note that
  this is the only case when .  If  is
  adjacent to both  and  in , then we get an
  FIS-1 (Fig.~\ref{fig:long-claw-2}).  If  is adjacent to
  only  or only  in , then we get a domino
  (Fig.~\ref{fig:long-claw-3}) or twin-
  (Fig.~\ref{fig:long-claw-4}), respectively.  The situation that
   is adjacent to \og{u_1} but not \og{u_2} is similar as
  above.

  {\em Whipping top.} The diameter is , and this distance is
  attained only by  or .  If both are bad
  pairs, then we have a domino.  If  is the only bad
  pair, then  induces a hole of , and it is
  nonadjacent to \og{t_2}; we get a .  A symmetrical argument
  applies if  is the only bad pair.

  {\em \dag.}  Consider first that  and , and let .  If \og{t_2} is nonadjacent to the hole
  induced by \og{P}, then we return \og{P} and \og{t_2} as a .
  If \og{t_2} is adjacent to \og{t_3} or \og{u_1}, then we get a
  domino.  If \og{t_2} is adjacent to \og{t_1}, then we get a
  twin-.  If \og{t_2} is adjacent to \og{t_1} and precisely one
  of , then we get an FIS-1.  If \og{t_2} is
  adjacent to both \og{t_3} and \og{u_1}, then we get a ;
  here the adjacency between \og{t_2} and \og{t_1} is immaterial.  A
  symmetric argument applies when  is a bad pair.  In
  the remaining case, neither \og{t_1} nor \og{t_2} is adjacent to
  \og{t_3}.  Therefore, a bad pair must be in the path ,
  which is nonadjacent to \og{t_3}, then we get a .

  {\em \ddag.} The only pair of vertices of distance  is .  Let  be the \stpath{t_1}{t_2} in .  Since
   cannot be adjacent to any vertex in , we can
  return \og{P} and \og{t_3} as a .
\end{proof}

\begin{lemma}\label{lem:at-with-w}
  Given a subgraph  of  in Fig.~\ref{fig:at} that contains
  , we can in  time find a \badgraph.
\end{lemma}
\begin{figure}[h]
\setbox4=\vbox{\hsize28pc \noindent\strut
\begin{quote}
  \vspace*{-5mm} \footnotesize

  \hspace*{2ex}  Note that .
  \\
  1 \hspace*{2ex} {\bf if}  and 
  {\bf then}
  \\
  \hspace*{7ex} {\bf call} Lemma~\ref{lem:non-consecutive-2} with () and ;
  \\
  \textcolor{white}{1} \hspace*{2ex} {\bf if} 
  and  {\bf then}
  \\
  \hspace*{7ex} {\bf call} Lemma~\ref{lem:non-consecutive-2} with () and ;
  \\
  \textcolor{white}{1} \hspace*{2ex} {\bf if}  {\bf then} {\sf symmetric as above};
  \\
  2 \hspace*{2ex} {\bf if}  {\bf
    then}
  \\
  \hspace*{6ex} {\bf return}  as a \dag;
  \comment{Fig.~\ref{fig:hole-1}.}
  \\
  \hspace*{2ex}  assume from now that
   and .
  \\
  3 \hspace*{2ex} {\bf if}  {\bf then return}
  () and  as a ;
  \\
  4 \hspace*{2ex} {\bf if}  {\bf then return}
   as a \ddag;
  \comment{Fig.~\ref{fig:hole-3}.}
  \\
  \textcolor{white}{4} \hspace*{2ex} {\bf if}  {\bf
    then return}  as a
  \dag. \comment{Fig.~\ref{fig:hole-4}.}

\end{quote} \vspace*{-3mm} \strut} 
\vspace*{-7mm}
\caption{Procedure for Lemma~\ref{lem:at-with-w}}
\label{fig:at-with-w}
\end{figure}
\begin{proof}
  Since  is simplicial, it has at most  neighbors in .  If
   has a unique neighbor in , then we can use a similar argument
  as Lemma~\ref{lem:at-no-w}.  Now let  be the two
  neighbors of  in .  If there exists some vertex  adjacent to both {\og{x_1}} and \og{x_2} in ,
  which can be found in linear time, then we can use it to replace .
  Hence we assume there exists no such vertex.  By assumption, we can
  find two distinct vertices  such that
  ; note that  and  in .  As a result,  and
   are nonadjacent; otherwise,  and the
  counterparts of  in  induce a hole of ,
  which contradicts the assumption that  is chordal.  We then
  apply the procedure described in Fig.~\ref{fig:at-with-w}.

  We now verify the correctness of the procedure.  Since each
  step---either directly or by calling a previously verified
  lemma---returns a \badgraph\ of , all conditions of previous
  steps are assumed to not hold in a later step.  By
  Lemma~\ref{lem:hole-conditions},  and
   are either  or .  Step 1 considers the case
  where .  By
  Lemma~\ref{lem:non-helly}, .  Thus,
  () or () is
  a hole of , depending on  is  or .  In
  the case (), only  and 
  can be adjacent to ; they are nonadjacent to .
  Likewise, in the case (), vertices
   and  are adjacent to  but not , while
   can be adjacent to only one of  and .  Thus, we
  can call Lemma~\ref{lem:non-consecutive-2}.  A symmetric argument
  applies when .  Now that the
  conditions of step~1 do not hold true, step~2 is clear from
  assumption.  Henceforth we may assume without loss of generality
  that  and .  Consequently,
   (Lemma~\ref{lem:non-consecutive-2}).
  Because we assume that the condition of step~1 does not hold,
  ; this justifies step~3.  Step~4 is clear as 
  is always adjacent to .
  \end{proof}
\begin{figure}[h!]
  \centering
  \subfloat[]{\label{fig:hole-1}
    \includegraphics[scale=.97]{5-1.pdf} 
  }
  \,
  \subfloat[]{\label{fig:hole-3}
    \includegraphics[scale=.97]{5-2.pdf} 
  }
  \,
  \subfloat[]{\label{fig:hole-4}
    \includegraphics[scale=.97]{5-3.pdf} 
}
\caption{Structures used in the proof of Lemma~\ref{lem:at-with-w}
  (dashed edges may or may not exist).}
  \label{fig:negative-certificate-2}
\end{figure}
\begin{thebibliography}{10}
\small
\bibitem{cao-14-recognizing-nhcag}
Yixin Cao.
\newblock Direct and certifying recognition of normal {H}elly circular-arc
  graphs in linear time.
\newblock In Jianer Chen, John Hopcroft, and Jianxin Wang, editors, {\em
  Proceedings of the 8th International Frontiers of Algorithmics Workshop, FAW
  2014}, 2014.
\newblock To appear.

\bibitem{cao-14-almost-interval-recognition}
Yixin Cao.
\newblock Linear recognition of almost (unit) interval graphs.
\newblock arXiv:1403.1515, 2014.

\bibitem{duran-14-survey}
Guillermo Dur\'an, Luciano~N. Grippo, and Mart\'\i n~D.~Safe.
\newblock Structural results on circular-arc graphs and circle graphs: A survey
  and the main open problems.
\newblock {\em Discrete Applied Mathematics}, 164(2):427--443, 2014.
\newblock LAGOS'11: Sixth Latin American Algorithms, Graphs, and Optimization
  Symposium, Bariloche, Argentina---2011.

\bibitem{grippo-12-cag-without-dominating-triple}
Luciano~N. Grippo and Mart\'in~D. Safe.
\newblock On circular-arc graphs having a model with no three arcs covering the
  circle.
\newblock CLAIO-SBPO 2012, Rio de Janeiro - Brazil, September, 24-28 2012.
  http://www2.claiosbpo2012.iltc.br/pdf/102012.pdf.

\bibitem{hadwiger-64-combinatorial-geometry}
Hugo Hadwiger, Hans Debrunner, and Victor Klee.
\newblock {\em Combinatorial geometry in the plane}.
\newblock Athena series. Holt, Rinehart and Winston, London, 1964.

\bibitem{heggernes-07-certifying-fis}
Pinar Heggernes and Dieter Kratsch.
\newblock Linear-time certifying recognition algorithms and forbidden induced
  subgraphs.
\newblock {\em Nordic Journal of Computing}, 14(1-2):87--108, 2007.

\bibitem{kratsch-06-certifying-interval-and-permutation}
Dieter Kratsch, Ross~M. McConnell, Kurt Mehlhorn, and Jeremy~P. Spinrad.
\newblock Certifying algorithms for recognizing interval graphs and permutation
  graphs.
\newblock {\em SIAM Journal on Computing}, 36(2):326--353, 2006.
\newblock A preliminary version appeared in SODA 2003.

\bibitem{lekkerkerker-62-interval-graphs}
Cornelis~G. Lekkerkerker and J.~Ch. Boland.
\newblock Representation of a finite graph by a set of intervals on the real
  line.
\newblock {\em Fundamenta Mathematicae}, 51:45--64, 1962.

\bibitem{lin--10-clique-operator-cag}
Min~Chih Lin, Francisco~J. Soulignac, and Jayme~L. Szwarcfiter.
\newblock The clique operator on circular-arc graphs.
\newblock {\em Discrete Applied Mathematics}, 158(12):1259--1267, 2010.

\bibitem{lin-13-nhcag-and-subclasses}
Min~Chih Lin, Francisco~J. Soulignac, and Jayme~L. Szwarcfiter.
\newblock Normal {H}elly circular-arc graphs and its subclasses.
\newblock {\em Discrete Applied Mathematics}, 161(7-8):1037--1059, 2013.

\bibitem{lin-09-cag-and-subclasses}
Min~Chih Lin and Jayme~L. Szwarcfiter.
\newblock Characterizations and recognition of circular-arc graphs and
  subclasses: A survey.
\newblock {\em Discrete Mathematics}, 309(18):5618--5635, 2009.

\bibitem{lindzey-13-find-forbidden-subgraphs}
Nathan Lindzey and Ross~M. McConnell.
\newblock On finding {Tucker} submatrices and {Lekkerkerker-Boland} subgraphs.
\newblock In Andreas Brandst\"{a}dt, Klaus Jansen, and R{\"u}diger Reischuk,
  editors, {\em Revised Papers of the 39th International Workshop on
  Graph-Theoretic Concepts in Computer Science, WG 2013}, volume 8165 of {\em
  LNCS}, pages 345--357, 2013.

\bibitem{mcconnell-03-recognition-cag}
Ross~M. McConnell.
\newblock Linear-time recognition of circular-arc graphs.
\newblock {\em Algorithmica}, 37(2):93--147, 2003.

\bibitem{mcconnell-11-survey-certifying-algorithms}
Ross~M. McConnell, Kurt Mehlhorn, Stefan N{\"a}her, and Pascal Schweitzer.
\newblock Certifying algorithms.
\newblock {\em Computer Science Review}, 5(2):119--161, 2011.

\bibitem{mckee-03-restricted-cag}
Terry~A. McKee.
\newblock Restricted circular-arc graphs and clique cycles.
\newblock {\em Discrete Mathematics}, 263(1-3):221--231, 2003.

\bibitem{tarjan-84-chordal-recognition}
Robert~E. Tarjan and Mihalis Yannakakis.
\newblock Simple linear-time algorithms to test chordality of graphs, test
  acyclicity of hypergraphs, and selectively reduce acyclic hypergraphs.
\newblock {\em SIAM Journal on Computing}, 13(3):566--579, 1984.
\newblock With Addendum in the same journal, 14(1):254-255, 1985.

\bibitem{tucker-74-structures-cag}
Alan~C. Tucker.
\newblock Structure theorems for some circular-arc graphs.
\newblock {\em Discrete Mathematics}, 7(1-2):167--195, 1974.

\bibitem{tucker-75-coloring-cag}
Alan~C. Tucker.
\newblock Coloring a family of circular arcs.
\newblock {\em SIAM Journal on Applied Mathematics}, 29(3):493--502, 1975.

\end{thebibliography}
\end{document}
