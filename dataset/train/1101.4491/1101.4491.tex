

In this section, we consider asymmetric and loopless directed graphs (digraphs for short). Let  be a digraph. An arc oriented from vertex  to  is denoted . The \emph{in-neighbourhood} of a vertex  is the set  and the \emph{out-neighbourhood} of  is . If , then  is the subdigraph \emph{induced} by , that is . Similarly, if , then  denotes the \emph{digraph} where . We say that  is \emph{acyclic} if it does not contain any circuit and that  is \emph{transitive} if for every triple of vertices  such that  and , then . A \emph{tournament}  is a complete asymmetric digraph and a \emph{bipartite tournament}  is a complete bipartite digraph (there is an arc between  and  if and only if  and ). We denote by  the size of a minimum feedback arc set of .

A vertex ordering  of  is a total order on the vertex set . We denote by  the fact that  is smaller than  in . It is well-known that every acyclic digraph has a \emph{topological ordering}  of the vertices: for every arc , . An \emph{ordered digraph} is a triple  where  is a digraph and  a vertex ordering of . An arc  in  is \emph{backward} if , \emph{forward} otherwise. Let  be a backward arc, then . A \emph{certificate}  in  is the vertex set of a directed path from  to  only composed of forward arcs. Observe that every vertex  distinct from  and  belongs to  and that  induces a circuit in which  is the unique backward arc. We abusively say that an arc  belongs to a certificate if its two vertices do. A subset  of \emph{backward arcs} can be \emph{certified} whenever there exists a collection  of \emph{arc-disjoint} certificates.


The following Lemma on matching and vertex cover in bipartite graphs will be used in three of our kernels.

\begin{lemma} \label{lem:matching}
Let  be a bipartite graph,  be a minimum vertex cover of ,  and . Any subset  can be matched into . 
\end{lemma}
\begin{proof}
By Hall's theorem~\cite{H35},  can be matched into  iff  for every subset . Hence, assume for a contradiction that there exists  such that . As there is no edge in  between  and  ( is a vertex cover of ), the set  is a vertex cover of  and : contradicting the minimality of  (see Figure~\ref{fig:vcrti}). Thereby for every subset , we have . This concludes the proof.  
\end{proof}

\begin{figure}[h]
\begin{center}
\centerline{\includegraphics[scale=1]{vc.pdf}}
\caption{The case . The gray sets represent the vertex cover .\label{fig:vcrti}}
\end{center}
\end{figure}








\subsection{Reduction rules for feedback arc set in digraphs}

The following definitions and reduction rules were the core tool of the kernelization algorithm in~\cite{BFG+11}. The purpose of this section is to show how they can be easily implemented in the case of tournaments and bipartite tournaments.

A vertex of a digraph  is {irrelevant} if it does not belong to any circuit of . It is easy to observe that  optimal feedback arc sets of a digraph avoid  arcs incident to irrelevant vertices, which implies the following reduction rule.

\begin{polyrule} \label{rule:uselessvertexfast}
Remove every irrelevant vertex.
\end{polyrule}

A partition  of  is an \emph{ordered partition} of an ordered digraph  if for every ,  is a set of consecutive vertices in . Hereafter an ordered partition of  will be denoted by , assuming that for every  and  with , . The set of \emph{external arcs} of  is . A certificate  of an external backward arc  is \emph{external} if the vertices of  belong to distinct parts of  (the arcs of the -path are external). We say that a subset  of backward arcs is \emph{externally certified} if there exists a set  of pairwise arc-disjoint external certificates.
	
	
\begin{definition}[Safe Partition] 
Let  be an ordered digraph. An ordered partition  of  is a \emph{safe partition} if the subset  of backward arcs of  is non-empty and can be can be externally certified (see Figure~\ref{fig:spfasbt}).
\end{definition}
	
\begin{figure}[h]
\centerline{\includegraphics[scale=2]{safepartition}}
\caption{The two external backward arcs of the partition can be certified using the paths  and . In this example, there are no other external backward arcs. \label{fig:spfasbt}}
\end{figure} 
	
\begin{polyrule}[Safe partition~\cite{BFG+11}] \label{rule:safepartition}
Let  be an instance of \FAS{},  be a vertex ordering of  and  be a safe partition of  . Return  where  is obtained from  by reversing the external backward arcs  and where .
\end{polyrule}


\subsection{Tournaments}

A \emph{directed-} (or circuit of size ) is a set of vertices  such that . As shown by the following well-known statement, irrelevant vertices in the case of tournaments can be defined as the vertices not contained in any directed-.

\begin{lemma}[Folklore]
\label{lem:folklore}
Let  be a tournament. Then the following properties are equivalent: (i)  is acyclic; (ii)  is transitive; (iii)  does not contain any directed-.
\end{lemma}

Consequently, it will be possible to certify any backward arc  of an ordered tournament  with a unique vertex  such that  induces a directed-. Clearly the cardinality of any set of arc-disjoint directed- provides a lower bound on the minimum feedback arc set of a tournament. We show how such a set can be used to compute in polynomial time a safe partition. 



\newcommand{\T}{T_{\sigma}}
\newcommand{\V}{V_{\sigma}}

\begin{definition}[Conflict packing] \label{def:cpfast}
A \emph{conflict packing} of a tournament  is a \emph{maximal} collection of arc-disjoint directed-'s.
\end{definition}

Observe that a conflict packing can be computed in polynomial time, for instance by using a greedy algorithm. Given a conflict packing  of an instance  of \FAST{},  denotes the set of vertices covered by . 

\begin{observation} \label{lem:cpfast}
Let  be a conflict packing of a tournament . If , then  and .
\end{observation}



\begin{lemma}[Conflict Packing] \label{lem:realcpfast}
Let  be a conflict packing of a tournament . There exists a vertex ordering  of  such that every backward arc  of  satisfies .
\end{lemma}

\begin{proof}
Let us consider the vertex set . By maximality of ,  is acyclic. Observe also that for every ,  is acyclic: otherwise  would contain a directed-  with  arc-disjoint from those of . Let  be the unique topological ordering of . For every vertex , there is a unique pair  of  consecutive vertices in  such that the insertion of  between  and  yields the topological ordering of . The pair  is called the \emph{locus} of . Consider any ordering  on  such that: 
\begin{itemize}
\vspace{-0.2cm}
\item for , if , then , and 
\vspace{-0.2cm}
\item if , and  is the locus of , then . 
\end{itemize}
By construction of , every backward arc  of  satisfies .
\end{proof}

\begin{theorem} \label{thm:fast}
The \FAST{} problem admits a kernel with at most  vertices.\end{theorem}
\begin{proof}
We prove the result by showing that any instance  of \FAST{} such that  is reduced under Rule~\ref{rule:uselessvertexfast} and  is either a negative instance or can be reduced using Rule~\ref{rule:safepartition}. \\ 

Let  be a conflict packing of  and  be the vertex ordering obtained through Lemma~\ref{lem:realcpfast}. 
Consider the bipartite graph  where:
\begin{enumerate}[(i)]
\item , 
\vspace{-0.2cm}
\item there is a vertex  in  for every backward arc  of  and, 
\vspace{-0.2cm}
\item  if  and  is a certificate of .
\end{enumerate}	 
	
Observe that any matching in  of size at least  corresponds to a conflict packing of size at least , in which case the instance is negative (Observation~\ref{lem:cpfast}). Hence, by K\"onig-Egervary's theorem~\cite{BM76}, a minimum vertex cover  of  has size at most . We denote  and .
Assume that . Since  and   (Lemma~\ref{lem:cpfast}), we have 
. Let  be the ordered partition of  such that every part  consists of either a vertex of  or a maximal subset of consecutive vertices (in ) of  (see Figure~\ref{fig:spfast}).

\begin{figure}[h]
\centerline{\includegraphics[scale=1]{spfast}}
\caption{The construction of the ordered partition  based on the bipartite graph . The vertices of  are alone in their parts while the other vertices are grouped in maximal subsets of consecutive vertices (w.r.t. .  \label{fig:spfast}}
\end{figure}

\begin{claim} \label{claim:pissafe}
 is a safe partition of .
\end{claim}
\begin{proofclaim}
Let  be a vertex of . By Lemma~\ref{lem:realcpfast},  is not incident to any backward arc.
Since  is reduced under Rule~\ref{rule:uselessvertexfast}, there must exist a backward arc  such that . It follows that the set  of external arc contains at least one backward arc. 
Let  be a backward arc of . By construction of , there exists a vertex . Then  is a certificate of  and  is an edge of . Observe that since  is a vertex cover and , the vertex  belongs to  (otherwise the edge  would not be covered). Thereby the subset  corresponding to the backward arcs of  is included in . 
By Lemma~\ref{lem:matching}, we know that  can be matched into  in . 
Since every vertex of  is a singleton in , the existence of the matching shows that the backward arcs of  can be certified using arcs of  only, and hence  is safe. 
\end{proofclaim}

This concludes the proof. 
\end{proof}
 
\subsection{Bipartite tournaments}

A  \emph{directed-} (or circuit of length ) is a subset  of vertices such that . The following statement shows that directed-'s will play the same role as directed-'s in tournaments. 

\begin{lemma}\cite{DGHNT10}
A bipartite tournament is acyclic if and only if it does not contain a directed-.
\end{lemma}

\begin{lemma} \cite{MRRS13}
If a vertex of a bipartite tournament belongs to a circuit, then it belongs to a directed-.
\end{lemma}

It follows that in the case of bipartite tournaments, an irrelevant vertex is a vertex not belonging to any directed-. Let  be an ordered bipartite tournament. A \emph{certificate} for a backward arc  of  (i.e. ) is a set  such that ,  and  are forward arcs. In other words,  is a directed- for which  is the unique backward arc. 

\begin{definition} \label{def:cpfasbt}
A \emph{conflict packing}  of a bipartite tournament is a \emph{maximal} collection of \emph{arc-disjoint} directed-'s.
\end{definition}

Remark that two directed-'s of  may intersect on two vertices, provided that these vertices both belong to  either  or . Here again, such a collection can be computed in polynomial time (using for instance a greedy algorithm). 

\begin{observation} \label{prop:cpfasbt}
Let  be a conflict packing of a tournament . If , then  and .
\end{observation}


The kernel in the case of bipartite tournaments requires a third reduction rule based on modules. A \emph{module} in a digraph  is a set of vertices  such that for every ,  and . A module  is \emph{non-trivial} if  and . A module  is \emph{maximal} if it is not included in any non-trivial module . It is easy to observe that in a bipartite tournament , any non-trivial module  is either a subset of  or of  (and thereby an independent set). 

\begin{polyrule}[\cite{MRRS13}] \label{rule:modulesfasbt}
Let  be a non-trivial module of  of size at least . Remove all but  arbitrary vertices from . 
\end{polyrule}

The correctness of this rule is a consequence of the following observation. Consider an instance without any irrelevant vertex. Then every vertex belongs to a  and whenever a module  of size at least  exists, there exists a set  of at least  directed-'s pairwise intersecting on the arcs  and . Thereby any solution of size at most  has to reverse one of those two arcs.

To compute a safe partition, we follow the main two steps described in the case of \FAST{}: 1) based on a conflict packing, compute a permutation  of  such that the vertices incident to the backward arcs are covered by the conflict packing; 2) compute from  a safe partition of . Let us show how to compute the vertex ordering .

\begin{lemma} \label{lem:module-ordering}
Let  be an acyclic bipartite tournament. There exists a partial order  on the vertices such that every topological ordering  of  is a linear extension of .
\end{lemma}
\begin{proof}
We define  as follows:  if and only if  or there exists  such that  and . It is easy to observe that two vertices are incomparable with respect to  if and only if they belong to the same module. Let us prove that if  is a topological ordering of , then the set of vertices of any maximal module  of  occurs consecutively in , implying the result.

Assume by contradiction that there exist  and  such that . W.l.o.g. assume that . First consider that . If   (the case   is similar) then , in other words the arc  is backward in : contradicting the fact that  is a topological ordering. Assume that . As , there exists a vertex  distinguishing  from : either ,  and  or ,  and . In any case, one of the three arcs between  and  or  is backward in : contradicting the fact that  is a topological ordering.
\end{proof}





\begin{lemma} \label{lem:cpfasbt}
Let  be a conflict packing of a bipartite tournament . There exists a vertex ordering  of  such that every backward arc  of  (i.e. ) satisfies . 
\end{lemma}
\begin{proof}
Let  be the set of vertices not covered by , i.e. . By maximality of ,  is acyclic. Let  be the partial order on  defined by Lemma~\ref{lem:module-ordering}. We define a vertex ordering  by extending  (adding comparabilities) and inserting vertices of .
Let  be two arbitrary vertices of :

\begin{enumerate}
\item  and :
\begin{enumerate}
\item  and : if , then  and  are incomparable, otherwise if , then ; 
\item   and : if   such that , , then , otherwise ;
\end{enumerate}

\item If  and : 
\begin{enumerate}
\item  and : 
if  such that , then , otherwise ;
\item   and : if , then , otherwise ;
\end{enumerate}

\item If  and : if  and  are not comparable in  and  with  and , then ;
\end{enumerate}

The following claims will help to prove that  is a partial order.
 
\begin{claim} \label{cl:comp}
Let  and  be two vertices of  (resp. ), then  or vice versa.
\end{claim}
\begin{proofclaim}
Assume there exist  and . Then  is a directed- which is arc-disjoint from every directed- of , contradicting the maximality of .
\end{proofclaim}

\begin{claim} \label{cl:consec}
Let  be a vertex of . If  and  are two comparable vertices in  such that  and , then .
\end{claim}
\begin{proofclaim}
Assume that . As  and  do not belong to the same module of  there exists  with  such that  and  are arcs of . It follows that the vertices  form a directed- arc-disjoint from those of : contradicting the maximality of .
\end{proofclaim}

\begin{claim} \label{cl:transitive}
Let  be two vertices such that  and  (resp.  and ). If there exists  such that  and , then  and .
\end{claim}
\begin{proofclaim}
We consider the case where  and , the other case being symmetric.
By condition 2 of definition of ,  implies that if , then  otherwise there exists  such that . Assume that . As ,  and  do not belong to the same maximal module. So there exists  such that . Observe that if , then the vertices  form a directed-. Likewise if , the vertices  form a directed-. In both cases we obtain a directed- arc-disjoint from those of : contradicting the maximality of . It follows that  and thereby  (condition 2b of definition of ).
\end{proofclaim}


\begin{claim} \label{cl:PO}
 is a partial order. 
\end{claim}
\begin{proofclaim}
Let us first observe that  is antisymmetric. If  or  belong to , antisymmetry is straightforward from definition of . Suppose that  and  both belong to . As new comparabilities are added only between pair of vertices non-comparable in , antisymmetry follows from Claim~\ref{cl:comp}. 

Let us prove that  is transitive with a case by case analysis.  Keep in mind that two vertices are incomparable with respect to  if and only if they belong to the same maximal module of . Let  be three vertices such that  and .
\begin{itemize}
\item : If  and , as  is transitive, we also have  and thereby . Suppose that  belong to the same maximal module of  and , then by construction of , we also have  implying . The same argument holds if   and  belong to the same maximal module of . The last case to consider is when  belong to the same maximal module of . Then  implies the existence of  such that  and  (condition 3 of definition of ). Likewise,  implies the existence of  such that  and . As , Claim~\ref{cl:comp} implies that  and thereby . As , it follows that  (condition 3 of definition of ).


\item  and  (or  and ): Assume that  and  belong to the same maximal module of . 
It follows that  and  both belong either to  or to . Assume w.l.o.g. that .
If , then  implies the existence of  such that  (condition 2a of definition of ). As  and  belongs to the same module of ,  implies that  and thereby . Assume that . Then  implies  (condition 2b of definition of ). Moreover,  implies the existence of  such that  and  (condition 3 of definition of ). Observe that as , . It follows that  as well, 
since otherwise  would form a directed- arc-disjoint from those of , contradicting the maximality of . Thereby we have .

If  and , as  and , by Claim~\ref{cl:transitive} we have . Suppose that . If , then  implies the existence of  such that  (condition 2a). It follows that . If , then  implies that  (condition 2b). In both cases, there exists a vertex  such that  and . It follows that  (condition 2a). The other cases are symmetric. 

\item  and : 
We first consider the case where  and  do not belong to the same module of , implying that either  or . We prove that  would contradict the maximality of . Suppose that ,  and that  (the case  is symmetric). As , there exists  such that  (condition 2a), or equivalently . It follows that if  or equivalently , we obtain a directed- on  arc-disjoint from those of . Suppose that  and . If , then there exists  such that . If , then  and  implies  (condition 2b). So  forms a directed- arc-disjoint from those of . If , then there exists  such that  (condition 2a). Observe that  would contradict  (condition 2b). Thereby  forms a directed- arc-disjoint from those of . So as announced, we prove that if  and  do not belong to the same module,  .

Let us now assume that  belong to the same maximal module  of . So w.l.o.g. assume that . Suppose that . Then  implies the existence of  with  (condition 2a). Observe that  is an arc of  ( and ) and thereby  is an arc of  as well ( belong to the same maximal module not including ). It follows that  (condition 2a): contradiction. So we have . Observe then that  and  implies that  and . Hence we obtain  (condition 3).


\item  and : If  and , we prove that  implying that  (condition 2b). Assume that . As  implies  (condition 1a) and as  implies  (condition 2b), we have . Assume that . Then  implies the existence of  such that for every ,  (condition 1b). Moreover  implies that for every ,  (condition 2a). It follows that . To conclude, observe now that  would contradict  (condition 1b). 



If , we prove that every  satisfies , implying that  (condition 2a). Assume that . Then  implies the existence of  such that for every , . Moreover  implies that  (condition 2b). By Claim~\ref{cl:consec}, as  and , we have . Thereby every  satisfies . Assume that . Since  implies that  (condition 1a) and since  implies that every  satisfies  (condition 2a), we have that every  satisfies . 



\item  and : If   and , we prove the existence of a vertex  such that for every , , implying that  (condition 1b). This is the case if , as  implies that every  satisfies  (condition 2a) and as  implies  (condition 2b). So assume that . Then  implies  (condition 2b). Moreover  implies the existence of  such that  (condition 2a). It follows by Claim~\ref{cl:consec} that every  satisfies .

If  , we prove that , implying that  (condition 1a). Assume that . Then   implies that every  satisfies  (condition 2a). Moreover  implies the existence of  such that  (condition 2a). It follows that . Assume that . As  implies  (condition 2b) and as  implies  (condition 2b), we have . In both cases, Claim~\ref{cl:comp} implies that .


\item  and : If   and , we prove that  implying that  (condition 2b). Assume that . Observe that  implies the existence of  such that  (condition 2a). Moreover  implies the existence of  such that every  satisfies  (condition 1b). It follows that  and by Claim~\ref{cl:transitive}, . Assume that . As  implies  (condition 2b) and as  implies  (condition 1a), we have that .


If , we prove the existence of  such that , implying that  (condition 2a). Assume that . Observe that  implies the existence of  such that  (condition 2a). Moreover  implies  (condition 1a). It follows that  and . Assume that . Then  implies  (condition 2b). Moreover  implies the existence of  such that every  satisfies  (condition 1a). It follows that  .


\item : If   and , we prove the existence of  such that every  satisfies , implying that  (condition 1b). Assume that . Then  implies  (condition 1a). Moreover  implies the existence of  such that every  satisfies  (condition 1b). It follows that for every  we have . Assume that . Then  implies the existence of  such that every  satisfies  (condition 1b). To conclude it suffices to observe that as  implies  (condition 1a), .


If , we prove that  implying that  (condition 1a). Assume that . As  implies  (condition 1a) and  as  implies   (condition 1a), we have that . Assume that . Then  implies the existence of  such that every  satisfies  (condition 1b). Moreover  implies the existence of  such that every  satisfies  (condition 1b). It follows that  and thereby Claim~\ref{cl:comp} shows that .
\end{itemize} \end{proofclaim}

\begin{claim} 
If  is an arc of  such that , then  and  both belong to .
\end{claim}
\begin{proofclaim}
Let  be an arc of . By construction,  restricted to  is an extension of . 
It follows that if  and  are two vertices of , then . If  and , then by construction of  (condition 2b of definition of ),  implies . The case  and  is symmetric.
\end{proofclaim}

It follows from the above claims that any linear extension of the partial order  (Claim~\ref{cl:PO}) is a vertex ordering satisfying the statement.
\end{proof} 	



\begin{theorem} \label{thm:kernelfasbt}
The \FASBT{} problem admits a kernel with  vertices. 
\end{theorem}

\begin{proof}

We prove the result by showing that any instance  of \FASBT{} such that  is reduced under Rule~\ref{rule:uselessvertexfast} and Rule~\ref{rule:modulesfasbt} and  is either a negative instance or can be reduced using Rule~\ref{rule:safepartition}. \\ 

Let  be a conflict packing of  and  be an ordering of  satisfying the conditions of Lemma~\ref{lem:cpfasbt} (notice that each can be computed in polynomial time). 
We will denote by  the set . A directed- is a triple of vertices  such that  and . A directed  is \emph{free} if  and it does not exist  such that  (observe that by construction of  we have ). If  is a free directed-, then there exist  such that  is a backward arc of  and , that is  (otherwise,  would not be reduced under Rule~\ref{rule:uselessvertexfast}). Observe that independent of whether  or , , the subset of vertices  contains a certificate of . Indeed suppose without loss of generality that  and .  If , then  is a directed-, otherwise  is a directed-.

We construct a bipartite graph  as follows: 1)  contains a vertex  for every backward arc  of ; 2)  contains a vertex  for every free directed- of ; and 3)  if and only if . Clearly  can be built in polynomial time. It is easy to observe that any matching in  witnesses an arc-disjoint set of directed-'s. It follows that if the size  of a maximum matching in  is  or more, then  is a negative instance. Let us now explain how to compute a safe partition when . We first show that as , we can extract from  a set  of at least  vertex disjoint free directed-'s. Let us call \emph{interval of } a maximal subset 
of vertices appearing consecutively in . 

\begin{claim} \label{cl:module-interval}
If  is a module of , then  is contained in a unique interval of .
\end{claim}
\begin{proofclaim}
First observe that  is a module of . To compute , we started from a topological ordering  of  and by Lemma~\ref{lem:module-ordering}, vertices of  are consecutive in .
Assume that vertices of  are not consecutive in . Then by construction of ,  contains a vertex  such that  and . Now to be a module of ,  has to contain vertex : contradicting the fact that every module of a bipartite tournament is an independent set.
\end{proofclaim}

As by Observation~\ref{prop:cpfasbt}, , the set  has size at least  and is partitioned in at most  intervals. Let  be the number of maximal modules of  intersecting . As  is reduced under Rule~\ref{rule:modulesfasbt}, every maximal module has size at most  and thus . By Claim~\ref{cl:module-interval},  where  is the set of intervals and  is the number of maximal modules of  intersecting . For every interval , define  and . Observe that  implying that . In other words, it is possible to extract from the maximal modules of  intersecting  a set  of at least  vertex disjoint free directed-'s, each directed- being composed by three vertices from distinct consecutive modules.

Recall that the bipartite graph has a maximum matching of size  and that . By K\"onig-Egervary's theorem~\cite{BM76},  has a vertex cover  of size . Let  and . It follows that . Let  be the ordered partition defined as follows: for every vertex , ,  and  form a singleton parts of ; the other parts are the maximal set of consecutive vertices not covered by the directed-'s of . 

\begin{claim} \label{cl:safe-partition}
The ordered partition  is a safe partition.
\end{claim}
\begin{proofclaim}
Let us first argue that the set  of external arcs contains at least one backward arc. As , the ordered partition  is non-trivial. Let  be a directed- such that . By construction, ,  and  are not in  and thereby not incident to any backward arc. As  is reduced by Rule~\ref{rule:uselessvertexfast},  belongs to a directed- and since the directed-  is free, there exists a backward arc  such that : proving that   contains at least one backward arc.

Let us now consider  an external backward arc of . By construction of , there exists  such that . Thereby  is an edge of  and  contains a certificate  (either  or  induces a directed-). Thereby as  is a vertex cover of  and ,  belongs to . It follows that the subset  corresponding to the external backward arcs is included in . By Lemma~\ref{lem:matching}, there is a matching between  and . This shows that the set external backward arcs can be certified using external arcs only, proving that  is safe.
\end{proofclaim}

This concludes the proof.
\end{proof}

