

In this section, we consider asymmetric and loopless directed graphs (digraphs for short). Let $D=(V,A)$ be a digraph. An arc oriented from vertex $u\in V$ to $v\in V$ is denoted $uv$. The \emph{in-neighbourhood} of a vertex $x$ is the set $N^-(x)=\{y\in V\mid yx\in A\}$ and the \emph{out-neighbourhood} of $x$ is $N^+(x)=\{y\in V\mid xy\in A\}$. If $V'\subseteq V$, then $D[V']=(V',A')$ is the subdigraph \emph{induced} by $V'$, that is $A'=\{uv \in A\mid u\in V', v\in V'\}$. Similarly, if $A'\subseteq A$, then $D[A']=(V',A')$ denotes the \emph{digraph} where $V'=\{v\in V\mid \exists uv\in A' \mbox{ or } vw\in A'\}$. We say that $D$ is \emph{acyclic} if it does not contain any circuit and that $D$ is \emph{transitive} if for every triple of vertices $\{u, v, w\}\subseteq V$ such that $uv\in A$ and $vw\in A$, then $uw\in A$. A \emph{tournament} $T=(V,A)$ is a complete asymmetric digraph and a \emph{bipartite tournament} $T=(X\uplus Y,A)$ is a complete bipartite digraph (there is an arc between $u$ and $v$ if and only if $u\in X$ and $v\in Y$). We denote by $\mathbf{FAS}(D)$ the size of a minimum feedback arc set of $D$.

A vertex ordering $\sigma$ of $D$ is a total order on the vertex set $V$. We denote by $u<_{\sigma} v$ the fact that $u$ is smaller than $v$ in $\sigma$. It is well-known that every acyclic digraph has a \emph{topological ordering} $\sigma$ of the vertices: for every arc $uv\in A$, $u<_{\sigma} v$. An \emph{ordered digraph} is a triple $D_{\sigma}=(V,A,\sigma)$ where $D=(V,A)$ is a digraph and $\sigma$ a vertex ordering of $D$. An arc $uv\in A$ in $D_{\sigma}$ is \emph{backward} if $v<_{\sigma} u$, \emph{forward} otherwise. Let $uv$ be a backward arc, then $span(uv)=\{w\in V\mid v<_{\sigma} w<_{\sigma} u\}$. A \emph{certificate} $c(uv)$ in $D_{\sigma}$ is the vertex set of a directed path from $v$ to $u$ only composed of forward arcs. Observe that every vertex $w\in c(uv)$ distinct from $u$ and $v$ belongs to $span(uv)$ and that $c(uv)$ induces a circuit in which $uv$ is the unique backward arc. We abusively say that an arc  belongs to a certificate if its two vertices do. A subset $B \subseteq A$ of \emph{backward arcs} can be \emph{certified} whenever there exists a collection $\{c(e) : e \in B\}$ of \emph{arc-disjoint} certificates.


The following Lemma on matching and vertex cover in bipartite graphs will be used in three of our kernels.

\begin{lemma} \label{lem:matching}
Let $ B = (X \uplus Y, E)$ be a bipartite graph, $S$ be a minimum vertex cover of $B$, $S_X = X \cap S$ and $S_Y = Y \cap S$. Any subset $I \subseteq S_X$ can be matched into $Y \setminus S_Y$. 
\end{lemma}
\begin{proof}
By Hall's theorem~\cite{H35}, $I$ can be matched into $Y\setminus S_Y$ iff $|I'| \leqslant |N(I') \cap (Y \setminus S_Y)|$ for every subset $I' \subseteq I$. Hence, assume for a contradiction that there exists $I'\subseteq I$ such that $|I'| > |N(I') \cap (Y \setminus S_Y)|$. As there is no edge in $B$ between $X\setminus S_X$ and $Y\setminus S_Y$ ($S$ is a vertex cover of $B$), the set $S'=(S \setminus I') \cup (N(I') \cap (Y \setminus S_Y))$ is a vertex cover of $B$ and $|S'|<|S|$: contradicting the minimality of $S$ (see Figure~\ref{fig:vcrti}). Thereby for every subset $I'\subseteq I$, we have $|I'| \leqslant |N(I') \cap (Y \setminus S_Y)|$. This concludes the proof.  
\end{proof}

\begin{figure}[h]
\begin{center}
\centerline{\includegraphics[scale=1]{vc.pdf}}
\caption{The case $|I'| > |N(I') \cap (Y \setminus S_Y)|$. The gray sets represent the vertex cover $S'$.\label{fig:vcrti}}
\end{center}
\end{figure}








\subsection{Reduction rules for feedback arc set in digraphs}

The following definitions and reduction rules were the core tool of the kernelization algorithm in~\cite{BFG+11}. The purpose of this section is to show how they can be easily implemented in the case of tournaments and bipartite tournaments.

A vertex of a digraph $D$ is {irrelevant} if it does not belong to any circuit of $D$. It is easy to observe that  optimal feedback arc sets of a digraph avoid  arcs incident to irrelevant vertices, which implies the following reduction rule.

\begin{polyrule} \label{rule:uselessvertexfast}
Remove every irrelevant vertex.
\end{polyrule}

A partition $\mathcal{P} = \{V_1, \dots, V_l\}$ of $V$ is an \emph{ordered partition} of an ordered digraph $D_\sigma$ if for every $i \in [l]$, $V_i$ is a set of consecutive vertices in $\sigma$. Hereafter an ordered partition of $D_\sigma$ will be denoted by $\mathcal{P}_\sigma=(V_1,\dots V_l)$, assuming that for every $x\in V_i$ and $y\in V_j$ with $1\leqslant i<j\leqslant l$, $x<_\sigma y$. The set of \emph{external arcs} of $\mathcal{P}_\sigma$ is $A_E(D_{\sigma},\mathcal{P}_\sigma)=\{uv\mid u\in V_i, v\in V_j, i\neq j\}$. A certificate $c(uv)$ of an external backward arc $uv$ is \emph{external} if the vertices of $c(uv)$ belong to distinct parts of $\mathcal{P}_\sigma$ (the arcs of the $v,u$-path are external). We say that a subset $B\subseteq A_E(D_{\sigma},\mathcal{P}_\sigma)$ of backward arcs is \emph{externally certified} if there exists a set $\{c(e)\mid e\in B\}$ of pairwise arc-disjoint external certificates.
	
	
\begin{definition}[Safe Partition] 
Let $D_\sigma = (V, A, \sigma)$ be an ordered digraph. An ordered partition $\mathcal{P}_\sigma = (V_1,\dots,V_l)$ of $D_\sigma$ is a \emph{safe partition} if the subset $B_E$ of backward arcs of $A_E(D_{\sigma},\mathcal{P}_\sigma)$ is non-empty and can be can be externally certified (see Figure~\ref{fig:spfasbt}).
\end{definition}
	
\begin{figure}[h]
\centerline{\includegraphics[scale=2]{safepartition}}
\caption{The two external backward arcs of the partition can be certified using the paths $\Pi$ and $\Pi_1$. In this example, there are no other external backward arcs. \label{fig:spfasbt}}
\end{figure} 
	
\begin{polyrule}[Safe partition~\cite{BFG+11}] \label{rule:safepartition}
Let $(D=(V,A),k)$ be an instance of \FAS{}, $\sigma$ be a vertex ordering of $V$ and $\mathcal{P}_\sigma = (V_1,\dots,V_l)$ be a safe partition of  $D_\sigma = (V, A, \sigma)$. Return $(D',k')$ where $D'$ is obtained from $D$ by reversing the external backward arcs $B_E\subseteq A_E(D_{\sigma},\mathcal{P}_\sigma)$ and where $k'=k-|B_E|$.
\end{polyrule}


\subsection{Tournaments}

A \emph{directed-$C_3$} (or circuit of size $3$) is a set of vertices $\{u,v,w\}$ such that $\{uv,vw,wu\}\subseteq A$. As shown by the following well-known statement, irrelevant vertices in the case of tournaments can be defined as the vertices not contained in any directed-$C_3$.

\begin{lemma}[Folklore]
\label{lem:folklore}
Let $T=(V,A)$ be a tournament. Then the following properties are equivalent: (i) $T$ is acyclic; (ii) $T$ is transitive; (iii) $T$ does not contain any directed-$C_3$.
\end{lemma}

Consequently, it will be possible to certify any backward arc $uv$ of an ordered tournament $T_{\sigma}=(V,A,\sigma)$ with a unique vertex $w\in span(uv)$ such that $\{u,v,w\}$ induces a directed-$C_3$. Clearly the cardinality of any set of arc-disjoint directed-$C_3$ provides a lower bound on the minimum feedback arc set of a tournament. We show how such a set can be used to compute in polynomial time a safe partition. 



\newcommand{\T}{T_{\sigma}}
\newcommand{\V}{V_{\sigma}}

\begin{definition}[Conflict packing] \label{def:cpfast}
A \emph{conflict packing} of a tournament $T$ is a \emph{maximal} collection of arc-disjoint directed-$C_3$'s.
\end{definition}

Observe that a conflict packing can be computed in polynomial time, for instance by using a greedy algorithm. Given a conflict packing $\mathcal{C}$ of an instance $(T = (V,A),k)$ of \FAST{}, $V(\mathcal{C})$ denotes the set of vertices covered by $\mathcal{C}$. 

\begin{observation} \label{lem:cpfast}
Let $\mathcal{C}$ be a conflict packing of a tournament $T$. If $\mathbf{FAS}(T)\leqslant k$, then $|\mathcal{C}| \leqslant k$ and $|V(\mathcal{C})| \leqslant 3k$.
\end{observation}



\begin{lemma}[Conflict Packing] \label{lem:realcpfast}
Let $\mathcal{C}$ be a conflict packing of a tournament $T=(V,A)$. There exists a vertex ordering $\sigma$ of $T$ such that every backward arc $uv$ of $T_{\sigma}$ satisfies $u,v \in V(\mathcal{C})$.
\end{lemma}

\begin{proof}
Let us consider the vertex set $V'=V \setminus V(\mathcal{C})$. By maximality of $\mathcal{C}$, $T'=T[V']$ is acyclic. Observe also that for every $v\in V(\mathcal{C})$, $T[V'\cup\{v\}]$ is acyclic: otherwise $T$ would contain a directed-$C_3$ $\{v,u,w\}$ with $u,w\notin V(\mathcal{C})$ arc-disjoint from those of $\mathcal{C}$. Let $\tau$ be the unique topological ordering of $T'$. For every vertex $v\in V(\mathcal{C})$, there is a unique pair $u,w$ of  consecutive vertices in $\tau$ such that the insertion of $v$ between $u$ and $w$ yields the topological ordering of $T[V'\cup\{v\}]$. The pair $\{u,w\}$ is called the \emph{locus} of $v$. Consider any ordering $\sigma$ on $V$ such that: 
\begin{itemize}
\vspace{-0.2cm}
\item for $u,w\in V'$, if $u<_{\tau} w$, then $u\<w$, and 
\vspace{-0.2cm}
\item if $v\in V(\mathcal{C})$, and $\{u,w\}$ is the locus of $v$, then $u\< v\<w$. 
\end{itemize}
By construction of $\sigma$, every backward arc $uv$ of $T_{\sigma}$ satisfies $u,v \in V(\mathcal{C})$.
\end{proof}

\begin{theorem} \label{thm:fast}
The \FAST{} problem admits a kernel with at most $4k$ vertices.\end{theorem}
\begin{proof}
We prove the result by showing that any instance $(T = (V,A), k)$ of \FAST{} such that $T$ is reduced under Rule~\ref{rule:uselessvertexfast} and $|V| > 4k$ is either a negative instance or can be reduced using Rule~\ref{rule:safepartition}. \\ 

Let $\mathcal{C}$ be a conflict packing of $T$ and $\sigma$ be the vertex ordering obtained through Lemma~\ref{lem:realcpfast}. 
Consider the bipartite graph $B = (I \cup V', E)$ where:
\begin{enumerate}[(i)]
\item $V' = V \setminus V(\mathcal{C})$, 
\vspace{-0.2cm}
\item there is a vertex $i_{vu}$ in $I$ for every backward arc $vu$ of $\T$ and, 
\vspace{-0.2cm}
\item $i_{vu}w \in E$ if $w \in V'$ and $\{u,v,w\}$ is a certificate of $vu$.
\end{enumerate}	 
	
Observe that any matching in $B$ of size at least $k+1$ corresponds to a conflict packing of size at least $k+1$, in which case the instance is negative (Observation~\ref{lem:cpfast}). Hence, by K\"onig-Egervary's theorem~\cite{BM76}, a minimum vertex cover $S$ of $B$ has size at most $k$. We denote $S_I=S\cap I$ and $S_{V'}=S\cap V'$.
Assume that $|V| > 4k$. Since $|S_{V'}| \leqslant k$ and  $|V(\mathcal{C})| \leqslant 3k$ (Lemma~\ref{lem:cpfast}), we have 
$V' \setminus S_{V'} \neq \emptyset$. Let $\mathcal{P}_\sigma=(V_1, \dots, V_l)$ be the ordered partition of $\T$ such that every part $V_i$ consists of either a vertex of $V' \setminus S_{V'}$ or a maximal subset of consecutive vertices (in $\sigma$) of $V\setminus (V' \setminus S_{V'})$ (see Figure~\ref{fig:spfast}).

\begin{figure}[h]
\centerline{\includegraphics[scale=1]{spfast}}
\caption{The construction of the ordered partition $\mathcal{P}_\sigma$ based on the bipartite graph $B$. The vertices of $V' \setminus S_{V'} = \{u,v,w\}$ are alone in their parts while the other vertices are grouped in maximal subsets of consecutive vertices (w.r.t. $\sigma)$.  \label{fig:spfast}}
\end{figure}

\begin{claim} \label{claim:pissafe}
$\mathcal{P}_\sigma$ is a safe partition of $\T$.
\end{claim}
\begin{proofclaim}
Let $w$ be a vertex of $V'\setminus S_{V'}$. By Lemma~\ref{lem:realcpfast}, $w$ is not incident to any backward arc.
Since $T$ is reduced under Rule~\ref{rule:uselessvertexfast}, there must exist a backward arc $f$ such that $w\in span(f)$. It follows that the set $A_E(T_{\sigma},\mathcal{P}_\sigma)$ of external arc contains at least one backward arc. 
Let $e=vu\in A_E(T_{\sigma},\mathcal{P}_\sigma)$ be a backward arc of $\sigma$. By construction of $\mathcal{P}_\sigma$, there exists a vertex $w_e \in (V' \setminus S_{V'})\cap span(e)$. Then $\{u,v,w_e\}$ is a certificate of $e$ and $i_{vu}w_e$ is an edge of $B$. Observe that since $S$ is a vertex cover and $w_e \notin S_{V'}$, the vertex $i_{vu}$ belongs to $S_I$ (otherwise the edge $i_{vu}w_e$ would not be covered). Thereby the subset $I'\subseteq I$ corresponding to the backward arcs of $A_E$ is included in $S_I$. 
By Lemma~\ref{lem:matching}, we know that $I'$ can be matched into $V' \setminus S_{V'}$ in $B$. 
Since every vertex of $V' \setminus S_{V'}$ is a singleton in $\mathcal{P}_\sigma$, the existence of the matching shows that the backward arcs of $A_E$ can be certified using arcs of $A_E$ only, and hence $\mathcal{P}_\sigma$ is safe. 
\end{proofclaim}

This concludes the proof. 
\end{proof}
 
\subsection{Bipartite tournaments}

A  \emph{directed-$C_4$} (or circuit of length $4$) is a subset $\{w,x,y,z\}$ of vertices such that $\{wx, xy, yz, zw\}\subseteq A$. The following statement shows that directed-$C_4$'s will play the same role as directed-$C_3$'s in tournaments. 

\begin{lemma}\cite{DGHNT10}
A bipartite tournament is acyclic if and only if it does not contain a directed-$C_4$.
\end{lemma}

\begin{lemma} \cite{MRRS13}
If a vertex of a bipartite tournament belongs to a circuit, then it belongs to a directed-$C_4$.
\end{lemma}

It follows that in the case of bipartite tournaments, an irrelevant vertex is a vertex not belonging to any directed-$C_4$. Let $T_\sigma = (X\uplus Y,A,\sigma)$ be an ordered bipartite tournament. A \emph{certificate} for a backward arc $e = wz$ of $T_\sigma$ (i.e. $z<_{\sigma} w$) is a set $c_B(e)=\{z,u,v,w\}$ such that $zu$, $uv$ and $vw$ are forward arcs. In other words, $c_B(e)$ is a directed-$C_4$ for which $e$ is the unique backward arc. 

\begin{definition} \label{def:cpfasbt}
A \emph{conflict packing} $\mathcal{C}$ of a bipartite tournament is a \emph{maximal} collection of \emph{arc-disjoint} directed-$C_4$'s.
\end{definition}

Remark that two directed-$C_4$'s of $\mathcal{C}$ may intersect on two vertices, provided that these vertices both belong to  either $X$ or $Y$. Here again, such a collection can be computed in polynomial time (using for instance a greedy algorithm). 

\begin{observation} \label{prop:cpfasbt}
Let $\mathcal{C}$ be a conflict packing of a tournament $T=(X\uplus Y, A)$. If $\mathbf{FAS}(T)\leqslant k$, then $|\mathcal{C}| \leqslant k$ and $|V(\mathcal{C})| \leqslant 4k$.
\end{observation}


The kernel in the case of bipartite tournaments requires a third reduction rule based on modules. A \emph{module} in a digraph $D=(V,A)$ is a set of vertices $M$ such that for every $u,v\in V$, $N^+(u)\setminus M=N^+(v)\setminus M$ and $N^-(u)\setminus M=N^-(v)\setminus M$. A module $M$ is \emph{non-trivial} if $|M| > 1$ and $M \neq V$. A module $M$ is \emph{maximal} if it is not included in any non-trivial module $M'$. It is easy to observe that in a bipartite tournament $T=(X\uplus Y,A)$, any non-trivial module $M$ is either a subset of $X$ or of $Y$ (and thereby an independent set). 

\begin{polyrule}[\cite{MRRS13}] \label{rule:modulesfasbt}
Let $M$ be a non-trivial module of $T$ of size at least $k + 1$. Remove all but $k + 1$ arbitrary vertices from $M$. 
\end{polyrule}

The correctness of this rule is a consequence of the following observation. Consider an instance without any irrelevant vertex. Then every vertex belongs to a $C_4$ and whenever a module $M$ of size at least $k+1$ exists, there exists a set $\{\{t,u,v,w\}\mid t\in M \}$ of at least $k+1$ directed-$C_4$'s pairwise intersecting on the arcs $uv$ and $vw$. Thereby any solution of size at most $k$ has to reverse one of those two arcs.

To compute a safe partition, we follow the main two steps described in the case of \FAST{}: 1) based on a conflict packing, compute a permutation $\sigma$ of $X\uplus Y$ such that the vertices incident to the backward arcs are covered by the conflict packing; 2) compute from $\sigma$ a safe partition of $X\uplus Y$. Let us show how to compute the vertex ordering $\sigma$.

\begin{lemma} \label{lem:module-ordering}
Let $T=(X\uplus Y,A)$ be an acyclic bipartite tournament. There exists a partial order $\tau$ on the vertices such that every topological ordering $\sigma$ of $T$ is a linear extension of $\tau$.
\end{lemma}
\begin{proof}
We define $\tau$ as follows: $x<_{\tau} y$ if and only if $xy\in A$ or there exists $z$ such that $xz\in A$ and $zy\in A$. It is easy to observe that two vertices are incomparable with respect to $\tau$ if and only if they belong to the same module. Let us prove that if $\sigma$ is a topological ordering of $T$, then the set of vertices of any maximal module $M$ of $T$ occurs consecutively in $\sigma$, implying the result.

Assume by contradiction that there exist $x,y\in M$ and $v\notin M$ such that $x<_{\sigma}v<_{\sigma} y$. W.l.o.g. assume that $M\subseteq X$. First consider that $v\in Y$. If  $v\in N^+(x)$ (the case  $v\in N^-(x)$ is similar) then $v\in N^+(y)$, in other words the arc $yv$ is backward in $\sigma$: contradicting the fact that $\sigma$ is a topological ordering. Assume that $v\in X$. As $v\notin M$, there exists a vertex $s\notin M$ distinguishing $v$ from $x,y$: either $s\in N^+(v)$, $s\in N^-(x)$ and $s\in N^-(y)$ or $s\in N^-(v)$, $s\in N^+(x)$ and $s\in N^+(y)$. In any case, one of the three arcs between $s$ and $x, u$ or $y$ is backward in $\sigma$: contradicting the fact that $\sigma$ is a topological ordering.
\end{proof}





\begin{lemma} \label{lem:cpfasbt}
Let $\mathcal{C}$ be a conflict packing of a bipartite tournament $T = (X\uplus Y,A)$. There exists a vertex ordering $\sigma$ of $T$ such that every backward arc $uv$ of $T_\sigma$ (i.e. $v<_{\sigma} u$) satisfies $u,v \in V(\mathcal{C})$. 
\end{lemma}
\begin{proof}
Let $V'$ be the set of vertices not covered by $\mathcal{C}$, i.e. $V'=(X\uplus Y)\setminus V(\mathcal{C})$. By maximality of $\mathcal{C}$, $T'=T[V']$ is acyclic. Let $\tau$ be the partial order on $V'$ defined by Lemma~\ref{lem:module-ordering}. We define a vertex ordering $\sigma$ by extending $\tau$ (adding comparabilities) and inserting vertices of $V(\mathcal{C})$.
Let $x,y$ be two arbitrary vertices of $T$:

\begin{enumerate}
\item $x\in V(\mathcal{C})$ and $y\in V(\mathcal{C})$:
\begin{enumerate}
\item $x\in X$ and $y\in X$: if $N^-(x)\cap V'=N^-(y)\cap V'$, then $x$ and $y$ are incomparable, otherwise if $N^-(x)\subset N^-(y)$, then $x<_{\sigma} y$; 
\item  $x\in X$ and $y\in Y$: if  $\exists v\in N^-(y)\cap V'$ such that $\forall u\in N^-(x)\cap V'$, $u<_{\tau} v$, then $x<_{\sigma} y$, otherwise $y<_{\sigma} x$;
\end{enumerate}

\item If $x\in V'$ and $y\in V(\mathcal{C})$: 
\begin{enumerate}
\item $x\in X$ and $y\in X$: 
if $\exists u\in N^-(y)\cap V'$ such that $x<_{\tau} u$, then $x<_{\sigma} y$, otherwise $y<_{\sigma} x$;
\item  $x\in X$ and $y\in Y$: if $x\in N^-(y)\cap V'$, then $x<_{\sigma} y$, otherwise $y<_{\sigma} x$;
\end{enumerate}

\item If $x\in V'$ and $y\in V'$: if $x$ and $y$ are not comparable in $\tau$ and $\exists u\in V(\mathcal{C})$ with $x\in N^-(u)$ and $y\in N^+(u)$, then $x<_{\sigma} y$;
\end{enumerate}

The following claims will help to prove that $\sigma$ is a partial order.
 
\begin{claim} \label{cl:comp}
Let $x$ and $y$ be two vertices of $V(\mathcal{C})\cap X$ (resp. $V(\mathcal{C})\cap Y$), then $(N^+(x)\cap V')\subseteq (N^+(y)\cap V')$ or vice versa.
\end{claim}
\begin{proofclaim}
Assume there exist $u\in (N^+(x)\cap V')\setminus (N^+(y)\cap V')$ and $v\in (N^+(x)\cap V')\setminus (N^+(y)\cap V')$. Then $\{x, u, y, v\}$ is a directed-$C_4$ which is arc-disjoint from every directed-$C_4$ of $\mathcal{C}$, contradicting the maximality of $\mathcal{C}$.
\end{proofclaim}

\begin{claim} \label{cl:consec}
Let $y$ be a vertex of $V(\mathcal{C})$. If $x\in V'$ and $z\in V'$ are two comparable vertices in $\tau$ such that $x\in N^-(y)$ and $z\in N^+(y)$, then $x<_{\tau} z$.
\end{claim}
\begin{proofclaim}
Assume that $z<_{\tau} x$. As $x$ and $z$ do not belong to the same module of $T'$ there exists $u\in V'$ with $z<_{\tau} u<_{\tau} x$ such that $zu$ and $ux$ are arcs of $T$. It follows that the vertices $zuxy$ form a directed-$C_4$ arc-disjoint from those of $\mathcal{C}$: contradicting the maximality of $\mathcal{C}$.
\end{proofclaim}

\begin{claim} \label{cl:transitive}
Let $x,z$ be two vertices such that $x\in V'\cap X$ and $z\in V(\mathcal{C})\cap Y$ (resp. $x\in V'\cap Y$ and $z\in V(\mathcal{C})\cap X$). If there exists $y\in V'$ such that $x<_{\tau} y$ and $y<_{\sigma} z$, then $x\in N^-(z)$ and $x<_{\sigma} z$.
\end{claim}
\begin{proofclaim}
We consider the case where $x \in V' \cap X$ and $z \in V(\mathcal{C}) \cap Y$, the other case being symmetric.
By condition 2 of definition of $\sigma$, $y<_{\sigma} z$ implies that if $y\in X$, then $y\in N^-(z)$ otherwise there exists $u\in N^-(z)\cap V'$ such that $y<_{\tau} u$. Assume that $y\in X$. As $x<_{\tau} y$, $x$ and $y$ do not belong to the same maximal module. So there exists $v\in Y\cap V'$ such that $x<_{\tau} v<_{\tau} y$. Observe that if $x\in N^+(z)$, then the vertices $\{x,v,y,z\}$ form a directed-$C_4$. Likewise if $y\in Y$, the vertices $\{x,y,u,z\}$ form a directed-$C_4$. In both cases we obtain a directed-$C_4$ arc-disjoint from those of $\mathcal{C}$: contradicting the maximality of $\mathcal{C}$. It follows that $x\in N^-(z)$ and thereby $x<_{\sigma} z$ (condition 2b of definition of $\sigma$).
\end{proofclaim}


\begin{claim} \label{cl:PO}
$\sigma$ is a partial order. 
\end{claim}
\begin{proofclaim}
Let us first observe that $\sigma$ is antisymmetric. If $x$ or $y$ belong to $V(\mathcal{C})$, antisymmetry is straightforward from definition of $\sigma$. Suppose that $x$ and $y$ both belong to $V'$. As new comparabilities are added only between pair of vertices non-comparable in $\tau$, antisymmetry follows from Claim~\ref{cl:comp}. 

Let us prove that $\sigma$ is transitive with a case by case analysis.  Keep in mind that two vertices are incomparable with respect to $\tau$ if and only if they belong to the same maximal module of $T$. Let $x, y, z$ be three vertices such that $x<_{\sigma} y$ and $y<_{\sigma} z$.
\begin{itemize}
\item $x,y,z\in V'$: If $x<_{\tau} y$ and $y<_{\tau} z$, as $\tau$ is transitive, we also have $x<_{\tau} z$ and thereby $x<_{\sigma} z$. Suppose that $x, y$ belong to the same maximal module of $T'$ and $y<_{\tau} z$, then by construction of $\tau$, we also have $x<_{\tau} z$ implying $x<_{\sigma} z$. The same argument holds if  $x<_{\tau} y$ and $y,z$ belong to the same maximal module of $T'$. The last case to consider is when $x,y,z$ belong to the same maximal module of $T'$. Then $x<_{\sigma} y$ implies the existence of $u\in V(\mathcal{C})$ such that $x\in N^-(u)$ and $y\in N^+(u)$ (condition 3 of definition of $\sigma$). Likewise, $y<_{\sigma} z$ implies the existence of $v\in V(\mathcal{C})$ such that $y\in N^-(v)$ and $z\in N^+(v)$. As $y\in N^-(v)\setminus N^-(u)$, Claim~\ref{cl:comp} implies that $N^-(u)\cap V'\subset N^-(v)\cap V'$ and thereby $x\in N^-(v)$. As $z\in N^+(v)$, it follows that $x<_{\sigma} z$ (condition 3 of definition of $\sigma$).


\item $x,y \in V'$ and $z\in V(\mathcal{C})$ (or $x\in V(\mathcal{C})$ and $y,z\in V'$): Assume that $x$ and $y$ belong to the same maximal module of $T'$. 
It follows that $x$ and $y$ both belong either to $X$ or to $Y$. Assume w.l.o.g. that $x,y\in X$.
If $z\in X$, then $y<_{\sigma} z$ implies the existence of $u\in N^-(z)\cap V'$ such that $y<_{\tau} u$ (condition 2a of definition of $\sigma$). As $x$ and $y$ belongs to the same module of $T'$, $y<_{\tau} u$ implies that $x<_{\tau} u$ and thereby $x<_{\sigma} z$. Assume that $z\in Y$. Then $y<_{\sigma} z$ implies $y\in N^-(z)$ (condition 2b of definition of $\sigma$). Moreover, $x <_\sigma y$ implies the existence of $u \in V(\mathcal{C})$ such that $x \in N^-(u)$ and $y \in N^+(u)$ (condition 3 of definition of $\sigma$). Observe that as $y\in N^+(u)\cap N^-(z)$, $u\neq z$. It follows that $x\in N^-(z)$ as well, 
since otherwise $\{x,u,y,z\}$ would form a directed-$C_4$ arc-disjoint from those of $\mathcal{C}$, contradicting the maximality of $\mathcal{C}$. Thereby we have $x<_{\sigma} z$.

If $x\in X$ and $z\in Y$, as $x<_{\tau} y$ and $y<_{\sigma} z$, by Claim~\ref{cl:transitive} we have $x<_{\sigma} z$. Suppose that $x,z\in X$. If $y\in X$, then $y<_{\sigma} z$ implies the existence of $u\in N^-(z)\cap V'$ such that $y<_{\tau} u$ (condition 2a). It follows that $x<_{\tau} u$. If $y\in Y$, then $y<_{\sigma} z$ implies that $y\in N^-(z)\cap V'$ (condition 2b). In both cases, there exists a vertex $v$ such that $v\in N^-(z)\cap V'$ and $x<_{\tau} v$. It follows that $x<_{\sigma} z$ (condition 2a). The other cases are symmetric. 

\item $x,z\in V'$ and $y\in V(\mathcal{C})$: 
We first consider the case where $x$ and $z$ do not belong to the same module of $T'$, implying that either $x<_\tau z$ or $z<_\tau x$. We prove that $z<_\tau x$ would contradict the maximality of $\mathcal{C}$. Suppose that $x\in X$, $z\in Y$ and that $y\in X$ (the case $y\in Y$ is symmetric). As $x<_\sigma y$, there exists $u\in N^-(y)\cap V'$ such that $x<_\tau u$ (condition 2a), or equivalently $x\in N^-(u)$. It follows that if $z<_\tau x$ or equivalently $z\in N^-(x)$, we obtain a directed-$C_4$ on $\{x,y,z,u\}$ arc-disjoint from those of $\mathcal{C}$. Suppose that $x\in X$ and $z\in X$. If $z<_\tau x$, then there exists $u\in Y\cap V'$ such that $u\in N^+(z)\cap N^-(x)$. If $y\in Y$, then $x<_\sigma y$ and $y<_{\sigma} z$ implies $y\in N^-(z)\cap N^+(x)$ (condition 2b). So $\{x,y,z,u\}$ forms a directed-$C_4$ arc-disjoint from those of $\mathcal{C}$. If $y\in X$, then there exists $v\in N^-(y)\cap V'$ such that $x<_\tau v$ (condition 2a). Observe that $u\in N^-(y)$ would contradict $y<_\sigma z$ (condition 2b). Thereby $\{x,y,u,v\}$ forms a directed-$C_4$ arc-disjoint from those of $\mathcal{C}$. So as announced, we prove that if $x$ and $z$ do not belong to the same module,  $x<_\tau z$.

Let us now assume that $x, z$ belong to the same maximal module $M$ of $T'$. So w.l.o.g. assume that $x,z\in X$. Suppose that $y\in X$. Then $x<_{\sigma} y$ implies the existence of $u\in N^-(y)\cap V$ with $x<_{\tau} u$ (condition 2a). Observe that $xu$ is an arc of $T$ ($u\in Y$ and $x<_{\tau} u$) and thereby $zu$ is an arc of $T$ as well ($x,z$ belong to the same maximal module not including $u$). It follows that $z<_{\sigma} y$ (condition 2a): contradiction. So we have $y\in Y$. Observe then that $x<_{\sigma} y$ and $y<_{\sigma} z$ implies that $x\in N^-(y)$ and $z\in N^+(y)$. Hence we obtain $x<_{\sigma} z$ (condition 3).


\item $x,y \in V(\mathcal{C})$ and $z\in V'$: If $x\in X$ and $z\in Y$, we prove that $z\in N^+(x)$ implying that $x<_{\sigma} z$ (condition 2b). Assume that $y\in X$. As $x<_{\sigma} y$ implies $N^-(x)\cap V' \subset N^-(y)\cap V'$ (condition 1a) and as $y<_{\sigma} z$ implies $z\in N^+(y)$ (condition 2b), we have $z\in N^+(x)$. Assume that $y\in Y$. Then $x<_{\sigma} y$ implies the existence of $v\in N^-(y)\cap V'$ such that for every $u\in N^-(x)\cap V'$, $u<_{\tau} v$ (condition 1b). Moreover $y<_{\sigma} z$ implies that for every $w\in N^-(y)\cap V'$, $w<_{\tau} z$ (condition 2a). It follows that $v<_{\tau} z$. To conclude, observe now that $z\in N^-(x)$ would contradict $v<_\tau z$ (condition 1b). 



If $x,z\in Y$, we prove that every $u\in N^-(x)\cap V'$ satisfies $u<_{\tau} z$, implying that $x<_{\sigma} z$ (condition 2a). Assume that $y\in X$. Then $x <_{\sigma} y$ implies the existence of $v\in N^-(y)\cap V'$ such that for every $u\in N^-(x)\cap V'$, $u<_{\tau} v$. Moreover $y<_{\sigma} z$ implies that $y\in N^-(z)$ (condition 2b). By Claim~\ref{cl:consec}, as $v\in N^-(y)$ and $z\in N^+(y)$, we have $v<_{\tau} z$. Thereby every $u\in N^-(x)\cap V'$ satisfies $u<_{\tau} z$. Assume that $y\in Y$. Since $x<_{\sigma} y$ implies that $N^-(x)\cap V'\subset N^-(y)\cap V'$ (condition 1a) and since $y<_{\sigma} z$ implies that every $v\in N^-(y)\cap V'$ satisfies $v<_{\tau} z$ (condition 2a), we have that every $u\in N^-(x)\cap V'$ satisfies $u<_{\tau} z$. 



\item $x,z\in V(\mathcal{C})$ and $y\in V'$: If  $x\in X$ and $z\in Y$, we prove the existence of a vertex $v\in N^-(z)\cap V'$ such that for every $u\in N^-(x)\cap V'$, $u<_{\tau} v$, implying that $x<_{\sigma} z$ (condition 1b). This is the case if $y\in X$, as $x<_{\sigma} y$ implies that every $u\in N^-(x)\cap V'$ satisfies $u<_{\tau} y$ (condition 2a) and as $y<_{\sigma} z$ implies $y\in N^-(z)$ (condition 2b). So assume that $y\in Y$. Then $x<_{\sigma} y$ implies $y\in N^+(x)$ (condition 2b). Moreover $y<_{\sigma} z$ implies the existence of $v\in N^-(z)\cap V'$ such that $y<_{\tau} v$ (condition 2a). It follows by Claim~\ref{cl:consec} that every $u\in N^-(x)\cap V$ satisfies $u<_{\tau} v$.

If  $x,z\in X$, we prove that $N^-(x)\cap V'\subset N^-(z)\cap V'$, implying that $x<_{\sigma} z$ (condition 1a). Assume that $y\in X$. Then  $x<_{\sigma} y$ implies that every $u\in N^-(x)\cap V'$ satisfies $u<_{\tau} y$ (condition 2a). Moreover $y<_{\sigma} z$ implies the existence of $v\in N^-(z)\cap V'$ such that $y<_{\tau} v$ (condition 2a). It follows that $v\in (N^-(z)\setminus N^-(x))\cap V'$. Assume that $y\in Y$. As $x<_{\sigma} y$ implies $y\in N^+(x)\cap V'$ (condition 2b) and as $y<_{\tau} z$ implies $y\in N^-(z)\cap V'$ (condition 2b), we have $y\in N^-(z)\setminus N^-(x))\cap V'$. In both cases, Claim~\ref{cl:comp} implies that $N^-(x)\cap V'\subset N^-(z)\cap V'$.


\item $x\in V'$ and $y,z\in V(\mathcal{C})$: If  $x\in X$ and $z\in Y$, we prove that $x\in N^-(z)$ implying that $x<_{\sigma} z$ (condition 2b). Assume that $y\in X$. Observe that $x<_{\sigma} y$ implies the existence of $u\in N^-(y)\cap V'$ such that $x<_{\tau} u$ (condition 2a). Moreover $y<_{\sigma} z$ implies the existence of $v\in N^-(z)\cap V'$ such that every $w\in N^-(y)\cap V$ satisfies $w<_{\tau} v$ (condition 1b). It follows that $x<_{\tau} v$ and by Claim~\ref{cl:transitive}, $x\in N^-(z)$. Assume that $y\in Y$. As $x<_{\sigma} y$ implies $x\in N^-(y)$ (condition 2b) and as $y<_{\sigma} z$ implies $N^-(y)\cap V'\subset N^-(z)\cap V'$ (condition 1a), we have that $x\in N^-(z)$.


If $x,z\in X$, we prove the existence of $u\in N^-(z)\cap V'$ such that $x<_{\tau} u$, implying that $x<_{\sigma} z$ (condition 2a). Assume that $y\in X$. Observe that $x<_{\sigma} y$ implies the existence of $u\in N^-(y)\cap V'$ such that $x<_{\tau} u$ (condition 2a). Moreover $y<_{\sigma} z$ implies $N^-(y)\cap V'\subset N^-(z)\cap V'$ (condition 1a). It follows that $u\in N^-(z)\cap V'$ and $x<_{\tau} u$. Assume that $y\in Y$. Then $x<_{\sigma} y$ implies $x\in N^-(y)$ (condition 2b). Moreover $y<_{\sigma} z$ implies the existence of $u\in N^-(z)$ such that every $v\in N^-(y)\cap V'$ satisfies $v<_{\tau} u$ (condition 1a). It follows that  $x<_{\tau} u$.


\item $x,y,z\in V(\mathcal{C})$: If  $x\in X$ and $z\in Y$, we prove the existence of $v\in N^-(z)\cap V'$ such that every $u\in N^-(x)\cap V'$ satisfies $u<_{\tau} v$, implying that $x<_{\sigma} z$ (condition 1b). Assume that $y\in X$. Then $x<_{\sigma} y$ implies $N^-(x)\cap V'\subset N^-(y)\cap V'$ (condition 1a). Moreover $y<_{\sigma} z$ implies the existence of $v\in N^-(z)\cap V'$ such that every $w\in N^-(y)\cap V'$ satisfies $w<_{\tau} v$ (condition 1b). It follows that for every $u\in N^-(x)\cap V'$ we have $u<_{\tau} v$. Assume that $y\in Y$. Then $x<_{\sigma} y$ implies the existence of $v\in N^-(y)\cap V'$ such that every $u\in N^-(x)\cap V'$ satisfies $u<_{\tau} v$ (condition 1b). To conclude it suffices to observe that as $y<_{\sigma} z$ implies $N^-(y)\cap V'\subset N^-(z)\cap V'$ (condition 1a), $v\in N^-(z)\cap V'$.


If $x,z\in X$, we prove that $N^-(x)\cap V'\subset N^-(z)\cap V'$ implying that $x<_{\sigma} z$ (condition 1a). Assume that $y\in X$. As $x<_{\sigma} y$ implies $N^-(x)\cap V'\subset N^-(y)\cap V'$ (condition 1a) and  as $y<_{\sigma} z$ implies  $N^-(y)\cap V'\subset N^-(z)\cap V'$ (condition 1a), we have that $N^-(x)\cap V'\subset N^-(z)\cap V'$. Assume that $y\in Y$. Then $x<_{\sigma} y$ implies the existence of $v\in N^-(y)\cap V'$ such that every $u\in N^-(x)\cap V'$ satisfies $u<_{\tau} v$ (condition 1b). Moreover $y<_{\sigma} z$ implies the existence of $w\in N^-(z)\cap V'$ such that every $v\in N^-(y)\cap V'$ satisfies $v<_{\tau} w$ (condition 1b). It follows that $w\in N^-(z)\setminus N^-(x)$ and thereby Claim~\ref{cl:comp} shows that $N^-(x)\cap V'\subset N^-(z)\cap V'$.
\end{itemize} \end{proofclaim}

\begin{claim} 
If $xy$ is an arc of $T$ such that $y<_{\sigma} x$, then $x$ and $y$ both belong to $V(\mathcal{C})$.
\end{claim}
\begin{proofclaim}
Let $xy$ be an arc of $T$. By construction, $\sigma$ restricted to $V'$ is an extension of $\tau$. 
It follows that if $x$ and $y$ are two vertices of $V'$, then $x<_{\sigma} y$. If $x\in V'$ and $y\in V(\mathcal{C})$, then by construction of $\sigma$ (condition 2b of definition of $\sigma$), $x\in N^-(y)$ implies $x<_{\sigma} y$. The case $x\in V(\mathcal{C})$ and $y\in V'$ is symmetric.
\end{proofclaim}

It follows from the above claims that any linear extension of the partial order $\sigma$ (Claim~\ref{cl:PO}) is a vertex ordering satisfying the statement.
\end{proof} 	



\begin{theorem} \label{thm:kernelfasbt}
The \FASBT{} problem admits a kernel with $O(k^2)$ vertices. 
\end{theorem}

\begin{proof}

We prove the result by showing that any instance $(T = (X\uplus Y,A), k)$ of \FASBT{} such that $T$ is reduced under Rule~\ref{rule:uselessvertexfast} and Rule~\ref{rule:modulesfasbt} and $|X\uplus Y| \geqslant 11k^2+16k+1$ is either a negative instance or can be reduced using Rule~\ref{rule:safepartition}. \\ 

Let $\mathcal{C}$ be a conflict packing of $T$ and $\sigma$ be an ordering of $X\uplus Y$ satisfying the conditions of Lemma~\ref{lem:cpfasbt} (notice that each can be computed in polynomial time). 
We will denote by $V'$ the set $(X\cup Y) \setminus V(\mathcal{C})$. A directed-$P_3$ is a triple of vertices $\{x,y,z\}$ such that $xy\in A$ and $yz\in A$. A directed $P_3$ is \emph{free} if $\{x,y,z\}\subseteq V'$ and it does not exist $u\in V(\mathcal{C})$ such that $x<_{\sigma} u<_{\sigma} z$ (observe that by construction of $\sigma$ we have $x<_{\sigma} y<_{\sigma} z$). If $\{x,y,z\}$ is a free directed-$P_3$, then there exist $u,v\in V(\mathcal{C})$ such that $uv$ is a backward arc of $T_{\sigma}$ and $\{x,y,z\}\subseteq span(uv)$, that is $v<_{\sigma} x<_{\sigma} y<_{\sigma} z<_{\sigma} u$ (otherwise, $T$ would not be reduced under Rule~\ref{rule:uselessvertexfast}). Observe that independent of whether $u\in X, v\in Y$ or $u\in Y$, $v\in X$, the subset of vertices $\{v,x,y,z,u\}$ contains a certificate of $uv$. Indeed suppose without loss of generality that $u\in X$ and $v\in Y$.  If $x\in X$, then $\{u,v,x,y\}$ is a directed-$C_4$, otherwise $\{u,v,y,z\}$ is a directed-$C_4$.

We construct a bipartite graph $B=(I\cup P,E)$ as follows: 1) $I$ contains a vertex $i_{uv}$ for every backward arc $uv$ of $T_{\sigma}$; 2) $P$ contains a vertex $p_{xyz}$ for every free directed-$P_3$ of $T_{\sigma}$; and 3) $(i_{uv},p_{xyz})\in E$ if and only if $\{x,y,z\}\subseteq span(uv)$. Clearly $B$ can be built in polynomial time. It is easy to observe that any matching in $B$ witnesses an arc-disjoint set of directed-$C_4$'s. It follows that if the size $\mu$ of a maximum matching in $B$ is $k+1$ or more, then $T$ is a negative instance. Let us now explain how to compute a safe partition when $\mu\leqslant k$. We first show that as $|X\uplus Y| \geqslant 11k^2+16k+1$, we can extract from $T_{\sigma}$ a set $\mathcal{Q}$ of at least $k+1$ vertex disjoint free directed-$P_3$'s. Let us call \emph{interval of $\sigma$} a maximal subset $I\subseteq V'$
of vertices appearing consecutively in $\sigma$. 

\begin{claim} \label{cl:module-interval}
If $M$ is a module of $T$, then $M\setminus V(\mathcal{C})$ is contained in a unique interval of $V'$.
\end{claim}
\begin{proofclaim}
First observe that $M'=M\setminus V(\mathcal{C})$ is a module of $T[V']$. To compute $\sigma$, we started from a topological ordering $\tau$ of $T[V']$ and by Lemma~\ref{lem:module-ordering}, vertices of $M'$ are consecutive in $\tau$.
Assume that vertices of $M'$ are not consecutive in $\sigma$. Then by construction of $\sigma$, $V(\mathcal{C})$ contains a vertex $u$ such that $N^-(u)\cap M'\neq\emptyset$ and $N^+(u)\cap M'\neq\emptyset$. Now to be a module of $T$, $M$ has to contain vertex $u$: contradicting the fact that every module of a bipartite tournament is an independent set.
\end{proofclaim}

As by Observation~\ref{prop:cpfasbt}, $|V(\mathcal{C})|\leqslant 4k$, the set $V'$ has size at least $11k^2+12k+1$ and is partitioned in at most $4k-1$ intervals. Let $m$ be the number of maximal modules of $T$ intersecting $V'$. As $T$ is reduced under Rule~\ref{rule:modulesfasbt}, every maximal module has size at most $k+1$ and thus $m\geqslant 11k+1$. By Claim~\ref{cl:module-interval}, $m=\sum_{I\in\mathcal{I}} m_I$ where $\mathcal{I}$ is the set of intervals and $m_I$ is the number of maximal modules of $T$ intersecting $I$. For every interval $I$, define $r_I=m_I\mod 3$ and $k_I=(m_I-r_I)/3$. Observe that $\sum_{I\in\mathcal{I}} r_I\leqslant 8k+2$ implying that $\sum_{I\in\mathcal{I}} k_I\geqslant k+1$. In other words, it is possible to extract from the maximal modules of $T$ intersecting $V'$ a set $\mathcal{Q}$ of at least $k+1$ vertex disjoint free directed-$P_3$'s, each directed-$P_3$ being composed by three vertices from distinct consecutive modules.

Recall that the bipartite graph has a maximum matching of size $\mu\leqslant k$ and that $|P|=|\mathcal{Q}|\geqslant k+1$. By K\"onig-Egervary's theorem~\cite{BM76}, $B$ has a vertex cover $S$ of size $\mu$. Let $S_I=S\cap I$ and $S_P=S\cap P$. It follows that $P\setminus S_P\neq\emptyset$. Let $\mathcal{P}_\sigma$ be the ordered partition defined as follows: for every vertex $p_{xyz}\in  P\setminus S_P$, $\{x\}$, $\{y\}$ and $\{z\}$ form a singleton parts of $\mathcal{P}_\sigma$; the other parts are the maximal set of consecutive vertices not covered by the directed-$P_3$'s of $S_P\setminus P$. 

\begin{claim} \label{cl:safe-partition}
The ordered partition $\mathcal{P}_\sigma$ is a safe partition.
\end{claim}
\begin{proofclaim}
Let us first argue that the set $A_E(T_\sigma,\mathcal{P}_\sigma)$ of external arcs contains at least one backward arc. As $P\setminus S_P\neq\emptyset$, the ordered partition $\mathcal{P}_\sigma$ is non-trivial. Let $\{x,y,z\}$ be a directed-$P_3$ such that $p_{xyz}\in P\setminus S_P$. By construction, $x$, $y$ and $z$ are not in $V(\mathcal{C})$ and thereby not incident to any backward arc. As $T$ is reduced by Rule~\ref{rule:uselessvertexfast}, $z$ belongs to a directed-$C_4$ and since the directed-$P_3$ $\{x,y,z\}$ is free, there exists a backward arc $uv$ such that $v<_{\sigma} x<_{\sigma} y<_{\sigma} z<_{\sigma} u$: proving that  $A_E(T_\sigma,\mathcal{P}_\sigma)$ contains at least one backward arc.

Let us now consider $uv$ an external backward arc of $\mathcal{P}_\sigma$. By construction of $\mathcal{P}_\sigma$, there exists $p_{xyz}\in P\setminus S_P$ such that $x,y,z\in span(uv)$. Thereby $(i_{uv},p_{xyz})$ is an edge of $B$ and $\{u,x,y,z,v\}$ contains a certificate $c_I(uv)$ (either $\{u,v,x,y\}$ or $\{u,v,y,z\}$ induces a directed-$C_4$). Thereby as $S$ is a vertex cover of $B$ and $p_{xyz}\in P\setminus S_P$, $i_{uv}$ belongs to $S$. It follows that the subset $I_E\subseteq I$ corresponding to the external backward arcs is included in $S_B$. By Lemma~\ref{lem:matching}, there is a matching between $I_E$ and $P\setminus S_P$. This shows that the set external backward arcs can be certified using external arcs only, proving that $\mathcal{P}_\sigma$ is safe.
\end{proofclaim}

This concludes the proof.
\end{proof}

