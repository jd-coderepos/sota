\documentclass[submission,copyright,creativecommons]{eptcs}
\providecommand{\event}{AFL 2014} \usepackage{breakurl}             

\usepackage{amsmath,amssymb,amsthm}
\usepackage{tikz}
\mathchardef\mhyphen="2D

\setlength{\parindent}{0pt}
\setlength{\parskip}{0.1in}

\title{Synchronizing weighted automata}
\author{Szabolcs Iv\'an
\institute{University of Szeged, Hungary}
\email{szabivan@inf.u-szeged.hu}
}
\def\titlerunning{Synchronizing weighted automata}
\def\authorrunning{Szabolcs Iv\'an}

\def\bK{{\mathbf{K}}}
\def\bN{{\mathbf{N}}}
\def\bZ{{\mathbf{Z}}}
\def\bB{{\mathbf{B}}}

\theoremstyle{plain} 
\newtheorem{theorem}{Theorem} 
\newtheorem{lemma}{Lemma} 
\newtheorem{proposition}{Proposition}

\theoremstyle{definition} 
\newtheorem{definition}{Definition}{}

\theoremstyle{remark} 
\newtheorem{remark}{Remark} 

\begin{document}

\maketitle

\begin{abstract}
We introduce two generalizations of synchronizability to automata with transitions weighted in an arbitrary semiring .
(or equivalently, to finite sets of matrices in .)
Let us call a matrix  location-synchronizing if there exists a column in  consisting of nonzero entries such that all the other columns of  are filled by zeros.
If additionally all the entries of this designated column are the same, we call  synchronizing.
Note that these notions coincide for stochastic matrices and also in the Boolean semiring.
A set  of matrices in  is called (location-)synchronizing if  generates a matrix subsemigroup containing a (location-)synchronizing matrix.
The -(location-)synchronizability problem is the following: given a finite set  of  matrices with entries in ,
is it (location-)synchronizing? 
Both problems are PSPACE-hard for any nontrivial semiring.
We give sufficient conditions for the semiring  when the problems are PSPACE-complete and show several undecidability results as well, e.g. synchronizability is undecidable if  has infinite order in  or when the free semigroup on two generators can be embedded into .
\end{abstract}

\section{Introduction}

The synchronization (directing, reseting) problem of classical, deterministic automata is a well-studied topic with a vast literature (see e.g.~\cite{volkov} for a survey). An automaton  is \emph{synchronizable} if some word  induces a constant function on its state set, in which case  is a synchronizing word of .
Deciding whether an automaton is synchronizable can be done in polynomial time and it is also known that for synchronizable automata, a synchronizing word of length  exists, where  denotes the number of its states.
(The famous \v{C}ern\'y conjecture from the sixties states that this bound is .)

The notion of synchronizability has been extended e.g. (in three different ways) to nondeterministic automata in~\cite{imreh}, to stochastic automata in~\cite{kfouri} and more recently in another way in~\cite{doyen}, to integer-weighted transitions in~\cite{larsen}.
To our knowledge, only ad-hoc notions have been defined so far, each for a particular underlying semiring.
We note that in~\cite{larsen} the notion has also been extended to timed automata as well.

In this paper we introduce several extensions of synchronizability to automata with transitions weighted in an arbitrary semiring . For states  and word , let  denote the sum of the weights of all -labeled paths from  to , with the weight of a path being the product of the weights of its edges, as usual.
Following the nomenclature of~\cite{larsen},
we call the automaton  \emph{location-synchronizable} if : \quad iff  and \emph{synchronizable} if : \quad  and  for each .

As an equivalent formulation, let us call a matrix  \emph{location synchronizing} if it contains a column entirely filled with nonzero values, and all its other entries are zero.
If in addition all the nonzero values are the same, we call  \emph{synchronizing}.
Then, an instance of the synchronizability problems is a finite set  of matrices, each in . 
The family  is called (location) synchronizable if it generates a (location) synchronizing matrix.
The question is to decide whether the instance is (location) synchronizing.

Note that these notions coincide for stochastic automata and also in the Boolean semiring.
For unconstrained automata, both problems are -hard for any nontrivial semiring, and in any semiring,
the length of the shortest directing word can be exponential.
We give sufficient conditions for the semiring  when the problems are in  (and hence are -complete)
and show several undecidability results as well.

\section{Notation}

A \emph{semiring} is an algebraic structure  where  is a commutative monoid with identity ,
 is a monoid with identity ,  is an annihilator for  and  distributes over , i.e.
,  and  for each . (When the context is clear, we usually omit the  sign.)
The case when  is that of the trivial semiring; when , the semiring is nontrivial.
Three semirings used in this paper are
the \emph{Boolean semiring} 
and the semirings  and  of the natural numbers  and the integers 
with the standard addition and product. Among these, only  is a ring since the other two have no additive inverses.
A semiring  is zero-sum-free if  implies ; is zero-divisor-free if  implies  or ;
is positive if it is both zero-sum-free and zero-divisor-free;
is locally finite if for any finite , the least subsemiring of  containing  (which is also called
the subsemiring of  generated by ) is finite.

An \emph{alphabet} is a finite nonempty set, usually denoted  in this paper.
When  is an integer,  stands for the set .
For a set ,  denotes its power set .
For any alphabet , the semiring of \emph{languages} over  is  where
product is concatenation of languages,  and  stands for the empty word.

When  is a semiring and  is an integer, then the set  of  matrices with entries in 
also forms a semiring with pointwise addition  (for clarity,  stands for the entry in the
th row and th column)
and the usual matrix product .
The zero element is the null matrix 
and the one element is the identity matrix  in .

In this article we only take products of matrices, no sums and thus use the notion  when
 is a set of matrices for the least sub\emph{monoid} of the monoid 
containing . That is,  contains all products of the form  with
 and  for each .

For a semiring , alphabet  and integer , an \emph{-state -weighted -automaton}
is a system 
where  are the \emph{initial} and \emph{final} vectors, respectively and for each ,
 is a \emph{transition matrix}. The mapping  extends in a unique way to a homomorphism
,  with . The automaton  above associates to
each word  a weight , where  is considered as a  row vector and  as an
 column vector. We usually do not specify the number  of states explicitly and omit  and  when
the weight structure and/or the alphabet is clear from the context.

\section{Synchronizability in various semirings}

Classical nondeterministic automata (with multiple initial states but no -transitions)
can be seen as automata with weights in the Boolean semiring.
For any semiring , a -automaton  is
\begin{itemize}
\item \emph{partial} if there is at most one nonzero entry in each row of each transition matrix, and  has exactly one nonzero entry,
\item \emph{deterministic} if it is partial and there is exactly one nonzero entry in each row of each matrix .
\end{itemize}
A classical deterministic automaton  is called
\emph{synchronizable} (directable, resetable etc) if there exists a word  (called a synchronizing word of ) such that 
has exactly one column that is filled with 's and all the other entries of  are zero.
(Traditionally, this property is formalized as  inducing a constant map on the state set.)

As an example, the -state automaton  with arbitrary  and  and with transition matrices

is synchronizable since for the word , the transition matrix is


The notion celebrates its 50th anniversary this year -- a very popular and intensively studied conjecture in the area is that of
\v{C}ern\'y stating if an -state classical deterministic automaton is synchronizable, then it admits a synchronizing word of length at most .
We remark here that it is decidable in polynomial time (it's actually in )
whether an input classical, deterministic automaton is synchronizable.

Synchronizability has been extended to nondeterminisic automata in~\cite{imreh} in three different ways.
Here we highlight the one entitled ``D3-directability'' there: a -automaton 
(that is, a classical nondeterministic automaton)
is called D3-directable if there exists a word  such that  has exactly one column that is filled with 's and
all the other entries of  are zero.
It is known (see e.g.~\cite{imreh}) that in general, the shortest synchronizing word of a synchronizable -state -automaton
can have length  with  being an upper bound~\cite{gazdag}.
For partial -automata, the best known bounds are  and , see~\cite{martyugin-bound,gazdag}.

In the next section of the paper we will frequently use the following results of~\cite{martyugin-pspace}:
\begin{theorem}
\label{thm-marty}
Deciding whether an input -automaton is synchronizable is complete for .
The problem remains -complete when restricted to partial -automata.
\end{theorem}

For the probabilistic semiring, in which case the weight structure is that of the nonnegative reals with the standard addition and product,
and the input automata's transition matrices are restricted to be stochastic, the notion has been also generalized by several authors:
\begin{itemize}
\item In~\cite{kfouri},  is synchronizable if there exists a word  such that all the rows of  are identical.
\item In~\cite{doyen},  is synchronizable if there exists a single infinite word  such that for any ,
  there exists an integer  such that for each finite prefix  of  having length at least ,
  in  there is a column in which each entry is at least .
\end{itemize}
The problem of checking synchronizability is undecidable in the former setting and -complete in the latter setting.

Most of these generalizations require (an arbitrary precise approximation of) a column consisting of ones and
zeros everywhere else in some matrix of the form . In fact, under these conditions it is a simple consequence of the structure of
the semiring and the constraint on the automata that if in a row of a transition matrix  there is exactly one nonzero element,
then it has to be . (The Boolean semiring has only two elements, while in the probability semiring the stochasticity of the matrices
guarantee that the row sum is preserved and is one.)

The authors of~\cite{larsen} worked in the semiring , with a different semantics notion, though:
according to the notions of the present paper they worked in the semiring , where the elements are finite sets of integers,
with union as addition and complex sum  being product.
There two different notions of synchronizability are introduced: a matrix  is \emph{location synchronizing} if there exists
a column in which each entry is nonzero, while all the other entries of the matrix are zeroes (recall that in this semiring
 plays as zero)
and is \emph{synchronizing} if additionally the nonzero entries all coincide and map every possible starting vector 
to some fixed vector (which is simply not possible in this semiring since this would require the presence of an -trivial element of the semiring).
An automaton  is location synchronizable if there exists a word  such that
 is location synchronizing.
Regarding the complexity issues, location synchronizability is -complete (which is due to the fact
that  is positive, cf. Proposition~\ref{prop-zsf-zdf})
and synchronizability is trivially false.

In this paper we extend the notion of synchronizability in spirit similar to~\cite{larsen}, covering most of the generalizations above
(the exception being the case of the probabilistic semiring, which seems to require a notion of metric).
\begin{definition}
Given a semiring  and a matrix , we say that  is
\begin{itemize}
\item \emph{location synchronizing} if there exists a (unique) integer  such that  iff ;
\item \emph{synchronizing} if it is location synchronizing and additionally,  for each  for the above index .
\end{itemize}
A finite set  of matrices is \emph{(location) synchronizable} if they generate a (location) synchronizing matrix,
i.e. when  is (location) synchronizing for some , .

A -automaton is (location) synchronizable if so is its set of transition matrices.

We formulate the -(location) synchronizing problem
( and  for short)
as follows: given a finite set  of matrices
in  for some , decide whether  is (location) synchronizable?

(Clearly, this is equivalent to having a single -automaton as input.)
\end{definition}

\section{Results on complexity of the two problems}

Given a semiring , call a matrix  a partial -matrix if in each row there is at most one nonzero entry,
which can have only a value of  if present, formally for each  there exists at most one  with  in which
case  has to hold. Observe that the product of two partial -matrices is still a partial -matrix, being the
same in any semiring. Moreover, a partial -matrix is synchronizing iff it is location synchronizing.
Thus the following are equivalent for any set  of partial -matrices:
\begin{enumerate}
\item  is synchronizable;
\item  is location synchronizable;
\item , viewed as a set of partial -matrices over , is synchronizable.
\end{enumerate}
Since by Theorem~\ref{thm-marty} the last condition is -hard to check, we immediately get the following:
\begin{proposition}
For any nontrivial semiring , both  and  are -hard.
\end{proposition}

\subsection{Decidable subcases}

First we make several (rather straightforward) observations on decidable subcases, generally involving finiteness conditions.

Of course if  is finite, we get -completeness: 
\begin{proposition}
For any finite semiring  both problems are in , thus are -complete.
\end{proposition}
\begin{proof}
Given an instance  of the problem,
we store a current matrix  initialized by the unit matrix  of .
In an endless loop, we nondeterministically choose an index  and let .
After each step we check whether  is (location) synchronizing. If so, we report acceptance, otherwise
continue the iteration.

If  is finite, storing an entry of  takes constant space, so storing  takes  memory,
as well as computation of the product matrix. In total, we have an  algorithm which
is  by Savitch's theorem~\cite{papadimitriou}.
\end{proof}

\begin{proposition}
For any locally finite semiring , both  and
 are decidable, provided that addition and product of  are computable.
\end{proposition}
\begin{proof}
Recall that a semiring  is locally finite if any finite subset of  generates a finite subsemiring of .

Now given an instance  of the problem,
let  stand for the finite set of the entries occurring in any of the matrices.
Then clearly,  where  is the subsemiring of 
generated by .
Since  is finitely generated, this
implies  is finite as well, hence there exists an integer  such that  which can be chosen to be the least integer 
 with . Hence by computing the sets  for 
and reporting acceptance when a witness is found and rejecting the input when 
gets satisfied without finding a witness we decide the respective problem.

(Note that computability of addition and product is needed for the effective computation of the sets above.)
\end{proof}

\begin{proposition}
\label{prop-zsf-zdf}
For any positive semiring ,  is in .
\end{proposition}
\begin{proof}
For any positive semiring  the mapping  which maps  to  and all other elements of  to ,
is a semiring morphism. Hence  can be extended pointwise to a semiring morphism ,
with . Then, a matrix  is \emph{location} synchronizing if and only if
 is (location) synchronizing. Hence  can be reduced to 
via the polytime reduction , which is solvable in ,
hence so is .
\end{proof}
\begin{remark}
One can use the above semiring morphism to decide any such property of matrices which cares only on the positions of zeroes
(i.e. when  satisfies the property if and only if so does ).
Examples of such properties are \emph{mortality} (whether the all-zero matrix is generated),
and the \emph{zero-in-the-upper-left-corner} (whether a matrix with a zero in the upper-left corner is generated).
Thus both properties are in  for positive semirings
(and are in fact undecidable for the semiring , which is \emph{not} zero-sum-free).

Synchronizability, on the other hand, as well as the ``equal entries problem'' asking whether a matrix is generated having
the same entry at two specified positions, is not such a property. The latter is well-known to be undecidable in 
while the former is shown to be undecidable in Theorem~\ref{thm-n-sync}.
\end{remark}

\subsection{Undecidable subcases}
Now we turn our attention to undecidability results.

A well-known undecidable problem is the \emph{Fixed Post Correspondence Problem}, or FPCP for short: given a finite set
 of pairs of nonempty words over a binary alphabet,
does there exist a nonempty index sequence , each  in ,  with  (i.e. we fix the
\emph{last} used tile) such that
? The problem is already undecidable for the fixed
constant  (also, it's known to be decidable for , see~\cite{halava} and has an unknown decidability status
for ).

\begin{proposition}
For any semiring  such that the semigroup  embeds into the multiplicative monoid  of ,
the  problem is undecidable, even for two-state deterministic WFA with an alphabet size of
 (i.e. for eight  matrices when the question is viewed as a problem for matrices).
\end{proposition}
\begin{proof}
In order to ease notation, suppose  is a subsemigroup of .
For words , let us define the matrices 
 and .
Then a direct computation shows that

Also, matrices  are not synchronizing while matrices  are synchronizing iff .
Moreover, a product  is synchronizing for  iff .
Thus we can derive that a product of the form   with each  being either  or 
and  is synchronizing iff there exists some  such that ,  for each  and
 holds.

Hence, a reduction from FPCP to  is given by the transformation

Since FPCP is undecidable, so is .
\end{proof}

Note that  is positive, so its location synchronization problem is decidable in polynomial space, while when , its synchronization problem becomes undecidable.

Now we give a polynomial-time reduction from the -mortality problem to both of the -synchronization and the -location synchronization problem. The -mortality problem is actively studied for the case :
\begin{definition}
For a fixed semiring , the -mortality problem is the following: given a finite set 
of matrices in  for some , does  contain the null matrix ?
\end{definition}

\begin{proposition}
For any semiring , the -mortality problem reduces to both of 
and .
Thus, in particular, when -mortality problem is undecidable, so are both synchronizability problems.
\end{proposition}
\begin{proof}
Let  be an instance of the -mortality problem.
We define the matrices , i.e. adding an all-zero top row and an all-zero
first row to each ,  and fill the upper-left corner by . Also, we define . We claim that the following are equivalent:
\begin{enumerate}
\item ;
\item  is synchronizable;
\item  is location synchronizable.
\end{enumerate}
Observe that each member of  is block-lower triangular with  in the upper left corner, hence for any product
 we have  for some column vector .
Note that in order to ease notation we define  as the unit matrix  and set  -- since  is not
synchronizing and is the unit element of , this neither affects mortality (of ) nor synchronizability
(of ).

Thus in particular the first column of any matrix  contains a nonzero entry, hence  is (location) synchronizing
only if , in which case  is indeed a positive instance of the -mortality problem,
showing iii) i). For i)ii), let , , . Then , thus  is a synchronizing matrix. Finally, ii)iii) is clear for any .
\end{proof}
In particular, since mortality is undecidable in , so are and .

Our most involved result on undecidability is the following one:
\begin{theorem}
\label{thm-n-sync}
 is undecidable.
Thus if  embeds into  (i.e. when  has infinite order in ), then so is .
\end{theorem}
\begin{proof}
We give a polynomial-time reduction from the FPCP problem to .
This time we use the variant of FPCP in which the \emph{first} tile is fixed to .
Let  be an instance of the FPCP, .
For a nonempty word  let  be its value when considered as a \emph{ternary} number, i.e.
.
Also, we define for each word  a matrix .
Then, since , we get that  and since the mapping 
is also injective, it is an embedding of the semigroup  into .

We define the following matrices , ,  and , all in :

that is,  has exactly two nonzero entries, namely .

Then for any sequence ,  we have

with  and  and also

and thus

which is synchronizing if and only if , hence if  is a positive
instance of FPCP, then  is synchronizable.

For the other direction, suppose  is synchronizable. We already argued that any member  of  has the form  for words  with  and
 for some , . These matrices are clearly not (location) synchronizing.

Considering the matrix , we have the following claims:

{\sl Claim A.} For any matrix  we have 0 for some .

{\sl Claim B.} If  is synchronizing for some matrices  and , then so is .

Indeed,  is the matrix whose first column is the sum of the third and the fifth column of , and whose other entries are all zero.
Also, if 0 then
 where  is the first row of .
If  is synchronizing, this implies  for each , hence  and  is synchronizing as well.

Thus, by ii) above we get that if  is synchronizable, then there is a synchronizing matrix of the form  with .

Inspecting members of  we get the following claim:

{\sl Claim C.} Let  stand for the matrix semigroup . Then for any , any member of
 has the form  for some
matrices .

Indeed, for the base case  we have matrices of the form 
satisfying the condition. Suppose the claim holds for  and consider a matrix . By the induction hypothesis,  with ,
and  for some  and words .
We can also write  for 
and  for .
Calculating the product we get

showing the claim.

Thus, since , we get by Claim B that
if  is synchronizable, then there is a synchronizing matrix of the form .
Writing  and
 we get that this product is further equal to
0 which is synchronizing if and only if
 and . By  we get that if  is synchronizable, then there is a synchronizing matrix of the form

with , , .
Writing , ,  and 
we can write
\mathrm{int}(u')+1)\cdot\mathrm{int}(u_1u)\\
  3^{|u'|}\mathrm{int}(u_1u)\\mathrm{int}(v')+1)\cdot\mathrm{int}(v_1v)\\
  \end{array}\textrm{\Huge{\quad\quad}}\right)

which is synchronizing only if  and , that is, 
implying .

Hence if  is synchronizable then there exists a synchronizing product of the form
, which in turn implies , thus in that case
 is indeed a positive instance of the FPCP problem.
\end{proof}

We note that the idea of encoding of a PCP variant within matrix semirings is not new, see e.g.~\cite{halavahirvensalo,potapov,gaubert}.
For example, -mortality can be shown to be undecidable for  integral matrices
via a similar embedding  as in 
the proof of Theorem~\ref{thm-n-sync}, with  being the base-4 value of . This mapping is also an injective monoid
homomorphism. Then, defining  which satisfies
 and  we get a similar construction (cf.~\cite{Halava97decidableand}),
also suitable for showing the undecidability of the zero-in-the-upper-left-corner problem.
However, the lack of substraction (in general, zero-sum-freeness of ) prevents us to apply this method.
Also, defining matrices of the form  for a suitable  (as in~\cite{Halava01mortalityin}, see also~\cite{paterson}) is again out of question
since in , only permutation matrices are invertible.
The most closest approach is that of the equal entries problem: in the proof we also showed undecidability of the
problem whether  generates a matrix having equal entries in the top-left corner and in entry .
Actually, the embedding  shows the same for  matrices.
However, we were unable to modify the construction for  matrices to \emph{shift} the values  and
 into, say, the first column and at the same time, \emph{overwrite} the values  and 
by  and , respectively. (Adding them or something similar did not seem to work, either.)
That's why we had to use  matrices -- it is quite plausible that the encoding is not the most compact possible
and the dimension can be further lowered.


\section{Conclusion, future directions}
We generalized the notion of synchronizability to automata with transitions weighted in an arbitrary semiring in two ways:
one of them, location synchronizability requires the existence of a word  and a state  such that starting from any state ,
 and only  has a nonzero weight after  is being read; synchronizability additionally requires that this nonzero weight
is the same for all states . In this paper we studied the \emph{complexity} of checking these properties, parametrized by the underlying
semiring.

Our results can be summarised as follows:
\begin{itemize}
\item Both problems are -hard for any nontrivial semiring.
\item For finite semirings, they are -complete.
\item For positive semirings, location synchronizability is -complete.
\item For locally finite semirings they are decidable (provided that the addition and product operations of the semiring are computable).
\item The mortality problem reduces to both problems in any semiring. Thus for semirings having an undecidable mortality problem,
  both variants of synchronization are undecidable. (This is the case for .)
\item If  embeds into the multiplicative structure of , then synchronizability is undecidable
  for , even for deterministic automata.
\item Synchronizability is undecidable for any semiring where  has infinite order in the additive semigroup. (This is the case for .
  Note that for , location synchronizability is in .)
\end{itemize}
We do not have any decidability results for -synchronizability when the semiring  is not locally finite, the element  has a finite
order in the additive structure, and  does not embed into the multiplicative semigroup. Also, it is not clear whether synchronizability
can be reduced to location synchronizability in general -- since in , location synchronizability is decidable but synchronizability is
undecidable, so in general, synchronizability cannot be Turing-reduced to location synchronizability. It is also an interesting question whether
-synchronizability of -state automata is decidable or not -- we conjecture that it is still undecidable and one can use a slightly more compact
encoding of FPCP. Also, to cover the existing generalizations of synchronizability for the case of the probabilistic semiring,
we could study semirings that are equipped with a metric -- our current investigations can be seen as the case of this perspective
where the metric is the dicrete unit-distance metric.

\subsubsection*{Acknowledgement}
The research was supported by the European Union and the State of Hungary,
co-financed by the European Social Fund in the framework of T\'AMOP-4.2.4.A/ 2-11/1-2012-0001 ``National Excellence Program''.

The author is thankful to the anonymous referees for their suggestions and detailed comments.

\bibliographystyle{eptcs}
\bibliography{biblio-afl}
\end{document}
