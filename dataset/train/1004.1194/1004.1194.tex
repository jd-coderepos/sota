

In the previous section, we have
seen how to construct an $x$-partition of the
dynamic programming grid $G$. Once this partition has been
built, the first step of the block edit-distance procedure is
to construct a repository of $DIST$ tables corresponding to
each block of the partition. In this section we discuss how to
construct this repository of $DIST$ tables efficiently.

We will be building the $DIST$ tables by utilizing the process
of merging two $DIST$ tables. That is, if $D_1$ and $D_2$ are
two $DIST$ tables, one between a pair of substrings $A'$ and
$B'$ and the other between $A'$ and $B''$, then we refer to the
$DIST$ table between $A'$ and $B'B''$ as the product of
\emph{merging} $D_1$ and $D_2$. For an arbitrary scoring
function, merging two $x\times x$ $DIST$ tables requires $O(x^2)$
time using $x$ iterations of the SMAWK algorithm discussed in
Section~\ref{Section: Block Edit-Distance via SLPs}. For the
rational scoring function, a recent important result of
Tiskin~\cite{Tiskin2009} shows how using the succinct
representation of $DIST$ tables, two $DIST$ tables can be
merged in $O(x \lg x)$ time.

Recall that the $x$-partitioning step described in
Section~\ref{Section: Constructing the x-partition}
associates SLP variables with substrings in $A$ and $B$. Each
substring which is associated with a variable is of length
$O(x)$. There are two types of associated substrings: Those
whose associated SLP variable derives them exactly, and those
whose associated variable derives a superstring of them. As a first step, we construct the
$DIST$ tables between any pair of substrings $A'$ and $B'$ in
$A$ and $B$ that are associated with a pair of SLP variables,
and are of the first type.
This is done in a recursive manner. We first construct
the  four $DIST$ tables that correspond to
the pairs $(A'_1,B'_1)$, $(A'_1,B'_2)$, $(A'_2,B'_1)$, and
$(A'_2,B'_1)$, where $A'_1$ and $A'_2$ (resp. $B'_1$ and $B'_2$) are the strings derived by the left and right
children of $A'$'s (resp. $B'$'s) associated variable.
We then merge these four $DIST$s together
to obtain a $DIST$ between $A'$ and $B'$.


Now assume $A'$ is a substring in $A$ of the second type, and
we want to construct the $DIST$ between $A'$ and a substring
$B'$ in $B$. For simplicity assume that $B'$ is a substring of
the first type. Then, the variable associated with $A'$ is a
variable of the form $v^j_i$ on a path between some two key
vertices $v_i$ and $v_{i+1}$.
For each $k \leq j$,
let $A^k_i$ denote the substring associated with  $v^k_i$. Notice that $A'$ is the concatenation of $A^{j-1}_i$ and the substring $A''$ that is derived from $v^j_i$'s right child.
The $DIST$
between $A'$ and $B'$ is thus constructed by first recursively
constructing the $DIST$ between $A^{j-1}_i$ and $B'$,
and then merging this with the $DIST$ between $A''$ and $B'$.
The latter already being available since the variables associated with $A''$ and $B'$ are of the first type.









\begin{lemma}
\label{Lemma: dist repository rational}Building the $DIST$ repository under a rational scoring function
can be done in $O(n^2x\lg x)$ time.
\end{lemma}
\begin{proof}
Recall that for rational scoring functions, merging two
succinct $DIST$s of size $O(x)$ each requires $O(x \lg x)$
time~\cite{Tiskin2009}. We perform exactly one merge per each
distinct pair of substrings in $A$ and $B$ induced by the
$x$-partition. Since each substring is associated with a unique
SLP variable, there can only be $O(n^2)$ distinct
substring pairs. Thus, we get that only $O(n^2)$ merges of
succinct $DIST$s need to be performed to create our
repository, hence we achieve the required bound. \qed
\end{proof}

\begin{lemma}
\label{Lemma: dist repository arbitrary}Building the $DIST$ repository under an arbitrary scoring
function can be done in $O(n^2x^2)$ time.
\end{lemma}
\begin{proof}
In the case of arbitrary scoring functions, merging two
$x\times x$ $DIST$ tables requires $O(x^2)$ time. The same
upper bound shown above of $O(n^2)$ merges may be performed in
this case as well, and therefore we achieve the required bound.
\qed
\end{proof}
