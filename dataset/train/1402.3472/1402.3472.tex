\providecommand{\jump}[1]{\mathtt{jump}(#1)}
\providecommand{\jumpset}[1]{X_{#1}}
\providecommand{\secp}[1]{A_{#1}}
\providecommand{\jumpfam}{\mathcal{J}}
\providecommand{\chainfam}{\mathcal{C}}

In this section we conclude the proof of Theorem~\ref{thm:spic}
by showing the following.
\begin{theorem}\label{thm:dp}
Given a \spic{} instance $\mathcal{I} = (G,k,(\pos_u)_{u \in V(G)},\Gdown,\Gup)$ with $n = |V(G)|$,
 a threshold $\tau$ and a family $\secfam \subseteq 2^{V(G)}$,
one can in $n^{\Oh(k/\tau+\tau)} |\secfam|^{\Oh(\tau)}$ time find the canonical ordering $\ord$ of $\mathcal{I}$, assuming that
\begin{enumerate}
\item $\cost(\ord) \leq k$;
\item for each $u \in V(G)$, $|\incsol{\sol^\ord}{u}| \leq \tau$;
\item each section consistent with $\ord$ belongs to $\secfam$.
\end{enumerate}
\end{theorem}
Observe that if we apply Theorem~\ref{thm:dp} to a branch with a \spic{} instance $\mathcal{I}$,
the threshold $\tau$ and family $\secfam$ output by Theorem~\ref{thm:sec-enum},
then we obtain the algorithm promised by Theorem~\ref{thm:spic}.

The algorithm of Theorem~\ref{thm:dp} is a dynamic programming algorithm.
Henceforth assume that the instance $\mathcal{I}$ with threshold $\tau$ and family $\secfam$
is as promised in the statement of Theorem~\ref{thm:dp}, and let $\ord$ be the canonical ordering of $\mathcal{I}$.
We develop two different ways of separating the graphs $G$ and $G^\ord$ into smaller parts,
suitable for dynamic programming. Consequently, the dynamic programming algorithm
has in some sense `two layers', and two different types of states.

\subsection{Layer one: jumps and jump sets}

We first develop a way to split the graphs $G$ and $G^\ord$ `vertically'.
To this end, first denote for any position $p$
the section $\secp{p} = \{v \in V(G): \ord(v) < p\}$; note that this definition also makes sense
for $p = \infty$ and $\secp{\infty} = V(G)$.
Second, for any position $p$ define
$$\jump{p} = \min\{q: q > p \wedge \ord^{-1}(p)\ord^{-1}(q) \notin E(G^\ord)\};$$
in this definition we follow the convention that the minimum of an empty set is $\infty$. Moreover, we define a \emph{jump set} for position $p$ as
$$\jumpset{p} = \ord^{-1}([p,\jump{p}-1]) = \secp{\jump{p}} \setminus \secp{p}.$$
See also Fig.~\ref{fig:jump} for an illustration.

\begin{figure}
\centering
\includegraphics{fig-jump}
\caption{A jump at position $p$ and the corresponding jump set.
  The jump set $\jumpset{p}$, denoted with gray, is a clique in $G^\ord$, and no edge of $G^\ord$
    connects $\secp{p}$ with $V(G) \setminus \secp{\jump{p}}$.}
\label{fig:jump}
\end{figure}

The next two lemmata follow directly from the definition of a jump and the properties
of  umbrella orderings.
\begin{lemma}\label{lem:jump-ineq}
For any positions $p$ and $q$, if $p \leq q$ then $\jump{p} \leq \jump{q}$.
\end{lemma}
\begin{lemma}\label{lem:jump-cut} Jump set 
$\jumpset{p}$ is  a clique in $G^\ord$, but no edge of $G^\ord$
connects a vertex of $\secp{p}$ with a vertex of $V(G) \setminus \secp{\jump{p}}$.
\end{lemma}

We now slightly augment the graph $G$ so that $\jump{p} \neq \infty$ for all interesting
positions; see also Figure~\ref{fig:augment}.
We take $\Oh(n^2)$ branches, guessing the first
and the last vertex of $G$ in the ordering $\ord$; denote them by $\alpha$ and $\omega$.
We introduce  new vertices, $\alpha_1,\alpha_2,\omega_1,\omega_2,\omega_3$ 
and new edges $\alpha_1\alpha_2,\alpha\alpha_1,\omega_2\omega_3,\omega_1\omega_2,\omega\omega_1$ in $G$.
We also introduce new positions $-1,0,n+1,n+2,n+3$, isolated in $\Gdown$ and connected
by edges $\{-1,0\},\{0,1\},\{n,n+1\},\{n+1,n+2\},\{n+2,n+3\}$ in $\Gup$.
We define $\pos_{\alpha_1} = \{0\}$, $\pos_{\alpha_2} = \{-1\}$, $\pos_{\omega_1} = \{n+1\}$,
   $\pos_{\omega_2} = \{n+2\}$ and $\pos_{\omega_3} = \{n+3\}$.
Moreover, we put $\alpha_2$ and $\alpha_1$ before all vertices of $G$ in the ordering $\ord_0$,
and $\omega_1$, $\omega_2$ and $\omega_3$ after them.
Note that, if we precede all the vertices in the ordering $\ord$ with
$\alpha_2,\alpha_1$ and succeed with $\omega_1,\omega_2,\omega_3$ we obtain an ordering with no
higher cost. Due to the way we have extended $\ord_0$ to the new vertices, the
extended ordering $\ord$ defined in this way is the canonical ordering
of the extended graph $G$. Hence, we may abuse the notation and denote by $G$
the graph after the addition of these five new vertices, and assume that $V(\Gdown) = V(\Gup) = \{1,2,\ldots,|V(G)|\}$ again.

\begin{figure}
\centering
\includegraphics{fig-augment}
\caption{Augmentation of the input graph $G$.}
\label{fig:augment}
\end{figure}


Observe now that $\jump{1} = 3$ and $\jumpset{1} = \{\alpha_2,\alpha_1\}$,
 as $\ord^{-1}(1) = \alpha_2$ and $\ord^{-1}(3) = \alpha$.
Moreover, $\jump{n-2} = n$ and $\jumpset{n-2} = \{\omega_1,\omega_2\}$,
as $\ord^{-1}(n-2) = \omega_1$, $\ord^{-1}(n-1) = \omega_2$, and $\ord^{-1}(n) = \omega_3$

The main observation now is that a jump set, together with all edges of $\sol^\ord$
incident with it (i.e., $\incsol{\sol^\ord}{\jumpset{p}}$) contains
all sufficient information to divide the problem into parts before and after
a jump set.
\begin{lemma}\label{lem:jump-equiv}
For any position $p$, the following holds.
\begin{enumerate}
\item
For any $u_1,u_2 \in \jumpset{p}$ such that $\ord(u_1) \leq \ord(u_2)$ we have
\begin{align}
N_{G^\ord}(u_1) \cap \secp{p} &\supseteq N_{G^\ord}(u_2) \cap \secp{p}, \label{eq:jump-left} \\
N_{G^\ord}(u_1) \setminus \secp{\jump{p}} &\subseteq N_{G^\ord}(u_2) \setminus \secp{\jump{p}}. \label{eq:jump-right}
\end{align}
\item For any bijection $\ord_p: \jumpset{p} \to [p,\jump{p}-1]$
such that $\ord_p(u) \in \pos_u$ for any $u \in \jumpset{p}$ and 
both inclusions~\eqref{eq:jump-left} and~\eqref{eq:jump-right} hold
for any $u_1,u_2 \in \jumpset{p}$ with $\ord_p(u_1)  \leq \ord_p(u_2)$,
if we define an ordering $\ord'$ of $V(G)$ as
$\ord'(u) = \ord_p(u)$ if $u \in \jumpset{p}$ and $\ord'(u) = \ord(u)$ otherwise,
then $\ord'$ is feasible and $\ord'(G^{\ord'})$ is a subgraph of $\ord(G^\ord)$. 
\end{enumerate}
\end{lemma}
\begin{proof}
The first statement is straightforward from the properties of an umbrella ordering.
Let $\ord_p$ and $\ord'$ be as in the second statement.
Observe that inclusions~\eqref{eq:jump-left} and~\eqref{eq:jump-right}, together with the fact that $X_p$ is a clique in $\ord(G^\ord)$,
imply that $\ord'$ and $\ord$ differ only on the internal order of twin classes of $G^\ord$ and consequently $\ord'(G^\ord) = \ord(G^\ord)$.
Together with the fact that $\ord'(u)\in \pos_u$ for any $u\in V(G)$, this means that $\ord'$ is a feasible ordering of $G$ and an umbrella ordering of $G^\ord$.
Consequently $\sol^{\ord'} \subseteq \sol^\ord$, $\ord'(G^{\ord'})$ is a subgraph of $\ord'(G^\ord) = \ord(G^\ord)$, and the lemma is proven.
\end{proof}

We use Lemma~\ref{lem:jump-equiv} to fit the task of computing $\ord|_{\jumpset{p}}$ into Lemma~\ref{lem:lex-match}.
\begin{lemma}\label{lem:jump-detect}
Given a position $p$ and the sets $\jumpset{p}$, $\secp{p}$ and
$\incsol{\sol^\ord}{\jumpset{p}}$, one can
in polynomial time compute the ordering $\ord|_{\jumpset{p}}$.
\end{lemma}
\begin{proof}
First, observe that the data promised in the lemma statement allows
us to compute $N_{G^\ord}(u) \cap \secp{p}$ and $N_{G^\ord}(u) \setminus \secp{\jump{p}}$
for every $u \in \jumpset{p}$.
Define a binary relation $\preceq$ on $\jumpset{p}$ as $u_1 \preceq u_2$ if and only if
both~\eqref{eq:jump-left} and~\eqref{eq:jump-right} hold for $u_1$ and $u_2$.
Lemma~\ref{lem:jump-equiv} asserts that $\preceq$ is a total quasi-order on $\jumpset{p}$.
That is, the set $\jumpset{p}$ can by partitioned into sets $U_1,U_2,\ldots,U_s$
such that $u_1 \preceq u_2$ and $u_2 \preceq u_1$ for any $1 \leq j \leq s$ and
$u_1,u_2 \in U_j$, and $u_1 \preceq u_2$, $u_2 \not\preceq u_1$ for any $1 \leq j_1 < j_2 \leq s$
and $u_1 \in U_{j_1}$, $u_2 \in U_{j_2}$.
(Formally, we terminate the current branch if $\preceq$ does not satisfy these properties.)

Observe that $\ord|_{\jumpset{p}}$ maps $\jumpset{p}$ onto $[p,\jump{p}-1]$.
Lemma~\ref{lem:jump-equiv} asserts that all vertices of $U_1$
are placed by $\ord$ on the first $|U_1|$ positions of the range of $\ord|_{\jumpset{p}}$,
all vertices of $U_2$ are placed on the next $|U_2|$ positions etc.
We use Lemma~\ref{lem:lex-match} to find a lexicographically minimum ordering $\ord_p$ that satisfies the above
and additionally $\ord_p(u) \in \pos_u$ for each $u \in \jumpset{p}$.
Define $\ord'$ as in Lemma~\ref{lem:jump-equiv}.
By the minimality of $\ord$, we have $\cost(\ord') \geq \cost(\ord)$, but Lemma~\ref{lem:jump-equiv}
asserts that $\ord'(G^{\ord'})$ is a subgraph of $\ord(G^\ord)$.
Hence, $\ord'$ is of minimum possible cost.
By the lexicographical minimality of $\ord_p$, we have $\ord_p = \ord|_{\jumpset{p}}$
and the lemma is proven.
\end{proof}



With help of family $\secfam$, Lemma~\ref{lem:jump-detect} allows us to efficiently enumerate
jump sets with their surroundings.
\begin{theorem}\label{thm:jump-enum}
One can in $n^{\Oh(k/\tau)}|\secfam|^2$ time enumerate a family $\jumpfam$
of at most $n^{\Oh(k/\tau)}|\secfam|^2$ tuples $(A,X,\ord_X)$ such that:
\begin{enumerate}
\item in each tuple $(A,X,\ord_X)$ we have
\begin{enumerate}
\item $A,X \subseteq V(G)$ and $A \cap X = \emptyset$,
\item $\Gdown[[|A|+1,|A|+|X|]]$ is a complete graph,
\item $\ord_X$ is a bijection between $X$ and $[|A|+1,|A|+|X|]$;
\end{enumerate}
\item for any position $p$, if there are at most $2k/\tau$
edges of $\sol^\ord$ incident to $\jumpset{p}$, then
the tuple $J^\ord(p) := (\secp{p},\jumpset{p},\ord|_{\jumpset{p}})$ belongs to $\jumpfam$.
\end{enumerate}
\end{theorem}
\begin{proof}
We provide a procedure of guessing at most $n^{\Oh(k/\tau)}|\secfam|^2$ candidate tuples that will constitute the family $\jumpfam$. Since the promised properties of elements of $\jumpfam$ can be checked in polynomial time, it suffices to argue that every triple of the form $(\secp{p},\jumpset{p},\ord|_{\jumpset{p}})$ will be among the guessed candidates.

The number of choices for $\secp{p}$ and $\secp{\jump{p}}$ is $|\secfam|^2$.
Observe that then $\jumpset{p} = \secp{\jump{p}} \setminus \secp{p}$.
Furthermore, there are $n^{\Oh(k/\tau)}$ ways to choose $\incsol{\sol^\ord}{\jumpset{p}}$
and, by Lemma~\ref{lem:jump-detect}, we can further deduce $\ord|_{\jumpset{p}}$. Finally,  observe that  by the definition of a jump it follows that every triple $(\secp{p},\jumpset{p},\ord|_{\jumpset{p}})$ satisfies the promised properties of the elements of $\jumpfam$.
\end{proof}

We are now ready to describe the first layer of our dynamic programming algorithm.
\begin{definition}[layer-one state]
A \emph{layer-one state} is a pair $(J^1,J^2)$ of two elements of $\jumpfam$,
$J^1 = (A^1,X^1,\ord_X^1)$, $J^2 = (A^2,X^2,\ord_X^2)$ such that 
$A^1 \subseteq A^2$ and $(A^1 \cup X^1) \subseteq (A^2 \cup X^2)$.
The \emph{value} of a layer-one state $(J^1,J^2)$ is a bijection
$f[J^1,J^2] : (A^2 \cup X^2) \setminus A^1 \to [|A^1|+1,|A^2\cup X^2|]$ satisfying the following:
\begin{enumerate}
\item $f[J^1,J^2]$ is a feasible ordering of its domain, that is,
  for any $u \in (A^2 \cup X^2) \setminus A^1$ we have $f[J^1,J^2](u) \in \pos_u$ and
  for any $u_1,u_2 \in (A^2 \cup X^2) \setminus A^1$ such that $u_1u_2\in E(G)$, we have
  $f[J^1,J^2](u_1)f[J^1,J^2](u_2) \in E(\Gup)$;
\item $f[J^1,J^2](u) = \ord_X^1(u)$ for any $u \in X^1$ and
$f[J^1,J^2](u) = \ord_X^2(u)$ for any $u \in X^2$;
  \label{p:layer-one-last}
\item among all functions $f$ satisfying the previous conditions,\label{p:layer-one-min}
  $f[J^1,J^2]$ minimizes the cardinality of $\sol^f$ 
(where in the expression $\sol^f$ the function $f$ is treated as an ordering of the set $(A^2 \cup X^2) \setminus A^1$
 in the \spic{} instance $(G,k,(\pos_u)_{u \in V(G)},\Gdown,\Gup)$);
\item among all functions $f$ satisfying the previous conditions, 
$f[J^1,J^2]$ is lexicographically minimum.
\end{enumerate}
\end{definition}

We first observe the following consequence of the above definition.
\begin{lemma}\label{lem:layer-one-1}
For any $p^1 \leq p^2$ such that $J^\ord(p^1),J^\ord(p^2) \in \jumpfam$, we have that $(J^\ord(p^1),J^\ord(p^2))$ is a layer-one state and
$$f[J^\ord(p^1),J^\ord(p^2)] = \ord|_{\secp{\jump{p^2}} \setminus \secp{p^1}}.$$
In particular, $\ord=f[J^\ord(1),J^\ord(n-2)]\cup \{(\omega_3,n)\}$.
\end{lemma}
\begin{proof}
Let $M:=\secp{\jump{p^2}} \setminus \secp{p^1}$. It is straightforward to verify that $(J^\ord(p^1),J^\ord(p^2))$ is a layer-one state
and $\ord|_M$ satisfies the first \ref{p:layer-one-last} properties of the value of a layer-one state.
Also, no edges of $\sol^\ord$ are incident to $X_1$ nor to $X_{n-2}$, and hence $J^\ord(1),J^\ord(n-2) \in \jumpfam$ and
$(J^\ord(1),J^\ord(n-2))$ is a layer-one state.

Let $f$ be any function satisfying the first \ref{p:layer-one-min} conditions of the definition
of a value of the layer-one state $(J^\ord(p^1),J^\ord(p^2))$.
Let $\ord'$ be an ordering of $V(G)$ defined as $\ord'(u) = f(u)$ if $u$ is the domain of $f$, and $\ord'(u) = \ord(u)$ otherwise.
It is straightforward to verify that $\ord'$ is feasible, using the separation property provided by Lemma~\ref{lem:jump-cut} and the fact that $\ord'|_{X^1\cup X^2}=\ord|_{X^1\cup X^2}$.
For the same reasons, by the definition of $\ord'$ we have that $\sol^{\ord'}=\left(\sol^{\ord}\setminus \binom{M}{2}\right)\cup \sol^f$. By the optimality of $f$ we have that $|\sol^f|\leq |\sol^{\ord|_{M}}|\leq |\sol^{\ord}\cap \binom{M}{2}|$, and so $|\sol^{\ord'}|\leq |\sol^\ord|$. By the optimality of $\ord$ we infer that $|\sol^{\ord'}|=|\sol^{\ord}|$, and $\sol^{\ord'}$ is also a minimum completion of $G$.
Since $\sol^\ord$ is also lexicographically minimum, it is easy to see that the last criterion of the definition of the value of the layer-one state $(J^\ord(p^1),J^\ord(p^2))$ indeed chooses $\ord|_{M}$.
\end{proof}
By Lemma~\ref{lem:layer-one-1}, our goal is to compute $f[J^\ord(1),J^\ord(n-2)]$ by dynamic programming. Observe
that both $J^\ord(1)$ and $J^\ord(n-2)$ are known, due to the augmentation performed at the beginning of this section.

Our dynamic programming algorithm computes value $g[J^1,J^2]$ for every
layer-one state $(J^1,J^2)$, and we will ensure that $g[J^\ord(p^1),J^\ord(p^2)] = f[J^\ord(p^1),J^\ord(p^2)]$ for any $p^1 \leq p^2$ with $J^\ord(p^1),J^\ord(p^2) \in \jumpfam$;
we will not necessarily guarantee that the values of $f$
and $g$ are equal for other states.
(Formally, $g[J^1,J^2]$ may also take value of $\bot$, which implies that either $J^1$ or $J^2$ is not consistent with $\ord$;
 we assign this value to $g[J^1,J^2]$ whenever we find no candidate for its value.)

Consider now one layer-one state $(J^1,J^2)$ with 
$J^1 = (A^1,X^1,\ord_X^1)$, $J^2 = (A^2,X^2,\ord_X^2)$.
The base case for computing $g[J^1,J^2]$ is the case where $A^2 \subseteq A^1 \cup X^1$.
Then $\ord_X^1 \cup \ord_X^2$ is the only candidate for the value $f[J^1,J^2]$, provided that $\ord_X^1$ and $\ord_X^2$ agree on the intersection of their domains.

In the other case, we iterate through all possible tuples
$J^3 = (A^3,X^3,\ord_X^3)$, with $A^1 \subset A^3 \subset A^2$
such that both $(J^1,J^3)$ and $(J^3,J^2)$ are layer-one states,
and try $g[J^1,J^3] \cup g[J^3,J^2]$ as a candidate value for $g[J^1,J^2]$.
That is, we temporarily
pick $g[J^1,J^2]$ with the same criteria as for $f[J^1,J^2]$, but taking into
account only values $g[J^1,J^3] \cup g[J^3,J^2]$ for different choices of $J^3$.

Since the minimization for $g[J^1,J^2]$ is taken over smaller set of functions
than for $f[J^1,J^2]$, we infer that
\begin{enumerate}
\item the cardinality of $\sol^{f[J^1,J^2]}$ is not larger than the cardinality of $\sol^{g[J^1,J^2]}$;
\item even if these two sets are of equal size, $f[J^1,J^2]$ is lexicographically
not larger than $g[J^1,J^2]$.
\end{enumerate}
However, observe that if $J^1 = J^\ord(p^1)$ and $J^2 = J^\ord(p^2)$
and there exists $p^3$ such that $p^1 < p^3 < p^2$ and $J^\ord(p^3) \in \jumpfam$,
    then $g[J^1,J^\ord(p^3)]\cup g[J^\ord(p^3),J^2]$ is taken into account when evaluating $g[J^1,J^2]$. If we compute the values for the states $(J^1,J^2)$ in the order of non-decreasing values of $|A^2\setminus A^1|$, then the values $g[J^1,J^\ord(p^3)], g[J^\ord(p^3),J^2]$ have been computed before, and moreover by induction hypothesis they are equal to $f[J^1,J^\ord(p^3)]$ and $f[J^\ord(p^3),J^2]$, respectively. Therefore,
$$f[J^1,J^\ord(p^3)] \cup f[J^\ord(p^3),J^2] = \ord|_{\secp{\jump{p^2}} \setminus \secp{p^1}}$$
is taken as a candidate value for $g[J^1,J^2]$ and, consequently, $g[J^1,J^2] = f[J^1,J^2] = \ord_{\secp{\jump{p^2}} \setminus \secp{p^1}}$.

Finally, we need to ensure that $g[J^1,J^2] = f[J^1,J^2]$ in the case
when such position $p^3$ does not exist. To this end, we take also more
candidate values for $g[J^1,J^2]$, computed by the layer-two dynamic programming
in the next section. We ensure that, if $J^1 = J^\ord(p^1)$,
$J^2 = J^\ord(p^2)$ but for any $p^1 < q < p^2$ we have $J^\ord(q) \notin \jumpfam$,
then the layer-two dynamic programming actually outputs $f[J^1,J^2]$ as one of the candidates,
     and runs in time $(n|\secfam|)^{\Oh(\tau)}$ for any choice of $J^1,J^2$. By Theorem~\ref{thm:jump-enum} there are at most $n^{\Oh(k/\tau)}|\secfam|^4$ layer-one states. Hence by using $(n|\secfam|)^{\Oh(\tau)}$ work for each of them will give the running time promised in Theorem~\ref{thm:dp}.

\subsection{Layer two: chains}

In this section we are given a layer-one state $(J^1,J^2)$ with $J^1 = (A^1,X^1,\ord_X^1)$, $J^2 = (A^2,X^2,\ord_X^2)$;
denote $p^\alpha = |A^\alpha| + 1$, $r^\alpha = |A^\alpha \cup X^\alpha|+1$ for $\alpha = 1,2$.
We are to compute, in time $(n|\secfam|)^{\Oh(\tau)}$, the value $f[J^1,J^2]$, assuming: $J^1 = J^\ord(p^1)$, $J^2 = J^\ord(p^2)$, and
for no position $p^1 < q < p^2$ it holds that $J^\ord(q) \in \jumpfam$. By Theorem~\ref{thm:jump-enum}, it implies that the number of edges of $\sol^\ord$
incident to any set $\jumpset{q}$ for $p^1 < q < p^2$ is more than $2k/\tau$. Observe that the following holds  $X^\alpha=\secp{\jump{p^\alpha}} \setminus \secp{p^\alpha}$, and hence $r^\alpha=\jump{p^\alpha}$ for $\alpha=1,2$. 

For any position $q$, consider the following sequence: $z_q(0) = q$ and $z_q(i+1) = \jump{z_q(i)}$ (with the convention that $\jump{\infty} = \infty$).
Observe the following.
\begin{lemma}\label{lem:short-z}
For any $q \geq p^1$ it holds that $z_q(\tau) \geq p^2$.
\end{lemma}
\begin{proof}
Consider any $q \geq p^1$. For any $i > 0$ such that $z_q(i) < p^2$ we have
that there are more than $2k/\tau$ edges of $\sol^\ord$ incident to $\jumpset{z_q(i)}$.
However, the sets $\jumpset{z_q(i)}$ are pairwise disjoint for different values of $i$.
Since $|\sol^\ord|\leq k$, we infer that for less than $\tau$ values $i > 0$ we may have $z_q(i) < p^2$, and the lemma is proven.
\end{proof}

By a straightforward induction from Lemma~\ref{lem:jump-ineq} we obtain the following.
\begin{lemma}\label{lem:z-interlace}
For any two positions $c,d$ with $c \leq d \leq \jump{c}$
and for any $i \geq 0$ it holds that
$$z_c(i) \leq z_d(i) \leq z_c(i+1).$$
\end{lemma}
The next observation gives us the crucial separation property for the layer-two dynamic programming (see also Figure~\ref{fig:CiDi}).
\begin{lemma}\label{lem:z-cut}
For any positions $c,d$ with $c \leq d \leq \jump{c}$
define
\begin{align*}
C_i &= \ord^{-1}([z_c(i),z_d(i)-1]),\\
D_i &= \ord^{-1}([z_d(i),z_c(i+1)-1]).
\end{align*}
Then
\begin{enumerate}
\item sets $C_i,D_i$ form a partition of $V(G)\setminus A_c$;
\item for any $i \geq 0$, it holds that both $C_i \cup D_i$ and $D_i \cup C_{i+1}$
are cliques in $G^\ord$;
\item for any $j > i \geq 0$ there is no edge in $G^\ord$ between $C_i$ and $D_j$;
\item for any $i > j+1 > 0$ there is no edge in $G^\ord$ between $C_i$ and $D_j$.
\end{enumerate}
\end{lemma}
\begin{proof}
All statements follow from the definitions $z_c(i+1) = \jump{z_c(i)}$
and $z_d(i+1) = \jump{z_d(i)}$, and from Lemmata~\ref{lem:jump-cut} and~\ref{lem:z-interlace}.
\end{proof}

\begin{figure}
\centering
\includegraphics{fig-CiDi}
\caption{The separation property provided by Lemma~\ref{lem:z-cut}.
The sequences $z_c(i)$ and $z_d(i)$ are denoted with rectangular and hexagonal shapes, respectively.
The sets $C_i$ and $D_i$ are denoted with dots and lines, respectively.}
\label{fig:CiDi}
\end{figure}


Intuitively, Lemma~\ref{lem:z-cut} implies that we may independently consider the vertices of
$\bigcup_{i \geq 0} C_i$ and of $\bigcup_{i \geq 0} D_i$: the sequences $z_c(i)$ and $z_d(i)$
give us some sort of `horizontal' partition of the graphs $G$ and $G^\ord$.
We now formalize this idea.

\begin{definition}[chain]
A \emph{chain} is a quadruple $(s,z,u,B)$, where
\begin{align*}
s &\in \{0,1,\ldots,\tau\}, \\
z &: \{0,1,\ldots,s\} \to [p^1,r^2], \\
u &: \{0,1,\ldots,s\} \to V(G), \\
B &: \{0,1,\ldots,s\} \to 2^{V(G)}
\end{align*}
with the following properties:
\begin{enumerate}
\item $z(i) \in [p^2,r^2]$ if and only if $i = s$;
\item $z(i) < z(i+1)$ for any $0 \leq i < s$;
\item $|B(i)| = z(i)-1$ for any $0 \leq i \leq s$;
\item $B(i) \subset B(i+1)$, for any $0 \leq i < s$;
\item $u(i) \in B(j)$ if and only if $0 \leq i < j \leq s$;
\item no edge of $G$ connects a vertex of $B(i)$ with a vertex
of $V(G) \setminus B(i+1)$ for any $0 \leq i < s$.
\end{enumerate}
A chain $(s,z,u,B)$ is \emph{consistent} with the ordering $\ord$
if $s = \min\{i: z_{z(0)}(i) \geq p^2\}$ and for all $0 \leq i \leq s$
\begin{enumerate}
\item $z(i) = z_{z(0)}(i)$;
\item $\ord(u(i)) = z(i)$;
\item $B(i) = \secp{z(i)}$.
\end{enumerate}
\end{definition}

We remark here that if $p^2 \leq n-2$ then $\jump{q} \leq r^2$ for any $q \leq p^2$, and hence $z_{z(0)}(s) \leq r^2$ 
for any $z(0) \leq r^2$ in the definition above.

Our next lemma follows immediately  from the definition of a chain.
\begin{lemma}\label{lem:ord-to-chain}
For  $q \in [p^1,r^2]$, let $s = \min \{i: z_q(i) \geq p^2\}$. For every $0 \leq i \leq s$, let 
\begin{align*}
z(i)  =  z_q(i),\\
u(i) = \ord^{-1}(z(i)) ,\\
B(i) = \secp{z(i)} .
\end{align*}
Then $I^\ord(q) := (s,z,u,B)$ is a chain consistent with $\ord$.

\end{lemma}

Moreover, the bound of Lemma~\ref{lem:short-z} gives us the following enumeration algorithm.
\begin{theorem}\label{thm:chain-enum}
In $(n|\secfam|)^{\Oh(\tau)}$ time one can enumerate a family $\chainfam$
of at most $(n|\secfam|)^{\Oh(\tau)}$ chains that contains all chains consistent with $\ord$.
\end{theorem}
\begin{proof}
There are $1+\tau \leq n$ possible values for $s$.
For each $0 \leq i \leq s$, there are at most $n$ choices for $z(i)$,
$n$ choices for $u(i)$ and $|\secfam|$ choices for $B(i)$.
The bound $s \leq \tau$ due to Lemma~\ref{lem:short-z} yields the desired bound.
Observe that the properties of a chain can be verified in polynomial time.
\end{proof}

We are now finally ready to state the definition of a layer-two state with its value.
\begin{definition}[layer-two state]
A \emph{layer-two state} consists of two chains $I^1 = (s^1,z^1,u^1,B^1)$, $I^2 = (s^2,z^2,u^2,B^2)$ with $I^1,I^2 \in \chainfam$ such that
\begin{enumerate}
\item $s^2 \leq s^1 \leq s^2 + 1$,
\item $z^1(i) \leq z^2(i)$, $B^1(i) \subseteq B^2(i)$ for any $1 \leq i \leq s^2$ and $z^2(i) \leq z^1(i+1)$, $B^2(i) \subseteq B^1(i+1)$ for any $1 \leq i < s^1$;
\item $u^1(i) = u^2(j)$ if and only if $z^1(i) = z^2(j)$ for any $1 \leq i \leq s^1$ and $1 \leq j \leq s^2$;
\end{enumerate}
Furthermore, we denote
\begin{align*}
C_i[I^1,I^2] &= B^2(i) \setminus B^1(i) &\mathrm{for\ any\ }0 \leq i\leq s^2,\\
D_i[I^1,I^2] &= B^1(i+1) \setminus B^2(i)&\mathrm{for\ any\ }0 \leq i < s^1,\\
Z_i[I^1,I^2] &= [z^1(i),z^2(i)-1] & \mathrm{for\ any\ }0 \leq  i \leq s^2,\\
C_{s^1}[I^1,I^2] &= (A^2 \cup X^2) \setminus B^1(s^1) &\mathrm{if\ }s^2 < s^1,\\
Z_{s^1}[I^1,I^2] &= [z^1(s^1), r^2-1] &\mathrm{if\ }s^2 < s^1,\\
\end{align*}
and require that for any $0 \leq i \leq s^1$ all positions of $Z_i[I^1,I^2]$ are pairwise adjacent in $\Gup$.
We define $\Gdown^\ast$ to be equal to $\Gdown$ with additionally $[p^2,r^2-1]$ and each $Z_i[I^1,I^2]$ turned into a clique, for every $0 \leq i \leq s^1$.
Note that by Lemma~\ref{lem:intersection-union}, $\Gdown^\ast$ is a proper interval graph with identity being an umbrella ordering. Moreover, it holds that $E(\Gdown^\ast) \subseteq E(\Gup)$ by the construction of $E(\Gdown^\ast)$ and the fact that $J^2\in \jumpfam$.

The \emph{value} of a layer-two state $(I^1,I^2)$ is a bijection
$f[I^1,I^2]: \bigcup_{i=0}^{s^1} C_i[I^1,I^2] \to \bigcup_{i=0}^{s^1} Z_i[I^1,I^2]$
such that:
\begin{enumerate}
\item $f[I^1,I^2]$ is a feasible ordering of its domain, that is,
for any $u$ in the domain of $f[I^1,I^2]$ we have $f[I^1,I^2](u) \in \pos_u$, and
  for any $u_1,u_2$ in the domain of $f[I^1,I^2]$ such that $u_1u_2\in E(G)$, it holds that
  $f[I^1,I^2](u_1)f[I^1,I^2](u_2) \in E(\Gup)$;
\item $f[I^1,I^2](u) \in Z_i[I^1,I^2]$ whenever $u \in C_i[I^1,I^2]$;
\item $f[I^1,I^2](u^1(i)) = z^1(i)$ for all $0 \leq i \leq s^1$;
\item $f[I^1,I^2](u) = \ord_X^1(u)$ whenever $u \in X^1$ and $f[I^1,I^2](u) = \ord_X^2(u)$ whenever $u \in X^2$;\label{p:layer-two-last}
\item among all functions $f$ satisfying the previous conditions, $f[I^1,I^2]$ minimizes the cardinality
of $\sol^{f,\ast}$,
where the set $\sol^{f,\ast}$ is defined as the unique minimal completion for the ordering $f$ of the subgraph of $G$ induced by the domain of $f$
and \spic{} instance $(G,k,(\pos_u)_{u \in V(G)},\Gdown^\ast,\Gup)$;\label{p:layer-two-min}
\item among all functions $f$ satisfying the previous conditions, 
$f[I^1,I^2]$ is lexicographically minimum.
\end{enumerate}
\end{definition}

Note that in the definition of a layer-two state we {\em{do not}} require that any of the chains begins in $[p^1,r^1]$, i.e. that $z^1(0)$ or $z^2(0)$ are in this interval. The values for the states where these chains begin at arbitrary positions within $[p^1,r^2]$ will be essential for computing the final value we are interested in.

Similarly as in the case of layer-one states, we have the following claim.
\begin{definition}[relevant pair]
A pair $(q^1,q^2)$ with $p^1 \leq q^1 \leq q^2 \leq \min(\jump{q^1},r^2)$ is called
\emph{relevant} if one of the following holds:
\begin{enumerate}
\item $q^2 \leq r^1$,
\item $q^1 = q^2$, or
\item there exists a position $q_{\leftarrow} \geq p^1$
such that $\jump{q_\leftarrow} \leq q^1 \leq q^2 \leq \jump{q_\leftarrow+1}$
(see also Figure~\ref{fig:relevant}).
\end{enumerate}
\end{definition}

\begin{figure}
\centering
\includegraphics{fig-relevant}
\caption{The last case of the definition of a relevant pair $(q_1,q_2)$.
  Both positions $q_1$ and $q_2$ need to belong to the gray area.}
\label{fig:relevant}
\end{figure}

\begin{lemma}\label{lem:layer-two-ord}
For any $q_1,q_2$ such that $p^1 \leq q^1 \leq q^2 \leq \min(\jump{q^1},r^2)$, the pair $(I^\ord(q^1),I^\ord(q^2))$ is a layer-two state.
If moreover $(q^1,q^2)$ is a relevant pair, then $f[I^\ord(q^1),I^\ord(q^2)]$ is a restriction of $\ord$ to the domain of $f[I^\ord(q^1),I^\ord(q^2)]$.
In particular, $f[I^\ord(p^1),I^\ord(r^1)] = f[J^\ord(p^1),J^\ord(p^2)]$.
\end{lemma}
\begin{proof}
By somehow abusing the notation, we denote $X^1=X_{p^1}$ and $X^2=X_{p^2}$. It is straightforward to verify from the definition that $(I^\ord(q^1),I^\ord(q^2))$ is a layer-two state
and the restriction of $\ord$ to $Y := \bigcup_i C_i[I^\ord(q^1),I^\ord(q^2)]$ satisfies the first~\ref{p:layer-two-last} requirements
of the definition of a value of a layer-two state, even if $(q^1,q^2)$ is not a relevant pair.
Moreover, observe that Lemma~\ref{lem:z-cut} implies that $\sol^\ord \cap \binom{Y}{2}$ is a completion for the ordering
$\ord|_Y$ of $Y$ in the instance $(G,k,(\pos_u)_{u \in V(G)},\Gdown^\ast,\Gup)$. Hence, $\sol^{\ord|_Y,\ast} \subseteq \sol^\ord \cap \binom{Y}{2}$.

Now assume that $(q^1,q^2)$ is a relevant pair and denote $f = f[I^\ord(q^1),I^\ord(q^2)]$ and $I^\ord(q^\alpha) = (s^\alpha,z^\alpha,u^\alpha,B^\alpha)$
for $\alpha=1,2$.
If $q^1 = q^2$, then observe that the sets $C_i[I^\ord(q^1),I^\ord(q^2)]$ are empty, and the state in question asks for an empty function. Hence, assume $q^1 < q^2$.
Define an ordering $\ord'$ of $V(G)$ so that $\ord'(u) = f(u)$ for any $u\in Y$, and $\ord'(u) = \ord(u)$ otherwise.

Let us define $\sol := (\sol^\ord \setminus \binom{Y}{2}) \cup \sol^{f,\ast}$. 
In the subsequent claims we establish some properties of the graph $G+\sol$ and ordering $\ord'$.
\begin{claim}\label{cl:layer-two:margins}
$\ord'(G+\sol)[[1,r^1-1]\cup[p^2,n]] = \ord(G^\ord)[[1,r^1-1]\cup[p^2,n]]$.
\end{claim}
\begin{proof}
Note here that $\ord'$ and $\ord$ agree on positions before $r^1$ and after $p^2-1$. Observe also that $[p^1,r^1-1]$ and $[p^2,r^2-1]$ are cliques in $\ord(G^\ord)$, and $[p^1,r^1-1]$ can have non-empty intersection only with the first of the intervals $Z_i[I^\ord(q^1),I^\ord(q^2)]$.
Since $\binom{[p^2,r^2-1]}{2},\binom{Z_i[I^\ord(q^1),I^\ord(q^2)]}{2}\subseteq E(\Gdown^\ast)\cap \binom{\sigma(Y)}{2}\subseteq E(\sigma'(G[Y]+\sol^{f,\ast}))$, it follows by the definition of $\sol$ that that intervals $[p^1,r^1-1]$ and $[p^2,r^2-1]$ are cliques in $\ord'(G+\sol)$ as well. Since $Y\subseteq \ord^{-1}([p^1,r^2-1])$, the claim follows.
\cqed\end{proof}

\begin{claim}\label{cl:layer-two:umbrella}
$\ord'$ is a umbrella ordering of $G+\sol$.
\end{claim}
\begin{proof}
Consider any $a,b,c \in V(G)$ with $ac \in E(G+\sol)$ and
$\ord'(a) < \ord'(b) < \ord'(c)$; we want to show the umbrella property for the triple
$a,b,c$ in the graph $G+\sol$. We consider a few cases, depending on the intersection $\{a,b,c\} \cap Y$.
\begin{enumerate}
\item If $a,b,c \in Y$ or $a,b,c \notin Y$, then the umbrella property holds by the definition of $\sol^\ord$ and $\sol^{f,\ast}$.
\item If $\ord'(a) \geq p^2$ or $\ord'(c) < r^1$, then recall that $\ord'(G+\sol)[[1,r^1-1]\cup[p^2,n]] = \ord(G^\ord)[[1,r^1-1]\cup[p^2,n]]$. Then the umbrella property for $a,b,c$ follows from the fact that $\ord$ is an umbrella ordering of $G^\ord$.

Hence, in the remaining cases we have in particular that $a\notin X^2$ and $c\notin X^1$. Observe also that the assumption $ac \in E(G+\sol)$ implies that $p^1 \leq \ord'(a) < \ord'(c) < r^2$, since $r^1=\jump{p^1}$ and $r^2=\jump{p^2}$.

\item If $a,c \in Y$ and $b \notin Y$ then, by the structure of $Y$, we have $a \in C_i[I^\ord(q^1),I^\ord(q^2)]$, $c \in C_j[I^\ord(q^1),I^\ord(q^2)]$ for some $1 \leq i < j \leq s^1$. We claim that $j = i+1$. Assume the contrary. Observe that if $i+1 < j$ then in particular $i < s^2$.
By Lemma~\ref{lem:z-cut}, no edge of $G^\ord$ connects $\secp{z_{q^2}(i)}$ with $V(G) \setminus\secp{z_{q^2}(i+1)}$, so in particular there is no such edge neither in $G$, which is subgraph of $G^\ord$. Likewise, there is no edge between $[1,z_{q^2}(i)]$ and $[z_{q^2}(i+1),n]$ in $\Gdown^\ast$. By the construction of $\sol^{f,\ast}$ it follows that also no edge of $\sol^{f,\ast}$ connects $\secp{z_{q^2}(i)}$ with $V(G) \setminus\secp{z_{q^2}(i+1)}$.
As $\ord$ and $\ord'$ differ only on the internal ordering of each set $C_i[I^\ord(q^1),I^\ord(q^2)]$, and $ac\in E(G+F)$, we have a contradiction,
and hence $c \in C_{i+1}[I^\ord(q^1),I^\ord(q^2)]$.
It follows that $b \in D_i[I^\ord(q^1),I^\ord(q^2)]$ and, by Lemma~\ref{lem:z-cut}, $ab,bc \in E(G^\ord)$.
  By the definition of $\sol$, $ab,bc \in E(G)\cup \sol$.

In the remaining cases we have that either $a$ or $c$ does not belong to $Y$. Hence $ac\in E(G^\ord)$ by the definition of $F$.

\item If $a \in Y \setminus X^2$ and $c \notin Y$, then, by Lemma~\ref{lem:z-cut}, we have $a \in C_i[I^\ord(q^1),I^\ord(q^2)]$
and $c \in D_i[I^\ord(q^1),I^\ord(q^2)]$ for some $0  \leq i < s^1$. By Lemma~\ref{lem:z-cut},
$C_i[I^\ord(q^1),I^\ord(q^2)] \cup D_i[I^\ord(q^1),I^\ord(q^2)]$ is a clique in $G^\ord$, and, by the definition of $\Gdown^\ast$,
$C_i[I^\ord(q^1),I^\ord(q^2)]$ is a clique in $G+\sol$. Hence, $ab,bc \in E(G) \cup \sol$ regardless whether $b\in Y$ or not.
\item If $a \notin Y$ and $c \in C_i[I^\ord(q^1),I^\ord(q^2)]$ for some $i > 0$, then, by Lemma~\ref{lem:z-cut},
  $a \in D_{i-1}[I^\ord(q^1),I^\ord(q^2)]$. As in the previous case, Lemma~\ref{lem:z-cut} asserts that 
$D_{i-1}[I^\ord(q^1),I^\ord(q^2)] \cup C_i[I^\ord(q^1),I^\ord(q^2)]$ is a clique in $G^\ord$, and the definition of $\Gdown^\ast$
gives us that $C_i[I^\ord(q^1),I^\ord(q^2)]$ is a clique in $G+\sol$. Consequently, $ab,bc \in E(G) \cup \sol$  regardless whether $b\in Y$ or not.
\item If $a \notin Y$ and $c \in C_0[I^\ord(q^1),I^\ord(q^2)] = \ord^{-1}([q^1,q^2-1])$ then, as $\ord'(c) \geq r^1$, we have that pair $(q^1,q^2)$ is 
a relevant pair due to existence of some position $q_\leftarrow$. Since $ac\in E(G^\ord)$, we have that $\ord'(a) = \ord(a) \geq q_\leftarrow+1$.
As $\jump{q_\leftarrow+1} \geq q^2$, we have that also $ab\in E(G^\ord)$ and $bc\in E(G^\ord)$. By the definition of $F$ we infer that $ab \in E(G)\cup \sol$ and, additionally, $bc \in E(G) \cup \sol$ in the case $b \notin Y$.
If $b \in Y$ then $b \in C_0[I^\ord(q^1),I^\ord(q^2)]$ and $bc \in E(G) \cup \sol$ by the definition of $\Gdown^\ast$.
\item If $a,c \notin Y$ and $b \in Y$, then let $b \in C_i[I^\ord(q^1),I^\ord(q^2)]$ for some
$0 \leq i \leq s^1$. Since $\ord$ and $\ord'$ differ only on internal ordering of sets $C_i[I^\ord(q^1),I^\ord(q^2)]$ and $a,c\notin Y$, then the condition $\ord'(a) < \ord'(b) < \ord'(c)$ implies also $\ord(a) < \ord(b) < \ord(c)$. Since $ac\in E(G^\ord)$ and $\ord$ is an umbrella ordering of $G^\ord$, we infer that $ab,bc\in E(G^\ord)$. By the definition of $F$ this implies that $ab,bc\in E(G+F)$.
\end{enumerate}
\cqed\end{proof}

\begin{claim}\label{cl:layer-two:Gdown}
$E(\Gdown^\ast) \subseteq E(\ord'(G+\sol))$.
\end{claim}
\begin{proof}
Consider any $pq \in E(\Gdown^\ast)$.
Denote $a = \ord^{-1}(p)$, $b = \ord^{-1}(q)$ and similarly denote $a'$ and $b'$ for the ordering $\ord'$; we want to show that $a'b' \in E(G) \cup \sol$.
As $E(\Gdown^\ast) \subseteq E(\ord(G^\ord))$ we have $ab \in E(G^\ord)$.
If $p,q \in \bigcup_i Z_i[I^\ord(q^1),I^\ord(q^2)]$ then $a'b' \in E(G) \cup \sol^{f,\ast}$ by the definition of $\sol^{f,\ast}$.
Otherwise, without loss of generality assume that $q \notin \bigcup_i Z_i[I^\ord(q^1),I^\ord(q^2)]$, and hence $b = b'$.
If additionally $a=a'$ then $a'b' \in E(G) \cup\sol$ follows directly from the definition of $\sol$ and the fact that $ab \in E(G^\ord)$.
In the remaining case, if $a \neq a'$, we have $p \in Z_i[I^\ord(q^1),I^\ord(q^2)]$ and $a,a' \in C_i[I^\ord(q^1),I^\ord(q^2)]$ for some
$0 \leq i \leq s^1$. Moreover, from the assumption $a \neq a'$ we infer that $r^1 \leq p < p^2$, and consequently $i < s^1$.
By the definition of $\sol$, we need to show that $a'b \in E(G^\ord)$.

We consider two cases, depending on the relative order of $p$ and $q$. If $p < q$, then we have $z^2(i) \leq q < z^1(i+1)$ by Lemma~\ref{lem:z-cut}
and consequently $b \in D_i[I^\ord(q^1),I^\ord(q^2)]$. By Lemma~\ref{lem:z-cut} again, $b$ is adjacent to all vertices of $C_i[I^\ord(q^1),I^\ord(q^2)]$
in the graph $G^\ord$, and $a'b \in E(G^\ord)$.
A similar argument holds if $q < p$ and $i > 0$: by Lemma~\ref{lem:z-cut}, we have first that $b \in D_{i-1}[I^\ord(q^1),I^\ord(q^2)]$ and, second,
that $b$ is adjacent in $G^\ord$ to all vertices of $C_i[I^\ord(q^1),I^\ord(q^2)]$, and hence $a'b \in E(G^\ord)$.
In the remaining case, if $q <p$ and $i=0$ (hence $p\in [q^1,q^2-1]$),  from $p \geq r^1$ it follows that the reason why $(q^1,q^2)$ is a relevant pair
is existence of some position $q_\leftarrow$. Since $ab \in E(G^\ord)$, we infer that $q \geq q_\leftarrow+1$. Hence, $b$ is
adjacent in $G^\ord$ to all vertices of $C_0[I^\ord(q^1),I^\ord(q^2)]$, in particular to $a'$, and the claim is proven.  
\cqed\end{proof}

\begin{claim}\label{cl:layer-two:Gup} 
$E(\ord'(G)) \subseteq \Gup$ and $\ord'$ is a feasible ordering of $G$.
\end{claim}
\begin{proof}
Observe that it follows directly from the definition of $\ord'$ that $\ord'(u) \in \pos_u$ for any vertex $u$.
Hence, to show feasibility of $\ord'$ it suffices to show that $E(\ord'(G)) \subseteq \Gup$.

Consider any $ab \in E(G)$. If both $a$ and $b$ belong to $Y$ or both do not belong,
then the claim is obvious by the feasibility of both $\ord$ and $f$.
Assume then $a \in Y$ and $b \notin Y$. If $\ord(a) = \ord'(a)$ then clearly $\ord'(a)\ord'(b) = \ord(a)\ord(b) \in E(\Gup)$. 
Otherwise, $a \notin X^2$ and $a \in C_i[I^\ord(q^1),I^\ord(q^2)]$ for some $0 \leq i < s^1$.
If $\ord(b) \geq z_{q^2}(i)$ then Lemma~\ref{lem:z-cut} implies that $b \in D_i[I^\ord(q^1),I^\ord(q^2)]$.
By Lemma~\ref{lem:z-cut} again, $[z_{q^1}(i),z_{q^1}(i+1)-1]$ is a clique in $\ord(G^\ord)$ and hence in $\Gup$ as well, so $\ord'(a)\ord'(b) \in E(\Gup)$.
A similar situation happens if $\ord(b) < z_{q^1}(i)$ and $i>0$: $b \in D_{i-1}[I^\ord(q^1),I^\ord(q^2)]$ and again Lemma~\ref{lem:z-cut}
together with feasibility of $\ord$ proves the claim.
In the remaining case $i = 0$ and $\ord(b) < q^1$. As $\ord(a) \neq \ord'(a)$ we have $a \notin X^1$ and hence the reason why $(q^1,q^2)$ is a relevant pair
must be existence of some position $q_\leftarrow$. As $ab \in E(G)$ we have $\ord(b) \geq q_\leftarrow+1$. As $\jump{q_\leftarrow+1} \geq q^2$,
the position $b$ is adjacent to all positions of $[q^1,q^2-1]$ in $\ord(G^\ord)$ and hence $\ord'(a)\ord'(b) \in E(\ord(G^\ord)) \subseteq E(\Gup)$ as claimed.
\cqed\end{proof}

From the above claims we infer that $|\sol^{\ord'}| \leq |\sol^{f,\ast}| + |\sol^\ord \setminus \binom{Y}{2}|$, whereas
$\sol^{\ord|_Y,\ast} \subseteq \sol^\ord \cap \binom{Y}{2}$.
By the minimality of both $f$ and $\ord$, including the lexicographical minimality, we have $f = \ord|_Y$ and the lemma is proven.
\end{proof}

The layer-two dynamic programming algorithm computes,
for any layer-two state $(I^1,I^2)$,   
a function $g[I^1,I^2]$
that satisfies the first~\ref{p:layer-two-last} conditions of $f[I^1,I^2]$, and we will inductively ensure that $g[I^\ord(q^1),I^\ord(q^2)] = f[I^\ord(q^1),I^\ord(q^2)]$
for any relevant pair $(q^1,q^2)$.
We compute the values $g[I^1,I^2]$ in the order of decreasing value of $z^1(0)$ and, subject to that, increasing value of $z^2(0)$.
(Formally, $g[I^1,I^2]$ may also take value of $\bot$, which implies that either $I^1$ or $I^2$ is not consistent with $\ord$;
 we assign this value to $g[I^1,I^2]$ whenever we find no candidate for its value.)

Consider now a fixed layer-two state $(I^1,I^2)$ with
$I^1 = (s^1,z^1,u^1,B^1)$ and $I^2 = (s^2,z^2,u^2,B^2)$.
We start with the the base case when we have that either $z^1(0) = z^2(0)$ or
$z^1(0) \geq p^2-1$. Observe that in this situation we have that the domain of $g[I^1,I^2]$ is either $X^2$ or $X^2$ with an additional element $u^1(0)$ which must be mapped to $z^1(0)=p_2-1$. Hence all the values of $f[I^1,I^2]$ are fixed by $\ord_X^2$, $u^1$ and $z^1$, and there is only one candidate for this value.
It is straightforward to verify that, in the case when $I^1 = I^\ord(q^1)$ and $I^2 = I^\ord(q^2)$, this
unique candidate is indeed a restriction of $\ord$ and hence equals $f[I^1,I^2]$.

In the inductive step we have $z^1(0) < z^2(0)$ and $z^1(0) < p^2-1$.
We consider two cases, depending on the value of $z^2(0) - z^1(0)$.

First assume $z^2(0) - z^1(0) > 1$.
In this case consider all possible chains $I^3 = (s^3,z^3,u^3,B^3)$ such that both $(I^1,I^3)$ and $(I^3,I^2)$ are layer-two states,
and $z^1(0) < z^3(0) < z^2(0)$.
We take as candidate value for $g[I^1,I^2]$ the union $g[I^1,I^3] \cup g[I^3,I^2]$, and pick $g[I^1,I^2]$ using the criteria
from the definition of the value $f[I^1,I^2]$, but taking only functions $g[I^1,I^3] \cup g[I^3,I^2]$ for all choices of $I^3$ as candidates.

We claim that if $I^1 = I^\ord(q^1)$, $I^2 = I^\ord(q^2)$ and $(q^1,q^2)$ is a relevant pair, then 
$g[I^1,I^2] = f[I^1,I^2]$. Note that it suffices to show that $f[I^1,I^2]$ is considered as a candidate for $g[I^1,I^2]$
in the aforementioned process for some choice of $I^3$.
Consider any $q^1 < q^3 < q^2$ and observe that if $(q^1,q^2)$ is a relevant pair,
then also $(q^1,q^3)$ and $(q^3,q^2)$ are relevant pairs: this is clearly true for
the case $q^2 \leq r^1$ and, in the last case of the definition of a relevant pair,
notice that the same position $q_\leftarrow$ witnesses also that $(q^1,q^3)$ and $(q^3,q^2)$ are relevant.
Denote $I^3 = I^\ord(q^3)$ and observe that we consider a candidate $g[I^1,I^3] \cup g[I^3,I^2]$ for $g[I^1,I^2]$.
By Lemma~\ref{lem:layer-two-ord} and the inductive assumption, this candidate is a restriction of $\ord$, and hence,
again by Lemma~\ref{lem:layer-two-ord}, equals $f[I^1,I^2]$.

We are left with the case $z^2(0) = z^1(0) + 1$. As $z^1(0) < p^2 - 1$, we have $z^2(0) < p^2$.
For $\alpha=1,2$ define $s^\alpha_\ast = s^\alpha-1$, and
$z^\alpha_\ast(i) = z^\alpha(i+1)$, $u^\alpha_\ast(i) = u^\alpha(i+1)$ and $B^\alpha_\ast(i) = B^\alpha(i+1)$
for any $0 \leq i \leq s^\alpha_\ast$, and $I^\alpha_\ast = (s^\alpha_\ast, z^\alpha_\ast, u^\alpha_\ast, B^\alpha_\ast)$.
In this case we consider only one candidate for $g[I^1,I^2]$, being $g[I^1_\ast,I^2_\ast]$, extended with 
$g[I^1,I^2](u^1(0)) = z^1(0)$.

It remains to show that if $I^1 = I^\ord(q^1)$, $I^2 = I^\ord(q^2)$ and $(q^1,q^2)$ is an relevant pair,
then $g[I^1,I^2] = f[I^1,I^2]$.
Observe that $I^1_\ast = I^\ord(\jump{q^1})$ and $I^2_\ast = I^\ord(\jump{q^2})$.
Moreover, the position $q^1$ witnesses that $(\jump{q^1},\jump{q^2})$ is a relevant pair, and hence
$g[I^1_\ast,I^2_\ast] = f[I^1_\ast,I^2_\ast]$ by induction.
This completes the proof that $g[I^\ord(q^1),I^\ord(q^2)] = f[I^\ord(q^1),I^\ord(q^2)]$ for all relevant pairs $(q^1,q^2)$.

As candidates for the value $f[J^1,J^2]$ of the layer-one state $(J^1,J^2)$ we are currently processing, we take all the values $g[I^1,I^2]$ for all the layer-two states $(I^1,I^2)$ for which the domain of $f[I^1,I^2]$ is equal to the domain of $f[J^1,J^2]$. By Theorem~\ref{thm:chain-enum}, there are at most $(n|\secfam|)^{\Oh(\tau)}$ guesses for such states, and they can be enumerated in $(n|\secfam|)^{\Oh(\tau)}$ time. Observe also that if indeed $J^1 = J^\ord(p^1)$ and $J^2 = J^\ord(p^2)$, then the layer-two state $(I^1,I^2) = (I^\ord(p^1),I^\ord(r^1))$ will be among the enumerated states. Since $(p^1,r^1)$ is a relevant pair, we have that $g[I^\ord(p^1),I^\ord(r^1)]=f[I^\ord(p^1),I^\ord(r^1)]$, while by Lemma~\ref{lem:layer-two-ord} we have that $f[I^\ord(p^1),I^\ord(r^1)]$ is equal to the restriction of $\ord$ to its domain, which in turn is equal to the domain of $g[J^1,J^2]$. Hence, the restriction of $\ord$ to the domain of $g[J^1,J^2]$, which is exactly equal to $f[J^1,J^2]$ by Lemma~\ref{lem:layer-one-1}, will be among the enumerated candidate values --- this was exactly the property needed by the layer-one dynamic program.


By Theorem~\ref{thm:chain-enum} there are $(n|\secfam|)^{\Oh(\tau)}$ layer-two states, thus the entire computation of $f[J^1,J^2]$
takes $(n|\secfam|)^{\Oh(\tau)}$ time, as was promised.
This concludes the proof of Theorem~\ref{thm:dp}, and hence finishes the proof of Theorem~\ref{thm:main}.
