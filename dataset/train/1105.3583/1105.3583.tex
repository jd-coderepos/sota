\documentclass{LMCS}

\usepackage{enumerate}
\usepackage{hyperref}

\usepackage[utf8]{inputenc} 
\usepackage{algorithmic}
\usepackage{amssymb}
\usepackage{amsmath, latexsym}
\usepackage{amsfonts}
\usepackage{xspace}
\usepackage[mathscr]{eucal}
\usepackage{caption}

\newcommand{\argkx}{\left( x_{1}, \ldots, x_{k} \right)}
\newcommand{\arglx}{\left( x_{1}, \ldots, x_{l} \right)}
\newcommand{\CDlin}{{\textsc{Constant-}\textsc{Delay}}\xspace}  
\newcommand\set[1]{\ensuremath{\{#1\}}\xspace}
\newcommand\cA{\ensuremath{{\mathcal A}}\xspace}
\newcommand\cN{\ensuremath{{\mathcal N}}\xspace}
\newcommand{\T}{\ensuremath{{\mathcal T}}}
\newcommand{\F}{\ensuremath{{\mathcal F}}}
\newcommand\FO{\textup{FO}\xspace}

\def\doi{7 (2:20) 2011}
\lmcsheading {\doi}
{1--8}
{}
{}
{Nov.~\phantom19, 2010}
{Jun.~29, 2011}
{}   
\begin{document}

\title[First-order query evaluation on structures of bounded degree]{First-order query evaluation on structures of bounded degree}

\author[W.~Kazana]{Wojciech Kazana\rsuper a}	\address{{\lsuper a}INRIA and ENS Cachan}	\email{kazana@lsv.ens-cachan.fr}  \thanks{{\lsuper a}This work has been partially funded by the European
  Research Council under the European Community's Seventh Framework
  Programme (FP7/2007-2013) / ERC grant Webdam, agreement
  226513. \url{http://webdam.inria.fr/}}	

\author[L.~Segoufin]{Luc Segoufin\rsuper b} \address{{\lsuper b}INRIA and ENS Cachan} \email{see \url{http://www-rocq.inria.fr/~segoufin}} \thanks{{\lsuper b}We acknowledge the financial support of the Future 
  and Emerging Technologies (FET) programme within the Seventh
  Framework Programme for Research of the European Commission, under
  the FET-Open grant agreement FOX, number FP7-ICT-233599.} 

\keywords{First-order, query evaluation, enumeration, constant delay}
\subjclass{F.4.1,F.1.3}

\begin{abstract}
  We consider the enumeration problem of first-order queries over
  structures of bounded degree. Durand and Grandjean have shown that
  this problem is in \CDlin. An enumeration problem belongs to \CDlin if
  for an input of size  it can be solved by (i) an  precomputation
  phase building an index structure, followed by (ii) a phase enumerating
  the answers with no repetition and a constant delay between two
  consecutive outputs.
  In this article we give a different proof of this result based on
  Gaifman's locality theorem for first-order logic. Moreover, the
  constants we obtain yield a total evaluation time that is triply
  exponential in the size of the input formula, matching the
  complexity of the best known evaluation algorithms.
\end{abstract}

\maketitle

\section{Introduction.}

Model checking is the problem of testing whether a given sentence is true in a
given model. It's a classical problem in many areas of computer science, in
particular in verification.  If the formula is no longer a sentence but has
free variables then we are faced with the query evaluation problem. In this
case the goal is to compute all the answers of a given query on a given
database.

As for model checking, query evaluation is a problem often requiring a time at
least exponential in the size of the query. Even worse, the evaluation often
requires a time of the form , where  is the size of the database
and  the size of the query. This is dramatic, even for small , when the
database is huge.

However there are restrictions on the structures that make things easier. For
instance MSO sentences can be tested in time linear in  over structures of
bounded tree-width~\cite{Courcelle90} and MSO queries can be evaluated in time
linear in , where  is the size of the output of the query (note that
 could be exponential in the number of free variables of the query, and
hence in )~\cite{FlumFrickGrohe02}.

In this paper we are concerned with first-order logic (FO) and structures of
bounded degree. In this case the model checking problem for FO sentences is known to be
linear in ~\cite{Seese96}. Moreover, the constant factor is at most triply
exponential in the size  of the formula~\cite{FrickGrohe04}. This last
algorithm easily extends to query evaluation obtaining an algorithm working in
time  where  is a triply exponential function.

As we already mentioned, the size  of the output may be exponential in the
arity of the formula and therefore may still be large. In many applications
enumerating all the answers may already consume too many of the allowed
resources. In this case it may be appropriate to first output a small subset of
the answers and then, on demand, output a subsequent small number of answers
and so on until all possible answers have been exhausted. To make this even
more attractive it is preferable to be able to minimize the time necessary to
output the first answers and, from a given set of answers, also minimize the
time necessary to output the next set of answers - this second time interval is
known as the \emph{delay}.

We say that a query can be evaluated in linear time and constant delay if
there exists an algorithm consisting of a preprocessing phase taking time
linear in  which is then followed by an output phase printing the answers
one by one, with no repetition and with a constant delay between each output.
Notice that if a linear time and constant delay algorithm exists then the time
needed for the total query evaluation problem is bounded by  for
some function . Hence this is indeed a restriction of the linear time query
evaluation algorithms mentioned above. From the best of our knowledge it is not
yet known whether a bound  for some function  on a query evaluation
problem implies the existence of a linear time and constant delay enumeration
algorithm. We conjecture this is not the case.

It was shown in~\cite{DurandGrandjean07} that linear time constant delay query
evaluation algorithms could be obtained for FO queries over structures
of bounded degree, hence improving the results of~\cite{Seese96}
and~\cite{FrickGrohe04}.


The proof of~\cite{DurandGrandjean07} is based on an intricate quantifier
elimination method. In this paper we provide a different proof of this result
based on Gaifman Locality of FO queries. Our algorithm can be seen as an extension
of the algorithm of~\cite{FrickGrohe04} to queries. However the index structure
built during the preprocessing phase is more complicated than the one
of~\cite{FrickGrohe04} in order to obtain the constant delay enumeration.
Moreover, our constant factor is triply exponential in the size of the formula,
while it is not clear whether the constant factor obtained
in~\cite{DurandGrandjean07} is elementary. Note that the triply exponential
constant factor cannot be significantly improved: it is shown
in~\cite{FrickGrohe04} that a constant factor only doubly exponential in the
size of the formula is not possible unless the parametrized complexity class
AW collapses to the parametrized class FPT.

\section{Definitions.}
\subsection{Gaifman locality and first-order logic.}

A relational signature is a tuple , each 
being a relation symbol of arity . A relational structure over  is
a tuple , where  is the set of elements of \cA and  is
a subset of .
We fix a reasonable encoding of structures by words over some finite
alphabet. The \emph{size} of \cA is denoted by  and is the length of
the encoding of \cA.

The \emph{Gaifman graph} of a relational structure , denoted by ,
is defined as follows: the set of vertices of  is  and there is an
edge  in  iff there exists a relation  and a tuple  such that both  and  occur in . Given , the
\emph{distance} between  and , denoted , is the length of a
shortest path between  and  in  or  if  and  are
not connected. The \emph{distance} between two tuples  and  of \cA, denoted , is the . For a given  and a given tuple of elements  of some structure , we denote by  the set of all elements in  such
that their distance from  is less or equal to .  The
\emph{-neighborhood} of , denoted as , is the
substructure of \cA induced by  and expanded with one constant
for each element of . 
Given two tuples of elements  and  we say that they have \emph{the same -neighborhood type}, written
, if there is an isomorphism between
 and .

We consider first-order logic (\FO) built from atomic formulas of the form
 or  for some relation  and closed
under the usual Boolean connectives () and existential and
universal quantifications ().  When writing  we
always mean that  are exactly the free variables of . Given a
structure  and a tuple  of elements of , we write  if the formula  is true in  after
replacing its free variables with . As usual  denotes the size
of .

We are now ready to state Gaifman locality for \FO.

\begin{thm}[Gaifman Locality Theorem \cite{Libkin04}]\label{hanf}
  For any first-order formula , for every structure  and
  tuples , , we have  implies
   iff , where .
\end{thm}

Given , a structure is said to be -degree-bounded, if
the degree of the Gaifman graph is bounded by . The following nice
algorithmic property of -degree-bounded structures can be proved using
Theorem~\ref{hanf}.

\begin{thm}[\cite{Seese96,FrickGrohe04}]\label{Seese}
  Fix . The problem of whether a given -degree-bounded
  structure  satisfies a given first-order sentence  is decidable in
  time .
\end{thm}

\subsection{Model of computation and \CDlin class.}

We use Random Access Machines (RAM) with addition and uniform cost measure as a
model of computation. For further details on this model and its use in logic
see \cite{DurandGrandjean07}.

An enumeration problem is a binary relation. Given an enumeration problem 
and an input , a \emph{solution for } is a  such that .
An enumeration problem  induces a computational problem as follows: Given an
input , output all its solutions. An enumeration problem is in the class
\CDlin if on input  it can be decomposed into two steps:
\begin{itemize}
	\item a precomputation phase that is performed in time ,
	\item an enumeration phase that outputs all the solutions for  with
          no repetition and a constant delay between two consecutive
          outputs. The enumeration phase has full access to the output of the
          precomputation phase but can use only a constant total amount of extra memory.
\end{itemize}
In particular if  is in \CDlin then the enumeration problem  can be
solved in time .  From the best of our knowledge it
is not known whether the converse is true or not. We conjecture that it is not.
More details about \CDlin can be found in \cite{DurandGrandjean07}.

\newcommand{\olex}{\ensuremath{<_{\text{lex}}}}
We are interested in the following enumeration problem for 
\FO and : 


We further denote by  the set  and by  the cardinality of this set.
We show that  is in \CDlin.

\begin{thm}[\cite{DurandGrandjean07}]\label{main-result}
  There is an algorithm that for all , all  \FO
  and all -degree-bounded structures \cA enumerates  with
  a precomputation phase taking time  and
  a delay during the enumeration phase that is triply exponential in .
  In particular, for all  and all  \FO the
  enumeration problem  is in \CDlin. Moreover, if the domain
  of \cA is linearly ordered, the algorithm enumerates  in increasing
  order relative to the induced lexicographical order on tuples.
\end{thm}

Hence the total query evaluation induced by the enumeration procedure of
Theorem~\ref{main-result} is in time 
thus matching the model checking complexity of Theorem~\ref{Seese}.  Our
proof of Theorem~\ref{main-result} is based on Gaifman Locality Theorem while the
proof of~\cite{DurandGrandjean07} uses a quantifier elimination procedure (see
also~\cite{Lindell08} for a similar argument).
Note that it is not clear from the proof of~\cite{DurandGrandjean07} that their
algorithm is triply exponential in the size of the formula.

\section{\FO query evaluation.}

In this section we assume  to be fixed and all our structures
are -degree bounded.

A formula  with  free variables  is said
to be \emph{connected around } if  logically implies
that  are in the -neighborhood of  for .

Let  be the set of all isomorphism types of -neighborhoods of
single elements, i.e. the isomorphism types of structures of the form
 for some element  of some structure \cA. By
\emph{-neighborhood-type} of an element  we mean the isomorphism type
of its -neighborhood. Because our structures are -degree-bounded each
-neighborhood has at most  elements. For each  we
denote by  the fact that the -neighborhood-type of  is . For
each type in  we fix a representative for the corresponding
-neighborhood and fix a linear order among its elements. This way, we can
speak of the first, second,\ldots, element of an -neighborhood. For
technical reasons, we actually fix a linear order for each -neighborhood for
 such that (i) it is compatible with the distance from the center of
the neighborhood: the center is first, then come all the elements at distance
, then all elements at distance  and so on\ldots and (ii) the order of a
-type is consistent with the order on the induced -type.

For some sequence  of  elements from
, we write  for the fact that, for
,  is the
-th element of the -neighborhood of . Let  be the
set of all possible such . Let .

For a given  a \emph{-partition} of 
is a set of pairs  such that
, ,   for
, and . For a given -partition  of  and
 we write  to represent variables from
,  to represent the first variable from ,  to represent second
variable and so on.

For a given -partition 
of  by  we mean a conjunction of formulas
saying that  for all
 and formulas .
Note that the latter part implies that  is connected around .

The following is an immediate consequence of
Theorem~\ref{hanf}.

\begin{lem}\label{lemma-decompose}
  Fix a structure \cA.  Then any formula  with  free variables
  is equivalent over \cA to a formula of the form
  
where ,  is the set of all -partitions of , 
and  is finite.
\end{lem}

\proof
Let  be a formula with  free variables and .
As in the statement of this lemma, we denote by  the set of all partitions
 of  with .

By taking all possible -partitions over  we see that
 is equivalent to:


Let  be a tuple of  such that . Thus for
exactly one , .  As  induces that variables from each  for
some  are connected, the -neighborhood of each 
is completely included into the -neighborhood of . Let
. For  let  be the -neighborhood-type of
. We now take  as the set of all such tuples  for all tuples  such that . By construction we have  implies
\eqref{eq-decompose}. The reverse inclusion is an immediate consequence of
Gaifman Locality Theorem: When  holds,  is
induced by  and . Moreover,  is the
disjoint union of  and is therefore induced by . \qed

\medskip
We are now ready to prove Theorem~\ref{main-result}.
\medskip

\proof[Proof of Theorem~\ref{main-result}] Fix a formula  with
 free variables. Let \cA be a structure. Let . By
Lemma~\ref{lemma-decompose},  is equivalent over \cA to a formula
of the form given by \eqref{eq-decompose}.  We assume that \cA comes with a
linear order over its elements. If not, we use the linear order induced by the
encoding of \cA.

  Intuitively the precomputation phase determines the disjunction given by
  \eqref{eq-decompose} and precomputes the
  -neighborhoods of each element of . The fact that this can be done
  in time linear in  and triply exponential in  will make use
  of Theorem~\ref{Seese}.

  In a first step, for each  we precompute the pairs of nodes at
  distance . In other words, for each  in \cA, we compute the set of
  elements  such that . This can easily be done in time
  linear in  by induction on : during the base case we compute
  the Gaifman graph of \cA and then we perform the classical computation of
  the transitive closure of this graph up to depth .

  In a second step, the precomputation phase computes for each element  of
   its -neighborhood: for each element  of \cA, we compute its
  -neighborhood-type and for all  a pointer from  to the -th
  element of its -neighborhood. We use an induction on the radius of the
  neighborhood to achieve this goal within the desired time constraints.
  
  As -neighborhoods all share the same isomorphism type and have just one
  pointer to their centers, the induction base is obvious. So let's assume that
  in linear time in the size of  we have computed all -neighborhoods
  for all nodes.  With one more linear pass we now compute the
  -neighborhoods. Fix . From the first step, we have all
  the elements of \cA at distance  from . As we already have computed
  the -neighborhood, it remains to try all possible orders among those
  elements and test isomorphism with the ordered types we have initially fixed.

  There are at most  nodes at distance  and . Hence the
  number of orders we need to test is bounded by . Once the order is
  fixed we try all possible -neighborhood-types that we have initially
  fixed (there are  possibilities) and then test that the two orders
  induce an isomorphism (each test simply requires going through all tuples of
  the neighborhood). Let  be the maximal size of a
  -neighborhood. Thus this step is altogether achieved in time
   which is triply exponential in 
  because ,  and .

  During the third step of the precomputation we determine the
  -neighborhood-types that are relevant for  over \cA. Fix a -partition
   of  and a sequence
  . This sequence is
  \emph{relevant for } if . Notice that the tests of the form
   have been precomputed during the second step and
  can therefore now be treated as unary symbols. Similarly the tests
   can be expressed using the graph computed during the
  first phase. Altogether, the first and second phase has replaced
  
  with a formula of size linear in . Hence
  we can apply Theorem~\ref{Seese} in order to test whether the sequence is
  relevant for  in time linear in  and triply exponential in the size of
  the formula.  We do this for all possible , investigating at most
   cases. The number of possible  is
  the number of possible splits of  variables into disjoint and nonempty subsets
  multiplied by , which altogether is again .
  For each  we store a list of all sequences relevant for it. We call a -partition
   \emph{relevant} if that list is nonempty.

  The last step of the precomputation phase orders, for each , the elements of  having that particular
  -neighborhood-type and stores a pointer from one element to the next
  one according to the linear order on the elements of \cA. To do that, we just
  need to enumerate through all the elements in , in the order provided by
  the linear order on its elements, and, using information obtained in the
  second step, add each of them to a proper list. In order to do this we need
  to be able to sort a set of elements in linear time and this can be done in
  our RAM model as explained in~\cite{Grandjean96}.

  Altogether we have a precomputation phase of the desired properties: it works
  in time linear in  and triply exponential in . We now turn to
  the enumeration phase.

  Fix relevant -partition  in . We show how to
  enumerate in lexicographical order, with no repetition, constant memory and
  constant delay, all the tuples  such that .
  The result will then follow from the following simple lemma, whose proof
  consist in merging two ordered lists.

  \begin{lem}[\cite{Bagan09}]\label{lemme-disjunct}
    If there is a linear order  such that  are in \CDlin and both
    output their answers in increasing order relative to , then 
    is also in \CDlin and the answers can be enumerated in increasing order
    relative to~. \qed
  \end{lem}

 The proof is by induction on the number  of classes in the -partition
 . The base case being a particular case of the inductive step, we only do
 the inductive step.

 Without loss of generality we assume that the most significant variable of
  is in the first variable of , that the most significant
 variable of  is the first variable of 
 and so on. We simultaneously do the following for each sequence
  relevant for  and use Lemma \ref{lemme-disjunct}
 to avoid duplicate answers.

 Fix  relevant for .
 Using the precomputed pointers we can enumerate one by one all elements 
 of \cA whose -neighborhood-type is . For each such element
 let  and we enumerate, by induction, the solutions for
 ,
 where  is  with  removed. For each
 solution  obtained by induction, we check whether 
 intersects with  or not (recall that this information has been
 precomputed during the first phase and therefore requires only constant time).
 If it does not, we have a solution
  for  because of \eqref{eq-decompose}. If it does then
 we move to the next solution to . Notice that the size of  is bounded
 by  hence the length of false hits is bounded by . As we
 consider only relevant sequences of pairs, for each  we are certain to find
 at least one matching  that gives us a solution  to .
 Altogether we get the desired constant delay for the enumeration process.

 The enumeration phase needs to process all possible -partitions  and all
 relevant sequences of , i.e. a number of cases triply
 exponential in . Note that each such choice yields disjoint solution
 sets and can therefore be considered sequentially. Altogether this yields a
 procedure linear in the size of the output and triply exponential in . \qed

\section{Conclusion}

We have given a new proof of the linear time and constant delay
enumeration problem of first-order queries over structures of bounded
degree.  Our procedure is based on Gaifman's locality theorem for
first-order logic and our constants are triply exponential in the size
of the query, and therefore induces the known complexity of the
associated model checking problem.

\section*{Acknowledgement}
The authors wish to thank Dietrich Kuske and the anonymous referees for their
constructive comments on earlier versions of this paper.

\bibliographystyle{plain}
\bibliography{bibliography}

\end{document}
