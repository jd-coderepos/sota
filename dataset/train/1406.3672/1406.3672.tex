\chapter{Main Results}

Let  be the \emph{square-free}, \emph{monic} and \emph{completely splitting} polynomial which we wish to factor. Before we deal with the notion of \emph{square balanced polynomials} and
their extension we will make a comment about .

\begin{lemma}\label{polyroot}
	Let  be a polynomial of degree bounded by some polynomial in  and . Then the polynomial  may be constructed in time polynomial in 
	and .
\end{lemma}

\begin{proof}
	Given  we can construct its companion matrix . Then  by definition. Since  is bounded by some polynomial in  and  the whole
	operation can be done in time polynomial in  and .
\end{proof}

\begin{lemma}\label{polyrootfactor}
	Let  be the polynomial as defined above. If we can obtain , a non-trivial factor of , then we can obtain a non-trivial factor of  in additional time that is polynomial in  and .
\end{lemma}

\begin{proof}
	It is easy to see that  gives us the required non-trivial factor.
\end{proof}

From these two simple lemmas we conclude that factoring  is equivalent to factoring a polynomial  where the degree of  is bounded by a polynomial in  and . These polynomials form the basis of Gao's
\emph{super square balance} condition in \cite{gao2001deterministic} and Saha's \emph{cross balance condition} in \cite{saha2008factoring}, since the symmetry condition on  can always be extended to a similar symmetry condition
on  (for suitable polynomial ) due to lemmas \ref{polyroot} and \ref{polyrootfactor}. Since we are interested in strengthening this symmetry condition, we will henceforth not talk about these polynomials but will note that
all the following results for  are also applicable to polynomials of the form .

\section{A Stronger Notion of Balanced Polynomials}\label{sec:stronger}
Let  be the  defined in definition \ref{algebraR}. Let
, where  is odd and let  be a \emph{quadratic non-residue} that can be computed effeciently due to ERH (see section~\ref{sec:erh}). Then  is a generator of the \emph{-Sylow} group .
In \cite{gao2001deterministic} Gao gave an algorithm  to compute the square roots of elements in the alegbra , where  is the square root given by the algorithm for any quadratic residue . Then we have the following lemma due to Gao.

\begin{lemma}\label{squareroot}
	Let  such that  where  has odd order. Let , where . Then  iff  and 
	 iff .
\end{lemma}

\begin{proof}The proof can be found in \cite{gao2001deterministic}. We will omit the proof here since we will not be using his algorithm.
\end{proof}

Based upon this algorithm Gao gave the following definition for \emph{square balanced} polynomials.

\begin{definition}\label{squarebalance}
	Let  be a \emph{square-free},\emph{completely splitting} polynomial. Then for each  define the set  as follows:

	

	Then the polynomial  is square-balanced if .
\end{definition}

\noindent
Lemma \ref{squareroot} also gives us this alternative definition of the sets 
:



We will now generalize this definition of the sets  given by Gao and give a ``balance condition'' that is stronger.

\begin{definition}\label{strongerbal}
	Consider the following sequence of sets for 


Then the polynomial  satisfies this stronger ``balance condition'' if .
\end{definition}

Note that Gao's \emph{square balance} condition is essentially  and hence is a special case of definition \ref{strongerbal}. The following lemma is an extension of Gao's 
result on \emph{square balanced} polynomials.

\begin{lemma}\label{g}
	A polynomial  can be factored in deterministic polynomial time under the assumption of ERH, if it does not satisfy the``balance condition'' of definition \ref{strongerbal}.
   \end{lemma}


\begin{proof}
	Let . Consider . We have

	

	We define a sequence of polynomials  in the following fashion:

	

	From definition \ref{strongerbal} it is not too difficult to see that each polynomial ,  corresponds to the sequence of sets ,  since  has its roots the
	set . 
	     Since (by the assumption of ERH) we know , each of these 
	     's may be calculated in time polynomial in  and
	      and the number of such polynomials ( of them) is bounded by . Now suppose if  such that   then the leading coefficient
	     of  is a \emph{zero divisor} in  which gives a decomposition of  over .
\end{proof}

The above lemma leads to algorithm \ref{alg:squarebal} which fails to factor a polynomial  if it satisfies the above stronger notion of symmetry or balance.
A slight modification of the same approach also allows us to infer that the roots of  must have the same \emph{-Sylow} component.

\begin{lemma}
	A polynomial  can be factored in deterministic polynomial time under the assumption of ERH if , or in other
	words if the \emph{-Sylow} component of  differs from that of .
\end{lemma}

\begin{proof}
	Consider a sequence of polynomials  given by:

	

	The only way we do not get a factor of  in this fashion is if ,  is either
	 or . Clearly this implies that the \emph{-Sylow} expansion of every root is the same.
\end{proof}

The notion of lemma \ref{g} leads us to the following algorithm which fails to factor a polynomial  if it satisfies the symmetry condition of definition \ref{strongerbal}.


\begin{algorithm}[H]
	\caption{The Stronger Square Balance Algorithm}
	\label{alg:squarebal}
\begin{algorithmic}
         \State  k   
	 \State  
	 \State  
 \If {} \State  \EndIf
 \If {} \State  \EndIf
 \For {}
	 \For {}
		  \State 
		  \State 
		  \If {} 
		      \If {} 
			\State  
		      \EndIf
		      \State  
		  \EndIf
		  \If {} 
		      \If {} 
		          \State 
		      \EndIf
		      \State  
		  \EndIf
	 \EndFor
\EndFor
\end{algorithmic}
\end{algorithm}

\vskip 1cm
This algorithm terminates in time polynomial in  and  since  and so atmost  of the branches may be explored. The length of each branch is  and
each of the gcd's are calculable in time polynomial in  and . Consider the set . Since  and  differ in atleast one bit of their \emph{-Sylow} expansion we have that
. Consider a polynomial .



These polynomials give rise to a sequence of  matrices  defined below.

\begin{definition}\label{matrixdef}
	 if  is a  root of the polynomial ; otherwise . In other words  iff   and  otherwise.
\end{definition}

Since the polynomial and matrix representation are equivalent, we will talk of the set  interchangeably as consisting of the polynomials  or the matrices .We then have the following two lemmas regarding some
properties of the set .

\begin{lemma} \label{welldef}
	 if  then . Conversely, if  and
	 then .
\end{lemma}

\begin{proof}
	The forward direction follows since if  then by algorithm \ref{alg:squarebal} we conclude that  and  must have the same \emph{-Sylow} component. To see the
	converse note that if  and  do have the same 
	\emph{-Sylow} component, then above algorithm fails to separate them and hence they would belong to the same polynomial in .
\end{proof}


\begin{lemma}\label{transpose}
	If  then .
\end{lemma}

\begin{proof}
	Let   for some . Then there exists some matrix  such that . If  or  do not contain any other non-zero entries besides this then  and we are done.
	Otherwise let  for some . Then we know that  or equivalently  so that from lemma \ref{welldef} we conclude that 
	, so that  and we are done.
\end{proof}


It is also possible to visualize this in a graph-theoretic setting where the vertices
are the set of roots of  and the set  form a disjoint multiset of edges.

\begin{definition}\label{multigraph}
	Let  a set of size  labelled by the integers from  to . Consider  to consist of the matrices  defined in \ref{matrixdef}. Then  is a multigraph on  vertices.
\end{definition}

\begin{lemma}
      If  fails to be factored by algorithm \ref{alg:squarebal} then for every , the restriction of  to  must be regular.
\end{lemma}

\begin{proof}
	This follows from algorithm \ref{alg:squarebal} and lemma \ref{strongerbal}.
\end{proof}


In this graph theoretic setting one can view algorithm \ref{alg:squarebal} in the following fashion. Given this set of graphs we wish to distinguish between and separate the vertices (the roots) in some fashion in order to obtain
a non-trivial
factor of . One possible way would be to look at the orbits of every vertex and compare them by size. Algorithm \ref{alg:squarebal} simply computes out the -dimensional \emph{Weisfeiler-Leman} approximation for the orbit 
of every vertex for each of the graphs ; the set of  colored neighbors of every vertex being the -dimensional approximation of its orbits. If the graph  is not \emph{regular} then
one can separate the roots or equivalently obtain a non-trivial factor of . A natural next step in generalizing this 
algorithm would be to come up with a better approximation for the set of orbits for each vertex and thereby tighten the symmetry condition under which factoring fails. 

\section{Weisfeiler-Leman and Factoring}\label{sec:weisfeiler}

In this section we build on the idea of the preceding section to come up with a better approximation for the size of orbits of each vertex by showing that it is possible to implicitly compute the -dimensional 
\emph{Weisfeiler-Leman} approximation of the orbits. We briefly touch upon the general -dimensional \emph{Weisfeiler-Leman} algorithm before discussing it in the context of polynomial factoring.

\subsection{2-dimensional Weisfeiler Leman}
A more thorough treatment of the general algorithm can be found here \cite{barbados}. Consider a multigraph  ( where  is a set of colors) which satisfies the following properties:

  \begin{enumerate}
	  \item For every , i.e. the colors are \emph{disjoint}.

	  \item  where  is the all 's matrix.

  \end{enumerate}

  We wish to further refine this set into a set of colors  which satisfy the following \emph{well-behaved} property:

 \begin{enumerate}
	 \item The entries of each  come from the set .

	 \item 
	  and ,  being the all 's matrix.

	 
	 \item Let  be any automorphism of the multigraph , i.e. . Then the colors should be unchanged by , i.e. .
		 
 \end{enumerate}

 
 It is clear from the definition that the initial multigraph  satisfies this \emph{well-behaved} property. This gives us the set of \emph{well-behaved} colors for the first iteration. Then the algorithm proceeds in the following
 fashion:

 \begin{algorithm}[H]
	 \caption{The 2-Dimenstional Weisfeiler Leman Algorithm}\label{wl}
 \begin{algorithmic}
     \State 
     \While{True} 
     \State 
      \If {,  for some set  of color indices} 
     \State Output  as the final set of colors
	 \Else 
	         \State  where each  is a  matrix.
		 \State . Let .
		 \While {}
		 \State 
		 \EndWhile
	\EndIf
     \EndWhile
 \end{algorithmic}
\end{algorithm}

This algorithm terminates in time polynomial in , since the set of colors either increases by atleast one or the algorithm terminates and the size of the set of colors is bounded by .
Let  be the final set of colors so obtained. It is clear that this set satisfies condition  and  of the \emph{well-behaved} properties. The next lemma shows that it is \emph{well-behaved}, 
 i.e. it satisfies all the properties.

 
 \begin{lemma}\label{wellbehaved}
	 The set of colors  so obtained by this algorithm is well behaved.
\end{lemma}

\begin{proof}
	Proof is by induction on the iteration steps. Since the set  is \emph{well behaved} the base assertion certainly holds. Suppose the claim holds for the set of colors  at the  step. 
	Then according to the inductive hypothesis , so that . However we have
	 so that . Since each  is a  matrix, by identifying the matrix whose entries are  on both sides we conclude that .
	Upon adding these new matrices let the set be . If  then we replace 
	 and  by ,  and 
	. Consider an edge  and any automorphism . Then since  and  preserves them both we have  preserves . It therefore follows that 
	preserves  and .
\end{proof}

There is another property besides the \emph{well-behavedness} of this set that we need to show. We call a set of colors  closed under taking transposes, iff .
Note that the set  of the graph  is closed under taking transposes due to lemma \ref{transpose}. We then have the following lemma.

\begin{lemma}\label{transposepol}
	If the original set of colors  is closed under taking transposes, then the set  obtained after every iteration of algorithm \ref{wl} remains closed under transposes.
\end{lemma}

\begin{proof}
	The proof is by induction on the iterations of the algorithm. The base case is true by assumption. Suppose at the  iteration this assertion holds. Let the set of colors at that stage be 
	. Consider the additional colors  that are introduced by multiplying  where . Suppose 
	 then we also have  so that if  is added to the set, then so is . Let this new set be . Then this set by the previous
	argument is closed under taking transpose. Suppose that , so that we replace  with ,  and .
	This does not affect the transpose property
	since ,  and . Therefore the set at the  step retains
	its closure property under taking transposes.
\end{proof}

\subsection{Weisfeiler-Leman and Polynomials}
	In this section we illustrate how the steps of the -dimensional \emph{Weisfeiler-Leman} may be carried out with polynomials. Consider the set , the set of polynomials  obtained
	from algorithm \ref{wl}. Let  be such a polynomial, then we have associated with it the matrix  which was defined in definition \ref{matrixdef}.
        In the preceding section we defined the multigraph . We then have the following lemma.

	\begin{lemma}
		The set  is well behaved.
	\end{lemma}


	\begin{proof}
		It is easily seen that every element of  when thought of as a matrix has its entries from  so that condition  of \emph{well behavedness} is satisfied. Further, we claim that for 
		any . Suppose if  is identity then this is clearly true since  is a factor of  and we know that  so that
		. If neither of them are the
		identity matrix then from lemma \ref{welldef} we have that . Further , since 
		. Hence it follows that  so that condition  is satisfied as well. The last condition follows from the defintion of .
	\end{proof}


	Hence one can think of applying the  \emph{Weisfeiler-Leman} algorithm in this case. Note however that we do not explicitly know the matrices  but rather have their polynomial forms .
	Therefore we must show that the algorithm \ref{wl} may be duplicated with just this set of polynomials.

	\subsubsection{Multiplication of Matrices}
	Consider a \emph{well behaved} set of colors  with respect to the complete graph on  vertices. Assume also that this set is closed under taking transposes.	
	Suppose they are given in their polynomial forms, i.e. the polynomial corresponding to  is 
	defined by 

	


        where  depends on . We now wish to show that from their polynomial forms we can recover the polynomial form of the matrix corresponding to  for some  and  in this \emph{well behaved} set. 
	Denote by  the 
	polynomial corresponding to . Consider the algebra , where . Since  is square-free and completely
	splitting over , we have that  is a semisimple -algebra. Let its primitive idempotents over  be , where . Suppose
	 splits as below over :
	
	


	where each of the  and hence can be written as . It then follows that . It also follows from the expression 
	for  above that the set  is a permutation of . Let the inverse permutation of indices be denoted by 
	, i.e. , where  and  indexes  for a fixed . Then  may equivalently be written as 

	

	where the set  is the same as  . Consider the polynomial ring  and let . Since the expression for  is

	

	Hence the expression for  is given by 

	

	Consider now the polynomial .

			        


	Where the last couple of steps follow from the previous step because  and , while , otherwise  being just . Note that the polynomial
	corresponding to  (denoted by ) would be of the form:

	
 

	We wish to obtain  from the polynomial . This is obtained by eliminating  from  as follows. Consider the ring  and . Consider  from which we construct the ring  where .
	Let  be the characteristic polynomial of  over the ring . Then we have the following lemma:

	\begin{lemma}
		 where  is the polynomial defined in lemma \ref{g}. In other words  may be obtained by merely substituting 
		for  in this  polynomial.
	\end{lemma}


	\begin{proof}
		Let the primitive idempotents of  over  be  ( is \emph{semisimple} over  and the primitive idempotents of  over 
		 be . We wish to consider  as a polynomial in ; hence expressing every  in terms of 's is enough to express 
		as a member of this ring. We have

		

		where  

		

		The expression for  follows from evaluating it at  and . This then gives us the expression for  as follows

		

		Observe that for any ,  for . Therefore we have that 

		
	\end{proof}


	\subsubsection{The Remaining Steps}
	Now that we have the polynomial  corresponding to  we need to show that we can express the product as  where each  has entries from the set . This is however easy to reproduce: 
	expressing  where each  is \emph{square-free} and \emph{mutually prime} gives us the polynomials corresponding to . Further we have the following relations

	

	where we use the  symbol to denote the equivalence between a color matrix and its corresponding polynomial form. Note also that from lemma \ref{transposepol} that at every step of the iteration the set 
	is closed under taking transposes, so that for every color polynomial  we may also find . Hence it is possible to repeat this process till we achieve a stabilization of the
	colors. This gives rise to the following algorithm for the polynomials.

	\begin{algorithm}[H]
		\caption{2D WL for Polynomials}
		\label{polywl}
		\begin{algorithmic}
			\State 
			\While{True}
			\If{\State  for some index  of }
			   \State Output  as the final set
			\Else
			   \State  where each  is \emph{square-free} and \emph{mutually prime} to each other
			   \State 
			   \While{}
			         \State 
				 \State 
			   \EndWhile
			\EndIf
			\EndWhile
		\end{algorithmic}
	\end{algorithm}

			
         Note that given two polynomials  and , the polynomial representation for  is computable in time polynomial in the degrees of  and . Since the degree of each  is always bounded above
	 by , algorithm \ref{polywl} terminates in time polynomial in  and . 
	 
	 
	 \subsection{Colors and Schemes}\label{sec:scheme}
	 Consider the final set of colors (or polynomials)  we obtain from algorithm \ref{polywl}. In this section we will prove some simple properties about this set .

	 \begin{lemma}\label{thin}
		 . Further if  then  may be factored.
	 \end{lemma}

	 \begin{proof}
		The first part follows because , since for any  that are roots of ,  and  differ in their \emph{-Sylow} expansion in atleast one place
		- namely, the most significant bit. This is because . Since  the first inequality follows. The second inequality 
		is trivial since the product of all the polynomials in  is  (or equivalently, ). If  then the degree of any polynomial
		 is  which gives us an endomorphism of the roots of  and hence  may be factored by \cite{evdokimov}.
	\end{proof}


	\begin{lemma} \label{identity}
		If  then  may be factored.
	\end{lemma}

	\begin{proof}
			Since the starting set contained , the only way  would be that  decomposes into two or more colors at some stage. Let  be the polynomial form of :
			
			
			Suppose a color  is a non-trivial decomposition of , i.e.  and . Then since  the degree of , . Also since ,
			. Hence the leading coefficient of  will be a zero divisor in  and we get a decomposition of .

		\end{proof}
			

		\begin{lemma}\label{regular}
		 Let  be any color in the set . If  then  may be factored. In other words, for
		  not to be factored each color  must be regular.
	 \end{lemma}


	 \begin{proof}
		 Consider the polynomial form of such a . 

		 

		 Then it is easy to see that . If , then the leading coefficient of  is a zero-divisor in the algebra  and hence
		 we obtain a decomposition of .
	 \end{proof}

	 It is possible to strengthen this lemma and prove that the colors are not just \emph{regular}, but also \emph{strongly regular}. We prove this in the following lemma.

	 \begin{lemma}\label{intersectionnum}
		 Consider any three colors . Consider any . Then the cardinality of the set  is independent of the choice of the edge .
		 This cardinality will henceforth be denoted by .
	 \end{lemma}
	
	\begin{proof}
		For any  we know that the set  is closed under their multiplication, i.e.  where  is a positive integer. It is easy
		to see from this expansion that .
	\end{proof}

	Note that lemma \ref{regular} is a special case of lemma \ref{intersectionnum}, where the out-degree of color  is given by  where  denotes the identity color . This brings us to the notion of 
	\emph{Association schemes} which we introduce below:

	\begin{definition}\label{scheme} Let  be a finite, non-empty set and let  be a set of relations on . Then the pair  form an association scheme if
		\begin{enumerate}
			\item  is a partition of .
			\item For all , .
			\item 
			\item  there is a number  such that for all , .
		\end{enumerate}
	\end{definition}



	\begin{example}\label{schurian}
			An important class of association schemes called \emph{Schurian} or \emph{group case} schemes arise from the -orbits of a transitive group action. Let  be a group and let  be a subgroup.
			Then the action of  on the cosets of  by left multiplication is transitive. Consider the action of  on  where . Then the orbits of this action form an association scheme. To see this observe that they do partition . Further  is
			one of the orbits (since  acts 
			transitively on the cosets of ). Consider a -orbit  then it's transpose  is also a -orbit. If  then  so that
			 which implies that . Suppose  are any three of these -orbits. Let  so that . Then there is a
			bijection between  given by . Thus the set of -orbits under this action form an association scheme 
			denoted by .
	\end{example}


	
	Let  where . We then make the following claim:
	\begin{lemma}
		If after the algorithm \ref{polywl}  remains unfactored, then the tuple  is an association scheme.
	 \end{lemma}

	\begin{proof}
		Condition  follows from the well-behavedness of the set  (see lemma \ref{wellbehaved}). Condition  follows from lemma \ref{transposepol}. Condition  follows from lemma \ref{identity}. Finally,
		condition  follows because of lemma \ref{intersectionnum}.
	\end{proof}




	\subsection{Closed Subsets}\label{sec:subset}
      From the definition of schemes it follows that schemes are a generalization of groups, i.e. all groups are schemes. This can be seen in the following fashion. Consider a group . For each element 
      we define . Let . Then the set  forms a scheme with  if  and  otherwise. If  then this scheme has  colors and is
      referred to as a \emph{thin scheme}. Notice that by lemma \ref{thin} \emph{thin schemes} are easily factored. 

      In this section we will talk about \emph{closed subsets} which are a generalization of subgroups to the scheme structure and prove some result about the closed subsets of the scheme  from the previous 
      section. We will using the notation of Zieschang's book on Association Schemes \cite{zieschang}; a more detailed discussion on schemes and closed subsets can also be found there. If  is a scheme and 
      and  then by  we mean the set .


      \begin{definition}
	      Let  be any arbitrary scheme. A nonempty subset  of  is called closed if , where .
      \end{definition}

      Note that  which implies that . Further , which implies that . Hence .	Seen in this fashion the choice of the name \emph{closed subsets}
      and the link with subgroups becomes clear. 

      \begin{definition}
	 If  where  is a scheme, then we define .
 \end{definition}


 	We then have a following simple lemma.

	\begin{lemma}\label{partition}
		If  then  is closed if and only if the set  is a partition of .
	\end{lemma}

	\begin{proof}
		To see the forward direction we note that the relation  is an equivalence relation on . Every  belongs to  since , so . If  then  such 
		that . Since ,  so that . If  and  then  since . Hence if  is closed,  is a partition of .
		The converse follows in a similar fashion: if  then  (since  is a partition) so that . From transitivity it follows that  so that  is \emph{closed}.
	\end{proof}

	\begin{definition}\label{primitivescheme}
		An association scheme  is called primitive if its only closed subsets are  and  itself.
        \end{definition}

	
	From the definition it is clear that \emph{primitive schemes} are generalizations of \emph{prime groups} - groups which have no non-trivial subgroup. In the case of groups we know that groups with no proper subgroups
	are precisely the prime groups - cyclic (hence commutative) groups of prime order.	
	Let  and . Therefore since  is a partition of  we have from lemma \ref{partition}
	. It is then obvious that schemes on a prime number of vertices (,  prime) must be \emph{primitive}. From \cite{Hanaki:2006} we also know that association schemes of prime order are commutative. It is 
	tempting to extend the analogy with groups and conjecture that \emph{primitive schemes} would always have a prime order (or atleast be commutative/cyclic). However this is \emph{false} (for an argument see Appendix
	\ref{sec:counterexample}).
	
	Consider the scheme  from the previous section. We wish to prove that the
	problem of factoring the polynomial  (and its associated scheme ) can always be reduced to factoring a polynomial of degree  whose associated scheme is \emph{primitive}. Before that we
	need a definition and a minor lemma.

 	\begin{definition}
		Let  be an arbitrary scheme and  any subset of . Then by  we denote the intersection of all closed subsets of  which contain .
	\end{definition}

	We set  and inductively define . Then we have the following

	\begin{lemma}
		The set  is the union of all sets  where  is a non-negative integer.
	\end{lemma}

	\begin{proof}
		Let  and let . We wish to show . Since , we have  since . Thus for any non-negative integers  we have

		

		Therefore  is closed. Further since , we must have  by definition. The converse follows because

		

		Hence .
	\end{proof}

	Note that when generating  by taking the union of  one only needs to go till . This is because at every step the size of this set grows by at least one and it cannot grow beyond
	, hence it must stabilize for some . We then come to the main lemma.

	\begin{theorem}\label{primitive}
		Let the association scheme we get from algorithm \ref{polywl} on the polynomial  be . Here . Then the problem of factoring  over  may be 
		polynomial time reduced	to that of factoring a polynomial ,  whose association scheme  is primitive.
	\end{theorem}

	\begin{proof}If the scheme  is \emph{primitive} then we are already done. Hence, suppose that  is not \emph{primitive}. Then we first claim that it is possible to find a non-trivial closed
		subset of  in time polynomial in  and . This is done by considering  for all such . Since  and to construct  we only need to take powers
		of  till at most  we can construct all such 's in time . Now suppose all of these sets are either  then we claim that  must be \emph{primitive}.
		Suppose not, then there exists a set ,  which is closed. Let  where  (such a  exists). Then by definition
		 and hence we arrive at a contradiction. Thus given  we either determine that it is primitive, or in polynomial time construct a \emph{closed
		subset} of this scheme. Let  be the closed subset so obtained. Define by  the following polynomial:

		

		Where  and . From lemma \ref{partition} we know that  is a partition of . Hence , 
		. If  then  so that  such that the coefficient of  in
		 is different from that of . Suppose the coefficient of such a  is . Then the resultant polynomial  has
		at most  distinct roots, so that on making it square-free we get a polynomial  over  whose degree is at most . Finding a root  of  then gives us a zero
		divisor in the algebra , namely . Also note that finding such a  and its coefficient is easy, all we need to do is ensure . Note that we may assume that
		the scheme  formed by  is \emph{primitive} or we can again reduce it in the above fashion.
	\end{proof}
	
	
\begin{comment}
\subsection{The Scheme }
In this section we will study some of the properties particular to the scheme .

\begin{lemma}\label{nonsymmetric}
	. In other words none of the elements of the scheme  is symmetric.
\end{lemma}

\begin{proof}
	Suppose  such that . This implies that if  then , so that  is even. By lemma \ref{regular} we have . Since  is odd
	(otherwise  may be factored over ) we infer that  must be even. By \emph{Evdokimov}'s method \cite{evdokimov}, we can factor  where each  is odd. Notice that 
	we can replace the polynomial  by this set of polynomials  in  (algorithm \ref{polywl}) without violating the \emph{well-behaved} property of . 
\end{proof}

\section{The Bose-Messner Algebra}
We will conclude this chapter by discussing some properties of the algebra formed by the adjacency matrices

We will conclude this chapter by discussing some properties of the algebra formed by the adjacency matrices of an association scheme . The algebra 




We will end by discussing some properties of the algebra formed by the adjacency matrices of an association scheme . The algebra  considered as a subalgebra of  is called the 
\emph{Bose-Messner algebra}. By property  of definition \ref{scheme} we know that  is a \emph{self-adjoint algebra} over  so that it is \emph{semisimple}. This gives us a decomposition of this
algebra into its simple modules.



Where  is a \emph{simple module} generated by a primitive idempotent in the center of  and by Wedderburn's theorem is isomorphic to  where  is the
\emph{degree} of the character . 
\end{comment}
