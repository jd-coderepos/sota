\chapter{Main Results}

Let $f=\prod_{i=1}^n (x-\xi_i) \in \mathbb{F}_p[x]$ be the \emph{square-free}, \emph{monic} and \emph{completely splitting} polynomial which we wish to factor. Before we deal with the notion of \emph{square balanced polynomials} and
their extension we will make a comment about $f$.

\begin{lemma}\label{polyroot}
	Let $q \in \mathbb{F}_p[x]$ be a polynomial of degree bounded by some polynomial in $n$ and $\log{p}$. Then the polynomial $f_q(x) = \prod_{i=1}^n (x-q(\xi_i)) \in \mathbb{F}_p[x]$ may be constructed in time polynomial in $n$
	and $\log{p}$.
\end{lemma}

\begin{proof}
	Given $f$ we can construct its companion matrix $C_f\in Mat_n(\mathbb{F}_p)$. Then $f_q(x) = det(xI-q(C_f))$ by definition. Since $deg(q)$ is bounded by some polynomial in $n$ and $\log{p}$ the whole
	operation can be done in time polynomial in $n$ and $\log{p}$.
\end{proof}

\begin{lemma}\label{polyrootfactor}
	Let $f_q \in \mathbb{F}_p[x]$ be the polynomial as defined above. If we can obtain $g_q$, a non-trivial factor of $f_q$, then we can obtain a non-trivial factor of $f$ in additional time that is polynomial in $n$ and $\log{p}$.
\end{lemma}

\begin{proof}
	It is easy to see that $\gcd(g_q(q(x)),f)$ gives us the required non-trivial factor.
\end{proof}

From these two simple lemmas we conclude that factoring $f$ is equivalent to factoring a polynomial $f_q$ where the degree of $q$ is bounded by a polynomial in $n$ and $\log{p}$. These polynomials form the basis of Gao's
\emph{super square balance} condition in \cite{gao2001deterministic} and Saha's \emph{cross balance condition} in \cite{saha2008factoring}, since the symmetry condition on $f$ can always be extended to a similar symmetry condition
on $f_q$ (for suitable polynomial $q$) due to lemmas \ref{polyroot} and \ref{polyrootfactor}. Since we are interested in strengthening this symmetry condition, we will henceforth not talk about these polynomials but will note that
all the following results for $f$ are also applicable to polynomials of the form $f_q$.

\section{A Stronger Notion of Balanced Polynomials}\label{sec:stronger}
Let $\mathcal{R}$ be the $\mathbb{F}_p-algebra$ defined in definition \ref{algebraR}. Let
$p-1=2^rw$, where $w$ is odd and let $\gamma$ be a \emph{quadratic non-residue} that can be computed effeciently due to ERH (see section~\ref{sec:erh}). Then $\eta=\gamma^w$ is a generator of the \emph{$2$-Sylow} group $\mathbb{F}_p^\times$.
In \cite{gao2001deterministic} Gao gave an algorithm $\sigma$ to compute the square roots of elements in the alegbra $\mathcal{R}$, where $\sigma(a)$ is the square root given by the algorithm for any quadratic residue $a \in
\mathcal{R}$. Then we have the following lemma due to Gao.

\begin{lemma}\label{squareroot}
	Let $a \in \mathbb{F}_p$ such that $a=\eta^u\theta$ where $\theta$ has odd order. Let $u = \sum_{i=0}^{r-1} u_i2^i$, where $\forall 1 \le i \le r-1, u_i \in \{0,1\}$. Then $\sigma(a^2)=a$ iff $u_{r-1}=0$ and 
	$\sigma(a^2)=-a$ iff $u_{r-1}=1$.
\end{lemma}

\begin{proof}The proof can be found in \cite{gao2001deterministic}. We will omit the proof here since we will not be using his algorithm.
\end{proof}

Based upon this algorithm Gao gave the following definition for \emph{square balanced} polynomials.

\begin{definition}\label{squarebalance}
	Let $f=\prod_{i=1}^n (x-\xi_i) \in \mathbb{F}_p[x]$ be a \emph{square-free},\emph{completely splitting} polynomial. Then for each $1 \le i \le n$ define the set $D_i$ as follows:

	\begin{align*}
		D_i &= \left\{\xi_j \mid \xi_i\neq \xi_j,
		\sigma( (\xi_i-\xi_j)^2)=\xi_i-\xi_j\right\}
	\end{align*}

	Then the polynomial $f$ is square-balanced if $\forall 1 \le i,j \le n, |D_i|=|D_j|=\frac{n-1}{2}$.
\end{definition}

\noindent
Lemma \ref{squareroot} also gives us this alternative definition of the sets 
$D_i$:

\begin{align*}
	D_i &= \left\{\xi_j |\xi_i \neq \xi_j,(\xi_i-\xi_j)= \eta^u \theta,2\nmid o(\theta),u=\sum_{k=0}^{r-1}u_k2^k,u_{r-1}=0\right\}.
\end{align*}

We will now generalize this definition of the sets $D_i$ given by Gao and give a ``balance condition'' that is stronger.

\begin{definition}\label{strongerbal}
	Consider the following sequence of sets for $0 \le k \le r-1$
\begin{align*}
	D^k_i &= \left\{\xi_j \mid\xi_i \neq \xi_j,(\xi_i-\xi_j)=\eta^u 
	\theta,2\nmid o(\theta),u=\sum_{k=0}^{r-1}u_k2^k,u_{k}=0\right\}.\\
\end{align*}

Then the polynomial $f$ satisfies this stronger ``balance condition'' if $\forall 1 \le i,j \le n, \forall 0 \le k \le r-1, |D^k_i|=|D^k_j|$.
\end{definition}

Note that Gao's \emph{square balance} condition is essentially $\forall 1 \le i,j \le n,|D^{r-1}_i|=|D^{r-1}_j|$ and hence is a special case of definition \ref{strongerbal}. The following lemma is an extension of Gao's 
result on \emph{square balanced} polynomials.

\begin{lemma}\label{g}
	A polynomial $f \in \mathbb{F}_p[x]$ can be factored in deterministic polynomial time under the assumption of ERH, if it does not satisfy the``balance condition'' of definition \ref{strongerbal}.
   \end{lemma}


\begin{proof}
	Let $X \equiv x \pmod{f}$. Consider $g(y,x)=\frac{f(-y+X)}{-y} \in \mathcal{R}[y]$. We have

	\begin{align*}
		g(y,x) &= \frac{f(-y+X)}{-y} = \sum_{i=1}^n \prod_{j\neq i} (y-(\xi_i-\xi_j)) \mu_i
	\end{align*}

	We define a sequence of polynomials $d_0,\cdots,d_k,\cdots,d_{r-1}$ in the following fashion:

	\begin{align*}
	       d_k &= \gcd\left(\prod_{i=0}^{2^k-1} \left(y^{\frac{p-1}{2^{k+1}}}-
	       \eta^i\right),g\right) = 
	       \sum_{i=1}\prod_{j\in D^k_i}(y-(\xi_i-\xi_j))\mu_i
	     \end{align*}

	From definition \ref{strongerbal} it is not too difficult to see that each polynomial $d_k$, $0 \le k \le r-1$ corresponds to the sequence of sets $D^k_i$, $1 \le i \le n$ since $d_k(-y+X)\mu_i$ has its roots the
	set $D^k_i$. 
	     Since (by the assumption of ERH) we know $\eta$, each of these 
	     \(\gcd\)'s may be calculated in time polynomial in $n$ and
	     $\log{p}$ and the number of such polynomials ($2r$ of them) is bounded by $O(\log{p})$. Now suppose if $\exists 0 \le k \le r-1,\exists 1\le i,j \le n$ such that  $|D^k_i|\neq |D^k_j|$ then the leading coefficient
	     of $d_k$ is a \emph{zero divisor} in $\mathcal{R}$ which gives a decomposition of $f$ over $\mathbb{F}_p$.
\end{proof}

The above lemma leads to algorithm \ref{alg:squarebal} which fails to factor a polynomial $f \in \mathbb{F}_p[x]$ if it satisfies the above stronger notion of symmetry or balance.
A slight modification of the same approach also allows us to infer that the roots of $f$ must have the same \emph{$2$-Sylow} component.

\begin{lemma}
	A polynomial $f = \prod_{i=1}^n (x-\xi_i) \in \mathbb{F}_p[x]$ can be factored in deterministic polynomial time under the assumption of ERH if $\exists i,j \in \{1,\cdots,n\},\xi_i^w \neq \xi_j^w$, or in other
	words if the \emph{$2$-Sylow} component of $\xi_i$ differs from that of $\xi_j$.
\end{lemma}

\begin{proof}
	Consider a sequence of polynomials $s_0,s_1,\cdots,s_k,\cdots,s_{r-1} \in \mathbb{F}_p[x]$ given by:

	\begin{align*}
		s_k &= \gcd\left(\prod_{i=0}^{2^k-1} \left(x^{\frac{p-1}{2^{k+1}}}-\eta^i
		\right), f \right) 
	\end{align*}

	The only way we do not get a factor of $f$ in this fashion is if $\forall 1 \le k \le r-1$, $s_k$ is either
	$1$ or $f$. Clearly this implies that the \emph{$2$-Sylow} expansion of every root is the same.
\end{proof}

The notion of lemma \ref{g} leads us to the following algorithm which fails to factor a polynomial $f$ if it satisfies the symmetry condition of definition \ref{strongerbal}.


\begin{algorithm}[H]
	\caption{The Stronger Square Balance Algorithm}
	\label{alg:squarebal}
\begin{algorithmic}
         \State  k $\leftarrow$ $0$ 
	 \State  $g_0 \leftarrow \gcd(y^{\frac{p-1}{2}}-1,g)$
	 \State  $g_1 \leftarrow \gcd(y^{\frac{p-1}{2}}+1,g)$
 \If {$g_0\neq 1$} \State $(0)\in S_0$ \EndIf
 \If {$g_1\neq 1$} \State $(1)\in S_0$ \EndIf
 \For {$1 \le k \le r-1$}
	 \For {$(u_0,\cdots,u_{k-1})\in S^{k-1}$}
		  \State $g_0 \leftarrow \gcd(y^{\frac{p-1}{2^{k+1}}}-\eta^{\sum_{j=0}^{k-1}u_{j}2^{r-(k-j+1)}},g_{k-1})$
		  \State $g_1 \leftarrow \gcd(y^{\frac{p-1}{2^{k+1}}}-\eta^{\sum_{j=0}^{k-1}u_{j}2^{r-(k-j+1)}+2^{r-1}},g_{k-1})$
		  \If {$g_0 \neq 1$} 
		      \If {$k=r-1$} 
			\State $g_0 \in \mathcal{E}$ 
		      \EndIf
		      \State $(u_0,\cdots,u_{k-1},0) \in S_k$ 
		  \EndIf
		  \If {$g_1 \neq 1$} 
		      \If {$k=r-1$} 
		          \State $g_1 \in \mathcal{E}$
		      \EndIf
		      \State $(u_0,\cdots,u_{k-1},1) \in S_k$ 
		  \EndIf
	 \EndFor
\EndFor
\end{algorithmic}
\end{algorithm}

\vskip 1cm
This algorithm terminates in time polynomial in $n$ and $\log{p}$ since $|\mathcal{E}|=|S_{r-1}| \le n-1$ and so atmost $n-1$ of the branches may be explored. The length of each branch is $r \le \log{p}$ and
each of the gcd's are calculable in time polynomial in $n$ and $\log{p}$. Consider the set $\mathcal{E}$. Since $\xi_i-\xi_j$ and $\xi_j-\xi_i$ differ in atleast one bit of their \emph{$2$-Sylow} expansion we have that
$2 \le |\mathcal{E}| \le n-1$. Consider a polynomial $g_l \in \mathcal{E}$.

\begin{align*}
	g_l = \sum_{i=1}^n \prod_{j} (y-(\xi_i-\xi_j)) \mu_i
\end{align*}

These polynomials give rise to a sequence of $n \times n$ matrices $\{E_l\}_{l=1}^{|\mathcal{E}|}$ defined below.

\begin{definition}\label{matrixdef}
	$E_l(i,j)=1$ if $\xi_i-\xi_j$ is a $\mathbb{F}_p$ root of the polynomial $g_l\mu_i \in \mathcal{R}[y]$; otherwise $E_l(i,j)=0$. In other words $E_l(i,j)=1$ iff  $g_l(\xi_i-\xi_j)\mu_i=0$ and $E_l(i,j)=0$ otherwise.
\end{definition}

Since the polynomial and matrix representation are equivalent, we will talk of the set $\mathcal{E}$ interchangeably as consisting of the polynomials $g_l$ or the matrices $E_l$.We then have the following two lemmas regarding some
properties of the set $\mathcal{E}$.

\begin{lemma} \label{welldef}
	$\forall 1\le i,j,s,t \le n$ if $E_l(i,j)=E_l(s,t)=1$ then $(\xi_i-\xi_j)^w=(\xi_s-\xi_t)^w$. Conversely, if $E_l(i,j)=E_m(s,t)=1$ and
	$(\xi_i-\xi_j)^w=(\xi_s-\xi_t)^w$ then $E_l=E_m$.
\end{lemma}

\begin{proof}
	The forward direction follows since if $g_l(\xi_i-\xi_j)=g_l(\xi_s-\xi_t)=0$ then by algorithm \ref{alg:squarebal} we conclude that $\xi_i-\xi_j$ and $\xi_s-\xi_t$ must have the same \emph{$2$-Sylow} component. To see the
	converse note that if $\xi_i-\xi_j$ and $\xi_s-\xi_t$ do have the same 
	\emph{$2$-Sylow} component, then above algorithm fails to separate them and hence they would belong to the same polynomial in $\mathcal{E}$.
\end{proof}


\begin{lemma}\label{transpose}
	If $E_l \in \mathcal{E}$ then $E_l^\top \in \mathcal{E}$.
\end{lemma}

\begin{proof}
	Let  $E_l(i,j)=1$ for some $1 \le i,j \le n$. Then there exists some matrix $E_m$ such that $E_m(j,i)=1$. If $E_l$ or $E_m$ do not contain any other non-zero entries besides this then $E_m=E_l^\top$ and we are done.
	Otherwise let $E_l(s,t)=1$ for some $1 \le s,t \le n,s\neq i,t\neq j$. Then we know that $(\xi_i-\xi_j)^w=(\xi_s-\xi_t)^w$ or equivalently $(\xi_j-\xi_i)^w=(\xi_t-\xi_s)^w$ so that from lemma \ref{welldef} we conclude that 
	$E_m(s,t)=1$, so that $E_m=E_l^\top$ and we are done.
\end{proof}


It is also possible to visualize this in a graph-theoretic setting where the vertices
are the set of roots of $f \in \mathbb{F}_p[x]$ and the set $\mathcal{E}$ form a disjoint multiset of edges.

\begin{definition}\label{multigraph}
	Let $V$ a set of size $n$ labelled by the integers from $1$ to $n$. Consider $\mathcal{E}$ to consist of the matrices $E_l$ defined in \ref{matrixdef}. Then $G_f=(V,\mathcal{E})$ is a multigraph on $n$ vertices.
\end{definition}

\begin{lemma}
      If $f$ fails to be factored by algorithm \ref{alg:squarebal} then for every $E_l \in \mathcal{E}$, the restriction of $G_f$ to $(V,E_l)$ must be regular.
\end{lemma}

\begin{proof}
	This follows from algorithm \ref{alg:squarebal} and lemma \ref{strongerbal}.
\end{proof}


In this graph theoretic setting one can view algorithm \ref{alg:squarebal} in the following fashion. Given this set of graphs we wish to distinguish between and separate the vertices (the roots) in some fashion in order to obtain
a non-trivial
factor of $f$. One possible way would be to look at the orbits of every vertex and compare them by size. Algorithm \ref{alg:squarebal} simply computes out the $1$-dimensional \emph{Weisfeiler-Leman} approximation for the orbit 
of every vertex for each of the graphs $(V,E_l), 1 \le l \le |\mathcal{E}|$; the set of $E_l$ colored neighbors of every vertex being the $1$-dimensional approximation of its orbits. If the graph $E_l$ is not \emph{regular} then
one can separate the roots or equivalently obtain a non-trivial factor of $f$. A natural next step in generalizing this 
algorithm would be to come up with a better approximation for the set of orbits for each vertex and thereby tighten the symmetry condition under which factoring fails. 

\section{Weisfeiler-Leman and Factoring}\label{sec:weisfeiler}

In this section we build on the idea of the preceding section to come up with a better approximation for the size of orbits of each vertex by showing that it is possible to implicitly compute the $2$-dimensional 
\emph{Weisfeiler-Leman} approximation of the orbits. We briefly touch upon the general $2$-dimensional \emph{Weisfeiler-Leman} algorithm before discussing it in the context of polynomial factoring.

\subsection{2-dimensional Weisfeiler Leman}
A more thorough treatment of the general algorithm can be found here \cite{barbados}. Consider a multigraph $G=(V,\mathcal{E})$ ( where $\mathcal{E}$ is a set of colors) which satisfies the following properties:

  \begin{enumerate}
	  \item For every $E_i,E_j \in \mathcal{E},i\neq j, E_i \cap E_j = 
	  \emptyset$, i.e. the colors are \emph{disjoint}.

	  \item $\displaystyle\sum_{i:E_i\in \mathcal{E}} E_i = J$ where $J$ is the all $1$'s matrix.

  \end{enumerate}

  We wish to further refine this set into a set of colors $\mathcal{S}$ which satisfy the following \emph{well-behaved} property:

 \begin{enumerate}
	 \item The entries of each $S_i \in \mathcal{S}$ come from the set $\{0,1\}$.

	 \item $\forall S_i,S_j \in \mathcal{S}, i \neq j, S_i \cap S_j = \emptyset$
	  and $\displaystyle\sum_{i:S_i\in \mathcal{S}}^k S_i =J$, $J$ being the all $1$'s matrix.

	 
	 \item Let $P$ be any automorphism of the multigraph $(V,\mathcal{E})$, i.e. $\forall E_i \in \mathcal{E}, PE_iP^{-1}=E_i$. Then the colors should be unchanged by $P$, i.e. $PS_iP^{-1}=S_i,\forall S_i \in \mathcal{S}$.
		 
 \end{enumerate}

 
 It is clear from the definition that the initial multigraph $G$ satisfies this \emph{well-behaved} property. This gives us the set of \emph{well-behaved} colors for the first iteration. Then the algorithm proceeds in the following
 fashion:

 \begin{algorithm}[H]
	 \caption{The 2-Dimenstional Weisfeiler Leman Algorithm}\label{wl}
 \begin{algorithmic}
     \State $\mathcal{C} \leftarrow \mathcal{E}$
     \While{True} 
     \State $k \leftarrow |\mathcal{C}|$
      \If {$\forall C_i,C_j \in \mathcal{C}$, $C_iC_j= \sum_{l=1}^k lC_{\alpha_l}$ for some set $\{\alpha_l\}$ of color indices} 
     \State Output $\mathcal{C}$ as the final set of colors
	 \Else 
	         \State $C_iC_j=\sum_{l=1}^k lD_l$ where each $D_l$ is a $\{0,1\}$ matrix.
		 \State $\mathcal{C} \leftarrow \mathcal{C} \cup_{l=1}^k \{D_l\}$. Let $\mathcal{C} = \{C'_1,\cdots,C'_{k'}\}$.
		 \While {$\exists 1\le i,j \le k',i \neq j, C'_i \cap C'_j \neq 
		 \emptyset$}
		 \State $\mathcal{C} \leftarrow (\mathcal{C}\setminus \{C'_i,C'_j\}) \cup \{D'_1=C'_i\setminus C'_j,D'_2=C'_j\setminus C'_i,D'_3=C'_i \cap C'_j\}$
		 \EndWhile
	\EndIf
     \EndWhile
 \end{algorithmic}
\end{algorithm}

This algorithm terminates in time polynomial in $n$, since the set of colors either increases by atleast one or the algorithm terminates and the size of the set of colors is bounded by $n$.
Let $\mathcal{C}$ be the final set of colors so obtained. It is clear that this set satisfies condition $1$ and $2$ of the \emph{well-behaved} properties. The next lemma shows that it is \emph{well-behaved}, 
 i.e. it satisfies all the properties.

 
 \begin{lemma}\label{wellbehaved}
	 The set of colors $\mathcal{C}$ so obtained by this algorithm is well behaved.
\end{lemma}

\begin{proof}
	Proof is by induction on the iteration steps. Since the set $\mathcal{E}$ is \emph{well behaved} the base assertion certainly holds. Suppose the claim holds for the set of colors $\mathcal{S}$ at the $i^{th}$ step. 
	Then according to the inductive hypothesis $\forall P\in Aut(G),\forall S_i \in \mathcal{S}, PS_iP^{-1}=S_i$, so that $\forall S_i,S_j \in \mathcal{S}, PS_iS_jP^{-1}=PS_iP^{-1}PS_jP^{-1}=S_iS_j$. However we have
	$S_iS_j = \sum_{l=1}^k lD_l$ so that $\sum_{l=1}^k lPD_lP^{-1}=\sum_{l=1}^klD_l$. Since each $D_l$ is a $0/1$ matrix, by identifying the matrix whose entries are $l$ on both sides we conclude that $PD_lP^{-1}=D_l$.
	Upon adding these new matrices let the set be $\mathcal{S}'$. If $\exists S'_i,S'_j \in \mathcal{S}',S'_i \cap S'_j\neq \emptyset$ then we replace $S'_i$
	 and $S'_j$ by $D'_1=S'_i\setminus S'_j$, $D'_2=S'_j\setminus S'_i$ and 
	$D'_3=S'_i\cap S'_j$. Consider an edge $(u,v) \in D'_3$ and any automorphism $P \in Aut(G)$. Then since $(u,v) \in S'_i,S'_j$ and $P$ preserves them both we have $P$ preserves $D'_3$. It therefore follows that $P$
	preserves $D'_1=S'_i\setminus D'_3$ and $D'_2=S'_j\setminus D'_3$.
\end{proof}

There is another property besides the \emph{well-behavedness} of this set that we need to show. We call a set of colors $\mathcal{S}$ closed under taking transposes, iff $S_i \in \mathcal{S} \Rightarrow S_i^\top \in \mathcal{S}$.
Note that the set $\mathcal{E}$ of the graph $G_f$ is closed under taking transposes due to lemma \ref{transpose}. We then have the following lemma.

\begin{lemma}\label{transposepol}
	If the original set of colors $\mathcal{E}$ is closed under taking transposes, then the set $\mathcal{S}$ obtained after every iteration of algorithm \ref{wl} remains closed under transposes.
\end{lemma}

\begin{proof}
	The proof is by induction on the iterations of the algorithm. The base case is true by assumption. Suppose at the $t^{th}$ iteration this assertion holds. Let the set of colors at that stage be 
	$\mathcal{S}$. Consider the additional colors $D_l$ that are introduced by multiplying $S_iS_j$ where $S_i,S_j \in \mathcal{S}$. Suppose 
	$S_iS_j=\sum_{l}lD_l$ then we also have $S_i^\top S_j^\top=\sum_{l}lD_l^\top$ so that if $D_l$ is added to the set, then so is $D_l^\top$. Let this new set be $\mathcal{T}$. Then this set by the previous
	argument is closed under taking transpose. Suppose that $\exists T_i,T_j \in \mathcal{T},T_i \cap T_j \neq \emptyset$, so that we replace $T_i,T_j$ with $T_i \cap T_j$, $T_i \setminus T_j$ and $T_j \setminus T_i$.
	This does not affect the transpose property
	since $T_i^\top \cap T_j^\top= (T_i\cap T_j)^\top$, $T_i^\top \setminus T_j^\top = (T_i \setminus T_j)^\top$ and $T_j^\top \setminus T_i^\top = (T_j \setminus T_i)^\top$. Therefore the set at the $t+1^{th}$ step retains
	its closure property under taking transposes.
\end{proof}

\subsection{Weisfeiler-Leman and Polynomials}
	In this section we illustrate how the steps of the $2$-dimensional \emph{Weisfeiler-Leman} may be carried out with polynomials. Consider the set $\mathcal{E}$, the set of polynomials $\in \mathcal{R}[y]$ obtained
	from algorithm \ref{wl}. Let $g_l \in \mathcal{E}$ be such a polynomial, then we have associated with it the matrix $E_l$ which was defined in definition \ref{matrixdef}.
        In the preceding section we defined the multigraph $G_f=(V,\mathcal{E})$. We then have the following lemma.

	\begin{lemma}
		The set $\{I\} \cup \mathcal{E}$ is well behaved.
	\end{lemma}


	\begin{proof}
		It is easily seen that every element of $\mathcal{E}$ when thought of as a matrix has its entries from $\{0,1\}$ so that condition $1$ of \emph{well behavedness} is satisfied. Further, we claim that for 
		any $E_l,E_m \in \mathcal{E},l\neq m, E_l \cap E_m = \emptyset$. Suppose if $E_l$ is identity then this is clearly true since $g_m$ is a factor of $g \in \mathcal{R}[y]$ and we know that $y \nmid g$ so that
		$\forall i,(\xi_i,\xi_i) \notin
		E_m$. If neither of them are the
		identity matrix then from lemma \ref{welldef} we have that $E_l=E_m$. Further $\forall 1 \le i,j \le n,i\neq j,\exists 1 \le l \le |\mathcal{E}|, E_l(i,j)=1$, since $\prod_{l=1}^{|\mathcal{E}|}g_l=g$
		. Hence it follows that $I+\sum_{E_l \in \mathcal{E}}E_l=J$ so that condition $2$ is satisfied as well. The last condition follows from the defintion of $G_f$.
	\end{proof}


	Hence one can think of applying the $2-dimensional$ \emph{Weisfeiler-Leman} algorithm in this case. Note however that we do not explicitly know the matrices $E_l$ but rather have their polynomial forms $g_l \in \mathcal{R}[y]$.
	Therefore we must show that the algorithm \ref{wl} may be duplicated with just this set of polynomials.

	\subsubsection{Multiplication of Matrices}
	Consider a \emph{well behaved} set of colors $\mathcal{S}=\{S_1,S_2,\cdots,S_m\}$ with respect to the complete graph on $n$ vertices. Assume also that this set is closed under taking transposes.	
	Suppose they are given in their polynomial forms, i.e. the polynomial corresponding to $S_l$ is 
	defined by 

	\begin{align*}
		g_l(y,x) &= \sum_{i=1}^n \prod_{j:S_l(i,j)=1} (y-(\xi_i-\xi_j)) \mu_i
	\end{align*}


        where $\mu_i$ depends on $x$. We now wish to show that from their polynomial forms we can recover the polynomial form of the matrix corresponding to $S_lS_t$ for some $S_l$ and $S_t$ in this \emph{well behaved} set. 
	Denote by $\bar{g}_l$ the 
	polynomial corresponding to $S_l^\top$. Consider the algebra $\mathcal{T} \equiv \mathcal{R}[y]/(\bar{g}_l) \equiv \mathcal{R}[Y]$, where $Y \equiv y \pmod{\bar{g}_l}$. Since $\bar{g}_l$ is square-free and completely
	splitting over $\mathcal{R}$, we have that $\mathcal{T}$ is a semisimple $\mathcal{R}$-algebra. Let its primitive idempotents over $\mathcal{R}$ be $\nu_1,\cdots,\nu_{d_l}$, where $deg(g_l)=deg(\bar{g}_l)=d_l$. Suppose
	$\bar{g}_l$ splits as below over $\mathcal{R}[y]$:
	
	\begin{align*}
		\bar{g}_l(y,x) &= \prod_{j'=1}^{d_l} (y-\chi_{j'}) = \sum_{i=1}^n \prod_{j:S_l^\top(i,j)=1} (y-(\xi_i-\xi_j))\mu_i
	\end{align*}


	where each of the $\chi_{j'} \in \mathcal{R}$ and hence can be written as $\chi_{j'}=\sum_{i=1}^n \chi_{ij'} \mu_i$. It then follows that $Y = \sum_{j'=1}^{d_l} \chi_{j'}\nu_{j'}$. It also follows from the expression 
	for $\bar{g}_l$ above that the set $\{\chi_{i1},\chi_{i2},\cdots,\chi_{ij'},\cdots,\chi_{d_li}\}$ is a permutation of $\{(\xi_i-\xi_j)|g_l(\xi_i-\xi_j)\mu_i=0\}$. Let the inverse permutation of indices be denoted by 
	$\pi_i$, i.e. $\pi_i(j)=j'$, where $1 \le j' \le d_l$ and $j$ indexes $S_l^\top(i,j)=1$ for a fixed $i$. Then $Y \in \mathcal{T}$ may equivalently be written as 

	\begin{align*}
		Y &= \sum_{j'=1}^{d_l}\chi_{j'}\nu_{j'} = \sum_{j'=1}^{d_l}\sum_{i=1}^n \chi_{ij'}\mu_i \nu_{j'}\\
		  &= \sum_{i=1}^n \sum_{j'=1}^{d_l} \chi_{ij'} \nu_{j'}\mu_i \\
		  &= \sum_{i=1}^n \sum_{j:S_l^\top(i,j)=1} (\xi_i-\xi_j) \nu_{\pi_i(j)} \mu_i 
	\end{align*}

	where the set $\{\nu_{\pi_i(j)}|S_l^\top(i,j)=1\}$ is the same as  $\{\nu_1,\cdots,\nu_{d_l}\}$. Consider the polynomial ring $\mathcal{T}[z]$ and let $g_t \in \mathcal{T}[z]$. Since the expression for $g_t \in 
	\mathcal{R}[z]$ is

	\begin{align*}
		g_t(z,x) &= \sum_{i=1}^n \prod_{k:S_t(i,k)=1} (z-(\xi_i-\xi_k)) \mu_i
	\end{align*}

	Hence the expression for $g_t \in \mathcal{T}[z]$ is given by 

	\begin{align*}
		g_t(z,y,x) &= \sum_{i=1}^n \sum_{j'=1}^{d_l} \prod_{k:S_t(i,k)=1} (z-(\xi_i-\xi_k)) \nu_{j'}\mu_i
	\end{align*}

	Consider now the polynomial $g_t(z+Y,y,x) \in \mathcal{T}[z]$.

	\begin{align*}
		g_t(z+Y,y,x) &= \sum_{i=1}^n \sum_{j'=1}^{d_l} \prod_{k:S_t(i,k)=1} 
		\left(z+Y-(\xi_i-\xi_k)\right) \nu_{j'}\mu_i \\
		&= \sum_{i=1}^n \sum_{j'=1}^{d_l} \prod_{k:S_t(i,k)=1} \left(z+
		 \sum_{i_1=1}^n \sum_{j_1:S_l^\top(i_1,j_1)=1} (\xi_{i_1}-\xi_{j_1})\nu_{\pi_{i_1}(j_1)} \mu_{i_1} -(\xi_i-\xi_k)\right) \nu_{j'}\mu_i \\
		&= \sum_{i=1}^n \sum_{j:S_l^\top(i,j)=1} \prod_{k:S_t(i,k)=1} (z - (\xi_j-\xi_k)) \nu_{\pi_i(j)}\mu_i \\
		&= \sum_{i=1}^n \sum_{j:S_l(j,i)=1} \prod_{k:S_t(i,k)=1} (z-(\xi_j-\xi_k)) \nu_{\pi_i(j)}\mu_i
	\end{align*}		        


	Where the last couple of steps follow from the previous step because $\mu_i\mu_{i'}=0, i \neq i'$ and $\mu_i^2=\mu_i$, while $\nu_{j}\nu_{j'}=0,j\neq j'$, otherwise $\nu_{j}^2$ being just $\nu_{j}$. Note that the polynomial
	corresponding to $S_lS_t$ (denoted by $g_{lt} \in \mathcal{R}[y]$) would be of the form:

	\begin{align*}
		g_{lt}(z,x) &= \sum_{j=1}^n \prod_{\substack{
		                                            i,k: \\
							    S_l(j,i)=1 \\
						            S_t(i,k)=1}} (z-(\xi_j-\xi_k))\mu_j
	\end{align*}
 

	We wish to obtain $g_{lt}$ from the polynomial $g_t(z+Y,y,x)=h(z,y,x)$. This is obtained by eliminating $z,x$ from $h$ as follows. Consider the ring $\mathcal{R}'\equiv \mathbb{F}_p[y]/(f(y))$ and $\mathcal{T}'\equiv
	\mathcal{R}'[x]/(g_l(x,y))$. Consider $h(z,y,x) \in \mathcal{T}'[z]$ from which we construct the ring $\mathcal{U}\equiv \mathcal{T}'[z]/(h(z,y,x)) \equiv \mathcal{T}'[Z]$ where $Z\equiv z \pmod{h}$.
	Let $c_{\mathcal{R}'}(w,y) \in \mathcal{R}'[w]$ be the characteristic polynomial of $Z$ over the ring $\mathcal{R}'$. Then we have the following lemma:

	\begin{lemma}
		$\gcd(c_{\mathcal{R}'}(z,y),g(z,y)) = g_{lt}(z,y) \in \mathcal{R}'[z]$ where $g$ is the polynomial defined in lemma \ref{g}. In other words $g_{lt}(z,x) \in \mathcal{R}[z]$ may be obtained by merely substituting $x$
		for $y$ in this $\gcd$ polynomial.
	\end{lemma}


	\begin{proof}
		Let the primitive idempotents of $\mathcal{R}'$ over $\mathbb{F}_p$ be $\nu'_1,\cdots,\nu'_n$ ($\mathcal{R}'$ is \emph{semisimple} over $\mathbb{F}_p)$ and the primitive idempotents of $\mathcal{T}'$ over 
		$\mathcal{R}'$ be $\mu'_1,\cdots,\mu'_{d_l}$. We wish to consider $h(z,y,x)$ as a polynomial in $\mathcal{T}'[z]$; hence expressing every $\nu_{j} \mu_i$ in terms of $\mu'_j\nu'_i$'s is enough to express $h$
		as a member of this ring. We have

		\begin{align*}
			\nu_{\pi_i(j)}\mu_i &= \nu'_{j}\mu'_{i} + \sum_{r:S_l(r,i)=0} 
			\prod_{\substack{
												 q: \\
												 q \neq j \\
												 S_l(j,i)=1 \\
											 S_l(q,i)=1}} 
											 \frac{\xi_r-\xi_q}{\xi_j-\xi_q} 
											 \nu'_r\mu'_i
		\end{align*}

		where  

		\begin{align*}
			\nu_{\pi_i(j)}\mu_i &= \prod_{k:k\neq i} \frac{x-\xi_k}{\xi_i-\xi_k}\prod_{\substack{
													     q: \\
													     q \neq j\\
													     S_l(j,i)=1\\
													     S_l(q,i)=1}} \frac{y-\xi_q}{\xi_j-\xi_q}\\
			\nu'_j\mu'_i &= \prod_{q:q\neq j} \frac{y-\xi_q}{\xi_i-\xi_q} \prod_{\substack{
													k:\\
													k \neq i\\
												        S_l(j,i)=1\\
												S_l(j,k)=1}} \frac{x-\xi_k}{\xi_j-\xi_k}
		\end{align*}

		The expression for $\nu_{\pi_i(j)}{\mu_i}$ follows from evaluating it at $x=\xi_j,y=\xi_i,\xi_j \in \{\xi_1,\cdots,\xi_n\}$ and $\xi_i:S_l(\xi_j,\xi_i)=1$. This then gives us the expression for $h(z,y,x) \in 
		\mathcal{T}'[z]$ as follows

		\begin{align*}
			h(z,y,x) &= \sum_{i=1}^n \sum_{j:S_l(j,i)=1} 
			\prod_{k:S_t(i,k)=1} \left(z-\left(\xi_j-\xi_k\right)\right)
			 \left(\nu'_j + \sum_{r:S_l(r,i)=0} \alpha_{rj}\nu'_r\right)\mu'_i\\
			\alpha_{rj} &= \prod_{\substack{
							q:\\
							q\neq j\\
							S_l(j,i)=1\\
						S_l(q,i)=1}} \frac{\xi_r-\xi_q}{\xi_j-\xi_q}
		\end{align*}

		Observe that for any $1 \le i \le n$, $\nu'_j\nu'_r=0$ for $j,r:S_l(j,i)=1,S_l(r,i)=0$. Therefore we have that 

		\begin{align*}
			\gcd(c_{\mathcal{R}'}(z,y),g(z,y)) &= \sum_{j=1}^n \prod_{\substack{
											i,k:\\
											S_l(j,i)=1\\
										S_t(i,k)=1\\}} (z-(\xi_j-\xi_k))\mu'_j = g_{lt}(z,x)
		\end{align*}
	\end{proof}


	\subsubsection{The Remaining Steps}
	Now that we have the polynomial $g_{lt}$ corresponding to $S_lS_t$ we need to show that we can express the product as $\sum_{k}kD_k$ where each $D_k$ has entries from the set $\{0,1\}$. This is however easy to reproduce: 
	expressing $g_{lt}=\prod_{k}g_k^k$ where each $g_k$ is \emph{square-free} and \emph{mutually prime} gives us the polynomials corresponding to $D_k$. Further we have the following relations

	\begin{align*}
		S_l \cap S_t &\equiv \gcd(g_l,g_t) \\
		S_l \setminus S_t &\equiv \frac{g_l}{\gcd(g_l,g_t)}\\
		S_t \setminus S_l &\equiv \frac{g_t}{\gcd(g_l,g_t)}
	\end{align*}

	where we use the $\equiv$ symbol to denote the equivalence between a color matrix and its corresponding polynomial form. Note also that from lemma \ref{transposepol} that at every step of the iteration the set $\mathcal{S}$
	is closed under taking transposes, so that for every color polynomial $g_l \in \mathcal{S}$ we may also find $\bar{g}_l \in \mathcal{S}$. Hence it is possible to repeat this process till we achieve a stabilization of the
	colors. This gives rise to the following algorithm for the polynomials.

	\begin{algorithm}[H]
		\caption{2D WL for Polynomials}
		\label{polywl}
		\begin{algorithmic}
			\State $\mathcal{C} \leftarrow \mathcal{E}$
			\While{True}
			\If{\State $\forall g_i,g_j \in \mathcal{C}, g_{ij} = \prod_{l}g_{\alpha_l}^l$ for some index $\{\alpha_l\}$ of $\mathcal{C}$}
			   \State Output $\mathcal{C}$ as the final set
			\Else
			   \State $g_{lt}= \prod_{l}(h_l)^l$ where each $h_l$ is \emph{square-free} and \emph{mutually prime} to each other
			   \State $\mathcal{C} \leftarrow \mathcal{C} \cup_{l} \{h_l\}$
			   \While{$\exists g_i,g_j \in \mathcal{C},\gcd(g_i,g_j)\neq 1$}
			         \State $g=\gcd(g_i,g_j)$
				 \State $\mathcal{C} \rightarrow (\mathcal{C} \setminus \{g_i,g_j\}) \cup \{g,\frac{g_i}{g},\frac{g_j}{g}\}$
			   \EndWhile
			\EndIf
			\EndWhile
		\end{algorithmic}
	\end{algorithm}

			
         Note that given two polynomials $g_l$ and $g_t$, the polynomial representation for $g_{lt}$ is computable in time polynomial in the degrees of $g_l,g_t$ and $\log{p}$. Since the degree of each $g_l$ is always bounded above
	 by $n$, algorithm \ref{polywl} terminates in time polynomial in $n$ and $\log{p}$. 
	 
	 
	 \subsection{Colors and Schemes}\label{sec:scheme}
	 Consider the final set of colors (or polynomials) $\mathcal{C}$ we obtain from algorithm \ref{polywl}. In this section we will prove some simple properties about this set $\mathcal{C}$.

	 \begin{lemma}\label{thin}
		 $2 \le |\mathcal{C}| \le n$. Further if $|\mathcal{C}|=n$ then $f$ may be factored.
	 \end{lemma}

	 \begin{proof}
		The first part follows because $|\mathcal{E}| \ge 2$, since for any $\xi_i,\xi_j$ that are roots of $f \in \mathbb{F}_p$, $\xi_i-\xi_j$ and $\xi_j-\xi_i$ differ in their \emph{$2$-Sylow} expansion in atleast one place
		- namely, the most significant bit. This is because $\xi_i-\xi_j = -1(\xi_j-\xi_i) = \eta^{2^{r-1}} (\xi_j-\xi_i)$. Since $|\mathcal{C}| \ge |\mathcal{E}| \ge 2$ the first inequality follows. The second inequality 
		is trivial since the product of all the polynomials in $\mathcal{C}$ is $f(y,x) \in \mathcal{R}[y]$ (or equivalently, $\sum_{C_l \in \mathcal{C}} C_l = J$). If $|\mathcal{C}|=n$ then the degree of any polynomial
		$g_l \in \mathcal{C}$ is $1$ which gives us an endomorphism of the roots of $f$ and hence $f$ may be factored by \cite{evdokimov}.
	\end{proof}


	\begin{lemma} \label{identity}
		If $I \notin \mathcal{C}$ then $f$ may be factored.
	\end{lemma}

	\begin{proof}
			Since the starting set contained $I$, the only way $I \notin \mathcal{C}$ would be that $I$ decomposes into two or more colors at some stage. Let $g_I(y,x)$ be the polynomial form of $I$:
			\begin{align*}
				g_I(y,x) &= \sum_{i=1}^n y \mu_i
			\end{align*}
			
			Suppose a color $C_l$ is a non-trivial decomposition of $I$, i.e. $C_l \subsetneq I$ and $C_l \neq \emptyset$. Then since $\forall 1 \le i \le n$ the degree of $g_I\mu_i=1$, $\exists 1 \le j \le n,
			deg(g_l\mu_j)=0$. Also since $C_l\neq \emptyset$,
			$\exists 1 \le i \le n, deg(g_l\mu_j)=1$. Hence the leading coefficient of $g_l$ will be a zero divisor in $\mathcal{R}$ and we get a decomposition of $f \in \mathbb{F}_p$.

		\end{proof}
			

		\begin{lemma}\label{regular}
		 Let $C_l$ be any color in the set $\mathcal{C}$. If $\exists C_l \in \mathcal{C},\exists 1 \le u,v \le n,\displaystyle \sum_{w=1}^n C_l(u,w) \neq \sum_{w=1}^n C_l(v,w)$ then $f$ may be factored. In other words, for
		 $f$ not to be factored each color $C_l \in \mathcal{C}$ must be regular.
	 \end{lemma}


	 \begin{proof}
		 Consider the polynomial form of such a $C_l \in \mathcal{C}$. 

		 \begin{align*}
			 g_l &= \sum_{i=1}^n \prod_{j:C_l(i,j)=1} (y-(\xi_i-\xi_j)) \mu_i
		 \end{align*}

		 Then it is easy to see that $\sum_{w=1}^n C_l(u,w) = deg(g_l\mu_u)$. If $deg(g_l\mu_u) \neq deg(g_l\mu_v)$, then the leading coefficient of $g_l$ is a zero-divisor in the algebra $\mathcal{R}$ and hence
		 we obtain a decomposition of $f \in \mathbb{F}_p$.
	 \end{proof}

	 It is possible to strengthen this lemma and prove that the colors are not just \emph{regular}, but also \emph{strongly regular}. We prove this in the following lemma.

	 \begin{lemma}\label{intersectionnum}
		 Consider any three colors $C_s,C_t,C_l \in \mathcal{C}$. Consider any $(i,j) \in C_l$. Then the cardinality of the set $\{k: (i,k) \in C_s,(k,j) \in C_t\}$ is independent of the choice of the edge $(i,j) \in C_l$.
		 This cardinality will henceforth be denoted by $a_{stl}$.
	 \end{lemma}
	
	\begin{proof}
		For any $C_s,C_t \in \mathcal{C}$ we know that the set $\mathcal{C}$ is closed under their multiplication, i.e. $C_sC_t = \sum_{C_k \in \mathcal{C}} \alpha_k C_k$ where $\alpha_k$ is a positive integer. It is easy
		to see from this expansion that $a_{stl} = \alpha_l$.
	\end{proof}

	Note that lemma \ref{regular} is a special case of lemma \ref{intersectionnum}, where the out-degree of color $C_l$ is given by $a_{ll^\top1}$ where $1$ denotes the identity color $I$. This brings us to the notion of 
	\emph{Association schemes} which we introduce below:

	\begin{definition}\label{scheme} Let $X$ be a finite, non-empty set and let $S$ be a set of relations on $X$. Then the pair $(X,S)$ form an association scheme if
		\begin{enumerate}
			\item $S$ is a partition of $X\times X$.
			\item For all $s \in S$, $s^* = \{(y,x)|(x,y) \in s\} \in S$.
			\item $I = \{(x,x)\}\in S$
			\item $\forall p,q,r \in S$ there is a number $a_{pqr}$ such that for all $(x,z) \in r$, $|\{y \in X| (x,y) \in p, (y,z) \in q\}|=a_{pqr}$.
		\end{enumerate}
	\end{definition}



	\begin{example}\label{schurian}
			An important class of association schemes called \emph{Schurian} or \emph{group case} schemes arise from the $2$-orbits of a transitive group action. Let $G$ be a group and let $H \le G$ be a subgroup.
			Then the action of $G$ on the cosets of $H$ by left multiplication is transitive. Consider the action of $G$ on $G/H \times
			G/H$ where $g(xH,yH) \rightarrow (gxH,gyH)$. Then the orbits of this action form an association scheme. To see this observe that they do partition $G/H \times G/H$. Further $I = \{(xH,xH)|x \in G\}$ is
			one of the orbits (since $G$ acts 
			transitively on the cosets of $H$). Consider a $2$-orbit $s=G(xH,yH)$ then it's transpose $G(yH,xH)$ is also a $2$-orbit. If $(xH,yH),(uH,vH)\in s$ then $\exists g \in G, (uH,vH)=(gxH,gyH)$ so that
			$(vH,uH)=(gyH,gxH)$ which implies that $(yH,xH),(vH,uH) \in s^*$. Suppose $p,q,r$ are any three of these $2$-orbits. Let $(xH,zH),(uH,vH) \in r$ so that $\exists g \in G (uH,vH)=(gxH,gyH)$. Then there is a
			bijection between $f:\{yH|(xH,yH)\in p, (yH,zH) \in q \} \rightarrow \{wH|(uH,wH) \in p,(wH,vH)\in q\}$ given by $f(yH)=gyH$. Thus the set of $2$-orbits under this action form an association scheme 
			denoted by $(G/H,G//H)$.
	\end{example}


	
	Let $X=\{1,\cdots,n\}$ where $deg(f)=n$. We then make the following claim:
	\begin{lemma}
		If after the algorithm \ref{polywl} $f$ remains unfactored, then the tuple $(X,\mathcal{C})$ is an association scheme.
	 \end{lemma}

	\begin{proof}
		Condition $1$ follows from the well-behavedness of the set $\mathcal{C}$ (see lemma \ref{wellbehaved}). Condition $2$ follows from lemma \ref{transposepol}. Condition $3$ follows from lemma \ref{identity}. Finally,
		condition $4$ follows because of lemma \ref{intersectionnum}.
	\end{proof}




	\subsection{Closed Subsets}\label{sec:subset}
      From the definition of schemes it follows that schemes are a generalization of groups, i.e. all groups are schemes. This can be seen in the following fashion. Consider a group $G$. For each element $g \in G$
      we define $C_g=\{(e,f)|e,f\in G,eg=f\}$. Let $\mathcal{C}_G=\cup_{g \in G}C_g$. Then the set $(G,\mathcal{C}_G)$ forms a scheme with $a_{pqr}=1$ if $pq=r$ and $0$ otherwise. If $|G|=n$ then this scheme has $n$ colors and is
      referred to as a \emph{thin scheme}. Notice that by lemma \ref{thin} \emph{thin schemes} are easily factored. 

      In this section we will talk about \emph{closed subsets} which are a generalization of subgroups to the scheme structure and prove some result about the closed subsets of the scheme $(X,\mathcal{C})$ from the previous 
      section. We will using the notation of Zieschang's book on Association Schemes \cite{zieschang}; a more detailed discussion on schemes and closed subsets can also be found there. If $(X,S)$ is a scheme and $R \subseteq S$
      and $x \in X$ then by $xR$ we mean the set $\{y|y \in X,(\exists s \in R,(x,y) \in s\}$.


      \begin{definition}
	      Let $(X,S)$ be any arbitrary scheme. A nonempty subset $R$ of $S$ is called closed if $R^*R \subseteq R$, where $R^*=\{s^*|s \in R\}$.
      \end{definition}

      Note that $I \in R^*R$ which implies that $I \in R$. Further $R^* = R^*I \subseteq R^*R \subseteq R$, which implies that $R=R^*$. Hence $RR \subseteq R$.	Seen in this fashion the choice of the name \emph{closed subsets}
      and the link with subgroups becomes clear. 

      \begin{definition}
	 If $R \subseteq S$ where $(X,S)$ is a scheme, then we define $X/R = \{xR|x \in X\}$.
 \end{definition}


 	We then have a following simple lemma.

	\begin{lemma}\label{partition}
		If $R \subseteq S$ then $R$ is closed if and only if the set $X/R$ is a partition of $X$.
	\end{lemma}

	\begin{proof}
		To see the forward direction we note that the relation $x~y \equiv y \in xR$ is an equivalence relation on $X$. Every $x \in X$ belongs to $xR$ since $I \in R$, so $x ~ x$. If $y \in xR$ then $\exists s \in R$ such 
		that $(x,y) \in s$. Since $R^* = R$, $s^* \in R$ so that $x \in yR$. If $y \in xR$ and $z \in yR$ then $z \in xR$ since $RR \subseteq R$. Hence if $R$ is closed, $X/R$ is a partition of $X$.
		The converse follows in a similar fashion: if $y \in xR$ then $x \in yR$ (since $X/R$ is a partition) so that $R^*=R$. From transitivity it follows that $RR = R^*R \subseteq R$ so that $R$ is \emph{closed}.
	\end{proof}

	\begin{definition}\label{primitivescheme}
		An association scheme $(X,S)$ is called primitive if its only closed subsets are $\{I\}$ and $S$ itself.
        \end{definition}

	
	From the definition it is clear that \emph{primitive schemes} are generalizations of \emph{prime groups} - groups which have no non-trivial subgroup. In the case of groups we know that groups with no proper subgroups
	are precisely the prime groups - cyclic (hence commutative) groups of prime order.	
	Let $n_R=\sum_{s \in R} a_{ss^*1}$ and $|X|=n$. Therefore since $X/R$ is a partition of $X$ we have from lemma \ref{partition}
	$n_R|n$. It is then obvious that schemes on a prime number of vertices ($|X|=p$, $p$ prime) must be \emph{primitive}. From \cite{Hanaki:2006} we also know that association schemes of prime order are commutative. It is 
	tempting to extend the analogy with groups and conjecture that \emph{primitive schemes} would always have a prime order (or atleast be commutative/cyclic). However this is \emph{false} (for an argument see Appendix
	\ref{sec:counterexample}).
	
	Consider the scheme $(X,\mathcal{C})$ from the previous section. We wish to prove that the
	problem of factoring the polynomial $f$ (and its associated scheme $(X,\mathcal{C})$) can always be reduced to factoring a polynomial of degree $\le deg(f)$ whose associated scheme is \emph{primitive}. Before that we
	need a definition and a minor lemma.

 	\begin{definition}
		Let $(X,S)$ be an arbitrary scheme and $R$ any subset of $S$. Then by $(R)$ we denote the intersection of all closed subsets of $(X,S)$ which contain $R$.
	\end{definition}

	We set $R^0 =\{I\}$ and inductively define $R^i=R^{i-1}R$. Then we have the following

	\begin{lemma}
		The set $(R)$ is the union of all sets $(R \cup R^*)^i$ where $i$ is a non-negative integer.
	\end{lemma}

	\begin{proof}
		Let $P=R \cup R^*$ and let $Q = \cup_{i \in \mathbb{Z}_+}(R^*\cup R)^i$. We wish to show $(R)=Q$. Since $(P^i)^*=(P^*)^i$, we have $(P^i)^*=P^i$ since $P^*=P$. Thus for any non-negative integers $l,m$ we have

		\begin{align*}
			(P^l)^*P^m = P^lP^m = P^{(l+m)} \subseteq Q
		\end{align*}

		Therefore $Q$ is closed. Further since $R \subseteq Q$, we must have $(R) \subseteq Q$ by definition. The converse follows because

		\begin{align*}
			\forall i \in \mathbb{Z}_+,P^i \subseteq (P^i) \subseteq (P) = (R)
		\end{align*}

		Hence $Q \subseteq (R)$.
	\end{proof}

	Note that when generating $(R)$ by taking the union of $(R^* \cup R)^i$ one only needs to go till $i \le |\mathcal{S}|$. This is because at every step the size of this set grows by at least one and it cannot grow beyond
	$|\mathcal{S}|$, hence it must stabilize for some $i \le |\mathcal{S}|$. We then come to the main lemma.

	\begin{theorem}\label{primitive}
		Let the association scheme we get from algorithm \ref{polywl} on the polynomial $f \in \mathbb{F}_p[x]$ be $(X,\mathcal{C})$. Here $|X|=deg(f)=n$. Then the problem of factoring $f$ over $\mathbb{F}_p$ may be 
		polynomial time reduced	to that of factoring a polynomial $g \in \mathbb{F}_p[x]$, $deg(g) \le deg(f)$ whose association scheme $(Y,\mathcal{S})$ is primitive.
	\end{theorem}

	\begin{proof}If the scheme $(X,\mathcal{C})$ is \emph{primitive} then we are already done. Hence, suppose that $(X,\mathcal{C})$ is not \emph{primitive}. Then we first claim that it is possible to find a non-trivial closed
		subset of $\mathcal{C}$ in time polynomial in $n$ and $\log{p}$. This is done by considering $(C_l)$ for all such $C_l \in \mathcal{C}$. Since $|\mathcal{C}| \le n$ and to construct $(C_l)$ we only need to take powers
		of $C_l$ till at most $|\mathcal{C}|$ we can construct all such $(C_l)$'s in time $O(n\log{p})$. Now suppose all of these sets are either $I,\mathcal{C}$ then we claim that $(X,\mathcal{C})$ must be \emph{primitive}.
		Suppose not, then there exists a set $\mathcal{D}$, $\{I\} \subsetneq \mathcal{D} \subsetneq \mathcal{C}$ which is closed. Let $C_l \in \mathcal{D}$ where $C_l \neq I$ (such a $C_l$ exists). Then by definition
		$(C_l) \subseteq \mathcal{D} \subsetneq \mathcal{C}$ and hence we arrive at a contradiction. Thus given $(X,\mathcal{C})$ we either determine that it is primitive, or in polynomial time construct a \emph{closed
		subset} of this scheme. Let $\mathcal{D}$ be the closed subset so obtained. Define by $g_{\mathcal{D}}$ the following polynomial:

		\begin{align*}
			g_{\mathcal{D}} &= \prod_{C_l \in \mathcal{D}} g_l = \sum_{i=1}^n g_{\mathcal{D}i} \mu_i
		\end{align*}

		Where $deg(g_{\mathcal{D}}) = n_{\mathcal{D}}$ and $g_{\mathcal{D}i} \in \mathbb{F}_p[y]$. From lemma \ref{partition} we know that $X/\mathcal{D}$ is a partition of $X$. Hence $\forall j \in i\mathcal{D}$, 
		$g_{\mathcal{D}j}=g_{\mathcal{D}i}$. If $j \notin i\mathcal{D}$ then $g_{\mathcal{D}i} \neq g_{\mathcal{D}j}$ so that $\exists 0 \le \alpha < n_{\mathcal{D}}$ such that the coefficient of $y^{\alpha}$ in
		$g_{\mathcal{D}i}$ is different from that of $g_{\mathcal{D}j}$. Suppose the coefficient of such a $y^{\alpha}$ is $h(x) \in \mathbb{F}_p[x]/(f)$. Then the resultant polynomial $Res(h(x)-z,f(x)) \in F_p[z]$ has
		at most $n/n_{\mathcal{D}}$ distinct roots, so that on making it square-free we get a polynomial $g$ over $\mathbb{F}_p$ whose degree is at most $n/n_{\mathcal{D}}$. Finding a root $\beta$ of $g$ then gives us a zero
		divisor in the algebra $\mathcal{R}$, namely $h(x)-\beta$. Also note that finding such a $y^\alpha$ and its coefficient is easy, all we need to do is ensure $h(x) \notin \mathbb{F}_p$. Note that we may assume that
		the scheme $(Y,\mathcal{S})$ formed by $g$ is \emph{primitive} or we can again reduce it in the above fashion.
	\end{proof}
	
	
\begin{comment}
\subsection{The Scheme $(X,\mathcal{C})$}
In this section we will study some of the properties particular to the scheme $(X,\mathcal{C})$.

\begin{lemma}\label{nonsymmetric}
	$\forall C_i \in \mathcal{C}, C_i \neq C_i^\top$. In other words none of the elements of the scheme $\mathcal{C}$ is symmetric.
\end{lemma}

\begin{proof}
	Suppose $\exists C_l \in \mathcal{C}$ such that $C_l = C_l^\top$. This implies that if $(i,j) \in C_l$ then $(j,i) \in C_l$, so that $|C_l|$ is even. By lemma \ref{regular} we have $|C_l|=n*a_{ll^\top1}$. Since $n$ is odd
	(otherwise $f$ may be factored over $\mathbb{F}_p$) we infer that $a_{ll^\top1}=deg(g_l)$ must be even. By \emph{Evdokimov}'s method \cite{evdokimov}, we can factor $g_l=\prod_{i}g_i$ where each $g_i$ is odd. Notice that 
	we can replace the polynomial $g_l$ by this set of polynomials $\{g_i\}$ in $\mathcal{C}$ (algorithm \ref{polywl}) without violating the \emph{well-behaved} property of $\mathcal{C}$. 
\end{proof}

\section{The Bose-Messner Algebra}
We will conclude this chapter by discussing some properties of the algebra formed by the adjacency matrices

We will conclude this chapter by discussing some properties of the algebra formed by the adjacency matrices of an association scheme $(X,S)$. The algebra $s$




We will end by discussing some properties of the algebra formed by the adjacency matrices of an association scheme $(X,S)$. The algebra $\mathbb{C}S$ considered as a subalgebra of $Mat_n(\mathbb{C})$ is called the 
\emph{Bose-Messner algebra}. By property $2$ of definition \ref{scheme} we know that $\mathbb{C}S$ is a \emph{self-adjoint algebra} over $\mathbb{C}$ so that it is \emph{semisimple}. This gives us a decomposition of this
algebra into its simple modules.

\begin{align*}
	\mathbb{C}S &= \bigoplus_{\chi \in Irr(\mathbb{C}S)} \mathbb{C}Se_{\chi}
\end{align*}

Where $\mathbb{C}Se_{\chi}$ is a \emph{simple module} generated by a primitive idempotent in the center of $\mathbb{C}S$ and by Wedderburn's theorem is isomorphic to $Mat_{n_{\chi}}(\mathbb{C})$ where $n_{\chi}=\chi(I)$ is the
\emph{degree} of the character $\chi$. 
\end{comment}
