


\documentclass[a4paper]{llncs}

\usepackage{amssymb}
\setcounter{tocdepth}{3}
\usepackage{graphicx}
\usepackage{subfigure} 
\usepackage{comment} 

\usepackage{url}


\let\llncssubparagraph\subparagraph
\let\subparagraph\paragraph
\usepackage[compact]{titlesec}
\let\subparagraph\llncssubparagraph

\newcommand{\keywords}[1]{\par\addvspace\baselineskip
\noindent\keywordname\enspace\ignorespaces#1}

\setlength{\textfloatsep}{10pt plus 5pt minus 5pt}
\setlength{\intextsep}{10pt plus 5pt minus 5pt}


\begin{document}

\mainmatter  
\title{Weighted Shortest Path Models: A Revisit to the Simulation of Internet Routing}


\author{Mingming Chen\inst{1} \and Jichang Zhao\inst{2,3} \and Xiao Liang\inst{3} \and Ke Xu\inst{3}}

\institute{Department of Computer Science, Rensselaer Polytechnic Institute
\and School of Economics and Management, Beihang University
\and State Key Laboratory of Software Development Environment, Beihang University
\\ chenm8@rpi.edu, jichang@buaa.edu.cn, \{liangxiao, kexu\}@nlsde.buaa.edu.cn
}


\maketitle


\begin{abstract}
Understanding how packets are routed in Internet is significantly important to Internet measurement and modeling. The conventional solution for route simulation is based on the assumption of unweighted shortest path. However, it has been found and widely accepted that a packet in Internet is usually not transmitted along the unweighted shortest path between its source and destination. To better simulate the routing behavior of a packet, we thoroughly explore the real-world Internet routes and present a novel local information based simulation model, with a tuning parameter, which assigns weights to links based on local information and then simulates the Internet route with weighted shortest path. Comparisons with baseline approaches show its capability in well replicating the route length distribution and other structural properties of the Internet topology. Meanwhile, the optimal parameter of this model locates in the range of , which implies that a packet inside the Internet inclines to move to nodes with small degrees. This behavior actually reflects the design philosophy of Internet routing, balancing between network efficiency and traffic congestion.

\keywords{Internet Routing, Shortest Path, Internet Measurement, Traceroute, Local Information}
\end{abstract}


\section{Introduction}
It is very important to track how a packet is routed inside Internet, which is of great help to Internet measurement and modeling. Specifically, it could assist the deployment of sources and destinations for traceroute which is a generally employed tool to sample Internet \cite{Skitter,Scamper,DIMES,iPlane,Heuristics,Rocketfuel,NTC,TraceroutePattern}. However, the realistic routing process in Internet is complicated, which makes the simulation of the Internet routes be a significant challenge. For simplicity, almost all previous approaches assumed that a route in Internet is just an unweighted shortest
path between the source and the destination \cite{TheoryAndSimulations,SamplingBiases,AccuracyScalingINETMap,RelevanceOfMassivelyQualitative,BiasTracerouteSampling}.
Based on this assumption, \cite{TheoryAndSimulations}
proposed three mechanisms for the exploration process of traceroute:
USP (Unique Shortest Path), RSP (Random Shortest Path), and ASP (All
Shortest Path). While USP model, also called
\textit{(k,m)-traceroute}, is the most widely used one. Namely, the
classical approach to simulate the Internet routing is to consider it as
an unweighted shortest path model (\textit{USPM}). However, generally, real routes do not have the same properties as the unweighted
shortest path model assumes, which has already been pointed out in
\cite{EndToEndRouting,SimulatingInternetRoute,RealSizeOfSampledNetwork} and could be learned from their route length
distribution difference in Subsection~\ref{subsec:motivation}. Instead, the real routes are impacted by many factors, including commercial agreements, traffic congestion, and administrative routing policies \cite{RouingPolicyImpact}.

To improve the route simulation, recently a node degree model (\textit{NDM}) \cite{SimulatingInternetRoute} was proposed, whose idea is as follows. Two paths are calculated initially, one starting from the source and the other from the destination, where the next selected node on the path is always the highest degree neighbor of the current node. The computation terminates when it reaches a situation where a node is the highest degree neighbor of its own highest degree neighbor. Then, one of two cases applies: either the two paths have met at a node, or they have not. In the first case, the route produced by the model is the discovered path. In the second case, the model tries to find a shortest path between the two paths, and then obtains the route by merging the two paths and the shortest path. However, there is still a large gap between this model and the real Internet routing. It is still somewhat based on the unweighted shortest path method and has no physical senses for its omitting the
principles of link dynamic \cite{PathDiversity}, traffic dynamic \cite{TrafficDynamicLocal}, etc. For
instance, it does not take the path attributes, such as delay, loss
rate, and available bandwidth, into account. Also, it always routes
to the large degree nodes, that is to say, routes to the core of the
network, without considering the traffic load balance \cite{LoadBalance,LoadBalance2}. Imagining that
there is a link between two nodes at the border of Internet, it is unnecessary
and unreasonable for one of them via the backbone of the
network in order to reach the other.


Thus, a new and more accurate model with physical meanings is still needed. In this paper, we thoroughly investigate the real-world
traceroute traces and propose two new simulation models, \textit{Local Information Based Model} (\textit{LIM}) and \textit{Path Feature Based Model} (\textit{PFM}) both with tunable parameters , which weigh links flexibly and then simulate the Internet routing with weighted shortest path. The weights to links that our models assign are intent to mimic the link dynamic \cite{PathDiversity}, traffic dynamic \cite{TrafficDynamicLocal}, etc. The comparison of our two models with \textit{USPM} and \textit{NDM} shows that our two models, especially \textit{LIM}, could more accurately simulate the real Internet routes. In addition, the optimal parameter of \textit{LIM} locates in the range of , which implies that a packet in Internet tends to move to small degree nodes. This behavior reflects the trade-off between network efficiency and traffic load balance in the design philosophy of Internet routing.

The rest of this paper is organized as follows. First, in
Section~\ref{sec:preliminaries} we provide the preliminaries. Then
we present the motivation of this work and give an introduction to
our new models in Section~\ref{sec:models}.
Section~\ref{sec:evaluation} evaluates our models and analyzes the
underlying mechanisms for optimal parameters.
Finally, we conclude this work briefly in
Section~\ref{sec:conclusion}.

\section{Preliminaries}
\label{sec:preliminaries}

The mapping of the topological structure is greatly important
for a better understanding of Internet. Current explorations
still rely on the extensive use of traceroute: one
collects routes from a limited set of sources to a large set of
destinations, and then merges the obtained paths into a graph. This
method has been adopted by many influential network topology discovery
projects
\cite{Skitter,Scamper,DIMES,iPlane,Heuristics,Rocketfuel,NTC}.
In order to better simulate the Internet routes, we deeply study the real
traceroute traces of two network topology datasets: iPlane
\cite{iPlane} and skitter \cite{Skitter}. The routes of both datasets are undirected and unweighted. Also, we evaluate
our two models and compare them with other models on these two datasets.

\textbf{iPlane:} The Information Plane performs traceroutes from about two hundreds vantage points every day to map the worldwide IPv4 network topology, and uses the structural information to predict path properties between arbitrary end-hosts. The dataset we use was collected on June, 30, 2011. However, the set of destinations varies among the sources. Therefore, for the convenience of comparison, we obtain a set of 140 sources to a common set of 20,419 destinations. The resulting graph has 266,317 nodes and 1,663,170 edges.

\textbf{skitter:} skitter has deployed more than 30 vantage points around the world to map the IPv4 network topology. The dataset we use was collected on October, 15, 2005. Yet, the set of destinations varies among the sources. Similarly, we obtain a set of 21 sources to a common set of 157,290 destinations. The final graph has 771,312 nodes and 1,785,922 edges.

The Internet topology can be naturally modeled as an undirected graph ,
where  denotes the set of nodes and  is the set of edges. The number of links of a node is defined as its {\it degree}. Then, the {\it average degree} of a network can be defined as .
The {\it degree distribution} of a graph is denoted as . For Internet,  follows a power-law distribution  \cite{PowerLawRelationship}.
In such case, the exponent  could be considered as
an indicator of how heterogeneous this distribution is.
{\it Heterogeneity} of a
network, defined as ,
is usually used to characterize the nonuniformity of degrees. The {\it clustering coefficient} of a graph is the probability that two nodes are connected, given that they are both linked to a same third node ,
where  denotes the number of triangles (set of three nodes with three edges) and  is the number of connected triples (set of three nodes with at least two edges) in the graph.
\begin{comment}
\begin{table*}[!t]
\scriptsize
\caption{Network topology properties of the two real datasets: iPlane and skitter.}
\label{Properties of iPlane and skitter}
\vspace{-1.2em}
\centering
\setlength{\tabcolsep}{5pt}
\begin{tabular}{c||c|c|c|c|c|c|c}
\hline \hline
    Dataset &  &  &  &  &  &  &  \\
\hline
    iPlane & 266,317 & 1,663,170 & 12.49 & 2.799 & 0.000047 & 0.0294 & 8.411 \\
\hline
    skitter & 771,312 & 1,785,922 & 4.63 & 2.39 & 0.000006 & 0.000782  & 130.72 \\
\hline \hline
\end{tabular}
\end{table*}
\end{comment}
\begin{comment}
\begin{figure}[!t]
\centering
\includegraphics[scale=0.3]{figure/figure1.eps}
\vspace{-1.5em}
\caption{ of  of the two real Internet datasets: iPlane and skitter, respectively.}
\label{figure1}
\vspace{-0.5em}
\end{figure}
\end{comment}
\begin{comment}
Table \ref{Properties of iPlane and skitter} lists detailed properties of  of the two datasets and we can easily find that the two datasets are different from each other, especially on the average degree.
\end{comment}
\begin{comment}
Besides, Fig. \ref{figure1} shows the  of  of the two datasets, respectively.
From Table \ref{Properties of iPlane and skitter},and Fig. \ref{figure1}, we can easily find that these two are both scale-free networks. We could also learn from Table \ref{Properties of iPlane and
skitter} that the two datasets are different from each
other on {\it average degree}, {\it density}, {\it clustering
coefficient}, and {\it heterogeneity}. However, the other three
features are somewhat related to {\it average degree}. Thus,
we assume that it may be the {\it average degree} that makes the two network
datasets distinguishing.
\end{comment}

\section{New Simulation Models}
\label{sec:models}

In this section, we first describe the motivation that propels us to propose better simulating models for the realistic Internet routing. Then, we give an introduction to our two weighted based shortest path models. 
\subsection{Motivation}
\label{subsec:motivation}
In this subsection, we deeply analyze the real traceroute traces of the two real datasets and the corresponding unweighted shortest paths from all the sources to all the destinations. Unfortunately, there are large differences in the routes length distribution and the node degrees visited probabilities distribution between the real traces and the corresponding unweighted shortest paths.

\subsubsection{Routes Length Distribution}
Fig. \ref{figure2} shows the routes length distributions of the real traces and the corresponding unweighted shortest paths of the two datasets, respectively. As can be seen, the two distributions have large difference with each other. Also, the real traceroute traces have a mean route length of 14.11 hops and 14.57 hops, while the unweighted shortest paths have an average route length of 6.36 hops and 7.39 hops on iPlane and skitter datasets, respectively. The former is almost two times longer than the latter. It strongly indicates that \textit{USPM} is an improper method to simulate the real Internet routes.

\begin{figure}[!t]
\centering
\subfigure[iPlane]{
\label{figure2:subfig:a}
\includegraphics[scale=0.265]{figure/figure2a.eps}
}
\hspace{2em}
\subfigure[skitter]{
\label{figure2:subfig:b}
\includegraphics[scale=0.265]{figure/figure2b.eps}
}
\vspace{-1.7em}
\caption{The routes length distribution of the real traceroute traces and the corresponding unweighted shortest paths of the two real datasets: iPlane and skitter, respectively. Here, the hop of routes is denoted as .}
\label{figure2}
\vspace{-0.8em}
\end{figure}


\begin{figure}[!t]
\centering
\subfigure[Real traces]{
\label{figure3:subfig:a}
\includegraphics[scale=0.265]{figure/figure3a.eps}
}
\hspace{2em}
\subfigure[Shortest paths]{
\label{figure3:subfig:b}
\includegraphics[scale=0.265]{figure/figure3b.eps}
}
\vspace{-1.7em}
\caption{The node degrees visited probabilities distributions of the real traces and the corresponding unweighted shortest paths on iPlane dataset.}
\label{figure3}
\vspace{-0.8em}
\end{figure}

\begin{figure}[!t]
\centering
\subfigure[Real traces]{
\label{figure4:subfig:a}
\includegraphics[scale=0.265]{figure/figure4a.eps}
}
\hspace{2em}
\subfigure[Shortest paths]{
\label{figure4:subfig:b}
\includegraphics[scale=0.265]{figure/figure4b.eps}
}
\vspace{-1.7em}
\caption{The node degrees visited probabilities distributions of the real traces and the corresponding unweighted shortest paths on skitter dataset.}
\label{figure4}
\vspace{-0.5em}
\end{figure}



\subsubsection{Node Degree Visited Probability Distribution}
The node degrees visited probabilities distributions of the real
traces and the unweighted shortest paths of the two
datasets are shown in Figures \ref{figure3} and \ref{figure4},
respectively. It is interesting to note that in Figures
\ref{figure3:subfig:a} and \ref{figure4:subfig:a}, for some nodes
with high degrees, they might be rarely visited in the real routes,
which in another way implies that the assumption \textit{NDM} has made is unsatisfactory. It is also worthy noting that for some high degree nodes in unweighted shortest paths, their visiting possibilities are as large as  (a magnitude larger than the real traces with probabilities being ), which indicates that \textit{USPM} is prone to visit high degree nodes more. In
addition, Figures \ref{figure3} and \ref{figure4}
show that when the node degrees are large, their visited probabilities observed from unweighted shortest paths, from 
to , fluctuate more severely than those observed from the
real traces, from  to . Hence, \textit{USPM} differs significantly from the real case.

The above two phenomena motivate us to put forward better simulating models to more accurately characterize the two kinds of distributions above, especially the routes length distribution. 
\subsection{Local Information Based Model (\textit{LIM})}
In scale-free networks with degree distributions following
power-law, previous researches \cite{TrafficDynamicLocal,EfficientRouting}
have pointed out that unweighted shortest path based routing
strategy, that is \textit{USPM}, would reduce the communication
efficiency of networks. The reason is that such kind of strategy inclines to transmit traffic to the hubs with large degrees, which would easily introduce traffic jam to these hubs thereby inducing the decrease of the network communication capability. The situation would be severer if using \textit{NDM} since it always routes packets to the highest degree neighbors of nodes. To avoid this issue and so forth to better simulate the Internet routing, we propose a new model that uses the local information

to assign weight to the edge between nodes  and  with  the current node and  the next routing node, where the sum runs over the neighbors of ,  is the degree of node , and  is a tunable parameter. We regard the graphs as bidirectional ones and
the weights of the bidirectional edges between two nodes are different, which accords to the different path features of up and down links in reality. After weighing, we carry out weighted shortest path method to simulate the real routes. In this model,
we could intuitively learn that here  means the links connected to the neighbors with
higher degrees might be preferentially introduced into the shortest
path, while , the situation would be reversed to choose the links connected to the lower degree neighbors with higher priority.

\subsection{Path Feature Based Model (\textit{PFM})}
In this subsection, we describe \textit{PFM} which uses the path features to add weights to the corresponding edges and then simulates the real Internet routes by conducting weighted shortest path algorithm. Also, the weights of the bidirectional edges between each pair of connected nodes are different. Weights of links () are assigned based on the bounded Pareto distribution, which is the simplest heavy-tailed distribution with the probability density function

where  determines the shape,  denotes the minimal value, and  denotes the maximal value.

This model is mainly driven by the fact that
in Internet heavy-tailed distributions have been observed in the context of traffic characterization, known as the self-similar nature of traffic \cite{WideAreaTraffic,WWWTraffic,ParetoModulatedModel}.
In addition, \cite{EndToEndRouting} observed high variability in path features, such as round-trip time, loss rate, and bandwidth, and wide variability is one of the characteristics of heavy-tailed distributions. Consequently, in this model, we allocate weights to edges according to the Pareto distribution of the corresponding path attributes, such as delay, loss rate, available bandwidth, and traffic load, and then simulate the real routes by performing weighted shortest path method. It is consistent with the real-world scenario that a packet is routed to the link with smaller delay, smaller loss rate, lighter traffic load, larger available bandwidth, and so on, in order to reach the target more efficiently and with a higher delivery success rate.

In summary, we present two new models with tunable parameters , which makes our models much more flexible than other models. In fact, different values of the parameter would generate different weights for the edges and stand for diverse physical link situations in real Internet routes. In the next section, we evaluate these models and determine the best configuration of our models.

\section{Evaluation and Discussion}
\label{sec:evaluation}
This section first evaluates our models, i.e., compares
the performance of \textit{LIM}, \textit{PFM}, \textit{NDM}
\cite{SimulatingInternetRoute}, and \textit{USPM} on the estimated accuracy of the structural properties of the real
network datasets. Then, we explain why our models are better than other existing models and discuss the implications of optimal parameters.

\subsection{Evaluation}
In this subsection, we evaluate our models on the two real datasets: iPlane and skitter. The evaluation examines their accuracies of simulating the routes length distribution and estimating the other network topological properties, including {\it average degree}, {\it degree distribution power-law exponent}, {\it clustering coefficient}, and {\it heterogeneity}. The simulation experiments of all the four models are conducted from 140 sources to common 20,419 destinations on iPlane dataset and from 21 sources to a common set of 157,290 destinations on skitter dataset as described in Section~\ref{sec:preliminaries}. For parameter configuration, we set  for \textit{LIM} and we configure , , and  for \textit{PFM}. We adopt the mean of 100
independent experiments as the final result for each .
\begin{comment}
\begin{figure*}[!t]
\centering
\subfigure[iPlane]{
\label{figure5:subfig:a}
\includegraphics[scale=0.265]{figure/figure5a.eps}
}
\subfigure[skitter]{
\label{figure5:subfig:b}
\includegraphics[scale=0.265]{figure/figure5b.eps}
}
\vspace{-1.7em}
\caption{The average routes length of the real traces and the four models on two real datasets: iPlane and skitter, respectively.}
\label{figure5}
\vspace{-0.5em}
\end{figure*}
\begin{figure*}[!t]
\centering
\subfigure[iPlane]{
\label{figure6:subfig:a}
\includegraphics[scale=0.265]{figure/figure6a.eps}
}
\subfigure[skitter]{
\label{figure6:subfig:b}
\includegraphics[scale=0.265]{figure/figure6b.eps}
}
\vspace{-1.7em}
\caption{The Kullback-Leibler divergence of the routes length distribution between the real traces and the four models on two real datasets: iPlane and skitter, respectively.}
\label{figure6}
\vspace{-0.5em}
\end{figure*}
\end{comment}
\subsubsection{Routes Length Distribution}
\begin{comment}
In Fig. \ref{figure5}, we present the average routes length of the
real traces and the corresponding four models on two
datasets: iPlane and skitter, respectively. As can be seen, the real
trace has a mean route length of 14.11 hops and 14.57 hops; \textit{NDM} has an average route length of 11.51 hops and 11.87 hops;
while \textit{USPM} has an average route length of 6.22 hops and 8.17 hops
on iPlane and skitter, respectively. However, \textit{LIM}, with certain
values of , could have the same mean route length with those of the real trace, which indicates that \textit{LIM} can better
simulate the average route length of the real routes if we tune
 to proper values. Regarding to \textit{PFM}, it performs poorly on iPlane, while when , it could also obtain the accurate average route length of the real traces on skitter. Besides, we calculate the Kullback-Leibler
divergence \cite{KullbackLeibler}

with the real traces, where  is the routes length distribution of the real traceroute traces,
and  is respectively the routes length distribution of the four models. The closer  to 0, the more similar the two distributions are. Fig. \ref{figure6} shows how it varies for different models on the two real network datasets, respectively. From Figure
\ref{figure6}, we could observe that when  and
,  is almost equal to 0 for \textit{LIM}
on iPlane and skitter datasets, respectively; when  and
, \textit{PFM} respectively gets the minimal
 on the two datasets. What's more, it illustrates that
when  and , \textit{LIM} is better
than \textit{NDM} on iPlane dataset; when  and
, \textit{LIM} is better than \textit{NDM} on
skitter dataset. In most time, \textit{PFM} is much worse than the
real one. While, \textit{USPM} is the worst one. Therefore, it implies
that, in many cases, \textit{LIM} could outstandingly simulate the
routes length distribution of the real Internet routes.
\end{comment}
\begin{figure*}[!t]
\centering
\subfigure[iPlane]{
\label{figure7:subfig:a}
\includegraphics[scale=0.265]{figure/figure7a.eps}
}
\hspace{2em}
\subfigure[skitter]{
\label{figure7:subfig:b}
\includegraphics[scale=0.265]{figure/figure7b.eps}
}
\vspace{-1.7em}
\caption{The routes length distribution of the real
traces and the four models on two real datasets: iPlane and skitter, respectively.
Here the hop of routes is denoted as .}
\label{figure7}
\vspace{-0.5em}
\end{figure*}

Figure~\ref{figure7} presents the routes length distribution of the real
traces and the four models on the two real datasets, respectively. The routes length distributions of \textit{LIM} and \textit{PFM} are their optimal ones obtained by tuning the parameter . On iPlane, the best  for \textit{LIM} and
\textit{PFM} is  and , respectively; when on skitter dataset,  and  is the best for \textit{LIM} and \textit{PFM}, respectively. It can be seen that the routes length
distributions of \textit{LIM} on iPlane and skitter datasets are
almost the same with those of the real trace. While, with regard to the other
three models, there is still a considerable gap as compared to the real one,
especially for \textit{NDM} and \textit{USPM}. Consequently, it
can be concluded that \textit{LIM} is the best model to
simulate the routes length distribution of the real routes. Note that we will not consider \textit{USPM} in the following evaluation for its extremely bad simulation of the routes length distribution of the real Internet routes.

\begin{figure*}[!t]
\centering
\subfigure[iPlane]{
\label{figure8:subfig:a}
\includegraphics[scale=0.265]{figure/figure8a.eps}
}
\hspace{2em}
\subfigure[skitter]{
\label{figure8:subfig:b}
\includegraphics[scale=0.265]{figure/figure8b.eps}
}
\vspace{-1.7em}
\caption{The {\it average degree} of the sampled graphs by using the three models, compared with that of , on two real datasets: iPlane and skitter, respectively.}
\label{figure8}
\vspace{-0.8em}
\end{figure*}



\subsubsection{Average Degree}
Fig. \ref{figure8} shows the {\it average degree} of  and
those of the sampled graphs by adopting the three models:
\textit{LIM}, \textit{PFM}, and \textit{NDM}. It indicates that our
two models, in most cases, are much better than \textit{NDM}. \textit{LIM} reaches its peaks on the two datasets when  and , respectively; \textit{PFM} reaches its peaks on both datasets when
. Besides,
\textit{PFM} is always better than \textit{NDM}; \textit{LIM} is much better than
\textit{NDM} on the two datasets when  and , respectively. Hence, our two models are more effective
than \textit{NDM} on sampling the {\it average degree} of the underlying network.


\begin{figure*}[!t]
\centering
\subfigure[iPlane]{
\label{figure9:subfig:a}
\includegraphics[scale=0.265]{figure/figure9a.eps}
}
\hspace{2em}
\subfigure[skitter]{
\label{figure9:subfig:b}
\includegraphics[scale=0.265]{figure/figure9b.eps}
}
\vspace{-1.7em}
\caption{The {\it degree distribution} power-law exponent  of the sampled graphs by using the three models, compared with that of , on the two real datasets, respectively.}
\label{figure9}
\vspace{-0.5em}
\end{figure*}



\subsubsection{Power-law Degree Exponent}
Fig. \ref{figure9} presents the {\it degree distribution} power-law exponent , which is calculated with the
least square method \cite{LeastSquareMethod}, of  and those of
the sampled graphs by adopting the three models. It can be observed that our models
could achieve the same values with the real trace.
\textit{LIM} obtains the same values with  on both iPlane and skitter datasets when ; \textit{PFM} gets the same values with  on
the two datasets when  and
, respectively. Moreover, \textit{PFM} is, in most cases, better
than \textit{NDM}; \textit{LIM} is better than \textit{NDM} on the two network datasets when  and , respectively. Consequently, we can infer that our two models perform much better than \textit{NDM} on simulating the {\it degree distribution}.



\begin{figure*}[!t]
\centering
\subfigure[iPlane]{
\label{figure10:subfig:a}
\includegraphics[scale=0.265]{figure/figure10a.eps}
}
\hspace{2em}
\subfigure[skitter]{
\label{figure10:subfig:b}
\includegraphics[scale=0.265]{figure/figure10b.eps}
}
\vspace{-1.7em}
\caption{The {\it clustering coefficient} of the sampled graphs by using the three models, compared with that of , on two real datasets: iPlane and skitter, respectively.}
\label{figure10}
\vspace{-0.8em}
\end{figure*}



\subsubsection{Clustering Coefficient}
Fig. \ref{figure10} shows the {\it clustering coefficient} of  and those of the sampled graphs by adopting the three models. It indicates that our two models, in most cases, are much better than \textit{NDM}. \textit{LIM} reaches its peaks on the two datasets, iPlane and skitter, when  and , respectively; \textit{PFM} reaches its peaks on both datasets when . Further, \textit{PFM} is always better than \textit{NDM} and \textit{LIM} is much more effective than \textit{NDM} on both datasets when . Therefore, our two models could obtain a more accurate estimation of the {\it clustering coefficient} of the underlying network.

\begin{figure*}[!t]
\centering
\subfigure[iPlane]{
\label{figure11:subfig:a}
\includegraphics[scale=0.265]{figure/figure11a.eps}
}
\hspace{2em}
\subfigure[skitter]{
\label{figure11:subfig:b}
\includegraphics[scale=0.265]{figure/figure11b.eps}
}
\vspace{-1.7em}
\caption{The {\it heterogeneity} of the sampled graphs by using the three models, compared with that of , on two real datasets: iPlane and skitter, respectively.}
\label{figure11}
\vspace{-0.5em}
\end{figure*}


\subsubsection{Heterogeneity}
Fig. \ref{figure11} presents the {\it heterogeneity} of 
and those of the sampled graphs by adopting the three models. It can be found that
our models could achieve almost the same values with the real trace. \textit{LIM} obtains the same values with  on both datasets when ; \textit{PFM} gets the same values with  on iPlane and skitter datasets when  and , respectively. Besides, \textit{PFM} is always better than \textit{NDM}; \textit{LIM} is better
than \textit{NDM} on iPlane dataset when  and , and it is better than \textit{NDM} on
skitter dataset when . Consequently, it could be concluded that
our models could sample the {\it heterogeneity} of the network with a convincing fidelity.

\subsection{Evaluation Conclusion}
In previous subsection, we evaluate our models both on routes length
distribution and on the other network topological properties, such as {\it
average degree}, {\it degree distribution}, {\it clustering
coefficient}, and {\it heterogeneity}, on the two real datasets: iPlane
and skitter. The evaluation implies that our two models are more effective than \textit{NDM} not only on the simulating of routes length distribution, but also on the sampling of the other structural properties. In addition, our models
do not give rise to more consuming time and have the similar
computing complexity as \textit{NDM} and \textit{USPM}. However,
\textit{PFM} is much worse than \textit{LIM} on modeling the routes
length distribution. Particularly, it has been found in reality that for self-similar traffic,
 for \textit{PFM} should be in the region of 
\cite{ParetoModulatedModel}, which is not overlapped with the proper
range we find above. Therefore, for a better simulation of the real
Internet routes, we recommend \textit{LIM} instead of \textit{PFM}.
But it is also important to point out that one has to
choose different proper  for \textit{LIM} in order to
achieve the best sampling of different network properties, which is shown in
Table \ref{Different alpha for different properties}. However,
it is interesting to observe from Table \ref{Different alpha for
different properties} and also from previous subsection that when
setting , almost all of the topological properties could be under
accurate estimation. Thus, to better model the Internet routing, \textit{LIM} with  is recommended; in
addition, researchers could set  to be around 0.5 to more
accurately estimate the other network structural properties except the routes
length distribution.
\begin{table}[!t]
\footnotesize
\caption{The best value of \textit{LIM}() for simulating and sampling different network topological properties of the real Internet, including iPlane and skitter datasets.}
\label{Different alpha for different properties}
\vspace{-1em}
\centering
\setlength{\tabcolsep}{4pt}
\begin{tabular}{c|c||c}
\hline \hline
    Dataset & Topological property & \textit{LIM}() \\
\hline
    & Routes length distribution & 1.6  \\
    \cline{2-3}
           &  & 0.4  \\
    \cline{2-3}
    iPlane &  & 0.5  \\
    \cline{2-3}
           &  & 0.5  \\
    \cline{2-3}
           &  & 0.4  \\
\hline
    & Routes length distribution & 0.9  \\
    \cline{2-3}
           &  & 0.6  \\
    \cline{2-3}
    skitter &  & 0.5  \\
    \cline{2-3}
           &  & 0.7  \\
    \cline{2-3}
           &  & 0.4  \\
\hline \hline
\end{tabular}
\end{table}
\begin{comment}
\begin{table*}[!t]
\tiny
\caption{The range of  that makes \textit{LIM} much better than \textit{NDM} and the best value of , in the bracket, for sampling different topological properties of BA and GLP models, both with {\it average degree} varying from 6 to 40. }
\label{DiffAlphaForDiffDegrees}
\vspace{-1.2em}
\centering
\setlength{\tabcolsep}{3.5pt}
\begin{tabular}{c|c||c|c|c|c}
\hline \hline
    Graph model & Graph  &   &  &  & \\
\hline
& 6 &  &  &  & \\
    \cline{2-6}
           & 12 &  &  &  & \\
    \cline{2-6}
    BA model & 20 &  &  &  & \\
    \cline{2-6}
           & 30 &  &  &  & \\
    \cline{2-6}
           & 40 &  &  &  & \\
\hline \hline
           & 6 &  &  &  & \\
    \cline{2-6}
           & 12 &  &  &  & \\
    \cline{2-6}
    GLP model & 20 &  &  &  & \\
    \cline{2-6}
           & 30 &  &  &  & \\
    \cline{2-6}
           & 40 &  &  &  & \\
\hline \hline
\end{tabular}
\vspace{-0.5em}
\end{table*}



\subsection{Validation with Synthetic Graph Models}
The evaluation on real network datasets implies that \textit{LIM} is much better than the previous models and also  in the range of , more specifically  around 0.5, is recommended for best estimating almost all the network structural properties. However, it is significant to answer the questions that is its good performance just specific to or are  in the region of  and around 0.5 only suitable to the certain two real datasets, iPlane and CAIDA skitter, above? To answer these questions, it is indispensable to examine its performance and also the universalities of  in the range of  and around 0.5 on other real datasets or synthetic graphs. Due to limited availability of the real datasets, we just adopt synthetic graphs, including BA model \cite{BAModel} and GLP model \cite{GLPModel}, here to further investigate its performance and determine whether  in the region of  and around 0.5 are general for all the networks or not.

First, we conduct experiments on BA and GLP models, both with {\it average degree} varying from 6 to 40. Table \ref{DiffAlphaForDiffDegrees} presents the range of  with which \textit{LIM} performs much better than \textit{NDM}. It also shows the values of  with estimated properties are equal to or the most close to the true properties of BA and GLP models. The number of nodes is 10,000 and the simulation experiments for the four models are conducted from 200 sources to 2,000 destinations that are randomly selected from the nodes. Also, the results are the average ones of 100 experiments. From this table, we could see that, in most cases, \textit{LIM} is much better than \textit{NDM} on estimating the network topological properties of the two synthetic graph models. Furthermore, as shown in Table \ref{DiffAlphaForDiffDegrees},  in the range of  and around 0.5 are the best configuration for sampling different structural properties although the best  for  of GLP model is a little small. Thus, through the examination on BA and GLP models with different {\it average degree}, it could be concluded that \textit{LIM} is actually the best model now and also  in the region of  and around 0.5 for \textit{LIM} are universal for best sampling of different network topological properties of the underlying networks.



\begin{table*}[!t]
\tiny
\caption{The range of  that makes \textit{LIM} much better than \textit{NDM} and the best value of , in the bracket, for estimating different topological properties of BA and GLP models, both with {\it average degree} 12 and with different {\it clustering coefficient}. }
\label{DiffAlphaForDiffClustering}
\vspace{-1.2em}
\centering
\setlength{\tabcolsep}{3.5pt}
\begin{tabular}{c|c||c|c|c|c}
\hline \hline
    Graph model & Graph  &   &  &  & \\
\hline
& 0.006841 &  &  &  & \\
    \cline{2-6}
           & 0.020909 &  &  &  & \\
    \cline{2-6}
           & 0.031463 &  &  &  & \\
    \cline{2-6}
    BA model & 0.039832 &  &  &  & \\
    \cline{2-6}
           & 0.049039 &  &  &  & \\
    \cline{2-6}
           & 0.059272 &  &  &  & \\
    \cline{2-6}
           & 0.069599 &  &  &  & \\
    \cline{2-6}
           & 0.080583 &  &  &  & \\
\hline \hline
           & 0.035493 &  &  &  & \\
    \cline{2-6}
           & 0.038996 &  &  &  & \\
     \cline{2-6}
           & 0.046869 &  &  &  & \\
     \cline{2-6}
     GLP model & 0.052238 &  &  &  & \\
     \cline{2-6}
           & 0.057649 &  &  &  & \\
     \cline{2-6}
           & 0.060974 &  &  &  & \\
     \cline{2-6}
           & 0.061797 &  &  &  & \\
     \cline{2-6}
           & 0.062267 &  &  &  & \\
\hline \hline
\end{tabular}
\vspace{-0.5em}
\end{table*}



Second, in order to further check the universalities of  in the range of  and around 0.5, we conduct experiments on two specific graphs of Table \ref{DiffAlphaForDiffDegrees}, both with {\it average degree} 12 and with initial {\it clustering coefficient} 0.006053 and 0.059725 for BA and GLP models, respectively. The experiments are done on different variations, with different {\it clustering coefficient}, of the two original graphs obtained by exchanging the edges between nodes but without changing the degree of the node. Table \ref{DiffAlphaForDiffClustering} presents the range of  with which \textit{LIM} performs much better than \textit{NDM}. Besides, it shows the values of  with estimated properties are equal to or the most close to the true properties of BA and GLP models. The number of nodes is 10,000 and the simulation experiments for the four models are conducted from 200 sources to 2,000 destinations that are randomly selected from the nodes. Also, the results are the average ones of 100 experiments. This table shows that, in most time, \textit{LIM} is much better than \textit{NDM} on estimating the network topological properties of the two synthetic graph models. Moreover, It implies the region of  and also the best value of  are almost unchanged on the synthetic graphs with different {\it clustering coefficient} corresponding to those of the two original graphs with {\it average degree} 12 in Table \ref{DiffAlphaForDiffDegrees}. Consequently, it could be concluded that \textit{LIM} is actually the best model now and also  in the region of  and around 0.5 for \textit{LIM} are general for best estimating of different topological properties of the underlying networks.

In conclusion, through the examination on both real network datasets and also various synthetic graphs, \textit{LIM} with  in the range of , more specifically  around 0.5, is recommended to best estimate almost all the topological properties of the underlying network.
\end{comment}

\subsection{Discussion}
In this subsection, we explain why the proper value of
 for \textit{LIM} should be in the range of  and the implications of such range.


\begin{figure*}[!t]
\centering
\subfigure[iPlane]{
\label{figure12:subfig:a}
\includegraphics[scale=0.3]{figure/figure12a.eps}
}
\hspace{2em}
\subfigure[skitter]{
\label{figure12:subfig:b}
\includegraphics[scale=0.3]{figure/figure12b.eps}
}
\vspace{-1.7em}
\caption{The degree distribution of routes in each hop extracted from the real traceroute traces of iPlane and skitter datasets, respectively. ( is node degree;  represents hop;  represents the possibility of  in the node degree distribution of .)}
\label{figure12}
\vspace{-0.8em}
\end{figure*}


As stated in
\cite{TrafficDynamicLocal,EfficientRouting}
that when  in Eq.~\ref{eq:lim} is positive, the traffic
load on each node and also the packet traveling time would follow power-law
distributions, which indicates the highly heterogeneous status; when
, the traffic load on each node and the packet traveling
time would display as Poisson distribution and exponential
distribution, respectively, which represents a homogeneous state.
Additionally, a positive  in \cite{TrafficDynamicLocal,EfficientRouting} represents that packets tend to be routed
to the nodes with large degrees, which induces the hubs easily turn
to be jammed and then decreases the communication capability of the
networks. While, a negative  in \cite{TrafficDynamicLocal,EfficientRouting} illustrates that the
communication networks, in some cases, try to shunt some traffic
from the hubs to the nodes with small degrees in order to reduce the
loads of those hubs. Therefore, in those research work, a negative  is better than a
positive . While the negative  in \cite{TrafficDynamicLocal,EfficientRouting} is corresponding to the positive  in our research because we adopt the values of Eq.~\ref{eq:lim}
as the weights of the edges and then perform the weighted shortest
path algorithm. There is also an interesting phenomenon that
in the real Internet routes, packets incline to move to nodes
with small degrees, which could be seen from Fig. \ref{figure12}
that presents the degree distribution of routes in each hop
extracted from the real traceroute traces of iPlane and skitter
datasets, respectively. It shows that, in most cases, routes with their routing at each hop tend to move
to the small degree nodes, especially the nodes with degrees from 1 to 10.
This phenomenon surprisingly justifies that the route strategy with  in \cite{TrafficDynamicLocal,EfficientRouting}
and \textit{LIM} with  in our work reflect the realistic
situation in the Internet routes. Meanwhile, it could be used to
explain the poor performance of \textit{USPM} and \textit{NDM}, which are
their intents of targeting the large degree nodes.


\begin{figure}[!t]
\centering
\includegraphics[scale=0.25]{figure/figure14.eps}
\vspace{-1.7em}
\caption{The entropy of each hop  extracted from the real traceroute traces on iPlane and skitter datasets, respectively.}
\label{figure14}
\vspace{-0.5em}
\end{figure}


Moreover, \cite{TrafficDynamicLocal,EfficientRouting}
declared that the packet traveling time increases as the decrease of the negative  in their route
strategy. It is the cost that the network should pay for reducing the burdens of hubs by transferring some traffic to low degree nodes, which, in purpose, enhances the communication capability of the network. This fact also reflects the phenomenon that the increase of the average length of the routes as the increase of the positive  in \textit{LIM}.
However, it is also stated in \cite{TrafficDynamicLocal,EfficientRouting} that  is far from the larger the better, because too large  may sharply increase the packet traveling time, which would in turn decrease the network's communication capability too. In addition, as shown in Fig. \ref{figure12}, especially in Fig. \ref{figure12:subfig:b}, the node degree (from 1 to 100) distributes relatively evenly for some small hops (from 5 to 15), which indicates that the low degree nodes are relatively less dominant in these hops as compared to other large ones. Moreover, we define the entropy of each hop as

where  is the hop,  is the node degree that appears in this hop, and  is its probability. It is worthy noting that here we omit the  with . According to the definition, larger  means in that hop , each node degree distributes relatively evenly, while lower  means some node degrees are dominant in this hop. Fig. \ref{figure14} shows that on both iPlane and skitter datasets, when the hop is in the range of , different node degrees distribute somewhat evenly, while in other hops, mainly the low node degrees dominate. It tells us that for many small hops, the nodes with intermediate degrees (from 10 to 100) would be considered in the routing, while too large  for \textit{LIM} would indeed ignore them entirely. Consequently,  in \textit{LIM} should be in the range of  instead of exceeding 2.

To sum up, our simulating model \textit{LIM}, with  in the range of , represents the real situation of the
Internet routes. It actually reflects the design philosophy of Internet, trying to achieve a trade-off between network communication efficiency (packet traveling time) and traffic load balance among nodes. The  range  for \textit{LIM} would reach a win-win situation on both aspects. Thus, \textit{LIM} is a better model to simulate the routes in the Internet.



\begin{comment}
\begin{figure*}[!t]
\centering
\subfigure[1 source, 1,000 destinations]{
\label{figure13:subfig:a}
\includegraphics[scale=0.18]{figure/figure13a.eps}
}
\subfigure[5 sources, 1,000 destinations]{
\label{figure13:subfig:b}
\includegraphics[scale=0.18]{figure/figure13b.eps}
}
\subfigure[10 sources, 1,000 destinations]{
\label{figure13:subfig:c}
\includegraphics[scale=0.18]{figure/figure13c.eps}
}
\vspace{-1.5em}
\caption{ of subgraphs sampled from \textit{ER} random
graph , with the number of sources equal
to 1, 5, and 10, and the number of destinations equal to 1,000.}
\label{figure13}
\vspace{-0.5em}
\end{figure*}

\section{About The Sampling Bias}
\label{sec:bias}
In Section~\ref{sec:evaluation}, we have demonstrated that
\textit{USPM} differs significantly from the real situation of routes in Internet and our model \textit{LIM} could better simulate the Internet routing. Additionally, traceroute, based on the realistic Internet routing, is an universally adopted tool to obtain the worldwide Internet topology. Therefore, it is naturally to post the question that is
traceroute really biased in sampling the Internet here? In prior
works~\cite{SamplingBiases,AccuracyScalingINETMap,BiasTracerouteSampling},
the problem of sampling bias has been deeply explored, which could
be stated that traceroute-like methods produce fundamental biases in
observed topological features, particularly the degree distribution.
Yet, in their works, traceroute simulation models are usually
\textit{USPM} based versions. For this reason, we suspect that it might
be \textit{USPM} that induces the corresponding sampling bias, while the true
traceroute might be unbiased or at least less biased.

In order to justify the conjecture we have just made, we
conduct the same experiments as that in \cite{SamplingBiases}. We
similarly adopt the classical Erd{\"{o}}s-R\'{e}nyi (\textit{ER}) random
graph~\cite{RandomGraphModel}, with 100,000 nodes, 749,076 edges (), and the {\it average degree} is 15. In~\cite{SamplingBiases},
the authors simulate traceroute by utilizing an \textit{USPM} based
model, which allocates tiny random weights to edges in the
network. Hence, we denote their model as \textit{RWSPM}. In the
following experiments, we perform simulation for each model 100
times and obtain the mean values as the final results. Figure
\ref{figure13} presents the  of subgraphs sampled from \textit{ER}
random graph by adopting \textit{RWSPM} and \textit{LIM}, with the number of sources equal to 1, 5, and 10, and
the number of destinations equal to 1,000. Here, we set 
for \textit{LIM}. It can be seen that for \textit{LIM}, these  are closer to the real degree
distributions instead of following obvious power-law for all
configurations of sources and destinations. For instance, when we
fit the curves of  obtained from \textit{LIM} to power-law, which
are shown as the solid lines, we surprisingly get . While
regarding to \textit{RWSPM}, its  curves obey apparently
power-law with  as shown by the dotted lines, especially
for large degrees. Thus, it is reasonable to suspect that it is
\textit{USPM} that leads to the sampling bias.

To conclude, using our model \textit{LIM}, with , the
sampling bias situation is less serious than that of adopting
\textit{USPM}. While \textit{LIM} is better than the existing
models, including \textit{USPM}, on simulating the routes length
distribution and on sampling the other topological properties. Hence, it
could be conjectured that it might be \textit{USPM} that incurs the
sampling bias, while the true traceroute might be less biased, at least.
\end{comment}

\section{Conclusion}
\label{sec:conclusion}

In this paper, we deeply study the real Internet routes and propose two novel models to well simulate the
Internet routing. Through thoroughly comparison with existing models, we
find that one of our models, \textit{LIM}, could outstandingly
simulate the routes length distribution and better estimate the other
topological properties of the Internet topology with proper
configurations of the parameter . Besides, we recommend research
community using \textit{LIM} with  in the range of  to achieve better estimation on the
overall properties of the real routes in Internet, more specifically with  around 0.5 to more
accurately estimate the other network structural properties except the routes
length distribution.
\begin{comment}
What's more, through deeply
resurveying the sampling bias problem of traceroute-like methods by
considering \textit{LIM}, we conjecture that the corresponding
sampling bias is due to \textit{USPM} adopted previously. While,
the true traceroute in reality might be less biased, at least.
\end{comment}

\begin{thebibliography}{10}
\providecommand{\url}[1]{\texttt{#1}}
\providecommand{\urlprefix}{URL }

\bibitem{Scamper}
\uppercase{CAIDA} \uppercase{S}camper tool,
  http://www.caida.org/tools/measurement/scamper/

\bibitem{Skitter}
\uppercase{CAIDA} \uppercase{S}kitter tool,
  http://www.caida.org/tools/measurement/skitter/

\bibitem{DIMES}
\uppercase{DIMES}@home project, http://www.netdimes.org/new/

\bibitem{BiasTracerouteSampling}
Achlioptas, D., Clauset, A., Kempe, D., Moore, C.: On the bias of traceroute
  sampling: Or, power-law degree distributions in regular graphs. J. ACM  56,
  21:1--21:28 (Jul 2009)

\bibitem{LoadBalance}
Augustin, B., Friedman, T., Teixeira, R.: Measuring load-balanced paths in the
  \uppercase{i}nternet. In: Proceedings of IMC. pp. 149--160. ACM, New York,
  NY, USA (2007)

\bibitem{NTC}
Bourgeau, T., Friedman, T.: Efficient \uppercase{IP}-level network topology
  capture. In: Passive and Active Measurement, vol. 7799, pp. 11--20 (2013)

\bibitem{TraceroutePattern}
Brownlee, N.: On searching for patterns in traceroute responses. In: Passive
  and Active Measurement, vol. 8362, pp. 67--76 (2014)

\bibitem{AccuracyScalingINETMap}
Clauset, A., Moore, C.: Accuracy and scaling phenomena in \uppercase{I}nternet
  mapping. Phys. Rev. Lett.  94(1),  018701 (Jan 2005)

\bibitem{LeastSquareMethod}
Clauset, A., Shalizi, C.R., Newman, M.E.J.: Power-law distributions in
  empirical data. SIAM Review  51,  661--703 (2009)

\bibitem{WWWTraffic}
Crovella, M.E., Bestavros, A.: Self-similarity in world wide web traffic:
  evidence and possible causes. IEEE/ACM Trans. Netw.  5,  835--846 (Dec 1997)

\bibitem{LoadBalance2}
Cunha, I., Teixeira, R., Diot, C.: Measuring and characterizing end-to-end
  route dynamics in the presence of load balancing. In: Passive and Active
  Measurement. pp. 235--244 (2011)

\bibitem{TheoryAndSimulations}
Dall'Asta, L., Alvarez-Hamelin, I., Barrat, A., V\'{a}zquez, A., Vespignani,
  A.: Exploring networks with traceroute-like probes: theory and simulations.
  Theor. Comput. Sci.  355(1),  6--24 (2006)

\bibitem{PowerLawRelationship}
Faloutsos, M., Faloutsos, P., Faloutsos, C.: On power-law relationships of the
  internet topology. SIGCOMM Comput. Commun. Rev.  29(4),  251--262 (Aug 1999)

\bibitem{Heuristics}
Govindan, R., Tangmunarunkit, H.: Heuristics for \uppercase{I}nternet map
  discovery. In: IEEE INFOCOM. pp. 1371--1380 (2000)

\bibitem{RelevanceOfMassivelyQualitative}
Jean-Loup, G., Latapy, M., Magoni, D.: Relevance of massively distributed
  explorations of the \uppercase{I}nternet topology: qualitative results.
  Comput. Netw.  50,  3197--3224 (Nov 2006)

\bibitem{SamplingBiases}
Lakhina, A., Byers, J.W., Crovella, M., Xie, P.: Sampling biases in
  \uppercase{IP} topology measurements. In: IEEE INFOCOM. pp. 332--341 (2003)

\bibitem{ParetoModulatedModel}
Le-Ngoc, T., Subramanian, S.: A pareto-modulated poisson process
  (\uppercase{PMPP}) model for long-range dependent traffic. Comput. Commun.
  23,  123--132 (Jan 2000)

\bibitem{SimulatingInternetRoute}
Leguay, J., Latapy, M., Friedman, T., Salamatian, K.: Describing and simulating
  \uppercase{i}nternet routes. Comput. Netw.  51,  2067--2085 (Jun 2007)

\bibitem{iPlane}
Madhyastha, H.V., Isdal, T., Piatek, M., Dixon, C., Anderson, T.,
  Krishnamurthy, A., Venkataramani, A.: i\uppercase{P}lane: an information
  plane for distributed services. In: Proceedings of OSDI. pp. 367--380.
  Berkeley, CA, USA (2006)

\bibitem{WideAreaTraffic}
Paxson, V.: Wide-area traffic: the failure of poisson modeling. IEEE/ACM Trans.
  Netw.  3,  226--244 (Jun 1995)

\bibitem{EndToEndRouting}
Paxson, V.: End-to-end routing behavior in the \uppercase{I}nternet. IEEE/ACM
  Trans. Netw.  5,  601--615 (Oct 1997)

\bibitem{Rocketfuel}
Spring, N., Mahajan, R., Wetherall, D., Anderson, T.: Measuring \uppercase{ISP}
  topologies with rocketfuel. IEEE/ACM Trans. Netw.  12(1),  2--16 (2004)

\bibitem{RouingPolicyImpact}
Tangmunarunkit, H., Govindan, R., Shenker, S., Estrin, D.: The impact of
  routing policy on \uppercase{i}nternet paths. In: in Proc. 20th IEEE INFOCOM.
  pp. 736--742 (2001)

\bibitem{PathDiversity}
Teixeira, R., Marzullo, K., Savage, S., Voelker, G.M.: Characterizing and
  measuring path diversity of \uppercase{i}nternet topologies. In: SIGMETRICS.
  pp. 304--305. ACM, New York, NY, USA (2003)

\bibitem{RealSizeOfSampledNetwork}
Viger, F., Barrat, A., {Dall'Asta}, L., Zhang, C.H., Kolaczyk, E.D.: What is
  the real size of a sampled network? the case of the \uppercase{I}nternet.
  Physical Review E  75(5),  056111 (2007)

\bibitem{TrafficDynamicLocal}
Wang, W.X., Wang, B.H., Yin, C.Y., Xie, Y.B., Zhou, T.: Traffic dynamics based
  on local routing protocol on a scale-free network. Physical Review E  73
  (2006)

\bibitem{EfficientRouting}
Yin, C.Y., Wang, B.H., Wang, W.X., Zhou, T., Yang, H.J.: Efficient routing on
  scale-free networks based on local information. Physics Letters A  351(4-5),
  ~4 (2005)

\end{thebibliography}


\end{document}
