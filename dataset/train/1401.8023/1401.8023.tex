\documentclass{article}
\usepackage{latexsym,graphicx,amsthm}
\newtheorem{lemma}{Lemma}
\newtheorem{theorem}{Theorem}
\usepackage[numbers,sort&compress]{natbib}
\renewcommand{\thefootnote}{\fnsymbol{footnote}}	

\begin{document}

\date{October 24, 2001}

\title{\textbf{Brooks' Vertex-Colouring\\ Theorem in Linear Time} \thanks{Technical Report CS-AAG-2001-05, Basser Department of Computer Science, The University of Sydney, 2001. Completed as an undergraduate research project at The University of Sydney (Sydney, Australia)  and McGill University (Montr\'eal, Canada).}}


\author{Bradley Baetz \qquad David R. Wood\footnotemark[2]}

\footnotetext[2]{School of Mathematical Sciences, Monash University, Melbourne Australia (\texttt{david.wood@monash.edu}). Research supported by the Australian Research Council.}


\maketitle

\begin{abstract} Brooks' Theorem  [R.\ L.\ Brooks, On Colouring the Nodes of a
Network, \emph{Proc.\ Cambridge Philos.\ Soc.} \textbf{37}:194-197, 1941]
states that every graph  with maximum degree , has a
vertex-colouring with  colours, unless  is a complete graph or an
odd cycle, in which case  colours are required. Lov{\'a}sz [L.\
Lov{\'a}sz, Three short proofs in graph theory, \emph{J.\ Combin.\ Theory Ser.}
\textbf{19}:269-271, 1975] gives an algorithmic proof of Brooks' Theorem. 
Unfortunately this proof is missing important details and it is thus unclear
whether it leads to a linear time algorithm. In this paper we give a complete
description of the proof of Lov{\'a}sz, and we derive a linear time algorithm
for determining the vertex-colouring guaranteed by Brooks' Theorem. 
\end{abstract}



\section{Introduction}

Let  be a simple graph with maximum degree . Undefined
graph-theoretic terms can be found in~\cite{CL96}. A \emph{vertex-colouring} of
 is a mapping from the vertex set of  to some set of colours such that
adjacent vertices receive different colours. For convenience we take the set of
colours to be the positive integers . A graph is said to be
-\emph{colourable} if it has a vertex-colouring with at most  colours.
Minimising the number of colours in a vertex-colouring of a given graph is a
fundamental problem in algorithmic graph theory with applications in register
allocation for example. Unfortunately, determining if a given graph is
-colourable is NP-complete~\cite{Karp72}. The sequential greedy algorithm,
which chooses for each vertex  in turn, the minimum colour not used by a
neighbour of , will use at most  colours, since at each vertex 
there is at most  different colours assigned to the neighbours of .
Brooks~\cite{Brooks41} proved the following improvement to this result.

\begin{theorem}[Brooks' Theorem~\cite{Brooks41}]
Every graph  with maximum degree , has a vertex-colouring with 
colours, unless  is a complete graph or an odd cycle, in which case  colours
are required.
\end{theorem}

We say a vertex-colouring of graph  with maximum degree  is a
\emph{Brooks-colouring} if the number of colours is at most , or  if
 is a complete graph or an odd cycle. 

The original proof by Brooks leads to a quadratic time algorithm for
calculating a Brooks-colouring. Since then  Ponstein~\cite{Ponstein69} and
Lov{\'a}sz~\cite{Lovasz75} (also see Bryant~\cite{Bryant96}) describe
algorithmic proofs of Brooks' Theorem. However, the running time of the
resulting algorithms are not analysed, and in the proof by
Lov{\'a}sz~\cite{Lovasz75}, many important details are omitted.
In this paper, we give a complete description of the proof of Brooks' Theorem
due to Lov{\'a}sz~\cite{Lovasz75}, and derive a linear time algorithm for
computing a Brooks-colouring.  


\section{The Details}

We start with the following well-known result,
which can be proved by performing a pre-order traversal of the block-cut-forest
of the given graph, and possibly swapping two colours in each biconnected
component.

\begin{lemma}
\label{lem:NonBiconnected}
Given -colourings of the biconnected components of a graph , a -colouring
of  can be determined in  time.\qed
\end{lemma}


Lemma~\ref{lem:NonBiconnected} implies that we need only describe a linear time
algorithm  for determining a Brooks-colouring in the case of a biconnected
graph. 

\begin{lemma}
\label{lem:SequentialColouring}
If a graph  with maximum degree  contains two vertices  and 
at distance 2 such that  is connected, then a
-colouring of  can be determined in  time.
\end{lemma}
\begin{proof}
Let  be an ordering of the vertices of
 such that  is a vertex adjacent to both  and ,
and for every , the vertex  has at least one neighbour  with
. Since  is connected, such an ordering can be
determined by a depth-first search of . Let the colour of
 and  be . This is valid since  and  are not adjacent. Now, for
each , , colour the vertex  with the minimum positive
integer  which is different from the colours assigned to the neighbours of
 which are already coloured. For each , , the colour
assigned to  is at most  since there are at most 
neighbours of  already coloured. The  colour assigned to  is at most
 since  has two neighbours (namely,  and ) receiving the
same colour. Thus, we have a vertex-colouring of  with at most 
colours. This procedure can be implemented in  time using standard
depth-first search algorithms. 
\end{proof}

\begin{lemma}
\label{lem:FindAB}
Let  be a biconnected graph which is not a complete graph or a cycle.
Then vertices  and  at distance 2 in  can be found in 
time such that  is connected.
\end{lemma}

\begin{proof}
Let . Since  is biconnected and not a 3-cycle, .

Suppose every vertex  has degree  or degree . Since  is not a
cycle, at least one vertex has degree . Thus, and since  is
biconnected, at least two vertices have degree , as otherwise  would
be  the 1-connected graph shown in Figure~(a). Since  is
not a complete graph there is at least one vertex of degree , which implies
there is exactly two vertices of degree ; that is,  is ,
as shown in Figure~(b). Since  is not a 3-cycle there are
at least two vertices of degree . Let  and  be any two degree 
vertices. Then  is connected, and we are done. Clearly this
case can be recognised and the vertices  and  determined in 
time. 

\begin{figure}[htb]
\begin{center}
\includegraphics{picture}
\end{center}
\end{figure}

Otherwise  has a vertex  with . We now consider two
cases depending on the biconnectivity of . First, suppose
 is biconnected. Let  and  be any vertex at distance 2
from . Then  is connected, and we are done. Second,
suppose  is not biconnected. Since  is biconnected,
 is connected. Let  and  be end-blocks of  with respective cut-points  and . (An end-block corresponds to a
leaf of the block-cut-tree, and since a tree has at least two leaves,  and
 exist.)\ Since  is biconnected but  is not biconnected,
 must be adjacent to vertices in  and  which are not  and
. Let  and  be these vertices. The only vertex adjacent to both 
and  is , and since ,  is connected.
Again, this algorithm can be implemented in  time using depth-first
search algorithms for determining biconnectivity and the biconnected components
of a graph~\cite{Tarjan72}. \end{proof}

Of course cycles (both odd and even) and complete graphs can be recognised and
Brooks-colourings for these graphs determined in linear time. Combining this
observation with Lemmata~\ref{lem:NonBiconnected},
\ref{lem:SequentialColouring} and \ref{lem:FindAB}, we obtain the following
result.

\begin{theorem}
There is an algorithm to determine a Brooks-colouring of a given graph  in
 time.\qed
\end{theorem}





\def\soft#1{\leavevmode\setbox0=\hbox{h}\dimen7=\ht0\advance \dimen7
  by-1ex\relax\if t#1\relax\rlap{\raise.6\dimen7
  \hbox{\kern.3ex\char'47}}#1\relax\else\if T#1\relax
  \rlap{\raise.5\dimen7\hbox{\kern1.3ex\char'47}}#1\relax \else\if
  d#1\relax\rlap{\raise.5\dimen7\hbox{\kern.9ex \char'47}}#1\relax\else\if
  D#1\relax\rlap{\raise.5\dimen7 \hbox{\kern1.4ex\char'47}}#1\relax\else\if
  l#1\relax \rlap{\raise.5\dimen7\hbox{\kern.4ex\char'47}}#1\relax \else\if
  L#1\relax\rlap{\raise.5\dimen7\hbox{\kern.7ex
  \char'47}}#1\relax\else\message{accent \string\soft \space #1 not
  defined!}#1\relax\fi\fi\fi\fi\fi\fi} \def\cprime{}
\begin{thebibliography}{7}
\expandafter\ifx\csname natexlab\endcsname\relax\def\natexlab#1{#1}\fi

\bibitem[{Brooks(1941)}]{Brooks41}
\textsc{R.~L. Brooks}, On colouring the nodes of a network. \emph{Proc.
  Cambridge Philos. Soc.}, \textbf{37}:194--197, 1941.

\bibitem[{Bryant(1996)}]{Bryant96}
\textsc{V.~Bryant}, A characterisation of some 2-connected graphs and a comment
  on an algorithmic proof of {B}rooks' theorem. \emph{Discrete Math.},
  \textbf{158}:279--281, 1996.

\bibitem[{Chartrand and Lesniak(1996)}]{CL96}
\textsc{G.~Chartrand and L.~Lesniak}, \emph{Graphs \& Digraphs}. Chapman and
  Hall, 1996.

\bibitem[{Karp(1972)}]{Karp72}
\textsc{R.~M. Karp}, Reducibility among combinatorial problems. In
  \textsc{R.~E. Miller and J.~W. Thatcher}, eds., \emph{Complexity of Computer
  Communications}, pp. 85--103, Plenum Press, 1972.

\bibitem[{Lov\'{a}sz(1975)}]{Lovasz75}
\textsc{L.~Lov\'{a}sz}, Three short proofs in graph theory. \emph{J. Combin.
  Theory Ser. B}, \textbf{19}:269--271, 1975.

\bibitem[{Ponstein(1969)}]{Ponstein69}
\textsc{J.~Ponstein}, A new proof of {B}rooks's chromatic number theorem for
  graphs. \emph{J. Combinatorial Theory}, \textbf{7}:255--257, 1969.

\bibitem[{Tarjan(1972)}]{Tarjan72}
\textsc{R.~E. Tarjan}, Depth-first search and linear graph algorithms.
  \emph{SIAM J. Comput.}, \textbf{1(2)}:146--160, 1972.

\end{thebibliography}

\end{document}
