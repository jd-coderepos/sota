\documentclass [11pt,a4paper]{article}
\usepackage{amsmath,amssymb,amsbsy,amsfonts,amsthm,latexsym,amsopn,amstext,                                                amsxtra,euscript,amscd}
\def\F{\mathbb{F}}
\def\N{\mathbb{N}}
\def\Z{\mathbb{Z}}
\def\K{\mathbb{K}}
\def\e{\mathbf{e}}
\usepackage{color,xcolor}
\usepackage{float}
\usepackage{graphicx}
\newtheorem{theorem}{Theorem}
\newtheorem{lemma}{Lemma}
\newtheorem{corollary}{Corollary}
\newtheorem{proposition}{Proposition}
\newtheorem{definition}{Definition}
\newtheorem{remark}{Remark}
\newtheorem{example}{Example}
\newtheorem{condition}{Condition}
\newtheorem{conjecture}{Conjecture}

\newcommand{\quash}[1]{}


\setlength{\evensidemargin}{0.135in}
\setlength{\oddsidemargin}{0.135in} \setlength{\textwidth}{6in}
\setlength{\topmargin}{0in} \setlength{\textheight}{8.5in}



\begin{document}

\title{On the -error linear complexity of binary
sequences derived from polynomial quotients}

\author{Zhixiong Chen\\
School of Applied Mathematics, Putian University, \\ Putian, Fujian
351100, P. R. China\\
ptczx@126.com\\
\\
Zhihua Niu\\
School of Computer Engineering and Science, Shanghai University,\\
Shangda Road, Shanghai 200444, P. R. China \\
zhniu@staff.shu.edu.cn\\
\\
Chenhuang Wu\\
School of Applied Mathematics, Putian University, \\ Putian, Fujian
351100, P. R. China}
\maketitle


\begin{abstract}
We investigate the -error linear complexity of -periodic
binary sequences defined from the polynomial quotients (including the well-studied Fermat quotients), which is defined by

where  is an odd prime and . Indeed, first for all integers , we determine exact values of the -error linear complexity over the finite field  for these binary sequences under the assumption of  being a primitive root modulo , and then we determine their -error linear complexity over the finite field  for either  when  or  when . Theoretical results obtained indicate that such sequences possess `good' error linear complexity.
\end{abstract}



\noindent {\bf Keywords:}  Fermat quotients, Polynomial quotients, Binary sequences, Linear complexity, -Error linear complexity, Cryptography

\noindent {\bf MSC(2010):} 94A55, 94A60, 65C10


\section{Introduction}\label{intro}

For an odd prime  and integers  with , the {\it
Fermat quotient  \/} is defined as the unique integer

and

An equivalent definition of the Fermat quotient is given below

For any fixed positive integer , by the fact that

from the Fermat Little Theorem, Chen and Winterhof extended (\ref{FFF})
to define

which is called a \emph{polynomial quotient} in \cite{CW2}. In
fact . It is easy to see that

if , and


Many number theoretic and cryptographic questions as well as
measures of pseudorandomness have been studied for Fermat quotients
and their generalizations
\cite{ADS,AW,BFKS,C,CD,CG,COW,CW2,CW,CW3,DCH,DKC,EM,GW,OS,Sha,Shk,S,S2010,S2011,S2011b,SW}.






In this paper, we still concentrate on certain binary sequences defined from the polynomial quotients (of course including the Fermat quotients) in the references.
The first one is the binary threshold sequence  studied in \cite{CD,CG,CHD,COW,CW4,DKC} by defining

The second one, by combining  with the Legendre symbol
, is defined in \cite{CHD,CW4,DKC,GW} by

(In fact, in \cite{DKC,GW} , a fixed multiplicative
character modulo  of order , is applied to defining -ary sequences  of
discrete logarithms modulo a divisor  of  by

and  otherwise. When , we have  for all .) We note that both  and  are -periodic by (\ref{addstruct}).


The authors of \cite{COW,GW} investigated measures of pseudorandomness as well as linear complexity profile of
 and  (of course including ) via certain character sums over Fermat quotients.
The authors of \cite{CHD,DKC}  determined the \emph{linear complexity} (see
 below for the definition) of  and  if  is a primitive element modulo , and later the authors of \cite{CD,CG,CW4}  extended to a more general setting of  when . The authors of \cite{CW4}  also determined the  trace representations of  and . In this paper, our main aim is to study the \emph{-error linear complexity} (see  below for the definition) for  and .
All results indicate that such sequences have desirable cryptographic features.



For our purpose, we need to describe   and  in an equivalent way. From (\ref{addstruct}),
 induces a surjective map from  (the group of invertible
elements modulo ) to  (the additive group of numbers modulo ). For each fixed , we define

for . Each  has the cardinality  by (\ref{addstruct}). Here and hereafter, we use  to denote the cardinality of a set . Let , for 
one can define   and  equivalently by

and

respectively, where  is the set of quadratic residues modulo  and  is the set of quadratic non-residues modulo .
For , it is easy to  define   and  similarly by only re-dividing the set .




We need to mention that,  the following relation holds  between
 and :

for all  with .
If  for any positive integer , we have  by
(\ref{poly-Fermat:relation}) and (\ref{value-2})  for all .
For any positive  with , write  with
 and , by (\ref{poly-Fermat:relation})
again one can get

Note that  since  . Hence, a polynomial quotient  with large  can be reduced to the one with  and we restrict ourselves to
 from now on.


We conclude this section by introducing the notions of the linear complexity and the -error linear complexity of periodic sequences.

Let  be a field.  For a -periodic
sequence  over , we recall that the
\emph{linear complexity} over , denoted by  , is the least order  of a linear
recurrence relation over 

which is satisfied by  and where .
Let

which is called the \emph{generating polynomial} of . Then the linear
complexity over  of  is computed by

see, e.g. \cite{LN} for details. For integers , the \emph{-error linear complexity} over  of , denoted by , is the smallest linear complexity (over ) that can be
obtained by changing at most  terms of the sequence per period, see \cite{SM,Meidl}, and see \cite{DXS} for the related even earlier defined sphere complexity.  Clearly  and

when  equals the number of nonzero terms of  per period, i.e., the weight of .


The linear complexity and the -error linear complexity are important cryptographic characteristics of sequences
and provide information on the predictability and thus unsuitability for cryptography. For a sequence to be cryptographically strong, its linear complexity
should be large, but not significantly reduced by changing a few
terms. And according to  the Berlekamp-Massey
algorithm \cite{Massey}, the linear complexity
should be at least a half of the period.


Instead of studying   and  directly, we define the -periodic binary sequence  by

for , and

for , where  is a non-empty subset of , and investigate the -error linear complexity over  for  in Section \ref{LC-2}. In Section \ref{LC-p}, we investigate the -error linear complexity over  for . Although  is a binary sequence, it is constructed based on the polynomial quotients modulo  (note that the linear complexity over  of the polynomial quotients is , see a proof in  \cite{OS} for the Fermat quotients), thus, it is natural to consider the -error linear complexity over  for . In fact, it is also motivated by the ideas of \cite{AMW,AW06} and partially \cite{AM,AW,BW,CY,ESK,GLSW,HKN,HMMS}.



\section{-Error Linear Complexity over }\label{LC-2}

First we present some auxiliary statements. Define

for .

\begin{lemma}\label{lemma-add}
Let  be  a  primitive -th root of
unity. For , we have

\end{lemma}
Proof. For any fixed , the numbers  belong to different  when  runs through the set  by  (\ref{addstruct}), hence we have

 We note that in the definition of , we restrict . For , we derive

which deduces the desired result for different  modulo . The calculations here are performed in
finite fields with characteristic two. ~\hfill 

\begin{lemma}\label{lemma-p}
Let  be  a  primitive -th root of
unity and  with . If   is a
primitive root modulo , we have

or

\end{lemma}
Proof. We only show the first assertion. Since  is a
primitive root modulo , we see that  is the minimal irreducible polynomial with the root .
So if , we derive

With the restriction on , we get . The converse is true after simple calculations.
~\hfill 



Now we present our main results.


\begin{theorem}\label{klc-2-primitive}
Let  be the binary sequence of period  defined in (\ref{hhhh}) using polynomial quotients (\ref{poly-def}) with  and a non-empty subset  of  with . If  is a primitive root modulo , then
the -error linear complexity  over  of   satisfies

if  is odd, and otherwise

\end{theorem}
Proof. From the construction of , there are  many 1's in one period of  since each  contains  many elements. Changing
all terms of 1's will lead to the zero sequence. So we always assume that . Let

be the generating polynomial of the sequence obtained from  by changing exactly  terms of  per period,
where  is the corresponding error polynomial with  many  monomials. We note that  is a nonzero polynomial due to . We will consider the common roots of  and ,
i.e.,  the roots of the form  for , where  is a  primitive -th root of unity. The number of the common roots will help us to derive the values of -error linear
complexity of   by  (\ref{licom}).

On the one hand, we assume that  for some . Since  is a primitive root modulo , for each , there exists a  such that . Then we have

that is, all ( many) elements  for  are roots of . Hence we have

where

the roots of which are exactly  for .  Let

Using the fact that

we restrict  and write

where  since  is a nonzero polynomial. Then the exponent of each monomial in  forms the set

which can be divided into two sets  and  with


We note that by (\ref{addstruct})  contains  many numbers with

and  contains  many numbers.

Hence, from (\ref{Hk}) and (\ref{pi}), we find that
the set of the exponents of monomials in  is

the cardinality of which is

That is,  since  contains  many terms. However, the
assumption of 
implies  and hence

a contradiction.
So  for all .

 On the other hand, by Lemma \ref{lemma-add} we get

Hence below we only need to consider the number of roots of the form  for .

First, if  is odd, for  (in this case  will not occur) it is easy to see that

For , we first consider  for some , i.e.,  is changed only one term at the position  per period, we have  for all  and there are exactly  many  such that . However, for any other  with  terms, the number of such kind of roots of  does not increase, since   satisfying

which guarantees all  are roots of , should be of the form

by Lemma \ref{lemma-p}, in other words,  should contain at least  many terms if the number of roots of   increases (from  to ). So we derive

For  and , one can choose  with  terms, as mentioned above,  of the form  such that all 's   are roots of .  Hence, we get



Second, we turn to the case of even . When , all   are roots of   from (\ref{roots}). No other possible roots of the form  will occur for .
So we have

We complete the proof.   ~\hfill \\

In Theorem \ref{klc-2-primitive}, we restrict . For , we can similarly consider the complementary sequence, denoted by , of , i.e.,  for all . The difference between the (-error) linear complexity of  and that of   is at most  by the fact that

where  is the generating polynomial of ,  is the generating polynomial of  and  is the error polynomial. In this case, one might ask how about the -error linear complexity for the complementary sequence , which in fact is defined by

where  is a non-empty subset of  with .
(Note that  for , but  in this case.)  In particular, we can get some balanced binary sequences when  for certain special applications.


Fortunately, following the same way as the proof of Theorem \ref{klc-2-primitive}, we get

if  is odd, and otherwise

if  is a primitive root modulo .


The statement of the -error linear complexities of  and  follows from Theorem \ref{klc-2-primitive} directly.
We describe it in the following corollary.

\begin{corollary}
Let  and  be the binary sequences of period  defined in
(\ref{binarythreshold}) and (\ref{binarylegendre}), respectively. If  is a primitive root modulo , then
their -error linear complexity  over  satisfies

if , and otherwise

\end{corollary}



For , the result is somewhat different because of (\ref{value-2}) and we present it in the following separate
theorem.


\begin{theorem}\label{klc-2-primitive-w=1}
Let  be the binary sequence of period  defined in (\ref{hhhh-w=1}) using polynomial quotients (\ref{poly-def}) with  and a non-empty subset  of  with . If  is a primitive root modulo , then
the -error linear complexity  over  of   satisfies

if  is odd, and otherwise

\end{theorem}
Proof. The proof is similar to that of Theorem \ref{klc-2-primitive}. Here we present a sketch. Let

be the generating polynomial of the sequence obtained from  by changing exactly  terms of  per period,
where  is the corresponding error polynomial with  many  monomials.

For , under the assumption of  being primitive root modulo , we can show  for all , as proved in  Theorem \ref{klc-2-primitive}. So we only need to determine the number of roots of the form  for .
By Lemma \ref{lemma-add}, we have

For odd , all  are roots of  when  and
 has one more root if  satisfies

from which we derive by Lemma \ref{lemma-p}

That is to say, only that  modulo  is of the form above, which contains  terms, can guarantee that all  are roots of ,
thus

and

For even , all  are roots of   and any  with  terms for   will not increase
the number of the common roots of  and . Then the result follows. ~\hfill \\


We restrict that  is a primitive root modulo  in the theorems above. A
conjecture of Artin suggests that approximately  of all primes
have  as a primitive element (\cite[p.81]{Shanks}), and it is
very seldom that a primitive element modulo the prime  is not
 a primitive element modulo .
If  is not a primitive root modulo , it seems that our method is not suitable for computing the exact number of the common roots of   and  without additional ideas, as mentioned in the proof of Theorem \ref{klc-2-primitive}. But we have some partial results, as described in the following theorem, under certain special conditions. We conjecture that Theorems \ref{klc-2-primitive} and \ref{klc-2-primitive-w=1} are true for most primes , e.g.  satisfying , see \cite{CDP1997,CD} for the applications of such primes. We note that  if and only if the order of  modulo  is lager than .



\begin{theorem}\label{klc-2-general}
Let  with  and the order of  modulo  be  with  and .

(i). Let  be the binary sequence of period  defined in (\ref{hhhh}) using polynomial quotients (\ref{poly-def}) with  and .

(ii). Let  be the binary sequence of period  defined in (\ref{hhhh-w=1})  using polynomial quotients (\ref{poly-def}) with   and .

If  for (i) or  for (ii), the -error linear complexity over  of   satisfies

 and otherwise .
\end{theorem}
Proof. First for  for (i), according to the proof of Theorem \ref{klc-2-primitive}, there do exist an  such that
  for , the generating polynomial of the sequence obtained from  by changing exactly  terms of  per period. (Otherwise, we will get a more accurate result, as described in Theorem \ref{klc-2-primitive}.)
Thus there are at least  many  such that . Then the result follows.

For the case of   for (ii), the discussion is similar by using  the proof of Theorem \ref{klc-2-primitive-w=1}. ~\hfill 


\section{-Error Linear Complexity over }\label{LC-p}

In this section, we view the binary sequences  defined in  (\ref{hhhh}) and (\ref{hhhh-w=1}) as sequences over 
and consider their (-error) linear complexity over , which is also an interesting problem for binary sequences. Such kind of work has been done in many references, such as \cite{AM,AMW,AW06,AW,BW,CY,ESK,GLSW,HKN,HMMS}.

We will employ the -th Hasse derivative of a polynomial , which is defined to be

The multiplicity of  as a root of  is  if  and , see e.g. \cite[Ch 6.4]{LN} for details.

Before presenting the main results, we introduce a technical lemma, which will be used in the proofs.

\begin{lemma}\label{D-derivative}
With notations of  defined in Section \ref{intro}. Let 
and  be the -th Hasse derivative of  for . Then for , we have

and hence

where .
\end{lemma}
Proof. In the proof of Lemma \ref{lemma-add}, we have shown that

i.e.,

from which we derive

Then write

for some , it is  easy to check the rest equalities by using

for , where we use . ~\hfill 


Now we present our main results.



\begin{theorem}\label{klc-p-w=1}
Let  be the (binary) sequence of period  defined in (\ref{hhhh-w=1}) using polynomial quotients (\ref{poly-def}) with  and a non-empty subset  of  with .
Then we have

for , and  for .
\end{theorem}
Proof. Let

be the generating polynomial of the sequence obtained from  by changing exactly  terms of  per period,
where  is the corresponding error polynomial with  many  monomials. In particular,  is the generating polynomial of . Since  over , we only need to consider the multiplicity of  as a root of .

It is easy to check by Lemma \ref{D-derivative} that

where  is the -th Hasse derivative of . So we have

where the notation `' means  but .
Hence the linear complexity of  is

by (\ref{licom}).

Now we consider the case of . For  with  terms, since  it is easy to see that

if , and

if   and . So for such , the multiplicity of  as a root of  is at most  and hence the -error linear complexity will not decrease. Now
we assume  and write

where ,  and .
Using the facts that

and

we get

That is to say,  modulo  should be of the form above and it has at least  terms if .

Hence we conclude that, if , the multiplicity of  as a root of , which contains  terms, is not equal to . (Otherwise,  modulo  has at most  terms, a contradiction.) So we have

and we derive the desired result.

For , one can choose

where . From

we find  and the value .     ~\hfill 




\begin{theorem}\label{klc-p-w=2}
Let  be the (binary) sequence of period  defined in (\ref{hhhh}) using polynomial quotients (\ref{poly-def}) with  and a non-empty subset  of  with .
Then we have

and  for .
\end{theorem}
Proof. Let

be the generating polynomial of the sequence obtained from  by changing exactly  terms of  per period,
where  is the corresponding error polynomial with  many  monomials.

We check that , hence  by (\ref{licom}).

Below we consider the case of . For  for any  and ,
we have

 and

by Lemma \ref{D-derivative}. So we derive

and

For  for any  and ,  we find that  and

hence the multiplicity of  as a root of  is . So we conclude that



Now we want to find the smallest  such that

From  and   for , we have

by Lemma \ref{D-derivative}, where   is the -th Hasse derivative of . We define a new polynomial  with

for some , and compute

Then we have . Following the proof of Theorem \ref{klc-p-w=1}, we derive

for . Hence  should be of the form

which contains at least  terms, since  can take  as its output. Hence .

So we conclude that if , , from which the first desired result follows.
For , one can directly choose

and then compute

and

for  by Lemma \ref{D-derivative}, so we have  and  .
 ~\hfill \\


It seems difficult for us to consider the case of larger  without additional ideas. We leave it open. However, motivated by \cite{AMW,AW06}, we have a more accurate upper bound for  being the set of quadratic non-residues modulo .
In this case,   in (\ref{hhhh}) or (\ref{hhhh-w=1}) is in fact  defined in (\ref{binarylegendre}).

Since

for all integers  with , according to \cite{AW06} we see that
 can be represented by

where the multivariate polynomial  is of the form

We reduce  modulo  and  such that the degree strictly less than  in each indeterminate.
And then the linear complexity over  of  equals to , we refer the reader to \cite[Theorem 8]{BEP} for the assertion and the definition of the \emph{degree} of multivariate polynomials.



For , Substituting  by  at those positions  with  in ,
we get a new sequence  represented by the polynomial

from which we derive after some simple calculations

by \cite[Theorem 8]{BEP}. Since , we obtain an upper bound on the -error linear complexity
of   defined in (\ref{binarylegendre}) as follows

for .



For , We only use  instead of  above and obtain

for .\\

Finally, we mention a lower bound on the -error linear complexity over  of 
defined in (\ref{hhhh})  or (\ref{hhhh-w=1}). From \cite[Theorem 8]{BEP}, each -periodic sequence over  can be represented by a unique polynomial

with , otherwise the period is reduced to . We find by (\ref{addstruct}) that
changing at most  (smaller than the weight of  per-period) terms from   will not reduce the period, hence
the -error linear complexity over  is .




\section{Concluding Remarks}

In this paper, we study the \emph{error linear complexity spectrum} (see \cite{EKKLP} for details) of
-periodic
binary sequences defined from the polynomial quotients, that is, we determine exact values of
their -error linear complexity over the finite field 
for all integers  under the assumption of  being a primitive root modulo . Main results can be described in the following figures, which visually reflect how the linear complexity of the binary sequences decreases as the number  of allowed bit changes increases. It is of interest to consider this problem for the case of  being not a primitive root modulo . We only estimate a lower bound on their -error linear complexity if , with which most primes  are satisfied, see \cite{CDP1997}.


\begin{figure}[H]
\centering
\includegraphics[width=110mm,height=60mm]{th111.pdf}
\caption{Error linear complexity spectrum of  when  (Theorem \ref{klc-2-primitive})}
\end{figure}

\begin{figure}[H]
\centering
\includegraphics[width=110mm,height=60mm]{th222.pdf}
\caption{Error linear complexity spectrum of  when  (Theorem \ref{klc-2-primitive-w=1})}
\end{figure}




We also view the binary sequences as sequences over the finite field 
and determine their -error linear complexity over  for either  when  or  when . Results indicate that the linear complexity is large (close to the period) and  not significantly reduced by changing a few terms. It is interesting to consider this problem for larger .





We finally remark that, the definition of binary sequences studied in this manuscript is related to
generalized cyclotomic classes modulo , as you can see in Section \ref{intro}. In particular, the Fermat quotient
  defines a group epimorphism from  to  by the fact, see e.g. \cite{OS}, that

So if  is a (fixed) primitive root modulo , we have

and

where   and the subscript of  is performed modulo . (Note that for , we don't have this property.) Sequences related to cyclotomic classes modulo a prime and generalized cyclotomic classes modulo the product of two
distinct primes have been widely investigated since several decades ago, the well-known basic examples are the Legendre sequences and the Jacobi sequences, see \cite{CDR,D97,D98,DHS} and references therein. As we know, the -error linear complexity of the Jacobi sequences and their generalizations \cite{D97,D98} has not been solved thoroughly.
Hence we hope that our idea and method might be helpful for considering this problem and lead to furtherly study applications of the theory of cyclotomy in cryptography.





\section*{Acknowledgements}

The authors wish  to thank Arne Winterhof for helpful suggestions.




Z.X.C. was partially supported by the National Natural Science
Foundation of China under grant No. 61170246 and the Special Scientific Research Program in Fujian Province Universities of China under grant No. JK2013044.

Z.H.N. was partially supported by the Shanghai Leading Academic Discipline Project under grant No. J50103,
the National Natural Science Foundation of China  under grants No. 61074135, 61272096 and 61202395, the Program for New Century Excellent Talents in University  under grant NCET-12-0620, and the Shanghai Municipal Education Commission Innovation Project.

C.H.W. was partially supported by the Foundation item of the Education Department of Fujian Province of China under grants No. JA12291 and JB12179.

Parts of this paper were written during a very pleasant visit of the
first author to RICAM in Linz. He wishes to thank for the hospitality.


\begin{thebibliography}{99}

\bibitem{ADS} T. Agoh, K. Dilcher and L. Skula.
Fermat quotients for composite moduli.  J.  Number Theory 66
(1997) 29--50.

\bibitem{AM} H. Aly and  W. Meidl. On the linear complexity and -error linear complexity over  of
the -ary Sidel'nikov sequence. IEEE Trans. Inform. Theory 53 (2007) 4755--4761.

\bibitem{AMW}
H. Aly,  W. Meidl and A. Winterhof. On the -error linear complexity
of cyclotomic sequences. J. Math. Cryptol. 1 (2007)
283--296.


\bibitem{AW06}H. Aly and  A. Winterhof. On the -error linear complexity over  of Legendre
and Sidel'nikov sequences. Des. Codes Cryptogr. 40 (2006) 369--374.

\bibitem{AW} H. Aly  and A. Winterhof. Boolean functions derived from Fermat
quotients. Cryptogr. Commun. 3 (2011) 165--174.

\bibitem{BEP}
S. R. Blackburn, T. Etzion, K. G. Paterson.  Permutation
polynomials, de Bruijn sequences, and linear complexity. J. Combin.
Theory Ser. A 76 (1996)  55--82.



\bibitem{BFKS}
J. Bourgain, K. Ford, S. Konyagin and I. E. Shparlinski. On the
divisibility of Fermat quotients. Michigan Math. J. 59 (2010) 313--328.

\bibitem{BW} N. Brandst\"{a}tter and A. Winterhof. -error linear complexity over   of subsequences of Sidelnikov sequences of period . J. Math. Cryptol. 3 (2009)  215--225.



\bibitem{C}
M. C. Chang. Short character sums with Fermat quotients. Acta Arith.
152 (2012) 23--38.

\bibitem{CD}  Z. X. Chen  and  X. N. Du. On the linear complexity of binary threshold sequences derived from
Fermat quotients. Des. Codes Cryptogr. 67 (2013) 317--323.



\bibitem{CG}Z. X. Chen  and  D. G\'{o}mez-P\'{e}rez. Linear complexity of
binary sequences derived from polynomial quotients. Sequences and Their Applications-SETA 2012, 181--189, Lecture Notes in Comput. Sci., 7280, Springer, Berlin, 2012.

\bibitem{CHD}  Z. X. Chen, L. Hu  and  X. N. Du. Linear complexity of some binary sequences derived from Fermat
quotients. China Commun.  9 (2012) 105--108.


\bibitem{COW}  Z. X. Chen, A. Ostafe and  A. Winterhof. Structure of
pseudorandom numbers derived from Fermat quotients. Arithmetic of Finite Fields-WAIFI 2010, 73--85, Lecture Notes in Comput. Sci., 6087, Springer, Berlin, 2010.




\bibitem{CW2}Z. X. Chen  and A.  Winterhof. Additive character sums of
polynomial quotients. Theory and Applications of Finite Fields-Fq10,
67--73, Contemp. Math., 579, Amer. Math. Soc., Providence, RI, 2012.

\bibitem{CW}
Z. X. Chen  and A.  Winterhof. On the distribution of pseudorandom
numbers and vectors derived from Euler-Fermat quotients. Int. J.
Number Theory 8 (2012) 631--641.

\bibitem{CW3} Z. X. Chen and A. Winterhof. Interpolation of Fermat quotients. SIAM J. Discr. Math. 2013 (to appear)



\bibitem{CW4} Z. X. Chen. Trace representation and linear complexity of binary
sequences derived from Fermat quotients.  http://arxiv.org/arXiv:1306.5648, 2013.



\bibitem{CY}
J. H. Chung, K. Yang. Bounds on the linear complexity and the 1-error linear complexity over  of -ary Sidel'nikov sequences. Sequences and Their Applications-SETA 2006, 74--87, Lecture Notes in Comput. Sci., vol. 4086, Springer, Berlin, 2006.

\bibitem{CDP1997}  R. Crandall, K. Dilcher and C. Pomerance. A search for Wieferich
and Wilson primes. Math. Comp. 66 (217) (1997) 433--449.

\bibitem{CDR}T. W. Cusick, C. S. Ding, A. Renvall. Stream ciphers and number theory.
North-Holland Mathematical Library, 55. North-Holland Publishing
Co., Amsterdam, 1998.

\bibitem{DXS}
C. S. Ding, G. Z. Xiao,  W. J. Shan.  The stability theory of stream
ciphers. Lecture Notes in Computer Science, 561. Springer-Verlag,
Berlin, 1991.

\bibitem{D97}C. S. Ding. Linear complexity of generalized cyclotomic binary sequences
of order 2. Finite Fields Appl. 3 (1997)  159--174.


\bibitem{D98}C. S. Ding. Autocorrelation values of generalized cyclotomic sequences of order two.
 IEEE Trans. Inform. Theory 44 (1998)  1699--1702.


\bibitem{DHS}
C. S. Ding,  T. Helleseth,  W. J. Shan.  On the linear complexity of
Legendre sequences. IEEE Trans. Inform. Theory 44 (1998)
1276--1278.

\bibitem{DCH}
X. N. Du, Z. X. Chen and L. Hu. Linear complexity of binary
sequences derived from Euler quotients with prime-power modulus.
Inform. Process. Lett. 112 (2012) 604--609.

\bibitem{DKC}
X. N. Du, A. Klapper and Z. X. Chen. Linear complexity of
pseudorandom sequences generated by Fermat quotients and their
generalizations. Inform. Process. Lett. 112 (2012) 233--237.


\bibitem{EKKLP}
T. Etzion, N. Kalouptsidis, N. Kolokotronis, K. Limniotis and K. G. Paterson. Properties of the error linear complexity spectrum. IEEE Trans. Inform. Theory 55 (2009)  4681--4686.

\bibitem{ESK}
Y. C. Eun, H. Y. Song and G. M. Kyureghyan. One-error linear complexity over  of Sidel'nikov sequences.
Sequences and Their Applications-SETA 2004, 154--165, Lecture Notes in Comput. Sci., vol. 3486, Springer, Berlin, 2005.


\bibitem{EM} R. Ernvall and T. Mets{\"a}nkyl{\"a}.
On the -divisibility of Fermat quotients. Math. Comp.  66 (1997)
1353--1365.


\bibitem{GLSW}
M. Z. Garaev, F. Luca, I. E. Shparlinski and A. Winterhof. On the lower bound of the linear complexity over   of Sidelnikov sequences. IEEE Trans. Inform. Theory 52 (2006)  3299--3304.





\bibitem{GW} D. G\'{o}mez-P\'{e}rez and A. Winterhof. Multiplicative character sums of
Fermat quotients and pseudorandom sequences. Period. Math. Hungar.
64 (2012) 161--168.






\bibitem{HKN}
T. Helleseth, S. H. Kim and  J. S. No. Linear complexity over    and trace representation of Lempel-Cohn-Eastman sequences. IEEE Trans. Inform. Theory 49 (2003)  1548--1552.

\bibitem{HMMS}
T. Helleseth, M. Maas, J. E. Mathiassen and  T. Segers. Linear complexity over  of Sidel'nikov sequences. IEEE Trans. Inform. Theory 50 (2004)  2468--2472.



\bibitem{LN} R. Lidl and H. Niederreiter. Finite Fields.
Second edition. Encyclopedia of Mathematics and its Applications, 20. Cambridge University Press, Cambridge, 1997.




\bibitem{Massey} J. L. Massey. Shift register synthesis and BCH decoding. IEEE Trans. Inform.
Theory 15 (1969) 122--127.



 \bibitem{Meidl}W. Meidl. How many bits have to be changed to decrease the linear
complexity? Des. Codes Cryptogr. 33 (2004) 109--122.







\bibitem{OS} A. Ostafe  and  I. E. Shparlinski. Pseudorandomness and dynamics of
Fermat quotients. SIAM J. Discr. Math. 25 (2011) 50--71.



\bibitem{Sha}M. Sha. The arithmetic of Carmichael quotients. http://arxiv.org/arXiv:1108.2579, 2011.

\bibitem{Shanks}D Shanks. Solved and Unsolved Problems in Number Theory. Second edition. Chelsea
Publishing Company, New York, 1978.

\bibitem{Shk}I. D. Shkredov. On Heilbronn's exponential sum. Quart. J. Math. (2012) doi: 10.1093/qmath/has037.



\bibitem{S}I. E. Shparlinski. Character sums with Fermat quotients. Quart. J. Math. 62 (2011) 1031--1043.

\bibitem{S2010} I. E. Shparlinski.  Bounds of multiplicative character sums with
Fermat quotients of primes. Bull. Aust. Math. Soc. 83 (2011)
456--462.

\bibitem{S2011} I. E. Shparlinski.  On the value set of Fermat
quotients. Proc. Amer. Math. Soc. 140 (2012) 1199--1206.

\bibitem{S2011b} I. E. Shparlinski.  Fermat quotients: Exponential sums, value set and primitive
roots. Bull. Lond. Math. Soc.  43 (2011) 1228--1238.

\bibitem{SW}
I. E. Shparlinski  and A. Winterhof. Distribution of values of
polynomial Fermat quotients. Finite Fields Appl. 19 (2013)
93--104.




\bibitem{SM}M. Stamp, C. F. Martin. An algorithm for the -error linear complexity of
 binary sequences with period . IEEE Trans. Inform. Theory 39 (1993)  1398--1401.


\end{thebibliography}






\end{document}
