

We mostly use the notation from~\cite{baader-nipkow}, with some additions from~\cite{Lopez-FraguasRS09-RTA09}. 
We consider a first order signature $\Sigma = \CS \uplus \FS$, where $\CS$ and $\FS$ are two disjoint
sets of \emph{constructor} and defined \emph{function} symbols respectively, all of them with associated
arity. We use $c,d,\ldots$ for constructors, $f,g,\ldots$ for functions and $X,Y,\ldots$ for variables of a
numerable set  $\var$. The notation $\overline{o}$ stands for tuples of any kind of syntactic objects. 
The set $\Exp$ of {\it total expressions} is defined as $\Exp \ni e::= X \mid h(e_1,\ldots,e_n)$, where 
$X\in\var$, $h\in \CS^n\cup \FS^n$ and $e_1,\ldots,e_n\in \Exp$. The set $\CTerm$ of
{\it total constructed terms}
(or {\it  c-terms}) is defined like $\Exp$, but with $h$ restricted to $\CS^n$ (so $\CTerm\subseteq \Exp$).
The intended meaning is that $\Exp$ stands for evaluable expressions, i.e., expressions that can contain 
function symbols, while $\CTerm$ stands for data terms representing {values}. 
We will write $e,e',\ldots$ for expressions and $t,s,\ldots$ for c-terms. 
We say that an expression $e$ is \emph{ground} iff no variable appears in $e$. 
We will frequently use \emph{one-hole contexts}, defined as $\Cntxt\ni {\cal C} ::= [\ ] \mid h(e_1,\ldots,{\cal C},\ldots,e_n)$. 


\begin{example}\label{ex:clerks}
We will use a simple example throughout this section to illustrate these definitions. Assume we want to represent
the staff of a shop, so we have $\CS = \{\mi{madrid}^0, \mi{vigo}^0,$
$\mi{man}^0, \mi{woman}^0, \mi{pepe}^0, \mi{luis}^0,$ $\mi{pilar}^0,$ $\mi{maria}^0, e^2, p^2\}$,
where $e$ will be the constructor for employees and $p$ the constructor for pairs, and
$\FS = \{\mi{branches}^0, \mi{search}^1, \mi{employees}^1\}$.
Using this signature, we can build the set $\Exp = \{\mi{madrid}, \mi{vigo},$ $\mi{employees}$ $(\mi{madrid}),
p(\mi{pilar}, X), \ldots\}$. From this set, we have $\CTerm = \{\mi{madrid}, \mi{vigo},
p(\mi{pilar}, X), \ldots\}$,
while the ground terms are $\{\mi{employees}(\mi{madrid}),$ $\mi{madrid}, \mi{vigo}, \ldots\}$.
Finally, a possible one-hole context is $p([\ ], X)$.
\end{example}

We also consider the extended signature $\Sigma_\bot
=\Sigma\cup \{\bot\}$, where $\bot$ is a new $0$-arity constructor symbol that 
does not appear in programs and which stands for the undefined value. Over this signature 
we define the sets $\Exp_\bot$ and $\CTerm_\bot$ of {\it partial} expressions and c-terms, respectively.
The intended meaning is that $\Exp_\bot$ and $\CTerm_\bot$ stand for 
partial expressions and values, respectively.  
Partial expressions are ordered by the {\em approximation} ordering $\sqsubseteq$ defined as the least
partial ordering satisfying $\perp \sqsubseteq e$ and $e \sqsubseteq e'
\Rightarrow {\cal C}[e] \sqsubseteq {\cal C}[e']$
for all $e,e' \in \Exp_\perp, {\cal C} \in {\it \Cntxt}$.
The {\it shell} $|e|$ of an expression $e$ represents the outer constructed part
of $e$ and is defined as: $|X|=X$; $|c(e_1,\ldots,e_n)|=c(|e_1|,\ldots,|e_n|)$; $|f(e_1,\ldots,e_n)|=\bot$. \longtxt{It is trivial to check that for any expression $e$ we have $|e| \in \CTerm_\perp$, that any total expression is maximal w.r.t.\ $\ordap$, and that as consequence if $t$ is total then $t \ordap |e|$ implies $t = e$.} 

\begin{example}
Using the signature from Example~\ref{ex:clerks}, we have
$\mi{employees(\bot)} \in \Exp_\bot$,
$p(\bot, X) \in \CTerm_\bot$, and
$|p(\mi{search}(\mi{branches}), X)| = p(\bot, X)$.
\end{example}

{\em Substitutions} $\theta \in \Subst$ are finite mappings $\theta:\var
\longrightarrow \Exp$, extending naturally to $\theta:\Exp \longrightarrow
\Exp$.  \longtxt{We write $\epsilon$ for the identity (or empty) substitution.} 
We write $e\theta$ to apply of $\theta$ to $e$, and $\theta\theta'$
for the composition, defined by $X(\theta\theta') = (X\theta)\theta'$. 
\longtxt{The domain and  variable range of $\theta$ are defined as $\dom(\theta) = \{X\in \var \mid X\theta \neq X\}$ and $\vran(\theta) = \bigcup_{X\in \mi{dom}(\theta)}\mi{var}(X\theta)$.}
\shorttxt{The domain of $\theta$ is defined as $\dom(\theta) = \{X\in \var \mid X\theta \neq X\}$.}
By $[X_1/e_1, \ldots, X_n/e_n]$ we denote a substitution $\sigma$ such that $\dom(\sigma) = \{X_1, \ldots, X_n\}$ and $\forall i. \sigma(X_i) = e_i$.  \longtxt{If $\dom(\theta_0) \cap \dom(\theta_1) = \emptyset$, their disjoint union $\theta_0 \uplus \theta_1$ is defined by $(\theta_0 \uplus \theta_1)(X) = \theta_i(X)$, if $X \in \dom(\theta_i)$ for some $\theta_i$; $(\theta_0 \uplus \theta_1)(X) = X$ otherwise.}
\longtxt{Given $W \subseteq {\cal V}$
we write $\theta|_{W}$ for the restriction of $\theta$ to $W$, i.e.\ $(\theta|_{W})(X) = \theta(X)$ if $X \in W$, and $(\theta|_{W})(X) = X$ otherwise; we use $\theta|_{\backslash D}$ as a shortcut for $\theta|_{({\cal V} \backslash D)}$.} 
\emph{C-substitutions}
$\theta \in \emph{CSubst}$ verify that $X\theta \in \CTerm$ for all $X\in \dom(\theta)$. \longtxt{We say a substitution $\sigma$ is ground  iff $\vran(\sigma) = \emptyset$, i.e.\ $\forall X \in \dom(\sigma)$ we have that $\sigma(X)$ is ground.}\shorttxt{We say that a substitution is ground when no variable appears in its range.} The sets $\Subst_\perp$ and $CSubst_\perp$ of partial substitutions and partial c-substitutions are the sets of finite mappings from variables to partial expressions and partial c-terms, respectively. 


\begin{example}
Using the signature from Example~\ref{ex:clerks}, we can define the C-substitutions
$\theta_1 \equiv X / \mi{woman}$, $\theta_2 \equiv X / \mi{man}$,
and $\theta_3 \equiv Y / \mi{pilar}$. We can define the restrictions 
$\theta_1|_{\{X\}} = \theta_1$ and
$\theta_1|_{\backslash \{X\}} = \epsilon$.
Finally, given the expression $p(X, Y)$ we have
$p(X, Y)\theta_1\theta_2 = p(\mi{woman}, Y)$
and
$p(X, Y)\theta_1\theta_3 = p(X, Y)\theta_3\theta_1 = p(\mi{woman}, \mi{pilar})$.
\end{example}

A {left-linear constructor-based term rewriting system} or just \emph{constructor system} (\emph{\ctrs}) or \emph{program} ${\cal P}$ is a set of c-rewrite rules of the form $f(\overline{t})\to r$ where $f\in \FS^n$, $r\in \Exp$ and $\overline{t}$ is a linear $n$-tuple of c-terms, where linearity means that variables occur only once in $\overline{t}$. Notice that we allow $r$ to contain so called \emph{extra variables}, i.e., variables not occurring in $f(\overline{t})$. 
\longtxt{To be precise, we say that $X \in \var$ is an extra variable in the rule $l \tor r$
iff $X \in \mi{var}(r) \setminus \mi{var}(l)$, and by} \shorttxt{By} $\vextra{R}$ we denote
the set of extra variables in a program rule $R$. 
We assume that every \ctrs\ contains the rules $\progq = \{X~?~Y \tor X, X~?~Y \tor Y\}$, defining the behavior of $? \in \FS^2$, used in infix mode\shorttxt{ and right-associative}, and that those are the only rules for $?$. \longtxt{Besides, $?$ is right-associative so $e_1~?~e_2~?~e_3$ is equivalent to $e_1~?~(e_2~?~e_3)$.} For the sake of conciseness we will often omit\longtxt{ these rules when presenting a \ctrs}.  A consequence of this is that we only consider non-confluent programs. Given a \trs\ ${\cal P}$, its associated \emph{term rewriting relation} $\rw_{\cal P}$ is defined as:
${\cal C}[l\sigma] \rw_{\cal P} {\cal C}[r\sigma]$
for any context ${\cal C}$, rule $l \tor r \in {\cal P}$ and $\sigma \in
\emph{Subst}$. We write $\stackrel{*}{\rw_{\cal P}}$ for the reflexive and
transitive closure of the relation $\rw_{\cal P}$. We will usually omit the reference to ${\cal  P}$ or denote it by $\prog \vdash e \rw e'$ and $\prog \vdash e \rw^* e'$.  

\begin{example}\label{ex:rules}
Using the signature from Example~\ref{ex:clerks}, we can describe the following
program:
$$
\begin{array}{lll}
  \mi{branches} & \to & \mi{madrid}\ ?\ \mi{vigo}\\
  \mi{employees}(\mi{madrid}) & \to & e(\mi{pepe}, \mi{men})\\
  \mi{employees}(\mi{madrid}) & \to & e(\mi{maria}, \mi{men})\\
  \mi{employees}(\mi{vigo}) & \to & e(\mi{pilar}, \mi{women})\ ?\ e(\mi{luis}, \mi{men})\\
\mi{search}(e(N,S)) & \to & p(N,N)
\end{array}
$$
In this example, the function symbol $\mi{branches}$ defines the different branches of the
company, $\mi{employees}$ defines the employees in each branch (built with the constructor
symbol $e$), and $\mi{search}$ returns a pair of names, built with the constructor symbol $p$.
Note that several different notations are possible; for example, it is possible to define
the employees of one branch by using just one rule and the $?$ operator or just several
different rules with the same lefthand side. 
\end{example}



\subsection{A proof calculus for constructor systems with extra variables}
In~\cite{Lopez-FraguasRS09-RTA09} an adequate semantics for reachability of c-terms by term rewriting in
\ctrss\ was presented. The key idea there was using a suitable notion of value, in this case
the notion of s-cterm.
$\scterm$ is the set of s-cterms, which are 
\emph{finite} sets of elemental s-cterms, while the set $\escterm$ of elemental s-cterms is defined as 
$\escterm \ni e\uscterm{} ::= X~|~c(\uscterm{1}, \ldots, \uscterm{n})$ for $X \in \var$, $c \in\cs^n$, 
$\uscterm{1}, \ldots, \uscterm{n} \in \scterm$. We extend this \longtxt{idea} to expressions obtaining 
the sets $\sexp$ of s-expressions or just s-exp, and $E\sexp$ of elemental s-expressions, which are defined 
the same but now using any symbol in $\Sigma$ in applications instead of just constructor symbols. Note 
that the s-expression $\emptyset$ corresponds to $\perp$, so s-exps are partial by default. The approximation
preorder $\sqsubseteq$ is defined for s-exps as the least preorder such that $\mi{se}\sqsubseteq \mi{se}'$
iff $\forall \ese\in \mi{se}.\exists \ese'\in \mi{se}'$ such that $\ese\sqsubseteq \ese'$, $X\sqsubseteq X$
for any $X\in\var$, and $h(\mi{se}_1, \ldots, \mi{se}_n)\sqsubseteq h(\mi{se}'_1, \ldots, \mi{se}'_n)$ 
iff $\forall i. \mi{se}_i\sqsubseteq \mi{se}'_i$.


\begin{example}
Using the signature from Example~\ref{ex:clerks}, and given the s-cterm
$\mi{sct} \equiv e(\{\mi{pepe}, \mi{pilar}\},$ $\{\mi{men},$ $\mi{women}\})$,
we have $\mi{sct} \in \escterm$, while $\{\mi{sct}\} \in \scterm$.
Similarly, given the es-exp
$\mi{esex} \equiv \mi{employees}$ $(\{\mi{madrid}, \mi{vigo}\})$ we have $\mi{esex} \in \mi{E\sexp}$
and $\mi{esex} \not\in \escterm$. Finally, we have that $\{\mi{esex}\} \in \sexp$.
\end{example}

The sets $\ssubst$ and $\scsubst$ of s-substitutions and s-csubstitutions (or just s-csubst) \longtxt{consist of finite mappings from variables to s-exps or s-cterms, respectively}\shorttxt{are substituions with s-exps or s-cterms in their range, respectively}.  \longtxt{We extend  s-substs to be applied to $\mi{ESExp}$ and $\mi{SExp}$ as $\sigma : \mi{E\sexp} \rightarrow \sexp$ defined by }\shorttxt{To apply  s-substs to s-exp we use }
$X\sigma = \sigma(X)$, $h(\overline{se})\sigma  =  \{h(\overline{se\sigma})\}$\longtxt{; and $\sigma : \sexp \rightarrow \sexp$ defined by }\shorttxt{, }$\mi{se}\sigma =  \bigcup_{\ese \in \mi{se}} \ese\sigma$. \longtxt{The approximation preorder $\ordap$ is defined for s-substs as $\sigma \sqsubseteq \theta$ iff $\forall X \in \var. \sigma(X) \sqsubseteq \theta(X)$.
 For any nonempty and finite set $\{\theta_1, \ldots, \theta_n\} \subseteq \scsubst\sbt$ we define $\bigcup \{\theta_1, \ldots, \theta_n\} \in \scsubst\sbt$ as $\bigcup \{\theta_1, \ldots, \theta_n\}(X) = \theta_1(X) \cup \ldots \cup \theta_n(X)$.}
 
\begin{example}
Using the signature from Example~\ref{ex:clerks}, we can define the s-csubstitution
$\sigma \equiv \{ X / \{\mi{pepe}, \mi{pilar}\},$
$Y / \{\mi{men}, \mi{women}\}\} \in \scsubst$. Hence, given
$\mi{esex} \equiv e(\{X\}, \{Y\}) \in \mi{E\sexp}$ we have that
$\mi{esex}\sigma \equiv e(\{\mi{pepe},$ $\mi{pilar}\},$ $\{\mi{men}, \mi{women}\})$.
\end{example}

\longtxt{We obtain the denotation of an expression  as }\shorttxt{The denotation of an expression is }the denotation of its associated s-expression, assigned by the operator $\etose{\_} : \mi{Exp}_\perp \tor \sexp$, defined as $\etose{\bot}=\emptyset$; $\etose{X} = \{X\}$ for any $X\in\var$; $\wetose{h(e_1, \ldots, e_n)} = \{h(\etose{e_1}, \ldots, \etose{e_n})\}$ for any $h\in\Sigma^n$\longtxt{. The operator $\etose{\_}$ is}\shorttxt{, } extended to s-substitutions as $\etose{\sigma}(X)=\wetose{\sigma(X)}$, for $\sigma \in Subst_\perp$. \longtxt{It is easy to check that $\wetose{e\sigma} = \etose{e}\etose{\sigma}$ (see~\cite{Lopez-FraguasRS09-RTA09}).} Conversely, we can flatten \longtxt{an s-expression $se$} \shorttxt{s-exps }to obtain the set $\miflat(se)$ of expressions ``contained'' in it, so $\miflat(\emptyset)=\{\bot\}$\longtxt{ and }\shorttxt{, }$\miflat(se)=\bigcup_{\ese\in se} \miflat(\ese)$ if $se\not=\emptyset$, \longtxt{ where the flattening of elemental s-exps is defined as} $\miflat(X)=\{X\}$\longtxt{ ; }\shorttxt{, }$\miflat(h(se_1,\ldots,se_n))=\{h(e_1,\ldots,e_n)~|~e_i\in \miflat(se_i)\mbox{ for } i=1..n\}$. 
\longtxt{\begin{figure}[tb]
\begin{center}
\framebox{
\begin{minipage}{.9\textwidth}
\begin{center}
\begin{tabular}{l@{~~}cl}
 \textbf{\crule{E}} & $\usexp{} \cltor \emptyset$ \\[.15cm]
 \textbf{\crule{RR}} & $\{X\} \cltor \{X\}$ & if $X \in \var$ \\[.15cm]
\textbf{\crule{DC}} & $\infer[]{\{c(\usexp{1}, \ldots, \usexp{n})\} \cltor \{c(\uscterm{1}, \ldots, \uscterm{n})\}}{\usexp{1} \cltor \uscterm{1} \ \ldots \ \usexp{n} \cltor \uscterm{n}}$ & if $c \in \CS$ \\[.15cm]
\textbf{\crule{More}} & $\infer[]{\usexp{} \cltor \uscterm{1} \cup \ldots \cup \uscterm{n}}{\usexp{} \cltor \uscterm{1} \ \ldots \usexp{} \cltor \uscterm{n}}$\\[.15cm]
\textbf{\crule{Less}} & $\infer[]{\{e\usexp{1}, \ldots, e\usexp{n}\} \cltor \uscterm{1} \cup \ldots \cup \uscterm{m}}{\{\mi{esa}_1\} \cltor \uscterm{1} \ \ldots \ \{\mi{esa}_m\} \cltor \uscterm{m}}$ & $\begin{array}{l}
\mbox{if } n \geq 2, m>0 \mbox{, for any }\\
\{\mi{esa}_1, \ldots, \mi{esa}_m\}  \\
\subseteq \{e\usexp{1}, \ldots, e\usexp{n}\}
\end{array}
$\\[.15cm]
 \textbf{\crule{ROR}} & $\infer[]{\{f(se_1,\ldots,se_n)\} \cltor st}{se_1 \cltor \etose{p_1}\theta \ \ldots \ se_n \cltor \etose{p_n}\theta \ ~\etose{r}\theta \cltor st}$ & if $\begin{array}{l}
(f(p_1, \ldots, p_n) \tor r)\in \prog \\
\theta \in \scsubst\sbt
\end{array}$
\end{tabular}
\end{center}
\end{minipage}
}
\end{center}
\vspace{-.5cm}
    \caption{A proof calculus for constructor systems}
    \label{fig:semCRW}
\end{figure}}
\shorttxt{The semantics of s-expressions is defined by a sequent calculus that proves reduction statements of the form $\prog \vdash se \clto st$ with $se \in \sexpb$ and $st \in \sctermb$, expressing that $st$ represents an approximation to one of the possible structured sets of values for $se$ under the $\ctrs$ $\prog$, with the rules \textbf{\crule{(E)}}  $\usexp{} \cltor \emptyset$; 
\textbf{\crule{(RR)}} $\{X\} \cltor \{X\}$ if $X \in \var$;
\textbf{\crule{(DC)}} $\{c(\usexp{1}, \ldots, \usexp{n})\} \cltor \{c(\uscterm{1}, \ldots, \uscterm{n})\}$ if $\forall i \in \{1, \ldots, n\}. \usexp{i} \cltor \uscterm{i}$ for $c \in \CS$; 
\textbf{\crule{(More)}}  ${\usexp{} \cltor \uscterm{1} \cup \ldots \cup \uscterm{n}} $ if $\forall i \in \{1, \ldots, n\}. \usexp{} \cltor \uscterm{i}$;
\textbf{\crule{(Less)}} $\{e\usexp{1}, \ldots, e\usexp{n}\} \cltor \uscterm{1}$ $\cup$ $\ldots \cup \uscterm{m}$ if $\forall i \in \{1, \ldots, m\}. \{\mi{esa}_i\} \cltor \uscterm{i}$ with $n \geq 2, m>0$ for any $\{\mi{esa}_1, \ldots, \mi{esa}_m\} \subseteq \{e\usexp{1}, \ldots, e\usexp{n}\}$; 
\textbf{\crule{(ROR)}} ${\{f(se_1,\ldots,se_n)\} \cltor st}$ if $\forall i \in \{1, \ldots, n\}. se_i \cltor \etose{p_i}\theta \wedge \etose{r}\theta \cltor st$ for $(f(p_1, \ldots, p_n) \tor r)\in \prog$ and  $\theta \in \scsubst\sbt$.}  

\begin{example}
Using the signature from Example~\ref{ex:clerks}, we have that $\etose{p(X, Y)} = \{p(\{X\}, \{Y\})\}$ and
$\mi{flat}(\{p(\{X\},$ $\{Y, Z\})\}) = \{p(X, Y), p(X, Z)\}$
\end{example}

\longtxt{In Figure \ref{fig:semCRW} we can find the proof calculus that defines the semantics of s-expressions.  Our proof calculus proves reduction statements of the form $se \clto st$ with $se \in \sexpb$ and $st \in \sctermb$, expressing that $st$ represents an approximation to one of the possible structured sets of values for $se$.} 
\longtxt{We refer the interested reader to~\cite{Lopez-FraguasRS09-RTA09} for detailed explanations about the calculus.}\longtxt{We write $\prog \vdash se \clto st$ to express that $se \clto st$ is derivable in our calculus under the \ctrs\ $\prog$.} \longtxt{We say that a proof for a statement $\prog \vdash se \clto st$ is ground iff $se$, $st$ and all the s-exp in the premises are ground.} The \emph{denotation} of an s-expression $se$ under a \ctrs\ $\prog$ is defined as $\den{se}^{\prog} = \{st \in \sctermb~|~\prog \vdash se \clto st\}$, so ${\den{e}}^{\prog} = {\den{\etose{e}}}^{\prog}$. \longtxt{In the following w}\shorttxt{W}e will usually omit the reference to $\prog$. The denotation of $\sigma \in \ssubst$ is defined as $\den{\sigma} = \{\theta \in \scsubst\sbt~|~ \forall X \in \var, \sigma(X) \clto \theta(X)\}$, so for $\theta \in Subst_\perp$ we define $\den{\theta} = \den{\etose{\theta}}$.

\begin{example}
Using the signature from Example~\ref{ex:clerks} and the rules from Example~\ref{ex:rules}, we have 
$\mi{employees}(\{\mi{X}\})$ $\clto \{e(\mi{pepe}, \mi{men})\}$, given the substitution
$X / \{\mi{madrid}\}$. Moreover, we can use this same substitution to reach
$\{e(\mi{maria}, \mi{men})\}$ by using a different program rule.
\end{example}
  
\shorttxt{In~\cite{Lopez-FraguasRS09-RTA09} we proved the adequacy of the calculus w.r.t.\ term rewriting for programs without extra variables. As extra variables are very common when using narrowing, we have extended the proof to deal with them. A remarkable point is that, as a consequence of the freely instantiation of extra variables performed in \textbf{\crule{ROR}}, then every program with extra variables turns into non-deterministic. For example under a program $\{f \tor (X, X)\}$ and using $0, 1 \in \CS^0$ we can prove $\etose{f} = \{f\} \clto \{(\{0\}, \{1\})\} = \etose{(0,1)}$.} 
\longtxt{The setting presented in~\cite{Lopez-FraguasRS09-RTA09} was not able to deal with extra variables. As programs with extra variables are very common when using narrowing, 
for this work we decided to extend 
the setting to deal with them. But then we realized that the semantics had the foundations
to deal with extra variables, 
as the rule \textbf{\crule{ROR}} from Figure \ref{fig:semCRW} allows to instantiate extra variables freely with s-cterms: therefore all that was left was proving the adecuacy of the semantics in this extended scenario.
Nevertheless, as a consequence of the freely instantiation of extra variables in \textbf{\crule{ROR}}, then
every program with extra variables turns into non-deterministic. For example consider a program $\{f \tor (X, X)\}$ for which the constructors $0, 1 \in \CS^0$ are available, then we can do:
$$
\begin{scriptsize}
\infer[\textbf{\crule{ROR}}]{\etose{f} = \{f\} \clto \{(\{0\}, \{1\})\} = \etose{(0,1)}}
{
\infer[\textbf{\crule{DC}}]{\etose{(X, X)}[X/\{0,1\}] = \{(\{0,1\}, \{0,1\})\} \clto \{(\{0\}, \{1\})\}}
{
\infer[\textbf{\crule{Less}}]{\{0,1\}  \clto \{0\}}
		{\infer[\textbf{\crule{DC}}]{\{0\} \clto \{0\}}{}} &
\infer[]{~\{0,1\}  \clto \{1\}}{\ldots}
 }
}
\end{scriptsize}
$$}

But in fact this is not very surprising, and it has to do with the relation between non-determinism and
extra variables~\cite{AntoyH06Extra}, but adapted to the run-time choice semantics~\cite{hussmann93,rodH08}
induced by term rewriting.  As a consequence of this we assume that all the programs contain the function $?$, so we only consider 
non-confluent \trss. We admit that this is a limitation of our setting, but we also conjecture that 
for confluent \trss\ a simpler semantics could be used, for which the packing of alternatives of c-terms
would not be needed. However, the important point to bear in mind is that
having $?$ at one's disposal is enough to express the non-determinism of any program~\cite{Han05TR}, so
we can use it to define the transformation $\setoe{\_}$ from s-exp and elemental s-exp to partial 
expressions that, contrary to $\miflat$, now takes care of the keeping the nested set structure by means 
of uses of the $?$ function. Then $\setoe{\_} : \esexp \rightarrow \Exp_\perp$ is defined by 
$\setoe{X} = X$, $\setoe{h(se_1, \ldots, se_n)} = h(\setoe{se_1}, \ldots, \setoe{se_n})$; and 
$\setoe{\_} : \sexp \rightarrow \Exp_\perp$ is defined by $\setoe{\emptyset} = \perp$, 
$\setoe{\{ese_1, \ldots, ese_n\}} = \setoe{ese_1}~?~\ldots~?~\setoe{ese_n}$ for $n > 0$, where in the 
case for $\setoe{\{ese_1, \ldots, ese_n\}}$ we use some fixed arbitrary order on terms in the line of 
Prolog~\cite{SterlingShapiro86} for arranging the arguments of $?$. This operator is also overloaded 
for substitutions as $\setoe{\_} : \ssubst \rightarrow \Subst_\perp$ as 
$(\setoe{\sigma})(X) = \setoe{\sigma(X)}$.
Thanks to the power of $?$ to express non-determinism, that transformation preserves the semantics from
Figure~\ref{fig:semCRW}, and we can use it to prove the following new result about the adequacy of the
semantics for programs with extra variables---see~\cite{generatorsProofs} for a detailed proof.

\begin{theorem}[Adequacy of $\den{\_}$] 
For all $e, e' \in \Exp, t \in \CTerm_\perp, st \in \sctermb$: \\
\textbf{Soundness }$st \in \den{\etose{e}}$ and $t \in \miflat(st)$ implies $e \rw^* e'$ for some $e' \in \Exp$ such that $t \ordap |e'|$. Therefore, $\etose{t} \in \den{\etose{e}}$ implies 
$e \rw^* e'$ for some $e' \in \Exp$ such that $t \ordap |e'|$. Besides, in any of the previous cases, if $t$ is total then $e \rw^* t$.\\
\textbf{Completeness} $e \rw^* e'$ implies $\wetose{|e'|} \in \den{\etose{e}}$. Hence, if $t$ is total then $e \rw^* t$ implies $\wetose{t} \in \den{\etose{e}}$. 
\end{theorem}
\longtxt{
\begin{figure}[bt]
\begin{center}
\framebox[\textwidth]{
\begin{tabular}[t]{c|c}
$
\begin{array}{l}
\_  \vdash \mrel{\_}{\_} \subseteq \ctrs \times \sctermb \times \Exp \\
\begin{array}{ll}
\prog \vdash \mrel{st}{e} & \mbox{ if } \forall est \in st, \prog \vdash \mrel{est}{e} \\
\end{array}
\; \\ \; \\ \; \\
\end{array}
$
 &
$
\begin{array}{l}
\_ \vdash \mrel{\_}{\_} \subseteq \ctrs \times \esctermb \times \Exp \\
\begin{array}{ll}
\prog \vdash \mrel{X}{e} & \mbox{ if } \prog \vdash e \rw^* X \\
\prog \vdash \mrel{c(\overline{st})}{e} & \mbox{ if } \prog \vdash e \rw^* c(\overline{e})  \mbox{ for some } \overline{e}
\end{array}\\
\mbox{ ~~~~~~~~~~~~~~~such that } \forall e_i \in \overline{e}, \prog \vdash \mrel{st_i}{e_i}\\
\end{array}
 $
\end{tabular}
}
 \end{center}
\vspace{-.5cm}
    \caption{Domination relation}
    \label{fig:DomRel}
\end{figure}
}
We refer the interested reader to~\cite{Lopez-FraguasRS09-RTA09} and~\cite{LRS09ReportFully}
(Theorems 2 and 3) for more properties of $\den{\_}$ like compositionality or monotonicity, some of which are used in the
proofs for the results in the present paper.
\longtxt{There is another characterization of $\den{\_}$ closer to term rewriting which is based of the  \emph{domination relation $\mrel{\_}{\_}$} presented in Figure~\ref{fig:DomRel} (we will omit the prefix ``$\prog \vdash$'' when it is implied by the context).}
\longtxt{With this relation we try to transfer to the rewriting world the finer distinction between sets of values that the structured representation of $\sctermb$ allows us to perform.  We extend the relation $\mrel{\_}{\_}$ to $\_\vdash \mrel{\_}{\_} \subseteq \ctrs \times \scsubst\sbt \times Subst$ by $\mrel{\theta}{\sigma}$ iff $\forall X \in \var, \mrel{\theta(X)}{\sigma(X)}$. As can be seen in~\cite{LRS09ReportFully}, this relation is a key ingredient to prove the soundness of $\den{\_}$, and its equivalence to $\den{\_}$ is stated in the following result. 
\begin{lemma}[Domination]\label{TMrelvsSem}\label{lemDenSubstMrel1}For all $ e \in \Exp, \mi{st} \in \sctermb$, $st \in \den{\etose{e}}$ iff $\mrel{st}{e}$. Besides, regarding substitutions, for all $\sigma \in \Subst$, $\theta \in \scsubst$ we have that $\theta \in \den{\etose{\sigma}}$ iff $\mrel{\theta}{\sigma}$.
\end{lemma}}


%
