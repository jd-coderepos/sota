

\documentclass[letterpaper, 10 pt, conference]{ieeetran}  



\IEEEoverridecommandlockouts                              

\overrideIEEEmargins                                      



\usepackage{graphics} \usepackage{epsfig} \usepackage{amsmath} \usepackage{amssymb}  \usepackage{amsfonts}                                
\usepackage{enumerate}
\usepackage{psfrag}

\def\qedp{\hspace*{\fill}~{\tiny }}
\def\salt{\vskip 0.2 true cm}

\newtheorem{theorem}{Theorem}
\newtheorem{itlemma}{Lemma}
\newtheorem{itdefinition}{Definition}
\newtheorem{itproposition}{Proposition}
\newtheorem{itfact}{Fact}
\newtheorem{itremark}{Remark}
\newtheorem{itassumption}{Assumption}
\newtheorem{itcorollary}{Corollary}
\newtheorem{itexample}{Example}
\newenvironment{definition}{\begin{itdefinition}\rm}{\end{itdefinition}}
\newenvironment{fact}{\begin{itfact}\rm}{\end{itfact}}
\newenvironment{proposition}{\begin{itproposition}\rm}{\end{itproposition}}
\newenvironment{remark}{\begin{itremark}\rm}{\end{itremark}}
\newenvironment{assumption}{\begin{itassumption}\rm}{\end{itassumption}}
\newenvironment{lemma}{\begin{itlemma}\rm}{\end{itlemma}}
\newenvironment{corollary}{\begin{itcorollary}\rm}{\end{itcorollary}}
\newenvironment{result}{\begin{itresult}\rm}{\end{itresult}}
\newenvironment{example}{\begin{itexample}\rm}{\end{itexample}}

\title{Switching Control for Parameter Identifiability of Uncertain Systems }




\author{Giorgio Battistelli  and Pietro Tesi
\thanks{G. Battistelli is with the
        University of Florence, Dipartimento di Ingegneria dell’Informazione (DINFO), Via di Santa Marta 3, 
        50139 Firenze, Italy (e-mail: giorgio.battistelli@unifi.it).  P. Tesi is with University of Groningen,  
        ENgineering and TEchnology institute Groningen (ENTEG), 
        Faculty of Mathematics and Natural Sciences, Nijenborgh 4, 9747 AG Groningen,
        The Netherlands (e-mail: p.tesi@rug.nl).
       }
}


\begin{document}



\maketitle
\thispagestyle{empty}
\pagestyle{empty}


\begin{abstract}
This paper considers the problem of identifying the parameters of an uncertain 
linear system by means of feedback control. The problem 
is approached by considering time-varying controllers.
It is shown that even when the uncertainty set is not finite, 
parameter identifiability can be generically ensured by 
switching among a finite number of linear time-invariant controllers. 
The results are shown to have several implications, ranging from 
fault detection and isolation to adaptive and supervisory control.
Practical aspects of the problem are also discussed in details.
\end{abstract}


\section{introduction}


Identifying the parameters of an uncertain 
system from input-output data is  
a problem of long-standing fundamental interest in control engineering. 
This problem is often referred to as the problem of \emph{parameter identifiability} 
\cite{Willems,glover}. 
This paper considers the identifiability problem 
with respect to uncertain linear systems where the uncertainty 
set consists of a known bounded set possibly containing 
a continuum of parameters. For this class of systems,
we address the problem of ensuring the identifiability 
of the unknown parameters of the system by means of feedback controllers.

The motivations for studying this problem are immense. For instance, parameter identifiability of a feedback loop can be of interest in
the context of {\em fault detection/isolation} for systems subjects to failures, in order to make it possible to promptly detect any departure from the nominal
behavior and to precisely identify the parameter variation \cite{kinnaert,cocquempot}.
Another application is {\em control reconfiguration} wherein the objective is that
of replacing the active controller (typically designed in order to ensure robust stability in all the uncertainty region) with a different one providing
enhanced (possibly optimized) performance \cite{liberzon,steffen}. Finally, on-line estimation of the uncertain parameters under feedback can be exploited when dealing with systems which naturally exhibit multiple operating conditions  for constructing {\em adaptive control} laws. In fact, many existing adaptive control techniques rely on the idea of \emph{certainty equivalence}, 
which amounts to applying at each instant  of time the controller designed for 
the model that best fits the available data \cite{Astrom,Ioannou}. 

In this paper, the problem of parameter identifiability 
is approached by searching for feedback control laws 
under which closed-loop behaviors obtained 
with different system parameters can be distinguished one from another.
To this end, we introduce a notion of \emph{discerning control}.
Parameter identifiability is then defined precisely in terms of 
discerning control.


In principle, parameter identifiability 
under feedback can always be ensured by means 
of a \emph{probing signal} injected into the plant as an additive perturbation input, 
superimposed to the control variable \cite{Astrom,Ioannou}. Nevertheless, in many contexts, such a solution should be avoided due to the inherent drawback of leading the feedback loop away from the desired behavior,
thus destroying regulation properties. This is especially true when the behavior of the feedback loop has to be monitored continuously as in the contexts of fault/detection isolation and adaptive control.
Then, a natural question arises on whether or not it is possible to guarantee parameter identifiability directly by means of a feedback control law, possibly designed also to satisfy
other control objectives (\emph{e.g.}, stability in nominal operating conditions). 
An affirmative answer to this question was given in \cite{Ba13,BaBaTe14} for the special case
of switching linear systems, \emph{i.e.}, when the uncertain parameters can take on only a finite number of possible values. Specifically, in \cite{Ba13,BaBaTe14}, it is shown that, under quite mild assumptions, for switching linear systems almost all linear time-invariant (LTI) controllers are discerning. 

In general, an analogous result cannot be established in case of 
continuously parameterized systems, \emph{i.e.}, when the uncertainty set is not finite.
The reason is inherently tied to the fact that LTI controllers do not generally 
provide a sufficient level of excitation to the loop \cite[Chapter 2]{Astrom}.
The problem of loss of identifiability due to feedback can be overcome 
by means of time-varying controllers, and one possibility 
is given by switching control \cite{MOPA,MOPA2}. 
In this paper, we exploit this property
to show that parameter identifiability can in fact be \emph{generically} ensured by 
switching among a finite number of LTI controllers (hereafter called modes), provided that the number of different controller modes is sufficiently large. 
Specifically, an upper bound on the number of controller modes needed for parameter identifiability is given in terms of the dimension of the uncertainty set.
Moreover, we show that the result remains
true even if we restrict the controller modes to be of a given fixed order
and to satisfy certain stability requirements. 
The latter result is perhaps surprising as it indicates that
the seemingly conflicting goals of ensuring parameter identifiability as well as a satisfactory behavior of the feedback system (at least under nominal conditions) 
can be simultaneously accomplished by means of switching control. 

As a further contribution, we analyze the properties of 
least-squares parameter estimation in connection with the use of discerning controllers.
Specifically, in order to ensure the practical applicability of the estimation technique, we focus on a {\em multi-model} approach wherein the estimate is selected among 
a finite number of possible values of the parameter vector (obtained by suitably sampling the uncertainty set).
In this context, a bound on the worst-case parameter estimation error is derived, which accounts also for the presence of unknown but bounded disturbances 
and measurement noises. This latter result is of special interest in the context of {\em multi-model adaptive switching control} (MMASC) 
of uncertain systems \cite{Ba13}, \cite{morse97}-\nocite{hespanha,debruye,bald}\cite{BaBaTe13}, of which multi-model least-square parameter estimation constitutes one of the key elements. In this respect, it has been shown in  \cite{Ba13,BaBaTe13} that, in the case of a finite uncertainty set, by employing discerning
controllers it is possible to construct MMASC schemes which enjoy quite 
strong stability properties, namely exponential input-to-state stability. Hence, the results of the paper suggest that similar stability properties could be achieved also in the
case of continuously parameterized uncertainty. This issue will be the subject of further research.

The remainder of the paper is organized as follows. In Section II, we describe the framework under 
consideration. In Section III,  the main results of the paper are given concerning the existence and genericity of switching controllers ensuring parameter identifiability.
Section IV analyzes the properties of multi-model least-squares parameter estimation. 
Examples
are finally given in Section V. 

For the sake of clarity, all the proofs are reported in the Appendix section.

\emph{Notation.} Before concluding this section, let us introduce some notations
and basic definitions. Given a vector ,  denotes its Euclidean norm.
Given a symmetric, positive semi-definite matrix , we denote by
 and  the minimum
and maximum eigenvalues of , respectively. Given a matrix ,  is its transpose and  its spectral
norm. Given a measurable time function  and a time interval ,
we denote the  and  norms of  on  as

and

respectively. When , we simply write  and . 
Finally, we let  and  denote the sets of square integrable and, respectively, (essentially) bounded time functions on .

\section{Framework and objectives}\label{overview}


We consider a process described by an uncertain linear system  

where  is the state, 
 is the input,  
is the output, and  is 
an unknown parameter vector belonging to the known
bounded set .

The problem of interest is that of designing a controller 
ensuring global discernibility, \emph{i.e.}, identifiability of 
the unknown parameter vector  from observations of the plant input/output data 

where {\em col} stands for column vector. 
In \cite{BaBaTe14}, the problem was addressed in the special case when the set  is finite and it was shown that, 
under mild assumption, global discernibility can be ensured by means of a LTI controller. When the set  is not finite, a single LTI controller
in general cannot ensure global discernibility by itself. Nevertheless, as will be shown in the following, it turns out that global discernibility can be achieved
by switching among a finite number of LTI controllers. Accordingly, let the controller be described by a switching linear system
where  is the controller state and
 is the 
switching signal, \emph{i.e.} the signal 
(right continuous) which identifies 
the index of the active system at each instant of time. Hereafter, it will be supposed that the switching signal 
is generated so as to have a finite number of discontinuity points in every finite time interval.
For any , , , , and  are constant matrices of appropriate dimensions.
In the sequel, we shall denote 
by  the LTI system with state-space representation  and by 
the control law associated with the switching signal .

Denote by  and 
the state and the output of the closed-loop system  resulting from 
the interconnection of (\ref{equazione:sistema}) and (\ref{equazione:controllore})
when the unknown parameter takes value  and the controller switching signal is .
The corresponding dynamics can be therefore expressed as
{\setlength\arraycolsep{1pt}
\label{}
}where, for any  and ,
 
Finally, let 

denote the value at time  of  
when the plant initial state at time  is , 
the controller initial state is , the unknown parameter vector takes value , and the controller switching signal is .
The following notions 
can be introduced. 

\begin{definition} \label{def:disc:CLD} 
Let the process be as in (\ref{equazione:sistema}) and assume that  and  are available for measurements.
Further, consider two distinct parameter vectors .
A switching controller of the form (\ref{equazione:controllore}) 
is said to be (,)-\emph{discerning} if, 
for any time-interval , with , 
there exists a switching signal  such that

for all nonzero quadruples of vectors
. 
In addition, the switching controller (\ref{equazione:controllore}) is said to be 
\emph{globally discerning} if it satisfies condition (\ref{eq:main}) for all 
pairs  of different parameter vectors.
\qedp
\end{definition} 
\salt

In words, when the switching controller (\ref{equazione:controllore}) is (,)-discerning and
a discerning switching signal  is adopted, 
then   and 
cannot give rise to the same observation data 
when  is in the feedback loop (unless the initial conditions are null). As a consequence, under global discernibility, 
it is possible to uniquely identify the unknown parameter vector  by observing  on the interval .



\section{Main results}\label{main}


In this section, we derive sufficient conditions for a switching control to be discerning and we show that, under mild assumptions, almost every switching controller
is discerning provided that the number  of controller modes is sufficiently large.

To this end, notice preliminarily that a necessary condition for the existence of a discerning controller is that all the pairs 
are observable. In fact, the presence of unobservable dynamics would entail the existence
of non-zero trajectories of the closed-loop state  corresponding to zero trajectories of the closed-loop output   and, hence, for which
it would be impossible to infer the plant mode. 
Accordingly, the following assumption is considered. 

\begin{enumerate}
\item[\bf A1.] The pair  is observable for all .
\end{enumerate} 

Let now  denote the characteristic polynomial of the closed-loop system  resulting from the
feedback interconnection of the plant  with the -th controller mode . The following result holds. \salt

\begin{lemma} \label{lem:poly} Let assumption \textbf{A1} hold and suppose that, for any pair of distinct parameter vectors , there exist at least one index  
such that the two closed-loop characteristic polynomials  and  are coprime. Then, the following properties are true.
\begin{enumerate}[(i)]
\item the switching controller  (\ref{equazione:controllore})  is globally discerning;
\item condition (\ref{eq:main}) holds for any switching signal  such that each controller mode  is active, at least, on an interval 
of positive measure.
\end{enumerate}
\qedp
\end{lemma} \salt

Let now  denotes the total number of elements of the controller matrices  when the controller
order is , and let   be the set of  controller matrices\footnote{Here, with a little abuse
of notation, we identify the quadruples  with an -dimensional vector containing all the elements of the matrices
  according to a given order.}  for which 
the two closed-loop polynomials  and  are coprime. Further, for a given number  of controller modes, let
 be the set of switching controllers   satisfying the hypothesis of Lemma 
\ref{lem:poly} (\emph{i.e.}, such that  for any pair of distinct parameter vectors , there exist at least one index  
such that the two closed-loop characteristic polynomials  and  are coprime). Since, in view of Lemma 
\ref{lem:poly}, all switching controllers belonging to  are globally discerning, we are now interested in studying the properties of the sets
 and .

To this end, recall that coprimeness of the two polynomials  and  is equivalent to the fact that their 
\emph{Sylvester resultant}  (\emph{i.e.}, the determinant
of the Sylvester matrix associated with the two polynomials) is different from . With this respect, we note that  depends polynomially on the elements
of the controller matrices . Hence, when only a single pair of distinct parameter vectors  is taken into account,
the set of controller matrices  for which , which is the complement of , is an algebraic set.
Then, as well known, only two situations may occur: either  is always satisfied; or  is satisfied on a set with zero Lebesgue measure.
In this latter case, the set  is \emph{generic}
\footnote{Recall that a subset  of a topological space is generic when it is open and dense: for any , then there exists a neighborhood of  contained in ; for any , then every neighborhood of  contains an element of .} (since it is the complement of a proper algebraic set). The following lemma, proved in \cite{Ba13}, provides sufficient conditions for such a favorable situation to occur.
 \salt

\begin{lemma} \label{lem:autp}
\cite{Ba13} Let assumption \textbf{A1} hold and consider two distinct parameter vectors . Then a  controller  ensuring 
coprimeness of the two polynomials  and  exists if and only if 
the following two conditions hold:
\begin{enumerate}[(a)]
\item the transfer functions of  and 
are different;
\item the characteristic polynomials of the uncontrollable parts of  and  are coprime.
\end{enumerate}
In addition, when conditions (a)-(b) holds, for any given controller order  the set  is generic and of full measure on . 
\qedp
\end{lemma}  

Building on the above lemmas, under suitable regularity assumptions for the functions , it is possible to derive conditions for the existence of a global
discerning switching controller on the whole uncertainty set . In particular,  by exploiting the results of \cite{Sontag}, the following theorem can be stated. \salt

\begin{theorem} \label{thm1}
Let the uncertainty set  be contained in an analytic manifold  of dimension  and let the elements of the system matrices 
 be analytic functions of  on . Further, let assumption \textbf{A1} and conditions (a)-(b) of Lemma \ref{lem:autp} hold on .
Then, provided that , the set  is generic and of full measure on .
\qedp
\end{theorem} \salt

A few remarks are in order. First of all, notice that Theorem \ref{thm1} provides a bound on the number  of controller modes
that may be needed in order to ensure identifiability of the unknown parameter  in . Such a bound is consistent with the results of \cite{Ba13,BaBaTe14} where it is shown
that when the set  is finite, i.e., it is a -dimensional manifold, one single LTI controller is generically discerning. As discussed in  \cite{Sontag}, the bound
 for parameter distinguishability is tight for general maps (in the sense that when  one can find counterexamples). However,
for specific cases, fewer controller modes can be sufficient. For instance, when  is a scalar parameter, one can consider a single LTI controller  
and plot the root locus of the closed loop polynomial  as a function of . Then, global discernibility is guaranteed provided that such a root locus never cross itself.

Notice finally that the set of analytic functions considered in the statement of Theorem \ref{thm1} is quite general as it captures many function classes of interest 
(\emph{e.g.}, polynomials, 
trigonometric functions, exponentials, and also rational functions as long as away from singularities). 

\subsection{Accounting for additional control objectives}

In the foregoing analysis, only the global discernibility objective has been taken into account. However
generally speaking, a control law should typically satisfy other control objectives, the most fundamental one being stability.
With this respect, suppose that we want the switching controller to ensure closed-loop stability in a given subset  of  together with
global discernibility. For instance,  can represent the neighborhood of the nominal operating condition and the switching controller should
be designed so as to ensure: a satisfactory behavior in nominal conditions as well as the possibility of promptly identifying any departure from the nominal behavior (e.g, 
for fault-detection and isolation or for control reconfiguration purposes). An extreme case is when  so that we want to design a robust and globally discerning 
switching controller ensuring stability for any possible operating condition (of course this may be possible or not depending on the of size of the uncertainty set ).

As well known, a sufficient condition to ensure stability under switching is the existence of a common Lyapunov function. For example, if we consider a quadratic parameter-dependent
Lyapunov function
, then in order for the closed-loop system  to be stable for any  it is sufficient that
there exists  such that

for any  and for any . When the set  is compact, by means of simple continuity arguments, it is immediate to show
that, for any given smooth , the set of controller matrices  satisfying (\ref{eq:stab2}) is an open subset of .
Then, recalling that, by definition, any non-empty open set contains a closed-ball of positive radius, the following result on the existence of global discerning controllers ensuring also stability
can be readily stated. 

\begin{theorem} \label{thm2}
Let the same hypotheses of Theorem \ref{thm1} hold. Further, let the set  be compact and suppose that there exists at least
one controller  for which conditions (\ref{eq:stab1}) and (\ref{eq:stab2}) are satisfied with the Lyapunov matrix 
depending continuously on .
Then, whenever , the set of switching controllers   that jointly satisfies (\ref{eq:stab1}) and (\ref{eq:stab2}) and ensures global discernibility
is non-negligible, in the sense that it contains a ball of positive radius in 
. \qedp 
\end{theorem} 


\section{Multi-model least-squares parameter estimation}


In this section, we discuss how the unknown parameter vector  can be estimated from the closed-loop data  on an interval
 and we show that, when the data result from application of a discerning switching controller, the resulting estimate enjoys some nice 
properties even in the presence of unknown disturbances and measurement noises.

To this end, recall that, when the unknown parameter vector takes value , the evolution of  on the interval
 takes the form . Then the set  of all possible
closed-loop data on the interval  associated with  and with the switching signal  can be written as
{\setlength\arraycolsep{0pt} 
}Hence a natural approach for estimating the plant unknown parameters is the {\em least-squares} one, which
amounts to selecting the parameter vector  for which the distance between the observed close-loop data  on the interval 
 and the set 
is minimal. Accordingly, the optimal least-squares estimate  can be obtained as

where


Concerning the computation of the distance (\ref{eq:distance}), for any possible feedback loop
, let   denote its state transition matrix  and let  be 
its observability Gramian on the interval , \emph{i.e.},

Notice that, for any globally discerning switching control law, the observability Gramian 
 turns out to be
positive definite for any  (otherwise there would be zero output trajectories corresponding to non-zero state trajectories and
parameter identification would not be possible). Hence, in this case, 
the minimization in (\ref{eq:distance}) yields
{\setlength\arraycolsep{0pt} 
}For further considerations on how this quantity can be computed in practice the interested reader is referred to Appendix A of \cite{BaBaTe13},
where a similar problem is addressed.

From Definition 1, it is immediately clear that when a globally discerning switching control law is adopted, for any non null  the  is zero if and only if
 coincides with , the true parameter vector.\salt

\begin{proposition}\label{prop:least-squares}
Let the switching controller (\ref{equazione:controllore}) be globally discerning and
a discerning switching signal  be adopted on the observation interval . Further, let the observed
data  be generated by the closed-loop system (\ref{eq:closed-loop}) from initial condition 
. Then, .
\qedp
\end{proposition}
\salt

While the above proposition illustrates the theoretical effectiveness of the leas-squares estimation criterion in ideal conditions, in practice computation of the minimum
in (\ref{eq:MDC}) can be a quite challenging task when the set  is not finite. In this case, a standard approach is the {\em multi-model} one which amounts to considering only a finite
number, say L, of possible parameter values by constructing the finite set 
. 
Typically,  is obtained by sampling the set  with a given guaranteed density , so that for any  there exists
at least one  such that . When such a condition is satisfied, we say that  is {\em -dense} in .
Accordingly, the following {\em multi-model least squares criterion}  can be used to estimate the unknown parameter vector 

as an alternative to (\ref{eq:MDC}). 

\begin{remark}
Guidelines on how to choose an {\em -dense} finite covering for 
can be found, for instance, in \cite{debruye2}-\nocite{baldi2}\cite{Buchstaller}. \qedp
\end{remark}

\subsection{Properties of multi-model least-squares parameter estimation}

When analyzing of the properties of an estimation criterion, either optimal as in (\ref{eq:MDC}) or approximate as in (\ref{eq:MDC:MM}),
it is important to take into account also the effects of process disturbances and measurement noises. With this respect, in the following analysis we suppose that the
plant state and measurement equations are
affected by additive disturbances  and ,
 respectively, \emph{i.e.},

with  and .
Then, it is an easy matter to verify that a state space
representation of the closed-loop system 
 takes the form

where  and

for any  and .
Further, thanks to linearity, the
closed-loop data  can be decomposed as

where  is the forced response and  is the natural response
which can be written as

with   the same function of the previous sections.
Notice also that the forced response  can be bounded in terms of the disturbance amplitude as follows. 

\begin{proposition}
Let the set  be compact and let the elements of , , and  depend continuously on .
Then, for any , there exists a positive real  such that

\qedp
\end{proposition}

Notice now that, by virtue of the triangular inequality, we have

for any . Hence, the properties of the least-squares estimate  can be investigated by deriving bounds on .
In this respect, the following result is relevant. 

\begin{proposition}
Consider the same assumptions as in Proposition \ref{prop:least-squares}. Then, for any 
,

where

\qedp
\end{proposition}

For any pair of parameter vectors , the {\em joint observability Gramian}  provides information concerning
the degree of distinguishability between the two closed-loop systems  and . In fact,
whenever the switching control law  is globally discerning, the matrix  is singular if and only if 
(this is a straightforward consequence of Proposition \ref{prop:least-squares}). Then,
we can derive the following result. \salt

\begin{lemma}\label{lemma:K-function}
Let the same assumptions as in Proposition \ref{prop:least-squares} hold. Further, let the set  be compact and let the elements of , , and  depend continuously on . 
Moreover, let the switching controller (\ref{equazione:controllore}) be globally discerning and
a discerning switching signal  be adopted on the observation interval . Then, there exist two class  functions\footnote{Recall
that a function  belongs to class  if it is continuous, strictly increasing, and .}  
and  such that

for any .
\qedp
\end{lemma}\salt

The importance of Lemma \ref{lemma:K-function} is that it allows to bound the distance , pertaining to the noise-free dynamics,
in terms of the initial state  and of the distance between the true parameter vector  and the candidate estimate . In particular, the left-most inequality in
(\ref{eq:K-function}) ensures that  cannot be small when the discrepancy  is large, whereas the right-most inequality
ensures that  nicely degrades to  as the estimate  approaches the true value . With respect to the latter observation,
from (\ref{eq:K-function}) it follows that when   is -dense in  there always exists a parameter vector  such that
. By exploiting inequalities (\ref{eq:triangular}) and (\ref{eq:K-function}), the main result of this section can finally be stated.
\salt

\begin{theorem}
Let the same assumptions as in Lemma \ref{lemma:K-function} hold. Further, let  be -dense in . Then, the estimate  obtained as in
(\ref{eq:MDC:MM}) is such that

where  is the inverse of . \qedp
\end{theorem} 

Concerning the upper bound on the estimation error provided by inequality (\ref{eq:bound}), it can be seen that  the term  
(which decreases the denser the sampling  is) accounts
for the fact that only a finite number of models is considered, while the term
 can be seen as a sort of noise-to-signal ratio, and indeed goes to  as the disturbance  goes to . 

\section{An example}


In the following, a simple example is provided in order to illustrate how identifiability of an uncertain parameter vector
can be achieved my means of switching control. To this end, consider an LTI plant with system matrices

with , and let the switching controller be a purely proportional one
 
Since, the plant transfer matrix is , each closed-loop characteristic polynomial takes the form


\subsection{One uncertain parameter} 

Suppose first, for illustration purpose, that only the gain  is uncertain whereas
 is perfectly known, i.e., . Notice that assumption \textbf{A1} holds whenever . Hence, in this case, we can exploit Lemma 1 and consider, for any pair  the two polynomials

As it can be easily verified, the Sylvester resultant of such polynomials is

Then, it can be seen that if , the resultant is different from  whenever  and  are different.  Hence, in this
case, a single proportional controller with non-null gain is globally discerning and there is no need for
considering a switching controller in that . Similar considerations hold when  is uncertain
and  is perfectly known.

\subsection{Two uncertain parameters} 
Suppose now that both  and  are uncertain, \emph{i.e.},
. Again, assumption \textbf{A1} holds provided that . Straightforward calculations allow to see that, in this case, the resultant of the two polynomials
 
is

Hence, a single controller is not sufficient for global discernibility as by choosing

one has . In fact, since  can be arbitrarily small, it is always possible to find  so as to satisfy (\ref{eq:ab}) regardless of the amplitude of the uncertainty set . Then . \\
On the contrary, it can be seen that a switching controller with two modes, , is generically globally discerning.
To see this, notice that

and

If we choose  and  such that  , , and  the above determinant turns out to be
different from . As a consequence, in this case, the two resultants  and  
can simultaneously vanish if and only if  and  which is equivalent to
 and . Hence, we have that  which is generic and of full measure in .

\section{Conclusions}

In this paper, we have addressed the problem of identifying the parameters of an uncertain 
linear system by means of switching control. 
It was shown that even when the uncertainty set is not finite, 
parameter identifiability can be generically ensured by 
switching among a finite number of linear time-invariant controllers. 
In particular, the results show that an upper bound on the number of controller modes needed for parameter identifiability can be given in terms of the dimension of the uncertainty set. 
The results also indicate that
the seemingly conflicting goals of ensuring parameter identifiability as well as a satisfactory behavior of the feedback system can be simultaneously accomplished by means of switching control. 

Several practical aspects have also been discussed. In particular, 
we have analyzed the properties of 
least-squares parameter estimation in connection with the use of discerning controllers,
providing bounds on the worst-case parameter estimation error
in the presence of: i) finite covering of the uncertainty set; and 
ii) bounded disturbances affecting the process dynamics as well as measurement noises. 
 
The results lend themselves to be extended in various directions.
Most notably, these results find a very natural application
in the context of switching control for uncertain systems. In this respect, 
we envision that the analysis tools introduced in this paper should 
lead to the development of novel control reconfiguration
algorithms capable of achieving input-to-state stability 
for uncertain systems even when the uncertainty set is described 
by a continuum.

\section*{Appendix}


{\em Proof of Lemma 1:} Consider two distinct parameter vectors  and consider an index  for which  and 
are coprime. Let the controller  be active on an interval .
Consider now a nonzero quadruples of vectors  representing possible initial states of the two feedback loops  
and  at time . Let  be the corresponding states 
that are reached at time , \emph{i.e.}, at the beginning of the time
interval , under the switching law . Suppose now the switching signal  is chosen so as to satisfy a dwell-time condition, \emph{i.e.}, in such a way that there exists a lower bound
 on the time interval between subsequent variations of the controller index. Then, it is immediate to see that, under such a switching law, 
when  then also . In fact, such a state is reached after switching a finite number of times between 
autonomous linear systems, i.e., the feedback loops, and
it is known that an autonomous linear system cannot reach the zero state in finite time starting from a non-zero initial state. Notice now that, under assumption \textbf{A1}, coprimeness of
the polynomials  and  implies observability of the parallel system

(see for instance Proposition 1 of \cite{Ba13}).
Then, if we initialize such a system as  at time ,
we have that  is different from  a.e. on ,
where ``a.e.'' stands for ``almost everywhere'', \emph{i.e.} everywhere 
except on a set of zero Lebesgue measure. This, in turn, implies that
, or equivalently, 
,  a.e. on . Then, by choosing a switching signal  which satisfies
a dwell-time condition and is such that each controller mode  is active, at least, on an interval  of positive measure, the same line reasoning can be repeated
for any pair  with , thus concluding the proof. \qedp

{\em Proof of Theorem 1:} 
Notice first that the resultant  of the two polynomials  and  is a polynomial (and hence analytic) function of the elements
of the matrices , , and . This, in turn, implies that  is an analytic function of  and
 (since the composition of analytic functions is analytic). Notice now that, under the stated hypotheses, Lemma 2 ensures that, for any pair  with ,
it is possible to find at least one set of matrices  such that  and  are coprime and, hence,
. Recall, finally, that the set  corresponds to the set of switching controllers  for which the vector function
  is different from  for any pair  with .
Then, proceeding as in the proof of Theorem 2 of \cite{Sontag}, we can conclude that  is generic and of full measure on  whenever .
\qedp

{\em Proof of Theorem 2:} 
Let  be a controller for which conditions (\ref{eq:stab1}) and (\ref{eq:stab2}) are satisfied with  continuous in . 
Further, consider a closed ball  in the controller parameter space  centered in  and with radius 
and let 

Note that, in view of the compactness of   and of the continuity of  and , we have that . Moreover, under the considered hypotheses, it is easy to show that 
 depends continuously on  (in this respect, notice that 
 is an affine function of the controller matrices ).
Hence, this implies  the existence of  such that , 
\emph{i.e.}, such that  all the controllers in  satisfies 
(\ref{eq:stab2}) with the same Lyapunov matrix . As a consequence, the set 
 of all controllers satisfying (\ref{eq:stab2})  with the 
Lyapunov matrix  turns out to be open, and  will be open as well. Finally, when , the set  is generic and
of full measure on  and, hence,  is non-negligible.
\qedp

{\em Proof of Proposition 1:} This  is a straightforward consequence of the fact that, when , we cannot have 
 a.e. on , since the 
observed closed-loop data are generated as  with 
and the control law  is supposed to be discerning. Hence,  whenever
, and  since by hypothesis .
\qedp

{\em Proof of Proposition 2:}
Recalling that the forced response  of the switching linear system  can be written
as

it is an easy matter to see that 

From the latter inequality, the bound in (\ref{eq:bound2}) can be readily obtained since, by hypothesis, the switching signal  contains only a finite number of discontinuity points in .
\qedp

{\em Proof of Proposition 3:} It follows from standard calculations by replacing  with  in the expression for the distance . \qedp

{\em Proof of Lemma 3:} 
In view of Proposition 3, we have that

with

Notice that   depends continuously on  and  and, in addition, 
if and only if  since the switching law is supposed to be discerning. Then,
the class  function  can be taken equal to .
As for the lower bound, notice that Proposition 3 implies also that
{\setlength\arraycolsep{1pt}
}Since  depends continuously on  and is equal to  if and only if
 (again thanks to the discernibility of the switching law), then a class  function  can be found
that satisfies the inequality  for any .
In particular,  can be constructed as in the proof of Theorem 2 of \cite{AlBaBaTAC} to which the reader is referred for additional details.
\qedp

{\em Proof of Theorem 3:}
Since  is -dense in , there exists at least one  such that . For such a
, one has

Since the estimate  is optimal in , one has also

Further, by exploiting the lower bound in Proposition 3, we can write

Combining the two latter inequalities, we obtain

which can be written as (\ref{eq:bound}), Proposition 2 and the fact that any class  function is invertible. 
\qedp

\bibliographystyle{IEEEtran} 
      
\bibliography{discerning}








\end{document}
