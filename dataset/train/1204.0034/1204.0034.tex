To analyze the time required to send  dofs from the source to the receiver with a non-systematic relay, a Markov chain model can be established. This approach is similar to that of \cite{lucani_fieldsize}, which characterizes transmission delays when RLNC is applied over a single link.

We define the state of the network by a 3-tuple , where  and  represent the number of dofs at the receiver and the relay, respectively, while  represents the number of dofs shared by these two nodes. Since there are in total  dofs,  is valid if and only if , , , and . Transmission starts in , and terminates in . There is an partial order of states  according to the validity condition of state transitions: ,  and .

\begin{figure}[t!]
\centering
\includegraphics[width=0.22\textwidth]{graphics/ijk}
\caption{A representation of the state . Given  degrees of freedom (dof) at the source, the receiver and the relay have  and  dofs, respectively; they also share  dofs.}\vspace*{-.3cm}
\label{fig:state}
\end{figure}

Let  be the probability of transiting from state  to state  when one packet is transmitted by the source. Let  be the expected amount of time to reach a terminating state from . Under the slotted transmission model,  can be defined recursively as

For terminating states, .


\subsection{Transition Probabilities}

For a non-systematic relay, Eq.~(\ref{eq:Tijk}) shows that to find the completion time, we only need to compute state transition probabilities , assuming that a single packet, coded or uncoded, is transmitted from the source. Depending on the values of , , and , we can divide the computation of  into 5 different cases as listed below. For each case, all possible state transitions are considered, as illustrated by Figure~\ref{fig:statetransitions}. In Figure \ref{fig:statetransitions}, an arrow represents a successful transmission of one packet from a node to another. 

We denote  to be the updates to , , and , where

 The transition probability  can be found by adding the probabilities of cases given in Figure~\ref{fig:statetransitions} corresponding to a given value of .

\begin{figure}[tbp]
\begin{center}
\includegraphics[width=0.49\textwidth]{graphics/figure2}
\caption{Possible transmission patterns and corresponding probabilities.}
\label{fig:statetransitions}\vspace*{-.4cm}
\end{center}
\end{figure}

\subsubsection{ and }\label{sec:case_a} since , the combined knowledge at the relay and the receiver is equal to . Therefore, , implying that any additional dof received at the receiver/relay contributes to the common knowledge  between them.
A dof new to the relay is already in the the knowledge space of the receiver, and vice versa. Thus, if both the relay and the receiver successfully receive a packet from the source,  increments by two. 

Since , the relay has at least one dof not known to the receiver; otherwise,  cannot be strictly less than both  and . In other words, a packet from the relay to the receiver is innovative to the receiver.



The probability  is given by:
	    \begin{itemize}
            \item : Cases 1, 2.

                In Case 1,  since the relay receives a  packet from the source;  since the receiver receives a coded packet from the relay.                 In Case 2, the relay and the receiver both receive the same coded packet from the source. This packet is innovative to both nodes.

            \item : Case 3. 
            
                since the relay receives a coded packet from the source.  since the receiver receives two packets, one from the source and one from the relay. Thus,  can increment by 3, since the single broadcast from the source increases  by 2. 

            \item : Case 4. 


            \item : Cases 5, 6.

            \item : Case 7.

	    \item : Case 8.
        \end{itemize}

\subsubsection{ and , then } the relay already has all dofs. Hence, even if the relay receives a packet from the source,  does not change. The relay can be considered as an additional source. As a result, any packet from the source or the relay is innovative to the receiver. Therefore,  and  for all cases.

The probability  is given by:
	    \begin{itemize}
            \item : Cases 1, 2, 5, 6.
            \item : Cases 3, 7.
            \item : Cases 4, 8.
	    \end{itemize}

\subsubsection{ and , then } the receiver has received all dofs. Therefore, transmission is completed and the state is absorbing, i.e. .

\subsubsection{, and  or } since , the relay has at least one dof which is not shared with the receiver. Therefore, a coded packet from the relay is innovative to the receiver. The key difference between this scenario and that of Section \ref{sec:case_a} is that the equality  does not necessarily hold here. The condition  implies that the combined knowledge at the receiver and the relay is less than . A coded packet from the source is innovative to both the relay and the receiver. If received successfully by both nodes,  increments by one.

The probability  is given by:
    \begin{itemize}
        \item : Cases 1, 2.

            In Case 1,  since the coded packet from the relay to the receiver is outside of the receiver's knowledge space. In Case 2,  since the source transmits the same dof to both the relay and the receiver. 

        \item : Case 3.
	
	    Unlike in Section~\ref{sec:case_a},  here since  and the same new dof is transmitted from the source to the relay and the receiver. 

        \item : Case 4.

             since the packet from the source to the relay is not in the knowledge space of receiver.

        \item : Case 5.

             since the packet from the source to the receiver is not in the knowledge space of the relay.

        \item : Case 6.
        \item : Case 7.

            Since the packet from the source to the receiver is not within the knowledge space of the relay, this packet does not count towards , resulting in . 
            
        \item : Case 8.
    \end{itemize}

\subsubsection{ and }
since , all dofs at relay are already known at the receiver. For  to increase, a new dof needs to be delivered from the source directly or indirectly to the receiver. Therefore, receiving a packet at the receiver from the relay does not increase  or  unless the source has successfully transmitted a new packet to the relay. Furthermore, if the receiver receives packets from both the source and the relay,  then .

Therefore,  is given by:

	    \begin{itemize}
            \item : Cases 1, 2, 3.
            \item : Case 4.
            \item : Cases 5, 7.
            \item : Cases 6, 8.
        \end{itemize}

	



\subsection{Mean Transmission Completion Time}

In the above analysis, we assume that the relay always codes packets in its queues. Since the system initiates in state , the mean completion time for a non-systematic relay network is given by

where  is defined recursively using Equation \eqref{eq:Tijk}. 