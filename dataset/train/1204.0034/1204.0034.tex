To analyze the time required to send $M$ dofs from the source to the receiver with a non-systematic relay, a Markov chain model can be established. This approach is similar to that of \cite{lucani_fieldsize}, which characterizes transmission delays when RLNC is applied over a single link.

We define the state of the network by a 3-tuple $(i,j,k)$, where $i$ and $j$ represent the number of dofs at the receiver and the relay, respectively, while $k$ represents the number of dofs shared by these two nodes. Since there are in total $M$ dofs, $(i,j,k)$ is valid if and only if $i+j-k \leq M$, $i\leq M$, $j\leq M$, and $ 0 \leq k \leq \min(i,j)$. Transmission starts in $(0,0,0)$, and terminates in $(M,j,k)$. There is an partial order of states $(i,j,k) \preceq (i',j',k')$ according to the validity condition of state transitions: $i\leq i'$, $j\leq j'$ and $k\leq k'$.

\begin{figure}[t!]
\centering
\includegraphics[width=0.22\textwidth]{graphics/ijk}
\caption{A representation of the state $(i,j,k)$. Given $M$ degrees of freedom (dof) at the source, the receiver and the relay have $i$ and $j$ dofs, respectively; they also share $k$ dofs.}\vspace*{-.3cm}
\label{fig:state}
\end{figure}

Let $P_{(i,j,k) \rightarrow (i',j',k')}$ be the probability of transiting from state $(i,j,k)$ to state $(i',j',k')$ when one packet is transmitted by the source. Let $T_{(i,j,k)}$ be the expected amount of time to reach a terminating state from $(i,j,k)$. Under the slotted transmission model, $T_{(i,j,k)}$ can be defined recursively as
\begin{align}
T_{(i,j,k)} & = 1 +\sum_{(i,j,k) \preceq (i',j',k')}P_{(i,j,k) \rightarrow (i',j',k')} T_{(i',j',k')}\,. \label{eq:Tijk}
\end{align}
For terminating states, $T_{(M,j,k)}=0$.


\subsection{Transition Probabilities}

For a non-systematic relay, Eq.~(\ref{eq:Tijk}) shows that to find the completion time, we only need to compute state transition probabilities $P_{(i,j,k) \rightarrow (i',j',k')}$, assuming that a single packet, coded or uncoded, is transmitted from the source. Depending on the values of $i$, $j$, and $k$, we can divide the computation of $P_{(i,j,k) \rightarrow (i',j',k')}$ into 5 different cases as listed below. For each case, all possible state transitions are considered, as illustrated by Figure~\ref{fig:statetransitions}. In Figure \ref{fig:statetransitions}, an arrow represents a successful transmission of one packet from a node to another. 

We denote $(\delta i,\delta j,\delta k)$ to be the updates to $i$, $j$, and $k$, where
\begin{align*}
i' &=\min(i+\delta i, M)\,, \quad j' = \min(j + \delta j, M)\,, \quad \text{ and}\\
k' &= \min(k+\delta k,M,i+ \delta j, k + \delta k)\,.
 \end{align*}
 The transition probability $P_{(i,j,k) \rightarrow (i',j',k')}$ can be found by adding the probabilities of cases given in Figure~\ref{fig:statetransitions} corresponding to a given value of $(\delta i,\delta j,\delta k)$.

\begin{figure}[tbp]
\begin{center}
\includegraphics[width=0.49\textwidth]{graphics/figure2}
\caption{Possible transmission patterns and corresponding probabilities.}
\label{fig:statetransitions}\vspace*{-.4cm}
\end{center}
\end{figure}

\subsubsection{$i+j-k=M$ and $k<\min(i,j)$}\label{sec:case_a} since $i+j-k=M$, the combined knowledge at the relay and the receiver is equal to $M$. Therefore, $\delta k = \delta i + \delta j$, implying that any additional dof received at the receiver/relay contributes to the common knowledge $k$ between them.
A dof new to the relay is already in the the knowledge space of the receiver, and vice versa. Thus, if both the relay and the receiver successfully receive a packet from the source, $k$ increments by two. 

Since $k < \min(i, j)$, the relay has at least one dof not known to the receiver; otherwise, $k$ cannot be strictly less than both $i$ and $j$. In other words, a packet from the relay to the receiver is innovative to the receiver.



The probability $P_{(i,j,k) \rightarrow (i',j' j,k')}$ is given by:
	    \begin{itemize}
            \item $(\delta i,\delta j,\delta k)=(1,1,2)$: Cases 1, 2.

                In Case 1, $\delta j = 1$ since the relay receives a  packet from the source; $\delta i = 1$ since the receiver receives a coded packet from the relay.                 In Case 2, the relay and the receiver both receive the same coded packet from the source. This packet is innovative to both nodes.

            \item $(\delta i,\delta j,\delta k)=(2,1,3)$: Case 3. 
            
               $\delta j = 1$ since the relay receives a coded packet from the source. $\delta i = 2$ since the receiver receives two packets, one from the source and one from the relay. Thus, $\delta k$ can increment by 3, since the single broadcast from the source increases $k$ by 2. 

            \item $(\delta i,\delta j,\delta k)=(0,1,1)$: Case 4. 


            \item $(\delta i,\delta j,\delta k)=(1,0,1)$: Cases 5, 6.

            \item $(\delta i,\delta j,\delta k)=(2,0,2)$: Case 7.

	    \item $(\delta i,\delta j,\delta k)=(0,0,0)$: Case 8.
        \end{itemize}

\subsubsection{$i+j-k=M$ and $k=i<M$, then $j=M$} the relay already has all dofs. Hence, even if the relay receives a packet from the source, $j$ does not change. The relay can be considered as an additional source. As a result, any packet from the source or the relay is innovative to the receiver. Therefore, $\delta j = 0$ and $\delta i = \delta k$ for all cases.

The probability $P_{(i,j,k) \rightarrow (i' i,j',k')}$ is given by:
	    \begin{itemize}
            \item $(\delta i,\delta j,\delta k)=(1,0,1)$: Cases 1, 2, 5, 6.
            \item $(\delta i,\delta j,\delta k)=(2,0,2)$: Cases 3, 7.
            \item $(\delta i,\delta j,\delta k)=(0,0,0)$: Cases 4, 8.
	    \end{itemize}

\subsubsection{$i+j-k=M$ and $k=j$, then $i=M$} the receiver has received all dofs. Therefore, transmission is completed and the state is absorbing, i.e. $P_{(i,j,k) \rightarrow (i, j, k)} = 1$.

\subsubsection{$i+j-k <M$, and $k < \min(i,j)$ or $k = i < j$} since $k<j$, the relay has at least one dof which is not shared with the receiver. Therefore, a coded packet from the relay is innovative to the receiver. The key difference between this scenario and that of Section \ref{sec:case_a} is that the equality $\delta i + \delta j = \delta k$ does not necessarily hold here. The condition $i + j - k< M$ implies that the combined knowledge at the receiver and the relay is less than $M$. A coded packet from the source is innovative to both the relay and the receiver. If received successfully by both nodes, $k$ increments by one.

The probability $P_{(i,j,k) \rightarrow (i',j',k')}$ is given by:
    \begin{itemize}
        \item $(\delta i,\delta j,\delta k)=(1,1,1)$: Cases 1, 2.

            In Case 1, $\delta k = 1$ since the coded packet from the relay to the receiver is outside of the receiver's knowledge space. In Case 2, $\delta i = \delta j =\delta k = 1$ since the source transmits the same dof to both the relay and the receiver. 

        \item $(\delta i,\delta j,\delta k)=(2,1,2)$: Case 3.
	
	    Unlike in Section~\ref{sec:case_a}, $\delta k=2$ here since $i+j-k <M$ and the same new dof is transmitted from the source to the relay and the receiver. 

        \item $(\delta i,\delta j,\delta k)=(0,1,0)$: Case 4.

            $\delta k = 0$ since the packet from the source to the relay is not in the knowledge space of receiver.

        \item $(\delta i,\delta j,\delta k)=(1,0,0)$: Case 5.

            $\delta k = 0$ since the packet from the source to the receiver is not in the knowledge space of the relay.

        \item $(\delta i,\delta j,\delta k)=(1,0,1)$: Case 6.
        \item $(\delta i,\delta j,\delta k)=(2,0,1)$: Case 7.

            Since the packet from the source to the receiver is not within the knowledge space of the relay, this packet does not count towards $k$, resulting in $\delta k = 1$. 
            
        \item $(\delta i,\delta j,\delta k)=(0,0,0)$: Case 8.
    \end{itemize}

\subsubsection{$i+j-k <M$ and $k=j\leq i$}
since $k = j$, all dofs at relay are already known at the receiver. For $i$ to increase, a new dof needs to be delivered from the source directly or indirectly to the receiver. Therefore, receiving a packet at the receiver from the relay does not increase $i$ or $k$ unless the source has successfully transmitted a new packet to the relay. Furthermore, if the receiver receives packets from both the source and the relay,  then $\delta i = 1$.

Therefore, $P_{(i,j,k) \rightarrow (i',j',k')}$ is given by:

	    \begin{itemize}
            \item $(\delta i,\delta j,\delta k)=(1,1,1)$: Cases 1, 2, 3.
            \item $(\delta i,\delta j,\delta k)=(0,1,0)$: Case 4.
            \item $(\delta i,\delta j,\delta k)=(1,0,0)$: Cases 5, 7.
            \item $(\delta i,\delta j,\delta k)=(0,0,0)$: Cases 6, 8.
        \end{itemize}

	



\subsection{Mean Transmission Completion Time}

In the above analysis, we assume that the relay always codes packets in its queues. Since the system initiates in state $(0,0,0)$, the mean completion time for a non-systematic relay network is given by
\begin{align}
T_{non-sys}=T_{(0,0,0)},\label{eq:TNS}
\end{align}
where $T_{(0,0,0)}$ is defined recursively using Equation \eqref{eq:Tijk}. 