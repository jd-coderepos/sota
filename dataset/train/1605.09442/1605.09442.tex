\section{Result 3: Axiomatization  of the Language }
\label{sec:soundness}

In this section, we present a consistent
axiomatization  for the functions and predicates of the
language  (as presented earlier in Section~\ref{sec:prelim}).

\subsection{The Axiomatization }
\label{sec:axioms}

We introduce an axiom system  for the language .
For the sake of readability, we choose not to specify the
sorts of various terms if they are clear from context.

\subsubsection{Axioms of Linear Arithmetic over the Natural Numbers}

The following axioms follow from the ones for Presburger
arithmetic. Note that both Presburger arithmetic and the linear
arithmetic as part of  include only the addition symbol, and
do not have full multiplication. (Multiplication by constants is
simply a short-hand for repeated addition up to a known constant
bound.)

\begin{enumerate}

\item 
  
\item 

\item  

\item  

\item  

\item  

\item 

\item 

\item 



\item 

\item 

\end{enumerate}

\subsubsection{Axioms of Equality for Strings and Natural Numbers}

It is assumed that the equality predicate for both string and numeric
sorts is reflexive, symmetric, and transitive. In addition, we have
the following axiom recursively defined over string terms. Below we
present the axiom for string constants over the alphabet .

\begin{enumerate}[resume]

\item 

\end{enumerate}

\subsubsection{Axioms of Concatenation}

Concatenation is associative, but not commutative.

\begin{enumerate}[resume]

\item 

\item 

\end{enumerate}

\subsubsection{Axioms of the  Function}

\begin{enumerate}[resume]

\item 

\item 

\item 

\item 

\end{enumerate}
  
\subsubsection{Axioms of }

The axioms for the  predicate essentially allow us to define a
natural mapping between natural numbers, represented in binary, and
strings over .

\begin{enumerate}[resume]

\item 

\item 

\item 

\item 
  
\item 

\item 



\item , where  and  are the binary
  digits of  and  respectively. (This describes distribution of  over a concatenation.)



\item 

\item 


\end{enumerate}

\subsection{Relationship between  and }

We refer to the set of sentences logically entailed by the axiom
system  as the theory . Note that this set
contains sentences with arbitrary quantifiers in them. We assume that
sentences are always written in prenex normal form.

The set  is a set of quantifier-free -formulas. As discussed
before, when we use the term ``quantifier-free'' formulas, we mean
that each free variable is implicitly existentially quantified and
there are no other explicit quantifiers in the formula. When the
formulas in  are existentially quantified, we get the same set
of sentences implied by  that have a single set of existential
quantifiers in prenex normal form. We also call this the existential
fragment of .

\subsection{Consistency of }

\begin{Theorem}
  The axiom system  presented in Section~\ref{sec:axioms} is consistent.
\end{Theorem}

\begin{proof}
  It is well known that a theory or axiom system is consistent if it
  has a model~\cite{HodgesModelTheory}.  We prove consistency by
  showing that the structure established in
  Section~\ref{sec:canonicalmodel} is in fact a model of . The
  remainder of the proof is structured in sections corresponding to
  those in the description of .

  \begin{enumerate}
    

  \item \textbf{Axioms of arithmetic over natural numbers:} These are
   standard axioms for natural number arithmetic.  Since we choose
    to model numeric terms, it follows that these axioms
   are true over the natural numbers.


 \item \textbf{Axioms of equality for strings and natural numbers:}
   This axiom states that if two strings  and  are equal, then
    and  have the same length, in addition to the standard
   axioms of equality. Our model of string terms states that two
   strings are equal if they have the same characters appearing in the
   same order, and that the length of a string is the natural number
   of characters in that string.  It follows that if two strings are
   equal, then they have the same characters, and therefore have the
   same length.


 \item \textbf{Axioms of concatenation:} The first axiom states that
   concatenating any string with the empty string, on either side,
   produces a result equal to the original string.  Our model
   represents the result of concatenating  and  as a string
   having all of 's characters (in the same order) followed by all
   of 's characters (also in the same order). If one of  or 
   is empty, it follows that the resulting string has the same
   characters and in the same order as the other string, and therefore
   the two are equal.

   The second axiom states that string concatenation is associative.
   Suppose strings  are composed of characters , ,  respectively. Then by
   definition of concatenation in our model, we have:

  

  Therefore the axiom holds in this model.


\item \textbf{Axioms of the Length Function:} The first axiom states
  that the only string having length 0 is the empty string ,
  which follows trivially from the definition of the set of string
  constants .

  The second axiom states that the length of the concatenation of 
  and  is equal to the sum of the lengths of  and  taken
  separately.  Our model represents the result of concatenating 
  and  as a string having all of 's characters (in the same
  order) followed by all of 's characters (also in the same
  order). Since characters are conserved by this process, it follows
  that the resulting string has length equal to the sum of the lengths
  of  and .

  The third axiom states that all single-character strings have length
  1, which holds trivially.


\item \textbf{Axioms of  string-numeric conversion predicate:}
  The first four axioms state some basic properties of string-number
  conversion:  is not the binary representation of any
  number, ``0'' is the binary representation of 0, ``1'' is the binary
  representation of 1, and single-character strings that are not ``0''
  or ``1'' are not the binary representation of any number. These
  axioms are true by inspection.

  The fifth and sixth axioms show that leading zeroes can be added to and removed from
  a string without changing its value. 
   We can show that this is true in our model by demonstrating that if  is a binary string and  is a string of all zeroes,
   the binary expansions of  and , denoted  and  respectively, both represent the same natural number:
  
   
  
  Hence adding or deleting leading zeroes has no effect on what number is represented by a given binary string, and so these axioms hold.
  
  The seventh axiom holds if we assume that all numbers are written in binary;
  concatenating the binary digits of two numbers is equivalent to
  concatenating the string representations of those numbers.
  


  The eighth axiom illustrates how to perform string-number conversion
  on an addition term .  It suffices to show
  that :

  

  \end{enumerate}
  
This completes the proof.
\end{proof}

\section{Result 4: Incompleteness of the Theory }
\label{sec:incompleteness}

We first state a number of useful definitions and theorems related to
completeness of first-order theories from the standard model theory
literature~\cite{HodgesModelTheory}.

\begin{Definition}
A first-order theory  in language  is \textbf{complete} if for
all -formulas , exactly one of  and  is a
consequence of .
\end{Definition}

\begin{Definition}
Two models  of a first-order theory are \textbf{elementarily
  equivalent} if for all first order -sentences , .
\end{Definition}

\begin{Theorem}
\label{thm:completeIffEquivalent}
A first-order theory  is complete iff all of its models
are elementarily equivalent~\cite{HodgesModelTheory}.
\end{Theorem}

\noindent{We are now in a position to prove the following result.}

\begin{Theorem}
 is incomplete.
\end{Theorem}

\begin{proof}
  Consider two models  of the theory , defined as
  follows:  is the canonical model given in
  Section~\ref{sec:canonicalmodel}, and  is a restricted version of
   where the only string constants that are allowed
  are nonempty string constants with no leading zeroes. (In other
  words, the only string constant in  that starts with `0' is
  ``0''.) It is easy to see that both of these are models of
  .

  Now consider the first-order sentence  which states ``the
   predicate describes a bijection between strings and natural
  numbers''.
\footnote{Note that as long as the alphabet  is finite and
  string constants are concatenations of a finite number of
  characters, in general there exists a bijection between strings and
  natural numbers. This follows from the fact that the set 
  of strings is countably infinite. The argument made in the proof
  above deals with a very particular bijection as defined by
  .}

We state this sentence  formally as follows:



  It follows that due to the restrictions on string constants imposed
  in ,  clearly defines a bijection between strings and
  natural numbers, where each integer is mapped to the unique string
  that is its minimal binary representation, and so  is true in the
  model .  However, in the model ,  does not define a
  bijection, as by counterexample,  and  are both true.  Therefore  is false in the model .
  
  From this we conclude that  is able to distinguish between 
  and , and hence  is not elementarily equivalent to ; by
  Theorem~\ref{thm:completeIffEquivalent},  is
  incomplete.
\end{proof}

