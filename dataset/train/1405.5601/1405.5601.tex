
\documentclass[submission,copyright,creativecommons]{eptcs}
\providecommand{\event}{AFL 2014}
\usepackage{amssymb}
\usepackage{breakurl}
\usepackage[utf8]{inputenc}

\def\fpartial{\stackrel{\circ}{\longrightarrow}}
\newcommand{\N}{{\mathbb N}}
\newcommand{\Z}{{\mathbb Z}}
\newcommand{\R}{{\mathbb R}}
\newcommand{\mod}{{\rm \,mod\,}}
\newtheorem{fact}{Fact}
\newtheorem{definition}{Definition}
\newtheorem{lemma}{Lemma}
\newtheorem{theorem}{Theorem}

\newenvironment{proof}[1][Proof]{\begin{trivlist}
\item[\hskip \labelsep {\bfseries #1}]}{\end{trivlist}}



\usepackage{url}


\title{Simplifying Nondeterministic Finite Cover Automata}

\author{Cezar C\^ampeanu
\institute{Department of Computer Science and Information Technology,\\
The University of Prince Edward Island, Canada}
\email{\quad ccampeanu@upei.ca}
}
\def\titlerunning{Simplifying NFCA}
\def\authorrunning{Cezar C\^ampeanu}
\begin{document}
\sloppy

\maketitle
\begin{abstract}
The concept of Deterministic Finite Cover Automata (DFCA) was introduced at WIA '98, 
as a more compact representation than Deterministic Finite Automata (DFA) for finite languages.
In some cases representing a finite language, Nondeterministic Finite Automata (NFA) may 
significantly reduce  the number of states used. 
The combined power of the succinctness of the representation of finite languages using 
both cover languages and non-determinism has been suggested, but never  systematically studied. 
In the present paper, for  nondeterministic finite cover automata (NFCA) and -nondeterministic 
finite cover automaton (-NFCA), we show that minimization can be as hard as minimizing NFAs for regular languages, 
even in the case of NFCAs using unary alphabets. Moreover, we show how we can adapt the 
methods used to reduce, or minimize the size of NFAs/DFCAs/-DFCAs, for simplifying NFCAs/-NFCAs.
\end{abstract}
\section{Introduction}

The race to find more compact representation for finite languages was started in 1959, 
when Michael O. Rabin and Dana Scott introduced the notion of Nondeterministic Finite Automata, 
and showed that the equivalent Deterministic Finite Automaton can be, 
in terms of number of states,  exponential larger than the NFA.
Since, it was proved in \cite{moore} that  we can obtain a polynomial algorithm for minimizing
DFAs, and in \cite{hopcroft} was proved that an  algorithm exists.
In the meantime, several heuristic approaches have been proposed to reduce 
the size of NFAs \cite{heuristic,ilieYunfa}, but it was proved by 
Jiang and  Ravikumar \cite{ravikumar} that  NFA minimization problems are
hard; even in case of regular languages over a one letter alphabet, 
the minimization is NP-complete \cite{GruberHolzerNFAHard,ravikumar}.

On the other hand, in case of finite languages, we can obtain minimizing algorithms \cite{maslov,revuz} 
that are in the order of , where  is the number of states of the original DFA.
In \cite{gapIJFCS,CoverAutomata,KornerGoeman} it has been shown that 
using Deterministic Finite Cover Automata to represent finite languages,
we have minimization algorithms as efficient as the best known algorithm for minimizing DFAs 
for regular languages. 

The study of the state complexity of operations on regular languages was initiated 
by Maslov in 1970 \cite{maslov,maslov1}, but has not become 
a subject of systematic study until 1992 \cite{KaiShengsc92}.
The special case of state complexity of operations 
on finite languages was studied in \cite{finiteop}.

Nondeterministic state complexity of regular languages 
was also subject of interest, for example in 
\cite{holzerKutribNFA,holzerKutribUnary,holzerKutribLata09,holzerKutribNFA09}.
To find lower bounds for the nondeterministic 
state complexity of regular languages, the fooling set technique,
or the extended fooling set technique may 
be used \cite{birget,Shallit,GruberHolzerNFAHard}.

In this paper we show that NFCA state complexity 
for a finite language  can be exponentially lower 
than NFA or DFCA state complexity of the same language.
We modify the fooling set technique for cover automata, 
to help us prove lower bounds for NFCA state complexity
in section \ref{slowerbounds}. We also show that the (extended) fooling set technique 
is not optimal, as we have minimal NFCAs with arbitrary number of states, and the largest fooling set 
has constant size.
In section \ref{shard} we show that minimizing NFCAs is hard, and in section~\ref{sheuristic}
we show that heuristic approaches for minimizing DFAs or NFAs need a special treatment
 when applied to NFCAs, as many results valid for the DFCAs are no longer true for NFCAs.
In section \ref{openproblems}, we formulate a few open problems and future research directions.

\section{Notations and definitions}

The number of elements of a set  is denoted by . 
In case  is an alphabet, i.e, finite non-empty set, the free monoid generated 
by  is , and it is the set of all words over .
The length of a word , , , 
, is .
The set of words of length equal to  is , the set of words of length less than or equal 
to  is denoted by . 
In a similar fashion, we define , , or .
A finite automaton is a structure  , where
 is a finite non-empty set called the set of states,  is an alphabet, ,  
 is the set of final states, and  is the transition function.
For delta, we distinguish the following cases:
\begin{itemize}
 \item if , the automaton is deterministic; 
in case  is always defined, the automaton is complete, otherwise it is incomplete;
 \item if , the automaton is non-deterministic.
\end{itemize}
The language accepted by an automaton is defined by:
, where
 is defined as follows:


Of course,  in case the automaton is deterministic, and 
, in case the automaton is non-deterministic.
\begin{definition}
Let  be a finite language, and  be the length of the longest word  in , i.e.,
\footnote{ We use the convention that .}.
If  is a finite language,  is a cover language for  if .

A cover automaton for a finite language  is an automaton that
recognizes a cover language, , for .
An -NFCA  is a cover automaton for the language .
\end{definition}

One could plainly see  that any automaton that recognizes  is also a cover automaton.

The level of a state  in a cover automaton  is
the length of the shortest word that can reach the state , i.e., 
.

Let us denote by  the smallest word , according to quasi-lexicographical order, such that
, see \cite{CoverAutomata} for a similar definition in case of DFCA.
Obviously,  .

For a regular language ,  denotes the Myhil-Nerode equivalence of words \cite{hopcroft79}.

The similarity relation induced by a finite language  is defined as follows\cite{CoverAutomata}:
, if
for all ,  iff .
A dissimilar sequence for a finite language  
is a sequence  such that , for all 
and
 .


Now, we need to define the similarity for states in an NFCA, 
since it was the main notion used for DFCA minimization.

\begin{definition}
In an NFCA , two states  are similar, written ,  if
 iff , 
for all .
\end{definition}

In case the NFCA  is understood, we may omit the subscript , i.e., we 
write  instead of , also we can write  instead of .

We consider only non-trivial NFCAs for , i.e., 
 NFCAs such that  for all states . In case
,  can be eliminated, and the resulting NFA is still a NFCA for .
In this case, if , then  either , or , because .


Deterministic state complexity of a regular language  is defined as the number of states of the minimal 
deterministic automaton recognizing , and it is denoted by :
 
Non-deterministic state complexity of a regular language  is defined as the number of states of the minimal 
non-deterministic automaton recognizing , and it is denoted by :
 
For finite  languages , we can also define deterministic cover state complexity 
and non-deterministic cover state complexity :


 

Obviously,  , but also 
.
Thus, non-deterministic finite cover automata can be considered to be one of 
the most compact representation of finite languages.

\section{Lower-bounds and Compression Ratio for NFCAs}
\label{slowerbounds}

We start this section analyzing few examples where nondeterminism, 
or the use of cover language, reduce the state complexity.
Let us first analyze the type of languages where non-determinism, combined with cover properties,
 reduce significantly the state complexity.

We choose the language 
 , where .
In Figure~\ref{f1} we can see an NFA recognizing  with  states.
We must note that the longest word in the language 
has  letters.
Let us analyze if the automaton in Figure~\ref{f1}  is minimal.
The fooling set technique, introduced in \cite{Chrobak} and \cite{gramlich}, and used to prove the lower-bound 
for state complexity of NFAs,  is  stated in \cite{birget,Chrobak} as follows:

\begin{lemma}
\label{lfst1}
Let  be a regular language, and suppose there  exists a set of pairs 
, with the following properties:
\begin{enumerate}
 \item 
  \label{sfst}
  If , for  and , for all 
, , then . The set  is called {\em a fooling set} for .
 \item 
 \label{extfst}
If , for  and for , if , implies  
either  
or , 
 then . The set  is called {\em an extended fooling set} for .
\end{enumerate}
\end{lemma}

Now consider the following set of pairs of words:
.

For , we have  that
\begin{enumerate}
 \item , or
 \item .
\end{enumerate}

Let us examine for each ,  if  the words  
 and 
 are also in . We have the following possibilities:
\begin{enumerate}
  \item Case I
  \begin{enumerate}
   \item , and
   \item .
  \end{enumerate}
 \item Case II
  \begin{enumerate}
   \item , if , but 
   \item , if  (because ).
  \end{enumerate}
 \item Case III
  \begin{enumerate}
   \item  (because ), or
   \item  if , 
         or 
          .
  \end{enumerate}
\end{enumerate}
From the statement \ref{extfst}. of Lemma~\ref{lfst1}, it follows that the NFA is minimal. 
We must note the following:
\begin{enumerate}
 \item we cannot use the weak form \ref{sfst} to prove the lower-bound;
 \item when proving the lower-bound, we concatenate words to obtain a word 
of length greater than the maximum length of the words in the language, and 
that's why  is rejected.
Since in case of cover automata such words will be automatically rejected, there is no doubt that any fooling set type 
technique we may use to prove the lower-bound for NFCAs must consider the length, and 
we should ignore the cases when the length exceeds the maximal one.
\end{enumerate}

Hence, the fooling set technique introduced in \cite{Chrobak} and \cite{gramlich}, and used to prove the lower-bound 
for state complexity of NFAs, can be modified to prove 
a lower-bound for minimal NFCAs, and it can be formulated  for cover languages
as an adaptation of Theorem~1 in \cite{GruberHolzerNFAHard}.
\begin{lemma}
\label{lcfst}
Let  be a finite language such that the longest word in  has the length , 
and suppose there  exists a set of pairs , with the following properties:
\begin{enumerate}
 \item 
  \label{scfst}
  If  for  and 
for , , and , we have that 
, then . 

The set  is called {\em a fooling set} for .
 \item 
 \label{cfstext}
If , for  and for , if , implies 
either  and , 
or  and  for all, 
 then . 

The set  is called {\em an extended fooling set} for .
\end{enumerate}
\end{lemma}
\begin{proof}
Assume there exists an NFCA , with  states accepting .
For each , , , therefore we must have a state  
and .
In other words, there exists a state  and . 
\begin{enumerate}
 \item We claim   for all .
If ,
then , and because , it follows that 
  , a contradiction.
\item We consider the function  
defined by ,  as above.
We claim that  is injective.
If , then
,
also 
.
Because , we also have that ,
and because , it follows that , a contradiction.
If , using the same reasoning, will follow that .
In both cases we have a contradiction, thus  must have at least  elements.
\end{enumerate}
\end{proof}

For the example above, we discover that we cannot have more than one pair 
of the form , thus, applying the extended fooling set technique for NFCAs,
 the minimum number of states in a minimal 
NFCA is at least . 
This proves that the NFCA presented in Figure~\ref{f2} is minimal.


It is easy to check that any two distinct words , , 
are not similar with respect to . It follows that for the language presented in 
Figure~\ref{f1}, . One can also verify that for two distinct words
 and ,
if  , , they are distinguishable;
also, in case 
, the word  will distinguish between all the words for which
 or , thus the number of states in the minimal DFA is even
 larger than  .
In case   and , the minimal DFCA is presented in Figure~\ref{f3}. A simple 
computation shows us that the corresponding minimal DFA has 15 states.

\begin{figure}[h]
\begin{center}
\begin{picture}(200,70)
\put(10,20){\oval(15,15)}
\put(7,17){}

\put(10,50){\oval(18,15)}
\put(3,47){}
\put(44,50){\vector(-1,0){24}}
\put(25,55){}
\put(10,43){\vector(0,-1){15}}
\put(-10,30){}
\put(14,43){\vector(2,-1){36}}
\put(35,35){}

\put(18,20){\vector(1,0){30}}
\put(25,25){}
\put(56,20){\oval(15,15)}
\put(54,17){}    

\put(54,50){\oval(18,15)}
\put(46,47){}    
\put(94,50){\vector(-1,0){30}}
\put(75,55){}
\put(54,43){\vector(0,-1){15}}
\put(56,35){}

\put(164,50){\oval(22,15)}
\put(156,47){}    
\put(185,50){\vector(-1,0){10}}
\put(152,50){\vector(-1,0){30}}
\put(125,55){}
\put(152,50){\vector(-4,-1){92}}
\put(90,38){}


\put(64,20){\vector(1,0){30}}
\put(76,25){}
\put(102,20){\oval(15,15)} 
\put(98,17){} 
\put(109,20){\vector(1,0){20}}
\put(112,25){}
\put(145,20){\vector(1,0){20}}
\put(145,25){}
\put(180,20){\oval(30,15)}
\put(170,17){}
\put(195,20){\vector(1,0){25}}
\put(200,25){}
\put(235,20){\oval(25,15)}
\put(224,17){}    
\put(235,20){\oval(30,18)}  
\end{picture}
\end{center}
\caption{An NFA with  states for the language .\newline
}
\label{f1}
\end{figure}

\begin{figure}[h]
\begin{center}
\begin{picture}(200,70)
\put(-5.5,30){\vector(1,0){5}}
\put(7,30){\oval(15,15)}
\put(5,27){0}
\put(7,49){\oval(15,15)[t]}
\put(-0.5,49){\line(1,-3){4}}
\put(14.5,49){\vector(-1,-3){4}}
\put(12,58){}
\put(15,30){\vector(1,0){30}}
\put(25,35){}
\put(52,30){\oval(15,15)}
\put(50,27){1}    
\put(60,30){\vector(1,0){30}}
\put(65,35){}
\put(97,30){\oval(15,15)} 
\put(95,27){2} 
\put(105,30){\vector(1,0){20}}
\put(110,35){}
\put(145,30){\vector(1,0){20}}
\put(145,35){}
\put(180,30){\oval(30,15)}
\put(170,27){}
\put(195,30){\vector(1,0){25}}
\put(200,35){}
\put(235,30){\oval(29,15)}
\put(225,27){}    
\put(235,30){\oval(32,18)}  
\end{picture}
\end{center}
\caption{An NFCA with  states for the language , that is the same as 
the one in Figure~\ref{f1}. In case  and , the language is the same as the one described in Figure~\ref{f3}.\newline
An equivalent minimal NFA has  states.
}
\label{f2}
\end{figure}


\begin{figure}[h]
\begin{center}
\begin{picture}(200,100)
\put(10,30){\oval(15,15)}
\put(7,27){0}
\put(18,30){\vector(1,0){15}}
\put(25,35){}

\put(10,13){\oval(15,15)[b]}
\put(17,13){\vector(-1,2){5}}
\put(3,13){\line(1,2){5}}
\put(12,0){}


\put(40,30){\oval(15,15)}
\put(37,27){1}    

\put(46,35){\vector(1,2){9}}
\put(40,45){}


\put(60,60){\oval(15,15)} 
\put(57,57){2} 

\put(68,60){\vector(1,0){15}}
\put(75,65){}


\put(75,48){\oval(30,10)[b]}
\put(60,48){\vector(0,1){5}}
\put(90,48){\line(0,1){5}}
\put(68,46){}

\put(66,66){\vector(1,1){18}}
\put(70,80){}

\put(90,90){\oval(15,15)} 
\put(87,87){4} 
\put(90,90){\oval(15,15)} 

\put(83,90){\line(-1,0){50}}
\put(50,50){\oval(80,80)[tl]}
\put(10,50){\vector(0,-1){12}}
\put(15,70){}

\put(90,60){\oval(15,15)}
\put(90,60){\oval(18,18)}  
\put(87,57){5} 


\put(48,30){\vector(1,0){24}}
\put(55,35){}

\put(80,30){\oval(15,15)}
\put(77,27){}

\put(88,30){\vector(1,0){24}}
\put(95,35){}


\put(120,30){\oval(15,15)}
\put(117,27){}    
\put(120,30){\oval(18,18)}  

\put(120,40){\vector(0,1){10}}
\put(125,40){}

\put(120,60){\oval(15,15)}
\put(117,57){}    
\put(120,60){\oval(18,18)}  

\put(115,68){\vector(-1,1){18}}
\put(110,75){}

\end{picture}
\end{center}
\caption{A minimal DFCA with  states for the language , .\newline
The equivalent minimal DFA has  states.}
\label{f3}
\end{figure}


This language example shows that NFCAs may be a much more compact representation 
for finite languages than NFAs, or even DFCAs, and motivates the study of such objects.
In terms of compression, clearly the number of states in the NFCA 
is exponentially smaller than the number of states in the DFA, and in some 
cases, even exponentially smaller than in an NFA.

\begin{figure}
 \begin{center}
\begin{picture}(200,50)
 \put(2,40){\vector(1,0){5}}
 \put(15,40){\oval(15,15)}
 \put(13,37){}
 \put(23,40){\vector(1,0){24}}
 \put(30,45){}
 \put(55,40){\oval(15,15)}
 \put(53,37){}
 \put(55,28){\vector(0,1){5}}
 \put(63,40){\vector(1,0){20}}
 \put(70,45){}
 \put(92,40){\oval(15,15)}
 \put(92,40){\oval(18,18)}
 \put(90,37){}
 \put(101,40){\vector(1,0){20}}
 \put(105,45){}
 \put(135,40){\vector(1,0){12}}
 \put(140,45){} 
 \put(161,40){\oval(25,15)}
 \put(161,40){\oval(28,18)}
 \put(150,37){}
 \put(175,40){\vector(1,0){25}}
 \put(180,45){}

 \put(210,40){\oval(20,15)}
\put(210,40){\oval(22,18)}
 \put(207,37){}
 \put(133,30){\oval(156,20)[b]}
 \put(133,22){}
\put(211,27){\line(0,1){5}}
\end{picture}
\end{center}
\caption{An NFA/NFCA  for .} 
\label{f4}
\end{figure}

Let's set , ,
and choose the following language:


In Figure~\ref{f4},  the NFCA  accepts the language , therefore
 .
It is known \cite{Chrobak,holzerKutribUnary,pighizzini} 
that the automaton   is minimal NFA for , if  is a prime number.
However, this may not be a minimal NFCA, as illustrated by the example in Figure~\ref{f5}, where  
is not minimal for , even if it is minimal NFA for the cover language.

\begin{figure}
\begin{center}
\begin{picture}(150,75)
 \put(4,20){\vector(1,0){5}}
 \put(15,20){\oval(12,12)}
 \put(13,17){}
 \put(22,20){\vector(1,0){25}}
 \put(30,25){}
 \put(20,23){\vector(1,1){28}}
 \put(30,40){}
 \put(54,52){\oval(12,12)}
 \put(52,50){}
 \put(60,52){\vector(1,0){25}}
 \put(65,55){}
 \put(92,52){\oval(12,12)}
 \put(92,52){\oval(15,15)}
 \put(90,50){}
 \put(100,52){\vector(1,0){25}}
 \put(110,55){}
 \put(132,52){\oval(12,12)}
 \put(132,52){\oval(15,15)}
 \put(130,50){}
\put(92,60){\oval(80,15)[t]}
 \put(90,70){}
\put(132,58){\line(0,1){5}}
\put(52,63){\vector(0,-1){5}}

 \put(55,20){\oval(12,12)}
 \put(53,17){}
 \put(55,8){\vector(0,1){5}}
 \put(61,20){\vector(1,0){25}}
 \put(70,25){}
 \put(92,20){\oval(12,12)}
 \put(92,20){\oval(15,15)}
 \put(90,17){}
 
  
 \put(75,10){\oval(40,15)[b]}
 \put(75,5){}
\put(95,9){\line(0,1){5}}
\end{picture}
\hspace*{2cm}
\begin{picture}(150,75)(20,0)
 \put(4,20){\vector(1,0){5}}
 \put(15,20){\oval(12,12)}
 \put(13,17){}
 \put(22,20){\vector(1,0){20}}
 \put(25,25){}
 \put(48,20){\oval(12,12)}
 \put(45,17){}
 \put(48,8){\vector(0,1){5}}
 \put(54,20){\vector(1,0){20}}
 \put(60,25){}
 \put(82,20){\oval(12,12)}
 \put(82,20){\oval(15,15)}
 \put(80,17){}
 \put(88,20){\vector(1,0){20}}
 \put(105,25){}
 \put(125,20){\vector(1,0){14}}
 \put(130,25){} 
 \put(145,20){\oval(12,12)}
 \put(145,20){\oval(15,15)}
 \put(142,17){}


\put(97,10){\oval(98,15)[b]}
 \put(133,5){}
\put(146,9){\line(0,1){5}}
\end{picture}
\end{center}
\caption{A minimal NFCA for , left, and a  minimal NFA for a cover language, right.} 
\label{f5}
\end{figure}


We apply the extended fooling set technique for the language . 
Because the alphabet is unary, all the  words in an extended fooling
 set   are powers of : 
, 
for some .
A simple computation shows that if , and  and   
for some , then  and 
, for any .
It follows that . 

Let  be an NFA accepting , and we can consider that it is already in Chrobak normal form, 
as it is ultimately periodic.
Thus, for each , , where  are primes, and each cycle has  states, 
. 
Now, let us prove that  is minimal for some language , .


Assume there exists an automaton  with  states,  such that
 .
It follows that the language  will contain words 
with a length  for , and all . For  large enough, one of these words will be 
of length multiple of  plus , therefore, for large enough , i.e., greater than some ,
. Thus, the number of states in  is at least .
 is also a minimal NFCA for languages , , hence it follows that Theorem~7 in 
\cite{GruberHolzerNFAHard} is also valid for cover automata:

\begin{theorem}
 There is a sequence of languages  such that the nondeterministic cover complexity
of  is at least , but the extended fooling set for  is of size ,
 where  is a constant.
\end{theorem}

Now, we are ready to check how hard is to obtain this minimal representation
of a finite language.


\section{Minimization Complexity}
\label{shard}

In this section we show that minimizing NFCAs is hard, and we'll show it with the exact same arguments from \cite{gruberholzerunary},
used to prove that minimizing NFAs is hard.
We will describe the construction from \cite{gramlich,gruberholzerunary}, showing 
that we can also use it with only a minor addition for cover NFAs.
To keep the paper self contained, we include a complete description,  
and emphasize the changes required for the cover automata, rather 
than just presenting the differences. 

Let us consider a logical formula , 
 in the conjunctive normal form, i.e., ,
where each clause , ,
is defined using variables , , and each ,  are either  or .
Let  be distinct prime numbers such that
. We set , and 
using Chinese Remainder Theorem \cite{china}\footnote{Theorem I.3.3, page 21},
it follows that the function  is bijective.
We need to define a language  and a natural number  such that
, if and only if   is unsatisfiable, therefore, the finite language 
 has  as a cover language.
We can construct an automaton  in  in a similar fashion as we build automata 
that recognizes the language
.
Let  be an automaton recognizing . 
It is clear that it can be constructed in  time.
For each clause  such that  is an assignment of its 
variables  for which 
 is not satisfied, we define .
An automaton  accepting   can be constructed 
in  time\footnote{Using Cartesian product construction, for example.}. 
Setting , it follows that 
 iff  is satisfiable. Moreover,  is a cyclic language with period at most ,
 thus setting , we have that 
 has 
as a cover language iff
  is satisfiable. 
Since according to \cite{Primtest}, primality test can be done in polynomial time, 
 we can find the first  prime numbers  in polynomial time, 
which means that our NFA construction can also be done in polynomial time.
If  is unsatisfiable, then , if  is satisfiable,
then the  minimal period of  is , according to \cite{Chrobak,gramlich}, 
 and the minimal number of states 
in an NFA is at least equal to the largest prime number dividing its period, which is . 
Using  the same argument as in \cite{gruberholzerunary}, it 
follows that the existence of a polynomial algorithm to decide if   implies that 
, therefore we can solve  in polynomial time, i.e., .
Consequently, we proved that
\begin{theorem}
 Minimizing either NFCAs or -NFCAs is at least NP-hard. 
\end{theorem}


\section{Reducing the Number of States of NFCAs}
\label{sheuristic}

Assume the DFA  is minimal for , and the minimal NFA is
, 
where , , if  and  if .
In other words, the minimal NFA is the same as the DFA, except that we delete the dead state.
We may have a minimal DFCA as , and  as a minimal NFA, but not as a minimal NFCA,  as illustrated by
 and .

We need to investigate if classical methods to reduce the number of states in an NFA or DFA/DFCA
can also be applied to NFCAs, thus, we first analyze the state  merging technique.
For NFAs,  we distinguish between  two main ways of merging states: (1) a weak method,
where two states are merged by simply collapsing one into the other, and consolidate
all their input and output transitions, and (2), a strong method, where one state is
merged into another one by redirecting its input transitions toward the other state,
and completely deleting it and all its output transitions. 
The same methods are considered for NFCAs.

\begin{definition}
 Let  be a NFCA for the finite language . 
\begin{enumerate}
 \item We say that the state  is {\em weakly mergible} in state 
if the automaton
, where
, , and 

 is also a NFCA for .
In this case we write .
\item We say that the state  is {\em strongly mergible} in state , 
if the automaton
, where
, , and 

 is also a NFCA for .
In this case we write .
\end{enumerate}
\end{definition}
In case , 

and in case , 
, where for 
 and
.


For the case of DFCAs, if  is a DFCA for  and two states are similar with respect
 to the similarity relation induced by , then all the words reaching these states are similar.
Moreover, if two words of minimal length reach two distinct states in a DFCA, and the words are similar
with respect to , then the states in the DFCA must be similar with respect to the similarity 
relation induced by . These results are used  for DFCA minimization, and we need to verify 
if they can be used in case of NFCAs. In the following lemmata we show that  
 the corresponding results are no longer true.

\begin{lemma}
 Let  be a NFCA for the finite language .
It is possible that , but  and  are not mergible.
\end{lemma}
\begin{proof}
 For the automaton in Figure~\ref{f5}, left, , but the states  and  are not mergible,
as the resulting automaton would not reject .
\end{proof}


\begin{lemma}
 Let  be a NFCA for the finite language , and , .
It is possible to have , , ,
 , and  .
\end{lemma}
\begin{proof}
Consider the language , where  is depicted in Figure~\ref{f6}.
\begin{figure}
\begin{center}
\begin{picture}(100,100)
 \put(10,50){\oval(15,15)}
 \put(7,47){}
 \put(16,50){\vector(1,1){25}}
 \put(25,65){}
 \put(16,50){\vector(1,0){25}}
 \put(30,55){}
 \put(16,50){\vector(1,-1){25}}
 \put(25,30){}
 \put(49,77){\oval(15,15)}
 \put(47,74){}
 \put(87,77){\oval(15,15)}
 \put(87,77){\oval(18,18)}
 \put(84,74){}
 \put(67,88){\oval(40,20)[t]}
 \put(47,90){\vector(0,-1){5}}
 \put(87,90){\line(0,-1){5}}
 \put(87,94){}

 \put(53,75){\vector(1,0){25}}
 \put(60,80){}
 \put(47,50){\oval(15,15)}
 \put(45,47){}
 \put(53,50){\vector(1,0){25}}
 \put(60,55){}
 \put(85,50){\oval(15,15)}
 \put(85,50){\oval(18,18)}
 \put(82,47){}
 \put(91,50){\vector(1,0){25}}
 \put(95,55){}
 \put(122,50){\oval(15,15)}
 \put(122,50){\oval(18,18)}
 \put(120,47){}
 \put(82,62){\oval(75,10)[t]}
 \put(45,63){\vector(0,-1){5}}
 \put(120,63){\line(0,-1){5}}
 \put(123,63){}

 \put(49,23){\oval(15,15)}
 \put(47,20){}
 \put(55,23){\vector(1,0){25}}
 \put(62,28){}
 \put(87,23){\oval(15,15)}
 \put(87,23){\oval(18,18)}
 \put(84,20){}
 \put(68,10){\oval(40,15)[b]}
 \put(48,10){\vector(0,1){5}}
 \put(88,10){\line(0,1){5}}
 \put(92,13){}

\end{picture}
\end{center}
\begin{center}
\caption{{An example where , , but  and ,
           , 
, \newline, and .}
          }
\end{center}
 \label{f6}
\end{figure}

We have that:
\begin{itemize}
 \item , because , but ;
 \item , , and
 \item , because
       , ,
       , ,
     and , for all .
\end{itemize}
\end{proof}

Let us verify the case when two states  are similar, or we can distinguish between them.

\begin{lemma}
  Let  be  a NFCA for the finite language , , ,
and either , or .
Assume  and .
\begin{enumerate}
 \item  If , for all possible choices of  and , then .
 \item  The converse is false, i.e., we may have  , for some  and , and .
\end{enumerate}
\end{lemma}
\begin{proof}
Assume  , and let . 
Because either , or , 
we have that  and , 
or   , and .
If  and , 
it follows that we have two states  and 
such that  , and .
This proves that the first implication is true. For the second implication, 
consider the automaton depicted in Figure~\ref{f6} with , and the following states :
 , , , and the letter . We have that 
, , 
but , because  
and .
\end{proof}

This result contrasts with the one for the deterministic case for cover automata, 
and the main reason is the nondeterminism, not the fact that we work with  cover languages.


Next, we would like to verify if similar states can be merged in case of NFCAs, also to check
 which type of merge works.
In case we have two similar states, we can strongly merge them as shown below.
In the case of DFCAs, if two states are similar, these can be merged. We must ensure that 
the same result is also true for  NFCAs, and the next theorem 
shows it. 


\begin{theorem}
\label{ltech1}
 Let  be an NFCA for ,
and  such that ,  and .
Then we have
\begin{enumerate}
 \item if , then .
 \item It is possible that .
\end{enumerate}
\end{theorem}

\begin{proof}
For the first part, let  be the automaton obtained from  by strongly merging  in .
We need to show that
 is a cover NFCA for .
Let  be a word in ,  and 
 for all , .
We now prove that  iff .

If we can find the states  such that
,
,
\ldots,
,
 and ,
then 
,
,
\ldots,
,
, i.e., 
.
Assume , and  is the smallest with this property.
If , then , which implies , then 
,
,
\ldots,
,
which means  .

Assume the statements hold for  for 
(), and 
consider the case when 
 .
If for every non-empty prefix of , ,
, then
 iff 
 , i.e., 

iff
.


Otherwise, let  be the smallest number such that 
.
Then  (and ).
By induction hypothesis,
 iff .
Therefore,  iff 
,
proving the first part.
For the second part, consider the automaton in Figure~\ref{f7} as a NFCA for . We have that
  and , because , and 
, .
We cannot weakly merge state  with state , as we would recognize .
In Figure~\ref{f8} we have the result for strongly merging state  in state .
\begin{figure}
\begin{picture}(150,80)
\put(-5,50){\vector(1,0){5}}  
\put(8,50){\oval(15,15)}  
\put(5,47){}
\put(16,50){\vector(1,0){27}}
\put(25,55){}    
\put(16,48){\vector(1,-1){26}}
\put(20,30){}
\put(50,20){\oval(15,15)}
\put(47,17){}    
\put(57,20){\vector(1,0){24}}
\put(65,25){}
\put(90,20){\oval(15,15)}
\put(90,20){\oval(18,18)}
\put(87,17){}    

\put(50,50){\oval(15,15)}  
\put(47,47){}
\put(57.5,50){\vector(1,0){25}}  
\put(65,55){}    

\put(90,50){\oval(15,15)}  
\put(87,47){}
\put(98,50){\vector(1,0){25}}  
\put(105,55){}    

\put(130,50){\oval(15,15)}  
\put(127,47){}
\put(138,50){\vector(1,0){25}}  
\put(145,55){}    
\put(130,71){\oval(15,15)[t]}  
\put(137,71){\line(-1,-3){5}}  
\put(123,71){\vector(1,-3){5}}  
\put(137,77){}    

\put(171,50){\oval(15,15)}
\put(171,50){\oval(18,18)}    
\put(168,47){}
\end{picture}
\hfill
\begin{picture}(150,80)
\put(-7,60){\vector(1,0){7}}
\put(8,60){\oval(15,15)}  
\put(5,57){}
\put(15,60){\vector(1,0){27}}
\put(25,65){}    
\put(15,57){\vector(1,-1){23}}
\put(18,43){}
\put(47,30){\oval(19,15)}
\put(39.5,27){}    
\put(37.5,30){\vector(-1,0){21}}  
\put(56.5,30){\vector(1,0){27}}
\put(65,33){}
\put(92,30){\oval(15,15)}
\put(92,30){\oval(18,18)}
\put(89,27){}    

\put(50,60){\oval(15,15)}  
\put(47,57){}
\put(58,60){\vector(1,0){26.5}}  
\put(65,64){}    

\put(92,60){\oval(15,15)}  
\put(89,57){}
\put(92,52){\vector(-2,-1){36}}  
\put(65,45){}    



\put(20,35){}    
\put(47,7){\oval(15,15)[b]}  
\put(55,7){\line(-1,3){5}}  
\put(40,7){\vector(1,3){5}}  
\put(57,5){}    

\put(8,30){\oval(15,15)}
\put(8,30){\oval(18,18)}    
\put(5,27){}
\end{picture}
\caption{Example for weakly merging failure and similar states.}
\label{f7}
\end{figure}
\end{proof}

\begin{figure}
\begin{center}
\begin{picture}(100,60)
\put(-5,50){\vector(1,0){5}}
\put(8,50){\oval(15,15)}  
\put(5,47){}
\put(15.5,50){\vector(1,0){26.5}}
\put(25,55){}    
\put(15.5,48){\vector(1,-1){22.5}}
\put(20,30){}
\put(46,20){\oval(18,15)}
\put(43,17){}    
\put(56,20){\vector(1,0){28}}
\put(65,22){}
\put(92,20){\oval(15,15)}
\put(92,20){\oval(18,18)}
\put(89,17){}    

\put(50,50){\oval(15,15)}  
\put(47,47){}
\put(58,50){\vector(1,0){26.5}}  
\put(65,55){}    

\put(92,50){\oval(15,15)}  
\put(89,47){}
\put(90,42){\vector(-2,-1){35}}  
\put(65,33){}    





\end{picture}
\end{center}
\caption{Example for strongly merging similar states for the example presented in Figure~\ref{f7}.}
\label{f8}
\end{figure}
We can observe that strongly merging states does not add words in the language, 
while weakly merging may add words. Because any DFCA is also a NFCA, then some smaller automata can be obtained from 
larger ones without using state merging technique, and the following lemma presents such a case. 
Also, the automaton in Figure~\ref{f2} is obtained from automaton in Figure~\ref{f1} by strongly merging states 
 into state .

\begin{lemma}
Let  be an NFCA for , and consider the reduced sub-automaton
generated by state , , i.e.,   
contains only reachable and useful states, and  is the induced transition function.
If , for all , we can find two regular languages , such that
\begin{itemize}
 \item , and
 \item ,
\end{itemize}
then  is not minimal.
\end{lemma}
\begin{proof}
Let ,  be two NFAs
for  and , and  .
We define the automaton  as follows:
, in case , 
and  in case .
For the transition function, we have  if ,
 if , ,
and
, if ,
and , if .
Obviously, the automaton  recognizes the cover language for , 
and its state complexity is lower.
\end{proof}
This technique was used to produce the minimal NFCA for  in Figure~\ref{f5}.

\section{Conclusion}
\label{openproblems}
In this paper we showed that NFCAs are a more compact representation of finite languages than both 
NFAs and DFCAs, therefore it is a subject worth investigating.
We presented a lower-bound technique for state complexity of NFCAs, and proved its limitations.
We showed that minimizing NFCAs has at least the same level of difficulty as minimizing 
general NFAs, and that extra information about the maximum length of the words in the language 
does not help reducing the time complexity.
We checked if some of the results involving reducing the size of automata for NFAs and DFCAs
are still valid for NFCAs, and showed that most of them are no longer valid. 
However, the method of strong merging states still works in case of NFCAs, and we showed 
that there are also other methods that could be investigated.

As future research, below is a list of problems we consider worth investigating:
\begin{enumerate}
 \item check if the bipartite graph lower-bound
 technique can be applied for \mbox{NFCAs;}
 \item find bounds for nondeterministic cover state complexity;
 \item investigate the problem of magic numbers for NFCAs. 
In this case, we can relate either to DFCAs, or DFAs. 
\end{enumerate}


\bibliographystyle{eptcs.bst}
\bibliography{MinNFCA}
\end{document}