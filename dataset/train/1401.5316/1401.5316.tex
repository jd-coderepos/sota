\documentclass[11pt]{article}
\usepackage[small, bf]{caption}
\usepackage{indentfirst,url,amsfonts, amsthm, amsmath, amssymb,color, enumerate,algorithmic,verbatim,printlen}
\usepackage[boxed]{algorithm}
\usepackage[top=1.0in, bottom=1.0in, left=1.0in, right=1.0in]{geometry}
\usepackage{graphicx}
\newcommand{\shortbar}{\begin{center}\rule{5ex}{0.1pt}\end{center}}
\newcommand{\ignore}[1]{}
\newcommand{\courseNumber}{EECS 574}
\newcommand{\courseTitle}{Computational Complexity}
\newcommand{\semester}{Fall 2010}
\newtheorem{theorem}{Theorem}[section]
\newtheorem{lemma}[theorem]{Lemma}
\newtheorem{property}[theorem]{Property}
\newtheorem{corollary}[theorem]{Corollary}
\newtheorem{proposition}[theorem]{Proposition}
\newtheorem{statement}[theorem]{Statement}
\newtheorem{conjecture}[theorem]{Conjecture}
\newtheorem{fact}[theorem]{Fact}
\newtheorem{definition}[theorem]{Definition}
\newtheorem{example}{Example}
\newtheorem{problem}[theorem]{Problem}
\newtheorem{exercise}{Exercise}
\newtheorem{remark}[theorem]{Remark}
\newtheorem{reduction}[theorem]{Reduction}
\newtheorem{question}{Question}


\newcommand{\istrut}[2][0]{\rule[- #1 mm]{0mm}{#1 mm}\rule{0mm}{#2 mm}}
\newcommand{\rb}[2]{\raisebox{#1 mm}[0mm][0mm]{#2}}
\newcommand{\vcm}[1][1]{\vspace*{#1 cm}}
\newcommand{\hcm}[1][1]{\hspace*{#1 cm}}
\newcommand{\MWM}{{\sc mwm}}
\newcommand{\MWPM}{{\sc mwpm}}
\newcommand{\f}[2]{\frac{#1}{#2}}
\newcommand{\poly}{\operatorname{poly}}
\newcommand{\polylog}{\operatorname{polylog}}
\newcommand{\E}{\operatorname{E}}
\newcommand{\Var}{\operatorname{Var}}
\newcommand{\ID}{\operatorname{ID}}
\newcommand{\defect}{\operatorname{def}}

\newcommand{\solutions}[6]{
\vspace{-2ex}
\begin{center}
{\small  \courseNumber, \courseTitle
\hfill {\small \bf {#1} }\\
\semester, University of Michigan, Ann Arbor \hfill
{\em Date: #3}}\\
\vspace{-1ex}
\hrulefill\\
\vspace{2ex}
{\LARGE Homework #2 Solutions}\\
\end{center}
\noindent
}
\newcommand{\defeq}{\stackrel{\textrm{def}}{=}}
\newcommand{\Prob}{\textrm{Prob}} 

\long\def\symbolfootnote[#1]#2{\begingroup \def\thefootnote{\fnsymbol{footnote}}\footnote[#1]{#2}\endgroup} 


\begin{document}
\begin{titlepage}
\title{A Distributed Minimum Cut Approximation Scheme}
\author{Hsin-Hao Su \footnote{This work is supported by NSF grants CCF-0746673, CCF-1217338, and CNS-1318294
and a grant from the US-Israel Binational Science Foundation. This work was done while visiting MADALGO at Aarhus University.}\\
University of Michigan}
\date{}
\end{titlepage}


\maketitle

\begin{abstract}
In this paper, we study the problem of approximating the minimum cut in a distributed message-passing model, the \textsf{CONGEST} model. The minimum cut problem has been well-studied in the context of centralized algorithms. However, there were no known non-trivial algorithms in the distributed model until the recent work of Ghaffari and Kuhn. They gave algorithms for finding cuts of size  and  in  rounds and  rounds respectively, where  is the size of the minimum cut. This matches the lower bound they provided up to a polylogarithmic factor. Yet, no scheme that achieves -approximation ratio is known. We give a distributed algorithm that finds a cut of size  in  time, which is optimal up to polylogarithmic factors.
\end{abstract}
\thispagestyle{empty}

\setcounter{page}{1}

\section{Introduction}
The minimum cut problem is a fundamental problem in graph algorithms and network design. Given a weighted undirected graph , a cut  where , is a partition of vertices into two non-empty sets. The weight of a cut, , is defined to be the sum of the edge weights crossing . The minimum cut problem is to find a cut with the minimum weight. The exact version of the problem as well as the approximate version have been studied for many years \cite{Karger93, KS93, Karger94b, NI92, Matula93, Gabow95, SW97, Karger2000} in the context of centralized models of computation, resulting in nearly linear time algorithms \cite{Karger2000, Matula93, Karger94b}. 

Elkin \cite{Elkin04} and Das Sarma et al.~\cite{DasSarma12} addressed the problem in the distributed message-passing model. The problem has trivial time complexity of  (unweighted diameter) in the \textsf{LOCAL} model, where the message size is unlimited. Ghaffari and Kuhn \cite{GK13} recently developed approximation algorithms for this problem in the \textsf{CONGEST} model where each message is bounded by  bits. They assume that the edges of  have integer weights from  and treat  as an unweighted multigraph, where an edge  with weight  is converted to  parellel edges, while still only  bits can be sent over these parallel edges together in each round. Let  be the value of the minimum cut, they give an algorithm that finds a cut of size at most  in  time. Moreover, they gave an algorithm that finds a cut of size at most  in  time. Das Sarma et al.~\cite{DasSarma12} showed -approximating the minimum cut requires  rounds for weighted graphs for any . Ghaffari and Kuhn extended their lower bound for unweighted multigraphs (which is equivalent to the setting where one is allowed to send messages of size  over an edge of weight  in weighted graphs). For unweighted simple graphs, they also gave a lower bound of . Therefore, the upper bound and lower bound provided by Ghaffari and Kuhn match up to a polylogarithmic factor. 

However, still no approximation algorithms exist for any approximation factor less than 2. In this paper, we give a simple algorithm that finds a minimum cut of size at most  in  time. In particular, our algorithm runs in  rounds. 

Our approach uses the semi-duality between minimum cuts and tree packings as in \cite{Karger2000, Thorup07}. Karger \cite{Karger2000} showed that if we greedily pack enough trees, then for any minimum cut, there is a tree crossing the cut at most twice. However, it is technically not easy to utilize this fact to find minimum cuts in the distributed model. Instead, we use a lemma by Thorup \cite{Thorup07}, which shows that if we pack more trees then there is at least one minimum cut that is crossed by a tree exactly once. We take some ingredients from Ghaffari and Kuhn's algorithm and Thurimella's algorithm \cite{Thurimella97} for identifying biconnected components to devise a procedure that is able to simultanously test the values of the  cuts induced by deleting one of the  edges in a tree. Note that the number of trees we have to pack is polynomial in the value of the minimum cut. Thus, we will first use the sampling lemma of Karger \cite{Karger94} to obtain a sampled graph that scales the value of the minimum cut down to . Then we only have to pack polylogarithmic number of trees. Finally, we combine the resampling procedure, the tree packing, and the procedure for testing tree-induced-cuts to find an approximate minimum cut.


\section{Distributed Minimum Cut Approximation}
Let  be a connected graph with integer weights from , where . We will treat  as a multigraph with uniform edge weights.
Let  be the weight of the minimum cut of . We show how to find such an approximate minimum cut whose weight is at most . 

An edge  is a {\it bridge} if it does not exist a cycle in  passing  (or equivalently, deleting  breaks  into two connected components). Given two graph  and  with the same vertex set,  is the multigraph obtained by including edges in  and edges in .

A {\it tree packing}  is a multiset of spanning trees. The {\it load} of an edge  with respect to  is the number of trees in  containing . Given a tree , we say a cut is {\it induced} by  if such a cut is obtained by deleting an edge . We will denote this cut by . A tree packing  is {\it greedy} if each  is a minimum spanning tree with respect to the loads induced by . Let  such that .


\begin{lemma}[Thorup \cite{Thorup07}]\label{lem:treepacking}A greedy tree packing with  trees contains a tree crossing some min-cut only once.
\end{lemma}
\begin{remark}The number of trees in the original statement of the lemma is , though the proof actually implies that  is enough. In particular, Thorup showed  trees is sufficient, where  satisfies  for some . We can choose  and  to make the inequality hold. Therefore,  trees is sufficient.\end{remark}

We describe our algorithm in Algorithm \ref{alg:approxmincut}. The subroutine Test() returns a cut whose weight is at most  w.h.p.~if there exists a cut in  induced by  with weight at most .

We show that w.h.p. the algorithm will output a cut  with . In particular, consider the iteration  where . Let  denote the value of the minimum cut in the sampled graph . If , then it is clear that . If , since we sampled with probability , we know that w.h.p. for any cut  \cite[Corollary 2.4]{Karger94}, 
Therefore, . If we pack  trees in , then by Lemma \ref{lem:treepacking} there exists a tree crossing some minimum cut  of  only once. Notice that for any other cut , 


\begin{algorithm}[H]
\caption{-approximate minimum cut}
\begin{algorithmic}[1]\label{alg:approxmincut}
\STATE 
\STATE 
\REPEAT
\STATE 
\STATE (We are assuming  in this iteration)
\STATE Let  be the subgraph sampled with probability  on each edge of .
\STATE \label{line:treepacking}Find a greedy tree packing  with  trees in 
\
\STATE 
\REPEAT \label{line:loopup}
\FOR{each }
\STATE Call Test().
\STATE If Test() returns a cut , output  and terminate.
\ENDFOR
\STATE 
\UNTIL{} \label{line:loopdown}
\STATE 
\UNTIL{}
\end{algorithmic}
\end{algorithm}


Therefore, one of the cuts induced by some  is an  approximate minimum cut. Denote this cut by , so . Therefore in the 'th iteration, there exists  in the loop (Line \ref{line:loopup}--Line \ref{line:loopdown}) such that . So w.h.p.\ we will output a cut with weight at most .

\subsection{Distributed Implmentation}

We have shown the correctness of this algorithm. It remains to show how to implement it in  distributed rounds, and in particular, to implement the tree packing (Line \ref{line:treepacking}) and Test() in Algorithm \ref{alg:approxmincut}. To pack  trees, it is striaghtfoward to apply  MST computations on the graph where the edge weights are equal to the number of trees including it. This can be done in  rounds \cite{KP95}.

Given a partition  of  into components, Ghaffari and Kuhn \cite{GK13} devised a testing procedure to test if there is a cut induced by a component in  that has weight less than  in  rounds. Given a spanning tree , we will show how to test the  cuts induced by  also in  rounds. 

\begin{algorithm}[H]
\caption{Test. Test() returns a cut whose weight is at most  w.h.p.~if there exists a cut in  induced by  with weight at most . Note that the sample probability .}
\begin{algorithmic}[1]\label{alg:testing}
\FOR{}
	\STATE Let  be the subgraph obtained by sampling each edge of  independently with probability .
	\STATE For each edge , determine if  is a {\bf bridge} in the graph . \STATE Let eG_i+T
\ENDFOR
\STATE If there is  such that , then return the cut 
\end{algorithmic}
\end{algorithm}

\begin{lemma} If  induces a cut  with weight at most , then Test() returns a cut w.h.p. Moreover, any cut returned by the algorithm has weight at most  w.h.p. \end{lemma}

\begin{proof}

Consider a cut . First observe that  contains an edge crossing  if and only if  is not a bridge in the graph . Therefore, . 

If there is , then  and . By Hoeffiding's inequality, . By taking the union bound over the  cuts induced by , we conclude that w.h.p. the algorithm will return a cut if there is cut whose weight is at most .

On the other hand if , then  when , since  when . So . By Hoeffiding's inequality, . By taking the union bound over the  cuts induced by , we conclude the cut returned by the algorithm has weight at most  w.h.p.
\end{proof}

\subsection{Computing the Bridges}
Given a subgraph  of , it remains to show how to determine what edges of  are bridges in the subgraph  in  rounds. Thurimella \cite{Thurimella97} gave an algorithm for computing the biconnected components of the underlying graph in  rounds. With simple modifications, it can be applied to compute which edges of  are bridges in the {\it subgraph}  of the underlying graph . Note that even we have the algorithm for computing the bridges of  in , it is not clearly whether we can directly simulate it to compute the bridges of  in , because we want the running time to depend on the diameter of  rather than that of . Therefore, we describe the algorithm and necessary changes below.

\renewcommand\thefootnote{\fnsymbol{footnote}}
Fix a root  in . Let  be the preorder number which denote the time  is visited if we perform a depth-first search on  starting at . Denote the subtree rooted at  by  and let  be the size of . Let
\addtocounter{footnote}{1} 

 \footnotetext[2]{It can be the case that  is the parent of  in , which happens when there are parallel edges between  and  in , and one of them is in . Note that an edge is not a bridge if it is a multiedge.}
\begin{lemma}Let  with  being a child of , i.e.~.  is a bridge if and only if  and .\end{lemma}
\begin{proof}
First notice that every vertex  must have . If  is a bridge, then no descendent of  will be adjacent to anything outside the subtree rooted at , for otherwise a cycle passing  will be created. Therefore, .

On the other hand, if  or if , then there exists a vertex  and  such that  and  are adjacent. Since , there must exists a path from  to  such that it does not pass . Therefore,  forms a cycle and  is not a bridge.
\end{proof}
\begin{remark}Note that the second condition  is needed because  is not necessarily a DFS tree. \end{remark}

Now it remains to show how to compute , , and  in  time. It is explicitly described in \cite{Thurimella97} how to compute . Note that  is independent of the . Although  and  depend on , they can be computed in a similar way in  time. For completeness, we describe how to compute these functions in the following.

\begin{lemma}[\cite{GKP93,KP95}]\label{lem:decompose} A tree of  vertices can be divided into  connected subgraphs each of diameter  in  time\end{lemma}

First, use Lemma \ref{lem:decompose} to decompose the rooted tree  into components . For each component , there is a root  which is either the root of , , or the unique vertex in  connecting to its parent outside . It is shown in \cite{Peleg} that the root  is able to downcast distinct messages of size  to each of  in  time. Conversely, it is possible for each of the  to upcast a message of size  to the root  in  time. 

Suppose each vertex has a unique ID. The component ID of  is defined to be the ID of . The component ID can be broadcast to every vertex in the component in  rounds. We can then assume that the root  knows the topology of the contracted tree where each component is contracted into a single vertex. This can be done if every root  upcasts a message about the component ID of its parent and itself.

To compute , each root  in each component first calculate the size of  then upcast it to . Since  knows the topology of the contracted tree,  can calculate the size of each subtree rooted at each of . Then  downcasts the size of subtree rooted at  back to . Now each  computes its preorder number internally in  time assuming  has number . During the computation, each  also records what its preorder number is supposed to be if the depth-first search started from the root of its parent component. Finally, each  upcasts this number to  and then  computes the correct offset for each subtree and downcasts the offsets back to the . After adding the offset internally, we get the correct preorder number.

To compute , initially each vertex  computes  in constant rounds. Then the problem becomes aggregating the minimum in the subtree  for each . First, each  computes the minimum in  in  time and then upcasts to . Using the information,  calculates the minimum of the subtrees rooted at each  and downcasts to each . Now each  sends the minimum to its parent via the inter-component links. The parent replace its minimum if it is smaller. Finally, each component  internally updates the minimum toward the root . Then each vertex has the correct minimum.  can be computed in the same way.

Therefore, the step of computing the bridges in  of  takes  time. Each invocation of Test() takes  time. 

\subsection{Running Time}
Now we analyze the running time of Algorithm \ref{alg:approxmincut}. The outerloop runs for  iterations. Therefore, the tree packing, Line \ref{line:treepacking}, is executed  times, each taking  rounds.

Let  be the largest index such that . The total number of iterations that the innerloop runs is at most 
Therefore, Test() is invoked at most  times, each taking  rounds. 

The total running time is


\begin{remark}\label{rmk:combinedGK} The total iterations of the outerloop and innerloop in Algorithm \ref{alg:approxmincut} can be reduced to  and  by first approximating  within constant factor by Ghaffari and Kuhn's algorithm. Then, we can reduce our running time to .\end{remark}

The exponent of the  and the  in our running time depends heavily on the size of the greedy tree packing in Lemma \ref{lem:treepacking}. If one can show that  trees is sufficient, then our running time can be improved to  rounds. Using Ghaffari and Kuhn's algorithm to approximate  within a constant (Remark \ref{rmk:combinedGK}), we can get a running time of . For comparison, Karger \cite{Karger2000} showed that a greedy tree packing of size  is enough for any minimum cut to be crossed at most twice by some tree. It will be interesting to see if the number of trees in Lemma \ref{lem:treepacking} can be reduced.

\bibliography{mybib_short}
\bibliographystyle{plain}

\end{document}
