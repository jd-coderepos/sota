\documentclass[submission,copyright,creativecommons,english]{eptcs}
\providecommand{\event}{DCM 2014} \usepackage{latexsym,amssymb,amsmath,amsthm,mathdots,ellipsis,multirow}
\usepackage[T1]{fontenc}
\usepackage[utf8]{inputenc}
\usepackage{stmaryrd}
\usepackage{amstext}
\usepackage{graphicx}
\usepackage{float}
\usepackage{algorithm}
\usepackage{xcolor}
\usepackage{versions}\excludeversion{ignore}

\newcommand{\nb}[1]{{\color{red}#1}}
\newtheorem{postulate}{Postulate}
\newtheorem{theorem}{Theorem}
\newtheorem{lemma}{Lemma}
\newtheorem{definition}[theorem]{Definition}

\DeclareMathOperator{\sgn}{sgn}

\newcommand{\F}{{F}}
\newcommand{\G}{{G}}
\newcommand{\M}{\mathcal{A}\!}
\newcommand{\ESM}{\mathcal{E}\!}
\newcommand{\ac}{\mathcal{AC}\!}
\newcommand{\Ops}{\mathrm{Ops}}
\newcommand{\dotdot}{\ensuremath{\mathpunct{\ldotp\ldotp}}}
\newcommand{\ESMprog}{\mathrm{\ESM}\mathrm{-prog}}
\newcommand{\IN}{\mbox{~\bf in~}}
\newcommand{\LET}{\mbox{\bf let}\;}
\newcommand{\IF}{\mbox{\bf if}\;}
\newcommand{\THEN}{\;\mbox{\bf then}\;}
\newcommand{\ELSE}{\;\mbox{\bf else}\;}
\newcommand{\DO}{\;\textbf{do}\;}
\newcommand{\True}{\textsc{true}}
\newcommand{\False}{\textsc{false}}
\newcommand{\Undef}{\textsc{undef}}
\newcommand{\evaluation}[2][]{\ensuremath{\llbracket #2\right\rrbracket_{#1}}}
\newcommand{\val}[2]{\evaluation[#2]{#1}}
\newcommand{\valf}[2]{{#1}_{#2}}
\newcommand{\DOm}{\mathrm{Dom}~}
\newcommand{\NN}{\ensuremath{\mathbb{N}}}
\newcommand{\GG}{\ensuremath{\mathbb{G}}}
\newcommand{\K}{\ensuremath{\mathrm{\Sigma}}}
\newcommand{\st}{\mbox{ \bf such that }}
\newcommand{\RAM}{\ensuremath{\mathcal{R}}}
\newcommand{\bb}{\ensuremath{\mathcal{B}}}
\newcommand{\cc}{\ensuremath{\mathcal{C}}}
\newcommand{\C}{\textsc{c}}
\newcommand{\Id}[1]{\widehat{#1}}
\newcommand{\Val}[1]{\llbracket{#1}\rrbracket}

\newcommand{\Goto}{\mbox{\bf goto}\;}
\newcommand{\New}{\mbox{\bf new}\;}

\newcommand{\Set}[2]{#1 := #2}
\newcommand{\IFExists}{\mbox{\bf if for }\;}
\newcommand{\For}{\mbox{\bf for}\;}
\newcommand{\rite}[1]{\mathop{\longrightarrow}\limits^{#1}}
\renewenvironment{cases}{\left\{\begin{array}{ll}}{\end{array}\right.}
\newcommand{\A}{\mathcal{A}~\!}
\newcommand{\B}{\mathcal{B}~\!}
\newcommand{\DOForall}{\mbox{{\bf do} \dots\ {\bf for all} \dots}}
\newcommand{\Choose}{\epsilon}   \newcommand{\TheUnique}{\;\mbox{\bf TheUnique}\;}
\sloppy

\title{Cellular Automata are Generic}
\def\titlerunning{Cellular Automata are Generic}
\def\authorrunning{N. Dershowitz \&  E. Falkovich}
\author{Nachum Dershowitz 
\institute{School of Computer Science\\Tel Aviv University\\Tel Aviv, Israel}
\email{nachum.dershowitz@cs.tau.ac.il}
\and Evgenia Falkovich\footnote{This work was carried out in partial fulfillment of the requirements for the Ph.D.\ degree
of the second author.}
\institute{School of Computer Science\\Tel Aviv University\\Tel Aviv, Israel}
\email{jenny.falkovich@gmail.com}}
\date{\today}
\usepackage{babel}
\begin{document}
\maketitle
\begin{abstract}
Any algorithm (in the sense of Gurevich's abstract-state-machine axiomatization of classical algorithms) operating over any arbitrary unordered domain can be simulated by a dynamic cellular automaton, that is, by a pattern-directed cellular automaton with unconstrained topology and with the power to create new cells.  
The advantage is that the latter is closer to physical reality.
The overhead of our simulation is quadratic.
\end{abstract}

\begin{quote}\raggedleft
\textit{Order gave each thing view.}\\includegraphics[scale = 0.4]{pics/transition_example.pdf}\includegraphics[scale = 0.4]{pics/non-deterministic_transition_example.pdf}f(s^{1},\ldots,s^\ell):=u\IF c\THEN p \mbox{\rm ~~~~or~~~~}
\IF c\THEN p\ELSE q\includegraphics[scale = 0.4]{pics/unordered-tree.pdf}\includegraphics[scale = 0.4]{pics/unordered-term-graph-eps-converted-to.pdf}\includegraphics[scale = 0.4]{pics/unordered-tangle-eps-converted-to.pdf}\includegraphics[scale = 0.4]{pics/unordered-tangle-pairs-eps-converted-to.pdf}\includegraphics[scale = 0.4]{pics/unordered-tangle-reversed-eps-converted-to.pdf}\includegraphics[scale = 0.4]{pics/unordered-transition-pair.pdf}\includegraphics[scale = 0.4]{pics/unordered-transition-choose.pdf}\vspace*{-5mm}\IF t\in p\THEN t:=p\includegraphics[scale = 0.4]{pics/unordered-transition-in.pdf}\IF t\neq p \THEN t:=f(t,p)\includegraphics[scale = 0.4]{pics/unordered-transition-functions.pdf}\includegraphics[scale = 0.4]{pics/unordered-transition-singleton-1.pdf}\includegraphics[scale = 0.4]{pics/unordered-transition-singleton-2.pdf}\includegraphics[scale = 0.4]{pics/unordered-transition-singleton-3.pdf}\includegraphics[scale = 0.4]{pics/unordered-transition-check-union1.pdf}\includegraphics[scale = 0.4]{pics/unordered-transition-check-union-v2-2.pdf}\includegraphics[scale = 0.4]{pics/unordered-transition-check-union-v2-3.pdf}\includegraphics[scale = 0.4]{pics/unordered-transition-check-union-v2-4.pdf}\includegraphics[scale = 0.4]{pics/unordered-transition-check-union-v2-5.pdf}\includegraphics[scale = 0.4]{pics/unordered-transition-check-union-v2-6.pdf}\includegraphics[scale = 0.4]{pics/unordered-transition-union-v2-2.pdf}\includegraphics[scale = 0.4]{pics/unordered-transition-union-v2-3.pdf}\includegraphics[scale = 0.4]{pics/unordered-transition-union-v2-4.pdf}\includegraphics[scale = 0.4]{pics/unordered-transition-union-v2-5.pdf}\includegraphics[scale = 0.4]{pics/unordered-transition-union-v2-6.pdf}\includegraphics[scale = 0.4]{pics/unordered-transition-example_union.pdf}\includegraphics[scale = 0.4]{pics/unordered-transition-union2.pdf}\includegraphics[scale = 0.4]{pics/unordered-transition-union5.pdf}
We do so for all rules in the transition, and replace the inclusion rules with the outcome.
\end{ignore}
\end{proof}

Every classical algorithm is emulated step-by-step, state-by-state by an ASM
consisting of  a fixed number of comparisons and assignments~\cite{ASM-Theorem-Gurevich}.
That fact, along with the previous lemmata, is what is needed  to achieve our goal:

\begin{theorem}[Main]\label{th:main}
Cellular automata with {bounded dynamics (i.e.\@ all the nodes in a pattern are within a bounded distance of the focus) and without loops (there are no directed cycles within patterns)} can simulate the performance of  any classical algorithm over an unordered domain
with {quadratic} multiplicand overhead.
\end{theorem}

\begin{proof}
{We  have  to ensure that, once the automaton starts to simulate the singleton or union operation,
it cannot be interrupted by the application of other transition rules.
Otherwise, foreign steps could affect the elements of the sets involved in these set operations.
This problem can be precluded, for instance, by changing the color of the \textsf{Criticals} node during the
simulation of those operations.}

{Each step of the original algorithm can only create a bounded number of new sets.
Hence the size of the sets involved in any union operation is bounded by the size of the sets in  the initial state plus some multiple of the algorithm's steps so far.
So the overall overhead caused by unions is quadratic.}
\end{proof}

\section{Discussion}

We have outlined the basic features of dynamic cellular automata and proved that this model is flexible enough
to simulate arbitrary computations over unordered domains.
It may perhaps be possible to reduce the cost of the simulation.
{In particular, allowing negative edges in patterns, for labeled edges that must not appear, can reduce the complexity of the union operation.
One may also consider allowing for duplication, which would increase the cost of comparisons but reduce the cost of union.}

{Another question is at what added expense could one  bound the degree of nodes.}

One task facing us now is to describe a natural extension to the parallel case. In this case, at each step, all  cells are active and can all affect their neighborhoods.

\subsection*{Acknowledgement}
This work benefited greatly from long discussions with Gilles Dowek.
{We thank the reviewer for a careful reading.}

\bibliographystyle{eptcs}
\bibliography{models}
 
\end{document}
