\chapter{Simulation Experiments, Results and Discussions}
\label{simulations}

In this chapter, we present and discuss a wide range of results of extensive simulations that we have conducted in NS-2~\cite{NS-2} (Version 2.30), to compare the performance of SQ-AODV with that of MDR~\cite{mdr} and AODV~\cite{aodv}. We have considered the following six parameters:

\begin{itemize}
\item \textbf{Packet Delivery Ratio (PDR)}: is the ratio of the number of packets successfully received
by all destinations to the total number of packets injected into the network by all sources. The PDR is
therefore a number between 0 and 1.

\item \textbf{Normalized Control Overhead:} is the ratio of number of routing packets transmitted 
(hop wise) by all the nodes to the total number of packets successfully received by all destinations in the 
network. The normalized control overhead is therefore a number greater than 0.

\item \textbf{Average Packet Delay:} is the sum of the times taken by the successful data packets to 
travel from their sources to destinations divided by the total number of successful packets. The average packet delay 
is measured in seconds.

\item \textbf{Average Hop Count:} is the sum of the number of hops taken by the successful data packets to 
travel from their sources to destinations divided by the total number of successful packets. The average hop count  
is measured in number of hops.

\item \textbf{Node Expiration Time (NET)}: is the time for which a node has been alive before it must halt transmission due to battery depletion. The node expiration time is plotted as number of nodes alive at a given time, for different points in time during the simulation.

\item \textbf{Connection Expiration Time (CET)}: is the time for which a connection has been active before it must cease transmission due to the non-availability of a route between source and destination.
This occurs when nodes along the path expire or become unreachable due to poor link conditions. The connection expiration time is expressed in seconds.
\end{itemize}

\noindent
We present our simulation results in 3 different parts, and are as follows:

\begin{enumerate}
\item Demonstration of SQ-AODV features
\item Validation of MDR~\cite{mdr} implementation
\item Performance comparison of SQ-AODV, MDR and AODV~\cite{aodv}
\end{enumerate}

\section{Demonstration of SQ-AODV Features}
\label{sq-aodv features}

In this section, we present our simulation results to demonstrate the energy awareness and make-before-break re-routing features of SQ-AODV by comparing the packet delivery ratio performance with that of AODV.

\subsection{Simulation Setup}
We have considered the 12-node topology in 500 m x 1000 m area as shown in Figure~\ref{fig51}. The nodes are identical in their capability, but are initialized with different energies in different experiments Expt1 to Expt5, that we have conducted, and there is no mobility in the network.

\begin{figure}[htbp]
	\centering
	\includegraphics[scale=0.40]{fig8.eps}
	\caption{12 Node Topology}
	\label{fig51}
\end{figure}

To calibrate the load that can be supported by the network, an extensive series of simulations with one, two, three, and five simultaneous sessions was conducted with varying average session data rates. We found that when the aggregate rate of sessions at the nodes exceeded about 225 Kbps on average, the packet drop rate in the network became excessive, indicating that the network was saturated beyond capacity. We have thus used this rate as 100\% load, and normalized using this.

The MAC layer protocol is IEEE 802.11 DCF  (Distributed Co-ordination Function) with the PLCP (Physical Layer Convergence Protocol)  data rate being 1 Mbps. The parameters used in the simulations for Expt1 to Expt5 are listed in Table~\ref{tab51}.

\begin{table}[htbp]
\centering
  \caption{Values of Parameters Used for the Simulation of the 12 Node Topology}
  \hfill \\	
  \begin{tabular}{|c|c|}
  \hline
  Packet size & 512 Bytes \\
  \hline
  Packets/Session & 1000 \\
  \hline
  Date traffic & Poisson with exponential\\
  \            & inter-arrival time \\
  \hline
  MAC Protocol & IEEE 802.11 DCF \\
  \hline
  PCLP Data rate & 1 Mbps \\
  \hline
  Buffer length & 50 Packets \\
  \hline
  Transmit power & 0.2818 W \\
  \hline
  Propagation model & Two-Ray Ground \\
  \hline
  \end{tabular}
  \label{tab51}
\end{table}

Here we demonstrate the PDR performance of SQ-AODV under a variety of different network loads and node energies, and assess the benefit of its make-before-break strategy. In each experiment Expt1 to Expt5, the initial energy levels of the nodes is randomly chosen between 10 Joules to 100 Joules, and compared the performance of PDR for SQ-AODV and AODV. The starting times of the sessions were chosen such that there were at most between 3-5 sessions in parallel in the network at any instant. The network load was varied from about 20\% to 90\%, so the session data rate is varied from 15 Kbps to 65 Kbps (3-5 sessions of 15 Kbps each equates to 20\% of 225 Kbps (225 Kbps equals 100\% load) and 3-5 sessions of 65 Kbps each equates to 90\% of 225 Kbps). Each experiment is run 50 times (each initialized with different seed), and the resulting PDR is averaged over these 50 runs. There are 12 sessions in each experiment, with 1000 packets per session, generated as per a Poisson process. The energy distribution of the nodes for each of the 5 experiments Expt1 to Expt5, as well as the source-destination pairs for each session are summarized in Table~\ref{tab52} and Table~\ref{tab53}, respectively.

\begin{table}[htbp]
\centering
  \caption{Initial Energy of Nodes for Each Experiments in Joules}
  \hfill \\	
  \begin{tabular}{|c|c|c|c|c|c|}
  \hline
  Node & Expt1 & Expt2 & Expt3 & Expt4 & Expt5 \\
  \hline
  $1$ & $45$ & $29$ & $94$ & $82$ & $93$ \\
  \hline
  $2$ & $20$ & $91$ & $40$ & $73$ & $39$ \\
  \hline
  $3$ & $24$ & $40$ & $29$ & $37$ & $64$ \\
  \hline
  $4$ & $62$ & $97$ & $87$ & $33$ & $31$ \\
  \hline
  $5$ & $59$ & $82$ & $73$ & $90$ & $73$ \\
  \hline
  $6$ & $93$ & $82$ & $50$ & $30$ & $35$ \\
  \hline
  $7$ & $85$ & $50$ & $87$ & $64$ & $100$ \\
  \hline
  $8$ & $68$ & $34$ & $54$ & $28$ & $19$ \\
  \hline
  $9$ & $24$ & $15$ & $24$ & $86$ & $61$ \\
  \hline
  $10$ & $30$ & $17$ & $18$ & $86$ & $49$ \\
  \hline
  $11$ & $55$ & $43$ & $35$ & $90$ & $87$ \\
  \hline
  $12$ & $39$ & $20$ & $95$ & $11$ & $97$ \\
  \hline
  \end{tabular}
  \label{tab52}
\end{table}

\begin{table}[htbp]
\centering
  \caption{Source-destination Pairs in the 12 Node Topology}
  \hfill \\	
  \begin{tabular}{|c|c|c|c|}
  \hline
  $Session\ No.$ & $\{src,\ dst\}$\\
  \hline
  $1$ & $\{1,\ 11\}$\\
  \hline
  $2$ & $\{2,\ 12\}$\\
  \hline
  $3$ & $\{3,\ 11\}$\\
  \hline
  $4$ & $\{8,\ 6\}$\\
  \hline
  $5$ & $\{10,\ 1\}$\\
  \hline
  $6$ & $\{2,\ 4\}$\\
  \hline
  $7$ & $\{12,\ 3\}$\\
  \hline
  $8$ & $\{4,\ 2\}$\\
  \hline
  $9$ & $\{11,\ 1\}$\\
  \hline
  $10$ & $\{7,\ 6\}$\\
  \hline
  $11$ & $\{8,\ 10\}$\\
  \hline
  $12$ & $\{12,\ 1\}$\\
  \hline
  \end{tabular}
  \label{tab53}
\end{table}

\subsection{Results and Discussions}

\begin{figure}[htbp]
	\centering
	\includegraphics[scale=0.75]{AVG_PDR_EXPT1.eps}
	\caption{Packet Delivery Ratio for EXPT1}
	\label{fig52}
\end{figure}


\begin{figure}[htbp]
	\centering
	\includegraphics[scale=0.75]{AVG_PDR_EXPT2.eps}
	\caption{Packet Delivery Ratio for EXPT2}
	\label{fig53}
\end{figure}


\begin{figure}[htbp]
	\centering
	\includegraphics[scale=0.75]{AVG_PDR_EXPT3.eps}
	\caption{Packet Delivery Ratio for EXPT3}
	\label{fig54}
\end{figure}


\begin{figure}[htbp]
	\centering
	\includegraphics[scale=0.75]{AVG_PDR_EXPT4.eps}
	\caption{Packet Delivery Ratio for EXPT4}
	\label{fig55}
\end{figure}


\begin{figure}[htbp]
	\centering
	\includegraphics[scale=0.75]{AVG_PDR_EXPT5.eps}
	\caption{Packet Delivery Ratio for EXPT5}
	\label{fig56}
\end{figure}

The packet delivery ratio for each of the experiments Expt1 to Expt5, for both SQ-AODV and AODV~\cite{aodv} is plotted in Figures~\ref{fig52} - \ref{fig56}. It can be observed that, in all cases, and for all loads, the packet delivery ratio is improved substantially, and SQ-AODV outperforms AODV. This is because there are some nodes whose battery life is limited. Choosing these nodes as intermediate nodes, as is done by AODV, leads to disconnections in the session. SQ-AODV, on the other hand, performas better because, (i) it chooses those intermediate nodes whose energy is sufficient to support the session for its entire durations or it chooses those intermediate nodes which maximize the life-time of the route, and (ii) its make-before-break strategy that re-routes a session proactively when a  link failure due to depletion of node energy is imminent. Thus, SQ-AODV is successful in reducing packet drops in the network drastically. Additionally, in SQ-AODV, the traffic of a source is stopped just as a destination is about to drain, leading to saving of network resources, by not transmitting packets that would, in any case, not be used by the destination.



\section{Validation of MDR Implementation}
\label{MDR-verification}
In this section we give a brief introduction to the Minimum Drain Rate (MDR)~\cite{mdr} routing protocol and its implementation in NS-2~\cite{NS-2}. We also present the simulation results to demonstrate the correctness of our implementation.
Since MDR is an energy-aware routing protocol with a system model and operation quite similar to that of SQ-AODV, we have chosen MDR for comparison with SQ-AODV.

\subsection{MDR Implementation}
\label{mdr-implementation}
\textit{Minimum drain rate} is basically a mechanism used to select a path between a source and destination that maximizes the life-time of a route. In MDR the life-time of a path is dictated by the life-time of the bottleneck node along the path. The life-time of a node is calculated using the current Average-Energy-Drain-Rate (AEDR). Each node calculates the Energy-Drain-Rate (EDR), by calculating the energy consumed by the node for the last $T$ seconds ($T=6 \ seconds$ as specified by MDR authors in Section 3 of~\cite{mdr}). This computed EDR is averaged using exponential weighted moving average (with $\alpha=0.3$ as specified by MDR authors in Section 3 of~\cite{mdr}).

Let $DR_{i}{(t)}$ be the drain rate of node $i$ at time $t$ and $E_{r}^{i}{(t)}$ is the residual battery power of node $i$ at time $t$. Thus, the life-time of a path is determined by the minimum $T_{r}^{i}{(t)}$ along that path, where,
\begin{equation*}
T_{r}^{i}{(t)} = \frac{E_{r}^{i}{(t)}}{DR_{i}{(t)}}. 
\end{equation*}
Thus, the Minimum Drain Rate (MDR) mechanism selects the route with maximum life-time.

We have used AODV~\cite{aodv} as the underlying routing protocol and made necessary modifications for our MDR implementation. In~\cite{mdr}, authors have used Dynamic Source Routing (DSR) as underlying routing protocol for implementation of MDR, however authors claim that underlying routing protocol does not make a difference in the performance of the MDR scheme. Since both DSR and AODV are on-demand routing protocols and we have already used AODV for implemention of SQ-AODV, we decided to use AODV for MDR implementation.

For MDR routing protocol implementation, we have modified the RREQ message of AODV to carry the bottleneck life-time information of the path. As the RREQ message travels from source to destination, the bottleneck life-time field of the RREQ messages is updated. The destination node waits either for \textbf{$0.25 \ seconds$} after the first RREQ receiption or for the receiption of 3 RREQs, and finally reply to the RREQ that maximizes the life-time of the path. MDR updates its routes every $10 \ seconds$ to maintain up-to-date routing information. Hence a source node keep updating the routes by periodic route discoveries. We have simulated the AODV-based MDR, and present the simulation setup and the results in the Section~\ref{setup-mdr} and Section~\ref{results-mdr} respectively.

\subsection{Simulation Setup}
\label{setup-mdr}
We consider the 49-node static topology (where there is no node mobility) with 12 sessions as shown in Fig.~\ref{fig57} for our simulations. This is the same dense network scenario considered in~\cite{mdr}. The nodes are distributed uniformly in an area of size 540 m x 540 m, and are identical in their capability, but are initialized with different energies. The source and destinations have higher initial energy than other nodes, this is to make sure that the communication between source and destination starts.

\begin{figure}[htbp]
	\centering
	\includegraphics[scale=0.5]{fig9_A.eps}
	\caption{49 Node Topology with 12 Sessions}
	\label{fig57}
\end{figure}

The MAC layer protocol is IEEE 802.11 DCF (Distributed Co-ordination Function) with the PLCP (Physical Layer Convergence Protocol)  data rate being 1 Mbps. The parameters used for these simulations are listed in Table~\ref{tab54}. The data traffic in all the sessions is CBR, with data packets at each node arriving at 3 packets/sec and all the sessions starting at the start of the simulation. Initial energy of the source/destinations are chosen randomly between 50 to 75 Joules and that of intermediate nodes are selected between 25 to 50 Joules. The simulation was run 50 times (each initialized with a different seed), and the resulting parameters are averaged over these 50 runs.

\begin{table}[htbp]
\centering
  \caption{Values of Parameters Used for MDR Verification}
  \hfill \\	
  \begin{tabular}{|c|c|}
  \hline
  Packet size & 512 Bytes \\
  \hline
  Simulation time & 800 sec \\
  \hline
  Date traffic & CBR with 3 pkts/sec\\
  \hline
  MAC Protocol & IEEE 802.11 DCF \\
  \hline
  PCLP Data rate & 1 Mbps \\
  \hline
  Buffer length & 50 Packets \\
  \hline
  $P_{t}consume$ & 0.2818 W \\
  \hline
  $P_{r}consume$ & 0.2818 W \\
  \hline
  Propagation model & Two-Ray Ground \\
  \hline
  \end{tabular}
  \label{tab54}
\end{table}

\subsection{Results and Discussions}
\label{results-mdr}
\begin{figure}[htbp]
	\centering
	\includegraphics[scale=0.75]{mdr_paper_node_expiry.eps}
	\caption{Node Expiration Time}
	\label{fig58}
\end{figure}

\begin{figure}[htbp]
	\centering
	\includegraphics[scale=0.75]{mdr_paper_connection_expiry.eps}
	\caption{Connection Expiration Time}
	\label{fig59}
\end{figure}

\begin{figure}[htbp]
	\centering
	\includegraphics[scale=0.2]{mdr_paper_result1.eps}
	\caption{Node Expiration Time~\cite{mdr}}
	\label{fig510}
\end{figure}

\begin{figure}[htbp]
	\centering
	\includegraphics[scale=0.2]{mdr_paper_result2.eps}
	\caption{Connection Expiration Time~\cite{mdr}}
	\label{fig511}
\end{figure}

The results of Node Expiration Time and Connection Expiration Time, generated with our implementation of MDR routing protocol are plotted in Figure~\ref{fig58} and Figure~\ref{fig59} respectively and Figure~\ref{fig510} and Figure~\ref{fig511} are the results from the MDR paper~\cite{mdr} for Node Expiration Time and Connection Expiration Time respectively with * sign as sample points. Although authors of MDR paper~\cite{mdr} have not given enough information to duplicate their results, we have reverse engineered the network parameters to get close to their results. We have simulated an AODV-based MDR and the plots in Figure~\ref{fig58} and Figure~\ref{fig59} are qualitatively similar to those in origional MDR paper, this verifies the behavior of our implementation of MDR is similar to that of original MDR paper.

\section{Performance Comparison of SQ-AODV, MDR and AODV}
\label{comp-sq-aodv-mdr-and-aodv}
In this section, we present the simulation results to compare the performance of the three protocols SQ-AODV, MDR and AODV. For performance comparison, we have conducted two set of experiments, \textbf{Set A} and \textbf{Set B}. Set A is designed to evaluate the overall performance of SQ-AODV, MDR and AODV, for CBR traffic sources, while Set B is designed to evaluate the overall performance SQ-AODV, MDR and AODV, for Poisson traffic sources and at {\it varying network loads.} We present the results of these 2 sets along with the simulation setup in the next section.

\subsection{Simulation Setup for Set A}

We consider the 49-node static topology (where there is no node mobility) with 12 sessions as shown in Figure~\ref{fig57}, This is the same dense network scenario considered in~\cite{mdr}. The nodes are distributed uniformly in an area of size 540 m x 540 m, and are identical in their capability, but 
are initialized with different energies.

\begin{table}[htbp]
 \centering
  \caption{Values of Parameters Used for Simulation Set A} 
  \hfill \\
  \begin{tabular}{|c|c|}
  \hline
  Packet size & 512 Bytes\\
  \hline
  Simulation time & 800 seconds\\
  \hline
  Date traffic & CBR with 3 pkts/sec\\
  \hline
  MAC Protocol & IEEE 802.11 DCF\\
  \hline
  PCLP Data rate & 1 Mbps\\
  \hline
  Buffer length & 50 Packets\\
  \hline
  Transmit power & 0.2818 W\\
  \hline
  Propagation model & Two-Ray Ground\\
  \hline
  \end{tabular}
  \label{tab55}
\end{table}

Simulation Set A involves 2 experiments. In the first, all sessions begin transmission at the start of the simulation, and the simulation runs for a fixed duration (800 s). In the second, the session start times are chosen 
randomly. In both experiments, the initial energy of the nodes is uniformly distributed between 25 J and 100 J, and data at each node arrives at
3 pkts/sec or about 12 Kbps. Every experiment was run 50 times (each initialized with a different seed), and the resulting parameters averaged over these 50 runs. The network parameters used in simulation Set A are detailed in Table~\ref{tab55}.

\subsection{Results and Discussions for Set A}
\label{results-seta}

The results of the two experiments from simulation Set A are presented in Table~\ref{tab56}, while the plots of NET and CET are presented in Figs.~\ref{fig512} -~\ref{fig515}

\begin{table}[htbp]
\centering
\caption{Simulation Results for Set A}
\hfill \\
\begin{tabular}{|c|c|c|c|c|c|c|}
\hline
Parameter  & \multicolumn{3}{c|}{Set-A(1)} & \multicolumn{3}{c|}{Set-A(2)} \\
\        & SQ-AODV & MDR & AODV & SQ-AODV & MDR & AODV  \\
\hline
PDR            & 0.9760   & 0.8456   & 0.8681 & 0.9892   & 0.9201   & 0.8926 \\
\hline
COH            & 0.7742   & 13.3207  & 1.0877 & 0.3402   & 4.1554   & 0.8256 \\
\hline
PD (sec)       & 0.0618   & 0.2429   & 0.0543 & 0.0348   & 0.0508   & 0.0353 \\
\hline
\end{tabular}
\label{tab56}
\end{table}

We see from Table~\ref{tab56} that the PDR for SQ-AODV in the two experiments is improved by 12.5\% and 10.8\%, respectively, relative to AODV. This is because choosing nodes with limited battery life, as happens in AODV, leads to (avoidable) disconnections of sessions. SQ-AODV, on the other hand, performs better because: (i) it chooses those intermediate nodes whose energy is sufficient to support the session for its entire duration or it chooses nodes to maximize the life-time of the route, and (ii) due to its make-before-break strategy, which re-routes a session proactively when link failure due to depletion of node energy is imminent. Thus, SQ-AODV successfully reduces packet drops in the network quite significantly.

Similarly, the PDR in MDR in the two cases is poorer by 15.4\% and 7.5\%, respectively as compared to SQ-AODV. This is because MDR's periodic route update feature adds substantial routing overhead in the network. In fact,  Table~\ref{tab56} shows that MDR overhead is approximately 17 times and 12 times worse than that of SQ-AODV, respectively, and  almost 12 times and 5 times worse than that of AODV, respectively. This leads to its much lower PDR.

The packet delay for both SQ-AODV and AODV is comparable in both cases. We posit that this is because the delay in SQ-AODV is the result of two opposing factors. On the one hand, finding stable routes, where the life-time of the bottleneck node is maximized, may lead to longer (but more stable) routes,
thus increasing delay. On the other hand, proactive route maintenance by way of make-before-break decreases delay, since no retransmissions need occur while an alternative route is located. These two factors have a compensatory effect, making the packet delay in SQ-AODV of the same order as that in AODV. Results in Section~\ref{results-setb} (Table~\ref{tab58}) demonstrate the compensatory effects which makes the delay in SQ-AODV and AODV comparable. MDR, by contrast, imposes a much higher load on the network due to its periodic route updates making the data packets to wait longer, leading to a delay that is about 4 times and 1.5 times, respectively, the delay for AODV or SQ-AODV. 

\begin{figure}[htbp]
	\centering
	\includegraphics[scale=0.75]{res1_node_expiry.eps}
	\caption{NET: Sessions Commence at Start of Simulation}
	\label{fig512}
\end{figure}


\begin{figure}[htbp]
	\centering
	\includegraphics[scale=0.75]{res2_node_expiry.eps}
	\caption{NET: Random Session Start Times}
	\label{fig513}
\end{figure}

We see from Figs.~\ref{fig512} and~\ref{fig513} that in our network setting, running SQ-AODV improves the node expiration time by between 25 to 100 seconds over AODV, and by between 100 to 150 seconds over MDR. In other words, for a given number of nodes alive, this equates to SQ-AODV extending the node life-time by between 10\%-25\% over AODV, and by between 25\%-35\% over MDR. Viewed another way, at a given simulation time, SQ-AODV typically has between 10\%-25\% more nodes alive than does AODV, and has between 20\%-60\% more
nodes alive than does MDR. This is because SQ-AODV's proactive route maintenance is very economical of node energy. In addition, due to the proactive mechanism in SQ-AODV, a source stops transmitting traffic if a destination is about to drain, which saves resources by minimizing the transmission of packets that would not have been received by the destination in any case (due to its expiring). The nodes in MDR, on the other hand, expire 
faster than they do in either AODV or SQ-AODV by a significant margin, this is because the periodic updates of MDR consume energy at a substantially
faster rate, causing nodes to expire much quicker, as our results demonstrate.

\begin{figure}[htbp]
	\centering
	\includegraphics[scale=0.75]{res1_connection_expiry.eps}
	\caption{CET: Sessions Commence at Start of Simulation}
	\label{fig514}
\end{figure}

\begin{figure}[htbp]
	\centering
	\includegraphics[scale=0.75]{res2_connection_expiry.eps}
	\caption{CET: Random Session Start Times}
	\label{fig515}
\end{figure}

Figs.~\ref{fig514} and~\ref{fig515} illustrate that in our network setting, in terms of CET, AODV performs better by about 10-50 seconds over both SQ-AODV and MDR. (Note that the x-axis in these figures simply indicates the {\it number of connections} that have expired, and is {\it not} the connection identifier . Thus, the connections that expire under each protocol can be different. So, for example, Fig.~\ref{fig514} shows that 6 connections expire at approximately 200 seconds with AODV, at 180 seconds with SQ-AODV, and at 165 seconds with MDR. However, the 6 sessions that expire under each protocol
are not the same sessions).

This equates to AODV connection expiration times being anywhere between 30\%-7\% better than those of SQ-AODV or MDR. This is because, in SQ-AODV: (i) a source on receiving an RCR from an intermediate node tries only a fixed (but configurable; in our case 3) times before it reaches the maximum number of
retries and ends the session, and (ii) intermediate nodes reject a new session once its residual-energy is bellow Threshold-1. By contrast AODV keeps retrying and so has a higher probability of finding a path, and keeping the session alive for longer. In the case of MDR, however, it is the control overhead packets that cause the node energy to drain faster, leading to the sessions expiring quicker than with AODV or SQ-AODV. We note that the slightly higher connection expiration times in AODV do come at the cost of lower PDR and lower node expiration times, which implies that even though the connections may be alive for a longer period in AODV, they do not successfully transmit as much data as SQ-AODV does.

\subsection{Simulation Setup for Set B}

We consider the same 49-node static topology (where there is no node mobility) with 12 sessions as shown in Figure~\ref{fig57} for Set B. The nodes are distributed uniformly in an area of size 540 m x 540 m, and are identical in their capability, but are initialized with different energies. In this set The traffic arrives as per a Poisson process, for different network loads. The initial node energies are uniformly distributed between 75 J and 300 J, and the simulation is run until each session has transmitted 3000 packets, while the session data rates vary from 15 Kbps to 65 Kbps. Here we again examine performance by comparing PDR, control overhead (COH) and packet delay (PD) at {\it{different network loads}}.

\begin{table}[htbp]
 \centering
  \caption{Parameters and Their Values Used for Simulation Set B} 
  \hfill \\
  \begin{tabular}{|c|c|}
  \hline
  Packet size & 512 Bytes\\
  \hline
  Packets/Session & 3000 \\
  \hline
  Date traffic & Poisson with $\lambda$ = 15 Kbps - 65Kbps\\
  \hline
  MAC Protocol & IEEE 802.11 DCF\\
  \hline
  PCLP Data rate & 1 Mbps \\
  \hline
  Buffer length & 50 Packets \\
  \hline
  Transmit power & 0.2818 W\\
  \hline
  Propagation model & Two-Ray Ground\\
  \hline
  \end{tabular}
  \label{tab57}
\end{table}

Every experiment was run 50 times (each initialized with a different seed), and the resulting parameters averaged over these 50 runs. The network parameters used in simulation Sets A and B are detailed in Table~\ref{tab57}.

\subsection{Results and Discussions for Set B}
\label{results-setb}
We observe from Fig.~\ref{fig516}, the PDR of SQ-AODV is substantially better than that for AODV or MDR. In fact, the PDR for SQ-AODV is improved by between 25\%-13\% over AODV and by between 22\%-18\% over MDR over the network loads considered. The key reason for this are  the two properties of SQ-AODV discussed in Chapter~\ref{intro-sq-aodv}, which induce stable routes for the sessions and bolster PDR.

\begin{figure}[htbp]
	\centering
	\includegraphics[scale=0.75]{new-avg-pdr-plot.eps}
	\caption{Average Packet Delevery Ratio for Simulation Set B}
	\label{fig516}
\end{figure}

The PDR of MDR is better than that of AODV by 5\%-10\% at lower loads, but is reduced by equal amount 
at higher loads because of its extra overhead, which degrades MDR performance at higher network loads.

\begin{figure}[htbp]
	\centering
	\includegraphics[scale=0.75]{new-avg-coh-plot.eps}
	\caption{Average Control Overhead for Simulation Set B}
	\label{fig517}
\end{figure}

Fig.~\ref{fig517} shows that SQ-AODV has marginally higher normalized control overhead (between 1\%-3\% higher) than AODV. This is because, as explained in Chapter~\ref{intro-sq-aodv}, to support stable routing, SQ-AODV uses per-session (or per-flow) based routing (as opposed to simple destination-based routing used in AODV). For this, control packets of SQ-AODV carry extra flow-id information along with source and destination, and also packets need travel all the way to destination to find a stable path, leading to a marginally higher control overhead.

We see that MDR has the highest control overhead, almost 300\% higher than either AODV or SQ-AODV at loads above 35 Kbps, rising to over 1000\% higher at lower loads. This is because of the control overhead of MDR. In particular,
at lower loads the control overhead of MDR becomes very high, because it takes substantial time for the sources to generate 3000 packets. In the meantime, the regular  periodic update procedures of MDR continue accumulating significant control overhead.

\begin{figure}[htbp]
	\centering
	\includegraphics[scale=0.75]{new-pkt-delay-plot.eps}
	\caption{Average Packet Delay for Simulation Set B}
	\label{fig518}
\end{figure}

Finally Fig.~\ref{fig518} illustrates that, the delay experienced by packets in SQ-AODV is almost the same or marginally better than that in AODV, at all loads under consideration. This is because the delay in SQ-AODV is the result of two opposing factors. On the one hand, finding stable routes, where the life-time of the bottleneck node is maximized, may lead to longer (but more stable) routes, thus increasing delay. On the other hand, proactive route maintenance by way of make-before-break decreases delay, since no retransmissions need occur while an alternative route is located. These two factors have a compensatory effect, making the packet delay in SQ-AODV of the same order as that in AODV. The delay experienced in MDR is between 250-500 ms higher, or between 20\%-50\% higher than that in AODV and SQ-AODV because, at higher loads data packets have to wait longer due to periodic route updates.
The advantage with SQ-AODV is that it is designed to provide stable routes and a fast re-routing capability to the nodes in ad-hoc networks at minimum overhead to the network. This helps in making effective use of network resources, as demonstated by our simulation results.

\begin{table}[htbp]
\centering
\caption{Average Number of Hops for Simulation Set B}
\hfill \\
\begin{tabular}{|c|c|c|c|c|c|c|}
\hline
Protocol & \multicolumn{6}{c|}{Session Data Rate}\\
\	 & 15 Kbps & 25 Kbps & 35 Kbps & 45 Kbps & 55 Kbps & 65 Kbps\\
\hline
SQ-AODV  & 4.9503 & 5.0165 & 5.0519 & 5.0487 & 5.0938 & 5.2244 \\
\hline
MDR      & 5.0147 & 5.0258 & 5.0800 & 5.1616 & 5.0861 & 5.1326 \\
\hline
AODV     & 4.5294 & 4.4972 & 4.5858 & 4.6581 & 4.6984 & 4.7311 \\
\hline
\end{tabular}
\label{tab58}
\end{table}


Table~\ref{tab58} gives the number of hops taken by the data packets to travel from source to destination in SQ-AODV, MDR and AODV, on an average. As we can see the number of hops taken by SQ-AODV and MDR is increased by about 11\% over AODV on an average over all loads. This should clearly indicate that the packet delay in case of SQ-AODV would have been more, but the proactive route maintenance by way of make-before-break decreases delay in SQ-AODV, leading to an almost similar delay in both SQ-AODV and AODV over all loads. On the other hand for MDR, the data packets experience a larger delay compared to both SQ-AODV and AODV as explained earlier in Section~\ref{results-seta}.