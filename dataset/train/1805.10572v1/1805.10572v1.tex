\documentclass{article}





\usepackage[preprint,nonatbib]{nips_2018}





\usepackage[utf8]{inputenc} \usepackage[T1]{fontenc}    \usepackage{hyperref}       \usepackage{url}            \usepackage{booktabs}       \usepackage{amsfonts}       \usepackage{nicefrac}       \usepackage{microtype}      \usepackage{xspace}
\usepackage{amsmath}
\usepackage{enumitem}
\usepackage{graphicx}
\usepackage{multirow}
\usepackage{color}
\usepackage{longtable}
\newcommand{\tabincell}[2]{\begin{tabular}{@{}#1@{}}#2\end{tabular}}

\newcommand{\ritsi}{\xspace{RITS-I}}
\newcommand{\britsi}{\xspace{BRITS-I}}
\newcommand{\rits}{\xspace{RITS}}
\newcommand{\brits}{\xspace{BRITS}}
\newcommand{\methodname}{\xspace{BRITS}}

\newtheorem{definition}{Definition}
\newtheorem{example}{Example}


\newcommand{\x}{{\mathbf x}}
\newcommand{\y}{{\mathbf y}}
\newcommand{\z}{{\mathbf z}}
\newcommand{\h}{{\mathbf h}}
\newcommand{\m}{{\mathbf m}}
\newcommand{\bias}{{\mathbf b}}
\newcommand{\X}{{\mathbf X}}
\newcommand{\W}{{\mathbf W}}
\newcommand{\Ux}{{\mathbf U}}

\newcommand{\red}[1]{\textcolor{red}{#1}}

\title{\methodname: Bidirectional Recurrent Imputation for Time Series}

\author{
Wei Cao \\
\And
Dong Wang \\
\And
Jian Li\\
\And
Hao Zhou\\
\And
Lei Li\\
\And
Yitan Li\\
\AND
Tsinghua University \quad Duke University \quad  Toutiao AILab
}


\begin{document}


\maketitle

\begin{abstract}

Time series are widely used as signals in many classification/regression tasks. It is ubiquitous that time series contains many missing values. Given multiple correlated time series data, how to fill in missing values and to predict their class labels? 
Existing imputation methods often impose strong assumptions of the underlying data generating process, such as linear dynamics in the state space. 
In this paper, we propose \methodname, a novel method based on recurrent neural networks for missing value imputation in time series data. 
Our proposed method directly learns the missing values in a bidirectional recurrent dynamical system, without any specific assumption. The imputed values are treated as variables of RNN graph and can be effectively updated during the backpropagation.
BRITS has three advantages: (a) it can handle multiple correlated missing values in time series; (b) it generalizes to time series with nonlinear dynamics underlying; (c) it provides a data-driven imputation procedure and applies to general settings with missing data.
We evaluate our model on three real-world datasets, including an air quality dataset, a health-care data, and a localization data for human activity.
Experiments show that our model outperforms the state-of-the-art methods in both imputation and classification/regression accuracies. 

\end{abstract} \section{Introduction}
\label{sec:introduction}

Multivariate time series data are abundant in many application areas, such as financial marketing ~\cite{bauer2016arrow,batres2015deep}, health-care~\cite{che2018recurrent,liu2016learning}, meteorology~\cite{xingjian2015convolutional,rani2012recent},
and traffic engineering~\cite{wang2017deepsd,zhang2017deep}.
Time series are widely used as signals for classification/regression in  various applications of corresponding areas.
However, missing values in time series are very common, due to unexpected accidents, such as equipment damage or communication error, and may significantly harm the performance of downstream applications.



Much prior work proposed to fix the missing data problem with statistics and machine learning approaches. However most of them require fairly strong assumptions on missing values.
We can fill the missing values using classical statistical time series models such as ARMA or ARIMA (e.g., \cite{ansley1984estimation}). But these models are essentially 
linear (after differencing).
Kreindler et al. \cite{kreindler2012effects} assume that the data are smoothable, i.e., there is no sudden wave in the periods of missing values,
hence imputing missing values can be done by smoothing over nearby values. 
Matrix completion has also been used to address missing value problems (e.g.,  \cite{wang2006unifying,yu2016temporal}). But it typically applies to only static data and requires strong assumptions such as low-rankness.
We can also predict missing values by fitting a parametric data-generating model with the observations~\cite{fung2006methods,azur2011multiple}, which assumes that data of time series follow the distribution of hypothetical models.
These assumptions make corresponding imputation algorithms less general, and the performance less desirable when the assumptions do not hold.







In this paper, we propose \methodname, a novel method for filling the missing values for multiple correlated time series. 
Internally, \methodname ~adapts recurrent neural networks~(RNN)~\cite{hochreiter1997long,cho2014learning} for imputing missing values, without any specific assumption over the data.
Much prior work uses non-linear dynamical systems for time series prediction~\cite{brakel2013training,ozaki19852,basharat2009time}.
In our method, we instantiate the dynamical system as a bidirectional RNN, i.e., imputing missing values with bidirectional recurrent dynamics.
In particular, we make the following technical contributions: 
\begin{itemize}
\item 
We design a bidirectional RNN model for imputing missing values.
We directly use RNN for predicting missing values, instead of tuning weights for smoothing as in \cite{che2018recurrent}. 
Our method does not impose specific assumption, hence works more generally than previous methods.
\item 
We regard missing values as variables of the bidirectional RNN graph, which are involved in the backpropagation process. 
In such case, missing values get delayed gradients in both forward and backward directions with consistency constraints, which makes the estimation of missing values more accurate.
\item We simultaneously perform missing value imputation and classification/regression of applications jointly in one neural graph.
This alleviates the error propagation problem from imputation to classification/regression. 
Additionally, the supervision of classification/regression makes the estimation of missing values more accurate.
\item We evaluate our model on three real-world datasets, including an air quality dataset, a health-care dataset and a localization dataset of human activities. Experimental results show that our model outperforms the state-of-the-art models for both imputation and classification/regression accuracies.


\end{itemize}

%
 \section{Related Work}
\label{sec:related_work}
There is a large body of literature on the imputation of missing values in time series. We only mention a few closely related ones. The {\em interpolate method} tries to fit a "smooth curve" to the observations and thus reconstruct the missing values by the local interpolations \cite{kreindler2012effects,fung2006methods}.  Such method discards any relationships between the variables over time. The {\em autoregressive method}, including ARIMA, SARIMA etc., eliminates the non-stationary parts in the time series data and fit a parameterized stationary model. The {\em state space model} further combines  ARIMA  and  Kalman Filter \cite{fung2006methods,harvey1990forecasting}, which provides more accurate results. {\em  Multivariate Imputation by Chained Equations} (MICE) \cite{azur2011multiple} first initializes the missing values arbitrarily and then estimates each missing variable based on the chain equations. The graphical model \cite{li2009dynammo} introduces a latent variable for each missing value, and finds the latent variables by learning their transition matrix.
There are also some data-driven methods for missing value imputation. Yi et al. \cite{yi2016st} imputed the missing values in air quality data with geographical features. Wang et al. \cite{wang2006unifying} imputed  missing values in recommendation system with collaborative filtering. Yu et al. \cite{yu2016temporal} utilized matrix factorization with temporal regularization to impute the missing values in regularly sampled time series data.

Recently, some researchers attempted to impute the missing values with recurrent neural networks \cite{berglund2015bidirectional,che2018recurrent,lipton2016directly,choi2016doctor,yoon2017multi}. The recurrent components are trained together with the classification/regression component, which significantly boosts the accuracy.  Che et al. \cite{che2018recurrent} proposed GRU-D, which imputes missing values in health-care data with a smooth fashion. It assumes that a missing variable can be represented as the combination of its corresponding last observed value and the global mean. GRU-D achieves remarkable performance on health-care data. However, it has many limitations on general datasets \cite{che2018recurrent}. A closely related work is M-RNN proposed by Yoon et al. \cite{yoon2017multi}. M-RNN also utilizes bi-directional RNN to impute missing values. Differing from our model, it drops the relationships between missing variables. The imputed values in M-RNN are treated as constants and cannot be sufficiently updated.

%
 \section{Preliminary}

We first present the problem formulation and some necessary preliminaries.

\begin{definition}[Multivariate Time Series] We denote a multivariate time series  as a sequence of  observations. The -th observation  consists of  features , and was observed at timestamp   (the time gap between different timestamps may  be not same).  In reality, due to unexpected accidents, such as equipment damage or communication error,  may have the missing values (e.g., in Fig. \ref{fig:time_series},  in  is missing). To represent the missing values in , we introduce a {\em masking vector}  where, 

In many cases, some features can be missing for consecutive timestamps (e.g., blue blocks in Fig. \ref{fig:time_series}). We define  to be the time gap from the last observation to the current timestamp , i.e., 


See Fig. \ref{fig:time_series} for an illustration.
\end{definition}


\begin{figure}[ht]
\centering
\includegraphics[width=0.95\textwidth]{./time_series_verticle.png}
\caption{An example of multivariate time series with missing values.  to  are observed at  respectively. Considering the nd feature in , the last observation of the nd feature took place at , and we have that .}
\label{fig:time_series}
\end{figure}


In this paper, we study a general setting for time series classification/regression problems with missing values. We use  to denote the label of corresponding classification/regression task. In general,  can be either a scalar or a vector. Our goal is to predict  based on the given time series . In the meantime, we impute the missing values in  as accurate as possible. In another word, we aim to design an effective multi-task learning algorithm for both classification/regression and imputation.
 \section{\methodname}
\label{sec:feature_independent}

In this section, we describe the \methodname.  Differing from the prior work which uses RNN to impute missing values in a smooth fashion \cite{che2018recurrent}, we learn the missing values directly in a recurrent dynamical system \cite{pascanu2013difficulty,sussillo2013opening} based on the observed data. The missing values are thus imputed according to the recurrent dynamics, which significantly boosts both the imputation accuracy and the final classification/regression accuracy.
We start the description with the simplest case: the variables observed at the same time are mutually uncorrelated. For such case, we propose algorithms for imputation with {\em unidirectional recurrent dynamics} and {\em bidirectional recurrent dynamics}, respectively. We further propose an algorithm for correlated multivariate time series in Section \ref{sec:feature_dependent}.

\subsection{Unidirectional Uncorrelated Recurrent Imputation}
\label{sec:uni_impute}
For the simplest case, we assume that for the -th step,  and  are uncorrelated if  (but  may be correlated with some ). We first propose an imputation algorithm by unidirectional recurrent dynamics, denoted as {\bf RITS-I}.

In a {\em unidirectional recurrent dynamical system}, each value in the time series can be derived by its predecessors with a fixed {\em arbitrary} function \cite{brakel2013training,ozaki19852,basharat2009time}. Thus, we iteratively impute all the variables in the time series according to the  recurrent dynamics. For the -th step, if  is actually observed, we use it to validate our imputation and pass  to the next recurrent steps. Otherwise, since the future observations are correlated with the current  value, 
we replace  with the obtained imputation, and validate it by the future observations. To be more concrete, let us consider an example.


\begin{example}
\label{example:forward}
Suppose we are given a time series , where  and  are missing.  According to the recurrent dynamics, at each time step , we can  obtain an estimation  based on  the previous  steps. In the first  steps, the estimation error can be obtained immediately by calculating the estimation loss function  for . However, when , we cannot calculate the error immediately since the true values are missing. Nevertheless, note that  at the -th step,  depends on  to . We thus obtain a ``delayed" error for  at the -th step.
\end{example}


\begin{figure}[t!]
\centering
\includegraphics[width=0.7\textwidth]{./gradient.jpg}
\caption{Imputation with unidirectional dynamics.}
\label{fig:backpropagate}
\end{figure}



\subsubsection{Algorithm}
We introduce a recurrent component and a regression component for imputation. The recurrent component is achieved by a recurrent neural network and the regression component is achieved by a fully-connected network.
A standard recurrent network \cite{DeepBook} can be represented as 

where  is the  function, ,  and  are parameters, and  is the hidden state of previous time steps.

In our case, since  may have missing values, we cannot use  as the input directly as in the above equation. Instead,  we use a ``complement" input  derived by our algorithm when  is missing. Formally,
we  initialize the initial hidden state  as an all-zero vector and then update the model by: 


Eq. \eqref{eq1} is the regression component which transfers the hidden state  to the estimated vector . In Eq. \eqref{eq2}, we replace missing values in  with corresponding values in , and obtain the complement vector . Besides, since the time series may be irregularly sampled, in Eq. \eqref{eq4}, we further introduce a {\em temporal decay factor}  to decay the hidden vector . Intuitively, if  is large (i.e., the values are missing for a long time), we expect a small  to decay the hidden state.
Such factor also represents the missing patterns in the time series which is critical to imputation \cite{che2018recurrent}. In Eq. \eqref{eq3}, based on the decayed hidden state, we predict the next state . Here, 
 indicates the concatenate operation.
In the mean time, we calculate the estimation error by the estimation loss function  in Eq. \eqref{eq5}. In our experiment, we use the {\em mean absolute error} for . Finally, we predict the task label  as
 where  can be either a fully-connected layer or a softmax layer depending on the specific task, and  is the weight for different hidden state which can be derived by the {\em attention mechanism}\footnote{The design of attention mechanism is out of this paper's scope.}  or the {\em mean pooling mechanism}, i.e., . We calculate the output loss by  . Our model is then updated by minimizing the accumulated loss .

\subsubsection{Practical Issues}


In practice, we use LSTM as the recurrent component in Eq. \eqref{eq3} to prevent the gradient vanishing/exploding problems in vanilla RNN~\cite{DeepBook}. Standard RNN models including LSTM treat   as a constant. During backpropagation, gradients are cut at . This makes the estimation errors backpropagate insufficiently.
For example, in Example \ref{example:forward}, the estimation errors of  to  are obtained at the -th step as the delayed errors. However, if we treat  to  as constants, such delayed error cannot be fully backpropagated. To overcome such issue, we treat  as a variable of RNN graph. We let the estimation error passes through  during the backpropagation.
Fig. \ref{fig:backpropagate} shows how \ritsi~ method works in Example \ref{example:forward}. In this example, the gradients are backpropagated through the opposite direction of solid lines. Thus, the delayed error  is passed to steps  and . In the experiment, we find that our models are unstable during training if we treat  as a constant. 
See Appendix \ref{appendix:cut} for details.




\subsection{Bidirectional Uncorrelated Recurrent Imputation}
\label{sec:bi-direct}
In the \ritsi, errors of estimated missing values are delayed until the presence of the next observation.
For example, in Example \ref{example:forward}, the error of  is delayed until  is observed. Such error delay makes the model converge slowly and in turn leads to inefficiency in training.  
In the meantime, it also leads to the {\em bias exploding} problem \cite{bengio2015scheduled}, i.e., the mistakes made early in the sequential prediction are fed as input to the model and may be quickly amplified. In this section, we propose an improved version called {\bf \britsi}. The  algorithm alleviates the above-mentioned issues by utilizing the {\em bidirectional recurrent dynamics} on the given time series, i.e., besides the forward direction, each value in time series can be also derived from the backward direction by another fixed arbitrary function.

To illustrate the intuition of \britsi~ algorithm, again, we consider Example \ref{example:forward}. Consider the backward direction of the time series. In bidirectional recurrent dynamics, the estimation  reversely depends on  to . Thus, the error in the -th step is back-propagated from not only the -th step in the forward direction (which is far from the current position), but also the -th step in the backward direction (which is closer).
Formally, the \britsi~ algorithm performs the \ritsi~ as shown in Eq. \eqref{eq1} to Eq. \eqref{eq5} in forward and backward directions, respectively. In the forward direction, we obtain the estimation sequence  and the loss sequence . Similarly, in the backward direction, we obtain another estimation sequence  and another loss sequence . We enforce the prediction in each step to be consistent in both directions by introducing the ``{\em consistency loss}'':

where we use the {\em mean squared error} as the discrepancy in our experiment. The final estimation loss is obtained by accumulating the forward loss , the backward loss  and the consistency loss . The final estimation in the -th step is the mean of  and .

















 \subsection{Correlated Recurrent Imputation}
\label{sec:feature_dependent}
The previously proposed algorithms \ritsi~ and \britsi~ assume that features observed at the same time are mutually uncorrelated.  This may  be not true in some applications. For example, in the air quality data \cite{yi2016st}, each feature represents one measurement in a monitoring station. Obviously,  the observed measurements are spatially correlated. In general, the measurement in one monitoring station is close to those values observed in its neighboring stations. In this case, we can estimate a missing measurement according to both its historical data, and the measurements in its neighbors.

In this section, we propose an algorithm, which utilizes the feature correlations in the unidirectional recurrent dynamical system. We refer to such algorithm as {\bf \rits}. The feature correlated algorithm for bidirectional case (i.e., {\bf \methodname})  can be derived in the same way. Note that in Section \ref{sec:uni_impute}, the estimation  is only correlated with the historical observations, but irrelevant with the the current observation. We refer to  as a ``history-based estimation". In this section, we derive another ``feature-based estimation" for each , based on the other features at time . Specifically, at the -th step, we first obtain the complement observation  by Eq. \eqref{eq1} and Eq. \eqref{eq2}. Then, we define our feature-based estimation as  where 

,  are corresponding parameters.
We restrict the diagonal of parameter matrix  to be all zeros. Thus, the -th element in  is exactly the estimation of , based on the other features. We further combine the historical-based estimation  and the feature-based estimation . We denote the combined vector as , and we have that

Here we use  as the weight of combining the history-based estimation  and the feature-based estimation . Note that  is derived from  by Eq. \eqref{equ:estimate_z}. The elements of  can be history-based estimations or truly observed values, depending on the presence of the observations. Thus, we learn the weight  by considering both the temporal decay  and the masking vector  as shown in Eq. \eqref{equ:beta}.
The rest parts are similar as the feature uncorrelated case. We first replace missing values in  with the corresponding values in . The obtained vector is then fed to the next recurrent step to predict memory :

However, the estimation loss is slightly different with the feature uncorrelated case. We find that simply using  leads to a very slow convergence speed. Instead, we accumulate all the estimation errors of ,  and :


 \section{Experiment}
\label{sec:exp}
Our proposed methods are applicable to a wide variety of applications. We evaluate our methods on three different real-world datasets. The download links of the datasets, as well as the implementation codes can be found in the GitHub page\footnote{\url{https://github.com/NIPS-BRITS/BRITS}}.

\subsection{Dataset Description}





\subsubsection{Air Quality Data}
We evaluate our models on the air quality dataset, which consists of PM2.5 measurements from  monitoring stations in Beijing. The measurements are hourly collected  from 2014/05/01 to 2015/04/30. Overall, there are  values are missing. For this dataset, we do pure imputation task. We use the exactly same train/test  setting as in prior work \cite{yi2016st}, i.e., we use the , ,  and  months as the test data and the other months as the training data. See Appendix \ref{appendix:air} for details. To train our model, we randomly select every  consecutive steps as one time series. 



\subsubsection{Health-care Data}
We evaluate our models on health-care data in {\em PhysioNet Challenge 2012} \cite{silva2012predicting}, which consists of  multivariate clinical time series from intensive care unit (ICU). Each time series contains  measurements such as Albumin, heart-rate etc., which are irregularly sampled at the first  hours after the patient's admission to ICU. We stress that this dataset is extremely sparse. There are up to  missing values in total.
For this dataset, we do both the imputation task and  the classification task. To evaluate the imputation performance, we randomly eliminate  of observed measurements from data and use them as the  ground-truth. At the same time, we predict the in-hospital death of each patient as a binary classification task. Note that the eliminated measurements are only used for validating the imputation, and they are never visible to the model.

\subsubsection{Localization for Human Activity Data}
The UCI localization data for human activity \cite{kaluvza2010agent} contains records of five people performing different activities such as  walking, falling, sitting down etc (there are  activities). Each person wore four sensors on her/his left/right ankle, chest, and belt. Each sensor recorded a 3-dimensional coordinates for about  to   millisecond. We randomly select  consecutive steps as one time series, and there are  time series in total. For this dataset, we do both imputation and classification tasks. Similarly, we randomly eliminate  observed data as the imputation ground-truth. We further predict the corresponding activity of observed time series (i.e., walking, sitting, etc).

\subsection{Experiment Setting}
\label{exp_setting}

\subsubsection{Model Implementations}
We fix the dimension of hidden state  in RNN to be . We train our model by an Adam optimizer with learning rate  and batch size . For all the tasks, we normalize the numerical values to have zero mean and unit variance for stable training. 

We use different early stopping strategies for pure imputation task and classification tasks. For the imputation tasks, we randomly select  of non-missing values as the validation data. The early stopping is thus performed based on the validation error.
For the classification tasks, we first pre-train the model as an imputation task. Then we use -fold cross validation to further optimize both the imputation and classification losses simultaneously.



We evaluate the imputation performance
in terms of {\em mean absolute error} () and {\em mean relative error} (). Suppose that  is the ground-truth of the -th  item,  is the output of the -th item, and 
there are  items in total. Then,  and  are defined as

For air quality data, the evaluation is performed on its original data. For
 heath-care data and activity data, since the numerical values are not in the same scale, we evaluate the performances on their normalized data. To further evaluate the classification performances, we use {\em area under ROC curve} () \cite{bradley1997use} for health-care data, since such data is highly unbalanced (there are  patients who died in hospital). We then use standard accuracy for the activity data, since different activities are relatively balanced.
 
\subsubsection{Baseline Methods}
We compare our model with both RNN-based methods and non-RNN based methods. The non-RNN based imputation methods include:
\begin{itemize}
\item {\bf Mean:} We simply replace the missing values with corresponding global mean.
\item {\bf KNN:} We use -nearest neighbor \cite{friedman2001elements} to find the similar samples, and impute the missing values with  weighted average of its neighbors. 
\item {\bf Matrix Factorization (MF):} We factorize the data matrix into two low-rank matrices, and fill the missing values by matrix completion \cite{friedman2001elements}.

\item {\bf MICE}: We use Multiple Imputation by Chained Equations (MICE), a widely used imputation method, to fill the missing values. MICE creates multiple imputations with chained equations \cite{azur2011multiple}.

\item {\bf ImputeTS}: We use ImputeTS package in R to impute the missing values. ImputeTS is a widely used package for missing value imputation, which utilizes the state space model and kalman smoothing \cite{RJ-2017-009}.

\item {\bf STMVL}: Specifically, we use STMVL for  the air quality data imputation. STMVL is the state-of-the-art method for air quality data imputation. It further utilizes the geo-locations when imputing missing values  \cite{yi2016st}.
\end{itemize}

We implement KNN, MF and MICE based on the python package fancyimpute\footnote{https://github.com/iskandr/fancyimpute}. In recent studies, RNN-based models in missing value imputations achieve remarkable performances \cite{che2018recurrent,lipton2016directly,choi2016doctor,yoon2017multi}. We also compare our model with existing RNN-based imputation methods, including:
\begin{itemize}
\item {\bf GRU-D:} GRU-D is proposed for handling missing values in health-care data. It imputes each missing value by the weighted combination of its last observation, and the global mean, together with a recurrent component \cite{che2018recurrent}.

\item {\bf M-RNN:} M-RNN also uses bi-directional RNN. It imputes the missing values according to hidden states in both directions in RNN. M-RNN treats the imputed values as constants. It does not consider the correlations among different missing values\cite{yoon2017multi}.
\end{itemize}
We compare the baseline methods with our four models: \ritsi~ (Section \ref{sec:uni_impute}), \rits~ (Section \ref{sec:bi-direct}), \britsi~(Section \ref{sec:feature_dependent}) and \brits~(Section \ref{sec:feature_dependent}).
We implement all the RNN-based models with PyTorch, a widely used package for deep learning. All models are trained with GPU GTX 1080.







\subsection{Experiment Results}
\begin{table}[t!]
\centering
\renewcommand\arraystretch{1.2}
\caption{Performance Comparison for Imputation Tasks (in ))}
\label{tab:imputation}
\begin{tabular}{c|c|c|c|c}
\toprule[1.6pt]
\multicolumn{2}{c|}{Method}                                            & Air Quality & Health-care                           & Human Activity                        \\ \hline
\multirow{6}{*}{Non-RNN} & Mean                                         &  ()                           &                    &                     \\ \cline{2-5} 
                         & KNN                                          &  ()                           &                    &                     \\ \cline{2-5} 
                         & MF                                           &  ()                           &                     &                    \\ \cline{2-5} 
                         & MICE                                         &  ()                           &  ()                   &                     \\ \cline{2-5} 
                         & ImputeTS                                     &  ()                           &                      &                     \\ \cline{2-5} 
                         & STMVL                                        &  ()                           & /                                     & /                                     \\ \hline
\multirow{2}{*}{RNN}     & GRU-D                                        & /                                             &                     &                     \\ \cline{2-5} 
                         & M-RNN                                        &  ()                           &                     &                     \\ \hline
\multirow{4}{*}{Ours}    & \ritsi                        &  ()                           &                     &                     \\ \cline{2-5} 
                         & \britsi                       &  ()                           &                     &                     \\ \cline{2-5} 
                         & \rits                         &  ()                           &                     &                     \\ \cline{2-5} 
                         & {\bf \brits} &  ()         &   &   \\ \hline
\end{tabular}
\end{table}

Table \ref{tab:imputation} shows the imputation results. 
As we can see, simply applying na\i ve mean imputation is very inaccurate. Alternatively, KNN, MF, and MICE perform much better than mean imputation. ImputeTS achieves the best performance among all the non-RNN methods, especially for the heath-care data (which is smooth and contains few sudden waves). STMVL performs well on the air quality data. However, it is specifically designed for geographical data, and cannot be applied in the other datasets. Most of RNN-based methods, except GRU-D, demonstrates significantly better performances in imputation tasks. We stress that GRU-D does not impute missing value explicitly. It actually performs well on classification tasks (See Appendix \ref{appendix:classification} for details).
 M-RNN uses explicitly imputation procedure, and achieves remarkable imputation results.  Our model \brits~ outperforms all the baseline models significantly. According to the performances of our four models, we also find that both bidirectional recurrent dynamics, and the feature correlations are helpful to enhance the model performance.

We also compare the accuracies of all RNN-based models in classification tasks.
GRU-D achieves an AUC of  on health-care data , and accuracy  of  on human activity; M-RNN is slightly worse. It achieves  and  on two tasks. Our model \brits~ outperforms the baseline methods. The accuracies of \brits~ are  and  respectively. See Appendix \ref{appendix:classification} for more classification results.

Due to the space limitation, we provide more experimental results in Appendix \ref{appendix:classification} and \ref{appendix:synthetic} for better understanding our model.









%
 
\section{Conclusion}
\label{sec:conclusion}
In this paper, we proposed \methodname, a novel method to use recurrent dynamics to effectively impute the missing values in multivariate time series. Instead of imposing assumptions over the data-generating process, our model directly learns the missing values in a bidirectional recurrent dynamical system, without any specific assumption. Our model treats missing values as variables of the bidirectional RNN graph. Thus, we get the delayed gradients for missing values in both forward and backward directions, which makes the imputation of missing values more accurate. We performed the missing value imputation and classification/regression simultaneously within a joint neural network. Experiment results show that our model demonstrates more accurate results for both imputation and classification/regression than state-of-the-art methods. 
\bibliographystyle{abbrv}
\bibliography{ref}
\newpage
\appendix

 \section{Performance Comparison for Classification Tasks}
 \label{appendix:classification}
 Table \ref{tab:classification} shows the performances of different RNN-based models on the classification tasks. As we can see, our model \brits~ outperforms other baseline methods for classifications. Comparing with Table \ref{tab:imputation}, GRU-D performs well for classification tasks. Furthermore, we find that it is very important for GRU-D to carefully apply the dropout techniques in order to prevent the overfitting (we use  dropout layer on the top classification layer). However, our models further utilize the imputation errors as the supervised signals. It seems that dropout is not necessary for our models during training.
 
  \begin{table}[htbp]
\centering
\renewcommand\arraystretch{1.2}
\caption{Performance Comparison for Classification Tasks}
\label{tab:classification}
\begin{tabular}{c|c|c}
\hline
Method                 & Health-care  (AUC)     & Human Activity (Accuracy)    \\ \hline
GRU-D                  &  &  \\ \hline
M-RNN                  &  &  \\ \hline\hline
\ritsi  &  &  \\ \hline 
\britsi &  &  \\ \hline
\rits   &  &  \\ \hline
{\bf \brits}  &  &  \\ \hline
\end{tabular}
\end{table}



\section{Performance Comparison for Univariate Synthetic Data}
\label{appendix:synthetic}
To better understand our model, we generate a set of univariate synthetic time series. 
Speficically, we randomly generate a time series with length , using the state-space representation \cite{harvey1990forecasting}: 

where  is the -th value in time series. The residual terms , ,  and  are randomly drawn from a normal distribution . We eliminate about  values from the generated series, and compare our model \britsi~(note the data is univariate) and ImputeTS. We show three examples in Fig. \ref{fig:synthetic_example}.

\begin{figure}[htbp]
\centering
\footnotesize
\includegraphics[width=0.95\textwidth]{./synthetic.pdf}
\caption{Example for synthetic time series imputation. Each column corresponds to one time series. The figures in the first row are imputations of ImputeTS algorithm, and the figures in the second row are imputations by \britsi.}
\label{fig:synthetic_example}
\end{figure}

The first row corresponds to the imputations of ImputeTS and the second row corresponds to our model \britsi. As we can see, our model demonstrates better performance than ImputeTS. Especially, ImputeTS fails when the start part of time series is missing. However, for our model, the imputation errors are backpropagated to previous time steps. Thus, our model can adjust the imputations in the start part with delayed gradient, which leads to much more accurate results.



\section{Performance for Non-differentiable }
\label{appendix:cut}
As we claimed in Section \ref{sec:uni_impute}, we regard the missing values as variables of RNN graph. During the backpropagation, the imputed values can be further updated sufficiently. In the experiment, we find that if we cut the gradient of  (i.e., regard it as constant), the models are unstable and easy to overfit during the training. We refer to the model with non-differentiate  as \brits-cut.
Fig. \ref{fig:brits_curve} shows the validation errors of \brits~ and \brits-cut during training for the health-care data imputation. At the first  iterations,  the validation error of \brits-cut decreases fast. However, it soon fails due to the overfitting.

\begin{figure}[t] 
\centering
\includegraphics[width=2.5in]{./brits_curve.pdf} 
\caption{Validation errors for \brits~ and \brits-cut during training}
\label{fig:brits_curve} 
\end{figure} 

\section{Test Data of Air Quality Imputation}
\label{appendix:air}
We use the same method as in prior work \cite{yi2016st} to select the test data for air quality imputation. Specifically, we use the , ,  and  months as the test set and the rest data as the training set. To evaluate our model, we select the test timeslots by the following rule: if a measurement is observed at a timeslot in one of these four months (e.g., 8 o'clock 2015/03/07), we check the corresponding position in its previous month (e.g., 8 o'clock 2015/02/07). If it is absent in the previous month, we eliminate such value  and use it as the imputation ground-truth of test data.  Recall that to train our model, we randomly select consecutive  time steps as one time series. Thus, 
 a test timeslot may be contained in multiple time series. We use the mean of the imputations in different time series as the final result.
  

\end{document}