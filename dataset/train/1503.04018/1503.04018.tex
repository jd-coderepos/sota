By definition of the displacement,
if $\deplacement{S}<+\infty$,
then there exists a word $w\in\lang{S}$
such that $\deplacement{S}=\sum w$.
The following lemma provides a way to bound the length of such a word $w$.

\begin{restatable}{lemma}{iterationsigma}
  \label{lem:iterationsigma}
  For every nonterminal $S\in V$ with $\deplacement{S}<+\infty$,
  there is a complete
  elementary parse tree with root labeled by $S$ and
  yield $w\in A^*$ such that $\deplacement{S}=\sum w$.
\end{restatable}
\begin{proof}
  Since $\deplacement{S}<+\infty$,
  there exists a complete parse tree with root labeled by $S$ and yield
  $w\in A^*$ such that $\sum w= \displ{S}$.
  Let $(T,\lsymoperator)$ be such a parse tree with the fewest possible number
  of nodes and assume towards a contradiction that $T$ is not elementary. 
  This means there exists $s\pprefix t$ in $T$ and $X\in V$ such
  that $\lsym{s}=X=\lsym{t}$.
  The subtree rooted in $s$ provides a derivation $X\gstep{*}uXv$
  for two words $u,v$ in $A^*$.
Notice that if $\sum u+\sum v>0$ then
  $\displ{X}=+\infty$.
  Then, \cref{lem:summary-composition-and-propagation} implies that
  $\displ{S}\ge \displ{uXv}= \displ{u} +\displ{X} +\displ{v} =+\infty$,
  which contradicts the assumption of the lemma. Therefore, $\sum u+\sum v\leq 0$. 
  By collapsing the subtree
  $\{t'\in T \mid s\prefix t' \wedge  t\not\prefix t'\}$,
  we get a new parse
  tree $(T',\lsymoperator')$ with $|T'|<|T|$, $\lsym[']{\eps}=S$ and
  yield $w'\in A^*$ satisfying
  $\sum w'= \sum w -(\sum u+\sum v)\geq \sum w \geq \displ{S}$.
  Since clearly, $w'\in\lang{S}$, by definition of the displacement
  it holds that $\sum w'\le \displ{S}$ and therefore that $\sum w'=\displ{S}$.
  This contradicts our assumed minimality of $T$.
  Hence $T$ is elementary.
  \qed
\end{proof}

The corollary below follows from \cref{lem:iterationsigma}
and the observation (\cref{rem:elem-parse-trees-leaves})
that the yield of an elementary parse tree is a word of length bounded by
$\degree^{|V|}$.

\begin{corollary}
  \label{cor:existence-of-elementary-complete-flow-trees}
  For every nonterminal $S \in V$ with $\displ{S} < +\infty$,
  and for every $c \in \setN$ with $c\geq \degree^{\card{V}}$,
  there exists a complete elementary flow tree with root
  $\lnode{\varepsilon}{c}{S}{d}$ such that
  $d = c + \displ{S}$.
\end{corollary}
\begin{proof}
  According to \cref{lem:iterationsigma},
  there exists a complete elementary parse tree $(T, \lsymoperator)$
  with root labeled by $S$ and
  yield $w \in A^*$ such that $\displ{S} = \sum w$.
  Since this parse tree is elementary,
  it has no more than $\degree^{\card{V}}$ leaves.
Hence, $\len{w} \leq \degree^{\card{V}}\leq c$,
  which entails that
  $c \vstep{w} c + \displ{S}$ since $A=\{-1,0,1\}$ by assumption.
  It is routinely checked that
  the parse tree $(T, \lsymoperator)$ induces
  a complete elementary flow tree with root $\lnode{\varepsilon}{c}{S}{d}$,
  where $d = c + \displ{S}$.
  \qed
\end{proof}



\existselementaryparsetree*
\begin{proof}
  Observe that $\summary{S}(n)\leq n+\deplacement{S}$ holds
  for every $S\in V$ and $n \in \setNbar$.
  The remaining inequality follows from
  \cref{cor:existence-of-elementary-complete-flow-trees,lem:correctness-of-flow-trees}.
\qed
\end{proof}







\computablesummaryfinitedispl*
\begin{proof}
  Let $S\in V$ with $\displ{S}<+\infty$,
  and let $c \in \setN$.
  Observe that $\summary{S}(c) \leq c + \displ{S}$.
  Therefore,
  the computation of $\summary{S}(c)$ reduces to the
  question whether $\summary{S}(c) \geq d$,
  given $d \in \setN$.
  To decide the latter,
  we show that $\summary{S}(c) \geq d$ if, and only if,
  there exists a complete flow tree with root
  $\lnode{\varepsilon}{b}{S}{e}$ satisfying $b \leq c$ and $e \geq d$,
  and of height bounded by $h \eqdef \card{V} \cdot (\degree^{\card{V}} + 1)$.
\todo{[PT] we could state the above as Lemma in order to highlight the bound
  for easier complexity analysis.
  [GS] Agreed, but  if we highlight a bound, we might want to provide a
  better bound.  See commented todo below.
}
  The ``if'' direction follows from \cref{lem:correctness-of-flow-trees}
  and the monotonicity of the summary function $\summary{S}$.
  For the ``only if'' direction,
  assume that $\summary{S}(c) \geq d$.
  By \cref{lem:existence-of-complete-flow-trees},
  there exists a complete flow tree with
  root $\lnode{\varepsilon}{b}{S}{e}$ satisfying $b \leq c$ and $e \geq d$.
  Pick one,
  say $(T, \lsymoperator, \linoperator, \loutoperator)$,
  that contains the least number of nodes $t \in T$ with $\len{t} > h$.
  We show that, in fact, $T$ contains no such node.
  Since $\displ{S}<+\infty$,
  we derive from \cref{lem:summary-composition-and-propagation} that
  $\displ{\lsym{r}} < +\infty$ for every node $r \in T$.
  Now,
  consider a leaf $t$ in $T$.
  Assume, towards a contradiction, that $\len{t} > h$.
  The main observation is that for every two nodes $r, s \in T$,
\begin{equation}
    \label{eq:distinct-outputs}
    r \pprefix s \pprefix t \,\wedge\, \lsym{r} = \lsym{s}
    \ \implies \
    \lin{r} \neq \lin{s}
  \end{equation}
For if this were not the case,
  then
  \begin{itemize}
  \item
    either $\lout{r} \leq \lout{s}$,
    in which case we could replace the subtree rooted in $r$
    by the subtree rooted in $s$,
    contradicting the minimality assumption on $T$.
  \item
    or  $\lout{r} > \lout{s}$,
    which would entail,
    with the same reasoning as in the proof of \cref{lem:iterationsigma},
    that $\displ{\lsym{r}} = +\infty$,
    which is impossible.
  \end{itemize}
By the pigeonhole principle,
  it follows from \cref{eq:distinct-outputs} that
  there exists an ancestor $s \pprefix t$
  such that $\len{s} \leq \card{V} \cdot \degree^{\card{V}}$ and
  $\lin{s} \geq \degree^{\card{V}}$.
  The height of the subtree rooted in $s$ is strictly larger than $\card{V}$,
  since $t$ is in it.
  Because $\displ{\lsym{s}} < +\infty$,
  we can use \cref{cor:existence-of-elementary-complete-flow-trees}
  and replace,
  without violating the flow conditions as $\lout{s} \leq \lin{s} + \displ{\lsym{s}}$,
  the subtree rooted in $s$ by a complete flow tree of
  height at most $\card{V}$.
  This contradicts the minimality assumption on $T$.

  \medskip

  The observation that $\lin{t}$ and $\lout{t}$ are both bounded by
$\lin{\varepsilon} + \degree^h$ for every node $t$ of a complete flow tree of height $h$
  concludes the proof the proposition.
  \qed
\end{proof}

\ratioinfinite*
\begin{proof}
  Assume that $X \gstep{*} u X v$ with $\displ{uv}=+\infty$.
  Let $\lambda \in \setR$ with $\lambda \geq 1$,
  and let us show that $\ratio{X} \geq \lambda$.
  It is routinely checked that,
  since $\displ{uv}=+\infty$,
  there exists $\mu \in \{\sum z \mid z \in \lang{u}\}$ and
  $\nu \in \{\sum z \mid z \in \lang{v}\}$ such that
  $\lambda \mu + \nu \geq 0$ and
  $\mu + \nu \geq 1$.
  Observe that
  $\displ{u} \geq \mu$,
  $\displ{X} \geq 0$ and
  $\displ{v} \geq \nu$.
  Therefore,
  there exists $m \in \setN$  such that
  $\summary{u}(m) \geq m + \mu$,
  $\summary{X}(m) \geq m$ and
  $\summary{v}(m) \geq m + \nu$.
  It follows from \cref{rem:summary-monotonicity} that
  these inequalities hold for all $n \geq m$ as well.
  Let $n, k \in \setN$ such that $n \geq m$ and
  $n + k\mu \geq m$.
  Note that $n + k\mu + k\nu \geq m$ since $\mu + \nu \geq 1$.
  Since $X \gstep{*} u^k X v^k$,
  we get, by monotonicity of the summary functions,
  that
  \begin{align*}
    \qquad
    \summary{X}(n)
    & \ \geq \ \summary{v^k} \circ \summary{X} \circ \summary{u^k} (n) & [\text{\cref{lem:summary-composition-and-propagation}}]\\
    & \ \geq \ \summary{v^k} \circ \summary{X} (n + k\mu)\\
    & \ \geq \ \summary{v^k} (n + k\mu)\\
    & \ \geq \ n + k\mu + k\nu\\
    & \ \geq \ n + k \cdot \max \{1, \mu (1 - \lambda)\} & [\mu + \nu \geq 1 \ \wedge \ \lambda \mu + \nu \geq 0]
  \end{align*}
  If $\mu \geq 0$ then, for every $k \in \setN$,
  it holds that $n + k\mu \geq m$, hence, $\summary{X}(n) \geq n + k$.
  We derive that $\summary{X}(n) = +\infty$ for every $n \geq m$,
  which entails that $\ratio{X} = +\infty$.
  Otherwise, $\mu < 0$.
  Take $k = \lfloor \frac{n-m}{-\mu}\rfloor$ and let
  $r = n - m + k\mu$.
  Observe that $0 \leq r \leq -\mu - 1$.
  Since $n + k\mu \geq m$,
  we get that
  $\summary{X}(n) \geq n - k\mu (\lambda - 1)$
  from the above inequalities.
  We derive that
  $\summary{X}(n) \geq \lambda n + (\lambda - 1)(\mu + 1 - m)$
  for every $n \geq m$,
  which entails that $\ratio{X} \geq \lambda$.
  \qed
\end{proof}

We now show that the transformations
used in our reduction to thin GVAS are indeed correct,
i.e., produce equivalent systems.
Recall that
two GVAS $G = (V, A, R)$ and $G' = (V', A', R')$ are called
\emph{equivalent} if
firstly $V = V'$,
secondly $\ratio[^{G}]{X}=\ratio[^{G'}]{X}$ for every nonterminal $X$, and
thirdly $\summary[^{G}]{X}=\summary[^{G'}]{X}$ for every nonterminal $X$ \emph{with finite ratio}.

\factsummarization*
\begin{proof}
Recall that
the \emph{unfolding}
of a nonterminal $X$ with $\displ[^G]{X}<+\infty$,
is the GVAS $H = (V, A, R')$ where
$R'$ is obtained from $R$ by
removing all production rules $X \pstep \alpha$ and
instead adding, for every $0\le i\le\degree^{\card{V}}$
with $j=\summary[^G]{X}(i) > -\infty$,
a rule $X \pstep (-1)^i (1)^j$.

\smallskip

  We first prove that $\summary[^G]{X} = \summary[^{H}]{X}$.
  First note that
  $\summary[^G]{X}(-\infty) = \summary[^{H}]{X}(-\infty) = -\infty$
  and
  $\summary[^G]{X}(+\infty) = \summary[^{H}]{X}(+\infty) = +\infty$.
  Let $n \in \setN$.
  By definition of $H$,
  we get that
  $\summary[^{H}]{X}(n) = \max
  \{n - i + \summary[^G]{X}(i) \mid 0 \leq i \leq \degree^{\card{V}} \wedge i \leq n\}$.
  It follows from \cref{rem:summary-monotonicity} that
  $\summary[^{H}]{X}(n) = n - m + \summary[^G]{X}(m)$ where
  $m = \min \{\degree^{\card{V}}, n\}$.
  If $n \leq \degree^{\card{V}}$ then we immediately get that
  $\summary[^{H}]{X}(n) = \summary[^G]{X}(n)$.
  Otherwise, $n > \degree^{\card{V}}$ and
  $\summary[^{H}]{X}(n) = n - \degree^{\card{V}} + \summary[^G]{X}(\degree^{\card{V}})$.
  We derive from \cref{lem:asymptotic-summary-finite-displ}
  that $\summary[^{H}]{X}(n) = \summary[^G]{X}(n)$.

  \smallskip

  We now prove that $\summary[^G]{S} = \summary[^{H}]{S}$
  for every nonterminal $S$.
  Let $c, d \in \setN$.
  Assume that $\summary[^G]{S}(c) \geq d$.
  By \cref{lem:existence-of-complete-flow-trees},
  there exists a complete flow tree
  $(T, \lsymoperator, \linoperator, \loutoperator)$ for $G$
  with root $\lnode{\varepsilon}{c}{S}{d}$.
  Let $U$ denote the set of all nodes $t \in T$ such that every
  proper ancestor $s \pprefix t$ verifies $\lsym{s} \neq X$.
  By definition,
  the set $U$ is a nonempty and prefix-closed subset of $T$.
  Moreover,
  $\lsym{t} \neq X$ for each internal node $t$ of $U$,
  and $\lsym{t} \in (\{X\} \cup A)$ for each leaf $t$ of $U$.
  It follows that $U$ is a flow tree for $H$,
  since $\summary[^G]{\#} = \summary[^{H}]{\#}$ for every $\# \in (\{X\} \cup A)$.
  Note that the root of $U$ also satisfies $\lnode{\varepsilon}{c}{S}{d}$.
  We derive from \cref{lem:correctness-of-flow-trees} that
  $\summary[^{H}]{S}(c) \geq d$.

  Conversely,
  the same reasoning as above shows that $\summary[^{H}]{S}(c) \geq d$
  implies $\summary[^G]{S}(c) \geq d$.
  We have thus shown that
  $\summary[^G]{S}(c) \geq d \Leftrightarrow \summary[^{H}]{S}(c) \geq d$,
  for every $c, d \in \setN$.
  It follows that $\summary[^G]{S} = \summary[^{H}]{S}$.
  By definition of the ratio,
  we also get that $\ratio[^G]{S} = \ratio[^{H}]{S}$.
  \qed
\end{proof}

\factabstraction*
\begin{proof}
Recall that the the \emph{abstraction} of a nonterminal $X \in V$ with
$\ratio[^G]{X} = +\infty$, is the GVAS $H = (V, A \cup \{1\}, R')$ where
$R'$ is obtained from $R$ by removing all production rules $X \pstep \alpha$ and
replacing them by the two rules $X \pstep 1 X \mid \eps$.
  
\smallskip

  Let $D_X$ denote the set of nonterminals $S \in V$ such that
  $X$ is derivable from $S$ in $G$.
  Note that $D_X$ is also the set of nonterminals $S \in V$ such that
  $X$ is derivable from $S$ in $H$.
  Recall that $\ratio[^G]{X} = +\infty$.
  By definition of $H$, it holds that $\ratio[^H]{X} = +\infty$.
  It follows from \cref{lem:summary-composition-and-propagation} that
  $\ratio[^G]{S} = \ratio[^H]{S} = +\infty$ for every $S \in D_X$.

  \smallskip

  Now consider a nonterminal $S \not\in D_X$.
  It is readily seen that $G$ and $H$ have the same derivations
  $S \gstep{*} w$ starting from $S$.
  Therefore, $\lang[^G]{S} = \lang[^H]{S}$.
  It follows that $\summary[^G]{S} = \summary[^H]{S}$.
  By definition of the ratio,
  we also get that $\ratio[^G]{S} = \ratio[^H]{S}$.
  The observation that every nonterminal with finite ratio is
  in $V \setminus D_X$ concludes the proof.
  \qed
\end{proof}


 
\summarycomputableboundedratio*
\begin{proof}
  By \cref{prop:reduction-to-thin}, it is enough show the claim for thin GVAS.
  Let us consider a thin GVAS $G = (V, A, R)$ and a nonterminal $X \in V$.
  By \cref{thm:thin},
  the relation $\vstep{X}$ is effectively definable in Presburger arithmetic.
  Therefore, so is the set
  $\Sigma_X(n) \eqdef \{d \mid \exists c \leq n : c \vstep{X} d\}$,
  for any given $n \in \setNbar$.
  We derive that its supremum $\summary{X}(n) = \sup \Sigma_X(n)$
  is computable.

  \smallskip

  We now prove that the question whether $\ratio{X} < +\infty$
  is decidable.
  Since
  the relation $\vstep{X}$ is effectively definable in
  Presburger arithmetic,
  it is effectively semilinear \cite{Ginsburg:1966:PACIF}.
  This means that we can compute a finite family
  $\{(\vec{b}_i, \vec{P}_i)\}_{i \in I}$ of
  vectors $\vec{b}_i$ in $\setN^2$ and
  finite subsets $\vec{P}_i$ of $\setN^2$,
  with $\vec{P}_i = \{\vec{p}_i^1, \ldots, \vec{p}_i^{\ell_i}\}$,
  such that
  $\vstep{X} {=} \,\bigcup_{i \in I}
   \left(
     \vec{b}_i + \setN\vec{p}_i^1+\cdots+\setN\vec{p}_i^{\ell_i}
   \right)$.
  We consider two cases.
\begin{itemize}
  \item
    If there exists $i \in I$ and a vector $\vec{p}$ in
    $\bigcup_{i \in I} \vec{P}_i$
    such that $\vec{p}(1) = 0$ and $\vec{p}(2) > 0$,
    then $\vec{b}_i(1) \vstep{X} (\vec{b}_i(2) + k \vec{p}(2))$
    for every $k \in \setN$.
    It follows that $\summary{X}(\vec{b}_i(1)) = +\infty$,
    which entails, by monotonicity of $\summary{X}$,
    that $\ratio{X} = +\infty$.
  \item
    Otherwise,
    there exists $\lambda \in \setR$ with $\lambda \geq 1$
    such that $\vec{p}(2) \leq \lambda \vec{p}(1)$ for every
    vector $\vec{p}$ in $\bigcup_{i \in I} \vec{P}_i$.
    Define $b = \max \{\vec{b}_i(2) \mid i \in I\}$.
    It is routinely checked that
    $d \leq \lambda c + b$
    for every $c, d$ with $c \vstep{X} d$.
    We derive that $\summary{X}(n) \leq \lambda n + b$
    for every $n \in \setN$,
    which implies that $\ratio{X} \leq \lambda$.
  \end{itemize}
We have shown that $\ratio{X} = +\infty$ if, and only if,
  there exists $\vec{p}$ in $\bigcup_{i \in I} \vec{P}_i$
  with $\vec{p}(1) = 0$ and $\vec{p}(2) > 0$.
  The latter condition is decidable,
  and so is the former.
  \qed
\end{proof}

\begin{lemma}
  \label{fact:certificate-decidable}
  Let $(T, \lsymoperator)$ be a parse tree and
  let $\linoperator, \loutoperator: T \to \setN$.
  Then $(T, \lsymoperator, \linoperator, \loutoperator)$ is a certificate if
  the three following conditions hold:
  \begin{enumerate}
  \item[$(i)$]
    All internal nodes satisfy the first flow condition,
  \item[$(ii)$]
    Every leaf $t \in T$ with $\ratio{\lsym{t}} < +\infty$
    satisfies the second flow condition, and
  \item[$(iii)$]
    Every leaf $t \in T$ with $\ratio{\lsym{t}} = +\infty$
    has a proper ancestor $s \pprefix t$ such that
    $\lsym{s} = \lsym{t}$ and $\lin{s} < \lin{t}$.
  \end{enumerate}
\end{lemma}
\begin{proof}
  Assume that $(i)$--$(iii)$ hold.
  We only need to show that every leaf of $T$ satisfies
  the second flow condition.
  By contradiction, assume that $T$ contains a leaf $t$ with
  $\lout{t} \not\leq \summary{\lsym{t}}(\lin{t})$.
  It follows from $(ii)$ and $(iii)$ that $\ratio{\lsym{t}} = +\infty$
  and that $t$ has a proper ancestor $s \pprefix t$ such that
  $\lsym{s} = \lsym{t}$ and $\lin{s} < \lin{t}$.
  Let $t_1, \ldots, t_\ell$,
  with $\lnode{t_i}{c_i}{\#_i}{d_i}$,
  denote the leaves of the subtree of $T$ rooted in $s$,
  in lexicographic order (informally, from left to right).
  Obviously, $t = t_k$ for some $k$ in $\{1, \ldots, \ell\}$.
  We may suppose, without loss of generality,
  that $t_1, \ldots, t_{k-1}$ satisfy the second flow condition.
  This means that
  $d_i \leq \summary{\#_i}(c_i)$ for all $i$ with $1 \leq i < k$.
  Since every internal node satisfies the first flow condition,
  it holds that $\lin{s} \geq c_1$ and
  $d_i \geq c_{i+1}$ for all $i$ with $1 \leq i < k$.
  We derive from the monotonicity of summary functions that
  \begin{align*}
    \summary{\#_1 \cdots \#_{k-1}}(\lin{s})
    & \ = \ \summary{\#_{k-1}} \circ \cdots \circ \summary{\#_1}(\lin{s}) & [\text{\cref{lem:summary-composition-and-propagation}}]\\
    & \ \geq \ \summary{\#_{k-1}} \circ \cdots \circ \summary{\#_1}(c_1) & [\lin{s} \geq c_1]\\
    & \ \geq \ c_k & [\summary{\#_i}(c_i) \geq d_i \geq c_{i+1}]\\
    & \ > \ \lin{s} & [c_k = \lin{t} > \lin{s}]
  \end{align*}
  Define $u = \#_1 \cdots \#_{k-1}$, $X = \lsym{s} = \#_k$, and
  $v = \#_{k+1} \cdots \#_\ell$.
  Recall that $t_1, \ldots, t_\ell$ are the leaves,
  in lexicographic order,
  of the subtree of $T$ rooted in $s$.
  Therefore, we have the derivation $X \gstep{*} u X v$.
  We obtain from \cref{lem:summary-for-infinite-ratio} that
  $\summary{X}(\lin{s}) = +\infty$.
  Since $\lin{t} \geq \lin{s}$, we get that $\summary{X}(\lin{t}) = +\infty$,
  which contradicts our assumption that
  $\lout{t} \not\leq \summary{X}(\lin{t})$.
  \qed
\end{proof}
