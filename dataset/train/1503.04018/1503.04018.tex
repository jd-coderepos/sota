By definition of the displacement,
if ,
then there exists a word 
such that .
The following lemma provides a way to bound the length of such a word .

\begin{restatable}{lemma}{iterationsigma}
  \label{lem:iterationsigma}
  For every nonterminal  with ,
  there is a complete
  elementary parse tree with root labeled by  and
  yield  such that .
\end{restatable}
\begin{proof}
  Since ,
  there exists a complete parse tree with root labeled by  and yield
   such that .
  Let  be such a parse tree with the fewest possible number
  of nodes and assume towards a contradiction that  is not elementary. 
  This means there exists  in  and  such
  that .
  The subtree rooted in  provides a derivation 
  for two words  in .
Notice that if  then
  .
  Then, \cref{lem:summary-composition-and-propagation} implies that
  ,
  which contradicts the assumption of the lemma. Therefore, . 
  By collapsing the subtree
  ,
  we get a new parse
  tree  with ,  and
  yield  satisfying
  .
  Since clearly, , by definition of the displacement
  it holds that  and therefore that .
  This contradicts our assumed minimality of .
  Hence  is elementary.
  \qed
\end{proof}

The corollary below follows from \cref{lem:iterationsigma}
and the observation (\cref{rem:elem-parse-trees-leaves})
that the yield of an elementary parse tree is a word of length bounded by
.

\begin{corollary}
  \label{cor:existence-of-elementary-complete-flow-trees}
  For every nonterminal  with ,
  and for every  with ,
  there exists a complete elementary flow tree with root
   such that
  .
\end{corollary}
\begin{proof}
  According to \cref{lem:iterationsigma},
  there exists a complete elementary parse tree 
  with root labeled by  and
  yield  such that .
  Since this parse tree is elementary,
  it has no more than  leaves.
Hence, ,
  which entails that
   since  by assumption.
  It is routinely checked that
  the parse tree  induces
  a complete elementary flow tree with root ,
  where .
  \qed
\end{proof}



\existselementaryparsetree*
\begin{proof}
  Observe that  holds
  for every  and .
  The remaining inequality follows from
  \cref{cor:existence-of-elementary-complete-flow-trees,lem:correctness-of-flow-trees}.
\qed
\end{proof}







\computablesummaryfinitedispl*
\begin{proof}
  Let  with ,
  and let .
  Observe that .
  Therefore,
  the computation of  reduces to the
  question whether ,
  given .
  To decide the latter,
  we show that  if, and only if,
  there exists a complete flow tree with root
   satisfying  and ,
  and of height bounded by .
\todo{[PT] we could state the above as Lemma in order to highlight the bound
  for easier complexity analysis.
  [GS] Agreed, but  if we highlight a bound, we might want to provide a
  better bound.  See commented todo below.
}
  The ``if'' direction follows from \cref{lem:correctness-of-flow-trees}
  and the monotonicity of the summary function .
  For the ``only if'' direction,
  assume that .
  By \cref{lem:existence-of-complete-flow-trees},
  there exists a complete flow tree with
  root  satisfying  and .
  Pick one,
  say ,
  that contains the least number of nodes  with .
  We show that, in fact,  contains no such node.
  Since ,
  we derive from \cref{lem:summary-composition-and-propagation} that
   for every node .
  Now,
  consider a leaf  in .
  Assume, towards a contradiction, that .
  The main observation is that for every two nodes ,

For if this were not the case,
  then
  \begin{itemize}
  \item
    either ,
    in which case we could replace the subtree rooted in 
    by the subtree rooted in ,
    contradicting the minimality assumption on .
  \item
    or  ,
    which would entail,
    with the same reasoning as in the proof of \cref{lem:iterationsigma},
    that ,
    which is impossible.
  \end{itemize}
By the pigeonhole principle,
  it follows from \cref{eq:distinct-outputs} that
  there exists an ancestor 
  such that  and
  .
  The height of the subtree rooted in  is strictly larger than ,
  since  is in it.
  Because ,
  we can use \cref{cor:existence-of-elementary-complete-flow-trees}
  and replace,
  without violating the flow conditions as ,
  the subtree rooted in  by a complete flow tree of
  height at most .
  This contradicts the minimality assumption on .

  \medskip

  The observation that  and  are both bounded by
 for every node  of a complete flow tree of height 
  concludes the proof the proposition.
  \qed
\end{proof}

\ratioinfinite*
\begin{proof}
  Assume that  with .
  Let  with ,
  and let us show that .
  It is routinely checked that,
  since ,
  there exists  and
   such that
   and
  .
  Observe that
  ,
   and
  .
  Therefore,
  there exists   such that
  ,
   and
  .
  It follows from \cref{rem:summary-monotonicity} that
  these inequalities hold for all  as well.
  Let  such that  and
  .
  Note that  since .
  Since ,
  we get, by monotonicity of the summary functions,
  that
  
  If  then, for every ,
  it holds that , hence, .
  We derive that  for every ,
  which entails that .
  Otherwise, .
  Take  and let
  .
  Observe that .
  Since ,
  we get that
  
  from the above inequalities.
  We derive that
  
  for every ,
  which entails that .
  \qed
\end{proof}

We now show that the transformations
used in our reduction to thin GVAS are indeed correct,
i.e., produce equivalent systems.
Recall that
two GVAS  and  are called
\emph{equivalent} if
firstly ,
secondly  for every nonterminal , and
thirdly  for every nonterminal  \emph{with finite ratio}.

\factsummarization*
\begin{proof}
Recall that
the \emph{unfolding}
of a nonterminal  with ,
is the GVAS  where
 is obtained from  by
removing all production rules  and
instead adding, for every 
with ,
a rule .

\smallskip

  We first prove that .
  First note that
  
  and
  .
  Let .
  By definition of ,
  we get that
  .
  It follows from \cref{rem:summary-monotonicity} that
   where
  .
  If  then we immediately get that
  .
  Otherwise,  and
  .
  We derive from \cref{lem:asymptotic-summary-finite-displ}
  that .

  \smallskip

  We now prove that 
  for every nonterminal .
  Let .
  Assume that .
  By \cref{lem:existence-of-complete-flow-trees},
  there exists a complete flow tree
   for 
  with root .
  Let  denote the set of all nodes  such that every
  proper ancestor  verifies .
  By definition,
  the set  is a nonempty and prefix-closed subset of .
  Moreover,
   for each internal node  of ,
  and  for each leaf  of .
  It follows that  is a flow tree for ,
  since  for every .
  Note that the root of  also satisfies .
  We derive from \cref{lem:correctness-of-flow-trees} that
  .

  Conversely,
  the same reasoning as above shows that 
  implies .
  We have thus shown that
  ,
  for every .
  It follows that .
  By definition of the ratio,
  we also get that .
  \qed
\end{proof}

\factabstraction*
\begin{proof}
Recall that the the \emph{abstraction} of a nonterminal  with
, is the GVAS  where
 is obtained from  by removing all production rules  and
replacing them by the two rules .
  
\smallskip

  Let  denote the set of nonterminals  such that
   is derivable from  in .
  Note that  is also the set of nonterminals  such that
   is derivable from  in .
  Recall that .
  By definition of , it holds that .
  It follows from \cref{lem:summary-composition-and-propagation} that
   for every .

  \smallskip

  Now consider a nonterminal .
  It is readily seen that  and  have the same derivations
   starting from .
  Therefore, .
  It follows that .
  By definition of the ratio,
  we also get that .
  The observation that every nonterminal with finite ratio is
  in  concludes the proof.
  \qed
\end{proof}


 
\summarycomputableboundedratio*
\begin{proof}
  By \cref{prop:reduction-to-thin}, it is enough show the claim for thin GVAS.
  Let us consider a thin GVAS  and a nonterminal .
  By \cref{thm:thin},
  the relation  is effectively definable in Presburger arithmetic.
  Therefore, so is the set
  ,
  for any given .
  We derive that its supremum 
  is computable.

  \smallskip

  We now prove that the question whether 
  is decidable.
  Since
  the relation  is effectively definable in
  Presburger arithmetic,
  it is effectively semilinear \cite{Ginsburg:1966:PACIF}.
  This means that we can compute a finite family
   of
  vectors  in  and
  finite subsets  of ,
  with ,
  such that
  .
  We consider two cases.
\begin{itemize}
  \item
    If there exists  and a vector  in
    
    such that  and ,
    then 
    for every .
    It follows that ,
    which entails, by monotonicity of ,
    that .
  \item
    Otherwise,
    there exists  with 
    such that  for every
    vector  in .
    Define .
    It is routinely checked that
    
    for every  with .
    We derive that 
    for every ,
    which implies that .
  \end{itemize}
We have shown that  if, and only if,
  there exists  in 
  with  and .
  The latter condition is decidable,
  and so is the former.
  \qed
\end{proof}

\begin{lemma}
  \label{fact:certificate-decidable}
  Let  be a parse tree and
  let .
  Then  is a certificate if
  the three following conditions hold:
  \begin{enumerate}
  \item[]
    All internal nodes satisfy the first flow condition,
  \item[]
    Every leaf  with 
    satisfies the second flow condition, and
  \item[]
    Every leaf  with 
    has a proper ancestor  such that
     and .
  \end{enumerate}
\end{lemma}
\begin{proof}
  Assume that -- hold.
  We only need to show that every leaf of  satisfies
  the second flow condition.
  By contradiction, assume that  contains a leaf  with
  .
  It follows from  and  that 
  and that  has a proper ancestor  such that
   and .
  Let ,
  with ,
  denote the leaves of the subtree of  rooted in ,
  in lexicographic order (informally, from left to right).
  Obviously,  for some  in .
  We may suppose, without loss of generality,
  that  satisfy the second flow condition.
  This means that
   for all  with .
  Since every internal node satisfies the first flow condition,
  it holds that  and
   for all  with .
  We derive from the monotonicity of summary functions that
  
  Define , , and
  .
  Recall that  are the leaves,
  in lexicographic order,
  of the subtree of  rooted in .
  Therefore, we have the derivation .
  We obtain from \cref{lem:summary-for-infinite-ratio} that
  .
  Since , we get that ,
  which contradicts our assumption that
  .
  \qed
\end{proof}
