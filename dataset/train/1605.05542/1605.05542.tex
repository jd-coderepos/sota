\documentclass[10pt]{article}



\usepackage{graphicx}



\usepackage{a4wide}  

\usepackage{url}  
\usepackage[utf8]{inputenc}  
\usepackage{subfigure}  
\usepackage{latexsym}  
\usepackage{amsmath} 
\usepackage{amssymb} 
\urlstyle{tt}  



\newtheorem{definition}{Definition}  
\newtheorem{proposition}{Proposition}  
\newtheorem{prop}{Property}  
\newtheorem{lemma}{Lemma}  
\newtheorem{cor}{Corollary}  
\newtheorem{corollary}{Corollary}  
\newtheorem{example}{Example}  
\newtheorem{invariant}{Invariant}  
\newtheorem{property}{Property}  
 
\newcommand{\pr}{\textrm{pref}}
\newcommand{\ov}{\textrm{ov}}
\newcommand{\pe}{\textrm{period}}
\newcommand{\cn}{\textrm{\it sp}}
\newcommand{\opt}{\mbox{\it OPT}}
\newcommand{\opts}{\mbox{\it OPT}_{\sigma}}
\newcommand{\be}{\beta}


\newtheorem{theorem}{Theorem}  
 
\newenvironment{proof}{\noindent {\it Proof.}}{\vskip1ex}  
\newenvironment{preuve}{\noindent {\it Proof.}}{\vskip1ex}  
\newenvironment{preuveA1}{\noindent {\it Proof of conservation of   
Property .}}{\vskip1ex}  
 \newenvironment{preuveA2}{\noindent {\it Proof of conservation of   
Property .}}{\vskip1ex}  
\newenvironment{preuveB}{\noindent {\it Proof of the invariants.}}  
{\vskip1ex}  
\newenvironment{OJO}{\noindent {\bf OJO:}}{\vskip1ex}  
  
 
\newcommand{\Nat}{{\mathbb N}}
\newcommand{\Real}{{\mathbb R}}
\def\lastname{Habib}
\graphicspath{{figures/}}

\begin{document}


\title{A note on the shortest common superstring\\ of NGS reads}

\author{Tristan Braquelaire\thanks{LaBRI and CBiB, University of Bordeaux, France.} \and Marie Gasparoux \and Mathieu Raffinot\thanks{CNRS, LaBRI and CBiB, University of Bordeaux, France.} \and Raluca Uricaru}

\maketitle


\begin{abstract}
The Shortest Superstring Problem (SSP) consists, for a set of strings
, to find a minimum length string that
contains all , as substrings.


 This problem is proved to be {\em NP-Complete} and APX-hard. Guaranteed
 approximation algorithms have been proposed, the current best ratio
 being , which has been achieved following a long and
 difficult quest. However, SSP is highly used in practice on next
 generation sequencing (NGS) data, which plays an increasingly important role in sequencing. In this note, we show that the SSP approximation ratio can be
 improved on NGS reads by assuming specific characteristics of NGS data
 that are experimentally verified on a very large sampling set.
\end{abstract}  


\section{Introduction}

The Shortest Superstring Problem (SSP) consists, for a set of strings
, in constructing a string  such that any
element of  is a substring of  and  is of minimal length.
For an arbitrary number of sequences , the problem is known to be {\em NP-Complete}~\cite{GALLANT198050, Garey1990}
and APX-hard~\cite{Blum:1994}. Lower bounds for the achievable approximation ratios
on a binary alphabet have been given by
Ott~\cite{Ott1999}. The best known approximation ratio so far is  \cite{Mucha13} after a long series of improvements
\cite{l-tdstls-90, Blum:1994,KPS94,
  Armen1995,Armen199829,Breslauer1997340,
  Czumaj199774,Sweedyk:1999,TengY97, KaplanS05, PaluchEZ12}.

In the meantime, SSP compression algorithms have been designed as
sub-routines of the previous ones. The idea is to ensure a fixed
compression ratio between the sum of the lengths of the sequences of
the set and the optimal superstring on this set. The greedy algorithm
is such a compression algorithm that is proven to achieve a
compression ratio of at least , while the best
compression algorithm achieves a ratio of 
\cite{KPS94}.

In this note, we focus on practical applications of SSP, like assembling biological sequences, mostly DNA sequences with an alphabet of  named {\em bases}, but also on proteome
sequences with a 26 letter alphabet corresponding to {\em amino acids}. SSP is
used in contig reconstruction step, contigs that subsequently need to be organised.

Over the past decade, the landscape of sequencing and assembly deeply
changed, with the increasing development of Next Generation Sequencing (NGS)
devices. These relatively cheap devices produce, from a ``soup'' of cells, millions of randomly read, short, equal length DNA sequences in a single {\em run}.
Each sequence is typically 32 to 1000 bases long, with a small and still decreasing cost per
base. Such sequences are named {\em reads}. NGS technology allows to
tackle new challenges in biology and medecine; the exponential
increase of sequencing demands leads to the creation of more and
more sequencing platforms, dealing with NGS data at 99\%.

Considering the specificity of read sequences, is it possible to propose better
approximation algorithms for this type of data? This research,
similar to the one targeting better algorithms for small-world graphs in
social networks, aims to better suit the actual data.

This note is a first step in this direction. We first model the read
sequences more finely thus, according to our examples,
better matching the experimental data.  Then, we derive a better
approximation ratio algorithm by using the properties of the
reads. For instance, on the set SRR069579, we reach a 
approximation ratio (see Table \ref{SRR069579}). To our knowledge, the
only related work is \cite{GolovnevKM13}, where the sequences have the
same length. Up to  bases, they propose a better approximation
ratio based on De Bruijn graphs. However, these sequences are way
shorter than real-world reads.

Note that some theoretical variations of SSP have also been studied
\cite{Yu16a, CrochemoreCIKRRW10}. Here we do not dwell on these
studies since their focus is far from ours, neither do we detail the
greedy algorithm approximation conjecture, which is a subject by
itself \cite{TARHIO1988131,KaplanS05,FiciKRRW16}.


\section{Modeling of reads}

NGS reads have some specific properties that we model and exhibit on
real sets of reads.

For a string  of length , any integer  is a
\emph{period} of  if  for all . Note that  always has at least one period, corresponding to its
length. The smallest period of  is called \emph{the period} of ,
and denoted .


We consider SSP on  reads  of length
, where . We now consider the period of each read. We
denote , , the number of reads of period . 



Let  be a parameter and let . We express  (for {\em s}mall {\em p}eriod) as a percentage
of  relatively to the value of  and we
denote  \\

A strong characteristic of a set of reads is that even for ratios ,  is very small compared to . The order of
magnitude is  being a few per hundred of . For a
large panel of sets of reads on which we tested our approach, we found
for  a  value inferior to . In
Figure \ref{plotsreads}, we show such four sets of reads with lengths
of 32, 36, 98 and 200.







\begin{figure}[t]
\begin{centering}
\includegraphics[width=5cm]{./SRR069579-plot_log10.pdf}
\includegraphics[width=5cm]{./ERR000009-plot_log10.pdf}\|w_{\sigma}|= \sum \left| \sigma_i \right| \leq \mbox{wt}(\mathcal{C})
  + \mbox{wt}(\mathcal{C}) \frac 1 \alpha + \frac {\cn \cdot m} 2 \leq (1+\frac
  1 \alpha) \opt + \frac{\cn \cdot m}{2}|\tau|  \leq 2\opt + \frac{38}{63} \left( \frac{1-\alpha}{\alpha} \right) \opt + \frac{38}{126} \mbox{\em perc}_{\alpha} \mbox{OPT}
\end{theorem}

\begin{table}[!htbp]
\clearpage{}{\small \begin{tabular}{|rrrrrrrr|}
\hline
period & nbseq &  cum. nbseq &  &  &     &   &  \\ \hline
1 & 4 & 4 & 0.0277778 & 37 & 23.1111 & 1.74749e-05 & 23.1111\\
2 & 6 & 10 & 0.0555556 & 19 & 12.254 & 3.0581e-05 & 12.254\\
 &  &  &  &  &  &   &  \\ 
32 & 23746 & 41326 & 0.888889 & 2.125 & 2.0754 & 0.00588868 & 2.08129\\
{\bf 33} & {\bf 98795} & {\bf 140121} & {\bf 0.916667} & {\bf 2.09091} & {\bf 2.05483} & {\bf 0.0189677} & {\bf 2.0738}\\
34 & 247451 & 387572 & 0.944444 & 2.05882 & 2.03548 & 0.0507631 & 2.08624\\
35 & 829535 & 1217107 & 0.972222 & 2.02857 & 2.01723 & 0.154306 & 2.17154\\
36 & 2485202 & 3702309 & 1 & 2 & 2 & 0.455893 & 2.45589\\
\hline
\end{tabular}
}
\clearpage{}
\caption{SRR069579 read set, 3702309 reads of size 36.  }
\label{SRR069579}
\end{table}

\begin{table}[!htbp]
\clearpage{}{\small \begin{tabular}{|rrrrrrrr|}
\hline
period & nbseq &  cum. nbseq &  &  &     &   &  \\ \hline

1&	4&	4&	0.03125&	33&	20.6984&	8.83862e-06&	20.6984\\
2&	8&	12&	0.0625&	17&	11.0476&	1.76772e-05&	11.0476\\


 &  &  &  &  &  &   &  \\ 

28&	30474&	61366&	0.875&	2.14286&	2.08617&	0.00518227&	2.09135\\
{\bf 29}&	{\bf 89474}&{\bf	150840}&{\bf	0.90625}&{\bf	2.10345}&{\bf	2.0624}&	{\bf 0.0119997}&{\bf	2.0744}\\
30&	341160&	492000&	0.9375&	2.06667&	2.04021&	0.0371279&	2.07734\\
31&	953389&	1445389&	0.96875&	2.03226&	2.01946&	0.105085	&2.12454\\
32&	2606944&	4052333&	1&	2&	2&	0.285099&	2.2851\\


\hline
\end{tabular}
}
\clearpage{}
\caption{ERR000009 read set, 4052333 reads of size 32}
\label{ERR000009}
\end{table}

\begin{table}[!htbp]
\clearpage{}{\small \begin{tabular}{|rrrrrrrr|}
\hline
period & nbseq &  cum. nbseq &  &  &     &   &  \\ \hline
1 &	2 &	2 &	0.005	  & 201 &	122.032 & 4.80545e-06 &	122.032\\
100 &	23 &	 25 &	0.5&	3 &	2.60317 &	5.35808e-06 &	2.60318\\
 &  &  &  &  &  &   &  \\ 
195 &	38013&	54574&	0.975&	2.02564&	2.01547&	0.00067972&	2.01615\\
{\bf 196}&	{\bf 134284}&	{\bf 188858}&	{\bf0.98	}& {\bf 2.02041}&	{\bf 2.01231} &	{\bf 0.00232588} &	{\bf 2.01464}\\
197 &	473686 &	662544 &	0.985 &	2.01523 &	2.00919 &	0.00810323&	2.01729\\
198 &	1685038 &	2347582 &	0.99  &	2.0101 & 	2.00609 &	0.0285511 &	2.03464\\
199 &	5811666 &	8159248 &	0.995 &	2.00503 &	2.00303 &	0.0987212 &	2.10175\\
200 &	16944518 &	25103766 &	 1 & 	       2 &	      2 &	0.302286&   2.30229\\
\hline
\end{tabular}
}
\clearpage{}
\caption{SRR211279 read set, 25103766 reads of size 200}
\label{SRR211279}
\end{table}

\begin{table}[!htbp]
\clearpage{}{\small \begin{tabular}{|rrrrrrrr|}
\hline
period & nbseq &  cum. nbseq &  &  &     &   &  \\ \hline

1&	1&	1&	0.0102041&	99&	60.5079&	7.13344e-06&	60.5079\\
50&	4&	5&	0.510204&	2.96	&2.57905&	7.70411e-06&	2.57906\\


 &  &  &  &  &  &   &  \\ 

94&	17083&	28491&	0.959184&	2.04255&	2.02567&	0.00218336&	2.02785\\
{\bf 95} &	{\bf 65228} &{\bf 93719} &{\bf 0.969388} &	{\bf 2.03158}&	{\bf 2.01905} &	{\bf 0.00708125} &	{\bf 2.02613}\\
96&	302973&	396692&	0.979592	&2.02083&	2.01257&	0.0295941&	2.04216\\
97&	942267&	1338959&	0.989796&	2.01031&	2.00622&	0.098889	&2.10511\\
98&	2804284&	4143243&	1&	2&	2&	0.303013&	2.30301\\


\hline
\end{tabular}
}
\clearpage{}
\caption{SRR959239 read set,  4143243 reads of size 98}
\label{SRR959239}
\end{table}

\section{Experimental results}
\label{exprest}

We present experimental results for the sets of reads SRR069579 (Table
\ref{SRR069579}), ERR000009 (Table \ref{ERR000009}), SRR211279 (Table
\ref{SRR211279}), and SRR959239 (Table \ref{SRR959239}). 

In each table, for each period  from  to  we show : (a)
, (b) the cumulative number of sequences, (c) the value of
 corresponding to , (d) the value of , (e)  which
corresponds to the term of equation \ref{lastequa} due to the large
cycles, (f)  which is the part of the
final ratio brought by the small cycles, and eventually (g) , the final ratio that can be reached by using the value of  from the
previous line in the table.


The resulting approximation ratios on the read sets cited above are
respectively 2.0738, 2.09, 2.01464 and 2.02623.
















{\small \bibliographystyle{plain}
\begin{thebibliography}{10}

\bibitem{Armen199829}
C.~Armen and C.~Stein.
\newblock A  superstring approximation algorithm.
\newblock {\em Discrete Applied Mathematics}, 88(1–3):29 -- 57, 1998.
\newblock Computational Molecular Biology \{DAM\} - \{CMB\} Series.

\bibitem{Blum:1994}
A.~Blum, T.~Jiang, M.~Li, J.~Tromp, and M.~Yannakakis.
\newblock Linear approximation of shortest superstrings.
\newblock {\em J. ACM}, 41(4):630--647, July 1994.

\bibitem{Breslauer1997340}
D.~Breslauer, T.~Jiang, and Z.~Jiang.
\newblock Rotations of periodic strings and short superstrings.
\newblock {\em Journal of Algorithms}, 24(2):340 -- 353, 1997.

\bibitem{Armen1995}
A.~Chris and S.~Clifford.
\newblock {\em Algorithms and Data Structures: 4th International Workshop, WADS
  '95 Kingston, Canada, August 16--18, 1995 Proceedings}, chapter Improved
  length bounds for the shortest superstring problem, pages 494--505.
\newblock Springer Berlin Heidelberg, Berlin, Heidelberg, 1995.

\bibitem{CrochemoreCIKRRW10}
M.~Crochemore, M.~Cygan, C.~S. Iliopoulos, M.~Kubica, J.~Radoszewski,
  W.~Rytter, and T.~Walen.
\newblock Algorithms for three versions of the shortest common superstring
  problem.
\newblock In {\em {CPM}}, volume 6129 of {\em Lecture Notes in Computer
  Science}, pages 299--309. Springer, 2010.

\bibitem{Czumaj199774}
A.~Czumaj, L.~Gasieniec, M.~Piotr\'ow, and W.~Rytter.
\newblock Sequential and parallel approximation of shortest superstrings.
\newblock {\em Journal of Algorithms}, 23(1):74 -- 100, 1997.

\bibitem{FiciKRRW16}
G.~Fici, T.~Kociumaka, J.~Radoszewski, W.~Rytter, and T.~Walen.
\newblock On the greedy algorithm for the shortest common superstring problem
  with reversals.
\newblock {\em Inf. Process. Lett.}, 116(3):245--251, 2016.

\bibitem{GALLANT198050}
J.~Gallant, D.~Maier, and J.~Astorer.
\newblock On finding minimal length superstrings.
\newblock {\em Journal of Computer and System Sciences}, 20(1):50 -- 58, 1980.

\bibitem{Garey1990}
M.~R. Garey and D.~S. Johnson.
\newblock {\em Computers and Intractability; A Guide to the Theory of
  NP-Completeness}.
\newblock W. H. Freeman \& Co., New York, NY, USA, 1990.

\bibitem{GolovnevKM13}
A.~Golovnev, A.~S. Kulikov, and I.~Mihajlin.
\newblock Approximating shortest superstring problem using de bruijn graphs.
\newblock In {\em {CPM}}, volume 7922 of {\em Lecture Notes in Computer
  Science}, pages 120--129. Springer, 2013.

\bibitem{KaplanS05}
H.~Kaplan and N.~Shafrir.
\newblock The greedy algorithm for shortest superstrings.
\newblock {\em Inf. Process. Lett.}, 93(1):13--17, 2005.

\bibitem{KPS94}
S.~R. Kosaraju, J.~K. Park, and C.~Stein.
\newblock Long tours and short superstrings.
\newblock In {\em Foundations of Computer Science, 1994 Proceedings., 35th
  Annual Symposium on}, pages 166--177, Nov 1994.

\bibitem{l-tdstls-90}
M.~Li.
\newblock Towards a {DNA} sequencing theory (learning a string).
\newblock In {\em Proc. of the 31st Symposium on the Foundations of Comp.
  Sci.}, pages 125--134. IEEE Computer Society Press, Los Alamitos, CA, 1990.

\bibitem{Mucha13}
M.~Mucha.
\newblock Lyndon words and short superstrings.
\newblock In {\em {SODA}}, pages 958--972. {SIAM}, 2013.

\bibitem{Ott1999}
Sascha Ott.
\newblock {\em Graph-Theoretic Concepts in Computer Science: 25th International
  Workshop, WG'99 Ascona, Switzerland, June 17--19, 1999 Proceedings}, chapter
  Lower Bounds for Approximating Shortest Superstrings over an Alphabet of Size
  2, pages 55--64.
\newblock Springer Berlin Heidelberg, Berlin, Heidelberg, 1999.

\bibitem{PaluchEZ12}
K.~E. Paluch, K.~M. Elbassioni, and A.~van Zuylen.
\newblock Simpler approximation of the maximum asymmetric traveling salesman
  problem.
\newblock In {\em {STACS}}, volume~14 of {\em LIPIcs}, pages 501--506. Schloss
  Dagstuhl - Leibniz-Zentrum fuer Informatik, 2012.

\bibitem{Sweedyk:1999}
Z.~Sweedyk.
\newblock \boldmath a -approximation algorithm for shortest
  superstring.
\newblock {\em SIAM J. Comput.}, 29(3):954--986, December 1999.

\bibitem{TARHIO1988131}
J.~Tarhio and E.~Ukkonen.
\newblock A greedy approximation algorithm for constructing shortest common
  superstrings.
\newblock {\em Theoretical Computer Science}, 57(1):131 -- 145, 1988.

\bibitem{TengY97}
S.-H. Teng and F.~F. Yao.
\newblock Approximating shortest superstrings.
\newblock {\em {SIAM} J. Comput.}, 26(2):410--417, 1997.

\bibitem{vazirani}
V.~V. Vazirani.
\newblock {\em Approximation Algorithms}.
\newblock Springer-Verlag New York, Inc., New York, NY, USA, 2001.

\bibitem{Yu16a}
Y.~W. Yu.
\newblock Approximation hardness of shortest common superstring variants.
\newblock {\em CoRR}, abs/1602.08648, 2016.

\end{thebibliography}


}
\end{document}
