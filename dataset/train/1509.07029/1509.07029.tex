\documentclass[journal,draftcls,onecolumn,12pt,twoside]{IEEEtran}

\usepackage{amscd, amssymb, amsthm, amsmath, color}
\usepackage{graphicx, psfrag, enumerate, multirow, algorithmic}
\usepackage[linesnumbered,ruled,vlined]{algorithm2e}



\renewcommand{\labelenumi}{(\roman{enumi})}

\newcommand{\seq
}[1]{\mathbf{#1}}


\newtheorem{theorem}{\bf Theorem}
\newtheorem{proposition}[theorem]{\bf Proposition}
\newtheorem{corollary}[theorem]{\bf Corollary}
\newtheorem{lemma}[theorem]{\bf Lemma}
\newtheorem{example}{Example}

\newcommand{\GP}{{\mathtt{GP}}}

\renewcommand{\d}{\displaystyle}

\long\def\symbolfootnote[#1]#2{\begingroup
\def\thefootnote{\fnsymbol{footnote}}\footnote[#1]{#2}\endgroup}

\begin{document}
\title{The Global Packing Number for an Optical Network}

\author{Yuan-Hsun Lo, Yijin Zhang,~\IEEEmembership{Member,~IEEE}, Wing Shing Wong,~\IEEEmembership{Fellow,~IEEE} and Hung-Lin Fu,~\IEEEmembership{Member,~IEEE}
\thanks{Partially supported by the National Natural Science Foundation of China (No. 61301107, 61174060), the Hong Kong RGC Earmarked Grant CUHK414012 and the Shenzhen Knowledge Innovation Program JCYJ20130401172046453 (Y.-H.~Lo, Y.~Zhang and W.~S.~Wong), and by Ministry of Science and Technology, Taiwan under grants MOST 104-2115-M-009-009 (H.-L.~Fu).}
\thanks{Y.-H. Lo is with the School of Mathematical Sciences, Xiamen University, Xiamen 361005, China. E-mail: yhlo0830@gmail.com}
\thanks{Y. Zhang is with the School of Electronic and Optical Engineering, Nanjing University of Science and Technology, Nanjing, China.
E-mail: yijin.zhang@gmail.com}
\thanks{W. S. Wong is with the Department of Information Engineering, The Chinese University of Hong Kong, Shatin, N.~T., Hong Kong.
E-mail: wswong@ie.cuhk.edu.hk}
\thanks{H.-L. Fu is with the Department of Applied Mathematics, National Chiao Tung University, Hsinchu 300, Taiwan, ROC. E-mail: hlfu@math.nctu.edu.tw}
}
\maketitle


\begin{abstract}
The {\em global packing number problem} arises from the investigation of optimal wavelength allocation in an optical network that employs Wavelength Division Multiplexing (WDM).
Consider an optical network that is represented by a connected, simple graph . 
We assume all communication channels are bidirectional, so that all links and paths
are undirected.
It follows that there are  distinct node pairs associated with ,
where  is the number of nodes in .
A path system  of  consists of  paths, one path to
connect each of the node pairs.  The global packing number of a path system , denoted by , is the minimum integer  to guarantee the existence of a mapping , such that  if  and  have common edge(s).
The global packing number of , denoted by , is defined to be the minimum  among all possible path systems .
If there is no wavelength conversion along any optical transmission path for any node pair in the network, the global packing number signifies the minimum number of wavelengths required to support simultaneous communication for all pairs in the network.

In this paper, the focus is on ring networks, so that  is a cycle.
Explicit formulas for the global packing number of a cycle is derived.
The investigation is further extended to chain networks.
A path system, , that enjoys  is called {\em ideal}.  A characterization of ideal path systems is also presented.
We also describe an efficient heuristic algorithm to assign wavelengths that can be applied
to a general network with more complicated traffic load.




\end{abstract}

\begin{IEEEkeywords}
Global packing number, Ring networks, WDM networks, Traffic capacity, Wavelength assignment.
\end{IEEEkeywords}


\section{Introduction} \label{sec:intro}



This paper investigates a class of resource allocation problem that deals with the computation of the {\em global packing number} of a communication network.  This
is an index that characterizes the number of wavelengths required to support a uniformly loaded optical network that employs Wavelength Division Multiplexing (WDM).  The global packing number problem is therefore well motivated by engineering application.

Let a graph, , represent a communication network so that the link between any two nodes represents a bidirectional communication channel.  
For simplicity, we assume a uniform traffic model for which the
traffic load between any node pair is identical.   
A classical problem of path routing deals with the question of finding the shortest path between any node pair.   
On the other hand, some networking problems deal with the issue of distributing the routed traffic evenly over the network.   
If the links have identical traffic capacity, , one formulation of this latter class of questions is to find routing paths that they can all be supported simultaneously by the minimum . 
 
In an optic network, links between nodes are implemented by optical fibers that carry optic carrier signals.   
The WDM technology \cite{GY_04,LXC_05,S_10,SSA_02} allows carrier signals of different wavelengths be mixed together for simultaneous transmission over the optic fibers.  
In the basic implementation architecture, the wavelength of a carrier signal is kept constant as it passes over the network nodes.  
Hence, one can envision the end-to-end transmission from a source to a destination node as implemented over a reserved {\em lightpath}.  
Different simultaneous transmissions that go through a given optical fiber link must use different wavelengths.  
Hence, there is a basic question of computing the number of wavelengths required in a WDM network in order to support a given traffic model.  
This defines the global packing number problem for  if we assume the traffic load
is simply one request for each distinct node pair. 
 
To facilitate subsequent discussion, we state the following notation and basic results.

\subsection{Notation} \label{sec:intro_notation}
Let  be a connected simple graph. 
For any two nodes, , one can assign a particular path to connect  and 
and denote it by .
Note that  is sometimes set to be one of the shortest paths that connects  with .
Let  denote a set of assigned paths,
one for each distinct node pair.
We say  is a \emph{path system} of .
Obviously, , where  is the number of nodes in .
Let  denote the collection of all path systems of .
A path system is called a \emph{shortest path system} if every path is a shortest path connecting the two corresponding nodes.

Given a path system  of . 
A \emph{global packing} of  is a mapping  from  to the set , for some , such that for any two paths ,  provided that  and  have one or more than one edge in common.
Each element in the image  is called a \emph{wavelength}, named for its appliction to optical networks.
The \emph{global packing number} of , denoted by , is the minimum integer  to guarantee the existence of a global packing.
The global packing number of , denoted by , is defined to be the smallest global packing number of  among all path systems ; i.e., 
A path system  with  is said to be \emph{ideal}.
An ideal shortest path system is said to be {\em perfect}.

Consider a path system .
In \emph{graph coloring} model, let  be the graph such that , and  is adjacent to  in  if and only if  and  have the some edge(s) in common in .
Then , the \emph{chromatic number} of .

All terminology and notation on graph theory used throughout this paper can be referred
to the textbook~\cite{West_01}.





\subsection{Bounds derived from basic principles} \label{sec:intro_bound}
Let  denote the number of edges in the graph .
A \emph{-packing} is a collection of mutually edge-disjoint subgraphs of .
A -packing is \emph{full} if it contains  edges.
Let  be a path system of , where , and  be a global packing of .
Then, for each , the preimage of  under , , forms a -packing.
Since a -packing consists of at most  edges, a natural lower bound of  is given as follows.

\begin{proposition}\label{pro:lowerbound}
Let  be a path system of .
Then,

\end{proposition}

On the other hand, assume that  can be partitioned into  subsets, , each of which forms a -packing. 
One can define a global packing by  if  for all . 
It follows that .

\begin{proposition}\label{pro:upperbound}
Let  be a path system of .
If  can be partitioned into  subsets such that each subset forms a -packing, then

\end{proposition}

In this paper, the focus is on ring networks, so that  is a cycle.
In spite of its simplicity, the ring topology is of fundamental interest in the study of optical networks~\cite{GCFPFGNP_01,GSCAC_09,PCF_05}.
Let  denote a cycle of  nodes.
For the sake of convenience, in this paper we assume the nodes of  are labelled clockwise by non-negative integers , and  denotes the path consists of edges  (mod ).
Note that  if .

The rest of this paper is organized as follows.
In Section~\ref{sec:evencycle} we compute the exact value of  for the case  is even by means of Proposition~\ref{pro:lowerbound} and Proposition~\ref{pro:upperbound}. 
As for odd  cases, the exact value of  is given in Section~\ref{sec:oddcycle}, where the derivation is based on a greedy construction, which we refer to as
{\em Intelligent Packing algorithm}.
The corresponding ideal path systems of  are characterized as well.
In Section~\ref{sec:simulation} we extend the Intelligent Packing algorithm to a general version, referred to as the {\em Length First Packing algorithm}, which is applicable to
general connected graphs.  To demonstrate the effectiveness of the Length First
Packing algorithm we apply it to chain networks and show that they can achieve the global packing number value. 
We also apply it to ring networks with more complicated traffic
load and compare its performance against a random algorithm.
Finally, concluding remarks are given in Section~\ref{sec:conclusion}.


\section{Global packing numbers of even cycles}\label{sec:evencycle}
To derive a lower bound of , by Proposition~\ref{pro:lowerbound}, it is natural to consider the shortest path system.
Fix two nodes  in a cycle .
If the distance of  and  is less than , there is only one shortest path that connects them; however, if the distance is equal to  (i.e.,  and  are on opposite side), there are two choices of shortest paths.


\begin{lemma}\label{lemma:cycle_even_lower}
For any integer , 
\end{lemma}
\begin{proof}
Let  be a shortest path system of .
Since  for any path system  with , we have .
Observe that there are exactly  paths of length  and  paths of length , for each . 
By Proposition~\ref{pro:lowerbound}, 

Hence it follows that  when  is odd.
In what follows, we consider the case  is even.

Let  be one of the ideal path systems of .
There are two cases:  is or is not a shortest path system.
If  is not a shortest path system, by the same argument in \eqref{eq:lower_even_1},  must be larger than , and then we are done.
If  is a shortest path system, we now claim .
Suppose not, i.e., .
By \eqref{eq:lower_even_1}, it must be .
Fix an edge , and consider the number of paths of length less than  which contain .
Then there are  paths: one of length , two of length , , and  of length .
Since all of them contain , they have to be assigned distinct wavelengths.
That is, , where  is the number of paths of length  which contain .
Note that this fact is true for all distinct edges.
Let  be the collection of  paths of length  in .
Now, it suffices to prove that there exists an edge which occurs in , at least  times.

\begin{figure}[h]
\centering
\includegraphics[width=2.5in]{Fig-even-proof.pdf}
\caption{The  paths of length  which contain .} 
\label{fig:even-proof}
\end{figure}

Suppose the assertion is not true.
Since there are  edges in , each edge occurs in  exactly  times.
Denote  and let  be an edge which occurs in paths , , .
For convenience, let  and , in which  denotes that  is ahead of  counterclocksiwely, as illustrated in Fig.~\ref{fig:even-proof}.
Hence , and all the edges in  occur  times as well.
Consider the edge  (note that  may be ).
Clearly, in these  paths,  only occurs  times.
Since  is one of the endpoints of , it can not be an endpoint of the other paths of length .
This implies that either  does not occur in the other paths of length  or if it occurs, some of the edges in  will occur more than  times, a contradiction.
For the first case,  occurs exactly  times among all paths of length , a contradiction.
Hence we complete the proof.
\end{proof}

Now, we will show that  by choosing some one particular shortest path system .
As an illustration, consider the following example.

\begin{example}\label{exa:20}
{\rm
Let  and  be the shortest path system of  which contains the following  paths of length :

Except for above  paths, in  there are  paths of length , for each .
By Proposition~\ref{pro:upperbound}, we aim to partition these  paths into  -packings.
First, partition the set  into subsets , ,  and , so that the sum of all elements of each subset is , a factor of .
For each subset, we construct  full -packings, which consist of all paths with lengths belonging to the subset.
For example, the  full -packings produced from  are listed below.


\noindent
These  full -packings can be viewed as the results of cyclic rotation of the first one, , called the \emph{base packing}.
More precisely, we first cover the edges of  with two paths of length  and two paths of length  alternatively to form the base packing, and then rotate it to obtain the others.
This ensures that there are no repeated paths among these rotations.
Note that this construction works for the other three subsets: ,  and .
We have a total of  full -packings, which consist of all paths of length .

Next, we shall replace some paths in the full -packings produced from the subset  with the paths of length .
In ,  and  are replaced by ; in ,  and  are replaced by ; in ,  and  are replaced by ; in ,  and  are replaced by ; and so on.
Following this procedure, in ,  and  are replaced by .
Finally, those  paths ( of which are of length  and the others are of length ) that are taken off will form  -packings as below.


\noindent
Hence  is completely partitioned into  -packings, which achieves the bound of Lemma~\ref{lemma:cycle_even_lower} for .
For a general proof, we need the following.
}
\end{example}

\begin{lemma}\label{lemma:partition}
Let  be a positive integer, and let  be a set of positive integers with  for all .
If  for some positive integer , where , then the collection of all paths of lengths belonging to  on the cycle  can be partitioned into  full -packings.
\end{lemma}
\begin{proof}
First, we construct a base -packing by periodically putting a path of length , a path of length , , a path of length  one after another on  in clockwise direction.
Since , the base -packing consists of  paths of length , for each , and then clearly is a full packing.
Then, rotate this base -packing step by step clockwise to produce the other  full -packings.
Since any two paths of length , for each , are completely overlapped if and only if the base -packing is rotated by  or a multiple of  steps, these  full -packings have no repeated paths and thus form a partition of the set of all paths of lengths belonging to .
\end{proof}

We are ready for the main result in this section.

\begin{theorem}\label{thm:cycle_even}
For any integer , 
\end{theorem}
\begin{proof}
By Lemma~\ref{lemma:cycle_even_lower} and Proposition~\ref{pro:upperbound}, it suffices to prove that there exists an ideal path system, , of  such that all paths of  can be partitioned into  -packings.
Let  be the shortest path system of  which contains paths  if  is odd, or  otherwise.

Consider the case that  is odd.
By Lemma~\ref{lemma:partition}, for , all paths with lengths belonging to  can be partitioned into  full -packings.
Since  in the case  is odd, the paths of length less than  will produce  full -packings.
Now, consider the  full -packings associated with the path lengths  and .
For , denote by  the full -packing that contains the paths  and .
For each , if  is even, replace the two paths  of  with ; otherwise, replace the two paths  of  with .
See Fig.~\ref{fig:replacement} for an illustration.
Let  be the set of paths in  which are replaced by a path of length , for .
More precisely,  if  is even, and  otherwise.
The paths in  are rearranged as follows.
For , let  
And, let 
It is easy to see that  form a partition of , and each of them is a -packing.
Hence we ultimately partition  into  -packings.

\begin{figure*}\centering
\includegraphics[width=6in]{Fig-replacement.pdf}
\caption{Replace some paths of  with a path of length .} \label{fig:replacement}
\end{figure*}

The case  is even can be dealt with in a similar fashion.
We omit the detail, and the proof is completed.
\end{proof}

We note here that `shortest' is not a necessary condition for a path system to be ideal.
In , the following path system, see Table~I, is ideal but not shortest, for instance.
The path marked by  indicates a non-shortest path.

\begin{table}\label{tab:counterexample}
\begin{tabular}{|c|c|c|}
\hline
node pair &
\begin{tabular}{c}
path \\
system
\end{tabular} & 
\begin{tabular}{c}
receiving \\
wavelength
\end{tabular} \\ \hline \hline
 &  & 1 \\ \hline
 &  & 2 \\ \hline
 &  & 2 \\ \hline
 &  & 3 \\ \hline
 &  & 3 \\ \hline
 &  & 2 \\ \hline
\end{tabular}
\caption{An example that a non-shortest path system is ideal.}
\end{table}
In the case that  is odd, however, an ideal path systems of  must be perfect.

\begin{corollary}\label{coro:even}
Let  be an odd integer.
If  is an ideal path system of , then  is perfect.
\end{corollary}
\begin{proof}
Suppose  is a non-shortest path system of  satisfying that .
Assume that there are  non-shortest paths in , and let , where  denotes the collection of all paths of length  and  refers to the disjoint union operation.
Let  denote the set of shortest paths of length less than  in .
Then,  where  is the operation that replaces some shortest path with its corresponding non-shortest path having the same endpoints.


Fix a pair of antipodal edges , and let  and  be the number of paths in  which contain  and , respectively, for .
By \eqref{eq:lower_even_1}, it is obvious that . 
Let  and  with the same endpoints, i.e.,  is a shortest path while  is not.
Since  contains at most one of  and , there are two cases as follows.
If  contains one of them, say , then  and .
If  contains neither  nor , then  and .
This concludes that , for some integer .
Note that this fact is true for all distinct pairs of antipodal edges.
Since  contains some non-shortest paths, by \eqref{eq:lower_even_1} there exists some pair of antipodal edges  such that .
In addition, since either  or  occurs on any paths in , one of  and  is contained in at least  distinct paths in .
Hence the path system  needs at least  wavelengths, and thus is not an ideal path system due to Theorem~\ref{thm:cycle_even}.
This completes the proof.
\end{proof}

\section{Global packing numbers of odd cycles}\label{sec:oddcycle}
In this section we deal with the global packing number of . 
Similar to Section~\ref{sec:evencycle}, we use shortest path systems to estimate the lower bound of .
Notice that in  the shortest path between any two nodes is unique, that is,

Hence there is no confusion on choosing the shortest path system.

\begin{lemma}\label{lemma:cycle_odd_lower}
For any integer , 
\end{lemma}
\begin{proof}
Let  be the shortest path system of .
In  there are exactly  paths of length , for . 
Since  for any path system , by Proposition~\ref{pro:lowerbound}, we have

\end{proof}

Now, we introduce an algorithm, named \emph{Intelligent Packing (IP)}, to produce a global packing for the shortest path system of .
We use a  symmetric array, denoted by , to represent the resulting global packing , i.e.,  and  indicate the value .
The addition or subtraction herein is taken modulo .

\medskip

\begin{algorithm}\SetAlgoLined
  \SetKwInOut{Input}{input}\SetKwInOut{Output}{output}
  \Input{the shortest path system of }
  \Output{the maximal visited wavelength index, \emph{total}}
  initialization: , , and  for all pairs  \;
  \While{}
  {
    \For{}
 	{
 		\If{}{ the least available wavelength index\;}
 		\If{}{ the least available wavelength index\;}
 	}
 	update \emph{total} \;
    \;
  }
    \Return \emph{total} \;
  \caption{Intelligent Packing (IP)}
  \label{alg:GGP}
\end{algorithm}

\medskip

The main idea of IP is to greedily assign the smallest free wavelength index to paths of  in the descending order of the path length.
For the paths with the same length, we arrange them in the order of their smaller endpoint labels.
After all paths of lengths lager than or equal to  receiving their wavelength indices, denote by  the maximal visited wavelength index, and by  the set of edges in which the wavelength indexed  is free, for .
For the sake of convenience, elements of , called \emph{idle bands} on wavelength  after the -path round, are represented as ordered pairs , indicating a series of adjacent edges  (mod ), whenever they are all in .
As an illustration, consider the following example: .

\begin{example}
{\rm 
Let  be the shortest path system of .
In the first round () of the IP algorithm, the  paths of length  are arranged in the order:  , , , , , , , , ,  and .
Then, paths  and  receive wavelength index , paths  and  receive wavelength index ,  , path  receives wavelength index .
Fig.~\ref{fig:11}(a) is a graphic representation of the wavelength assignment.
In Fig.~\ref{fig:11}, the - and -axis respectively represent the node labels and wavelength indices, and an  rectangle with bold boundary on the -layer indicates that the corresponding path of length  receives wavelength index .
In addition, the circled numbers represent the order of edges in receiving wavelength indices in each round.
One can see that the maximal visited wavelength index , and the corresponding idle bands , for , and .

\begin{figure*}\centering
\includegraphics[width=6in]{Fig-11new.pdf}
\caption{A round-by-round illustration of Algorithm~\ref{alg:GGP} for .} \label{fig:11}
\end{figure*}

In the second round (), the  paths of length  are arranged in the order:  , , , , , , , , ,  and .
Under the wavelength assignment of the the paths of length , path  can receive wavelength index , path  can receive wavelength index , path  can receive wavelength index , and the others' wavelength indices are illustrated in Figure~\ref{fig:11}(b).
After the first two rounds, one can see that , and the corresponding idle bands are listed as follows.
\begin{itemize}
\item , for .
\item .
\item  for .
\item  for .
\end{itemize}
The remaining three rounds ( and ) are shown in Figure~\ref{fig:11}(c) -- (e).
The maximal visited wavelength index is , attaining the bound proposed in Lemma~\ref{lemma:cycle_odd_lower}.
This implies .
The resulting  is listed in Table~II.  }
\end{example}

\begin{table}[h]\label{tab:C11}
\scriptsize
\begin{tabular}{c|ccccccccccc}
 & 0 & 1 & 2 & 3 & 4 & 5 & 6 & 7 & 8 & 9 & 10 \\ \hline
0 &  & 7 & 13 & 11 & 6 & 1 & 1 & 7 & 13 & 11 & 6 \\
1 & 7 &  & 8 & 14 & 12 & 7 & 2 & 2 & 8 & 14 & 12 \\
2 & 13 & 8 &  & 9 & 15 & 13 & 8 & 3 & 3 & 9 & 15 \\
3 & 11 & 14 & 9 &  & 10 & 11 & 14 & 9 & 4 & 4 & 10 \\
4 & 6 & 12 & 15 & 10 &  & 6 & 12 & 15 & 10 & 5 & 5 \\
5 & 1 & 7 & 13 & 11 & 6 &  & 1 & 7 & 13 & 11 & 6 \\
6 & 1 & 2 & 8 & 14 & 12 & 1 &  & 2 & 8 & 14 & 12 \\
7 & 7 & 2 & 3 & 9 & 15 & 7 & 2 &  & 3 & 9 & 15 \\
8 & 13 & 8 & 3 & 4 & 10 & 13 & 8 & 3 &  & 4 & 10 \\
9 & 11 & 14 & 9 & 4 & 5 & 11 & 14 & 9 & 4 &  & 5 \\
10 & 6 & 12 & 15 & 10 & 5 & 6 & 12 & 15 & 10 & 5 &  \\
\end{tabular}
\normalsize
\caption{The global packing array produced from the IP algorithm for .}
\end{table}

The wavelength arrangement of the IP algorithm can be completely characterized in the following three lemmas.

\begin{lemma}\label{lemma:GP_largelength}
For  we have
\begin{enumerate}[(a)]
\item 
\smallskip
\item ;
\smallskip
\item  \\ 

\end{enumerate}
For  we have
\begin{enumerate}[(d)]
\item , .
\end{enumerate}
\end{lemma}
\begin{proof}
These statements are proved by induction on .
When  (the first round), the paths of length  are considered in the order , , , , , ,  and .
Following the greedy strategy, it is easy to see that , , , ,  and .
Then we have .
Observe that for each , the wavelength index  is used on  edges, in which the only exceptional edge is , i.e., .
Then (a), (b) and (c) hold for .
Consider .
Since  for , the wavelength indices  can not be assigned to the paths of length .
That is, the idle bands on wavelength less than or equal to  will not be changed after the -path round.
Hence (d) holds for .

Assume (a), (b), (c) and (d) hold for , .
Consider .
By induction hypotheses (c) and (d), the longest idle band on wavelength index from  to  after the -path round is of length , which is smaller than .
So, the paths of length  can not be assigned the wavelengths with indices less than or equal to . 
This implies (d): , .
In the wavelength assignment, the paths of length  are considered in the order: , , , , , , , and , , , .
By induction hypothesis (a), the wavelength index  is only occupied by the path  for .
Then, by the greedy selection strategy,  and ,  and , and so on.
As the process goes to  and , 
each wavelength index between  and  is occupied by one path of length  and one path of length , and each wavelength index between  and  is occupied by two paths of length .
Consider the endpoints of these occupied paths.
Let  denote the set of idle bands on wavelength index  so far.
We have

Notice that the longest idle band in \eqref{eq:idle_band} is of length , which is less than  due to .
Hence the remaining unassigned paths of length  must receive wavelength index larger than .
This implies that  for .
Thus, condition (c) holds.
Moreover, this forces us to have , , and then up to .
Then , and thus (a) and (b) also hold.
Hence the result follows by induction.
\end{proof}

\medskip

\begin{lemma}\label{lemma:GP_even}
Let  be an even integer.
We have
\begin{enumerate}[(a)]
\item  , for ;
\smallskip
\item , for ;
\smallskip
\item , for ; 
\smallskip
\item , for ;
\smallskip
\item .
\end{enumerate}
\end{lemma}
\begin{proof}
By Lemma~\ref{lemma:GP_largelength}(c)--(d), the idle bands on wavelength indices less than  after the -path round are of length at most , and each  contains exactly one idle band of length  for .
More precisely, for ,

Consider the wavelength index , where .
By Lemma~\ref{lemma:GP_largelength}(a), only the edges of the path  use the wavelength index .
Hence from the greedy selection strategy we have 

for .
Now, the wavelength index , , is free on edges of the path .
This implies

for .
So we derive (a) by combining \eqref{eq:GP_even_1} and \eqref{eq:GP_even_2}, and then obtain (b) directly from \eqref{eq:idle_even_1}.

Since each wavelength index , for , is occupied by three paths of length  and one path of length , there is no idle bands on it after the -path round, which implies (c).
As the idle band  in \eqref{eq:idle_even_1} is filled with a path of length , we get (d).
Finally, (e) is obtained from the assignment of wavelengths to the paths of length  in (a) and (b) as well as the fact that  by Lemma~\ref{lemma:GP_largelength}.
Then we complete the proof.
\end{proof}	

See Figure~\ref{fig:11}(c) for an example () of Lemma~\ref{lemma:GP_even}.
A parallel result of Lemma~\ref{lemma:GP_even} for the case odd is  is proposed below, where the proof is omitted due to the similarity between these two cases.

\begin{lemma}\label{lemma:GP_odd}
Let  be an odd integer.
We have
\begin{enumerate}[(a)]
\item , for ;
\smallskip
\item , for ;
\smallskip
\item , for ; 
\smallskip
\item , for ;
\smallskip
\item .
\end{enumerate}
\end{lemma}

\medskip

We are ready for the main result of this section. 

\begin{theorem}\label{th:cycle_odd_optimal}
For any integer , 
\end{theorem}
\begin{proof}
By Lemma~\ref{lemma:cycle_odd_lower} and Proposition~\ref{pro:upperbound}, it suffices to show that the Algorithm~\ref{alg:GGP} returns .
After assigning wavelengths to the paths of length , by Lemma~\ref{lemma:GP_even}(e) and Lemma~\ref{lemma:GP_odd}(e), the maximal used wavelength index is , i.e., .
Due to the greedy selection strategy, we only need to prove that for any pair of integers  and  with  and , there exists exactly one wavelength index  such that  and .
As the case  is odd can be dealt with in the same way, we only consider the even case.

First, the assignment of wavelengths to the paths of length  shown in Lemma~\ref{lemma:GP_even}(a)--(b) states that the associated wavelength indices are less than .
Then  for .
Next, by Lemma~\ref{lemma:GP_largelength}(c), the -path round will produce  idle bands:  of them are of length  and the others are of length .
By the recursive relation in Lemma~\ref{lemma:GP_largelength}(d), in order to find all idle bands of length , a fixed integer between  and , it is sufficient to consider the -path and -path rounds.
Plugging  into Lemma~\ref{lemma:GP_largelength}(c) leads to the following:

Therefore, the  idle bands of length  herein are 
Similarly, by plugging  into Lemma~\ref{lemma:GP_largelength}(c), the  idle bands of length  produced from the -path round are 
We find that for any pair of integers  and  with  and , there is exactly one idle band  on some wavelength .
Then the result follows by applying iteratively the greedy selection strategy.
\end{proof}

One can conclude from the proof of Lemma~\ref{lemma:cycle_odd_lower} that, if there exists a path system  such that , then  must consist of shortest paths.
Combining this fact with Theorem~\ref{th:cycle_odd_optimal} yields the following.

\begin{corollary}\label{coro:odd}
If  is an ideal path system of , then  is perfect.
\end{corollary}

\section{Length First Packing Algorithm}\label{sec:simulation}
The optimal global packing for odd cycles is completely characterized by means of Algorithm~\ref{alg:GGP}, IP, in Section~\ref{sec:oddcycle}.
In order to handle general graphs, we extend IP to Algorithm~\ref{alg:LFP}, \emph{Length First Packing (LFP)}.
The idea of LFP is to greedily assign the smallest free wavelength index to paths of  in the descending order of the path length.
In contrast to IP, there is no additional ordering relation, instead a random selection
is adopted among paths with the same length in the LFP scheme.
Clearly, IP can be viewed as a special case of LFP when the objective, , is the (unique) shortest path system of an odd cycle.

In order to demonstrate the power of LFP, we apply it to two classes of networks
in this section: chain networks and ring networks with random and quasi-random traffic
loads.

\begin{algorithm}\SetAlgoLined
  \SetKwInOut{Input}{input}\SetKwInOut{Output}{output}
  \Input{a path system, }
  \Output{the maximal visited wavelength index, \emph{total}}
  initialization: ,  and  \;
  \While{}
  {
    \For{}
 	{
 		\If{}
 		{
 		 the least available wavelength index\;
 		\;
 		}
 	}
 	update \emph{total} \;
    \;
  }
    \Return \emph{total}\;
  \caption{Length First Packing (LFP)}
  \label{alg:LFP}
\end{algorithm}

\subsection{Chain networks}
A chain network (or a path) is another fundamental topology in the study of optical networks \cite{Agbinya_06}.
Let  denote a chain of  nodes.
The LFP algorithm can be used to provide an assignment which yields
the global packing number of , .

\begin{theorem}\label{thm:path}
For any integer ,

\end{theorem}
\begin{proof}
Let  be the unique path system of , as there is only one path that connects any two fixed nodes.
Since the middle-most edge of  occurs on  paths in , we have .

For , let  denote the collection of paths of length .
Obviously, .
The LFP algorithm assigns wavelengths to the path in , the two paths in , the three paths in , and so on, in a greedy fashion.
Notice that there is no addition constraint on the order of those paths which have the same length.
Denote by  the maximal visited wavelength index after the LFP algorithm finishes the wavelength assignment of the paths in .
We claim that 


We only consider the case  is even, as the odd case can be dealt with in the same way.
Let , and denote by  the middle-most edge on .
Since all paths of length  contain , they must receive distinct wavelength indices.
Then .
Now, assume \eqref{eq:path} holds for , where , and consider the paths of length .
In  there are exactly  paths which contain .
For each of those paths in  which do not contain , it is not hard to see that there always exists at least one free wavelength index  with .
Therefore, , and \eqref{eq:path} follows by induction.
Hence , as desired.
\end{proof}

\noindent \emph{Remark.} 
Given a chain, one can easily find a greedy wavelength assignment on its path system which does not produce its global packing number.
Take  as an example.
Let  denote the node set, where  and  are adjacent if .
The following global packing, , needs  wavelengths, while  due to Theorem~\ref{thm:path}.

This example states that the order of paths is the essential issue, if one tries to apply a greedy algorithm to derive the global packing numbers.


\subsection{Ring networks with more random traffic load}
We originally introduced the global packing number as the minimum number of wavelengths
required to support simultaneous communication over a WDM network when the traffic
load is uniformly defined as one request per node pair.  Obviously, this traffic condition can be relaxed so that the global packing number corresponds to the minimal
number of wavelengths required for a general given traffic load for the network.
The LFP algorithm can be applied to these traffic scenarios.

First, we carry out a performance study of the LFP scheme on ring networks in comparison
with the \emph{Random Packing(RP)} scheme.  The latter scheme loops through
the wavelength set once starting with wavelength 1.  For each wavelength, , there
is an iterative loop which randomly selects an unassigned path and assigns to  if it does not cause any violation until no more paths can be assigned to .

We considered ring networks of node sizes  and three kinds of traffic models defined as below:
\begin{itemize}
\item \emph{uniform}: each node pair has exactly one connection;
\item \emph{full-random}: there are  random connections;
\item \emph{quasi-random}: based on the uniform model, there are  extra random connections.
\end{itemize}
In all three models, for any node pair a shortest path connecting the nodes is chosen.
This choice is unique for odd cycles.  In the case that  is even and the nodes are antipodal,
the shortest path is chosen randomly among the two candidates. 
For full-random and quasi-random traffic types,  traffic models are instantiated for each
type; and for each instantiated traffic model, a simulation run consists of  tests 
and the averaged required wavelength number is reported.

\begin{table}[h]\label{tab:uniform}
\begin{tabular}{|c||c|c|c|c|c|c|}
\hline
 &  & LFP & RP \\ \hline \hline 
5 & 3 & 3 & 3.47 \\ \hline
10  & 13 & 13.47 & 14.92 \\ \hline
15 & 28 & 29.69 &  33.05 \\ \hline
20 & 51 & 53.11 & 58.42 \\ \hline
25 & 78 & 82.27 & 90.28 \\ \hline
30 & 113 & 118.08 & 129.29 \\ \hline
35 & 153 & 160.31 & 174.77 \\ \hline
40 & 201 & 209.02 & 227.20 \\ \hline
\end{tabular}
\caption{Global packing performance comparison among the optimal solution, LFP and RP schemes in uniform traffic model.}
\end{table}

Table~III reports the global packing performance for the uniform traffic model.
The columns indicate node sizes, exact values of  which are derived in Section~\ref{sec:evencycle} and Section~\ref{sec:oddcycle}, and the average number of required wavelengths by applying the LFP and RP schemes.
Compare to , the simulation results indicate that the LFP scheme requires
 to  additional wavelengths, and is better than the RP scheme, which requires
an additional  to .

\begin{table}[h]\label{tab:random}
\begin{tabular}{|c||c|c||c|c|}
\hline
\multirow{2}{*}{} &
\multicolumn{2}{c||}{full-random} &
\multicolumn{2}{c|}{quasi-random} \\
\cline{2-5}
  & LFP & RP & LFP & RP \\
\hline \hline
5 & 10.43 & 10.48 & 5.81 & 5.90 \\
\hline
10 & 34.49 & 35.06 & 17.85 & 18.61 \\
\hline
15 & 71.14 & 72.67 & 35.43 & 38.01 \\
\hline
20 & 120.94 & 124.12 & 60.55 & 64.83 \\
\hline
25 & 183.55 & 188.51 & 90.77 & 97.91 \\
\hline
30 & 258.91 & 266.59 & 128.23 & 138.18 \\
\hline
35 & 347.23 & 357.52 & 171.62 & 184.96 \\
\hline
40 & 448.05 & 462.06 & 223.26 & 240.13 \\
\hline
\end{tabular}
\caption{Global packing performance comparison between LFP and RP schemes in full- and quasi-random traffic models.}
\end{table}

Table~IV lists the average number of wavelengths required by applying LFP and RP schemes to the full-random and quasi-random traffic models.
In either case, the LFP scheme is obviously better than the RP scheme; the former uses less than  and  of the wavelengths that are required by the latter when  in the full-random and quasi-random scenarios, respectively.

\medskip



\section{Concluding remarks}\label{sec:conclusion}
The global packing number is a newly defined index on a graph, motivated by
the wavelength assignment problem in a WDM optical network.
This paper focuses on a ring networks, which have underlying graphs defined by cycles.
Based on graph packing method and a newly proposed greedy construction, IP algorithm, we completely characterize the global packing number of a cycle as well as its ideal path system.
Furthermore, the LFP algorithm, which can be viewed as a general version of IP algorithm, has been applied to the chain networks and networks with random traffic loads.
Simulation results show that the LFP scheme is more efficient than the random scheme in wavelength usage.  Results obtained here point to several new directions for future research,
for example, extending the concept of global packing number to networks that allow
wavelength conversion. 

\begin{thebibliography}{1}

\bibitem{Agbinya_06}
J.~I.~Agbinya, "Design considerations of MoHotS and wireless chain networks", Wirel. Pers. Commun. 40 (2006) 91--106.

\bibitem{GCFPFGNP_01}
R.~Gaudino, A.~Carena, V.~Ferrero, A.~Pozzi, V.~De~Feo, P.~Gigante, F.~Neri and P.~Poggiolini, "RINGO: a WDM ring optical packet network demonstrator", ECOC 2001, paper Th.L.2.6, Oct. 2001, Amsterdam, The Netherlands, 620--621.

\bibitem{GSCAC_09}
A.~Giorgetti, N.~Sambo, I.~Cerutti, N.~Andriolli and P.~Castoldi, "Label preference schemes for lightpath provisioning and restoration in distributed GMPLS networks", J. Lightwave Techno. 27 (2009) 688--697.

\bibitem{GY_04}
Y.~Guo and C.~Yuan, "An efficient control protocol for lightpath establishment in wavelength-routed WDM networks", Proc. SPIE  5281 (2004) 326--333.

\bibitem{LXC_05}
K.~Lu, G.~Xiao, I.~Chlamtac, "Analysis of blocking probability for distributed lightpath establishment in WDM optical networks," IEEE/ACM Trans. Netw. 13 (2005) 187--197.

\bibitem{PCF_05}
S.~Pitchumani, I.~Cerutti, A.~Fumagalli, "Destination-initiated wavelength-weighted reservation protocols: Scalable solutions for WDM rings," Proc. IEEE Int. Conf. Communications (ICC), 3 (2005) 1776--1780.



\bibitem{S_10}
D.~Saha, "An intelligent destination initiated reservation protocol for wavelength management in WDM optical networks", Proc. of the ICACT, 2010.

\bibitem{SSA_02}
G.~Sahin, S.~Subramaniam, M.~Azizoglu, "Signaling and capacity assignment for restoration in optical networks," OSA J. Optical Networking, 1 (2002) 188--206.

\bibitem{West_01}
D.~B.~West, Introduction to graph theory, Prentice Hall, Upper Saddle River, NJ 07458, 2001.

\end{thebibliography}

\end{document}
