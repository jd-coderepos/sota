
This section describes our main constructions for solving timed orchestration synthesis problems.
We first translate timed orchestration synthesis problems for LTL
properties to corresponding synthesis problems for safety properties
(Sec.~\ref{sub.sec.monitor})\@. Second,
SMT constraints are generated for the latter problem, whereby existential 
variables quantify over the parameters to be synthesized and universal variables quantify over system states
(Sec.~\ref{sub.sec.parameterized.invariants}). 
Third, the 
SMT constraints are solved by means of two alternating quantifier-free SMT solvers (Sec.~\ref{sub.sec.efsmt}) for each polarity\@.
In order to simplify the exposition below, we omit  as it ranges only over the existentially-quantified parameters  in , and 
concentrate on the properties  and \@. 



\subsection{Transforming Interaction-level to Safety Properties}\label{sub.sec.monitor}

To effectively synthesize parameters such that interaction-level
properties  are satisfied, we adapt bounded
LTL synthesis~\cite{ScheweF07a} to our context. The underlying
strategy is to construct a deterministic progress monitor from
.  The monitor is meant to keep track of the final states
visited in the B\"uchi automaton 
corresponding to  during system execution. To achieve
this, we equip the monitor with a dedicated \sig{risk} state
representing that a final state in  has
been visited for  times. When the risk state is never reached for
all possible runs, all final states in 
are visited finitely often (i.e., less than  times). This
observation is sufficient to conclude that the system satisfies
. This is the intuition behind Algorithm~\ref{alg:int}.

Algorithm~\ref{alg:int} uses  to
be the set of interactions from  and  as a symbol
not within . 
On Line~\ref{l2}, the symbol  is used to mark labels corresponding
to interactions  not appearing in . On Line~\ref{l3} a
deterministic progress monitor is constructed  by unrolling
 via function , which is similar to the
approach in bounded LTL synthesis~\cite{ScheweF07a}. Consequently, we
omit it and instead provide a high-level description of what it does (see below example for understanding):
Starting from the initial state of ,
 is used to \emph{unroll} all traces
of  and to create a deterministic
monitor . Each location\footnote{To avoid
  ambiguity, we call a state in the monitor component ``location''
  while keeping the name ``state'' for B\"uchi automaton.}  in
 records the set of states being visited in the
B\"uchi automaton. For each location, the number of times a final state
in  has been visited previously is counted.
The algorithm maintains a queue of unprocessed locations. For each
unprocessed location in the queue, every interaction  is selected to create a successor
location respectively. A state  is stored in the successor
location, if state  is in the unprocessed location and if in the
post-processed B\"uchi automaton, a transition from  to  via
edge labeled  exists. In addition, the number of visited final
states is updated. Whenever a final state in
 has been visited  times, the unroll
process replaces the location of  by
\sig{risk}, a dedicated location with no outgoing edges. 

Once the monitor is constructed, an augmented system  is
created from  (Line~\ref{l6}). The interaction set in the
augmented system  is the one from Line~\ref{l5}
where all property-unrelated interactions  are marked with
. Finally, on Line~\ref{l7} the state predicate 
expressing the deadlock condition is constructed from the new set of
interactions.
\vspace*{.3cm}
\begin{algorithm}
\vspace*{.1cm}
\caption{Translate  into  and construct a monitored system }\label{alg:int}
\vspace*{.1cm}
\vspace*{.1cm}
\begin{algorithmic}[1]
\Statex \textbf{Input: }, , 
\Statex \textbf{Output: }
\vspace*{.1cm}
\State construct a B\"uchi automaton  for the negated property of  \label{l1}
\State postprocess  by replacing every label  with \label{l2}
\State  \textbf{monitor}() 
\label{l3}
\State  {} ? {} : {} \label{l5}
\State  \label{l6}
\State   \label{l7}
\State \textbf{return } , \text{ } 
\end{algorithmic}
\end{algorithm}
\mycomment{
\vspace*{.3cm}
\begin{algorithm}
\vspace*{.1cm}
\caption{Deterministic Progress Monitor}\label{alg:mon}
\vspace*{.1cm}
\vspace*{.1cm}
\begin{algorithmic}[1]
\Statex \textbf{Input: }, 
\Statex \textbf{Output: }
\vspace*{.1cm}
\State todo: extract an algo from the description above
\State  
\State ...
\end{algorithmic}
\vspace*{.3cm}
\end{algorithm}
\vspace*{.3cm}
}
\begin{example}
We illustrate the steps of the algorithm~\ref{alg:int} using the robot running example. 
Figure~\ref{fig:negated.property}-(a) illustrates the
result  for property 
(Line~\ref{l1}), and  (b) displays the result after post-processing
(Line~\ref{l2})\@. 

To illustrate the result of unrolling in Line~\ref{l3},
Figure~\ref{fig:negated.property}-(c) shows it for . There, the
initial location stores \{[(0)]\}, where [(0)] is to
indicate that at , one has not yet reached  previously. When
the initial location \{[]\} takes interaction
\sig{take1l}, it goes to \{[], []\}, as in
Figure~\ref{fig:negated.property}-(b), state  can move to 
or . Notice  that it a destination location has possibly been created previously. 
For example, in
Figure~\ref{fig:negated.property}-(c), for the initial location
\{[]\} to take interaction \#, it goes back to
\{[]\}. For
\{[], []\} to take interaction \sig{take2l},
it moves to a new location \{[], [],
[]\}. This new location is then replaced by \sig{risk},
as in this example, we have .

As for the new interaction set in the monitored system, we show two
examples with respect to whether the interaction is in :


\end{example}	

\begin{figure}[htp]
\centering
\includegraphics[width=\columnwidth]{fig/NegatedProperty}
\caption{(a)(b) B\"uchi automaton for  before and
  after postprocessing. (c) 
.
}
\label{fig:negated.property}
\end{figure}
Notice that the introduction of  symbol simplifies the unroll construction in bounded synthesis. 
Another difference to vanilla bounded synthesis is that, in the context of unrolling, every state
has outgoing edges of size . 
In contrast, in  bounded LTL synthesis, each  is viewed as a Boolean
variable, which creates, in the worst case, on the order of  outgoing edges.

The following result reduces timed orchestration synthesis for interaction-level properties to a corresponding timed orchestration synthesis problem on state-based properties only.
\begin{lemma}
\label{lem:correct}
  Given  an assignment of ,  satisfies  if all time runs of  reach neither the location \sig{risk} in  nor a state where  holds. 


\end{lemma}


\proof (Sketch) Assume that any time run  in
 does not visit location \sig{risk} or
any state where  holds. We need to show that
 satisfies .
\begin{enumerate}
\item 	Because  does not hold and because time runs are maximal, any such time run  is infinite. 
\item From an infinite time run , we show that  defines an
  -word : as  is
  never reached, 
  is an invariant for all reachable states. Recall that 
  is the necessary condition for enabling a location to trigger
   by only allowing \emph{finite-time} evolving. Therefore,
  for all reachable states, one of the interaction (discrete jump)
  must appear after finite time. Thus,  contains infinitely many
  discrete jumps and consequently,  defines an -word
  .
\item By construction, every location in the monitor has edges labeled
  in , and \# marks each
  property-unrelated interaction in  (Line~\ref{l5}). From this observation,
  together with the fact that  does not
  restrict the behavior of , we have that a time
  run  in  not reaching \sig{risk}
  is bisimilar to a time run  in , with
   and  defining the same -word .
\item Recall that  is an unroll of
  . From this, together with the
  existence of  and the fact that while running
   in  no final state is
  reached infinitely many times, we have that  does not
  satisfy . Consequently, we can conclude that for
  every time run  in , the corresponding
   satisfies . \qed
\end{enumerate}
\newcommand{\mk}{k^*}
By Lemma~\ref{lem:correct}, it is sufficient to only consider safety
properties when performing orchestration synthesis.
Notice howerver that,  if for a given fixed  Lemma~\ref{lem:correct} fails for all possible
assignments  then one may not conclude that no solution exists for
the orchestration synthesis problem as there might be a larger  for
which Lemma~\ref{lem:correct} does hold.



Next we show that it is futile to go beyond a reasonable bound. More precisely, if the domains of the parameters in the input system  are bounded, then one can effectively compute a limit  on  such that: if Lemma~\ref{lem:correct} is not applicable for  then it is also not applicable for any strictly larger \@. 













\begin{lemma}\label{lem:k}
Let all parameters in  have bounded integer domains with common
upper bound  and  be the number of regions in 
when all parameters within the location and guard conditions are
assigned . Let  be the number
of locations in , and  be the
number of discrete location combinations in , i.e., . Finally, let .

Given  an
assignment of , if there exists a time run of
 reaching either \sig{risk} in
 or a state where 
holds, then  does not satisfy .







	
\end{lemma}

\proof (Sketch) Let  be a time run of
 which reaches either \sig{risk} in
 or a state where 
holds. We have two cases:
\begin{itemize}
\item \textbf{Case 1}: \sig{risk} is not reached, equally,  reaches a state where  holds. The deadlock of  under  is irrelevant to the monitor component, as the monitor component does not hinder any execution apart from \sig{risk}.  Therefore,  contains deadlock states, and 
  does not satisfy , as reaching a deadlock state means that one can not create an  -word from that time run. 
	






	\item \textbf{Case 2}: \sig{risk} is reached in the -th
          unroll. We need to show that with a prefix of a violating
          time run  in  that
          visited one final state  in
           for  times, one proves,
          by tailoring a fragment of , the existence of a time
          run  in  such that the
          -word of , when applying to
          , guarantees to visit 
          arbitrary many times.  This is done with the help of two results:
          (1) the number of regions in a timed automaton is finite,
          and (2) the pigeonhole principle.
concrete values, the system is a timed automaton and the
          number of regions is finite. The total number of regions is
          bounded by . Recall that the executions in  reflect executions in the B\"uchi automaton
          . Consider a discrete jump in
          the system. With a specific destination state  in
          , one can actually capture a
          discrete jump in  by only
          viewing its change in regions and locations. To reflect such
          changes, we use tuples , where
           and  are source and destination
          regions in ,  and
           are source and destination location in
          ,  is the source state in
           with interaction .  The total number for all such  tuples is bounded by .  Therefore, when
          the violating  visits a particular final state  in
           for  times, in the
          corresponding region representation, one particular tuple
           should
          have appeared twice (due to pigeonhole principle). The clock
          valuations associated to , resp.  may
          be different in the two tuples. However, the corresponding
          states are region-equivalent and consequently
          bisimilar. Thanks to this,  can evolve
          region-bisimilarly until the tuple  appears for the third time, and so on. While
          repeating this pattern a time run visiting  infinitely
          often is constructed.






		
		
	
		
		




		


	
		


	
\end{itemize}










