
This section describes our main constructions for solving timed orchestration synthesis problems.
We first translate timed orchestration synthesis problems for LTL
properties to corresponding synthesis problems for safety properties
(Sec.~\ref{sub.sec.monitor})\@. Second,
$\exists\forall$SMT constraints are generated for the latter problem, whereby existential 
variables quantify over the parameters to be synthesized and universal variables quantify over system states
(Sec.~\ref{sub.sec.parameterized.invariants}). 
Third, the 
$\exists\forall$SMT constraints are solved by means of two alternating quantifier-free SMT solvers (Sec.~\ref{sub.sec.efsmt}) for each polarity\@.
In order to simplify the exposition below, we omit $\phi_C$ as it ranges only over the existentially-quantified parameters  in $V$, and 
concentrate on the properties $\phi_{state}$ and $\phi_{int}$\@. 



\subsection{Transforming Interaction-level to Safety Properties}\label{sub.sec.monitor}

To effectively synthesize parameters such that interaction-level
properties $\phi_{int}$ are satisfied, we adapt bounded
LTL synthesis~\cite{ScheweF07a} to our context. The underlying
strategy is to construct a deterministic progress monitor from
$\phi_{int}$.  The monitor is meant to keep track of the final states
visited in the B\"uchi automaton $\mathcal{A}_{\neg \phi_{int}}$
corresponding to $\neg\phi_{int}$ during system execution. To achieve
this, we equip the monitor with a dedicated \sig{risk} state
representing that a final state in $\mathcal{A}_{\neg \phi_{int}}$ has
been visited for $k$ times. When the risk state is never reached for
all possible runs, all final states in $\mathcal{A}_{\neg \phi_{int}}$
are visited finitely often (i.e., less than $k$ times). This
observation is sufficient to conclude that the system satisfies
$\phi_{int}$. This is the intuition behind Algorithm~\ref{alg:int}.

Algorithm~\ref{alg:int} uses $\Sigma_{\phi_{int}} \subseteq \Sigma$ to
be the set of interactions from $\phi_{int}$ and $\#$ as a symbol
not within $\Sigma$. 
On Line~\ref{l2}, the symbol $\#$ is used to mark labels corresponding
to interactions $\sigma$ not appearing in $\pintt$. On Line~\ref{l3} a
deterministic progress monitor is constructed $\mon{k}$ by unrolling
$\mathcal{A}_{\neg\phi_{int}}$ via function $\textbf{monitor}(\mathcal{A}_{\neg\phi_{int}}, k)$, which is similar to the
approach in bounded LTL synthesis~\cite{ScheweF07a}. Consequently, we
omit it and instead provide a high-level description of what it does (see below example for understanding):
Starting from the initial state of $\mathcal{A}_{\neg\phi_{int}}$,
$\Sigma_{\phi_{int}}\cup \{\#\}$ is used to \emph{unroll} all traces
of $\mathcal{A}_{\neg\phi_{int}}$ and to create a deterministic
monitor $C_{\neg \phi_{int},k}$. Each location\footnote{To avoid
  ambiguity, we call a state in the monitor component ``location''
  while keeping the name ``state'' for B\"uchi automaton.}  in
$C_{\neg \phi_{int},k}$ records the set of states being visited in the
B\"uchi automaton. For each location, the number of times a final state
in $\mathcal{A}_{\neg\phi_{int}}$ has been visited previously is counted.
The algorithm maintains a queue of unprocessed locations. For each
unprocessed location in the queue, every interaction $\sigma \in
\Sigma_{\phi_{int}} \cup \{\#\}$ is selected to create a successor
location respectively. A state $s'$ is stored in the successor
location, if state $s$ is in the unprocessed location and if in the
post-processed B\"uchi automaton, a transition from $s$ to $s'$ via
edge labeled $\sigma$ exists. In addition, the number of visited final
states is updated. Whenever a final state in
$\mathcal{A}_{\neg\phi_{int}}$ has been visited $k$ times, the unroll
process replaces the location of $C_{\neg \phi_{int},k}$ by
\sig{risk}, a dedicated location with no outgoing edges. 

Once the monitor is constructed, an augmented system $\mathcal{S}_{inv,k}$ is
created from $\mathcal{S}$ (Line~\ref{l6}). The interaction set in the
augmented system $\mathcal{S}_{inv,k}$ is the one from Line~\ref{l5}
where all property-unrelated interactions $\sigma$ are marked with
$\#$. Finally, on Line~\ref{l7} the state predicate $\phi_{deadlock}$
expressing the deadlock condition is constructed from the new set of
interactions.
\vspace*{.3cm}
\begin{algorithm}
\vspace*{.1cm}
\caption{Translate $\pintt$ into $\deadd$ and construct a monitored system $\systik$}\label{alg:int}
\vspace*{.1cm}
\vspace*{.1cm}
\begin{algorithmic}[1]
\Statex \textbf{Input: }$\syst$, $\pintt$, $k$
\Statex \textbf{Output: }$\systik, \deadd$
\vspace*{.1cm}
\State construct a B\"uchi automaton $\mathcal{A}_{\neg\phi_{int}}$ for the negated property of  $\phi_{int}$\label{l1}
\State postprocess $\mathcal{A}_{\neg\phi_{int}}$ by replacing every label $\neg \sigma$ with $\Sigma_{\phi_{int}}\setminus \{\sigma\} \cup \{\#\}$\label{l2}
\State $\mon{k} := $ \textbf{monitor}($\mathcal{A}_{\neg\phi_{int}}, k$) 
\label{l3}
\State $\Delta_{inv,k}(\sigma):=$ {$\sigma \in \Sigma_{\phi_{int}}$} ? {$\Delta(\sigma)\cup\{C_{\neg\phi_{int},k}.\sigma\}$} : {$\Delta(\sigma)\cup\{C_{\neg\phi_{int},k}.\#\}$} \label{l5}
\State $\mathcal{S}_{inv,k} = (V, C\cup \{C_{\neg\phi_{int},k}\}, \Sigma, \Delta_{inv,k})$ \label{l6}
\State $\phi_{deadlock} :=\bigwedge_{\sigma  \in \Sigma} \neg \tEn{\sigma}$  \label{l7}
\State \textbf{return } $\mathcal{S}_{inv,k}$, \text{ } $\deadd$
\end{algorithmic}
\end{algorithm}
\mycomment{
\vspace*{.3cm}
\begin{algorithm}
\vspace*{.1cm}
\caption{Deterministic Progress Monitor}\label{alg:mon}
\vspace*{.1cm}
\vspace*{.1cm}
\begin{algorithmic}[1]
\Statex \textbf{Input: }$\mathcal{A}_{\neg\phi_{int}}$, $k$
\Statex \textbf{Output: }$\mon{k}$
\vspace*{.1cm}
\State todo: extract an algo from the description above
\State $tovisit.push(q_0)$ 
\State ...
\end{algorithmic}
\vspace*{.3cm}
\end{algorithm}
\vspace*{.3cm}
}
\begin{example}
We illustrate the steps of the algorithm~\ref{alg:int} using the robot running example. 
Figure~\ref{fig:negated.property}-(a) illustrates the
result $\mathcal{A}_{\neg \phi_{prompt}}$ for property $\phi_{prompt}$
(Line~\ref{l1}), and  (b) displays the result after post-processing
(Line~\ref{l2})\@. 

To illustrate the result of unrolling in Line~\ref{l3},
Figure~\ref{fig:negated.property}-(c) shows it for $k=1$. There, the
initial location stores \{$s_0$[$s_3$(0)]\}, where [$s_3$(0)] is to
indicate that at $s_0$, one has not yet reached $s_3$ previously. When
the initial location \{$s_0$[$s_3(0)$]\} takes interaction
\sig{take1l}, it goes to \{$s_0$[$s_3(0)$], $s_1$[$s_3(0)$]\}, as in
Figure~\ref{fig:negated.property}-(b), state $s_0$ can move to $s_0$
or $s_1$. Notice  that it a destination location has possibly been created previously. 
For example, in
Figure~\ref{fig:negated.property}-(c), for the initial location
\{$s_0$[$s_3(0)$]\} to take interaction \#, it goes back to
\{$s_0$[$s_3(0)$]\}. For
\{$s_0$[$s_3(0)$], $s_1$[$s_3(0)$]\} to take interaction \sig{take2l},
it moves to a new location \{$s_0$[$s_3(0)$], $s_3$[$s_3(1)$],
$s_2$[$s_3(0)$]\}. This new location is then replaced by \sig{risk},
as in this example, we have $k=1$.

As for the new interaction set in the monitored system, we show two
examples with respect to whether the interaction is in $\phi_{prompt}$:
\begin{align*}
& \Delta_{inv,k}(\sig{take1l}) = \{\phio_1.\port{occupy1-l}, \buffer_1.\port{take}, C_{\neg\phi_{prompt}}.\port{take1l}\}\\
& \Delta_{inv,k}(\sig{take2r}) = \{\phio_2.\port{occupy2-r}, \buffer_1.\port{take}, C_{\neg \phi_{prompt}}.\#\}
\end{align*}

\end{example}	

\begin{figure}[htp]
\centering
\includegraphics[width=\columnwidth]{fig/NegatedProperty}
\caption{(a)(b) B\"uchi automaton for $\neg \phi_{prompt}$ before and
  after postprocessing. (c) 
$\mathcal{C}_{\neg \phi_{prompt}, 1}$.
}
\label{fig:negated.property}
\end{figure}
Notice that the introduction of $\#$ symbol simplifies the unroll construction in bounded synthesis. 
Another difference to vanilla bounded synthesis is that, in the context of unrolling, every state
has outgoing edges of size $|\Sigma_{\phi_{int}}\cup \{\#\}|$. 
In contrast, in  bounded LTL synthesis, each $\sigma$ is viewed as a Boolean
variable, which creates, in the worst case, on the order of $2^{|\Sigma|}$ outgoing edges.

The following result reduces timed orchestration synthesis for interaction-level properties to a corresponding timed orchestration synthesis problem on state-based properties only.
\begin{lemma}
\label{lem:correct}
  Given $\vec{v}$ an assignment of $V$, $\mathcal{S}(\vec{v})$ satisfies $\phi_{int}$ if all time runs of $\mathcal{S}_{inv,k}(\vec{v})$ reach neither the location \sig{risk} in $C_{\neg\phi_{int},k}$ nor a state where $\phi_{deadlock}(\vec{v})$ holds. 


\end{lemma}


\proof (Sketch) Assume that any time run $\rho$ in
$\mathcal{S}_{inv,k}(\vec{v})$ does not visit location \sig{risk} or
any state where $\phi_{deadlock}(\vec{v})$ holds. We need to show that
$\mathcal{S}(\vec{v})$ satisfies $\phi_{int}$.
\begin{enumerate}
\item 	Because $\dead$ does not hold and because time runs are maximal, any such time run $\rho$ is infinite. 
\item From an infinite time run $\rho$, we show that $\rho$ defines an
  $\omega$-word $\rho_{\Sigma}$: as $\phi_{deadlock}(\vec{v})$ is
  never reached, $\bigvee_{\sigma \in \Sigma} \tEn{\sigma}(\vec{v})$
  is an invariant for all reachable states. Recall that $\tEn{\sigma}$
  is the necessary condition for enabling a location to trigger
  $\sigma$ by only allowing \emph{finite-time} evolving. Therefore,
  for all reachable states, one of the interaction (discrete jump)
  must appear after finite time. Thus, $\rho$ contains infinitely many
  discrete jumps and consequently, $\rho$ defines an $\omega$-word
  $\rho_{\Sigma}$.
\item By construction, every location in the monitor has edges labeled
  in $\sigma \in \Sigma_{\phi_{int}} \cup \{\#\}$, and \# marks each
  property-unrelated interaction in $\Sigma\setminus
  \Sigma_{\phi_{int}}$ (Line~\ref{l5}). From this observation,
  together with the fact that $\mathcal{S}_{inv,k}(\vec{v})$ does not
  restrict the behavior of $\mathcal{S}(\vec{v})$, we have that a time
  run $\rho$ in $\mathcal{S}_{inv,k}(\vec{v})$ not reaching \sig{risk}
  is bisimilar to a time run $\rho'$ in $\mathcal{S}(\vec{v})$, with
  $\rho$ and $\rho'$ defining the same $\omega$-word $\rho_{\Sigma}$.
\item Recall that $C_{\neg\phi_{int},k}$ is an unroll of
  $\mathcal{A}_{\neg\phi_{int}}$. From this, together with the
  existence of $\rho_{\Sigma}$ and the fact that while running
  $\rho_{\Sigma}$ in $\mathcal{A}_{\neg\phi_{int}}$ no final state is
  reached infinitely many times, we have that $\rho_{\Sigma}$ does not
  satisfy $\neg\phi_{int}$. Consequently, we can conclude that for
  every time run $\rho'$ in $\mathcal{S}(\vec{v})$, the corresponding
  $\rho_{\Sigma}$ satisfies $\phi_{int}$. \qed
\end{enumerate}
\newcommand{\mk}{k^*}
By Lemma~\ref{lem:correct}, it is sufficient to only consider safety
properties when performing orchestration synthesis.
Notice howerver that,  if for a given fixed $k$ Lemma~\ref{lem:correct} fails for all possible
assignments $\vec{v}$ then one may not conclude that no solution exists for
the orchestration synthesis problem as there might be a larger $k$ for
which Lemma~\ref{lem:correct} does hold.



Next we show that it is futile to go beyond a reasonable bound. More precisely, if the domains of the parameters in the input system $\mathcal{S}$ are bounded, then one can effectively compute a limit $\mk$ on $k$ such that: if Lemma~\ref{lem:correct} is not applicable for $\mk$ then it is also not applicable for any strictly larger $k$\@. 













\begin{lemma}\label{lem:k}
Let all parameters in $\syst$ have bounded integer domains with common
upper bound $\lambda$ and $\delta$ be the number of regions in $\syst$
when all parameters within the location and guard conditions are
assigned $\lambda$. Let $|\mathcal{A}_{\neg\phi_{int}}|$ be the number
of locations in $\mathcal{A}_{\neg\phi_{int}}$, and $\eta$ be the
number of discrete location combinations in $\mathcal{S}$, i.e., $\eta
= |Q_1||Q_2|\dots|Q_m|$. Finally, let $\mk =
\delta^2\eta^2|\mathcal{A}_{\neg\phi_{int}}||\Sigma|+1$.

Given $\vec{v}$ an
assignment of $V$, if there exists a time run of
$\mathcal{S}_{inv,\mk}(\vec{v})$ reaching either \sig{risk} in
$C_{\neg\phi_{int},\mk}$ or a state where $\phi_{deadlock}(\vec{v})$
holds, then $\mathcal{S}(\vec{v})$ does not satisfy $\phi_{int}$.







	
\end{lemma}

\proof (Sketch) Let $\rho$ be a time run of
$\mathcal{S}_{inv,\mk}(\vec{v})$ which reaches either \sig{risk} in
$C_{\neg\phi_{int},\mk}$ or a state where $\phi_{deadlock}(\vec{v})$
holds. We have two cases:
\begin{itemize}
\item \textbf{Case 1}: \sig{risk} is not reached, equally, $\rho$ reaches a state where $\phi_{deadlock}(\vec{v})$ holds. The deadlock of $\mathcal{S}_{inv,\mk}(\vec{v})$ under $\rho$ is irrelevant to the monitor component, as the monitor component does not hinder any execution apart from \sig{risk}.  Therefore, $\mathcal{S}(\vec{v})$ contains deadlock states, and 
 $\mathcal{S}(\vec{v})$ does not satisfy $\phi_{int}$, as reaching a deadlock state means that one can not create an  $\omega$-word from that time run. 
	






	\item \textbf{Case 2}: \sig{risk} is reached in the $\mk$-th
          unroll. We need to show that with a prefix of a violating
          time run $\rho$ in $\mathcal{S}_{inv,\mk}(\vec{v})$ that
          visited one final state $s$ in
          $\mathcal{A}_{\neg\phi_{int}}$ for $\mk$ times, one proves,
          by tailoring a fragment of $\rho$, the existence of a time
          run $\rho'$ in $\mathcal{S}(\vec{v})$ such that the
          $\omega$-word of $\rho'$, when applying to
          $\mathcal{A}_{\neg\phi_{int}}$, guarantees to visit $s$
          arbitrary many times.  This is done with the help of two results:
          (1) the number of regions in a timed automaton is finite,
          and (2) the pigeonhole principle.
concrete values, the system is a timed automaton and the
          number of regions is finite. The total number of regions is
          bounded by $\delta$. Recall that the executions in $C_{\neg
            \phi_{int},k}$ reflect executions in the B\"uchi automaton
          $\mathcal{A}_{\neg\phi_{int}}$. Consider a discrete jump in
          the system. With a specific destination state $s$ in
          $\mathcal{A}_{\neg\phi_{int}}$, one can actually capture a
          discrete jump in $\mathcal{S}_{inv,k}(\vec{v})$ by only
          viewing its change in regions and locations. To reflect such
          changes, we use tuples $\langle(r_{source}, l_{source},
          s_{source}),\sigma,(r_{dest}, l_{dest}, s)\rangle$, where
          $r_{source}$ and $r_{dest}$ are source and destination
          regions in $\mathcal{S}(\vec{v})$, $l_{source}$ and
          $l_{dest}$ are source and destination location in
          $\mathcal{S}(\vec{v})$, $s_{source}$ is the source state in
          $\mathcal{A}_{\neg\phi_{int}}$ with interaction $\sigma \in
          \Sigma$.  The total number for all such $\langle(r_{source},
          l_{source}, s_{source}),\sigma,(r_{dest}, l_{dest},
          s)\rangle$ tuples is bounded by $\delta^2 \eta^2
          |\mathcal{A}_{\neg\phi_{int}}| |\Sigma|$.  Therefore, when
          the violating $\rho$ visits a particular final state $s$ in
          $\mathcal{A}_{\neg\phi_{int}}$ for $\mk$ times, in the
          corresponding region representation, one particular tuple
          $\langle(r_{source}, l_{source},
          s_{source}),\sigma,(r_{dest}, l_{dest}, s)\rangle$ should
          have appeared twice (due to pigeonhole principle). The clock
          valuations associated to $l_{source}$, resp. $l_{dest}$ may
          be different in the two tuples. However, the corresponding
          states are region-equivalent and consequently
          bisimilar. Thanks to this, $\mathcal{S}$ can evolve
          region-bisimilarly until the tuple $\langle(r_{source},
          l_{source}, s_{source}),\sigma,(r_{dest}, l_{dest},
          s)\rangle$ appears for the third time, and so on. While
          repeating this pattern a time run visiting $s$ infinitely
          often is constructed.






		
		
	
		
		




		


	
		


	
\end{itemize}










