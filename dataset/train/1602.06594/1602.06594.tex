
\documentclass[letterpaper, 10 pt, conference]{ieeeconf}  

\IEEEoverridecommandlockouts                              
\overrideIEEEmargins                                      

\usepackage{times} \usepackage{amsmath} \usepackage{amssymb}  \usepackage{algorithmic}
\usepackage{algorithm}

\newcommand{\st}{ \;|\;}
\newcommand{\sfour}{ \;\;\;\; }
\newcommand{\es}{x}
\newcommand{\Lag}{{\cal L}}
\newcommand{\C}{{\cal C}}
\newcommand{\B}{{\cal B}}
\newcommand{\J}{{\cal J}}
\newcommand{\Z}{{\mathbb Z}}
\newcommand{\F}{\mathbb{F}}
\newcommand{\supp}{\mbox{supp }}
\newcommand{\spark}{\mbox{spark }}
\newcommand{\dubbel}[1]{{\mathbb #1}}
\newcommand{\X}{{\cal X}}
\newcommand{\A}{{\cal A}}
\newcommand{\sbold}{\mathbf{s}}
\newcommand{\wbold}{\mathbf{w}}
\newcommand{\ebold}{\mathbf{e}}
\newcommand{\bbold}{\mathbf{b}}
\newcommand{\ybold}{\mathbf{y}}
\newcommand{\ubold}{\mathbf{u}}
\newcommand{\etabold}{\mathbf{\boldsymbol{\eta}}}
\newcommand{\rbold}{\mathbf{r}}
\newcommand{\beq}{}
\newcommand{\bmat}{\left[ \begin{array}}
\newcommand{\emat}{\end{array} \right]}
\newcommand{\twee}[2]{\left[ #1 \sfour #2 \right]}
\newcommand{\COL}{\mathrm{col }\;}
\newcommand{\AND}{\;\mbox{and }}
\newcommand{\FOR}{\;\;\mbox{for }}
\newcommand{\WITH}{\;\mbox{with }}
\newcommand{\DET}{\mathrm{det }\;}
\newcommand{\DIAG}{\mathrm{diag }\;}
\newcommand{\DEG}{\mathrm{deg}\;}
\newcommand{\WDEG}{\mathrm{wdeg }\;}
\newcommand{\MAX}{\;\mbox{max }}
\newcommand{\MIN}{\;\mbox{min }}
\newcommand{\ORD}{\mathrm{ord }\;}
\newcommand{\MOD}{\;\;\mbox{mod }\;}
\newcommand{\IM}{\;\mbox{im }}
\newcommand{\DIM}{\mathrm{dim }\;}
\newcommand{\RANK}{\mathrm{rank}\;}
\newcommand{\PDIM}{p\mathrm{dim }\;}
\newcommand{\PRANK}{p\mathrm{rank}\;}
\newcommand{\PSPAN}{p\mathrm{-span }\;}
\newcommand{\SPAN}{\;\mathrm{span }\;}
\newcommand{\ellL}{L}
\newcommand{\Sseq}{S_1 , \ldots , S_N}
\newcommand{\param}{\xi}

\newtheorem{definition}{Definition}[section]
\newtheorem{Remark}{Remark}[section]
\newtheorem{thm}{Theorem}[section]
\newtheorem{lem}[thm]{Lemma}
\newtheorem{cor}[thm]{Corollary}
\newtheorem{example}[thm]{Example}
\newtheorem{prop}[thm]{Property}
\newtheorem{rem}[thm]{Remark}


\title{\LARGE \bf
Vulnerability of linear systems against sensor attacks--a system's security index
}


\author{Michelle S. Chong and Margreta Kuijper\thanks{Michelle S. Chong is with the Department of Automatic Control, Lund University, SE-221 00 Lund, Sweden.
        {\tt\small michelle.chong@control.lth.se}}\thanks{Margreta Kuijper is with the Department of Electrical and Electronic Engineering, University of Melbourne, Australia.
        {\tt\small mkuijper@unimelb.edu.au}}}


\begin{document}



\maketitle
\thispagestyle{empty}
\pagestyle{empty}


\begin{abstract}
The `security index' of a discrete-time LTI system under sensor attacks is introduced as a quantitative measure on the security of an observable system. We derive ideas from error control coding theory to provide sufficient conditions for attack detection and correction. 
\end{abstract}


\section{Introduction}
The security of control systems against adversarial attacks is a challenge to maintain when the adversary knows the workings of any component of the system and has garnered access, with the malicious intent of causing disruption. This has lead to a proliferation of works in tackling this issue, in particular in detecting the occurrence of an attack \cite{pasqualetti2013attack, pasqualetti2012attack, pasqualetti2015divide}, or in designing resilient control or estimation algorithms, see \cite{fawziTD14, shoukry2013event, ChongWakaikiHespanhaACC15, chenKarMouraICASSP15, sandbergTJ2010} and many more.

In this paper, we concentrate on LTI systems where the sensing component has been compromised by the attacker, who has full knowledge of the system. The vulnerability of the sensors is modelled by an additive attack signal to the sensor measurements, which is non-zero when the particular sensor is compromised. Inspired by ideas in coding theory, we introduce the notion of the `security index' for linear systems, a quantitative measure of the vulnerability of a system to sensor attacks. While ideas from error control coding have already been employed to this context in recent literature \cite{fawziTD14}, our aim is to further strengthen this link. Our notion of a `security index' is formulated based on the measurement time series from all sensors and is analogous to the notion of the `minimum distance' of a code in error control coding theory. We demonstrate that by using ideas from coding theory, the formulation simplifies the approach in \cite{fawziTD14}, leading to new results. Particularly, we express the `security index' of a system in terms of different representations of the system concerned. 
 
Previous works in state estimation for systems under sensor attacks include \cite{ChongWakaikiHespanhaACC15, fawziTD14, chenKarMouraICASSP15, pasqualetti2013attack, pasqualetti2015divide, pajic2015attack}. There is a consensus with \cite{fawziTD14} and \cite{ChongWakaikiHespanhaACC15} that the states of an LTI system can only be reconstructed if strictly less than half of the sensors are under attack. We will see in this paper that this condition is also derived when approached with ideas from coding theory. Other related works are \cite{HendrickxJJSS14, sandbergTJ2010} which focus on power networks. It is in this specialised setting that the authors of \cite{sandbergTJ2010} introduce the terminology `security index', which we adopted for a broader context in this paper. The presence of measurement noise has been considered in \cite{pajic2015attack}, which we do not consider, but is the subject of further work. 
\newline
\noindent{\em Notation}: We denote the set of integers and complex numbers as  and , respectively. The notation  is used to denote the set of positive integers including .   


\section{Problem formulation}
We consider a discrete-time, observable linear time-invariant (LTI) system  given by a  state matrix  and a  observation matrix , defined as follows:

The {\em behavior}  of the system is defined as the set of all possible output trajectories  that satisfy the system's equations for some initial condition . Due to the time-invariant finite dimensional nature of the underlying system, the behavior  has the following two properties:
\begin{itemize}
\item  is left-shift invariant: if  then , where the shift operator  is defined via .
\item  is autonomous: there exists  such that for any  and  we have that  implies that .
\end{itemize} 
We assess the vulnerability of an LTI system via its measurable outputs, which may have been compromised by an attacker. While the usual assumption for many applications is that the matrix  has full row rank, we do not make such an assumption in our setting. This setting occurs in the case where each sensor is measuring a local part of the system, such as in sensor networks implemented in a large geographical location. To aid in the introduction of a measure of a system's security against sensor attacks, we define the following for a system's trajectory.

\begin{definition} \label{def:supp_y_trajectory} The {\em support} of a trajectory , denoted by , is defined as the set of indices  in  such that its component trajectory  is not the zero trajectory. 
\end{definition}
\begin{definition} The {\em weight} of a trajectory , denoted by , is defined as , i.e., the number of components of  that are not the zero trajectory.
\end{definition}

We now introduce a concept that is central to this paper:
\begin{definition}\label{def_security} The {\em security index} of the system  is defined as 

\end{definition}
This notion plays a paramount role in our investigation into the resilience of the system under adversarial attack. More precisely, we consider attacks on the system  that result in the system  given by:

where  is the unknown attack signal and  is the known received signal. Thus we are focusing exclusively on scenarios where the system's outputs (= sensors) are attacked. The behavior  of the system  is defined as the set of all possible trajectories  that satisfy the above equations for some initial condition  and some attack signal . We consider the following two problems:

\begin{quote}
{\em Problem 1 (attack detection):} Given received signal , detect that . 
\end{quote}
\begin{quote}
{\em Problem 2 (attack correction):} Given received signal , find  such that  is minimal. 
\end{quote}
In the sections that follow, we derive conditions such that the problems above are solvable in a tractable manner.

\section{Conditions for attack detection and correction using a system's security index}
The first question that arises is: under which conditions on the attack signal  are these problems solvable? We have the following results. 

\begin{thm}\label{thm_detect}
Suppose the received signal  corresponds to an attack  with  an unknown non-zero value . Then , i.e., attack detection is possible. 
\end{thm}
\begin{proof}
Let  and  be as stated in the theorem and let  be such that . Then  because of Definition~\ref{def_security} and the assumption that . Since  is linear it then follows that .	
\end{proof}

Consequently, we can interpret the security index  as the minimum number of sensors that an attacker needs to compromise without being detected. We call the system  {\em maximally secure} if . It is easily seen that generically, systems described by equations~\eqref{eq_Asystem}-\eqref{eq_Csystem} are maximally secure.

\begin{thm}\label{thm_correct}
Suppose the received signal  corresponds to an attack  with  an unknown value . Then there exists a unique  such that  is minimal, i.e. unique attack correction is possible.   
\end{thm}
\begin{proof}
		Let  be as stated in the theorem and let  and  be such that  with . Suppose that there also exist  and  such that  with . Clearly , so that  is a trajectory in  of weight . Definition~\ref{def_security} now implies that 
		 is the zero trajectory, in other words,  is unique.  It follows that  is minimal.
\end{proof}

We have formulated system 's security index  and results in therms of the output trajectory , instead of 's initial condition  to transparently draw an analogy with error control coding. In fact, this choice is natural because the recovery of  is equivalent to having  during attack correction, due to the observability assumption on system .

\section{Computing a system's security index}
In this section, we show how system 's security index  can be computed. To this end, we introduce the {\em coding matrix}  of the system  defined as follows 
\beq
G = \bmat{c} G_1 \\ G_2 \\ \vdots \\ G_N \emat,\;\;\;\mbox{where }G_i := \bmat{c} C_i \\
C_iA
\\
C_iA^2 \\
\vdots \\
C_iA^{n-1}
\emat\label{eq_Gi}
\eeq
with  defined as the 'th row of  for .

We call the above matrix  the {\em coding matrix} of the system as it exhibits the link with error control coding \cite{leeSE2015ECC}. Note however in contrast to error control coding, the coding matrix  cannot be chosen freely---instead, it is fixed and given by the system . In particular, the number of sensors  is fixed.  In the following theorem, we use  to denote the matrix that is obtained by stacking the matrices  defined in~(\ref{eq_Gi}), for .

\begin{thm}\label{thm_GJ}

where  is the largest integer in  for which there exists a subset  of  of cardinality  such that .
\end{thm}
\begin{proof}
		Let  be a subset of  of cardinality  such that . Let  be such that . Then   because of left invertibility of . Furthermore, the trajectory  that corresponds to initial condition  satisfies  (use Cayley-Hamilton). Thus  is a trajectory in  of weight , so that . Further, by definition of , trajectories in  cannot have more than  of their component trajectories equal to the zero trajectory. Therefore .
\end{proof}
\begin{cor} \label{cor:max_secure_G}
The system  given by equations~(\ref{eq_Asystem})-(\ref{eq_Csystem}) is maximally secure if and only if each matrix , as defined in~(\ref{eq_Gi}), has full column rank ().
\end{cor}

Not surprisingly, a system is maximally secure if and only if the system is observable via each sensor. In this case, we obtain from Theorem \ref{thm_correct} that the sufficient condition for the attack on the system to be correctable is that the number of compromised sensors is strictly less than half of the total number of sensors. This conforms with the results in \cite{fawziTD14} and \cite{ChongWakaikiHespanhaACC15} for discrete-time and continuous-time LTI systems, respectively.

We further provide ways of computing a system's security index . Since we assume that the LTI system is observable and hence the  matrix  given by~(\ref{eq_Gi}) has full rank, there exists a full rank  matrix , written as 
\beq
H = \bmat{cccc} H_1 & H_2 & \cdots & H_N \emat ,\label{eq_Hi}
\eeq
such that ; mindful of the analogous coding theoretic terminology, in this paper we call such a matrix  a {\em check matrix} of the system. In the next theorem  denotes the matrix that is obtained by juxtaposing the matrices  defined in~(\ref{eq_Hi}), for .
\begin{thm}\label{thm_HJ}

where  is defined as the smallest integer  in  for which there exists a subset  of  of cardinality  such that . 
\end{thm} 
\begin{proof}
		Let  be a subset of  of cardinality  such that . Let  be a non-zero trajectory such that . Let  be the trajectory that coincides with  at the appropriate locations and that has zero component trajectories at all other locations. Then  is a trajectory in  of weight , so that . Further, by definition of , trajectories in  have weight  (use Cayley-Hamilton) and therefore . 
\end{proof}

The terminology "spark" stems from the compressed sensing literature~\cite{TillmannP14}.

\begin{cor}
The system  given by equations~(\ref{eq_Asystem})-(\ref{eq_Csystem}) is maximally secure if and only if all square  submatrices of  of the form , where  is a subset of  of cardinality , are nonsingular.
\end{cor}

An alternative representation of the system  is given by a set of  difference equations
\beq
R(\sigma ) \ybold = 0 ,\label{eq_Rkernel}
\eeq
where  is a  polynomial matrix and  represents the left shift, as before.
In the special case where  corresponds to a minimal lag representation, its  row degrees are the observability indices of the system . 

Recall that a square polynomial matrix is called {\em unimodular} if it has a polynomial inverse; a nonsquare polynomial matrix is called {\em left unimodular} if it has a polynomial left inverse. Two polynomial matrices  and  of the same size are called {\em left unimodularly equivalent} if there exists a unimodular matrix  such that .

\begin{thm}\label{thm_RJ}
Let the system  be given by~(\ref{eq_Rkernel}). Then its security index  is given by the smallest integer  in  for which there exists a subset  of  of cardinality  such that  is not left unimodular (here  denotes the matrix that consists of the th columns of  where ).
\end{thm}
\begin{proof}
		Let  be a subset of  of cardinality  such that  is not left unimodular. Then there exists  such that . Let  be the trajectory that coincides with  at the appropriate locations and that has zero component trajectories at all other locations. Then  is a trajectory in  of weight , so that . By definition of  we must also have  which proves the theorem.
\end{proof}

In the remainder of this section, we show that the system 's security index  can be computed more easily by exploiting the system's special structure. Below,  denotes the support of a vector , i.e. the set of indices  in  such that . As mentioned in Definition \ref{def:supp_y_trajectory},  denotes the support of a trajectory , i.e. the set of indices  in  such that  is not the zero trajectory. 

\begin{lem}\label{lemma_eigen}
Let the system  be given by equations~(\ref{eq_Asystem})-(\ref{eq_Csystem}). Let  be a linear combination of eigenvectors of  that correspond to different eigenvalues, so
\beq
x(0) = \alpha_1 v_1 + \alpha_2 v_2 + \cdots + \alpha_m v_m ,\label{eq_alphas}
\eeq
where ,  for  and  for .
Let  be the trajectory in the behavior  of  that corresponds to . Then
\beq
\supp ( \ybold ) = \cup_{j=1}^m \supp (Cv_j ) . \label{resultSupp}
\eeq
\end{lem}
\begin{proof}
It follows immediately from the definition of  that 
\beq
\supp ( \ybold ) \subset \cup_{j=1}^m \supp (Cv_j ) . \label{pfSupp1}
\eeq
To prove the reverse inclusion, let . Then there exists  such that . Since  is assumed non-zero it follows that . Then

is non-zero because of the Vandermonde structure and the fact that all 's  are distinct. Thus  is not the zero trajectory so that .
This implies that
\beq
\cup_{j=1}^m \supp (Cv_j ) \subset \supp ( \ybold ) . \label{pfSupp2}
\eeq
From~(\ref{pfSupp1}) and~(\ref{pfSupp2}) we conclude that~(\ref{resultSupp}) holds. 	
\end{proof}
A special consequence of the above lemma is that the system's security index  is determined by the weight of the special trajectories that have initial conditions in an eigenspace  of : 
\begin{thm}\label{thm_eigen}
Let the system  be given by equations~(\ref{eq_Asystem})-(\ref{eq_Csystem}). Let  be the  distinct eigenvalues of the matrix . Let , where  is the subspace spanned by all eigenvectors corresponding to  for . 
Then

In particular, if , meaning that all  eigenvalues of  are distinct, then

where  is a basis of eigenvectors of .
\end{thm}
\begin{proof}
Let  be an arbitrary trajectory in , corresponding to initial condition , written as in~(\ref{eq_alphas}). It follows from Lemma~\ref{lemma_eigen} that 

This implies that

		To prove the reverse inequality, let  and  be such that  is minimal. Let  be the system's trajectory that corresponds to initial condition . From  it follows immediately that . Therefore 
		
		which proves the theorem.
\end{proof}
From the theorem above, we see that the computation of the security index is straightforward for a system that has  distinct eigenvalues : simply use a diagonal -matrix; the security index of the system is then given by the size of the support of the sparsest column of the corresponding -matrix.  In particular, the system is maximally secure if and only if this -matrix has no zero values. We now illustrate the computation of system 's security index through an example. 
\begin{example}
Let  and consider

Then it follows from Theorem~\ref{thm_eigen} that for  the system's security index is  so that by Theorem~\ref{thm_detect} detection of attacks on one output is possible. Minimal lag equations for this system are given by

We observe that when , every column of  is left unimodular. Hence, by Theorem \ref{thm_RJ}, we obtain  and thus, the security index  equals .
\end{example}


\section{Attack detection, correction and other future work}
Using the results developed in the previous sections, we now discuss how they can be employed in detecting the occurrence of an attack and how correction can be performed.

To start with detection, we first recall that the observability assumption on our system implies that a check matrix  (defined in \eqref{eq_Hi}) exists. Attack detection is most easily formulated in terms of  as follows.

\begin{quote} 
{\em Attack detection rule:} Given received signal , compute the `syndrome trajectory'  and conclude that an attack has taken place (meaning ) if and only if . 
\end{quote}

Alternatively, we can use the description of the form  as the main ingredient of our decision rule to detect/correct.   
\begin{quote}
{\em Attack detection rule:} Given received signal , compute the `syndrome trajectory'  and conclude that an attack has taken place (meaning ) if and only if . 
\end{quote}
In this paper, we choose to formulate all fundamental notions and results in terms of the trajectory  rather than in terms of the initial condition  so as to have a clear fundamental theory that exhibits the link with error control coding in a transparent way.  Indeed, we are able to make this choice because for attack correction the recovery of  is equivalent to the recovery of  due to the observability assumption on the system . It is however important to note that in practical situations the recovery of  from the received trajectory  is the main objective. In fact, we only need to consider attack scenarios where attacks happen in the first  time instances (otherwise we can simply reconstruct  from  because of the observability of ).

For attack correction, the above syndrome trajectory  should be used to identify the attack locations.  Once the attack locations have been found,  can be computed on the basis of the remaining attack-free components of . How to identify the attack locations from  in terms of our knowledge of the system matrices  and  is a topic of future research. Another topic of future research is the effect of feedback control on the security index of an LTI system with inputs.




\addtolength{\textheight}{-12cm}   









\bibliographystyle{plain}
\bibliography{code}


\end{document}
