\documentclass{article}
\usepackage{graphicx}
\usepackage{amsmath,amssymb,amsfonts,amsthm}
\usepackage{mathrsfs}
\usepackage[hang,small,bf]{caption}
\usepackage{subfigure}
\usepackage[shortlabels]{enumitem}
\usepackage{bm}
\usepackage[a4paper,top=25mm,bottom=25mm,inner=30mm,outer=20mm]{geometry}
\usepackage{tikz}
\usetikzlibrary{arrows,calc}
\usetikzlibrary{arrows,shapes}
\usetikzlibrary{matrix}
\tikzset{
>=stealth',
help lines/.style={dashed, thick},
axis/.style={<->},
important line/.style={thick},
connection/.style={thick, dotted},
}
\usepackage{hyperref}
\newtheorem{thm}{Theorem}[section]
\newtheorem{rem}[thm]{Remark}
\newtheorem{rems}[thm]{Remarks}
\newtheorem{defi}[thm]{Definition}
\newtheorem{defn}[thm]{Definition}
\newtheorem{prop}[thm]{Proposition}
\newtheorem{lem}[thm]{Lemma}
\newtheorem{cor}[thm]{Corollary}
\newtheorem{assump}[thm]{Assumption}
\newtheorem{algo}[thm]{Algorithm}

\def\OOO{{\mathcal O}}
\def\MM{{\mathcal M}}
\def\CC{{\mathcal C}}
\def\BB{{\mathcal B}}
\def\AA{{\mathcal A}}
\def\DD{{\mathcal D}}
\def\ee{{\epsilon}}
\def\ve{{\varepsilon}}
\newcommand{\ue}{u_\epsilon}
\newcommand{\pe}{p_\epsilon}
\newcommand{\lj}{[\![}
\newcommand{\rj}{]\!]}
\def\RR{{\mathbb{R}}}
\def\div{ {\rm div}}
\def\jac{ {\rm Jac}}
\def\id{\, {\rm Id}}
\def\Uadmis{{\mathcal U}_{ad}}
\newcommand{\dsp}{\displaystyle}
\def\Aeps{A_{\Omega^0,\varepsilon}}
\def\Aepst{A_{(Id+\theta)\Omega^0,\varepsilon}}
\newcommand{\dJ}{\ensuremath{\text{D}J}}
\newcommand{\Div}{\ensuremath{\text{div}}}
\newcommand{\AS}{\ensuremath{\mbox{}}}
\newcommand{\ES}{\ensuremath{\mbox{}}}
\newcommand{\bfE} {{E}}
\newcommand{\dip} {\! :\!}
\newcommand{\Tsup}{^{\text{\tiny T}}}
\newcommand{\Dsup}{^{\text{\tiny D}}}
\newcommand{\Nsup}{^{\text{\tiny N}}}
\newcommand{\CS}{\mbox{}}
\newcommand{\CSB}{\CS_{\Bcal}}
\newcommand{\CSr}{\CS_{\incsize}}
\newcommand{\rhor}{\rho_{\incsize}}
\newcommand{\CSs}{\CS^{\star}}
\newcommand{\rhos}{\rho^{\star}}
\newcommand{\iBcal}{\int_{\Bcal}}
\newcommand{\idBcal}{\int_{\dBcal}}
\newcommand{\Bcal}{\mathcal{B}}
\newcommand{\Ccal}{\mathcal{C}}
\newcommand{\Dcal}{\mathcal{D}}
\newcommand{\Ical}{\mathcal{I}}
\newcommand{\Ccalr}{\mathcal{C}_{\incsize}}
\newcommand{\dBcal}{\partial\Bcal}
\newcommand{\bfna}{{\nabla}}
\newcommand{\bfxib} {\bar{\bfxi}}
\newcommand{\bfxb} {\bar{{x}}{}}
\newcommand{\Vcal}{\mathcal{V}}
\newcommand{\dSb}{\;\text{d}\bar{S}}
\newcommand{\dVb}{\;\text{d}\bar{V}}
\newcommand{\dVxib}{\dV(\xib)}
\newcommand{\bfd} {{d}}
\newcommand{\bfh} {{h}}
\newcommand{\bfp} {{p}}
\newcommand{\bfw} {{w}}
\newcommand{\bfW} {{W}}
\newcommand{\sheq}{\hspace*{-0.1em}=\hspace*{-0.1em}}
\newcommand{\shdeq}{\hspace*{-0.1em}:=\hspace*{-0.1em}}
\newcommand{\bfD} {{D}}
\newcommand{\bfu} {{u}}
\newcommand{\bfxZ} {\bfz}
\newcommand{\bfy}{{y}}
\newcommand{\bfs}{{s}}
\newcommand{\bfxi}{{\xi}}
\newcommand{\bfel}{{\ell}}
\newcommand{\xb} {\bar{x}}
\newcommand{\xib} {\bar{\xi}}
\newcommand{\bfz} {{z}}
\newcommand{\shm}{\hspace*{-0.1em}-\hspace*{-0.1em}}
\newcommand{\Kcal}{\mathcal{K}}
\newcommand{\GS}{\mbox{}}
\newcommand{\bfEh} {{\hat{E}}{}}
\newcommand{\shp}{\hspace*{-0.1em}+\hspace*{-0.1em}}
\newcommand{\bfv} {{v}}
\newcommand{\bfvBh} {\hat{\bfv}_{\Bcal}}
\newcommand{\Rbb} {\mathbb{R}}
\newcommand{\Jbb} {\mathbb{J}}
\newcommand{\Jbbr} {\Jbb_{\incsize}}
\newcommand{\bfph}{{\varphi}}
\newcommand{\bfze}{{0}}
\newcommand{\bft} {{t}}
\newcommand{\sip} {\! \cdot\!}
\newcommand{\shg}{\hspace*{-0.1em}>\hspace*{-0.1em}}
\newcommand{\shgeq}{\hspace*{-0.1em}\geq\hspace*{-0.1em}}
\newcommand{\shleq}{\hspace*{-0.1em}\leq\hspace*{-0.1em}}
\newcommand{\shl}{\hspace*{-0.1em}<\hspace*{-0.1em}}
\newcommand{\shin}{\hspace*{-0.1em}\in\hspace*{-0.1em}}
\newcommand{\shneq}{\hspace*{-0.1em}\not=\hspace*{-0.1em}}
\newcommand{\Ibb} {\mathbb{I}}
\newcommand{\PS}{\mbox{}}
\newcommand{\IS}{\mbox{}}
\newcommand{\JS}{\mbox{}}
\newcommand{\KS}{\mbox{}}
\newcommand{\Tens}{\hspace*{-1pt}\otimes\hspace*{-1pt}}
\newcommand{\bfI} {{I}}
\newcommand{\iB}{\int_{B}}
\newcommand{\iOr}{\int_{\OOr}}
\newcommand{\iOBr}{\int_{\OBr}}
\newcommand{\iBr}{\int_{\Br}}
\newcommand{\iO}{\int_{\OO}}
\newcommand{\OO}{\Omega}
\newcommand{\G}{\Gamma}
\newcommand{\GD}{\G_{\text{D}}}
\newcommand{\GN}{\G_{\text{N}}}
\newcommand{\iGN}{\int_{\GN}}
\newcommand{\shsubs}{\hspace*{-0.1em}\subset\hspace*{-0.1em}}
\newcommand{\dV}{\;\text{d}V}
\newcommand{\bfeps}{{e}}
\newcommand{\iD}{\int_{D}}
\newcommand{\idD} {\int_{\partial D}}
\newcommand{\dS}{\,\text{d}S}
\newcommand{\bfG} {{G}}
\newcommand{\shsetm}{\hspace*{-0.1em}\setminus\hspace*{-0.1em}}
\newcommand{\bfr} {{r}}
\newcommand{\bfeta}{{\eta}}
\newcommand{\bfal}{{\alpha}}
\newcommand{\bfN} {{N}}
\newcommand{\rmi}{\mathrm{i}}
\newcommand{\bfK} {{K}}
\newcommand{\inv}[1]{\dfrac{1}{#1}}
\newcommand{\shcup}{\hspace*{-0.1em}\cup\hspace*{-0.1em}}
\newcommand{\shcap}{\hspace*{-0.1em}\cap\hspace*{-0.1em}}
\newcommand{\dO}{\partial\OO}
\newcommand{\bfsig}{{\sigma}}
\newcommand{\bff} {{f}}
\newcommand{\bfg} {{g}}
\newcommand{\Br} {B_{\incsize}}
\newcommand{\Brb} {\bar{B}_{\incsize}}
\newcommand{\incsize}{a}
\newcommand{\OOr}{\Omega_{\incsize}}
\newcommand{\OBr}{\Omega\setminus\Br}
\newcommand{\bfur}{\bfu_{\incsize}}
\newcommand{\bfvr}{\bfv_{\incsize}}
\newcommand{\bfvb}{\bar{\bfv}}
\newcommand{\bfvrt}{\tilde{\bfv}_{\incsize}}
\newcommand{\bfvrh}{\hat{\bfv}_{\incsize}}
\newcommand{\bfvrb}{\bfvb_{\incsize}}
\newcommand{\bfX} {{X}}
\newcommand{\bfx} {{x}}
\newcommand{\bfe} {{e}}
\newcommand{\Rsub}{_{\text{R}}}
\newcommand{\Usub}{_{\text{U}}}
\newcommand{\Csub}{_{\text{C}}}
\newcommand{\Ssub}{_{\text{S}}}
\newcommand{\Vsub}{_{\text{V}}}
\newcommand{\Dsub}{_{\text{D}}}
\newcommand{\dVxi}{\;\text{d}V(\xi)}
\newcommand{\Lcal}{\mathcal{L}}
\newcommand{\Lcalr}{\Lcal_{\incsize}}
\newcommand{\LcalB}{\Lcal_{\Bcal}}
\newcommand{\bfuB}{\bfu_{\Bcal}}
\newcommand{\bfvB}{\bfv_{\Bcal}}
\newcommand{\bfvBz}{\bfv_{\Bcal_{0}}}
\newcommand{\DS}{\mbox{}}
\newcommand{\bfn} {{n}}
\newcommand{\sym}{_{\text{sym}}}
\renewcommand{\SS}{\ensuremath{\mbox{}}}
\newcommand{\dt}{\,\text{d}t}
\newcommand{\du}{\,\text{d}u}
\newcommand{\dz}{\,\text{d}z}
\newcommand{\dw}{\,\text{d}w}
\newcommand{\dth}{\;\text{d}\theta}
\newcommand{\dphi}{\;\text{d}\phi}
\newcommand{\Scal}{\mathcal{S}}
\newcommand{\bfGb} {\bar{{G}}{}}
\newcommand{\bfub} {\bar{{u}}{}}
\newcommand{\bfwb} {\bar{{w}}{}}
\newcommand{\shnin}{\hspace*{-0.1em}\not\in\hspace*{-0.1em}}
\newcommand{\shtimes}{\hspace*{-0.1em}\times\hspace*{-0.1em}}
\newcommand{\bfF} {{F}}
\newcommand{\Ebb} {\mathbb{E}}
\newcommand{\bfurN}{\bfu_{\incsize}\Nsup}
\newcommand{\bfurD}{\bfu_{\incsize}\Dsup}
\newcommand{\bfvrN}{\bfv_{\incsize}\Nsup}
\newcommand{\bfvrD}{\bfv_{\incsize}\Dsup}
\newcommand{\bfuN}{\bfu\Nsup}
\newcommand{\bfuD}{\bfu\Dsup}
\newcommand{\bfwN}{\bfw\Nsup}
\newcommand{\bfwD}{\bfw\Dsup}
\newcommand{\bfwNb}{\bfwb\Nsup}
\newcommand{\bfwDb}{\bfwb\Dsup}
\newcommand{\bfurNb}{\bfub_{\incsize}\Nsup}
\newcommand{\bfurDb}{\bfub_{\incsize}\Dsup}
\newcommand{\bfvrNb}{\bfvb_{\incsize}\Nsup}
\newcommand{\bfvrDb}{\bfvb_{\incsize}\Dsup}
\newcommand{\bfuNb}{\bfub\Nsup}
\newcommand{\bfuDb}{\bfub\Dsup}
\newcommand{\suite}[1][0ex]{\notag \
\label{equation_eps}
\left\{
\begin{array}{ll}
\text{div}(\ue)=0  &  x\in \Omega_\epsilon,\\
-\epsilon^2 \mu \Delta \ue = \nabla \pe & x\in \Omega_\epsilon,\\
\frac{\partial X_\epsilon}{\partial t}+ \ue\cdot \nabla X_\epsilon=\lambda \Delta X_\epsilon 
 & t>0, x\in \Omega_\epsilon,\\
 X_\epsilon =X_{\text{init}} & t=0, x\in \Omega_\epsilon,
\end{array}
\right.

\label{boundary_conditions_eps}
\left\{
\begin{array}{lll}
\ue=u_D & X_\epsilon=X_D & t>0,x\in \Gamma_D,\\
\ue\cdot n=0 & D_\epsilon \nabla X_\epsilon \cdot n=0 & t>0, x\in \Gamma_N,\\
\ue=0 &  \nabla X_\epsilon \cdot n=-\epsilon \frac{1}{4e}\mathcal{R}(X_\epsilon) & t>0, x\in \Gamma_\epsilon,\\
\end{array}
\right.

\mathcal{R}(X_\epsilon)=j_0 (X^m_\epsilon,T) \exp\Big(-\alpha_c \frac{zF}{RT}\eta\Big).

\Omega_{\epsilon}=\Omega\backslash \bigcup_{i=1}^{N(\epsilon)} \omega_i^{\epsilon},

\label{equation_homo}
\left\{
\begin{array}{ll}
\mbox{div}(u^*)=0 & x\in \Omega\\
u^*= -\frac{K}{\mu} \nabla p^* & x\in \Omega\\
\frac{\partial X^*}{\partial t}+ u^*\cdot \nabla  X^*=\lambda \mbox{div} (D \nabla X^*) +|\partial \omega| \mathcal{R}(X^*) & x\in \Omega\\
X^* =X_{\text{init}} & t=0, x\in \Omega,
\end{array}
\right.

\label{boundary_conditions_homo}
\left\{
\begin{array}{lll}
u^*=u_D & X^*=X_D & t>0,x\in \Gamma_D,\\
u^*\cdot n=0 & \nabla X^* \cdot n=0 & t>0, x\in \Gamma_N.
\end{array}
\right.

\label{def:tensors}
K_{ij}=\int_{Y\backslash \omega}\nabla u_i(y) : \nabla u_j(y)dy, \quad D_{ik}=
\int_{Y\backslash \omega}\Big(e_i+\nabla \pi_i\Big)\cdot \Big(e_k+\nabla \pi_k\Big) dy

A:B=\sum \limits_{k,\ell}A_{k\ell}B_{k\ell},

\label{cell_problem}
\left\{ \begin{array}{rcll}
&&\nabla p_i-\Delta u_i=e_i, \mbox{ }x\in Y\backslash \omega,\\
&&\mbox{div}(u_i)=0, \mbox{ }x\in Y\backslash \omega,\\
&&u_i=0,\mbox{ } x\in \partial \omega,\\
&&y\rightarrow p_i(y),u_i(y) \mbox{ Y-periodic,}
\end{array} \right.
\hspace{1cm}
\left\{ \begin{array}{rcll}
&&-\div(\nabla \pi_j+e_j)=0 , \mbox{ }x\in Y\backslash \omega,\\
&&\nabla \pi_j \cdot n=-e_j \cdot n,\mbox{ } x\in \partial \omega,\\
&&y\rightarrow \pi_j(y) \mbox{ Y-periodic,}
\end{array} \right.

\label{optim_problem}
\left\{ \begin{array}{rcll}
&&\max\limits_{\omega\subset Y} |\partial \omega|\\
&&\mbox{s.t.}\\
 &&|\omega| \geq C_f|\partial \omega|\,\\ && \frac{tr(K)}{d}\geq k_{min},\\ \end{array} \right.

d^{\mathbb{D}}(\omega_0,\omega)= \inf \limits_{T\in \mathbb{D}| T(\omega_0)=\omega}\Big(\left\|T-Id\right\|_{W^{2,\infty}}+ \left\|T^{-1}-Id\right\|_{W^{2,\infty}}\Big),

\mathbb{D}= \left\{T \mbox{ such that }(T-Id)\in W^{2,\infty}(\Rbb^d;\Rbb^d),(T^{-1}-Id)\in W^{2,\infty}(\Rbb^d;\Rbb^d)  \right\}.

\big( Id + \theta \big) (\omega) := \left\{ x + \theta(x) 
\mbox{ for } \ x \in \omega\right\} ,

J\big( \big( Id + \theta \big) \omega \big)  = J (\omega)
+ J^\prime(\omega)(\theta) + o(\theta) \quad\mbox{with}\quad
\lim_{\theta\to 0} \frac{|o(\theta)|}{\quad \Vert\theta\Vert_{W^{1,\infty}}}=0\;,

J_{surf}(\omega)=|\partial \omega|=\int_{\partial \omega} ds \;\;\;\mbox{   and  }\;\;\; J_{vol}(\omega)=|\omega|=\int_{\omega} dx.

J'_{surf}(\omega)(\theta)=\int_{\partial \omega} \theta \cdot n H\,ds,\quad 
J'_{vol}(\omega)(\theta)=\int_{\partial \omega} \theta \cdot n\, ds,

&&H_{0,\#}^1(Y\backslash \omega)^d= \left\{ v\in H^1(Y\backslash \omega)^d: v|_{\partial \omega}=0, y\rightarrow v(y) \mbox{ is periodic}\right\},\\
&&L_{0,\#}^2(Y\backslash \omega)= \left\{ q\in L^2(Y\backslash \omega): \int_{Y\backslash \omega} q dx=0,\; y\rightarrow q(y) \mbox{ is periodic}\right\}.

\label{variational_stokes}
\int_{Y\backslash \omega} \Big(\nabla u_i : \nabla v_i - \text{div}(v_i)p_i - \mbox{div}(u_i) q_i \Big)dx = \int_{Y\backslash \omega} e_i \cdot v_i dx, \;\;\forall (v_i,q_i)\in (H_{0,\#}^1(Y\backslash \omega)^d,L^2_{0,\#}(Y\backslash \omega)).

J(\omega)=\int_{Y\backslash \omega} j(x,u,\nabla u)dx, \text{ where }u=(u_i)_{i=1...d},

\label{variational_adjoint}
\int_{Y\backslash \omega} \Big(\nabla U_i : \nabla v_i -\text{div}(v_i)P_i - \mbox{div}(U_i) q_i \Big)dx = -\int_{Y\backslash \omega} \Big( \frac{\partial j}{\partial u_i}(x,u_i,\nabla u_i) \cdot v_i+ \frac{\partial j}{\partial \nabla u_i}(x,u_i,\nabla u_i):\nabla v_i\Big) dx,

J'(\omega)(\theta)=\int_{\partial \omega} \Big\{j(x,u,\nabla u)-\sum_{i=1}^d \Big(\frac{\partial U_i}{\partial n}\cdot \frac{\partial u_i}{\partial n}+\frac{\partial j}{\partial \nabla u_i}\cdot n \cdot \frac{\partial u_i}{\partial n} \Big)\Big\}\theta \cdot n  ds.

\frac{\mbox{tr(K)}}{d}=\frac{1}{d}\sum_{i=1}^d K_{ii}=\frac{1}{d}\int_{Y\backslash \omega}\sum_{i=1}^d |\nabla u_i|^2dx,

\frac{1}{d}\text{tr}(K)'(\omega)(\theta)=-\frac{1}{d}\sum_{i=1}^d\int_{\partial \omega}  \Big(\big(\frac{\partial u_i}{\partial n}\big)^2+\frac{\partial  u_i}{\partial n} \cdot \frac{\partial U_i}{\partial n}\Big)\theta \cdot nds,

\int_{Y\backslash \omega}\Big(\nabla U_i : \nabla v_i -\text{div}(v_i)P_i - \mbox{div}(U_i) q_i\Big) dx =-2 \int_{Y\backslash \omega} \nabla u_i: \nabla v_i,
\forall (v_i,q_i)\in (H_{0,\#}^1(Y\backslash \omega)^2,L^2_{0,\#}(Y\backslash \omega))

\left\{ \begin{array}{ll}
\psi(x) = 0 & \mbox{ for } x\in\partial \omega,  \\
\psi(x) < 0 &  \mbox{ for } x\in\omega, \\
\psi(x) > 0 & \mbox{ for } x\in Y\backslash\omega,
\end{array}
\right.

\label{HJ_eq}
\frac{\partial \psi}{\partial t}+\mathcal{V} |\nabla \psi|=0,

\label{general_shape_derivative}
J'(\omega)(\theta)=\int_{\partial \omega}\mathcal{T}\theta \cdot n ds,

\theta = \mathcal{V} n \quad \mbox{ and } \quad 
J'(\omega)(\theta)=\int_{\partial \omega}\mathcal{T} \mathcal{V} ds\leq 0.

\label{artificial_momentum}
\int_{Y}\frac{5}{2\rho^2}(u_i \cdot v_i)dx,\quad \int_{Y}\frac{5}{2\rho^2}(U_i \cdot v_i)dx,

\mathcal{L}(\omega,\ell,\mu) = |\partial \omega|-\ell_{1}(C_f|\partial \omega|-|\omega|)-\ell_2 \left (k_{min}-\frac{\text{tr}(K)}{d} \right)+
\frac{\mu_1}{2}(C_f|\partial \omega|-|\omega|)^2+\frac{\mu_2}{2} \left(k_{min}-\frac{\text{tr}(K)}{d} \right)^2,

\ell_{1}^{n+1}=\ell_{1}^{n}-\mu_{1}(C_f|\partial \omega_n|-|\omega_n|)
\quad \mbox{and} \quad \ell_{2}^{n+1}=\ell_{2}^{n}-\mu_{2}\left(k_{min}-\frac{\text{tr}(K)}{d}\right).

 The penalty parameters are augmented every  iterations. With such an algorithm the constraints are enforced only at convergence.\\ 
 
Fig. \ref{homogenized} shows different optimal periodic layouts starting from an intuitive circular arrangement. The radius of the circle in  was set up to , meanwhile the coefficients  and  were calculated so as both constraints in \eqref{optim_problem} were active within this layout. The first optimal design (second row Fig. \ref{homogenized}) corresponds to the optimal solution of the algorithm for the foregoing parameters. Then the second design (third row Fig. \ref{homogenized}) was established by reducing  and  to the half. The perimeter gain w.r.t. the circular layout is  for the former design and  for the latter one.\\



\begin{figure}[h]
\centering
\begin{tikzpicture}[      
        every node/.style={anchor=south west,inner sep=0pt},
        x=1mm, y=1mm,
      ]   
     \node (fig5) at (0cm,0cm){\includegraphics[width=0.25\textwidth,height=0.24\textwidth]{layout_1}};
     \node (fig6) at (4.5cm,0cm)
     {\includegraphics[width=0.25\textwidth,height=0.24\textwidth]{periodic_layout_1}}; 
     \node (fig1) at (0cm,-4.5cm){\includegraphics[width=0.25\textwidth,height=0.24\textwidth]{layout_2}};
     \node (fig2) at (4.5cm,-4.5cm)
     {\includegraphics[width=0.25\textwidth,height=0.24\textwidth]{periodic_layout_2}};
     \node (fig3) at (0cm,-9cm){\includegraphics[width=0.25\textwidth,height=0.24\textwidth]{layout_3}};
     \node (fig4) at (4.5cm,-9cm)
     {\includegraphics[width=0.25\textwidth,height=0.24\textwidth]{periodic_layout_3}};
\end{tikzpicture}
\caption{Two micro-tubular optimal designs for different values of  and  starting from a circular layout. The base cell results are on the left and the corresponding periodic structures on the right.}\label{homogenized}
\end{figure}

{\section{Summary and Outlook}}

The above results demonstrate the capabilities of topology optimization via a level-set method to enhance the design of micro-tubular fuel cells for the aeronautic industry. Thanks the application of an inverse homogenization technique and the level-set method, periodic optimal micro-tubular fuel cells with a sharp contour can easily be designed and then manufactured by 3D printing. The foregoing study thus suggests a promising use of these technologies in the future  computer aided design of fuel cells.\\


\textbf{Acknowledgments:} {The author would like to thank Airbus Group for its financial support in the framework of his PhD thesis, Ch. Nespoulous and E. Moullet of Airbus Innovations for their important contribution through the joint work developed during E. Mullet's internship \cite{etienne}, and also G. Allaire of Centre de Math\'ematiques Appliqu\'ees of \'Ecole Polytechnique for many helpful suggestions after kindly reading this manuscript.}

\newpage
\bibliographystyle{abbrv}
\bibliography{FC_biblio,background}

\end{document}
