

\documentclass[letterpaper, 12pt, draftcls, onecolumn]{ieeeconf}




\IEEEoverridecommandlockouts                              

\overrideIEEEmargins                                      



\usepackage[dvipdfmx]{graphicx}\usepackage{amsmath, amssymb, bm, mathrsfs}
\usepackage{cite, color, graphicx, subcaption}


\def\proof{\noindent\hspace{1em}{\itshape {\bf Proof:} }}
\def\endproof{\hspace*{\fill}~\QED\par\endtrivlist\unskip}



\newtheorem{theorem}{Theorem}[section]
\newtheorem{proposition}[theorem]{Proposition}
\newtheorem{definition}[theorem]{Definition}
\newtheorem{problem}[theorem]{Problem}
\newtheorem{assumption}[theorem]{Assumption}
\newtheorem{example}[theorem]{Example}
\newtheorem{lemma}[theorem]{Lemma}
\newtheorem{remark}[theorem]{Remark}
\newtheorem{corollary}[theorem]{Corollary}
\newtheorem{algorithm}[theorem]{Algorithm}

\newtheorem{propositionB1}{Proposition B.\hspace{-5pt}}


\title{\LARGE \bf
Stabilization of Systems with Asynchronous 
Sensors and Controllers\thanks{}
}


\author{Masashi Wakaiki, Kunihisa Okano, and Jo\~ao~P.~Hespanha
\thanks{
This paper was partially
presented at the 2015 American Control Conference,
July 1-3, 2015, the USA.
This material is based upon work supported by 
the National Science
Foundation under Grant No. CNS-1329650.
M. Wakaiki acknowledges Murata Overseas Scholarship Foundation
for the support of this work.
K. Okano is supported by 
JSPS Postdoctoral Fellowships for Research Abroad.
Corresponding author M. Wakaiki. Tel. +1 805 893 7785. 
Fax +1 805 893 3262.
}
\thanks{
M. Wakaiki is with the Department of Electrical and Electronic Engineering, Chiba University,
Chiba, 263-8522, Japan (e-mail:{\tt \small wakaiki@chiba-u.jp}).
K. Okano is with the Graduate School of Natural Science and Technology,
Okayama University, Okayama, 700-8530, Japan 
(e-mail:{\tt \small kokano@okayama-u.ac.jp}).
J.~P.~Hespanha is with the Center for Control,
Dynamical-systems and Computation (CCDC), University of California,
Santa Barbara, CA 93106-9560 USA
(e-mail:{\tt \small hespanha@ece.ucsb.edu}).
}}

\begin{document}



\maketitle
\thispagestyle{empty}
\pagestyle{empty}


\begin{abstract}
We study the stabilization of networked control systems with
asynchronous sensors and controllers. 
Offsets between the sensor and controller clocks
are unknown and
modeled as parametric uncertainty.
First we consider multi-input linear systems and
provide a sufficient condition 
for the existence of linear time-invariant controllers that are 
capable of stabilizing the closed-loop system for 
every clock offset in a given range of admissible values. 
For first-order systems, we next
obtain the maximum length of the offset range
for which the system can be stabilized 
by a single controller. 
Finally,
this bound is compared with the offset bounds that would be allowed if
we restricted our attention to static output feedback controllers.
\end{abstract}


\section{Introduction}
In networked and embedded control systems,
the outputs of plants are often sampled in a nonperiodic fashion and sent to
controllers with time-varying delays.
To address robust control with such imperfections,
various techniques have been developed, for example, the input-delay approach \cite{Fridman2004, Mirkin2007},
the gridding approach \cite{Fujioka2009, Oishi2010, Donkers2011},
and the impulsive systems approach based on
Lyapunov functionals \cite{Naghshtabrizi2010}, on
looped functionals \cite{Briat2012}, and on 
clock-dependent Lyapunov functions \cite{Briat2013};
see also the surveys \cite{Hespanha2007, Hetel2017}.
In contrast to the references mentioned above, 
here we assume that time-stamps are used to provide the
controller with information about the sampling times and the communication delays
incurred by each measurement.
In this approach,
sensors send measurements to controllers
together with time-stamps,
and the controllers
exploit this information to mitigate the effect of variable delays and sampling periods
\cite{Graham2004, Nakamura2008, Garcia2014}.
However, when the local clocks at
the sensors and at the controllers are not synchronized, 
the time-stamps and 
the true sampling instants 
do not match.
Protocols to establish synchronization have been actively
studied as surveyed in \cite{Rhee2009}, 
and synchronization by the global positioning system
(GPS) or radio clocks has been utilized in some systems.
Nevertheless, synchronizing clocks over networks 
has fundamental limits~\cite{Freris2011}, and
a recent study \cite{Jiang2013} 
has shown that synchronization
based on GPS signals is vulnerable against attacks.



In this paper, we study the stabilization problem of
systems with asynchronous sensing and control.
We assume that
the controller can use the time-stamps but does not know
the offset between
the sensor and controller clocks, but
we do assume that this offset is essentially constant over the time
scales of interest.
Our objective is to find linear time-invariant (LTI) controllers 
that achieve closed-loop stability
for every clock offset in a given range.





We formulate the stabilization of systems with clock offsets as the problem
of stabilizing systems with parametric uncertainty,
which can be regarded as the
simultaneous stabilization of a family of plants, as 
studied in \cite[Sec. 5.4]{vidyasagar1985} and
\cite{Vidyasagar1982}.
However, we had to overcome a few technical difficulties that distinguish the problem considered here from previously published results:

{\it Infinitely many plants:}
We consider a family of plant models that is indexed by 
a continuous-valued parameter.
Such a family
includes infinitely many plants, but
the approaches for simultaneous stabilization 
e.g., in \cite{Shi2009}
exploit the property that the number of plant models is finite.

{\it Nonlinearity of the uncertain parameter:}
In this work, 
the uncertain parameter appears in a non-linear form.
Therefore, it is not suitable to use the techniques 
based on linear matrix inequalities (LMIs) in 
\cite{Oliveira1999}
for the robust stabilization
of systems with polytopic uncertainties.
Although the robust stability analysis based on continuous paths of
systems with respect to the -gap metric
was developed in \cite{Cantoni2012},
controller designs based on this approach have not been fully investigated.

{\it Common unstable poles and zeros:}
Earlier studies on simultaneous stabilization consider
a restricted class of plants.
For example, 
the sufficient condition 
in \cite{Blondel1993Suf}
is obtained for a family of plants with
no common unstable zeros or poles. 
The set of plants in \cite{Maeda1984} has
common unstable zeros (or poles) but
all the plants are stable (or minimum-phase).
These assumptions are not satisfied for the systems 
in the present paper.


We make the following technical contributions for 
multi-input systems and first-order systems:
First we consider multi-input systems and 
obtain a sufficient condition for stabilization with asynchronous sensing and control.
We construct a stabilizing controller from
the solution of an appropriately defined   control problem.
The above mentioned difficulties found in 
the simultaneous stabilization problem we consider 
is circumvented by exploiting geometric properties on .
For first-order systems,
we obtain
an explicit formula for the exact bound 
on the clock offset that can be allowed for stability.
This result is based on the stabilization of
interval systems \cite{Ghosh1988, Olbrot1994}, 
to which our problem can be reduced for first-order plants.
We start by formulating the problem in the context of state feedback without
disturbances and noise, but we show in Section 3.2 that
the above results also apply for output feedback with disturbances and 
noise.






The authors in the previous study \cite{Okano2015}
have considered systems with time-varying clock offsets
and have
proposed a stabilization method with causal controllers, 
based on the analysis of data rate limitations in quantized control.
The stability analysis and the -gain analysis of systems with variable 
clock offsets have been investigated in \cite{WakaikiCDC2015} and
\cite{Wakaiki2016ACC}, respectively. 
The major difference with respect to those studies is that
here we consider only constant offsets but
design stabilizing LTI controllers. 
This paper is based on the conference paper \cite{Wakaiki2015ACC}, but
here we extend the preliminary results for single-input systems to
the multi-input case.



The remainder of the paper is organized as follows.
Section 2 introduces the closed-loop system we consider 
and presents the problem formulation.
Section 3 is devoted to the discretization of the closed-loop system.
In Section~4, we obtain
a sufficient condition for the stabilizability
of general-order systems.
In Section 5,
we derive the exact bound
on the permissible clock offset for first-order systems. 
In Section 6, we discuss
stabilizability with static controllers and
the comparison of 
the offset bounds obtained for LTI controllers 
and static controllers.





{\it Notation and definitions:~}
We denote by  the set of non-negative integers.
The symbols 
, , and  denote
the open unit disc ,
the closed unit disc , and
the unit circle , respectively.
We denote by  the complement of the open unit disc
.


A square matrix  is said to be {\em Schur stable}
if all its eigenvalues lie in the unit disc .
We say that a discrete-time LTI system 
is stabilizable (detectable) if 
there exists a matrix  () such that  () is Schur stable.
We also use the terminology  is 
stabilizable (respectively,  is detectable) to denote
this same concept.

We denote by  the space of all bounded holomorphic
real-rational
functions in .
The field of fractions of 
 is denoted by .
For a commutative ring , 
 denotes the set of
matrices with entries in , of any order.
For ,
 denotes the induced 2-norm.
For ,
the -norm is defined as .
For  and
,
we define a lower linear fractional transformation of  and 
as
.


A pair  in  is said to be 
{\em right coprime} if  the Bezout identity 

holds for some , .
 admits a {\em right coprime
	factorization} if there exist ,  
such that 
 and the pair  is right coprime.
Similarly, a pair  
in  
is {\em left coprime} if the Bezout identity 

holds for some , .
 admits a {\em left coprime
	factorization} if there exist ,  
such that  and the pair 
 is left coprime.
If  is a scalar-valued function, then we use the expressions {\em coprime} and
{\em coprime factorization}. 











\section{Problem Statement}
Consider the following LTI plant:

where  and
 are the state and the input of the plant, respectively.
As shown in 
Fig.~\ref{fig:CLS},
this plant is connected 
through a sampler and a zero-order hold (ZOH)
to a time-stamp aware estimator and a controller,
which will be described soon.
\begin{figure}[tb]
	\centering
	\includegraphics[width = 8cm]{Closed_loop_verB.pdf}
	\caption{Closed-loop system with a time-stamp aware estimator.}
	\label{fig:CLS}
\end{figure}

Let  be sampling instants
from the perspective of the controller clock.
A sensor measures the state  and sends it to a controller
together with a time-stamp.
However, since the sensor and the controller may not be synchronized,
the time-stamp determined by the sensor 
typically includes an 
unknown offset with respect to the controller clock. 
In this paper, we assume that the clock offset is constant. 
Although
clock properties are affected by environment such as temperature
and humidity,
the change of such properties is slow 
for the time scales of interest.
Furthermore, the difference of 
clock frequencies can be ignored. 
This is justified by noting that 
time synchronization techniques, like the one proposed 
in \cite{He2014},
can achieve asymptotic convergence of the clock frequencies (in the mean-square sense),
even in the presence of random network delays.
We thus assume that
the time-stamp  reported by the sensor is given by

for some unknown constant .

Let  be the update period of the ZOH.
The control signal  
is assumed to be piecewise constant and updated periodically at times  () with values  computed by the controller:
 for .
We place a basic assumption for stabilization of sampled-data systems.
\begin{assumption}
	\label{ass:stabilizability}
	{\em(}Stabilizability and non-pathological control update{\em)}
	{\em
The plant  is stabilizable and the update period  is non-pathological,
		that is,  ()
		for each pair  of eigenvalues of .
	}
\end{assumption}


While the ZOH updates the control signal  periodically, 
the true sampling times  and the reported sampling times 
 may not be periodic. 
However, we do assume that both  and  do not fall 
behind  by more than the ZOH update period .
This assumption is formally stated as follows.
\begin{assumption}
	\label{ass:sample}
	{\em(}Bounded clock offset{\em)}
	{\em
For every ,
. 
	}
\end{assumption}
This assumption implies that
the clock offset  is smaller than the control update period ,
which holds 
in most mechatronics systems. In fact,
control update periods 
for mechatronics systems
generally take values from 100 s to 10 ms,
while
recent clock
synchronization algorithms such as the IEEE 1588 Precision Time Protocol (PTP)
\cite{PTP}
make clock offsets smaller than a few tens of microseconds.


\begin{figure}[tb]
	\centering
	\includegraphics[width = 8.7cm,clip]{clockmismatchA.pdf}
	\caption{Sampling instants , reported time-stamps , 
		and updating instants  of the zero-order hold.}
	\label{fig:clockmismatch}
\end{figure}


Fig.~\ref{fig:clockmismatch} shows the timing diagram
of the sampling instants ,
the reported time-stamps , 
and updating instants  of the control inputs.

The controller side is comprised of a time-stamp aware estimator and a
controller
as in the model-based or emulation-based control of 
networked control systems \cite{Garcia2014}.
The time-stamp aware estimator generates 
the state estimate 
from the data  according to the following dynamics:
 
Note that if the time-stamp is correct, i.e., 
, then this estimator consistently produces 
 for all , perfectly compensating
transmission delays.
Time-stamp aware estimators have been used to compensate for
network-induced imperfections, e.g., in \cite{Graham2004,Nakamura2008,Garcia2014}.


The controller is a discrete-time LTI system and 
generates the control input  based on the state estimate :
 
where  is the state of the controller.


The objective of the present paper is to find
a discrete-time LTI controller  as in \eqref{eq:controller} 
that achieves closed-loop stability for every
clock offset in a given range of admissible values.
Specifically, we want to solve the following problem:
\begin{problem}
	\label{problem:clockoffset}
	{\it
		Given an offset interval ,
		determine if there exists a controller  as in \eqref{eq:controller} such that
		,  as  and
		 as  
		for every 
		and for every initial states  and .
		Furthermore, if one exists, find such a controller .
	}
\end{problem}




\section{Discretization of the Closed-loop System}
To solve Problem \ref{problem:clockoffset},
we discretize the system comprised of 
the plant , the estimator , the ZOH, and the sampler.
In this section, we obtain a realization for the discretized system and 
describe its basic properties related to stability, stabilizability, and detectability.
Moreover, we extend the discretized system to scenarios with
disturbances/noise and output feedback.
\subsection{Discretized system and its basic properties}
The following lemma provides a realization for the discretized system:
\begin{lemma}
	\label{lem:discretization}
	Define 
	
	The dynamics of 
	the discretized system  
	comprised of the plant , the estimator , 
	the ZOH, and the sampler
	can be described by the following equations:
	
	where ,
	, and
	
\end{lemma}
\begin{proof}
	Using ,
	we have from the state equation
	\eqref{eq:original_plant} that
	
	
	We compute  in terms of  and .
	It follows from 
	the dynamics of the estimator  in \eqref{eq:estimator} that
	
	and
	
Since  and , it follows that
	
	and also that 
	
	Using  and ,
	we conclude 
from \eqref{eq:hat_x_t}--\eqref{eq:int_indentity} that
	
	
	From \eqref{eq:discretiztion_plant} and \eqref{eq:hat_x_t_final},
	we obtain the  and  in \eqref{eq:FGH_def}.
	Moreover, 
	we have 
	by the definition of the extended state . 
	\hspace*{\fill} 
\end{proof}


Next we show that if the extended state  and the controller state 
converge to the origin, then
the intersample values of  and  also converge to the origin.
\begin{proposition}
	\label{prop:stability_equivalence}
	For the discreteized system  in Lemma \ref{lem:discretization}, we have
	that  as  if and only if
	 as  and  as .
\end{proposition}
\begin{proof}
	The statement that 
	 as  and  as 
	imply 
	 as , follows directly
	from the definition of .
	
	To prove the converse statement,
	assume that  as .
	Then  and
	,  as .
	Since

	for all  and all ,
	we derive  ().
	Similarly, we see from the dynamics of the estimator  that
 as .
	This completes the proof.
	\hspace*{\fill} 
\end{proof}
This proposition allows us to conclude
Problem \ref{problem:clockoffset} can be solved by
finding LTI controllers  achieving ,  ()
for every 
and for every initial states  and .

The following result allows us to conclude that the
discretized system 
is detectable and stabilizable for all  and almost all 
if the plant  is stabilizable.
\begin{proposition}
	\label{prop:detectability_stabilizability}
	The discretized system  in 
	\eqref{eq:extended_sys} is detectable for all  and .
	Moreover,  is stabilizable for all  if 
	Assumption \ref{ass:stabilizability} holds.
\end{proposition}
\begin{proof}
	Let us first obtain another realization 
	 of 
	the discretized system  in \eqref{eq:extended_sys}.
	We can transform  into
	
Furthermore,
	if we define
	
	then we obtain
	
	and

We have thus another realization 
	
	for .
	
	Next we check detectability and stabilizability
	by using the realization 
	. 
Define
	
	Then we have that
	
	and clearly
	 is Schur stable.
Therefore,
	the discreteized system  is detectable 
	for all  and .
	
	To show stabilizability, we use
	the well-known rank conditions (see, e.g., \cite[Sec.~3.2]{zhou1996}).
	We have that 
is full row rank for all  if and only if
	
	is full row rank for all .
	Hence, the discretized system  is stabilizable for all  if 
	Assumption \ref{ass:stabilizability} holds.
\end{proof}




\subsection{Extension to the output feedback case 
	with disturbances and noise}
Instead of  in \eqref{eq:original_plant}, consider 
a plant  with disturbances, noise, and output feedback:

where  and , 
are the disturbance, measurement noise, and output of the plant, respectively.
As in \cite[Chap.~3]{Garcia2014}, \cite{Xu2005CDC}, 
and the references therein, we assume that
a smart sensor is co-located with the
plant and that the sensor has the following observer to 
generate the state estimate, which is sampled and sent to the controller side:

where  is the state estimate and  is an observer gain such that
 is Hurwitz.
The sampler sends the state estimate , 
and
the resulting dynamics of the time-stamp aware estimator  is provided by
 
where  is the quantization noise. 
A calculation similar to the one performed in the proof of
in Lemma \ref{lem:discretization} can be used to show
that the dynamics of the discretized system  is given by

where  and

The only difference from
the original idealized system  in \eqref{eq:extended_sys} is
that  has the disturbance . 
Hence, 
for the output feedback case with
bounded disturbances and noise,
solutions of Problem~\ref{problem:clockoffset} achieve
the boundedness of the closed-loop state.
\begin{proposition}
	Assume that  as 
	for the idealized system  in Lemma \ref{lem:discretization}
	(in the context of state feedback without distubances and measurement noise).
	If , , and  are bounded for all  and 
	all ,
then the states
	, , , and  are also bounded for all  and all
	.
	Moreover, if  for all  and 
	all , then
	, , , and  converge to the origin.
\end{proposition}
\begin{proof}
	Since 
	is bounded for every  and every , 
	it follows that  and  are also bounded
	for all .
	The rest of the proof follows the similar lines as that of 
	Proposition~\ref{prop:stability_equivalence}, 
	and hence it is omitted.
\end{proof}






See also \cite{Wakaiki2016ACC} for the -gain analysis 
of systems with time-varying offsets.




\section{Controller Design via Simultaneous Stabilization}
\subsection{Preliminaries}
We first consider a general simultaneous stabilization problem 
not limited to the system introduced in Section 2.

The transfer function  
of the system 
is usually defined by the Z-transform of the system's impulse response,
i.e.,
,
but in this paper, we define the transfer function  
by  for 
consistency of the Hardy space theory;
see \cite[Sec. 2.2]{vidyasagar1985} for details.
Hence the transfer function of a causal system is not proper.
We say that  
{\it stabilizes} 
if , , and  belong
to . 
We recall that when these three transfer functions belong to 
, 
they will have no poles in the closed unit disk.


Consider the family of plants  
parameterized by , 
where  is a nonempty parameter set,
and assume that we have a doubly coprime factorization of 
 over  

where  and 
 are
a right coprime factorization and a left coprime factorization,
respectively.
We explicitly construct the matrices in \eqref{eq:doubly_coprime} using
a stabilizable and detectable realization of ;
see, e.g.,
\cite[Theorem 4.2.1]{vidyasagar1985}.

The following theorem provides a necessary and sufficient condition 
for simultaneous stabilization:
\begin{theorem}[\hspace{-0.001pt}\cite{Vidyasagar1982, vidyasagar1985}]
	\label{thm:SS_CM}
	{\it
Given a nonempty set , consider the plant  
		having a doubly coprime factorization \eqref{eq:doubly_coprime} 
		for each .
		Fix  and define
		
		Then  is right coprime for every
		. Moreover, 
		there exists a controller that stabilizes 
		 for every  if and only if
		there exists  such that
		for all ,
		
		Such a stabilizing controller is given by
		
	}
\end{theorem}




\begin{remark}
		Although the simultaneous stabilization of a {\em finite} family of plants is considered in
		\cite[Sec. 5.4]{vidyasagar1985} and 
		\cite{Vidyasagar1982},
		generalization to an {\em arbitrary} family of plants is readily apparent,
		as mentioned in the last paragraph of Section 3 in \cite{Vidyasagar1982}.
\end{remark}
\begin{remark}
		A {\em left} coprime factorization of stabilizing controllers
		is used in 
		\cite[Sec. 5.4]{vidyasagar1985} and 
		\cite{Vidyasagar1982},
		whereas we represent controllers by a {\em right} coprime 
		factorization in \eqref{eq:SScontroller}.
		Therefore, Theorem \ref{thm:SS_CM} is slightly different
		from its counterpart in 
		\cite[Sec. 5.4]{vidyasagar1985} and
		\cite{Vidyasagar1982}.
\end{remark}

\subsection{Robust Controller Design}
It is generally not easy to verify 
in a computationally efficient fashion
that
a transfer function  satisfying 
\eqref{eq:SS_Unimodular} exists.
In the next theorem, we develop a simple sufficient condition for
\eqref{eq:SS_Unimodular} to hold, by exploiting geometric properties
on  inspired by results
on strong stabilization \cite{zeren2000}.
\begin{theorem}
	\label{thm:suff}
	{\it
		Given a nonempty set , 
		assume that each plant  ()
		has a doubly coprime factorization \eqref{eq:doubly_coprime} 
		such that 
		there exist , 
		,
		and  satisfying 
		 and
		
		for all .
If there exists  
		satisfying
		the following -norm condition:
		
		then  satisfies \eqref{eq:SS_Unimodular}, and hence
		the controller  in \eqref{eq:SScontroller} stabilizes 
		 for every .
	}
\end{theorem}
\begin{proof}
	We define  and  as in
	\eqref{eq:UVdef}. Since , 
	it follows from \eqref{eq:UVdef} and
	the Bezout identity  
	in \eqref{eq:doubly_coprime} that
	
	Moreover, since , we obtain
	
	Hence 
	
	Since  for all
	 satisfying ,
	it follows that
	if
	
	then \eqref{eq:SS_Unimodular} holds for all . 
	From the assumption \eqref{eq:N_cond}, 
	
	Hence if  satisfies \eqref{eq:Nehari_problem} 
	for all ,
	then 
	\eqref{eq:Q_N_norm} holds, and consequently
	
	is simultaneously stabilizable by  in \eqref{eq:SScontroller}
	from Theorem \ref{thm:SS_CM}.
	\hspace*{\fill}  
\end{proof}

The proposition below shows that our discretized system  in \eqref{eq:extended_sys}
always satisfies
the assumptions on  and 
that appear in Theorem \ref{thm:suff}. 
This result also provides the matrices  and  in \eqref{eq:N_cond}
without explicitly calculating a coprime factorization of  
for all .
\begin{proposition}
	\label{prop:DN_cond}
Define the transfer function .
	For all ,
	there exists a doubly coprime factorization \eqref{eq:doubly_coprime}
	such that ,
and 
	\eqref{eq:N_cond} holds with
	
\end{proposition}
\begin{proof}
	Consider the realization  in
	the proof of 
	Proposition \ref{prop:detectability_stabilizability}.
	For every ,
	the matrix  in \eqref{eqw:L_def}
	achieves the Schur stability of
	 as shown in \eqref{eq:FLH}.
	From the realization of , e.g., in
	\cite[Theorem 4.2.1]{vidyasagar1985},
	we can write
	 as
	
Noticing that the far right-hand side of the equation above
	does not depend on ,
	we have .
	
	It follows that . 
	From the realization ,
	we see that
	
	Since  and  for ,
	it follows that

	
	On the other hand,
	we have

and
	
	Since ,
	it follows that .
	Therefore we derive from \eqref{eq:PD-P0_first}
	
	Similarly to \eqref{eq:int_exp}, we have
	,
and hence

Since , , and 
	are commutative, we derive 
	
	and 

Thus \eqref{eq:N_cond} holds with
	 in \eqref{eq:R_def} and
	 in \eqref{eq:W_def}
	\hspace*{\fill}  
\end{proof}

Define 

From Theorem \ref{thm:suff}, to obtain a controller  as in  \eqref{eq:controller},
it is enough to solve the following suboptimal problem:
Find  satisfying
.
This problem is equivalent to a standard suboptimal  control problem
\cite[Chaps.~16,~17]{zhou1996}:
Find  such that 
, where  is defined by

The results of this section can be summarized through the following controller design algorithm:
\begin{algorithm}
	\label{alg:controller_design}
	\begin{enumerate}
		\item
		Using the realization 
		
		the matrix 
		
and an arbitrary matrix  such that  is Schur stable,
		set
		
		
		\item 
		For a given offset interval ,
		set  as in \eqref{eq:gamma}, and
		solve the  control problem \cite[Chaps.~16,~17]{zhou1996}:
		Find  such that 
		,
		where  is defined by \eqref{eq:G_def}.
		
		
		\item 
		If the  control problem is not solvable,
		then the algorithm fails.
		Otherwise the transfer function  of the 
		controller 
		is given by .
	\end{enumerate}
\end{algorithm}



\begin{remark}
		We have from Proposition \ref{prop:DN_cond}
		that  for constant , where
		 is expressed as the nominal component  
		plus the uncertainty block .
		If we obtain a similar formula for the case of time-varying offsets as
		studied for systems with aperiodic sampling in \cite{Fujioka2009},
		we can deal with the stabilization problem of systems with time-varying offsets
		through a small gain theorem. Although
		the uncertainty part of the discretized system  may be non-causal, 
		the small gain theorem for systems with non-causal uncertainty in \cite{Unal2008Automatica}
		can be used.
		This extension is a subject for future research.
\end{remark}


\begin{example}
Consider the unstable batch reactor studied 
in \cite{Rosenbrock1972}, where
the system matrices  and  in \eqref{eq:original_plant}
are given by

This example has been developed over the years as a benchmark example for networked
control systems, and
its data were transformed by a change of basis
and time scale \cite{Rosenbrock1972}.

Here we compare the proposed method with the robust stabilization method in
\cite{doyle1981} and \cite[Chap.~7]{vidyasagar1985} based on
the following fact: {\em
	Consider a family of plants  with .
	Assume 
	that 
	 has no poles on  and the same number of unstable poles for every 
	and
	that a function  satisfies
	 
	for all 
	and all .
	If the controller  
	stabilizes  and
	satisfies 
	
	then  stabilizes  for all .}
The order of such a controller is typically equal to
the order of the following transfer function:


We compute the length of the allowable offset interval 
obtained from the sufficient condition \eqref{eq:Nehari_problem} for each ,
which is shown as the solid line in Fig.~\ref{fig:MIMO_ex}.
On the other hand, the dashed line in the figure represents 
the length of the offset interval 
obtained from the robust control approach that leads to the condition
\eqref{eq:suff_cond_conv} with an appropriate function 
satisfying \eqref{eq:r_def}.
For example, we use  for  and , and this  satisfies \eqref{eq:r_def} and

for all .
The solid line is obtained by finding the maximum and minimum of
 that satisfies
the condition

whereas to derive the dashed line, we first calculate  satisfying
\eqref{eq:r_def} for a fixed  and
then check the existence of a controller  that stabilizes  and achieves the
-norm condition \eqref{eq:suff_cond_conv}.
We see from Fig.~\ref{fig:MIMO_ex}
that the proposed sufficient condition \eqref{eq:Nehari_problem} 
is less conservative
than \eqref{eq:suff_cond_conv}.

Consider the case , and let   and  be
controllers that are obtained from the sufficient conditions 
\eqref{eq:Nehari_problem}  and  \eqref{eq:suff_cond_conv} with
the maximum offset length, respectively.
The order of the controller  
is 7, but 
applying balanced model truncation \cite[Chap.~6]{zhou1996}
to the controller ,
we can obtain
an approximated controller  with order 5,
which 
satisfies 
.
From
iterative calculations of the eigenvalues of the 
discretized closed-loop system for each ,
we find that both  and 
stabilize the discretized system  in \eqref{eq:extended_sys} 
for all .
The controller 
has order  and
allows the offsets  without compromising the closed-loop stability.
Approximated controllers with any order obtained by applying balanced 
model truncation to 
do not achieve the closed-loop stability even in the case .
For comparison, a linear quadratic regulator whose state weighting matrix  and
input weighting matrix are identity matrices with appropriate dimension stabilizes
the discretized system  in \eqref{eq:extended_sys} 
only for .
From this numerical result, we see that
the derived controller achieves better robust performance against clock offsets 
than
a linear quadratic regulator designed without regard to the clock offset and
also than the robust controller based on \eqref{eq:suff_cond_conv}.

\begin{figure}[tb]
	\centering
	\includegraphics[width = 8cm,clip]{MIMO_example.pdf}
	\caption{Allowable offset length  versus sampling period .}
	\label{fig:MIMO_ex}
\end{figure}	
\end{example}


\section{Exact Bound on Offsets for First-order Systems}
In this section, the bounds
on the clock offset
that were obtained for LTI controllers 
(from Theorems \ref{thm:scalar_LTIcontroller} 
and \ref{thm:suff}) are compared with
the exact bound that would be allowed 
if we restricted our attention to a time-invariant 
static output feedback controller, and with
an offset range that gives a sufficient condition 
for the existence of a time-varying 
2-periodic static output feedback controller.
We also derive a bound obtained using 
standard robust control tools, regarding the clock offset as an
additive uncertainty.


In this section, the plant class is restricted to scalar systems, 
and we reduce the necessary and sufficient condition for stabilizability
in Theorem \ref{thm:SS_CM} to a computationally verifiable one,
which gives an explicit formula for the exact bound on
the clock offset that LTI controllers can allow.


Consider an unstable scalar plant:
 
with .
If , the stabilization problem is trivial because a
zero control input  leads to the stability of the closed-loop system.
So in the reminder o this section, 
we will focus our attention to the case .
The case  will be addressed separately later.

Solving explicitly the integrals that appear in \eqref{eq:FGH_def},
the extended system \eqref{eq:extended_sys} is given by

where
 and

In what follows, we take  for simplicity of notation,
because stabilizability does not depend on the value of this ratio.


The extended system \eqref{eq:scalar_sys}
is stabilizable and detectable except for , at which
point the system loses detectability. 
Since , 
it follows that 

We have from Assumption \ref{ass:sample} 
that , and hence
the set  on  is a subset of


As in Section 3,
taking the Z-transform of
\eqref{eq:scalar_sys} and then
mapping , we obtain the transfer function :

The system \eqref{eq:scalar_TF}
belongs to a class of the so-called interval systems.
The stabilization of general interval systems
has been studied, e.g., in
\cite{Ghosh1988, Olbrot1994}.
Here we shall develop a new approach based on
Theorem \ref{thm:SS_CM}.

\subsection{Main result for scalar plants}
The following theorem gives the exact bound on the 
clock offset for scalar systems:
\begin{theorem}
	\label{thm:scalar_LTIcontroller}
	{\it
Define  and .
		Let  and 
		consider the set  in \eqref{eq:Delta_theta} of the form
		.
		There exists a controller that stabilizes 
		 in \eqref{eq:scalar_TF} 
		for all , that is,
		there exists  satisfying
		\eqref{eq:SS_Unimodular} for all 
		if and only if
		
		In particular, if , then \eqref{eq:M1M2cond}
		is equivalent to
		
		
		Furthermore, 
		define a conformal mapping  from
		
		to 
		 by
		
		If \eqref{eq:M1M2cond} holds, then
		a finite-dimensional stabilizing controller  is given by 
		
		where the  functions 
		, , , , , and  are defined by
		
		for any arbitrarily fixed  with , and
		any rational function 
		 that satisfies
		the interpolation conditions ,
		, and .
		Such a function  always exists if \eqref{eq:M1M2cond} holds.
	}
\end{theorem}
\begin{proof}
	See Section 4.2.
\end{proof}

\begin{remark}
	The rational function  in Theorem \ref{thm:scalar_LTIcontroller}
	can be obtained from the Schur-Nevanlinna algorithm; see, e.g., 
	\cite{luxemburg2010, wakaiki2012}.
\end{remark}	

\begin{remark}
	\label{rem:related_work}
	Since the inverse mapping  is given by
	the following rational function:
	
	the stabilizing controller  is finite dimensional for 
	a rational function .
\end{remark}


If we change the offset variable from  to ,
then \eqref{eq:Delta_theta} and \eqref{eq:M1M2cond} give
the maximum 
length of the offset interval  allowed 
by an LTI controller.
\begin{corollary}
	\label{coro:maximum_length}
	Assume .
	There exists a controller that stabilizes 
	the extended system \eqref{eq:scalar_sys}
	for all  if and only if 

\end{corollary}
\begin{proof}
	See Section 4.2.
\end{proof}

\begin{remark}	
	\label{rem:a_equal_zero}	
	In the case , the extended system  is given by
	
	Similarly to the case , one can show that
	there exists a controller stabilizing 
	for all .
	This result is consistent with that in the case when
	 in
	Corollary \ref{coro:maximum_length}, but
	we omit the proof for brevity.
\end{remark}

\begin{example}
	Consider a scalar plant with unstable pole . 
	In Fig.~\ref{fig:h_dependency}, 
	we plot the maximum length of the offset interval 
	versus the ZOH-update period . 
	The solid line is the maximum length 
	and 
	the vertical dotted lines indicate .
	If , then the restriction  arising from Assumption \ref{ass:sample} gives the bound
	.
	On the other hand, if , then
	 is bounded by \eqref{eq:Delt_bound_scalar}.
	The maximum offset length 
	exponentially decreases as  becomes larger. 
	\begin{figure}[bt]
		\centering
		\includegraphics[width = 9.5cm]{h_dependency.pdf}
		\caption{Maximum offset length  
			versus ZOH-update period  (unstable pole ):
			The solid line is 
			the maximum offset length. The vertical dotted line
			indicates .}
		\label{fig:h_dependency}
	\end{figure}
\end{example}

\subsection{Proofs of Theorem \ref{thm:scalar_LTIcontroller}
	and Corollary \ref{coro:maximum_length}}
We prove Theorem \ref{thm:scalar_LTIcontroller}
by reducing the stabilization problem 
to a
Nevanlinna-Pick interpolation problem.
This reduction relies on results stated in 
Lemmas \ref{thm:SStoInt} and \ref{thm:ReductionNPInt}
have appeared in \cite{Ghosh1988, Olbrot1994}, 
but we give new proofs of these results 
based on Theorem \ref{thm:SS_CM}.

First we show that
the stabilization problem is equivalent
to an interpolation problem with a specified codomain:
\begin{lemma}
	\label{thm:SStoInt}
	{\it 
		Let  and 
		.
		There exists a controller that stabilizes
		 in \eqref{eq:scalar_TF} 
		for all , that is,
		there exists  satisfying
		\eqref{eq:SS_Unimodular} for all 
		if and only if
		there exists  such that 
		 is a map from  to
		 and
		satisfies
		the interpolation conditions , , 
		and . 
	}
\end{lemma}
\begin{proof}
We obtain an  coprime factorization 
 with

where  is a fixed complex number with . 
If we define  and  as in \eqref{eq:XY_def},
then the Bezout identity  holds.
Hence defining  and  by \eqref{eq:XY_def},
we see that
 in \eqref{eq:SS_Unimodular} satisfies


Theorem \ref{thm:SS_CM} and \eqref{eq:separation} show that
the plant 
is simultaneously stabilizable by a single LTI controller
if and only if there exists  such that

We have \eqref{eq:thetaT1T2_inv} if and only if 
 has no zero in  
for all , that is,

for all .

It is now enough to show that
 if and only if
 satisfies  
and the interpolation conditions in the lemma.

Suppose that . Since
, we have

Moreover,
since the unstable zeros of  are , , and 
and since  has no unstable poles,
it follows that , , and
.

Conversely, let  satisfy 
, , and .
If we define 

then  belongs to .
Assume, to get a contradiction, 
that . Since
 
it follows that
 has some
unstable poles that are zeros of  in . 
Let  be one of the poles.
Since  has only simple zeros in , it follows that
. The interpolation conditions of  lead to
, which contradicts the equality in \eqref{eq:QT2}.
\end{proof}

From Lemma \ref{thm:SStoInt},
it suffices to study the following interpolation problem
for stabilizability:
\begin{problem}
	\label{prob:interpolation}
	{\it
		Let  be distinct points in  and
		let  belong to
		.
		Find a function  such that
		 
}
\end{problem}

We solve Problem \ref{prob:interpolation}
by reducing it to the Nevanlinna-Pick interpolation.
To this effect,
we need a conformal map from 

to .
In \cite{Olbrot1994}, 
\cite[Section 4.1]{foias1996},
such a conformal map  is given in
\eqref{eq:conformal_map}.


Using
the conformal map defined in \eqref{eq:conformal_map},
we see that
Problem \ref{prob:interpolation} can be reduced to
the Nevanlinna-Pick interpolation problem.
\begin{lemma}
	\label{thm:ReductionNPInt}
	{\it
		Problem \ref{prob:interpolation} is solvable if and only if
		the following Nevanlinna-Pick interpolatoin problem is solvable:
		Find a function  
		such that
		
	}
\end{lemma}

\begin{proof}
Let  be a solution to Problem \ref{prob:interpolation}, and
set . Then we derive
the equivalence between \eqref{eq:F_int_cond} and \eqref{eq:G_int_cond}.
Since  is a conformal map,
we see that  is holomorphic in  if and only if
 is so.

Regarding the rationality of solutions, 
since  is given by
a rational function in \eqref{eq:phi_inv}
and since , 
it follows that  is rational for every rational solution .

Conversely,  may not be rational for a
rational function . 
However,  is holomorpic in , 
and hence it is an irrational
solution of
the Nevanlinna-Pick interpolation problem.
If the Nevanlinna-Pick interpolation problem is solvable, 
then there exists a rational solution,
which can be obtained from the Schur-Nevanlinna algorithm, e.g., in
\cite{luxemburg2010, wakaiki2012}
and the explicit formula of the solutions in 
\cite[Sec. 2.11]{foias1996}.
We therefore have the desired rational function .
\end{proof}

Interpolating functions and a conformal map in Lemma \ref{thm:ReductionNPInt} 
are illustrated by the commutative diagram in Fig.~\ref{fig:diagram}.

\begin{figure}[tb]
	\centering
	\includegraphics[width = 6cm,clip]{Diagram_clockA.pdf}
	\caption{ Interpolating functions  and a conformal map .}
	\label{fig:diagram}
\end{figure}



Finally we obtain the proof of Theorem \ref{thm:scalar_LTIcontroller}.

\noindent\hspace{1em}{\itshape 
	{\bf Proof of Theorem \ref{thm:scalar_LTIcontroller}:} }
Lemmas \ref{thm:SStoInt} and \ref{thm:ReductionNPInt} show that
the stabilization problem for systems with clock offsets
can be reduced to 
the Nevanlinna-Pick interpolation problem 
with a boundary condition; see, e.g., 
\cite[Sec. 2.11]{foias1996} for the interpolation problem.
We therefore obtain a necessary and sufficient condition 
based on the positive definiteness of
the associated Pick matrix:

From the Schur complement formula, \eqref{eq:PickMatrix} is equivalent to

We see that \eqref{eq:kappa_p_1} is 

Since , 
it follows that \eqref{eq:p_sqrt} is equivalent to

After rearranging this, we derive \eqref{eq:M1M2cond}.
\hspace*{\fill} 

\noindent\hspace{1em}{\itshape {\bf Proof of Corollary 
		\ref{coro:maximum_length}:} }
Substituting  and
 into \eqref{eq:1_theta}, 
we obtain

and hence

Taking the logarithm function of both sides gives the desired conclusion.
\hspace*{\fill} 




\subsection{Comparison with time-invariant/2-periodic 
	static controllers}
The proposition below gives the exact bound
on the clock offset that could be obtained using a static
stabilizer for a scalar plant.
\begin{proposition}
	\label{prop:static}
	{\it
		Consider the extended system \eqref{eq:scalar_sys}.
		Define , , and
		 as in \eqref{eq:Imax_def}.
		There exists  a static output feedback controller
		 that achieves 
		for every  
		if and only if
		
	}
\end{proposition}
\begin{proof}
	Without loss of generality, we assume that .
	Introducing the static controller
	
	into the extended system \eqref{eq:scalar_sys}, we have
	
	From the Jury stability criterion, the above system is stable if and only
	if the following three inequalities hold:
	
	
From (\ref{staticJury-1}) and , we have
	
	Therefore \eqref{staticJury-2}
	and \eqref{staticJury-3} give a lower and upper bound on ,
	respectively:
	
	Notice that the lower (upper)
	bound in (\ref{staticJury-2-1}) is increasing (decreasing) with
	respect to .
Hence these bounds take the inifimum and the supremum under
	(\ref{staticJury-1-1}) when , and
	
	Thus,
	there exists a
	static controller  that stabilizes 
	the extended system \eqref{eq:scalar_sys}
	for all
	 if and only if  satisfies \eqref{eq:static_bound}.
\end{proof}

The next result provides a sufficient condition on the offset range

for the existence of time-varying
2-periodic controllers that 
stabilize the extended system \eqref{eq:scalar_sys}.
\begin{proposition}
	\label{prop:static_periodic}
	{\it
		Consider the extended system \eqref{eq:scalar_sys}.
		Define , ,
		and  as in \eqref{eq:Imax_def}.
		There exists a 2-periodic static controller
		
		that achieves 
		for every 
		if
		
		where 
		
	}
\end{proposition}

\begin{proof}
This is also based on the Jury stability criterion.

Without loss of generality, we assume that .
With the 2-periodic controller \eqref{eq:class_periodic}, 
the extended system \eqref{eq:scalar_sys} can be written as

where



Denote the characteristic polynomial  of the matrix in
(\ref{extsys_2p}) by .
The coefficients  are given as

where  and .
From the Jury stability test, we have that stability of (\ref{extsys_2p}) is
equivalent to the following three inequalities i)--iii):
i) The first condition is given by


ii) Furthermore,

This inequality is equivalent to the following inequalities:


iii) Finally,



In what follows, we fix a controller, or , and then evaluate the
range of permissible  with the controller.
Suppose that  and .
For such parameters , (\ref{2pJury-1}) and (\ref{2pJury-2}) are
reduced to

and

respectively.
Select the parameters  so that the lower bounds on  in
(\ref{2pJury-1-1}) and (\ref{2pJury-2-1}) coincide with each other.
That is,  and  are chosen to satisfy the following relation:

Moreover, we select  so that (\ref{2pJury-3}) holds for any
.
This implies that

With the above class of controllers,
where  and  satisfy , , (\ref{LBcond}),
and (\ref{2pJury3-1}), the conditions (\ref{2pJury-1})--(\ref{2pJury-3}) for
stability hold if and only if
(\ref{2pJury-1-1}) follows.
We will show that  and  satisfy , , (\ref{LBcond}),
and (\ref{2pJury3-1}) if and only if

where  is defined in \eqref{def:kappa}.
We then analyze the bounds on  followed by (\ref{2pJury-1-1}) when the
controller belongs the class characterized by (\ref{class_gamma}) and
(\ref{class_eta}).

The equation (\ref{class_eta}) is obtained from (\ref{LBcond}) with
the fact that .

We now aim to show (\ref{class_gamma}).
Substituting (\ref{LBcond}) into (\ref{2pJury3-1}) and using ,
we have that (\ref{2pJury3-1}) is satisfied if and only if


From (\ref{class_eta}) and (\ref{2pJury3-2}),  satisfies

where  is given as

Note that  from .
Thus, from (\ref{2pJury-3-3}),  satisfies one of the following two cases:
i)  and , or
ii)  and .
A routine calculation shows that .
Thus, i) is reduced to  and ii) is 
to .
On the other hand, from  and (\ref{class_eta}), it follows that
 and .
Therefore, we arrive at (\ref{class_gamma}). 


Conversely, (\ref{class_gamma}) implies that .
Moreover, from (\ref{class_eta}), we have (\ref{LBcond}) and

The right-hand side is negative by (\ref{class_gamma}) and the fact
.
Note also that (\ref{2pJury3-1}) holds if  and  are taken as
(\ref{class_gamma}) and (\ref{class_eta}).

Finally, taking the supremum and the infimum of the upper bound and the lower
bound on  in (\ref{2pJury-1-1}) over (\ref{class_gamma}), we conclude
the proof.
\end{proof}


We are now in a position to compare the bounds
\eqref{eq:LTI_bound_M}, \eqref{eq:static_bound}, and
\eqref{eq:2peridic_bound}.
For all , a routine calculation shows that 

As expected, the offset condition \eqref{eq:static_bound}
for time-invariant static controllers 
results in the smallest range for values of 
because the set of all time-invariant static controllers
is a subset of the class of LTI controllers and 
that of 2-periodic static controllers in \eqref{eq:class_periodic}.
On the other hand,
2-periodic static stabilizers do not belong to the class of LTI controllers,
and vice versa.
The second inequality in \eqref{eq:comparison_static_periodic}
always holds for all , but the bound \eqref{eq:comparison_static_periodic}
is a sufficient condition.
In order to compare 
the ability to robustly stabilize the closed loop of
2-periodic static controllers versus
LTI controllers, we need to do a brute-force computation for 
the exact bound on clock offset that would be allowed by
a 2-periodic static controller. 

\begin{example}
		Consider a scalar plant with ZOH-update period . 
		Fig.~\ref{fig:comparison} shows the maximum offset lengths 
		 allowed by
		LTI stabilizers and static ones, which are
		obtained by 
		Theorem \ref{thm:scalar_LTIcontroller}
		and
		Proposition \ref{prop:static}, respectively. 
		The figure also 
		gives a lower bound on the maximum offset length obtained
		using 2-periodic static stabilizers,
		which is derived from Proposition \ref{prop:static_periodic}.
		All lines decreases exponentially as the unstable pole 
		 grows to .
		If the unstable pole  is smaller than 0.9, then
		Assumption \ref{ass:sample} gives the bound 
		.
		We also observe that LTI controllers double the robustness 
		with respect to that
		achieved by time-invariant static controllers.
		For example, for the unstable pole ,
		the maximum offset length by LTI controllers
		is ,
		whereas that by time-invariant static controllers is 
		. 
		
		\begin{figure}[bt]
			\centering
			\includegraphics[width = 9.5cm]{scalar_comparison.pdf}
			\caption{Comparison among the maximum offset length 
				 obtained using 
				each class of controllers (): 
				The solid (dashed) line is the exact
				bound on the maximum offset length allowed by LTI 
				(time-invariant static) controllers.
				The dashed-dotted line is a lower bound on the maximum offset length
				by 2-periodic static controllers.}
			\label{fig:comparison}
		\end{figure}
\end{example}



\subsection{Conservativeness due to the small gain theorem}
The sufficient condition in Theorem \ref{thm:suff} with
 in
\eqref{eq:N_cond}, shows that 
there exists a controller stabilizing 
for all .
This bound of  
is the same as that was obtained in
\eqref{eq:static_bound} for a static controller,
which shows that 
the use of the small gain theorem is as conservative as
the restriction of controllers to static gains.

This conservativeness arises from the codomain of
interpolating functions.
For simplicity, let the offset bound 
.
In Theorem \ref{thm:suff},
the stabilization problem we consider is equivalent to
finding  satisfying ,
, .
Recall that a necessary and sufficient condition for stabilization in 
Lemma \ref{thm:SStoInt}
is the existence of a function 
 satisfying the same interpolation
conditions.
The difference between the codomains  and
 leads to
conservativeness in the stabilization analysis.


\subsection{Regarding a clock offset as an 
	additive uncertainty}
The transfer function  
in \eqref{eq:scalar_TF} can be viewed
as a perturbation of the nominal transfer function

by the following additive uncertainty:

where

Since 
,
it follows that

for all  and 
with .
Hence, 
as shown in \cite[Sec. 3.5]{foias1996},
there exists a controller stabilizing 
for all  
if  satisfies

for some ,
where the  functions
 and  are part of the coprime factorization 
 
satisfying the Bezout identity .
Reducing the problem of finding  with
\eqref{eq:robut_one_block} to the Nevanlina-Pick interpolation problem
as in \cite[Sec. 4.3]{foias1996}, we see that 
 satisfies
\eqref{eq:robut_one_block} for some 
if and only if


This bound derived from the classical approach of  
robust control turns out to also coincide with the one
in \eqref{eq:static_bound} obtained for a static controller.
This shows that
the classical  robust control approach uses
an unnecessarily large class of parametric uncertainty. 
In conjunction with the observation in Section 5.2, 
this discussion implies that
the use of the small gain theorem, the over-approximation of a
offset uncertainty
by an -additive uncertainty, and
the restriction of controllers to static gains have
the same level of conservativeness for scalar plants.


\section{Concluding Remarks}
We studied the problem of stabilizing systems
in which the sensor and the controller have a constant
clock offset. 
We formulated the problem as the stabilization problem for systems with
parametric uncertainty.
For multi-input systems, we derived 
a sufficient condition that 
is numerically testable,
based on the results of simultaneous stabilization.
For first-order systems,
we obtained
the maximum offset length
that can be allowed by an LTI controller.
However, a full investigation of the problem 
for general-order systems 
and systems with model uncertainty  is still 
an open area for future research.




\begin{thebibliography}{10}
	
	\bibitem{PTP}
	{\em Precision Clock Synchronization Protocol for Networked Measurement and
		Control Systems}.
	\newblock IEC 61588(E):2004--IEEE Std. 1588(E), 2002.
	
	\bibitem{Blondel1993Suf}
	V.~Blondel, G.~Campion, and M.~Gevers.
	\newblock {A sufficient condition for simultaneous stabilization}.
	\newblock {\em IEEE Trans. Automat. Control}, 38:1264--1266, 1993.
	
	\bibitem{Briat2013}
	C.~Briat.
	\newblock Convex conditions for robust stability analysis and stabilization of
	linear aperiodic impulsive and sampled-data systems under dwell-time
	constraints.
	\newblock {\em Automatica}, 49:3449--357, 2013.
	
	\bibitem{Briat2012}
	C.~Briat and A.~Seuret.
	\newblock Convex dwell-time characterizations for uncertain linear impulsive
	systems.
	\newblock {\em IEEE Trans. Automat. Control}, 57:3241--3246, 2012.
	
	\bibitem{Cantoni2012}
	M.~Cantoni, U.~T. J\"onsson, and C.-K. Kao.
	\newblock Robustness analysis for feedback interconnections of distributed
	systems via integral quadratic constraints.
	\newblock {\em IEEE Trans. Automat. Control}, 57:302--317, 2012.
	
	\bibitem{Oliveira1999}
	M.~C. de~Oliveira, J.~Bernussou, and J.~C. Geromel.
	\newblock A new discrete-time robust stability condition.
	\newblock {\em Systems Control Lett.}, 37:261--265, 1999.
	
	\bibitem{Donkers2011}
	M.~C.~F. Donkers, W.~P. M.~H. Heemels, N.~van~de Wouw, and L.~Hetel.
	\newblock Stability analysis of networked control systems using a switched
	linear systems approach.
	\newblock {\em IEEE Trans. Automat. Control}, 56:2101--2115, 2011.
	
	\bibitem{doyle1981}
	J.~C. Doyle and G.~Stein.
	\newblock {Multivariable feedback design: Concepts for a classical/modern
		synthesis}.
	\newblock {\em IEEE Trans. Automat. Control}, 26:4--16, 1981.
	
	\bibitem{foias1996}
	C.~Foia\c{s}, H.~\"Ozbay, and A.~Tannenbaum.
	\newblock {\em {Robust Control of Infinite Dimensional Systems:~Frequency
			Domain Methods}}.
	\newblock London: Springer, 1996.
	
	\bibitem{Freris2011}
	N.~M. Freris, S.~R. Graham, and P.~R. Kumar.
	\newblock {Fundamental limits on synchronizing clocks over network}.
	\newblock {\em IEEE Trans. Automat. Control}, 56:1352--1364, 2011.
	
	\bibitem{Fridman2004}
	E.~Fridman, A.~Seuret, and J.-P. Richard.
	\newblock {Robust sampled-data stabilization of linear systems: An input delay
		approach}.
	\newblock {\em Automatica}, 40:1441--1446, 2004.
	
	\bibitem{Fujioka2009}
	H.~Fujioka.
	\newblock A discrete-time approach to stability analysis of systems with
	aperiodic sample-and-hold devices.
	\newblock {\em IEEE Trans. Automat. Control}, 54:2440--2445, 2009.
	
	\bibitem{Garcia2014}
	E.~Garcia, P.~J. Antsaklis, and A.~Montestruque.
	\newblock {\em Model-Based Control of Networked Systems}.
	\newblock Springer, 2014.
	
	\bibitem{Ghosh1988}
	B.~K. Ghosh.
	\newblock {An approach to simultaneous system design. Part II: Nonswitching
		gain and dynamic feedback compensation by algebraic geometric methods}.
	\newblock {\em SIAM J. Control Optim.}, 26:919--963, 1988.
	
	\bibitem{Graham2004}
	S.~Graham and P.~R. Kumar.
	\newblock {Time in general-purpose control systems: The Control Time Protocol
		and an experimental evaluation}.
	\newblock In {\em Proc. 43rd IEEE CDC}, 2004.
	
	\bibitem{He2014}
	J.~He, P.~Cheng, L.~Shi, C.~Chen, and Y.~Sun.
	\newblock {Time synchronization in WSNs: A maximum-value-based consensus
		approach}.
	\newblock {\em IEEE Trans. Automat. Control}, 59:660--675, 2014.
	
	\bibitem{Hespanha2007}
	J.~P. Hespanha, P.~Naghshtabrizi, and Y.~Xu.
	\newblock {A survey of recent results in networked control systems}.
	\newblock {\em Proc. IEEE}, 95:138--162, 2007.
	
	\bibitem{Hetel2017}
	L.~Hetel, C.~Fiter, H.~Omran, A.~Seuret, E.~Fridman, J.-P. Richard, and S.-I.
	Niculescu.
	\newblock {Recent developments on the stability of systems with aperiodic
		sampling: An overview}.
	\newblock {\em Automatica}, 76:309--335, 2017.
	
	\bibitem{Jiang2013}
	X.~Jiang, J.~Zhang, J.~J. Harding, B. J.~Makela, and A.~D.
	Dom{\'\i}ngues-Garc{\'\i}a.
	\newblock {Spoofing GPS receiver clock offset of phasor measurement units}.
	\newblock {\em IEEE Trans. Power Systems}, 28:3253--3262, 2013.
	
	\bibitem{luxemburg2010}
	L.~A. Luxemburg and P.~R. Brown.
	\newblock {The scalar Nevanlinna-Pick interpolation problem with boundary
		conditions}.
	\newblock {\em J. Comput. Appl. Math.}, 235:2615--2625, 2011.
	
	\bibitem{Maeda1984}
	H.~Maeda and M.~Vidyasagar.
	\newblock {Some results on simultaneous stabilization}.
	\newblock {\em Systems Control Lett.}, 5:205--208, 1984.
	
	\bibitem{Mirkin2007}
	L.~Mirkin.
	\newblock Some remarks on the use of time-varying delay to model
	sample-and-hold circuits.
	\newblock {\em IEEE Trans. Automat. Control}, 52:1109--1112, 2007.
	
	\bibitem{Naghshtabrizi2010}
	P.~Naghshtabrizi, J.~P. Hespanha, and A.~R. Teel.
	\newblock {Stability of delay impulsive systems with application to networked
		control systems}.
	\newblock {\em Trans. Inst. Meas. Control}, 32:511--528, 2010.
	
	\bibitem{Nakamura2008}
	Y.~Nakamura, K.~Hirata, and K.~Sugimoto.
	\newblock Synchronization of multiple plants over networks via switching
	observer with time-stamp information.
	\newblock In {\em Proc. SICE Annu. Conf.}, 2008.
	
	\bibitem{Oishi2010}
	H.~Oishi, Y.~Fujioka.
	\newblock Stability and stabilization of aperiodic sampled-data control systems
	using robust linear matrix inequalities.
	\newblock {\em Automatica}, 46:1327--1333, 2010.
	
	\bibitem{Okano2015}
	K.~Okano, M.~Wakaiki, and J.~P. Hespanha.
	\newblock Real-time control under clock offsets between sensors and
	controllers.
	\newblock In {\em Proc. HSCC'15}, 2015.
	
	\bibitem{Olbrot1994}
	A.~W. Olbrot and M.~Nikodem.
	\newblock {Robust stabilization: Some extensions of the gain margin
		maximization problem}.
	\newblock {\em IEEE Trans. Automat. Control}, 39:652--657, 1994.
	
	\bibitem{Rhee2009}
	I.-K. Rhee, J.~Lee, J.~Kim, E.~Serpedin, and Y.-C. Wu.
	\newblock {Clock synchronization in wireless sensor networks: An overview}.
	\newblock {\em Sensors}, 9:56--85, 2009.
	
	\bibitem{Rosenbrock1972}
	H.~H Rosenbrock.
	\newblock {\em Computer-Aided Control System Design}.
	\newblock New York: Academic Press, 1974.
	
	\bibitem{Shi2009}
	H.-B. Shi and L.~Qi.
	\newblock {Static output feedback simultaneous stabilisation via coordinates
		transformations with free variables}.
	\newblock {\em IET Control Theory Appl.}, 3:1051--1058, 2009.
	
	\bibitem{Unal2008Automatica}
	H.~U. \"Unal and A.~\.Iftar.
	\newblock A small gain theorem for systems with non-causal subsystems.
	\newblock {\em Automatica}, 44:2950--2953, 2008.
	
	\bibitem{vidyasagar1985}
	M.~Vidyasagar.
	\newblock {\em {Control System Synthesis: A Factorization Approach}}.
	\newblock Cambridge, MA: MIT Press, 1985, Republished in Morgan \& Claypool,
	2011.
	
	\bibitem{Vidyasagar1982}
	M.~Vidyasagar and N.~Viswanadham.
	\newblock {Algebraic design techniques for reliable stabilization}.
	\newblock {\em IEEE Trans. Automat. Control}, 27:1085--1095, 1982.
	
	\bibitem{WakaikiCDC2015}
	M.~Wakaiki, K.~Okano, and J.~P. Hespanha.
	\newblock Control under clock offsets and actuator saturation.
	\newblock In {\em Proc. 54th IEEE CDC}, 2015.
	
	\bibitem{Wakaiki2015ACC}
	M.~Wakaiki, K.~Okano, and J.~P. Hespanha.
	\newblock Stabilization of networked control systems with clock offsets.
	\newblock In {\em Proc. ACC'15}, 2015.
	
	\bibitem{Wakaiki2016ACC}
	M.~Wakaiki, K.~Okano, and J.~P. Hespanha.
	\newblock {-gain analysis of systems with clock offsets}.
	\newblock In {\em Proc. ACC'16}, 2016.
	
	\bibitem{wakaiki2012}
	M.~Wakaiki, Y.~Yamamoto, and H.~\"Ozbay.
	\newblock {Sensitivity reduction by strongly stabilizing controllers for MIMO
		distributed parameter systems}.
	\newblock {\em IEEE Trans. Automat. Control}, 57:2089--2094, 2012.
	
	\bibitem{Xu2005CDC}
	Y.~Xu and J.~P. Hespanha.
	\newblock Estimation under controlled and uncontrolled communications in
	networked control systems.
	\newblock In {\em Proc. 44th IEEE CDC}, 2005.
	
	\bibitem{zeren2000}
	M.~Zeren and H.~\"Ozbay.
	\newblock {On the strong stabilization and stable -controller
		design problems for MIMO systems}.
	\newblock {\em Automatica}, 36:1675--1684, 2000.
	
	\bibitem{zhou1996}
	K.~Zhou, J.~C. Doyle, and K.~Glover.
	\newblock {\em {Robust and Optimal Control}}.
	\newblock Prentice Hall, 1996.
	
\end{thebibliography}


\end{document}
