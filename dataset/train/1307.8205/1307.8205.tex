We start by introducing \STI\ (Soft Type assignment with Intersection), a type assignment system for $\lam$-calculus assigning to $\lambda$-terms non-idempotent and not associative intersection types. The system assigns types to all and only strongly normalizing terms.

\begin{definition}
\begin{enumerate}[i)]
\item Terms of $\lam$-calculus are defined by the following grammar:
$$\M ::= \x \mid \M\M \mid \lam \x. \M$$
where $\x$ ranges over a countable set ot variables. The symbol $\equiv$ denotes the syntactical equality modulo renaming of bound variables.
\item The reduction relation $\redbeta$ is the contextual closure of the rule $(\lam \x. \M) \N \rightarrow \M[\N/\x]$,
where $\M[\N/\x]$ is the capture-free substitution of $\N$ to all the free occurrences of $\x$ in $\M$.  $\redbetas$ is the reflexive and transitive closure of $\redbeta$.

\item A term $\M$ is an instance of $\N$ if it is obtained from $\N$ by renaming a subset of its free variables with a unique fresh name.

\item The set of \STI\ types is defined as follows:
\begin{align*}
\A ::= & \tvar \mid \sigma \lin \A & \mbox{(linear types)}\\
\sigma ::= & \A \mid \underbrace{ \sigma \wedge ... \wedge \sigma}_n \quad (n > 1) & \mbox{(intersection types)}
\end{align*}
where $\tvar$ ranges over a countable set of type variables. Linear types are ranged over by $\A, \B, \C$, intersection types by 
$\sigma, \tau, \rho$.  The connective $\wedge$ is commutative, but it is not idempotent nor associative.

The number of elements of a type is defined inductively as $l(\A) = 1$, $l(\stra{\sigma}{n})= l(\sigma_{1}) + ... + l(\sigma_{n})$.

\item A context is a finite set of assumptions of the shape $\x: \sigma$, where $\x$ is a variable and $\sigma$ is a type. Variables in a context are all distinct, and
contexts are ranged over by $\Gamma, \Delta$. $dom(\Gamma)$ is the set $\{x \mid x:\sigma \in \Gamma \}$.
The intersection of contexts is given by
$$\Gamma \wedge \Delta = \{ \x : \sigma \mid \x : \sigma \in \Gamma, \x \not\in \dom{\Delta} \} \cup \{ \x : \tau \mid \x : \tau \in \Delta, \x \not\in \dom{\Gamma} \} \cup \{ \x : \sigma \wedge \tau \mid \x:\sigma \in \Gamma, \x:\tau \in \Delta \}$$
while $\Gamma, \Delta$ represents the union of sets $\Gamma$ and $\Delta$, provided that $\Gamma \# \Delta$, i.e. $\dom{\Gamma} \cap \dom{\Delta} = \emptyset$.

\item The system \STI\ proves sequents of the shape $\Gamma \der \M : \sigma$,
where $\Gamma$ is a context, $\M$ is a term of $\lambda$-calculus, and $\sigma$ is a type. The rules are given in Table $\ref{tab:rules}$.

\item Derivations are denoted by $\Pi, \Sigma$. $\Pi \dem \Gamma \der \M: \sigma$ denotes a derivation $\Pi$ with conclusion 
$\Gamma \der \M: \sigma$.

\end{enumerate}
\end{definition}

\begin{table}
\small
\begin{center}
\begin{tabular}{cc}
\hline
\\
$\infer[(Ax)]
     {\x: \A \der \x: \A}
     {}
\qquad
\infer[(w)]
     {\Gamma, \x: \A \der \M : \sigma}
     {\Gamma \der \M : \sigma & \x \notin \dom{\Gamma}}
$
\\
\\
$\infer[(\lin I)]
     {\Gamma \der \lambda \x. \M: \sigma \lin \A}
     {\Gamma, \x: \sigma  \der \M: \A}
\qquad
\infer[(\lin E)]
     {\Gamma, \Delta \der \M\N: \A}
     {\Gamma \der \M: \sigma \lin \A & \Delta \der \N: \sigma & \Gamma \# \Delta}$
     \\
     \\
     $\infer[(\wedge_n)]
     {\bigwedge_{i=1}^n \Gamma_i \der \M: \stra{\sigma}{n}}
     {\Gamma_{1} \der \M: \sigma_{1} & ... & \Gamma_{n} \der \M: \sigma_{n} & n > 1}$
     \\
     \\
     $\infer[(m)]
     {\Gamma, \x:  \stra{\sigma}{n} \der \M [\x/\x_{1}, ...\ , \x/\x_{n}]: \tau}
     {\Gamma, \x_{1} : \sigma_{1}, ..., \x_{n}: \sigma_{n} \der \M: \tau}$
\\
\\
\hline

\end{tabular}
\end{center}
\caption{The type assignment system}
\label{tab:rules}
\normalsize
\end{table} 

Some comments are in order. Since the condition on contexts in rule $(\lin E)$, terms are built in a linear form, and an explicit multiplexor rule is present (rule $(m)$). This allows to control the number of (multiple) contractions, which is responsible for the growth of the reduction time. The counterpart of the contraction on the right side of a derivation is the rule $(\wedge_n)$, which is parametric in $n$. In doing this, we were inspired by the Soft Linear Logic of Lafont.

Let us define {\em constructive} the rules, which contribute in building the subject, i.e., either $(Ax)$, or $(\lin I)$ or $(\lin E)$). 

\begin{definition}[Intersection trees]
Let $(\delta)$ be a (possibly empty) sequence of applications of rules $(w)$ and $(m)$. An {\em intersection tree} is a maximal (sub)proof of the shape defined inductively in the following way:
\begin{itemize}
\item Let the last rule of $\Sigma$ be a constructive rule . Then
\small
$$
\infer=[(\delta)]{\Gamma \der M:\sigma }{\Sigma}
$$
\normalsize
is an empty intersection tree, with conclusion $\Gamma \der M:\sigma$ and one leaf $\Sigma$.
\item 
If $\Sigma_i$ is a (possibly empty) intersection tree ($1\leq i\leq n$), then
\small
$$
\infer=[(\delta)]{\Gamma \der \M':\sigma }{\infer[(\wedge_n)] {\bigwedge_{i=1}^n \Gamma_i \der \M: \stra{\sigma}{n}}
{\Sigma_i\dem\Gamma_{i} \der \M: \sigma_{i} \quad (1 \leq i \leq n)}}
$$
\normalsize
is an intersection tree, with conclusion $\Gamma\der \M':\sigma$, where $\M'$ is an instance of $\M$, $\Gamma$ is a contraction of $\bigwedge_{i=1}^n \Gamma_i $, and its leaves are the leaves of all the $\Sigma_i$.
\end{itemize}
\end{definition}

Since the $(\wedge_n)$ rule is the only rule building an intersection type on the right of the turnstile symbol, it is possible to state the following, which is a key property for proving the normalization bound.


\begin{property}[Subject with intersection type]
\label{prop:stra}
Let $\Pi \dem \Gamma \der \M: \stra{\sigma}{m}$ with $m > 1$. Then $\Pi$ ends with a non empty intersection tree. 
\end{property}

\begin{proof}
By induction on the shape of $\Pi$.
If the last applied rule is $(\wedge_n)$, then the statement is trivially true and $\delta$ is the empty sequence.
Otherwise, the derivation needs to contain at least one application of rule $(\wedge_n)$, with subject $\M'$, such that
$\M$ is an instance of $\M'$. Then this application can be followed only by $\delta$ of rules, which can contain only applications of rule $(w)$ or rule $(m)$.
\end{proof}


The substitution property holds for terms having disjoint free variables sets.

\begin{lemma}[Substitution]
\label{lem:subs}
Let $\Pi \dem \Gamma, \x : \sigma \der \M : \tau$, $\Sigma \dem \Delta \der \N : \sigma$, $\Gamma \# \Delta$
and $\x \not \in dom(\Delta)$.

Then there exists $S(\Sigma, \Pi)$ such that $S(\Sigma, \Pi) \dem \Gamma, \Delta \der \sub{\M}{\N}{\x} : \tau$.
\end{lemma}

\begin{proof} By induction on the shape of $\Pi$. The proof is trivial except for the cases of $(w)$, $(\wedge_n)$ or $(m)$.

If $(w)$ is the last applied rule introducing a variable $\y \not= \x$, then the proof follows by induction. Otherwise, let $\Pi$ be the proof
\small
$$\infer[(w)]{\Pi \dem \Gamma, \x: \A \der \M: \sigma}{\Pi' \dem \Gamma \der \M: \sigma & \x \notin \dom{\Gamma}}$$
\normalsize
and let $\Sigma \dem \Delta \der \N : \A$. 
If $\Delta$ contains only bindings of variables to linear types, then $S(\Sigma, \Pi)$ is the proof
\small
$$\infer=[(w)]{S(\Sigma, \Pi) \dem \Gamma, \Delta \der \M: \sigma}{\Pi' \dem \Gamma \der \M: \sigma}$$
\normalsize
Otherwise, let us assume, without loss of generality, $\Delta = \Delta', \y : \tau$ such that $\A_{1}, ...\ , \A_{n}$ are the elements of $\tau$, and let $\Delta'$ contain only bindings of variables to linear types. Then $S(\Sigma, \Pi)$ is the proof
\small
$$\infer=[(w)]{\infer=[(m)]{S(\Sigma, \Pi) \dem \Gamma, \Delta', \y: \tau \der \M : \sigma}{\Gamma, \Delta', \y_1: \A_1, ...\ , \y_n: \A_n \der \M: \sigma}}{\Pi' \dem \Gamma \der \M: \sigma}$$
\normalsize
where the sequence of applications of rule $(m)$ is constructiong $\tau$.

If the last applied rule is $(\wedge_n)$, with $n > 1$, then $\Pi$ is of the shape 
\small
$$\infer[(\wedge_{n})]{\Gamma, \x : \stra{\sigma}{n} \der \M: \stra{\tau}{n}}{\Pi_{1} \dem \Gamma_{1}, \x: \sigma_{1} \der \M: \tau_{1} & ... & \Pi_{n} \dem \Gamma_{n}, \x: \sigma_{n} \der \M: \tau_{n}}$$ 
\normalsize
By Property \ref{prop:stra}, $\Sigma$ is of the shape
\small
$$\infer[(\wedge_{n})]{\infer=[(\delta)]{\Sigma \dem \Delta \der \N: \stra{\sigma}{n}}{\Delta' \der \N': \stra{\sigma}{n}}}{\Sigma_{1} \dem \Delta_{1} \der \N': \sigma_{1} & ... & \Sigma_n \dem \Delta_n \der \N': \sigma_{n}}$$
\normalsize
where $\delta$ is a sequence of applications of $(w)$ and $(m)$ rules, and $\N$ is an instance of $\N'$.

By inductive hypothesis $S(\Sigma_i, \Pi_i) \dem \Gamma_i, \Delta_i \der \sub{\M}{\N'}{\x} : \tau_i$, since $\Gamma \# \Delta$ implies $\Gamma_i \# \Delta_i$ for all $i$ ,
so $S(\Sigma, \Pi)$ is given by
\small
$$\infer=[(\delta)]{\Gamma, \Delta \der \sub{\M}{\N}{\x}: \stra{\tau}{n}}{\infer[(\wedge_n)]{\Gamma, \Delta' \der \sub{\M}{\N'}{\x}: \stra{\tau}{n}}{S(\Sigma_{1}, \Pi_{1}) \dem \Gamma_{1}, \Delta_{1} \der \sub{\M}{\N'}{\x}: \tau_{1} & ... & S(\Sigma_{n}, \Pi_{n}) \dem \Gamma_{n}, \Delta_{n} \der \sub{\M}{\N'}{\x}: \tau_{n}}}$$
\normalsize
If the last applied rule is $(m)$, then $\Pi$ is of the shape 
\small
$$\infer[(m)]
     {\Gamma, \x: \stra{\sigma}{n} \der \M[\x/\x_{1}, ...\ , \x/\x_{n}]: \tau}
     {\Pi' \dem \Gamma, \x_1: \sigma_{1}, \dots, \x_n: \sigma_{n} \der \M: \tau}$$
\normalsize
Exactly as in the previous case, we can apply Property \ref{prop:stra} to $\Sigma$, thus obtaining $\Sigma_{i} \dem \Delta_{i} \der \N' : \sigma_{i}$, for $1 \leq i \leq n$.
Also, we must rename the variables in $\Delta_{i}$, so that 
we actually get proofs $\Sigma'_{i} \dem \Delta'_{i} \der \N'_{i}: \sigma_{i}$ where $\N'_{i}$ is an instance of $\N'$ and all $\dom{\Delta'_{i}}$ are disjoint from each other; this is not a trouble as we will be able to recover $\Delta'$ and $\N'$ easily by a suitable sequence $\rho$ of applications of $(m)$ rules.

By induction we can now build

$$S(\Sigma'_1, \Pi') \dem \Gamma, \Delta'_1, \x_2: \sigma_2, ... , \x_n: \sigma_n \der \sub{\M}{\N'_{1}}{\x_{1}}: \tau$$
$$S(\Sigma'_2, S(\Sigma'_1, \Pi')) \dem \Gamma, \Delta'_1, \Delta'_2, \x_3: \sigma_3, ... , \x_n: \sigma_n \der \M[\N'_{1}/\x_{1}, ...\ , \N'_{2}/\x_{2}]: \tau$$
$$\vdots$$
$$S(\Sigma'_n, S(\Sigma'_{n-1}, ... S(\Sigma'_1, \Pi') ... )) \dem \Gamma, \Delta'_{1}, ... , \Delta'_{n} \der \M[\N'_{1}/\x_{1}, ...\ , \N'_{n}/\x_{n}]: \tau$$

and by applying sequences $\rho$ and $\delta$ of rule $(m)$, we get the desired proof
\small
$$\infer=[(\delta)]{S(\Sigma, \Pi) \dem \Gamma, \Delta \der \M[\N/\x_{1}, ...\ , \N/\x_{n}]: \tau}{
\infer=[(\rho)]{\Gamma, \Delta' \der \M[\N'/\x_{1}, ...\ , \N'/\x_{n}]: \tau}
{\Gamma, \Delta'_{1}, ... , \Delta'_{n} \der \M[\N'_{1}/\x_{1}, ...\ , \N'_{n}/\x_{n}]: \tau}}$$
\normalsize
\end{proof}

The substitution property is sufficient for proving the subject reduction property, but we need to take into account that one step of $\beta$-reduction on the subject can be matched by a \textsl{set} of $n \geq 1$ parallel simplification steps in the underlying derivation, corresponding to reducing virtual copies of the same redex having different types.


\begin{property}[Subject reduction]
\label{prop:subjred}
$\Pi \dem \Gamma \der \M :\sigma$ and $\M \redbeta \M'$ implies $\Pi' \dem \Gamma \der \M': \sigma$.
\end{property}

\begin{proof}
$\M \redbeta \M'$ means $\M=C[(\lambda \x.\Q) \N]$ and $\M'=C[\Q[\N/\x]]$, for some context $C[.]$. The proof is by induction on $C[.]$. 
Let us consider just the base case in which $C[.]=[.]$, i.e., $\M=(\lambda \x.\Q)\N$. Then the most difficult case is when $\Pi$ ends by a non empty intersection tree. Note that the shape of $\M$ implies each leaf $\Pi_i$ of the intersection tree be of the shape:  
\small
$$\infer[(\lin E)]{\Gamma_i, \Delta_i \der (\lambda \x. \Q_i) \N_i: \A_i}{\infer=[(\delta_i)]{\Gamma_i \der \lambda \x. \Q_i : \sigma_i \lin \A}{\infer[(\lin I)]{\Gamma'_i \der \lambda \x. \Q'_i : \sigma_i \lin \A_i}{\Sigma'_i\dem \Gamma'_i, \x: \sigma_i \der \Q'_i : \A_i}} & \Sigma''_i \dem \Delta_i \der \N_i : \sigma_i & \Gamma_i \# \Delta_i}$$
\normalsize
where $1\leq i \leq n$, for some $n >1$, $(\lambda \x.\Q)\N$ is an instance of $(\lambda \x. \Q_i) \N_i$, and $\delta_i$ is a (possibly empty) sequence of applications of $(w)$ and $(m)$ rules.  Since all $(m)$ rules in $\delta_i$ deal with variables in $\dom{\Gamma}$, sequence $\delta_i$ can be delayed to obtain the proof
\small
$$\infer=[(\delta_i)]{\Gamma_i, \Delta_i \der  (\lambda \x. \Q_i) \N_i: \A_i}{\infer[(\lin E)]{\Gamma'_i, \Delta_i \der (\lambda \x. \Q'_i) \N_i: \A_i}{\infer[(\lin I)]{\Gamma'_i \der \lambda \x. \Q'_i : \sigma_i \lin \A_i}{\Sigma'_i \dem \Gamma'_i, \x: \sigma_i \der \Q'_i : \A} & \Sigma_i'' \dem \Delta_i \der \N_i : \sigma_i}}$$
\normalsize
By Lemma \ref{lem:subs}, there are proofs $S(\Sigma''_i,\Sigma'_i) \dem \Gamma_i, \Delta_i \der
\Q_i[\N_i/x]$, and then the result is obtained by replacing the leafs $\Pi_i$ of the intersection tree by
$S(\Sigma''_i,\Sigma'_i)$ ($1\leq i \leq n$). 



\end{proof}





Moreover the system is strongly normalizing. Formally:

\begin{property}[Strong normalization]
\label{prop:sn}
$\Pi \dem \Gamma \der \M :\sigma$ if and only if $\M$ is strongly normalizing.
\end{property}

For the right implication, the proof is obtained in the next section by observing that the measure of $\Pi$ decreases with each reduction step, and this does not depend on any particular strategy. As for the left implication, the proof can be obtained by adapting Neergaard's proof \cite{Neergaard05} to system \STI. In fact, Neergaard proved the strong normalization property for a system with rigid intersection types, i.e. intersection without commutativity, associativity nor idempotency.

