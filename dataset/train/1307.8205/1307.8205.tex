We start by introducing \STI\ (Soft Type assignment with Intersection), a type assignment system for -calculus assigning to -terms non-idempotent and not associative intersection types. The system assigns types to all and only strongly normalizing terms.

\begin{definition}
\begin{enumerate}[i)]
\item Terms of -calculus are defined by the following grammar:

where  ranges over a countable set ot variables. The symbol  denotes the syntactical equality modulo renaming of bound variables.
\item The reduction relation  is the contextual closure of the rule ,
where  is the capture-free substitution of  to all the free occurrences of  in .   is the reflexive and transitive closure of .

\item A term  is an instance of  if it is obtained from  by renaming a subset of its free variables with a unique fresh name.

\item The set of \STI\ types is defined as follows:

where  ranges over a countable set of type variables. Linear types are ranged over by , intersection types by 
.  The connective  is commutative, but it is not idempotent nor associative.

The number of elements of a type is defined inductively as , .

\item A context is a finite set of assumptions of the shape , where  is a variable and  is a type. Variables in a context are all distinct, and
contexts are ranged over by .  is the set .
The intersection of contexts is given by

while  represents the union of sets  and , provided that , i.e. .

\item The system \STI\ proves sequents of the shape ,
where  is a context,  is a term of -calculus, and  is a type. The rules are given in Table .

\item Derivations are denoted by .  denotes a derivation  with conclusion 
.

\end{enumerate}
\end{definition}

\begin{table}
\small
\begin{center}
\begin{tabular}{cc}
\hline
\\

\\
\\

     \\
     \\
     
     \\
     \\
     
\\
\\
\hline

\end{tabular}
\end{center}
\caption{The type assignment system}
\label{tab:rules}
\normalsize
\end{table} 

Some comments are in order. Since the condition on contexts in rule , terms are built in a linear form, and an explicit multiplexor rule is present (rule ). This allows to control the number of (multiple) contractions, which is responsible for the growth of the reduction time. The counterpart of the contraction on the right side of a derivation is the rule , which is parametric in . In doing this, we were inspired by the Soft Linear Logic of Lafont.

Let us define {\em constructive} the rules, which contribute in building the subject, i.e., either , or  or ). 

\begin{definition}[Intersection trees]
Let  be a (possibly empty) sequence of applications of rules  and . An {\em intersection tree} is a maximal (sub)proof of the shape defined inductively in the following way:
\begin{itemize}
\item Let the last rule of  be a constructive rule . Then
\small

\normalsize
is an empty intersection tree, with conclusion  and one leaf .
\item 
If  is a (possibly empty) intersection tree (), then
\small

\normalsize
is an intersection tree, with conclusion , where  is an instance of ,  is a contraction of , and its leaves are the leaves of all the .
\end{itemize}
\end{definition}

Since the  rule is the only rule building an intersection type on the right of the turnstile symbol, it is possible to state the following, which is a key property for proving the normalization bound.


\begin{property}[Subject with intersection type]
\label{prop:stra}
Let  with . Then  ends with a non empty intersection tree. 
\end{property}

\begin{proof}
By induction on the shape of .
If the last applied rule is , then the statement is trivially true and  is the empty sequence.
Otherwise, the derivation needs to contain at least one application of rule , with subject , such that
 is an instance of . Then this application can be followed only by  of rules, which can contain only applications of rule  or rule .
\end{proof}


The substitution property holds for terms having disjoint free variables sets.

\begin{lemma}[Substitution]
\label{lem:subs}
Let , , 
and .

Then there exists  such that .
\end{lemma}

\begin{proof} By induction on the shape of . The proof is trivial except for the cases of ,  or .

If  is the last applied rule introducing a variable , then the proof follows by induction. Otherwise, let  be the proof
\small

\normalsize
and let . 
If  contains only bindings of variables to linear types, then  is the proof
\small

\normalsize
Otherwise, let us assume, without loss of generality,  such that  are the elements of , and let  contain only bindings of variables to linear types. Then  is the proof
\small

\normalsize
where the sequence of applications of rule  is constructiong .

If the last applied rule is , with , then  is of the shape 
\small
 
\normalsize
By Property \ref{prop:stra},  is of the shape
\small

\normalsize
where  is a sequence of applications of  and  rules, and  is an instance of .

By inductive hypothesis , since  implies  for all  ,
so  is given by
\small

\normalsize
If the last applied rule is , then  is of the shape 
\small

\normalsize
Exactly as in the previous case, we can apply Property \ref{prop:stra} to , thus obtaining , for .
Also, we must rename the variables in , so that 
we actually get proofs  where  is an instance of  and all  are disjoint from each other; this is not a trouble as we will be able to recover  and  easily by a suitable sequence  of applications of  rules.

By induction we can now build






and by applying sequences  and  of rule , we get the desired proof
\small

\normalsize
\end{proof}

The substitution property is sufficient for proving the subject reduction property, but we need to take into account that one step of -reduction on the subject can be matched by a \textsl{set} of  parallel simplification steps in the underlying derivation, corresponding to reducing virtual copies of the same redex having different types.


\begin{property}[Subject reduction]
\label{prop:subjred}
 and  implies .
\end{property}

\begin{proof}
 means  and , for some context . The proof is by induction on . 
Let us consider just the base case in which , i.e., . Then the most difficult case is when  ends by a non empty intersection tree. Note that the shape of  implies each leaf  of the intersection tree be of the shape:  
\small

\normalsize
where , for some ,  is an instance of , and  is a (possibly empty) sequence of applications of  and  rules.  Since all  rules in  deal with variables in , sequence  can be delayed to obtain the proof
\small

\normalsize
By Lemma \ref{lem:subs}, there are proofs , and then the result is obtained by replacing the leafs  of the intersection tree by
 (). 



\end{proof}





Moreover the system is strongly normalizing. Formally:

\begin{property}[Strong normalization]
\label{prop:sn}
 if and only if  is strongly normalizing.
\end{property}

For the right implication, the proof is obtained in the next section by observing that the measure of  decreases with each reduction step, and this does not depend on any particular strategy. As for the left implication, the proof can be obtained by adapting Neergaard's proof \cite{Neergaard05} to system \STI. In fact, Neergaard proved the strong normalization property for a system with rigid intersection types, i.e. intersection without commutativity, associativity nor idempotency.

