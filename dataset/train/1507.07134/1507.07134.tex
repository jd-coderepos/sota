


















\documentclass[twocolumn]{autart}    

\usepackage{graphicx}          



\usepackage[T1]{fontenc}
\usepackage[utf8]{inputenc}
\usepackage{ctable}
\usepackage{tikz}
\usetikzlibrary{arrows,automata}
\usepackage{graphicx}
\usepackage{subcaption}
\usepackage{amsfonts}
\usepackage{amssymb}
\usepackage{amsmath}
\usepackage{tabulary}
\usepackage{multirow}

\usepackage[noprefix]{nomencl}
\makenomenclature

\usepackage{url}
\usepackage{acronym}
\usepackage{enumerate}
\usepackage{breakurl} 
\newtheorem{theorem}{\bf Theorem}[section]
\usepackage[noend]{algpseudocode}
\usepackage{soul}
\usepackage{algorithm}
\newcommand{\algrule}[1][.2pt]{\par\vskip.5\baselineskip\hrule height #1\par\vskip.5\baselineskip}
\newtheorem{lemma}[theorem]{\bf Lemma}
\newtheorem{proposition}[theorem]{\bf Proposition}
\newtheorem{corollary}[theorem]{\bf Corollary}
\newtheorem{definition}{\bf Definition}[section]

\newcommand{\boxthm}[2]{\vspace{2ex} \noindent \fbox{ \begin{minipage}{.95\columnwidth} \centering \begin{#1} #2 \end{#1} \end{minipage}}\ 
 h_{i,t+1} &= \frac{1}{2} \big[ h_{i-1,t} + h_{i+1,t} + b \left( q_{i-1,t} - q_{i+1,t} \right) \nonumber \\
 &\qquad {} + r \left( q_{i+1,t} |q_{i+1,t}| - q_{i-1,t} |q_{i-1,t}| \right) \big] \label{eq:3a}\\
q_{i,t+1} &= \frac{1}{b} \big[ h_{i,t+1} -h_{i+1,t} + q_{i+1,t} - r |q_{i+1,t}| \big],  \label{eq:3b}
 \label{eq:3}
y_{S_i}(t,\ell_j) =
\left\{
\begin{array}{lcl}
1 & \;\;\;\text{if}\;\; \xi \left( p_{i,t} - \hat{p}_{i,t}\right) \geq \varepsilon  \\
0 & \;\;\;\text{otherwise}\\
\end{array}
\right.

\mathbf{y}_{S_i}(\ell_j) = 
\left\{
\begin{array}{lcl}
 1 & \;\;\;\text{if}\;\;y_{S_i}(t,\ell_j) = 1,\;\;\text{for any }t>0 \\0 & \;\;\;\text{otherwise}\\
\end{array}
\right.

\label{eq:sensing_model}
\mathcal{M}\left(\mathcal{L},{\mathcal{S}}\right) =
\left[
\begin{array}{c}
\mathbf{y}_{{\mathcal{S}}}(\ell_1)\\
\mathbf{y}_{\mathcal{S}}(\ell_2)\\
\vdots\\
\mathbf{y}_{\mathcal{S}}(\ell_n)\\
\end{array}
\right]
\mathcal{M}( \mathcal{L},\mathcal{S}) = 
\bordermatrix{
& S_1& S_2& S_3& S_4& S_5& S_6& S_7& S_8\cr
\ell_1&1 & 1 & 1 & 0 & 1 & 0 & 0 & 0 \cr 
\ell_2&1 & 1 & 1 & 1 & 0 & 1 & 0 & 0 \cr
\ell_3&1 & 1 & 0 & 1 & 1 & 0 & 0 & 1 \cr 
\ell_4&1 & 0 & 1 & 1 & 1 & 1 & 1 & 0 \cr
\ell_5&1 & 0 & 1 & 1 & 0 & 1 & 1 & 0 \cr
\ell_6&0 & 1 & 1 & 1 & 1 & 0 & 1 & 1 \cr 
\ell_7&0 & 0 & 1 & 1 & 1 & 1 & 1 & 1 \cr
\ell_8&0 & 1 & 0 & 1 & 1 & 0 & 1 & 1 \cr
\ell_9&0 & 0 & 1 & 1 & 0 & 1 & 1 & 1 \cr
\ell_{10}& 0 & 0 & 0 & 1 & 1 & 1 & 1 & 1}.
\label{eq:6}
f_D(\mathcal{C}_S) = \left\lvert \bigcup\limits_{C_i\in\mathcal{C}_S}C_i\right\rvert.

\label{eq:submodular}
f\left(\mathcal{C}_s\cup\{C_i\}\right) - f(\mathcal{C}_s) \ge f\left(\mathcal{C}_r\cup\{C_i\}\right) - f(\mathcal{C}_r)
 \label{eq:objidnt}
f_I(\mathcal{C}_S) =f_D(\mathcal{C}_S^t),
\beta(X) =  \text{set of all 2-element subsets of}\; X,
\label{eq:beta}
\alpha(Y,\beta(X)) = \{a\in\beta(X):\;\lvert Y \cap a \rvert = 1\}.

y_i = \sum\limits_{C_u\in C^{\ast}}\left\vert \alpha(Y_i, G_{u})\right\vert \vspace{-0.35cm}

\label{eq:one}
\sum\limits_{i}{\dbinom{k_i}{2}}\le \frac{k}{n}\dbinom{n}{2}

\begin{split}
\sum\limits_{i}{\dbinom{k_i}{2}}  & = \frac{1}{2}\left(\sum\limits_i k_i^2 - \sum\limits_i k_i\right) \le \frac{1}{2}\left(k\sum\limits_i k_i -n\right) \\
& = \frac{1}{2}\left(kn - n\right)\le \frac{1}{2}\left(kn-k\right) =\frac{k}{n}\dbinom{n}{2}.\qed
\end{split}

\frac{\partial h}{\partial t} +  \frac{a^2}{gA}\frac{\partial q}{\partial x} = 0

\frac{1}{gA}\frac{\partial q}{\partial t} + \frac{\partial h}{\partial x} + \frac{cq|q|}{2gDA^2} = 0
 \label{eq:a3}
 
 \label{eq:a4}
 

 

b = \frac{a}{gA}

r = \frac{c \Delta x}{2gDA^2}
 
 h_{i,t+1} &= \frac{1}{2} \big[ h_{i-1,t} + h_{i+1,t} + b \left( q_{i-1,t} - q_{i+1,t} \right) \nonumber \\
 &\qquad {} + r \left( q_{i+1,t} |q_{i+1,t}| - q_{i-1,t} |q_{i-1,t}| \right) \big] \label{eq:3a}\\
q_{i,t+1} &= \frac{1}{b} \big[ h_{i,t+1} -h_{i+1,t} + q_{i+1,t} - r |q_{i+1,t}| \big]  \label{eq:3b}
 \label{eq:5a}
h_{i,t+1}  + \frac{b}{2}C_dA_{d,t+1} \sqrt{2gh_{i,t+1}}
- \frac{C_M +C_P }{2} = 0
f_D\left(\mathcal{C}_s\cup\{C_i\}\right) - f_D(\mathcal{C}_s) \ge f_D\left(\mathcal{C}_r\cup\{C_i\}\right) - f_D(\mathcal{C}_r)
\label{eq:p1}
f_D(\mathcal{C}_s\cup\{C_i\}) = f_D(\mathcal{C}_s\cup\{C_i'\}) = f_D(\mathcal{C}_s) + f_D(\{C_i'\})

\label{eq:p2}
f_D(\mathcal{C}_r \cup\{C_i\}) = f_D(\mathcal{C}_r\cup\{C_i'\}) = f_D(\mathcal{C}_s\cup\{C_i'\}) + f_D(\{\lambda\}),

\label{eq:p3}
f_D(\mathcal{C}_r) = f_D(\mathcal{C}_s) + f_D(\{\lambda\}) +f_D(\{\mu\}).

f_D(\mathcal{C}_s\cup\{C_i\}) -  f_D(\mathcal{C}_s) -f_D(\{\mu\}) = f_D(\mathcal{C}_r \cup\{C_i\}) - f_D(\mathcal{C}_r) 
\mathcal{M}( \mathcal{L},\mathcal{S}) = 
\bordermatrix{
& S_1& S_2& S_3& S_4& S_5& S_6& S_7& S_8\cr
\ell_1&1 & 1 & 1 & 0 & 1 & 0 & 0 & 0 \cr 
\ell_2&1 & 1 & 1 & 1 & 0 & 1 & 0 & 0 \cr
\ell_3&1 & 1 & 0 & 1 & 1 & 0 & 0 & 1 \cr 
\ell_4&1 & 0 & 1 & 1 & 1 & 1 & 1 & 0 \cr
\ell_5&1 & 0 & 1 & 1 & 0 & 1 & 1 & 0 \cr
\ell_6&0 & 1 & 1 & 1 & 1 & 0 & 1 & 1 \cr 
\ell_7&0 & 0 & 1 & 1 & 1 & 1 & 1 & 1 \cr
\ell_8&0 & 1 & 0 & 1 & 1 & 0 & 1 & 1 \cr
\ell_9&0 & 0 & 1 & 1 & 0 & 1 & 1 & 1 \cr
\ell_{10}& 0 & 0 & 0 & 1 & 1 & 1 & 1 & 1}

\begin{split}
G_1 &\gets G_1 \setminus   \left\{\{1,4\},\{1,5\},\{2,4\},\{2,5\},\{3,4\},\{3,5\} \right\}\\
&= \{\{1,2\},\{1,3\},\{2,3\},\{4,5\}\}.
\end{split}

At the same time, a new set  is created, which contains the set of pair-wise events in . Since , we get .

\item[Next iteration.] We continue with the same steps until no improvement can be made, i.e.  for each sensor. At the end of the algorithm, sensors in the set  are included in the test cover.
\end{description}

For this example, a complete account of the values of variables in each iteration of the algorithm is given in Table 1.

\section{Evaluation on real networks (cont.)}
For all networks \cite{datab,doi:10.1061/(ASCE)WR.1943-5452}, the layouts and the simulation plots illustrating the four performance metrics are shown in Table \ref{tab:10}. For the ease of presentation, the worst localization set size, , is normalized by dividing it by the number of pipes.


\centering
\begin{table*}[ht]
\caption{Evaluation on real netowrks}
\centering
\begin{tabular}{|c|c|}
\hline
\includegraphics[trim = 10mm 60mm 10mm 60mm, clip, scale = 0.25]{Fig_1_net.pdf}&\includegraphics[trim = 10mm 60mm 10mm 60mm, clip, scale = 0.25]{Fig_1.pdf}\\
 \hline
\includegraphics[trim = 10mm 60mm 10mm 60mm, clip, scale = 0.25]{Fig_2_net.pdf}&\includegraphics[trim = 10mm 60mm 10mm 60mm, clip, scale = 0.25]{Fig_2.pdf}\\
\hline
\includegraphics[trim = 10mm 60mm 10mm 60mm, clip, scale = 0.25]{Fig_3_net.pdf}&\includegraphics[trim = 10mm 60mm 10mm 60mm, clip, scale = 0.25]{Fig_3.pdf}\\
 \hline
\includegraphics[trim = 10mm 60mm 10mm 60mm, clip, scale = 0.25]{Fig_4_net.pdf}&\includegraphics[trim = 10mm 60mm 10mm 60mm, clip, scale = 0.25]{Fig_4.pdf}\\
\hline
\includegraphics[trim = 10mm 60mm 10mm 60mm, clip, scale = 0.25]{Fig_5_net.pdf}&\includegraphics[trim = 10mm 60mm 10mm 60mm, clip, scale = 0.25]{Fig_5.pdf}\\
\hline

\end{tabular}
\label{tab:10}
\end{table*}

\begin{table*}[ht]
\caption*{Evaluation on real netowrks}
\centering
\begin{tabular}{|c|c|}
\hline
\includegraphics[trim = 10mm 60mm 10mm 60mm, clip, scale = 0.25]{Fig_6_net.pdf}&\includegraphics[trim = 10mm 60mm 10mm 60mm, clip, scale = 0.25]{Fig_6.pdf}\\
\hline
\includegraphics[trim = 10mm 60mm 10mm 60mm, clip, scale = 0.25]{Fig_7_net.pdf}&\includegraphics[trim = 10mm 60mm 10mm 60mm, clip, scale = 0.25]{Fig_7.pdf}\\
 \hline
\includegraphics[trim = 10mm 60mm 10mm 60mm, clip, scale = 0.25]{Fig_8_net.pdf}&\includegraphics[trim = 10mm 60mm 10mm 60mm, clip, scale = 0.25]{Fig_8.pdf}\\
\hline
\includegraphics[trim = 10mm 60mm 10mm 60mm, clip, scale = 0.25]{Fig_9_net.pdf}&\includegraphics[trim = 10mm 60mm 10mm 60mm, clip, scale = 0.25]{Fig_9.pdf}\\
 \hline
\includegraphics[trim = 10mm 60mm 10mm 60mm, clip, scale = 0.25]{Fig_10_net.pdf}&\includegraphics[trim = 10mm 60mm 10mm 60mm, clip, scale = 0.25]{Fig_10.pdf}\\
\hline
\end{tabular}
\label{tab:gt}
\end{table*}

\newpage
\begin{table*}[tp]
\caption*{Evaluation on real netowrks}
\centering
\begin{tabular}{|c|c|}
 \hline
\includegraphics[trim = 10mm 60mm 10mm 60mm, clip, scale = 0.25]{Fig_11_net.pdf}&\includegraphics[trim = 10mm 60mm 10mm 60mm, clip, scale = 0.25]{Fig_11.pdf}\\
\hline
\includegraphics[trim = 10mm 60mm 10mm 60mm, clip, scale = 0.25]{Fig_12_net.pdf}&\includegraphics[trim = 10mm 60mm 10mm 60mm, clip, scale = 0.25]{Fig_12.pdf}\\
\hline
\end{tabular}
\label{tab:gt}
\end{table*}

\begin{thebibliography}{1}

\bibitem{datab}
Centre of Water Systems University of EXETER. 
http://emps.exeter.ac.uk/engineering/research/cws/
downloads/benchmarks/. Accessed: 2015-04-14.


\bibitem{visent}
Visenti. http://www.visenti.com/.

\bibitem{doi:10.1061/(ASCE)WR.1943-5452}
M. D. Jolly, A. D. Lothes, and L. Ormsbee. Research database of water distribution system models. \emph{Journal of Water Resources Planning and Management}, 140(4):410--416, 2014.

\bibitem{Dal}
D. Misiunas. \emph{Failure monitoring and asset condition assessment in water supply systems.} PhD thesis, LUND University, Sweden, 2005.

\bibitem{todi}
E. Todini and L. Rossman. Unified framework for deriving simultaneous equation algorithms for water distribution networks. \emph{Journal of Hydraulic Engineering}, 139(5):511--526, 2013.

\bibitem{wylie1993}
E. B. Wylie, V. L. Streeter, and L. Suo, \emph{Fluid transients in systems.} Prentice Hall, 1993.

\bibitem{69012}
T. T. Zan, H. B. Lim, K. Wong, A. J. Whittle, and B. Lee. Event detection and localization in urban water distribution network. \emph{IEEE Sensors Journal}, 14(12):4134--4142, 2014.

 
 \end{thebibliography}

\end{document}