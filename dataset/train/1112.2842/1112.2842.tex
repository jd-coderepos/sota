\documentclass[journal]{IEEEtran}
\usepackage{graphicx}
\usepackage{subfigure}
\usepackage{amsmath}
\usepackage{theorem}
\usepackage{cite}
\usepackage{color}
\newtheorem{Theorem}{Theorem}
\newtheorem{lemma}{\textbf{Lemma}}
\newtheorem{Corollary}{Corollary}
\bibliographystyle{IEEEtran}
\newtheorem{DD}{Definition}
\usepackage[]{algorithm2e}
\bibliographystyle{IEEEtran}



\begin{document}
\title{Joint Rate Selection and Wireless Network Coding for Time Critical Applications}
\author{
\small Xiumin Wang, Chau Yuen, Yinlong Xu\\
 Singapore University of Technology and Design, Singapore\\
 School of Computer Science, University of Science and Technology of China, China\\
\small Email: wangxiumin@sutd.edu.sg, yuenchau@sutd.edu.sg, ylxu@ustc.edu.cn
\thanks{This research is supported by the International Design Center at Singapore University of Technology and Design, Singapore (Grant No. IDG31100102 \& IDD11100101).}}
\maketitle

\begin{abstract}
In this paper, we  dynamically select the transmission rate and design wireless network coding to improve the quality of services such as delay for time critical applications.
With low transmission rate, and hence longer transmission range, more packets may be encoded together, which increases the coding opportunity. However, low transmission rate may incur extra transmission delay, which is intolerable for time critical applications.
We design a novel joint rate selection and wireless network coding (RSNC) scheme with delay constraint, so as to minimize the total number of packets that miss their deadlines at the destination nodes. We prove that the proposed problem is NP-hard, and propose a novel graph model and transmission metric which consider
both the heterogenous transmission rates and the packet deadline constraints during the graph construction. Using the graph model, we mathematically formulate the problem and design an efficient
algorithm to determine the transmission rate and coding strategy for each transmission. Finally, simulation results demonstrate the superiority of the RSNC scheme.
\end{abstract}

\vspace{-0.1in}
\section{Introduction}
With the increase in both wireless channel bandwidth and the computational capability of wireless devices, wireless networks
now can be used to support time critical applications such as video streaming or interactive gaming. Such time critical applications require the data content to reach the destination node(s) in a timely fashion, i.e., a delay deadline is imposed on packet reception, beyond which the reception becomes useless (or invalid) \cite{XTL2006Time-critical14}.



Recently, network coding becomes a promising approach to improve wireless network performance \cite{Sagduyu2007,ACL+2000Network1216,KRH+2008XORs510}. Specifically, the work in \cite{KRH+2008XORs510} proposed the first network coding based packet forwarding architecture, named {\em COPE}, to improve the throughput of wireless networks. With COPE, each node opportunistically overhears some of the packets transmitted by its neighbors, which are not intended to itself. The relay node can then intelligently XOR multiple packets and forward it to multiple next hops with only one transmission, which results in a significant throughput improvement.


In most recent works for wireless network coding, network nodes always transmit packets at a fixed rate. However, most wireless systems are now capable of performing adaptive modulation to vary the link transmission rate in response to the
signal to interference plus noise at the receivers. Transmission rate diversity exhibits a rate-range tradeoff: the higher the transmission rate, the shorter the transmission range for a given transmission power \cite{KV2009Is646}. To aid overhearing, one may use the lowest transmission rate, so as to successfully deliver packet to more receivers/overhearing nodes.
Although this may increase the coding opportunity, it may not yield good performance, especially for time critical applications, as the arrival times of packets may be delayed.

In the literature, only a few works studied the relationships between adapting the transmission rate and the network coding gain \cite{KV2009Is646,CJH2010Joint2444,Kim2010,Ni2008}. The work in \cite{KV2009Is646} showed that compared with pure network coding scheme, joint rate adaptation and network coding is more effective in throughput performance. They also proposed a joint rate selection and coding scheme to minimize the sum of the uplink and the downlink costs in star network topology. The work in \cite{CJH2010Joint2444} mathematically formulated the optimal packet coding and rate selection problem as an integer programming problem, and proposed an efficient heuristic algorithm to jointly find a good combination of coding solution and the transmission rate. However, there are only a few works considered the delay guarantee of packet receptions, which is especially important for time critical applications. So far, only \cite{ZX2010Broadcast6} considered the delay constraint of packet reception, and proposed a coding scheme to minimize the number of packets that miss their deadlines. However, they assume that the transmission rates on all the links are fixed and the same.

In this paper, by considering the impact of both transmission rate and network coding on the packet reception delay, we design a joint rate selection and network coding (RSNC) scheme for wireless time critical applications, so as to minimize the total number of packets that will miss their deadlines at the destination nodes. The main contributions of our paper can be concluded as follows.
\begin{itemize}
\item We propose a novel graph model, which considers both the heterogenous transmission rates and the deadlines of the packet receptions during the graph construction. Based on the graph model, we mathematically formulate the problem of minimizing the total number of packets that miss their deadlines by joint rate selection and network coding, as an integer programming problem.
\item We propose a metric to determine the coded packet and the transmission rate for each packet transmission. By considering the impact of the transmission rate on both delay and network coding gain, we also design an efficient algorithm to optimize the proposed metric.



\item We compare the performance of the proposed RSNC scheme with some existing algorithms. Simulation results show that the proposed scheme can significantly reduce the packet deadline miss ratio.
\end{itemize}

The rest of the paper is organized as follows. We define our problem in Sec.~\ref{Sec.formulation}. Sec.~\ref{Sec.solution} gives the graph model and problem formulation. The algorithm design for each transmission is given in Sec.~\ref{Sec.algorithm}. We show the simulation results in Sec.~\ref{Sec.simulation}, and conclude the paper in Sec.~\ref{Sec.conclusion}.

\begin{figure}[t]
\begin{center}\vspace{-0.04in}
\includegraphics[height=27mm,width=80mm]{rsnc}\vspace{-0.1in}
\caption{Motivation illustration}\vspace{-0.08in} \label{Fig.rsnc}
\end{center}\vspace{-0.15in}
\end{figure}

\vspace{-0.1in}
\section{Problem Definition}\label{Sec.formulation}
In this section, we first illustrate the motivation of our problem. We then give the problem description and its complexity.
\vspace{-0.24in}
\subsection{Motivation Illustration}\label{Sec.formulation.motivation}\vspace{-0.02in}
We now give an example to show how joint rate selection and network coding affect the time critical applications.

Take Fig.~\ref{Fig.rsnc} as an example, where source/transmitting node node  needs to transmit packet  to node 
respectively. Fig.~\ref{Fig.rsnc}(a) gives the set of overheard packets  at destination . Suppose that the size of each packet is , and the maximum transmission rates from  to  are  and , respectively.
Fig.~\ref{Fig.rsnc}(b) shows the reception deadline of each ``wanted" packet at its destination. For the current transmission, according to the work in \cite{KRH+2008XORs510,ZX2010Broadcast6},  will send the encoded packet , as the most number of destinations can decode it. However, there is a problem for selecting the transmission rate at . If  is selected,  can not successfully receive the packet, as the maximum transmission rates from  to them are both . If  is selected, although all of the three receivers  can receive and decode one ``wanted" packet,   will miss its deadline at , as its arrival time is .

As an alternative, we may choose to first send packet  with transmission rate , where destinations  will obtain a ``wanted" packet in . After this transmission, the encoded packet  can be sent with transmission rate , where destination  and  will obtain a ``wanted" packet after  (including the waiting time of the first transmission). Obviously, the latter solution is better than the first one, as no packet will miss their deadline.

\vspace{-0.15in}
\subsection{Problem Description}\vspace{-0.02in}
In this paper, we consider the application of network coding in wireless networks. Each network node knows the overheard/routed packets that its neighbors have such that it can perform network coding operations. Such information can be achieved by using {\em reception reports}, as introduced in \cite{KRH+2008XORs510}. We also assume that the forwarding/relaying node knows the deadlines of the packet receptions at its receivers. Specifically, we consider the transmission scheme within a single hop since multi hop can be regarded as multiple single hops. As in COPE \cite{KRH+2008XORs510}, only XORs coding is performed at the node in our work.

Without loss of generality, let  be the current source node, and  be the set of packets required to be transmitted from . Suppose that  is the set of 's neighbors which requires packets in  sent from node . Let  be the set of ``wanted" packets at , and  be the set of overheard packets at , where . For each , let  be the reception deadline of packet  at node . We also assume that  is the maximum transmission rate on link , and only if the transmission rate from  to  is less than , the packet sent from  can be successfully received by  \cite{KV2009Is646}. We also assume that the size of each packet is .

Our problem is that given the set of overheard packets at each node , , the set of packets required by , , the deadline of required packet  at node , , and the maximum transmission rate  on the link from  to , design the encoding strategy of the packets and select the transmission rate for each propagation, such that the total number of packets that miss their deadlines at each destination is minimized.

Let  be  if packet  misses its deadline at , otherwise, let it be , where . Thus, our objective is to minimize
 In this paper, we refer such a problem of joint Rate Selection and Network Coding (RSNC) for time critical applications as RSNC problem.


\vspace{-0.1in}
\subsection{Problem Complexity}
\begin{lemma}
The RSNC problem is NP-hard.
\end{lemma}\vspace{-0.1in}
\begin{proof}
We can consider a special case of the RSNC problem:  and the maximum transmission rates on all the links are the same. Then, this special case is equivalent to finding a maximum weight clique problem as in \cite{ZX2010Broadcast6}, which is known as an NP-hard problem. Thus, the RSNC problem is also NP-hard.
\end{proof}

\vspace{-0.1in}
\section{Graph Model and RSNC Formulation}\label{Sec.solution}
\subsection{Graph Model}\label{graph.model}
Although the graph model in \cite{ZX2010Broadcast6} works well for the case where the transmission rates on all the links are the same and fixed, it can not be used directly for our RSNC problem. Here, we construct a novel graph model , which considers both the transmission rates and the packet reception deadlines.

We define  as the minimum transmission rate that can be used to meet the deadline of  at .
We add a vertex  in , only if the following two conditions can be met.

(1) ;

(2) .

Note that, if , packet  will definitely miss its deadline at .
Thus, conditions (1) and (2) ensure that we add a vertex  in  only if the ``wanted" packet  will not miss its deadline at .
That is, .

Then, for any two different vertices , there is a link  if all the following conditions can be satisfied.

(a) ;

(b)  or  and ;

(c)  and .

For any clique  in , let . According to the work in \cite{ZX2010Broadcast6}, if node  successfully receives the encode packet , where ,  can decode a ``wanted" packet , where .

Next, we will use an example to show the novelty of our graph model as compared to others in the literature, e.g., \cite{ZX2010Broadcast6}.

Still take Fig.~\ref{Fig.rsnc} as an example. The graph constructed by \cite{ZX2010Broadcast6} is shown in Fig.~\ref{Fig.example} (a). According to \cite{ZX2010Broadcast6}, any clique in the graph represents a feasible encoded packet. Thus,  can be sent and its intended next hops are , because ,, forms a clique. As described in Sec.~\ref{Sec.formulation.motivation}, it is not a good choice.
However, with our graph model shown in Fig.~\ref{Fig.example}(b),  will not be encoded as vertices
,, do not form a clique in the graph. In addition, for the current transmission, the encoded packet derived from any clique in the graph can be sent without missing the deadlines at its intended destinations. For example, if , which is derived from the clique , is sent with the minimum of the maximum transmission rates among  and , , its intended next hops  can successfully decode the packets  respectively, without missing their deadlines.

\begin{figure}[t]
\begin{center}
\includegraphics[height=23mm,width=70mm]{jointrate}\vspace{-0.08in}
\caption{Different graph model comparison}\vspace{-0.15in}\label{Fig.example}
\end{center}\vspace{-0.1in}
\end{figure}

With the graph , we have the following lemma.
\vspace{-0.02in}\begin{lemma}\vspace{-0.1in}\label{lemma_graph}
For the current packet transmission, if the encode packet , where , is sent with the transmission rate , it will be received by all the nodes in . In addition, for each , the packet  will be decoded by  without missing its deadline.
\end{lemma}\vspace{-0.02in}
\begin{proof}
Firstly, we can easily obtain that with transmission rate , all the receivers in  can successfully receive the sending packet. This is because the transmission rate  must be lower than the maximum transmission rate from  to .

Secondly, our graph is the subgraph of that constructed in~\cite{ZX2010Broadcast6}. According to \cite{ZX2010Broadcast6}, if  is successfully received by ,  can decode its ``wanted" packet , where .
Thus, any receiver  can obtain a ``wanted" packet  from  with transmission rate , where .

Thirdly, according to the condition (c), we have
{\small}
So, its arrival time at receiver  is
{\small}
In other words, the arrival time of the packet  will not miss its deadline at its receiver , where .
\end{proof}

As in Lemma~\ref{lemma_graph}, a clique  in the graph represents a feasible transmission solution for the current propagation, with the encoded packet , transmission rate , intended next hops in , and the propagation delay .

\vspace{-0.1in}
\subsection{RSNC Formulation}
While Lemma~\ref{lemma_graph} ensures that any encoding strategy based on any clique in the graph will be delivered within deadline for the current packet transmission, the transmission orders of the encoded packets, represented by the cliques in , also affects the timely packet receptions at their destinations.

As shown in Fig.~\ref{Fig.example}(b), if we first schedule packet  with transmission rate , represented by clique , and then schedule packet  with transmission rate , represented by clique , all the packets will be received at their destinations without missing deadlines. However, if we first schedule packet , and then packet , packet  will miss its deadline at .

Thus, the next task for us is to find a set of cliques in the graph and schedule the transmissions of the encoded packets represented by each clique, so as to minimize the number of missed packets for the whole transmission process.

Suppose that  is a clique in the graph, and the encoded packet represented by it is sent as the -th transmission at node . We also assume that , . Thus, the -th transmission at  is  where , and the transmission rate is . Let  be the transmission delay of the -th transmission.

We firstly define the following variant.
{\small }
Then, we can formulate the RSNC problem based on the graph model as follows.
{\small }
subject to{\small }\vspace{-0.04in}
where  is a sufficient large constant.

In the above formulation, the term of the objective represents the number of packets that miss their deadlines at the receivers, which needs to be minimized.
Constraint~(\ref{ch.1}) denotes that each vertex in the graph can only belong to one clique. Constraint~(\ref{ch.2}) means that if there is no link between vertex  and , vertices  can not be in the same clique. Constraint~(\ref{ch.3}) gives the transmission delay for the -th transmission, which is equal to the transmission delay with the minimum transmission rate among the rates from  to all intended receivers. The sufficient large constant  is used to guarantee that if ,  must be , as denoted in Constraint~\ref{ch.4}, and if ,  must be , as denoted in Constraint~\ref{ch.5}. Note that the arrival time of the packet in the -th transmission should consist of both the waiting time of the previous  transmissions and the transmission time of the -th transmission, i.e., .
Thus, Constraint~(\ref{ch.4}) and~(\ref{ch.5}) show that only if the arrival time of  at , i.e., , is no more than the reception deadline of  at ,  can be .

With the above integer nonlinear programming, we can get the optimal solution of RSNC problem. However, the computational complexity for the above integer nonlinear programming is too high when the graph is large. Thus, we need to find an efficient algorithm to solve it.

\vspace{-0.04in}
\section{Joint Rate Selection and Network Coding Algorithm}\label{Sec.algorithm}\vspace{-0.03in}
Since each clique in the graph represents a feasible transmission strategy for the current transmission, instead of determining the whole transmission sequence at once, we first design the algorithm to determine the encoding strategy and rate selection scheme for each packet propagation, by selecting a clique at a time. The whole transmission process consists of multiple packets transmission/cliques selection.
\vspace{-0.07in}
\subsection{Metric Consideration for Each Packet Propagation}\vspace{-0.03in}
First of all, in order to measure the ``goodness" of transmitting an encoded packet at a specific transmission rate for each packet propagation, it is necessary for us to adopt a reasonable metric which should take into account the impact of the transmission rate and the packet reception deadlines.
In this section, we shall design a metric, which not only satisfies as more requests as possible, but also minimizes the number of packets missing the deadlines after the current transmission.

For the current transmission, given an encoded packet and a selected transmission rate, let  be  if  is decoded/received by  from the current propagation without missing its deadline, otherwise, let it be . In addition, as described in Fig.~\ref{Fig.rsnc}, the current encoding strategy and transmission rate also affect the timely receptions of the packets in the following propagations. Let  be 1 if  will definitely miss its deadline at  after the current propagation, otherwise, let it be . Later, we will introduce how to calculate  and  for a given encoded packet and transmission rate. Let  be the transmission rate selected for the current propagation.

Our metric can be defined as follows.
\begin{DD}
For a coding solution , define the metric  when using the transmission rate  as follows:
\vspace{-0.05in}
{\small}
where  is the parameter, which can be defined as the benefit (e.g., importance) of the packet .
\end{DD}\vspace{-0.05in}

Firstly, for a given encoded packet ,  is  only if all the following conditions are met: 1) , which means  can successfully receive the sending packet; 2) , which means  is required at ; 3) All the other packets combined in the encoded packet except  are available at , which denotes the decoding requirement of  at ; 4) , which shows the requirement of the reception deadline.
Secondly, for each packet  which is not successfully received/decoded by  from the current transmission without missing deadlines (),  is  only if
{\small }
Here,  is the transmission delay of the current transmission, and  denotes the minimum delay to meet 's deadline at  in the next transmission. If the sum of the current transmission delay and the next minimum transmission delay is larger than the deadline of  at ,  will definitely miss its deadline, i.e., .
Thus, given an encoded packet and its transmission rate,  and  are both determined.

Hence the meaning of metric  in (\ref{equation.U}) can be explained as follows.
The first term  denotes the benefit obtained from the packets that are received without missing their deadlines from the current transmission. The second term  represents the lost due to packets that will definitely miss their deadlines after the current transmission. So, the metric  denotes the net benefit obtained from the current encoded packet and the transmission rate.
Thus, for each packet propagation, we aim to determine an encoded packet and select the transmission rate , that will maximize the metric .

Note that, the problem of maximizing the defined metric  is also NP-hard. We can prove it by considering its special case: the transmission rates on all the links are the same,  the reception deadlines for all the packets are the transmission time of one packet, and each packet has the same benefit . The special case of maximizing the defined metric  becomes to maximize the total number of the receivers that can decode one ``wanted" packet from the current encoded packet, which has been proved to be NP-hard in \cite{XiuminWang2010}.

\vspace{-0.1in}
\subsection{Heuristic Algorithm Design for Each Packet Propagation}\label{Sec.algorithm.design}
Although maximizing the defined metric  is also NP-hard, we can easily obtain the following observations, based on which we can design the heuristic algorithm.

P1: Maximizing the first term of the metric  is equal to find a maximum weight clique in the graph, where the weight at vertex  is defined as the benefit .

P2: The transmission rate is a parameter that adjusts the trade-off between delay and network coding gain. If  uses a low transmission rate, more receivers can successfully receive the sending packet, and the current transmission may satisfy more receivers' requirements, denoted by the first term in . However, low transmission rate means high transmission delay, which may cause more packets to miss their deadlines in the following transmission, denoted by the second term in .

Based on the above observations, we then design a heuristic algorithm for each packet propagation, by gradually increasing the transmission rate. Initially, the transmission rate is set to be no less than the lowest one from  to its receivers. Let  be the set of available transmission rates from  to all the destinations, and let  be the -th lowest rate in . As in Sec.~\ref{graph.model}, we construct the auxiliary graph with the given information. We also assign the weight  in vertex  for  to denote the benefit of .

In the -th step, we restrict that the transmission rate used at  must be no less than . For , if its maximum transmission rate from  is less than , it can not successfully receive the sending packet. This restriction can be realized by omitting the vertex  in  if . Then, we find the maximum weight clique in the subgraph , and adopt the transmission rate represented by the found clique. Each vertex  in the found clique denotes that  will be successfully obtained by  without missing its deadline, for the given transmission rate. For each of the other packets that can not be obtained at their receivers from the current transmission, we then judge whether it will definitely miss its deadline at its destinations, by (\ref{eq.miss}). Thus, in each step, we calculate . Such process continues until all the rates in  are considered. Finally, we compare the values of  obtained from each step and adopt the one with the largest value as solution.
Note that, if there are more than one solution with the maximum value of , we will choose the one with the smaller lost represented by the second term of (\ref{equation.U}).
The detailed of the algorithm is shown in Algorithm 1 of Fig.~\ref{alg}.
\vspace{-0.1in}
\subsection{Algorithm for the Whole Transmission Process}\vspace{-0.02in}
While algorithm 1 in Sec.~\ref{Sec.algorithm.design} describes the encoding of packets and selection of rate for every transmission, the whole transmission process will consist of multiple of such single process. We will first construct the graph  based on model in Sec.~\ref{graph.model}, and the graph will be updated by removing the selected vertices in the found clique by Algorithm 1, and the vertex  if  will definitely miss its deadline at destination . The packet reception deadlines for the packets also need to be updated after each transmission. The whole transmission process continues until the vertices set  of  becomes empty. The detail algorithm for the whole transmission is given in Algorithm 2 of Fig.~\ref{alg}.

\begin{figure}[t]\center
\scriptsize{
\begin{tabular}{|l|}
\hline {\bf Algorithm 1: one packet propagation process}\\
\hspace{2mm} , ;\\
\hspace{2mm} , ;\\
\hspace{2mm}{\bf for}  to \\
\hspace{4mm} find max weight clique  in subgraph ;\\
\hspace{4mm} , if , for ;\\
\hspace{4mm} ;\\
\hspace{4mm} {\bf For} each \\
\hspace{6mm} , if ;\\
\hspace{4mm} {\bf Endfor}\\
\hspace{4mm} ;\\
\hspace{2mm}{\bf Endfor}\\
\hspace{2mm} add  into  if  is the maximum among ;\\
\hspace{2mm} ;\\
\hspace{2mm} ;;\\
\hspace{2mm} the encoded packet is  for the current transmission;\\
\hspace{2mm} ;\\
\hline
\hline {\bf Algorithm 2: the whole packet transmission process}\\
\hspace{2mm} construct graph ;\\
\hspace{2mm} {\bf while} ( is not empty)\\
\hspace{4mm} conduct Algorithm 1 for the current packet propagation;\\
\hspace{4mm} remove the selected clique from ;\\
\hspace{4mm} remove the vertex  from  if ;\\
\hspace{4mm} update the packet reception deadline, e.g., ;\\
\hspace{4mm} update  based on the remaining  and ;\\
\hspace{2mm}{\bf Endwhile}\\
\hline
\end{tabular}\vspace{-0.05in}
\caption{Algorithm Design}\vspace{-0.15in} \label{alg}}
\end{figure}

\vspace{-0.1in}
\section{Simulation Results}\label{Sec.simulation}\vspace{-0.02in}
In this section, we demonstrate the effectiveness of our RSNC scheme through simulations.
We randomly generate a set of available packets in  and the ``wanted" packets in  at destination , where . The maximum transmission rate from  to  is randomly selected in , and the packet reception deadline is randomly generated in .

For comparison purpose, we include two baseline algorithms,
namely, {\em DSF (deadline smallest first) coding} algorithm \cite{ZX2010Broadcast6} and {\em SIN-1} algorithm \cite{XTL2006Time-critical14}. DSF coding algorithm does not consider the heterogenous transmission rates on the links, and in each time slot, always finds the maximum weight clique in the defined graph. SIN-1 algorithm always sends the packet with the minimum ``SIN-1" in each transmission, where ``SIN-1" of packet  is defined as the ratio of the duration from the current time to the deadline of the most urgent request for packet , to the number of requests for . In the simulation, we compare the deadline miss ratio under different transmission schemes, where deadline miss ratio is defined as the ratio of the number of packets missing their deadlines to the total number of requests.
For each setting, we present the average result of 100 samples.

\vspace{-0.1in}
\subsection{The Impact of the Transmission Rate}\vspace{-0.02in}
We first investigate the impact of the transmission rate on the performance of random one packet propagation during the whole transmission process. Given the rate for the current transmission, we run the maximum weight clique algorithm in the graph model to find the maximum number of packets that can be obtained at their destinations without missing deadline, i.e., satisfied requests, based on which we derive the number of packets that will definitely lose their deadlines in the next transmissions according to (\ref{eq.miss}), i.e., failed requests. We set , , and .
\begin{figure}[h]
\begin{center}\vspace{-0.1in}
\includegraphics[height=35mm,width=63mm]{tradeoff}\vspace{-0.1in}
\caption{The impact of the transmission rate on the performance of one transmission.}\vspace{-0.15in} \label{sim.tradeoff}
\end{center}\vspace{-0.1in}
\end{figure}

As shown in Fig.~\ref{sim.tradeoff}, with the increase of the transmission rate, the number of packets that can be successfully received/decoded by their receivers from the current transmission decreases. The reason is that with the increase of the transmission rate, higher number of destinations may not receive the packets due to shorter transmission range, which thus decreases the encoding opportunity at . We can also see that with the increase of the transmission rate, the number of packets that will definitely miss their deadlines at the receivers decreases. This is because, higher transmission rates incur less transmission delay, which gives more chances for the other packets received timely in the following transmissions. The above observation motivates the algorithm design in Sec.~\ref{Sec.algorithm.design}, by deciding the tradeoff between transmission rate and network encoding strategy.

We then investigate the impact of the transmission rate on the performance of the whole transmission. We set  and vary the scale of the transmission rates, i.e., . As shown in Fig.~\ref{sim.rate}, the deadline miss ratio with our RSNC scheme is much lower than with other schemes. In addition, the deadline miss ratio also decreases with the increase of the transmission rate. This is because higher transmission rates incur less transmission delay, which satisfies more successful transmissions.
\begin{figure}[t]
\begin{center}\vspace{-0.07in}
\includegraphics[height=35mm,width=60mm]{rate}\vspace{-0.1in}
\caption{The impact of the transmission rate on the performance of the whole transmission process.}\vspace{-0.15in} \label{sim.rate}
\end{center}\vspace{-0.05in}
\end{figure}

\vspace{-0.12in}
\subsection{The Impact of the Number of Destinations }
\begin{figure}[t]
\begin{center}
\includegraphics[height=31mm,width=88mm]{m}\vspace{-0.08in}
\caption{The miss deadline ratio vs. the number of destinations.}\vspace{-0.15in} \label{sim.m}
\end{center}\vspace{-0.08in}
\end{figure}

We then investigate the impact of the number of destinations  and the transmission rates on the deadline miss ratio. In this case, we set  by varying  in  for  and .

As shown in Fig.~\ref{sim.m}, the deadline miss ratio with our RSNC scheme is much lower than with other schemes. With the increase of , the gain of our RSNC scheme increases. We can also see that the DSF algorithm does not show significant gain over SIN-1 algorithm. This is because, although with network coding in DSF, more packets can be combined together, the encoded packet may still miss its deadline at some destinations, due to inappropriate transmission rate used. From Fig.~\ref{sim.m}, we see that, with the increase of , the deadline miss ratio increases. The reason is that there are more packets to be sent at  within the same deadline scale.
\vspace{-0.12in}
\subsection{The Impact of the Number of Packets }\vspace{-0.02in}
\begin{figure}[t]
\begin{center}\vspace{-0.08in}
\includegraphics[height=31mm,width=85mm]{n}\vspace{-0.1in}
\caption{The miss deadline ratio vs. the total number of packets in P.} \vspace{-0.1in}\vspace{-0.15in}\label{sim.n}
\end{center}\vspace{-0.05in}
\end{figure}

Finally, we investigate the impact of the total number of packets  and the reception deadlines on the deadline miss ratio. We set  by varying  in  for the cases of  and .

Again, from Fig.~\ref{sim.n}, we can see that our proposed RSNC scheme has the lowest deadline miss ratio.
In addition, with the increase of , the deadline miss ratio increases. This is because more packets need to be transmitted
at node . From Fig.~\ref{sim.n}, it is easy to see that the deadline miss ratio is smaller when , compared with . It is reasonable because with the increase of the deadlines, less packet will lose its deadline.

\vspace{-0.12in}
\section{Conclusion}\label{Sec.conclusion}
In this paper, we propose a novel joint rate selection and network coding (RSNC) scheme for time critical applications. We first prove that the proposed problem is NP-hard, and design a novel graph model to model the problem. Using the graph model, we mathematically formulate the problem. We also propose a metric, based on which we design an efficient algorithm to determine transmission rate and coding strategy. Finally, simulation results demonstrate the proposed RSNC algorithm effectively reduces the packet deadline miss ratio.

\vspace{-0.07in}


\begin{thebibliography}{10}
\providecommand{\url}[1]{#1}
\csname url@samestyle\endcsname
\providecommand{\newblock}{\relax}
\providecommand{\bibinfo}[2]{#2}
\providecommand{\BIBentrySTDinterwordspacing}{\spaceskip=0pt\relax}
\providecommand{\BIBentryALTinterwordstretchfactor}{4}
\providecommand{\BIBentryALTinterwordspacing}{\spaceskip=\fontdimen2\font plus
\BIBentryALTinterwordstretchfactor\fontdimen3\font minus
  \fontdimen4\font\relax}
\providecommand{\BIBforeignlanguage}[2]{{\expandafter\ifx\csname l@#1\endcsname\relax
\typeout{** WARNING: IEEEtran.bst: No hyphenation pattern has been}\typeout{** loaded for the language `#1'. Using the pattern for}\typeout{** the default language instead.}\else
\language=\csname l@#1\endcsname
\fi
#2}}
\providecommand{\BIBdecl}{\relax}
\BIBdecl

\bibitem{XTL2006Time-critical14}
J.~Xu, X.~Tang, and W.~Lee, ``Time-critical on-demand data broadcast:
  algorithms, analysis, and performance evaluation,'' \emph{{IEEE} Trans. on
  Parallel and Distributed Systems}, vol.~17, no.~1, pp. 3 -- 14, 2006.

\bibitem{Sagduyu2007}
Y.~Sagduyu and A.~Ephremides, ``On joint {MAC} and network coding in wireless
  ad hoc networks,'' \emph{{IEEE} Trans. on Information Theory}, vol.~53,
  no.~10, pp. 3697 --3713, Oct. 2007.

\bibitem{ACL+2000Network1216}
R.~Ahlswede, N.~Cai, S.~Li, and R.~Yeung, ``Network information flow,''
  \emph{{IEEE} Trans. on Information Theory}, vol.~46, pp. 1204--1216, 2000.

\bibitem{KRH+2008XORs510}
S.~Katti, H.~Rahul, W.~Hu, D.~Katabi, M.~Medard, and J.~Crowcroft, ``{XORs} in
  the air: Practical wireless network coding,'' \emph{{IEEE/ACM} Trans. on
  Networking}, vol.~16, no.~3, pp. 497--510, 2008.

\bibitem{KV2009Is646}
Y.~Kim and G.~D. Veciana, ``Is rate adaptation beneficial for inter-session
  network coding?'' \emph{{IEEE} Journal on Selected Areas in Communications},
  vol.~27, pp. 635 --646, Jun. 2009.

\bibitem{CJH2010Joint2444}
K.~Chi, X.~Jiang, and S.~Horiguchi, ``Joint design of network coding and
  transmission rate selection for multihop wireless networks,'' \emph{{IEEE}
  Trans. on Vehicular Technology}, vol.~59, pp. 2435 --2444, 2010.

\bibitem{Kim2010}
T.~Kim, S.~Vural, I.~Broustis, D.~Syrivelis, S.~Krishnamurthy, and T.~La~Porta,
  ``A framework for joint network coding and transmission rate control in
  wireless networks,'' in \emph{Proceedingsof 2010 {IEEE} {INFOCOM}}, Mar.
  2010, pp. 1 --9.

\bibitem{Ni2008}
B.~Ni, N.~Santhapuri, C.~Gray, and S.~Nelakuditi, ``Selection of {Bit-Rate} for
  wireless network coding,'' in \emph{Proceedings of 5th {IEEE} SECON}, 2008.

\bibitem{ZX2010Broadcast6}
C.~Zhan and Y.~Xu, ``Broadcast scheduling based on network coding in time
  critical wireless networks,'' in \emph{2010 {IEEE} International Symposium on
  Network Coding {(NetCod)}}, Jun. 2010, pp. 1 --6.

\bibitem{XiuminWang2010}
X.~Wang, J.~Wang, and Y.~Xu, ``Data dissemination in wireless sensor networks
  with network coding,'' \emph{EURASIP Journal on Wireless Communications and
  Networking}, vol. 2010, 2010.

\end{thebibliography}

\end{document}
