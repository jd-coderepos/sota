\documentclass[english]{IEEEtran}
\usepackage[T1]{fontenc}
\usepackage[utf8x]{inputenc}
\usepackage{array}
\usepackage{textcomp}
\usepackage{graphicx}

\makeatletter

\DeclareRobustCommand{\greektext}{\fontencoding{LGR}\selectfont\def\encodingdefault{LGR}}
\DeclareRobustCommand{\textgreek}[1]{\leavevmode{\greektext #1}}
\DeclareFontEncoding{LGR}{}{}
\DeclareTextSymbol{\~}{LGR}{126}
\providecommand{\tabularnewline}{\\}

\usepackage[utf8x]{inputenc}
\usepackage[T1]{fontenc}
\usepackage{textcomp}
\usepackage{cite}

\makeatother

\usepackage{babel}
\begin{document}

\title{On the Security of the Automatic Dependent Surveillance-Broadcast
Protocol}


\author{Martin Strohmeier, Vincent Lenders, Ivan Martinovic\\
University of Oxford, United Kingdom\\
armasuisse, Switzerland}
\maketitle
\begin{abstract}
Automatic dependent surveillance-broadcast (ADS-B) is the communications
protocol currently being rolled out as part of next generation air
transportation systems. As the heart of modern air traffic control,
it will play an essential role in the protection of two billion passengers
per year, besides being crucial to many other interest groups in aviation.
The inherent lack of security measures in the ADS-B protocol has long
been a topic in both the aviation circles and in the academic community.
Due to recently published proof-of-concept attacks, the topic is becoming
ever more pressing, especially with the deadline for mandatory implementation
in most airspaces fast approaching. 

This survey first summarizes the attacks and problems that have been
reported in relation to ADS-B security. Thereafter, it surveys both
the theoretical and practical efforts which have been previously conducted
concerning these issues, including possible countermeasures. In addition,
the survey seeks to go beyond the current state of the art and gives
a detailed assessment of security measures which have been developed
more generally for related wireless networks such as sensor networks
and vehicular ad hoc networks, including a taxonomy of all considered
approaches.\end{abstract}
\begin{IEEEkeywords}
ADS-B; aviation; air traffic control; NextGen; security; wireless;
privacy; broadcast
\end{IEEEkeywords}

\section{Introduction}

\thispagestyle{empty}The world of air traffic control (ATC) is moving
from uncooperative and independent (primary surveillance radar, PSR)
to cooperative and dependent air traffic surveillance (secondary surveillance
radar, SSR). This paradigm shift holds the promise of reducing the
total cost of deployment and improving the detection accuracy of aircraft.
However, it is well known in the aviation community that the ATC system,
which is currently being rolled out, called \emph{automatic dependent
surveillance-broadcast} (ADS-B), has not been developed with security
in mind and is susceptible to a number of different radio frequency
(RF) attacks. The problem has recently been widely reported in the
press \cite{NewScientist,CNN,Forbes,Wired,NPR} and at hacker conventions
\cite{Costin,bradhaines2012,defconkunkel}. Academic researchers,
too, proved the ease of compromising the security of ADS-B with current
off-the-shelf hard- and software \cite{schafer2013experimental}.
This broad news exposure led the International Civil Aviation Organization
(ICAO) to put the security of civil aviation on their agenda of the
12th air navigation conference, identifying ``cyber security as a
high-level impediment to implementation that should be considered
as part of the roadmap development process'' \cite{ICAO2012} and
creating a task force to help with the future coordination of the
efforts of involved stakeholders. 

This shows that there is a widespread concern about the topic, created
by the fact that ADS-B will be mandatory for all new aircraft in the
European airspace by 2015\footnote{Older aircraft need to be retrofit by 2017. The FAA mandates ADS-B
in the US airspace by 2020.} and has already been embraced by many airlines worldwide. Reports
from manufacturers and regulation bodies show that around 70-80 percent
of commercial aircraft worldwide have been equipped with ADS-B transponders
as of 2013 \cite{adsbimplementation,airbusstatus}. Countries such
as Australia have already deployed full continental coverage, with
ADS-B sensors being the single means of ATC in low population parts
of the country \cite{cascade}.

This paper gives an overview of the research that has been conducted
regarding the security of ADS-B and describes the potential vulnerabilities
identified by the community. Since much relevant security research
has been conducted in related fields such as wireless sensor networks
or general ad hoc networks, we analyze proposed countermeasures from
other areas that could be adapted for use in ADS-B or whether they
are not applicable for reasons inherent to the system. Furthermore,
the present survey provides a threat catalogue, analysis and vulnerability
categorization of the mainly used data link \textit{Mode S}. We focus
primarily on the security of ADS-B and not on other air traffic control
(sub-)\,systems, such as GPS. Among other questions, we seek to answer
why existing ideas for securing (wireless) networks such as traditional
cryptography cannot simply be transferred and used in the protection
of systems such as ADS-B.

While there were a number of reasons behind the switch to a modern
air traffic management system, cost has consistently been mentioned
as one of the most important ones throughout the process; existing
radar infrastructures are simply much more expensive to deploy and
maintain \cite{Smith2006}. ADS-B, on the other hand, provides significant
operational enhancements for both airlines and air traffic managers.
The increased accuracy and precision improves safety and decreases
the likelihood for incidents by a large margin, unless the system's
weak security is exploited by malicious in- and outsiders.

The remainder of the survey is organized as follows: Section \ref{sec:Problem-Statement}
gives a detailed overview over the security problems related to ADS-B
and the requirements of the system environment. Section \ref{sec:Outlining-Solutions}
outlines solutions proposed in previous works and looks at the sister
protocol ADS-C and military versions of ADS-B. Section \ref{sec:Secure-Broadcast-Authentication}
surveys secure broadcast authentication methods to address the problem
while Section \ref{sec:Secure-Location-Verification} reviews means
to establish secure location verification with ADS-B. Section \ref{sec:Summary}
summarizes and Section \ref{sec:Conclusion} concludes the survey. 


\section{Problem Statement \label{sec:Problem-Statement}}

This section defines the problems related to security in ADS-B more
thoroughly. First, we give a short overview over the currently used
ADS-B protocol and its existing vulnerabilities. Building on this,
a model of the ADS-B environment is outlined and the required attributes
of possible solutions are identified.\\



\subsection{ADS-B Protocol Overview}

The American Federal Aviation Administration (FAA) as well as its
European pendant EUROCONTROL named ADS-B as the satellite-based successor
of radar. At its introduction, ADS-B was a completely new paradigm
for air-traffic control. Every participant retrieves their own position
and velocity by using an onboard GPS receiver. The position is then
periodically broadcasted in a message (typically twice per second)
by the transmitting subsystem called ADS-B Out.\emph{ }The messages
are then received and processed by ATC stations on the ground as well
as by nearby aircraft, if equipped with the receiving subsystem ADS-B
In.\emph{ }Messages can integrate further fields such as ID, intent,
urgency code, and uncertainty level.\emph{ }

\begin{figure}
\includegraphics[scale=0.75]{Figures/ssr_overview}\caption{ADS-B hierarchy \cite{schafer2013experimental}.\label{fig:ADS-B-hierarchy.}
The 1090 MHz Extended Squitter is based on the traditional Mode S
system and provides the data link for ADS-B in commercial aviation.
UAT is a new development but currently only mandated for general aviation. }
\end{figure}


Two competing ADS-B data link standards have been proposed: Universal
Access Transceiver (UAT) and 1090 MHz Extended Squitter (1090ES).
UAT has been created specifically for the use with aviation services
such as ADS-B, utilizing the 978MHz frequency with a bandwidth of
1Mbps. Since UAT requires fitting new hardware, as opposed to 1090ES,
it is currently only used for general aviation\emph{ }in EUROCONTROL
and FAA-mandated airspaces. Commercial aircraft, on the other hand,
employ SSR Mode S with Extended Squitter, a combination of ADS-B and
traditional Mode S known as 1090ES (see Fig.\,\ref{fig:ADS-B-hierarchy.}).\emph{
}In other words, the ADS-B function can be integrated into traditional
Mode S transponders. From here on, we focus on the commercially used
1090ES data link. The complete overview over the ADS-B protocol can
be found in the specification documents \cite{DO2422,DO260B,DO282B2}
while various other works give succinct, higher level descriptions
of the protocol (e.g.\,\cite{Costin,schafer2013experimental,McCallie2011}). 


\subsubsection*{The 1090ES Data Link}

\begin{figure}
\includegraphics[scale=0.5]{Figures/system-overview}

\caption{Overview of the ADS-B system architecture. Aircraft receive positional
data that is transmitted via the ADS-B Out subsystem over the 1090ES
or the UAT data link. It is then received and processed by ground
stations and by other aircraft via the ADS-B In subsystem. \label{fig:Overview-ADS-B}}
\end{figure}


As the name suggests, the 1090ES data link predominantly uses the
1090MHz frequency for communication sent out by aircraft, to both
other aircraft and ground stations\emph{ }(Mode S also uses ground
to air communication at 1030MHz for interrogations and information
services). Figure \ref{fig:1090-ES-Data} provides a graphical view
of a 1090ES transmission, which starts off with a preamble of two
synchronization pulses. The data block is then transmitted by utilizing
pulse position modulation (PPM). With every time slot being 1\textmu{}s
long, a bit is indicated by either sending a 0.5\textmu{}s pulse in
the first half of the slot (1-bit) or in the second half (0-bit).
It is important to note that PPM is very sensitive to reflected signals
and multipath dispersion, a fact that can play a major role in security
and protocol considerations.\footnote{See \cite{barry} for more information on PPM and multipath.}

There are two different possible message lengths specified in Mode
S, 56 bit and 112 bit \cite{DO2422}, whereas ADS-B solely uses the
longer format. The downlink format field DF (alternatively UF for
uplink messages) assigns the type of the message.\emph{ }1090ES uses
a multipurpose format as shown in Fig.\,\ref{fig:1090-ES-Data}.
When set to 17, it indicates that the message is an extended squitter,
enabling the transmission of 56 arbitrary bits in the ME field.\emph{
}The CA field indicates information about the capabilities of the
employed transponder, while the\emph{ }24 bit AA field carries the
unique ICAO aircraft address which enables aircraft identification.\emph{
}Finally, the PI-field provides a 24 bit CRC to detect and correct
possible transmission errors.\emph{ }It is possible for recipients
to correct up to 5 bit errors in 1090ES messages using a fixed generator
polynomial of degree 24.

This quick overview shows that only the 56 bit ME field can be used
to transmit arbitrary data, i.e. utilized for a secure ADS-B solution.
However, not only is it very limited in size but it is also typically
occupied by positional and other data. Thus, the format as currently
in practical use is intuitively very limiting to most types of security
solutions as we will explain in more detail in this survey.


\subsection*{Relation to Legacy Systems}

\begin{figure}
\includegraphics[scale=0.65]{Figures/extended_squitter}

\caption{1090 ES Data Link \cite{schafer2013experimental}\label{fig:1090-ES-Data}}
\end{figure}
Traditionally, aircraft localization has been relying on radar systems
which had been developed for military applications, namely identification,
friend or foe (IFF) systems. There are two different concepts in conventional
radars: primary surveillance radars and secondary surveillance radars
\cite{Skolnik07}. PSRs are \textit{independent}; they work without
cooperation from the aircraft by transmitting high-frequency signals,
which the target object reflects. The echo identifies range, angular
direction, velocity, size and shape of the object. SSR, on the other
hand, uses interrogations from ground stations which are responded
to by transponders in aircraft. The reply includes information such
as the precise aircraft altitude, identification codes or information
about technical issues. In contrast to PSR, this approach is also
much more accurate in terms of localization and identification. As
all surveillance data such as position and status are derived directly
by the aircraft, SSR is \textit{dependent}. Furthermore, cooperation
by the aircraft is a requirement.

Before ADS-B, all SSR systems in ATC have been interrogation-based.
So called \textit{modes} are being used to query the identification
and altitude of an aircraft. There are three modes (A, C and S) currently
in use in civil aviation, Table \ref{tab:Comparison-of-different}
compares their characteristics with ADS-B. The latter embodies a paradigm
shift in ATC as air traffic surveillance is now cooperative and dependent,
i.e.~every aircraft collects their own data such as position and
velocity by using onboard measurement devices. ADS-B based surveillance
infrastructure is also much more cost-effective compared to conventional
PSR, which is a much more complex technology that also suffers from
wear and tear due to rotating parts. ICAO specifies the technological
cost of using primary radar to monitor an en-route airspace (200NM
radius) at \6 million and ADS-B is significantly cheaper at \T_{C}\mu sp_{m}\geq1-(1-\frac{c_{s}}{c})^{c_{r}}cc_{s}c_{r}ANNAK_{n}FK_{i}=F(K_{i+1}),\,0\leq i\leq n-1K_{i},i>0idK_{i}i<jFK_{i}K_{i}i+dFF'iP_{j}P_{j+2}K'_{i}SAC_{p}PVPVPV'P'ABABVPPVPPVEPBPVPP'BVPEPBP{V}Gr_{0}0.5r_{0}r_{group}x_{O}y_{O}v_{O}x_{E}y_{E}I\Delta t_{o}nl_{1},...,l_{n}l_{1}^{'},...,l_{n}^{'}\mu_{d}\sigma_{0}^{2}$. If these hypotheses
are true, then it can be assumed that the claimant has given valid
positional reports within the chosen level of significance (see \cite{Xiao}
for further explanations and simulation results in VANETs). As this
method is only practical for fixed receivers, the goal is to use ground
stations to detect unusual behaviour, which are also more likely to
be able to collect enough samples to make the verification process
as sound as possible.

On top of this, there are numerous comparably simple rules that can
be utilized as potential red flags by an intrusion detection system
without resorting to more complex measures. Neither of these rules
are necessary nor sufficient by themselves to detect an ongoing attack.
But depending on the scenario and the attacker's savvy, they can indicate
unusual behaviour that should be further investigated either by a
human or handled through additional technical means. For example,
it is very plausible to outright drop a number of packets where either
the data or the meta data is technically or physically impossible.
Doing so can significantly reduce the strain on the ADS-B system and
even prevent both spoofing and DoS attacks which are not crafted carefully
enough. If a large number of potential red flags are checked for an
intrusion detection system, not only the risk for an attacker to cause
an alert increases significantly, but also the cost and complexity
of an attack rise. While this does not constitute theoretical perfect
security, plausibility checks can be very useful in practice. 

Mitigating factors exist across various layers, from the physical
to the application. Some are available to ground stations/air traffic
control only, others also to aircraft-to-aircraft (A2A) ADS-B IN communication.
Such cases include but are not limited to \cite{Leinmuller2006}:\\

\begin{itemize}
\item Investigating\textbf{ airplanes which suddenly appear} well within
the maximum communication range of a receiver.
\item Dropping \textbf{aircraft which are violating a given acceptance range
threshold}, producing impossible locations.
\item \textbf{Aircraft violating a given mobility grade threshold}, producing
impossible minimum or maximum velocities.
\item \textbf{Maximum Density Threshold}: If too many aircraft are in a
given area, ATC software will typically alarm the user.
\item \textbf{Map-based Verification}: Aircraft in unusual places such as
no-fly areas or outside typical airways (this might possibly be better
handled at the ATC software layer).
\item \textbf{Flight plan-based Verification}: Flooded/attacked ground stations
are able to check ADS-B messages against the existing flight plan.
\item Obvious \textbf{discontinuities in one of the 9 ADS-B state vector
data fields} (also see the related Section \ref{sub:Kalman-Filtering})
.
\end{itemize}
As explained before, such potential red flags need to be handled with
utmost care and typically they can not be automated but require much
additional scrutiny before any action is taken (e.g., the packet/flight
is considered an attack and dropped from ATC monitors). Yet, they
also enable the opportunity to follow up and activate further means
to secure the airspace. For example, when using such centralized detection
at the ground stations, these same stations could then destroy messages
sent by the detected offenders as outlined in \cite{Martinovic2009,Wilhelm2011}.\\



\section{Summary \label{sec:Summary}}

The Tables \ref{tab:Overview1}-\ref{tab:Overview3} provide a compact
overview over the effectiveness of all examined solutions in combating
the various proposed attacks as well as in offering advantages and
disadvantages concerning the feasibility to implement each approach
in the real world. As it has been laid out, there is no single optimal
or even good solution when considering means that have no or little
impact on the currently employed ADS-B software and hardware. Table
\ref{tab:Overview1} shows the attacks the discussed approaches can
counteract. We see that most security schemes focus on attacks of
the message injection/modification class. This has two main reasons
that have been mentioned throughout this survey: First of all, the
open nature of ADS-B has been considered a desirable feature in most
scenarios. So unless there is a major paradigm shift in the way air
traffic communication and control is handled currently, there is no
interest in protecting against passive listeners, despite this being
the first stepping stone for more sophisticated and problematic attacks.
Second, passive attacks such as eavesdropping are simply much more
difficult to protect against without having a full cryptographic solution.
Similarly, attacks on the physical layer, such as continuously jamming
the well-known frequency or the more surgical message deletion are
hard to defend against, with measures on the same layer (e.g. uncoordinated
spread spectrum) providing some of the only approaches to this general
wireless security problem. All discussed approaches do however address
message insertion and tampering, either by protecting outright against
it through verification (cryptographic methods) or by detecting anomalies
in the data (e.g.~Kalman filtering, multilateration).
\begin{table}[t]
\begin{tabular}{|>{\centering}p{24mm}|>{\centering}p{1.5cm}|c|>{\centering}p{1.5cm}|}
\hline 
 & Injection / Modification & Eavesdropping & Jamming / Deletion\tabularnewline
\hline 
\hline 
Physical Layer Authentication & + & - & -\tabularnewline
\hline 
Uncoordinated Spread Spectrum & - & + & +\tabularnewline
\hline 
(Lightweight) PKI & + & + & -\tabularnewline
\hline 
μTESLA & + & - & -\tabularnewline
\hline 
Wide Area Multilateration & + & - & -\tabularnewline
\hline 
Distance Bounding & + & - & -\tabularnewline
\hline 
Kalman Filtering & + & - & -\tabularnewline
\hline 
Group Verification & + & - & -\tabularnewline
\hline 
Data Fusion & + & - & -\tabularnewline
\hline 
Traffic Modeling & + & - & -\tabularnewline
\hline 
\end{tabular}

\caption{Overview of capabilities of various security approaches against feasible
attacks on ADS-B. \label{tab:Overview1}}
\end{table}
\begin{table}[t]
\begin{tabular}{|>{\centering}p{2.4cm}|>{\centering}p{1.1cm}|>{\centering}p{1.1cm}|>{\centering}p{1.1cm}|>{\centering}p{1.1cm}|}
\hline 
 & Data Integrity & Source Integrity & Location Integrity & DoS\tabularnewline
\hline 
\hline 
Physical Layer Authentication & No & Yes & Possibly & Partly\tabularnewline
\hline 
Uncoordinated Spread Spectrum & No & No & No & Yes\tabularnewline
\hline 
(Lightweight) PKI & Yes & Yes & Yes & Partly\tabularnewline
\hline 
μTESLA & No & Yes & No & No\tabularnewline
\hline 
Wide Area Multilateration & No & No & Yes & No\tabularnewline
\hline 
Distance Bounding & No & No & Partly & No\tabularnewline
\hline 
Kalman Filtering & No & Partly & Partly & No\tabularnewline
\hline 
Group Verification & No & Possibly & Yes & No\tabularnewline
\hline 
Data Fusion & No & Partly & Yes & Backup\tabularnewline
\hline 
Traffic Modeling & No & No & Yes & No\tabularnewline
\hline 
\end{tabular}

\caption{Overview of security features of various approaches for use with ADS-B.
\label{tab:Overview2}}
\end{table}


Table \ref{tab:Overview2} takes a look at the security features the
discussed schemes can provide. As discussed before, only a full cryptographic
public key infrastructure can guarantee the integrity of received
data. All other approaches either aim to secure the integrity of the
source (e.g. μTESLA, many of the discussed physical layer schemes)
or seek to verify the provided location data independently. Additional
protection against flood denial of service attacks against ATC systems
can be directly provided by spread spectrum approaches and cryptography,
while other methods rely on higher layers to sort out false aircraft
claims.

Table \ref{tab:Overview3} provides an overview over the feasibility
of the different approaches in practical settings, especially considering
the current state of air traffic control in the aviation industry.
As is to be expected, the difficulty and cost columns are mostly correlated.
The difficulty to overcome technical challenges is particularly high
for distance bounding, which is yet in its beginnings and a full-blown
public key infrastructure. In contrast, we see wide-area multilateration,
Kalman filters and data fusion techniques already in use in the field.
This naturally translates to the cost factor which plays an important
role in industry decisions. One can choose between a completely new
protocol which addresses the security question better than the current
installment of ADS-B does, slight modifications such as new message
types, or a transparent, parallel system which requires new software
and/or new hardware in different scales. This touches also on the
question of scalability. For example, the fusion of various ATC systems
and their data (PSR, SSR, ADS-C, WAMLAT, FANS) is an obvious and necessary
idea. Yet, it is common knowledge in the aviation community that a
``major part of the business case for Automatic Dependent Surveillance
- Broadcast (ADS-B) is attributed to the savings generated by decommissioning
or reducing reliance on conventional radar systems`` \cite{Smith2006}.
Thus, it seems inefficient and unlikely to have legacy and/or new
backup systems around on a broad scale, simply to fix the inadequateness
of ADS-B related to security.
\begin{table*}[t]
\begin{centering}
\begin{tabular}{|>{\centering}p{2.5cm}|>{\centering}p{1.3cm}|>{\centering}p{1.2cm}|>{\centering}p{1.2cm}|>{\centering}p{4cm}|>{\centering}p{3cm}|}
\hline 
 & Difficulty  & Cost & Scalability & Compatibility & References\tabularnewline
\hline 
\hline 
Physical Layer 

Authentication & Variable & Variable & Variable & Requires additional hard-/software. No modifications to the ADS-B
protocol. & \cite{Danev2012,Danev2010,Zeng2010,hall2005radio,Jana2010,Maes2010,Devadas2008,Mathur2008,Zhang2010a,Wang2011a,Xiong,Qiu,Laurendeau2008,Laurendeau2009}\tabularnewline
\hline 
Uncoordinated Spread Spectrum & Medium & Medium & Medium & Requires new hardware and a new physical layer. & \cite{Strasser2008,Christina2009,Liu2010}\tabularnewline
\hline 
(Lightweight) PKI & High & High & Medium & Distribution infrastructure and changes in protocol and message handling
needed.  & \cite{Costin,Finke2013a,wessoncan,Samuelson2006,viggiano2010secure,schuchman2011automatic,Ziliang2010,Raya2005,Raya2007,Robinson2007,Parno,Zhang2010}\tabularnewline
\hline 
μTESLA & Medium & Medium & High & New message type required for key publishing, MAC added. & \cite{Perrig2003,Perrig2000,Perrig2005,Perrig2002,Liu,Eldefrawy2010,Haas2009,Hu2006}\tabularnewline
\hline 
Wide Area Multi\-lateration & Low & Medium & Medium & No change to ADS-B required. Separate hardware system. & \cite{Smith2006,Savvides2002,Neven2005,Purton2010,Johnson2012,Kaune2012,Thomas2011a,Daskalakis2003,Niles2012,Galati2005}\tabularnewline
\hline 
Distance Bounding & High & Medium & Low & New messages and protocol needed. & \cite{Brands1994,Song2008,Chiang2009,Chiang2012,Ranganathan,Tippenhauer2009}\tabularnewline
\hline 
Kalman Filtering & Low & Low & High & No additional messages needed. Separate software system. & \cite{Kovell2012,Kalman1960a,Welch1995,Fox2003,Krozel2004}\tabularnewline
\hline 
Group Verification & High & Medium & Low & New messages and protocol needed. & \cite{Sampigethaya2011,Kovell2012}\tabularnewline
\hline 
Data Fusion & Low & Medium - High & Medium & No change in ADS-B required. Separate system. & \cite{Baud2006,liu2013multi,Wei2003,Yan2008,smith2008method}\tabularnewline
\hline 
Traffic Modeling & Medium & Low & High & Additional, separate entities for ground stations needed. & \cite{schafer2013experimental,Xiao,Leinmuller2006}\tabularnewline
\hline 
\end{tabular}
\par\end{centering}

\caption{Overview of feasibility attributes of various approaches for use with
ADS-B. \label{tab:Overview3}}
\end{table*}



\subsection*{Future Research Directions}

Considering the fact that the invention, certification and large-scale
deployment of air-traffic systems takes decades, as currently seen
in the example of ADS-B, it seems equally non-sensible to present
a completely overhauled ATC-system at this point in time. Yet, future
aviation protocol development is important as there will eventually
be a successor to ADS-B and the responsible community must learn from
the ADS-B case study. Authentication should be considered right from
the beginning when planning new protocols, this includes choosing
the right cryptographic primitives, an appropriate communication pattern
considering A2A and ground stations, and, most importantly, a solution
to the key management problem. Furthermore, the challenges and realities
of communication in the avionic environment need to be taken into
account, for example the extremely lossy environment (as recently
examined and illustrated with OpenSky, an open research sensor network
in \cite{Strohmeier14,Schaefer14}) that might render many traditional
approaches infeasible.

For the urgent problem at hand, this means that incremental changes
with backwards compatibility kept in mind as a main factor are more
useful than completely new proposals. Improvement on transparent secure
location verification approaches such as multilateration can help
bridge the security gap in the very near future, although this means
losing some of the advantages for which ADS-B was originally developed.
In the same vain, fingerprinting methods on various layers can help
build an effective intrusion detection system against all but the
most sophisticated attackers without affecting the protocol as it
is currently deployed. Improvements on current data fusion algorithms
can also further both safety and security of currently deployed ATC
by reducing error margins and uncertainties for controllers.

Of course, cost and complexity of deployment cannot be the only factors
taken into consideration - not having security is famously even much
more expensive.\footnote{``If you think safety is expensive, try an accident”, an insight
by easyJet owner Stelios Haji-Ioannou that came after facing manslaughter
charges for the deaths of employees in a shipping disaster at a former
company.} As mentioned throughout this survey, it can be a useful dichotomy
to distinguish between attack detection, attack prevention and dealing
with suspected attacks. We focused mainly on attack detection and
prevention, leaving attack reaction for future research. \\



\section{Conclusion \label{sec:Conclusion}}

This survey sought to review the available research on the topic of
securing the ADS-B protocol in particular and air traffic control
communication in general. We provided an in-depth overview of the
existing work, both specific to ADS-B as well as ideas brought in
from related fields such as VANETs. After reviewing the literature,
it seems that the solutions currently under consideration (and in
use in practice such as multilateration) can only be a fill-in, providing
a quick improvement to the security of the current system. For all-encompassing
security (and possibly privacy), new message types and/or completely
new protocols would need to be defined. Taking this into account for
the creation of a long-term security solution in dependent air traffic
surveillance, it makes sense to consider the impact of both secure
broadcast authentication approaches as well as of secure location
verification. To avoid new hard challenges in the foreseeable future,
this should include a thorough analysis of the predicted traffic density
on today's wireless navigation channels as well as the possible impact
of the communication and message overhead of a new protocol.\\


\bibliographystyle{IEEEtran}
\bibliography{ads-b_overview_revision}

\end{document}
