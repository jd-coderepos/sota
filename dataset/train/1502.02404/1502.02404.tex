\documentclass{article}

\usepackage{fullpage}
\usepackage{amsthm}
\usepackage{amsmath}
\usepackage{amsfonts}
\usepackage{amssymb}
\usepackage{stmaryrd}
\usepackage{verbatim}
\usepackage{epic}

\usepackage{amscd}
\usepackage[dvips]{epsfig}
\usepackage{wrapfig}
\usepackage{enumitem}
\usepackage{pst-tree}
\usepackage{bussproofs}

\usepackage{cmll}


\bibliographystyle{plainurl}


\newcommand{\lone}{1}
	\newcommand{\ltens}{\otimes}
	\newcommand{\lbot}{\bot}
	\newcommand{\lplus}{\oplus}
	\newcommand{\lpar}{\parr}
	\newcommand{\lwith}{\with}



\newcommand{\target}[1]{t_{#1}}

\theoremstyle{plain}

\newtheorem*{notations}{Notations}
\newtheorem{definition}{Definition}
\newtheorem{theorem}{Theorem}
\newtheorem{example}{Example}
\newtheorem{prop}[theorem]{Proposition}
\newtheorem{lem}[theorem]{Lemma}
\newtheorem{fact}[theorem]{Fact}
\newtheorem{rem}{Remark}



\newcommand{\Nat}{\ensuremath{\mathbb{N}}}
\newcommand{\atoms}[1]{\textit{At}(#1)}
\newcommand{\nontrivialconnected}[3]{\mathcal{S}_{#1}^{#3}(#2)}
\newcommand{\connectedcomponents}[2]{\mathcal{C}^{#2}(#1)}
\newcommand{\size}[1]{\textit{size}(#1)}
\newcommand{\leftwires}[1]{\mathcal{L}(#1)}
\newcommand{\labelofcell}[1]{l_{#1}}
\newcommand{\taylor}[2]{\mathcal{T}(#1)[#2]}
\newcommand{\criticalports}[3]{\mathcal{K}_{#2, #3}(#1)}
\newcommand{\groundof}[1]{\mathcal{G}(#1)}
\newcommand{\finitemultisets}[1]{\mathcal{M}_\textit{fin}(#1)}
\newcommand{\finitesequences}[1]{{#1}^{< \infty}}
\newcommand{\cosize}[1]{\textit{cosize}(#1)}
\newcommand{\depthof}[1]{\textit{depth}(#1)}
\psset{treemode=U,levelsep=5mm}

\def\restriction#1#2{\mathchoice
              {\setbox1\hbox{}
              \restrictionaux{#1}{#2}}
              {\setbox1\hbox{}
              \restrictionaux{#1}{#2}}
              {\setbox1\hbox{}
              \restrictionaux{#1}{#2}}
              {\setbox1\hbox{}
              \restrictionaux{#1}{#2}}}
\def\restrictionaux#1#2{{#1\,\smash{\vrule height .8\ht1 depth .85\dp1}}_{\,#2}} 

\newcommand{\dom}[1]{\textsf{dom}(#1)}
\newcommand{\codom}[1]{\textsf{codom}(#1)}
\newcommand{\im}[1]{\textsf{im}(#1)}
\newcommand{\emptysequence}{\varepsilon}
\newcommand{\typesoflinks}{\mathfrak{T}}
\newcommand{\tens}{\otimes}
\newcommand{\one}{1}
\newcommand{\bottom}{\bot}
\newcommand{\cod}{\oc}
\newcommand{\contr}{\wn}

\newcommand{\portsatzero}[1]{\mathcal{P}_0(#1)}
\newcommand{\wiresatzero}[1]{\mathcal{W}_0(#1)}
\newcommand{\axiomsatzero}[1]{\mathcal{A}_0(#1)}
\newcommand{\arity}[1]{{\textit{a}}_{#1}}
\newcommand{\ports}[1]{\mathcal{P}(#1)}
\newcommand{\conclusions}[1]{\mathcal{P}^{\textsf{f}}(#1)}
\newcommand{\axioms}[1]{\mathcal{A}(#1)}
\newcommand{\multiplicativeports}[1]{\mathcal{P}^\textit{m}(#1)}
\newcommand{\multiplicativeportsatzero}[1]{\mathcal{P}_0^\textit{m}(#1)}

\newcommand{\sm}[1]{\llbracket #1 \rrbracket}
\newcommand{\Card}[1]{\textsf{Card}\left( #1 \right)}
\newcommand{\portsoftype}[2]{\mathcal{P}^{#1}(#2)}
\newcommand{\portsatzerooftype}[2]{\mathcal{P}_0^{#1}(#2)}
\newcommand{\exponentialportsatzero}[1]{\mathcal{P}_0^\textit{e}(#1)}
\newcommand{\Taylor}[2]{\mathcal{T}_{#1}(#2)}
\newcommand{\setofmultisets}[1]{\mathcal{M}_\textsf{fin}(#1)}
\newcommand{\conclusionscirc}[1]{\mathcal{P}_\circ^{\textsf{f}}(#1)}
\newcommand{\conclusionsnotcirc}[1]{\mathcal{P}_\bullet^{\textsf{f}}(#1)}
\newcommand{\cellsof}[1]{\mathcal{C}(#1)}
\newcommand{\setofboxesatzero}[1]{\mathcal{B}_0(#1)}
\newcommand{\boxes}[1]{\mathcal{B}(#1)}
\newcommand{\boxesatzero}[1]{\mathcal{B}_{0}(#1)}
\newcommand{\exactboxes}[2]{\mathcal{B}^{=#2}(#1)}
\newcommand{\exactboxesatzero}[2]{\mathcal{B}_{0}^{=#2}(#1)}
\newcommand{\boxesgeq}[2]{\mathcal{B}^{\geq #2}(#1)}
\newcommand{\boxesatzerogeq}[2]{\mathcal{B}_0^{\geq #2}(#1)}
\newcommand{\boxesatzerosmaller}[2]{\mathcal{B}_0^{< #2}(#1)}
\newcommand{\exponentialports}[1]{\mathcal{P}^{\textit{e}}(#1)}
\newcommand{\wires}[1]{\mathcal{W}(#1)}
\newcommand{\supp}[1]{\textit{Supp}(#1)}
\newcommand{\scalefact}{0.26}

\newcommand{\pict}[1]{\scalebox{\scalefact}{\input{#1.pstex_t}}}
\newcommand{\scalefactbis}{0.4}
\newcommand{\pictbis}[1]{\scalebox{\scalefactbis}{\input{#1.pstex_t}}}
\newcommand{\scalefactter}{0.33}
\newcommand{\pictter}[1]{\scalebox{\scalefactter}{\input{#1.pstex_t}}}

\newcommand{\scalefactnine}{0.3}
\newcommand{\pictnine}[1]{\scalebox{\scalefactnine}{\input{#1.pstex_t}}}

\newcommand{\scalefactten}{0.45}
\newcommand{\pictten}[1]{\scalebox{\scalefactten}{\input{#1.pstex_t}}}


\newcommand{\scalefacteight}{0.30}
\newcommand{\picteight}[1]{\scalebox{\scalefacteight}{\input{#1.pstex_t}}}


\newcommand{\scalefactseven}{0.23}
\newcommand{\pictseven}[1]{\scalebox{\scalefactseven}{\input{#1.pstex_t}}}


\newcommand{\scalefactsix}{0.26}
\newcommand{\pictsix}[1]{\scalebox{\scalefactter}{\input{#1.pstex_t}}}

\newcommand{\scalefactfour}{0.7}
\newcommand{\pictfour}[1]{\scalebox{\scalefactfour}{\input{#1.pstex_t}}}

\newcommand{\scalefactfive}{0.4}
\newcommand{\pictfive}[1]{\scalebox{\scalefactfive}{\input{#1.pstex_t}}}


\newenvironment{scprooftree}[1]{\gdef\scalefactor{#1}\begin{center}\proofSkipAmount \leavevmode}{\scalebox{\scalefactor}{\DisplayProof}\proofSkipAmount \end{center} }




\title{The relational model is injective for Multiplicative Exponential Linear Logic}

\author{Daniel de Carvalho}


\begin{document}

\maketitle

\begin{abstract}
We prove a completeness result for Multiplicative Exponential Linear Logic (MELL): we show that the relational model is injective for MELL proof-nets, i.e. the equality between MELL proof-nets in the relational model is exactly axiomatized by cut-elimination.
\end{abstract}



In the seminal paper by Harvey Friedman \cite{Friedman}, it has been shown that equality between simply-typed lambda terms in the full typed structure  over an infnite set  is completely axiomatized by  and : we have . A natural problem is to know whether a similar result could be obtained for Linear Logic. 

Such a result can be seen as a ``separation'' theorem. To obtain such separation theorems, it is a prerequesite to have a ``canonical'' syntax. 
When Jean-Yves Girard introduced Linear Logic (LL) \cite{ll}, he not only introduced a sequent calculus system but also ``proof-nets''. Indeed, as for LJ and LK (sequent calculus systems for intuitionnistic and classical logic, respectively), different proofs in LL sequent calculus can represent ``morally'' the same proof: proof-nets were introduced to find a unique representative for these proofs. 


The technology of proof-nets was completely satisfactory for the multiplicative fragment without units.\footnote{For the multiplicative fragment with units, it has been recently shown \cite{MLLwithunits} that, in some sense, no satisfactory notion of proof-net can exist. Our proof-nets have no jump, so they identify too many sequent calculus proofs, but not more than the relational semantics.} For proof-nets having additives, contractions or weakenings, it was easy to exhibit different proof-nets that should be identified. Despite some flaws, the discovery of proof-nets was striking. In particular, Vincent Danos proved by syntactical means in \cite{phddanos} the confluence of these proof-nets for the Multiplicative Exponential Linear Logic fragment (MELL). For additives, the problem to have a satisfactory notion of proof-net has been addressed in \cite{mallpn}. For MELL, a ``new syntax'' was introduced in \cite{hilbert}. In the original syntax, the following properties of the weakening and of the contraction did not hold:
\begin{itemize}
\item the associativity of the contraction;
\item the neutrality of the weakening for the contraction;
\item the contraction and the weakening as morphisms of coalgebras.
\end{itemize}
But they hold in the new syntax; at least for MELL, we got a syntax that was a good candidate to deserve to be considered as being ``canonical''. Then trying to prove that any two (-expanded) MELL proof-nets that are equal in some denotational semantics are -joinable has become sensible and had at least the two following motivations:
\begin{itemize}
\item to prove the canonicity of the ``new syntax'' (if we quotient more normal proof-nets, then we would identify proof-nets having different semantics);
\item to prove by semantics means the confluence (if a proof-net reduces to two cut-free proof-nets, then they have the same semantics, so they would be -joinable, hence equal).
\end{itemize}
The problem of \emph{injectivity}\footnote{The tradition of the lambda-calculus community rather suggests the word ``completeness'' and the terminology of category theory rather suggets the word ``faithfulness'', but we follow here the tradition of the Linear Logic community.} of the denotational semantics for MELL, which is the question whether equality in the denotational semantics between (-expanded) MELL proof-nets is exactly axiomatized by cut-elimination or not, can be seen as a study of the separation property with a semantic approach. The first work on the study of this property in the framework of proof-nets is \cite{Marco} where the
authors deal with the translation into LL of the pure -calculus; it has been studied more recently for the intuitionistic multiplicative
fragment of LL \cite{typedbohm} and for differential nets \cite{separationdiff}. For Parigot's -calculus, see \cite{lmbohm} and \cite{separationsaurin}.

Finally the precise problem of  injectivity for MELL has been adressed by Lorenzo Tortora de Falco in his PhD thesis \cite{phdtortora} and in \cite{injectcoh} for the (multiset based) coherence semantics and the multiset based relational semantics. 
He gave partial results and counter-examples for the coherence semantics: the (multiset based) coherence semantics is not injective for MELL.  
Also, it was conjectured that the relational model is
injective for MELL. We prove the conjecture in the present paper.



In \cite{injectcoh}, a proof of the injectivity of the relational model is given for a weak fragment. 
But despite many efforts (\cite{phdtortora}, \cite{injectcoh}, \cite{boudesunifying}, \cite{pagani06a}, \cite{separationdiff}, \cite{taylorexpansioninverse}...), all
the attempts to prove the conjecture failed up to now. 
New progress was made in \cite{LPSinjectivity}, where it has been proved that the relational semantics is injective for ``connected'' MELL proof-nets. Still, there, ``connected'' is understood as a very strong assumption, the set of ``connected'' MELL proof-nets contains the fragment of MELL defined by removing weakenings and units. Actually \cite{LPSinjectivity} proved a much stronger result: in the full MELL fragment two proof-nets  and  with the same
interpretation are the same ``up to the connections between the doors of exponential
boxes'' (we say that they have the same LPS\footnote{The LPS of a proof-net is the graph obtained by forgetting the outline of the boxes but keeping the auxiliary doors.} - see Figures~\ref{fig: R_1}, \ref{fig: R_2} and \ref{fig: R_3} for an example of three different proof-nets having the same LPS). We wrote: ``This result can be expressed in terms of differential nets: two cut-free proof-nets with different LPS have different Taylor expansions. We also believe this work is an essential step
towards the proof of the full conjecture.'' Despite the fact we obtained a very interesting result about \emph{all} the proof-nets (i.e. also for non-``connected'' proof-nets\footnote{and even adding the MIX rule}), the last sentence was a bit too optimistic, since, in the present paper, which presents a proof of the full conjecture, we could not use any previous result nor any previous technic/idea.


The result of the present paper can be seen as
\begin{itemize}
\item a semantic separation property in the sense of \cite{Friedman};
\item a semantic proof of the confluence property;
\item a proof of the ``canonicity'' of the new syntax of MELL proof-nets;
\item a proof of the fact that if the Taylor expansions of two cut-free MELL proof-nets into differential nets \cite{EhrhardRegnier:DiffNets} coincide, then the two proof-nets coincide.
\end{itemize}
\begin{comment}
The relational semantics is very important for Linear Logic:\begin{itemize}
\item the interpretations of the proofs in the relational model are the same ones as in the non-uniform coherence semantics (\cite{bucciarelliehrhard01, Boudes11}) and in the category of weighted sets (\cite{classicalPCF});
\item we can closely relate the size of the points of the interpretations of the proof-nets in the relational model with the execution time of the proof-nets (\cite{phddecarvalho}, \cite{Carvalhoexecution} and \cite{CarvPagTdF10});
\item the relational semantics is a denotational semantics for differential nets \cite{EhrhardRegnier:DiffNets};\footnote{This is not the case with the coherence semantics.} these ones can be seen as a refinement of Linear Logic - the translation from LL proof-nets into differential nets is the Taylor expansion.
\end{itemize}
\end{comment}


Let us give one more interpretation of its signifance. First, notice that a proof of this result should consist in showing that, given two non -equivalent proof-nets  and , their respective semantics  and  are not equal, i.e.  or .\footnote{The converse, i.e. two -equivalent proof-nets have the same semantics, holds by definition of soundness.} But, actually, we prove something much stronger: we prove that, given a proof-net , there exist two points  and  such that, for any proof-net , we have . 

Now, the points of the relational model can be seen as non-idempotent intersection types\footnote{Idempotency of intersection () does not hold.}  (see \cite{phddecarvalho} and \cite{Carvalhoexecution} for a correspondance between points of the relational model and System R - System R has also been studied recently in \cite{inhabitation}). And the proof given in the present paper uses the types only to derive the normalization property; actually we prove the injectivity for cut-free proof-nets in an untyped framework:\footnote{For cut-free proof-nets, types guarantee that they are not cyclic as graphs - instead of typing, it is enough to assume this property. Our proof even works for ``non-correct'' proof-structures (correctness is the property characterizing nets corresponding in a typed framework with proofs in sequent calculus): we could expect that if the injectivity of the relational semantics holds for proof-nets corresponding with MELL sequent calculus, then it still holds for proof-nets corresponding with MELL+MIX sequent calculus, since the category \textbf{Rel} of sets and relations is a compact closed category. \cite{k=2} assuming correctness substituted in the proof the ``bridges'' of \cite{LPSinjectivity}  by ``empires''.} substituting the assumption that proof-nets are typed by the assumption that proof-nets are normalizable does not change anything to the proof.\footnote{Except that we have to consider the \emph{atomic} subset of the interpretation instead of the full interpretation (see Remark~\ref{remark: untyped framework}).} 
In \cite{CarvPagTdF10}, we gave a semantic characterization of normalizable untyped proof-nets and we characterized ``head-normalizable'' proof-nets as proof-nets having a non-empty interpretation in the relational semantics. Principal typings in untyped -calculus are intersection types which allow to recover all the intersection types of some term. If, for instance, we consider the System  of \cite{phddecarvalho} and \cite{Carvalhoexecution}, it is enough to consider some \emph{injective -point}\footnote{An \emph{injective -point} is a point in which all the positive multisets have cardinality  and in which each atom occurring in it occurs exactly twice.} to obtain the principal typing of an untyped -term. But, generally, for normalizable MELL proof-nets, \emph{injective -points}, for any , are not principal typings; indeed, two cut-free MELL proof-nets having the same LPS have the same injective -points for any . In the current paper we show that a -point and a \emph{-injective point}\footnote{The reader should not confuse \emph{-injective points} with \emph{injective -points}. \emph{-injective points} are points in which every positive multiset has cardinality  for some  and, for any , there is at most one occurrence of a positive multiset having cardinality  - they are obtained by \emph{-injective experiments} (see our Definition~\ref{defin: k-injective}).} together allow to recover the interpretation of any normalizable MELL proof-net. So, the result of the current paper can be seen as a first attempt to find a right notion of ``principal typing'' of intersection types in Linear Logic. As a consequence, normalization by evaluation, as in \cite{Rocca88} for -calculus, finally becomes possible in Linear Logic too.

Section~\ref{section: Syntax} formalizes PS's (our cut-free proof-nets). Section~\ref{section: Experiments} gives a sketch of our algorithm leading from  to the rebuilding of . Section~\ref{section: one step} describes more precisely one step of the algorithm and states our theorem (Theorem~\ref{thm: injectivity}): 
, 
where  is the reflexive symmetric transitive closure of the cut-elimination relation.
 





\begin{notations} 
We denote by  any empty sequence. If  is a sequence , then  denotes the sequence ; otherwise, it denotes the sequence  of length . The set of finite sequences of elements of some set  is denoted by .

A multiset  of elements of some set  is a function ; we denote by  \emph{the support of } i.e. the set . A multiset  is said to be \emph{finite} if  is finite. The set of finite multisets of elements of some set  is denoted by .

If  is a function ,  and , then we denote by  the function  defined by x \not= x_0 x = x_0
\end{notations}

\section{Syntax}\label{section: Syntax}

We introduce the syntactical objects we are interested in. As recalled in the introduction, simple types guarantee normalization, so we can limit ourselves to nets without any cut. Correctness does not play any role, that is why we do not restrict our nets to be correct and we rather consider proof-structures (\emph{PS's}). Since our proof is easily extended to MELL with axioms, we remove them for simplicity. Moreover, since it is convenient to represent formally our proof using differential nets with boxes (\emph{differential PS's}), we define PS's as differential PS's satisfying some conditions (Definition~\ref{defin: PS}). More generally, \emph{differential -PS's} are defined by induction on the depth: Definition~\ref{defin: diff ground-structure} concerns what happens at depth .

\begin{comment}
Moreover, since it is convenient to present our algorithm 

Differential PS's are differential nets possibly with boxes that contain proof-structures (Definition~\ref{defin: differential PS}). They are defined by induction on the depth; Definition~\ref{defin: diff ground-structure} concerns what happens at depth . 
\end{comment}

We define the set  of types as follows: . We set    . Pre-contractions (-ports) are an artefact of our inductive definition on the depth and are used to ensure the canonicity of our syntactical objects (see Example~\ref{example: old versus new}).


\begin{definition}\label{defin: diff ground-structure}
A \emph{differential ground-structure} is a -tuple , where
\begin{itemize}
\item  is a finite set; the elements of  are the \emph{ports of };
\item  is a function ;the element  of  is the \emph{label of  in };
\item  is a subset of ; the elements of  are the \emph{wires of };
\item  is a function  such that, for any port  of , we have     ; 
if , then  is a \emph{premise of }; the \emph{arity  of } is the number of its premises;
\item  is a subset of  such that    ; if  s.t. , then  is \emph{the left premise of ;}
\item and  is a function  such that, for any ,
\begin{itemize}
\item if , then ;
\item if  (resp. ), then, for any  and any  such that , we have  (resp. );
\item if , then ;
\item and if , then .
\end{itemize}
\end{itemize}
We set , , , , , . The set  is the set of \emph{conclusions of }. For any , we set ; we set ; the set  of \emph{exponential ports of } is . 

A \emph{ground-structure} is a differential ground-structure  s.t. .
\end{definition}

Notice that, for any differential ground-structure , we have .

\begin{example}
The ground-structure  defined by: 
 ; 
; 
, , , ; 
, ; 
; 
and , , , ; 
is the ground-structure of the content of the box  of  (the leftmost box of Figure~\ref{fig: new_example}).
\end{example}


The content of every box of our \emph{differential -PS's} is a \emph{-PS}: every -port inside is always the main door of some box.

\begin{definition}\label{defin: differential PS}
For any , we define, by induction on , the set of \emph{differential -PS of depth } (resp. the set of \emph{-PS of depth }). 
A \emph{differential -PS of depth } (resp. a \emph{-PS of depth }) is a 4-tuple , where
\begin{itemize}
\item  is a differential ground-structure (resp. a ground-structure);
\item  (resp. ) and is the set of \emph{boxes of  at depth }; 
\item  is a function that associates with every  a -PS     of depth  that enjoys the following property: if , then there exists  s.t.  is a -PS of depth ; the -port  is \emph{the main door of the box };
\item and  is a function that associates with every  a function  such that (resp.  and), for any , 
\begin{itemize}
\item  is injective\footnote{So one cannot (pre-)contract several -ports of the same box.} (resp.\footnote{This stronger condition on -PS's is \emph{ad hoc}, but it allows to lighten the notations.} ); if  with , then we set ;
\item  and ;
\item for any , we have   ;
\item for any , we have ;
\end{itemize}
\end{itemize}
(resp. moreover no  is a sequence)\footnote{This condition on -PS's is \emph{ad hoc}, but it allows to simplify Definition~\ref{defin: Taylor}.}.
\begin{comment}
we require that
\begin{itemize}
\item 
\item and, for any  with , we have .
\end{itemize}
\end{comment}
For any differential -PS , we set ,  and  is the set of \emph{boxes of }. We denote by  the extension of the function  that associates with each , where , the -PS . We denote by  the extension of the function  that associates with each , where , the function .

We set 
 and ; the elements of  (resp. of ) are the \emph{ports of  at depth } (resp. the \emph{wires of  at depth }). 
For any , we set . 
We set ,  and ; the elements of  are the \emph{conclusions of } and the elements of  are the \emph{-conclusions of }. For any relation  on , for any , we set  and .
\end{definition}

\begin{figure}\centering
\pictter{new_example}
\caption{PS }
\label{fig: new_example}
\end{figure}

\begin{comment}
\begin{figure}\centering
\pictsix{new_example_bis}
\caption{PS  having the same LPS as }
\label{fig: same LPS}
\end{figure}

\begin{figure}\begin{minipage}{11cm}
\centering
\pictseven{new_LPS}
\caption{LPS of  and }
\label{fig: LPS}
\end{minipage}
\end{figure}
\end{comment}

\begin{comment}
Pseudo differential -PS's will be used in Definition \ref{defin: Taylor} as an auxiliary construction. They are differential -PS's up to the names of the -conclusions of the boxes. One can always associate a differential -PS  with a pseudo differential -PS  as follows:

\begin{definition}
For any pseudo differential -PS , we define, by induction on , a differential -PS  such that
\begin{itemize}
\item 
\item 
\item 
\item and 
\end{itemize}
as follows:
\begin{itemize}
\item for any , , where  is the following pseudo -PS:
\begin{itemize}
\item 
\item , , , , 
\item  and, for any , 
\begin{itemize}
\item 
\item \labelofcell{\groundof{B_S(o)}}(b_{B_S(o)}(o')(p)) \not= \circ\labelofcell{\groundof{B_S(o)}}(b_{B_S(o)}(o')(p)) = \circ
\end{itemize}
\item for any , \labelofcell{\groundof{\underline{S(o)}}}(p) \not= \circ\labelofcell{\groundof{\underline{S(o)}}}(p) = \circ
\end{itemize}
\end{itemize}
\end{definition}
\end{comment}



\begin{comment}
\begin{definition}
For any differential -PS , we set 
\begin{itemize}
\item  and ; the elements of  (resp. of ) are the \emph{ports of  at depth } (resp. the \emph{wires of  at depth });
\item for any , ;
\item ; the elements of  are the \emph{conclusions of }; 
\item ; the elements of  are the \emph{-conclusions of };
\item and 
\end{itemize}
and, for any relation  on , for any , we set  and .
\end{definition}
\end{comment}




\begin{comment}
\begin{definition}
For any , we define, by induction on , the set of \emph{typed differential -PS of depth } (resp. the set of \emph{typed -PS of depth }) and a function  from the set of  typed differential -PS of depth  to the set of differential -PS of depth . 
A \emph{typed differential -PS of depth } (resp. a \emph{typed -PS of depth }) is a 4-tuple , where
\begin{itemize}
\item  is a typed differential ground-structure (resp. a typed ground-structure); 
\item ;
\item  is a function that associates with every  a typed-PS  of depth ;
\item and  is a function that associates with every  a function  such that
\begin{itemize}
\item for any , we have  
\item for any , there exists  such that  
\item for any , there exists  such that ;
\item and  is a differential -PS of depth  (resp. a -PS of depth );
\end{itemize}
\end{itemize}
where  with  the function that associates with every  the -PS .
\end{definition}
\end{comment}

PS's are the MELL proof-nets studied in the present paper: there is no cut and no assumption of correctness property.

\begin{definition}\label{defin: PS}
A PS is a -PS  such that .
\end{definition}

\begin{example}
Consider the PS  of Figure~\ref{fig: new_example}. 
We have ,        , . We have , ,  and .
\end{example}

\begin{example}~\label{example: old versus new}
In order to understand the role of the -ports, consider how the proof-nets  (Figure~\ref{figure: O1}) and  (Figure~\ref{figure: O2}) in the ``old syntax'' (we denoted derelictions, contractions and auxiliary doors of the ``old syntax'' by \textsf{d}, \textsf{c} and \textsf{a}, respectively) are represented by the same PS  (Figure~\ref{figure: N}). Roughly speaking, in our formalism, one pre-contracts (using -ports) as soon as possible and one contracts (using -ports) as late as possible.

\begin{figure}
\centering
\begin{minipage}{4.5cm}
\centering
\pictter{O1}
\caption{ (``old syntax'')}
\label{figure: O1}
\end{minipage}\hfill
\begin{minipage}{4.5cm}
\centering
\pictter{O2}
\caption{ (``old syntax'')}
\label{figure: O2}
\end{minipage}
\begin{minipage}{4.5cm}
\centering
\pictter{N}
\caption{PS }
\label{figure: N}
\end{minipage}
\end{figure}
\end{example}



We write  (resp. ) if  and  are the same differential PS's up to the names of their ports (resp. that are not conclusions):

\begin{definition}
An \emph{isomorphism  of ground-structures} is a structure-preserving bijection . We define, by induction on , when  holds for two differential -PS's  and : it holds whenever  is a pair  s.t. ,  and, 
for any , 
 and . 
We set  and, for any , . 
We write  if  s.t. .

We write  (resp. ) if there exists  s.t.  (resp. ).
\end{definition}


\begin{comment}
\begin{definition}
Let  be a PS of depth . Let  be a ground-structure such that 
\begin{itemize}
\item 
\item 
\item 
\end{itemize}
Let  be a surjection . Then we define a differential -PS  of depth  as follows:
\begin{itemize}
\item 
\item 
 \item 
\item q \in \wiresatzero{R}q \in \conclusionsnotcirc{R}
\item 
\item 
\item 
\item b_R(o)(q) \notin \conclusionscirc{R}b_R(o)(q) \in \conclusionscirc{R}
\end{itemize}
\end{definition}
\end{comment}




The \emph{arity  of a port  in a differential -PS } is computed  by ``ignoring'' the -conclusions of the boxes of :

\begin{definition}
Let  be a differential -PS. We define, by induction on , the integers  for any  and : we set 
       and 
  .
\end{definition}

\begin{example}
We have  (and not ) and  (see Figure~\ref{fig: new_example}).
\end{example}

\begin{comment}
\begin{definition}
Let  be a differential -PS. We define, by induction on , : .
\end{definition}

\begin{example}
We have , where  is the PS of Figure~\ref{fig: new_example}.
\end{example}
\end{comment}

\section{Experiments and their partial expansions}\label{section: Experiments}




When Jean-Yves Girard introduced proof-nets in \cite{ll}, he also introduced \emph{experiments of proof-nets}. Experiments (see our Definition~\ref{defin: experiment}) are a technology allowing to compute pointwise the interpretation  of a proof-net  in the model directly on the proof-net rather than through some sequent calculus proof obtained from one of its sequentializations: the set of \emph{results} of all the experiments of a given proof-net is its interpretation . In an untyped framework, experiments correspond with type derivations and results correspond with intersection types. 



\begin{definition}\label{defin: experiment}
For any , we define, by induction on , the set : ; ; .

Let  be a differential -PS. We define, by induction on , the set of \emph{experiments of }: it is is the set of triples , where  is a function that associates with every  an element of  and  is a function which associates to every  a finite multiset of experiments of  such that
\begin{itemize}
\item for any , for any  such that ,  and , we have ;
\item for any ,
we have  .
\end{itemize}
For any experiment , we set  and . We set   .
\end{definition}

We encode in a more compact way the ``relevant'' information given by an experiment \emph{via} \emph{pseudo-experiments} and the functions :

\begin{definition}\label{defin: experiment induces pseudo-experiment}
For any differential -PS , we define, by induction on , the set of \emph{pseudo-experiments of }: it is the set of functions that associate with every  a finite set of pseudo-experiments of  and with  a pair  for some .

Given an experiment  of some differential -PS , we define, by induction on , a pseudo-experiment  of  as follows:  and   for any .

Given a pseudo-experiment  of a differential -PS , we define, by induction on , the function  as follows: for any ,  and, for any , .
\end{definition}


There are different kinds of experiments:
\begin{itemize}
\item In \cite{injectcoh}, it was shown that given the result of an \emph{injective -obsessional experiment} ( big enough) of a cut-free proof-net in the fragment , one can rebuild the entire experiment and, so, the entire proof-net. There, ``injective'' means that the experiment labels two different axioms with different atoms and ``obsessional'' means that different copies of the same axiom are labeled by the same atom.\item In \cite{LPSinjectivity}, it was shown that for any two cut-free MELL proof-nets  and , we have  iff, for  big enough\footnote{Interestingly, \cite{k=2}, following the approach of \cite{LPSinjectivity}, showed that, if these two proof-nets are assumed to be (recursively) connected, then we can take .}, there exist an \emph{injective -experiment} of  and an \emph{injective -experiment} of  having the same result; as an immediate corollary we obtained the injectivity of the set of (recursively) connected proof-nets. There, ``injective'' means that not only the experiment labels two different axioms with different atoms, but it labels also different copies of the same axiom by different atoms. Given some proof-net , there is exactly one injective -experiment of  up to the names of the atoms.
\item In the present paper we show that, for any two PS's  and , given the result  of a \emph{-injective experiment} of  for  big enough, if , then  is the same PS as . The conditions on  are given by the result of a -experiment, so we show that two (well-chosen) points are enough to determine a PS. The expression ``-injective'' means that, for any two different occurrences of boxes, the experiment never takes the same number of copies: it takes  copies and  copies with  (\emph{a contrario}, in \cite{injectcoh} and \cite{LPSinjectivity}, the experiments always take the same number of copies). As shown by the proof-net S of Figure~\ref{fig: no reconstruction of the experiment}, it is impossible to rebuild the experiment from its result, since there exist four different -injective experiments  and  such that, for any , we have ,  and . For instance  takes  copies of the box  and  copies of the box , while  takes  copies of the box  and  copies of the box .
\end{itemize}

\begin{definition}\label{defin: k-injective}
Let . A pseudo-experiment  of a -PS  is said to be \emph{-injective} if
\begin{itemize}
\item for any , for any , there exists  such that ;
\item for any , for any , we have  ;
\item and, for any , we have .
\end{itemize}
An experiment  is said to be \emph{-injective} if  is -injective.
\end{definition}

\begin{example}\label{example: pseudo}
There exists a -injective pseudo-experiment  of the proof-net  of Figure~\ref{fig: new_example} such that
, 
, 
, 
, 
  , 
  , 
   
and   .
\end{example}



In \cite{LPSinjectivity}, the interest for \emph{injective} experiments came from the remark that the result of an \emph{injective} experiment of a \emph{cut-free} proof-net can be easily identified with a differential net of its Taylor expansion in a sum of differential nets \cite{EhrhardRegnier:DiffNets} (it is essentialy the content of our Lemma~\ref{lem: taylor expansion}). Thus any proof using injective experiments can be straightforwardly expressed in terms of differential nets and conversely. Since this identification is trivial, besides the idea of considering injective experiments instead of obsessional experiments, the use of the terminology of differential nets does not bring any new insight\footnote{For proof-nets with cuts, the situation is completely different: the great novelty of differential nets is that differential nets have a cut-elimination; the differential nets appearing in the Taylor expansion of a proof-net with cuts have cuts, while the semantics does not see these cuts. But the proofs of the injectivity only consider cut-free proof-nets.}, it just superficially changes the presentation.That is why we decided in \cite{LPSinjectivity} to avoid introducing explicitely differential nets. In the present paper, we made the opposite choice for the following reason: 
the algorithm leading from the result of a -injective experiment of  to the entire rebuilding of  is done in several steps: in the intermediate steps, we obtain a partial rebuilding where some boxes have been recovered but not all of them; a convenient way to represent this information is the use of ``differential nets with boxes'' (called ``differential PS's'' in the present paper). Now, the differential net representing the result and the proof-net  are both instances of the more general notion of ``differential nets with boxes''.


The rebuilding of the proof-net  is done in  steps, where  is the depth of . We first rebuild the occurrences of the boxes of depth  
(the deepest ones) and next we rebuild the occurrences of the boxes of depth  
and so on... This can be formalized using differential nets (with boxes) as follows: if  is an injective experiment of , then  is the differential net corresponding with  in which only boxes of depth  are expanded,\footnote{Boxes \emph{of} depth  are boxes whose content is a proof-net of depth ; the reader should not confuse \emph{boxes of depth } with \emph{boxes at depth }.} so  is (essentially) the same as the result of the experiment and ; the first step of the algorithm builds  from , the second step builds  from , and so on... We thus reduced the problem of the injectivity to the problem of rebuilding  from  for any -injective experiment  ( big enough).
\begin{comment}
The algorithm building  from  is presented in Definitions~\ref{defin: the algorithm (a)} and \ref{defin: the algorithm (b)}. Lemma~\ref{lem: we have an algorithm} states that:
\begin{itemize}
\item we do not cheat, i.e. we never use the names of the ports of ;\item and the algorithm is deterministic: at each step we can obtain only one differential net with boxes (up to the names of the ports), so, eventually, only one proof-net.
\end{itemize}
\end{comment}








\begin{definition}\label{defin: Taylor}
Let  be a -PS of depth . Let  be a pseudo-experiment of . Let . 
We define, by induction on , a differential -PS          of depth   s.t. 
 
and    as follows: we set   ;
\begin{itemize}
\item   and  ;
\item p \in \portsatzero{R}p = (o_1, e_1) : p'o_1 \in \boxesatzerogeq{R}{i}
\item  is the extension of  that associates with each , where , the port w' \in \mathcal{W}_{e_1, i}t_{e_1, i}(w') \notin \conclusionscirc{B_R(o_1)}w' \in \mathcal{W}_{e_1, i}t_{e_1, i}(w') \in \conclusionscirc{B_R(o_1)}w' \in \conclusionsnotcirc{B_R(o_1)}
\item 
\item  \mbox{ ;} 
\item o \in \boxesatzerosmaller{R}{i}o = (o_1, e_1):o'o_1 \in \boxesatzerogeq{R}{i}
\item  is the extension of  that associates with each , where , the function b_{e_1, i}(o')(p) \notin \conclusionscirc{B_R(o_1)}b_{e_1, i}(o')(p) \in \conclusionscirc{B_R(o_1)}
\end{itemize}
\end{definition}


\begin{figure*}\pictfive{T_f_0_csl}
\caption{}
\label{fig: T(f)[0]}
\end{figure*}

\begin{figure*}\picteight{T_f_1_bis}
\caption{}
\label{fig: T(f)[1]}
\end{figure*}


\begin{example}
If  is a pseudo-experiment of the proof-net  of Figure~\ref{fig: new_example} with  like in Example~\ref{example: pseudo}, then Figures~\ref{fig: T(f)[0]} and~\ref{fig: T(f)[1]} represent respectively  and .
\end{example}



The injectivity of the relational semantics for differential PS's of depth  is trivial (one can proceed by induction on the cardinality of the set of ports). Since , one can easily identify the result  of an experiment  with the differential net :

\begin{lem}\label{lem: taylor expansion}
Let  and  be two -PS's such that . Let  be an experiment of  and let  be an experiment of  such that . Then .
\end{lem}


Now, the following fact shows that 
if we are able to recover  from , then we are done.


\begin{fact}\label{fact: Taylor[i] with i large}
Let  be a -PS. Let  be a pseudo-experiment of . Then .
\end{fact}

\begin{comment}
\begin{proof}
Just notice that  and .
\end{proof}
\end{comment}


If  is a -injective experiment of , then, for any , there exists a bijection  such that, for any , we have  . In Subsection~\ref{subsection: outline}, we will show how to recover   from . There are two kinds of boxes of  at depth : the ``new'' boxes of depth  and the boxes of depth , which are the ``old'' boxes (i.e. that already were in ) that do not go inside some ``new'' box:

\begin{fact}\label{fact : boxes of taylor{e}{i+1} of depth < i}
Let  be a -PS. Let  be a pseudo-experiment of . Let . Then we have 
\begin{itemize}
\item ;
\item ; 
\item and .
\end{itemize}
\end{fact}

The challenge is the rebuilding of the ``new'' boxes at depth  of depth .

\section{From  to }\label{section: one step}

\subsection{The outline of the boxes}\label{subsection: outline}

In this subsection we first show how to recover the set  and, therefore, the set  (Lemma~\ref{lem: M_i}). Next, we show how to determine, from , the set  of ``new'' boxes and, for any such ``new'' box , the set  of exponential ports that are immediately below (Proposition~\ref{prop: critical ports below new boxes}). In particular, we have , where the set  is defined from the set  of the numbers of copies of boxes taken by the pseudo--experiment :






\begin{definition}\label{defin: the algorithm (b)}
Let  be a differential -PS. Let . Let  be a -injective pseudo-experiment of . For any , we define, by induction on ,  and  as follows. 
We set  and we write  in base : ; we set  .


For any , we set .
\end{definition}

Notice that all the sets  and  can be computed from , since
we have   .

\begin{example}\label{example: N_i(e)}
If  is a -injective pseudo-experiment as in Example~\ref{example: pseudo}, then . We have , hence  and . We have , hence  and .  
\end{example}


The following lemma shows that, for any -injective pseudo-experiment  of , for any , the function  is actually a bijection  such that, for any ,  we have  . 

\begin{lem}\label{lem: M_i}
Let  be a -PS. 
Let . 
For any -injective pseudo-experiment  of , for any , we have , hence .
\end{lem}

\begin{example}~\label{example: N_i(e) - 2}
(Continuation of Example~\ref{example: N_i(e)}) We thus have  and  with    and    (see Figure~\ref{fig: T(f)[1]}).
\end{example}

The set  of ``critical ports'' is a set of exponential ports that will play a crucial role in our algorithm.



\begin{definition}\label{definition: critical}
Let  be a differential -PS. Let . For any , we define the sequence  as follows: . For any , we set   and, for any , we set   . 
\end{definition}

\begin{example}
We have  and , where  is the PS of Figure~\ref{fig: T(f)[1]}. So we have .
\end{example}



Critical ports are defined by their arities. We show that they are exponential ports that are immediately below the ``new'' boxes:

\begin{prop}\label{prop: critical ports below new boxes}
Let  be a -PS. 
Let . Let  be a -injective pseudo-experiment of  and let . Then we have . Furthermore, for any , we have  and, if , then there exist  and  such that . In particular, we have .
\end{prop}

\begin{comment}
The proof is in four steps. Step~\ref{item: step1} is used in Step~\ref{item: step2}. Step~\ref{item: step2} is used in Steps~\ref{item: step3} and~\ref{item: step4}. Steps~\ref{item: step3} and~\ref{item: step4} together show the statement of the proposition.
\begin{enumerate}
\item\label{item: step1} We first prove that, for any , for any , for any , we have 

 
For any , we have

So, by Lemma~\ref{lem: M_i}, for any , we have .
\item\label{item: step2} Second we prove, by induction on , that
\begin{itemize}
\item for any , there exists  such that  and
\begin{itemize}
\item either ; moreover, in this case,  and ;
\item or there exist  and  such that ; moreover, in this case, there exists  such that  and .
\end{itemize}
\item and, for any  such that , we have :
\end{itemize}

\begin{itemize}
\item If , then  and, since , we have .
\item If  and , then, by Lemma~\ref{lem: M_i}, there exists a unique  such that :
\begin{itemize}
\item either : in this case,  is a singleton, hence ; now,
\begin{itemize}
\item for any , for any , we have 
, hence, by induction hypothesis, 
, so, for any , we have ;
\item and, by Lemma~\ref{lem: arity{taylor{e}{i}}(c) for c at depth 0}, we have 
;
\end{itemize}
\item or there exist  and  such that : apply~\ref{item: step1}.\end{itemize}
\item If  and , then:
\begin{itemize}
\item if there exist ,  and  such that , then , hence, by induction hypothesis, ;
\item if , then, by Lemma~\ref{lem: arity{taylor{e}{i}}(c) for c at depth 0}, there exists  such that , hence .
\end{itemize}
\end{itemize}
\item\label{item: step3} We prove, by induction on , that, for any , there exists  such that . Let . We distinguish between two cases:
\begin{itemize}
\item : let  such that ; by~\ref{item: step2}, we have ;
\item there exist ,  and  such that : 
we have  for some  
(by induction hypothesis); by Lemma~\ref{lem: M_i}, there exists  such that 
 and we have ; by~\ref{item: step2}, there exists  such that  and ; by Lemma~\ref{lem: M_i}, we have .
\end{itemize}

We showed that  and, for any , we have .
\item\label{item: step4} Finally we prove, by induction on , that, for any , we have . 
If , then . If  and , then, by~\ref{item: step2}, we are in one of the two following cases:
\begin{itemize}
\item there exists  such that : by Lemma~\ref{lem: arity{taylor{e}{i}}(c) for c at depth 0}, we have ;
\item there exist ,  and  such that : by Lemma~\ref{lem: arity{taylor{e}{i}}(c) for c at depth 0}, we have .
\end{itemize}
\end{enumerate}
\end{proof}
\end{comment}



\begin{example}
(Continuation of Example~\ref{example: N_i(e) - 2}) We thus have  ; and indeed  and  are the boxes of depth  at depth  of  (see Figure~\ref{fig: new_example}). Moreover we have  and ; and indeed, in Figure~\ref{fig: new_example}, we have    and   .
\end{example}

\begin{comment}
Notice, thanks to Proposition~\ref{prop: critical ports below new boxes}, that the set  does not depend on  and ; it is the set  defined as follows:

\begin{definition}
Let  be a -PS. Let .  Let . We define, by induction on  the set  as follows: 
  \\ 
\end{definition}
\end{comment}

As the following example shows, the information we obtain is already strong, but not strong enough.

\begin{example}
The PS's ,  and  of Figures~\ref{fig: R_1}, \ref{fig: R_2} and \ref{fig: R_3} respectively have the same LPS. But if we know that , then we know that . Still we are not able to distinguish between  and .

\begin{figure}
\centering
\begin{minipage}{0.3\textwidth}
\centering
\pictter{R_1}
\caption{}
\label{fig: R_1}
\end{minipage}\hfill
\begin{minipage}{0.3\textwidth}
\centering
\pictter{R_2}
\caption{}
\label{fig: R_2}
\end{minipage}
\begin{minipage}{0.3\textwidth}
\centering
\pictter{R_3}
\caption{}
\label{fig: R_3}
\end{minipage}
\end{figure}
\end{example}


\subsection{Connected components}

In order to rebuild the content of the boxes, we introduce our notion of \emph{connected component} (Definition~\ref{defin: connected components}), which uses the auxiliary notions of \emph{substructure} (Definition~\ref{defin: substructure}) and \emph{connected substructure} (Definition~\ref{defin: connected substructure}). A differential -PS  is a substructure of a differential -PS  (we write ) if  is obtained from  by erasing some ports and wires. More precisely:

\begin{definition}\label{defin: substructure}
Let  and  be two differential -PS's. Let . We write  to denote that , , , , , , 
,  and . We write  if there exists  such that .  
\end{definition}

\begin{rem}
Let  and . We have  iff .
\end{rem}


\begin{rem}
If  and , then .
\end{rem}

Let us explain with the following example why we sometimes need to erase some wires (so the notion of  is not enough).  

\begin{example}
We need to erase some wires whenever there exist a box  and  such that . Consider, for instance, Figure~\ref{fig: R'}. If  is a -injective pseudo-experiment of , then we want to be able to consider the PS  of Figure~\ref{fig: U} as a substructure of , so we need to erase the wire ; we thus have .
\begin{figure}
\begin{minipage}{0.25\textwidth}
\centering
\pictbis{simple_net_bis}
\caption{PS }\label{fig: no reconstruction of the experiment}
\end{minipage}
\centering
\begin{minipage}{0.17\textwidth}
\centering
\pictnine{S}
\caption{PS }
\label{fig: R'}
\end{minipage}\hfill
\begin{minipage}{0.16\textwidth}
\centering
\pictnine{U}
\caption{PS }
\label{fig: U}
\end{minipage}
\begin{minipage}{0.21\textwidth}
\centering
\pictter{T_e_bis_1}
\caption{PS }
\label{fig: T(e')[1]}
\end{minipage}
\begin{minipage}{0.14\textwidth}
\centering
\pictten{T}
\caption{PS }
\label{fig: T}
\end{minipage}\hfill
\end{figure}
\end{example}

\begin{comment}
\begin{fact}\label{fact: adequate}
Let  be a differential -PS. Let  such that
\begin{enumerate} 
\item 
\item and .
\end{enumerate}
Then there exists a substructure  of  such that  and .
\end{fact}


If the set  satisfies the conditions of the previous fact, then we denote by  the unique substructure  of  such that  and . \end{comment}



\begin{comment}
\begin{proof}
By Fact~\ref{fact: sqsubseteq}, we already know that there exists a differential ground-structure  such that  and . We set , where  is a function that associates with every  the function: 
\end{proof}
\end{comment}




The relation  formalizes the notion of ``connectness" between two ports of  at depth . But be aware that, here, ``connected'' has nothing to do with ``connected'' in the sense of \cite{LPSinjectivity}: here, any two doors of the same box are always ``connected''.

\begin{definition}
Let  be a differential -PS. We define the binary relation  on  as follows: for any , we have  iff  and  or  and  or .
\end{definition}

\begin{definition}\label{defin: connected substructure}
Let  and  be two differential -PS's. Let  such that . We write  if, for any , there exists a finite sequence  of elements of  such that ,  and, for any , we have  and .
\end{definition}

\begin{rem}
If ,  and , then .
\end{rem}

The sets  of ``components  of  above  and  that are connected \emph{via} other ports than  and such that '' will play a crucial role in the algorithm of the rebuilding of  from . The reader already knows that, here, ``connected'' has nothing to do with the ``connected proof-nets" of \cite{LPSinjectivity}: there, the crucial tool used was rather the ``bridges'' that put together two doors of the same copy of some box only if they are connected in the LPS of the proof-net.

\begin{definition}\label{defin: connected components}
Let . 
Let  be a differential -PS. Let  and . We set 
. 
We write also  instead of  and .
\end{definition}

\begin{comment}
\begin{rem}
We have .
\end{rem}

\begin{rem}\label{remark: nonrivialconnected}
If  and , then .
\end{rem}
\end{comment}

\begin{comment}
\begin{remark}
For any -PS , for any , we have .
\end{remark}
\end{comment}

A port at depth  of  that is not in  cannot belong to two different components:

\begin{fact}\label{fact: two components with a common port}
Let . Let  be a differential -PS. Let . Let  such that . Then .
\end{fact}

\begin{comment}
\begin{proof}
By Remark~\ref{remark: substructures characterized by ports}, it is enough to prove .

Let  and let . There exists a finite sequence  of elements of  such that
,  and\begin{itemize}
\item for any , we have ;
\item for any , we have .
\end{itemize}
We prove, by induction on , that, for any , we have . By assumption, we have . Now, assume that  and . We have , hence . But we have  and , so .
\end{proof}
\end{comment}

\begin{example}
We have  and , where  is the PS of Figure~\ref{fig: T(f)[1]}.
\end{example}

\begin{comment}
\begin{fact}\label{fact: cosize of connected}
Let . Let  be a differential -PS. Let . Let . Let . Then .
\end{fact}
\end{comment}

The operator  glues together several -PS's that share only -conclusions:

\begin{definition}
Let  be a set of -PS's. We say that  is \emph{gluable} if, for any 
 s.t. , we have . 
If  is gluable, then  is the -PS such that  obtained by glueing all the elements of .
\end{definition}



The set  (for  big enough) is an alternative way to describe a -PS :

\begin{fact}\label{fact: connected components}
Let  be a -PS. Let . We have .
\end{fact}

\begin{comment}
\begin{proof}
We prove, by induction on , that, for any , there exists  such that . If , then we set : since , we cannot have , hence .
\end{proof}
\end{comment}

\begin{comment}
\begin{definition}
Let  be a differential -PS. Let  and let . We set 

\end{definition}

\begin{example}
We have , where  is the PS of Figure~\ref{fig: taylor}.
\end{example}

\begin{fact}\label{fact: pouvoir definir G_{k, J}}
Let  be a differential -PS and let . We have
\begin{itemize}
\item 
\item and .
\end{itemize}
\end{fact}

So, by Fact~\ref{fact: sqsubseteq}, we can set .

\begin{example}
The differential ground-structure , where  is the PS of Figure~\ref{fig: taylor}, is represented in Figure~\ref{fig: G}.
\end{example}
\end{comment}


\begin{comment}
\begin{figure*}[!t]
\pictter{G}
\caption{, where  is the PS of Figure~\ref{fig: taylor}}
\label{fig: G}
\end{figure*}
\end{comment}



Definition~\ref{defin: overline} allows to formalize the operation of ``putting a connected component inside a box'', which will be useful for building the boxes of depth   of : from some \emph{boxable} differential -PS , we build a -PS  such that, for some , there exists  such that .




\begin{definition}\label{defin: overline}Let  be a differential -PS. If ,  and , then one says that  is \emph{boxable} and we define a -PS  s.t. ,  and  as follows:
\begin{itemize}
\item ;
\item 
\item p \in \wiresatzero{R}
\item ; ; ; .
\end{itemize}
If  is a set of boxable differential -PS's, 
then we set .
\end{definition}

In the proof of the following proposition, we finally describe the complete algorithm leading from  to .
 Informally: for every , for every equivalence class , if  (with ), then we remove  elements of  from  and we put  such elements inside the (new) box  of depth . For every , the set  of the proof is the union of the sets of such  elements for all the equivalence classes .


\begin{prop}\label{prop: from i to i+1}
Let  and  be two PS's. Let , , , . Let  be a -injective pseudo-experiment of  and let  be a -injective pseudo-experiment of  s.t. . Then, for any , we have .
\end{prop}

\begin{proof}
(Sketch) By induction on . We assume that . We set  and .

Let . There is a bijection  such that, for any ,  we have  . 
For any , we set  and . We set . For any , we define  as follows: . We set  . 
For any , we are given  such that, for any , we have . Let  be some differential PS such that , where  is the unique  s.t. , and:
\begin{itemize}
\item 
\item for any , we have 
;
\item 
\item for any , we have  and 
\item for any , there exists  such that \\ q \in \conclusionscirc{B_{S'}(\cod(j))}q \in \conclusionsnotcirc{B_{S'}(\cod(j))}
\end{itemize}
Then one can show that .
\end{proof}



\begin{example}
Consider Figure~\ref{fig: T(f)[1]}. We set . We have . We set . Let  be the -PS of Figure~\ref{fig: T}. We have . We have . We set . The -PS  is the -PS of depth  that consists of only one port: the port  labelled by . There exists  such that  defined by  (see Figure~\ref{fig: new_example}).
\end{example}

\subsection{Injectivity}

\begin{theorem}\label{thm: injectivity}
Let  and  be two PS's s.t. . If , then .
\end{theorem}

\begin{proof}
We set  . 
For any , there exist a -injective experiment  of  and a -injective experiment  of \footnote{This is not necessary true for the multiset based coherent semantics.} such that . By Lemma~\ref{lem: taylor expansion}, we have . Therefore if , then, by Proposition~\ref{prop: from i to i+1}, we have . Now, by Fact~\ref{fact: Taylor[i] with i large}, we have .
\end{proof}

\begin{rem}
From any -point of  (i.e. the result of some -experiment of ) one can recover  and . This remark shows that for characterizing , two points are enough: a -point of its interpretation, from which one can bound  and , and a -injective point of its interpretation with .
\end{rem}

\begin{rem}\label{remark: with axioms}
If we want to extend our theorem to PS's with axioms, then we assume that the interpretation of any ground type is an infinite set and we consider a -injective experiment  that is ``injective'' in the sense that every atom (the atoms are the elements of the interpretations of the ground types) occurring in  occurs exactly twice.
\end{rem}

\begin{rem}\label{remark: untyped framework}
In an untyped framework with axioms, we need to add the constraint on the injective -injective point one considers to be {-atomic}, i.e. a point of  that cannot be obtained from another point of  by some substitution that is not a renaming. Atomic points are results of atomic experiments (experiments that label axioms with atoms) - the converse does not necessarily hold.
\end{rem}

\bibliography{ll_new}

\onecolumn

\appendix





\section{Proof of Lemma~\ref{lem: M_i}}




\begin{fact}\label{fact: M_0 pairwise disjoint}
Let  be a -PS. 
Let . Let  be a -injective pseudo-experiment of . Then 
\begin{itemize}
\item for any , for any , we have ;
\item and, for any , for any , for any , we have .
\end{itemize}
\end{fact}

\begin{proof}
Let  and let . For any , there exists  such that . So, if there exists  such that , then there exists  such that ; but, by the definition of \emph{-injective experiment} (Definition~\ref{defin: k-injective}), this entails that ; now, since , we obtain a contradiction with the following requirement of the definition of -PS's (Definition~\ref{defin: differential PS}): no  is a sequence.
\end{proof}

\begin{lem}\label{lem: M_1}
Let . 
For any -injective pseudo-experiment  of , we have 
\begin{itemize}
\item if , then 
\item 
\item 
\item .
\end{itemize}
\end{lem}

\begin{proof}
The first item follows from the fact that if , then  and if , then . 

We prove the three other items by induction on . We have:

By the induction hypothesis, for any , for any , we have , hence, by Fact~\ref{fact: M_0 pairwise disjoint},
\begin{itemize}
\item for any , for any , for any , we have ;
\item and, for any , for any , for any , we have .
\end{itemize}
We obtain . Since, by the induction hypothesis, for any , for any , we have , we obtain .
\end{proof}




\begin{lem}\label{lemma: M_j(e)}
Let  be a -PS. 
Let . 
For any -injective pseudo-experiment  of , for any , we have 
\begin{itemize}
\item for any , for any , we have ;
\item for any , for any , for any , we have ;
\item if , then ;
\item ;
\item ;
\item and .
\end{itemize}
\end{lem}

\begin{proof}
By induction on . 

If , then the items hold by Fact~\ref{fact: M_0 pairwise disjoint} and Lemma~\ref{lem: M_1}. Now, we assume that they hold for some .

Since , we have 
\begin{enumerate}
\item for any , for any , we have ;
\item and, for any , for any , for any , we have .
\end{enumerate}
If , then, since, , 
we obtain , hence .

Now, we prove, by by induction on , that:
\begin{itemize}
\item 
\item 
\item and .
\end{itemize}
We have

By the induction hypothesis, for any , for any , we have , hence, by 1. and 2., 
\begin{itemize}
\item for any , for any , we have ;
\item and, for any , for any , for any , we have .
\end{itemize}
We obtain 
\begin{itemize}
\item 
\item and .
\end{itemize}
Since, by the induction hypothesis, for any , for any , we have , we obtain .
\end{proof}

\begin{comment}
\begin{cor}\label{cor: M_i}
Let  be a -PS. Let . Let  be a -injective pseudo-experiment of . Let . Then, for any , for any , we have .
\end{cor}

\begin{proof}
In the case , notice that whenever , we have .
\end{proof}
\end{comment}

\textbf{Proof of Lemma~\ref{lem: M_i}}

\begin{proof}
By induction on . We have

\end{proof}



\section{Proof of Proposition~\ref{prop: critical ports below new boxes}}


We prove Proposition~\ref{prop: critical ports below new boxes} through Lemma~\ref{lem: arity{taylor{e}{i}}(c) for c at depth 0}. 


\begin{definition}
Let  be a -PS. 
For any , for any , we set 
\\
. 



For any , for any , we define, by induction on , a subset  of : we set 

\end{definition}

\begin{rem}
If ,  then  .
\end{rem}


\begin{lem}\label{lem: arity{taylor{e}{i}}(c) for c at depth 0}
Let  be a -PS. Let . Let  be a -injective pseudo-experiment of . Let . Let . Then, for any , we have  if, and only if, there exists  such that . Moreover we have .
\end{lem}

\subsection{Proof of Lemma~\ref{lem: arity{taylor{e}{i}}(c) for c at depth 0}}

\begin{fact}\label{fact: arity}
Let  be a -PS. Let  be a pseudo-experiment of . Let . Let . 
If , then

If , then

\end{fact}

\begin{proof}
For any , for any , we have

hence

\end{proof}

\begin{lem}\label{lemma: m_{k, 0}(c)}
Let  be a -PS. Let . Let  be a -injective pseudo-experiment of . Let . Let . Then .
\end{lem}

\begin{proof}
First, notice that 

Now, we prove the statement by induction on . If , then we just apply . If , then, by induction hypothesis, we have

hence

\end{proof}

\textbf{Proof of Lemma~\ref{lem: arity{taylor{e}{i}}(c) for c at depth 0}:}

\begin{proof}
By Lemma~\ref{lemma: m_{k, 0}(c)}, we already know that . We prove, by induction on , that, for any , we have  if, and only if, there exists  such that . If , then we have

\end{proof}

\subsection{Proof of Proposition~\ref{prop: critical ports below new boxes}}

\begin{proof}
(Sketch) The proof is in four steps, using Lemmas~\ref{lem: M_i} and \ref{lem: arity{taylor{e}{i}}(c) for c at depth 0}. \begin{enumerate}
\item\label{item: step1} We first prove that, for any , for any , for any , we have  .
\item\label{item: step2} Second we prove, by induction on , that
\begin{itemize}
\item for any , there exists  such that  and
\begin{itemize}
\item either ; moreover, in this case,  and ;
\item or there exist  and  such that  \mbox{;}  moreover, in this case, there exists  such that  and  
      ;
\end{itemize}
\item and, for any  such that , we have .
\end{itemize}
\item\label{item: step3} We prove, by induction on , that, for any , there exists  such that . \item\label{item: step4} Finally we prove, by induction on , that, for any , we have . 
\end{enumerate}
\end{proof}



\section{Rebuilding the boxes: Proposition~\ref{prop: rebuilding boxes}}\label{appendix: rebuilding boxes}


\begin{prop}\label{prop: rebuilding boxes}
Let  be a -PS. Let . 
Let  be a -injective pseudo-experiment of . Let . Let . 
We set . 
For any , let  such that   . 
Let  such that, 
for any ,    . 
Then there exists     such that 
\\ q \in \conclusionscirc{B_{\taylor{e}{i+1}}(\cod_{e, i}(j_0))}q \in \conclusionsnotcirc{B_{\taylor{e}{i+1}}(\cod_{e, i}(j_0))}
\end{prop}


The proof of Proposition~\ref{prop: rebuilding boxes} uses Proposition~\ref{prop: crucial}, which justifies that in the algorithm leading from  to , for every , for every equivalence class , if  (with ), then we remove  elements of  from . Its proof is quite technical, we just state the right induction hypothesis and give some remarks. 

\begin{definition}
Let  be a -PS. Let . Let . Let  be a -injective pseudo-experiment of  and let . Let . Let . We denote by  the following subset of : 

\end{definition}



\begin{fact}\label{fact: adequate_extended}
Let  be a differential -PS. Let  and let  such that
\begin{enumerate} 
\item 
\item 
\item and .
\end{enumerate}
Then there exists a substructure  of  such that  and .
\end{fact}

\begin{proof}
We set  and  with:
\begin{itemize}
\item 
\item 
\item 
\item 
\item 
\item 
\item 
\item and .
\end{itemize}
\end{proof}

If the pair  satisfies the conditions of the previous fact, then we say that the pair  is \emph{adequate} with respect to  and we denote by  the unique substructure  of  such that  and . We have .


\begin{fact}\label{fact: from taylor{e1}{i} to taylor{e}{i}}
Let  be a -PS. Let . Let  be a -injective pseudo-experiment of . Let . 
Let . Let . 
Let  and  such that the pair  is adequate with respect to . Then the pair  is adequate with respect to . Moreover we have 
\begin{itemize}
\item ;
\item ;
\item and, for any , .
\end{itemize}
\end{fact}

This fact allows the following definition:

\begin{definition}\label{defin: T[R, e, i]}
Let  be a -PS. Let  be a pseudo-experiment of . Let . 
Let . Let . 
Let . We set .
\end{definition}

Before of the proof of Proposition~\ref{prop: crucial}, notice that the proof of Proposition~\ref{prop: critical ports below new boxes} actually proves something more than its statement: it proves also that, for any  such that , there exist  and  such that  and . From now on, whenever we refer to Proposition~\ref{prop: critical ports below new boxes}, we refer to the statement thus completed.

\begin{prop}\label{prop: crucial}
Let  be a -PS. Let . 
Let  be a -injective pseudo-experiment of . Let . 
Let . We set 

and

Let  such that  
and . Then 
\begin{itemize}
\item 
\item 
\item and .
\end{itemize}
Moreover, if , then .
\end{prop}

\begin{proof}
By induction on . We distinguish between two cases:
\begin{enumerate}
\item Case : by Proposition~\ref{prop: critical ports below new boxes}, there exist ,  and  such that . 
For any , we set  \mbox{}  and  we have , hence ; therefore . 
\begin{itemize}
\item Now, notice that if there exists  such that , then: 
 we thus have  and ; we obtain  and . In the same way, we have
 we thus have  and ; we obtain  and .
\item Otherwise, for any , we have , hence . So, 

and

We apply the induction hypothesis and we obtain:
\begin{itemize}
\item 
\item 
\item and .
\end{itemize}
\end{itemize}
\item Case : Let . We set . We set . We have . Let . 

Notice that, for any  such that , we have .

Now, for any   such that , we have . Indeed assume that  such that  and ; let  and let ; there exist  such that , ,  and . Let . We have , which is contradictory.



We thus obtain . 
We distinguish between two cases:
\begin{itemize}
\item : for any , we set .
\item : let  such that  and ; let  such that

and

By induction hypothesis, we have
\begin{itemize}
\item 
\item 
\item and .
\end{itemize}
For any   such that , we have , hence  and . 
\end{itemize}
Finally:

and

Therefore  and . Moreover . Lastly, for any , we have .
\end{enumerate}
\end{proof}


We need a variant of the notion of equivalence denoted by :

\begin{definition}
Let  be a -PS. Let . For any -injective pseudo-experiment  of , for any , for any , for any , for any -PS  such that , we write  if  such that 
 and  if there exists  such that .
\end{definition}


\textbf{Sketch of the proof of Proposition~\ref{prop: rebuilding boxes}}


\begin{proof}(Sketch) 
We set . Let  such that, for any , we have . 
For any , there exist  and  such that, for any , we have   q \in \conclusionscirc{B_{\taylor{e}{i+1}}(\cod_{e, i}(j_0))}q \in \conclusionsnotcirc{B_{\taylor{e}{i+1}}(\cod_{e, i}(j_0))} We check the existence of such  and  by induction on :
\begin{itemize}
\item In the case , for any , we have , hence one can set  for some .
\item In the case  for some  and some , by induction hypothesis, there exists  and  such that, for any , we have   q \in \conclusionscirc{B_{\taylor{e}{i+1}}(\cod_{e, i}(j_0))}q \in \conclusionsnotcirc{B_{\taylor{e}{i+1}}(\cod_{e, i}(j_0))} We set . Moreover we set 
\groundof{\varphi'_{\mathcal{V}}}(p) \in \conclusionscirc{B_R(o_1)} and .
\end{itemize}

We set .

Let us show that there exists a bijection : 
there exist a partition  of  and a bijection  such that, for any  such that , we have , where  is the integer . We have . By Proposition~\ref{prop: crucial}, we have . We thus have .

For any , there exists  such that, for any ,  q \in \conclusionscirc{B_{\taylor{e}{i+1}}(\cod_{e, i}(j_0))}q \in \conclusionsnotcirc{B_{\taylor{e}{i+1}}(\cod_{e, i}(j_0))}
We have: for any , there exists  such that, for any , ; 
the sets  and  are gluable; 
there exists  such that, 
for any , we have  q \in \conclusionscirc{B_{\taylor{e}{i+1}}(\cod_{e, i}(j_0))}q \in \conclusionsnotcirc{B_{\taylor{e}{i+1}}(\cod_{e, i}(j_0))}
But we have 

and .
\end{proof}

\section{Proposition~\ref{prop: from i to i+1}}

In the sketch of the proof of Proposition~\ref{prop: from i to i+1}, we gave the complete formalized algorithm leading from  to  (up to the names of the ports). We only gave some arguments in favour of its correctness. In Section~\ref{appendix: rebuilding boxes}, we already stated Proposition~\ref{prop: crucial}. This proposition is actually used in the proof of Proposition~\ref{prop: rebuilding boxes} and again in Proposition~\ref{prop: from i to i+1}. Subsections~\ref{subsection: deterministic}, ~\ref{subsection: cond_sup_1} and~\ref{subsection: cond_sup_2} give some more arguments in favour of the correctness of the algorithm. 

\subsection{The algorithm is deterministic}\label{subsection: deterministic}

We should show that, if  is some differential PS that enjoys the conditions given on  in the proof of Proposition~\ref{prop: from i to i+1}, then . The proof is quite easy but tedious.

\subsection{}\label{subsection: cond_sup_1}

As said in its informal description before the statement of the proposition, in order to obtain the differential PS , we consider the set  and we remove some of these elements and put some other of these elements inside new boxes. This means in particular that we keep all the ports that are not caught by some element of , what is formally stated by:

with 


It is essentially what Lemma~\ref{lem: P_k subseteq ports of taylor} says. 



\begin{lem}\label{lem: P_k subseteq ports of taylor}
Let  be a -PS. Let . 
Let  be a -injective pseudo-experiment of . Let . For any , there exist  and  such that, for any , we have .
\end{lem}

\begin{proof}
(Sketch) 
By induction on :

If , then .

If , then we distinguish between two cases:
\begin{itemize}
\item In the case there exist ,  and  such that , 
there exists  such that  and, for any , we have . 
Now, by Proposition~\ref{prop: critical ports below new boxes}, we have  with .
\item In the case there exist ,  and  such that , by induction hypothesis, there exists  and  such that  and, for any , we have . 

By Fact~\ref{fact: from taylor{e1}{i} to taylor{e}{i}}, we can set . By 
Proposition~\ref{prop: critical ports below new boxes}, we have .


Let . Notice that . We have 

Futhermore, notice that
we have ; so we have:
\begin{itemize}
\item if , then 

\item if , then

\end{itemize}
hence, in the two cases, .

Lastly, let . By Fact~\ref{fact: from taylor{e1}{i} to taylor{e}{i}}, we can distinguish between the two following cases:
\begin{itemize}
\item there exists  such that : by Proposition~\ref{prop: critical ports below new boxes}, we have ;
\item : since, for any , we have , we obtain , hence ; by Proposition~\ref{prop: critical ports below new boxes}, we obtain .
\end{itemize}
\end{itemize}
\end{proof}

\subsection{}\label{subsection: cond_sup_2}

We said also that we do not add new connected components (the new boxes are never caught by the elements of the ). It is the content of Lemma~\ref{lem: second condition}.

\begin{fact}\label{fact: coh[i]}
Let  be a -PS. Let . 
Let  be a -injective pseudo-experiment of . Let . 
Let . 
Let  such that .
 Then we have .
\end{fact}

\begin{proof}
We distinguish between four cases:
\begin{itemize}
\item  and : we have ;
\item  and : we have ;
\item : we have ;
\item there exists  such that : by Fact~\ref{fact : boxes of taylor{e}{i+1} of depth < i}, we obtain .
\end{itemize}
\end{proof}

\begin{fact}\label{fact: coh[i+1]}
Let  be a -PS. Let . 
Let  be a -injective pseudo-experiment of . Let . 
Let . 
Let  such that .
 Then we have  or .
\end{fact}

\begin{proof}
We distinguish between three cases:
\begin{itemize}
\item ( and ) or ( and ) or : we have ;
\item there exists  such that : by  Fact~\ref{fact : boxes of taylor{e}{i+1} of depth < i}, we have ;
\item there exists  such that : by  Proposition~\ref{prop: critical ports below new boxes}, we obtain .
\end{itemize}
\end{proof}

\begin{fact}\label{fact: critical ports for M => crtical ports for j}
Let  be a -PS. Let . 
Let  be a -injective pseudo-experiment of . Let . Let . Let  such that . Then we have .
\end{fact}

\begin{proof}
Let . By Proposition~\ref{prop: critical ports below new boxes}, there exists  such that . We have , hence . But, since , we have .
\end{proof}

\begin{lem}\label{lem: second condition}
Let  be a -PS. Let . 
Let  be a -injective pseudo-experiment of . Let . 
For any , we have .
\end{lem}

\begin{proof}
(Sketch) 
Let  and . 
By assumption we have .

First, notice that . Indeed, assume that . Let . For any  such that , we have . Indeed, consider the case : we have ; by Proposition~\ref{prop: critical ports below new boxes}, we have . This shows that , which contradicts . 

Now, using Fact~\ref{fact : boxes of taylor{e}{i+1} of depth < i}, we could easily show that . 
So, by Fact~\ref{fact: coh[i+1]} and Fact~\ref{fact: critical ports for M => crtical ports for j}, we have . 
Finally, by Fact~\ref{fact: coh[i]}, we obtain .
\end{proof}

\section{With axioms (Remark~\ref{remark: with axioms})}

With axioms, we need to slightly modify Definition~\ref{defin: experiment induces pseudo-experiment}, since different experiments can induce the same pseudo-experiment:

\begin{definition}\label{defin: experiment induces pseudo-experiment - with axioms}
Given an experiment  of some differential -PS , we define, by induction on , a pseudo-experiment  of  as follows:  and 
 for any 
\end{definition}

Notice that, if there is no axiom, Definitions~\ref{defin: experiment induces pseudo-experiment} and~\ref{defin: experiment induces pseudo-experiment - with axioms} induce the same pseudo-experiment  for an experiment .

\section{Untyped framework (Remark~\ref{remark: untyped framework})}

Since there is no type, we define (differential) ground-structures \emph{via} the auxiliary definition of \emph{(differential) pre-ground-structures}:

\begin{definition}\label{defin: diff pre-ground-structure}
A \emph{differential pre-ground-structure} is a -tuple , where
\begin{itemize}
\item  is a finite set; the elements of  are the \emph{ports of };
\item  is a subset of ; the elements of  are the \emph{wires of };
\item  is a function  such that ; the element  of  is the \emph{label of  in };
\item  is a function  such that, for any , we have
\begin{itemize}
\item ;
\item ;
\end{itemize}
if , then  is a \emph{premise of };
\item  is a subset of  such that, for any  such that , we have ; if  such that , we say that  is a \emph{left premise of ;}
\item and  is a partition of  such that, for any , ; the elements of  are the \emph{axioms of }.
\end{itemize}
We set , , , , ,  and . The elements of  are the \emph{conclusions of }.

We set , , 
, , , 
, ,  and .


A \emph{pre-ground-structure} is a differential pre-ground-structure  such that .

A \emph{differential ground-structure} (resp. a \emph{ground-structure}) is a differential pre-ground structure (resp. a pre-ground structure)  such that the reflexive transitive closure  of the binary relation  on  defined by  is antisymmetric.
\end{definition}

For the semantics of PS's, we are given a set  that does not contain any couple nor any -tuple and such that . We define, by induction on , the set  for any :
\begin{itemize}
\item 
\item 
\end{itemize}
We set .

Definition~\ref{defin: experiment in an untyped framework} is an adaptation of Definition~\ref{defin: experiment} in an untyped framework.

\begin{definition}
For any , we define  as follows:
\begin{itemize}
\item if  and , then ;
\item if  with , then ;
\item if  with  and , then ;
\item if  with  and , then ;
\end{itemize}
where  and .
\end{definition}

\begin{definition}\label{defin: experiment in an untyped framework}
For any differential -PS , we define, by induction on  the set of \emph{experiments of }: it is is the set of triples , where  is a function  and  is a function which associates to every  a finite multiset of experiments of  such that
\begin{itemize}
\item for any , we have ,  for some ;
\item for any  (resp. ), for any  such that ,  and , we have  (resp. );
\item for any  (resp. ), we have  (resp. );
\item for any ,
we have p \in \conclusionscirc{R}p \in \portsatzerooftype{\contr}{R}p \in \portsatzerooftype{\cod}{R}
where 

\end{itemize}
For any experiment , we set  and .

For any differential -PS , we set .
\end{definition}

\begin{definition}\label{defin: injective}
Let . We say that  is \emph{injective} if, for every , there are at most two occurrences of  in . 

For any set , for any function , we say that  is \emph{injective} if  is injective.

An experiment  of a differential -PS  is said to be \emph{injective} if  is injective.
\end{definition}

\begin{definition}
Let . For any , we define  as follows:
\begin{itemize}
\item if , then  and ;
\item if  and , then ;
\item if  and , then .
\end{itemize}
For any set , for any function , we define a function  by setting:  for any .
\end{definition}

\begin{rem}
For any functions , for any function , we have .
\end{rem}

\begin{definition}
Let  be a differential -PS. Let  be an experiment of . Let . We define, by induction of , an experiment  of  by setting 
\begin{itemize}
\item 
\item  for any .
\end{itemize}
\end{definition}

Since we deal with untyped proof-nets, we cannot assume that the proof-nets are -expanded and that experiments label the axioms only by atoms. That is why we introduce the notion of \emph{atomic experiment}:

\begin{definition}
For any differential -PS , we define, by induction on , the set of \emph{atomic experiments of }: it is the set of experiments  of  such that
\begin{itemize}
\item for any , we have ;
\item and, for any , the multiset  is a multiset of atomic experiments of .
\end{itemize}
\end{definition}

\begin{fact}\label{fact: an experiment of a flat-pS induces an experiment of differential flat-PS}
Let  be a -PS. Let  be an experiment of . If  is atomic, then  is atomic.
\end{fact}

\begin{proof}
By induction on .
\end{proof}

\begin{definition}
Let  be a set. Let . 
Let , we say that  is \emph{-atomic} if we have 

where  is the set of atoms occurring in .

We denote by  the subset of  consisting of the -atomic elements of .
\end{definition}

For any PS , any -atomic injective point is the result of some atomic experiment of :

\begin{fact}\label{fact: from point to experiment}
Let  be a -PS. Let  injective. Then there exists an atomic experiment  of  such that .
\end{fact}

\begin{proof}
We prove, by induction on  lexicographically ordered, that, for any non-atomic injective experiment  of , there exist an experiment  of , a function  such that  and  such that .
\end{proof}

The converse does not necessarily hold: for some PS's , there are results of atomic injective experiments of  that are not -atomic. Indeed, consider Figure~\ref{fig: atomic experiments}. 
\begin{figure}[!t]
\centering
\pictfour{atomic}
\caption{PS }
\label{fig: atomic experiments}
\end{figure}
There exists an atomic injective experiment  of  such that 
\begin{itemize}
\item , 
\item 
\item and ,
\end{itemize}
where . But  is not in : there exists an atomic injective experiment  of  such that 
\begin{itemize}
\item , 
\item  
\item and ;
\end{itemize}
we set \gamma \in A \setminus \{ \gamma_8 \}\gamma = \gamma_8 - we have .

But it does not matter, because there are many enough atomic points:

\begin{fact}\label{fact: atomic are enough}
Let  be a -PS. For any , there exist  and  such that . \end{fact}

\begin{proof}
By induction on , where  is defined for any  as follows: : if , then we can set  and ; if , then there exist a function ,  such that  and  such that , hence . By induction hypothesis, there exist  and  such that . We set : we have .
\end{proof}

\end{document}
