
\appendix

\begin{center} {\Large
\textbf{A polyhedral approach for the Equitable Coloring Problem}}\\[12pt]
Isabel M\'endez-D\'iaz,~~Graciela Nasini,~~Daniel Sever\'in\\[12pt]
\textsc{Online Appendix}
\end{center}

\section{Introduction} 

In this appendix we present sufficient conditions for some valid inequalities related to the \emph{Equitable Coloring Problem}
to be facet-defining inequalities.



All the proofs are based in the same technique, frequently used in the
literature for this kind of results, which is described in the following
remark.

\begin{trem} \label{TECHNIQUE}
Let $\pi^X x + \pi^W w \leq \pi_0$ be a valid inequality for $\ECP$ defining a proper face $\F'$.
In order to prove that $\F'$ is a facet of $\ECP$ we have to show that, given any face $\F = \{ (x,w) \in \ECP ~:~ \lambda^X x + \lambda^W w = \lambda_0\}$ such that $\F' \subset \F$, $\lambda^X x + \lambda^W w = \lambda_0$ can be written as a linear combination of $\pi^X x + \pi^W w = \pi_0$ and the minimal equation system for $\ECP$ given in Theorem (\ref{TDIM}). This last condition becomes equivalent to prove that $(\lambda^X, \lambda^W)$ verifies an equation system of $dim(\ECP)-1$ equalities.
The validity of each equality in the system is derived from the condition 
$\lambda^X x^1 + \lambda^W w^1 = \lambda_0 = \lambda^X x^2 + \lambda^W w^2$ applied on a suitable pair of equitable colorings $(x^1, w^1), (x^2,w^2)$ lying on $\F'$.
\end{trem}

For the sake of simplicity, we directly present the corresponding equation system on $(\lambda^X, \lambda^W)$ and the proposed equitable colorings used to derive each equation, bypassing how to get that equation system from the minimal equation system given in Theorem (\ref{TDIM}) and the inequality at hand.

As we have mentioned in Section \ref{SPOLYT}, we present equitable colorings by using mappings, color classes or binary vectors, according to our convenience.







\subsection{2-rank inequalities}

\begin{tthm} \label{T2RANK1}
Let $G$ be a monotone graph, $S \subset V$ such that $\alpha(S) = 2$ and $j \leq \lceil n/2 \rceil - 1$. If
\begin{enumerate}
\item[(i)] there exists a stable set $H$ of size 3 in $G$ such that:
\begin{itemize}
\item if $n$ is odd, the complement of $G-H$ has a perfect matching $M$ and both endpoints of some edge
of $M$ belong to $S$,
\item if $n$ is even, there exists another stable set $H'$ of size 3 in $G$ such that
$H \cap H' = \varnothing$, the complement of $G - (H \cup H')$ has a perfect matching $M$, both
endpoints of some edge of $M$ belong to $S$ and there exist vertices $h \in H$, $h' \in H'$ not
adjacent each other, 
\end{itemize}
\item[(ii)] for all $v \in V \backslash S$, there exist different vertices $s, s' \in S$
and a stable set $H_v = \{v, s, s'\}$ in $G$ such that:
\begin{itemize}
\item if $n$ is odd, the complement of $G-H_v$ has a perfect matching,
\item if $n$ is even, there exists another stable set $H'_v$ of size 3 in $G$ such that
$H_v \cap H'_v = \varnothing$ and the complement of $G - (H_v \cup H'_v)$ has a perfect matching,
\end{itemize}
\item[(iii)] for all $k$ such that $\max \{\chi_{eq}, j\} \leq k \leq \lceil n/2 \rceil - 2$, there exists a $k$-eqcol
where two vertices of $S$ share the same color,
\end{enumerate}
then the $(S,j)$-rank inequality, \ie
\begin{equation} \label{R2RANK1}
\sum_{v \in S} x_{vj} + \sum_{v \in V} x_{vn-1} \leq 2 w_j + w_{n-1} - w_n,
\end{equation}
defines a facet of $\ECP$.
\end{tthm}
\begin{proof}
Let $\F'$ be the face of $\ECP$ defined by (\ref{R2RANK1}) and
$\F = \{ (x,w) \in \ECP ~:~ \lambda^X x + \lambda^W w = \lambda_0\}$ be a face such that $\F' \subset \F$.
According to Remark \ref{TECHNIQUE}, we have to prove that $(\lambda^X, \lambda^W)$ verifies the following equation system: 
\begin{enumerate}
\item[(a)] $\lambda^X_{vj} = \lambda^X_{vn} + \lambda^W_n,~~~ \forall~ v \in S$.
\item[(b)] $\lambda^X_{vn-1} = \lambda^X_{vn} + \lambda^W_n,~~~ \forall~ v \in V$.
\item[(c)] $\lambda^X_{vk} = \lambda^X_{vn-1} + \lambda^W_{n-1},~~~
                   \forall~ v \in V,~ 1 \leq k \leq n-2,~ k \neq j$.
\item[(d)] $\lambda^X_{vj} = \lambda^X_{vn-1} + \lambda^W_{n-1},~~~
                                            \forall~ v \in V \backslash S$.
\item[(e)] $\lambda^W_k = 0,~~~ \forall~ \chi_{eq} + 1 \leq k \leq n-2,~ k \neq j$.
\item[(f)] If $j \geq \chi_{eq} + 1$ then $\lambda^W_j = - 2 \lambda^W_{n-1}$.
\end{enumerate}
We present pairs of equitable colorings lying on $\F'$ that allow us to
prove the validity of each equation in the previous system.
\begin{enumerate}
\item[(a)] Let $s, s' \in S$ be non adjacent vertices.\\
\textbf{Case $v = s$}. Let $c^1$ be a $(n-1)$-eqcol such that $c^1(s) = c^1(s') = j$ and
$c^2 = intro(c^1,s)$. Then, $\lambda^X_{sj} = \lambda^X_{sn} + \lambda^W_n$.\\
\textbf{Case $v \neq s$}. 
Let $c^1$ be a $n$-eqcol such that $c^1(v) = j$, $c^1(s) = n$ and $c^2 = swap_{j,n}(c^1)$. Then, 
$\lambda^X_{vj} + \lambda^X_{sn} = \lambda^X_{vn} + \lambda^X_{sj}$. Since $\lambda^X_{sj} = \lambda^X_{sn} + \lambda^W_n$, we obtain $\lambda^X_{vj} = \lambda^X_{vn} + \lambda^W_n$.
\item[(b)] \textbf{Case $v \notin S$}. 
By hypothesis (ii), there exist $s, s' \in S$ such that $\{v, s, s'\}$ is a stable set.
Let $c^1$ be a $(n-1)$-eqcol such that $c^1(v) = c^1(s) = n-1$, $c^1(s') = j$ and
$c^2 = intro(c^1,v)$. Therefore, $\lambda^X_{vn-1} = \lambda^X_{vn} + \lambda^W_n$.\\
\textbf{Case $v \in S$ and $|S| = 2$}. By hypothesis (ii), there exist $u \in V \backslash S$ and $v' \in S$ such
that  $\{u, v, v'\}$ is a stable set. Let $c^1$ be a $(n-1)$-eqcol such that $c^1(u) = c^1(v) = n-1$, $c^1(v') = j$
and $c^2 = intro(c^1,v)$. Therefore, $\lambda^X_{vn-1} = \lambda^X_{vn} + \lambda^W_n$.\\
\textbf{Case $v \in S$ and $|S| \geq 3$}. Let $s, s' \in S$ be non adjacent vertices,
$c^1$ be a $(n-1)$-eqcol such that $c^1(s) = c^1(s') = n-1$ and other vertex
of $S$ is painted with color $j$, and $c^2 = intro(c^1,s)$.
Then, $\lambda^X_{sn-1} = \lambda^X_{sn} + \lambda^W_n$ and the condition is proved for the case $v = s$.
If instead $v \neq s$, let $c^1$ be a $n$-eqcol such that $c^1(v) = n-1$, $c^1(s) = n$, other vertex of $S$ is
painted with color $j$ and $c^2 = swap_{n,n-1}(c^1)$. Then, 
$\lambda^X_{vn-1} + \lambda^X_{sn} = \lambda^X_{vn} + \lambda^X_{sn-1}$.
Since $\lambda^X_{sn-1} = \lambda^X_{sn} + \lambda^W_n$, we conclude that $\lambda^X_{vn-1} = \lambda^X_{vn} + \lambda^W_n$.
\item[(c)] Let $H$ and $M$ be the stable set and the matching given by hypothesis (i). Let $s, s' \in S$ be
the endpoints of an edge of $M$ and let $u, u' \in H$.\\
\textbf{Case $v = u$}. Let $c^1$ be a $(n-2)$-eqcol such that $c^1(u) = c^1(u') = k$, $c^1(s) = c^1(s') = j$ and
$c^2 = intro(c^1,u)$. We conclude that $\lambda^X_{uk} = \lambda^X_{un-1} + \lambda^W_{n-1}$.\\
\textbf{Case $v \neq u$}. Let $c^1$ be a $n$-eqcol such that $c^1(u) = k$, $c^1(v) = n-1$, a vertex of $S$ is
painted with color $j$ and $c^2 = swap_{k,n-1}(c^1)$. Then,
$\lambda^X_{u k} + \lambda^X_{v n-1} = \lambda^X_{u n-1} + \lambda^X_{vk}$.
Since $\lambda^X_{uk} = \lambda^X_{un-1} + \lambda^W_{n-1}$, we conclude that
$\lambda^X_{vk} = \lambda^X_{vn-1} + \lambda^W_{n-1}$.
\item[(d)] Let $H_v = \{v, s, s'\}$, $H'_v$ (if $n$ is even) and $M_v$ be the stable sets and the matching given
by hypothesis (ii), and let $H$, $H'$ (if $n$ is even) and $M$ be the stable sets and the matching given by hypothesis
(i). Let $c^1$ be a $(\lceil n/2 \rceil - 1)$-eqcol such that the color class $j$ is $H_v$ and the
remaining color classes are $H'_v$ (if $n$ is even) and the endpoints of edges of $M_v$.
Let $\hat{s}, \hat{s}' \in S$ be the endpoints of an edge of $M$ and
let $c^2$ be a $(\lceil n/2 \rceil - 1)$-eqcol such that the color class $j$ is $\{\hat{s}, \hat{s}'\}$ and the
remaining color classes are $H$, $H'$ (if $n$ is even) and the endpoints of edges of $M$ except
$(\hat{s}, \hat{s}')$. These colorings imply
\[ \lambda^X_{vj} + \lambda^X_{sj} + \lambda^X_{s'j} +
   \sum_{w \in V \backslash \{v, s, s'\}} \lambda^X_{w c^1(w)}
	= \lambda^X_{\hat{s}j} + \lambda^X_{\hat{s}'j} +
   \sum_{w \in V \backslash \{\hat{s}, \hat{s}'\}} \lambda^X_{w c^2(w)}. \]
Applying conditions (a)-(c), this last equality becomes
\[ \lambda^X_{vj} + \sum_{w \in V \backslash \{v\}} \lambda^X_{w n} + (n-3) \lambda^W_{n-1}
+ (n-1) \lambda^W_n = \sum_{w \in V} \lambda^X_{w n} + (n-2) \lambda^W_{n-1} + n \lambda^W_n,\]
giving rise to the desired result.
\item[(e)] Let us observe that from any $k$-eqcol $(x^1,w^1)$ and any $(k-1)$-eqcol
$(x^2,w^2)$ lying on $\F'$ we get $\lambda^X x^1 + \lambda^W_k = \lambda^X x^2$.
Then, applying conditions (a)-(d) yields $\lambda^W_k = 0$.\\
Thus we only need to prove that, for any $r$ such that $\chi_{eq} \leq r \leq n-2$, there exists an $r$-eqcol $c$
lying on $\F'$.\\
\textbf{Case $r < j$}. The existence of $c$ is guaranteed by the monotonicity of $G$.\\
\textbf{Case $j \leq r \leq \lceil n/2 \rceil-2$}. Hypothesis (iii) guarantees the existence of an $r$-eqcol $c'$
where two vertices $s, s' \in S$ satisfy $c'(s) = c'(s')$. Then, $c = swap_{c'(s),j}(c')$ is an $r$-eqcol that lies
on $\F'$.\\
\textbf{Case $r = \lceil n/2 \rceil-1$}. $c$ may be one of the colorings given in condition (d).\\
\textbf{Case $r = \lceil n/2 \rceil$}. Let $H$, $H'$ (if $n$ is even) and $M$ be the stable sets and the matching
given by hypothesis (i). Let $s, s' \in S$ be the endpoints of an edge of $M$ and let $h \in H$, $h' \in H'$ (if $n$
is even) be non adjacent vertices.\\
If $n$ is odd, color classes of $c$ are $\{h\}$, $H \backslash \{h\}$ and the endpoints of edges of $M$ where
$\{s,s'\}$ is the class $j$. If instead $n$ is even, color classes of $c$ are $\{h,h'\}$, $H \backslash \{h\}$,
$H' \backslash \{h'\}$ and the endpoints of edges of $M$ where $\{s,s'\}$ is the class $j$.\\
\textbf{Case $r \geq \lceil n/2 \rceil+1$}. Let us consider the $\lceil n/2 \rceil$-eqcol yielded in the
previous case and let $v_1, v_2$ be vertices sharing a color different from $j$.
In order to generate a $(\lceil n/2 \rceil + 1)$-eqcol $c$, we introduce a new color on $v_1$, \ie $c = intro(c', v_1)$
where $c'$ is the $\lceil n/2 \rceil$-eqcol. By repeating this procedure, we can generate a
$(\lceil n/2 \rceil + 2)$-eqcol and so on.
\item[(f)] Let $c^1$ be a $j$-eqcol such that $c^1(s) = c^1(s') = j$ for some $s, s' \in S$ and
$c^2$ be a $(j-1)$-eqcol (the existence of these colorings is proved above). Hence,
\[ \lambda^X_{sj} + \lambda^X_{s'j} +
   \sum_{v \in V \backslash \{s, s'\}} \lambda^X_{v c^1(v)} + \lambda^W_j
	= \sum_{v \in V} \lambda^X_{v c^2(v)}. \]
Application of conditions (a)-(d) yields $\lambda^W_j = - 2 \lambda^W_{n-1}$.
\end{enumerate}
\end{proof}

Let us present an example where the previous theorem is applied.\\

\noindent \textbf{Example.}
Let $G$ be the graph presented in Figure \ref{minigraph1}. We have that $G$ is monotone and $\chi_{eq}(G) = 5$. 
If $S = \{1,2,\ldots,7\}$, $\alpha(S) = 2$ and $H = \{4, 7, 8\}$ is a stable set such that $\overline{G - H}$ has the perfect matching $\{(1,10), (2,11), (3,5), (6,9)\}$ with $\{3,5\}\subset S$. Moreover, it is not hard to verify that for all $v \in \{8,9,10,11\}$ there exists a stable set $H_v = \{ 4, 7, v\}$ such that $\overline{G - H_v}$ has a perfect matching.
Then, if $1 \leq j \leq 5=\lceil 11/2 \rceil-1$, the $(S,j)$-rank inequality is a facet-defining inequality of $\ECP(G)$.  
\begin{figure}[h]
  \centering
\begin{graph}(4, 2)(-2, -1)
	\roundnode{v1}(-2.3,-0.1)
	\autonodetext{v1}[w]{1}
	\roundnode{v2}(-1.7,0.1)
	\autonodetext{v2}[se]{2}
	\roundnode{v3}(-1.2,0)
	\autonodetext{v3}[n]{3}
	\roundnode{v4}(-1.7528,0.7608)
	\autonodetext{v4}[n]{4}
	\roundnode{v5}(-2.6472,0.4702)
	\autonodetext{v5}[nw]{5}
	\roundnode{v6}(-2.6472,-0.4702)
	\autonodetext{v6}[sw]{6}
	\roundnode{v7}(-1.7528,-0.7608)
	\autonodetext{v7}[s]{7}
	\roundnode{v8}(-0.4,0)
	\autonodetext{v8}[n]{8}
	\roundnode{v9}(0.4,0)
	\autonodetext{v9}[n]{9}
	\roundnode{v10}(1.2,0)
	\autonodetext{v10}[n]{10}
	\roundnode{v11}(2,0)
	\autonodetext{v11}[n]{11}
	\edge{v1}{v2}
	\edge{v1}{v3}
	\edge{v1}{v4}
	\edge{v1}{v5}
	\edge{v1}{v6}
	\edge{v1}{v7}
	\edge{v2}{v3}
	\edge{v2}{v4}
	\edge{v2}{v5}
	\edge{v2}{v6}
	\edge{v2}{v7}
	\edge{v3}{v7}
	\edge{v3}{v4}
	\edge{v4}{v5}
	\edge{v5}{v6}
	\edge{v6}{v7}
	\edge{v3}{v8}
	\edge{v8}{v9}
	\edge{v9}{v10}
	\edge{v10}{v11}
\end{graph}
  \caption{}
  \label{minigraph1}
\end{figure}

\begin{tthm} \label{T2RANK2}
Let $G$ be a monotone graph, $S \subset V$ such that $\alpha(S) = 2$ and $Q = \{ q \in S ~:~ S \subset N[q] \}$.
If $|Q| \geq 2$ and
\begin{enumerate}
\item[(i)] no connected component of the complement of $G[S \backslash Q]$ is bipartite,
\item[(ii)] for all $v \in V \backslash S$ verifying $Q \subset N(v)$, there exist two
vertices $s, s' \in S \backslash Q$ and a stable set $H_v = \{v, s, s'\}$ in $G$ such that:
\begin{itemize}
\item if $n$ is odd, the complement of $G-H_v$ has a perfect matching,
\item if $n$ is even, there exists another stable set $H'_v$ of size 3 in $G$ such that
$H_v \cap H'_v = \varnothing$ and the complement of $G - (H_v \cup H'_v)$ has a perfect matching,
\end{itemize}
\end{enumerate}
then, for all $j \leq \lceil n/2 \rceil - 1$, the $(S,Q,j)$-2-rank inequality, \ie
\begin{equation} \label{R2RANK2AGAIN}
\sum_{v \in S\backslash Q} x_{vj} + 2 \sum_{v \in Q} x_{vj} \leq 2 w_j,
\end{equation}
defines a facet of $\ECP$.
\end{tthm}
\begin{proof}
Let $q, q'$ be different vertices of $Q$.

Let $\F'$ be the face of $\ECP$ defined by (\ref{R2RANK2AGAIN}) and
$\F = \{ (x,w) \in \ECP ~:~ \lambda^X x + \lambda^W w = \lambda_0\}$ be a face such that $\F' \subset \F$.
According to Remark \ref{TECHNIQUE}, we have to prove that $(\lambda^X, \lambda^W)$ verifies the following equation system: 
\begin{enumerate}
\item[(a)] $\lambda^X_{vj} = \lambda^X_{vn} + \lambda^W_n,~~~
	      \forall~ v \in V \backslash S ~\textrm{such that}~ Q \backslash N(v) \neq \varnothing$.
\item[(b)] $\lambda^X_{vk} = \lambda^X_{vn} + \lambda^W_n,~~~
	      \forall~ v \in V,~ 1 \leq k \leq n-1,~ k \neq j$.
\item[(c)] $\lambda^X_{qn} + \lambda^X_{vj} = \lambda^X_{qj} + \lambda^X_{vn},~~~
	      \forall~ v \in Q \backslash \{q\}$.
\item[(d)] $\lambda^X_{qn} + 2\lambda^X_{vj} = 2\lambda^X_{vn} + \lambda^X_{qj} + \lambda^W_n,~~~
	      \forall~ v \in S \backslash Q$.
\item[(e)] $\lambda^X_{vj} = \lambda^X_{vn} + \lambda^W_n,~~~
	      \forall~ v \in V \backslash S ~\textrm{such that}~ Q \subset N(v)$.
\item[(f)] $\lambda^W_k = 0,~~~ \forall~ \chi_{eq} + 1 \leq k \leq n-1,~ k \neq j$.
\item[(g)] If $j \geq \chi_{eq} + 1$ then
           $\lambda^X_{qn} + \lambda^W_n = \lambda^X_{qj} + \lambda^W_j$.
\end{enumerate}
We present pairs of equitable colorings lying on $\F'$ that allow us to
prove the validity of each equation in the previous system.
\begin{enumerate}
\item[(a)] Let $\hat{q} \in Q \backslash N(v)$ and let $c^1$ be a $(n-1)$-eqcol such that
$c^1(\hat{q}) = c^1(v) = j$ and $c^2 = intro(c^1,v)$.
We conclude that $\lambda^X_{vj} = \lambda^X_{vn} + \lambda^W_n$.
\item[(b)] Let $s, s' \in S \backslash Q$ be non adjacent vertices.\\
\textbf{Case $v = s$}. Let $c^1$ be a $(n-1)$-eqcol such that $c^1(s) = c^1(s') = k$ and $c^1(q) = j$,
and $c^2 = intro(c^1,s)$. Then, $\lambda^X_{sk} = \lambda^X_{sn} + \lambda^W_n$.\\
\textbf{Case $v \neq s$}. Let $c^1$ be a $n$-eqcol such that $c^1(v) = k$, $c^1(s) = n$. If $v = q$, we make
$c^1(q') = j$. Otherwise, we make $c^1(q) = j$. From the coloring $c^2 = swap_{k,n}(c^1)$ we have
$\lambda^X_{vk} + \lambda^X_{sn} = \lambda^X_{vn} + \lambda^X_{sk}$ and since $\lambda^X_{sk} = \lambda^X_{sn} + \lambda^W_n$ we obtain $\lambda^X_{vk} = \lambda^X_{vn} + \lambda^W_n$.
\item[(c)] Let $c^1$ be a $n$-eqcol such that $c^1(q) = n$, $c^1(v) = j$ and $c^2 = swap_{j,n}(c^1)$.
Therefore, $\lambda^X_{qn} + \lambda^X_{vj} = \lambda^X_{qj} + \lambda^X_{vn}$.
\item[(d)] Let $J$ be the connected component in the complement of $G[S \backslash Q]$
such that $v$ is a vertex of $J$. Since $\alpha(S) = 2$, $J$ does not have triangles.
By hypothesis (i), $J$ is not bipartite and therefore there exists at least an odd cycle in $J$ of size $p$ with $p \geq 5$.\\
Now, let $d(v)$ be the minimum distance in $J$ between $v$ and all the odd cycles in $J$, where the \emph{distance} from a vertex $v$ to an odd cycle is defined as the minimum number of vertices of a path between $v$ and a vertex of the odd cycle.
Condition (d) is proved by induction on $d(v)$.\\
\textbf{Case $d(v) = 0$}. Then, $v$ belongs to an odd cycle of size $p \geq 5$ in $J$. Let $v_1 = v, v_2, \ldots, v_p \in S \backslash Q$ be the vertices of that odd cycle, and let $k_1, k_2, \ldots, k_{p+1}$
be colors different from $j$.\\
We denote by $\oplus$ the sum of two integers modulo $p$.
Let $c^1$, $c^2$, $\ldots$, $c^p$ be $(n-1)$-eqcols such that,
for each $1 \leq i \leq p$, $c^i(v_i) = j$, $c^i(v_{i \oplus 1}) = j$,
$c^i(v_r) = k_r ~~\forall~ r \in \{1,\ldots,p\} \backslash \{i, i \oplus 1\}$, $c^i(q) = k_{p+1}$, and let
$c^{p+1}$ be a $n$-eqcol such that $c^{p+1}(v_1) = n$, $c^{p+1}(v_r) = k_r ~~\forall~ r \in \{2,\ldots,p\}$, $c^{p+1}(q) = j$. For instance, if $p = 5$, colors of $v_1$, $\ldots$, $v_5$ and $q$ would be:
\begin{center} \small
\begin{tabular}{|c|c@{\hspace{2pt}}c@{\hspace{2pt}}c@{\hspace{2pt}}c@{\hspace{2pt}}c@{\hspace{2pt}}c|c|c@{\hspace{2pt}}c@{\hspace{2pt}}c@{\hspace{2pt}}c@{\hspace{2pt}}c@{\hspace{2pt}}c|}
\hline
 \multicolumn{7}{|c|}{$c^i$ with $i$ odd} & \multicolumn{7}{|c|}{$c^i$ with $i$ even} \\
\hline
 size & $v_1$ & $v_2$ & $v_3$ & $v_4$ & $v_5$ & $q$ &
 size & $v_1$ & $v_2$ & $v_3$ & $v_4$ & $v_5$ & $q$ \\
\hline
 $n-1$ & $j$ & $j$ & $k_3$ & $k_4$ & $k_5$ & $k_6$ &
 $n-1$ & $k_1$ & $j$ & $j$ & $k_4$ & $k_5$ & $k_6$ \\
 $n-1$ & $k_1$ & $k_2$ & $j$ & $j$ & $k_5$ & $k_6$ &
 $n-1$ & $k_1$ & $k_2$ & $k_3$ & $j$ & $j$ & $k_6$ \\
 $n-1$ & $j$ & $k_2$ & $k_3$ & $k_4$ & $j$ & $k_6$ &
 $n$ & $n$ & $k_2$ & $k_3$ & $k_4$ & $k_5$ & $j$ \\
\hline
\end{tabular}
\end{center}
We assume that the remaining vertices have the same color in all the colorings.
Thus, we obtain
\[ \sum_{\substack{i = 1\\i~\textrm{odd}}}^{p+1} \sum_{v \in V} \lambda^X_{vc^i(v)} =
\sum_{\substack{i = 1\\i~\textrm{even}}}^{p+1} \sum_{v \in V} \lambda^X_{vc^i(v)} + \lambda^W_n. \]
By condition (b), we get 
$\lambda^X_{qn} + 2\lambda^X_{vj} = 2\lambda^X_{vn} + \lambda^X_{qj} + \lambda^W_n$.\\
\textbf{Case $d(v) \geq 1$}. Let $v' \in J$ be a vertex adjacent to $v$ in $J$ such that
$d(v') = d(v) - 1$. By inductive hypothesis, $\lambda^X_{qn} + 2\lambda^X_{v'j} = 2\lambda^X_{v'n} + \lambda^X_{qj} + \lambda^W_n$.\\
Let $c^1$ be a $(n-1)$-eqcol such that $c^1(v) = c^1(v') = j$ and $c^1(q) = k$, where
$k \neq j$. Let $c^2$ be a $n$-eqcol such that $c^2(v) = k$, $c^2(v') = n$, $c^2(q) = j$ and
$c^2(i) = c^1(i) ~~\forall~i \in V \backslash \{v, v', q\}$. Hence
$\lambda^X_{vj} + \lambda^X_{v'j} + \lambda^X_{qk} = \lambda^X_{vk} + \lambda^X_{v'n} + \lambda^X_{qj}
+ \lambda^W_n$. Multiplying this equality by 2, subtracting
$\lambda^X_{qn} + 2\lambda^X_{v'j} = 2\lambda^X_{v'n} + \lambda^X_{qj} + \lambda^W_n$ and
applying condition (b) yields
$\lambda^X_{qn} + 2\lambda^X_{vj} = 2\lambda^X_{vn} + \lambda^X_{qj} + \lambda^W_n$. 
\item[(e)] By hypothesis (ii), we can establish a $(\lceil n/2 \rceil - 1)$-eqcol $c^1$ such
that color class $j$ is $\{v, s, s'\}$ where $s, s' \in S$ (as we did in condition (d) of Theorem \ref{T2RANK1}).
Let $k$ be the color of $q$ in $c^1$ and $c^2 = swap_{j,k}(c^1)$. We get
$\lambda^X_{vj} = \lambda^X_{vn} + \lambda^W_n$ by applying conditions (a)-(d).
\item[(f)] Since $G$ is monotone, there exist a $k$-eqcol $c$ and a $(k-1)$-eqcol $c'$.
If $k < j$, we consider $c^1 = c$ and $c^2 = c'$. If $k > j$, we consider
$c^1 = swap_{c(q),j}(c)$ and $c^2 = swap_{c'(q),j}(c')$. Then, we apply conditions (a)-(e)
to $\lambda^X x^1 + \lambda^W_k = \lambda^X x^2$, where $x^1$ and $x^2$ are the binary variables representing
colorings $c^1$ and $c^2$ respectively.
\item[(g)] Let $c^1$ be a $j$-eqcol such that $c^1(q) = j$ and $c^2$ be a $(j-1)$-eqcol (the existence of these
colorings is proved above). Then, we apply conditions (a)-(e) to $\lambda^X x^1 + \lambda^W_j = \lambda^X x^2$,
where $x^1$ and $x^2$ are the binary variables representing colorings $c^1$ and $c^2$ respectively.
\end{enumerate}
\end{proof}

Theorem \ref{T2RANK2} states that, among other things, $j \leq \lceil n/2 \rceil - 1$ for the
$(S,Q,j)$-2-rank-inequality to define a facet of $\ECP$. Indeed,
this condition is only used in Theorem \ref{T2RANK2} for proving equations given in (e),
\ie $\lambda^X_{vj} = \lambda^X_{vn} + \lambda^W_n,~ \textrm{for all}~ v \in V \backslash S ~\textrm{such that}~ Q \subset N(v)$. So, if every vertex $v \in V \backslash S$ verifies
$Q \backslash N(v) \neq \varnothing$, these equations
vanish from the equation system on $(\lambda^X, \lambda^W)$ and the inequality (\ref{R2RANK2AGAIN}) defines a facet of $\ECP$ even though $j > \lceil n/2 \rceil - 1$. We have proved the following result.

\begin{tcor} \label{T2RANK2COL}
Let $G$ be a monotone graph, $S \subset V$ such that $\alpha(S) = 2$ and $Q = \{ q \in S ~:~ S \subset N[q] \}$.
If $|Q| \geq 2$, no connected component of the complement of $G[S \backslash Q]$
is bipartite and for all $v \in V \backslash S$, $Q \backslash N(v) \neq \varnothing$, then the $(S,Q,j)$-2-rank
inequality defines a facet of $\ECP$ for all $j \leq n-1$.
\end{tcor}

Let us present an example where the previous result is applied.\\

\noindent \textbf{Example.} Let $G$ be the graph presented in Figure \ref{minigraph1}, $S = \{1,2,\ldots,7\}$ and $Q = \{1,2\}$.
The $(S,Q,j)$-2-rank inequality is a facet-defining inequality of $\ECP(G)$ for $1 \leq j \leq 10$ since the
assumptions of Corollary \ref{T2RANK2COL} are satisfied: vertices $3,\ldots,7$ induce an odd cycle in
$\overline{G}$ and for all $v \in \{8,9,10,11\}$, $Q \backslash N(v) = \{1,2\}$.

\subsection{Subneighborhood inequalities}

\begin{tthm} \label{TNEIGHBOR1}
Let $G$ be a monotone graph, $u \in V$, $j \leq n - 1$ such that $\lceil n/j \rceil \leq \lceil n/\chi_{eq} \rceil$
and $S \subset N(u)$ such that $S$ is not a clique of $G$ and, if $S \neq N(u)$ then $\alpha(S) \leq \lceil n/j \rceil - 1$.\\
If
\begin{enumerate}
\item[(i)] for all $3 \leq i \leq \min \{\lceil n/j \rceil, \alpha(S)\}$, there exists a
$\biggl(\biggl\lceil \dfrac{n}{i-1} \biggr\rceil - 1\biggl)$-eqcol whose color class $C_j$ satisfies $|C_j\cap S| = i$,
\item[(ii)] for all $v \in N(u) \backslash S$, there exists an equitable coloring whose color class $C_j$ satisfies
$|C_j \cap S| = \alpha(S)$ and $(C_j\cap N(u)) \backslash S=\{v\}$,
\end{enumerate}
then the $(u,j,S)$-subneighborhood inequality, \ie
\begin{equation} \label{RNEIGHBOR1AGAIN}
\gamma_{jS} x_{uj} + \sum_{v \in S} x_{vj} +
\sum_{k = j+1}^n (\gamma_{jS} - \gamma_{kS}) x_{uk} \leq \gamma_{jS} w_j,
\end{equation}
defines a facet of $\ECP$, where
$\gamma_{kS} = \min \{\lceil n/k \rceil, \alpha(S)\}$.
\end{tthm}
\begin{proof}
Let $\F'$ be the face of $\ECP$ defined by (\ref{RNEIGHBOR1AGAIN}) and
$\F = \{ (x,w) \in \ECP ~:~ \lambda^X x + \lambda^W w = \lambda_0\}$ be a face such that $\F' \subset \F$.
According to Remark \ref{TECHNIQUE}, we have to prove that $(\lambda^X, \lambda^W)$ verifies the following equation system: 
\begin{enumerate}
\item[(a)] $\lambda^X_{vj} = \lambda^X_{vn} + \lambda^W_n,~~~ \forall~ v \in V \backslash N[u]$.
\item[(b)] $\lambda^X_{vk} = \lambda^X_{vn} + \lambda^W_n,~~~
	      \forall~ v \in V \backslash \{u\},~ 1 \leq k \leq n-1,~ k \neq j$.
\item[(c)] $\lambda^X_{vj} + \lambda^X_{un} = \lambda^X_{vn} + \lambda^X_{uj},~~~
	      \forall~ v \in S$.
\item[(d)] $\lambda^X_{uk} + (\gamma_{jS} - 1) \lambda^X_{uj} = \gamma_{jS} \lambda^X_{un}
              + \gamma_{jS} \lambda^W_n,~~~ \forall~ 1 \leq k \leq j-1$.
\item[(e)] $\lambda^X_{uk} + (\gamma_{kS} - 1) \lambda^X_{uj} =
	    \gamma_{kS} \lambda^X_{un} + \gamma_{kS} \lambda^W_n,~~~
	      \forall~ j+1 \leq k \leq n-1$.
\item[(f)] $\lambda^X_{vj} = \lambda^X_{vn} + \lambda^W_n,~~~ \forall~ v \in N(u) \backslash S$.
\item[(g)] $\lambda^W_k = 0,~~~ \forall~ \chi_{eq} + 1 \leq k \leq n-1,~ k \neq j$.
\item[(h)] If $j \geq \chi_{eq} + 1$ then
           $\gamma_{jS} \lambda^X_{un} + \gamma_{jS} \lambda^W_n = \gamma_{jS} \lambda^X_{uj} + \lambda^W_j$.
\end{enumerate}
We present pairs of equitable colorings lying on $\F'$ that allow us to
prove the validity of each equation in the previous system.
\begin{enumerate}
\item[(a)] Let $c^1$ be a $(n-1)$-eqcol such that $c^1(u) = c^1(v) = j$ and $c^2 = intro(c^1,v)$. We conclude that
$\lambda^X_{vj} = \lambda^X_{vn} + \lambda^W_n$. 
\item[(b)] Let $s, s' \in S$ be non adjacent vertices.\\
\textbf{Case $v = s$}. Let $c^1$ be a $(n-1)$-eqcol such that $c^1(s) = c^1(s') = k$, $c^1(u) = j$
and $c^2 = intro(c^1,s)$. Then, $\lambda^X_{sk} = \lambda^X_{sn} + \lambda^W_n$.\\
\textbf{Case $v \neq s$}. Let $c^1$ be a $n$-eqcol such that $c^1(v) = k$, $c^1(s) = n$, $c^1(u) = j$ and
$c^2 = swap_{k,n}(c^1)$. We have $\lambda^X_{vk} + \lambda^X_{sn} = \lambda^X_{vn} + \lambda^X_{sk}$.
Since $\lambda^X_{sk} = \lambda^X_{sn} + \lambda^W_n$, we conclude that
$\lambda^X_{vk} = \lambda^X_{vn} + \lambda^W_n$.
\item[(c)] Let $c^1$ be a $n$-eqcol such that $c^1(v) = j$, $c^1(u) = n$ and $c^2 = swap_{j,n}(c^1)$.
Therefore, $\lambda^X_{vj} + \lambda^X_{un} = \lambda^X_{vn} + \lambda^X_{uj}$.
\item[(d)] \textbf{Case $\gamma_{jS} = 2$}. Let $c^1$ be a $(n-1)$-eqcol such that $c^1(u) = k$ and
$c^1(s) = c^1(s') = j$ where $s, s' \in S$.\\
\textbf{Case $\gamma_{jS} \geq 3$}. Let $c$ be the $(\lceil \frac{n}{\gamma_{jS} - 1} \rceil - 1)$-eqcol given
by hypothesis (i) and $c^1 = swap_{c(u),k}(c)$.\\
In both cases, $c^1(u) = k$. Now, let $C_j$ and $C_k$ be the color classes $j$
and $k$ of $c^1$ respectively. Considering $c^2 = swap_{j,k}(c^1)$ give rise to
\[ \lambda^X_{uk} + \sum_{v \in C_j} \lambda^X_{vj} + \sum_{v \in C_k \backslash \{u\}} \lambda^X_{vk} =
   \lambda^X_{uj} + \sum_{v \in C_j} \lambda^X_{vk} + \sum_{v \in C_k \backslash \{u\}} \lambda^X_{vj}. \]
Since $|C_j \cap S| = \gamma_{jS}$, we have $C_j \subset S$ and we can apply (a)-(c) in order to
get $\lambda^X_{uk} + (\gamma_{jS} - 1) \lambda^X_{uj} = \gamma_{jS} \lambda^X_{un} + \gamma_{jS} \lambda^W_n$.
\item[(e)] We proceed in the same way as in (d) except that, for the case $\gamma_{jS} \geq 3$, we use the
$(\lceil \frac{n}{\gamma_{kS} - 1} \rceil - 1)$-eqcol given by hypothesis (i) instead of the
$(\lceil \frac{n}{\gamma_{jS} - 1} \rceil - 1)$-eqcol.
\item[(f)] In first place, let us note that $v \in N(u) \backslash S$ implies $S \subsetneqq N(u)$.
Then, $\alpha(S) \leq \lceil n/j \rceil - 1$ and, by hypothesis (ii), there exists a coloring $c^1$ that paints $v$
and $\alpha(S)$ vertices of $S$ with color $j$ but the remaining vertices of $N(u)$ do not use $j$.
Let $k$ be the color used by vertex $u$ in $c^1$ and let $C_j$, $C_k$ be the color classes $j$ and $k$ in
$c^1$ respectively, and $c^2 = swap_{j,k}(c^1)$. We have
\[ \lambda^X_{uk} + \lambda^X_{vj} + \sum_{w \in C_j\backslash \{v\}} \lambda^X_{wj} +
                                    \sum_{w \in C_k \backslash \{u\}} \lambda^X_{wk} =
   \lambda^X_{uj} + \lambda^X_{vk} + \sum_{w \in C_j\backslash \{v\}} \lambda^X_{wk} +
                                    \sum_{w \in C_k \backslash \{u\}} \lambda^X_{wj}. \]
In virtue of conditions (a)-(e), we obtain $\lambda^X_{vj} = \lambda^X_{vn} + \lambda^W_n$.
\item[(g)] Since $G$ is monotone, there exist a $k$-eqcol $c$ and a $(k-1)$-eqcol $c'$.
If $k < j$, we consider $c^1 = c$ and $c^2 = c'$. If $k > j$, we consider
$c^1 = swap_{c(u),j}(c)$ and $c^2 = swap_{c'(u),j}(c')$. Then, we apply conditions (a)-(f)
to $\lambda^X x^1 + \lambda^W_k = \lambda^X x^2$, where $x^1$ and $x^2$ are the binary variables representing
colorings $c^1$ and $c^2$ respectively.
\item[(h)] Let $c^1$ be a $j$-eqcol such that $c^1(u) = j$ and $c^2$ be a $(j-1)$-eqcol (the existence of these
colorings is proved above). Then, we apply conditions (a)-(f) to $\lambda^X x^1 + \lambda^W_j = \lambda^X x^2$,
where $x^1$ and $x^2$ are the binary variables representing colorings $c^1$ and $c^2$ respectively.
\end{enumerate}
\end{proof}

Let us present two examples where the previous theorem is applied.\\

\noindent \textbf{Example.} Let $G$ be the graph given in Figure \ref{minigraph2}(a). We have that $G$ is
monotone and $\chi_{eq}(G) = 3$. Let us consider $u = 1$, $S = N(1)$ and $j = 3$. In order to prove that the $(u,j,S)$-subneighborhood inequality
defines a facet of $\ECP(G)$, it is enough to exhibit a ($\lceil \frac{11}{2} \rceil - 1$)-eqcol
such that $|C_3 \cap S| = 3$ and a ($\lceil \frac{11}{3} \rceil - 1$)-eqcol such that $|C_3 \cap S| = 4$.
Both colorings are shown in Figure \ref{minigraph2} (b) and (c) respectively.

It is not hard to see that the $(u,j,S)$-subneighborhood inequality is also facet-defining for
$4 \leq j \leq 10$.
On the other hand, $(u,j,S)$-subneighborhood inequality with $j \in \{ 1,2 \}$
is facet-defining by Theorem \ref{TIGHT}.

Therefore, the $(u,j,S)$-subneighborhood inequality defines a facet of $\ECP(G)$ for all $1 \leq j \leq 10$.\\

\noindent \textbf{Example.} Let us consider again the graph given in Figure \ref{minigraph2}(a). The $(u,j,S)$-subneighborhood inequality with $u = 1$, $j = 3$ and $S = \{3,4,5\}$
is facet-defining since $\alpha(S) \leq \lceil \frac{11}{3} \rceil - 1$ and there exist the following colorings:
a ($\lceil \frac{11}{2} \rceil - 1$)-eqcol such that
$|C_3 \cap S| = 3$, an equitable coloring such that $|C_3 \cap S| = 3$ and $(C_3 \cap N(1)) \backslash S = \{ 2 \}$, and
an equitable coloring such that $|C_3 \cap S| = 3$ and $(C_3 \cap N(1)) \backslash S = \{ 6 \}$.
These colorings are shown in Figure \ref{minigraph2} (b), (c) and (d) respectively.

\begin{figure}[h]
  \centering ~~~~~~~~~~~
\begin{graph}(5.8, 2)(-2, -1)
	\freetext(0.4,-0.75){(a) labeling of $G$}
	\roundnode{v1}(-2,0)
	\autonodetext{v1}[w]{1}
	\roundnode{v2}(-1.2,0)
	\autonodetext{v2}[n]{2}
	\roundnode{v3}(-1.7528,0.7608)
	\autonodetext{v3}[n]{3}
	\roundnode{v4}(-2.6472,0.4702)
	\autonodetext{v4}[nw]{4}
	\roundnode{v5}(-2.6472,-0.4702)
	\autonodetext{v5}[sw]{5}
	\roundnode{v6}(-1.7528,-0.7608)
	\autonodetext{v6}[s]{6}
	\roundnode{v7}(-0.4,0)
	\autonodetext{v7}[n]{7}
	\roundnode{v8}(0.4,0)
	\autonodetext{v8}[n]{8}
	\roundnode{v9}(1.2,0)
	\autonodetext{v9}[n]{9}
	\roundnode{v10}(2,0)
	\autonodetext{v10}[n]{10}
	\roundnode{v11}(2.8,0)
	\autonodetext{v11}[n]{11}
	\edge{v1}{v2}
	\edge{v1}{v3}
	\edge{v1}{v4}
	\edge{v1}{v5}
	\edge{v1}{v6}
	\edge{v2}{v7}
	\edge{v7}{v8}
	\edge{v8}{v9}
	\edge{v9}{v10}
	\edge{v10}{v11}
\end{graph}~~~~~~~
\begin{graph}(5.8, 2)(-2, -1)
	\freetext(0.4,-0.75){(b) 5-eqcol in $G$}
	\roundnode{v1}(-2,0)
	\autonodetext{v1}[w]{5}
	\roundnode{v2}(-1.2,0)
	\autonodetext{v2}[n]{1}
	\roundnode{v3}(-1.7528,0.7608)
	\autonodetext{v3}[n]{3}
	\roundnode{v4}(-2.6472,0.4702)
	\autonodetext{v4}[nw]{3}
	\roundnode{v5}(-2.6472,-0.4702)
	\autonodetext{v5}[sw]{3}
	\roundnode{v6}(-1.7528,-0.7608)
	\autonodetext{v6}[s]{2}
	\roundnode{v7}(-0.4,0)
	\autonodetext{v7}[n]{4}
	\roundnode{v8}(0.4,0)
	\autonodetext{v8}[n]{5}
	\roundnode{v9}(1.2,0)
	\autonodetext{v9}[n]{1}
	\roundnode{v10}(2,0)
	\autonodetext{v10}[n]{2}
	\roundnode{v11}(2.8,0)
	\autonodetext{v11}[n]{4}
	\edge{v1}{v2}
	\edge{v1}{v3}
	\edge{v1}{v4}
	\edge{v1}{v5}
	\edge{v1}{v6}
	\edge{v2}{v7}
	\edge{v7}{v8}
	\edge{v8}{v9}
	\edge{v9}{v10}
	\edge{v10}{v11}
\end{graph}\\~~~~~~~~~~~
\begin{graph}(5.8, 3)(-2, -1)
	\freetext(0.8,-0.75){(c) 3-eqcol with $c(2)=3$}
	\roundnode{v1}(-2,0)
	\autonodetext{v1}[w]{1}
	\roundnode{v2}(-1.2,0)
	\autonodetext{v2}[n]{3}
	\roundnode{v3}(-1.7528,0.7608)
	\autonodetext{v3}[n]{3}
	\roundnode{v4}(-2.6472,0.4702)
	\autonodetext{v4}[nw]{3}
	\roundnode{v5}(-2.6472,-0.4702)
	\autonodetext{v5}[sw]{3}
	\roundnode{v6}(-1.7528,-0.7608)
	\autonodetext{v6}[s]{2}
	\roundnode{v7}(-0.4,0)
	\autonodetext{v7}[n]{1}
	\roundnode{v8}(0.4,0)
	\autonodetext{v8}[n]{2}
	\roundnode{v9}(1.2,0)
	\autonodetext{v9}[n]{1}
	\roundnode{v10}(2,0)
	\autonodetext{v10}[n]{2}
	\roundnode{v11}(2.8,0)
	\autonodetext{v11}[n]{1}
	\edge{v1}{v2}
	\edge{v1}{v3}
	\edge{v1}{v4}
	\edge{v1}{v5}
	\edge{v1}{v6}
	\edge{v2}{v7}
	\edge{v7}{v8}
	\edge{v8}{v9}
	\edge{v9}{v10}
	\edge{v10}{v11}
\end{graph}~~~~~~~~~~~
\begin{graph}(5.8, 3)(-2, -1)
	\freetext(0.8,-0.75){(d) 3-eqcol with $c(6)=3$}
	\roundnode{v1}(-2,0)
	\autonodetext{v1}[w]{1}
	\roundnode{v2}(-1.2,0)
	\autonodetext{v2}[n]{2}
	\roundnode{v3}(-1.7528,0.7608)
	\autonodetext{v3}[n]{3}
	\roundnode{v4}(-2.6472,0.4702)
	\autonodetext{v4}[nw]{3}
	\roundnode{v5}(-2.6472,-0.4702)
	\autonodetext{v5}[sw]{3}
	\roundnode{v6}(-1.7528,-0.7608)
	\autonodetext{v6}[s]{3}
	\roundnode{v7}(-0.4,0)
	\autonodetext{v7}[n]{1}
	\roundnode{v8}(0.4,0)
	\autonodetext{v8}[n]{2}
	\roundnode{v9}(1.2,0)
	\autonodetext{v9}[n]{1}
	\roundnode{v10}(2,0)
	\autonodetext{v10}[n]{2}
	\roundnode{v11}(2.8,0)
	\autonodetext{v11}[n]{1}
	\edge{v1}{v2}
	\edge{v1}{v3}
	\edge{v1}{v4}
	\edge{v1}{v5}
	\edge{v1}{v6}
	\edge{v2}{v7}
	\edge{v7}{v8}
	\edge{v8}{v9}
	\edge{v9}{v10}
	\edge{v10}{v11}
\end{graph}
  \caption{}
  \label{minigraph2}
\end{figure}

\begin{tcor} \label{T2RANK2CASEQ1}
Let $G$ be a monotone graph, $j \leq n - 1$ and $q \in V$ such that $\alpha(N(q)) = 2$. Then, the $(q,j,N(q))$-subneighborhood inequality defines a facet of $\ECP$.\\
Moreover, let $S \subset V$ with $\alpha(S) = 2$ and $S \subsetneqq N(q)$.
If $j \leq \lceil n/2 \rceil - 1$ and
for all $v \in N(q) \backslash S$, there exist different vertices $s, s' \in S$
and a stable set $H_v = \{v, s, s'\}$ in $G$ such that:
\begin{itemize}
\item If $n$ is odd, the complement of $G - H_v$ has a perfect matching,
\item If $n$ is even, there exists another stable set $H'_v$ of size 3 in $G$ such that $H_v \cap H'_v = \varnothing$
and the complement of $G - (H_v \cup H'_v)$ has a perfect matching,
\end{itemize}
then the $(q,j,S)$-subneighborhood inequality defines a facet of $\ECP$.
\end{tcor}
\begin{proof}
\textbf{Case $\lceil n/j \rceil \leq \lceil n/\chi_{eq} \rceil$}. The $(q,j,N(q))$-subneighborhood inequality defines
a facet of $\ECP$ since hypotheses (i) and (ii) from Theorem \ref{TNEIGHBOR1} hold trivially.\\
Now, let us consider the $(q,j,S)$-subneighborhood inequality. Since $j \leq \lceil n/2 \rceil - 1$, we have that
$\alpha(S) = 2 \leq \lceil n/j \rceil - 1$. Moreover, hypothesis (i) from Theorem \ref{TNEIGHBOR1}
holds trivially.\\
Let $v \in N(q) \backslash S$ and $M_v$, $H_v = \{v, s, s'\}$ and $H'_v$ (if $n$ is even) be the matching and the stable sets given by the hypothesis. Consider the $(\lceil n/2 \rceil - 1)$-eqcol such that the
color class $C_j$ is $H_v$ and the remaining color classes are $H'_v$ (if $n$ is even) and the endpoints of edges of $M_v$. Then, $|C_j \cap S| = 2$, $(C_j \cap N(q)) \backslash S = \{v\}$ and hypothesis (ii) from Theorem \ref{TNEIGHBOR1} holds. Therefore, the $(q,j,S)$-subneighborhood inequality defines a facet of $\ECP$.\\
\textbf{Case $\lceil n/j \rceil > \lceil n/\chi_{eq} \rceil$}. In virtue of the previous case, we know that the
$(q,\chi_{eq},N(q))$-subneighborhood and the $(q,\chi_{eq},S)$-subneighborhood are facet-defining inequalities of
$\ECP$. Hence, the $(q,j,N(q))$-subneighborhood and the $(q,j,S)$-subneighborhood inequality define facets of $\ECP$ due to Theorem \ref{TIGHT}.
\end{proof}

\subsection{Outside-neighborhood inequalities}

\begin{tthm} \label{TNEIGHBOR2}
Let $G$ be a monotone graph, $u \in V$ such that $N(u)$ is not a clique and $\chi_{eq} \leq j \leq \lfloor n/2 \rfloor$.
If
\begin{enumerate}
\item[(i)] there exists $\hat{v} \in V \backslash N[u]$ not universal in $G - u$, 
\item[(ii)] if $n$ is odd, the complement of $G - u$ has a perfect matching,
\item[(iii)] for all $v \in V \backslash N[u]$, the following conditions hold:
\begin{itemize}
\item if $n$ is even, the complement of $G - \{u, v\}$ has a perfect matching,
\item if $n$ is odd, there exists a stable set $H_v \subset V \backslash \{u, v\}$ of size 3 such that
the complement of $G - (H_v \cup \{u, v\})$ has a perfect matching, \end{itemize}
\item[(iv)] for all $r$ such that $j \leq r \leq \lfloor n/2 \rfloor$, we have the following:
\begin{itemize}
\item if $\biggl\lfloor \dfrac{n}{r} \biggr\rfloor > \biggl\lfloor \dfrac{n}{r+1} \biggr\rfloor$, then there exists an $r$-eqcol such that
      $C_j \subset N(u)$ and an $r$-eqcol such that $u \in C_j$ and $|C_j| = \lfloor n/r \rfloor$,
\item if $\biggl\lfloor \dfrac{n}{r} \biggr\rfloor = \biggl\lfloor \dfrac{n}{r+1} \biggr\rfloor$, then there exists an $r$-eqcol satisfying conditions given in Remark \ref{NEIGHBOR2POINTS}, \ie lying on the face defined by (\ref{RNEIGHBOR2AGAIN}),
\end{itemize}
\end{enumerate}
then the $(u,j)$-outside-neighborhood inequality, \ie
\begin{equation} \label{RNEIGHBOR2AGAIN}
\biggl(\biggl\lfloor \dfrac{n}{j} \biggr\rfloor - 1 \biggr) x_{uj} - \sum_{v \in V \backslash N[u]} x_{vj}
+ \sum_{k = j+1}^n b_{jk} x_{uk} \leq \sum_{k = j+1}^{n} b_{jk} (w_k - w_{k+1}),
\end{equation}
defines a facet of $\ECP$, where $b_{jk} = \lfloor n/j \rfloor - \lfloor n/k \rfloor$.
\end{tthm}
\begin{proof}
Let $\F'$ be the face of $\ECP$ defined by (\ref{RNEIGHBOR2AGAIN}) and
$\F = \{ (x,w) \in \ECP ~:~ \lambda^X x + \lambda^W w = \lambda_0\}$ be a face such that $\F' \subset \F$.
According to Remark \ref{TECHNIQUE}, we have to prove that $(\lambda^X, \lambda^W)$ verifies the following equation system: 
\begin{enumerate}
\item[(a)] $\lambda^X_{vk} = \lambda^X_{vn} + \lambda^W_n,
                           ~~~\forall~v \in V \backslash N[u],~ 1 \leq k \leq n - 1,~ k \neq j$.
\item[(b)] $\lambda^X_{vk} = \lambda^X_{vn} + \lambda^W_n,
                           ~~~\forall~v \in N(u),~ 1 \leq k \leq n - 1$.
\item[(c)] $\lambda^X_{uj} = \lambda^X_{un} + \lambda^W_n$.
\item[(d)] $\lambda^X_{vj} = \lambda^X_{vn} + \lambda^W_n +
\lambda^W_{\lfloor n/2 \rfloor + 1},~~~\forall~v \in V \backslash N[u]$.
\item[(e)] $\lambda^X_{uk} = \lambda^X_{un} + \lambda^W_n +
(\lfloor n/j \rfloor - 1)\lambda^W_{\lfloor n/2 \rfloor + 1},~~~\forall~1 \leq k \leq j - 1$.
\item[(f)] $\lambda^X_{uk} = \lambda^X_{un} + \lambda^W_n +
(\lfloor n/k \rfloor - 1)\lambda^W_{\lfloor n/2 \rfloor + 1},~~~\forall~j + 1 \leq k \leq n - 1$.
\item[(g)] If $j \neq \chi_{eq}$, then $\lambda^W_k = 0,~~~\forall~\chi_{eq} + 1 \leq k \leq j$.
\item[(h)] $\lambda^W_k = \biggl(\biggl\lfloor \dfrac{n}{k-1} \biggr\rfloor - \biggl\lfloor \dfrac{n}{k} \biggr\rfloor \biggr)
\lambda^W_{\lfloor n/2 \rfloor + 1},~~~\forall~ j + 1 \leq k \leq n - 1,~ k \neq \lfloor n/2 \rfloor+1$.
\end{enumerate}
We present pairs of equitable colorings lying on $\F'$ that allow us to
prove the validity of each equation in the previous system.
\begin{enumerate}
\item[(a)] By hypothesis (i), there exist $\hat{v} \in V \backslash N[u]$ and $\hat{v}' \in V \backslash \{u, \hat{v}\}$ not adjacent to $\hat{v}$.\\
\textbf{Case $v = \hat{v}$}. Let $c^1$ be a $(n-1)$-eqcol such that
$c^1(u) = j$, $c^1(\hat{v}) = c^1(\hat{v}') = k$ and $c^2 = intro(c^1,\hat{v})$.
Then, $\lambda^X_{\hat{v} k} = \lambda^X_{\hat{v} n} + \lambda^W_n$.\\
\textbf{Case $v \neq \hat{v}$}. Let $c^1$ be a $n$-eqcol such that $c^1(u) = j$, $c^1(v) = k$, $c^1(\hat{v}) = n$ and $c^2 = swap_{k,n}(c^1)$. We have $\lambda^X_{vk} + \lambda^X_{\hat{v} n} = \lambda^X_{vn} + \lambda^X_{\hat{v} k}$ and
therefore $\lambda^X_{vk} = \lambda^X_{vn} + \lambda^W_n$.
\item[(b)] Let $u_1, u_2 \in N(u)$ be non adjacent vertices.\\
\textbf{Case $v = u_1$}. Let $c^1$ be a $(n-1)$-eqcol such that $c^1(u_1) = c^1(u_2) = k$. If
$k = j$ we set $c^1(u) = n-1$, otherwise $c^1(u) = j$. Let $c^2 = intro(c^1,u_1)$.
Then, $\lambda^X_{u_1 k} = \lambda^X_{u_1 n} + \lambda^W_n$.\\
\textbf{Case $v \neq u_1$}. Let $c^1$ be a $n$-eqcol such that $c^1(v) = k$ and $c^1(u_1) = n$. If $k = j$ we set
$c^1(u) = n-1$, otherwise $c^1(u) = j$. Let $c^2 = swap_{k,n}(c^1)$.
We have $\lambda^X_{vk} + \lambda^X_{u_1 n} = \lambda^X_{vn} + \lambda^X_{u_1 k}$ and therefore
$\lambda^X_{vk} = \lambda^X_{vn} + \lambda^W_n$.
\item[(c)] Let $v \in N(u)$, $c^1$ be a $n$-eqcol such that $c^1(u) = j$, $c^1(v) = n$ and
$c^2 = swap_{j,n}(c^1)$. In virtue of condition (b), we obtain $\lambda^X_{uj} = \lambda^X_{un} + \lambda^W_n$.
\item[(d)]  \textbf{Case $n$ even}. Let $M_v$ be the matching given by hypothesis (iii). Let $c^1$ be the
$\lfloor n/2 \rfloor$-eqcol whose color classes are the endpoints of $M_v$ and $C_j = \{u, v\}$.
Let $c^2 = intro(c^1,v)$. We deduce that
$\lambda^X_{vj} = \lambda^X_{v\lfloor n/2 \rfloor + 1} + \lambda^W_{\lfloor n/2 \rfloor + 1}
= \lambda^X_{vn} + \lambda^W_n + \lambda^W_{\lfloor n/2 \rfloor + 1}$.\\
\textbf{Case $n$ odd}. Let $M_v$ and $H_v$ be the matching and the stable set given by hypothesis (iii). Let
$c^1$ be the $\lfloor n/2 \rfloor$-eqcol whose color classes are $H_v$, the endpoints of $M_v$ and $C_j = \{u, v\}$.
Now, let $M$ be the matching given by hypothesis (ii) and let $v'$ be a vertex such that $(v, v')$ belongs to $M$.
Let $c^2$ be the $(\lfloor n/2 \rfloor + 1)$-eqcol
whose color classes are the endpoints of $M \backslash (v, v')$, $C_j = \{u\}$ and
$C_{\lfloor n/2 \rfloor+1} = \{ v, v' \}$.
Thus,
\[ \lambda^X_{vj} + \sum_{i \in V \backslash \{u, v\}} \lambda^X_{ic^1(i)} =
\lambda^X_{v\lfloor n/2 \rfloor + 1} + \sum_{i \in V \backslash \{u, v\}} \lambda^X_{ic^2(i)}
+ \lambda^W_{\lfloor n/2 \rfloor + 1}. \]
Conditions (a) and (b) allow us to reach the desired result.
\item[(e)] Let us notice that, if $r = \lfloor n/\lfloor n/j\rfloor \rfloor$ then
$\lfloor n/j \rfloor = \lfloor n/r \rfloor$, $j \leq r \leq \lfloor n/2 \rfloor$ and
$\lfloor \frac{n}{r} \rfloor > \lfloor \frac{n}{r+1} \rfloor$. By hypothesis (iv), there exists an $r$-eqcol
$c$ such that $N(u)$ contains all the vertices painted with color $j$. Let $c^1 = swap_{c(u),k}(c)$ and
$c^2$ be the $r$-eqcol that paints vertex $u$ and $\lfloor n/j \rfloor-1$ vertices of
$V \backslash N[u]$ with color $j$ also given by hypothesis (iv).
By condition (c), we have $\lambda^X_{uk} + \sum_{v \in V \backslash \{u\}} \lambda^X_{vc^1(v)} =
\lambda^X_{un} + \lambda^W_n + \sum_{v \in V \backslash \{u\}} \lambda^X_{vc^2(v)}$.
Applying conditions (a), (b) and (d), we get
$\lambda^X_{uk} = \lambda^X_{un} + \lambda^W_n +
(\lfloor n/j \rfloor - 1)\lambda^W_{\lfloor n/2 \rfloor + 1}$.
\item[(f)] \textbf{Case $k \leq \lfloor n/2 \rfloor$}. We proceed in the same way as in (e),
but using $r = \lfloor n/\lfloor n/k\rfloor \rfloor$ instead of $\lfloor n/\lfloor n/j\rfloor \rfloor$.\\
\textbf{Case $k \geq \lfloor n/2 \rfloor + 1$}. Then, $\lfloor n/k \rfloor = 1$. Let $v \in N(u)$, $c^1$ be a
$n$-eqcol such that $c^1(u) = k$, $c^1(v) = j$ and $c^2 = swap_{k,j}(c^1)$.
Conditions (b) and (c) allow us to obtain $\lambda^X_{uk} = \lambda^X_{un} + \lambda^W_n$.
\item[(g)-(h)] This condition can be verified by providing a $k$-eqcol $(x^1,w^1)$ and a $(k-1)$-eqcol
$(x^2,w^2)$ lying on $\F'$ and applying conditions (a)-(f) to equation $\lambda^X x^1 + \lambda^W_k = \lambda^X x^2$.\\
Thus, we only need to prove that, for any $\chi_{eq} \leq r \leq n-1$, there exists an $r$-eqcol $c$ lying on $\F'$ .\\ 
\textbf{Case $r < j$}. The existence of $c$ is guaranteed by the monotonicity of G.\\
\textbf{Case $j \leq r \leq \lfloor n/2 \rfloor$}. The existence of $c$ is guaranteed by hypothesis (iv).\\
\textbf{Case $r = \lfloor n/2 \rfloor + 1$}. $c$ may be the $(\lfloor n/2 \rfloor + 1)$-eqcol yielded by condition (d).\\
\textbf{Case $\lfloor n/2 \rfloor+2 \leq r \leq n-1$}. Let us consider the
$(\lfloor n/2 \rfloor + 1)$-eqcol yielded in the previous case and let $v_1, v_2$ be vertices sharing a color
different from $j$. In order to generate a $(\lfloor n/2 \rfloor + 2)$-eqcol $c$, we introduce a new color on $v_1$,
\ie $c = intro(c',v_1)$ where $c'$ is the $(\lfloor n/2 \rfloor + 1)$-eqcol. By repeating this
procedure, we can generate a $(\lfloor n/2 \rfloor + 3)$-eqcol and so on.
\end{enumerate}
\end{proof}

Let us present an example where the previous theorem is applied.\\

\noindent \textbf{Example.} Let $G$ be the graph given in Figure \ref{minigraph2}(a). Let us recall that $G$ is
monotone and $\chi_{eq}(G) = 3$.
We apply Theorem \ref{TNEIGHBOR2} considering $u = 1$ and $j = 3$. It is not hard to see that the assumptions of
this theorem are satisfied. Below, we present some examples of colorings related to hypothesis (iv) of
Theorem \ref{TNEIGHBOR2}. Figure \ref{minigraph3}(a) shows a 3-eqcol of $G$ such that $C_3 \subset N(1)$
and Figure \ref{minigraph3}(b) shows a 3-eqcol of $G$ such that $1 \in C_3$ and $|C_3| = 3$.


By Theorem \ref{TIGHT2}, $(1,j)$-outside-neighborhood inequalities with $j \in \{1,2\}$ are also facet-defining.

\begin{figure}[h]
  \centering ~~~~~~~~~~~
\begin{graph}(5.8, 3)(-2, -1)
	\freetext(0.8,-0.75){(a) 3-eqcol, $C_3 \subset N(1)$}
	\roundnode{v1}(-2,0)
	\autonodetext{v1}[w]{1}
	\roundnode{v2}(-1.2,0)
	\autonodetext{v2}[n]{3}
	\roundnode{v3}(-1.7528,0.7608)
	\autonodetext{v3}[n]{3}
	\roundnode{v4}(-2.6472,0.4702)
	\autonodetext{v4}[nw]{3}
	\roundnode{v5}(-2.6472,-0.4702)
	\autonodetext{v5}[sw]{3}
	\roundnode{v6}(-1.7528,-0.7608)
	\autonodetext{v6}[s]{2}
	\roundnode{v7}(-0.4,0)
	\autonodetext{v7}[n]{1}
	\roundnode{v8}(0.4,0)
	\autonodetext{v8}[n]{2}
	\roundnode{v9}(1.2,0)
	\autonodetext{v9}[n]{1}
	\roundnode{v10}(2,0)
	\autonodetext{v10}[n]{2}
	\roundnode{v11}(2.8,0)
	\autonodetext{v11}[n]{1}
	\edge{v1}{v2}
	\edge{v1}{v3}
	\edge{v1}{v4}
	\edge{v1}{v5}
	\edge{v1}{v6}
	\edge{v2}{v7}
	\edge{v7}{v8}
	\edge{v8}{v9}
	\edge{v9}{v10}
	\edge{v10}{v11}
\end{graph}~~~~~~~
\begin{graph}(5.8, 3)(-2, -1)
	\freetext(1.0,-0.75){(b) 3-eqcol, $1 \in C_3$, $|C_3| = 3$}
	\roundnode{v1}(-2,0)
	\autonodetext{v1}[w]{3}
	\roundnode{v2}(-1.2,0)
	\autonodetext{v2}[n]{2}
	\roundnode{v3}(-1.7528,0.7608)
	\autonodetext{v3}[n]{1}
	\roundnode{v4}(-2.6472,0.4702)
	\autonodetext{v4}[nw]{1}
	\roundnode{v5}(-2.6472,-0.4702)
	\autonodetext{v5}[sw]{1}
	\roundnode{v6}(-1.7528,-0.7608)
	\autonodetext{v6}[s]{2}
	\roundnode{v7}(-0.4,0)
	\autonodetext{v7}[n]{3}
	\roundnode{v8}(0.4,0)
	\autonodetext{v8}[n]{2}
	\roundnode{v9}(1.2,0)
	\autonodetext{v9}[n]{3}
	\roundnode{v10}(2,0)
	\autonodetext{v10}[n]{2}
	\roundnode{v11}(2.8,0)
	\autonodetext{v11}[n]{1}
	\edge{v1}{v2}
	\edge{v1}{v3}
	\edge{v1}{v4}
	\edge{v1}{v5}
	\edge{v1}{v6}
	\edge{v2}{v7}
	\edge{v7}{v8}
	\edge{v8}{v9}
	\edge{v9}{v10}
	\edge{v10}{v11}
\end{graph}
  \caption{}
  \label{minigraph3}
\end{figure}

\subsection{Clique-neighborhood inequalities}

\begin{tthm} \label{TNEIGHBOR3}
Let $G$ be a monotone graph, $u \in V$, $Q$ be a clique of $G$ such that $Q \cap N[u]=\varnothing$ and $j,k$
be numbers verifying $3 \leq k \leq \min \{\alpha(N(u)) + 1, \lceil n/\chi_{eq} \rceil\}$ and
$1 \leq j \leq \biggl\lceil \dfrac{n}{k-1} \biggr\rceil - 1$. If
\begin{enumerate}
\item[(i)] for all $v \in V \backslash (N[u] \cup Q)$, there exist
$\lceil n/3 \rceil \leq r \leq \lceil n/2 \rceil - 1$, $q_1, q_2 \in V$ and two $r$-eqcols such that in one of them
$C_j = \{u, v, q_1\}$ and in the other $C_j = \{u , q_2\}$ ($q_1$ and $q_2$ may be the same vertex),
\item[(ii)] for all $t$ such that $max \{ j, \chi_{eq} \} \leq t \leq n - 3$, we have the following:
\begin{itemize}
\item if $\biggl\lceil \dfrac{n}{t} \biggr\rceil > \biggl\lceil \dfrac{n}{t+1} \biggr\rceil$, there exist $q \in Q$ and
a $t$-eqcol such that $C_j \subset N(u)$, $|C_j| = \lceil n/t \rceil$ and $u, q \in C_t$,
\item if $\biggl\lceil \dfrac{n}{t} \biggr\rceil = \biggl\lceil \dfrac{n}{t+1} \biggr\rceil$, there exists a $t$-eqcol
satisfying conditions given in Remark \ref{NEIGHBOR3POINTS}, \ie lying on the face defined by (\ref{RNEIGHBOR3AGAIN}),
\end{itemize} \end{enumerate}
then the $(u,j,k,Q)$-clique-neighborhood inequality, \ie
\begin{multline} \label{RNEIGHBOR3AGAIN}
(k - 1) x_{uj} +
\sum_{l = \lceil \frac{n}{k-1} \rceil}^{n-2} \biggl(k - \biggl\lceil \dfrac{n}{l} \biggr\rceil \biggr) x_{ul} +
(k - 1) \bigl(x_{u n-1} + x_{un} \bigr) +
\!\!\!\sum_{v \in N(u)\cup Q}\!\!\!x_{vj} \\ + \sum_{v \in V \backslash \{u\}}\!\!\!(x_{v n-1} + x_{vn}) \leq
\sum_{l = j}^n b_{ul} (w_l - w_{l+1}),
\end{multline}
defines a facet of $\ECP$, where
\[ b_{ul} = \begin{cases}
 \min\{\lceil n / l \rceil, \alpha(N(u)) + 1\}, &\textrm{if $j \leq l \leq \lceil n/k \rceil - 1$} \\
 k, &\textrm{if $\lceil n/k \rceil \leq l \leq n - 2$} \\
 k + 1, &\textrm{if $l \geq n - 1$}
\end{cases} \]
\end{tthm}
\begin{proof}
Let $\F'$ be the face of $\ECP$ defined by (\ref{RNEIGHBOR3AGAIN}) and
$\F = \{ (x,w) \in \ECP ~:~ \lambda^X x + \lambda^W w = \lambda_0\}$ be a face such that $\F' \subset \F$.
According to Remark \ref{TECHNIQUE}, we have to prove that $(\lambda^X, \lambda^W)$ verifies the following equation system: 
\begin{enumerate}
\item[(a)] $\lambda^X_{uj} = \lambda^X_{un} + \lambda^W_n$.
\item[(b)] $\lambda^X_{vn-1} = \lambda^X_{vn} + \lambda^W_n,~~~ \forall~ v \in V$.
\item[(c)] $\lambda^X_{vr} = \lambda^X_{vn-1} + \lambda^W_{n-1},~~~ \forall~ v \in V \backslash \{u\},~ 1 \leq r \leq n - 2,~ r \neq j$.
\item[(d)] $\lambda^X_{vj} = \lambda^X_{vn} + \lambda^W_n,~~~ \forall~ v \in N(u) \cup Q$.
\item[(e)] $\lambda^X_{vj} = \lambda^X_{vn-1} + \lambda^W_{n-1},~~~ \forall~ v \in V \backslash (N[u] \cup Q)$.
\item[(f)] $\lambda^X_{ur} = \lambda^X_{un} + (k - 1) \lambda^W_{n-1} + \lambda^W_n,~~~ \forall~ 1 \leq r \leq \lceil \frac{n}{k-1} \rceil - 1,~ r \neq j$.
\item[(g)] $\lambda^X_{ur} = \lambda^X_{un} + (\lceil n / r \rceil - 1) \lambda^W_{n-1} + \lambda^W_n,~~~ \forall~ \lceil \frac{n}{k-1} \rceil \leq r \leq n - 2$.
\item[(h)] $\lambda^W_r = (b_{ur} - b_{ur-1}) \lambda^W_{n-1},~~~ \forall~ \chi_{eq} + 1 \leq r \leq n-2$.
\end{enumerate}
We present pairs of equitable colorings lying on $\F'$ that allow us to
prove the validity of each equation in the previous system.
\begin{enumerate}
\item[(a)] Let $q \in Q$, $c^1$ be a $(n-1)$-eqcol such that $c^1(u) = c^1(q) = j$ and $c^2 = intro(c^1,u)$.
Then, $\lambda^X_{uj} = \lambda^X_{un} + \lambda^W_n$.
\item[(b)] \textbf{Case $v = u$}. Let $q \in Q$, $w \in N(u) \cup Q \backslash \{q\}$, $c^1$ be a $(n-1)$-eqcol
such that $c^1(u) = c^1(q) = n-1$, $c^1(w) = j$ and $c^2 = intro(c^1,u)$. Then, $\lambda^X_{un-1} = \lambda^X_{un} + \lambda^W_n$.\\
Now, let $v_1, v_2 \in N(u)$ be non adjacent vertices.\\
\textbf{Case $v = v_1$}. Let $c^1$ be a $(n-1)$-eqcol such that $c^1(v_1) = c^1(v_2) = n-1$,
$c^1(u) = j$ and $c^2 = intro(c^1,v_1)$. We have $\lambda^X_{v_1 n-1} = \lambda^X_{v_1 n} + \lambda^W_n$.\\
\textbf{Case $v \in V \backslash \{u, v_1\}$}. Let $c^1$ be a $n$-eqcol such that $c^1(u) = j$, $c^1(v) = n-1$, $c^1(v_1) = n$ and $c^2 = swap_{n-1,n}(c^1)$.
We have $\lambda^X_{vn-1} + \lambda^X_{v_1 n} = \lambda^X_{vn} + \lambda^X_{v_1 n-1}$ and, since
$\lambda^X_{v_1 n-1} = \lambda^X_{v_1 n} + \lambda^W_n$, we obtain $\lambda^X_{v n-1} = \lambda^X_{v n} + \lambda^W_n$.
\item[(c)] Let $v_1, v_2 \in N(u)$ be non adjacent vertices and $q \in Q$.\\
\textbf{Case $v = v_1$}. Let $c^1$ be a $(n-2)$-eqcol such that $c^1(v_1) = c^1(v_2) = r$,
$c^1(u) = c^1(q) = j$ and $c^2 = intro(c^1,v_1)$. Then, $\lambda^X_{v_1 r} = \lambda^X_{v_1 n-1} + \lambda^W_{n-1}$.\\
\textbf{Case $v \neq v_1$}. Let $c^1$ be a $n$-eqcol such that $c^1(u) = j$, $c^1(v) = r$, $c^1(v_1) = n-1$ and
$c^2 = swap_{r,n-1}(c^1)$. We have $\lambda^X_{vr} + \lambda^X_{v_1 n-1} = \lambda^X_{vn-1} + \lambda^X_{v_1 r}$ and,
since $\lambda^X_{v_1 r} = \lambda^X_{v_1 n-1} + \lambda^W_{n-1}$, we obtain
$\lambda^X_{vr} = \lambda^X_{vn-1} + \lambda^W_{n-1}$.
\item[(d)] Let $q \in Q$.\\
\textbf{Case $v = q$}. Let $c^1$ be a $(n-1)$-eqcol such that $c^1(u) = c^1(q) = j$ and
$c^2 = intro(c^1,q)$. Then, $\lambda^X_{qj} = \lambda^X_{qn} + \lambda^W_n$.\\
\textbf{Case $v \neq q$}. Let $c^1$ be a $n$-eqcol such that $c^1(u) = n-1$, $c^1(q) = n$, $c^1(v) = j$ and $c^2 = swap_{j,n}(c^1)$.
We have $\lambda^X_{vj} + \lambda^X_{qn} = \lambda^X_{vn} + \lambda^X_{qj}$ and, since $\lambda^X_{qj} = \lambda^X_{qn} + \lambda^W_n$, we obtain $\lambda^X_{vj} = \lambda^X_{vn} + \lambda^W_n$.
\item[(e)] Hypothesis (i) ensures that there exists an equitable coloring $c^1$ such that
$c^1(u) = c^1(v) = c^1(q_1) = j$ and the remaining vertices do not use color $j$, and there exists
another equitable coloring $c^2$ (with the same number of colors) such that $c^2(u) = c^2(q_2) = j$
and the remaining vertices do not use color $j$, where $q_1, q_2 \in Q$. We have
\[ \sum_{w \in V\backslash\{u,v,q_1\}} \lambda^X_{w c^1(w)} + \lambda^X_{q_1 j} + \lambda^X_{vj}
= \sum_{w \in V\backslash\{u,v,q_2\}} \lambda^X_{w c^2(w)} + \lambda^X_{q_2 j} + \lambda^X_{vc^2(v)} \]
and, by conditions (b)-(d), we derive
$\lambda^X_{vj} = \lambda^X_{vc^2(v)} = \lambda^X_{vn} + \lambda^W_{n-1}$.
\item[(f)] Let $t = \lceil \frac{n}{k - 1} \rceil - 1$. Clearly, $max \{ j, \chi_{eq} \} \leq t \leq n - 3$ and
$\lceil \frac{n}{t} \rceil > \lceil \frac{n}{t+1} \rceil$. By hypothesis (ii), there exists a $t$-eqcol $c$ whose
class color $C_j$ satisfies $C_j \subset N(u)$ and $|C_j| = \lceil n/t \rceil$ and $u$ and a vertex of $Q$ use
color $t$. Let $c^1 = swap_{j,t}(c)$ and $c^2 = swap_{r,t}(c)$ (since $t \geq j$ and $t \geq r$, both colorings are
well-defined). Hence, $c^1(u) = j$ and $c^2(u) = r$. We apply conditions proved before to
$\lambda^X x^1 = \lambda^X x^2$, where $x^1$ and $x^2$ are the binary variables representing colorings
$c^1$ and $c^2$ respectively, and we conclude that
$\lambda^X_{ur} = \lambda^X_{un} + (k - 1) \lambda^W_{n-1} + \lambda^W_n$.
\item[(g)] \textbf{Case $r \leq \lceil \frac{n}{2} \rceil-1$}. We proceed in the same way as in (f), but using
$t = \lceil \frac{n}{\lceil n/r \rceil - 1} \rceil - 1$ instead of
$\lceil \frac{n}{k - 1} \rceil - 1$.\\
\textbf{Case $r \geq \lceil \frac{n}{2} \rceil$}. Let $v_1, v_2 \in N(u)$ be non adjacent vertices and $q \in Q$. Let $c^1$ be a $(n-2)$-eqcol such that
$c^1(v_1) = c^1(v_2) = r$, $c^1(u) = c^1(q) = j$ and $c^2 = swap_{j,r}(c^1)$. We apply conditions proved before
to $\lambda^X x^1 = \lambda^X x^2$, where $x^1$ and $x^2$ are the binary variables representing colorings
$c^1$ and $c^2$ respectively, and we conclude that $\lambda^X_{ur} = \lambda^X_{un} + \lambda^W_{n-1} + \lambda^W_n$.
\item[(h)] This condition can be verified by providing an $r$-eqcol $(x^1,w^1)$ and an $(r-1)$-eqcol
$(x^2,w^2)$ lying on $\F'$ and applying conditions (a)-(g) to equation $\lambda^X x^1 + \lambda^W_r = \lambda^X x^2$.\\
Thus, we only need to prove that, for any $\chi_{eq} \leq t \leq n-2$, there exists a $t$-eqcol $c$ lying on $\F'$ .\\ 
\textbf{Case $t < j$}. The existence of $c$ is guaranteed by the monotonicity of G.\\
\textbf{Case $j \leq t \leq n-3$}. The existence of $c$ is guaranteed by hypothesis (ii).\\
\textbf{Case $t=n-2$}. $c$ may be the $(n-2)$-eqcol yielded by condition (c).
\end{enumerate}
\end{proof}

\begin{tcor} \label{TNEIGHBOR3COR}
Let $G$ be a monotone graph and let $u$, $j$, $k$, $Q$ be defined as in Theorem \ref{TNEIGHBOR3}.
If hypothesis (ii) of Theorem \ref{TNEIGHBOR3} holds and for all $v \in V \backslash (N[u] \cup Q)$:
\begin{itemize}
\item if $n$ is odd,
\begin{itemize}
\item there exists a vertex $q_1 \in Q$ and a stable set $H^1_v = \{u, v, q_1\}$ such that the complement of $G - H^1_v$
has a perfect matching $M_v$,
\item there exists a vertex $q_2 \in Q$ and two disjoint stable sets $H^2_v = \{u, q_2\}$, $H^3_v$ such that
$|H^3_v| = 3$ and the complement of $G - (H^2_v \cup H^3_v)$ has a perfect matching  $M'_v$,
\end{itemize}
\item if $n$ is even,
\begin{itemize}
\item there exists a vertex $q_1 \in Q$ and two disjoint stable sets $H^1_v = \{u, v, q_1\}$, $H^2_v$ such that
$|H^2_v| = 3$ and the complement of $G - (H^1_v \cup H^2_v)$ has a perfect matching $M_v$,
\item there exists a vertex $q_2 \in Q$ and three disjoint stable sets $H^3_v = \{u, q_2\}$, $H^4_v$, $H^5_v$
such that $|H^4_v| = |H^5_v| = 3$ and the complement of $G - (H^3_v \cup H^4_v \cup H^5_v)$ has a perfect matching  $M'_v$,
\end{itemize}
\end{itemize}
then the $(u,j,k,Q)$-clique-neighborhood inequality defines a facet of $\ECP$.
\end{tcor}
\begin{proof}
Let us suppose that $n$ is odd.
Let $v \in V \backslash (N[u] \cup Q)$ and let $M_v$, $M'_v$, $H^1_v$, $H^2_v$ and $H^3_v$ be the matchings and the
stable sets given in the hypothesis. Consider an $(\lceil n/2 \rceil - 1)$-eqcol such that the color class
$j$ is $H^1_v$ and the remaining color classes are the endpoints of edges of $M_v$, and an
$(\lceil n/2 \rceil - 1)$-eqcol such that the color class $j$ is $H^2_v$ and the remaining color classes are $H^3_v$
and the endpoints of edges of $M'_v$. Therefore, hypothesis (i) of Theorem \ref{TNEIGHBOR3} holds and
the $(u,j,k,Q)$-clique-neighborhood inequality defines a facet of $\ECP$.

The proof for $n$ even is analogous to the previous one.
\end{proof}

Let us present an example where the previous result is applied.\\

\noindent \textbf{Example.} Let $G$ be the graph given in Figure \ref{minigraph2}(a). Let us recall that $G$ is
monotone and $\chi_{eq}(G) = 3$. We apply Corollary \ref{TNEIGHBOR3COR} considering $u = 1$, $j = 1$, $k = 4$ and
$Q = \{7,8\}$. It is not hard to see that the assumptions of this corollary are satisfied.
Below, we present some examples of colorings related to hypothesis (ii) of Theorem \ref{TNEIGHBOR3}.
Figure \ref{minigraph4}(a) shows a 3-eqcol of $G$ such that $1,7 \in C_3$, $C_1 \subset N(1)$, $|C_1| = 4$ and
Figure \ref{minigraph4}(b) shows a 5-eqcol of $G$ such that $1,7 \in C_5$, $C_1 \subset N(1)$, $|C_1| = 3$.


\begin{figure}[h]
  \centering ~~~~~~~~~~~
\begin{graph}(5.8, 2)(-2, -1)
	\freetext(0.6,-0.75){(a) 3-eqcol of $G$}
	\roundnode{v1}(-2,0)
	\autonodetext{v1}[w]{3}
	\roundnode{v2}(-1.2,0)
	\autonodetext{v2}[n]{1}
	\roundnode{v3}(-1.7528,0.7608)
	\autonodetext{v3}[n]{1}
	\roundnode{v4}(-2.6472,0.4702)
	\autonodetext{v4}[nw]{1}
	\roundnode{v5}(-2.6472,-0.4702)
	\autonodetext{v5}[sw]{1}
	\roundnode{v6}(-1.7528,-0.7608)
	\autonodetext{v6}[s]{2}
	\roundnode{v7}(-0.4,0)
	\autonodetext{v7}[n]{3}
	\roundnode{v8}(0.4,0)
	\autonodetext{v8}[n]{2}
	\roundnode{v9}(1.2,0)
	\autonodetext{v9}[n]{3}
	\roundnode{v10}(2,0)
	\autonodetext{v10}[n]{2}
	\roundnode{v11}(2.8,0)
	\autonodetext{v11}[n]{3}
	\edge{v1}{v2}
	\edge{v1}{v3}
	\edge{v1}{v4}
	\edge{v1}{v5}
	\edge{v1}{v6}
	\edge{v2}{v7}
	\edge{v7}{v8}
	\edge{v8}{v9}
	\edge{v9}{v10}
	\edge{v10}{v11}
\end{graph}~~~~~~~
\begin{graph}(5.8, 2)(-2, -1)
	\freetext(0.6,-0.75){(b) 5-eqcol of $G$}
	\roundnode{v1}(-2,0)
	\autonodetext{v1}[w]{5}
	\roundnode{v2}(-1.2,0)
	\autonodetext{v2}[n]{1}
	\roundnode{v3}(-1.7528,0.7608)
	\autonodetext{v3}[n]{1}
	\roundnode{v4}(-2.6472,0.4702)
	\autonodetext{v4}[nw]{1}
	\roundnode{v5}(-2.6472,-0.4702)
	\autonodetext{v5}[sw]{2}
	\roundnode{v6}(-1.7528,-0.7608)
	\autonodetext{v6}[s]{4}
	\roundnode{v7}(-0.4,0)
	\autonodetext{v7}[n]{5}
	\roundnode{v8}(0.4,0)
	\autonodetext{v8}[n]{3}
	\roundnode{v9}(1.2,0)
	\autonodetext{v9}[n]{2}
	\roundnode{v10}(2,0)
	\autonodetext{v10}[n]{3}
	\roundnode{v11}(2.8,0)
	\autonodetext{v11}[n]{4}
	\edge{v1}{v2}
	\edge{v1}{v3}
	\edge{v1}{v4}
	\edge{v1}{v5}
	\edge{v1}{v6}
	\edge{v2}{v7}
	\edge{v7}{v8}
	\edge{v8}{v9}
	\edge{v9}{v10}
	\edge{v10}{v11}
\end{graph}
\caption{}
  \label{minigraph4}
\end{figure}

\subsection{$S$-color inequalities}

\begin{tthm} \label{TONLYCOLORS}
Let $M$ be a matching of the complement of $G$ such that $2 \leq |M| \leq \lfloor \frac{n-1}{2} \rfloor$
and let $S \subset \{1,\ldots,n\}$ such that $|S| = 2|M| - r$ with $r \in \{1,2\}$ and $S$ contains all the colors
greater than $n - |M|$. Then, the $S$-color inequality, \ie
\begin{equation} \label{RONLYCOLORSAGAIN}
\sum_{j \in S} \sum_{v \in V} x_{vj} \leq \sum_{k=1}^{n} b_{Sk} (w_k - w_{k+1}),
\end{equation}
defines a facet of $\ECP$, where
$$b_{Sk} = |S \cap \{1,\ldots,k\}| \biggl\lfloor \dfrac{n}{k} \biggr\rfloor + \min \biggl\{ |S \cap \{1,\ldots,k\}|, n - k \biggl\lfloor \dfrac{n}{k} \biggr\rfloor \biggr\}.$$
\end{tthm}
\begin{proof}
Let us note that $|S| \geq 2$. If $|S| = 2$, $S = \{n-1,n\}$ and the $S$-color inequality defines the same
face as the $\{n-1\}$-color inequality as stated in Remark \ref{REMARKONLY}.1. But the $\{n-1\}$-color inequality
is the constraint (\ref{RUPPER}) with $j = n-1$ by Remark \ref{REMARKONLY}.2, which is facet-defining by Theorem \ref{TORIGINAL4}. Then, the $S$-color inequality defines a facet of $\ECP$. So, from now on we assume that
$|S| \geq 3$.

For the sake of simplicity, we define $p = n - |M|$.

Now, let $\F'$ be the face of $\ECP$ defined by (\ref{RONLYCOLORSAGAIN}) and
$\F = \{ (x,w) \in \ECP ~:~ \lambda^X x + \lambda^W w = \lambda_0\}$ be a face such that $\F' \subset \F$.
According to Remark \ref{TECHNIQUE}, we have to prove that $(\lambda^X, \lambda^W)$ verifies the following equation system: 
\begin{align*}
&\textrm{(a)}~~ \lambda^X_{vj} = \lambda^X_{vn} + \lambda^W_n,~~~\forall~v \in V,~
                   j \in S \backslash \{n\}. & \\
&\textrm{(b)}~~ \lambda^X_{vj} = \lambda^X_{vn} + \lambda^W_n + \frac{1}{r} \lambda^W_{p+1},~~~\forall~v \in V,~
                   j \notin S. & \\
&\textrm{(c)}~~ \sum_{k = \theta+1}^j \lambda^W_k = (b_{Sj} - b_{S\theta}) \frac{1}{r} \lambda^W_{p+1}, ~~~\forall~\chi_{eq}+1 \leq j \leq n - 1 ~\textrm{such that} &  \\
&~~~~~~~~~~~~~~~~ j \neq p+1,~ j \notin \S ~\textrm{and}~ \theta = \max \{j' \in \mathbb{Z} : j' \leq j-1,~j' \notin \S\}. &
\end{align*}
We present pairs of equitable colorings lying on $\F'$ that allow us to
prove the validity of each equation in the previous system.
\begin{enumerate}
\item[(a)] Let $v'$ be a vertex not adjacent to $v$. It exists since $G$ does not have universal vertices.
Let $c^1$ be a $(n-1)$-eqcol such that $c^1(v) = c^1(v') = j$ and $c^2 = intro(c^1,v)$. We conclude that
$\lambda^X_{v j} = \lambda^X_{v n} + \lambda^W_n$.
\item[(b)] Since $j \notin S$, we know that $j \leq p$ so we can propose $p$-colorings using $j$. Let $\{ (u_1, u'_1), (u_2, u'_2), \ldots, (u_{|M|}, u'_{|M|}) \}$ be the matching $M$ of the complement of $G$ and
let $T = S \backslash \{ p + 1, \ldots, n \}$. Since $\{ p + 1, \ldots, n \} \subset S$ and $|S| = 2|M| - r$, we
have $|T| = |S| - (n - p) = |M| - r$. Moreover, $T \neq \varnothing$.\\
In order to prove $\lambda^X_{vj} = \lambda^X_{vn} + \lambda^W_n + \frac{1}{r} \lambda^W_{p+1}$, we consider
three cases:\\
\textbf{Case $v = u_1$ and $r = 1$}. Let us consider that
$T = \{ t_1, t_2, \ldots, t_{|M|-1} \}$. Let $c^1$ be a $p$-eqcol such that $c^1(u_{i+1}) = c^1(u'_{i+1}) = t_i$ for
$1 \leq i \leq |M|-1$, $c^1(u_1) = c^1(u'_1) = j$ and $c^2 = intro(c^1,u_1)$.
Therefore, $\lambda^X_{u_1 j} = \lambda^X_{u_1 p+1} + \lambda^W_{p+1}$. As condition (a) asserts that
$\lambda^X_{u_1 p+1} = \lambda^X_{u_1 n} + \lambda^W_n$, we conclude that
$\lambda^X_{u_1 j} = \lambda^X_{u_1 n} + \lambda^W_n + \lambda^W_{p+1}$.\\
\textbf{Case $v = u_1$ and $r = 2$}. Since $|M| \leq \lfloor \frac{n-1}{2} \rfloor$, we have
$|\{1,\ldots,p\} \backslash T| = p - |M| + 2 \geq 3$ and we can ensure that there exist different colors
$k, l \in \{1,\ldots,p\} \backslash (T \cup \{j\})$. Moreover, there exists a vertex
$w \in V \backslash \{u_1, u'_1, \ldots, u_{|M|}, u'_{|M|}\}$ because $M$ is not perfect.\\
Now, we propose a pair of equitable colorings (namely $c^1$ and $c^2$)
in order to obtain several equalities. Let us consider $T = \{ t_1, t_2, \ldots, t_{|M|-2} \}$ and $c^1$, $c^2$ be equitable colorings such
that $c^1(u_{i+2}) = c^1(u'_{i+2}) = t_i$ for $1 \leq i \leq |M|-2$, $c^2(i) = c^1(i)$ for
$i \in V \backslash \{ u_1, u'_1, u_2, u'_2, w \}$ and
the colors of vertices $u_1$, $u'_1$, $u_2$, $u'_2$ and $w$ are:
\begin{center} \small
\begin{tabular}{|c|c@{\hspace{3pt}}c@{\hspace{3pt}}c@{\hspace{3pt}}c@{\hspace{3pt}}c|c|c@{\hspace{3pt}}c@{\hspace{3pt}}c@{\hspace{3pt}}c@{\hspace{3pt}}c|}
\hline
 \multicolumn{6}{|c|}{$c^1$} & \multicolumn{6}{|c|}{$c^2$} \\
\hline
 size & $u_1$ & $u'_1$ & $u_2$ & $u'_2$ & $w$ &
 size & $u_1$ & $u'_1$ & $u_2$ & $u'_2$ & $w$ \\
\hline
 $p$ & $j$ & $j$   & $k$ & $k$ & $l$ & $p+1$ & $p+1$ & $p+1$ & $k$ & $j$ & $l$ \\
 $p$ & $l$ & $l$   & $j$ & $j$ & $k$ & $p$ & $l$   & $l$   & $k$ & $k$ & $j$ \\
 $n$   & $j$ & $p+1$ & $l$ & $k$ & $n$ & $n$   & $p+1$ & $j$   & $l$ & $k$ & $n$ \\
 $n$   & $l$ & $p+1$ & $k$ & $n$ & $j$ & $n$   & $l$   & $p+1$ & $j$ & $n$ & $k$ \\
\hline
\end{tabular}
\end{center}
Each combination gives us a different equality of the form $\lambda^X x_1 + \lambda^W w_1 = \lambda^X x_2 + \lambda^W w_2$, namely 
\begin{enumerate}
\item[1.] $\lambda^X_{u_1 j} + \lambda^X_{u'_1 j} + \lambda^X_{u'_2 k} =
 \lambda^X_{u_1 p+1} + \lambda^X_{u'_1 p+1} + \lambda^X_{u'_2 j} + \lambda^W_{p+1}$
\item[2.] $\lambda^X_{u_2 j} + \lambda^X_{u'_2 j} + \lambda^X_{w k} =
 \lambda^X_{u_2 k} + \lambda^X_{u'_2 k} + \lambda^X_{w j}$
\item[3.] $\lambda^X_{u_1 j} + \lambda^X_{u'_1 p+1} =
 \lambda^X_{u_1 p+1} + \lambda^X_{u'_1 j}$
\item[4.] $\lambda^X_{u_2 k} + \lambda^X_{w j} =
 \lambda^X_{u_2 j} + \lambda^X_{w k}$
\end{enumerate}
Let us note that the addition
of the previous equalities gives $2 \lambda^X_{u_1 j} = 2 \lambda^X_{u_1 p+1} + \lambda^W_{p+1}$.
Since condition (a) asserts that $\lambda^X_{u_1 p+1} = \lambda^X_{u_1 n} + \lambda^W_n$, we conclude that
$2 \lambda^X_{u_1 j} = 2 \lambda^X_{u_1 n} + 2 \lambda^W_n + \lambda^W_{p+1}$.\\
\textbf{Case $v \neq u_1$}. Let $c^1$ be a $n$-eqcol such that $c^1(v) = j$, $c^1(u_1) = n$ and
$c^2 = swap_{j,n}(c^1)$. The conditions proved recently allows us to conclude that
$\lambda^X_{vj} = \lambda^X_{vn} + \lambda^W_n + \frac{1}{r} \lambda^W_{p+1}$.
\item[(c)] Let $(x^1,w^1)$ be a $j$-eqcol and $(x^2,w^2)$ be a $\theta$-eqcol. If any of these colorings does not
lie on $\F'$, we can always swap its color classes so that it belongs to the face.
Thus $\lambda^X x^1 + \sum_{k = \theta+1}^j \lambda^W_k = \lambda^X x^2$.
In virtue of conditions (a) and (b), the previous equation becomes
\[ \sum_{v \in V} \lambda^X_{vn} + n \lambda^W_n + (n - b_{Sj}) \frac{1}{r} \lambda^W_{p+1} + \sum_{k = \theta+1}^j \lambda^W_k = \sum_{v \in V} \lambda^X_{vn} + n \lambda^W_n + (n - b_{S\theta}) \frac{1}{r} \lambda^W_{p+1}, \]
and this leads to $\sum_{k = \theta+1}^j \lambda^W_k = (b_{Sj} - b_{S\theta}) \frac{1}{r} \lambda^W_{p+1}$.
\end{enumerate}
\end{proof}

Let us present an example where the previous theorem is applied.\\

\noindent \textbf{Example.} We assume that $G$ is the graph presented in Figure \ref{minigraph2}(a). Let us note that $\overline{G}$ has the matching $\{(4,5)$, $(3,6)$, $(1,7)$, $(2,8)$, $(9,11)\}$. So, for all $S$ such that
$8 \leq |S| \leq 9$ and $\{7,\ldots,11\} \subset S$, the assumptions of Theorem \ref{TONLYCOLORS}
hold and the $S$-color inequality defines a facet of $\ECP$ as expected. Furthermore,
since $\overline{G}$ has also matchings of sizes between 2 and 5, the $S$-color
inequality defines a facet for all $S$ such that $3 \leq |S| \leq 9$ and
$\{ 11 - \lceil \frac{|S|+1}{2} \rceil, \ldots, 11\} \subset S$.\\

Unlike the last theorem, Theorems \ref{T2RANK1}, \ref{T2RANK2}, \ref{TNEIGHBOR1}, \ref{TNEIGHBOR2}
and \ref{TNEIGHBOR3} are restricted to monotone graphs. However, these results might be extended to
the general case, but the proofs behind them turn very cryptic.

%