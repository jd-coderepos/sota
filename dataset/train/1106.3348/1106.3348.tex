
\appendix

\begin{center} {\Large
\textbf{A polyhedral approach for the Equitable Coloring Problem}}\12pt]
\textsc{Online Appendix}
\end{center}

\section{Introduction} 

In this appendix we present sufficient conditions for some valid inequalities related to the \emph{Equitable Coloring Problem}
to be facet-defining inequalities.



All the proofs are based in the same technique, frequently used in the
literature for this kind of results, which is described in the following
remark.

\begin{trem} \label{TECHNIQUE}
Let  be a valid inequality for  defining a proper face .
In order to prove that  is a facet of  we have to show that, given any face  such that ,  can be written as a linear combination of  and the minimal equation system for  given in Theorem (\ref{TDIM}). This last condition becomes equivalent to prove that  verifies an equation system of  equalities.
The validity of each equality in the system is derived from the condition 
 applied on a suitable pair of equitable colorings  lying on .
\end{trem}

For the sake of simplicity, we directly present the corresponding equation system on  and the proposed equitable colorings used to derive each equation, bypassing how to get that equation system from the minimal equation system given in Theorem (\ref{TDIM}) and the inequality at hand.

As we have mentioned in Section \ref{SPOLYT}, we present equitable colorings by using mappings, color classes or binary vectors, according to our convenience.







\subsection{2-rank inequalities}

\begin{tthm} \label{T2RANK1}
Let  be a monotone graph,  such that  and . If
\begin{enumerate}
\item[(i)] there exists a stable set  of size 3 in  such that:
\begin{itemize}
\item if  is odd, the complement of  has a perfect matching  and both endpoints of some edge
of  belong to ,
\item if  is even, there exists another stable set  of size 3 in  such that
, the complement of  has a perfect matching , both
endpoints of some edge of  belong to  and there exist vertices ,  not
adjacent each other, 
\end{itemize}
\item[(ii)] for all , there exist different vertices 
and a stable set  in  such that:
\begin{itemize}
\item if  is odd, the complement of  has a perfect matching,
\item if  is even, there exists another stable set  of size 3 in  such that
 and the complement of  has a perfect matching,
\end{itemize}
\item[(iii)] for all  such that , there exists a -eqcol
where two vertices of  share the same color,
\end{enumerate}
then the -rank inequality, \ie

defines a facet of .
\end{tthm}
\begin{proof}
Let  be the face of  defined by (\ref{R2RANK1}) and
 be a face such that .
According to Remark \ref{TECHNIQUE}, we have to prove that  verifies the following equation system: 
\begin{enumerate}
\item[(a)] .
\item[(b)] .
\item[(c)] .
\item[(d)] .
\item[(e)] .
\item[(f)] If  then .
\end{enumerate}
We present pairs of equitable colorings lying on  that allow us to
prove the validity of each equation in the previous system.
\begin{enumerate}
\item[(a)] Let  be non adjacent vertices.\\
\textbf{Case }. Let  be a -eqcol such that  and
. Then, .\\
\textbf{Case }. 
Let  be a -eqcol such that ,  and . Then, 
. Since , we obtain .
\item[(b)] \textbf{Case }. 
By hypothesis (ii), there exist  such that  is a stable set.
Let  be a -eqcol such that ,  and
. Therefore, .\\
\textbf{Case  and }. By hypothesis (ii), there exist  and  such
that   is a stable set. Let  be a -eqcol such that , 
and . Therefore, .\\
\textbf{Case  and }. Let  be non adjacent vertices,
 be a -eqcol such that  and other vertex
of  is painted with color , and .
Then,  and the condition is proved for the case .
If instead , let  be a -eqcol such that , , other vertex of  is
painted with color  and . Then, 
.
Since , we conclude that .
\item[(c)] Let  and  be the stable set and the matching given by hypothesis (i). Let  be
the endpoints of an edge of  and let .\\
\textbf{Case }. Let  be a -eqcol such that ,  and
. We conclude that .\\
\textbf{Case }. Let  be a -eqcol such that , , a vertex of  is
painted with color  and . Then,
.
Since , we conclude that
.
\item[(d)] Let ,  (if  is even) and  be the stable sets and the matching given
by hypothesis (ii), and let ,  (if  is even) and  be the stable sets and the matching given by hypothesis
(i). Let  be a -eqcol such that the color class  is  and the
remaining color classes are  (if  is even) and the endpoints of edges of .
Let  be the endpoints of an edge of  and
let  be a -eqcol such that the color class  is  and the
remaining color classes are ,  (if  is even) and the endpoints of edges of  except
. These colorings imply

Applying conditions (a)-(c), this last equality becomes

giving rise to the desired result.
\item[(e)] Let us observe that from any -eqcol  and any -eqcol
 lying on  we get .
Then, applying conditions (a)-(d) yields .\\
Thus we only need to prove that, for any  such that , there exists an -eqcol 
lying on .\\
\textbf{Case }. The existence of  is guaranteed by the monotonicity of .\\
\textbf{Case }. Hypothesis (iii) guarantees the existence of an -eqcol 
where two vertices  satisfy . Then,  is an -eqcol that lies
on .\\
\textbf{Case }.  may be one of the colorings given in condition (d).\\
\textbf{Case }. Let ,  (if  is even) and  be the stable sets and the matching
given by hypothesis (i). Let  be the endpoints of an edge of  and let ,  (if 
is even) be non adjacent vertices.\\
If  is odd, color classes of  are ,  and the endpoints of edges of  where
 is the class . If instead  is even, color classes of  are , ,
 and the endpoints of edges of  where  is the class .\\
\textbf{Case }. Let us consider the -eqcol yielded in the
previous case and let  be vertices sharing a color different from .
In order to generate a -eqcol , we introduce a new color on , \ie 
where  is the -eqcol. By repeating this procedure, we can generate a
-eqcol and so on.
\item[(f)] Let  be a -eqcol such that  for some  and
 be a -eqcol (the existence of these colorings is proved above). Hence,

Application of conditions (a)-(d) yields .
\end{enumerate}
\end{proof}

Let us present an example where the previous theorem is applied.\\

\noindent \textbf{Example.}
Let  be the graph presented in Figure \ref{minigraph1}. We have that  is monotone and . 
If ,  and  is a stable set such that  has the perfect matching  with . Moreover, it is not hard to verify that for all  there exists a stable set  such that  has a perfect matching.
Then, if , the -rank inequality is a facet-defining inequality of .  
\begin{figure}[h]
  \centering
\begin{graph}(4, 2)(-2, -1)
	\roundnode{v1}(-2.3,-0.1)
	\autonodetext{v1}[w]{1}
	\roundnode{v2}(-1.7,0.1)
	\autonodetext{v2}[se]{2}
	\roundnode{v3}(-1.2,0)
	\autonodetext{v3}[n]{3}
	\roundnode{v4}(-1.7528,0.7608)
	\autonodetext{v4}[n]{4}
	\roundnode{v5}(-2.6472,0.4702)
	\autonodetext{v5}[nw]{5}
	\roundnode{v6}(-2.6472,-0.4702)
	\autonodetext{v6}[sw]{6}
	\roundnode{v7}(-1.7528,-0.7608)
	\autonodetext{v7}[s]{7}
	\roundnode{v8}(-0.4,0)
	\autonodetext{v8}[n]{8}
	\roundnode{v9}(0.4,0)
	\autonodetext{v9}[n]{9}
	\roundnode{v10}(1.2,0)
	\autonodetext{v10}[n]{10}
	\roundnode{v11}(2,0)
	\autonodetext{v11}[n]{11}
	\edge{v1}{v2}
	\edge{v1}{v3}
	\edge{v1}{v4}
	\edge{v1}{v5}
	\edge{v1}{v6}
	\edge{v1}{v7}
	\edge{v2}{v3}
	\edge{v2}{v4}
	\edge{v2}{v5}
	\edge{v2}{v6}
	\edge{v2}{v7}
	\edge{v3}{v7}
	\edge{v3}{v4}
	\edge{v4}{v5}
	\edge{v5}{v6}
	\edge{v6}{v7}
	\edge{v3}{v8}
	\edge{v8}{v9}
	\edge{v9}{v10}
	\edge{v10}{v11}
\end{graph}
  \caption{}
  \label{minigraph1}
\end{figure}

\begin{tthm} \label{T2RANK2}
Let  be a monotone graph,  such that  and .
If  and
\begin{enumerate}
\item[(i)] no connected component of the complement of  is bipartite,
\item[(ii)] for all  verifying , there exist two
vertices  and a stable set  in  such that:
\begin{itemize}
\item if  is odd, the complement of  has a perfect matching,
\item if  is even, there exists another stable set  of size 3 in  such that
 and the complement of  has a perfect matching,
\end{itemize}
\end{enumerate}
then, for all , the -2-rank inequality, \ie

defines a facet of .
\end{tthm}
\begin{proof}
Let  be different vertices of .

Let  be the face of  defined by (\ref{R2RANK2AGAIN}) and
 be a face such that .
According to Remark \ref{TECHNIQUE}, we have to prove that  verifies the following equation system: 
\begin{enumerate}
\item[(a)] .
\item[(b)] .
\item[(c)] .
\item[(d)] .
\item[(e)] .
\item[(f)] .
\item[(g)] If  then
           .
\end{enumerate}
We present pairs of equitable colorings lying on  that allow us to
prove the validity of each equation in the previous system.
\begin{enumerate}
\item[(a)] Let  and let  be a -eqcol such that
 and .
We conclude that .
\item[(b)] Let  be non adjacent vertices.\\
\textbf{Case }. Let  be a -eqcol such that  and ,
and . Then, .\\
\textbf{Case }. Let  be a -eqcol such that , . If , we make
. Otherwise, we make . From the coloring  we have
 and since  we obtain .
\item[(c)] Let  be a -eqcol such that ,  and .
Therefore, .
\item[(d)] Let  be the connected component in the complement of 
such that  is a vertex of . Since ,  does not have triangles.
By hypothesis (i),  is not bipartite and therefore there exists at least an odd cycle in  of size  with .\\
Now, let  be the minimum distance in  between  and all the odd cycles in , where the \emph{distance} from a vertex  to an odd cycle is defined as the minimum number of vertices of a path between  and a vertex of the odd cycle.
Condition (d) is proved by induction on .\\
\textbf{Case }. Then,  belongs to an odd cycle of size  in . Let  be the vertices of that odd cycle, and let 
be colors different from .\\
We denote by  the sum of two integers modulo .
Let , , ,  be -eqcols such that,
for each , , ,
, , and let
 be a -eqcol such that , , . For instance, if , colors of , ,  and  would be:
\begin{center} \small
\begin{tabular}{|c|c@{\hspace{2pt}}c@{\hspace{2pt}}c@{\hspace{2pt}}c@{\hspace{2pt}}c@{\hspace{2pt}}c|c|c@{\hspace{2pt}}c@{\hspace{2pt}}c@{\hspace{2pt}}c@{\hspace{2pt}}c@{\hspace{2pt}}c|}
\hline
 \multicolumn{7}{|c|}{ with  odd} & \multicolumn{7}{|c|}{ with  even} \\
\hline
 size &  &  &  &  &  &  &
 size &  &  &  &  &  &  \\
\hline
  &  &  &  &  &  &  &
  &  &  &  &  &  &  \\
  &  &  &  &  &  &  &
  &  &  &  &  &  &  \\
  &  &  &  &  &  &  &
  &  &  &  &  &  &  \\
\hline
\end{tabular}
\end{center}
We assume that the remaining vertices have the same color in all the colorings.
Thus, we obtain

By condition (b), we get 
.\\
\textbf{Case }. Let  be a vertex adjacent to  in  such that
. By inductive hypothesis, .\\
Let  be a -eqcol such that  and , where
. Let  be a -eqcol such that , ,  and
. Hence
. Multiplying this equality by 2, subtracting
 and
applying condition (b) yields
. 
\item[(e)] By hypothesis (ii), we can establish a -eqcol  such
that color class  is  where  (as we did in condition (d) of Theorem \ref{T2RANK1}).
Let  be the color of  in  and . We get
 by applying conditions (a)-(d).
\item[(f)] Since  is monotone, there exist a -eqcol  and a -eqcol .
If , we consider  and . If , we consider
 and . Then, we apply conditions (a)-(e)
to , where  and  are the binary variables representing
colorings  and  respectively.
\item[(g)] Let  be a -eqcol such that  and  be a -eqcol (the existence of these
colorings is proved above). Then, we apply conditions (a)-(e) to ,
where  and  are the binary variables representing colorings  and  respectively.
\end{enumerate}
\end{proof}

Theorem \ref{T2RANK2} states that, among other things,  for the
-2-rank-inequality to define a facet of . Indeed,
this condition is only used in Theorem \ref{T2RANK2} for proving equations given in (e),
\ie . So, if every vertex  verifies
, these equations
vanish from the equation system on  and the inequality (\ref{R2RANK2AGAIN}) defines a facet of  even though . We have proved the following result.

\begin{tcor} \label{T2RANK2COL}
Let  be a monotone graph,  such that  and .
If , no connected component of the complement of 
is bipartite and for all , , then the -2-rank
inequality defines a facet of  for all .
\end{tcor}

Let us present an example where the previous result is applied.\\

\noindent \textbf{Example.} Let  be the graph presented in Figure \ref{minigraph1},  and .
The -2-rank inequality is a facet-defining inequality of  for  since the
assumptions of Corollary \ref{T2RANK2COL} are satisfied: vertices  induce an odd cycle in
 and for all , .

\subsection{Subneighborhood inequalities}

\begin{tthm} \label{TNEIGHBOR1}
Let  be a monotone graph, ,  such that 
and  such that  is not a clique of  and, if  then .\\
If
\begin{enumerate}
\item[(i)] for all , there exists a
-eqcol whose color class  satisfies ,
\item[(ii)] for all , there exists an equitable coloring whose color class  satisfies
 and ,
\end{enumerate}
then the -subneighborhood inequality, \ie

defines a facet of , where
.
\end{tthm}
\begin{proof}
Let  be the face of  defined by (\ref{RNEIGHBOR1AGAIN}) and
 be a face such that .
According to Remark \ref{TECHNIQUE}, we have to prove that  verifies the following equation system: 
\begin{enumerate}
\item[(a)] .
\item[(b)] .
\item[(c)] .
\item[(d)] .
\item[(e)] .
\item[(f)] .
\item[(g)] .
\item[(h)] If  then
           .
\end{enumerate}
We present pairs of equitable colorings lying on  that allow us to
prove the validity of each equation in the previous system.
\begin{enumerate}
\item[(a)] Let  be a -eqcol such that  and . We conclude that
. 
\item[(b)] Let  be non adjacent vertices.\\
\textbf{Case }. Let  be a -eqcol such that , 
and . Then, .\\
\textbf{Case }. Let  be a -eqcol such that , ,  and
. We have .
Since , we conclude that
.
\item[(c)] Let  be a -eqcol such that ,  and .
Therefore, .
\item[(d)] \textbf{Case }. Let  be a -eqcol such that  and
 where .\\
\textbf{Case }. Let  be the -eqcol given
by hypothesis (i) and .\\
In both cases, . Now, let  and  be the color classes 
and  of  respectively. Considering  give rise to

Since , we have  and we can apply (a)-(c) in order to
get .
\item[(e)] We proceed in the same way as in (d) except that, for the case , we use the
-eqcol given by hypothesis (i) instead of the
-eqcol.
\item[(f)] In first place, let us note that  implies .
Then,  and, by hypothesis (ii), there exists a coloring  that paints 
and  vertices of  with color  but the remaining vertices of  do not use .
Let  be the color used by vertex  in  and let ,  be the color classes  and  in
 respectively, and . We have

In virtue of conditions (a)-(e), we obtain .
\item[(g)] Since  is monotone, there exist a -eqcol  and a -eqcol .
If , we consider  and . If , we consider
 and . Then, we apply conditions (a)-(f)
to , where  and  are the binary variables representing
colorings  and  respectively.
\item[(h)] Let  be a -eqcol such that  and  be a -eqcol (the existence of these
colorings is proved above). Then, we apply conditions (a)-(f) to ,
where  and  are the binary variables representing colorings  and  respectively.
\end{enumerate}
\end{proof}

Let us present two examples where the previous theorem is applied.\\

\noindent \textbf{Example.} Let  be the graph given in Figure \ref{minigraph2}(a). We have that  is
monotone and . Let us consider ,  and . In order to prove that the -subneighborhood inequality
defines a facet of , it is enough to exhibit a ()-eqcol
such that  and a ()-eqcol such that .
Both colorings are shown in Figure \ref{minigraph2} (b) and (c) respectively.

It is not hard to see that the -subneighborhood inequality is also facet-defining for
.
On the other hand, -subneighborhood inequality with 
is facet-defining by Theorem \ref{TIGHT}.

Therefore, the -subneighborhood inequality defines a facet of  for all .\\

\noindent \textbf{Example.} Let us consider again the graph given in Figure \ref{minigraph2}(a). The -subneighborhood inequality with ,  and 
is facet-defining since  and there exist the following colorings:
a ()-eqcol such that
, an equitable coloring such that  and , and
an equitable coloring such that  and .
These colorings are shown in Figure \ref{minigraph2} (b), (c) and (d) respectively.

\begin{figure}[h]
  \centering ~~~~~~~~~~~
\begin{graph}(5.8, 2)(-2, -1)
	\freetext(0.4,-0.75){(a) labeling of }
	\roundnode{v1}(-2,0)
	\autonodetext{v1}[w]{1}
	\roundnode{v2}(-1.2,0)
	\autonodetext{v2}[n]{2}
	\roundnode{v3}(-1.7528,0.7608)
	\autonodetext{v3}[n]{3}
	\roundnode{v4}(-2.6472,0.4702)
	\autonodetext{v4}[nw]{4}
	\roundnode{v5}(-2.6472,-0.4702)
	\autonodetext{v5}[sw]{5}
	\roundnode{v6}(-1.7528,-0.7608)
	\autonodetext{v6}[s]{6}
	\roundnode{v7}(-0.4,0)
	\autonodetext{v7}[n]{7}
	\roundnode{v8}(0.4,0)
	\autonodetext{v8}[n]{8}
	\roundnode{v9}(1.2,0)
	\autonodetext{v9}[n]{9}
	\roundnode{v10}(2,0)
	\autonodetext{v10}[n]{10}
	\roundnode{v11}(2.8,0)
	\autonodetext{v11}[n]{11}
	\edge{v1}{v2}
	\edge{v1}{v3}
	\edge{v1}{v4}
	\edge{v1}{v5}
	\edge{v1}{v6}
	\edge{v2}{v7}
	\edge{v7}{v8}
	\edge{v8}{v9}
	\edge{v9}{v10}
	\edge{v10}{v11}
\end{graph}~~~~~~~
\begin{graph}(5.8, 2)(-2, -1)
	\freetext(0.4,-0.75){(b) 5-eqcol in }
	\roundnode{v1}(-2,0)
	\autonodetext{v1}[w]{5}
	\roundnode{v2}(-1.2,0)
	\autonodetext{v2}[n]{1}
	\roundnode{v3}(-1.7528,0.7608)
	\autonodetext{v3}[n]{3}
	\roundnode{v4}(-2.6472,0.4702)
	\autonodetext{v4}[nw]{3}
	\roundnode{v5}(-2.6472,-0.4702)
	\autonodetext{v5}[sw]{3}
	\roundnode{v6}(-1.7528,-0.7608)
	\autonodetext{v6}[s]{2}
	\roundnode{v7}(-0.4,0)
	\autonodetext{v7}[n]{4}
	\roundnode{v8}(0.4,0)
	\autonodetext{v8}[n]{5}
	\roundnode{v9}(1.2,0)
	\autonodetext{v9}[n]{1}
	\roundnode{v10}(2,0)
	\autonodetext{v10}[n]{2}
	\roundnode{v11}(2.8,0)
	\autonodetext{v11}[n]{4}
	\edge{v1}{v2}
	\edge{v1}{v3}
	\edge{v1}{v4}
	\edge{v1}{v5}
	\edge{v1}{v6}
	\edge{v2}{v7}
	\edge{v7}{v8}
	\edge{v8}{v9}
	\edge{v9}{v10}
	\edge{v10}{v11}
\end{graph}\\~~~~~~~~~~~
\begin{graph}(5.8, 3)(-2, -1)
	\freetext(0.8,-0.75){(c) 3-eqcol with }
	\roundnode{v1}(-2,0)
	\autonodetext{v1}[w]{1}
	\roundnode{v2}(-1.2,0)
	\autonodetext{v2}[n]{3}
	\roundnode{v3}(-1.7528,0.7608)
	\autonodetext{v3}[n]{3}
	\roundnode{v4}(-2.6472,0.4702)
	\autonodetext{v4}[nw]{3}
	\roundnode{v5}(-2.6472,-0.4702)
	\autonodetext{v5}[sw]{3}
	\roundnode{v6}(-1.7528,-0.7608)
	\autonodetext{v6}[s]{2}
	\roundnode{v7}(-0.4,0)
	\autonodetext{v7}[n]{1}
	\roundnode{v8}(0.4,0)
	\autonodetext{v8}[n]{2}
	\roundnode{v9}(1.2,0)
	\autonodetext{v9}[n]{1}
	\roundnode{v10}(2,0)
	\autonodetext{v10}[n]{2}
	\roundnode{v11}(2.8,0)
	\autonodetext{v11}[n]{1}
	\edge{v1}{v2}
	\edge{v1}{v3}
	\edge{v1}{v4}
	\edge{v1}{v5}
	\edge{v1}{v6}
	\edge{v2}{v7}
	\edge{v7}{v8}
	\edge{v8}{v9}
	\edge{v9}{v10}
	\edge{v10}{v11}
\end{graph}~~~~~~~~~~~
\begin{graph}(5.8, 3)(-2, -1)
	\freetext(0.8,-0.75){(d) 3-eqcol with }
	\roundnode{v1}(-2,0)
	\autonodetext{v1}[w]{1}
	\roundnode{v2}(-1.2,0)
	\autonodetext{v2}[n]{2}
	\roundnode{v3}(-1.7528,0.7608)
	\autonodetext{v3}[n]{3}
	\roundnode{v4}(-2.6472,0.4702)
	\autonodetext{v4}[nw]{3}
	\roundnode{v5}(-2.6472,-0.4702)
	\autonodetext{v5}[sw]{3}
	\roundnode{v6}(-1.7528,-0.7608)
	\autonodetext{v6}[s]{3}
	\roundnode{v7}(-0.4,0)
	\autonodetext{v7}[n]{1}
	\roundnode{v8}(0.4,0)
	\autonodetext{v8}[n]{2}
	\roundnode{v9}(1.2,0)
	\autonodetext{v9}[n]{1}
	\roundnode{v10}(2,0)
	\autonodetext{v10}[n]{2}
	\roundnode{v11}(2.8,0)
	\autonodetext{v11}[n]{1}
	\edge{v1}{v2}
	\edge{v1}{v3}
	\edge{v1}{v4}
	\edge{v1}{v5}
	\edge{v1}{v6}
	\edge{v2}{v7}
	\edge{v7}{v8}
	\edge{v8}{v9}
	\edge{v9}{v10}
	\edge{v10}{v11}
\end{graph}
  \caption{}
  \label{minigraph2}
\end{figure}

\begin{tcor} \label{T2RANK2CASEQ1}
Let  be a monotone graph,  and  such that . Then, the -subneighborhood inequality defines a facet of .\\
Moreover, let  with  and .
If  and
for all , there exist different vertices 
and a stable set  in  such that:
\begin{itemize}
\item If  is odd, the complement of  has a perfect matching,
\item If  is even, there exists another stable set  of size 3 in  such that 
and the complement of  has a perfect matching,
\end{itemize}
then the -subneighborhood inequality defines a facet of .
\end{tcor}
\begin{proof}
\textbf{Case }. The -subneighborhood inequality defines
a facet of  since hypotheses (i) and (ii) from Theorem \ref{TNEIGHBOR1} hold trivially.\\
Now, let us consider the -subneighborhood inequality. Since , we have that
. Moreover, hypothesis (i) from Theorem \ref{TNEIGHBOR1}
holds trivially.\\
Let  and ,  and  (if  is even) be the matching and the stable sets given by the hypothesis. Consider the -eqcol such that the
color class  is  and the remaining color classes are  (if  is even) and the endpoints of edges of . Then, ,  and hypothesis (ii) from Theorem \ref{TNEIGHBOR1} holds. Therefore, the -subneighborhood inequality defines a facet of .\\
\textbf{Case }. In virtue of the previous case, we know that the
-subneighborhood and the -subneighborhood are facet-defining inequalities of
. Hence, the -subneighborhood and the -subneighborhood inequality define facets of  due to Theorem \ref{TIGHT}.
\end{proof}

\subsection{Outside-neighborhood inequalities}

\begin{tthm} \label{TNEIGHBOR2}
Let  be a monotone graph,  such that  is not a clique and .
If
\begin{enumerate}
\item[(i)] there exists  not universal in , 
\item[(ii)] if  is odd, the complement of  has a perfect matching,
\item[(iii)] for all , the following conditions hold:
\begin{itemize}
\item if  is even, the complement of  has a perfect matching,
\item if  is odd, there exists a stable set  of size 3 such that
the complement of  has a perfect matching, \end{itemize}
\item[(iv)] for all  such that , we have the following:
\begin{itemize}
\item if , then there exists an -eqcol such that
       and an -eqcol such that  and ,
\item if , then there exists an -eqcol satisfying conditions given in Remark \ref{NEIGHBOR2POINTS}, \ie lying on the face defined by (\ref{RNEIGHBOR2AGAIN}),
\end{itemize}
\end{enumerate}
then the -outside-neighborhood inequality, \ie

defines a facet of , where .
\end{tthm}
\begin{proof}
Let  be the face of  defined by (\ref{RNEIGHBOR2AGAIN}) and
 be a face such that .
According to Remark \ref{TECHNIQUE}, we have to prove that  verifies the following equation system: 
\begin{enumerate}
\item[(a)] .
\item[(b)] .
\item[(c)] .
\item[(d)] .
\item[(e)] .
\item[(f)] .
\item[(g)] If , then .
\item[(h)] .
\end{enumerate}
We present pairs of equitable colorings lying on  that allow us to
prove the validity of each equation in the previous system.
\begin{enumerate}
\item[(a)] By hypothesis (i), there exist  and  not adjacent to .\\
\textbf{Case }. Let  be a -eqcol such that
,  and .
Then, .\\
\textbf{Case }. Let  be a -eqcol such that , ,  and . We have  and
therefore .
\item[(b)] Let  be non adjacent vertices.\\
\textbf{Case }. Let  be a -eqcol such that . If
 we set , otherwise . Let .
Then, .\\
\textbf{Case }. Let  be a -eqcol such that  and . If  we set
, otherwise . Let .
We have  and therefore
.
\item[(c)] Let ,  be a -eqcol such that ,  and
. In virtue of condition (b), we obtain .
\item[(d)]  \textbf{Case  even}. Let  be the matching given by hypothesis (iii). Let  be the
-eqcol whose color classes are the endpoints of  and .
Let . We deduce that
.\\
\textbf{Case  odd}. Let  and  be the matching and the stable set given by hypothesis (iii). Let
 be the -eqcol whose color classes are , the endpoints of  and .
Now, let  be the matching given by hypothesis (ii) and let  be a vertex such that  belongs to .
Let  be the -eqcol
whose color classes are the endpoints of ,  and
.
Thus,

Conditions (a) and (b) allow us to reach the desired result.
\item[(e)] Let us notice that, if  then
,  and
. By hypothesis (iv), there exists an -eqcol
 such that  contains all the vertices painted with color . Let  and
 be the -eqcol that paints vertex  and  vertices of
 with color  also given by hypothesis (iv).
By condition (c), we have .
Applying conditions (a), (b) and (d), we get
.
\item[(f)] \textbf{Case }. We proceed in the same way as in (e),
but using  instead of .\\
\textbf{Case }. Then, . Let ,  be a
-eqcol such that ,  and .
Conditions (b) and (c) allow us to obtain .
\item[(g)-(h)] This condition can be verified by providing a -eqcol  and a -eqcol
 lying on  and applying conditions (a)-(f) to equation .\\
Thus, we only need to prove that, for any , there exists an -eqcol  lying on  .\\ 
\textbf{Case }. The existence of  is guaranteed by the monotonicity of G.\\
\textbf{Case }. The existence of  is guaranteed by hypothesis (iv).\\
\textbf{Case }.  may be the -eqcol yielded by condition (d).\\
\textbf{Case }. Let us consider the
-eqcol yielded in the previous case and let  be vertices sharing a color
different from . In order to generate a -eqcol , we introduce a new color on ,
\ie  where  is the -eqcol. By repeating this
procedure, we can generate a -eqcol and so on.
\end{enumerate}
\end{proof}

Let us present an example where the previous theorem is applied.\\

\noindent \textbf{Example.} Let  be the graph given in Figure \ref{minigraph2}(a). Let us recall that  is
monotone and .
We apply Theorem \ref{TNEIGHBOR2} considering  and . It is not hard to see that the assumptions of
this theorem are satisfied. Below, we present some examples of colorings related to hypothesis (iv) of
Theorem \ref{TNEIGHBOR2}. Figure \ref{minigraph3}(a) shows a 3-eqcol of  such that 
and Figure \ref{minigraph3}(b) shows a 3-eqcol of  such that  and .


By Theorem \ref{TIGHT2}, -outside-neighborhood inequalities with  are also facet-defining.

\begin{figure}[h]
  \centering ~~~~~~~~~~~
\begin{graph}(5.8, 3)(-2, -1)
	\freetext(0.8,-0.75){(a) 3-eqcol, }
	\roundnode{v1}(-2,0)
	\autonodetext{v1}[w]{1}
	\roundnode{v2}(-1.2,0)
	\autonodetext{v2}[n]{3}
	\roundnode{v3}(-1.7528,0.7608)
	\autonodetext{v3}[n]{3}
	\roundnode{v4}(-2.6472,0.4702)
	\autonodetext{v4}[nw]{3}
	\roundnode{v5}(-2.6472,-0.4702)
	\autonodetext{v5}[sw]{3}
	\roundnode{v6}(-1.7528,-0.7608)
	\autonodetext{v6}[s]{2}
	\roundnode{v7}(-0.4,0)
	\autonodetext{v7}[n]{1}
	\roundnode{v8}(0.4,0)
	\autonodetext{v8}[n]{2}
	\roundnode{v9}(1.2,0)
	\autonodetext{v9}[n]{1}
	\roundnode{v10}(2,0)
	\autonodetext{v10}[n]{2}
	\roundnode{v11}(2.8,0)
	\autonodetext{v11}[n]{1}
	\edge{v1}{v2}
	\edge{v1}{v3}
	\edge{v1}{v4}
	\edge{v1}{v5}
	\edge{v1}{v6}
	\edge{v2}{v7}
	\edge{v7}{v8}
	\edge{v8}{v9}
	\edge{v9}{v10}
	\edge{v10}{v11}
\end{graph}~~~~~~~
\begin{graph}(5.8, 3)(-2, -1)
	\freetext(1.0,-0.75){(b) 3-eqcol, , }
	\roundnode{v1}(-2,0)
	\autonodetext{v1}[w]{3}
	\roundnode{v2}(-1.2,0)
	\autonodetext{v2}[n]{2}
	\roundnode{v3}(-1.7528,0.7608)
	\autonodetext{v3}[n]{1}
	\roundnode{v4}(-2.6472,0.4702)
	\autonodetext{v4}[nw]{1}
	\roundnode{v5}(-2.6472,-0.4702)
	\autonodetext{v5}[sw]{1}
	\roundnode{v6}(-1.7528,-0.7608)
	\autonodetext{v6}[s]{2}
	\roundnode{v7}(-0.4,0)
	\autonodetext{v7}[n]{3}
	\roundnode{v8}(0.4,0)
	\autonodetext{v8}[n]{2}
	\roundnode{v9}(1.2,0)
	\autonodetext{v9}[n]{3}
	\roundnode{v10}(2,0)
	\autonodetext{v10}[n]{2}
	\roundnode{v11}(2.8,0)
	\autonodetext{v11}[n]{1}
	\edge{v1}{v2}
	\edge{v1}{v3}
	\edge{v1}{v4}
	\edge{v1}{v5}
	\edge{v1}{v6}
	\edge{v2}{v7}
	\edge{v7}{v8}
	\edge{v8}{v9}
	\edge{v9}{v10}
	\edge{v10}{v11}
\end{graph}
  \caption{}
  \label{minigraph3}
\end{figure}

\subsection{Clique-neighborhood inequalities}

\begin{tthm} \label{TNEIGHBOR3}
Let  be a monotone graph, ,  be a clique of  such that  and 
be numbers verifying  and
. If
\begin{enumerate}
\item[(i)] for all , there exist
,  and two -eqcols such that in one of them
 and in the other  ( and  may be the same vertex),
\item[(ii)] for all  such that , we have the following:
\begin{itemize}
\item if , there exist  and
a -eqcol such that ,  and ,
\item if , there exists a -eqcol
satisfying conditions given in Remark \ref{NEIGHBOR3POINTS}, \ie lying on the face defined by (\ref{RNEIGHBOR3AGAIN}),
\end{itemize} \end{enumerate}
then the -clique-neighborhood inequality, \ie

defines a facet of , where

\end{tthm}
\begin{proof}
Let  be the face of  defined by (\ref{RNEIGHBOR3AGAIN}) and
 be a face such that .
According to Remark \ref{TECHNIQUE}, we have to prove that  verifies the following equation system: 
\begin{enumerate}
\item[(a)] .
\item[(b)] .
\item[(c)] .
\item[(d)] .
\item[(e)] .
\item[(f)] .
\item[(g)] .
\item[(h)] .
\end{enumerate}
We present pairs of equitable colorings lying on  that allow us to
prove the validity of each equation in the previous system.
\begin{enumerate}
\item[(a)] Let ,  be a -eqcol such that  and .
Then, .
\item[(b)] \textbf{Case }. Let , ,  be a -eqcol
such that ,  and . Then, .\\
Now, let  be non adjacent vertices.\\
\textbf{Case }. Let  be a -eqcol such that ,
 and . We have .\\
\textbf{Case }. Let  be a -eqcol such that , ,  and .
We have  and, since
, we obtain .
\item[(c)] Let  be non adjacent vertices and .\\
\textbf{Case }. Let  be a -eqcol such that ,
 and . Then, .\\
\textbf{Case }. Let  be a -eqcol such that , ,  and
. We have  and,
since , we obtain
.
\item[(d)] Let .\\
\textbf{Case }. Let  be a -eqcol such that  and
. Then, .\\
\textbf{Case }. Let  be a -eqcol such that , ,  and .
We have  and, since , we obtain .
\item[(e)] Hypothesis (i) ensures that there exists an equitable coloring  such that
 and the remaining vertices do not use color , and there exists
another equitable coloring  (with the same number of colors) such that 
and the remaining vertices do not use color , where . We have

and, by conditions (b)-(d), we derive
.
\item[(f)] Let . Clearly,  and
. By hypothesis (ii), there exists a -eqcol  whose
class color  satisfies  and  and  and a vertex of  use
color . Let  and  (since  and , both colorings are
well-defined). Hence,  and . We apply conditions proved before to
, where  and  are the binary variables representing colorings
 and  respectively, and we conclude that
.
\item[(g)] \textbf{Case }. We proceed in the same way as in (f), but using
 instead of
.\\
\textbf{Case }. Let  be non adjacent vertices and . Let  be a -eqcol such that
,  and . We apply conditions proved before
to , where  and  are the binary variables representing colorings
 and  respectively, and we conclude that .
\item[(h)] This condition can be verified by providing an -eqcol  and an -eqcol
 lying on  and applying conditions (a)-(g) to equation .\\
Thus, we only need to prove that, for any , there exists a -eqcol  lying on  .\\ 
\textbf{Case }. The existence of  is guaranteed by the monotonicity of G.\\
\textbf{Case }. The existence of  is guaranteed by hypothesis (ii).\\
\textbf{Case }.  may be the -eqcol yielded by condition (c).
\end{enumerate}
\end{proof}

\begin{tcor} \label{TNEIGHBOR3COR}
Let  be a monotone graph and let , , ,  be defined as in Theorem \ref{TNEIGHBOR3}.
If hypothesis (ii) of Theorem \ref{TNEIGHBOR3} holds and for all :
\begin{itemize}
\item if  is odd,
\begin{itemize}
\item there exists a vertex  and a stable set  such that the complement of 
has a perfect matching ,
\item there exists a vertex  and two disjoint stable sets ,  such that
 and the complement of  has a perfect matching  ,
\end{itemize}
\item if  is even,
\begin{itemize}
\item there exists a vertex  and two disjoint stable sets ,  such that
 and the complement of  has a perfect matching ,
\item there exists a vertex  and three disjoint stable sets , , 
such that  and the complement of  has a perfect matching  ,
\end{itemize}
\end{itemize}
then the -clique-neighborhood inequality defines a facet of .
\end{tcor}
\begin{proof}
Let us suppose that  is odd.
Let  and let , , ,  and  be the matchings and the
stable sets given in the hypothesis. Consider an -eqcol such that the color class
 is  and the remaining color classes are the endpoints of edges of , and an
-eqcol such that the color class  is  and the remaining color classes are 
and the endpoints of edges of . Therefore, hypothesis (i) of Theorem \ref{TNEIGHBOR3} holds and
the -clique-neighborhood inequality defines a facet of .

The proof for  even is analogous to the previous one.
\end{proof}

Let us present an example where the previous result is applied.\\

\noindent \textbf{Example.} Let  be the graph given in Figure \ref{minigraph2}(a). Let us recall that  is
monotone and . We apply Corollary \ref{TNEIGHBOR3COR} considering , ,  and
. It is not hard to see that the assumptions of this corollary are satisfied.
Below, we present some examples of colorings related to hypothesis (ii) of Theorem \ref{TNEIGHBOR3}.
Figure \ref{minigraph4}(a) shows a 3-eqcol of  such that , ,  and
Figure \ref{minigraph4}(b) shows a 5-eqcol of  such that , , .


\begin{figure}[h]
  \centering ~~~~~~~~~~~
\begin{graph}(5.8, 2)(-2, -1)
	\freetext(0.6,-0.75){(a) 3-eqcol of }
	\roundnode{v1}(-2,0)
	\autonodetext{v1}[w]{3}
	\roundnode{v2}(-1.2,0)
	\autonodetext{v2}[n]{1}
	\roundnode{v3}(-1.7528,0.7608)
	\autonodetext{v3}[n]{1}
	\roundnode{v4}(-2.6472,0.4702)
	\autonodetext{v4}[nw]{1}
	\roundnode{v5}(-2.6472,-0.4702)
	\autonodetext{v5}[sw]{1}
	\roundnode{v6}(-1.7528,-0.7608)
	\autonodetext{v6}[s]{2}
	\roundnode{v7}(-0.4,0)
	\autonodetext{v7}[n]{3}
	\roundnode{v8}(0.4,0)
	\autonodetext{v8}[n]{2}
	\roundnode{v9}(1.2,0)
	\autonodetext{v9}[n]{3}
	\roundnode{v10}(2,0)
	\autonodetext{v10}[n]{2}
	\roundnode{v11}(2.8,0)
	\autonodetext{v11}[n]{3}
	\edge{v1}{v2}
	\edge{v1}{v3}
	\edge{v1}{v4}
	\edge{v1}{v5}
	\edge{v1}{v6}
	\edge{v2}{v7}
	\edge{v7}{v8}
	\edge{v8}{v9}
	\edge{v9}{v10}
	\edge{v10}{v11}
\end{graph}~~~~~~~
\begin{graph}(5.8, 2)(-2, -1)
	\freetext(0.6,-0.75){(b) 5-eqcol of }
	\roundnode{v1}(-2,0)
	\autonodetext{v1}[w]{5}
	\roundnode{v2}(-1.2,0)
	\autonodetext{v2}[n]{1}
	\roundnode{v3}(-1.7528,0.7608)
	\autonodetext{v3}[n]{1}
	\roundnode{v4}(-2.6472,0.4702)
	\autonodetext{v4}[nw]{1}
	\roundnode{v5}(-2.6472,-0.4702)
	\autonodetext{v5}[sw]{2}
	\roundnode{v6}(-1.7528,-0.7608)
	\autonodetext{v6}[s]{4}
	\roundnode{v7}(-0.4,0)
	\autonodetext{v7}[n]{5}
	\roundnode{v8}(0.4,0)
	\autonodetext{v8}[n]{3}
	\roundnode{v9}(1.2,0)
	\autonodetext{v9}[n]{2}
	\roundnode{v10}(2,0)
	\autonodetext{v10}[n]{3}
	\roundnode{v11}(2.8,0)
	\autonodetext{v11}[n]{4}
	\edge{v1}{v2}
	\edge{v1}{v3}
	\edge{v1}{v4}
	\edge{v1}{v5}
	\edge{v1}{v6}
	\edge{v2}{v7}
	\edge{v7}{v8}
	\edge{v8}{v9}
	\edge{v9}{v10}
	\edge{v10}{v11}
\end{graph}
\caption{}
  \label{minigraph4}
\end{figure}

\subsection{-color inequalities}

\begin{tthm} \label{TONLYCOLORS}
Let  be a matching of the complement of  such that 
and let  such that  with  and  contains all the colors
greater than . Then, the -color inequality, \ie

defines a facet of , where

\end{tthm}
\begin{proof}
Let us note that . If ,  and the -color inequality defines the same
face as the -color inequality as stated in Remark \ref{REMARKONLY}.1. But the -color inequality
is the constraint (\ref{RUPPER}) with  by Remark \ref{REMARKONLY}.2, which is facet-defining by Theorem \ref{TORIGINAL4}. Then, the -color inequality defines a facet of . So, from now on we assume that
.

For the sake of simplicity, we define .

Now, let  be the face of  defined by (\ref{RONLYCOLORSAGAIN}) and
 be a face such that .
According to Remark \ref{TECHNIQUE}, we have to prove that  verifies the following equation system: 

We present pairs of equitable colorings lying on  that allow us to
prove the validity of each equation in the previous system.
\begin{enumerate}
\item[(a)] Let  be a vertex not adjacent to . It exists since  does not have universal vertices.
Let  be a -eqcol such that  and . We conclude that
.
\item[(b)] Since , we know that  so we can propose -colorings using . Let  be the matching  of the complement of  and
let . Since  and , we
have . Moreover, .\\
In order to prove , we consider
three cases:\\
\textbf{Case  and }. Let us consider that
. Let  be a -eqcol such that  for
,  and .
Therefore, . As condition (a) asserts that
, we conclude that
.\\
\textbf{Case  and }. Since , we have
 and we can ensure that there exist different colors
. Moreover, there exists a vertex
 because  is not perfect.\\
Now, we propose a pair of equitable colorings (namely  and )
in order to obtain several equalities. Let us consider  and ,  be equitable colorings such
that  for ,  for
 and
the colors of vertices , , ,  and  are:
\begin{center} \small
\begin{tabular}{|c|c@{\hspace{3pt}}c@{\hspace{3pt}}c@{\hspace{3pt}}c@{\hspace{3pt}}c|c|c@{\hspace{3pt}}c@{\hspace{3pt}}c@{\hspace{3pt}}c@{\hspace{3pt}}c|}
\hline
 \multicolumn{6}{|c|}{} & \multicolumn{6}{|c|}{} \\
\hline
 size &  &  &  &  &  &
 size &  &  &  &  &  \\
\hline
  &  &    &  &  &  &  &  &  &  &  &  \\
  &  &    &  &  &  &  &    &    &  &  &  \\
    &  &  &  &  &  &    &  &    &  &  &  \\
    &  &  &  &  &  &    &    &  &  &  &  \\
\hline
\end{tabular}
\end{center}
Each combination gives us a different equality of the form , namely 
\begin{enumerate}
\item[1.] 
\item[2.] 
\item[3.] 
\item[4.] 
\end{enumerate}
Let us note that the addition
of the previous equalities gives .
Since condition (a) asserts that , we conclude that
.\\
\textbf{Case }. Let  be a -eqcol such that ,  and
. The conditions proved recently allows us to conclude that
.
\item[(c)] Let  be a -eqcol and  be a -eqcol. If any of these colorings does not
lie on , we can always swap its color classes so that it belongs to the face.
Thus .
In virtue of conditions (a) and (b), the previous equation becomes

and this leads to .
\end{enumerate}
\end{proof}

Let us present an example where the previous theorem is applied.\\

\noindent \textbf{Example.} We assume that  is the graph presented in Figure \ref{minigraph2}(a). Let us note that  has the matching , , , , . So, for all  such that
 and , the assumptions of Theorem \ref{TONLYCOLORS}
hold and the -color inequality defines a facet of  as expected. Furthermore,
since  has also matchings of sizes between 2 and 5, the -color
inequality defines a facet for all  such that  and
.\\

Unlike the last theorem, Theorems \ref{T2RANK1}, \ref{T2RANK2}, \ref{TNEIGHBOR1}, \ref{TNEIGHBOR2}
and \ref{TNEIGHBOR3} are restricted to monotone graphs. However, these results might be extended to
the general case, but the proofs behind them turn very cryptic.

%