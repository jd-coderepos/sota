\documentclass{llncs}
\usepackage{times}
\usepackage{amssymb}
\usepackage{url}
\usepackage{epsfig}
\usepackage{ifthen}
\usepackage{cite}
\usepackage{color}


\pagestyle{headings}

\newcommand{\eps}{\ensuremath{\varepsilon}}
\newcommand{\XX}[1]
    {\marginpar{\raggedright{} #1}}

\begin{document}

\mainmatter

\title{Detecting Useless Transitions in Pushdown Automata}

\author{Wan Fokkink \and Dick Grune \and Brinio Hond \and Peter Rutgers}
\institute{VU University Amsterdam, Department of Computer Science}

\maketitle

\begin{abstract}
Pushdown automata may contain transitions that are never used in any accepting run of the automaton.
We present an algorithm for detecting such useless transitions.
A finite automaton that captures the possible stack content during runs of the pushdown automaton,
is first constructed in a forward procedure to determine which transitions are reachable, and then
employed in a backward procedure to determine which of these transitions can lead to a final state.
\end{abstract}


\section{Introduction}\label{sec:intro2}

Context-free languages are used in language specification, parsing, and code optimization.
They are defined by means of a context-free grammar or a pushdown automaton (pda).
Some languages can be specified more efficiently by a pda
than by a context-free grammar, as shown by Goldstine, Price, and Wotschke \cite{GoldstinePW82econ}.
Pda's are at the root of deterministic parsers for context-free languages (notably LL, LR),
see e.g.\ \cite{Substring.Nederhof.1996,LR.Bertsch.1999}.
We consider pda's in which any number of symbols can be popped
from as well as pushed onto the stack in one transition.
Popping zero or multiple symbols is useful in bottom-up parsing, and facilitates the reversal of a pda.


For context-free grammars, it is rather straightforward to determine
whether a production is \emph{useless}, i.e., cannot occur in a derivation from
the start variable to a string of terminal symbols; such a method is discussed
in many textbooks on formal languages (e.g., \cite[Theorem 6.2]{Linz11}).
It consists of two parts: Detect which variables are reachable from the start variable,
and which variables can be transformed into a string of terminal symbols.
Productions that contain a useless variable, not satisfying these two properties,
can be removed from the grammar without changing the associated language.

We present an algorithm to detect whether a transition in a pda is useless,
meaning that no run of the pda from the initial configuration to a final state
includes this transition. Such a transition can be removed from the pda without
changing the language accepted by the pda. Removing useless transitions, which
may improve the performance of running the pda, is especially sensible if
the pda has been generated automatically, because then there tend to be useless
transitions present.

Similar to detecting useless variables in context-free grammars,
our algorithm for detecting useless transitions in a pda consists of
two parts. The first part finds which transitions are not reachable from
the initial configuration.
Here we exploit an algorithm by Finkel, Willems and Wolper~\cite{FinkelWW97}
to construct a finite automaton (nfa) that captures exactly all
possible stacks in the reachable configurations of a pda.
Their approach is modified to take into account that multiple symbols may be popped from
the stack at once. The second part of our algorithm finds after which transitions
it is impossible to reach a final state.  Here we use the nfa
constructed in the first part to compute in a backward fashion which transitions
can lead to a final state in the pda.

We prove that the algorithm marks exactly the useless transitions.
The worst-case time complexity of the algorithm is , with
 the number of states and  the number of transitions of the pda.
This worst case actually only occurs in the unlikely case that the nfa
is constructed over a large number of iterations, is saturated with
-transitions, and contains a lot of backward nondeterminism.
A prototype implementation of the algorithm exhibits a good performance.


An alternative approach is to transform the pda under consideration into an equivalent
context-free grammar, and then determine the useless productions.
Another alternative approach is to check for each transition separately whether it is useless:
Provide the transition with a special input symbol , all other transitions in the pda
with empty input , and check whether the language accepted by the resulting pda intersected
with the regular language  is empty. Checking emptiness of pda's is generally
performed by a conversion to a context-free grammar.
Disadvantage of these approaches is that the resulting grammar tends to be much larger than the original pda.

Bouajjani, Esparza and Maler \cite{BouajjaniEM97} employed a method similar to the one in \cite{FinkelWW97}
to capture the reachable configurations of a pda via an nfa, in the context of model checking infinite-state systems.
Griffin~\cite{Griffin06} showed how to detect which transitions are reachable
from the initial configuration in a deterministic pushdown automaton (dpda).
For each transition, the algorithm creates a temporary dpda in which the
successive state of the transition is set to a new, final state; all other
states in the temporary dpda are made non-final. Then it is checked whether the
language generated by the dpda is empty; if it is, the transition is unreachable.
This algorithm determines which transitions are reachable
from the initial configuration, but not which transitions can lead to a final state.
Vice versa, Kutrib and Malcher \cite{KutribMalcher12} studied reversibility of dpda's.




\section{Preliminaries}

\begin{definition}\label{def:NPDA}
A nondeterministic pushdown automaton (pda) consists of
a finite set of states , a finite input alphabet , a finite stack alphabet ,
a finite transition relation ,
an initial state , and a set  of final states.
\end{definition}

\noindent
In this definition,  denotes the empty string. Note that
zero or multiple symbols can be popped from the stack in one transition.
It is assumed that the initial stack is empty. (An arbitrary initial stack  can be constructed
by adding a new initial state  and a transition .)

A configuration consists of a state from  together with a stack from .
We let  denote elements in , and  strings in .
The reverse of a string  is denoted by .
A transition  (or )
gives rise to moves  (or )
between configurations, for any .
The language accepted by a pda consists of the strings in  that give rise to a run
of the pda from the initial configuration  to a configuration  with .

A transition in a pda is \emph{useless} if no run of the pda from the initial configuration
 to a configuration  with , for any input string from ,
includes this transition. To determine the useless transitions, input strings from  are irrelevant.
The point is that, since a run for any input string suffices to make a transition useful,
we can assume that any desired terminal symbol from  is available as input at any time.
Input strings from  are therefore disregarded in our algorithm to detect useless transitions.
(In the context of model-checking infinite-state systems, pda's in which input strings are disregarded
are called ``pushdown systems'' or ``pushdown processes'' \cite{Walukiewicz96}.)

A transition  is written as .
It gives rise to moves .
We write  if there is a run from  to  of the pda,
consisting of zero or more moves.

\begin{definition}\label{def:NFA}
A nondeterministic finite automaton (nfa) consists of
a finite set of states , a finite input alphabet ,
a transition relation , an initial state ,
and a set  of final states.
\end{definition}

\noindent
In our application of nfa's, the input alphabet is the stack alphabet  from the pda.

A transition  is written as .
We write  if there is a path from  to  in the nfa
with consecutive labels .
We write  if there is such a path from  to ,
possibly intertwined with transitions labeled by .
The language accepted by an nfa consists of the strings  in  for which there
exists a path  with .


\section{Detecting the useless transitions in a pda}

Our algorithm for detecting useless transitions in a pda summarizes all
reachable configurations of the pda in an nfa. As a first step, an nfa is constructed
that accepts the stacks that can occur during any run of the pda.
A second step determines which transitions can lead to a configuration
from which a final state can be reached. Transitions that cannot be reached from the
initial start (as determined in step ), or that cannot lead to a final state (as determined in step )
are useless.


\subsection{Detecting the unreachable transitions}
\label{sec:forward}

A configuration or transition in a pda  is reachable if it is employed in a run of ,
starting from the initial configuration. The reachable configurations of  are captured by means of an nfa .
The stacks in  that can occur at a state  in  are accepted at the state  in ,
in reverse order. During the construction of , intermediate non-final states
are created when multiple symbols are pushed onto the stack in one transition.
They are denoted by , to distinguish them from the final states  that are
inherited from . A state in  that may be either final or non-final is denoted by .

Fix a pda ; as said, we will disregard .
To achieve a single final state without outgoing transitions that is only reached with an empty stack, a fresh stack symbol 
is added to , and the initial stack is  (instead of ). In each run of the pda,  is always at the bottom of the stack.
Fresh states  and  are added to , and  is extended with transitions  for every ,
 for every , and .
We change  to . The resulting pda is called .

Initially the nfa  under construction consists of the transition ;
the fresh state  is non-final and  final. Intuitively, this transition builds the initial stack of .
The set  of unreachable transitions in  initially, as an overapproximation, contains all transitions in .
The nfa  and the set  are constructed as follows.

\vspace{1.5mm}

\noindent
Procedure {\em forward}: For each transition  in  do:
\begin{enumerate}
\item[1.~~]
If  is not a state in , then stop this iteration step.\vspace{1mm}
\item[2.~~]
Determine the set , which either consists only of , if , or of the states  for which
there exists a path  in , if .\vspace{1mm}
\item[3.~~]
If , then stop this iteration step.
\item[4.~~]
If , then delete  from , and establish a path  in  (see below); the state  in  is final.\vspace{1mm}
\item[5.~~]
Let  be the first state in the path . For each state , if the transition 
is not yet present in , then add this transition to .
\end{enumerate}

\vspace{1mm}

\noindent
If  changed during this run, then perform the {\em forward} procedure again, over all transitions in .
Else, stop, and return the constructed nfa  and the set  of unreachable transitions in .
These transitions are then culled from , producing the pda .
The sets  need to be recomputed in every run of the {\em forward} procedure.
The sets  computed in the last run of the {\em forward} procedure are stored,
as they will be used in the {\em backward} procedure in the next section.

The procedure called in step 4, which establishes a path  in 
and returns the first state in this path, is defined as follows.
\begin{itemize}
\item[4.1]
If , then return , and stop.\vspace{1mm}
\item[4.2]
If there is a transition  in  with  non-final (there is at most one such transition),
then establish a path  in , return the first state in this path, and stop.\vspace{1mm}
\item[4.3]
Add non-final states  and transitions  to ,
and return .
\end{itemize}

The idea behind the construction of  is as follows. Given a transition
 in , pushing  onto the stack and moving to state 
corresponds to a path  in . A state  in  can jump
to the first state in this path if there is a path  in ,
because then we can reach  from  by pushing  onto the stack.
By executing , we pop  from the stack, leading back to , then jump to , and push  onto the stack
via the path .
This jump is captured in  by an -transition from (every possible)  to .
To reduce the number of -transitions in , we only consider those  with a path 
in  that does not start with an -transition.

For each transition in  at most one path is established in , and for the rest  consists of
-transitions (between states in such a path), so the construction of  always terminates.
The set  returned at the end contains exactly the unreachable transitions in .
A proof of this fact is presented at the end of this section.

We give an example construction of nfa  from a pda.
As usual, the initial state in pda's and nfa's is drawn with an incoming arrow, and final states as a double circle.

\begin{example}
\label{exa:nfa1}
Consider the following pda .

\vspace{3mm}

\centerline{\begin{picture}(0,0)\includegraphics{npda.pdf}\end{picture}\setlength{\unitlength}{3947sp}\begingroup\makeatletter\ifx\SetFigFont\undefined \gdef\SetFigFont#1#2#3#4#5{\reset@font\fontsize{#1}{#2pt}\fontfamily{#3}\fontseries{#4}\fontshape{#5}\selectfont}\fi\endgroup \begin{picture}(4704,1006)(2314,-760)
\put(2650,-536){\makebox(0,0)[b]{\smash{{\SetFigFont{10}{12.0}{\rmdefault}{\mddefault}{\updefault}{\color[rgb]{0,0,0}}}}}}
\put(3565, 57){\makebox(0,0)[b]{\smash{{\SetFigFont{10}{12.0}{\rmdefault}{\mddefault}{\updefault}{\color[rgb]{0,0,0}}}}}}
\put(3516,-633){\makebox(0,0)[b]{\smash{{\SetFigFont{10}{12.0}{\rmdefault}{\mddefault}{\updefault}{\color[rgb]{0,0,0}}}}}}
\put(3957,-322){\makebox(0,0)[rb]{\smash{{\SetFigFont{10}{12.0}{\rmdefault}{\mddefault}{\updefault}{\color[rgb]{0,0,0}}}}}}
\put(3080,-134){\makebox(0,0)[rb]{\smash{{\SetFigFont{10}{12.0}{\rmdefault}{\mddefault}{\updefault}{\color[rgb]{0,0,0}}}}}}
\put(3142,-322){\makebox(0,0)[lb]{\smash{{\SetFigFont{10}{12.0}{\rmdefault}{\mddefault}{\updefault}{\color[rgb]{0,0,0}}}}}}
\put(4485,-536){\makebox(0,0)[b]{\smash{{\SetFigFont{10}{12.0}{\rmdefault}{\mddefault}{\updefault}{\color[rgb]{0,0,0}}}}}}
\put(5045,-696){\makebox(0,0)[b]{\smash{{\SetFigFont{10}{12.0}{\rmdefault}{\mddefault}{\updefault}{\color[rgb]{0,0,0}}}}}}
\put(5045,-385){\makebox(0,0)[b]{\smash{{\SetFigFont{10}{12.0}{\rmdefault}{\mddefault}{\updefault}{\color[rgb]{0,0,0}}}}}}
\put(4112,-134){\makebox(0,0)[lb]{\smash{{\SetFigFont{10}{12.0}{\rmdefault}{\mddefault}{\updefault}{\color[rgb]{0,0,0}}}}}}
\put(6216,-422){\makebox(0,0)[b]{\smash{{\SetFigFont{10}{12.0}{\rmdefault}{\mddefault}{\updefault}{\color[rgb]{0,0,0}}}}}}
\put(5678,-536){\makebox(0,0)[b]{\smash{{\SetFigFont{10}{12.0}{\rmdefault}{\mddefault}{\updefault}{\color[rgb]{0,0,0}}}}}}
\put(6869,-536){\makebox(0,0)[b]{\smash{{\SetFigFont{10}{12.0}{\rmdefault}{\mddefault}{\updefault}{\color[rgb]{0,0,0}}}}}}
\end{picture} }

\vspace{3mm}

\noindent
We have taken the liberty to omit the state  from ,
to keep the example small, and since the state  is always reached with the stack .

To determine the reachable transitions in , the following nfa  is constructed.

\vspace{4mm}

\centerline{\begin{picture}(0,0)\includegraphics{nfa1.pdf}\end{picture}\setlength{\unitlength}{3947sp}\begingroup\makeatletter\ifx\SetFigFont\undefined \gdef\SetFigFont#1#2#3#4#5{\reset@font\fontsize{#1}{#2pt}\fontfamily{#3}\fontseries{#4}\fontshape{#5}\selectfont}\fi\endgroup \begin{picture}(5059,2010)(967,-3338)
\put(2053,-2134){\makebox(0,0)[b]{\smash{{\SetFigFont{10}{12.0}{\rmdefault}{\mddefault}{\updefault}{\color[rgb]{0,0,0}}}}}}
\put(2466,-3132){\makebox(0,0)[rb]{\smash{{\SetFigFont{10}{12.0}{\rmdefault}{\mddefault}{\updefault}{\color[rgb]{0,0,0}}}}}}
\put(4347,-3068){\makebox(0,0)[lb]{\smash{{\SetFigFont{10}{12.0}{\rmdefault}{\mddefault}{\updefault}{\color[rgb]{0,0,0}}}}}}
\put(3906,-3230){\makebox(0,0)[b]{\smash{{\SetFigFont{10}{12.0}{\rmdefault}{\mddefault}{\updefault}{\color[rgb]{0,0,0}}}}}}
\put(4432,-1721){\makebox(0,0)[rb]{\smash{{\SetFigFont{10}{12.0}{\rmdefault}{\mddefault}{\updefault}{\color[rgb]{0,0,0}}}}}}
\put(4917,-2723){\makebox(0,0)[b]{\smash{{\SetFigFont{10}{12.0}{\rmdefault}{\mddefault}{\updefault}{\color[rgb]{0,0,0}}}}}}
\put(4432,-2555){\makebox(0,0)[rb]{\smash{{\SetFigFont{10}{12.0}{\rmdefault}{\mddefault}{\updefault}{\color[rgb]{0,0,0}}}}}}
\put(5877,-2134){\makebox(0,0)[b]{\smash{{\SetFigFont{10}{12.0}{\rmdefault}{\mddefault}{\updefault}{\color[rgb]{0,0,0}}}}}}
\put(5474,-1721){\makebox(0,0)[lb]{\smash{{\SetFigFont{10}{12.0}{\rmdefault}{\mddefault}{\updefault}{\color[rgb]{0,0,0}}}}}}
\put(5474,-2555){\makebox(0,0)[lb]{\smash{{\SetFigFont{10}{12.0}{\rmdefault}{\mddefault}{\updefault}{\color[rgb]{0,0,0}}}}}}
\put(4917,-1524){\makebox(0,0)[b]{\smash{{\SetFigFont{10}{12.0}{\rmdefault}{\mddefault}{\updefault}{\color[rgb]{0,0,0}}}}}}
\put(2507,-1721){\makebox(0,0)[rb]{\smash{{\SetFigFont{10}{12.0}{\rmdefault}{\mddefault}{\updefault}{\color[rgb]{0,0,0}}}}}}
\put(2993,-2723){\makebox(0,0)[b]{\smash{{\SetFigFont{10}{12.0}{\rmdefault}{\mddefault}{\updefault}{\color[rgb]{0,0,0}}}}}}
\put(3051,-2481){\makebox(0,0)[lb]{\smash{{\SetFigFont{10}{12.0}{\rmdefault}{\mddefault}{\updefault}{\color[rgb]{0,0,0}}}}}}
\put(2507,-2555){\makebox(0,0)[rb]{\smash{{\SetFigFont{10}{12.0}{\rmdefault}{\mddefault}{\updefault}{\color[rgb]{0,0,0}}}}}}
\put(3953,-2134){\makebox(0,0)[b]{\smash{{\SetFigFont{10}{12.0}{\rmdefault}{\mddefault}{\updefault}{\color[rgb]{0,0,0}}}}}}
\put(3550,-1721){\makebox(0,0)[lb]{\smash{{\SetFigFont{10}{12.0}{\rmdefault}{\mddefault}{\updefault}{\color[rgb]{0,0,0}}}}}}
\put(3550,-2555){\makebox(0,0)[lb]{\smash{{\SetFigFont{10}{12.0}{\rmdefault}{\mddefault}{\updefault}{\color[rgb]{0,0,0}}}}}}
\put(2993,-1524){\makebox(0,0)[b]{\smash{{\SetFigFont{10}{12.0}{\rmdefault}{\mddefault}{\updefault}{\color[rgb]{0,0,0}}}}}}
\put(3051,-1785){\makebox(0,0)[lb]{\smash{{\SetFigFont{10}{12.0}{\rmdefault}{\mddefault}{\updefault}{\color[rgb]{0,0,0}}}}}}
\put(2987,-2134){\makebox(0,0)[b]{\smash{{\SetFigFont{10}{12.0}{\rmdefault}{\mddefault}{\updefault}{\color[rgb]{0,0,0}}}}}}
\put(1565,-2042){\makebox(0,0)[b]{\smash{{\SetFigFont{10}{12.0}{\rmdefault}{\mddefault}{\updefault}{\color[rgb]{0,0,0}}}}}}
\put(1117,-2133){\makebox(0,0)[b]{\smash{{\SetFigFont{10}{12.0}{\rmdefault}{\mddefault}{\updefault}{\color[rgb]{0,0,0}}}}}}
\put(1156,-2555){\makebox(0,0)[lb]{\smash{{\SetFigFont{10}{12.0}{\rmdefault}{\mddefault}{\updefault}{\color[rgb]{0,0,0}}}}}}
\put(1116,-3063){\makebox(0,0)[b]{\smash{{\SetFigFont{10}{12.0}{\rmdefault}{\mddefault}{\updefault}{\color[rgb]{0,0,0}}}}}}
\end{picture} }

\begin{itemize}
\item
Initially  consists of .\vspace{1mm}
\item
First the paths  and 
and  are added to , by the transitions
 and  and , respectively, in .
These transitions are deleted from .\vspace{1mm}
\item
Next the paths  and  are added
to , by the transitions  and , respectively, in ,
which are deleted from .\vspace{1mm}
\item
Next the transitions  and  are added to ,
by the transitions  and , respectively, in ,
which are deleted from .\vspace{1mm}
\item
Finally the transition  is added to ,
by the transition  in ,
which is deleted from .
\end{itemize}
Since all transitions in  are applied in the construction of , they are all reachable. That is, at the end ,
and  coincides with .
\end{example}

\paragraph{Correctness proof}
The following two properties of , which follow immediately from its construction, give insight into the structure of .
In particular, Lemma \ref{lem:nfa1} implies that in , the outgoing transitions of a final state always carry the label ,
while each non-final state has exactly one outgoing transition with a label from .


\begin{lemma}
\label{lem:nfa-reachable}
For each state  in , there is a path  in , for some .
\end{lemma}

\begin{lemma}
\label{lem:nfa1}
For each state  in  there is exactly one path  in , for some , .
\end{lemma}

\begin{proof}
We prove both lemmas in one go, by induction on the construction of .

Initially they hold trivially, because then  only consists of the transition .

When a path  is added to , then also a transition 
is added, where  was already present in . Since by induction  for some ,
clearly Lemma \ref{lem:nfa-reachable} still holds in the extended nfa .
Moreover, since only (part of) the path  is added to , together with some
-transitions to the first state in this path, clearly Lemma \ref{lem:nfa1} still holds in the extended nfa .
\qed
\end{proof}

\noindent
The following lemma and proposition are corner stones in the correctness proof.

\begin{lemma}
\label{lem:strengthening}
If there are paths  and  in , then there is a run  of .
\end{lemma}

\begin{proof}
The lemma is proved by induction on the path . A well-founded partial ordering on paths in  is defined as follows.
Suppose that during the construction of , each -transition is at its creation provided with a sequence number, being one
higher than the previously created -transition (the first created -transition gets sequence number ).
Now  is defined to be smaller than  if:

\vspace*{-2mm}
\begin{itemize}
\item[(i)]
either  contains an -transition with a higher sequence number than any of the
-transitions in ;
\item[(ii)]
or  contains  as a strict subsequence.
\end{itemize}

\vspace*{-2mm}

\noindent
In the base case of the induction, the path  consists of zero transitions, so that  and .
By Lemma \ref{lem:nfa1}, the path  in  implies  and . And trivially there is a run
 of .

In the inductive case, the path  consists of one or more transitions.
We distinguish two cases, depending on whether the path  starts with an -transition.

\vspace{1mm}

\noindent
{\sc Case 1}: The path  is of the form , with .
By Lemma \ref{lem:nfa1},  is non-final, and  is of the form ,
with .
By (ii),  is smaller than .
Since there are paths  and  in ,
by induction, there is a run  of .
So there is a run  of .

\vspace{1mm}

\noindent
{\sc Case 2}: The path  is of the form .
Suppose the transition  was created in  due to a transition  in .
Then there must be paths  and  in .
By (i),  is smaller than , because
 has a higher sequence number than any of the -transitions in the path .
Since  and ,
by induction, there is a run  of .
The transition  in  gives rise to the move .
By (ii),  is smaller than .
Since  and ,
by induction, there is a run  of .
Concluding, there is a run  of .
\qed
\end{proof}

\begin{proposition}
\label{prop:nfa1}
There is a path  in  if, and only if,  is reachable in .
\end{proposition}

\begin{proof}
()
The transition  is in , and by assumption there is a path  in .
So by Lemma \ref{lem:strengthening}, there is a run  of , and so of .

\vspace{2mm}

\noindent
() By induction on the number of moves in a run  of .
In the base case, no transition is applied, so  and .
The transition  is in .

In the inductive case, suppose  is the last transition applied in the run .
Then  for some , and there is a run  of . Since this run
takes one move less than the one to , by induction there is a path  in .
This path splits into  in , where we choose  such that
 does not start with an -transition.
In view of the transition  in  and the path  in ,
during the construction of , a path  was created, together with
the transition . Concluding, there is a path ,
so , in .
\qed
\end{proof}

\begin{theorem}
\label{thm:nfa1}
The returned set  consists of the unreachable transitions in .
\end{theorem}

\begin{proof}
Suppose  in  is not in . Then during the construction of ,
 was used in the creation of a path , together with one or more transitions
. We choose one such . The construction requires that there is a path 
in . And by Lemma \ref{lem:nfa-reachable},  for some . So by Proposition \ref{prop:nfa1},
 is reachable in . In this configuration,  can be applied, to reach .
So  is reachable in .

Vice versa, suppose  is reachable in . Then a configuration  is reachable in , for some .
So by Proposition \ref{prop:nfa1} there is a path  in .
This path splits into  in , where we choose  such that
 does not start with an -transition.
In view of , a path  and a transition  were added to .
And as a result,  was deleted from .
\qed
\end{proof}

\paragraph{Complexity analysis}
Let  be the number of states and  the number of transitions in the pda .
In the analysis of the worst-case time complexity of the algorithm we assume that the number of elements popped from and
pushed onto the stack in one transition, as well as the size of the stack alphabet, are bounded by some constant.
Then the nfa  contains at most  states.

Building  takes at most : During a run of the {\em forward} procedure over the transitions in ,
at most  times (once for each transition in ), in step 2 backward scans over the -transitions are performed,
which take at most  (because there are at most  -transitions);
and there are at most  runs of the {\em forward} procedure over the transitions in  (because  contains at most  transitions).


\subsection{Detecting which transitions can lead to the final state}
\label{sec:backward}

If the transition  is not in , then by Proposition \ref{prop:nfa1} the language accepted by 
is empty. So then all transitions in  are reported as useless.

If the transition  is in , then
the set  of useless transitions in  is constructed by running the following {\em backward} procedure
over -transitions in  that are in a set ; at the start of such a run an -transition from  is
copied to the set , while on the other hand during the run -transitions from  may be added to .

Initially , as an overapproximation, contains all transitions in , 
and . We recall from step 2 of the {\em forward} procedure that the set  equals  if ,
or the states  for which there exists a path  in  if .
These sets have already been computed in (the last run of) the {\em forward} procedure.

\vspace{4mm}

\noindent
Procedure {\em backward}: While  do:
\begin{enumerate}
\item[1.~~]
Pick an  and add it to .\vspace{1mm}
\item[2.~~]
Find the path  in  (for some , ); since  denotes a final state, according to Lemma \ref{lem:nfa1}, exactly one such path exists.\vspace{1mm}
\item[3.~~]
For each transition  in  (for any , ) do:\vspace{1mm}
\item[3.1]
If , then stop this iteration step (i.e., return to step 3).\vspace{1mm}
\item[3.2]
If , then delete  from .\vspace{1mm}
\item[3.3]
If  (i.e., ), then add to  the -transitions that occur in any path
 in .
\end{enumerate}

\vspace{1mm}

\noindent
At the end, return the set  of useless transitions in . The transitions in  that stem from the original pda 
(i.e., not those from the preprocessing step in which  and  were added) are the useless transitions in , so can be culled without
changing the associated language.

The idea behind the {\em backward} procedure is that a transition  in  is added to  when
we are certain that, for some ,  and , there is a path
 in 
and a run  of .
Each transition  in  may in turn give rise to adding -transitions in  to ,
and removing transitions from . Namely, by Lemma \ref{lem:nfa1} there is exactly one path  in  (for some , ).
For each transition  (for any , ) in ,
all -transitions in any path  in 
can be added to . The reason is that there is a path  in ,
as well as a run  of : The transition  gives rise to the move ;
in view of the paths  and  in ,
by Lemma \ref{lem:strengthening} there is a run  of ; and
by assumption there is a run  of . So if there is a path  in ,
then  is useful: In view of the path  in , by Proposition \ref{prop:nfa1},
the configuration  is reachable in ; and we argued there is a run  of ,
which starts with an application of . Hence  can be removed from .
To try and avoid adding the same -transition to  more than once,
we only consider those paths  in  that do not start with an -transition.

The {\em backward} procedure is executed only a finite number of times, as it is performed at most once for each -transition in .
The set  returned at the end contains exactly the useless transitions in .
The corresponding theorem is presented at the end of this section, together with two propositions needed in its proof.

\begin{example}
\label{exa:nfa2}
We perform the {\em backward} procedure on the nfa  from Example \ref{exa:nfa1}.
\begin{itemize}
\item
Initially  and .\vspace{2mm}
\item
 is added to ; then  and .
Since , the transition  is deleted from ,
and the -transitions in paths  in  are added to :
 and .\vspace{2mm}
\item
 is added to ; then  and . Since , the transition  is deleted from .\vspace{2mm}
\item
 is added to ; then  and . Since , the transition  is deleted from .\vspace{2mm}
\item
 is added to ; then  and .
Since , the transition  is deleted from ,
and the -transitions in paths  in  are added to : .\vspace{2mm}
\item
 is added to ; then  and .
Since , the transition  is deleted from ,
and the -transitions in paths  in  are added to : .\vspace{2mm}
\item
 is added to ; then  and . Since , the transition  is deleted from .\vspace{2mm}
\item
 is added to ; then  and . Since , the transition  is deleted from .
\end{itemize}
At the end,  consists of . So this transition is useless in .
\end{example}

Example \ref{exa:nfa2} shows that in step 2 of the {\em backward} procedure,
the path  cannot be replaced by all paths  in .
Else  would be erroneously deleted from , in view of
the transition  in  and
the path  in .

\paragraph{Correctness proof}
The following proposition is needed to show that only useful transitions in  are deleted from  during runs
of the {\em backward} procedure.

\vspace*{-2mm}

\begin{proposition}
\label{prop:nfa2}
Let  be a transition in , and  a path in .
Then there is a path  in , for some , such that
there is a run  of .
\end{proposition}

\begin{proof}
By induction on the construction of .
In the base case, . So , ,  and .
So we can take .

In the inductive case, suppose that due to a transition  in , a path  in ,
and a transition  in ,
-transitions in paths  in  are added to .
We show that the proposition holds for any transition  in any path ;
say  with .
Since  and  are paths in , by Lemma \ref{lem:strengthening},
there is a run  of . And by ,  in .
The transition  was already present in  before the current extension of .
Since  is a path in , by induction there is a path  in , for some ,
such that there is a run  of .
Concluding, there is a path  in ,
and a run  of .
\qed
\end{proof}

The idea behind the next lemma is that if  is a reachable configuration of ,
and , then any run  of  contains a move
 in which a non-empty part  of  is removed from the stack.

\begin{lemma}
\label{lem:nfa3}
Let  be a path in , and
 a run of , with .
Then there is a path  in  for some  and , and
the path  in  splits into
 for some ,  and ,
where  does not start with an -transition, and
there is a path  in  for some ,  and ,
where  is a transition in 
that is applied to the configuration  in the run .
\end{lemma}

\begin{proof}
By induction on the number of moves in the run .
Note that  implies . Let  in 
be the first transition that is applied in the run . We distinguish two cases:

\vspace{1mm}

\noindent
{\sc Case 1}:  with , for some  and .
We take  and . The path  in  splits into
, where we choose  such that
 does not start with an -transition.
Since there is a path  in , ,
and  is in , by the {\em forward} procedure, there is a path  in .
So if we take  and , we are done.

\vspace{1mm}

\noindent
{\sc Case 2}:  for some . The path  in 
splits into , where we choose  such that
 does not start with an -transition.
Since  is in , by the {\em forward} procedure, there is a path  in .
So there is a path  in .
Moreover, the run  is of the form
.
By induction, there is a path  in ,
and the path  in  splits into 
with ,
where  does not start with an -transition, and
there is a path  in ,
where  is a transition in  that is applied to the configuration 
in the run .
\qed
\end{proof}

The following proposition is needed to show that all useful transitions in  are eventually deleted from .

\begin{proposition}
\label{prop:nfa3}
Let  be a path in ,
and  a run of . Then  is in  when the
{\em backward} procedure terminates.
\end{proposition}

\begin{proof}
We apply induction on the length of .
In the base case, . Since  is a transition in , and 
is the only transition in  with an occurrence of , by the {\em forward} procedure, the only possible outgoing -transition of
 in  is . Hence the path 
implies that , ,  and .
Since  is in , initially .

In the inductive case, . Since there is a path  in 
and a run  of , by Lemma \ref{lem:nfa3},
there is a path  in  for some  and , and
the path  in  splits into 
for some ,  and ,
where  does not start with an -transition, and
there is a path  in  for some ,  and ,
where  is a transition in 
that is applied to the configuration  in the run .
Since  is shorter than , by induction,  is eventually in .
During the iteration of the {\em backward} procedure in which  is added to ,
in view of the transition  in  and the paths  and
 in , the -transitions in
the latter path are added to . So in particular,  is added to .
\qed
\end{proof}

\begin{theorem}
\label{thm:nfa2}
The returned set  consists of the useless transitions in .
\end{theorem}

\begin{proof}
Suppose the transition  in  is not in . Since  is in ,
by the {\em forward} procedure, there is a path  in . Since , while running
the {\em backward} procedure, for some , the transition  was found to be in ,
and a path  was found to be in .
In view of the transition  in  and the path  in ,
by Proposition \ref{prop:nfa2}, there is a path  in , for some ,
such that there is a run  of .
In view of the path  in ,
by Proposition \ref{prop:nfa1},  is reachable in .
By applying  to this configuration,  is reached. Since moreover
there is a run  of ,  is useful in .

Vice versa, suppose  is useful in . Then there is a run
 of .
In view of the run ,
by Proposition \ref{prop:nfa1}, there is a path  in ,
where we choose  such that  does not start with an -transition.
Since  is in , by the {\em forward} procedure, there is a path  in .
In view of the run , by Proposition \ref{prop:nfa3},  is eventually in .
In view of the paths  and  in , during the iteration
of the {\em backward} procedure in which  is added to ,  is deleted from .
\qed
\end{proof}

\paragraph{Complexity analysis}
Computing  takes at most :
For each of the at most  -transitions in , and for at most 
transitions in , in step 3.3 a forward scan is performed over the -transitions in , which takes at most .

\section{Implementation}

We made a prototype implementation of the algorithm, using a test suite of more than twenty pda's.
The largest pda, with 295 transitions, was obtained from the grammar of the programming language C.
This resulted in an nfa with 339 states and 1030 transitions, of which 695 -transitions,
and took 11 seconds on a 2GHz processor.

Achieving this performance required two optimizations, both limiting
the influence of -transitions.
The first concerns determining the set of states leading to  in
step 2 of the {\em forward} procedure.
This set is constructed by following paths backwards from ,
which may lead through webs of -transitions, causing a considerable slow-down.
These -transitions were created in step 5 of the {\em forward} procedure.
The optimization consists of computing for each state , in step 5,
the set  of states that can reach  through -transitions only.
If more -transitions are added in step 5 during a next iteration,  is updated.
The second optimization concerns memoization of -transitions as they are encountered
on paths to  in step 3.3 of the {\em backward} procedure.


No further optimizations were applied. In fact, all sets were implemented as arrays;
choosing more advanced data structures would certainly improve the efficiency.











\bibliographystyle{plain}

\begin{thebibliography}{1}

\bibitem{LR.Bertsch.1999}
Eberhard Bertsch and Mark-Jan Nederhof.
\newblock Regular closure of deterministic languages.
\newblock {\em SIAM Journal on Computing}, 29(1):81--102, 1999.

\bibitem{BouajjaniEM97}
Ahmed Bouajjani, Javier Esparza, and Oded Maler.
\newblock Reachability analysis of pushdown automata: Application to
  model-checking.
\newblock In {\em Proc. CONCUR'97}, {\em LNCS} 1243, pp.\ 135--150. Springer, 1997.

\bibitem{FinkelWW97}
Alain Finkel, Bernard Willems, and Pierre Wolper.
\newblock A direct symbolic approach to model checking pushdown systems.
\newblock In {\em Proc.\ INFINITY'97}, {\em ENTCS} 9, pp.\ 27--37.
  Elsevier, 1997.

\bibitem{GoldstinePW82econ}
Jonathan Goldstine, John~K. Price, and Detlef Wotschke.
\newblock A pushdown automaton or a context-free grammar - which is more
  economical?
\newblock {\em Theoretical Computer Science}, 18:33--40, 1982.

\bibitem{Griffin06}
Christopher Griffin.
\newblock A note on deciding controllability in pushdown systems.
\newblock {\em IEEE Transactions on Automatic Control}, 51(2):334--337, 2006.

\bibitem{KutribMalcher12}
Martin Kutrib and Andreas Malcher.
\newblock Reversible pushdown automata.
\newblock {\em Journal of Computer and System Sciences}, 78(6):1814--1827, 2012.

\bibitem{Linz11}
Peter Linz.
\newblock {\em An Introduction to Formal Languages and Automata}.
\newblock Jones \& Bartlett, 2011.

\bibitem{Substring.Nederhof.1996}
Mark-Jan Nederhof and Eberhard Bertsch.
\newblock Linear-time suffix parsing for deterministic languages.
\newblock {\em Journal of the ACM}, 43(3):524--554, 1996.

\bibitem{Walukiewicz96}
Igor Walukiewicz.
\newblock Pushdown processes: Games and model checking.
\newblock In {\em Proc. CAV'96}, {\em LNCS} 1102, pp.\ 62--74. Springer, 1996.

\end{thebibliography}



\end{document}
