\label{sec:implem}

We have implemented DNStamp by extending and adapting the EphPub command-line tool \cite{Castelluccia2011}. The prototype takes as input the file to be timestamped, the duration the timestamp should stay valid and an option indicating whether to request or verify the timestamp. Our prototype works with IPv4 addresses only and supports a timestamp duration up to 1 day. The case of timestamps
different than 1 day is discussed in Section~\ref{sec:moredays}.

\subsection{Selecting domain names}
\label{sec:domainselection}

In our experiments we require that each timestamp generates  resolution requests. We need to generate  random domain names with a reference TTL equal to 1 day. We do so by executing the domain name selection process described below.

First, we generate a list of random IP addresses and try a reverse resolution for each of them. The reverse resolution fails for an important number of IPv4 addresses. This is normal, since not all of the possible IPv4 addresses have an assigned domain name. In addition, we require that (i) the returned domain name is valid, i.e. that it does not generate an \domain{NXDOMAIN} error \cite{Mockapetris1987} when queried and that (ii) an authoritative server exists for the returned domain. Our measurements over 100 timestamping processes shows that on average  of the randomly generated IPv4 can be successfully reversed to a valid domain name.

Then, we retrieve the reference TTL for each domain name and verify whether the reference TTL is strictly equal to 1 day. We drop a domain name if it does not satisfy this constraint. We continue to search for domains that satisfy this constraint until we reach the  expected domain names.
To retrieve the reference TTL for a given domain, we emit an \domain{SOA} query \cite{Mockapetris1987}. This returns the authoritative server of the domain name and related information such as the \domain{minimum TTL}. According to the standard, this TTL should be used as the reference TTL. However, we observed that this was not always the case. Therefore, we decided to use a more accurate source of information for the reference TTL consisting in sending a DNS resolution (\domain{A} query) to the authoritative server itself. 

Figure~\ref{fig:distribTTL} shows the distribution of reference TTL we observe during our experiment. This distribution is consistent with the results presented by \cite{Sit2002}. According to this distribution, if the targeted reference TTL is equal to 1 day, we must drop on average  of the returned domain names. 


Figure~\ref{fig:domainGen} shows that on average our prototype tests  random IP addresses to generate  domain names with a TTL equal to 1 day. Approx.\  IP addresses are successfully reversed to a domain name that is valid and has an authoritative server.  These numbers stay constant over time.


\begin{figure*}
\begin{center}
\subfigure[]{
\includegraphics[width=\columnwidth]{figures/domainGeneration.pdf}
\label{fig:domainGen}
}
\subfigure[]{
\includegraphics[width=\columnwidth]{figures/TSRead.pdf}
\label{fig:ReadTS}
}
\caption{\subref{fig:domainGen}  IPv4 addresses and valid domain names retrieved per timestamp verification to generate 100 valid domain names with 1 day TTL. \subref{fig:ReadTS}  Cache entries read.  Numbers are averages from five 1 day timestamps  continuously verified during 40 hours.}
\label{fig:TSgeneration}
\end{center}
\end{figure*}

\begin{figure}
\begin{center}
\includegraphics[width=\columnwidth]{figures/TTLCDF.pdf}
\caption{CDF of reference TTL of domains observed during timestamp validations, representing a total of {\raise.17ex\hbox{}}757K domain names.}
\label{fig:distribTTL}
\end{center}
\end{figure}






\subsection{Load and delay considerations}
The important number of IP addresses and domain names that needs to be generated and tested for one timestamp could seem prohibitive. Our implementation requires about  minutes to iterate through and test all IP addresses and domain names and to eventually set the timestamp. We believe that this delay comes from our unoptimized implementation and that we can reduce the timestamping delay: 
(i) by reducing the number of cache entries required for one timestamp, and (ii) by dispatching and multi-threading batches of IP reverse-resolutions and search of authority requests among a larger set of DNS server.
This delay generates a timestamp that is shifted compared to . The requester may estimate and add this delay to  before selecting the domain names.
Otherwise, verifiers will observe the shift and would have to tolerate a delay representative of a requesting time.

The verification process does not need to reiterate through all IP addresses and domains, thus generating less load than the timestamp request. The verifier may directly use the domain names and resolvers described in the structure .
Once the timestamping time has been retrieved, the verifier may test whether the domains and resolvers are actually bound to the timestamped data  by executing the domain selection algorithm.

\begin{table}
\footnotesize{
\begin{center}
\begin{tabular}{|l|c|c|c|}
\hline
Experiment & Duration & Requests & Approx. Load \\
	& (sec.) & & (requests/sec.) \\ \hline
1 hour \DNStamp & 94 & 1954 & 20.8 \\ \hline
2 hours \DNStamp & 154 & 2214 & 14.4 \\ \hline
1 day \DNStamp & 189 & 3564 & 18.8 \\ \hline
2 days \DNStamp & 1836 & 25911 & 14.1 \\ \hline
Start surf & 12 & 94 & 7.8 \\ \hline
Regular surf & 600 & 2543 & 4.1 \\ \hline
Lab activity & 9 hours & 207377 & 6.4 \\ \hline
\end{tabular}
\end{center}
\caption{Comparison of DNS loads}
\label{tab:load}
}
\end{table}

We also evaluate the additional load of \DNStamp\ on the Domain Name System compared to other DNS consuming activities. Table~\ref{tab:load} summarizes the experimental results. With the current implementation of \DNStamp\ a request generates a peak load of 15 to 20 DNS requests per second. 
The experiment called start surf consists in starting the Internet browser Google Chrome and searching for the term "timestamping". The experiment called regular surf consists in consulting a new webpage per minute during 10 minutes.
The experiment called lab activity consists in recording the DNS activity of a 65 hosts lab during 9 working hours. \DNStamp\ generates a higher DNS peak activity than regular surf, yet it remains in the same order of magnitude.


\subsection{Selecting resolvers}
\label{sec:selectDNSres}
The selection of resolvers relies on a list of publicly accessible resolvers.
For our experiments, we use a list of {\raise.17ex\hbox{}}11k resolvers.
We obtained this list by keeping the resolvers that were still online at the time of the experiment from the list of {\raise.17ex\hbox{}}22k resolvers provided by EphPub \cite{Castelluccia2011}.
The latter list dates back to November 2009.
This highlights that the list of resolvers needs to be maintained and continuously updated: old resolvers may go offline and new ones may be added.
The methods for constructing the resolver list may rely on the methods proposed by \cite{Dagon2008} and \cite{Castelluccia2011}. Castelluccia \etal \cite{Castelluccia2011} estimated the number of resolvers that perform caching properly to 1.7 million.
As discussed in Section~\ref{sec:softwarattack}, the list of resolvers has to be considered as an intrinsic part the timestamping software. Updating this list is equivalent to changing the domain selection process. If the list changes, the resolvers selected for a particular timestamp changes, thus making existing timestamps invalid.

\subsection{Retrieving a timestamp}
Figure~\ref{fig:ReadTS} shows the number of cache entries our prototype could successfully read while continuously retrieving a timestamp over 40 hours. The timestamp completely vanishes after 24 hours. This behavior is normal since all remaining TTL reach 0 after 24 hours.

We observe that about  of the cache entries can be read right after the timestamp has been set. The missing  correspond to the cache entries that could not be successfully set by the requester. Indeed, we noted that some resolvers either refuse a connection (returning a DNS \domain{REFUSED} message) or simply time out. This ratio slightly decreases, and after  hours and until the expiration of the timestamp, about  of the cache entries are successfully retrieved. The decrease of valid cache entries typically comes from cache leaks \cite{Kumar1993}. Still, the remaining fraction of valid cache entries is sufficient to validate the timestamping time.



\myparagraph{Homogeneity}
We now verify for a given timestamp , if the computed time  is the same for all cache entries. Figure~\ref{fig:timestamp1day} shows an example distribution of computed timestamping times for each individual cache entry with a timestamp composed of 100 cache entries. We notice that the large majority of times hold in the interval between 17h11min41sec and 17h11min50sec. Thus the accuracy of the timestamp is around  seconds. 

While Figure~\ref{fig:timestamp1day} depicts a typical distribution, outliers in the distribution of computed  timestamping times of individual cache entries can occur. 
In a few cases,  of cache entries, \DNStamp\ computed timestamping times far before or after the actual timestamping time. 
We identified two causes for these outliers: 
(i) the domain name was already cached by the resolver when requesting the timestamp and its TTL is not reset by the timestamp request; the verification process computes a timestamping time prior to the actual timestamping time for this particular cache entry.
(ii) the authoritative server returns a wrong reference TTL during the verification process. 
We observed that a small number of authoritative servers do not implement DNS caching correctly. An authoritative server should always return the reference TTL when receiving a domain resolution request (\domain{A} query) for the domain it is authoritative for. Yet, we noted that a small fraction of authoritative servers does not conform to this behavior and instead decrements the returned TTL as a normal resolver. Thus, the verification process computes a timestamping time prior to the actual timestamping time for this particular domain name.

\begin{figure}
\begin{center}
\includegraphics[width=\columnwidth]{figures/timstamp_1day.pdf}
\caption{Example distribution of computed timestamping times for each cached domain name with a 1 day timestamp.}
\label{fig:timestamp1day}
\end{center}
\end{figure}

\myparagraph{Measurements from several locations}
We performed additional experiments to verify whether timestamps can be requested and verified from different locations in the Internet. The different locations that we use for our tests are: (i) Amazon cloud, by running the requesters and verifiers on EC2 instances (ii) broadband Internet access, by testing two standard ADSL lines provided by two different ISP providers.
The tests consist in requesting a different timestamp on each of these locations and verify the generated timestamps from all other locations. We observe no significant differences between the different locations. All locations can verify the timestamps requested on the other locations. The number of retrieved cache entries is similar from one location to another and in line with the observations of Figure~\ref{fig:timestamp1day}.
We observe that the calculated timestamping time for a specific cache entry has at most 1 second difference from one location to another. This difference may be explained by the network delays for some DNS replies or responses. 




\subsection{Overwriting an existing timestamp}
\label{sec:overwrite}

\begin{figure*}
\begin{center}
\subfigure[]{
\includegraphics[width=\columnwidth]{figures/overwriteTS.pdf}
\label{fig:overwriteTS}
}
\subfigure[]{
\includegraphics[width=\columnwidth]{figures/remainingTTL.pdf}
\label{fig:overwriteTTL}
}
\caption{\subref{fig:domainGen2day}  Cache entries successfully set to the reference TTL during 70 hours continuous overwriting of an existing timestamp. \subref{fig:overwriteTTL} Remaining TTL representing  of measured cache entries during the validity period (24 hours) of the initial timestamp. }
\end{center}
\end{figure*}

In this subsection, we experimentally demonstrate that a weak adversary is not capable of overwriting the remaining TTL of an existing timestamp.
We conduct the following experiment to demonstrate this property. We first generate a valid 1 day timestamp consisting of 100 cache entries. Then, we continuously try to overwrite the existing timestamp during 70 hours. We verify whether we can set cache entries to the reference TTL and monitor the remaining TTL each time we try to overwrite a cache entry.

Figure~\ref{fig:overwriteTS} shows the number of cache entries that could be successfully set to the reference TTL. At  and at multiples of 24 hours between 80 and 90 cache entries could be updated. This is normal, since this corresponds to the initial timestamp request (at ) and the subsequent expirations of the timestamp every 24 hours. At these times, the DNS caches do not cache the requested domain names or has just reached a remaining TTL of 0.

Between these peaks and from time to time, the weak adversary is able to set one or two cache entries. This is possible because the domain name selection process encountered some error such as a timeout while reversing an IP address or while requesting the reference TTL. The adversary will thus select a list of domain names different from the list selected by the requester.
This has only a limited impact on the verification. A verifier relies on the domain names and resolver included in  and will not use the cache entries set by the weak adversary. Even if the verifier regenerates the domain names, it is very unlikely that she will select a majority of cache entries the weak adversary has set.
We verified the latter assertion by continuously executing the domain name generation process during the validity period (24 hours) of the initial timestamp and checking the remaining TTL.
Figure~\ref{fig:overwriteTTL} shows that  of the remaining TTL decrement exactly as expected. Thus, the attacker cannot modify these cache entries.






\subsection{Timestamps having duration different than one day}
\label{sec:moredays}





We describe experimental results using timestamp durations of 2 days and one hour.
The domain selection process works exactly as described in Section~\ref{sec:domainselection}. The only difference concerns the reference TTL. We accept reference TTL with a value greater or equal to the desired duration.

Our experiments show that it is possible to use \DNStamp\ for timestamps having duration different than one day. Yet, the verifier needs to eliminate falsely computed timestamping times due to DNS cache resolver that set a wrong reference TTL.


\myparagraph{Requesting and verifying timestamps}
We generate 5 two day timestamps which we continuously verify during 72 hours. Similarly, we generate 5 one hour timestamps which we continuously verify during 26 hours.
Figure~\ref{fig:domainGen2day} and  Figure~\ref{fig:domainGen1hour} shows the number of IP addresses and the number of domain names that the domain name generation process has to generate for each verification. 
With a two day timestamp (Figure~\ref{fig:domainGen2day}), about 6500 random IP addresses and 1200 domain names have to be generated, in order to end up with 100 domain names with a reference TTL greater or equal to two days. This confirms the distribution of reference TTL shown in Figure~\ref{fig:distribTTL} where approx.\  of domains have a reference TTL smaller than two days. With a one hour timestamp (Figure~\ref{fig:domainGen1hour}), about 525 random IP addresses and 110 domain names have to be generated.

The high number of IP addresses and domain names required for a two day timestamps can seem prohibitive. Indeed, we require about 30 minutes to request a two day timestamp. In contrast, we only require about 1 minute 30 seconds for the one hour timestamp. We believe that this delay mainly comes from our unoptimized implementation. The optimizations already discussed for one day timestamps should considerably decrease the timestamping delay.

Verifying a timestamp does not necessarily require to reiterate through all IP addresses and domain names.
Instead of executing the whole domain name generation process, as in our experimentation, the verifier may directly use the domain names and DNS cache resolvers indices of . Thus, verification is almost immediate.

\myparagraph{Persistence of timestamps}
Figure~\ref{fig:ReadTS2day} and Figure~\ref{fig:ReadTS1hour} show that the timestamp continues to exist even when the desired reference TTL has exceeded. In the case of two day timestamps, approx.\ one third of the DNS cache entries continue to exist after 48 hours. This portion corresponds to the domain names with a reference TTL strictly greater than two days. Similarly, with one hour timestamp we observe that approx.\  of the DNS cache entries last for one day, reflecting the reference TTL distribution of Figure~\ref{fig:distribTTL}.

To avoid this behavior, it is possible to force reference TTL to exactly the desired duration, resulting in all DNS cache entries to disappear after this duration.


\begin{figure*}
\begin{center}
\subfigure[]{
\includegraphics[width=\columnwidth]{figures/domainGeneration_2days.pdf}
\label{fig:domainGen2day}
}
\subfigure[]{
\includegraphics[width=\columnwidth]{figures/domainGeneration_1hour.pdf}
\label{fig:domainGen1hour}
}
\subfigure[]{
\includegraphics[width=\columnwidth]{figures/TSRead_2days.pdf}
\label{fig:ReadTS2day}
}
\subfigure[]{
\includegraphics[width=\columnwidth]{figures/TSRead_1hour.pdf}
\label{fig:ReadTS1hour}
}
\caption{\subref{fig:domainGen2day}, \subref{fig:domainGen1hour}:  Number of IPv4 addresses and valid domain names retrieved per verification in order to generate 100 valid domain names with a reference TTL greater or equal to \subref{fig:domainGen2day} two days and \subref{fig:domainGen1hour} one hour. \subref{fig:ReadTS2day}, \subref{fig:ReadTS1hour}:  Number of DNS cache entries read with a timestamp of \subref{fig:ReadTS2day} two days and \subref{fig:ReadTS1hour} one hour. Numbers are averages from 5 two day timestamps  continuously verified during 72 hours, and from 5 one hour timestamps  continuously verified during 26 hours.}
\end{center}
\end{figure*}

\myparagraph{Errors due to DNS cache resolvers ignoring reference TTL greater than one day}
We observed that some DNS cache resolvers set the remaining TTL to maximum one day, even if the reference TTL is greater than one day. Our experiments show that this represents about one third of the DNS cache resolvers. For instance, Figure~\ref{fig:ReadTS2day} shows that one third of the returned timestamps disappear after 24h, even if the desired reference TTL is greater or equal to two days. This indicates that the remaining TTL for these DNS cache resolvers was set to one day only, and that the remaining TTL reaches  after 24 hours.

\myparagraph{Homogeneity of the computed timestamping times for the different DNS cache entries}
Figure~\ref{fig:timestamp2days} shows the distribution of computed timestamping times for the different DNS cache entries with one example of 2 day timestamp. We notice that the majority of the computed times tends towards the original timestamping time which is Feb 4th at 14h33. Some peaks appear before that date, e.g. at one day and at 6 days prior to the actual timestamping date. Similar behavior holds for one hour timestamp as depicted in Figure~\ref{fig:timestamp1hour}. The outliers represent about one third of the DNS cache entries in the two day timestamp, and only a very small fraction for one hour timestamp.

The outliers can be explained by the fact that some DNS cache resolvers set a false remaining TTL when a new DNS cache entry is added. If the reference TTL of the cached domain name is 7 days and the DNS cache resolver only sets a remaining TTL of one day, the computed time is 6 days prior to the actual timestamping time.
In order to make the timestamping scheme robust to these errors, a verifier must eliminate these outliers by selecting only the most frequent timestamping times.


\begin{figure*}
\begin{center}
\subfigure[]{
\includegraphics[width=\columnwidth]{figures/timstamp_2days.pdf}
\label{fig:timestamp2days}
}
\subfigure[]{
\includegraphics[width=\columnwidth]{figures/timstamp_1hour.pdf}
\label{fig:timestamp1hour}
}
\caption{Example distribution of computed timestamping times for each cached domain name with \subref{fig:timestamp2days} a 2 day timestamp and \subref{fig:timestamp1hour} a one hour timestamp.}
\end{center}
\end{figure*}