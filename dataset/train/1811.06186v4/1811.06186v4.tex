\ifx\allinone\undefined
\input{../preamble}
\begin{document}
\fi

\section{Conclusion}
\label{sec:con}
In this paper, we presented a novel perspective that regards gait as a set and thus proposed a GaitSet approach. The GaitSet can extract both spatial and temporal information more effectively and efficiently than those existing methods regarding gait as a template or sequence.
It also provide a novel way to aggregate valuable information from different sequences to enhance the recognition accuracy.
Experiments on two benchmark gait datasets has indicated that compared with other state-of-the-art algorithms, GaitSet achieves the highest recognition accuracy, and reveals a wide range of flexibility on various complex environments, showing a great potential in practical applications. In the future, we will investigate a more effective instantiation for Set Pooling~(SP) and further improve the performance in complex scenarios.

\ifx\allinone\undefined
\end{document}
