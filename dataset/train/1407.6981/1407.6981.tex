\documentclass{sig-alternate-2013}
\usepackage{epsfig,endnotes}
\usepackage{ifpdf}
\usepackage{multirow}
\usepackage{color}
\usepackage{etoolbox}
\newcommand{\subparagraph}{}
\usepackage{titlesec}

\usepackage{amsmath, hyperref}
\usepackage[numbers]{natbib}


\hyphenation{squar-es}

\newcommand\TODO[1]{\textcolor{red}{\emph{#1}}}
\newcommand\TODOMARGIN[1]{\marginpar{\textcolor{red}{\emph{#1}}}}

\newtheorem{theorem}{Theorem}
\newtheorem{lemma}{Lemma}
\newtheorem{obs}{Observation}
\newcommand\RAPPOR{{RAPPOR}}
\newcommand\RAPPORSPELLEDOUT{{Randomized Aggregatable Privacy-Preserving Ordinal Response}}

\newcommand\USC{{University of Southern California}}



\title{\RAPPOR{}: \RAPPORSPELLEDOUT{}}
\author{{\'{U}}lfar Erlingsson, Aleksandra Korolova and Vasyl Pihur}

\numberofauthors{3} \author{
\alignauthor
{\'{U}l}far Erlingsson \\
       \affaddr{Google, Inc.} \\
       \email{ulfar@google.com}
\alignauthor
Vasyl Pihur \\
       \affaddr{Google, Inc.}\\
       \email{vpihur@google.com}
\alignauthor Aleksandra Korolova \0.2ex]
       \email{korolova@usc.edu}
}


\newfont{\mycrnotice}{ptmr8t at 7pt}
\newfont{\myconfname}{ptmri8t at 7pt}
\let\crnotice\mycrnotice \let\confname\myconfname 

\permission{Permission to make digital or hard copies of part or all of this work for personal or classroom use is granted without fee provided that copies are not made or distributed for profit or commercial advantage, and that copies bear this notice and the full citation on the first page. Copyrights for third-party components of this work must be honored. For all other uses, contact the owner/authors. Copyright is held by the authors.}
\conferenceinfo{CCS'14,}{November 3--7, 2014, Scottsdale, Arizona, USA.} 
\copyrightetc{ACM 978-1-4503-2957-6/14/11, http://dx.doi.org/10.1145/2660267.2660348}


\clubpenalty=10000 
\widowpenalty = 10000


\begin{document}
\maketitle


\begin{abstract}
\RAPPORSPELLEDOUT{},
or \RAPPOR{}, is a technology for crowdsourcing statistics from end-user client software, 
anonymously, with strong privacy guarantees.  
In short, \RAPPOR{}s allow the forest of client data to be studied, 
without permitting the possibility of looking at individual trees.
By applying randomized response in a novel manner,
\RAPPOR{} provides the mechanisms for such collection as well as for efficient, 
high-utility analysis of the collected data.
In particular, \RAPPOR{} permits statistics to be collected
on the population of client-side strings
with strong privacy guarantees for each client,
and without linkability of their reports.

This paper
describes and motivates \RAPPOR{},
details its differential-privacy and utility guarantees,
discusses its practical deployment and properties in the face of different attack models,
and, finally, gives results of its application
to both synthetic and real-world data.
\end{abstract}



\section{Introduction}
Crowdsourcing data to make better, more informed decisions is becoming increasingly commonplace. 
For any such crowdsourcing,
privacy-preservation mechanisms should be applied 
to reduce and control the privacy risks introduced by the data collection process,
and balance that risk against the beneficial utility of the collected data.
For this purpose we introduce
\RAPPORSPELLEDOUT{},
or \RAPPOR{},
a widely-applicable, practical new mechanism that provides strong privacy guarantees 
combined with high utility,
yet is not founded on the use of trusted third parties.


\RAPPOR{} builds on the ideas of \textit{randomized response}, a surveying technique developed in the 1960s for collecting statistics on sensitive topics where survey respondents wish to retain confidentiality~\citep{warner}.
An example commonly used to describe this technique involves a question on a sensitive topic, such as ``Are you a member of the Communist party?''~\cite{WikipediaRR}. For this question, the survey respondent is asked to flip a fair coin, in secret, and answer ``Yes'' if it comes up heads, but tell the truth otherwise (if the coin comes up tails).
Using this procedure, each respondent retains very strong deniability for any ``Yes'' answers, since such answers are most likely attributable to the coin coming up heads; as a refinement, respondents can also choose the untruthful answer by flipping another coin in secret, and get strong deniability for both ``Yes'' and ``No'' answers.

Surveys relying on randomized response enable easy computations of accurate population statistics while preserving the privacy of the individuals. Assuming absolute compliance with the randomization protocol (an assumption that may not hold for human subjects, and can even be non-trivial for algorithmic implementations~\cite{mironov-CCS12}), it is easy to see that in a case where both ``Yes'' and ``No'' answers can be denied (flipping two fair coins), the true number of ``Yes'' answers can be accurately estimated by , where  is the proportion of ``Yes'' responses. 
In expectation,
respondents will provide the true answer 75\% of the time,
as is easy to see by
a case analysis of the two fair coin flips.

Importantly,
for one-time collection,
the above randomized survey mechanism
will protect the privacy
of any specific respondent,
irrespective of any attacker's prior knowledge, 
as assessed via the -differential privacy guarantee~\cite{dwork06}.
Specifically,
the respondents will have
differential privacy at the level .
This said,
this privacy guarantee degrades
if the survey is repeated---e.g.,
to get fresh, daily statistics---and
data is collected multiple times from the same respondent.
In this case,
to maintain both differential privacy and utility,
better mechanisms are needed,
like those we present in this paper.


Privacy-Preserving Aggregatable Randomized Response, 
or \RAPPOR{}s, is a new mechanism for collecting statistics from end-user, client-side software, 
in a manner that provides strong privacy protection
using randomized response techniques.  
\RAPPOR{} is designed to permit collecting, over large numbers of clients, 
statistics on client-side values and strings, 
such as their categories, frequencies, histograms, 
and other set statistics.
For any given value reported,
\RAPPOR{}
gives 
a strong deniability guarantee for the reporting client,
which strictly limits private information disclosed,
as measured by an -differential privacy bound,
and holds even for 
a single client that reports often on the same value.

A distinct contribution is \RAPPOR{}'s ability 
to collect statistics about an arbitrary set of strings 
by applying randomized response to Bloom filters~\cite{bloom} 
with strong -differential privacy guarantees. 
Another contribution is the elegant manner
in which \RAPPOR{}
protects the privacy of clients 
from whom data is collected repeatedly (or even infinitely often), and
how \RAPPOR{}
avoids addition of privacy externalities,
such as those that might be created by
maintaining a database of contributing respondents (which might be breached),
or repeating a single, memoized response (which would be linkable, and might be tracked).
In comparison,
traditional randomized response does not provide any longitudinal privacy 
in the case when multiple responses are collected from the same participant. 
Yet another contribution is that the \RAPPOR{}  mechanism is performed locally on the client, and does not require a trusted third party.


Finally, \RAPPOR{}
provides a novel, high-utility 
decoding framework for learning statistics
based on a sophisticated combination of
hypotheses testing,
least-squares solving, and LASSO regression~\cite{lasso}.


\subsection{The Motivating Application Domain}\label{sec:motivation}
\RAPPOR{} is
a general technology for privacy-preserving data collection and crowdsourcing of statistics,
which could be applied in a broad range of contexts. 


In this paper, however, we focus on the specific application domain
that motivated the development of \RAPPOR{}:
the need for Cloud service operators to collect up-to-date statistics 
about the activity of their users and their client-side software.
In this domain,
\RAPPOR{} has already seen limited deployment in Google's Chrome Web browser,
where it has been used
to improve the data sent by users that have opted-in to reporting statistics~\cite{ChromeRAPPORpage}.
Section~\ref{sec:chromehome} briefly describes this real-world application,
and the benefits \RAPPOR{} has provided
by shining a light on the unwanted or malicious hijacking of user settings.


For a variety of reasons,
understanding population statistics is a key part  
of an effective, reliable operation of online services
by Cloud service and software platform operators.
These reasons are often as simple as 
observing how frequently certain software features are used,
and measuring their performance and failure characteristics.
Another, important set of reasons
involve
providing better security and abuse protection to the users, their clients, and the service itself.
For example, to assess the prevalence of botnets or hijacked clients,
an operator may wish to monitor how many clients
have---in the last 24 hours---had critical preferences overridden, 
e.g., to redirect
the users' Web searches to the URL of a known-to-be-malicious search provider.



The collection of up-to-date crowdsourced statistics
raises a dilemma for service operators.
On one hand,
it will likely be detrimental to the end-users' privacy
to directly collect their information.
(Note that even the search-provider preferences of a user
may be uniquely identifying, incriminating, 
or otherwise compromising for that user.)
On the other hand,
not collecting any such information 
will also be to the users' detriment:
if operators cannot
gather the right statistics,
they cannot make
many software and service improvements that benefit users
(e.g., by detecting or preventing malicious client-side activity).
Typically, operators resolve this dilemma
by using techniques
that derive only the necessary high-order statistics,
using mechanisms
that limit the users' privacy risks---for example,
by collecting only coarse-granularity data, 
and by eliding data that is not shared by a certain number of users.


Unfortunately,
even for careful operators, 
willing to utilize state-of-the-art techniques,
there are few existing, practical mechanisms 
that offer both privacy and utility,
and even fewer that provide clear privacy-protection guarantees.
Therefore, 
to reduce privacy risks,
operators rely to a great extent on pragmatic means and processes,
that, for example, avoid the collection of data, 
remove unique identifiers, or otherwise systematically scrub data, 
perform mandatory deletion of data after a certain time period, 
and, in general, enforce access-control and auditing policies on data use.
However, these approaches are limited in their ability to provide provably-strong privacy guarantees. 
In addition, privacy externalities from individual data collections, 
such as timestamps or linkable identifiers, may arise;
the privacy impact of those externalities may
be even greater than that of the data collected.


\RAPPOR{} can help operators
handle the significant challenges, and potential privacy pitfalls,
raised by this dilemma.


\subsection{Crowdsourcing Statistics with \RAPPOR{}}
Service operators may apply \RAPPOR{}
to crowdsource statistics
in a manner that protects their users' privacy,
and thus address the challenges described above.


As a simplification,
\RAPPOR{} responses can be assumed to be \emph{bit strings},
where each bit corresponds to a randomized response 
for some logical predicate on the reporting client's properties, such as its values, context, or history.
(Without loss of generality,
this assumption
is used for the remainder of this paper.)
For example, one bit in a \RAPPOR{} response may correspond to a predicate 
that indicates the stated gender, male or female, of the client user,
or---just as well---their membership in the Communist party.


The structure of a \RAPPOR{} response need not be otherwise constrained;
in particular, 
(i) the response bits may be sequential, or unordered,
(ii) the response predicates may be independent, disjoint, or correlated,
and (iii) the client's properties may be immutable, or changing over time.
However,
those details (e.g., any correlation of the response bits)
must be correctly accounted for,
as they impact both the utilization and privacy guarantees of \RAPPOR{}---as
outlined in the next section, and detailed in later sections.



In particular,
\RAPPOR{} can be used to
collect 
statistics on 
categorical client properties,
by having each bit in a client's response
represent whether, or not, that client belongs to a category.
For example, those categorical predicates 
might represent 
whether, or not, the client is utilizing a software feature.
In this case,
if each client can use only one of three disjoint features, , and , 
the collection of a three-bit \RAPPOR{} response from clients
will allow measuring
the relative frequency 
by which the features are used by clients.
As regards to privacy,
each client will be protected
by the manner in which 
the three bits are derived from a single (at most) true predicate;
as regards to utility,
it will suffice to count how many responses had the bit set, for each distinct response bit,
to get a good statistical estimate of the empirical distribution of the features' use.


\RAPPOR{}
can also be used to 
gather
population statistics on 
numerical and ordinal values,
e.g., 
by associating response bits
with predicates for different ranges of numerical values,
or by reporting on
disjoint categories for different logarithmic magnitudes of the values.
For such numerical \RAPPOR{} statistics,
the estimate may be improved
by collecting and utilizing
relevant information about the priors and shape of
the empirical distribution, such as its smoothness.


Finally,
\RAPPOR{}
also allows collecting statistics on
non-categorical domains, or categories that cannot be enumerated ahead of time,
through the use of Bloom filters~\cite{bloom}.
In particular, \RAPPOR{}
allows collection of compact Bloom-filter-based randomized responses on strings,
instead of 
having clients report
when they match a set of hand-picked strings, predefined by the operator.
Subsequently,
those responses can be matched against candidate strings, as they become known to the operator,
and used to estimate both known and unknown strings in the population.
Advanced statistical decoding techniques must be applied
to accurately interpret the randomized, noisy data in Bloom-filter-based \RAPPOR{} responses.
However, as in the case of categories,
this analysis needs only consider the aggregate counts of distinct bits set in \RAPPOR{} responses
to provide good estimators for population statistics,
as detailed in Section~\ref{sec:decoding}.


Without loss of privacy,
\RAPPOR{} analysis can be re-run on a collection of responses, e.g., 
to consider new strings and cases missed in previous analyses,
without the need to re-run the data collection step.
Individual responses
can be especially useful for exploratory or custom data analyses. 
For example, if the geolocation of clients' IP addresses are collected
alongside the \RAPPOR{} reports of their sensitive values, 
then the observed distributions
of those values 
could be compared across different geolocations,
e.g., by analyzing different subsets separately.
Such analysis
is compatible with \RAPPOR{}'s privacy guarantees,
which hold true even in the presence of auxiliary data,
such as geolocation.
By limiting the number of correlated categories, or Bloom filter hash functions,
reported by any single client,
\RAPPOR{} can maintain its differential-privacy guarantees even when 
statistics are collected on multiple aspects of clients,
as outlined next, and detailed in Sections~\ref{sec:diffprivacy} and~\ref{sec:attacks}.











\subsection{\RAPPOR{} and (Longitudinal) Attacks}
Protecting privacy for both one-time and multiple collections requires consideration of several distinct attack models.
 A basic attacker is assumed to have access to a single report and can be stopped with a single round of randomized response.
A windowed attacker has access to multiple reports over time from the same user.
Without careful modification of the traditional randomized response techniques, almost certainly full disclosure of private information would happen.
This is especially true if the window of observation is large and the underlying value does not change much.
An attacker with complete access to all clients' reports (for example, an insider with unlimited access rights),
is the hardest to stop, yet such an attack is also the most difficult to execute in practice. \RAPPOR{} provides explicit trade-offs between
different attack models in terms of tunable privacy protection for all three types of attackers.

\RAPPOR{} builds on the basic idea of memoization and provides a framework for one-time and longitudinal privacy protection by playing the randomized response game twice with a memoization step in between. The first step, called a Permanent randomized response, is used to create a ``noisy'' answer which is memoized by the client and permanently reused in place of the real answer. The second step, called an Instantaneous randomized response, reports on the ``noisy'' answer over time, eventually completely revealing it. 
Long-term, longitudinal privacy is ensured by the use of the Permanent randomized response, while the use of an Instantaneous randomized response provides protection against possible tracking externalities.

The idea of \emph{underlying memoization} turns out to be crucial for privacy protection in the case where multiple responses are collected from the same participant over time. For example, in the case of the question about the Communist party from the start of the paper, memoization can allow us to provide -differential privacy even with an \emph{infinite} number of responses, as long as the underlying memoized response has that level of differential privacy.

On the other hand,
without memoization or other limitation on responses,
randomization is not sufficient to
maintain plausible deniability
in the face of multiple collections.
For example,
if 75 out of 100 responses are ``Yes'' for a single client
in the randomized-response scheme
at the very start of this paper,
the true answer will have been ``No''
in a vanishingly unlikely 
fraction of cases.


Memoization is absolutely effective in providing longitudinal privacy only in cases when the underlying true value does not change
 or changes in an uncorrelated fashion. When users' consecutive reports are temporally correlated, differential privacy guarantees
 deviate from their nominal levels and become progressively weaker as correlations increase. Taken to the extreme, when asking users
 to report daily on their age in days, additional measures are required to prevent full disclosure over time, such as stopping collection after
 a certain number of reports or increasing the noise levels exponentially, 
as discussed further in Section~\ref{sec:attacks}.
 
For a client that reports on a property that strictly alternates between two true values, (), the two memoized Permanent randomized
 responses for  and  will be reused, again and again, to generate \RAPPOR{} report data. Thus, an attacker that obtains a large enough number of reports,
 could learn those memoized ``noisy" values with arbitrary certainty---e.g.,
 by separately analyzing the even and odd subsequences.
However, even in this case, 
the attacker cannot be certain of the values of  and 
because of memoization.
This said,
if  and  are correlated,
the attacker may still learn more
than they otherwise would have;
maintaining privacy in the face of 
any such correlation 
is discussed further in Sections~\ref{sec:diffprivacy} and~\ref{sec:attacks} (see also~\cite{KiferM11}).

\begin{figure*}[!t]
\centering
\includegraphics[trim=0 0.2in 0 0,clip=true,width=2\columnwidth]{life.pdf}
\caption{Life of a \RAPPOR{} report: The client value of the string ``The number 68'' is hashed onto the Bloom filter  using  (here 4) hash functions. For this string, a Permanent randomized response  is produces and memoized by the client, and this  is used (and reused in the future) to generate Instantaneous randomized responses  (the bottom row), which are sent to the collecting service.}\vspace*{-2ex}
\label{fig:life}
\end{figure*}

In the next section we will describe the \RAPPOR{} algorithm in detail. We then provide intuition and formal justification for the reasons why the proposed algorithm satisfies the rigorous privacy guarantees of differential privacy. We then devote several sections to discussion of the additional technical aspects of \RAPPOR{} that are crucial for its potential uses in practice, such as parameter selection, interpretation of results via advanced statistical decoding, and experiments illustrating what can be learned in practice. The remaining sections discuss our experimental evaluation, the attack models we consider, the limitations of the \RAPPOR{} technique, as well as related work.


\section{The Fundamental \RAPPOR{} Algorithm}
Given a client's value , the \RAPPOR{} algorithm executed by the client's machine, reports to the server a bit array of size , that encodes a ``noisy" representation of its true value . The noisy representation of  is chosen in such a way so as to reveal a \emph{controlled} amount of information about , limiting the server's ability to learn with confidence what  was. This remains true even for a client that submits an infinite number of reports on a particular value .

To provide such strong privacy guarantees, the \RAPPOR{} algorithm implements two separate defense mechanisms, both of which are based on the idea of randomized response and can be separately tuned depending on the desired level of privacy protection at each level. 
Furthermore, additional uncertainty is added through the use of Bloom filters which serve not only to make reports compact, but also to complicate the life of any attacker (since any one bit in the Bloom filter may have multiple data items in its pre-image).

The \RAPPOR{} algorithm takes in the client's true value  and parameters of execution , and is executed locally on the client's machine performing the following steps:

\begin{enumerate}
\item {\bf Signal.} Hash client's value  onto the Bloom filter  of size  using  hash functions.
\item {\bf Permanent randomized response.} For each client's value  and bit  in , create a binary reporting value  which equals to

where  is a user-tunable parameter controlling the level of longitudinal privacy guarantee.\
P(S_i = 1) = \begin{cases}
q, & \text{if }. \\
p, & \text{if }.
\end{cases}

P(A(v_1) \in R) \le e^\epsilon P(A(v_2) \in R).

P(S = s | V = v) & = & P(S = s | B, B', v)\cdot P(B' | B, v)\cdot P(B | v) \\
                         & = & P(S = s | B')\cdot P(B' | B)\cdot P(B | v) \\
                         & = & P(S = s | B')\cdot P(B' | B).

P(b'_i = 1 | b_i = 1) & = & \frac{1}{2}f + 1 - f = 1 - \frac{1}{2}f \;\;\;\text{~and~}\\
P(b'_i = 1 | b_i = 0) & = & \frac{1}{2}f.

P(B' = b' | B = b^*) & = & \left(\frac{1}{2}f\right)^{b'_1}\left(1 - \frac{1}{2}f\right)^{1 - b'_1} \times \ldots \\
                   &   & \times \left(\frac{1}{2}f\right)^{b'_h}\left(1 - \frac{1}{2}f\right)^{1 - b'_h} \times \ldots \\
                   &   & \times \left(1 - \frac{1}{2}f\right)^{b'_{h + 1}}\left(\frac{1}{2}f\right)^{1 - b'_{h + 1}} \times \ldots \\
                   &   & \times \left(1 - \frac{1}{2}f\right)^{b'_k}\left(\frac{1}{2}f\right)^{1 - b'_k}.

RR_{\infty} & = &  \frac{P(B'\in R^* | B = B_1)}{P(B' \in R^* | B = B_2)} \\
                  & =  & \frac{\sum_{B'_i \in R^*}P(B' = B'_i |B = B_1)}{\sum_{B'_i \in R^*} P(B' = B'_i | B = B_2)} \\
                  & \le & \max_{B'_i \in R^*} \frac{P(B' = B'_i | B = B_1)}{P(B' = B'_i| B = B_2)}  \;\;\;\;\; \text{(by Observation~\ref{obs-ratios})} \\
                 & =  & \left(\frac{1}{2}f\right)^{2(b'_1 + b'_2 + \ldots + b'_h - b'_{h+1} - b'_{h+2} - \ldots - b'_{2h})} \\
                  & & \times \left(1 - \frac{1}{2}f\right)^{2(b'_{h+1} + b'_{h+2} + \ldots + b'_{2h} - b'_1 - b'_2 - \ldots - b'_h)}.

q^* = P(S_i = 1 | b_i = 1) = \frac{1}{2}f(p + q) + (1 - f)q.

p^* = P(S_i = 1 | b_i = 0) = \frac{1}{2}f(p + q) + (1 - f)p.

RR_1 &=& \frac{P(S \in R | B = B_1)}{P(S \in R | B = B_2)} \\
   &=& \frac{\sum_{s_j \in R}P(S = s_j | B = B_1)}{\sum_{s_j \in R}P(S = s_j | B = B_2)} \\
   &\le& \max_{s_j \in R} \frac{P(S= s_j | B = B_1)}{P(S = s_j | B = B_2)} \\
   &=& \left[\frac{q^*(1 - p^*)}{p^*(1 - q^*)}\right]^h

\epsilon_1 = h\log\left(\frac{q^*(1 - p^*)}{p^*(1 - q^*)}\right).

\frac{P(S_1 = s_1, S_2 = s_2, \ldots, S_j = s_j, \ldots, S_N = s_N | B_1)}{P(S_1 = s_1, S_2 = s_2, \ldots, S_j = s_j, \ldots, S_N = s_N | B_2)}  =\\
   \frac{\prod_{i=1}^N P(S_i = s_i | B_1)}{\prod_{i=1}^N P(S_i = s_i | B_2)} = \frac{P(S_j = s_j | B_1)}{P(S_j = s_j | B_2)}.

t_{ij} = \frac{c_{ij} - (p + \frac{1}{2}fq - \frac{1}{2}fp)N_j}{(1 - f)(q - p)}.

E(C_i) = qT_i + p(N - T_i),

\hat{T}_i = \frac{C_i - pN}{q - p}.

Var(\hat{T}_i) = \frac{p(1 - p)N}{(q - p)^2}.

\text{sd}(\hat{T}_i) = \frac{\sqrt{N}}{2q - 1}.

\hat{T}_i & > & Q \times \text{sd}(\hat{T}_i) \\
          & > & \frac{Q\sqrt{N}}{2q - 1},

\frac{N}{x} > \frac{Q\sqrt{N}}{2q - 1},

x \le \frac{(2q - 1)\sqrt{N}}{Q}.


\end{document}
