\documentclass{llncs}
\usepackage{amssymb,graphicx}
\usepackage{cite}




\spnewtheorem{Claim}{Claim}{\bfseries}{\it}
\newtheorem{observation}[theorem]{Observation}


\title{Recognizing -equistable graphs in FPT time\thanks{This research is supported in part by ``Agencija za raziskovalno dejavnost Republike Slovenije'', research program P-- and research projects J-, J-, J-, and BI-FR/----PROTEUS--.}}

\author{Eun Jung Kim\inst{1} \and Martin Milani\v{c}\inst{2} \and Oliver Schaudt\inst{3}}
\institute{CNRS-Universit\'e Paris-Dauphine\\ 
Place du Mar\'echal de Lattre de Tassigny, 75775 Paris Cedex 16\\ \email{eun-jung.kim@dauphine.fr}  \and
University of Primorska, UP IAM and UP FAMNIT\\
Muzejski trg 2, SI-6000 Koper, Slovenia.\\ \email{martin.milanic@upr.si}
\and
Universit\"at zu K\"oln,
Institut f\"ur Informatik,
Weyertal 80, 50931 K\"oln, Germany.\\ \email{schaudto@uni-koeln.de}}



\begin{document}

\maketitle

\begin{abstract}
A graph  is called \emph{equistable} if there exist a
positive integer  and a weight function  such that  is a
maximal stable set of  if and only if . Such a function 
is called an \emph{equistable function} of . For a positive integer , a graph  is said to
be \emph{-equistable} if it admits an equistable function which is bounded by .

We prove that the problem of recognizing -equistable graphs is fixed parameter tractable when parameterized by , affirmatively answering a question of Levit et al. In fact, the problem admits an -vertex kernel that can be computed in linear time.\\

\noindent{\textbf{Keywords:}} equistable graphs,
recognition algorithm,
fixed parameter tractability.
\end{abstract}

\section{Introduction}

The main notion studied in this paper is the class of equistable graphs, introduced by Payan in 1980~\cite{MR553649} as a generalization of the well known and well studied class of threshold graphs~\cite{MR0479384,MR1417258}. A {\it stable} (or {\it independent}) {\it set} in a (\hbox{finite}, simple, undirected) graph  is a set of pairwise non-adjacent vertices. A {\it maximal stable set} is a stable set not contained in any other stable set. A graph  is said to be {\it equistable} if there exists a function  such that for every , set  is a maximal stable set of  if and only if . Equivalently,  is equistable if and only if there exist a positive integer  and a weight function  such that  is a maximal stable set of  if and only if . Such a function  is called an {\em equistable function} of , while the pair  is called an {\em equistable structure}. Equistable graphs were studied in a series of papers~\cite{MR3040147,MR2794315,MR2379078,MR3162288,MR2024271,MR2024275,MR2823204,MR,MR1268776,MR553649,MT}; 
besides threshold graphs and cographs, they also generalize the class of general partition graphs~\cite{McAvaney1993131,MR2080087,MR2794315}. The complexity status of recognizing equistable graphs is open, and no combinatorial characterization of equistable graphs is known.

Levit et al.~introduced in~\cite{MR3040147} the notion of -equistable graphs. For a positive integer , a graph  is said to be {\em -equistable} if it admits an equistable function . Such a weight function is called a {\em -equistable function}, and the corresponding structure  is a {\em -equistable structure}. We remark that there exist equistable graphs such that the smallest  for which the graph is -equistable is exponential in the number of vertices of ~\cite{MR2823204}.

For a positive integer , an equistable graph  is said to be {\em \hbox{target-} equistable} if it admits an equistable
function  with equistable structure .
Clearly, every target- equistable graph is also -equistable (but not vice versa).

\begin{sloppypar}
As mentioned above, the complexity of recognizing equistable graphs is open, but it seems \hbox{plausible} that the problem could be NP-hard. It thus makes sense to search ways to simplify the recognition problem. To this end, we consider the following two parameterized problems related to equistability.
\end{sloppypar}

\medskip
\begin{center}
\fbox{\parbox{0.82\linewidth}{\noindent
{\sc -Equistability}\.8ex]
\begin{tabular*}{.9\textwidth}{rl}
{\em Input:} & A graph , a positive integer .\\
{\em Parameter:} & .\\
{\em Question:} & Is  target- equistable?
\end{tabular*}
}}
\end{center}

Apart from being natural parameterizations of the equistability problem, the
first problem has been tackled before (in a non-parameterized variant)
in a paper by Levit et al.~\cite{MR3040147}. There they prove the following.

\begin{theorem}[Levit et al.~\cite{MR3040147}]\label{thm:k-equistable}
For every fixed , there is an  algorithm
to decide whether a given -vertex graph is -equistable.
In case of a positive instance, the algorithm also produces a -equistable structure of~.
\end{theorem}

Also, the authors ask whether Theorem~\ref{thm:k-equistable} can be strengthened in the sense that there is an FPT-algorithm for recognizing -equistable graphs.
We answer this question affirmatively.

More precisely, we prove the following results:
\begin{itemize}
  \item There is an -vertex kernel for the {\sc -Equistability} problem that can be computed in linear time.
	      This yields an FPT algorithm for the {\sc -Equistability} problem of running time , given a graph with  vertices and  edges. This affirmatively answers the question posed by Levit et al.~\cite{MR3040147}.
  \item The {\sc Target- Equistability} problem admits an -vertex kernel, computable in linear time.
Moreover, there is an  time algorithm to solve the {\sc Target- Equistability} problem.
\end{itemize}
The first result we prove in Section~\ref{sec:k-equi}, and the second in Section~\ref{sec:t-equi}.

In order to achieve the above mentioned running times of our FPT algorithms, we present a refinement of the algorithm proposed
by Levit et al.~in~\cite{MR3040147} in their proof of Theorem~\ref{thm:k-equistable}. This we present in Section~\ref{sec:XP}.

\section{Preliminaries}

\subsection{Twin classes}\label{subsec:twin-classes}

Following~\cite{MR3040147}, we say that vertices  and  of a graph  are {\em twins} if they have exactly the same set of neighbors other
than  and . It is easy to verify that the twin relation is an equivalence relation.
We recall some basic properties of the twin relation~(see~\cite{MR3040147}):

\begin{lemma}
Let  be a graph. The twin relation is an equivalence relation, and every equivalence class is either a clique or a stable set.
\end{lemma}

An equivalence class of the twin relation will be referred to as a {\em twin class}.
Twin classes that are cliques will be referred to {\em clique classes}, and the remaining classes will be referred to as {\em stable set classes}.
We say that two disjoint sets of vertices  and  in a graph  {\em see} each other if every vertex of  is adjacent to every vertex of , and they {\em miss} each other if every vertex of  is non-adjacent to every vertex of . A vertex  {\em sees}
a set  if the singleton  sees , and similarly  {\em misses}  if  misses~.
The set of all twin classes will be denoted by  and referred to as the {\em twin partition} of . The number of twin classes of  will be denoted by . The following observation is an immediate consequence of the fact that the twin classes are equivalence classes under the twin relation.

\begin{observation}\label{j34fskf}
Every two distinct twin classes either see each other or miss each other.
\end{observation}

By Observation~\ref{j34fskf}, the {\em quotient graph} of , denoted , is thus well defined: Its vertex set is , and two twin classes are adjacent if and only if they see each other in . Given a graph , it is possible to find in linear time the twin partition , the quotient graph  and , using any of the linear time algorithms for modular decomposition \cite{MR2500307, DBLP:conf/caap/CournierH94, MR1687819}.

The following two lemmas due to Levit et al.~\cite{MR3040147} show why twin partitions are important in the study of equistable graphs.

\begin{lemma}\label{lem:same-weight}
For every equistable function  of  and for every , every set of the form  is a subset of a twin class of .
In particular, if  is a -equistable graph, then .
\end{lemma}

\begin{corollary}\label{cor:target-t-number-of-classes}
If  is a target- equistable graph, then .
\end{corollary}

\begin{lemma}\label{lem:twin-same-weight}
For every equistable function  of an equistable graph  and for every clique class  there exists an  such that
.
\end{lemma}

\subsection{Parameterized complexity}

\begin{sloppypar}
A decision problem parameterized by a problem-specific parameter~ is called \emph{fixed-parameter tractable} if there exists an algorithm that solves it in time \hbox{}, where~ is the instance size.
The function~ is typically super-polynomial and depends only on~.
One of the main tools to design such algorithms is the kernelization technique.
A \emph{kernelization} is a polynomial-time algorithm which transforms an instance~ of a parameterized problem into an equivalent instance~ of the same problem such that the size of~ is bounded by~ for some computable function~ and  is bounded by a function of .
The instance~ is said to be a \emph{kernel} of size~.
It is a folklore that a parameterized problem is fixed-parameter tractable if and only if it admits a kernelization. In the remainder of this paper, the kernel size is expressed in terms of the number of vertices.
For more background on parameterized complexity the reader is referred to
Downey and Fellows~\cite{DF13}.
\end{sloppypar}

\section{A refined XP-algorithm for {\sc -Equistability}}\label{sec:XP}

In this section we propose a revised version of the algorithm of Levit et al.~\cite{MR3040147} for checking whether a given graph is -equistable.
We implement some speed-ups and give a more careful analysis of the running time. Let us remark that this improvement does not speed up the running time when  is fixed, and it is thus not relevant for the main result of Levit et al.~\cite{MR3040147}.
However, the improved running time is essential when the algorithm is applied to a kernelized instance, for the
{\sc -Equistability} resp.~{\sc Target- Equistability} problem.
We refrain from formally restating the whole algorithm from~\cite{MR3040147} in order not to create redundancy.

\begin{theorem}\label{thm:refined-XP}
Let  be a graph on  vertices and  edges, and let .
Then there is an algorithm of running time

to check whether  is -equistable.
This algorithm computes a -equistable structure, if one exists, and the same holds if a target  is prescribed.
\end{theorem}

We emphasize that unlike in the statement of Theorem~\ref{thm:k-equistable}, the constant hidden in the -notation
in Theorem~\ref{thm:refined-XP} does not depend on  (in Theorem~\ref{thm:refined-XP},  is not restricted to be a constant).

Before we prove Theorem~\ref{thm:refined-XP}, we state the following observation.

\begin{lemma}\label{lem:technicalbound}
Let  and let  with .
Then 
\end{lemma}

\begin{sloppypar}
\begin{proof}
If , the statement is immediate.
So, let , and assume the statement is true for .
Let  with .
We know that ,
and thus \hbox{} \hbox{}.
A straightforward calculation shows that the right hand side is maximized (over )
for .
Thus,

which completes the proof.
\qed \end{proof}
\end    {sloppypar}

We can now prove Theorem~\ref{thm:refined-XP}.

\begin{proof}[of Theorem~\ref{thm:refined-XP}]
Recall that by Lemma~\ref{lem:same-weight}, any equistable weight function for  assigns the same weight only to vertices of the same twin class. Following the algorithm of Levit et al.~\cite{MR3040147}, we proceed as follows.
First, we compute in time  the twin partition of 
and the quotient graph  (cf.~Section~\ref{subsec:twin-classes}).
Fix any ordering  such that vertices in each twin class appear consecutively in this ordering. Clearly, the permutation of the weights within a twin class produces an equivalent weight function, i.e., a weight function is an equistable function of  if and only if after any permutation of the weights within a twin class we still have an equistable function. We aim to produce a family  which contains all equistable functions up to permutations of the weights within a twin class. It suffices to produce all mappings  such that the vertices in the set , , appear consecutively in the ordering of . Let  be the number of partitions of  into  labeled intervals, where some of the intervals may be empty. It is straightforward to verify that  is bounded by .
A standard counting argument yields .

The set  can be computed in time  as follows.
Generate all one-to-one mappings from the set  to an -element set. Using the above ordering of ,
each such mapping determines a partition of . If the partition refines the twin partition of , then compute all the  one-to-one mappings from the resulting set of (at most ) non-empty intervals to the set . Each of these mappings specifies, in a natural way, a function in .

Let us now estimate more carefully the size of . Let . We have

implying .
We thus obtain

Thus, we have to consider only  many weight functions, which
can be computed in time
.

It remains to check if any of these  weight functions in  is an equistable function.
For every weight function , the algorithm from~\cite{MR3040147} first computes the target value 
by evaluating the -weight of an arbitrary (fixed) maximal stable set of  (see~\cite{MR3040147} for details);
in our setting, this computation can be implemented in time .
The algorithm then computes the set  of all -dimensional vectors  with integer coordinates such that
 for all . A vector  represents the set of all subsets of  such that
the number of vertices of -weight  in the set equals .

Note that the number of vectors in  is bounded by , which, by Lemma~\ref{lem:technicalbound},
is in turn bounded by , for each function . For each vector , the algorithm then checks whether the corresponding sets are of the right weight, that is, whether  if  and only if the vector encodes a set of maximal stable sets. This latter condition can be verified in time  using the quotient graph  (see~\cite{MR3040147} for details).

\begin{sloppypar}
The running time of this algorithm is thus

This expression simplifies to , as desired.
\end{sloppypar}

We remark that, in case of a prescribed target value , the above algorithm can be modified in an obvious way to accept only those equistable functions under which all maximal stable sets have total weight . This completes the proof.
\qed \end{proof}


\section{An -vertex kernel for the {\sc Target- Equistability} problem}\label{sec:t-equi}

Given a graph , the following reduction rule is specified by a positive integer  as a parameter.

\begin{itemize}
  \item[] {\bf -Clique Reduction}. If a clique class  contains more than  vertices, delete from  all but  vertices.
\end{itemize}
The following lemma shows why -Clique Reduction rule is safe for both problems, {\sc Target- Equistability} and {\sc -Equistability}.

\begin{lemma}\label{lem:clique-reduction}
Let  be a graph,  a finite set,  a clique class of  with  where , and  a positive integer.
Then, for every , graph  is target- -equistable if and only if  is target- -equistable, where  is a graph obtained after the -Clique Reduction rule has been applied to  with respect to the clique class .
\end{lemma}

\begin{proof}
Let . First assume that  is target- -equistable, say with a -equistable structure .
It is immediate that the restriction  of  to  yields a -equistable structure  of .
Therefore  is target- -equistable.

Now assume that  is target- -equistable, with a -equistable structure .
We define a function  by extending  to the set .
Indeed, we simply put  for all , and  for all  where .
The choice of  is arbitrary, since  is constant on  by Lemma~\ref{lem:twin-same-weight}.

We claim that  is an equistable structure of .
To show this, pick an arbitrary maximal stable set  of .
Then , and so we may assume that .
Clearly  is a maximal stable set of , and so .
Therefore .

Conversely, let  be a set with .
Since , we have .
As  is constant on  and , we may w.l.o.g.~assume that .
Hence, , and so  is a maximal stable set of .
Thus,  is a maximal stable set of  which completes the proof.
\qed \end{proof}

In particular, -Clique Reduction rule is safe for the {\sc Target- Equistability} problem.
This is seen by putting  and  in the statement of Lemma~\ref{lem:clique-reduction}.

\begin{theorem}\label{thm:t-equistable}
The {\sc Target- Equistability} problem admits a kernel of at most  vertices, computable in linear time.
Moreover, there is an  time algorithm to solve the {\sc Target- Equistability} problem,
given a graph with  vertices and  edges.
\end{theorem}
\begin{proof}
Let  be a graph on  vertices and  edges.
Using one of the linear time algorithms for modular decomposition~\cite{MR2500307,DBLP:conf/caap/CournierH94,MR1687819}, we can compute  and  in linear time.
If , then we conclude that  is not target- equistable, by Corollary~\ref{cor:target-t-number-of-classes}.
Similarly, if there exists a stable set class  with , then we conclude that  is not target- equistable.
Also, we can apply -Clique Reduction rule with parameter , to every clique class, in linear time.
Afterward, the graph has at most  vertices, which proves the first statement of the theorem.

Our FPT algorithm works as follows.
First we compute in time  a kernel  with  many vertices.
Then we apply Theorem~\ref{thm:refined-XP} to check whether  is target- equistable.
For this, we can put  and decide whether  is -equistable with target .
We thus obtain a running time of
\hbox{}.
\qed \end{proof}



\section{An -vertex kernel for the {\sc -Equistability} problem}\label{sec:k-equi}

This section is devoted to the proof of the following result.

\begin{theorem}\label{thm:k-equistable-fpt}
The {\sc -Equistability} problem admits an -vertex kernel, computable in linear time.
Moreover, there is an  time algorithm to solve the {\sc -Equistability} problem, given a graph with  vertices and  edges.
\end{theorem}

\begin{proof}
Let us first prove that the second statement follows from the first one. Assume that we can compute an -vertex kernel for the {\sc -Equistability} problem in linear time. By Theorem~\ref{thm:refined-XP}, we can then decide whether this kernel is -equistable in time .

We now turn to the construction of the -vertex kernel.
In case of a no-instance, our algorithm simply returns a non-equistable graph, say the -vertex path .
In what follows, we will assume that the input graph  satisfies , since otherwise  is not -equistable, by Lemma~\ref{lem:same-weight}.
The following claim is the main step of our kernelization.

\begin{Claim}\label{clm:big-classes}
If there exist
two distinct twin classes  and  such that
one of them is a stable set and ,
then  is not -equistable.
\end{Claim}

\begin{proof}
Suppose for a contradiction that  is -equistable, with an equistable weight function , and that there exist
two distinct twin classes  and  with  such that  is a stable set.
If the set  is contained in every maximal stable set of , then  forms a twin class, a contradiction.
Thus, we may assume without loss of generality that there exists a maximal stable set  of 
such that  and .

Recall that  is either a clique class or a stable set class.
Since every clique intersects every stable set in at most one vertex
and every stable set class is either entirely contained in  or disjoint from it,
the fact that  implies .
Let  be weights such that , and
. Since , there exists a set  of  vertices in  of weight .
Since  and , there exists a set  of  vertices in  of weight  such that
. Then, the set  is not a stable set, since
otherwise by Observation~\ref{j34fskf} the set  would be a stable set properly containing , contrary to the  maximality of .
Note that , contradicting the assumption that  is an equistable weight function of .
\qed \end{proof}

We consider the following two cases.\\

\noindent
\textbf{Case 1.} \emph{Every twin class  with  is a clique class.}\\

In this case, every stable set class has less than  vertices, which implies that every maximal stable set of  contains at most
 vertices from each twin class and is thus of total size at most .
This implies that in every -equistable structure  of , we have .

We now perform -Clique Reduction rule from Section~\ref{sec:t-equi} with .
By Lemma~\ref{lem:clique-reduction} applied with  and , the application of -Clique Reduction rule is safe.
When the rule can no more be applied, we have a graph  with at most  twin classes, each of size at most
.
We are done since .\\
	
\noindent
\textbf{Case 2.} \emph{There exists a stable set twin class  with .}\\

\noindent
By Claim~\ref{clm:big-classes}, we may assume that  is the unique twin class of size at least 
(since otherwise  is not -equistable).

Note that  contains at most  twin classes, each containing less than  vertices, 
hence .

Suppose first that  corresponds to an isolated vertex in the quotient graph .
If , then  and we are done.

So suppose that .

\begin{Claim}\label{claim}
 is -equistable if and only if it admits a -equistable function that is constant on .
\end{Claim}

\begin{proof}
The if part being trivial, assume that  is -equistable, and let  be a -equistable structure of .
Let  be such that , where .
Now we define a weight function  that equals  outside , and is constantly  on .
We claim that  is a -equistable function of . Clearly,  is bounded by .
Under , all maximal stable sets of  have weight .

The only possible problem is that  for some vertex set  that is not a maximal stable set of .
In this case, we claim that . To see this, suppose . 
Since  , we get .
Therefore 
.
But , since  and . Thus
, a contradiction.

So, , and since  and  is constant on  we may assume that
. But this yields

A contradiction.
\qed \end{proof}

According to Claim~\ref{claim}, it suffices to test if  is -equistable by
considering all possible functions  that are constant on , and test for each of them whether it is a -equistable function.

Before that, we reduce size of .
For this, we compute a graph  from  by deleting all but  many vertices from .
Note that, since  is a twin class,  is unique up to isomorphism.

\begin{Claim}\label{clm:G-vs-G-prime}
 is -equistable if and only if  is -equistable.
\end{Claim}

\begin{proof}
Let  and .

First we assume that  is -equistable, say with an equistable structure .
By Claim~\ref{claim}, we may assume that  is constant on , say .
We now consider the weight function  with target value , and claim that  is a -equistable structure of .
Since every maximal stable set of  (resp.,~) contains  (resp.,~) as a subset,
it is straightforward that every maximal stable set of  has weight .
Suppose that there is a set  with  that is not a maximal stable set of .
Then the set  has total weight , but is not a maximal stable set of , a contradiction.
This proves that  is -equistable.

Now we assume that  is -equistable, say with an equistable structure .
By Claim~\ref{claim} applied to , we may assume that  is constant on , say .
Consider the weight function  defined as  for all  and  for all  with target
value . We claim that  is a -equistable structure of .
Again it is straightforward that any maximal stable set of  has weight .
Suppose that there is a set  with  that is not a maximal stable set of .

Recall that  and consequently .
If , we thus obtain

a contradiction.
Thus, , and so we may assume that .
Let .
Then , but  is not a maximal stable set of .
This is contradictory, and so  is -equistable.
\qed \end{proof}

By Claim~\ref{clm:G-vs-G-prime}, it suffices to check whether  is -equistable.
Since , we are done.

\bigskip
\begin{sloppypar}
Now, suppose that  corresponds to a non-isolated vertex in the quotient graph .
Then, there exists a twin class  that sees .
Let  be a maximal stable set of  containing a vertex of .
Then, . Since
\hbox{}, we have in particular
that .
\end{sloppypar}

If , then
for every -equistable function  of  and
every maximal stable set, say , such that , we have
, hence  is not -equistable.

If , then .

Since it is clear that the above algorithm runs in time , the proof is complete.
\qed \end{proof}





\section{Future work}

Several open problems surrounding our work remain, some of which we want to mention here in order to stimulate
research on this topic.

Firstly, we believe it is NP-hard to determine, given a graph  and an integer , whether  is -equistable.
It would be satisfying to see this proven, especially for the purpose of this paper. As mentioned in the introduction, the smallest such  (if existing) might have to be exponential in the number of vertices of ~\cite{MR2823204}, which might serve as a hint for the hardness of this problem.

The analogous question is open also for the problem of {\sc Target- Equistability}: what is the computational complexity of
determining, given a graph  and an integer , whether  is target- equistable? Again, the smallest such  (if existing) might have to be exponential in the number of vertices of the input graph~\cite{MR2823204}.

A different computational problem in this context would be the following: given a graph  and a number , does it admit an equistable weight function using at most  different weights? Here, both the parameterized and classical complexity are unknown.
Although we did not study this problem in depth, our impression is that it should be NP-hard, but FPT when parameterized by .
In view of the results of the present paper, there might very well be a polynomial kernel for this problem.
Another problem that seems similar at first sight is whether equistability is FPT when parameterized by , the number of twin-classes of .

Apart from these recognition problems, it is apparently open whether the maximum stable set problem
is FPT in the class of equistable graphs. Here we do at least know that this problem is APX-hard in this class~\cite{MR2823204}.

\bibliographystyle{abbrv}
\bibliography{bibliography}
\end{document}
