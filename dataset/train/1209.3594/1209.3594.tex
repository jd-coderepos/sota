

\documentclass[11pt]{article}
\usepackage[english]{babel} \usepackage[T1]{fontenc}
\usepackage{graphicx,xspace,amsmath,marvosym,enumerate}
\usepackage{wrapfig,enumerate,dsfont,hyperref,subfig,multirow}
\usepackage[table]{xcolor}
\usepackage{booktabs}

\usepackage[boldsans]{concmath} \usepackage{euler} 

\usepackage{amsfonts,amssymb,bm,theorem,fullpage}

\newtheorem{lemma}[equation]{Lemma}
\newtheorem{corollary}[equation]{Corollary}
\newtheorem{proposition}[equation]{Proposition}
\newtheorem{theorem}[equation]{Theorem}
\newtheorem{definition}[equation]{Definition}
\newtheorem{observation}[equation]{Observation}
\newtheorem{remark}[equation]{Remark}
\newtheorem{assumption}[equation]{Assumption}
\newtheorem{claim}[equation]{Claim}
\newtheorem{property}[equation]{Property}
\newcommand{\qed}{\hfill\ensuremath{\Box}}
\newenvironment{proof}{\noindent\textbf{Proof.}
}{\par\medskip}

\newcommand{\N}{\ensuremath{\mathds N}}
\newcommand{\Z}{\ensuremath{\mathds Z}}
\newcommand{\Q}{\ensuremath{\mathds Q}}
\newcommand{\R}{\ensuremath{\mathds R}}
\newcommand{\T}{\ensuremath{\mathcal T}}
\newcommand{\F}{\ensuremath{\mathcal F}}
\newcommand{\figscale}{1.1}



\graphicspath{{figures/}}



\begin{document}

\title{On Universal Point Sets for Planar Graphs}

\author{Jean Cardinal\thanks{Partially supported by the ESF EUROCORES
    programme EuroGIGA, CRP ComPoSe. The research was carried out while the
    first author was at ETH Z\"urich.} \and Michael
  Hoffmann\thanks{Partially supported by the ESF EUROCORES programme
    EuroGIGA, CRP GraDR and the Swiss National Science Foundation, SNF Project
    20GG21-134306.} \and Vincent Kusters}

\date{\small D\'epartement d'Informatique, Universit\'e Libre de
  Bruxelles, \texttt{jcardin@ulb.ac.be}\\
  Institute of Theoretical Computer Science, ETH Z\"urich,
  \texttt{\{hoffmann,kustersv\}@inf.ethz.ch}}



\maketitle \begin{abstract}
  A set  of points in  is -universal, if every planar graph on 
  vertices admits a plane straight-line embedding on . Answering a question
  by Kobourov, we show that there is no -universal point set of size , for
  any . Conversely, we use a computer program to show that there exist
  universal point sets for all  and to enumerate all corresponding
  order types. Finally, we describe a collection  of  planar
  graphs on  vertices that do not admit a simultaneous geometric embedding
  without mapping, that is, no set of  points in the plane supports a plane
  straight-line embedding of all graphs in .
\end{abstract}
\sloppy

\section{Introduction}
\label{sec:introduction}

We consider plane, straight-line embeddings of graphs in . In those
embeddings, vertices are represented by pairwise distinct points, every edge is
represented by a line segment connecting its endpoints, and no two edges
intersect except at a common endpoint.

An {\em -universal} (or short \emph{universal}) point set for planar graphs
admits a plane straight-line embedding of all graphs on  vertices. A
longstanding open problem is to give precise bounds on the minimum number of
points in an -universal point set. The currently known asymptotic bounds are
apart by a linear factor. On the one hand, it is known that every planar graph
can be embedded on a grid of size ~\cite{fpp-hdpgg-90,s-epgg-90}. On the other hand, it was shown that at
least  points are necessary~\cite{Kurowski04}, improving earlier bounds
of ~\cite{ck-lbsuspg-89} and ~\cite{fpp-hdpgg-90}.

The following, somewhat simpler question was asked ten years ago by
Kobourov~\cite{dmo-topp-}: what is the largest value of  for which a
universal point set of size  exists? We prove the following.
\begin{theorem}
\label{thm:main}
There is no -universal point set of size , for any .
\end{theorem}

At some point, the Open Problem Project page dedicated to the
problem~\cite{dmo-topp-} mentioned that Kobourov proved there exist 14-universal
point sets of size 14. If this is correct, our bound is tight, and the answer to
the above question is . After verification, however, this claim appears to
be unsubstantiated~\cite{Kobourov}. We managed to check that there exist
universal point sets only up to . Further investigations are ongoing.

\paragraph{Overview.}
Section~\ref{sec:uni} is devoted to the proof of Theorem~\ref{thm:main}.  It
combines a labeled counting scheme for \emph{planar 3-trees} (also known as
\emph{stacked triangulations}) that is very similar to the one used by Kurowski
in his asymptotic lower bound argument~\cite{Kurowski04} with known lower bounds
on the rectilinear crossing number~\cite{AF05,lvww-cqk-04}.  Note that although
planar 3-trees seem to be useful for lower bounds, a recent preprint from Fulek
and T\'oth~\cite{FT12} shows that there exist -universal point sets of size
 for planar 3-trees.

For a collection  of planar graphs on 
vertices, a \emph{simultaneous geometric embedding without mapping} for
 is a collection of plane straight-line embeddings  onto the same set  of  points.

In Section~\ref{sec:sim}, we consider the following problem: what is the largest
natural number  such that every collection of  planar graphs on
the same number of vertices admit a simultaneous geometric embedding without
mapping? From the F\'ary-Wagner Theorem~\cite{f-slrpg-48,w-bzv-36} we know that
.  We prove the following upper bound:
\begin{theorem}\label{thm:sim}
  There is a collection of  planar graphs on  vertices that do not
  admit a simultaneous plane straight-line embedding without mapping, hence
  .
\end{theorem}
To our knowledge these are the best bounds currently known. It is a very
interesting and probably challenging open problem to determine the exact value
of~.

Finally, in Section~\ref{sec:small}, we use a computer program to show that
there exist -universal point sets of size  for all  and give the
total number of such point sets for each . As a side remark, note that it is
not clear that the property ``there exists an -universal point set of size
'' is monotone in .

\section{Large Universal Point Sets}
\label{sec:uni}

A planar 3-tree is a maximal planar graph obtained by iteratively splitting a
facial triangle into three new triangles with a degree-three vertex, starting
from a single triangle. Since a planar 3-tree is a maximal planar graph, it has
 vertices and  triangular faces and its combinatorial embedding is
fixed up to the choice of the outer face. 

For every integer , we define a family  of labeled planar 3-trees
on the set of vertices  as follows:
\begin{enumerate}[(i)]
\item  contains only the complete graph ,
\item  contains every graph that can be constructed by making the new
vertex  adjacent to the three vertices of one of the  facial triangles
of some .
\end{enumerate}
We insist on the fact that  is a set of {\em labeled} abstract graphs,
many of which can in fact be isomorphic if considered as abstract (unlabeled)
graphs. We also point out that for , the class  does not contain
\emph{all} labeled planar 3-trees on  vertices. For instance, the four graphs
in  are shown in \figurename~\ref{fig:T5}, and there is no graph for which
both Vertex~ and Vertex~ have degree three.

\begin{lemma}\label{lem:tn_size_lower}
  For , we have .
\end{lemma}
\begin{proof}
  By definition, . Every graph in  is constructed by splitting
  one of the  faces of a graph in . We therefore have:
  
\end{proof}

\begin{figure}[htbp]
  \centering \includegraphics[scale=\figscale]{t5a-1}\hfill \includegraphics[scale=\figscale]{t5a-2}\hfill \includegraphics[scale=\figscale]{t5a-3}\hfill \includegraphics[scale=\figscale]{t5a-4}
  \caption{\label{fig:T5}The four planar 3-trees in , with vertex set .}
\end{figure}

\begin{lemma}
  \label{lem:labeled_points_unique_stacked}
  Given a set  of labeled points in the plane and a
  bijection , there is \emph{at most one}  such that
   is a plane straight-line embedding of .
\end{lemma}
\begin{proof}
  Consider any such labeled point set  and assume without loss of
  generality that  for all . In all  the
  vertices  form a . Hence, for all , the
  straight-line embedding  connects all pairs of points in
   with line segments. If these points are in
  convex position, there is a crossing and there is no  for
  which  is a plane straight-line embedding
  (\figurename~\ref{fig:crossingornot}). Otherwise, there is a unique
  graph  for which  is a plane
  straight-line drawing. We proceed as follows.

  \begin{figure}[htbp]
\hfil \includegraphics[scale=\figscale]{crossingornot-2}\hfil \includegraphics[scale=\figscale]{crossingornot-1}\hfil \caption{\label{fig:crossingornot}Some permutations of a given point set do
      not define any planar 3-tree in , because they generate a crossing
      (left). On the other hand, when no such crossing occurs, the permutation
      defines a unique planar 3-tree in  (right). At any rate, a single
      permutation can be associated with at most one planar 3-tree in .}
  \end{figure}

  Given a plane straight-line drawing for some graph  on the first
   points, the next point  is located in some triangular
  region of the drawing; denote this region by . Only if
  during the construction of  we decided to connect the next vertex  to
  exactly the vertices , there is no crossing introduced by
  mapping  to . (An edge to any other vertex would cross one of
  the bounding edges of the triangle .) In other words,
  for every  the role of vertex  is completely determined. If no
  crossing is ever introduced, this process determines exactly one graph
   for which  forms a plane straight-line embedding. (Note that
  a crossing can be introduced only if  is located outside of the
  convex hull of . And also in that case there need not be
  a crossing, as the example in \figurename~\ref{fig:crossingornot} (right)
  shows.)\qed
\end{proof}

\noindent We use the following theorem by \'{A}brego and Fern\'{a}ndez-Merchant.
\begin{theorem}[\cite{AF05}]\label{t:cn}
  Every plane straight-line drawing of the complete graph  has at
  least  crossings.
\end{theorem}
Note that for  at least one of the floor expressions is zero, whereas
for  the theorem states that every straight-line drawing of  has at
least one crossing. Any pair of crossing edges corresponds to a four-tuple of
points in convex position. Using this interpretation we can easily derive a
floor-free lower bound on the number of convex four-gons contained in every
planar point set.
\begin{corollary}\label{cor:convex_4_tuples}
  Given a point set  of  points in general position,
  more than a -fraction of all four
  element subsets of  is in convex position.
\end{corollary}
\begin{proof}
  By Theorem~\ref{t:cn} at least  four element subsets of  are in convex position. For  odd we
  have

and for  even we have

and so

for all .\qed
\end{proof}
We will use this fact to prove the following lemma.
\begin{lemma}\label{lem:tn_on_point_set}
  On any set  of  points fewer than
   graphs from  admit a plane
  straight-line embedding.
\end{lemma}
\begin{proof}
  Let  be a set of  points and denote by
   the set of labeled planar -trees from 
  that admit a plane straight-line embedding onto . Note that a
  straight-line embedding can be represented by a permutation  of
  the points of , where each vertex  is mapped to point
  . Let  be the set of all permutations of . We define
  a map  by assigning to each  some
   such that  is a plane straight-line
  embedding of  (such an embedding exists by definition of ).

  By Lemma~\ref{lem:labeled_points_unique_stacked}, every permutation
   is a plane straight-line embedding of \emph{at most
    one} . It follows that  is a injection, and hence
  , with , is a bijection and so  
  .

  Next we can quantify the difference between  and  using
  Corollary~\ref{cor:convex_4_tuples}. Note that the general position
  assumption is not a restriction, since in case of collinearities, a
  slight perturbation of the point set yields a new point set that
  still admits all plane straight-line drawings of the original point
  set. Consider a permutation  such that
   form a convex quadrilateral. As argued in the
  first paragraph of the proof of
  Lemma~\ref{lem:labeled_points_unique_stacked},  is not a plane
  straight-line embedding for any . It follows that . We know from Corollary~\ref{cor:convex_4_tuples}
  that more than a fraction of  of the 4-tuples of
   are in convex position and therefore a corresponding fraction of
  all permutations does \emph{not} correspond to a plane straight-line
  drawing. So we can bound the number of possible labeled plane
  straight-line drawings by
  
\end{proof}

\begin{proof}[of Theorem~\ref{thm:main}]
  Consider an -universal point set  with
  . Being universal, in particular  has to accommodate all
  graphs from . By Lemma~\ref{lem:tn_size_lower}, there are
  exactly  graphs in , whereas by
  Lemma~\ref{lem:tn_on_point_set} no more than
   graphs from  admit a plane
  straight-line drawing on . Combining both bounds we obtain
  . Setting  yields
  , which is a
  contradiction and so there is no -universal set of  points.

  For  the inequality reads  and so there is no indication that there cannot be a
  -universal set of  points. To prove the claim for any
  , consider the two functions  and
   that constitute the inequality. As  is
  exponential in  whereas  is just a cubic polynomial, 
  certainly dominates , for sufficiently large .  Moreover, we
  know that . Noting that  and ,
  for , it suffices to show that , for all .  

  \noindent We can bound
  
  which is easily seen to be upper bounded by two, for .\qed
\end{proof}

\section{Simultaneous Geometric Embeddings}
\label{sec:sim}

The number of non-isomorphic planar 3-trees on  vertices was computed by
Beineke and Pippert~\cite{BP74}, and appears as sequence A027610 on Sloane's
Encyclopedia of Integer Sequences. For , this number is .  Hence
we can also phrase our result in the language of simultaneous
embeddings~\cite{bcdeeiklm-spge-07}.
\begin{corollary}
  There is a collection of  planar graphs that do not admit a
  simultaneous (plane straight-line) embedding without mapping.
\end{corollary}
In the following we will give an explicit construction for a much smaller family
of graphs that not admit a simultaneous embedding without mapping. As a first
observation, note that the freedom to select the outer face is essential in
order to embed graphs onto a given point set. In fact, for planar 3-trees, the
mapping for the outer face is the only choice there is. We prove this in two
steps.
\begin{lemma}
  \label{lem:stacked_any_face}
  Let  be a labeled planar 3-tree on the vertex set , for , and
  let  denote any triangle in . Then  can be constructed starting from
   by iteratively inserting a degree-three vertex into some facial triangle
  of the partial graph constructed so far.
\end{lemma}
\begin{proof}
  We prove the statement by induction on . For  there is nothing to
  show. Hence let . By definition  can be constructed iteratively from
  \emph{some} triangle in the way described. Without loss of generality suppose
  that adding vertices in the order  yields such a construction
  sequence. Denote by  the graph that is constructed by the sequence
  , for .

  Let  such that . Consider the graph : In the last step,
   is added as a new vertex into some facial triangle  of . As
   is a neighbor of both  and  in , both  and  are vertices of
  ; denote the third vertex of  by . Note that all of  and
   and  are facial triangles in .

  If , then exchanging the role of  and  yields a construction
  sequence  for , as claimed. If , then 
  is a separating triangle in . By the inductive hypothesis we can obtain a
  construction sequence  for  starting with the triangle
  . The desired sequence for  is obtained as
  , where  is the suffix of  that excludes the
  starting triangle .\qed
\end{proof}
And now we can prove the desired property:
\begin{lemma}
  \label{lem:stacked_unique}
  Given a labeled planar 3-tree  on vertex set , a triangle
   in , and a set  of  points with
  , there is at most one way to complete the partial
  embedding  to a plane
  straight-line embedding of  on .
\end{lemma}
\begin{proof}
  We use Lemma~\ref{lem:stacked_any_face} to relabel the vertices in such a way
  that  becomes  and the order  is a construction
  sequence for . Embed vertices  onto . We iteratively
  embed the remaining vertices as follows. Vertex  was inserted into some
  face  during the construction given by
  Lemma~\ref{lem:stacked_any_face}. Note that  have already been
  embedded on points . The vertices contained in the triangle
   (except ) are partitioned into three sets by the cycles 
  ( vertices) and  ( vertices) and  (
  vertices). We want to embed  on a point  such that 
  contains exactly  points,  contains exactly  points
  and  contains exactly  points. Note that it is necessary
  to embed  on a point with this property: if some triangle has too few
  points, then it will not be possible to embed the subgraph of  enclosed by
  the corresponding cycle there. It remains to show that there is always at most
  one choice for . Suppose that there are two candidates for , say
   and . Then  must be contained in  or
   or  (or vice versa). Without loss of
  generality, let it be contained in : now  contains
  fewer points than , which is a contradiction. The lemma follows by
  induction.\qed
\end{proof}
Therefore it is not surprising that it is very easy to find three graphs that do
not admit a simultaneous (plane straight-line) embedding without mapping, if the
mapping for the outer face is specified for each of them.
\begin{figure}[htbp]
  \hfil \includegraphics[scale=\figscale]{threegraphs-1}\hfil \includegraphics[scale=\figscale]{threegraphs-2}\hfil \includegraphics[scale=\figscale]{threegraphs-3}\hfil \caption{\label{fig:three}Three planar graphs that do not admit a simultaneous
    geometric embedding with a fixed mapping for the outer face.}
\end{figure}
\begin{lemma}\label{prop:three}
  There is no set  of five points with convex hull 
  such that every graph shown in \figurename~\ref{fig:three} has a (plane
  straight-line) embedding on  where the vertices ,  and  are mapped
  to the points ,  and , respectively.
\end{lemma}
\begin{proof}
  The point  for the central vertex that is connected to all of  must
  be chosen so that (i) it is not in convex position with ,  and 
  and (ii) the number of points in the three resulting triangles is one in one
  triangle and zero in the other two. That requires three distinct choices for
  , but there are only two points available.\qed
\end{proof}
In fact, there are many such triples of graphs. The following lemma can be
verified with help of a computer program that exhaustively checks all order
types. Point set order types~\cite{gp-ms-83} are a combinatorial abstraction of
planar point sets that encode the orientation of all point triples, which in
particular determines whether or not any two line segments cross. For a small
number of points, there is a database with realizations of every (realizable)
order type~\cite{ak-psotd-01}.
\begin{lemma}\label{prop:seven}
  There is no set  of eight points with convex hull 
  such that every graph shown in \figurename~\ref{fig:seven} has a (plane
  straight-line) embedding on  where the vertices ,  and  are mapped
  to the points ,  and , respectively.
\end{lemma}
\begin{figure}[htbp]
  \hfil \subfloat[]{\includegraphics[scale=1.1]{8-7-1}}\hfil \subfloat[]{\includegraphics[scale=1.1]{8-7-7}}\hfil \subfloat[]{\includegraphics[scale=1.1]{8-7-4}}\hfil \subfloat[]{\includegraphics[scale=1.1]{8-7-5}}\hfil\\
  \hspace*{1pt}\hfil \subfloat[]{\includegraphics[scale=1.1]{8-7-2}}\hfil \subfloat[]{\includegraphics[scale=1.1]{8-7-6}}\hfil \subfloat[]{\includegraphics[scale=1.1]{8-7-3}}\hfil \subfloat[]{\includegraphics[scale=1.1]{bipyramid}\hfil}\\\caption{\label{fig:seven}{\small (a)--(g)}: Seven planar graphs, no three of
    which admit a simultaneous geometric embedding with a fixed mapping for the
    outer face; {\small (h)}: the skeleton  of a triangular bipyramid.}
\end{figure}
Denote by  the family of seven graphs on eight
vertices depicted in \figurename~\ref{fig:seven}. We consider these graphs as
abstract but \emph{rooted} graphs, that is, one face is designated as the outer
face and the counterclockwise order of the vertices along the outer face (the
\emph{orientation} of the face) is  in each case. Observe that all graphs
in  are planar 3-trees.

Using  we construct a family  of graphs as follows.
Start from the skeleton  of a triangular bipyramid, that is, a triangle and
two additional vertices, each of which is connected to all vertices of the
triangle. The graph  has five vertices and six faces and it is a planar
3-tree.

We obtain  from  by planting one of the graphs from
 onto each of the six faces of . Each face of  is a
(combinatorial) triangle where one vertex has degree three (one of the pyramid
tips) and the other two vertices have degree four (the vertices of the starting
triangle). On each face  of  a selected graph  from  is
planted by identifying the three vertices bounding  with the three vertices
bounding the outer face of  in such a way that vertex  (which appears at
the top in \figurename~\ref{fig:seven}) is mapped to the vertex of degree three
(in ) of . In the next paragraph, we will see why we do not have to
specify how  and  are matched to . The family  consists of
all graphs on  vertices that can be obtained in this way. By
construction all these graphs are planar 3-trees. Therefore by
Lemma~\ref{lem:stacked_unique} on any given set of  points, the plane
straight-line embedding is unique (if it exists), once the mapping for the outer
face is determined.

Observe that  is \emph{flip-symmetric} with respect to horizontal
reflection. In other (more combinatorial) words, for every  we
can exchange the role of the bottom two vertices  and  of the outer face
(and thereby also its orientation) to obtain a graph that is also in
. The graphs form symmetric pairs of siblings ,
, , and  flips to itself. Therefore, regardless of
the orientation in which we plant a graph from  onto a face of ,
we obtain a graph in , and so  is well-defined.

Next, we give a lower bound on the number of nonisomorphic graphs in
.
\begin{lemma}
  The family  contains at least  pairwise nonisomorphic
  graphs.
\end{lemma}
\begin{proof}
  Consider the bipyramid  as a face-labeled object. There are  different
  ways to assign a graph from  to each of the six now
  distinguishable faces. Denote this class of face-labeled graphs by
  . For many of these assignments the corresponding graphs are
  isomorphic if considered as abstract (unlabeled) graphs. However, the
  following argument shows that every isomorphism between two such graphs maps
  the vertex set of  to itself.

  The two tips of  have degree three and are incident to three faces. Onto
  each of the faces one graph from  is planted, which increases the
  degree by four (for ) or three (for ) to a total of at
  least twelve. The three triangle vertices start with degree four and are
  incident to four faces. Every graph from  planted there adds at
  least one more edge, to a total degree of at least eight. But the highest
  degree among the interior vertices of the graphs in  is seven,
  which proves the claim.

  Hence we have to look for isomorphisms only among the symmetries of the
  bipyramid . The tips are distinguishable from the triangle vertices,
  because the former are incident to three high degree vertices, whereas the
  latter are incident to four high degree vertices. Selecting the mapping for
  one face of  determines the whole isomorphism. Since there are at most two
  ways to map a face to a face (we can select the mapping for the two non-tip
  vertices, that is, the orientation of the triangle), every graph in
   is isomorphic to at most  graphs from
  . It follows that there are at least  pairwise
  nonisomorphic graphs in .\qed
\end{proof}
We now give an upper bound on the number of graphs of  that can be
simultaneously embedded on a common point set.
\begin{lemma}
  At most  pairwise nonisomorphic graphs of  admit a
  simultaneous (plane straight-line) embedding without mapping.
\end{lemma}
\begin{proof}
  Consider a subset  of pairwise
  nonisomorphic graphs and a point set  that admits a simultaneous embedding
  of . Since  is a class of maximal planar graphs,
  the convex hull of  must be a triangle. For each  we can
  select an outer face  and a mapping  for the vertices bounding
   to the convex hull of  so that the resulting straight-line
  embedding, which by Lemma~\ref{lem:stacked_unique} is completely determined by
   and , is plane.

  Let us group the graphs from  into bins, according to the maps
   and . For , there are  possible choices: one of the
  eleven faces of one of the seven graphs in . For  there are
  three choices: one of the three possible rotations to map the face chosen by
   to the convex hull of . Note that regarding  there is no
  additional factor of two for the orientation of the face, because by
  flip-symmetry such a change corresponds to a different graph (for
  ) or a different face of the graph (for ), that is, a
  different choice for . Altogether this yields a partition of 
  into  bins.

  The crucial observation (and ultimate reason for this subdivision) is that for
  all graphs in a single bin the vertices of  (the bipyramid) are mapped to
  the same points. This is a consequence of the uniqueness of the embedding up
  to the mapping for the outer face (Lemma~\ref{lem:stacked_unique}), which is
  identical for all graphs in the same bin. Therefore, the triangle  of 
  in which the outer face is located is mapped to the same oriented triple of
  points in  for all graphs in the same bin. From there the pattern repeats,
  noting that every face of  contains the same number of points (five) and
  that the polyhedron  is face-transitive so that there is no difference as
  to which face of  was selected to contain the outer face.

  It follows that for all graphs in the same bin the graphs from 
  planted onto the faces of  are mapped to the same point sets. Any two
  (nonisomorphic) graphs from  differ in at least one of those
  faces -- and by definition not in the one in which the outer face was selected
  by . In order for the graphs in a bin to be simultaneously embeddable on
  , by Lemma~\ref{prop:seven} there are at most two different graphs from
   mapped to any of the remaining five faces of . Therefore
  there cannot be more than  graphs from  in any bin.
  Hence , as claimed.\qed
\end{proof}
Since there are strictly more nonisomorphic graphs in  than can
possibly be simultaneously embedded, not all graphs of  admit a
simultaneous embedding.  In particular, any subset of  nonisomorphic
graphs in  is a collection that does not have a simultaneous
embedding. This proves our Theorem~\ref{thm:sim}.

\section{Small -universal point sets}
\label{sec:small}

As we have seen in the previous sections, there are no -universal point sets
of size  for . In this section, we consider the case .
Specifically, we used a computer program~\cite{universalsrc} to show the
following:
\begin{theorem}
  \label{thm:at_most_10}
  There exist -universal point sets of size  for all .
\end{theorem}
We use a straightforward brute-force approach. The two main ingredients are the
aforementioned order type database~\cite{ak-psotd-01} with point sets of size
 and the \emph{plantri} program for generating maximal planar
graphs~\cite{brinkmann2007fast,plantrisrc}. To determine if a point set  of
size  is -universal, our program tests if for all maximal planar graphs
 on  vertices, there exists a bijection  such
that straight-line drawing of  induced by  is plane. If such a
bijection exists for all , then  is universal. Otherwise, there is a graph
 that has no plane straight-line embedding on . Note that it is sufficient
to consider maximal planar graphs since adding edges only makes the embedding
problem more difficult. Work on the case  is still in progress at the time
of writing. For  the approach unfortunately becomes infeasible; it is
unknown whether or not there exist -universal point sets of size  for
. Table~\ref{tab:ups_numbers} gives an overview of the results
of this paper and \figurename~\ref{fig:universal_5_10} shows one universal point
set for each .

\begin{table}
  \centering
  \setlength{\tabcolsep}{2.5pt}
  \begin{tabular}{@{}r*{15}{c}@{}}
    \toprule : &  &  &  &  &  &  &  &  &  &  &  &  &
     &  & \\
    \# universal point sets: &  &  &  &  &  &  &  &  &  &  &  &
     &  &  & \\
    \bottomrule\\
  \end{tabular}
  \caption{The number of (non-equivalent) -universal point sets of size .}
  \label{tab:ups_numbers}
\end{table}

\begin{figure}[htbp]
  \centering \newcommand{\thisfigwidth}{0.3\textwidth}\includegraphics[width=\thisfigwidth]{universal_05_0001}\hfil \includegraphics[width=\thisfigwidth]{universal_06_0004}\hfil \includegraphics[width=\thisfigwidth]{universal_07_0016}\\baselineskip]
  \includegraphics[width=\thisfigwidth]{universal_08_0074}\hfil \includegraphics[width=\thisfigwidth]{universal_09_0072}\hfil \includegraphics[width=\thisfigwidth]{universal_10_0018}
  \caption{One universal point set for each . Each pair of points
    is connected with a line segment.}
  \label{fig:universal_5_10}
\end{figure}

\newpage\bibliographystyle{plain}
\bibliography{universal}


\end{document}
