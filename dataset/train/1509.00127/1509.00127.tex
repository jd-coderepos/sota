\documentclass[twocolumn]{article}

\usepackage{framed}
\usepackage{hyperref}


\newtheorem{theorem}{Theorem}
\newtheorem{definition}{Definition}
\newtheorem{assumption}{Assumption}
\newtheorem{corollary}{Corollary}

\newcommand{\comment}[1]{}

\newcommand{\floor}[1]{\lfloor #1 \rfloor}
\newcommand{\ceil}[1]{\lceil #1 \rceil}

\def\denseformat{
\setlength{\textheight}{9.2in}
\setlength{\textwidth}{7.1in}
\setlength{\evensidemargin}{-0.2in}
\setlength{\oddsidemargin}{-0.2in}
\setlength{\headsep}{10pt}
\setlength{\topmargin}{-0.3in}
\setlength{\columnsep}{0.375in}
\setlength{\itemsep}{0pt}
\renewcommand{\baselinestretch}{0.99}
}
\def\midformat{
\setlength{\textheight}{8.9in}
\setlength{\textwidth}{6.7in}
\setlength{\evensidemargin}{-0.19in}
\setlength{\oddsidemargin}{-0.19in}
\setlength{\headheight}{0in}
\setlength{\headsep}{10pt}
\setlength{\topsep}{0in}
\setlength{\topmargin}{0.0in}
\setlength{\itemsep}{0in}       \renewcommand{\baselinestretch}{1.1}
\parskip=0.070in
}
\def\spacyformat{
\setlength{\textheight}{8.8in}
\setlength{\textwidth}{6.5in}
\setlength{\evensidemargin}{-0.18in}
\setlength{\oddsidemargin}{-0.18in}
\setlength{\headheight}{0in}
\setlength{\headsep}{10pt}
\setlength{\topsep}{0in}
\setlength{\topmargin}{0.0in}
\setlength{\itemsep}{0in}      \renewcommand{\baselinestretch}{1.2}
\parskip=0.080in
}

\denseformat

\begin{document}
\bibliographystyle{plain}

\title{{\sc DiffSum} -- A Simple Post-Election Risk-Limiting Audit}
\author{Ronald L. Rivest\\
  MIT CSAIL, {\texttt{rivest@mit.edu}}}
\date{\today}
\maketitle

\thispagestyle{empty}


We present {\sc DiffSum}, a simple risk-limiting post-election ballot-polling audit.
See~\cite{NBHC07,LindemanStYa12,LindemanSt12}
for background.

You wish to check that candidate A really won a plurality election against candidate B.
You may sample the  cast paper ballots without replacement.

\begin{leftbar}
\noindent {\bf Procedure {\sc DiffSum:}}
\begin{enumerate}

\item {\bf [Choose ]} Let  be the number of decimal digits in , and
  choose  where  controls the error rate (the chance of the audit accepting
  an incorrect outcome):
\begin{tabular} {c|c|c|c|c|c}
                &  0   &  1   &  2   &  3   & 4 \\ \hline
      \hbox{max error rate} & 22\% & 15\% & 10\% &  6\% & 4\% 
    \end{tabular}


  \item {\bf [Begin]} Draw an initial sample of  ballots.

  \item {\bf [Tally]} Determine the number  of votes for  in your sample, and the number  of
    votes for .

  \item {\bf [Stop?]} Stop the audit (accept A as winner) if  and 
    

  \item {\bf [Continue?]} 
    If , stop (you have just completed a full recount).
    Otherwise, enlarge your random sample and return to step~3.
\end{enumerate}
\vspace{-0.15in}
\end{leftbar}

\noindent{\bf Remarks:} 
The initial size 24 of the sample in step 2 is arbitrary.
In step 5 the increase in sample size is also arbitrary; it could be by a
single ballot.

The name ``{\sc DiffSum}'' was chosen because
(\ref{eq:diffsum1}) says


\noindent{\bf Efficiency:} Let  be the true margin (the
fraction of votes cast for A minus the fraction cast for B).
In a sample of size ,
the expected value of  is .  Thus, {\sc DiffSum} is expected to
stop when

or


{\sc DiffSum} is approximately as efficient as {\sc Bravo}---compare
(\ref{eq:s-bound}) with the estimate  for {\sc
  Bravo}~\cite{LindemanStYa12} (here  is the risk limit).
Moreover, {\sc DiffSum} does not need an initial estimate of the vote
shares, and {\sc Bravo} is inefficient when this estimate is
inaccurate.

\noindent{\bf Error rate:}
The error rate bounds given in Step 1 are based on
extensive simulations for 
 to , 
 to ,
, and . 
We measured the error rate over 10,000 simulated elections
in each case.
Each simulation estimated the error rate when the election was
a tie, a worst-case scenario;
with more realistic margins the error
rate drops dramatically, so that in practice
even  should give very reliable audits.

\smallskip
\noindent{\bf Example:} 
An election with  votes can be audited using  for 
a risk limit
of .
For , {\sc DiffSum} examines about 175 ballots (estimated),
{\sc Bravo} (with ) examines about 115 (estimated).
In simulations for this election,
{\sc DiffSum} with 
examines about 157 ballots on average, and has an error rate of less than 0.04\%;  
{\sc DiffSum} with 
examines about 112 ballots on average, and has an error rate of less than 0.2\%.
Bravo examines about 119 ballots on average, and has an error rate of approximately 2.5\%.

\smallskip
\noindent{\bf Extension:}
In practice, one should cease random sampling once a significant
number (say 4\%) of the ballots have been sampled, when switching over
to a full hand recount becomes more economical.

With more candidates, let {\sc DiffSum} check that the
sample winner beats the sample's strongest loser.

\smallskip
\noindent {\bf Conclusion:}
{\sc DiffSum} is exceptionally simple, and 
appears quite comparable to {\sc Bravo}
in terms of efficiency and error rate.  Further simulations and
analysis would be helpful.

\smallskip
\noindent {\bf Acknowledgment:} I thank Philip Stark for helpful comments.

\small
\vspace*{-0.2in}

\bibliography{audit}

\end{document}
