In this section we use a general combinatorial approach for solving prize-collecting problems introduced by Hajiaghayi and Jain \cite{Hajiaghayi}. In their work they obtained the primal-dual $3$-approximation algorithm for edge-weighted Prize-collecting Steiner Forest problem. We repeat their argumentation in planar node-weighted setting resulting in the $4$-approximation algorithm. 

Consider a graph $G=(V,E)$ with a non-negative cost function on nodes $w : V \rightarrow Q_+$, a set of pairs of vertices (demands) $D = {(s_1,t_1), (s_2,t_2),\dots, (s_k,t_k)}$ and a non-negative penalty function $\pi : D \rightarrow Q_+$. In the Node-weighted Prize-collecting Steiner Forest problem we are asked to find a set of vertices $F\subseteq V$ which minimizes the sum of costs of vertices in $F$ plus penalties for pairs of vertices which are not connected in a subgraph of $G$ induced by $F$.

Note that we can give an equivalent definition of demands and penalties by specifying penalties for each unordered pair of vertices. Simply set penalties for pairs of vertices which are not in $D$ to $0$. From now on we will use values $\pi_{ij}$ to denote penalties. Let also $\Gamma(S)$ denote the set of vertices in $V-S$ incident to vertices from $S \subseteq V$ and let $S \odot (i,j)$ means that $|(i,j)\cap S| = 1$ (i.e., $S$ separates vertices $i$ and $j$)
Using this notation, we can formulate our problem with the following integer program
\begin{align*}
\text{min} & \sum_{v\in V} w_v x_v + \sum_{(i,j)\in V\times V}\pi_{ij}z_{ij} & (IP_{SF})\\
s.t.\\
& \sum_{v\in \Gamma(S)} x_v + z_{i,j} \geq  1			& \forall S \subseteq V, \hspace{3mm} \forall (i,j) \in V \times V : \hspace{3mm} S \odot (i,j) \\
& x_v \in \{0,1\}										& \forall v \in V\\
& z_{i,j} \in \{0,1\}									& \forall (i,j) \in S \times S
\end{align*}
Setting $x_v = 1$ corresponds to buying a vertex $v$ (including $v$ into solution $F$) and setting $z_{i,j} = 1$ corresponds to paying a penalty instead of connecting vertices $i$ and $j$.

The dual of the linear relaxation of this program is:
\begin{align*}
\text{max} & \sum_{S \subseteq V, S\odot (i,j)} y_{S_{ij}} & (DLP_{SF} 1)\\
s.t.\\
& \sum_{S:v\in\Gamma(S), S\odot(i,j)} y_{S_{ij}} \leq w_v	& \forall v \in V\\
& \sum_{S:S\odot(i,j)} y_{S_{ij}} \leq \pi_{i,j}			& \forall (i,j) \in V \times V\\
& y_{S_{ij}} \geq 0											& \forall S \subseteq V, S \odot (i,j)\\
\end{align*}
The problem with this dual program is that it has many different variables for each pair of vertices. Hence in a moat growing approach we have to decide how to split the growth of a moat corresponding to a set $S$ between variables $y_{S_{ij}}$. It seems to be a difficult task (see \cite{Hajiaghayi} for a detailed discussion) and may require decreasing some dual variables throughout the course of the algorithm.

Fortunately Hajiaghayi and Jain in \cite{Hajiaghayi} proposed a general approach of handling this issue of different variables induced by prize-collecting setting by circumventing it using Farkas' Lemma. The following arguments are repetitions of their work in the node-weighted setting and we conduct them for the sake of the completeness.

First, we create new variables $y_S = \sum_{(i,j): S\odot(i,j)} y_{S_{i,j}}$. Now the dual $DLP_{SF} 1$ becomes
\begin{align*}
\text{max} & \sum_{S \subseteq V} y_S & (DLP_{SF} 2)\\
s.t.\\
& y_S \leq \sum_{(i,j): S\odot(i,j)} y_{S_{i,j}}			& \forall S \subseteq V\\
& \sum_{S:v\in\Gamma(S)} y_S \leq w_v						& \forall v \in V\\
& \sum_{S:S\odot(i,j)} y_{S_{ij}} \leq \pi_{i,j}			& \forall (i,j) \in V \times V\\
& y_{S_{ij}} \geq 0											& \forall S \subseteq V, S \odot (i,j)\\
& y_S \geq 0												& \forall S \subseteq V\\
\end{align*}
\begin{fact}
Linear programs $DLP_{SF} 1$ and $DLP_{SF} 2$ are equivalent
\end{fact}
\begin{proof}
Take a feasible solution $y$ to $DLP_{SF} 1$. Let $y_S = \sum_{(i,j): S\odot(i,j)} y_{S_{i,j}}$. This together with $y$ constitutes a feasible solution to $DLP_{SF} 2$ of the same cost.\\
To see the other direction, take a feasible solution $y$ to $DLP_{SF} 2$. We can assume that the first constraint in $DLP_{SF} 2$ is tight, because otherwise we could decrease $y_{S{i,j}}$ until $y_S = \sum_{(i,j): S\odot(i,j)} y_{S_{i,j}}$ without affecting the objective function and not violating other constraints. The $y_{S{i,j}}$ is feasible to $DLP_{SF} 1$ and the objective function is the same. \qed
\end{proof}

Now we will use Farkas' Lemma to get a rid of dual variables $y_{S_{i,j}}$. Observe that they are not included in the objective function of $DLP_{SF} 2$. The idea is to replace constraints involving $y_{S_{i,j}}$ with different inequalities which for fixed $y_S$ check whether feasible $y_{S_{i,j}}$ exists.

\begin{fact} Farkas' lemma (variant)\\
	\label{FarkasLemma}
	Consider a matrix $A \in R^{m \times n}$ and a vector $b \in R^m$. The system $Ax\leq b$ has a solution $x \geq 0$, if and only if for all $y\geq 0$ with $yA \geq 0$ one has $yb \geq 0$.
\end{fact}

Consider a feasible solution to $DLP_{SF} 2$ and a system defined by constraints of $DLP_{SF} 2$ containing $y_{S_{i,j}}$, i.e.
\begin{align*}
& \sum_{(i,j): S\odot(i,j)} y_{S_{i,j}} \geq y_S			& \forall S \subseteq V\\
& -\sum_{S:S\odot(i,j)} y_{S_{ij}} \geq -\pi_{i,j}			& \forall (i,j) \in V \times V\\
\end{align*}
Farkas Lemma (see Fact~\ref{FarkasLemma}) says that this system has a solution $y_{S_{ij}} \geq 0$ if and only if for each vector $[\alpha | \beta] \geq 0$ with $\alpha_S - \beta_{i,j} \geq 0$ (for each $S, i,j$ such that $S\odot(i,j)$) we have that $\sum_{S\subseteq V} \alpha_S - \sum_{i,j} \beta_{i,j} \pi_{i,j} \leq 0$.
Notice that we can safely replace $\beta_{i,j}$ with $\max_{S: S\odot(i,j)}{\alpha_S}$ which gives us the following constraint:
\begin{align*}
& \sum_{S\subseteq V} \alpha_S \cdot y_S \leq \sum_{i,j} \max_{S: S\odot(i,j)}{\alpha_S} \cdot \pi_{i,j} & \text{for each} \hspace{3mm}\alpha: 2^V \rightarrow \mathbb{R}^+\\
\end{align*}
So our new dual is
\begin{align*}
\text{max} & \sum_{S \subseteq V} y_S & (DLP_{SF} 3)\\
s.t.\\
& \sum_{S:v\in\Gamma(S)} y_S \leq w_v						& \forall v \in V\\
& \sum_{S\subseteq V} \alpha_S \cdot y_S \leq \sum_{i,j} \max_{S: S\odot(i,j)}{\alpha_S} \cdot \pi_{i,j} & \text{for each} \hspace{3mm}\alpha: 2^V \rightarrow \mathbb{R}^+\\
& y_S \geq 0												& \forall S \subseteq V\\
\end{align*}
and it has only one dual variable $y_S$ for each set $S$. On the other hand, it has infinitely many constraints. However, as the lemma below says, many of them are redundant.
\begin{lemma} (Lemma 2.2 in \cite{Hajiaghayi})
It is sufficient to consider $\alpha's$ having only one positive value in its range.
\end{lemma}

The above lemma allows us to think about $\alpha$'s as families of subsets of $V$. Hence we can write our dual as follows
\begin{align}
\text{max} & \sum_{S \subset V} y_S & (DLP_{SF} 4) \nonumber\\
s.t. \nonumber\\
& \sum_{S:v\in\Gamma(S)} y_S \leq w_v																& \forall v \in V \label{ForestConstraint1}\\
& \sum_{S\in \mathbb{S}} y_S \leq \sum_{(i,j) \in V \times V, \mathbb{S}\odot(i,j)} \pi_{i,j}		& \forall \mathbb{S} \in 2^{2^V}\label{ForestConstraint2}\\
& y_S \geq 0																						& \forall S \subset V \nonumber
\end{align}
where $\mathbb{S}\odot(i,j)$ denotes that there exists $S \in \mathbb{S}$ such that $S \odot (i,j)$ (we say that family $\mathbb{S}$ separates vertices $i$ and $j$ if and only if there exists at least one set $S \in \mathbb{S}$ which separates vertices $i$ and $j$).

Note that $\mathbb{S}$ is a family of subsets of vertices and our dual has double exponential number of constraints. But we have now only one dual variable for each set. Intuitively this double exponential number of constraints implicitly ensures that for given variables $y_S$ there exist feasible variables $y_{S_{ij}}$ of the former dual program which sum to $y_S$.

Although the double exponential number of constraints does not sound good, we will be able to construct polynomial-time primal-dual algorithm based on this dual.

We can define a function $f : 2^{2^V} \rightarrow \mathbb{R}_+$ which for every family $\mathbb{S}$ define $f(\mathbb{S})$ to be the right-hand side of the corresponding constraint, i.e.:
$$f(\mathbb{S}) = \sum_{(i,j) \in V \times V, \mathbb{S}\odot(i,j)} \pi_{i,j}$$
In their paper, Hajiaghayi and Jain show that $f$ is submodular. This property allows them to prove the following fact.
\begin{fact} (Corollary 2.2 in \cite{Hajiaghayi})
\label{SumTight}
Suppose $y$ is a feasible solution to dual $DLP_{SF} 4$. Suppose the constraints corresponding to families $\mathbb{S}_1$ and $\mathbb{S}_2$ are tight. Then the constraint corresponding to the family $\mathbb{S}_1 \cup \mathbb{S}_2$ is also tight.
\end{fact}
\newpage
\subsection{Algorithm}
Without loss of generality we can assume that each terminal $v$ belongs to exactly one demand and its weight $w_v$ is $0$. To see this, construct a new graph where for each vertex $v_i, v_j$ of each demand $(i,j) \in D$ we have additional two vertices $v_i^{ij}$ and $v_j^{ij}$ connected by a single edge to original vertices ($v_i$ and $v_j$ correspondingly). The penalties are now only between vertices $v_i^{ij}$ and $v_j^{ij}$. Weights of new vertices are now $0$ while weights of original vertices $v_i, v_j$ remain the same. It is easy to see that every solution for the new graph can be used to construct a solution of the same cost for the original graph and vice-versa.

Now we are ready to give a primal-dual algorithm for the NWPCSF problem on planar graphs.
The algorithm starts with an initial solution $F$ in which there are all vertices of cost $0$ (hence all terminals). In each iteration the algorithm maintains moats which are the connected components of graph $G$ induced by the vertices of the current solution $F$. Demands can be marked (meaning that we decide to pay a penalty for them) or unmarked. At the beginning all demands are unmarked. Once demand is marked, it stays marked forever. A moat (denoted by the corresponding set $S\subseteq V$) is active in the current iteration if and only if there is at least one unmarked demand $(i,j)$ such that $S \odot (i,j)$. Now in each iteration we simultaneously grow each active moat until one of the following two events occur:
\begin{itemize}
	\item a vertex $v$ goes tight (constraint (\ref{ForestConstraint1}) becomes equality), or
	\item a family $\mathbb{S}$ goes tight (constraint (\ref{ForestConstraint2}) becomes equality).
\end{itemize}

In the first case we simply add $v$ to our solution $F$ (which may make some moats inactive) and continue to the next iteration.

In the second case, we mark each demand $(i,j)$ such that $\mathbb{S} \odot (i,j)$. Hence in the following iterations all moats from $\mathbb{S}$ will be inactive, and we will not violate any constraint during the growth process. We repeat this process until all moats become inactive.

After that we have an additional pruning phase in which we process all vertices of $F$ in the reverse order of buying. We remove a vertex $v$ from $F$ if after its removal from $F$, all unmarked demands are still connected in the graph induced by $F$. We output this pruned set of vertices as $F'$ which is our final solution.

Obtaining $\epsilon_1$ and a tight vertex in line $7$ is straightforward. On the other hand obtaining $\epsilon_2$ in line $8$ and a tight family $\mathbb{S}$ seems to be much harder, since the number of corresponding constraint is double exponential. Fortunately Hajiaghayi and Jain in section $4$ of \cite{Hajiaghayi} gave a polynomial time algorithm for computing $\epsilon_2$ and the corresponding tight family $\mathbb{S}$.

Since the algorithm terminates after at most $2|V|-1$ iterations (in each iteration the number of active moats or the number of connected components decreases), the running time of this algorithm is polynomial.

\begin{algorithm}[H]
	\SetKwData{Left}{left}\SetKwData{This}{this}\SetKwData{Up}{up}
	\SetKwInOut{Input}{input}\SetKwInOut{Output}{output}
	\Input{A planar graph $G=(V,E)$ with non-negative weights $w_i$ on the nodes and non-negative penalties $\pi_{ij}$ between each pair of vertices such that if $\pi_{ij} > 0$ then $w_i=0$ and $w_j=0$}
	\Output{A set of vertices $F'$ representing a forest and a set of pairs $Q'$ representing not connected demands}
	\BlankLine
	\Begin{
		$F\leftarrow \{v_i\in V: w_i = 0\}$\;
		$Q\leftarrow \emptyset$ \tcp{set all demands unmarked}
		$y_S\leftarrow 0$ \tcp{implicitly}
		\BlankLine
		\BlankLine
		$AM \leftarrow \left\{S\subseteq V : S \in SCC\left(G[F]\right) \wedge \displaystyle\mathop{\exists}_{(i,j) \in V\times V - Q } \pi_{ij}>0 \wedge S \odot (i,j) \right\}$\;
		\tcp{identify active moats as components of subgraph of $G$ induced by vertices $F$ for which there is at least one unmarked demand $(i,j)$ which is separated by the corresponding set}
		\BlankLine
		\BlankLine
		\While{AM $ \neq \emptyset$}{
			find minimum $\epsilon_1$ s.t if we increase $y_S$ for each $S\in AM$ by $\epsilon_1$ we get a new tight vertex $v$\;
			find minimum $\epsilon_2$ s.t if we increase $y_S$ for each $S\in AM$ by $\epsilon_2$ we get a new tight family $\mathbb{S}$\;
			$\epsilon \leftarrow min(\epsilon_1,\epsilon_2)$\;
			$y_S \leftarrow y_S + \epsilon$ for all $S\in AM$\;
			\eIf{$\epsilon = \epsilon_1$}{
				$F \leftarrow F \cup \{v\}$\;
			}{
				$Q \leftarrow Q \cup \{(i,j) \in V\times V : \mathbb{S} \odot (i,j)\}$
			}
			$AM \leftarrow \left\{S\subseteq V : S \in SCC\left(G[F]\right) \wedge \displaystyle\mathop{\exists}_{(i,j) \in V\times V - Q } \pi_{ij}>0 \wedge S \odot (i,j) \right\}$\;
		}
		\tcp{pruning phase}
		Derive $F'$ from $F$ by removing vertices in reverse order of purchase so that every unmarked demand is connected in $F'$.\\
		Let $Q'$ be all demands not connected via $F'$
	}
	\caption{Primal-Dual Algorithm for NWPCSF on planar graphs}
\end{algorithm}


\subsection{Analysis}
\begin{theorem}
\label{4aprox}
The algorithm outputs a set of vertices $F'$ and a set of demands $Q'$ which are not connected via $F'$ such that
$$\sum_{v\in F'}w_v + \sum_{(i,j)\in Q'}\pi_{ij} \leq 4\sum_{S\subseteq V}y_S \leq 4\ OPT$$
\end{theorem}
In order to prove Theorem~\ref{4aprox} it is enough to prove the following two lemmas:
\begin{lemma}
\label{pc1lemma}
$$\sum_{(i,j)\in Q'}\pi_{ij} \leq \sum_{S\subseteq V}y_S$$
\end{lemma}
\begin{proof}
First observe that $Q' \subseteq Q$ where $Q$ are marked pairs. Now consider families $\mathbb{S}_1,\dots,\mathbb{S}_f$ which went tight during the run of the algorithm. Observe that each marked pair was separated by some $\mathbb{S}_i$.\\
Hence family $\mathbb{S}_{all} = \displaystyle\mathop{\cup}_{j=1}^{f}\mathbb{S}_j$ separates each marked pair.
From Fact~\ref{SumTight} the union of tight families is tight. Putting it all together gives:\\
$$\sum_{(i,j)\in Q'}\pi_{ij} \leq \sum_{(i,j)\in Q}\pi_{ij} \leq \sum_{\mathbb{S}_{all}\odot (i,j)}\pi_{ij} = \sum_{S\in\mathbb{S}}y_S \leq \sum_{S\subseteq V}y_S$$ \qed
\end{proof}

\begin{lemma}
\label{general3lemma}
$$\sum_{v\in F'}w_v \leq 3\sum_{S\subseteq V}y_S$$
\end{lemma}
To prove Lemma~\ref{general3lemma} we will use an auxiliary lemma. But first, let us introduce one definition.\\
For a set of nodes $F$ and the set of unmarked demands $R=D-Q$ define a minimal feasible augmentation $F_{aug}$ of $F$ with respect to $R$ to be a set of vertices $F_{aug}$ containing $F$ as a subset such that every pair of vertices from $R$ is connected in the subgraph of $G$ induced by $F_{aug}$ and such that removal of any $v\in F_{aug} - F$ from $F_{aug}$ disconnects some pair from $R$.

\begin{lemma}
\label{Carsten3lemma}
Let $G$ be planar, $R$ be the set of unmarked demands after running the above algorithm, $F_j$ be the set of bought vertices before running iteration $j$ and $F_{aug}$ be a minimal feasible augmentation of $F_j$ with respect to $R$. Let also $A_j$ be the set of active moats before running iteration $j$. Then\\
$$\sum_{S\in{A_j}}|F_{aug} \cap \Gamma(S)| \leq 3|A_j|$$
\end{lemma}

Before we prove Lemma~\ref{Carsten3lemma} we will show how it helps us in proving Lemma~\ref{general3lemma}.\\
\begin{proof}[Proof of Lemma~\ref{general3lemma}]\\
Since we add a vertex $v$ to $F$ only if it is tight, and after that we do not modify variables corresponding to sets adjacent to $v$, we have the following equality
\begin{align*}
\sum_{v\in F'}w_v & = \sum_{v\in F'} \sum_{S:v\in\Gamma(S)}y_S = \sum_{S\subseteq V}|F' \cap \Gamma(S)|y_S
\end{align*}
(in the last step we changed the order of the summation).

Now let $F'$ be the output of the algorithm, $R=D-Q$ be the set of all unmarked demands, $F_j$ be the set of bought vertices before iteration $j$, $A_j$ be the set of active moats before running iteration $j$ and $\epsilon_j$ be the increase of the dual variables in iteration $j$. Then for each $S\subseteq V$ we have $y_S = \sum_{j:S \in A_j} \epsilon_j$ hence the following holds:
$$\sum_{S\subseteq V} y_S = \sum_j |A_j|\epsilon_j$$
and
\begin{align*}
\sum_{v\in F'}w_v & = \sum_{S\subseteq V}|F' \cap \Gamma(S)|y_S  = \sum_{S\subseteq V}|F' \cap \Gamma(S)|\sum_{j : S\in A_j}\epsilon_j\\
	& = \sum_j\left(\sum_{S\in A_j}|F' \cap \Gamma(S)|\right)\epsilon_j
\end{align*}

Observe now, that $F_{aug} = F_j \cup F'$ is a minimal feasible augmentation of $F_j$ with respect to $R$. Obviously, every demand from $R$ is connected in $F_{aug}$. Consider any vertex $v \in F' - F_j$. Removing $v$ from $F_{aug}$ will make some pair from $R$ disconnected because otherwise $v$ would be deleted in the pruning phase. Hence we can use Lemma~\ref{Carsten3lemma}:
$$\sum_{S\in{A_j}}|(F_j \cup F') \cap \Gamma(S)| \leq 3|A_j|$$
Since $|F' \cap \Gamma(S)| \leq |(F_j \cup F') \cap \Gamma(S)|$ we have
\begin{align*}
\sum_{v\in F'}w_v & = \sum_j\left(\sum_{S\in A_j}|F' \cap \Gamma(S)|\right)\epsilon_j\\
	& \leq \sum_j3|A_j|\epsilon_j = 3\sum_{S\subseteq V}y_S
\end{align*} \qed
\end{proof}

Now we conclude with the sketch of the proof of Lemma~\ref{Carsten3lemma}.
\begin{proof}[of Lemma~\ref{Carsten3lemma}]
The proof is conducted in a similar way as the proof of Lemma~\ref{lemMainCC} and the analysis is essentially the same as in \cite{Moldenhauer}.
We need to count the adjacencies between active moats and vertices from $F_{aug} - F_j$. Consider the graph $G'$ obtained from $G$ in the following way:
\begin{enumerate}
	\item take the subgraph of $G$ induced by vertices from $F_{aug}$
	\item discard isolated inactive moats
	\item contract each inactive moat with a neighboring vertex 
	\item contract each active moat
\end{enumerate}
Next color vertices of $G'$ in two colors:
\begin{itemize}
	\item white color for vertices obtained from contracting active moats
	\item black color for all other vertices, i.e. $F_{aug} - F_j$
\end{itemize}
Observe now that deleting a black vertex in $G'$ disconnects two white vertices, because otherwise it would be deleted in the pruning phase. $G'$ remains planar, since deletions and contractions preserve planarity. Moreover, it is easy to see that the number of adjacencies $\sum_{S\in{A_j}}|(F_{aug}) \cap \Gamma(S)|$ in $G$ is the same as the number of edges between white and black vertices in $G'$. To bound this number we use Lemma~\ref{lemma-graph-white-black} for each component of $G'$. Therefore we have $$\sum_{S\in{A_j}}|(F_{aug}) \cap \Gamma(S)| \leq 3|A_j|$$ \qed
\end{proof}
