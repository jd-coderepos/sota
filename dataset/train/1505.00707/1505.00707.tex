\documentclass[preprint,12pt]{elsarticle}

\usepackage{amssymb}
\usepackage{gastex}
\usepackage{amsmath}
\usepackage{upref}
\usepackage{multirow}
\usepackage{multicol}
\usepackage{url}
\usepackage[all]{xy}
\usepackage{vaucanson-g}
\usepackage{color}
\usepackage[utf8]{inputenc}
\usepackage{makeidx}
\usepackage{theorem}

\newtheorem{theorem}{Theorem}
\newtheorem{lemma}[theorem]{Lemma}
\newtheorem{proposition}[theorem]{Proposition}
\newtheorem{corollary}[theorem]{Corollary}
{\theorembodyfont{\rmfamily}\newtheorem{example}[theorem]{Example}
  \newtheorem{remark}[theorem]{Remark} }
\newenvironment{proof}{\noindent\textit{Proof.}}
{\QED\vskip\theorempostskipamount} 
\newenvironment{proofof}[1]{\noindent\textit{Proof
    \protect{#1}.}}
                       {\QED\vskip\theorempostskipamount}
\def\petitcarre{\vrule height4pt width 4pt depth0pt}
\def\QED{\relax\ifmmode\eqno{\hbox{\petitcarre}}\else{\unskip\nobreak\hfil\penalty50\hskip2em\hbox{}\nobreak\hfil
  \petitcarre
  \parfillskip=0pt \finalhyphendemerits=0\par\smallskip}
  \fi}

\newcommand\alp{\mathop{\rm alph}\nolimits}
\newcommand\A{\mathcal{A}}
\newcommand\B{\mathcal{B}}
\newcommand\E{\mathcal{E}}
\newcommand\T{\mathcal{T}}
\newcommand\C{\mathcal{C}}
\newcommand\D{\mathcal{D}}
\newcommand\F{\mathcal{F}}
\newcommand\I{\mathcal{I}}
\newcommand\G{\mathcal{G}}
\newcommand\K{\mathcal{K}}

\newcommand\LL{\mathcal{L}}
\newcommand\RR{\mathcal{R}}
\newcommand\CR{\mathcal{CR}}
\newcommand\MR{\mathcal{MR}}

\newcommand{\N}{\mathbb{N}}
\newcommand{\Z}{\mathbb{Z}}
\newcommand{\R}{\mathbb{R}}

\let\pars\pi  
\let\proba\delta
\let\edge\xrightarrow
\def\un(#1){\underline{#1}\,}
\DeclareMathOperator{\Card}{Card}
\DeclareMathOperator{\rep}{rep}
\DeclareMathOperator{\lcp}{lcp}
\DeclareMathOperator{\Div}{Div}
\DeclareMathOperator{\Sep}{Sep}
\DeclareMathOperator{\Span}{Span}
\DeclareMathOperator{\Pal}{Pal}
\DeclareMathOperator{\Zigzag}{Zigzag}
\DeclareMathOperator{\Closure}{Closure}
\DeclareMathOperator{\End}{End}
\DeclareMathOperator{\Fact}{Fact}
\DeclareMathOperator{\Stab}{Stab}
\DeclareMathOperator{\Group}{Group}
\DeclareMathOperator{\rank}{rank}
\DeclareMathOperator{\ord}{ord}
\DeclareMathOperator{\supp}{supp}
\def\Im{\text{\upshape{Im}}}
\def\Ker{\text{\upshape{Ker}}}
\definecolor{ivoire}{rgb}{0.99,0.99,0.8}
\DeclareMathOperator{\Fac}{Fac}
\def\bs{\mathbf{s}}
\def\id{{\rm id}}
\def\bw{\mathbf{w}}
\definecolor{light-gray}{gray}{0.7}
\definecolor{forestgreen}{RGB}{34,139,34}
\definecolor{golden}{rgb}{1.0, 0.84, 0.0}





\numberwithin{theorem}{section}
\numberwithin{equation}{section}
\numberwithin{figure}{section}
\numberwithin{table}{section}








\begin{document}
\begin{frontmatter}

\title{Specular sets}

\author{Val\'erie Berth\'e, Clelia De Felice, Vincent Delecroix, \\ Francesco Dolce, Julien Leroy, Dominique Perrin, \\ Christophe Reutenauer, Giuseppina Rindone \\ \\  CNRS, Universit\'e Paris Diderot \\  Universit\`a degli Studi di Salerno \\  CNRS, Universit\'e de Bordeaux \\  Universit\'e Paris-Est \\  Universit\'e de Li\`ege \\  Universit\'e du Qu\'ebec \`a Montr\'eal}

\begin{abstract}
We introduce specular sets.
These are subsets of groups which form a natural generalization of free groups.
These sets of words are an abstract generalization of the natural codings of interval exchanges and of linear involutions.
We consider two important families of sets contained in specular sets: sets of return words and maximal bifix codes.
For both families we prove several cardinality results as well as results concerning the subgroup generated by these sets.
\end{abstract}

\begin{keyword}
Tree sets; Return words; Bifix codes; Linear involutions; Specular groups; Free group.
\end{keyword}


\end{frontmatter}







\tableofcontents





\section{Introduction}

We have studied in a series of papers initiated in~\cite{BerstelDeFelicePerrinReutenauerRindone2012} the links between uniformly recurrent languages, subgroups of free groups and bifix codes.
In this paper, we continue this investigation in a situation which involves groups which are not free anymore.
These groups, named here specular, are free products of a free group and of a finite number of cyclic groups of order two.
These groups are close to free groups and, in particular, the notion of a basis in such groups is clearly defined.
It follows from the Kurosh subgroup theorem that any subgroup of a specular group is specular.
A specular set is a subset of such a group which generalizes the natural codings of linear involutions studied in~\cite{BertheDelecroixDolcePerrinReutenauerRindone2014}.
It is a set of words stable by taking the inverse and defined in terms of restrictions on the extensions of its elements.

The main results of this paper are Theorems~\ref{theoremReturns} and~\ref{theoremFiniteIndex}, referred to as the First Return Theorem and the Finite Index Basis Theorem.
The first one asserts that the set of return words to a given word in a recurrent specular set is a basis of a subgroup of index 2, called the even subgroup.
The last one characterizes the symmetric bases of subgroups of finite index of specular groups contained in a specular set  as the finite -maximal symmetric bifix codes contained in . This generalizes the analogous result proved initially in~\cite{BerstelDeFelicePerrinReutenauerRindone2012} for Sturmian sets and extended in~\cite{BertheDeFeliceDolceLeroyPerrinReutenauerRindone2013b} to a more general class of sets, containing both Sturmian sets and interval exchange sets.

There are two interesting features of the subject of this paper.

In the first place, some of the statements concerning the natural codings of linear involutions can be proved using geometric methods, as shown in a separate paper~\cite{BertheDelecroixDolcePerrinReutenauerRindone2014}.
This provides an interesting interpretation of the groups playing a role in the natural codings (these groups are generated either by return words or by maximal bifix codes) as fundamental groups of some surfaces.
The methods used here are, however, purely combinatorial.

In the second place, the abstract notion of specular set gives rise to groups called here specular.
These groups are natural generalizations of free groups, and are free products of a finite number of copies of  and of .
They are called \emph{free-like} in~\cite{Bartholdi2014} and appear at several places in~\cite{Harpe2000}.

The idea of considering recurrent sets of reduced words invariant by taking inverses is connected with the notion of -full words of~\cite{PelantovaStarosta2014} (see Section~\ref{sec:palindromes}).

The paper is organized as follows.
In Section~\ref{sectionPreliminaries}, we recall some notions concerning words, extension graphs and bifix codes.
We define the notion of characteristic which is the Euler characteristic of the extension graph of the empty word.
We consider tree sets of characteristic  or  (tree sets of characteristic  are introduced in~\cite{BertheDeFeliceDolceLeroyPerrinReutenauerRindone2013a}, while the case of arbitrary characteristic is treated in~\cite{DolcePerrin2016}).


In Section~\ref{sectionSpecularGroups}, we introduce specular groups, which form a family with properties very close to free groups.
We deduce from the Kurosh subgroup theorem that any subgroup of a specular group is specular (Theorem~\ref{theoremKurosh}).

In Section~\ref{sectionSpecular} we introduce specular sets.
We recall several results from~\cite{DolcePerrin2015} and~\cite{DolcePerrin2016} concerning the cardinality of some sets included in neutral sets, namely bifix codes (Theorems~\ref{theoremCardinality} and~\ref{corollaryCardinality}).
We give a construction which allows to build specular sets from a tree set of characteristic  using a transducer called doubling transducer (Theorem~\ref{theoremDoubling}).
We make a connection with the notion of -full words introduced in~\cite{PelantovaStarosta2014} and related to the palindromic complexity of~\cite{DroubayJustinPirillo2001}.

In Section~\ref{sectionInvolutions} we recall the definition of a linear involution introduced in~\cite{DanthonyNogueira1988} and we show that the natural coding of a linear involution without connections is a specular set (Theorem~\ref{theoremInvolutionSpecular}).

In Section~\ref{sec:return} we introduce three variants of the notion of set of return words.
We prove several cardinality results concerning these sets (Theorems~\ref{theoremCardRightReturns}, \ref{theoremCardReturns}, \ref{theoremCardFirstMixed}).
We prove that the set of return words to a given word forms a basis of the even subgroup (Theorem~\ref{theoremReturns} referred to as the First Return Theorem) and that the mixed return words form a monoidal basis of the specular group (Theorem \ref{theoremFirstMixed}).

In Section~\ref{sec:groups} we prove several results concerning subgroups generated by bifix codes.
We prove that a set closed by taking inverses is acyclic if and only if any symmetric bifix code is free (Theorem~\ref{theoremFreeness}).
Moreover, we prove that in such a set, for any finite symmetric bifix code , the free monoid  and the free subgroup  have the same intersection with  (Theorem~\ref{theoremSaturation}).

Finally, in Section~\ref{sectionFiniteIndex}, we prove the Finite Index Basis Theorem (Theorem~\ref{theoremFiniteIndex}) and a converse (Theorem~\ref{propositionConverseFIB}).





\paragraph{Acknowledgments}
This paper is an extended version of a conference paper~\cite{words2015}.
The authors thank  Laurent Bartholdi and Pierre de la Harpe for useful indications.
This work was supported by grants from R\'egion \^{I}le-de-France, the ANR projects Dyna3S and Eqinocs.










\section{Preliminaries}
\label{sectionPreliminaries}
In this section, we first recall some notions on sets of words including recurrent, uniformly recurrent and tree sets.
We also recall some definitions and properties concerning bifix codes.





\subsection{Extension graphs}
Let  be a finite alphabet.
We denote by  the free monoid on .
We denote by  the empty word.
The \emph{reversal} of a word  with  is the word .
A word  is said to be a \emph{palindrome} if .

A set of words on the alphabet  is said to be \emph{factorial} if it contains the alphabet  and all the factors of its elements. 

An \emph{internal factor} of a word  is a word  such that  with  nonempty.

Let  be a  set of words on the alphabet .
For , we denote

and further

We omit the subscript  when it is clear from the context.
A word  is \emph{right-extendable} if ,
\emph{left-extendable} if  and \emph{biextendable} if
. A 
factorial set
 is called \emph{right-extendable}
(resp. \emph{left-extendable}, resp. \emph{biextendable}) if every word in  is
right-extendable (resp. left-extendable, resp. biextendable).

A word  is called \emph{right-special} if . 
It is called \emph{left-special} if . 
It is called \emph{bispecial} if it is
both left-special and right-special.

For , we denote 

The word  is called \emph{weak} if , \emph{neutral} if  and \emph{strong} if .

We say that a factorial set  is \emph{neutral} if every nonempty word in  is neutral.
The \emph{characteristic} of  is the integer .
Thus a neutral set of characteristic  is such that all words (including the empty word) are neutral. This what is called a neutral
set in~\cite{BertheDeFeliceDolceLeroyPerrinReutenauerRindone2013a}.

A set of words  is \emph{recurrent} if it is factorial and if for any , there is a  such that .
An infinite factorial set is said to be \emph{uniformly recurrent} if for any word  there is an integer  such that  is a factor of any word of  of length .
A uniformly recurrent set is recurrent.

In~\cite{DolcePerrin2016} it is proved that the converse is true for neutral sets.
As all sets we will deal with are neutral, we usually omit the term ``uniformly'' and just mention whenever we suppose them to be recurrent.

\begin{theorem}[\cite{DolcePerrin2016}]
\label{theo:recurrentur}
A recurrent neutral set is uniformly recurrent
\end{theorem}

The \emph{factor complexity} of a factorial set  of words on an alphabet  is the sequence .
Let  and  be respectively the first and second order differences sequences of the sequence . 

The following result is from~\cite{Cassaigne1997} (see also \cite{BertheRigo2010}, Theorem 4.5.4).

\begin{proposition}
\label{propCANT}
Let  be a factorial set on the alphabet .
One has  and  for all .
\end{proposition}


Let  be a biextendable set of words.
For , we consider define the undirected graph , or simply  when  is clear from the context, having as set of vertices the disjoint union of  and  and edges the pairs .
This graph is called the \emph{extension graph} of .
We sometimes denote  and  the copies of  and  used to define the set of vertices of .
We note that, since  has  vertices and  edges, the number  is the Euler characteristic of the graph \footnote{We consider here graphs as 1-dimensional complexes and thus they have no faces.}.

If the extension graph  is acyclic, then .
Thus  is weak or neutral.
More precisely, one has in this case that  is the number of connected components of the graph .

A biextendable set  is called \emph{acyclic} if for every , the graph  is acyclic.

A biextendable set  is called a \emph{tree set} of characteristic  if for any nonempty , the graph  is a tree and if  is a union of  trees (the definition of tree set in~\cite{BertheDeFeliceDolceLeroyPerrinReutenauerRindone2013a} corresponds to a tree set of characteristic ).
Note that a tree set of characteristic  is a neutral set of characteristic . We focus here on characteristic  or  (specular sets, that we will introduce in Section~\ref{sectionSpecular}, are tree sets of characteristic  with some symmetric properties).


An infinite word is \emph{episturmian} if the set of its factors is closed under reversal and if it contains for each  at most one word of length  which is right-special.
It is a \emph{strict episturmian} word if the set of its factors has exactly one right-special word of each length and moreover each of these words  is such that  (see~\cite{BerstelDeFelicePerrinReutenauerRindone2012}).

A \emph{Sturmian set} is the set of factors of a strict episturmian word.
Any Sturmian set is a recurrent tree set of characteristic  (see~\cite{BertheDeFeliceDolceLeroyPerrinReutenauerRindone2013a}). 

\begin{example}
\label{exampleFibo}
Let . The \emph{Fibonacci morphism} is the morphism  defined by  and .
The \emph{Fibonacci word} is the fixed-point  of the Fibonacci morphism.
Its set of factors is a Sturmian set (see~\cite{Lothaire2002}).
\end{example}





\subsection{Bifix codes}
\label{sectionBifix}

A \emph{prefix code} is a set of nonempty words which does not contain any
proper prefix of its elements. A \emph{suffix code} is defined symmetrically.
A \emph{bifix code} is a set which is both a prefix code and a suffix code
(see~\cite{BerstelPerrinReutenauer2009} for a more detailed introduction).

A \emph{coding morphism} for a prefix code  on the alphabet 
is a morphism  which maps bijectively  onto .

Let  be a recurrent set.
A prefix (resp. bifix) code  is -maximal if it is not properly contained
in a prefix (resp. bifix) code . 
Since  is recurrent, a finite -maximal bifix code is also an
-maximal prefix code 
(see~\cite{BerstelDeFelicePerrinReutenauerRindone2012}, Theorem 4.2.2).

For example, for any , the set  is an -maximal
bifix code.

Let  be a bifix code. Let  be the set of words without any suffix in 
and let  be the set of words without any prefix in .
A \emph{parse} of a word  with respect to a bifix code 
is a triple  such that . 
We denote by  the number of parses of a word 
with respect to . The -degree of , denoted , is
the maximal number of parses with respect to  of a word of .

For example, the set 
has -degree .

Let  be a recurrent set and let  be a finite bifix code. By Theorem 4.2.8
in~\cite{BerstelDeFelicePerrinReutenauerRindone2012},  is -maximal
if and only if its -degree is finite. Moreover, in this case,
a word  is such that  if and only if it is an internal factor of a word of . 

The \emph{kernel} of a bifix code  is the set of words of  which
are internal factors of .

We will use bifix codes in relation with
a more general version of extension graphs (see
\cite{BertheDeFeliceDolceLeroyPerrinReutenauerRindone2013a}). For two sets of words  and a word , we denote .
We also define 
as the undirected graph on the set of vertices which is the disjoint union of  and  and edges in .
Set further

Finally, for a word , we denote .
Note that , , and so on.

We will use below the following result.

\begin{proposition}
\label{propositionExtended}
Let  be a recurrent set, let  be a finite -maximal suffix code and let  be a finite -maximal prefix code.
\begin{enumerate}
\item If  is acyclic, then   is acyclic.
\item If  is neutral, then   for every .
\end{enumerate}
\end{proposition}
\begin{proof}
Statement 1 follows from Proposition 3.7 in~\cite{BertheDeFeliceDolceLeroyPerrinReutenauerRindone2013a}. 
Statement 2 is Proposition 6.2 in~\cite{DolcePerrin2016}.
\end{proof}

Observe that the condition that  (resp. ) is an -maximal suffix (resp. prefix) code is only necessary for Assertion 2
(for Assertion 1,  (resp. ) may be an arbitrary suffix (resp. prefix)
code). Observe also
 that this condition can be replaced by the condition that  (resp. ) is an
-maximal suffix code (resp. a -maximal prefix code),
where  and symmetrically .










\section{Specular groups}
\label{sectionSpecularGroups}
In this section, we introduce specular groups and we prove some properties of this family of groups.
In particular, using the Kurosh subgroup theorem, we prove that any subgroup of a specular group is specular (Theorem~\ref{theoremKurosh}).





\subsection{Definitions}
\label{subsec:speculardefinitions}
We consider an alphabet  with an involution , possibly with some fixed points.
We also consider the group  generated by  with the relations  for every .
Thus  for .
The set  is called a \emph{natural} set of generators of . 

When  has no fixed point, we can set 
by choosing a set of representatives of the orbits of 
for the set . The group  is then the free
group on , denoted . In general, the group  is a free product of
a free group and a finite number of copies of , that
is  where   is the number
of orbits of  with two elements  and  the number of its fixed points. 
Such a group will be called a \emph{specular group} of type .
These groups are very close to free groups, as we will see.
The integer  is called the \emph{symmetric rank} of the specular group .

\begin{proposition}
\label{pro:isospecular}
Two specular groups are isomorphic if and only if they have the same type.
\end{proposition}
\begin{proof}
The commutative image of a group of type  is  and the uniqueness of  follows from the fundamental theorem of finitely generated Abelian groups. 
\end{proof}

\begin{example}
\label{exampleSpecularGroup}
Let  and let  be the involution which exchanges 
and fixes . Then  is a specular group of symmetric
rank
.
\end{example}
The Cayley graph of a specular group  with respect to the set
of natural generators  is a regular tree where each vertex has degree
.
The specular groups are actually
characterized by this property (see~\cite{Harpe2000}).





\subsection{Subgroups}
\label{subsec:sub}
By the Kurosh subgroup theorem, any subgroup of a free product  is itself a free product of a free group and of groups conjugate to subgroups of the  (see~\cite{MagnusKarrassSolitar2004}).
Thus, we have, replacing  the Nielsen-Schreier Theorem of free groups, the following result.

\begin{theorem}
\label{theoremKurosh}
Any subgroup of a specular group is specular.
\end{theorem}

It also follows from the Kurosh subgroup theorem that the elements of order  in a specular group  are the conjugates of the  fixed points of  and this number is thus the number
of conjugacy classes of elements of order .
Indeed, an element of order  generates a subgroup conjugate to one of the subgroups generated by the letters of order 2.

Any specular group  has a free subgroup of index . Indeed, let
 be the subgroup formed of the reduced words of even length. It has
clearly index . It is free because it does not contain any element
of order  (such an element is conjugate to a fixed point of 
and thus is of odd length). 

A group having a free subgroup of finite index is called \emph{virtually
free} (see \cite{Harpe2000}). On the other hand, a finitely generated group is
said to be \emph{context-free} if, for some presentation, the set
of words equivalent to  is a context-free language.
By Muller and Schupp's theorem,
a finitely generated group is virtually free if and only if
it is context-free \cite{MullerSchupp1983}. 
Thus a specular group is context-free. One may
verify this directly as follows. A
context-free grammar generating the words equivalent to  for the
natural presentation of a specular group  is the
grammar with one nonterminal symbol  and the rules

The proof that this grammar generates the set of words
equivalent to  is similar to that used in~\cite{Berstel1979}
for the so-called Dyck-like languages.

We will need two more properties of specular groups.
Both are well-known to hold for free groups.

A group  is called \emph{residually finite} if for every element  of , there is a morphism  from  onto a finite group such that .

\begin{proposition}\label{propResiduallyFinite}
Any specular group is residually finite.
\end{proposition}
\begin{proof}
Let  be a free subgroup of index  in the specular group .
Let  be in . If , then the image of  in 
is nontrivial. Assume . Since  is free, it is residually finite.
Let  be a normal subgroup of finite index
of  such that . Consider the representation of 
on the right cosets of . Since , the image of 
in this finite group is nontrivial.
\end{proof}

A group  is said to be \emph{Hopfian} if any surjective morphism from  onto  is also injective.
By a result of Malcev, any finitely generated residually finite group is Hopfian (see~\cite{LyndonSchupp2001}, p. 197).
We thus deduce from Proposition~\ref{propResiduallyFinite} the following result.

\begin{proposition}
\label{pro:hopf}
A specular group is Hopfian.
\end{proposition}





\subsection{Monoidal basis}
\label{subsec:basis}
A word on the alphabet  is -\emph{reduced} (or simply \emph{reduced}) if it has no factor of the form  for .
It is clear that any element of a specular
group is represented by a unique reduced word.

A subset of a group  is called \emph{symmetric} if it is closed under taking inverses.
A set  in a specular group  is called a \emph{monoidal basis} of  if it is symmetric, if the monoid that it generates is   and if any product  of elements of  such that  for  is distinct of .

\begin{example}
The alphabet  is a monoidal basis of .
\end{example}

The previous example shows that the symmetric rank of a specular group is the cardinality of any monoidal basis (two monoidal bases have the same cardinality since the type is invariant by isomorphism by Proposition~\ref{pro:isospecular}).

Let  be a subgroup of a specular group .
Let  be a set of reduced words on  which is a prefix-closed set of representatives of the right cosets 
of .
Such a set is traditionally called a \emph{Schreier transversal} for 
(the proof of its existence is classical in the free group and it
is the same in any specular group).

Let

Each word  of  has a unique factorization  with 
and . The letter  is called the \emph{central part} of .
The set 
is a monoidal basis of , called the \emph{Schreier basis}
relative to .

\begin{proposition}
Let  and  be as above and let  be a Schreier basis relative to .
Then  is closed by taking inverses.
\end{proposition}
\begin{proof}
Let , then  .
We cannot have  since otherwise  implies
 by uniqueness of the coset representative and finally .
It generates  as a monoid because if 
with ,
then  with 
for  is a factorization of  in elements of .
Finally, if a product  of elements of  
is equal to , then  for some index 
since the central part  never
cancels in a product of two elements of .
\end{proof}

One can deduce directly Theorem~\ref{theoremKurosh} from these properties of .

\begin{proofof}{of Theorem~\ref{theoremKurosh}}
Let  be a subgroup of a specular group ,  be a Shreier transversal for  and  be the Schreier basis relative to .
Let  be a bijection from a set  onto  which extends to a morphism from  onto .
Let  be the involution sending each  to  where .
Since the central parts never cancel, if a nonempty word   is -reduced then .
This shows that  is isomorphic to the group .
Thus  is specular.
\end{proofof}


If  is a subgroup of index  of a specular group  of symmetric rank ,
the symmetric rank  of  is

This formula replaces Schreier's Formula (which corresponds to the case ). It can be proved as follows.
Let  be a Schreier transversal for  and let  be the
corresponding Schreier basis.
 The number
of elements of  is . Indeed, this
is the number of pairs 
 minus the  pairs  such that  with 
reduced or  with  not reduced.  This gives Formula
\eqref{SchreierSpecular}.

\begin{example}
\label{exampleEvenLength}
Let  be the specular group of Example~\ref{exampleSpecularGroup}.
Let  be the subgroup formed by the elements represented by a reduced word of even length.
The set  is a prefix-closed set of representatives of the two cosets of .
The representation of  by permutations on the cosets of  is represented in Figure~\ref{figureSchreier}.

\begin{figure}[hbt]
\centering\gasset{Nadjust=wh}
\begin{picture}(20,20)(0,-7)
\node(1)(0,0){}
\node(a)(20,0){}


\drawedge[curvedepth=5](1,a){}
\drawedge[curvedepth=5](a,1){}
\end{picture}
\caption{The representation of  by permutations on the cosets of .}
\label{figureSchreier}
\end{figure}
The monoidal basis corresponding to Formula \eqref{setSchreier} is 
.
The symmetric rank of  is , in agreement with Formula~\eqref{SchreierSpecular} and  is a free group of rank .
\end{example}

\begin{example}
Let again  be the specular group of Example~\ref{exampleSpecularGroup}.
Consider now the subgroup  stabilizing  in the representation of  by permutations on the set  of Figure~\ref{figureSchreier2}.

\begin{figure}[hbt]
\centering\gasset{Nadjust=wh}
\begin{picture}(20,20)(0,-7)
\node(1)(0,0){}\node(2)(20,0){}

\drawloop[loopangle=180](1){}
\drawedge[curvedepth=5](1,2){}
\drawedge[curvedepth=5](2,1){}
\drawloop[loopangle=0](2){}
\end{picture}
\caption{The representation of  by permutations on the cosets of .}
\label{figureSchreier2}
\end{figure}

We choose .
The set  corresponding to Formula \eqref{setSchreier} is .
The group  is isomorphic to .
\end{example}

The following result, which will be used later (Section~\ref{sec:return}), is a consequence of Proposition~\ref{pro:hopf}.

\begin{proposition}\label{propSymBasis}
Let  be a specular group of type  and let  be a symmetric set with  elements. If 
generates , it is a monoidal basis of .
\end{proposition}
\begin{proof}
Let  be a set of natural generators of .
Considering the commutative image of , we obtain that 
contains  elements of order . Thus there is a bijection
 from  onto  such that 
for every . The map  extends to a morphism
 from  to  which is surjective since  generates . Then  being surjective, it also injective
since  is Hopfian, and thus  is a monoidal basis of .
\end{proof}










\section{Specular sets}
\label{sectionSpecular}
In this section, we introduce specular sets.
We introduce odd and even words and the even code which play an important part in the sequel.
We prove that the decoding of a recurrent specular set by the even code is a union of two recurrent tree sets of characteristic  (Theorem~\ref{theoremDecodingEven}).
We exhibit a family of specular sets obtained as the result of a transformation called doubling, starting from a tree set of characteristic  and invariant by reversal (Theorem~\ref{theoremDoubling}).
In the last part, we relate specular sets with full and -full words, a notion linked with palindromic complexity and introduced in~\cite{PelantovaStarosta2014}.





\subsection{Definition}

We assume given an involution  on the alphabet  generating the specular group .

A symmetric biextendable (and thus factorial)
set  of reduced words on the alphabet 
is called
a \emph{laminary set} on  relative to  (following~\cite{HilionCoulboisLustig2008}
and~\cite{LopezNarbel2013}). Thus the elements of a laminary set 
are elements of the specular group  and the set  is contained
in .


A \emph{specular set} is a laminary set on  which is a tree set of characteristic . Thus, in a specular set, the extension graph of every nonempty word
is a tree and the extension graph of the empty word is a union of two
disjoint trees.
 

The following is a very simple example of a specular set.
\begin{example}\label{exampleabab}
Let  and let  be the identity on . Then the set
of factors of  is a specular set
(we denote by  the word  infinitely repeated).
\end{example}
The next example is due to Julien Cassaigne. 
We frequently refer to it in next sections.

\begin{example}\label{exampleJulien}
Let  and let  be
the set of factors of the fixed point 
of the morphism  from  into
itself defined by 

\begin{figure}[hbt]
\centering\gasset{Nadjust=wh,AHnb=0}
\begin{picture}(50,10)
\put(0,0){
\begin{picture}(20,10)
\node(bl)(0,0){}\node(al)(0,10){}
\node(cr)(20,0){}\node(br)(20,10){}

\drawedge(bl,cr){}\drawedge(al,cr){}
\drawedge(al,br){}
\end{picture}
}
\put(30,0){
\begin{picture}(20,10)
\node(dl)(0,0){}\node(cl)(0,10){}
\node(ar)(20,0){}\node(dr)(20,10){}

\drawedge(bl,cr){}\drawedge(al,cr){}
\drawedge(al,br){}
\end{picture}
}
\end{picture}
\caption{The extension graph .}\label{figureExtensionGraph}
\end{figure}

The extension graph of  is shown in Figure~\ref{figureExtensionGraph}.
It is shown in~\cite[Example 3.4]{BertheDeFeliceDolceLeroyPerrinReutenauerRindone2013a} that  is a tree set
of characteristic . We will see
later (Example~\ref{exampleJulien2}) that  is a specular set relative to the involution .
\end{example}
The following result shows in particular
 that in a specular set the two trees forming 
are isomorphic since they are exchanged by the bijection . 
\begin{proposition}\label{propositionClelia}
Let  be a specular set.
Let  be the two trees such that .
For any  and , one has 
if and only if  .
\end{proposition}
\begin{proof}
Assume that 
and   are both in . Since 
is a tree, there is a path from  to . 
We may assume that this path is reduced, that is, does not use
consecutively twice the same edge. Since this path is of odd
length, it has the form  with  and .
Since  is symmetric, we also have a reduced path  which is in  (for , we denote  and similarly for ) and thus in  since  and  are disjoint.
Since , these two paths have the same origin and end.
But if a path of odd length is its own inverse, its central
edge has the form  with , as one verifies easily
by induction on the length of the path. This is a contradiction with
the fact that the words of  are reduced.
Thus the two paths  are distinct.
This implies that  has
a cycle, a contradiction.
\end{proof}

Following again the terminology of~\cite{HilionCoulboisLustig2008}, we say
that a laminary set  is \emph{orientable} if there exist two factorial
sets  such that  with 
and for any , one has  if and only if 
(where  is the inverse of  in ).


The following result shows in particular that for 
any tree set  of characteristic 1
on the alphabet , the set 
is a specular set on the alphabet .

\begin{theorem}
Let  be a specular set on the alphabet .
Then,  is orientable if and only if there is a partition
 of the alphabet  and a tree set  of characteristic 1 on the alphabet  such that .
\end{theorem}

\begin{proof}
The condition is trivially sufficient.
Let us prove it is necessary and suppose that  is a specular set on the alphabet  which is orientable.
Let  be the corresponding pair of subsets of .
 The sets  are biextendable, since  is.
Set  and . Then  is a partition
of  and, since  are factorial, we have 
and .
Let  be the two trees such that .
Assume that a vertex of  is in . Then all vertices
of  are in  and all vertices of  are in .
Moreover,  and .
Thus  are tree sets of characteristic 1.
\end{proof} 

The following result follows easily from Proposition~\ref{propCANT} (see~\cite[Proposition 2.4]{DolcePerrin2016} for details).

\begin{proposition}\label{propComplexity}
The factor complexity of a specular set containing the alphabet  is given by  and  for  with . 
\end{proposition}





\subsection{Odd and even words}\label{sectionOddEven}
We introduce a notion which plays, as  we shall see, an important role in the study of specular sets.
Let  be a specular set.
Since a specular set is biextendable, any letter  occurs exactly twice as a vertex of , one as an element of  and one as an element of .
A letter  is said to be \emph{even} if its two occurrences appear in the same tree.
Otherwise, it is said to be \emph{odd}.
Observe that if a specular  is recurrent, there is at least one odd letter. 

\begin{example}
Let  be the set of factors of  as in Example~\ref{exampleabab}.
Then  and  are odd.
\end{example}

\begin{example}
Let  be the set of Example~\ref{exampleJulien}.
The letters  are even, while  and  are odd.
\end{example}


Let  be a specular set.
A word  is said to be \emph{even} if it has an even number
of odd letters. Otherwise it is said to be \emph{odd}.
The set of even words has the form  where 
 is a bifix code, called the \emph{even code}. The set 
is the set of even words without a nonempty even
prefix (or suffix). 
\begin{proposition}\label{propositionEvenCode}
 Let  be a recurrent specular set.
The even code is an -maximal bifix code of -degree .
\end{proposition}
\begin{proof} 
Let us verify that
any  is comparable for the prefix order with an element of
the even code
. If  is even, it is in . Otherwise, since  is recurrent,
there is a word  such that . If  is even, then
 is even and thus . Otherwise  is even and
thus . This shows that  is -maximal. 
 The fact
that it has -degree  follows from the fact that
any product of two odd letters is a word of  which is not an
internal factor of  and has two parses.
\end{proof}

\begin{example}\label{exampleEvenCode}
Let  be the specular set of Example~\ref{exampleJulien}. The letters
 are even and the letters  are odd. The even code is

\end{example}

Denote by  the two trees such that .
We consider the directed graph  with vertices  and edges all the triples
 for  and  such that  and
  for some . 
The graph  is called the
\emph{parity graph} of . Observe that for every letter  there
is exactly one edge labeled  because  appears exactly once as a left
(resp. right) vertex in .

\begin{example}\label{exampleparitygraph}
Let  be the specular set of Example~\ref{exampleJulien}. 
The parity graph of  is represented in Figure~\ref{figureParityGraph}, where we assume that  is the tree on the left of Figure~\ref{figureExtensionGraph} and  is the tree on the right of Figure~\ref{figureExtensionGraph}.
\begin{figure}[hbt]
\centering\gasset{Nadjust=wh}
\begin{picture}(20,10)
\node(0)(0,5){}\node(1)(20,5){}

\drawloop[loopangle=180](0){}
\drawedge[curvedepth=3](0,1){}\drawedge[curvedepth=3](1,0){}
\drawloop[loopangle=0](1){}
\end{picture}
\caption{The parity graph.}\label{figureParityGraph}
\end{figure}
\end{example}


\begin{proposition}\label{propositionPartition}
Let  be a  specular set and let  be its parity graph.
Let  be the set of words in  which are the label of a path
from  to  in the graph . 
\begin{enumerate}
\item[(1)] The family  is a partition of .
\item[(2)] For  and , if ,
then . 
\item[(3)]  is the set of even words.
\item[(4)] .
\end{enumerate}
\end{proposition}
\begin{proof}
We first note that for  such that , there is a path
in  labeled . Since ,
there is a   such that . Then we have
 and  for some . This shows
that  is the label of a path from  to  in .


Let us prove by induction on the length of a nonempty word  
that there exists a unique pair  such that .
The property is true for a letter, by definition of the extension
graph  and for words of length 
by the above argument. Let next  be in  with 
and  nonempty. By induction hypothesis, there is a unique pair
 such that . Let  be the first letter
of . Then the edge of  with label  starts in . Since
 is the label of a path, we have  for some
 and thus .
The other assertions follow easily
(Assertion (4) follows from Proposition~\ref{propositionClelia}).
\end{proof}
Note that Assertion (4) implies that no nonempty even word is its own
inverse. Indeed,  and .


\begin{proposition}\label{propositionOddReturns}
Let  be a specular set. 
If  are nonempty words such that , then  is odd.
\end{proposition}
\begin{proof}
Let  be such that . Then  by 
Assertion (4) of
Proposition~\ref{propositionPartition} and thus  by Assertion
 (2).
 Thus  is odd by Assertion (3).
\end{proof}

The following result is the counterpart for recurrent specular sets of the main result of~\cite[Theorem 6.1]{BertheDeFeliceDolceLeroyPerrinReutenauerRindone2013m} asserting that the family of (uniformly) recurrent tree sets of characteristic  is closed under maximal bifix decoding.
Let  be a recurrent set and let  be a coding morphism for a finite -maximal bifix code . The set  is called a \emph{decoding} of  by .

\begin{theorem}[Even code decoding Theorem]
\label{theoremDecodingEven}
The decoding of a  recurrent specular set by the even code is  a union of two recurrent tree sets of characteristic .
More precisely, let  be a recurrent specular set and let  be a coding morphism for the even code.
Then  and  are recurrent tree sets of characteristic .
\end{theorem}
\begin{proof}
We show that  is a recurrent tree set of characteristic .
The proof for  is the same.

First,  is biextendable, as one may easily verify.
Next, since  is recurrent, it is uniformly recurrent by Theorem~\ref{theo:recurrentur}.
Thus for every  there exists  such that  is a factor of any word  in  of length .
But if  are such that , then .
Thus  is recurrent.

We now show that  is a tree set of characteristic .
Let  be the even code and set  and .

It is enough to show that  is a tree for any .
Note first that .
Indeed, for  and  such that , one has  and .

First, for any nonempty word , since , the graph  is a tree by Proposition~\ref{propositionExtended}.

Next, let us show that the graph  is a tree.
First, since  is a union of two trees, it is acyclic, and thus the graph  is acyclic by Proposition~\ref{propositionExtended}.
Next, since  is neutral, by Proposition~\ref{propositionExtended}, we have .
This implies that  is a union of two trees.
Since  is the disjoint union of  and , this implies that each one is a tree.
\end{proof}

\begin{example}
Let  be the set of Example~\ref{exampleJulien}. 
Recall that it is the set of
factors of the fixed point of the morphism

The even code 
is given in Example~\ref{exampleEvenCode}. Let 
and let  be the
coding morphism for  given by 

The decoding of 
by  is a union of two tree sets of characteristic  which are the set of factors of the
fixed point of the two morphisms

and 

These two morphisms are actually the restrictions to  and 
of
the morphism .
\end{example}





\subsection{Bifix codes in specular sets}
Recall from Section~\ref{sectionPreliminaries} that the characteristic of a set  is given by .

The following result is from~\cite{DolcePerrin2016}.
We will use it for specular sets.

\begin{theorem}
\label{theoremCardinality}
Let  be a recurrent neutral set containing the alphabet .
For any finite -maximal bifix code  of -degree , one has

\end{theorem}

We can apply Theorem~\ref{theoremCardinality} to recurrent specular sets.

\begin{theorem}[Cardinality Theorem for bifix codes]
\label{corollaryCardinality}
Let  be a recurrent specular set containing the alphabet .
For any finite -maximal bifix code , one has 

\end{theorem}
\begin{proof}
Since  is specular, we have  and thus the statement follows directly from Theorem~\ref{theoremCardinality}.
\end{proof}
\begin{example}
Let  be the specular set of Example~\ref{exampleJulien}.
The even code (given in Example~\ref{exampleEvenCode}) is an -maximal code of -degree .
We have  in agreement with Theorem~\ref{theoremCardinality}.
\end{example}

The following statement is a partial converse of Theorem~\ref{theoremCardinality}. 

\begin{theorem}
\label{theoremConverseCard}
Let  be a uniformly recurrent laminary set containing the alphabet .
If the graph  is acyclic and if any finite -maximal bifix code of -degree  has  elements, then  is specular.
\end{theorem}

To prove Theorem~\ref{theoremConverseCard}, we use the following result, which can be proved in the same way as Theorem 3.12 in~\cite{BertheDeFeliceDolceLeroyPerrinReutenauerRindone2013b}, using internal transformations.

\begin{proposition}
\label{propositionConverseCard}
Let  be a recurrent set containing the alphabet  and let .
If all finite -maximal bifix codes of -degree  have the same cardinality, then any word of length greater than or equal to  is neutral.
\end{proposition}

Theorem~\ref{theoremConverseCard} results from Proposition~\ref{propositionConverseCard} applied with .







\subsection{Doubling maps}
\label{sec:doubling}
We now introduce a construction which allows one to build specular sets.
This is a particular case of the multiplying maps introduced in~\cite{DolcePerrin2016}.

A \emph{transducer} is a labeled graph with vertices
in a set  and edges labeled in . The set  is called the set of states, the set  is called the \emph{input alphabet}
and  is called the \emph{output alphabet}.
The graph obtained by erasing the output
letters is called the \emph{input automaton} (with an unspecified initial state). Similarly, the \emph{output automaton} is obtained by erasing the input letters.

Let  be a transducer with set of states  on the input alphabet
 and the output alphabet . We assume that
\begin{enumerate}
\item the input automaton is a group automaton,
 that is, every
letter of  acts on  as a permutation
\item the output labels of the edges are all distinct.
\end{enumerate}
  We define two maps  corresponding to initial states
 and  respectively. 
Thus  (resp. ) if the path starting at state  (resp. ) with input label  has output .
The pair  is called a \emph{doubling map}
and the transducer  a \emph{doubling transducer}.

The \emph{image} of a set  on the alphabet 
by the doubling map  is the set .

If  is a doubling transducer, we define an involution 
as follows. For any , let  be the edge with input
label  and output label . We define  as the output
label of the edge starting at  with input label . Thus, 
if  and  if .

Recall that the reversal of a word  is the word .

One can prove by induction on the length of  that if  and if  is the end of the path starting at  and with input label , then .
Observe that since the input automaton is a group automaton, there is always a path starting at  with input label .

A set  of words is closed under reversal if  implies  for every .
By definition, any Sturmian set is closed under reversal (see~\cite{BerstelDeFelicePerrinReutenauerRindone2012}).


\begin{theorem}\label{theoremDoubling}
For any tree set  of characteristic 
on the alphabet , closed under reversal
and any doubling map , the image of  by 
  is a specular set relative to the
involution .
\end{theorem}
\begin{proof}
Set .
By Theorem 3.1 of~\cite{DolcePerrin2016},  is a tree set of characteristic 2.
By construction, it is also clear the any word in  is -reduced.

Let now prove that  is a symmetric language.
Assume that   for  and .
Let  be the end of the path starting at  and with input
label . 
Since  and  is closed under reversal,
we have .
This shows that  is symmetric and so that it is laminary.
Thus,  is a specular set.
\end{proof}


We now give two examples of specular sets obtained by doubling maps (doubling the Fibonacci set).

\begin{example}
\label{ex:fibodouble2}
Let  and let  be the Fibonacci set over .
Let  be the doubling map given by the transducer of Figure~\ref{fig:fibodouble2} on the left.

\begin{figure}[hbt]
\centering\gasset{Nadjust=wh}
\begin{picture}(100,15)
\put(0,0){
\begin{picture}(50,15)
\node(0)(0,7){}
\node(1)(30,7){}

\drawloop[loopangle=30](0){}
\drawloop[loopangle=210](0){}
\drawloop[loopangle=30](1){}
\drawloop[loopangle=210](1){}
\end{picture}
}
\gasset{AHnb=0}
\put(50,0){
\begin{picture}(50,10)
\put(0,0){
\begin{picture}(20,10)
\node(dl)(0,0){}
\node(cl)(0,10){}
\node(ar)(20,0){}
\node(dr)(20,10){}

\drawedge(bl,cr){}
\drawedge(al,cr){}
\drawedge(al,br){}
\end{picture}
}
\put(30,0){
\begin{picture}(20,10)
\node(bl)(0,0){}
\node(al)(0,10){}
\node(cr)(20,0){}
\node(br)(20,10){}

\drawedge(bl,cr){}
\drawedge(al,cr){}
\drawedge(al,br){}
\end{picture}
}
\end{picture}
}
\end{picture}
\caption{A doubling transducer and the extension graph .}
\label{fig:fibodouble2}
\end{figure}

Both letters in  act as the identity on the two states .

Then  is the involution defined by .
The image of  by  is a specular set  on the alphabet .
The graph  is represented in Figure~\ref{figureFiboDouble} on the right.
All letters are even.

Note that the set  of Example~\ref{ex:fibodouble2} is not recurrent.
The set  is actually just a union of two Fibonacci sets, one over the alphabet  and the second over the alphabet .
\end{example}


\begin{example}
\label{exampleFiboDouble}
Let  and let  be the Fibonacci set.
Let  be the doubling map given by the transducer of Figure~\ref{figureFiboDouble} on the left.
The letter  acts as the transposition of the two states , while  acts as the identity.

\begin{figure}[hbt]
\centering
\gasset{Nadjust=wh}
\begin{picture}(100,12)
\put(5,0){
\begin{picture}(20,10)
\node(0)(0,5){}\node(1)(20,5){}

\drawloop[loopangle=180](0){}
\drawedge[curvedepth=3](0,1){}\drawedge[curvedepth=3](1,0){}
\drawloop[loopangle=0](1){}
\end{picture}
}
\gasset{AHnb=0}
\put(50,0){
\begin{picture}(50,10)
\put(30,0){
\begin{picture}(20,10)
\node(bl)(0,0){}\node(al)(0,10){}
\node(cr)(20,0){}\node(br)(20,10){}

\drawedge(bl,cr){}\drawedge(al,cr){}
\drawedge(al,br){}
\end{picture}
}
\put(0,0){
\begin{picture}(20,10)
\node(dl)(0,0){}\node(cl)(0,10){}
\node(ar)(20,0){}\node(dr)(20,10){}

\drawedge(bl,cr){}\drawedge(al,cr){}
\drawedge(al,br){}
\end{picture}
}
\end{picture}
}
\end{picture}
\caption{A doubling transducer and the extension graph .}
\label{figureFiboDouble}
\end{figure}

Then  is the involution  of Example~\ref{exampleSpecularGroup} and the image of  by  is a specular set 
on the alphabet .
The graph  is represented
in Figure~\ref{figureFiboDouble} on the right.

The letters  are odd and  are even.

Note that  is the set of factors of the fixed point 
of the morphism

The morphism  is obtained by applying the doubling map
to the cube  of the Fibonacci morphism  in such a way that
.
\end{example}


In the next example (due to Julien Cassaigne), the specular set
is obtained using a morphism of smaller size.
\begin{example}\label{exampleJulien2}
Let .
Let  be the set of factors of the fixed point  of
the morphism . It is a Sturmian set.
Indeed,  is the characteristic word of slope 
(see~\cite{Lothaire2002}).
The sequence  satisfies  for .
The image  of  by
the doubling  automaton of Figure~\ref{figureFiboDouble} is
the set of factors of the fixed point 
of the morphism  from  into
itself defined by 

Thus the set  is the same as that of Example~\ref{exampleJulien}.
\end{example}

Note that, when  is a specular set obtained by a doubling map using a transducer , the parity graph of  is the output automaton of  (see for instance Figures~\ref{figureParityGraph} and~\ref{figureFiboDouble}).



\subsection{Palindromes}
\label{sec:palindromes}

The notion of palindromic complexity originates in~\cite{DroubayJustinPirillo2001} where it is proved that a word of length  has at most  palindrome factors.
A word of length  is full if it has  palindrome factors and a factorial set is \emph{full} (or rich) if all its elements are full.
By a result of~\cite{GlenJustinWidmerZamboni2009}, a recurrent set closed under reversal is full if and only if every complete return word to a palindrome in  is a palindrome (a complete return word to a set  of words of the same length is a word of  which has exactly two factors in , one as a proper prefix and one as a proper suffix, see Section~\ref{subsec:cardret}).
It is known that all Sturmian sets are full~\cite{DroubayJustinPirillo2001} and also all natural codings of interval exchange defined by a symmetric permutation \cite{BalaziMaskovaPelantova2007}.

The fact that a tree set of characteristic  is full in the following result generalizes results of~\cite{DroubayJustinPirillo2001,BalaziMaskovaPelantova2007}.

\begin{proposition}
\label{propositionfull}
Let  be a recurrent tree set of characteristic , closed under reversal.
Then  is full.
\end{proposition}
\begin{proof}
We use the following equivalent definition of full sets (see~\cite{PelantovaStarosta2014}): for any ,
\begin{enumerate}
\item[(i)]
if  is not a palindrome, it is neutral. 
\item[(ii)] Otherwise,  is equal to the number of letters  such that  is a palindrome in  (the so-called \emph{palindromic extensions}).
\end{enumerate}
Since  is a tree set of characteristic , every word is neutral.
We thus only have to show that every palindrome has exactly one palindromic extension.
Let  be a palindrome.
It may be verified that since  is palindrome and  is closed under reversal, the graph  is closed under reversal in the sense that it contains an edge  if and only if it contains the edge .
One may verify that, as a consequence, there is at least one  such that .
Indeed, this can be proved as follows by induction on .
It is true if .
Otherwise, let  be such that  is a leaf of .
Then, since the graph is closed under reversal, the vertex  is also a leaf.
Set .
The restriction of the graph to the vertices in  is a tree closed under reversal, and thus the property follows by induction.
But if there is another one, the graph would have a cycle.
Indeed, assume that .
Consider a simple path  of minimal length from one of  to one of .
This path cannot contain the edges corresponding to .
Using these edges and the symmetric of , one obtains a cycle.
Thus  is full.
\end{proof}

In~\cite{PelantovaStarosta2014}, this notion was extended to that of -full, where  is a finite group of morphisms and antimorphisms of  (an antimorphism is the composition of a morphism
and reversal) containing at least one antimorphism.
As one of the equivalent definitions, a set  closed under  is -full if for every , every complete return word to the -orbit of  is fixed by a nontrivial element of .

Let us consider a tree set  of characteristic  and a specular set  obtained as the image of  by a doubling map .

Let us define the antimorphism  for .
From Section~\ref{sec:doubling} it follows that both edges  and  are in the doubling transducer.
Let us define also the morphism  obtained by replacing each letter  by  if there are edges  and  in the doubling transducer.

We denote by  the group generated by the  and .
Actually, we have .

\begin{example}
\label{ex:h2}
Let  be the specular set defined in Example~\ref{ex:fibodouble2}.
The group  is generated by

and

Note that, even if the images of  and  over the alphabet are the same, the latter is a morphism, while the first is an antimorphism.
Moreover, in that case, we have  for every .
\end{example}

\begin{example}
\label{ex:h}
Let  be the recurrent specular set defined in Example~\ref{exampleFiboDouble}.
The group  is generated by the antimorphism

and the morphism

We have , where  is the antimorphism fixing  and exchanging  and .

\end{example}

We now connect the notions of fullness and -fullness, proving an analogous result of Proposition~\ref{propositionfull} for specular sets.

\begin{proposition}
\label{pro:Hfull}
Let  be a recurrent tree set of characteristic  on the alphabet , closed under reversal and let  be the image of  under a doubling map.
Then  is -full.
\end{proposition}
\begin{proof}
By Proposition~\ref{propositionfull} we know that  is full.

To show that  is -full, we will use several properties of the map .
We note that it is injective, that it preserves prefixes and conversely:  is a prefix of  if and only if  is a prefix of .
Also, for any  and , the images of  by  form the -orbit of .

Consider  and a word  which is a complete return word to the -orbit of .
We may assume that  is a prefix of  and that  is a prefix of , with .
Let  and  be such that  and .
Then  is a prefix of .

We first show that  is a palindrome.
First observe that  has a suffix in the set .
Indeed, if  then  is a suffix of . Otherwise, if , one has that  is a suffix of .
Let now  be the longest palindrome prefix of .
Then  is a prefix of  since otherwise  would have a second occurrence in  (in a full set, the longest palindrome prefix of a word is unioccurrent, see~\cite{GlenJustinWidmerZamboni2009}).
Consequently  is a suffix of  and  cannot have another occurrence of  or  except as a prefix or a suffix (otherwise,  would have an internal factor in the -orbit of ).
Thus  is a complete return word to .
Consequently,  is a complete return word to the -orbit of  and thus , which implies that  and that  is a palindrome.

Now, the -orbit of any word  with  palindrome has two elements.
Indeed, either  is even and , or  is odd and .
Thus such a  is fixed by a nontrivial element of .
\end{proof}

\begin{example}
\label{ex:fibodoubleh2}
Let  be the specular set of Example~\ref{ex:fibodouble2}.
Since it is a doubling of the Fibonacci set (which is Sturmian and thus full), it is -full with respect to the group  generated by the antimorphism  and the morphism  of Example~\ref{ex:h2}.
The -orbit of  is the set .
The set of complete return words to  (see also Section~\ref{sec:return}) is given by

The four words are palindromes and thus they are fixed by .

As another example, consider .
Its -orbit is the set  and the set of complete return words to  is given by

Each of them is a palindrome, thus is fixed by .
\end{example}

\begin{example}
Let  be the specular set of Example~\ref{exampleFiboDouble}.
Since it is  a doubling of the Fibonacci set (which is Sturmian and thus full), it is -full with respect to the group  generated by the map  taking the inverse (that is fixing  and exchanging  and ) and the morphism  (which exchanges  and  respectively).
The -orbit of  is the set .
We have

The four words are fixed by .
 As another example, consider . Then
 and .
Each of them is fixed by some nontrivial element of .
\end{example}










\section{Linear involutions}
\label{sectionInvolutions}
In this section we define linear involutions and connections.
We prove that the natural coding of a linear involution without connections is a specular set (Theorem~\ref{theoremInvolutionSpecular}).





\subsection{Definition}
Let  be an alphabet of cardinality  with an involution  and the corresponding specular group . 
Note that we allow
 to have fixed points. 
This leads to a definition of linear involutions which is somewhat more general than the one used in~\cite{DanthonyNogueira1988,BertheDelecroixDolcePerrinReutenauerRindone2014}.


We consider two copies  and
 of an open interval  of the real line and denote  .
We call the sets  and  the two
\emph{components} of . We consider each component as an open interval.


A \emph{generalized permutation} on  of type , with ,
  is a bijection .
We represent it by a two line array

A \emph{length data} associated with  is a nonnegative
vector  such that


We consider a partition of  (minus 
points) in  open intervals
 of lengths 
and a partition of  (minus  points) in  open intervals
 of lengths . Let  be the set of  \emph{division points} separating
the intervals  for .


The \emph{linear involution} on  relative to these data is the
map  defined on the set
 as 
the composition
of two involutions defined as follows. 
\begin{enumerate}
\item[(i)]The first involution  is defined on .
It is such that for each , its restriction to 
is either a translation or a symmetry from  onto .

\item[(ii)]The second involution exchanges the two components of
  .
It  is defined for 
by . The image of  by 
is called the \emph{mirror image} of .
\end{enumerate}
We also say that  is a linear involution on  and relative to
 the alphabet 
or that it is a -linear involution to express the fact
that the alphabet  has  elements.

\begin{example}\label{exampleLinear}
Let  and

Let  be the -linear involution corresponding to the length data
represented in Figure~\ref{figureLinear} (we represent 
above ) with the assumption that the restriction
of  to  and  is a symmetry while its restriction
to  is a translation.


\begin{figure}[hbt]
\centering
\gasset{Nadjust=wh,AHnb=0}
\begin{picture}(115,17)(0,-1)
\node(h0)(0,10){}\node(h1)(15,10){}\node(h2)(45,10){}\node(h3)(60,10){}\node(h4)(100,10){}\node[Nframe=n](0)(110,10){}
\node(b0)(0,0){}\node(b1)(40,0){}\node(b2)(55,0){}\node(b3)(85,0){}\node(b4)(100,0){}\node[Nframe=n](1)(110,0){}
\gasset{Nh=.1,Nw=.1,Nadjust=n}
\node[ExtNL=y,Nh=.6,Nw=.6,Nfill=y,NLangle=-90,NLdist=2](z)(3,10){}\node[Nh=.6,Nw=.6,Nfill=y](sz)(57,10){}\node[ExtNL=y,Nh=.6,Nw=.6,Nfill=y,NLangle=-90,NLdist=2](Tz)(57,0){}
\node[Nh=.6,Nw=.6,Nfill=y](sTz)(17,10){}\node[ExtNL=y,Nh=.6,Nw=.6,Nfill=y,NLangle=-90,NLdist=2](TTz)(17,0){}
\node(m)(35,5){}




\drawedge[linecolor=red,linewidth=1,ELpos=70](h0,h1){}
\drawedge[linecolor=blue,linewidth=1](h1,h2){}
\drawedge[linecolor=magenta,linewidth=1](h2,h3){}
\drawedge[linecolor=forestgreen,linewidth=1](h3,h4){}
\drawedge[linecolor=green,linewidth=1,ELpos=60](b0,b1){}
\drawedge[linecolor=golden,linewidth=1](b1,b2){}
\drawedge[linecolor=cyan,linewidth=1](b2,b3){}
\drawedge[linecolor=yellow,linewidth=1](b3,b4){}
\gasset{AHnb=1}
\drawedge[AHnb=1,curvedepth=7](z,sz){}\drawedge[AHnb=1](sz,Tz){}
\drawedge[curvedepth=-3,AHnb=0](Tz,m){}\drawedge[curvedepth=3](m,sTz){}\drawedge(sTz,TTz){}
\end{picture}
\caption{A linear involution.}\label{figureLinear}
\end{figure}
We indicate on the figure the effect of the transformation  on a point
 located in the left part of the interval . The point
 is located in the right part of  and the point
 is just below on the left of .
 Next, the point  is located on the left part of 
and the point  just below.
\end{example}
Thus the notion of linear involution is an extension of the notion 
of  interval exchange transformation in the following sense.
Assume that 
\begin{enumerate}
\item[(i)] , 
\item[(ii)] for each letter , the interval
 belongs to  if and only if  belongs
to ,  
\item[(iii)] the restriction of 
to each subinterval is a translation. 
\end{enumerate}
Then, the restriction of 
to  is an interval exchange (and so is its restriction to 
 which is the inverse of the first one). Thus,
in this case,  is a pair of mutually inverse interval exchange transformations.

It is also an extension of the notion of interval exchange with flip
\cite{Nogueira1989,NogueiraPiresTroubetzkoy2013}. Assume again  conditions (i)
and (ii), but now that the restriction of 
to at least one  subinterval is a symmetry. Then the restriction of 
to  is an interval exchange  with flip.


Note that for convenience we consider in this paper interval exchange transformations
defined by a partition of an open interval  minus
 points in 
 open intervals. The usual notion of interval exchange transformation
uses a partition of a semi-interval in a finite number of semi-intervals.


A linear involution  is a bijection from 
onto .
Since  are involutions and ,
 the inverse of 
is .

The set  of division points is also the set of singular points
of  and their mirror images are the singular points of 
(which are the points where  (resp. ) is not defined).
Note that these singular points  may be `false' singularities, in the sense
that  can have a continuous extension to an open neighborhood of .



Two particular cases of linear involutions deserve attention.

A linear involution  on the alphabet  
relative to a generalized permutation  of type
 
is said to be \emph{nonorientable} if there are indices  such that
  (and thus indices 
such that ).  In other words,  there is  some  for which  
and  belong to  the same component  of . Otherwise
 is said to be \emph{orientable}.  


A linear involution  on 
relative to the alphabet 
is said to be \emph{coherent} if, for each , the restriction
of  to  is a translation if and only if 
and  belong to distinct components of . 

\begin{example}
The linear involution of Example~\ref{exampleLinear} is coherent.
\end{example}

Linear involutions which are orientable and coherent
correspond to interval exchange transformations, 
whereas  orientable but noncoherent  linear
involutions are interval exchanges with flip.

Orientable linear involutions correspond to orientable laminations (see
\cite{BertheDelecroixDolcePerrinReutenauerRindone2014}), whereas 
 coherent
linear involutions correspond to orientable surfaces. Thus coherent
nonorientable involutions correspond to nonorientable laminations
on orientable surfaces.





\subsection{Minimal involutions}

A \emph{connection} of a linear involution  is a triple  
where  is a singularity of ,  is a singularity of ,
 and . 

\begin{example}\label{exampleConnection}
Let us consider the linear involution  which is the same as in Example~\ref{exampleLinear} but such that the restriction of  to  
is a symmetry. Thus  is not coherent. We assume that ,
that . Let  and .

Then  is a singularity of  ( is the left endpoint
of ),  is a singularity of  (it is the right endpoint of )
and . Thus  is a connection.
\end{example}


\begin{example}\label{exampleInvolution3}
Let  be the  linear involution on 
represented in Figure~\ref{figureLinear3}. We assume that the
restriction of  to  is a translation
whereas the restriction to  and  is a symmetry.
We choose  for the length of the interval 
(or ). With this choice,  has no connections.

\begin{figure}[hbt]
\centering
\gasset{AHnb=0,Nadjust=wh}
\begin{picture}(100,15)
\node(h0)(0,10){}\node(b)(23.6,10){}\node(bbar)(61.8,10){}\node(h1)(100,10){}
\node(b0)(0,0){}\node(cbar)(38.2,0){}\node(abar)(76.4,0){}\node(b1)(100,0){}

\drawedge[linecolor=red,linewidth=1](h0,b){}
\drawedge[linecolor=blue,linewidth=1](b,bbar){}
\drawedge[linecolor=cyan,linewidth=1](bbar,h1){}
\drawedge[linecolor=forestgreen,linewidth=1](b0,cbar){}
\drawedge[linecolor=green,linewidth=1](cbar,abar){}
\drawedge[linecolor=magenta,linewidth=1](abar,b1){}
\end{picture}
\caption{A linear involution without connections.}\label{figureLinear3}
\end{figure}
\end{example}


Let  be a  linear involution without connections. Let 

be respectively the negative orbit of the singular points and its
closure under mirror image.
Then  is a bijection from  onto itself.
Indeed, assume that . If  then
. Next if , then
 for some . We cannot
have  since  is not in the image of .
Thus . Therefore in both cases . The converse
implication
is proved in the same way.

 A linear involution  
on  without connections
is  minimal if for any point  
the nonnegative orbit of   is dense in . 

Note that when a linear involution is  orientable, that is,  when it is a pair
of interval exchange transformations (with or without flips),   the interval exchange transformations
can be minimal although the linear involution is not since each component
of  is stable by the action of . 
Moreover, it is shown in~\cite{DanthonyNogueira1990} that noncoherent linear 
involutions are almost surely not minimal.

\begin{example}\label{exampleNonCoherent}
Let us consider the noncoherent linear involution  which is the same as in Example~\ref{exampleLinear} but such that the restriction of  to  
is a symmetry, as in Example~\ref{exampleConnection}. We assume that ,
that  and that 
 and that .
 Let  and  (see Figure~\ref{figureLinearnonCoherent}).
We have then , showing that  is not minimal. Indeed,
since , we have
. Since
 we have .
Finally, since , we obtain 
and thus .
\begin{figure}[hbt]
\centering
\gasset{Nadjust=wh,AHnb=0}
\begin{picture}(100,18)(0,-2)
\node(h0)(0,10){}\node(h1)(15,10){}\node(h2)(45,10){}\node(h3)(60,10){}\node(h4)(100,10){}
\node(b0)(0,0){}\node(b1)(40,0){}\node(b2)(55,0){}\node(b3)(85,0){}\node(b4)(100,0){}
\gasset{Nh=.6,Nw=.6,Nfill=y,Nadjust=n,ExtNL=y}
\node[NLdist=2](z)(90,10){}\node[Nh=.1,Nw=.1](m)(50,5){}\node(sz)(10,0){}\node[NLangle=-140,NLdist=2](Tz)(10,10){}\node(sTz)(50,10){}\node[ExtNL=y,NLangle=-90,NLdist=2](TTz)(50,0){}\node(sTTz)(90,0){}
\drawedge[linecolor=red,linewidth=1,ELpos=70](h0,h1){}
\drawedge[linecolor=blue,linewidth=1](h1,h2){}
\drawedge[linecolor=magenta,linewidth=1](h2,h3){}
\drawedge[linecolor=forestgreen,linewidth=1](h3,h4){}
\drawedge[linecolor=green,linewidth=1,ELpos=60](b0,b1){}
\drawedge[linecolor=yellow,linewidth=1](b1,b2){}
\drawedge[linecolor=cyan,linewidth=1](b2,b3){}
\drawedge[linecolor=golden,linewidth=1](b3,b4){}
\gasset{AHnb=1}
\drawedge[curvedepth=3,AHnb=0](z,m){}\drawedge[curvedepth=-3](m,sz){}
\drawedge(sz,Tz){}\drawedge[curvedepth=4](Tz,sTz){}\drawedge(sTz,TTz){}
\drawedge[curvedepth=-4](TTz,sTTz){}\drawedge(sTTz,z){}
\end{picture}
\caption{A noncoherent linear involution.}\label{figureLinearnonCoherent}
\end{figure}

\end{example}

The following result (already  proved in~\cite[Proposition 4.2]{BoissyLanneau2009}
 for the class of coherent involutions)
is~\cite[Proposition 3.7]{BertheDelecroixDolcePerrinReutenauerRindone2014}.
The proof uses Keane's theorem proving that an
interval exchange transformation without connections is minimal~\cite{Keane1975}. 
\begin{proposition}\label{propBL}
Let  be a linear involution without connections on . If 
is nonorientable, it is minimal. Otherwise, its restriction to each component
of  is minimal. 
\end{proposition}





\subsection{Natural coding}
Let  be a linear involution on , let  and let
 be the set defined by Equation~\eqref{eqO}.

Given , the \emph{infinite natural
  coding} of  relative to  is the infinite word

on the alphabet  defined by

We first observe that the infinite word  is
reduced. Indeed, assume that  and  with . Set  and . Then 
and . But  with . Since , we have . This implies that
 and  belong to the same component of , a contradiction.

We denote by  the set of factors of the infinite natural codings
of . We say that  is 
the \emph{natural coding} of . 

\begin{example}
Let  be the linear involution of Example~\ref{exampleInvolution3}.
The words of length at most  of  are represented 
in Figure~\ref{figureSetS}.
\begin{figure}[hbt]
\centering\gasset{Nadjust=wh,AHnb=0}
\begin{picture}(120,50)
\put(0,0){
\begin{picture}(60,40)(0,-5)
\node(1)(0,20){}
\node(a)(15,40){}
\node(b)(15,20){}\node(bbar)(15,0){}
\node(abbar)(30,40){}
\node(babar)(30,30){}\node(bcbar)(30,20){}
\node(bbarc)(30,10){}\node(bbarcbar)(30,0){}
\node(abbarc)(45,45){}\node(abbarcbar)(45,35){}
\node(babarc)(45,30){}\node(bcbara)(45,20){}
\node(bbarcb)(45,10){}
\node(bbarcbara)(45,5){}\node(bbarcbarb)(45,-5){}

\drawedge(1,a){}\drawedge(1,b){}\drawedge(1,bbar){}
\drawedge(a,abbar){}
\drawedge(b,bcbar){}\drawedge(b,babar){}
\drawedge(bbar,bbarc){}\drawedge(bbar,bbarcbar){}
\drawedge(abbar,abbarc){}\drawedge(abbar,abbarcbar){}
\drawedge(babar,babarc){}\drawedge(bcbar,bcbara){}
\drawedge(bbarc,bbarcb){}
\drawedge(bbarcbar,bbarcbara){}\drawedge(bbarcbar,bbarcbarb){}
\end{picture}
}
\put(60,0){
\begin{picture}(60,40)
\node(1)(0,20){}
\node(abar)(15,40){}\node(c)(15,20){}\node(cbar)(15,10){}
\node(abarc)(30,40){}
\node(cb)(30,30){}\node(cbbar)(30,20){}
\node(cbara)(30,10){}\node(cbarb)(30,0){}
\node(abarcb)(45,50){}\node(abarcbbar)(45,40){}
\node(cbabar)(45,35){}\node(cbcbar)(45,25){}
\node(cbbarcbar)(45,20){}
\node(cbarabbar)(45,10){}\node(cbarbabar)(45,0){}

\drawedge(1,abar){}\drawedge(1,c){}\drawedge(1,cbar){}
\drawedge(abar,abarc){}
\drawedge(c,cb){}\drawedge(c,cbbar){}
\drawedge(cbar,cbara){}\drawedge(cbar,cbarb){}
\drawedge(abarc,abarcb){}\drawedge(abarc,abarcbbar){}
\drawedge(cb,cbabar){}\drawedge(cb,cbcbar){}
\drawedge(cbbar,cbbarcbar){}
\drawedge(cbara,cbarabbar){}\drawedge(cbarb,cbarbabar){}
\end{picture}
}
\end{picture}
\caption{The words of length at most  of .}\label{figureSetS}
\end{figure}

The set  can  actually be defined directly as the set of factors
of the substitution

which extends to an automorphism of the free group on 
(see~\cite{BertheDelecroixDolcePerrinReutenauerRindone2014}).

\end{example}


The following is Proposition 5.3 in~\cite{BertheDelecroixDolcePerrinReutenauerRindone2014}.
\begin{proposition} \label{prop:inverse}
The natural coding of a linear involution is closed under taking inverses.
\end{proposition}

We prove the following result.
\begin{theorem}\label{theoremInvolutionSpecular}
The natural coding of a linear involution without connections
 is a specular set.
\end{theorem}
\begin{proof}
Let  be a linear involution without connections. By Proposition~\ref{prop:inverse}, the set  is symmetric. Since it is by definition biextendable
and formed of reduced words,
it is a laminary set. 
By~\cite[Theorem 9.5]{DolcePerrin2016},  is a tree set of characteristic . Thus  is specular.
\end{proof}

We now present an example of a linear involution on an alphabet  where the involution  has fixed points.

\begin{example}
\label{exampleFiboDoubleInvolution}
Let  be as in Example~\ref{exampleSpecularGroup} (in particular, , , ).
\begin{figure}[hbt]
\centering
\gasset{AHnb=0,Nadjust=wh}
\begin{picture}(100,15)
\node(h0)(0,10){}\node(bbar)(61.8,10){}\node(h1)(100,10){}
\node(b0)(0,0){}\node(c)(38.2,0){}\node(b1)(100,0){}

\drawedge[linecolor=red,linewidth=1](h0,bbar){}
\drawedge[linecolor=yellow,linewidth=1](bbar,h1){}
\drawedge[linecolor=blue,linewidth=1](b0,c){}
\drawedge[linecolor=green,linewidth=1](c,b1){}
\end{picture}
\caption{A linear involution on .}\label{figureLinearFiboDouble}
\end{figure}
Let  be the linear involution represented in Figure~\ref{figureLinearFiboDouble} with  being a translation on  and a symmetry on .
Choosing  for the length of , the involution is without connections. 
Thus  is a specular set.
Let us show it is equal to the specular set obtained by the doubling transducer in Example~\ref{exampleFiboDouble}.
Indeed, consider the interval exchange  on the interval  represented in Figure~\ref{figureFibonacciDouble} on the right, which is obtained by using two copies of the  interval exchange  defining the Fibonacci set (represented in Figure~\ref{figureFibonacciDouble} on the left).

\begin{figure}[hbt]
\centering\gasset{AHnb=0,Nh=2,Nw=2,ExtNL=y,NLdist=2}
\begin{picture}(100,15)
\put(0,0){
\begin{picture}(40,15)
\node(h0)(0,10){}\node(b)(20.6,10){}\node(h1)(33.3,10){}
\node(b0)(0,0){}\node(a)(12.7,0){}\node(b1)(33.3,0){}

\drawedge[linecolor=red,linewidth=1](h0,b){}\drawedge[linecolor=blue,linewidth=1](b,h1){}
\drawedge[linecolor=blue,linewidth=1](b0,a){}\drawedge[linecolor=red,linewidth=1](a,b1){}
\end{picture}
}
\put(40,0){
\begin{picture}(40,15)
\node(h0)(0,10){}\node(b)(20.6,10){}\node(h1)(33.3,10){}
\node(d)(53.9,10){}\node(h2)(66.6,10){}

\node(b0)(0,0){}\node(cbar)(12.7,0){}\node(b1)(33.3,0){}
\node(abar)(46,0){}\node(b2)(66.6,0){}

\drawedge[linecolor=red,linewidth=1](h0,b){}\drawedge[linecolor=blue,linewidth=1](b,h1){}
\drawedge[linecolor=green,linewidth=1](h1,d){}
\drawedge[linecolor=yellow,linewidth=1](d,h2){}

\drawedge[linecolor=yellow,linewidth=1](b0,a){}
\drawedge[linecolor=green,linewidth=1](a,b1){}
\drawedge[linecolor=blue,linewidth=1](b1,abar){}
\drawedge[linecolor=red,linewidth=1](abar,b2){}
\end{picture}
}
\end{picture}
\caption{Interval exchanges  and  for the Fibonacci set and its doubling.}
\label{figureFibonacciDouble}
\end{figure}

Let  and let  be the
map defined by
2-z,1)&\text{otherwise.}
\end{cases}

\CR_S(a)&=&\{abca,abcda,acda\}\\
\CR_S(b)&=&\{bcab,bcdacdab,bcdacdacdab\}\\
\CR_S(c)&=&\{cabc,cdabc,cdac\}\\
\CR_S(d)&=&\{dabcabcabcd,dabcabcd,dacd\}.

\Card(\CR_S(X)) = \Card(X)+\Card(A)-\chi(S).
\label{formulaComplete}

\Card(\CR_S(X)) = \Card(X)+\Card(A)-2.

\CR_S(\{a,b\})=\{ab,acda,bca,bcda\}.

\Card(Y)=(d+1)(\Card(A)-\chi(S))+\chi(S)=\Card(X)+\Card(A)-\chi(S).

\Card(\CR_S(X\setminus K))&=&\Card(Y)-\Card(K)\\
&=&\Card(X)-\Card(K)+\Card(A)-\chi(S)\\
&=&\Card(X\setminus K)+\Card(A)-\chi(S)\\

\RR_S(a)&=&\{bca,bcda,cda\},\\
\RR_S(b)&=&\{cab,cdacdab,cdacdacdab\},\\
\RR_S(c)&=&\{abc,dabc,dac\},\\
\RR_S(d)&=&\{abcabcd,abcabcabcd,acd\}.

\CR_S(\{a,a^{-1}\})&=&\{ab^{-1}cba^{-1},ab^{-1}cbc^{-1}a,a^{-1}cb^{-1}c^{-1}a,\\
&&\qquad  ab^{-1}c^{-1}ba^{-1},a^{-1}cbc^{-1}a,a^{-1}cb^{-1}c^{-1}ba^{-1}\}\\
\CR_S(\{b,b^{-1}\})&=&\{ba^{-1}cb,ba^{-1}cb^{-1},bc^{-1}ab^{-1},b^{-1}cb,
b^{-1}c^{-1}ab^{-1},b^{-1}c^{-1}b\},\\
\CR_S(\{c,c^{-1}\})&=&\{cba^{-1}c,cbc^{-1},cb^{-1}c^{-1},c^{-1}ab^{-1}c,
c^{-1}ab^{-1}c^{-1},c^{-1}ba^{-1}c\}.

\CR_S(\{b,d\})=\{bcab,bcd,dab,dacd\}.

\MR_S(w)=\RR_S(w)\cup \RR_S(w)^{-1}=\RR_S(w)\cup \RR'_S(w^{-1}).

\MR_S(a)&=&\{b^{-1}cb,b^{-1}cbc^{-1}a,a^{-1}cb^{-1}c^{-1}a
,b^{-1}c^{-1}b,a^{-1}cbc^{-1}a,a^{-1}cb^{-1}c^{-1}b\}\\
\MR_S(b)&=&\{a^{-1}cb,a^{-1}c,c^{-1}a,b^{-1}cb,
b^{-1}c^{-1}a,b^{-1}c^{-1}b\},\\
\MR_S(c)&=&\{ba^{-1}c,b,b^{-1},c^{-1}ab^{-1}c,
c^{-1}ab^{-1},c^{-1}ba^{-1}c\}.

\MR_S(b)=\{cab,c,dac,dab\}

\MR_S(b)=\{c,cab,dab,dac\}.

  V=\{v\in G_\theta\mid Qv\subset HQ\}

X=\{a,ba^{-1}c,bc^{-1},b^{-1}c^{-1},b^{-1}c\}.

a_1wb_1(a_2wb_1)^{-1}a_2wb_2\cdots a_pwb_p(a_1wb_p)^{-1} = \varepsilon,

with  (otherwise ), contradicting the fact that  is a monoidal basis.

Since , we have  and  for all .
By Proposition~\ref{propCANT}, it implies that  for all nonempty words .
Since  is acyclic, we conclude that  is a tree.

Finally, since  is acyclic, and since , the graph  has two connected components which are trees.
\end{proof}



\section*{References}
\addcontentsline{toc}{section}{References}

\bibliographystyle{plain}
\bibliography{specular}



\end{document}
