

\documentclass{llncs} 

\usepackage{amssymb}
\usepackage{amsmath}
\usepackage{psboxit}
\bibliographystyle{splncs}
\usepackage{graphics}

\newcommand{\rr}{{\cal R}}
\newcommand{\pp}{{\cal P}}
\newcommand{\gateway}{Gateway }




\begin{document}

\author{Sara Miner More \and Pavel Naumov}

\institute{ Department of Mathematics and Computer Science\\ McDaniel College, Westminster, Maryland 21157, USA
\email{ \{smore,pnaumov\}@mcdaniel.edu}}
\title{Functional Dependence of Secrets in a Collaboration Network}

\maketitle

\begin{abstract}
A collaboration network is a graph formed by communication channels between parties. Parties communicate over these channels
to establish secrets, simultaneously enforcing interdependencies between the secrets. The paper studies properties of these interdependencies that are induced by the topology of the network. In previous work, the authors developed a complete logical system for one such property, independence, also known in the information flow literature as nondeducibility.  This work describes a complete and decidable logical system for the functional dependence relation between sets of secrets over a collaboration network. The system extends Armstrong's system of axioms for functional dependency in databases.
\end{abstract}


\section{Introduction}
In this paper, we study properties of interdependencies between pieces of information. We call these pieces {\em secrets} to emphasize the fact that they might be unknown to some parties.  Below, we first describe two relations for expressing interdependencies between secrets.  Next, we discuss these relations in the context of collaboration networks which specify the available communication channels for the parties establishing the secrets.


\subsection{Relations on Secrets}
One of the simplest relations between two secrets is {\em functional dependence}, which we denote by . This means that the value of secret  reveals the value of secret . This relation is reflexive and transitive. 
A more general and less trivial form of functional dependence is functional dependence between sets of secrets. If  and  are two sets of secrets, then  means that, together, the values of all secrets in  reveal the values of all secrets in . Armstrong~\cite{a74} presented the following sound and complete axiomatization of this relation:
\begin{enumerate}
\item {\em Reflexivity}: , if ,
\item {\em Augmentation}: ,
\item {\em Transitivity}: ,
\end{enumerate}
where here and everywhere below  denotes the union of sets  and . The above axioms are known in database literature as Armstrong's axioms \cite[p.~81]{guw09}. Beeri, Fagin, and Howard~\cite{bfh77} suggested a variation of Armstrong's axioms that describe properties of multi-valued dependency.

Not all dependencies between two secrets are functional. For example, if secret  is a pair  and 
secret  is a pair , then there is an interdependency between these secrets in the sense that not every value of secret  is compatible with every value of secret . However, neither  nor  is necessarily true. If there is no interdependency between two secrets, then we will say that the two secrets are {\em independent}. In other words, secrets  and  are independent if any possible value of secret  is compatible with any possible value of secret . We denote this relation between two secrets by . This relation was introduced by Sutherland~\cite{s86} and is also known as {\em nondeducibility} in the study of information flow. Halpern and O'Neill~\cite{ho08} proposed a closely related notion called -secrecy. 

Like functional dependence, independence also can be generalized to relate two sets of secrets. If  and  are two such sets, then  means that any consistent combination of values of the secrets in  is compatible with any consistent combination of values of the secrets in . Note that ``consistent combination" is an important condition here, since some interdependency may exist between secrets in set  even while the entire set of secrets  is independent from the secrets in set . A sound and complete axiomatization of this independence relation between sets was given by More and Naumov~\cite{mn10}:
\begin{enumerate}
\item {\em Empty Set}: ,
\item {\em Monotonicity}: ,
\item {\em Symmetry}: ,
\item {\em Public Knowledge}: ,
\item {\em Exchange}: .
\end{enumerate}
The assumption  in the Public Knowledge axiom guarantees that each secret in the set  has a fixed value and, thus, is ``public knowledge". Details can be found in the original work~\cite{mn10}. Essentially the same axioms were shown by Geiger, Paz, and Pearl~\cite{gpp91} to provide a complete axiomatization of the independence relation between sets of random variables in probability theory. 

A complete logical system that combines the independence and functional dependence predicates for {\em single} secrets was described by Kelvey, More, Naumov, and Sapp~\cite{kmns10}:
\begin{enumerate}
\item {\em Reflexivity:} ,
\item {\em Transitivity:} ,
\item {\em Symmetry:} ,
\item {\em Universal Independence:} , 
\item {\em Universal Dependence:} ,
\item {\em Substitution:} ,
\end{enumerate}
where  and , unlike  and  above, stand for single secrets, not sets of secrets.

\subsection{Secrets in Collaboration Networks}

So far, we have assumed that the values of secrets are determined a priori. In the physical world, however, secret values are often generated, or at least disseminated, via interaction between several parties. Quite often such interaction happens over a fixed network. For example, in social networks, interaction between nodes happens along connections formed by friendship, kinship, financial relationship, etc.  In distributed computer systems, interaction happens over computer networks. Exchange of genetic information happens along the edges of the genealogical tree. Corporate secrets normally flow over an organization chart. In cryptographic protocols, it is often assumed that values are transmitted over well-defined channels. On social networking websites, information is shared between ``friends". Messages between objects on an UML interaction diagram are sent along connections defined by associations between the classes of the objects.

We attempt to capture this type of information flow over a graph by the notion of a {\em collaboration network}. Such a network consists of several parties connected by communication channels that form a network with a fixed topology. A pair of parties connected by a channel uses this channel to establish a secret. If the pairs of parties establish their secrets completely independently from other pairs, then possession of one or several of these secrets reveals no information about the other secrets. Assume, however, that secrets are not picked completely independently. Instead, each party with access to multiple channels may enforce some desired interdependency between the secrets it shares with other parties. These ``local" interdependencies between secrets known to a single party may result in a ``global" interdependency between several secrets, not all of which are known to any single party. Given the fixed topology of the collaboration network, we study what global interdependencies between secrets may exist in the system.
\begin{figure}[htbp]
   \centering
	\scalebox{.5}{\includegraphics{computation_graph.pdf}}
   \caption{Collaboration network .}
   \label{computation_graph}
\end{figure}

Consider, for example, the collaboration network  depicted in Figure~\ref{computation_graph}.  Suppose that the parties collaborate according to the following protocol.  Party  picks a random value  from  and sends it to party . Party  picks values  and  from  in such a way that  and sends both of these values to . Party  computes  and sends value  to party . In this protocol, it is clear that the values of  and  will always match. Hence, for this specific protocol, we can say that , but at the same time  and .

Note that in the above example, all channels transmit secret messages in one direction and, thus, the channel network forms a directed graph. However, in the more general setting, two parties might establish the value of a secret through a dialog over their communication channel, with messages traveling in both directions. Thus, in general, we will not assume any specific direction on a channel.



  





\subsection{Data Streams and Collaboration Networks}
In this section, we will consider a more sophisticated example of collaboration network from network coding theory.

\begin{figure}[htbp]
   \centering
	\scalebox{.5}{\includegraphics{butterfly_graph.pdf}}
   \caption{Butterfly network .}
   \label{butterfly_graph}
\end{figure}


Network coding studies methods of attaining maximum information flow in a network where channels have limited throughput. 
A standard example of network coding is given in terms of the butterfly network \cite{acly00} depicted in Figure~\ref{butterfly_graph} as . Suppose that parties  and  generate streams of 1-bit messages  and , respectively, with rate one message per second. They need to transmit both sequences of messages to both  and  using only the available communication channels. Each channel's throughput is one bit per second. Note that any protocol over  that attempts to independently transmit streams of messages  and  will fail due to the limited combined capacity of the three channels connecting parties , , and , with parties , , and .  



The desired result, however, can be easily achieved by a ``network coding" protocol that combines the two streams. 
Under this protocol, at time , party  transmits bit  to both  and . At the same time, party  transmits bit  to both  and . At time 2, party  already possesses bits  and , so can compute the bit  and send it to . At time 3, party  forwards this bit to  and . Note that party  received bit  directly from party , and after receiving  from  one second later,  can reconstruct the value of , since 
 Similarly, party  receives  directly from , and can reconstruct the Boolean value  after receiving the sum from .  For each time , the propagation of bits  and  is carried out in a similar fashion.

The coding protocol described above can be viewed as a protocol over a collaboration network if the whole stream of messages sent over a single channel in the coding network is interpreted as a single message in the collaboration network.  The computation rules
of the coding protocol are viewed as the local conditions of the collaboration network. For example, if the notation  denotes the entire secret value shared between parties  and , and  denotes its -th bit, then, for example, the local condition at party  can be described as 
 
The desired properties of the protocol can be stated in our notation as 
 and
 Other network protocols that deal with data streams, such as, for example, the alternating bit protocol~\cite{bsw69}, can similarly be interpreted in terms of collaboration networks.



\subsection{Network Topology}

The independence and functional dependence examples we have given so far are for a single protocol, subject to a particular set of local interdependencies between secrets. If the topology remains fixed, but the protocol is changed, then secrets which were previously functionally dependent may no longer be so, and vice versa. For example, for network  above, the claim  will no longer be true if, say, party  switches from enforcing the local condition  to enforcing the local condition . In this paper, we study properties of relations between secrets that follow from the topological structure of the network of channels, no matter which specific protocol is used, as long as it is specified in terms of interdependencies between adjacent channels. Examples of such properties for network  are  and . 

In an earlier work~\cite{mn09a}, we gave a complete axiomatic system for the independence relation between single secrets over a collaboration network. In fact, we axiomatized a slightly more general relation  between multiple single secrets.  One can also consider collaboration networks in which a secret is known to any arbitrary subset of parties, rather than a pair of parties.  In a recent paper~\cite{mn10clima}, we generalized the earlier independence results~\cite{mn09a} to this ``hypergraph" setting.
 
In this article, we turn our attention to functional dependence in (non-hypergraph) collaboration networks.  Here, we present a sound and complete logical system that describes the properties of the functional dependence relation  between sets of secrets over any fixed network topology .  This system includes Armstrong's {\em Reflexivity}, {\em Augmentation}, and {\em Transitivity} axioms.  To these, we add a {\em \gateway} axiom. The above-mentioned statement  is an instance of this new axiom for network .  We prove additional statements about different collaboration networks in Section~\ref{sec:example}. 

From the point of view of verification of a specific protocol, the logical calculus introduced in this paper allows us to separate arguments about properties of the protocol itself from the topological properties of the underlying network. For example, since  is a property of network , if the designers of a particular cryptographic protocol over  can guarantee that the value of  can not be reconstructed from the values of  and , then using the axioms of our logical system, one can prove that the value of  is not revealed by the value of  for the same protocol.


\section{Formal Setting}

Throughout this paper, we assume a fixed  infinite alphabet of variables , which we refer to as ``secret variables". By a network topology, we mean a finite graph whose edges, or ``channels", are labeled by secret variables.  We allow loop edges and multiple edges between the same pair of parties. The set of all channels of network  will be denoted by .  One channel may have (finitely) many labels, but the same label can be assigned to only one channel. Given this, we will informally refer to ``the channel labeled with " as simply ``channel ". 

\begin{definition}\label{}
A semi-protocol over a network  is a pair  such that
\begin{enumerate}
\item  is an arbitrary set of ``values" for each channel ,
\item  is a family of predicates, indexed by set  of all parties of the network , which we call ``local conditions".  If  is the list of all channels incident with party , then   is a predicate on .
\end{enumerate}
\end{definition}

\begin{definition}\label{}
A run of a semi-protocol  is a function  such that 
\begin{enumerate}
\item  for any channel , 
\item If  is the list of all channels incident with a party , then predicate  is true.
\end{enumerate}
\end{definition}

\begin{definition}\label{protocol}
A protocol is any semi-protocol that has at least one run.
\end{definition}
The set of all runs of a protocol  is denoted by .

\begin{definition}\label{rank}
A protocol  is called finite if the set  is finite for every . 
\end{definition}

We conclude this section with the key definition of this paper. It is the definition of functional
dependence between sets of channels. 

\begin{definition}\label{dependence}
A set of channels  functionally determines a set of channels , with respect to a fixed protocol , if

\end{definition}

We find it convenient to use the notation  if functions  and  are equal on every argument from set . 
Using this notation, we can say that a set of channels  functionally determines a set of channels  if 




\section{Language of Secrets}

By , we denote the set of all properties of secrets in collaboration network  definable through the predicate .  More formally,  is a minimal set of formulas defined recursively as follows: (i)  for any two finite sets of secret variables (labels of channels in network )  and , formula  is in , (ii) the false constant  is in set , and (iii) for any formulas  and , the implication  is in . As usual, we assume that conjunction, disjunction, and negation are defined through  and . 

Next, we define a relation  between a protocol and a formula from .  Informally,  means that formula  is true under protocol . 
\begin{definition}\label{}
For any protocol  over a network , and any formula , we define the relation  recursively as follows:
\begin{enumerate}
\item ,
\item  if the set of channels  functionally determines set of channels  under protocol ,
\item  if  or .
\end{enumerate}
\end{definition}
In this paper, we study the formulas  that are true under {\em any} protocol  over fixed network . 
Below we describe a formal logical system for such formulas.  This system, like earlier systems defined by Armstrong~\cite{a74}, More and Naumov~\cite{mn09,mn09a,mn10clima} and by Kelvey, More, Naumov, and Sapp~\cite{kmns10},  belongs to the set of deductive systems that capture properties of secrets.  In general, we refer to such systems as {\em logics of secrets}. Since this paper is focused on only one such system, here we call it {\em the Logic of Secrets}. Before stating the axioms of the Logic of Secrets, we need one more technical definition.

By a path in a network, we mean any undirected path in the graph formed by the channels of the network.
We say that a set of channels  is a {\em gateway} between sets of channels  and  if any path from  to  goes through .  We state this more formally below:

\begin{definition}\label{gateway}
Let , , and  be any three sets of channels in . Set  is a gateway between sets  and  if
for any path  in network ,

\end{definition}
Note that in the above definition sets , , and  are not necessarily disjoint. Thus, for example, for any set ,
set  is a gateway between  and itself. Also, note that the empty set is a gateway between any two components of the network that are not connected to one another.

\section{Axioms}

For a fixed collaboration network , the Logic of Secrets, in addition to propositional tautologies and the Modus Ponens inference rule, contains the following axioms:

\begin{enumerate}
\item {\em Reflexivity}: , if ,
\item {\em Augmentation}: ,
\item {\em Transitivity}: ,
\item {\em \gateway}: , if  is a gateway between sets  and  in network .
\end{enumerate}
Recall that the first three of these axioms were introduced by Armstong~\cite{a74}, and they are known in database theory as Armstrong's axioms \cite[p.~81]{guw09}. The soundness of all four axioms will be shown in Section~\ref{sec:sound}.

We use the notation  to state that formula  is derivable from the set of formulas  in the Logic of Secrets for network .


\section{Examples of Proofs}\label{sec:example}
We will give three examples of proofs in the Logic of Secrets. Our first example refers to square collaboration network  depicted in Figure~\ref{square_graph}.

\begin{figure}[htbp]
   \centering
	\scalebox{.5}{\includegraphics{square_graph.pdf}}
   \caption{Network .}
   \label{square_graph}
\end{figure}



\begin{proposition}\label{square example theorem}
.
\end{proposition}

\begin{proof}
Due to the symmetry of the network, it is sufficient to show that .
Note that  is a gateway between sets  and . Thus, by the \gateway axiom,
 implies . On the other hand, by the Augmentation axiom, the assumption  yields . 
By the Transitivity axiom,   and  imply . \qed
\end{proof}
For the second example, consider the linear network  shown in Figure~\ref{linear_graph_5edges}.
\begin{figure}[htbp]
   \centering
	\scalebox{.5}{\includegraphics{linear_graph_5edges.pdf}}
   \caption{Network .}
   \label{linear_graph_5edges}
\end{figure}


\begin{proposition}\label{linear example theorem}
.
\end{proposition}

\begin{proof}
We begin with the assumption that .  Since  is a gateway between sets  and , by the \gateway axiom, .  Next, using the assumption that , the Transitivity axiom yields .  Finally, we note that  is a gateway between  and , and apply the \gateway axiom once again to conclude that .
\qed \end{proof}

Note that the second hypothesis in the example above is significant.  Indeed, imagine a protocol on  where , the set of values allowed on all other channels is , and the local condition at each party  is simply .  In this protocol,  since the value of  on any run clearly determines the (constant) value of .   However, the value of  is of no help in determining the value of , so the conclusion  does not hold.

\begin{figure}[htbp]
\begin{center}
\scalebox{.5}{\includegraphics{benzene_graph.pdf}}
\caption{Network .}
\label{benzene_graph}
\end{center}
\end{figure}
As our final example, we prove a property of hexagonal collaboration network  shown in Figure~\ref{benzene_graph}.

\begin{proposition}\label{}
.
\end{proposition}
\begin{proof}
Note that  is a gateway between sets  and . Thus, by the Gateway axiom,
. Hence, by the assumption, , we have that . Similarly one
can show that  and  using the assumptions  and .

Consider statements  and . By the Augmentation axiom, they, respectively, imply that
 and . Thus, by the Transitivity axiom, .

Now consider  and statement , established earlier.
By the Augmentation axiom, they, respectively, imply that
 and . Thus, by the Transitivity axiom, 
.
\qed
\end{proof}





\section{Soundness}\label{sec:sound}
In this section, we demonstrate the soundness of each of the four axioms in the Logic of Secrets.

\begin{theorem}[Reflexivity]\label{}
, for any protocol  and any .
\end{theorem}
\begin{proof}
Consider any two runs  such that . Thus  for any .\qed
\end{proof}

\begin{theorem}[Augmentation]\label{}
, for any protocol  and any sets of channels , , and .
\end{theorem}
\begin{proof}
Assume  and consider any two runs  such that .
By our assumption, . Therefore, . \qed
\end{proof}

\begin{theorem}[Transitivity]\label{}
, for any protocol  and any sets of channels , , and .
\end{theorem}
\begin{proof}
Assume  and . Consider any two runs  such that .
By the first assumption, . By the second assumption, . \qed
\end{proof}

\begin{theorem}[Gateway]\label{}
, for any protocol  and any gateway  between sets  and .
\end{theorem}
\begin{proof}
Assume  and consider any two runs  such that . We will show that .
Consider the network  obtained by removing from  all channels in set . By the definition of a gateway, no single connected component of network  can contain channels from set  and set  at the same time. Let us divide all connected components of  into two subgraphs  and  such that  contains no channels from  and  contains no channels from . Components that do not contain channels from either  or  can be arbitrarily assigned to either  or .

Next, define a function  on each  as follows:

We will prove that  is a run of protocol . We need to show that  satisfies the local conditions of protocol  at each party . The connected component of  containing a party  either belongs to  or . Without loss of generality, assume that it belongs to . Thus, , the set of all channels in  incident with party , is a subset of . Hence, . Therefore,  satisfies the local condition at party  simply because  does.

By the definition of , we have  and . Together, the first of these statements and the assumption that  imply that . Thus, due to the second statement, . \qed

\end{proof}





\section{Completeness}
In this section, we demonstrate that the Logic of Secrets is complete with respect to the semantics defined above.  To do so, we first describe the construction of a protocol called , which is implicitly parameterized by a collaboration network  and a set  of formulas in .


\subsection{Protocol }
Throughout this section, we will assume that  is a fixed collaboration network, and  is a fixed set of formulas.

\begin{definition}\label{control closure}
For any , we define  to be the set of all channels  such that .
\end{definition}



\begin{theorem}\label{AsubA*}
, for any .
\end{theorem}
\begin{proof}
Let . By the Reflexivity axiom, . Hence, .
\qed \end{proof}

\begin{theorem}\label{ArhdA*}
, for any .
\end{theorem}
\begin{proof}
Let . By the definition of ,  , for any . We will
prove, by induction on , that  for any . 

\noindent {\em Base Case}:  by the Reflexivity axiom.

\noindent {\em Induction Step}: Assume that . By the Augmentation axiom, 

Recall that . Again by the Augmentation axiom, .
Hence, , by (\ref{eq0}) and the Transitivity axiom.
\qed \end{proof}

We now proceed to define our protocol .  We will first specify the set of values  for each channel .  In this construction, the value of each channel  on a particular run will be a function from the set  into the set . 
Thus, for any  and any , we have . 
We will find it more convenient, however, to think about  as a two-argument Boolean function, where . 

Furthermore, we will not allow the value of a channel on a particular run to be just {\it any} function from the set  into .  Instead, for any channel , we will restrict set  so that, for any run , if , then . 


To complete the description of protocol , we will specify the local conditions for each party in the network.
At each party , we define the local condition  as

That is, when two channels are incident with a party  and neither channel is in , the values of the functions assigned to those channels on argument  must match on any given run.

To show that  is indeed a protocol, we only need to show that it has at least one run. Indeed, 
the constant function  trivially satisfies the local condition at every party .

Now that the definition of protocol  is complete, we make the following two claims about its relationship to the given set of formulas .

\begin{theorem}\label{th1}
If , then .
\end{theorem}
\begin{proof} 
Assume  and consider two specific runs of .
The first of these two runs will be the constant run . The second run is defined as

To show that  satisfies the local condition at a party , consider any  and 
any . If , then . If , then, since , we have 
. Thus, . Therefore,  is a run of protocol .

Notice that by Theorem~\ref{AsubA*}, . Thus, by equality~(\ref{r2}),  for any  and any . Hence,  for any  and . Thus, by the assumption that , we have  for any  and . In particular,  for any .
Since, by definition, , we get  for any . By the definition of , this means that
. By the Reflexivity axiom, . By Theorem~\ref{ArhdA*} and the Transitivity axiom, .
\qed 

\end{proof}

\begin{theorem}\label{th2}
If , then . 
\end{theorem}
\begin{proof}
Assume that , but . Thus, there are runs  and  of  such that  for any  and any , but there is  and 
such that 

First, assume that network , obtained from  by the removal of all channels in set , contains a path  connecting channel  with a channel . Thus, this case implicitly assumes that . Let functions  and  on the channels of network  be defined as  and . Due to the local conditions of protocol , all channels along path   must have the same value  of function . The same is also true about function . Therefore,

This is a contradiction with statement (\ref{runs not equal}).

Next, suppose that there is no path in  connecting  with a channel in . Thus, set  is a gateway between sets  and . By the \gateway axiom,

By the Reflexivity axiom, . Recall the assumption . Thus, by the Transitivity axiom, . Taking into account (\ref{eq1}), . By Theorem~\ref{ArhdA*}, . Hence, again by Transitivity, 
. Thus, by Definition~\ref{control closure}, . Hence, by the definition of protocol ,  has value 0 for any run . Therefore, . This is a contradiction with statement (\ref{runs not equal}).
\qed 
\end{proof}

\subsection{Main Result}
Now, we are ready to finish the proof of completeness.

\begin{theorem}\label{}
If , then there is a finite protocol  such that . 
\end{theorem}
\begin{proof}
Assume . Let  be a maximal consistent set of formulas such that . Consider the finite  protocol  parameterized by network  and set of formulas . 
We will show that for any formula ,  if and only if  by induction on the structural complexity
of formula . The base case follows from Theorems~\ref{th1} and~\ref{th2}. The induction case follows from the maximality and consistency of set . To finish the proof of the theorem, select  to be .
\qed \end{proof}

\begin{corollary}\label{}
Binary relation  is decidable.
\end{corollary}

\begin{proof}
This statement follows from the completeness of the Logic of Secrets with respect to {\em finite} protocols and the recursive enumerability of all theorems in the logic. \qed
\end{proof}

\section{Conclusion}

We have presented a complete axiomatization of the properties of the functional dependence relation over secrets on collaboration networks.  In light of previous results capturing properties of the independence relation in the same setting~\cite{mn09a}, it would be interesting to describe properties that connect these two predicates on collaboration networks.  

An example of such a property for the network  in Figure~\ref{linear_graph} is given in the following theorem.
\begin{figure}[htbp]
   \centering
	\scalebox{.5}{\includegraphics{linear_graph.pdf}}
   \caption{Network .}
   \label{linear_graph}
\end{figure}
\begin{theorem}\label{concl thm}
For any protocol  over network , 

\end{theorem}
\begin{proof}
For any two runs  where , we must show that .  The assumption  guarantees that values  and  coexist in some run in ; call this run .  Thus, we have  and .  

Next, we create a new function  which ``glues" together runs  and  at party .  Formally, we define  as 



We claim that function  satisfies the local conditions of protocol , since at each party in , it behaves locally like an existing run.  Indeed, at party ,  matches run , and at parties  and ,  matches run .  At party ,  matches  exactly, since .  Thus, .  To complete the proof, we note that  and .  By the assumption that , we have .  The definition of  is such that , so , as desired.\qed 
\end{proof}

A complete axiomatization of properties that connect the functional dependence relation and the independence relation between secrets on a collaboration network remains an open problem. 

\section{Acknowledgment}

The authors would like to thank Andrea Mills and Benjamin Sapp for discussions of the functional dependence relation on sets of secrets during earlier stages of this work.

\bibliography{../sp}

\end{document}