\documentclass{LMCS}




\usepackage{graphicx,color}
\usepackage{stmaryrd}
\usepackage{amsfonts}

\usepackage{mathtools}
\usepackage{enumerate}
\usepackage{hyperref}


\input xy
\xyoption{all}




\itemsep\smallskipamount\parsep0pt\topsep\medskipamount
\newcounter{lister}
\newcounter{underlister}

\newenvironment{gqenumerate}
{\begin{list}{\arabic{lister}. }{\usecounter{lister}\parsep0pt plus 1pt\itemsep 0pt plus
2pt\topsep 0pt plus 3pt\def\thelister{\arabic{lister}}\listparindent 1.5em}}{\end{list}}\newcommand{\ealist}{\end{gqenumerate}\noindent}

\itemsep\smallskipamount\parsep0pt\topsep\medskipamount
\newcounter{lister2}
\newcounter{underlister2}

\newenvironment{gqitemize}
{\begin{list}{\raisebox{1pt}{}}{\usecounter{lister2}\parsep0pt plus 1 pt\itemsep 0pt plus
2pt\topsep 0pt plus 3pt\def\thelister{\arabic{lister2}}\listparindent 1.5em}}{\end{list}}\newcommand{\ealistb}{\end{gqitemize}\noindent}

\usepackage{color}
\newcommand{\todo}{\textcolor{red}}



\let\pf\proof
\let\epf\endproof


\newtheorem{ax}{Axiom}
\newtheorem{ex}[thm]{Example}
\newtheorem{const}[thm]{Construction}

\newtheorem{defn}{Definition}[section]

\newtheorem{notation}{Notation}

\newcommand{\chum}[3]{\xymatrix @=6mm{ \ar@<0.4ex>[r]^{{#1}_{#3}} & \ar@<0.4ex>[l]^{{#2}_{#3}}}}
\newcommand{\cphi}[2]{\xymatrix @=6mm{ \ar@<0ex>[r]^{{\smash{#1}}_{\smash{#2}}} &}}
\newcommand{\thmref}[1]{Theorem~\ref{#1}}
\newcommand{\secref}[1]{\S\ref{#1}}
\newcommand{\lemref}[1]{Lemma~\ref{#1}}
\newcommand{\propref}[1]{Proposition~\ref{#1}}
\newcommand{\remref}[1]{Remark~\ref{#1}}
\newcommand{\corref}[1]{Corollary~\ref{#1}}
\newcommand{\probref}[1]{Problem~\ref{#1}}
\newcommand{\defref}[1]{Definition~\ref{#1}}
\newcommand{\exref}[1]{Example~\ref{#1}}




\newcommand{\upclose}[1]{\,\uparrow\!\!#1}
\newcommand{\downclose}[1]{\,\downarrow\!\!#1}
\def\Con{C\!on}
\newcommand{\denote}[1]{[\![ #1 ]\!]}
\newcommand{\gq}{\textcolor{red}}



\def\doi{6 (1:3) 2010}
\lmcsheading {\doi}
{1--20}
{}
{}
{Apr.~16, 2007}
{Jan.~14, 2010}
{}   

\begin{document}

\title{Bifinite Chu Spaces}

\author[M.~Droste]{Manfred Droste}
\address{Institute of Computer Science,
Leipzig University,
04158 Leipzig, Germany}
\email{droste@informatik.uni-leipzig.de}

\author{Guo-Qiang Zhang}\
\address{Department of Electrical Engineering and Computer Science,
Case Western Reserve University,
Cleveland, OH 44106, U.S.A.}
\email{gq@case.edu}

\begin{abstract}
   This paper studies colimits of sequences of finite Chu spaces and
   their ramifications.  Besides generic Chu spaces, we consider
   extensional and biextensional variants.  In the corresponding
   categories we first characterize the monics and then the existence
   (or the lack thereof) of the desired colimits.  In each case, we
   provide a characterization of the finite objects in terms of
   monomorphisms/injections.  Bifinite Chu spaces are then expressed
   with respect to the monics of generic Chu spaces, and universal,
   homogeneous Chu spaces are shown to exist in this category.
   Unanticipated results driving this development include the fact that
   while for generic Chu spaces monics consist of an injective first
   and a surjective second component, in the extensional and
   biextensional cases the surjectivity requirement can be dropped.
   Furthermore, the desired colimits are only guaranteed to exist in
   the extensional case.  Finally, not all finite Chu spaces
   (considered set-theoretically) are finite objects in their
   categories.  This study opens up opportunities for further
   investigations into recursively defined Chu spaces, as well as
   constructive models of linear logic.
\end{abstract}

\keywords{Domain theory, approximation, category theory, Chu spaces}
\amsclass{ 03B70, 06A15, 06B23, 08A70, 68P99, 68Q55}
\subjclass{F.3.2}



\maketitle

\section{Introduction}\label{intro}

\noindent Within semantic frameworks for programming languages, a
basic approach to the study of infinite objects is through their
finite approximations. This is true both within an individual domain,
as well as with domains collectively. A salient example of the latter
is Plotkin's approach to SFP~\cite{plotkin}, where a class of domains
is constructed systematically by taking colimits of sequences of
finite partial orders.  An important component of this framework is
the notion of embedding-projection pair, capturing when one partial
order is an approximation of another.  An interesting outcome of this
process is that completeness of an individual domain, the property
that makes a cpo complete, becomes a natural by-product obtained by
taking colimits of finite structures.  In domain theory, the SFP- (or
bifinite) domains now form an important cartesian closed category of
domains, see~\cite{curien}.

In this paper we study colimits of sequences of finite Chu spaces.
This entails the use of monic morphisms (or monomorphisms) as a way to
formulate the substructure relationship.  Such an innocuous attempt
led to striking differentiations of the notion dictated by the
extensionality properties of the underlying spaces. We consider three
base categories of Chu spaces: the generic Chu spaces ({\bf C}), the
extensional Chu spaces ({\bf E}), and the biextensional Chu spaces
({\bf B}).  The main results are: (1) a characterization of monics in
each of the three categories; (2) existence (or the lack thereof) of
colimits and a characterization of finite objects in each of the
corresponding categories using monomorphisms/injections (denoted as
{\bf iC}, {\bf iE}, and {\bf iB}, respectively); (3) a formulation of
bifinite Chu spaces with respect to {\bf iC}; (4) the existence of
universal, homogeneous Chu spaces in this category.  Unanticipated
results driving this development include the fact that: (a) in {\bf
  C}, a morphism  is monic iff  is injective and  is
surjective while for {\bf E} and {\bf B},  is monic iff  is
injective (but  is not necessarily surjective); (b) while colimits
always exist in {\bf iE}, it is not the case for {\bf iC} and {\bf
  iB}; (c) not all finite Chu spaces (considered set-theoretically)
are finite objects in their categories.

Bifinite Chu spaces can be viewed, in an intuitive category-theoretic
sense, as ``countable'' objects which are approximable by the finite
objects of the category. The class of bifinite Chu spaces is very rich
(up to isomorphism, there are uncountably many such spaces). However,
we show that there is a single bifinite Chu space  which contains
any other bifinite Chu space as a subspace. Moreover,  can be
chosen to be homogeneous, i.e. to bear maximal possible degree of
symmetry, and with this additional property  is unique up to
isomorphism.


Our interest in Chu spaces stems from a number of recent developments.
Chu spaces provide a suitable model of linear logic, originating from
a general categorical construction introduced by Barr and his
student~\cite{barr1,barr2}.  The rich mathematical content of Chu
spaces has been extensively illustrated by Pratt and his collaborators
in a variety of settings, ranging from concurrency to logic and
category theory~\cite{pratt1,pratt2,pratt3,pratt4,pratt6,pratt7}.  In
particular, Pratt shows that all small categories can be embedded in
a certain category of Chu spaces~\cite{pratt5}.

Chu spaces are closely related to the topic of Formal Concept
Analysis (FCA~\cite{ganter,zhang-mfps}).  Both areas use the same
objects but the morphisms considered in FCA are different.
Chu spaces are also related to domains~\cite{zhang-mfps}.  In
\cite{lamarche}, a class called casuistries was introduced as a
``continuous'' version of Chu spaces, and yet maintaining the
constructions desired as a model of linear logic. On the other hand,
if instead of Chu transformations, Chu spaces are equipped with what
are called {\em approximable mappings}~\cite{pascal,scott}, one
obtains a cartesian closed category equivalent to the category of
algebraic lattices and Scott continuous functions~\cite{alga}. For
this to work properly, a modified notion of formal concept, called
{\em approximable concept}, needs to be used~\cite{tac}. This way, an
infinite concept can be approximated by finite ones.

Universal objects have played an important role in the development of
domain theory. For example,
the early work of Scott~\cite{scott76} and Plotkin~\cite{plotkin78}
showed that with universal objects, domain equations can be
treated by a calculus of retracts.

By studying Chu spaces that  are colimits of sequences of finite objects,
we hope to understand these spaces from a constructive angle, formulate a notion of
completeness, and study the existence of universal, homogeneous objects.
Recursively defined Chu spaces as well as models of linear logic within bifinite Chu
spaces are some topics worth revisiting in light of this paper.

\smallskip

The rest of the paper is organized as follows. Section 2 recalls basic terminologies and
gives a  characterization of monic morphisms in the categories {\bf C}, {\bf E}, and {\bf B}.
Section 3 studies colimits in the categories  {\bf iC}, {\bf iE}, and {\bf iB}.
Section 4 characterizes finite objects in  {\bf iC}, {\bf iE}, and {\bf iB}.
Section 5 introduces bifinite Chu spaces
and shows the existence of universal, homogeneous bifinite Chu spaces using
finite amalgamation.  


Remark: The shortened conference version of the paper was presented at CALCO 2007 in Bergen, Norway.

\section{Chu spaces and monic morphisms}

\noindent We recall some basic definitions to fix notation,
following~\cite{pratt0}.  Readers interested in more details should
consult~\cite{pratt2}.  In~\cite{pratt2}, the morphisms are called Chu
transformations.

\begin{defn}
A Chu space over a set  is a triple 
where  is a set whose elements can be considered as {\em objects} and  is a set
whose elements can be regarded as {\em attributes}.
The satisfaction relation  is a
function .
A morphism   from a Chu space 
to a Chu space 
is a pair of functions , with
 and 
such that for any  and ,

To alleviate the notational burden, we refer to a morphism
by , and refer to the forward component by
 and the backward component by .
\end{defn}



For all the examples we consider in this paper, .
If  is left unspecified, then it is assumed to contain at least two
elements, denoted as  and .
A Chu space   has two equivalence relations built-in.
One is on the rows, where the -th row corresponds
to a function .
Two rows  are {\em equivalent} if . Similarly,
an equivalence relation exists on columns, defined by equality 
for .
A Chu space   is  called {\em extensional} if  implies ,
i.e.,  does not contain repeated columns. Similarly, a  Chu space 
is {\em separable} if it does not contain repeated rows. Using  topological
analogy, if we think of objects in  as points and attributes in  as open sets,
then separable Chu spaces are those for which distinct points can be differentiated by
the open sets containing them (such spaces are called ).
A Chu space is {\em biextensional} if it is
both separable and extensional.

We denote by {\bf C}  the category of Chu spaces and morphisms defined above, and {\bf E} and
 the full subcategories of extensional and biextensional  Chu spaces, respectively.
Composition of morphisms reduces to functional compositions of the components:
,
noting that the second component goes backwards.
For abbreviation, objects are denoted as  for short, where .
We refer to  the {\em object set}, and  as the {\em attribute set} of , respectively.
As a refinement of an observation in~\cite{lamarche}, we have the following
result which will be useful for subsequent developments of the paper.

\begin{prop}\label{lamarche}
Suppose  are morphisms in {\bf C}.
Then
\begin{enumerate}[\em(1)]
\item if    is extensional, then  implies ;
\item if     is separable, then  implies ;
\item if    and  are biextensional, then  iff .
\end{enumerate}
\end{prop}

Thus,  the forward and backward components in a morphism determine each other
uniquely in the category of biextensional Chu spaces.

\pf
Let us write  and .  First
we show (1). Suppose  is extensional and . Then for all  and  we have

Hence  by extensionality of
. Now (2) follows from (1) by duality, and (1) and (2) imply
(3).  \epf

As a first order of business, we consider monic morphisms,
which capture the notion of a ``substructure''.
In categorical terms, a morphism  is {\em monic} (or mono)
if for any other morphisms  () such that
, we have
.

{\bf Remark}. To make a distinction in our reference to
morphisms at different levels, we reserve the term monic, mono, epi, etc for
Chu spaces, and use one-to-one, onto, injective, surjective for the functions
on the underlying sets. When properties on the underlying functions carry over to
Chu spaces, we occasionally mix the terms.

\begin{prop}\label{CB} We have:
\begin{enumerate}[\em(1)]
\item A morphism  in {\bf C} is monic  iff
 is injective and  is surjective.
\item A morphism  in {\bf E} is monic  iff
 is injective.
\item A morphism   in  is monic  iff
 is injective.
\item Suppose  is a
morphism in  {\bf C} and  is extensional.
If  is surjective, then  is injective.
\end{enumerate}
\end{prop}

\pf
(1) The ``If'' part is straightforward. We check the ``Only If'' part. 
Suppose  is such that for any pair of
morphisms , , if , then .  We show that  is injective and  is
surjective.  Let's write  and . 



First we show that  is surjective. Suppose otherwise. 
Choose two sets  of the same cardinality as  such that  are pairwise
disjoint.  Put  and . 
For , choose a bijection  and let  be the
joint extension of the identity  on  and .  Let .  For  and , let  if ,  if , and  if . 

Let . Then , with , are morphisms.  Indeed,  if , and  also, if , for
 by the definition of .  Moreover,  yield the
same composition with  because s behave the same on
the image set .  Now  being monic implies that
, a contradiction because . Hence .  Note that for this
construction to work,  cannot be required to be
extensional. 

The proof for the injectivity of  is the same as the ``only
if'' part for item (2), given next. 

(2) and (3).  \emph{If.} Let  be a morphism
and  an injection.  Consider two morphisms  (), which yield the same compositions
with .  Then , since 
is injective.  By Prop.~\ref{lamarche}, we have .  Hence  is monic. 

\emph{Only If.} Let    be monic and
write   and  . 
Assume    are such that  . 
Construct a Chu space  as follows. 
Let  , a singleton, and  . 
Also, let   for . Clearly,   is biextensional. 

Now define , , as follows. 
Let  for .  For , put
 .  Then
 are morphisms. 
Clearly, . 
We claim that also . Indeed, if  and , then


Hence  
yield the same compositions with  . 
Since    is monic, it follows that  . 
Thus  , and    is injective. 

(4) Let  with , say. Choose
any  and then  with . Then
. Then  by extensionality.  \epf

The second and third items above would not be so surprising
  if the injectivity of  implied the surjectivity of
   in {\bf B} and {\bf E}.  But this is not the case.

  \begin{ex} Consider , with
   and ; , with . Then the constraints  and  satisfy the property that  and
  the pair  gives rise to a morphism. Clearly  and  is injective, but  is not surjective.
  With respect to items (2) and (3) in the proposition, this means
  that the backward component of a monic morphism in {\bf E} and {\bf
  B} need not be surjective.
\end{ex}


{\bf Remark}. Using a similar proof,
we can show that a morphism  is monic  if and only if
 is surjective, in the category of separable Chu spaces. We omitted
this statement in Prop.~\ref{CB} because we do not consider the category of
separable Chu spaces in the rest of the paper.


\section{Colimits of -Chains}

\noindent We are interested in the subcategories of {\bf C}, {\bf E},
and {\bf B} with monic morphisms, denoted as {\bf iC}, {\bf iE}, and
{\bf iB}, respectively.  Let us begin with an unexpected observation
that colimits do not exist in {\bf iC} in general.  For this purpose,
we recall the definition of colimits, here formulated in {\bf iC}, but
it can easily be seen as an instantiation of a general
notion~\cite{maclane}.  We then show that colimits do exist in {\bf
  iE} and {\bf iB}.

\begin{defn}
An -sequence in   is a family 

\end{defn}



\begin{defn}\label{cocone}
  A cocone from an -sequence 
  to a Chu space  is a family of mappings  such that  for all , i.e., the diagram


commutes.

A cocone  is universal if
for any other cocone 
such that  for all
, there exists a unique  such
that  for all .  Such a
universal cocone, if it exists, is called the colimit of the family , while  is called the mediating
map. In this case we write .
\end{defn}

\begin{thm}\label{non-existence}
Colimits of -chains of finite Chu spaces do not always exist in  {\bf iC}.
\end{thm}

\pf Consider finite Chu spaces , such that  if , and  otherwise. Observe that  is biextensional.  Define
 such that 
for , and , but  otherwise.   It is
straightforward to verify that the s are indeed morphisms: for
all  and ,
 iff  iff  iff
. Hence  is monic.

Consider , with 
iff , and  otherwise. Define  by letting  be inclusions and  if  and  for .  One readily
checks that  is a cocone.
Consider another cocone defined by , where  extends  with  for
all . Define  by
 and letting  extend
 with .  Clearly,  is also a cocone.

Now we can infer that the colimit does not exist. More specifically,
suppose  with  were a colimit.  First consider a mediating
map .  We have 
for some . Then for each  we obtain , thus . Next consider a mediating map .  Since  must be onto,  for some . We obtain
, a contradiction.
\epf

Subsequently we will show that particular -sequences of Chu
spaces do have colimits. For this we provide a generic
construction. It is the standard construction in the category of sets,
assimilated into the context of Chu spaces. We phrase it explicitly
since we will often refer to it.

\begin{const}\label{const:colimit}
  Let  be an -sequence of
  Chu spaces where  and  is the inclusion mapping, for each .
  Consider  where
  

  \noindent Subsequently, we will denote a sequence  often by .

  For each , define  by
   and  for all
   and .
\end{const}

In Construction~\ref{const:colimit}, observe that possibly . Note that the relation  is well-defined since if  and , then  so
; inductively we obtain  for each .

Clearly,  is a morphism.  Then we have, for each ,
  
  and for any ,
  
  Therefore,  and  is indeed a cocone.  We note:

\begin{prop}
  If an -sequence  in  has a colimit, then this colimit is provided, up to
  isomorphism, by the cocone  of Construction~\ref{const:colimit}.
\end{prop}

\pf
Let  have a colimit  in  where . By Proposition~\ref{CB}(1), the mappings 
are injective and the mappings  are surjective, and we
may assume the s to be inclusions. Now construct  and . as in
Construction~\ref{const:colimit}. We claim that each   is a morphism in . By Proposition~\ref{CB}(1), it
remains to show that  is onto. Using that the
s are onto, for any  we can easily find
 with .

Since  is the colimit, there is a unique  in  such that  for
all . Then  is injective. If 
, then  and
, so  is onto. Further, 
is onto, and we claim that  is injective. Let  with .  For
each , then , showing . Hence
 is an isomorphism. \epf

In contrast to Theorem~\ref{non-existence}, we have the following.

\begin{thm}\label{limit-thm}
  Colimits exist in {\bf iE}, as given by Construction~\ref{const:colimit}.
\end{thm}




\pf Let  be an -sequence in
{\bf iE} where  for each . By
Proposition~\ref{CB}(2), the mappings  are injective, and
we may assume the s to be inclusions. Now construct  and  as in
Construction~\ref{const:colimit}. We claim that  is
extensional. Let  and assume that
. We need to show that . Indeed, let  and choose any . Then
.  So
 and thus  as  is
extensional. Hence , and  is
extensional.





For universality, let 
be a cocone, where .  Define  by letting  be such that  if , and letting  be
given as .  Then 
is well-defined because for any , ; also  is well-defined because for any
, , and hence the sequence 
belongs to .  Further,  is a morphism. We have  for all  because , and  for
all  and , by definitions.

The mediating morphism
 is a morphism in {\bf iE} because    is injective,
and by Prop.~\ref{CB}(2), it is monic. The mediating morphism
is unique because its values are fixed by the commutativity requirements of the colimit diagram.
\epf

We now consider the biextensional case.
In order to avoid potential confusion of terminology, we call
a Chu space  with finite  and  a finite Chu structure.
Finite objects in categorical terms will be studied in the next section, as we will
learn that finite objects and finite Chu structures do not always agree.

The proof of the following result involves a typical 
K{\"o}nig's-lemma argument for finite Chu structures which we will encounter
again later.

\begin{thm}\label{limit-thmB}
  Colimits exist in {\bf iB} for -sequences of finite Chu structures. 
  They do not exist in general for -sequences of arbitrary (non-finite) Chu structures in {\bf iB}.
   If for an
  -sequence
in  there is a cocone to some Chu space in {\bf iB}, then
  the sequence has a colimit in {\bf iB}. 
\end{thm}


\pf The proof is similar to that of Theorem~\ref{limit-thm}, except
that we need to show that in case the  constitute a
sequence composed of finite biextensional structures,  is biextensional as well.  Since extensionality is already
shown to be preserved by this limit structure, it remains to show that
separability is preserved.

Suppose all s are separable.  Let  with . Let  be minimal with .  We claim
that  in . The separability of  then gives us . 

Suppose . Then for any  we have
. For each element  such that
 let  be
the sequence defined inductively by  and
 for each . 
Consider the set

Clearly,  is an infinite set.  Consider elements of  as nodes,
and an edge from  to  exists if , and
 is an extension of , that is, . 
Observe that if , , and , then  iff
. So, in this case 
is defined iff  is defined, and then they are connected by
an edge. Since each  is finite,  is a finite branching,
infinite tree.  By K\"{o}nig's Lemma, this tree has an infinite
branch, say, .  Consequently,  for all .  Clearly, by the construction
above, we have  and , a contradiction. 

For the second part of the theorem, we construct a counterexample as
follows. Let  for , with  iff  and
 otherwise. For example,  is displayed below as
a countable matrix, which will be referred to as  for future
reference. 

\begin{center}
\begin{tabular}{| c c c  c  c  c c c c c  |}\hline
   1 & 1 & 1 & 1 & 1 & 1 & 1 &   \multicolumn{3}{c|}{} \\
   0 & 1 & 0 & 1 & 0 & 1 & 0 &   \multicolumn{3}{c|}{} \\
      0 & 0 & 1 & 0 & 0 & 1 & 0 &   \multicolumn{3}{c|}{}\\
            0 & 0 & 0 & 1 & 0 & 0 & 0 &   \multicolumn{3}{c|}{}\\
                        0 & 0 & 0 & 0 & 1 & 0 & 0 &   \multicolumn{3}{c|}{}\\
                         0 & 0 & 0 & 0 & 0 & 1 & 0 &   \multicolumn{3}{c|}{} \\
                               0 & 0 & 0 & 0 & 0 & 0 & 1 &  \multicolumn{3}{c|}{} \\
           \multicolumn{10}{|c|}{} \\ \hline
   \end{tabular}
\end{center}
Intuitively,  is obtained by starting
from the countable matrix  from the -th column. Clearly,  is biextensional. The morphism  is defined by , and
. Then,

and so  is indeed a Chu morphism for each . 


Now assume that  is a cocone
where . We show that then . Suppose
there is . Then, for any ,  by
the definition of . 
Inductively, we have  for all
.  But  for all ,
a contradiction.  Hence  is empty. But then  is not
separable. 

For the last part of the theorem, let  be an -sequence in  with some cocone  in {\bf iB}. By
Theorem~\ref{limit-thm}, the sequence  has a colimit  in
{\bf iE}. We claim that this is also the colimit of the sequence in
{\bf iB}. Now there is a morphism  in {\bf
  iE} making the diagram commute. We show that  is separable. 
Let  and . Choose any  with  . For any  we have
, so  since  is
biextensional, and  as  is injective. 

Hence  is biextensional, and by Proposition~\ref{CB}(2) and
(3), the morphisms   belong
to {\bf iB}, and our claim follows. 
\epf

It is informative to think about the example given in the proof of
Theorem~\ref{non-existence}.  By the colimit construction,  is the colimit both in {\bf
  iE} and in {\bf iB}. 
There is indeed a monic morphism  from  to , where  is the identity
(injection), and  is the inclusion (but not onto). 

Also note that even though Theorem~\ref{limit-thm}
confirms that colimit always exists for extensional Chu spaces,
the counterexample for Theorem~\ref{limit-thmB} shows that
colimits for infinite structures may have weird behaviors with unintended
effects. This invites us to look more into objects constructed as colimits of finite structures,
in the next sections. 

\section{Finite objects}


\noindent In studying patterns of approximation in Chu spaces, finite
objects play an important role since they serve as the basis of
approximation.  In most cases, one expects finite objects to
correspond to finite structures, objects whose constituents are finite
sets.  In categorical terms, finite objects are captured using
colimits in a standard way, and the notion of ``approximation'' is
captured by monic morphisms.  Therefore, we work with categories {\bf
  iC}, {\bf iE}, and {\bf iB}.  However, since Prop.~\ref{CB}
indicates that what counts as monic morphisms depends on
extensionality, the existence of colimits and the characterization of
finite objects are not straightforward set-theoretic generalizations
obtained by treating each component of Chu spaces separately.

We give a characterization of the finite objects of . 
Surprisingly, not all finite structures in  are finite
objects; finite objects are characterized as extensional structures
with finite object set instead. 
The following definition is phrased in ; but as a general
categorical concept it can be made explicit in {\bf iE} and {\bf iB}
as well, and we do not repeat this here. 

\begin{defn}\label{fi-def}
  An object ~ of ~ is finite if for every
  -sequence  of Chu spaces
  having a colimit, for every morphism  in 
  there exist  and a morphism  such that the diagram

commutes, i.e.,  is such that . 
\end{defn}

If  is finite and  is a finite object in
{\bf iC}, one can show that both  and  are finite sets,
i.e.  is a finite Chu space. However, somewhat surprisingly (at
least to us), the converse does not hold, as already simple examples
show, see Example~\ref{finite-example} below. The following result
characterizes the finite objects of {\bf iC}. 

\begin{thm}\label{both-finite}
  An object  is finite in  iff  is
  finite and  is extensional. In this case, ; in particular, if  is finite, so is . 
\end{thm}



\pf \emph{(Only if.)}~  Suppose  is a finite object of
 as in Definition~\ref{fi-def}.  If  is infinite, then we
can write , where .  Let , where  and  is  restricted to the
product .  It can then be checked that  is the colimit of the
-sequence , with
 inclusion and  identity
for all . We have  a
morphism.  Since  is a finite object, there exist an  and , such that . 
Then , a contradiction. 
Therefore,  must be finite. 

Next we show that  is extensional.  Suppose there are
 with  and . 
Put . We may assume that . Now let  with  and . Put
 for each  and
 for each 
and . We let  be the identities, and
 leave everything unchanged except . 
Also, let  be the identity map,  if  or , and  if . Then the
-sequence  has  as its colimit. Now define
 with , 
mapping odd numbers to , even numbers to , and leaving
elements in  unchanged.  Since  is finite, there are an
 and a morphism  such that
. 
Then , contradicting the assumption that . 


\emph{(If)}~ Suppose  in {\bf iC}, where  is
finite,  is extensional and  . Since the colimit of an -sequence is unique up to
isomorphism, we may assume that all s are inclusions and
that  is the
structure  with  given in Construction 3.2.  We can
further assume that  is an inclusion by renaming the
elements of .  As  is a finite set,  implies that  for some . 
Now define  by  if and only if
 satisfies . 

We show that   is a function. For this,
let  such that
.  Observe that .  For any , we have

Therefore,  and by extensionality,  . 

 (where ))

We check that   is a Chu morphism. 
Indeed, for any  and , we have


as required. 
Since  is onto,  is onto, and  is
monic. Finally,
we have  since for any ,
by the definition of , 
Hence , and  is shown to be
finite. 

Finally, let  denote the set of all functions from  into
. Note that  for each , and if
 is extensional, the mapping  provides an
injection of  into , showing  by cardinal arithmetic. \epf


Next we give two examples to illustrate Theorem~\ref{both-finite}. 

\begin{ex}\label{finite-example}
  Let  and , a
  finite Chu space. If , then  is not
  extensional and thus, by Theorem~\ref{both-finite}, not a finite
  object of . To see this more explicitly, one can
  construct a sequence  as in the
  proof of Theorem~\ref{both-finite}, with . 
\end{ex}

\begin{ex}\label{ex:Sig-infinite}
  Only in this example, let  be an arbitrary (possibly
  infinite) set, and let  with
   for each . Then  is extensional, and by Theorem~\ref{both-finite},  is a
  finite object of {\bf iC}. Trivially, if  is infinite, {\sf
    F} is not a finite Chu space. 
\end{ex}



\begin{thm}\label{finite-E-onlyif}
  In the category , if  is finite then 
  is finite. 
\end{thm}

An independent proof is needed even though we follow a similar path as the proof of Theorem~\ref{both-finite}. 
Not only should we make sure that the Chu spaces involved are all extensional, but also the
monic morphisms are characterized differently. 
These entail non-trivial modifications from  the proof of Theorem~\ref{both-finite}. 

\pf Suppose  is a finite object in . 
Suppose  is infinite. Then we can write , where . Fix . Let , where  and  is  restricted to the
product .  Clearly, all
s are extensional.  For morphisms , define  as inclusions, and
 the identity.  By
Theorem~\ref{limit-thm}, the colimit  with  of the sequence  exists, and can be taken as the one
given in Construction~\ref{const:colimit}.  Since each  is a
singleton,  is a singleton as well.  Thus we may assume . 
With  identity and  inclusion, we obtain a
monic morphism  from  to .  Hence there is
a monic morphism  from  to some  which
makes the required diagram commute. But then , a contradiction.  \epf

The converse of Theorem~\ref{finite-E-onlyif} is not true.  To show
this, we adapt the counterexample for the second part of
Theorem~\ref{limit-thmB} as follows. Let , with  iff  
for , and . 
Intuitively,  is obtained by starting from the countable
matrix  from the -th column.  The morphism  is defined by , and , but we keep  constant. 
Then, the colimit of this sequence is , with .  Now let ,
with inclusion and identity paired to form a morphism  from
 to .  There cannot be a morphism  from  to any , because  cannot be defined, simply because 's column contains
two 1s, and each . Hence  is a finite extensional Chu
space and thus a finite object of  but not of
. 



\begin{defn}
  A Chu space  over  is called \emph{discrete}, if
  for any mapping  there is  with . 
\end{defn}

\begin{thm}\label{finite-if}
  In the category ,  is finite iff  is
  finite and  is discrete. 
\end{thm}




\pf {\em (Only if)}.  By Theorem~\ref{finite-E-onlyif}, we know that
 is finite.  Suppose  is not discrete.  Let  be such that  for any .  Let , with  for  and , and  iff 
 for . Furthermore, for all
 and , we let , and for all  and , we let .  Viewed as an
infinite matrix,  is obtained by placing  at the upper-left
corner, and the infinite matrixes used in Theorem~\ref{limit-thmB} on
the lower-right corner. The lower-left corner is filled with zeros,
and the upper-right corner is filled with repeated columns duplicating
.  A rendering of  is given next, 
where  is a  matrix of all
s, and  is the countable matrix used in the proof of
Theorem~\ref{limit-thmB}. 


\begin{center}
\begin{tabular}{c c c  c  c  c c c}
& \multicolumn{3}{c}{} &  \multicolumn{4}{c}{} \\  \cline{2-8}
         \raisebox{-1.5ex}[0pt]{}  &     \multicolumn{3}{|c|}{\raisebox{-1.5ex}[0pt]{}} &
                           \multicolumn{1}{|c|}{\raisebox{-1.5ex}[0pt]{}}
                              & \multicolumn{1}{|c|}{\raisebox{-1.5ex}[0pt]{}} &
                             \multicolumn{1}{|c|}{\raisebox{-1.5ex}[0pt]{}}
                                      & \multicolumn{1}{|c|}{\raisebox{-1.5ex}[0pt]{}}  \\
   &       \multicolumn{3}{|c|}{}   &  \multicolumn{1}{|c|}{}  &   \multicolumn{1}{|c|}{} &
                   \multicolumn{1}{|c|}{} &  \multicolumn{1}{|c|}{} \\   \cline{2-8}
       &       \multicolumn{3}{|c|}{}   &  \multicolumn{4}{|c|}{}  \\
\raisebox{0ex}[0pt]{}  &     \multicolumn{3}{|c|}{} &      \multicolumn{4}{c|}{}  \\
& \multicolumn{1}{|c}{}&\multicolumn{1}{c}{}&\multicolumn{1}{c|}{}&   \multicolumn{4}{c|}{}  \\   \cline{2-8}
   \end{tabular}
\end{center}


We define morphisms  by letting ,
 and  for each .  One can check that up to isomorphism the colimit of this
sequence is  where
 such that  coincides with
 on , and  for each , ; further,  and . Clearly,  is a monomorphism from 
to . Since  is finite, there are 
and a morphism  which make the
diagram commute. Then , and for any  and  we obtain ,
a contradiction. 


{\em (If)}.  We follow a pattern  similar to the ``if'' part for
Theorem~\ref{both-finite}.  Suppose  in {\bf iE},
where  is discrete and  is finite. 
By Theorem~\ref{limit-thm},  we may
assume that 
is the structure   given in Construction~\ref{const:colimit}. 
  We can further assume that  is an
inclusion by renaming its elements.  As  is a finite set,  implies that  for some
.  Since  is discrete, we can define  such that for each ,  is
such that for all , . 
This entails that  is a
Chu morphism. 

To check the condition , note
that for any  and , we have


and by the extensionality of , we have , as required. 
\epf


The following easy remark shows that the structure of discrete
extensional Chu spaces  is very restricted: it
is completely determined,  up to isomorphism, by the cardinality of the
object set . 

\begin{rem}\label{rem:complext-iso}
  Let  and  be two
  discrete extensional Chu spaces with . Then  and
   are isomorphic in {\bf iC}. 
\end{rem}

\pf
Choose a bijection . By the assumption on
, for each  there is a uniquely determined  with . The mapping  with  yields a Chu morphism
, and  is bijective by the
assumption on . 
\epf

Almost similar to Theorem~\ref{finite-if}, we have the following. 
However, an independent proof is needed because
the structures used in the proof for Theorem~\ref{finite-if}
are not biextensional, and an extra case arises.. 


\begin{thm}\label{finite-if-finite}
   In the category ,  is finite iff 
   is finite and  is discrete, or else  is a singleton, , and  is finite.
\end{thm}




\pf (\emph{If.}) ~In the first case, by Theorem 4.5,  is
finite in , and  is biextensional. Note that
a colimit of an -sequence taken in 
coincides with the colimit of this sequence taken in
. Hence  is finite in .

Secondly, assume  is a singleton, , and  is
finite.  Suppose  in .  By Theorem
3.5, we may name  as the
structure  given in Construction 3.2. Since
 is a mapping and , we also obtain
that . Thus  is a singleton, since  is
biextensional. Hence we may assume that  for each .  Since each  is biextensional and  is finite, we
obtain that  is finite, too.

Suppose that  for each .  For each  and each element  define the sequence  as
in the proof of Theorem 3.5, and let again . Then with the extension order,  is an infinite
finite-branching tree since each  is finite, and therefore 
contains, by K\"onig's Lemma, an infinite branch. This implies that , a contradiction.

Hence we have  for some .  Then  makes the diagram
commute, showing that  is finite.

(\emph{Only if.}) ~First we assume that .  Since  is biextensional,  must be a singleton, say, . We
claim that  is finite.  Suppose  was infinite. We may
assume that .
We define Chu spaces    such that
  and    for each    and
. Also, let    be the identity mapping and 
  be the inclusion.
Then    is an -chain in   having    as its
colimit. So there exists some    such  that . In particular,
  is a mapping, which contradicts the assumption 
that .

Next we assume that    and we show that  is 
discrete.

 For this, we refine the arguments employed for
Theorem~\ref{finite-if}. If  is not discrete,
choose  such that  for any .
Now choose an infinite set  of size at least .  Put ; we write  for  for conciseness.

Let   and  where  and   are defined as follows:


On each  , define  and  precisely as
 in  on . Next, choose a bijection . On  define 
and  as ``unit matrix'', i.e.,  for any 
and  let  if , and
 otherwise. In the picture,  is denoted as . 

Further, for all  and  let . On , let  be the same
relation as used in the proof of Theorem~\ref{limit-thmB}. Recall that
 is biextensional. Finally, put  on . 



\begin{center}
\begin{tabular}{c ccc  ccc ccc cc  ccccc}
& \multicolumn{3}{c}{} &  \multicolumn{3}{c}{}  &  \multicolumn{3}{c}{}
                                       &  \multicolumn{2}{c}{}
                                       & \multicolumn{4}{c}{}
\\ \cline{2-16}
\raisebox{-1.5ex}[0pt]{} &  \multicolumn{3}{|c|}{\raisebox{-1.5ex}[0pt]{}}
     & \multicolumn{3}{|c|}{\raisebox{-1.5ex}[0pt]{}} & \multicolumn{3}{|c|}{\raisebox{-1.5ex}[0pt]{}}
                                &  \multicolumn{2}{c}{\raisebox{-1.5ex}[0pt]{}}
                                      &   \multicolumn{1}{|c|}{\raisebox{-1.5ex}[0pt]{}}
                                         &   \multicolumn{1}{|c|}{\raisebox{-1.5ex}[0pt]{}}
                                            &   \multicolumn{1}{|c|}{\raisebox{-1.5ex}[0pt]{}}
                                            & \multicolumn{1}{|c|}{\raisebox{-1.5ex}[0pt]{}} \\  & \multicolumn{1}{|c}{} &    \multicolumn{1}{c}{} & \multicolumn{1}{c|}{}   &  \multicolumn{1}{|c}{} &    \multicolumn{1}{c}{} & \multicolumn{1}{c|}{}     &  \multicolumn{1}{|c}{} &    \multicolumn{1}{c}{} & \multicolumn{1}{c|}{}     &  \multicolumn{1}{c}{}  &  \multicolumn{1}{c}{} &                    \multicolumn{1}{|c|}{} & \multicolumn{1}{|c|}{}  & \multicolumn{1}{|c|}{} & \multicolumn{1}{c|}{} \\  \cline{2-16}
     & \multicolumn{1}{|c}{} &    \multicolumn{1}{c}{} & \multicolumn{1}{c}{}      &  \multicolumn{1}{c}{} &    \multicolumn{1}{c}{} & \multicolumn{1}{c}{}      &  \multicolumn{1}{c}{} &    \multicolumn{1}{c}{} & \multicolumn{1}{c}{}     &  \multicolumn{1}{c}{}  &  \multicolumn{1}{c|}{} &                    \multicolumn{1}{c}{} & \multicolumn{1}{c}{}  & \multicolumn{1}{c}{} & \multicolumn{1}{c|}{} \\  \raisebox{-8ex}[0pt]{}           & \multicolumn{1}{|c}{} &    \multicolumn{1}{c}{} & \multicolumn{1}{c}{}      &  \multicolumn{1}{c}{} &    \multicolumn{1}{c}{} & \multicolumn{1}{c}{\raisebox{-12.5ex}[0pt]{}}      &  \multicolumn{1}{c}{} &    \multicolumn{1}{c}{} & \multicolumn{1}{c}{}     &  \multicolumn{1}{c}{}  &  \multicolumn{1}{c|}{}     &  \multicolumn{1}{c}{}
       &  \multicolumn{1}{c}{}  & \multicolumn{1}{c}{\raisebox{-7.5ex}[0pt]{}}
        &   \multicolumn{1}{c|}{}     \\  & \multicolumn{1}{|c}{} &    \multicolumn{1}{c}{} & \multicolumn{1}{c}{}      &  \multicolumn{1}{c}{} &    \multicolumn{1}{c}{} & \multicolumn{1}{c}{}      &  \multicolumn{1}{c}{} &    \multicolumn{1}{c}{} & \multicolumn{1}{c}{}     &  \multicolumn{1}{c}{}  &  \multicolumn{1}{c|}{} &                    \multicolumn{1}{c}{} & \multicolumn{1}{c}{}  & \multicolumn{1}{c}{} & \multicolumn{1}{c|}{} \\  & \multicolumn{1}{|c}{} &    \multicolumn{1}{c}{} & \multicolumn{1}{c}{}      &  \multicolumn{1}{c}{} &    \multicolumn{1}{c}{} & \multicolumn{1}{c}{}      &  \multicolumn{1}{c}{} &    \multicolumn{1}{c}{} & \multicolumn{1}{c}{}     &  \multicolumn{1}{c}{}  &  \multicolumn{1}{c|}{} &                    \multicolumn{1}{c}{} & \multicolumn{1}{c}{}  & \multicolumn{1}{c}{} & \multicolumn{1}{c|}{} \\  \cline{13-16}
        \raisebox{-4.5ex}[0pt]{}           & \multicolumn{1}{|c}{} &    \multicolumn{1}{c}{} & \multicolumn{1}{c}{}      &  \multicolumn{1}{c}{} &    \multicolumn{1}{c}{} & \multicolumn{1}{c}{}      &  \multicolumn{1}{c}{} &    \multicolumn{1}{c}{} & \multicolumn{1}{c}{}     &  \multicolumn{1}{c}{}  &  \multicolumn{1}{c|}{}     &  \multicolumn{1}{c}{}
       &  \multicolumn{1}{c}{}  & \multicolumn{1}{c}{\raisebox{-4.5ex}[0pt]{}}
        &   \multicolumn{1}{c|}{}     \\  & \multicolumn{1}{|c}{} &    \multicolumn{1}{c}{} & \multicolumn{1}{c}{}      &  \multicolumn{1}{c}{} &    \multicolumn{1}{c}{} & \multicolumn{1}{c}{}      &  \multicolumn{1}{c}{} &    \multicolumn{1}{c}{} & \multicolumn{1}{c}{}     &  \multicolumn{1}{c}{}  &  \multicolumn{1}{c|}{} &                    \multicolumn{1}{c}{} & \multicolumn{1}{c}{}  & \multicolumn{1}{c}{} & \multicolumn{1}{c|}{} \\  & \multicolumn{1}{|c}{} &    \multicolumn{1}{c}{} & \multicolumn{1}{c}{}      &  \multicolumn{1}{c}{} &    \multicolumn{1}{c}{} & \multicolumn{1}{c}{}      &  \multicolumn{1}{c}{} &    \multicolumn{1}{c}{} & \multicolumn{1}{c}{}     &  \multicolumn{1}{c}{}  &  \multicolumn{1}{c|}{} &                    \multicolumn{1}{c}{} & \multicolumn{1}{c}{}  & \multicolumn{1}{c}{} & \multicolumn{1}{c|}{} \\  \cline{2-16}

    \end{tabular}
\end{center}
Observe that for each  either  is the constant-
function, which implies that  is constantly  on
, or else  for infinitely many . It
follows that  and  are biextensional.

Now define morphisms 
and  such that ,  and  are the
identity on , and  for each . Then 
is the colimit of the sequence . 
Next we define a morphism  by letting
 and  for each , . 
Since  is finite, there is a morphism from 
into some , which implies a contradiction by the choice
of . 

Secondly, suppose that  is infinite. For a subset ,
we call two elements  -equivalent if  and
 coincide on .  Now split  with .  For each , let , and let  contain from each -equivalence class
in  exactly one element. 

Let    be    restricted to  , and put

Then    is separable, since    is separable and    intersects
each  -equivalence class, and    is extensional since  
contains from each  -equivalence class at most one element. 

Now let    be the identity, and for  
let   if   lies in the
-equivalence class of  . 
The sequence    has a colimit  
in  {\bf iE}, and clearly  is separable. Also, we may assume that    is
obtained by Construction 3.4; then  . 

There is a unique Chu morphism  from  to  with
 the identity on ; here the existence of 
follows from  being discrete and the uniqueness from 
being extensional.  Hence there are an  and  which make the diagram commute, and this implies a
contradiction about  as in the proof of
Theorem~\ref{both-finite}.  \epf

\section{Bifinite Chu spaces}

\noindent In this section, we will investigate Chu spaces which are,
intuitively and in a category-theoretic sense, countable objects and
approximable by the finite objects in the category. That is, we will
define bifinite Chu spaces as colimits of a sequence of (strongly)
finite Chu spaces. We will then show that this subcategory of
\textbf{iC} contains a universal homogeneous object.


Recall that the finite objects of \textbf{iC} may have an infinite
attribute set, if  is infinite (cf. Example~\ref{ex:Sig-infinite}). 
For technical reasons (cf. the proofs
of Theorem~\ref{thm:colimit-icbif} and Proposition~\ref{amalg}),
we will need that the objects employed here have a finite and
non-empty set of attributes. We will call a space  in
\textbf{iC} \emph{strongly finite}, if  is a finite object
in \textbf{iC} and a finite Chu space with non-empty set of
attributes. Clearly, if  is finite, the finite and the
strongly finite objects of  with non-empty sets of
attributes coincide. 


\begin{defn}
  A Chu space in {\bf iC} is called bifinite if it is isomorphic to
  the colimit (with respect to {\bf iE}) of a chain of stronly finite
  objects in {\bf iC}.  The corresponding full subcategory of bifinite
  Chu spaces of {\bf C} and {\bf iC} are denoted as  and , respectively. 
\end{defn}

As an example, consider the sequence of strongly finite biextensional
spaces  described in the proof of
Theorem~\ref{non-existence}.  As shown there, this sequence has no
colimit in the category .  But by
Theorem~\ref{limit-thm}, the sequence has a colimit with respect to
the category . This colimit thus belongs to
.  Moreover, we will see below in
Theorem~\ref{thm:colimit-icbif}, that this space is also a colimit
of the given sequence with respect to the category . 

Recall that any finite object of  is extensional, hence
any bifinite Chu space is also extensional. It would not be
interesting to formulate the concept of bifinite spaces in \textbf{iE}
or \textbf{iB}, i.e. as colimits of chains of finite objects of
\textbf{iE} resp. \textbf{iB}: By Theorems~\ref{finite-if} and
\ref{finite-if-finite}, these finite objects are discrete. One can show that
colimits of chains of discrete extensional objects are again discrete
and extensional. Hence any two such `bifinite' objects (in \textbf{iE}
or \textbf{iB}) with countably infinite object set are isomorphic by
Remark~\ref{rem:complext-iso}. In contrast, we show that
 is very large:

\begin{prop}
 contains at least
continuum  many non-isomorphic objects. 
\end{prop}

\pf Consider a strictly increasing sequence of finite subsets  of . We
define a sequence  as follows. 
For each , let  with
 and  if  and  otherwise, for any , . We let
 be the inclusion mapping,  if , and . As colimit of this
sequence of strongly finite objects we obtain, up to isomorphism,
 with
,  if
 and  otherwise, for any ,
further , and  inclusion, 
if  and  if .  Note that in  the set  equals  if , and
 if . Hence the bifinite space 
constructed in this way determines the sequence of subsets  uniquely, and two different sequences give rise to
non-isomorphic bifinite spaces.  Since there are continuum  many such
sequences, the result follows.  \epf

\begin{rem}
  By cardinality arguments, one can show that up to isomorphism
   has size ; this equals the
  continuum   if  has size at most continuum. 
\end{rem}

With the restriction of objects to bifinite Chu spaces, colimits now
exist, in contrast to Theorem~\ref{non-existence}. 

\begin{thm}\label{thm:colimit-icbif}
  Colimits exist in . 
\end{thm}

The technical content of the result is that  is
closed in  and in  with respect to taking
colimits of sequences in , and these colimits
taken in  constitute the colimits of the given sequences
with respect to . 

\pf First, let  be an
-sequence in \textbf{iC} with finite objects
, , and each
 being an inclusion. By Theorem~\ref{both-finite} each
 is extensional. Define  and
  as in
Construction~\ref{const:colimit}. By Theorem~\ref{limit-thm}, the cocone
 is a colimit of
 in . We claim that
it is also a colimit of this sequence in . 

First we show that each  is a
morphism in , i.e., that  is onto. 
Choose any . Clearly, since all   are onto, there is a sequence  with  and  for each
. Then  and
, as needed. 



For universality, let  and  be
a cocone.  Define  as in the proof
of Theorem~\ref{limit-thm}. Then  is a mediating morphism in
, and it only remains to show surjectivity of . Choose any .  Since , we can choose a chain of finite
objects  and monics  such
that  is a colimit of
this chain in .  Again we can assume that the
s are inclusions and that the colimit  is given as in
Construction~\ref{const:colimit}.  Each  is a finite
object.  Observing the morphisms , we
obtain a sequence of numbers  and monics  which make the diagram commute, that is,
for each  we have .
For  let  be the composition of  up
to  from  to .  Since the
diagram commutes and the  are monic, we obtain



Now consider the collection  of all finite sequences  such that

and


We claim that there are arbitrarily long sequences. Choose any .  Since  is surjective, there is  satisfying . Now put  for each .  By (*),
inductively we obtain

and our requirements (**) follow. 



Now consider  with the extension order. The sets  are finite
since the  are strongly finite objects. 
So,  is an infinite finite-branching tree. By K\"onig's lemma, there is
an infinite  branch in this tree. Hence there is an infinite sequence
 satisfying requirements (**) for each  . Now fill up this sequence
with the necessary additional elements from    to obtain an element
 satisfying 
for each . We claim that . Indeed, choose any . Then
. This proves our claim. 


Secondly, consider an arbitrary -sequence  in , and let
 be its colimit
in . We claim that this is also the colimit in
. As before, we can show that each  is a morphism in . Now we
continue in a standard way. 

For each , since ,
we can write  as a colimit of a sequence of strongly
finite objects  in  and in
. By a diagonal argument, there is a sequence of numbers
 such that the spaces  form a sequence in  whose colimit, if it
exists, is also colimit of the sequence  in , and whose colimit in  is
. Hence , and by our
first part,  is the colimit of the sequence
 also in . Thus
 is the colimit of 
in .  \epf

To make the paper self-contained, we recall briefly a
result of Droste and G\"{o}bel~\cite{droste1} concerning the existence of a universal,
homogeneous object in an algebroidal category. 
Let   be a  category  in which   all the morphisms   are  monic,
and  a full subcategory of .   Individually, an object 
of  is called
\begin{enumerate}[]
\item {\em -universal} if
for any object  in , there is a morphism ;
\item  {\em -homogeneous}  if for any
 in   and any pair  ,
there is an  isomorphism 
such that ;
\end{enumerate}

Intuitively, -homogeneity means that each isomorphism
between two -substructures of  extends to an
automorphism of ; this means that  has maximal possible degree
of symmetry. 

Collectively, the category   is said to have the
{\em amalgamation property} if for any ,  in , there exist ,  in
 such that . 


\begin{defn}
  Let  be a category in which all morphisms are monic. 
  Then  is called \emph{algebroidal}, if  has
  the following properties:

\begin{enumerate}[(1)]
\item  has a weakly initial object,
\item Every object of  is a colimit of an -chain of
finite objects,
\item Every -sequence of finite objects has a colimit, and
\item The number of finite objects of , up to isomorphism,
  is countable and between any pair of finite objects there exist only
  countably many morphisms. 
\end{enumerate}

\end{defn}


\begin{thm}\label{th:univ} (Droste and G\"{o}bel)
  Let  be an algebroidal category with all morphisms monic. 
  Let  be the full subcategory of finite objects of .  Then there exists a -universal, -homogeneous object iff  has the amalgamation
  property.  Moreover, in this case the -universal, -homogeneous object is unique up to isomorphism. 
\end{thm}


\begin{prop}\label{amalg} The category 
  contains an initial object. The strongly finite objects of
   are precisely the finite objects of . If  is countable, there are only countably many
  non-isomorphic finite objects in . 
  Between any pair of finite objects there are only finitely
  many injections. Moreover, the finite objects of  have the amalgamation property. 
\end{prop}

\pf The space  is the initial object of
, since all spaces in 
have non-empty attribute sets. Next, we show only the amalgamation
property; the rest is easy to see.  Suppose ,
, and  are
strongly finite objects in {\bf iC} such that , and let
 and  be morphisms in . 


Construct  as:


Note that in case ,  we have

To see that  is extensional,
suppose  are such that
 for all . 
By the definition of  , then,
for each , we have
. By the extensionality of  , we have
. Similarly, by the extensionality of , we have
 and so , as required. 

It is easy to check further that

and  
are morphisms in . 
\epf

By Proposition~\ref{amalg}, the following result immediately follows from
Theorem~\ref{th:univ}. 

\begin{thm}\label{thm:iCbifalgebroidal}
  Let  be countable. Then  is an
  algebroidal category containing a universal homogeneous object . 
  Moreover,  is unique up to isomorphism. 
\end{thm}

  Since  contains spaces with an attribute set of
  size continuum, it follows that the attribute set of  also has
  size continuum. However, we just note that since the proof of
  Theorem~\ref{th:univ} is constructive, we can
  \emph{construct a sequence}  whose
  colimit is the universal homogeneous object . 



We remark that  does not contain all countable extensional Chu spaces
(just as not all countable cpos are SFP). 
Let  
be the biextensional Chu space
described in the proof of Theorem 3.1. We claim that
  is not bifinite. 

Indeed, choose the sequence , the
space  and the
monics  as in the proof of Theorem
3.1.  By Theorem 3.4,  is the colimit of the chain 
in .  By Theorem 5.3, this is also the colimit of the
sequence of finite spaces  in the category .  Consider the morphisms  described in the proof of Theorem 3.1.  Now if  was bifinite, there would be a unique morphism  in  making the diagram commute. But
then , and  for some , yielding , a contradiction. 

\section{Concluding Remarks}


\noindent One specific motivation for considering the notion of
bifinite Chu space is for modeling linear logic~\cite{lamarche0}, for
which tensor and linear negation should also be brought into the
picture.  Although the category of bifinite Chu spaces is monoidal, it
is not monoidal closed~\cite{huang}.  Also, the construction of linear
negation cannot be accounted for nicely in bifinite Chu spaces either.
In spite of these, one has to look at the bigger picture and consider
our results in the context of Chu spaces as a general framework for
studying the dualities of objects and properties, points and open
sets, and terms and types, under rich mathematical contexts. This view
has already been made amply clear by
Pratt~\cite{pratt1,pratt2,pratt3,pratt4,pratt5,pratt6,pratt7}.
Traditionally, the study on Chu spaces had a non-constructive
flavor. This paper provides (1) a basis for a more constructive
analysis of categories of Chu spaces collectively; (2) a framework in
which finite Chu spaces can be used to approximate infinite ones as
colimits of -chains of finite Chu spaces; (3) the
``completeness'' of the delineated categories under the colimit
construction.  On the technical side, the development of our framework
hinges upon the adoption of monic morphisms as the basic steps for
approximation at the structural level.  It is certainly reasonable to
consider other possible notions of ``approximation'' as well, such as
``regular mono'' or more generally ``embedding-projection pair'', of
which monic morphism can be regarded as a special case.

\bigskip
{\bf Acknowledgment}. The authors would like to thank F. Lamarche and the anonymous 
referees for valuable feedback.
\begin{thebibliography}{10}


\bibitem{curien}
R. Amadio and P.-L. Curien,
\emph{Domains and Lambda-Calculi.} 
Cambridge University Press, 1998. 

\bibitem{barr1} M. Barr. *-Autonomous categories, with an appendix by Po Hsiang Chu. 
{\em  Lecture Notes in Mathematics}, Vol. 752,  Springer-Verlag, 1979. 

\bibitem{barr2} M. Barr. *-Autonomous categories and linear logic. 
{\em Mathematical Structures in Computer Science}, Vol. 1, pp. 159-178, 1991. 

\bibitem{pratt0}
H. Devarajan, D. Hughes, G. Plotkin, V. Pratt. 
Full completeness of the multiplicative linear logic of Chu spaces. 
14th Symposium on Logic in Computer Science (Trento, 1999), 234--243,
IEEE Computer Soc., Los Alamitos, CA, 1999. 


\bibitem{erne}
M. Ern\'{e}. General Stone duality. {\em Topology and Its Applications}, Vol. 137, pp. 125-158, 2004. 

\bibitem{droste1}
M. Droste and R. G\"{o}bel. Universal domains and the amalgamation
property. {\em Mathematical Structures in Computer Science}. 3:137-159,
1993. 

\bibitem{droste2}
M. Droste. 
Universal homogeneous causal sets. 
{\em Journal of Mathematical Physics},  46:1-10, 2005.


\bibitem{ganter}
B. Ganter and R. Wille. 
{\em Formal Concept Analysis}. 
Springer-Verlag, 1999. 


\bibitem{pascal}
P. Hitzler and G.-Q. Zhang. 
A cartesian closed category of approximable concept structures. 
In Pfeiffer and Wolff (eds.), Proceedings of the International Conference on Conceptual Structures,
Huntsville, Alabama, USA, July 2004. Lecture Notes in Artificial Intelligence, Vol. 3127,   pages 170-185, 2004.


\bibitem{alga}
P. Hitzler, M. Kr\"{o}tzsch and G.-Q. Zhang. 
A categorical view on algebraic lattices in Formal Concept Analysis. 
{\em Fundamenta Informaticae}, Volume 74 (2-3), pp. 301 - 328, 2006. 

\bibitem{huang}
F. Huang, M. Droste and G.-Q. Zhang.
	A monoidal category of bifinite Chu spaces. 
	Electronic Notes in Theoretical Computer Science, Vol. 212, pp. 285-297, 2008.

\bibitem{lamarche0}
F. Lamarche. 
Dialectics: a model of linear logic and PCF. Submitted to MSCS.

\bibitem{lamarche}
F. Lamarche. 
From Chu spaces to cpos. {\em Theory and Formal Methods of Computing 94}, Imperial College Press,
pp. 283-305, 1994. 



\bibitem{maclane}
S.~Mac~Lane. 
\emph{Categories for the Working Mathematician.} 
Springer-Verlag, 1971. 


\bibitem{plotkin}
G. Plotkin. 
A powerdomain construction. {\em SIAM J. Comput.,} Vol. 5, pp. 452-487, 1976. 

\bibitem{plotkin78}
G. Plotkin. 
 as a universal domain. 
{\em J. Comp. Sys. Sci.} Vol. 17, pp. 209-236, 1978. 


\bibitem{plotkin-chu}
G. Plotkin. Notes on the Chu construction and recursion. 
\verb+http://boole.stanford.edu/pub/gdp.pdf+ (accessed Jan 2007.) 

\bibitem{pratt1}
V. Pratt. 
Chu spaces. {\em School on Category Theory and Applications},  {\em Textos Mat. SCr. B, 21}, 39-100,  Univ. Coimbra, Coimbra, 1999. 


\bibitem{pratt2}
V.   Pratt. 
Higher dimensional automata revisited. {\em Math. Structures Comput. Sci.} 
Vol. 10, pp.  525-548, 2000. 


\bibitem{pratt3}
V. Pratt. 
Chu spaces from the representational viewpoint. {\em Ann. Pure Appl. Logic}, Vol. 96, pp.  319-333, 1999. 

\bibitem{pratt4}
V. Pratt. 
Towards full completeness of the linear logic of Chu spaces. 
{\em Mathematical {F}oundations of {P}rogramming {S}emantics} (Pittsburgh, PA, 1997),
{\em Electronic Notes in Theoretical Computer Science}, Vol. 7, 18 pp., 1997. 


\bibitem{pratt5}
V. Pratt. 
The Stone gamut: a coordinatization of mathematics. {\em Proceedings of 10th Annual Symposium on
Logic in Computer Science}, pp. 444-454, 1995. 

\bibitem{pratt6}
V. Pratt. 
Chu spaces and their interpretation as concurrent objects. {\em Lecture Notes in Comput. Sci.} Vol. 1000,
pp. 392-405, 1995. 

\bibitem{pratt7}
V. Pratt. 
Chu spaces as a semantic bridge between linear logic and mathematics. 
{\em Theoretical Computer Science},  Vol. 294, pp.  439-471, 2003. 

\bibitem{scott76}
D. Scott. Data types as lattices. {\em SIAM J. Comput.,} Vol. 5, pp. 522-586, 1976. 

\bibitem{scott}
D. Scott. 
Domains for denotational semantics. Automata, {L}anguages and {P}rogramming (Aarhus, 1982),
pp. 577--613, Lecture Notes in Comput. Sci., 140, Springer, Berlin-New York, 1982. 


\bibitem{zhang-mfps}
G.-Q. Zhang. Chu spaces, concept lattices, and  domains. 
In: Proceedings of the 19th Conference on the Mathematical Foundations of
Programming Semantics, March 2003, Montreal, Canada. {\em Electronic Notes
in Theoretical Computer Science} Vol. 83, 2004, 17 pages. 

\bibitem{tac}
G.-Q. Zhang and G. Shen. 
Approximable concepts, Chu spaces, and information systems. 
In V. De Paiva and V. Pratt (edts.) 
{\em  Theory and Applications of Categories},
Special Volume on Chu Spaces: Theory and Applications, Vol. 17, No. 5, pp. 80-102, 2006. 
\end{thebibliography}


\end{document}
