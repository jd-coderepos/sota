\makeatletter
\@ifundefined{OPTIONAppendix}{\def\OPTIONAppendix{0}}{}
\makeatother
\def\OPTIONConf{0}
\ifnum\OPTIONAppendix=0
    \documentclass[a4paper]{article}
    \def\OPTIONLoudLabels{0}
\def\OPTIONArxiv{1}

 
    \usepackage{srcltx}
    \usepackage{fancyhdr}
    \usepackage{xspace}
    \usepackage{enumerate}
    \usepackage[hyperref,svgnames]{xcolor}
    \usepackage{graphics,pstricks,color}
    \usepackage{relsize}

    \usepackage{booktabs}

    \usepackage{phaistos}
    \newcommand{\alittlefishy}{\text{\PHtunny}}
    \newcommand{\alotfishy}{\flaming{\alittlefishy}}

    \renewcommand{\bfdefault}{b}
    \usepackage[hyphens]{url}


    \usepackage{hyperref} \usepackage{latexsym,stmaryrd}
    \usepackage{amsmath,amsthm,amssymb}
    \usepackage{thmtools,thm-restate}



    \usepackage{mathpartir}
    \usepackage{charter}
    \usepackage{euler}
    \usepackage{marvosym}
    \usepackage{fullpage}
    \usepackage[authoryear]{natbib}

    \bibpunct{(}{)}{;}{a}{}{,}

    \declaretheoremstyle[
      bodyfont=\sl
    ]{mytheoremstyle}

    \declaretheorem[style=mytheoremstyle]{theorem}
    \declaretheorem[style=mytheoremstyle, sibling=theorem]{property}
\declaretheorem[style=mytheoremstyle]{lemma}
    \declaretheorem[style=mytheoremstyle, sibling=lemma]{corollary}
    \declaretheorem[style=mytheoremstyle, sibling=lemma]{conjecture}
    \declaretheorem{example}
    \makeatletter
    \declaretheorem[style=mytheoremstyle, sibling=lemma, 
]{proposition}
    \makeatother
    \declaretheorem{remark}
    \declaretheorem[style=mytheoremstyle]{definition}
\else
\fi


\newcommand{\addmytocentry}[2]{\addtocontents{toc}{\protect\contentsline {subsubsection}{\protect\numberline {#1}#2}{\thepage}{xx}}}
\newcommand{\MyThmRestateHook}[3]{\relax
}

\newcommand{\lxbel}[1]{\label{#1}}

\makeatletter
\newcommand{\XLabel}[1]{\ifcsname IDEMPOTFLAG#1\endcsname \lxbel{PROOF#1}\LoudLabel{#1}\addmytocentry{\ref{PROOF#1}}{\hyperref[PROOF#1]{Proof of \thmt@thmname~(\thmt@optarg)}}\else \Label{#1}\addmytocentry{\ref{#1}}{\hyperref[#1]{\thmt@thmname~(\thmt@optarg)}}\expandafter\gdef\csname IDEMPOTFLAG#1\endcsname{d}\fi}
\makeatother


\newcounter{zzbackby}
\newcommand{\backby}[2]{\setcounter{zzbackby}{\uccode`#1}\addtocounter{zzbackby}{-#2}\char\value{zzbackby}}



\newcommand{\XAlph}[1]{\backby{\Alph{section}}{#1}}



\ifnum\OPTIONAppendix=0
        \usepackage[T1]{fontenc}

\definecolor{dHilite}{rgb}{0.9, 0.9, 0.6}
\definecolor{dRed}{rgb}{0.65, 0.0, 0.0}
\definecolor{dGreen}{rgb}{0.0, 0.65, 0.0}
\definecolor{dDkGreen}{rgb}{0.0, 0.35, 0.0}
\definecolor{dBlue}{rgb}{0.0, 0.0, 0.65}
\definecolor{dPurple}{rgb}{0.65, 0.0, 0.65}
\definecolor{dDigPurple}{rgb}{0.5, 0.0, 0.5}
\definecolor{dFaint}{rgb}{0.7, 0.7, 0.7}
\definecolor{dGray}{rgb}{0.5, 0.5, 0.5}
\definecolor{dDark}{rgb}{0.2, 0.2, 0.2}
\definecolor{dAlmostBlack}{rgb}{0.1, 0.1, 0.1}

\makeatletter
\def\url@MGstyle{\def\UrlFont{\tiny\huge\ttfamily}\Url@do
}
\makeatother
\newcommand{\marginPseudoURL}[1]{\tt #1}
\newcommand{\marginnote}[1]{\marginparwidth=40pt \marginpar{\raisebox{-2ex}{\parbox{40pt}{\raggedright\scriptsize #1}}}}

\makeatletter
\def\url@vttstyle{\@ifundefined{selectfont}{\def\UrlFont{\tt}}{\def\UrlFont{\normalfont\fontfamily{cmvtt}\selectfont}}}
\makeatother
\urlstyle{vtt}



\newcommand\textvtt[1]{{\normalfont\fontfamily{cmvtt}\selectfont #1}}

\newcommand{\LoudLabel}[1]{\idempotentlabel{#1}\ifnum\OPTIONLoudLabels=1\ifnum\OPTIONConf=1\marginnote{\tiny\textvtt{#1}}\else \marginnote{\textvtt{#1}}\fi \fi }

\newcommand{\idempotentlabel}[1]{\ifcsname IDEMPFLAG#1\endcsname \message{YYY ALREADY DEFINED: #1}
    \else \message{YYZ NOT ALREADY DEFINED: #1}
      \expandafter\gdef\csname IDEMPFLAG#1\endcsname{d}\label{#1}\fi}



\ifnum\OPTIONLoudLabels=1\newcommand{\Label}[1]{\LoudLabel{#1}}\newcommand{\FLabel}[1]{\idempotentlabel{#1}{\tt\scriptsize{#1}}}\else \newcommand{\Label}[1]{\idempotentlabel{#1}}\newcommand{\FLabel}[1]{\idempotentlabel{#1}}\fi

\newdimen\zzfontsz
\newcommand{\fontsz}[2]{\zzfontsz=#1{\fontsize{\zzfontsz}{1.2\zzfontsz}\selectfont{#2}}}

\newcommand{\texfontsz}[1]{\zzfontsz=#1\fontsize{\zzfontsz}{1.25\zzfontsz}\selectfont}

\newcommand{\mathsz}[2]{\text{\fontsz{#1}{$#2$}}}

\newcommand{\XX}{\text{\ding{55}}}
\newcommand{\redXX}{\text{\textcolor{dRed}{\XX}}}
\newcommand{\greencheck}{\text{\textcolor{dGreen}{\checkmark}}}

\newcommand{\fixme}[1]{\textcolor{red}{\texttt{FIXME: {#1}}}}

\newcommand{\flaming}[1]{\textcolor{red}{\fontsz{18pt}{\bf #1}}}
\newcommand{\flamingmath}[1]{\textcolor{red}{\fontsz{18pt}{\bf \ensuremath{#1}}}}

\newcommand{\semiflaming}[1]{\textcolor{dRed}{\sl #1}}

\newcommand{\mathcolor}[2]{\text{\textcolor{#1}{\ensuremath{#2}}}}

\newcommand{\smallblacktriangle}{\text{\textscale{0.7}{$\blacktriangleright$}}}

\newcommand{\setof}[1]{\left\{#1\right\}}
\newcommand{\comprehend}[2]{\setof{{#1} \;\middle|\; {#2}}}

\newcommand{\assign}{\ensuremath{\,{:=}\,}}

\newcommand{\arr}{\rightarrow}
\def\CompactJudgments{0}
\newcommand{\entails}{\mathrel{\ifnum\CompactJudgments=1\vdash \else \vdash\,\fi}}
\newcommand{\ctxoutsym}{\ifnum\CompactJudgments=1\dashv \else \,\dashv \fi}
\newcommand{\ctxout}[1]{\mathrel{\ctxoutsym}{#1}}

\newcommand{\J}{\mathcal{J}}
\newcommand{\judg}{\J}

\newcommand{\MonnierCommaSym}{{\smallblacktriangle}}
\newcommand{\MonnierComma}[1]{{\MonnierCommaSym}_{#1}}

\newcommand{\FV}[1]{\mathrm{FV}(#1)}
\newcommand{\xfev}{\mathsf{FEV}}
\newcommand{\fev}[1]{\xfev(#1)}
\newcommand{\FEV}[1]{\fev{#1}}
\newcommand{\xfsolev}{\mathsf{FSolEV}}
\newcommand{\fsolev}[2]{\xfsolev_{#1}(#2)}

\newcommand{\beeq}{=_{\beta\eta}}

\newcommand{\kindstar}{\star}
\newcommand{\kindnat}{\mathbb{N}}

\newcommand{\inversefn}[1]{{#1}^{-1}}

\newcommand{\emptysig}{\cdot}
\newcommand{\emptyctx}{\cdot}

\newcommand{\natzero}{\mathsf{zero}}
\newcommand{\xnatsucc}{\mathsf{succ}}
\newcommand{\natsucc}[1]{\xnatsucc\texttt{(}{#1}\texttt{)}}


\newcommand{\instantiate}[1]{{#1}\texttt{[-]}}
\newcommand{\quantify}[1]{\Lambda{#1}.\,}


\newcommand{\exvar}[1]{\widehat{#1}}
\newcommand{\exalpha}{\exvar{\alpha}}
\newcommand{\exbeta}{\exvar{\beta}}

\newcommand{\alln}[1]{\forall{#1}{:}\kindnat.\,}

\newcommand{\Match}[2]{{#1} \Rightarrow {#2}}
\newcommand{\matchor}{\ensuremath{\normalfont\,\texttt{|}\hspace{-5.35pt}\texttt{|}\,}}
\newcommand{\ind}[3]{\mathsf{ind}\texttt{(}\Match{\natzero}{#1}\texttt{,\;}\Match{\natsucc{#2}}{#3}
                     \texttt{)}}

\newcommand{\exunk}[2]{{#1} : {#2}}
\newcommand{\exsol}[3]{{#1} : {#2} \texttt{=} {#3}}
\newcommand{\exsolnokind}[2]{{#1} \texttt{=} {#2}}
\newcommand{\exsolwild}[2]{({#1} : {#2}\dots)}
\newcommand{\rexvar}[2]{\exsolnokind{#1}{#2}}

\newcommand{\tyname}[1]{\textsf{\normalfont #1}}
\newcommand{\unitexp}{\text{\normalfont \tt()}}
\newcommand{\unitty}{\tyname{1}}

\newcommand{\trientails}{\mathrel{{\rhd}\,}}
\newcommand{\trictxoutsym}{{\lhd}}
\newcommand{\trictxout}[1]{\mathrel{\trictxoutsym}{#1}}

\newcommand{\subtypingycolor}[1]{\textcolor{dDigPurple}{#1}}
\newcommand{\subtype}{\mathrel{\normalfont\texttt{\subtypingycolor{<:}}}}  \newcommand{\declsubtype}{\mathrel{\leq}}

\newcommand{\etal}{{et al.}}
\newcommand{\eg}{e.g.\ }
\newcommand{\ie}{i.e.\ }
\newcommand{\ala}{\`a la\ }
\newcommand{\Wlog}{w.l.o.g.\ }
\newcommand{\visavis}{vis-\`a-vis\ }
\newcommand{\Ocaml}{{OCaml}\xspace}
\newcommand{\OCaml}{\Ocaml}

\newcommand{\naive}{na\"ive\xspace}
\newcommand{\Backslash}{\char"5C}
\newcommand{\Lbrack}{\char"5B}
\newcommand{\Rbrack}{\char"5D}
\newcommand{\Lbrace}{\char"7B}
\newcommand{\Rbrace}{\char"7D}

\newcommand{\Appendixref}[1]{Appendix \ref{#1}}
\newcommand{\Figureref}[1]{Figure \ref{#1}}
\newcommand{\Figref}[1]{Fig.\ \ref{#1}}
\newcommand{\Sectionref}[1]{Section \ref{#1}}
\newcommand{\Secref}[1]{Sec.\ \ref{#1}}
\newcommand{\Chapterref}[1]{Chapter \ref{#1}}

\newcommand{\Listingref}[1]{Listing \ref{#1}}

\newcommand{\Theoremref}[1]{Theorem \ref{#1}}
\newcommand{\Thmref}[1]{Thm.\ \ref{#1}}
\newcommand{\Corollaryref}[1]{Corollary \ref{#1}}
\newcommand{\Corref}[1]{Cor.\ \ref{#1}}
\newcommand{\Lemmaref}[1]{Lemma \ref{#1} (\nameref{#1})}   \newcommand{\Lemref}[1]{\Lemma \ref{#1}}   \newcommand{\Conjectureref}[1]{Conjecture \ref{#1}}
\newcommand{\Propositionref}[1]{Proposition \ref{#1}}
\newcommand{\Propertyref}[1]{Property \ref{#1}}
\newcommand{\Remarkref}[1]{Remark \ref{#1}}
\newcommand{\Tableref}[1]{Table \ref{#1}}

\newcommand{\Definitionref}[1]{Definition \ref{#1}}
\newcommand{\Defnref}[1]{Def.\ \ref{#1}}

\newcommand{\ProofCaseRule}[1]{\item \textbf{Case }\textrm{{#1}}: ~ }
\newcommand{\ProofCaseThing}[1]{\ProofCaseRule{\ensuremath{#1}}}
\newcommand{\ProofCasesRules}[1]{\item \textbf{Cases }\textrm{{#1}}: ~ }
\newcommand{\ProofCaseRuleNoColon}[1]{\item \textbf{Case }\textrm{{#1}}}

\gdef\xxDerivationProofCaseColor{N}
\newcommand{\Begincolorcases}[1]{\gdef\xxDerivationProofCaseColor{#1}}
\newcommand{\Endcolorcases}{\gdef\xxDerivationProofCaseColor{N}}

\newcommand{\DerivationProofCase}[3]{\smallskip
     \item \parbox[t]{100ex}{\textbf{Case } \\[-0.5em]
       $~$\hspace{5ex}
       \if\xxDerivationProofCaseColor N\ensuremath{\Infer{#1}{#2}{#3}}
       \else \colorbox{\xxDerivationProofCaseColor}{\ensuremath{\Infer{#1}{#2}{#3}}}\fi }\nopagebreak \\[-0.8ex]
  }

\newcommand{\DoubleDerivationProofCase}[6]{\smallskip
     \item \parbox[t]{100ex}{\textbf{Case } \\[-0.5em]
       $~$\hspace{5ex}
       \if\xxDerivationProofCaseColor N\ensuremath{\Infer{#1}{#2}{#3}~~~~~
              \Infer{#4}{#5}{#6}}
       \else \colorbox{\xxDerivationProofCaseColor}{\ensuremath{\Infer{#1}{#2}{#3}~~~~~
                \Infer{#4}{#5}{#6}}}\fi }\nopagebreak \\[-0.8ex]
  }

\newcommand{\Dee}{\mathcal{D}}
\newcommand{\D}{\mathcal{\Dee}}

\newenvironment{displ}{\vspace{1pt} \begin{center} ~\!\!}{\end{center}}
\newenvironment{mathdispl}{\vspace{1pt} \begin{center} ~\!\!\(}{\)\end{center}}

\newenvironment{ctabular}{\renewcommand{\arraystretch}{1}\vspace{1pt}\begin{center} ~\!\!\begin{tabular}[t]{l}}{\end{tabular}\end{center}}

\newcommand{\arrayenv}[1]{\renewcommand{\arraystretch}{1} \begin{array}[t]{@{}c@{}}#1\end{array}}
\newcommand{\arrayenvc}[1]{\renewcommand{\arraystretch}{1} \begin{array}[c]{@{}c@{}}#1\end{array}}
\newcommand{\arrayenvcl}[1]{\renewcommand{\arraystretch}{1} \begin{array}[c]{@{}l@{}}#1\end{array}}
\newcommand{\arrayenvr}[1]{\renewcommand{\arraystretch}{1} \begin{array}[t]{@{}r@{}}#1\end{array}}
\newcommand{\arrayenvbr}[1]{\renewcommand{\arraystretch}{1} \begin{array}[b]{@{}r@{}}#1\end{array}}
\newcommand{\arrayenvl}[1]{\renewcommand{\arraystretch}{1} \begin{array}[t]{@{}l@{}}#1\end{array}}
\newcommand{\arrayenvb}[1]{\renewcommand{\arraystretch}{1}  \begin{array}[b]{@{}c@{}}#1\end{array}} 
\newcommand{\arrayenvbl}[1]{\renewcommand{\arraystretch}{1}  \begin{array}[b]{@{}l@{}}#1\end{array}}
\newcommand{\arrayenvblll}[1]{\renewcommand{\arraystretch}{1}  \begin{array}[b]{@{}lll@{}}#1\end{array}}
\newcommand{\pfarr}[1]{\begin{array}[b]{@{}l@{}}#1\end{array}}

\newcommand{\BeginProof}{\renewcommand{\arraystretch}{1.1} \begin{tabular}[b]{r@{}r @{} l  l}}
\newcommand{\EndProof}{\end{tabular} \renewcommand{\arraystretch}{\mydefaultarraystretch}}

\newcommand{\Hand}{\text{\Pointinghand~~~~}}

\newcommand{\Pf}[4] {&$#1$ $#2$\, & $#3$ & #4 \\}
\newcommand{\Pfmrg}[3] {&$#1$\, & $#2$ & #3 \\}
\newcommand{\stepPf}[3] {\Pf{#1}{\,\step\,}{#2}{#3}}
\newcommand{\EPf}[3] {\Pf{#1}{\Entails}{#2}{#3}}
\newcommand{\LetPf}[3] {\Pf{\text{Let}\,~{#1}}{=\,}{#2\text{.}}{#3}}
\newcommand{\ForallPf}[3] {\Pf{\text{For all}\,~{#1}}{\in\,}{#2}{#3}}
\newcommand{\mkpf}[4] {\Pf{#2}{#1\,}{#3}{#4}}

\newcommand{\ePf}[3] {\mkpf{\entails}{#1}{#2}{#3}}
\newcommand{\eqPf}[3] {\mkpf{=}{#1}{#2}{#3}}
\newcommand{\continueeqPf}[2] {\mkpf{=}{~}{#1}{#2}}
\newcommand{\rightstarteqPf}[1] {\mkpf{~}{~}{#1}{~}}
\newcommand{\neqPf}[3] {\mkpf{\neq}{#1}{#2}{#3}}
\newcommand{\ltPf}[3] {\mkpf{<}{#1}{#2}{#3}}
\newcommand{\leqPf}[3] {\mkpf{\leq}{#1}{#2}{#3}}
\newcommand{\inPf}[3] {\mkpf{\in}{#1}{#2}{#3}}
\newcommand{\notinPf}[3] {\mkpf{\notin}{#1}{#2}{#3}}

\newcommand{\trailingjust}[1]{\Pf{}{}{}{~~{#1}}}
\newcommand{\derivesPf}[1]{${#1} \derives~~$}

\newcommand{\contraPf}[1] {\Pf{\Rightarrow\Leftarrow}{}{} {}\Pf{#1}{}{} {By contradiction}}

\newcommand{\NOTePf}[3] {\Pf{#1}{\not\entails\;}{#2}{#3}}
\newcommand{\proofsep}{\,\\[-0.5em]}
\newcommand{\PfTwo}[7]{\Pf{\arrayenvr{{#1}\\{#4}}}{\arrayenvl{{#2}\\{#5}}}{\arrayenvl{{#3}\\{#6}}}{\drophalf{\!\ensuremath{\left\} \begin{array}{r l} \,\\ \, \end{array}\!\!\!\!\!\text{#7} \right.}}}}
\newenvironment{llproof}{\BeginProof}{\EndProof}
\newcommand{\decolumnizePf}{\end{llproof} ~\\ \begin{llproof}}

\newcommand{\proofheading}[1]{}  

\newcommand{\ditto}{\ensuremath{''}}


\newcommand{\xdom}{\mathsf{dom}}
\newcommand{\dom}[1]{\xdom(#1)}

\newcommand{\xweight}{\mathsf{weight}}
\newcommand{\weight}[2]{\xweight_{#1}(#2)}

\newcommand{\xangst}{\mathsf{angst}}
\newcommand{\angst}[2]{\xangst_{#1}(#2)}

\newcommand{\xunsol}{\mathsf{unsol}}
\newcommand{\unsol}[2]{\xunsol_{#1}(#2)}

\newcommand{\xunsolved}{\mathsf{unsolved}}
\newcommand{\unsolved}[1]{\xunsolved(#1)}

\newcommand{\xGtypesize}[1]{\mathsf{size}_{#1}}
\newcommand{\Gtypesize}[2]{\xGtypesize{#1}(#2)}

\newcommand{\union}{\mathrel{\cup}}
\newcommand{\sect}{\mathrel{\cap}}


\newcommand{\normalize}{\mathrel{\Downarrow}}


\newcommand{\textgraybox}[1]{\boxed{#1}}
\newcommand{\graybox}[1]{\textgraybox{\ensuremath{#1}}}

\newcommand{\tightcolorbox}[2]{\setlength{\fboxsep}{1pt}\colorbox{#1}{#2}}

\newgray{grayboxgray}{0.85}
\newcommand{\textcolorbox}[2]{\tightcolorbox{#1}{#2}}
\newcommand{\textshadebox}[1]{\textcolorbox{grayboxgray}{#1}}
\newcommand{\shadebox}[1]{\text{\textshadebox{\ensuremath{#1}}}}



\newcommand{\mathcolorbox}[2]{\text{\tightcolorbox{#1}{$\displaystyle {#2}$}}}



\newcommand{\judgboxfontsize}[1]{\ifnum\OPTIONConf=1\mathsz{11pt}{#1}\else \mathsz{14pt}{#1}\fi}
\newcommand{\judgbox}[2]{{\raggedright \textgraybox{\ensuremath{\judgboxfontsize{#1}}}\!\fontsz{9pt}{\begin{tabular}[c]{l} #2 \end{tabular}} }}


\newcommand{\bnfas}{\mathrel{::=}}
\newcommand{\bnfalt}{\mathrel{\mid}}

\newcommand{\derives}{\mathrel{::}}

\newcommand{\AllSym}{\forall}
\newcommand{\xAll}[1]{\AllSym#1}
\newcommand{\All}[1]{\xAll{#1}.\:}

\newcommand{\AND}{\text{~~and~~}}

\newcommand{\Infer}[3]{\inferrule*[right={\text{\strut#1}}]{{}#2\mathstrut}{{}#3\mathstrut}}




\newcommand{\keyword}[1]{\textsf{#1}}
\newcommand{\textkw}[1]{\keyword{#1}}
\newcommand{\lam}[1]{\lambda #1.\,}
\newcommand{\fun}[2]{\lam{#1}{#2}}
\newcommand{\typelam}[1]{\Lambda #1.\,}
\newcommand{\LAM}[1]{\typelam{#1}}
\newcommand{\Let}[2]{\textkw{let}\;{#1}\,\texttt{=}\,{#2}\;\textkw{in}\;}

\newcommand{\bigprec}{\mathrel{\mathsz{14pt}{\prec}}}


\newcommand{\atomic}[1]{{#1}\;\mathrm{atomic}}

\newcommand{\declsubjudg}[3]{\ensuremath{{#1} \entails {#2} \declsubtype {#3}}}
\newcommand{\typejudg}[3]{\ensuremath{{#1} \entails {#2} : {#3}}}
\newcommand{\subjudg}[4]{\ensuremath{{#1} \entails {#2} \subtype {#3} \ctxout{#4}}}

\newdimen\zzinstsymLTwidth
\newdimen\zzinstsymEQwidth
\newdimen\zzinstsymDiff
\newcommand{\instsymLeq}{\settowidth{\zzinstsymLTwidth}{\text{\normalfont\tt<}}\settowidth{\zzinstsymEQwidth}{\text{\normalfont=}}\setlength{\zzinstsymDiff}{\zzinstsymEQwidth}\addtolength{\zzinstsymDiff}{-\zzinstsymLTwidth}\text{\raisebox{-0.22ex}{\normalfont=}\hspace{-\zzinstsymEQwidth}\hspace{0.5\zzinstsymDiff}\raisebox{0.77ex}{\normalfont\tt<}}}
\newcommand{\instsymColon}{\raisebox{-0.09ex}{\text{\normalfont{:}}}}
\newcommand{\instsyml}{\subtypingycolor{\instsymColon\hspace{0.05ex}\instsymLeq}}
\newcommand{\instsymr}{\subtypingycolor{\instsymLeq\hspace{0.05ex}\instsymColon}}
\newcommand{\instsymlop}{\mathrel{\instsyml}}
\newcommand{\instsymrop}{\mathrel{\instsymr}}

\newcommand{\instjudg}[4]{\ensuremath{{#1} \entails {#2} \instsymlop {#3} \ctxout{#4}}}
\newcommand{\instjudgr}[4]{\ensuremath{{#1} \entails {#3} \instsymrop {#2} \ctxout{#4}}}

\newcommand{\declsubjudgPf}[4] {\Pf{#1}{\entails}{{#2} \declsubtype {#3}}{#4}}
\newcommand{\subjudgPf}[5] {\Pf{#1}{\entails}{{#2} \subtype {#3} \ctxout{#4}}{#5}}
\newcommand{\substextendPf}[3] {\Pfmrg{{#1} \extendssym\,}{#2}{#3}}
\newcommand{\instjudgPf}[5]{\Pf{#1}{\entails}{{#2} {\;\instsyml\;} {#3} \ctxout{#4}}{#5}}
\newcommand{\instjudgrPf}[5]{\Pf{#1}{\entails}{{#3} {\;\instsymr\;} {#2} \ctxout{#4}}{#5}}

\newcommand{\chkcolor}{dBlue}
\newcommand{\syncolor}{dRed}
\newcommand{\appcolor}{dDkGreen}
\newcommand{\chk}{\mathrel{\mathcolor{\chkcolor}{\Leftarrow}}}
\newcommand{\uncoloredsyn}{{\Rightarrow}}
\newcommand{\syn}{\mathrel{\mathcolor{\syncolor}{\uncoloredsyn}}}
\newcommand{\appsep}{\;{\mathcolor{\appcolor}{\bullet}}\;}
\newcommand{\app}{\mathrel{\mathcolor{\appcolor}{{\uncoloredsyn}\hspace{-1.2ex}{\uncoloredsyn}}}}

\newcommand{\chkjudg}[4]{\ensuremath{{#1} \entails {#2} \chk {#3} \ctxout{#4}}}
\newcommand{\appjudg}[5]{\ensuremath{{#1} \entails {#3} \appsep {#2} \app {#4} \ctxout{#5}}}
\newcommand{\synjudg}[4]{\ensuremath{{#1} \entails {#2} \syn {#3} \ctxout{#4}}}

\newcommand{\declchkjudg}[3]{\ensuremath{{#1} \entails {#2} \chk {#3}}}
\newcommand{\declappjudg}[4]{\ensuremath{#1} \entails {#3} \appsep {#2}  \app {#4}}
\newcommand{\declsynjudg}[3]{\ensuremath{{#1} \entails {#2} \syn {#3}}}

\newcommand{\chkjudgPf}[5]{\Pf{#1}{\entails}{{#2} \chk {#3} \ctxout{#4}}{#5}}
\newcommand{\appjudgPf}[6]{\Pf{#1}{\entails}{{#3} \appsep {#2} \app {#4} \ctxout{#5}}{#6}}
\newcommand{\synjudgPf}[5]{\Pf{#1}{\entails}{{#2} \syn {#3} \ctxout{#4}}{#5}}
\newcommand{\declchkjudgPf}[4]{\Pf{#1}{\entails}{{#2} \chk {#3}}{#4}}
\newcommand{\declappjudgPf}[5]{\Pf{#1}{\entails}{{#3} \appsep {#2} \app {#4}}{#5}}
\newcommand{\declsynjudgPf}[4]{\Pf{#1}{\entails}{{#2} \syn {#3}}{#4}}


\newcommand{\hyp}[2]{{#1}:{#2}}
\newcommand{\hypeq}[2]{{#1} = {#2}}
\newcommand{\tighthypeq}[2]{{#1}{=}{#2}}
\newcommand{\alltype}[1]{\All{#1}}
\newcommand{\num}{\mathsf{int}}
\newcommand{\bool}{\mathsf{bool}}

\newcommand{\grow}[2]{{#1} \sqsupseteq {#2}}

\newcommand{\idsubst}{\mathsf{id}}

\newcommand{\extendssym}{\longrightarrow}
\newcommand{\extends}[2]{{#1} \extendssym {#2}}
\newcommand{\substextend}[2]{\extends{#1}{#2}}

\newcommand{\judgetp}[2]{{#1} \entails {#2}}
\newcommand{\judgetpPf}[3]{\Pf{#1}{\entails}{#2}{#3}}
\newcommand{\typesize}[2]{|{#1} {\;\entails} {#2}|}

\newcommand{\judgectx}[1]{{#1}~\mathit{ctx}}
\newcommand{\judgectxPf}[2]{\Pf{}{}{\judgectx{#1}}{#2}}

\newcommand{\xcompletes}{\ensuremath{\mathsf{completes}}}
\newcommand{\xcompletedby}{\ensuremath{\mathsf{completed\;by}}}
\newcommand{\completedby}[2]{{#1} \;\flamingmath{\xcompletedby}\; {#2}}
\newcommand{\apply}[2]{{[{#1}]}{#2}}

\newcommand{\ahat}{\hat{\alpha}}
\newcommand{\bhat}{\hat{\beta}}
\newcommand{\chat}{\hat{\gamma}}
\newcommand{\dhat}{\hat{\delta}}

\newcommand{\hsubst}[5]{[{#2}/{#3}]^{#1}_{#2}{#4}}

\newcommand{\rulename}[1]{\text{\normalfont\textsf{#1}}}

\newcommand{\substextendrulename}[1]{\ensuremath{{\extendssym}{\rulename{#1}}}\xspace}
\newcommand{\substextendId}{\substextendrulename{ID}}

\newcommand{\substextendUU}{\substextendrulename{Uvar}}
\newcommand{\substextendVV}{\substextendrulename{Var}}
\newcommand{\substextendEE}{\substextendrulename{Unsolved}}
\newcommand{\substextendSolSol}{\substextendrulename{Solved}}
\newcommand{\substextendMonMon}{\substextendrulename{Marker}}

\newcommand{\substextendSolve}{\substextendrulename{Solve}}
\newcommand{\substextendAdd}{\substextendrulename{Add}}
\newcommand{\substextendAddSolved}{\substextendrulename{AddSolved}}


\newcommand{\Dsubrulename}[1]{\ensuremath{{\declsubtype}\rulename{#1}}\xspace}
\newcommand{\DsubVar}{\Dsubrulename{Var}}
\newcommand{\DsubUnit}{\Dsubrulename{Unit}}
\newcommand{\DsubArr}{\Dsubrulename{\ensuremath{\arr}}}
\newcommand{\DsubImp}{\DsubArr} \newcommand{\DsubAllL}{\Dsubrulename{\ensuremath{\forall}{L}}}
\newcommand{\DsubAllR}{\Dsubrulename{\ensuremath{\forall}{R}}}

\newcommand{\Subrulename}[1]{\ensuremath{{\subtype}\rulename{#1}}\xspace}
\newcommand{\SubVar}{\Subrulename{Var}}
\newcommand{\SubUnit}{\Subrulename{Unit}}
\newcommand{\SubExvar}{\Subrulename{Exvar}}
\newcommand{\SubArr}{\Subrulename{\ensuremath{\arr}}}
\newcommand{\SubImp}{\SubArr} \newcommand{\SubAllL}{\Subrulename{\ensuremath{\forall}{L}}}
\newcommand{\SubAllR}{\Subrulename{\ensuremath{\forall}{R}}}

\newcommand{\SubSubst}[1]{\Subrulename{\flaming{Subst{#1}}}}
\newcommand{\SubSubstL}{\SubSubst{L}}
\newcommand{\SubSubstR}{\SubSubst{R}}

\newcommand{\SubInst}[1]{\Subrulename{Instantiate{#1}}}
\newcommand{\SubInstL}{\SubInst{L}}
\newcommand{\SubInstR}{\SubInst{R}}

\newcommand{\DeclWFrulename}[1]{\ensuremath{\rulename{Decl{#1}WF}}\xspace}
\newcommand{\DeclUvarWF}{\DeclWFrulename{Uvar}}
\newcommand{\DeclUnitWF}{\DeclWFrulename{Unit}}
\newcommand{\DeclArrowWF}{\DeclWFrulename{Arrow}}
\newcommand{\DeclForallWF}{\DeclWFrulename{Forall}}

\newcommand{\WFrulename}[1]{\ensuremath{\rulename{{#1}WF}}\xspace}
\newcommand{\UvarWF}{\WFrulename{Uvar}}
\newcommand{\UnitWF}{\WFrulename{Unit}}
\newcommand{\EvarWF}{\WFrulename{Evar}}
\newcommand{\SolvedEvarWF}{\WFrulename{SolvedEvar}}
\newcommand{\ArrowWF}{\WFrulename{Arrow}}
\newcommand{\ForallWF}{\WFrulename{Forall}}


\newcommand{\CWFrulename}[1]{\ensuremath{\rulename{{#1}Ctx}}\xspace}
\newcommand{\EmptyCWF}{\CWFrulename{Empty}}
\newcommand{\UvarCWF}{\CWFrulename{Uvar}}
\newcommand{\EvarCWF}{\CWFrulename{Evar}}
\newcommand{\SolvedEvarCWF}{\CWFrulename{SolvedEvar}}
\newcommand{\VarCWF}{\CWFrulename{Var}}
\newcommand{\MarkerCWF}{\CWFrulename{Marker}}


\newcommand{\Instrulename}[1]{\ensuremath{\rulename{Inst{#1}}}\xspace}
\newcommand{\InstLrulename}[1]{\Instrulename{L{#1}}}
\newcommand{\InstRrulename}[1]{\Instrulename{R{#1}}}

\newcommand{\InstLSolve}{\InstLrulename{Solve}}
\newcommand{\InstLReach}{\InstLrulename{Reach}}
\newcommand{\InstLArr}{\InstLrulename{Arr}}
\newcommand{\InstLAllR}{\InstLrulename{AllR}}

\newcommand{\InstRSolve}{\InstRrulename{Solve}}
\newcommand{\InstRReach}{\InstRrulename{Reach}}
\newcommand{\InstRArr}{\InstRrulename{Arr}}
\newcommand{\InstRAllL}{\InstRrulename{AllL}}


\newcommand{\CompletesRule}{\rulename{completes}}





\newcommand{\Decltyrulename}[1]{\ensuremath{\rulename{Decl#1}}\xspace}

\newcommand{\DeclIntrorulename}[1]{\Decltyrulename{\ensuremath{#1}I}}
\newcommand{\DeclIntroSynrulename}[1]{\Decltyrulename{\ensuremath{#1}I$\syn$}}
\newcommand{\DeclElimrulename}[1]{\Decltyrulename{\ensuremath{#1}E}}
\newcommand{\DeclApprulename}[1]{\Decltyrulename{\ensuremath{#1}App}}

\newcommand{\DeclVar}{\Decltyrulename{Var}}
\newcommand{\DeclSub}{\Decltyrulename{Sub}}
\newcommand{\DeclAnno}{\Decltyrulename{Anno}}

\newcommand{\DeclUnitIntro}{\DeclIntrorulename{\unitty}}
\newcommand{\DeclUnitIntroSyn}{\DeclIntroSynrulename{\unitty}}

\newcommand{\DeclArrIntro}{\DeclIntrorulename{\arr}}
\newcommand{\DeclArrIntroSyn}{\DeclIntroSynrulename{\arr}}
\newcommand{\DeclArrElim}{\DeclElimrulename{\arr}}

\newcommand{\DeclAllIntro}{\DeclIntrorulename{\AllSym}}
\newcommand{\DeclAllElim}{\DeclElimrulename{\AllSym}}

\newcommand{\DeclArrApp}{\DeclApprulename{\arr}}
\newcommand{\DeclAllApp}{\DeclApprulename{\forall}}


\newcommand{\Tyrulename}[1]{\ensuremath{\rulename{#1}}\xspace}

\newcommand{\Introrulename}[1]{\Tyrulename{\ensuremath{#1}I}}
\newcommand{\IntroSynrulename}[1]{\Tyrulename{\ensuremath{#1}I$\syn$}}
\newcommand{\Elimrulename}[1]{\Tyrulename{\ensuremath{#1}E}}
\newcommand{\Apprulename}[1]{\Tyrulename{\ensuremath{#1}App}}

\newcommand{\Var}{\Tyrulename{Var}}
\newcommand{\Sub}{\Tyrulename{Sub}}
\newcommand{\Anno}{\Tyrulename{Anno}}

\newcommand{\SubstSyn}{\Tyrulename{Subst$\syn$}}
\newcommand{\SubstChk}{\Tyrulename{Subst$\chk$}}

\newcommand{\UnitIntro}{\Introrulename{\unitty}}
\newcommand{\UnitIntroSyn}{\IntroSynrulename{\unitty}}

\newcommand{\ArrIntro}{\Introrulename{\arr}}
\newcommand{\ArrIntroSyn}{\IntroSynrulename{\arr}}
\newcommand{\ArrElim}{\Elimrulename{\arr}}

\newcommand{\AllIntro}{\Introrulename{\AllSym}}
\newcommand{\AllElim}{\Elimrulename{\AllSym}}

\newcommand{\ArrApp}{\Apprulename{\arr}}
\newcommand{\AllApp}{\Apprulename{\forall}}
\newcommand{\SubstApp}{\Apprulename{\rulename{Subst}}}
\newcommand{\SolveApp}{\Apprulename{\ahat}}


\newcommand{\subtermofsym}{\preceq}
\newcommand{\subtermof}{\mathrel{\subtermofsym}}
\newcommand{\propersubtermofsym}{\prec}
\newcommand{\propersubtermof}{\mathrel{\propersubtermofsym}}
\newcommand{\subtermofPf}[3] {\mkpf{\subtermof}{#1}{#2}{#3}}
\newcommand{\propersubtermofPf}[3] {\mkpf{\propersubtermof}{#1}{#2}{#3}}
\newcommand{\notsubtermofPf}[3] {\mkpf{\not\subtermof}{#1}{#2}{#3}}
\newcommand{\notpropersubtermofPf}[3] {\mkpf{\not\propersubtermof}{#1}{#2}{#3}}

\newcommand{\occursinsidearrow}{{\hspace{0.6ex}\raisebox{-0.4ex}{\ensuremath{\propersubtermof\rput[b](-1.35ex,1.2ex){\ensuremath{\mathsz{1.4ex}{\arr}}}}}}}
\newcommand{\notoccursinsidearrow}{{\hspace{0.6ex}\raisebox{-0.4ex}{\ensuremath{\propersubtermof{\rput[b](-2.3ex,0.0ex){\ensuremath{\not}}}\rput[b](-1.35ex,1.2ex){\ensuremath{\mathsz{1.4ex}{\arr}}}}}}}

\newcommand{\occursinsidearrowPf}[3] {\mkpf{\!\occursinsidearrow\!}{#1}{#2}{#3}}
\newcommand{\notoccursinsidearrowPf}[3] {\mkpf{\!\notoccursinsidearrow\!}{#1}{#2}{#3}}


\newcommand{\Ctxsubrulename}[1]{\ensuremath{\rulename{CtxSub#1}}\xspace}
\newcommand{\CtxsubEmpty}{\Ctxsubrulename{Empty}}
\newcommand{\CtxsubUvar}{\Ctxsubrulename{Uvar}}
\newcommand{\CtxsubVar}{\Ctxsubrulename{Var}}

\newcommand{\AssignRuleName}[1]{\ensuremath{\rulename{A#1}}\xspace}
\newcommand{\AssignIntroName}[1]{\AssignRuleName{\ensuremath{#1}I}}
\newcommand{\AssignElimName}[1]{\AssignRuleName{\ensuremath{#1}E}}
\newcommand{\AssignVar}{\AssignRuleName{Var}}
\newcommand{\AssignUnit}{\AssignRuleName{Unit}}
\newcommand{\AssignArrIntro}{\AssignIntroName{\arr}}
\newcommand{\AssignArrElim}{\AssignElimName{\arr}}
\newcommand{\AssignAllIntro}{\AssignIntroName{\forall}}
\newcommand{\AssignAllElim}{\AssignElimName{\forall}}
\newcommand{\judge}[3]{{#1} \vdash {#2} : {#3}}

\newcommand{\citepSequentCalculus}{\citep{Gentzen35}}
\newcommand{\citetSequentCalculus}{\citet{Gentzen35}}
\newcommand{\citepSubformulaProperty}{\citep[p.\ 87]{Gentzen35}}
\newcommand{\citetSubformulaProperty}{\citet[p.\ 87]{Gentzen35}}


\newcommand{\Fsub}{\ensuremath{{\text{F}}_{\texttt{<:}}}\xspace}

\newcommand{\Csharp}{\ensuremath{\text{\textrm{C}}^\sharp}}

\newcommand{\MLF}{\ensuremath{\mathsf{ML}^\mathsf{F}}\xspace}


\newcommand{\contextappvar}[2]{{[{#1}]}^\dagger{#2}}
\newcommand{\contextapp}[2]{{[{#1}]{#2}}}

\newcommand{\soln}[1]{\left|{#1}\right|}    

\newcommand{\equivctxsym}{\simeq}
\newcommand{\equivctx}[2]{{#1} \equivctxsym {#2}}

\newcommand{\LOCALCOPY}[1]{\href{papers/#1}{\bf \textcolor{dGreen}{local copy}}}
         \usepackage{etoolbox}
        \title{Lemmas and Proofs for \\ ``Complete and Easy Bidirectional Typechecking
            \\
             for Higher-Rank Polymorphism''
            \\
            }
        \author{Jana Dunfield \and Neelakantan R. Krishnaswami}
        \date{June 2013\footnote{Recompiled in 2020 to correct the name of the first author.}}
        \renewcommand\listtheoremname{}
        \begin{document}
        \maketitle
        \appendix
\else
  \makeatletter
       \onecolumn

\renewcommand{\rmdefault}{bch}
       \renewcommand{\bfdefault}{b}

\@setfontsize{\normalsize}{10pt}{12pt}

        \appendix
  \makeatother
\fi




\tableofcontents

\clearpage











\section{Declarative Subtyping}




\subsection{Properties of Well-Formedness}

\begin{restatable}[Weakening]{proposition}{propweakening}
\XLabel{prop:weakening}
            If $\judgetp{\Psi}{A}$
            then
            $\judgetp{\Psi, \Psi'}{A}$
            by a derivation of the same size.
\end{restatable}

\begin{restatable}[Substitution]{proposition}{propsubst}
\XLabel{prop:subst}
            If $\judgetp{\Psi}{A}$ and $\judgetp{\Psi, \alpha, \Psi'}{B}$
            then
            $\judgetp{\Psi,\Psi'}{[A/\alpha]B}$.
\end{restatable}




\subsection{Reflexivity}

\begin{restatable}[Reflexivity of Declarative Subtyping]{lemma}{declarativereflexivity}
\XLabel{lem:declarative-reflexivity}
  Subtyping is reflexive: if $\judgetp{\Psi}{A}$ then
        $\declsubjudg{\Psi}{A}{A}$.
\end{restatable}


\subsection{Subtyping Implies Well-Formedness}

\begin{restatable}[Well-Formedness]{lemma}{declarativewellformed}
\XLabel{lem:declarative-well-formed}
  If $\declsubjudg{\Psi}{A}{B}$ then $\judgetp{\Psi}{A}$ and $\judgetp{\Psi}{B}$.
\end{restatable}


\subsection{Substitution}

\begin{restatable}[Substitution]{lemma}{declsubsubstitution}
\XLabel{lem:decl-sub-substitution}
   If $\judgetp{\Psi}{\tau}$ and $\declsubjudg{\Psi, \alpha, \Psi'}{A}{B}$
   then $\declsubjudg{\Psi, [\tau/\alpha]\Psi'}{[\tau/\alpha]A}{[\tau/\alpha]B}$.
\end{restatable}


\subsection{Transitivity}

\begin{restatable}[Transitivity of Declarative Subtyping]{lemma}{declarativetransitivity}
\XLabel{lem:declarative-transitivity}
   If $\declsubjudg{\Psi}{A}{B}$ and $\declsubjudg{\Psi}{B}{C}$ then $\declsubjudg{\Psi}{A}{C}$.
\end{restatable}






\subsection{Invertibility of \DsubAllR}

\begin{restatable}[Invertibility]{lemma}{declinvertibility}
\XLabel{lem:decl-invertibility}
~\\
If $\Dee$ derives $\declsubjudg{\Psi}{A}{\alltype{\beta}{B}}$
then
$\Dee'$ derives $\declsubjudg{\Psi, \beta}{A}{B}$
where $\Dee' < \Dee$.
\end{restatable}


\subsection{Non-Circularity and Equality}

\begin{definition}[Subterm Occurrence]  \XLabel{def:occurrence}
 ~\\  
  Let $A \subtermof B$ iff $A$ is a subterm of $B$. \\
  Let $A \propersubtermof B$ iff $A$ is a proper subterm of $B$ (that is, $A \subtermof B$ and $A \neq B$). \\
  Let $A \occursinsidearrow B$ iff $A$ occurs in $B$ inside an arrow, that is, there exist $B_1$, $B_2$
  such that $(B_1{\arr}B_2) \subtermof B$ and $A \subtermof B_k$ for some $k \in \{1, 2\}$.
\end{definition}

\begin{restatable}[Occurrence]{lemma}{occurrencelemma}  \XLabel{lem:occurrence}
  ~\\[-10pt]
  \begin{enumerate}[(i)]
  \item  If $\declsubjudg{\Psi}{A}{\tau}$ then $\tau \notoccursinsidearrow A$.
  \item  If $\declsubjudg{\Psi}{\tau}{B}$ then $\tau \notoccursinsidearrow B$.
  \end{enumerate}
\end{restatable}

\begin{restatable}[Monotype Equality]{lemma}{declmonotypeequality}
  \XLabel{lem:decl-monotype-equality}
   If $\declsubjudg{\Psi}{\sigma}{\tau}$ then $\sigma = \tau$.
\end{restatable}













\begin{definition}[Contextual Size] \XLabel{def:typesize}
  The size of $A$ with respect to a context $\Gamma$, written $\typesize{\Gamma}{A}$, is defined by
  \begin{mathpar}
    \begin{array}{lcl}
      \typesize{\Gamma}{\alpha}  & = &   1
      \\
      \typesize{{\Gamma[\ahat]}}{\ahat}  & = &   1
      \\
      \typesize{{\Gamma[\hypeq{\ahat}{\tau}]}}{\ahat} & = &   1 + \typesize{\Gamma[\hypeq{\ahat}{\tau}]}{\tau} 
      \\
      \typesize{\Gamma}{\alltype{\alpha}{A}} & = &   1 + \typesize{\Gamma, \alpha}{A}
      \\
      \typesize{\Gamma}{A \arr B} & = &   1 + \typesize{\Gamma}{A} + \typesize{\Gamma}{B}
    \end{array}
  \end{mathpar}  
\end{definition}










\section{Type Assignment}



\begin{restatable}[Well-Formedness]{lemma}{declarativetypingwellformed}
\XLabel{lem:declarative-typing-well-formed}
~\\
  If $\declchkjudg{\Psi}{e}{A}$
  or $\declsynjudg{\Psi}{e}{A}$
  or $\declappjudg{\Psi}{e}{A}{C}$ then
  $\judgetp{\Psi}{A}$ (and in the last case, $\judgetp{\Psi}{C}$).
\end{restatable}

\begin{restatable}[Completeness of Bidirectional Typing]{theorem}{completenessbidirectional}
  \XLabel{thm:completeness-bidirectional}
~\\
  If $\judge{\Psi}{e}{A}$ then there exists $e'$ such that $\declsynjudg{\Psi}{e'}{A}$
  and $|e'| = e$. 
\end{restatable}

\begin{restatable}[Subtyping Coercion]{lemma}{subtypingcoercion}
\XLabel{lem:subtyping-coercion}
  If $\declsubjudg{\Psi}{A}{B}$ then there exists $f$ which is $\beta\eta$-equal to the identity such that 
  $\judge{\Psi}{f}{A \arr B}$. 
\end{restatable}

\begin{restatable}[Application Subtyping]{lemma}{applicationsubtyping}
\XLabel{lem:application-subtyping}
  If $\declappjudg{\Psi}{e}{A}{C}$
  then there exists $B$ such that $\declsubjudg{\Psi}{A}{B \arr C}$
  and $\declchkjudg{\Psi}{e}{B}$ by a smaller derivation.
\end{restatable}

\begin{restatable}[Soundness of Bidirectional Typing]{theorem}{soundnessbidirectional}
  \XLabel{thm:soundness-bidirectional}
  We have that:

  \begin{itemize}
  \item  If $\declchkjudg{\Psi}{e}{A}$, then there is an $e'$ such that $\judge{\Psi}{e'}{A}$
    and $e' \beeq  |e|$.
  \item  If $\declsynjudg{\Psi}{e}{A}$, then there is an $e'$ such that $\judge{\Psi}{e'}{A}$
    and $e' \beeq  |e|$.
  \end{itemize}
\end{restatable}



\section{Robustness of Typing}

\begin{restatable}[Type Substitution]{lemma}{typesubstitution}
\XLabel{lem:type-substitution}
~\\
  Assume $\judgetp{\Psi}{\tau}$.

  \begin{itemize}
  \item If\, $\declchkjudg{\Psi, \alpha, \Psi'}{e'}{C}$ then\, $\declchkjudg{\Psi, [\tau/\alpha]\Psi'}{[\tau/\alpha]e'}{[\tau/\alpha]C}$.
  \item If\, $\declsynjudg{\Psi, \alpha, \Psi'}{e'}{C}$ then\, $\declsynjudg{\Psi, [\tau/\alpha]\Psi'}{[\tau/\alpha]e'}{[\tau/\alpha]C}$.
  \item If\, $\declappjudg{\Psi, \alpha, \Psi'}{e'}{B}{C}$ then\, $\declappjudg{\Psi, [\tau/\alpha]\Psi'}{[\tau/\alpha]e'}{[\tau/\alpha]B}{[A/\alpha]C}$.
  \end{itemize}   

  Moreover, the resulting derivation contains no more applications of typing rules than the given one.
  (Internal subtyping derivations, however, may grow.)
\end{restatable}

\bigskip

\begin{definition}[Context Subtyping]
\XLabel{def:context-subtyping}
    We define the judgment $\Psi' \leq \Psi$ with the following rules:
    \\[-0.5ex]
    \begin{mathpar}
      \Infer{\CtxsubEmpty}
                { }
                {\cdot \leq \cdot}
      \and
      \Infer{\CtxsubUvar}
                {\Psi' \leq \Psi }
                {\Psi', \alpha \leq \Psi, \alpha}
      \and
      \Infer{\CtxsubVar}
                {\Psi' \leq \Psi
                  \\
                  \declsubjudg{\Psi}{A'}{A}}
                {\Psi', x:A' \leq \Psi, x:A}
    \end{mathpar}
\end{definition}

\medskip

\begin{restatable}[Subsumption]{lemma}{termsubsumption}
\XLabel{lem:term-subsumption} 
    Suppose $\Psi' \leq \Psi$. Then:

    \begin{enumerate}[(i)]
    \item If $\declchkjudg{\Psi}{e}{A}$ and $\declsubjudg{\Psi}{A}{A'}$ then $\declchkjudg{\Psi'}{e}{A'}$.
    \item If $\declsynjudg{\Psi}{e}{A}$ then 
          there exists $A'$ such that $\declsubjudg{\Psi}{A'}{A}$ and $\declsynjudg{\Psi'}{e}{A'}$.
    \item If $\declappjudg{\Psi}{e}{C}{A}$ and $\declsubjudg{\Psi}{C'}{C}$
      \\ then 
       there exists $A'$ such that $\declsubjudg{\Psi}{A'}{A}$ and $\declappjudg{\Psi'}{e}{C'}{A'}$.
    \end{enumerate}
\end{restatable}


\begin{restatable}[Substitution]{theorem}{termsubstitution}
\XLabel{thm:term-substitution}
~\\
  Assume $\declsynjudg{\Psi}{e}{A}$.
~\\[-3ex]
  \begin{enumerate}[(i)]
  \item If $\,\declchkjudg{\Psi, x:A}{e'}{C}$ then $\,\declchkjudg{\Psi}{[e/x]e'}{C}$.
  \item If $\,\declsynjudg{\Psi, x:A}{e'}{C}$ then $\,\declsynjudg{\Psi}{[e/x]e'}{C}$.
  \item If $\,\declappjudg{\Psi, x:A}{e'}{B}{C}$ then $\,\declappjudg{\Psi}{[e/x]e'}{B}{C}$.
\end{enumerate}
\end{restatable}

\begin{restatable}[Inverse Substitution]{theorem}{termUnsubstitution}
\XLabel{thm:term-unsubstitution}
~\\
Assume $\declchkjudg{\Psi}{e}{A}$.
\begin{enumerate}[(i)]
  \item If $\,\declchkjudg{\Psi}{[(e:A)/x]e'}{C}$ then $\,\declchkjudg{\Psi, x:A}{e'}{C}$.
  \item If $\,\declsynjudg{\Psi}{[(e:A)/x]e'}{C}$ then $\,\declsynjudg{\Psi, x:A}{e'}{C}$.
  \item If $\,\declappjudg{\Psi}{[(e:A)/x]e'}{B}{C}$ then $\,\declappjudg{\Psi, x:A}{e'}{B}{C}$.
\end{enumerate}
\end{restatable}


\begin{restatable}[Annotation Removal]{theorem}{annotationRemoval}
\XLabel{thm:annotation-dropping}
  We have that:

  \begin{itemize}
    \item If\, $\declchkjudg{\Psi}{\big((\fun{x}{e}) : A\big)}{C}$ then $\declchkjudg{\Psi}{\fun{x}{e}}{C}$.
    \item If\, $\declchkjudg{\Psi}{(\unitexp : A)}{C}$ then $\declchkjudg{\Psi}{\unitexp}{C}$.
    \item If\, $\declsynjudg{\Psi}{e_1\;(e_2 : A)}{C}$ then $\declsynjudg{\Psi}{e_1\;e_2}{C}$.
    \item If\, $\declsynjudg{\Psi}{(x : A)}{A}$ then $\declsynjudg{\Psi}{x}{B}$ and $\declsubjudg{\Psi}{B}{A}$.
    \item If\, $\declsynjudg{\Psi}{\big((e_1\,e_2) : A\big)}{A}$
       then $\declsynjudg{\Psi}{e_1\,e_2}{B}$
      and $\declsubjudg{\Psi}{B}{A}$. 
    \item If\, $\declsynjudg{\Psi}{\big((e : B) : A\big)}{A}$
      then $\declsynjudg{\Psi}{(e : B)}{B}$
      and $\declsubjudg{\Psi}{B}{A}$. 
    \item If $\declsynjudg{\Psi}{\big((\fun{x}{e}) : \sigma \arr \tau\big)}{\sigma \arr \tau}$
      then $\declsynjudg{\Psi}{\fun{x}{e}}{\sigma \arr \tau}$.
  \end{itemize}
\end{restatable}


\begin{restatable}[Soundness of Eta]{theorem}{soundnessofeta}
~\\\Label{thm:soundness-of-eta}
  If\, $\declchkjudg{\Psi}{\fun{x}{e\;x}}{A}$ and $x \not\in \FV{e}$, then $\declchkjudg{\Psi}{e}{A}$.   
\end{restatable}



\section{Properties of Context Extension}




\subsection{Syntactic Properties}

\begin{restatable}[Declaration Preservation]{lemma}{declarationpreservation}
\XLabel{lem:declaration-preservation}
  If $\substextend{\Gamma}{\Delta}$,
  and $u$ is a variable or marker $\MonnierComma{\ahat}$ declared in $\Gamma$, then $u$ is declared in $\Delta$. 
\end{restatable}

\begin{restatable}[Declaration Order Preservation]{lemma}{declarationorderpreservation}
\XLabel{lem:declaration-order-preservation}
  If $\substextend{\Gamma}{\Delta}$ 
  and $u$ is declared to the left of $v$ in $\Gamma$,
  then $u$ is declared to the left of $v$ in $\Delta$. 
\end{restatable}

\begin{restatable}[Reverse Declaration Order Preservation]{lemma}{reversedeclarationorderpreservation}
\XLabel{lem:reverse-declaration-order-preservation}
  If $\substextend{\Gamma}{\Delta}$ and $u$ and $v$ are both declared in $\Gamma$
  and $u$ is declared to the left of $v$ in $\Delta$,
  then $u$ is declared to the left of $v$ in $\Gamma$. 
\end{restatable}


\begin{restatable}[Substitution Extension Invariance]{lemma}{substitutionextensioninvariance}
\XLabel{lem:subst-extension-invariance}
   If $\judgetp{\Theta}{A}$ and $\substextend{\Theta}{\Gamma}$
   then
   $[\Gamma]A = [\Gamma]([\Theta]A)$
   and $[\Gamma]A = [\Theta]([\Gamma]A)$.
\end{restatable}

\begin{restatable}[Extension Equality Preservation]{lemma}{extensionequalitypreservation}
\XLabel{lem:extension-equality-preservation}
~\\
   If $\judgetp{\Gamma}{A}$
   and $\judgetp{\Gamma}{B}$ 
   and $[\Gamma]A = [\Gamma]B$
   and $\substextend{\Gamma}{\Delta}$,
   then 
   $[\Delta]A = [\Delta]B$.   
\end{restatable}


\begin{restatable}[Reflexivity]{lemma}{substextendreflexivity}
\XLabel{lem:substextend-reflexivity}
  If $\Gamma$ is well-formed, then $\substextend{\Gamma}{\Gamma}$. 
\end{restatable}



\begin{restatable}[Transitivity]{lemma}{substextendtransitivity}
\XLabel{lem:substextend-transitivity}
  If $\substextend{\Gamma}{\Delta}$
  and $\substextend{\Delta}{\Theta}$,
  then $\substextend{\Gamma}{\Theta}$.
\end{restatable}




\begin{definition}[Softness]  \XLabel{def:soft}
  A context $\Theta$ is \emph{soft} iff it consists only of $\ahat$ and $\hypeq{\ahat}{\tau}$
  declarations.
\end{definition}

\begin{restatable}[Right Softness]{lemma}{rightsoftness}  \XLabel{lem:softness}
  If $\substextend{\Gamma}{\Delta}$ and $\Theta$ is soft
  (and $(\Delta, \Theta)$ is well-formed)
  then $\substextend{\Gamma}{\Delta, \Theta}$.
\end{restatable}




\begin{restatable}[Evar Input]{lemma}{evarinput}  \XLabel{lem:evar-input}
~\\
  If $\substextend{\Gamma, \ahat}{\Delta}$
  then $\Delta = (\Delta_0, \Delta_{\ahat}, \Theta)$ where
  $\substextend{\Gamma}{\Delta_0}$,
  and $\Delta_{\ahat}$ is either $\ahat$ or $\hypeq{\ahat}{\tau}$,
  and $\Theta$ is soft.
\end{restatable}









\begin{restatable}[Extension Order]{lemma}{extensionorder} 
\XLabel{lem:extension-order}
~\\[-10pt]
  \begin{enumerate}[(i)]
    \item
      If $\substextend{\Gamma_L, \alpha, \Gamma_R}{\Delta}$
      then $\Delta = (\Delta_L, \alpha, \Delta_R)$
      where $\substextend{\Gamma_L}{\Delta_L}$. \\
      Moreover, if $\Gamma_R$ is soft then $\Delta_R$ is soft.
      
    \item
      If $\substextend{\Gamma_L, \MonnierComma{\ahat}, \Gamma_R}{\Delta}$
      then $\Delta = (\Delta_L, \MonnierComma{\ahat}, \Delta_R)$
      where $\substextend{\Gamma_L}{\Delta_L}$. \\
      Moreover, if $\Gamma_R$ is soft then $\Delta_R$ is soft.

    \item
      If $\substextend{\Gamma_L, \ahat, \Gamma_R}{\Delta}$
      then $\Delta = \Delta_L, \Theta, \Delta_R$
      where $\substextend{\Gamma_L}{\Delta_L}$
      and $\Theta$ is either $\ahat$ or $\hypeq{\ahat}{\tau}$ for some $\tau$.

    \item
      If $\substextend{\Gamma_L, \hypeq{\ahat}{\tau}, \Gamma_R}{\Delta}$
      then $\Delta = \Delta_L, \hypeq{\ahat}{\tau'}, \Delta_R$
      where $\substextend{\Gamma_L}{\Delta_L}$
      and $[\Delta_L]\tau = [\Delta_L]\tau'$.

    \item
      If $\substextend{\Gamma_L, x : A, \Gamma_R}{\Delta}$
      then $\Delta = (\Delta_L, x : A', \Delta_R)$
      where $\substextend{\Gamma_L}{\Delta_L}$
          and $[\Delta_L]A = [\Delta_L]A'$. \\
      Moreover, $\Gamma_R$ is soft if and only if $\Delta_R$ is soft.
  \end{enumerate}
\end{restatable}

\begin{restatable}[Extension Weakening]{lemma}{extensionweakening}
  \XLabel{lem:extension-weakening}
  If $\judgetp{\Gamma}{A}$ and $\substextend{\Gamma}{\Delta}$ then $\judgetp{\Delta}{A}$. 
\end{restatable}



\begin{restatable}[Solution Admissibility for Extension]{lemma}{extensionsolve}
\XLabel{lem:extension-solve}
  If $\judgetp{\Gamma_L}{\tau}$
  then
  $\substextend{\Gamma_L, \ahat, \Gamma_R}{\Gamma_L, \hypeq{\ahat}{\tau}, \Gamma_R}$. 
\end{restatable}

\begin{restatable}[Solved Variable Addition for Extension]{lemma}{extensionaddsolve}
\XLabel{lem:extension-addsolve}
  If $\judgetp{\Gamma_L}{\tau}$
  then
  $\substextend{\Gamma_L, \Gamma_R}{\Gamma_L, \hypeq{\ahat}{\tau}, \Gamma_R}$. 
\end{restatable}

\begin{restatable}[Unsolved Variable Addition for Extension]{lemma}{extensionadd}
\XLabel{lem:extension-add}
   We have that $\substextend{\Gamma_L, \Gamma_R}{\Gamma_L, \ahat, \Gamma_R}$. 
\end{restatable}


\begin{restatable}[Parallel Admissibility]{lemma}{parallelevaradmissibility}
\XLabel{lem:parallel-admissibility} ~\\
    If $\substextend{\Gamma_L}{\Delta_L}$ and $\substextend{\Gamma_L, \Gamma_R}{\Delta_L, \Delta_R}$ then:
    
    \begin{enumerate}[(i)]
    \item $\substextend{\Gamma_L, \ahat, \Gamma_R}{\Delta_L, \ahat, \Delta_R}$
    \item If $\judgetp{\Delta_L}{\tau'}$ then 
          $\substextend{\Gamma_L, \ahat, \Gamma_R}{\Delta_L, \ahat=\tau', \Delta_R}$.
    \item If $\judgetp{\Gamma_L}{\tau}$ and $\judgetp{\Delta_L}{\tau'}$ and $[\Delta_L]\tau = [\Delta_L]\tau'$,
          then $\substextend{\Gamma_L, \ahat=\tau, \Gamma_R}{\Delta_L, \ahat=\tau', \Delta_R}$. 
    \end{enumerate}
\end{restatable}

\begin{restatable}[Parallel Extension Solution]{lemma}{parallelextensionsolve}
\XLabel{lem:parallel-extension-solve} ~\\
  If $\substextend{\Gamma_L, \ahat, \Gamma_R}{\Delta_L, \ahat=\tau', \Delta_R}$
  and $\judgetp{\Gamma_L}{\tau}$
  and $[\Delta_L]\tau = [\Delta_L]\tau'$
  then $\substextend{\Gamma_L, \ahat=\tau, \Gamma_R}{\Delta_L, \ahat=\tau', \Delta_R}$. 
\end{restatable}



\begin{restatable}[Parallel Variable Update]{lemma}{parallelextensionupdate}
\XLabel{lem:parallel-extension-update} ~\\
  If ~~~\, $\substextend{\Gamma_L, \ahat, \Gamma_R}{\Delta_L, \ahat=\tau_0, \Delta_R}$
  and $\judgetp{\Gamma_L}{\tau_1}$
  and $\judgetp{\Delta_L}{\tau_2}$
  and $[\Delta_L]\tau_0 = [\Delta_L]\tau_1 = [\Delta_L]\tau_2$
  \\
  then $\substextend{\Gamma_L, \ahat=\tau_1, \Gamma_R}{\Delta_L, \ahat=\tau_2, \Delta_R}$. 
\end{restatable}



\subsection{Instantiation Extends}

\begin{restatable}[Instantiation Extension]{lemma}{instantiationextension}   \XLabel{lem:instantiation-extension}
~\\
   If $\instjudg{\Gamma}{\ahat}{\tau}{\Delta}$ or $\instjudgr{\Gamma}{\ahat}{\tau}{\Delta}$
   then
$\substextend{\Gamma}{\Delta}$.
\end{restatable}


\subsection{Subtyping Extends}

\begin{restatable}[Subtyping Extension]{lemma}{subtypingextension}   \XLabel{lem:subtyping-extension}
~\\
   If $\subjudg{\Gamma}{A}{B}{\Delta}$ then
$\substextend{\Gamma}{\Delta}$.
 \end{restatable}








\section{Decidability of Instantiation}




\begin{restatable}[Left Unsolvedness Preservation]{lemma}{leftunsolvednesspreservation}
\XLabel{lem:left-unsolvedness-preservation}
~\\
  If $\instjudg{\underbrace{\Gamma_0, \ahat, \Gamma_1}_\Gamma}{\ahat}{A}{\Delta}$
     or $\instjudgr{\underbrace{\Gamma_0, \ahat, \Gamma_1}_\Gamma}{\ahat}{A}{\Delta}$,
  and $\bhat \in \unsolved{\Gamma_0}$, then $\bhat \in \unsolved{\Delta}$.
\end{restatable}


\begin{restatable}[Left Free Variable Preservation]{lemma}{leftfreevariablepreservation}
\XLabel{lem:left-free-variable-preservation}
  If $\instjudg{\overbrace{\Gamma_0, \ahat, \Gamma_1}^\Gamma}{\ahat}{A}{\Delta}$
     or $\instjudgr{\overbrace{\Gamma_0, \ahat, \Gamma_1}^\Gamma}{\ahat}{A}{\Delta}$,
     and $\judgetp{\Gamma}{B}$ and $\ahat \notin \FV{[\Gamma]B}$
     and $\bhat \in \unsolved{\Gamma_0}$ and $\bhat \notin \FV{[\Gamma]B}$,
     then $\bhat \notin \FV{[\Delta]B}$. 
\end{restatable}


\begin{restatable}[Instantiation Size Preservation]{lemma}{instantiationsizepreservation}
\XLabel{lem:instantiation-size-preservation}
  If $\instjudg{\overbrace{\Gamma_0, \ahat, \Gamma_1}^\Gamma}{\ahat}{A}{\Delta}$
     or $\instjudgr{\overbrace{\Gamma_0, \ahat, \Gamma_1}^\Gamma}{\ahat}{A}{\Delta}$,
     and $\judgetp{\Gamma}{B}$ and $\ahat \notin \FV{[\Gamma]B}$, 
     then $|[\Gamma]B| = |[\Delta]B|$, where $|C|$ is the plain size of the term $C$.
\end{restatable}

This lemma lets us show decidability by taking the size of the type argument as the induction
metric. 

\begin{restatable}[Decidability of Instantiation]{theorem}{instantiationdecidability}
\XLabel{thm:decidability-of-instantiation}
    If $\Gamma = \Gamma_0[\ahat]$
    and
    $\judgetp{\Gamma}{A}$
    such that $[\Gamma]A = A$
    and $\ahat \notin \FV{A}$, then:
    
    \begin{enumerate}[(1)]
    \item Either there exists $\Delta$ such that $\instjudg{\Gamma_0[\ahat]}{\ahat}{A}{\Delta}$,
      or not.
    \item Either there exists $\Delta$ such that $\instjudgr{\Gamma_0[\ahat]}{\ahat}{A}{\Delta}$,
      or not.
    \end{enumerate}
\end{restatable}



\section{Decidability of Algorithmic Subtyping}

\subsection{Lemmas for Decidability of Subtyping}

\begin{restatable}[Monotypes Solve Variables]{lemma}{monotypessolvevariables}
\XLabel{lem:monotypes-solve-variables}
If $\instjudg{\Gamma}{\ahat}{\tau}{\Delta}$ or $\instjudgr{\Gamma}{\ahat}{\tau}{\Delta}$, then if $[\Gamma]\tau = \tau$ and $\ahat \notin \FV{[\Gamma]\tau}$, then $|\unsolved{\Gamma}| = |\unsolved{\Delta}| + 1$. 
\end{restatable}


\begin{restatable}[Monotype Monotonicity]{lemma}{monotypemonotonicity}  \XLabel{lem:monotype-monotonicity}
  If $\subjudg{\Gamma}{\tau_1}{\tau_2}{\Delta}$
  then
  $|\unsolved{\Delta}| \leq |\unsolved{\Gamma}|$.
\end{restatable}


\begin{restatable}[Substitution Decreases Size]{lemma}{typesizesubst}
\XLabel{lem:typesize-subst}
  If $\judgetp{\Gamma}{A}$ then $\typesize{\Gamma}{[\Gamma]A} \leq \typesize{\Gamma}{A}$.
\end{restatable}

\begin{restatable}[Monotype Context Invariance]{lemma}{monotypecontextinvariance}
\XLabel{lem:monotype-context-invariance}
~\\
   If $\subjudg{\Gamma}{\tau}{\tau'}{\Delta}$
   where $[\Gamma]\tau = \tau$ and $[\Gamma]\tau' = \tau'$
   and $|\unsolved{\Gamma}| = |\unsolved{\Delta}|$
   then $\Gamma = \Delta$. 
\end{restatable}


\subsection{Decidability of Subtyping}

\begin{restatable}[Decidability of Subtyping]{theorem}{subtypingdecidability}   
\XLabel{thm:subtyping-decidable}
~\\
  Given a context $\Gamma$ and types $A$, $B$ such that $\judgetp{\Gamma}{A}$ and
  $\judgetp{\Gamma}{B}$ and $[\Gamma]A = A$ and $[\Gamma]B = B$,
  it is decidable whether there exists $\Delta$ such that $\subjudg{\Gamma}{A}{B}{\Delta}$.
\end{restatable}



\section{Decidability of Typing}



\begin{restatable}[Decidability of Typing]{theorem}{typingdecidable}
\XLabel{thm:typing-decidable}
~\\[-2ex]
  \begin{enumerate}[(i)]
      \item \emph{Synthesis:}  
        Given a context $\Gamma$
        and a term $e$,
        \\
        it is decidable whether there exist a type $A$ and a context $\Delta$ such that \\
        $\synjudg{\Gamma}{e}{A}{\Delta}$.

      \item \emph{Checking:}
        Given a context $\Gamma$,
        a term $e$,
        and a type $B$ such that $\judgetp{\Gamma}{B}$,
        \\
        it is decidable whether there is a context $\Delta$ such that \\
        $\chkjudg{\Gamma}{e}{B}{\Delta}$.

      \item \emph{Application:}  
        Given a context $\Gamma$,
        a term $e$,
        and a type $A$ such that $\judgetp{\Gamma}{A}$,
        \\
        it is decidable whether there exist a type $C$ and a context $\Delta$ such that \\
        $\appjudg{\Gamma}{e}{A}{C}{\Delta}$.
  \end{enumerate}
\end{restatable}



\section{Soundness of Subtyping}





\begin{definition}[Filling]
  The \emph{filling} of a context $\soln{\Gamma}$ solves all unsolved variables:

  \begin{displ}
        \begin{array}[t]{lcll}
            \soln{\cdot} & = &   \cdot
            \\
            \soln{\Gamma, x : A} & = &   \soln{\Gamma}, x : A
            \\
            \soln{\Gamma, \alpha} & = &   \soln{\Gamma}, \alpha
            \\
            \soln{\Gamma, \hypeq{\ahat}{\tau}} & = &   \soln{\Gamma}, \hypeq{\ahat}{\tau}
            \\
            \soln{\Gamma, \MonnierComma{\ahat}} & = &   \soln{\Gamma}, \MonnierComma{\ahat}
            \\
            \soln{\Gamma, \ahat} & = &   \soln{\Gamma}, \hypeq{\ahat}{\unitty}
        \end{array}
  \end{displ}
\end{definition}


\subsection{Lemmas for Soundness}

\begin{restatable}[Uvar Preservation]{lemma}{ctxappuvarpreservation}
\XLabel{lem:ctxapp-uvar-preservation} ~\\
  If $\alpha \in \Omega$
  and $\substextend{\Delta}{\Omega}$
  then
  $\alpha \in [\Omega]\Delta$.
\end{restatable}
\begin{proof}  By induction on $\Omega$, following the definition of context application.
\end{proof}


\begin{restatable}[Variable Preservation]{lemma}{ctxappvarpreservation}
\XLabel{lem:ctxapp-var-preservation} ~\\
  If $(x : A) \in \Delta$ or $(x : A) \in \Omega$
  and $\substextend{\Delta}{\Omega}$
  then
  $(x : [\Omega]A) \in [\Omega]\Delta$.
\end{restatable}


\begin{restatable}[Substitution Typing]{lemma}{substitutiontyping}
\XLabel{lem:substitution-typing}
  If $\judgetp{\Gamma}{A}$ then $\judgetp{\Gamma}{[\Gamma]A}$. 
\end{restatable}






\begin{restatable}[Substitution for Well-Formedness]{lemma}{completionwf}
\XLabel{lem:completion-wf}
  If $\judgetp{\Omega}{A}$
  then
  $\judgetp{[\Omega]\Omega}{[\Omega]A}$.
\end{restatable}


\begin{restatable}[Substitution Stability]{lemma}{substitutionstability}   \XLabel{lem:substitution-stability}
~\\
  For any well-formed complete context $(\Omega, \Omega_Z)$,
  if $\judgetp{\Omega}{A}$ then $[\Omega]A = [\Omega, \Omega_Z]A$.
\end{restatable}

\begin{restatable}[Context Partitioning]{lemma}{contextpartitioning}   \XLabel{lem:context-partitioning}
  ~\\
  If $\substextend{\Delta, \MonnierComma{\ahat}, \Theta}{\Omega, \MonnierComma{\ahat}, \Omega_Z}$
  then
  there is a $\Psi$ such that
  $[\Omega, \MonnierComma{\ahat}, \Omega_Z](\Delta, \MonnierComma{\ahat}, \Theta)
   =
   [\Omega]\Delta, \Psi$. 
\end{restatable}

\begin{restatable}[Softness Goes Away]{lemma}{softnessgoesaway}  \XLabel{lem:softness-goes-away}
  ~\\
  If $\substextend{\Delta, \Theta}{\Omega, \Omega_Z}$
  where $\substextend{\Delta}{\Omega}$ and $\Theta$ is soft,
  then
  $[\Omega, \Omega_Z](\Delta, \Theta) = [\Omega]\Delta$.
\end{restatable}
\begin{proof}
  By induction on $\Theta$, following the definition of $[\Omega]\Gamma$.
\end{proof}








\begin{lemma}[Filling Completes]  \XLabel{lem:soln-completes}
  If $\substextend{\Gamma}{\Omega}$
  and $(\Gamma, \Theta)$ is well-formed,
  then
  $\substextend{\Gamma, \Theta}{\Omega, \soln{\Theta}}$.
\end{lemma}
\begin{proof}
  By induction on $\Theta$, following the definition of $\soln{-}$ and applying the rules
  for $\extendssym$.
\end{proof}








\begin{restatable}[Stability of Complete Contexts]{lemma}{completesstability}
  \XLabel{lem:completes-stability}  ~\\
  If $\substextend{\Gamma}{\Omega}$
  then $[\Omega]\Gamma = [\Omega]\Omega$. 
\end{restatable}

\begin{restatable}[Finishing Types]{lemma}{finishingtypes}
  \XLabel{lem:finishing-types} ~\\
  If
$\judgetp{\Omega}{A}$
and $\substextend{\Omega}{\Omega'}$
  then
  $[\Omega]A = [\Omega']A$. 
\end{restatable} 

\begin{restatable}[Finishing Completions]{lemma}{finishingcompletions}
  \XLabel{lem:finishing-completions}  ~\\
  If $\substextend{\Omega}{\Omega'}$
  then
  $[\Omega]\Omega = [\Omega']\Omega'$. 
\end{restatable}


\begin{restatable}[Confluence of Completeness]{lemma}{confluenceofcompleteness}    \XLabel{lem:completes-confluence}  ~\\
  If $\substextend{\Delta_1}{\Omega}$
  and $\substextend{\Delta_2}{\Omega}$
  then
  $[\Omega]\Delta_1 = [\Omega]\Delta_2$.
\end{restatable}



\subsection{Instantiation Soundness}

\begin{restatable}[Instantiation Soundness]{theorem}{instantiationsoundness}  \XLabel{thm:instantiation-soundness}
   ~\\
    Given $\substextend{\Delta}{\Omega}$ and $[\Gamma]B = B$ and $\ahat \notin \FV{B}$: \\[-2.5ex]
\begin{enumerate}[(1)]
            \item If $\instjudg{\Gamma}{\ahat}{B}{\Delta}$ 
              then
              $\declsubjudg{[\Omega]\Delta}{[\Omega]\ahat}{[\Omega]B}$.

            \item If $\instjudgr{\Gamma}{\ahat}{B}{\Delta}$ 
              then
              $\declsubjudg{[\Omega]\Delta}{[\Omega]B}{[\Omega]\ahat}$.
        \end{enumerate}
\end{restatable}

\subsection{Soundness of Subtyping}

\begin{restatable}[Soundness of Algorithmic Subtyping]{theorem}{soundness} 
\XLabel{thm:subtyping-soundness} ~\\
If $\subjudg{\Gamma}{A}{B}{\Delta}$
  where $[\Gamma]A = A$ and $[\Gamma]B = B$
  and $\substextend{\Delta}{\Omega}$
then
  $\declsubjudg{\contextapp{\Omega}{\Delta}}{\contextapp{\Omega}{A}}{\contextapp{\Omega}{B}}$. 
\end{restatable}


\begin{restatable}[Soundness, Pretty Version]{corollary}{soundnesspretty}
  If $\subjudg{\Psi}{A}{B}{\Delta}$, then $\declsubjudg{\Psi}{A}{B}$. 
\end{restatable}



\section{Typing Extension}




\begin{restatable}[Typing Extension]{lemma}{typingextension}
\XLabel{lem:typing-extension}
 ~\\
  If $\chkjudg{\Gamma}{e}{A}{\Delta}$
  or $\synjudg{\Gamma}{e}{A}{\Delta}$
  or $\appjudg{\Gamma}{e}{A}{C}{\Delta}$
  then
  $\substextend{\Gamma}{\Delta}$.
\end{restatable}


\section{Soundness of Typing}

\begin{restatable}[Soundness of Algorithmic Typing]{theorem}{typingsoundness}
\XLabel{lem:typing-soundness} Given $\substextend{\Delta}{\Omega}$: \\[-2.5ex]
\begin{enumerate}[(i)]
      \item 
          If $\chkjudg{\Gamma}{e}{A}{\Delta}$
then
          $\declchkjudg{[\Omega]\Delta}{e}{[\Omega]A}$. 
      
      \item
          If $\synjudg{\Gamma}{e}{A}{\Delta}$
then
          $\declsynjudg{[\Omega]\Delta}{e}{[\Omega]A}$. 

      \item
          If $\appjudg{\Gamma}{e}{A}{C}{\Delta}$
then
          $\declappjudg{[\Omega]\Delta}{e}{[\Omega]A}{[\Omega]C}$.
    \end{enumerate}
\end{restatable}



\section{Completeness of Subtyping}





\subsection{Instantiation Completeness}

\begin{restatable}[Instantiation Completeness]{theorem}{instantiationcompletes}
   \XLabel{thm:instantiation-completes}
   ~\\
   Given $\substextend{\Gamma}{\Omega}$ 
   and $A = [\Gamma]A$ 
   and $\ahat \in \unsolved{\Gamma}$  and $\ahat \notin \FV{A}$:
   
       \begin{enumerate}[(1)]
           \item If $\declsubjudg{[\Omega]\Gamma}{[\Omega]\ahat}{[\Omega]A}$
                 \\
                 then there are $\Delta$, $\Omega'$ such that
                 $\substextend{\Omega}{\Omega'}$
                 and $\substextend{\Delta}{\Omega'}$
                 and $\instjudg{\Gamma}{\ahat}{A}{\Delta}$. 

           \item If $\declsubjudg{[\Omega]\Gamma}{[\Omega]A}{[\Omega]\ahat}$
                 \\
                 then there are $\Delta$, $\Omega'$ such that
                 $\substextend{\Omega}{\Omega'}$
                 and $\substextend{\Delta}{\Omega'}$
                 and $\instjudgr{\Gamma}{\ahat}{A}{\Delta}$. 
      \end{enumerate}                                                    
\end{restatable}





\subsection{Completeness of Subtyping}

\begin{restatable}[Generalized Completeness of Subtyping]{theorem}{completingcompleteness}
  \XLabel{thm:completing-completeness}
  If $\substextend{\Gamma}{\Omega}$
  and $\judgetp{\Gamma}{A}$
  and $\judgetp{\Gamma}{B}$
  and $\declsubjudg{[\Omega]\Gamma}{[\Omega]A}{[\Omega]B}$
  then
  there exist $\Delta$ and $\Omega'$ such that 
  $\substextend{\Delta}{\Omega'}$ and 
  $\substextend{\Omega}{\Omega'}$ and 
  $\subjudg{\Gamma}{[\Gamma]A}{[\Gamma]B}{\Delta}$.
\end{restatable}


\begin{restatable}[Completeness of Subtyping]{corollary}{completeness}
  If $\declsubjudg{\Psi}{A}{B}$
  then
  there is a $\Delta$ such that
  $\subjudg{\Psi}{A}{B}{\Delta}$. 
\end{restatable}





\section{Completeness of Typing}
\Label{sec:last-section-before-proofs}


\begin{restatable}[Completeness of Algorithmic Typing]{theorem}{typingcompleteness}
  \XLabel{thm:typing-completeness}
  Given $\substextend{\Gamma}{\Omega}$ and $\judgetp{\Gamma}{A}$:
  \\[-2ex]
    \begin{enumerate}[(i)]
      \item
           If $\declchkjudg{[\Omega]\Gamma}{e}{[\Omega]A}$
           \\
           then there exist $\Delta$ and $\Omega'$
           \\
           such that
           $\substextend{\Delta}{\Omega'}$ and 
           $\substextend{\Omega}{\Omega'}$ and 
           $\chkjudg{\Gamma}{e}{[\Gamma]A}{\Delta}$.
      
      \item
           If $\declsynjudg{[\Omega]\Gamma}{e}{A}$
           \\
           then there exist $\Delta$, $\Omega'$, and $A'$
           \\
           such that
           $\substextend{\Delta}{\Omega'}$ and 
           $\substextend{\Omega}{\Omega'}$ and 
           $\synjudg{\Gamma}{e}{A'}{\Delta}$ and 
           $A = [\Omega']A'$. 
      
      \item
           If $\declappjudg{[\Omega]\Gamma}{e}{[\Omega]A}{C}$
           \\
           then there exist $\Delta$, $\Omega'$, and $C'$
           \\
           such that
           $\substextend{\Delta}{\Omega'}$ and 
           $\substextend{\Omega}{\Omega'}$ and 
           $\appjudg{\Gamma}{e}{[\Gamma]A}{C'}{\Delta}$ and 
           $C = [\Omega']C'$. 
    \end{enumerate}
\end{restatable}



\clearpage

\section*{Proofs}
\addcontentsline{toc}{section}{\fontsz{14pt}{Proofs}}


\newcounter{savedsection}
\setcounter{savedsection}{\value{section}}


\renewcommand{\thesection}{\XAlph{12}$'$}

In the rest of this document, we prove the results stated above, with the same sectioning.



\section{Declarative Subtyping}

\propweakening*

\propsubst*

The proofs of these two propositions are routine inductions. 


\subsection{Properties of Well-Formedness}

\subsection{Reflexivity}

\declarativereflexivity*
\begin{proof}  By induction on $A$.

  \begin{itemize}
  \ProofCaseThing{A = \unitty}  Apply rule \DsubUnit.
   
  \ProofCaseThing{A = \alpha}  Apply rule \DsubVar.

  \ProofCaseThing{A = A_1 \arr A_2}

      \begin{llproof}
      \declsubjudgPf{\Psi}{A_1}{A_1}   {By i.h.}
      \declsubjudgPf{\Psi}{A_2}{A_2}   {By i.h.}
      \declsubjudgPf{\Psi}{A_1 \arr A_2}{A_1 \arr A_2}   {By \DsubArr}
      \end{llproof}

  \ProofCaseThing{A = \alltype{\alpha}{A_0}}
  
      \begin{llproof}
      \declsubjudgPf{\Psi, \alpha}{A_0}{A_0}   {By i.h.}
      \judgetpPf{\Psi, \alpha}{\alpha}   {By \DeclUvarWF}
      \declsubjudgPf{\Psi, \alpha}{[\alpha/\alpha]A_0}{A_0}   {By def.\ of substitution}
      \declsubjudgPf{\Psi, \alpha}{\alltype{\alpha}{A_0}}{A_0}   {By \DsubAllL}
      \declsubjudgPf{\Psi}{\alltype{\alpha}{A_0}}{\alltype{\alpha}{A_0}}   {By \DsubAllR}
      \end{llproof}
      \qedhere
  \end{itemize}
\end{proof}


\subsection{Subtyping Implies Well-Formedness}

\declarativewellformed*
\begin{proof}
  By induction on the given derivation.  All 5 cases are straightforward.
\end{proof}


\subsection{Substitution}

\declsubsubstitution*
\begin{proof}
  By induction on the given derivation.

  \begin{itemize}
    \DerivationProofCase{\DsubVar}
              {\beta \in (\Psi, \alpha, \Psi')}
              {\declsubjudg{\Psi, \alpha, \Psi'}{\beta}{\beta}}

              It is given that $\Psi \entails \tau$.

              Either $\beta = \alpha$ or $\beta \neq \alpha$.  In the former case:
              We need to show $\declsubjudg{\Psi, \Psi'}{[\tau/\alpha]\alpha}{[\tau/\alpha]\alpha}$,
              that is, $\declsubjudg{\Psi, \Psi'}{\tau}{\tau}$, which follows by \Lemmaref{lem:declarative-reflexivity}.
              In the latter case: We need to show 
               $\declsubjudg{\Psi, \Psi'}{[\tau/\alpha]\beta}{[\tau/\alpha]\beta}$,
               that is,
               $\declsubjudg{\Psi, \Psi'}{\beta}{\beta}$.  Since $\beta \in (\Psi, \alpha, \Psi')$
               and $\beta \neq \alpha$, we have $\beta \in (\Psi, \Psi')$, so applying \DsubVar
               gives the result.

    \DerivationProofCase{\DsubUnit}
              { }
              {\declsubjudg{\Psi, \alpha, \Psi'}{\unitty}{\unitty}}

              For all $\tau$, substituting $[\tau/\alpha]\unitty = \unitty$, and applying
              \DsubUnit gives the result.

    \DerivationProofCase{\DsubArr}
              {\declsubjudg{\Psi, \alpha, \Psi'}{B_1}{A_1}
                \\
                \declsubjudg{\Psi, \alpha, \Psi'}{A_2}{B_2}
              }
              {\declsubjudg{\Psi, \alpha, \Psi'}{A_1 \arr A_2}{B_1 \arr B_2}}

              \begin{llproof}
                \declsubjudgPf{\Psi, \alpha, \Psi'}{B_1}{A_1}  {Subderivation}
                \declsubjudgPf{\Psi, \Psi'}{[\tau/\alpha]B_1}{[\tau/\alpha]A_1}  {By i.h.}
                \proofsep
                \declsubjudgPf{\Psi, \alpha, \Psi'}{A_2}{B_2}  {Subderivation}
                \declsubjudgPf{\Psi, \Psi'}{[\tau/\alpha]A_2}{[\tau/\alpha]B_2}  {By i.h.}
                \proofsep
                \declsubjudgPf{\Psi, \Psi'}{([\tau/\alpha]A_1) \arr ([\tau/\alpha]A_2)}
                                           {([\tau/\alpha]B_1) \arr ([\tau/\alpha]B_2)}  {By \DsubArr}
\Hand                \declsubjudgPf{\Psi, \Psi'}{[\tau/\alpha](A_1 \arr A_2)}
                                           {[\tau/\alpha](B_1 \arr B_2)}  {By definition of subst.}
              \end{llproof}

    \DerivationProofCase{\DsubAllL}
              {\judgetp{\Psi, \alpha, \Psi'}{\sigma}
                \\
                \declsubjudg{\Psi, \alpha, \Psi'}{[\sigma/\beta]A_0}{B}}
              {\declsubjudg{\Psi, \alpha, \Psi'}{\alltype{\beta}{A_0}}{B}}

              \begin{llproof}
                 \declsubjudgPf{\Psi, \alpha, \Psi'}{[\sigma/\beta]A_0}{B}  {Subderivation}
                 \declsubjudgPf{\Psi, \Psi'}{[\tau/\alpha][\sigma/\beta]A_0}{[\tau/\alpha]B}  {By i.h.}
                 \declsubjudgPf{\Psi, \Psi'}{\big[[\tau/\alpha]\sigma \,/\, \beta\big][\tau/\alpha]A_0}{[\tau/\alpha]B}  {By distributivity of substitution}
                 \proofsep
                 \judgetpPf{\Psi, \alpha, \Psi'}{\sigma}  {Premise}
                 \judgetpPf{\Psi}{\tau}  {Given}
                 \judgetpPf{\Psi, \Psi'}{[\tau/\alpha]\sigma}  {By \Propositionref{prop:subst}}
                 \proofsep
                 \declsubjudgPf{\Psi, \Psi'}{\alltype{\beta}{[\tau/\alpha]A_0}}{[\tau/\alpha]B}  {By \DsubAllL}
\Hand            \declsubjudgPf{\Psi, \Psi'}{[\tau/\alpha]\;\big(\alltype{\beta}{A_0}\big)}{[\tau/\alpha]B}  {By definition of substitution}
              \end{llproof}

    \DerivationProofCase{\DsubAllR}
              {\declsubjudg{\Psi, \alpha, \Psi', \beta}{A}{B_0}}
              {\declsubjudg{\Psi, \alpha, \Psi'}{A}{\alltype{\beta}{B_0}}}

              \begin{llproof}
                \declsubjudgPf{\Psi, \alpha, \Psi', \beta}{A}{B_0} {Subderivation}
                \declsubjudgPf{\Psi, \Psi', \beta}{[\tau/\alpha]A}{[\tau/\alpha]B_0} {By i.h.}
                \declsubjudgPf{\Psi, \Psi'}{[\tau/\alpha]A}{\alltype{\beta} [\tau/\alpha]B_0} {By \DsubAllR}
\Hand           \declsubjudgPf{\Psi, \Psi'}{[\tau/\alpha]A}{[\tau/\alpha](\alltype{\beta} B_0)} {By definition of substitution}
              \end{llproof}   \qedhere
  \end{itemize}
\end{proof}


\subsection{Transitivity}

To prove transitivity, we use a metric that adapts ideas from a
proof of cut elimination by \citet{Pfenning95}.

\declarativetransitivity*
\begin{proof}
  By induction with the following metric:
\[
      \left\langle
        \#\forall(B),~~~
\Dee_1 + \Dee_2
      \right\rangle
    \]
    where $\langle \dots \rangle$ denotes lexicographic order,
    the first part $\#\forall(B)$ is the number of quantifiers in $B$,
and the second part is the (simultaneous) size of the derivations
    $\Dee_1 \derives \declsubjudg{\Psi}{A}{B}$
    and $\Dee_2 \derives \declsubjudg{\Psi}{B}{C}$.
    We need to consider the number of quantifiers first in one case:
    when \DsubAllR concluded $\Dee_1$ and \DsubAllL concluded $\Dee_2$,
    because in that case, the derivations on which the i.h.\ must be applied are
    not necessarily smaller.


  \begin{itemize}
    \DoubleDerivationProofCase
          {\DsubVar}
              {\alpha \in \Psi}
              {\declsubjudg{\Psi}{\alpha}{\alpha}}
          {\DsubVar}
              {\alpha \in \Psi}
              {\declsubjudg{\Psi}{\alpha}{\alpha}}

          Apply rule \DsubVar.

    \ProofCaseRule{{\DsubUnit} / \DsubUnit}  Similar to the {\DsubVar} / \DsubVar case.
    
    \DoubleDerivationProofCase{\DsubArr}
              {\declsubjudg{\Psi}{B_1}{A_1}
                \\
                \declsubjudg{\Psi}{A_2}{B_2}
              }
              {\declsubjudg{\Psi}{A_1 \arr A_2}{B_1 \arr B_2}}
          {\DsubArr}
              {\declsubjudg{\Psi}{C_1}{B_1}
                \\
                \declsubjudg{\Psi}{B_2}{C_2}
              }
              {\declsubjudg{\Psi}{B_1 \arr B_2}{C_1 \arr C_2}}

              By i.h. on the 3rd and 1st subderivations, $\declsubjudg{\Psi}{C_1}{A_1}$. \\
              By i.h. on the 2nd and 4th subderivations, $\declsubjudg{\Psi}{A_2}{C_2}$. \\
              By \DsubArr, $\declsubjudg{\Psi}{A_1 \arr A_2}{C_1 \arr C_2}$.
  \end{itemize}

  If \DsubAllL concluded $\Dee_1$:

  \begin{itemize}
    \DerivationProofCase{\DsubAllL}
              {\judgetp{\Psi}{\tau}
                \\
                \declsubjudg{\Psi}{[\tau/\alpha]A_0}{B}}
              {\declsubjudg{\Psi}{\alltype{\alpha}{A_0}}{B}}

              \begin{llproof}
                \judgetpPf{\Psi}{\tau}  {Premise}
                \declsubjudgPf{\Psi}{[\tau/\alpha]A_0}{B}  {Subderivation}
                \declsubjudgPf{\Psi}{B}{C}  {Given ($\Dee_2$)}
                \declsubjudgPf{\Psi}{[\tau/\alpha]A_0}{C}  {By i.h.}
\Hand            \declsubjudgPf{\Psi}{\alltype{\alpha}{A_0}}{C}  {By \DsubAllL}
              \end{llproof}
  \end{itemize}

  If \DsubAllR concluded $\Dee_2$:

  \begin{itemize}
    \DerivationProofCase{\DsubAllR}
              {\declsubjudg{\Psi, \beta}{B}{C}}
              {\declsubjudg{\Psi}{B}{\alltype{\beta}{C}}}

              \begin{llproof}
                \judgetpPf{\Psi}{\tau}  {Premise}
                \declsubjudgPf{\Psi, \beta}{B}{C}  {Subderivation}
                \declsubjudgPf{\Psi}{A}{B}  {Given ($\Dee_1$)}
                \declsubjudgPf{\Psi, \beta}{A}{B}  {By \Propositionref{prop:weakening}}
                \declsubjudgPf{\Psi, \beta}{A}{C}  {By i.h.}
\Hand            \declsubjudgPf{\Psi}{A}{\alltype{\beta} C}  {By \DsubAllL}
              \end{llproof}
  \end{itemize}

  The only remaining possible case is {\DsubAllR} / \DsubAllL.

  \begin{itemize}
    \DoubleDerivationProofCase
          {\DsubAllR}
              {\declsubjudg{\Psi, \beta}{A}{B_0}}
              {\declsubjudg{\Psi}{A}{\alltype{\beta}{B_0}}}
          {\DsubAllL}
              {\judgetp{\Psi}{\tau}
                \\
                \declsubjudg{\Psi}{[\tau/\beta]B_0}{C}}
              {\declsubjudg{\Psi}{\alltype{\beta}{B_0}}{C}}
              
              \medskip

              \begin{llproof}
                \declsubjudgPf{\Psi, \beta}{A}{B_0}   {Subderivation of $\Dee_1$}
                \judgetpPf{\Psi}{\tau}  {Premise of $\Dee_2$}
                \declsubjudgPf{\Psi}{[\tau/\beta]A}{[\tau/\beta]B_0}  {By \Lemmaref{lem:decl-sub-substitution}}
                \eqPf{[\tau/\beta]A} {A}   {$\beta$ cannot appear in $A$}
                \declsubjudgPf{\Psi}{A}{[\tau/\beta]B_0}  {By above equality}
                \declsubjudgPf{\Psi}{[\tau/\beta]B_0}{C} {Subderivation of $\Dee_2$}
\Hand           \declsubjudgPf{\Psi}{A}{C}  {By i.h. (one less $\forall$ quantifier in $B$)}
              \end{llproof}
  \qedhere
  \end{itemize}
\end{proof}





\subsection{Invertibility of \DsubAllR}

\declinvertibility*
\begin{proof}
  By induction on the given derivation $\Dee$.
  
  \begin{itemize}
      \ProofCasesRules{\DsubVar, \DsubUnit, \DsubArr}
          Impossible: the supertype cannot have the form $\alltype{\beta}{B}$.

      \DerivationProofCase{\DsubAllR}
                {\declsubjudg{\Psi, \beta}{A}{B}}
                {\declsubjudg{\Psi}{A}{\alltype{\beta}{B}}}

          The subderivation is exactly what we need, and is strictly smaller than $\Dee$.

      \DerivationProofCase{\DsubAllL}
                {\judgetp{\Psi}{\tau}
                  \\
                  \arrayenvb{
                    \Dee_0
                    \\ \declsubjudg{\Psi}{[\tau/\alpha]A_0}{\alltype{\beta}{B}}
                  }
                }
                {\declsubjudg{\Psi}{\alltype{\alpha}{A_0}}{\alltype{\beta}{B}}}

          By i.h., $\Dee_0'$ derives $\declsubjudg{\Psi, \beta}{[\tau/\alpha]A_0}{B}$
          where $\Dee_0' < \Dee_0$. \\
          By \DsubAllL, $\Dee'$ derives $\declsubjudg{\Psi, \beta}{\alltype{\alpha}{A_0}}{B}$;
          since $\Dee_0' < \Dee_0$, we have $\Dee'  < \Dee$.
\qedhere
  \end{itemize}
\end{proof}


\subsection{Non-Circularity and Equality}

\occurrencelemma*
\begin{proof}   By induction on the given derivation.



  \begin{itemize}
    \ProofCasesRules{\DsubVar, \DsubUnit}
        (i), (ii): Here $A$ and $B$ have no subterms at all, so the result is immediate.
    
    \DerivationProofCase{\DsubArr}
          {\declsubjudg{\Psi}{B_1}{A_1}
            \\
            \declsubjudg{\Psi}{A_2}{B_2}
          }
          {\declsubjudg{\Psi}{A_1 \arr A_2}{B_1 \arr B_2}}
          
          \begin{enumerate}[(i)]
          \item Here, $A = A_1 \arr A_2$ and $\tau = B_1 \arr B_2$.
                 
                 \begin{llproof}
                     \Pf{B_1}{\notoccursinsidearrow}{A_1}    {By i.h. (ii)}
                     \Pf{B_1 \arr B_2}{\not\subtermof}{A_1}    
                                   {Suppose $B_1 \arr B_2 \subtermof A_1$.  Then $B_1 \occursinsidearrow A_1$: contradiction.}
                     \Pf{B_2}{\notoccursinsidearrow}{A_2}    {By i.h. (i)}
                     \Pf{B_1 \arr B_2}{\not\subtermof}{A_2}  
                                   {Similar}
                 \end{llproof}
                 
                 Suppose (for a contradiction) that $B_1 \arr B_2 \occursinsidearrow A_1 \arr A_2$. \\
                 $~$~~Now $B_1 \arr B_2 \subtermof A_1$ or $B_1 \arr B_2 \subtermof A_2$. \\
                 $~$~~But above, we showed that both were false: contradiction. \\
                 Therefore, $B_1 \arr B_2 \not\propersubtermof A_1 \arr A_2$. \\
                 Therefore, $B_1 \arr B_2 \notoccursinsidearrow A_1 \arr A_2$.
          
          \item Here, $A = \tau$ and $B = B_1 \arr B_2$. \\
                 Symmetric to the previous case.
          \end{enumerate}

      \DerivationProofCase{\DsubAllL}
          {\judgetp{\Psi}{\tau'}
            \\
            \declsubjudg{\Psi}{[\tau'/\alpha]A_0}{\tau}}
          {\declsubjudg{\Psi}{\alltype{\alpha}{A_0}}{\tau}}

          In part (ii), this case cannot arise, so we prove part (i).

          By i.h. (i), $\tau \notoccursinsidearrow [\tau'/\alpha]A_0$. \\
          It follows from the definition of $\occursinsidearrow$ that
          $\tau \notoccursinsidearrow \alltype{\alpha}{A_0}$.

      \DerivationProofCase{\DsubAllR}
          {\declsubjudg{\Psi, \beta}{\tau}{B_0}}
          {\declsubjudg{\Psi}{\tau}{\alltype{\beta}{B_0}}}
          
          In part (i), this case cannot arise, so we prove part (ii).
          
          Similar to the \DsubAllL case.
          \qedhere
  \end{itemize}
\end{proof}


\declmonotypeequality*
\begin{proof}
  By induction on the given derivation.

  \begin{itemize}
      \ProofCaseRule{\DsubVar}   Immediate.

      \ProofCaseRule{\DsubUnit}   Immediate.

      \DerivationProofCase{\DsubArr}
            {\declsubjudg{\Psi}{B_1}{A_1}
              \\
              \declsubjudg{\Psi}{A_2}{B_2}
            }
            {\declsubjudg{\Psi}{A_1 \arr A_2}{B_1 \arr B_2}}

            By i.h. on each subderivation, $B_1 = A_1$ and $A_2 = B_2$.
            Therefore $A_1 \arr A_2 = B_1 \arr B_2$.

       \ProofCaseRule{\DsubAllL}
            Here $\sigma = \alltype{\alpha} A_0$, which is not a monotype, so this case is impossible.

       \ProofCaseRule{\DsubAllR}
            Here $\tau = \alltype{\beta} B_0$, which is not a monotype, so this case is impossible.
\qedhere
  \end{itemize}
\end{proof}


\section{Type Assignment}





\declarativetypingwellformed*
\begin{proof}
  By induction on the given derivation.
  
  In all cases, we apply the induction hypothesis to all subderivations.

  \begin{itemize}
  \item   In the \DeclVar and \DeclArrIntro cases, we use our standard assumption
    that every context appearing in a derivation is well-formed.
  \item  In the \DeclArrIntroSyn case, we use inversion on the $\judgetp{\Psi}{\sigma \arr \tau}$ premise.
  \item  In the \DeclAllApp case, we use the property that if $\judgetp{\Psi}{[\tau/\alpha]A_0}$ then
    $\judgetp{\Psi}{\alltype{\alpha} A_0}$.
  \item  In the \DeclAnno case, we use its premise.
  \qedhere
  \end{itemize}
\end{proof}


\completenessbidirectional*
\begin{proof}
  By induction on the derivation of $\judge{\Psi}{e}{A}$. 

    \begin{itemize}
      \DerivationProofCase{\AssignVar}
            {x:A \in \Psi}
            {\judge{\Psi}{x}{A}}

            Immediate, by rule \DeclVar. 

      \DerivationProofCase{\AssignArrIntro}
            {\judge{\Psi, x:A}{e}{B}}
            {\judge{\Psi}{\lam{x} e}{A \arr B}}

            By inversion, we have $\judge{\Psi, x:A}{e}{B}$. \\
            By induction, we have $\declsynjudg{\Psi, x:A}{e'}{B}$, where $|e'| = e$. \\ 
            By \Lemmaref{lem:declarative-reflexivity}, $\declsubjudg{\Psi}{B}{B}$. \\
            By rule \DeclSub, $\declchkjudg{\Psi, x:A}{e'}{B}$. \\ 
            By rule \DeclArrIntro, $\declchkjudg{\Psi}{\fun{x}{e'}}{A \arr B}$. \\ 
            By \Lemmaref{lem:declarative-typing-well-formed}, $\judgetp{\Psi}{A \arr B}$. \\
            By rule \DeclAnno, $\declsynjudg{\Psi}{((\fun{x}{e'}) : A \arr B)}{A \arr B}$. \\ 
            By definition, $|((\fun{x}{e'}) : A \arr B)| = |\fun{x}{e'}| = \fun{x}{|e'|} = \fun{x}{e}$. 
      
      \DerivationProofCase{\AssignArrElim}
            {\judge{\Psi}{e_1}{A \arr B} \\
             \judge{\Psi}{e_2}{A}}
            {\judge{\Psi}{e_1\;e_2}{B}}

            By induction, $\declsynjudg{\Psi}{e'_1}{A \arr B}$ and $|e'_1| = e_1$. \\
            By induction, $\declsynjudg{\Psi}{e'_2}{A}$ and $|e'_2| = e_2$. \\
            By \Lemmaref{lem:declarative-reflexivity}, $\declsubjudg{\Psi}{A}{A}$. \\
            By rule \DeclSub, $\declchkjudg{\Psi}{e'_2}{A}$. \\
            By rule \DeclArrApp, $\declappjudg{\Psi}{e'_2}{A \arr B}{B}$. \\
            By rule \DeclArrElim, $\declsynjudg{\Psi}{e'_1\;e'_2}{B}$. \\ 
            By definition, $|e'_1\;e'_2| = |e'_1|\;|e'_2| = e_1\;e_2$. 

      \DerivationProofCase{\AssignAllIntro}
            {\judge{\Psi, \alpha}{e}{A}}
            {\judge{\Psi}{e}{\alltype{\alpha}{A}}}

            By induction, $\declsynjudg{\Psi, \alpha}{e'}{A}$ where $|e'| = e$. \\ 
            By \Lemmaref{lem:declarative-reflexivity}, $\declsubjudg{\Psi, \alpha}{A}{A}$. \\
            By rule \DeclSub, $\declchkjudg{\Psi, \alpha}{e'}{A}$. \\ 
            By rule \DeclAllIntro, $\declchkjudg{\Psi}{e'}{\alltype{\alpha}{A}}$. \\ 
            By \Lemmaref{lem:declarative-typing-well-formed}, $\judgetp{\Psi}{\alltype{\alpha}{A}}$. \\
            By rule \DeclAnno, $\declsynjudg{\Psi}{(e':\alltype{\alpha}{A})}{\alltype{\alpha}{A}}$. \\ 
            By definition, $|e':\alltype{\alpha}{A}| = |e'| = e$. 

      \DerivationProofCase{\AssignAllElim}
            {\judge{\Psi}{e}{\alltype{\alpha}{A}} \\
             \judgetp{\Psi}{\tau}}
            {\judge{\Psi}{e}{[\tau/\alpha]A}}    

            By induction, $\declsynjudg{\Psi}{e'}{\alltype{\alpha}{A}}$ where $|e'| = e$. \\
            By \Lemmaref{lem:declarative-reflexivity}, $\declsubjudg{\Psi}{[\tau/\alpha]A}{[\tau/\alpha]A}$. \\
            By \DsubAllL, $\declsubjudg{\Psi}{\alltype{\alpha}{A}}{[\tau/\alpha]A}$. \\
            By rule \DeclSub, $\declchkjudg{\Psi}{e'}{[\tau/\alpha]A}$. \\
            By \Lemmaref{lem:declarative-typing-well-formed}, $\judgetp{\Psi}{[\tau/\alpha]A}$. \\
            By rule \DeclAnno, $\declchkjudg{\Psi}{(e' : [\tau/\alpha]A)}{[\tau/\alpha]A}$. \\
            By definition, $|e' : [\tau/\alpha]A| = |e'| = e$. 
    \qedhere
    \end{itemize}
\end{proof}


\subtypingcoercion*
\begin{proof}
  By induction on the derivation of $\declsubjudg{\Psi}{A}{B}$. 

\begin{itemize}
    \DerivationProofCase{\DsubVar}
              {\alpha \in \Psi}
              {\declsubjudg{\Psi}{\alpha}{\alpha}}

   Choose $f = \fun{x}{x}$. \\
   Clearly $\judge{\Psi}{\fun{x}{x}}{\alpha \arr \alpha}$. 

    \DerivationProofCase{\DsubUnit}
              { }
              {\declsubjudg{\Psi}{\unitty}{\unitty}}

   Choose $f = \fun{x}{x}$. \\
   Clearly $\judge{\Psi}{\fun{x}{x}}{\unitty \arr \unitty}$. 

    \DerivationProofCase{\DsubArr}
              {\declsubjudg{\Psi}{B_1}{A_1}
                \\
                \declsubjudg{\Psi}{A_2}{B_2}
              }
              {\declsubjudg{\Psi}{A_1 \arr A_2}{B_1 \arr B_2}}

   By induction, we have $g : B_1 \arr A_1$, which is $\beta\eta$-equal to the identity. \\
   By induction, we have $k : A_2 \arr B_2$, which is $\beta\eta$-equal to the identity. \\ 
   Let $f$ be $\lam{h} k \circ h \circ g$. \\
   It is easy to verify that $\judge{\Psi}{f}{(A_1 \arr A_2) \arr (B_1 \arr B_2)}$. \\
   Since $k$ and $g$ are identities, $f =_{\beta\eta} \fun{h}{h}$. 

    \DerivationProofCase{\DsubAllL}
              {\judgetp{\Psi}{\tau}
                \\
                \declsubjudg{\Psi}{[\tau/\alpha]A}{B}}
              {\declsubjudg{\Psi}{\alltype{\alpha}{A}}{B}}

    By induction, $g : [\tau/\alpha]A \arr B$. \\ 
    Let $f \triangleq \fun{x}{g\;x}$. \\ 
    $f$ is an eta-expansion of $g$, which is $\beta\eta$-equal to the identity. Hence $f$ is too. \\
    Also, $\fun{x}{g\;x} : (\alltype{\alpha}{A}) \arr B$, using the \DeclAllElim rule on $x$. 

    \DerivationProofCase{\DsubAllR}
              {\declsubjudg{\Psi, \beta}{A}{B}}
              {\declsubjudg{\Psi}{A}{\alltype{\beta}{B}}}
  
   By induction, we have $g$ such that $\judge{\Psi, \beta}{g}{A \arr B}$. \\ 
   Let $f \triangleq \fun{x}{g\;x}$. \\
   Use the following derivation:
\begin{mathpar}
                               \inferrule*
                                         { 
                                           \inferrule*
                                                     { \inferrule*[Left=Weaken] 
                                                                 { \vdots }
                                                                 {\judge{\Psi, \beta}{g}{A \arr B}}
                                                     }
                                                     {\judge{\Psi, x:A, \beta}{g}{A \arr B}}
                                           \\
                                           \judge{\Psi, x:A, \beta}{x}{A} 
                                         }
                             {
                               \inferrule*
                                         {\judge{\Psi, x:A, \beta}{g\;x}{B}}
                              {\inferrule*
                                       {
                                         \judge{\Psi, x:A}{g\;x}{\alltype{\beta}{B}}
                                       }
                                       {\judge{\Psi}{\fun{x}{g\;x}}{A \arr \alltype{\beta}{B}}}
                            }}
   \qedhere
   \end{mathpar}
\end{itemize}
\end{proof}



\applicationsubtyping*
\begin{proof}
  By induction on the given derivation $\Dee$.

  \begin{itemize}
     \DerivationProofCase{\DeclArrApp}
            {\declchkjudg{\Psi}{e}{B}}
            {\declappjudg{\Psi}{e}{B \arr C}{C}}

            \begin{llproof}
\Hand              \declchkjudgPf{\Dee' \derives \Psi}{e}{B}  {Subderivation}
\Hand              \ltPf{\Dee'}{\Dee}   {$\Dee'$ is a subderivation of $\Dee$}
\Hand              \declsubjudgPf{\Psi}{\underbrace{B \arr C}_{A}}{B \arr C}  {By \Lemmaref{lem:declarative-reflexivity}}
            \end{llproof}

     \DerivationProofCase{\DeclAllApp}
            {\judgetp{\Psi}{\tau} \\ 
             \declappjudg{\Psi}{e}{[\tau/\alpha]A_0}{C}}
            {\declappjudg{\Psi}{e}{\alltype{\alpha} A_0}{C}}

            \begin{llproof}
              \judgetpPf{\Psi}{\tau}  {Subderivation}
              \declappjudgPf{\Psi}{e}{[\tau/\alpha]A_0}{C}  {Subderivation}
              \declsubjudgPf{\Psi}{[\tau/\alpha]A_0}{B \arr C}  {By i.h.}
\Hand      \declchkjudgPf{\Dee' \derives \Psi}{e}{B}  {\ditto}
\Hand      \ltPf{\Dee'}{\Dee}   {\ditto}
\Hand      \declsubjudgPf{\Psi}{\alltype{\alpha} A_0}{B \arr C}   {By \DsubAllL}
            \end{llproof}
  \qedhere
  \end{itemize}
\end{proof}


\soundnessbidirectional*
\begin{proof}
  \begin{itemize}
     \DerivationProofCase{\DeclVar}
          {(x : A) \in \Psi}
          {\declsynjudg{\Psi}{x}{A}}

      By rule \AssignVar, $\judge{\Psi}{x}{A}$. \\ 
      Note $x \beeq x$.

     \DerivationProofCase{\DeclSub}
          {\declsynjudg{\Psi}{e}{A}
            \\
            \declsubjudg{\Psi}{A}{B}
          }
          {\declchkjudg{\Psi}{e}{B}}

     By induction, $\judge{\Psi}{e'}{A}$ and $e' \beeq |e|$. \\
     By \Lemmaref{lem:subtyping-coercion}, $f : A \arr B$ such that $f \beeq id$. \\
     By \AssignArrElim, $\judge{\Psi}{f\;e'}{B}$. \\ 
     Note $f\;e' \beeq id\; e' \beeq e' \beeq |e|$. 

     \DerivationProofCase{\DeclAnno}
          {\judgetp{\Psi}{A}
            \\
            \declchkjudg{\Psi}{e}{A}
          }
          {\declsynjudg{\Psi}{(e : A)}{A}}

     By induction, $\judge{\Psi}{e'}{A}$ such that $e' \beeq |e|$. \\
     Note $e' \beeq |e| = |e:A|$. 

     \DerivationProofCase{\DeclUnitIntro}
          {}
          {\declchkjudg{\Psi}{\unitexp}{\unitty}}

      By \AssignUnit, $\judge{\Psi}{\unitexp}{\unitty}$. \\
      Note $\unitexp \beeq \unitexp$. 

     \DerivationProofCase{\DeclUnitIntroSyn}
                { }
                {\declsynjudg{\Psi}{\unitexp}{\unitty}}

      By \AssignUnit, $\judge{\Psi}{\unitexp}{\unitty}$. \\
      Note $\unitexp \beeq \unitexp$. 

     \DerivationProofCase{\DeclAllIntro}
           {\declchkjudg{\Psi, \alpha}{e}{A}
           }
           {\declchkjudg{\Psi}{e}{\alltype{\alpha}{A}}}

     By induction, $\judge{\Psi, \alpha}{e'}{A}$ such that $e' \beeq |e|$. \\ 
     By rule \AssignAllIntro, $\judge{\Psi}{e'}{\alltype{\alpha}{A}}$. 

     \DerivationProofCase{\DeclArrIntro}
          {\declchkjudg{\Psi, x : A}{e}{B}
          }
          {\declchkjudg{\Psi}{\lam{x} e}{A \arr B}}

     By induction, $\judge{\Psi, x:A}{e'}{B}$ such that $e' \beeq |e|$. \\
     By \AssignArrIntro, $\judge{\Psi}{\fun{x}{e'}}{A \arr B}$. \\ 
     Note $\fun{x}{e'} \beeq \fun{x}{|e|} = |\fun{x}{e}|$. 

     \DerivationProofCase{\DeclArrIntroSyn}
                {\judgetp{\Psi}{\sigma \arr \tau}
                  \\
                  \declchkjudg{\Psi, x : \sigma}{e}{\tau}
                }
                {\declsynjudg{\Psi}{\lam{x} e}{\sigma \arr \tau}}

     By induction, $\judge{\Psi, x:\sigma}{e'}{\tau}$ such that $e' \beeq |e|$. \\
     By \AssignArrIntro, $\judge{\Psi}{\fun{x}{e'}}{\sigma \arr \tau}$. \\ 
     Note $\fun{x}{e'} \beeq \fun{x}{|e|} = |\fun{x}{e}|$. 

     \DerivationProofCase{\DeclArrElim}
          {\declsynjudg{\Psi}{e_1}{A}
            \\
            \declappjudg{\Psi}{e_2}{A}{C}
          }
          {\declsynjudg{\Psi}{e_1\,e_2}{C}}

     By induction, $\judge{\Psi}{e'_1}{A}$ such that $e'_1 \beeq |e_1|$. \\ 
     By \Lemmaref{lem:application-subtyping}, there is a $B$ such that \\
     1. $\declsubjudg{\Psi}{A}{B \arr C}$, and \\
     2. $\declchkjudg{\Psi}{e_2}{B}$,  which is no bigger than $\declappjudg{\Psi}{e_2}{A}{C}$. \\ 
     By \Lemmaref{lem:subtyping-coercion}, we have $f$ such that $\judge{\Psi}{f}{A \arr B \arr C}$ and $f \beeq id$. \\ 
     By induction, we get $\judge{\Psi}{e'_2}{B}$ and $e'_2 \beeq |e_2|$. \\
     By \AssignArrElim twice, $\judge{\Psi}{f\;e'_1\;e'_2}{C}$. \\
     Note $f\;e'_1\;e'_2 \beeq id\;e'_1\;e'_2 \beeq e'_1\;e'_2 \beeq |e_1|\;e'_2 \beeq |e_1|\;|e_2| = |e_1\;e_2|$. 
  \qedhere
  \end{itemize}
\end{proof}



\section{Robustness of Typing}

\typesubstitution*
\begin{proof}
  By induction on the given derivation.

  In the \DeclVar case, split on whether the variable being
  typed is in $\Psi$ or $\Psi'$; the former is immediate, and in the latter, use the fact that
  $(x : C) \in \Psi'$ implies $(x : [\tau/\alpha]C) \in [\tau/\alpha]\Psi'$.

  In the \DeclSub case, use the i.h. and \Lemmaref{lem:decl-sub-substitution}.

  In the \DeclAnno case, we are substituting in the annotation in the term, as well as in the type;
  we also need \Propositionref{prop:subst}.

  In \DeclArrIntro, \DeclArrIntroSyn and \DeclAllIntro, we add to the context in the premise,
  which is why the statement is generalized for nonempty $\Psi'$.
\end{proof}



\termsubsumption*
\begin{proof}
  By mutual induction:
  in (i), by lexicographic induction on the derivation of the checking judgment, then of the subtyping judgment;
  in (ii), by induction on the derivation of the synthesis judgment;
  in (iii), by lexicographic induction on the derivation of the application judgment, then of the subtyping judgment.

  For part (i), checking:

  \begin{itemize}
     \DerivationProofCase{\DeclSub}
          {\declsynjudg{\Psi}{e}{B}
            \\
            \declsubjudg{\Psi}{B}{A}
          }
          {\declchkjudg{\Psi}{e}{A}}

          \begin{llproof}
            \declsynjudgPf{\Psi}{e}{B}   {Subderivation}
            \declsynjudgPf{\Psi'}{e}{B'}   {By i.h.}
            \declsubjudgPf{\Psi}{B'}{B}   {\ditto}
            \proofsep
            \declsubjudgPf{\Psi}{B}{A}   {Subderivation}
            \declsubjudgPf{\Psi}{A}{A'}   {Given}
            \declsubjudgPf{\Psi}{B'}{A'}  {By \Lemmaref{lem:declarative-transitivity} (twice)}
            \declsubjudgPf{\Psi'}{B'}{A'}  {By weakening}
            \proofsep
\Hand       \declchkjudgPf{\Psi'}{e}{A'}  {By \DeclSub}
          \end{llproof}

     \DerivationProofCase{\DeclUnitIntro}
          {}
          {\declchkjudg{\Psi}{\unitexp}{\unitty}}

          \begin{llproof}
              \declsynjudgPf{\Psi'}{\unitexp}{\unitty}  {By \DeclUnitIntroSyn}
              \declsubjudgPf{\Psi}{\unitty}{A'}   {Given}
              \declsubjudgPf{\Psi'}{\unitty}{A'}   {By weakening}
\Hand            \declchkjudgPf{\Psi'}{\unitexp}{A'}   {By \DeclSub}
          \end{llproof}

     \DerivationProofCase{\DeclAllIntro}
           {\declchkjudg{\Psi, \alpha}{e}{A_0}
           }
           {\declchkjudg{\Psi}{e}{\alltype{\alpha}A_0}}

           We consider cases of $\declsubjudg{\Psi}{\alltype{\alpha} A_0}{A'}$:

           \begin{itemize}
                  \DerivationProofCase{\DsubAllR}
                        {\declsubjudg{\Psi, \beta}{\alltype{\alpha} A_0}{B}}
                        {\declsubjudg{\Psi}{\alltype{\alpha} A_0}{\alltype{\beta} B}}

                        \smallskip
                        
                        \begin{llproof}
                          \declsubjudgPf{\Psi, \beta}{\alltype{\alpha} A_0}{B}  {Subderivation}
                          \declchkjudgPf{\Psi}{e}{\alltype{\alpha} A_0}   {Given}
                          \declchkjudgPf{\Psi'}{e}{B}   {By i.h. (i)}
                        \Hand  \declchkjudgPf{\Psi'}{e}{\underbrace{\alltype{\beta} B}_{A'}}   {By \DeclAllIntro}
                        \end{llproof}

                  \DerivationProofCase{\DsubAllL}
                        {\judgetp{\Psi}{\tau}
                          \\
                          \declsubjudg{\Psi}{[\tau/\alpha]A_0}{A'}}
                        {\declsubjudg{\Psi}{\alltype{\alpha} A_0}{A'}}

                        \smallskip

                        \begin{llproof}
\declchkjudgPf{\Psi, \alpha}{e}{A_0}   {Subderivation}
                          \declchkjudgPf{\Psi}{e}{[\tau/\alpha]A_0}   {By \Lemmaref{lem:type-substitution}}
                          \declsubjudgPf{\Psi}{[\tau/\alpha]A_0}{A'}   {Subderivation}
                        \Hand  \declchkjudgPf{\Psi'}{e}{A'}   {By i.h. (i)}
                        \end{llproof}

           \end{itemize}


     \DerivationProofCase{\DeclArrIntro}
          {\declchkjudg{\Psi, x : A_1}{e_0}{A_2}
          }
          {\declchkjudg{\Psi}{\lam{x} e_0}{A_1 \arr A_2}}

          We consider cases of $\declsubjudg{\Psi}{A_1 \arr A_2}{A'}$:

          \begin{itemize}
              \DerivationProofCase{\DsubArr}
                   {\declsubjudg{\Psi}{B_1}{A_1}
                    \\
                    \declsubjudg{\Psi}{A_2}{B_2}}
                   {\declsubjudg{\Psi}{A_1 \arr A_2}{B_1 \arr B_2}}

                   \begin{llproof}
                     \leqPf{\Psi}{\Psi'}  {Given}
                     \declsubjudgPf{\Psi}{B_1}{A_1}   {Subderivation}
                     \leqPf{\Psi', x : B_1}{\Psi, x : A_1}  {By \CtxsubVar}
                     \declchkjudgPf{\Psi', x : B_1} {e_0} {B_2}   {By i.h. (i)}
                   \Hand  \declchkjudgPf{\Psi'} {\lam{x} e_0} {B_1 \arr B_2}   {By \DeclArrIntro}
                   \end{llproof}

              \DerivationProofCase{\DsubAllR}
                    {\declsubjudg{\Psi, \beta}{A_1 \arr A_2}{B'}}
                    {\declsubjudg{\Psi}{A_1 \arr A_2}{\alltype{\beta} B'}}
                    
                    \begin{llproof}
                      \declsubjudgPf{\Psi, \beta}{A_1 \arr A_2}{B'}  {Subderivation}
                      \declchkjudgPf{\Psi, \beta}{\lam{x} e_0}{A_1 \arr A_2}  {By weakening}
                      \declchkjudgPf{\Psi', \beta}{\lam{x} e_0}{B'}  {By i.h. (i)}
                    \Hand  \declchkjudgPf{\Psi'}{\lam{x} e_0}{\alltype{\beta} B'}  {By \DeclAllIntro}
                    \end{llproof}
          \end{itemize}
  \end{itemize}


  For part (ii), synthesis:

  \begin{itemize}
     \DerivationProofCase{\DeclVar}
          {(x : A) \in \Psi}
          {\declsynjudg{\Psi}{x}{A}}

          By inversion on $\Psi' \leq \Psi$, we have $(x : A') \in \Psi'$ where $\declsubjudg{\Psi}{A'}{A}$. \\
          By \DeclVar, $\declsynjudg{\Psi'}{x}{A'}$.

     \DerivationProofCase{\DeclAnno}
          {\judgetp{\Psi}{A}
            \\
            \declchkjudg{\Psi}{e_0}{A}
          }
          {\declsynjudg{\Psi}{(e_0 : A)}{A}}

          \begin{llproof}
            \LetPf{A'}{A}  {}
            \judgetpPf{\Psi}{A}  {Subderivation}
            \judgetpPf{\Psi'}{A}  {By weakening}
            \proofsep
            \declchkjudgPf{\Psi}{e_0}{A}  {Subderivation}
            \declchkjudgPf{\Psi'}{e_0}{A}  {By i.h.}
            \proofsep
\Hand            \declsynjudgPf{\Psi'}{(e_0 : A)}{A'} {By \DeclAnno and $A' = A$}
\Hand            \declsubjudgPf{\Psi}{A'}{A}  {By \Lemmaref{lem:declarative-reflexivity}}
          \end{llproof}


     \DerivationProofCase{\DeclUnitIntroSyn}
          { }
          {\declsynjudg{\Psi}{\unitexp}{\unitty}}
          
          \begin{llproof}
            \LetPf{A'}{\unitty}   {}
\Hand        \declsynjudgPf{\Psi'}{\unitexp}{\unitty}  {By \DeclUnitIntroSyn}
\Hand        \declsubjudgPf{\Psi}{\unitty}{\unitty}  {By \DsubUnit}
          \end{llproof}

     \DerivationProofCase{\DeclArrIntroSyn}
           {\judgetp{\Psi}{\sigma \arr \tau}
             \\
             \declchkjudg{\Psi, x : \sigma}{e_0}{\tau}
           }
           {\declsynjudg{\Psi}{\lam{x} e_0}{\sigma \arr \tau}}

           \begin{llproof}
             \LetPf{A'}{\sigma \arr \tau} {}
             \leqPf{\Psi'}{\Psi}  {Given}
             \declsubjudgPf{\Psi}{\sigma}{\sigma}  {By \Lemmaref{lem:declarative-reflexivity}}
             \leqPf{\Psi', x : \sigma}  {\Psi, x : \sigma}   {By \CtxsubVar}
             \declchkjudgPf{\Psi, x : \sigma}{e_0}{\tau}  {Subderivation}
             \declsubjudgPf{\Psi}{\tau}{\tau}   {By \Lemmaref{lem:declarative-reflexivity}}
             \declchkjudgPf{\Psi', x : \sigma}{e_0}{\tau}  {By i.h. (i) with $\tau$}
\Hand             \declsubjudgPf{\Psi}{A'}{\sigma \arr \tau}    {By \Lemmaref{lem:declarative-reflexivity}}
\Hand             \declsynjudgPf{\Psi'}{\lam{x} e_0}{A'}    {By \DeclArrIntroSyn}
           \end{llproof}

     \DerivationProofCase{\DeclArrElim}
          {\declsynjudg{\Psi}{e_1}{C}
            \\
            \declappjudg{\Psi}{e_2}{C}{A}
          }
          {\declsynjudg{\Psi}{e_1\,e_2}{A}}

          \begin{llproof}
            \declsynjudgPf{\Psi}{e_1}{C}   {Subderivation}
            \declsynjudgPf{\Psi'}{e_1}{C'}   {By i.h. (ii)}
            \declsubjudgPf{\Psi}{C'}{C}   {\ditto}
            \declappjudgPf{\Psi}{e_2}{C}{A}   {Subderivation}
\Hand    \declsubjudgPf{\Psi}{A'}{A}  {By i.h. (iii)}
           \declappjudgPf{\Psi'}{e_2}{C'}{A'}  {\ditto}
\Hand    \declsynjudgPf{\Psi'}{e_1\,e_2}{A'}  {By \DeclArrElim}
          \end{llproof}
  \end{itemize}

  For part (iii), application:

  \begin{itemize}
     \DerivationProofCase{\DeclAllApp}
            {\judgetp{\Psi}{\tau} \\ 
             \declappjudg{\Psi}{e}{[\tau/\alpha]C_0}{A}}
            {\declappjudg{\Psi}{e}{\alltype{\alpha} C_0}{A}}

            \begin{llproof}
              \declsubjudgPf{\Psi}{C'}{\alltype{\alpha} C_0}   {Given}
              \declsubjudgPf{\Psi, \alpha}{C'}{C_0}  {By \Lemmaref{lem:decl-invertibility}}
              \declsubjudgPf{\Psi}{[\tau/\alpha]C'}{[\tau/\alpha]C_0}  {By \Lemmaref{lem:decl-sub-substitution}}
              \declsubjudgPf{\Psi}{C'}{[\tau/\alpha]C_0}  {$\alpha$ cannot appear in $C'$}
              \declappjudgPf{\Psi}{e}{[\tau/\alpha]C_0}{A} {Subderivation}
            \Hand  \declappjudgPf{\Psi'}{e}{C'}{A'}  {By i.h. (iii)}
            \Hand  \declsubjudgPf{\Psi'}{A'}{A}  {\ditto}
            \end{llproof}

     \DerivationProofCase{\DeclArrApp}
            {\declchkjudg{\Psi}{e}{C_0}}
            {\declappjudg{\Psi}{e}{C_0 \arr A}{A}}

            \begin{llproof}
              \declsubjudgPf{\Psi}{C'}{C_0 \arr A}    {Given}
            \end{llproof}

            \begin{itemize}
               \DerivationProofCase{\DsubArr}
                     {\declsubjudg{\Psi}{C_0}{C_1'}
                       \\
                       \declsubjudg{\Psi}{C_2'}{A}
                     }
                     {\declsubjudg{\Psi}{C_1' \arr C_2'}{C_0 \arr A}}

                     \begin{llproof}
                       \LetPf{A'}{C_2'} {}
                       \declchkjudgPf{\Psi}{e}{C_0}    {Subderivation}
                       \declsubjudgPf{\Psi}{C_0}{C_1'}   {Subderivation}
                       \declchkjudgPf{\Psi'}{e}{C_1'}   {By i.h.}
                       \declappjudgPf{\Psi'}{e}{C_1' \arr C_2'}{C_2'}  {By \DeclArrApp}
                     \Hand  \declappjudgPf{\Psi'}{e}{C_1' \arr A'}{A'}  {$A' = C_2'$}
                        \declsubjudgPf{\Psi}{C_2'}{A} {Subderivation}
                     \Hand  \declsubjudgPf{\Psi}{A'}{A} {$A' = C_2'$}
                     \end{llproof}
            
               \DerivationProofCase{\DsubAllL}
                    {\judgetp{\Psi}{\tau}
                      \\
                      \declsubjudg{\Psi}{[\tau/\beta]B}{C_0 \arr A}
                    }
                    {\declsubjudg{\Psi}{\alltype{\beta} B}{C_0 \arr A}}

                    \begin{llproof}
                      \declsubjudgPf{\Psi}{[\tau/\beta]B}{C_0 \arr A}  {Subderivation}
                      \declappjudgPf{\Psi'}{e}{[\tau/\beta]B}{A'}   {By i.h. (iii)}
                      \Hand \declsubjudgPf{\Psi}{A'}{A}   {\ditto}
                      \judgetpPf{\Psi}{\tau}   {Subderivation}
                      \judgetpPf{\Psi'}{\tau}   {By weakening}
                     \Hand  \declappjudgPf{\Psi'}{e}{\alltype{\beta} B}{A'}  {By \DeclAllApp}
                    \end{llproof}
           \qedhere
           \end{itemize}
   \end{itemize}
\end{proof}


\termsubstitution*
\begin{proof}
  By a straightforward mutual induction on the given derivation.
\end{proof}


\termUnsubstitution*
\begin{proof}
  By mutual induction on the given derivation.
  
  \begin{enumerate}[(i)]
      \item We have $\declchkjudg{\Psi}{[(e : A)/x]e'}{C}$.

          \begin{itemize}
              \DerivationProofCase{\DeclSub}
                       {\declsynjudg{\Psi}{[(e : A)/x]e'}{B}
                         \\
                         \declsubjudg{\Psi}{B}{C}
                       }
                       {\declchkjudg{\Psi}{[(e : A)/x]e'}{C}}

                  By i.h.\ (ii), $\declsynjudg{\Psi, x:A}{e'}{B}$. \\
                  By \DeclSub, $\declchkjudg{\Psi, x:A}{e'}{C}$.
                  
              \DerivationProofCase{\DeclUnitIntro}
                   {}
                   {\declchkjudg{\Psi}{\unitexp}{\underbrace{\unitty}_{C}}}
                  
                   We have $[(e : A)/x]e' = \unitexp$.  Therefore $e' = \unitexp$,
                   and the result follows by \DeclUnitIntro.

              \DerivationProofCase{\DeclAllIntro}
                    {\declchkjudg{\Psi, \alpha}{[(e : A)/x]e'}{C'}
                    }
                    {\declchkjudg{\Psi}{[(e : A)/x]e'}{\alltype{\alpha}{C'}}}
                  
                  By i.h.\ (i), $\declchkjudg{\Psi, \alpha, x:A}{e'}{C'}$. \\ 
                  By exchange, $\declchkjudg{\Psi, x:A, \alpha}{e'}{C'}$. \\ 
                  By \DeclAllIntro, $\declchkjudg{\Psi, x:A}{e'}{\alltype{\alpha}{C'}}$. 
                  
              \DerivationProofCase{\DeclArrIntro}
                   {\declchkjudg{\Psi, y : C_1}{e''}{C_2}
                   }
                   {\declchkjudg{\Psi}{\lam{y} e''}{C_1 \arr C_2}}

                  We have $[(e : A)/x]e' = \lam{y}{e''}$. \\
                  By the definition of substitution, $e' = \lam{y} e_0$ and $e'' = [(e : A)/x]e_0$.

                  \smallskip

                  \begin{llproof}
                    \declchkjudgPf{\Psi, y : C_1}{e''}{C_2}   {Subderivation}
                    \declchkjudgPf{\Psi, y : C_1}{[(e : A)/x]e_0}{C_2}  {By above equality}
                    \declchkjudgPf{\Psi, y : C_1, x:A}{e_0}{C_2} {By i.h. (i)}
                    \declchkjudgPf{\Psi, x : A, y:C_1}{e_0}{C_2} {By exchange}
                  \Hand  \declchkjudgPf{\Psi, x : A}{\underbrace{\lam{y} e_0}_{e'}}{\underbrace{C_1 \arr C_2}_{C}}  {By \DeclArrIntro}
                  \end{llproof}
          \end{itemize}



      \item We have $\declsynjudg{\Psi}{[(e:A)/x]e'}{C}$.

        \begin{itemize}
          \ProofCaseRule{$e' = x$}

              Note $[(e:A)/x]x = (e:A)$. \\
              Hence $\declsynjudg{\Psi}{(e:A)}{C}$; by inversion, $C = A$. \\
              By \Lemmaref{lem:declarative-typing-well-formed}, $\judgetp{\Psi}{C}$, which is $\judgetp{\Psi}{A}$. \\
              By \DeclAnno, $\declsynjudg{\Psi}{(e:A)}{A}$. \\
              By \DeclVar, $\declsynjudg{\Psi, x:A}{\underbrace{x}_{e'}}{A}$.

          \ProofCaseRule{$e' \neq x$}

              We now proceed by cases on the derivation of $\declsynjudg{\Psi}{[(e:A)/x]e'}{C}$.

              \begin{itemize}
                   \DerivationProofCase{\DeclVar}
                        {(y : C) \in \Psi}
                        {\declsynjudg{\Psi}{y}{C}}

                       Since $[(e:A)/x]e' = y$, we know that $e' = y$. \\ 
                       By \DeclVar, $\declsynjudg{\Psi, x:A}{y}{C}$. 

                   \DerivationProofCase{\DeclAnno}
                        {\declchkjudg{\Psi}{e''}{C}
                        }
                        {\declsynjudg{\Psi}{\underbrace{(e'' : C)}_{[(e:A)/x]e'}}{C}}

                       We know $[(e:A)/x]e' = (e'' : C)$ and $e' \neq x$. \\
                       Hence there is $e_0$ such that $e' = (e_0 : C)$ and $[(e:A)/x]e_0 = e''$.

                       \smallskip

                       \begin{llproof}
                         \declchkjudgPf{\Psi}{e''}{C}  {Subderivation}
                         \declchkjudgPf{\Psi}{[(e:A)/x]e_0}{C}  {By above equality}
                         \declchkjudgPf{\Psi, x:A}{e_0}{C} {By i.h.\ (i)}
                         \judgetpPf{\Psi, x:A}{C} {By \Lemmaref{lem:declarative-typing-well-formed}}
                         \declsynjudgPf{\Psi, x:A}{(e_0 : C)}{C} {By \DeclAnno}
                       \Hand  \declsynjudgPf{\Psi, x:A}{e'}{C} {By above equality}
                       \end{llproof}

                       \smallskip

                   \DerivationProofCase{\DeclUnitIntroSyn}
                              { }
                              {\declsynjudg{\Psi}{\unitexp}{\unitty}}

                       Since $[(e:A)/x]e' = \unitexp$, it follows that $e' = \unitexp$. \\ 
                       By \DeclUnitIntroSyn, $\declsynjudg{\Psi, x:A}{\unitexp}{\unitty}$. 

                   \DerivationProofCase{\DeclArrIntroSyn}
                              {\judgetp{\Psi}{\sigma \arr \tau}
                                \\
                                \declchkjudg{\Psi, y : \sigma}{e''}{\tau}
                              }
                              {\declsynjudg{\Psi}{\lam{y} e''}{\sigma \arr \tau}}

                       We have $[(e:A)/x]e' = \lam{y}{e''}$. \\
                       By definition of substitution,
                           there exists $e_0$ such that $e' = \lam{y}{e_0}$ and $e'' = [(e:A)/x]e_0$. \\ 
                       So $\declchkjudg{\Psi, y : \sigma}{[(e:A)/x]e_0}{\tau}$. \\ 
                       By i.h.\ (i), $\declchkjudg{\Psi, y : \sigma, x:A}{e_0}{\tau}$. \\
                       By exchange and \DeclArrIntro, $\declchkjudg{\Psi, x:A}{\lam{y}e_0}{\sigma \arr \tau}$. \\
                       Hence \DeclArrIntroSyn, $\declsynjudg{\Psi, x:A}{e'}{\sigma \arr \tau}$. 

                   \DerivationProofCase{\DeclArrElim}
                        {\declsynjudg{\Psi}{e_1}{B}
                          \\
                          \declappjudg{\Psi}{e_2}{B}{C}
                        }
                        {\declsynjudg{\Psi}{\underbrace{e_1\,e_2}_{[(e:A)/x]e'}}{C}}

                      Note that $[(e:A)/x]e' = e_1 \; e_2$. \\
                      So there exist $e'_1$, $e'_2$ such that
                      $e' = e'_1\;e'_2$ and $[(e:A)/x]e'_k = e_k$ for $k \in \{1, 2\}$. \\
                      Applying these equalities to each subderivation gives
                      \[
                         \declsynjudg{\Psi}{[(e:A)/x]e'_1}{B}
                         \AND
                         \declappjudg{\Psi}{[(e:A)/x]e'_2}{B}{C}
                      \]
                      By i.h.\ (ii) and (iii), $\declsynjudg{\Psi, x:A}{e'_1}{B}$
                         and  $\declappjudg{\Psi, x:A}{e'_2}{B}{C}$.\\ 
                      By \DeclArrElim, $\declsynjudg{\Psi, x:A}{e'_1\;e'_2}{C}$, which is
                         $\declsynjudg{\Psi, x:A}{e'}{C}$. 
              \end{itemize}
        \end{itemize}

    \item We have $\declappjudg{\Psi}{A}{[(e:A)/x]e'}{C}$. \\[-1.8em]
        \begin{itemize}
           \DerivationProofCase{\DeclAllApp}
                  {\judgetp{\Psi}{\tau} \\ 
                   \declappjudg{\Psi}{[(e:A)/x]e'}{[\tau/\alpha]B}{C}}
                  {\declappjudg{\Psi}{[(e:A)/x]e'}{\alltype{\alpha}{B}}{C} }

              Follows by i.h.\ (iii) and \DeclAllApp.

           \DerivationProofCase{\DeclArrApp}
                  {\declchkjudg{\Psi}{[(e:A)/x]e'}{B}}
                  {\declappjudg{\Psi}{[(e:A)/x]e'}{B \arr C}{C}}

              Follows by i.h.\ (i) and \DeclArrApp.
        \qedhere
        \end{itemize}
  \end{enumerate}
\end{proof}


\annotationRemoval*
\begin{proof}
  All of these follow directly from inversion and \Lemmaref{lem:term-subsumption}. The one exception is the
  third, which additionally requires a small induction on the application judgment. 
\end{proof}

\soundnessofeta*
\begin{proof}
  By induction on the derivation of $\declchkjudg{\Psi}{\fun{x}{e\;x}}{A}$. There are three
  non-impossible cases: 
  \begin{itemize}
       \DerivationProofCase{\DeclArrIntro}
          {\declchkjudg{\Psi, x : B}{e\;x}{C}
          }
          {\declchkjudg{\Psi}{\lam{x} e\;x}{B \arr C}}

        We have $\declchkjudg{\Psi, x : B}{e\;x}{C}$. \\ 
        By inversion on \DeclSub, we get $\declsynjudg{\Psi, x : B}{e\;x}{C'}$ and $\declsubjudg{\Psi}{C'}{C}$. \\ 
        By inversion on \DeclArrElim, we get $\declsynjudg{\Psi, x:B}{e}{A'}$ and $\declappjudg{\Psi, x:B}{x}{A'}{C'}$. \\ 
        By thinning, we know that $\declsynjudg{\Psi}{e}{A'}$. \\
        By \Lemmaref{lem:application-subtyping}, we get $B'$ so $\declsubjudg{\Psi, x:B}{A'}{B' \arr C'}$
        and $\declchkjudg{\Psi, x:B}{x}{B'}$. \\ 
        By inversion, we know that $\declsynjudg{\Psi, x:B}{x}{B}$ and $\declsubjudg{\Psi}{B}{B'}$. \\
        By \DsubArr, $\declsubjudg{\Psi, x:B}{B' \arr C'}{B \arr C}$. \\ 
        Hence by \Lemmaref{lem:declarative-transitivity}, $\declsubjudg{\Psi, x:B}{A'}{B \arr C}$. \\
        Hence $\declsubjudg{\Psi}{A'}{B \arr C}$. \\
        By \DeclSub, $\declchkjudg{\Psi}{e}{B \arr C}$. 


       \DerivationProofCase{\DeclAllIntro}
           {\declchkjudg{\Psi, \alpha}{\fun{x}{e\;x}}{B}
           }
           {\declchkjudg{\Psi}{\fun{x}{e\;x}}{\alltype{\alpha}{B}}}

         By induction, $\declchkjudg{\Psi, \alpha}{\fun{x}{e\;x}}{B}$. \\ 
         By \DeclAllIntro, $\declchkjudg{\Psi}{\fun{x}{e\;x}}{\alltype{\alpha}{B}}$. 

       
       \DerivationProofCase{\DeclSub}
          {\declsynjudg{\Psi}{\fun{x}{e\;x}}{B}
            \\
            \declsubjudg{\Psi}{B}{A}
          }
          {\declchkjudg{\Psi}{\fun{x}{e\;x}}{A}}

          We have $\declsynjudg{\Psi}{\fun{x} e\;x}{B}$ and $\declsubjudg{\Psi}{B}{A}$. \\ 
          By inversion on \DeclArrIntroSyn, $\declchkjudg{\Psi, x:\sigma}{e\;x}{\tau}$ and $B = \sigma \arr \tau$.\\
          By inversion on \DeclSub, we get $\declsynjudg{\Psi, x : \sigma}{e\;x}{C_2}$ and $\declsubjudg{\Psi}{C_2}{\tau}$. \\ 
          By inversion on \DeclArrElim, we get $\declsynjudg{\Psi, x:\sigma}{e}{C}$ and $\declappjudg{\Psi, x:\sigma}{x}{C}{C_2}$. \\ 
          By thinning, we know that $\declsynjudg{\Psi}{e}{C}$. \\
          By \Lemmaref{lem:application-subtyping},
          we get $C_1$ such that $\declsubjudg{\Psi, x:\sigma}{C}{C_1 \arr C_2}$
          and $\declchkjudg{\Psi, x:\sigma}{x}{C_1}$. \\ 
          By inversion on \DeclSub, $\declsynjudg{\Psi, x:\sigma}{x}{\sigma}$ and $\declsubjudg{\Psi}{\sigma}{C_1}$. \\
          By \DsubArr, $\declsubjudg{\Psi, x:\sigma}{C_1 \arr C_2}{\sigma \arr \tau}$. \\ 
          Hence by \Lemmaref{lem:declarative-transitivity}, $\declsubjudg{\Psi, x:\sigma}{C}{\sigma \arr \tau}$. \\
          Hence $\declsubjudg{\Psi}{C}{\sigma \arr \tau}$. \\
          Hence by \Lemmaref{lem:declarative-transitivity}, $\declsubjudg{\Psi}{C}{A}$. \\
          By \DeclSub, $\declchkjudg{\Psi}{e}{A}$. 
  \qedhere
  \end{itemize}
\end{proof}

\section{Properties of Context Extension}

\subsection{Syntactic Properties}

\declarationpreservation* 
\begin{proof}
  By a routine induction on $\substextend{\Gamma}{\Delta}$. 
\end{proof}


\declarationorderpreservation*
\begin{proof}
    By induction on the derivation of $\substextend{\Gamma}{\Delta}$. 
    
    \begin{itemize}
        \DerivationProofCase{\substextendId}
            { }
            { \substextend{\emptyctx}{\emptyctx} }

           This case is impossible. 

        \DerivationProofCase{\substextendVV}
              {\substextend{\Gamma}{\Delta}}
              {\substextend{\Gamma, x:A}{\Delta, x:A}}

            There are two cases, depending on whether or not $v = x$. 
            \begin{itemize}
            \item Case $v = x$: \\
              Since $u$ is declared to the left of $v$, $u$ is declared in $\Gamma$. \\
              By \Lemmaref{lem:declaration-preservation}, $u$ is declared in $\Delta$. \\
              Hence $u$ is declared to the left of $x$ in $\Delta, x:A$.

            \item Case $v \neq x$: \\
              Then $v$ is declared in $\Gamma$, and $u$ is declared to the left of $v$ in $\Gamma$.  \\
              By induction, $u$ is declared to the left of $v$ in $\Delta$. \\
              Hence $u$ is declared to the left of $v$ in $\Delta, x:A$. 
            \end{itemize}

        \DerivationProofCase{\substextendUU}
            { \substextend{\Gamma}{\Delta} }
            { \substextend{\Gamma, \alpha}{\Delta, \alpha} }

            This case is similar to the \substextendVV case. 

        \DerivationProofCase{\substextendEE}
            { \substextend{\Gamma}{\Delta} }
            { \substextend{\Gamma, \ahat}{\Delta, \ahat} }

            This case is similar to the \substextendVV case. 

        \DerivationProofCase{\substextendSolSol}
            { \substextend{\Gamma}{\Delta} \\ [\Delta]\tau = [\Delta]\tau'}
            { \substextend{\Gamma, \hypeq{\ahat}{\tau}}{\Delta, \hypeq{\ahat}{\tau'}} }

            This case is similar to the \substextendVV case. 

        \DerivationProofCase{\substextendMonMon}
            { \substextend{\Gamma}{\Delta} }
            { \substextend{\Gamma, \MonnierComma{\ahat}}{\Delta, \MonnierComma{\ahat}} }

            This case is similar to the \substextendVV case. 

        \DerivationProofCase{\substextendSolve}
            { \substextend{\Gamma}{\Delta} }
            { \substextend{\Gamma, \ahat}{\Delta, \hypeq{\ahat}{\tau}} }

            This case is similar to the \substextendVV case. 

        \DerivationProofCase{\substextendAdd}
            { \substextend{\Gamma}{\Delta} }
            { \substextend{\Gamma}{\Delta, \ahat} }

            By induction, $u$ is declared to the left of $v$ in $\Delta$. \\
            Therefore $u$ is declared to the left of $v$ in $\Delta, \ahat$.

        \DerivationProofCase{\substextendAddSolved}
            { \substextend{\Gamma}{\Delta} }
            { \substextend{\Gamma}{\Delta, \hypeq{\ahat}{\tau}} }

            By induction, $u$ is declared to the left of $v$ in $\Delta$. \\
            Therefore $u$ is declared to the left of $v$ in $\Delta, \hypeq{\ahat}{\tau}$.
    \qedhere
    \end{itemize}
\end{proof}

\reversedeclarationorderpreservation*
\begin{proof}
  It is given that $u$ and $v$ are declared in $\Gamma$.  Either $u$ is declared to the left of $v$
  in $\Gamma$, or $v$ is declared to the left of $u$.  Suppose the latter (for a contradiction).
  By \Lemmaref{lem:declaration-order-preservation}, $v$ is declared to
  the left of $u$ in $\Delta$.  But we know that $u$ is declared to the left of $v$ in $\Delta$:
  contradiction.   Therefore $u$ is declared to the left of $v$ in $\Gamma$.
\end{proof}

\substitutionextensioninvariance*
\begin{proof}
  To show that $[\Gamma]A = [\Theta][\Gamma]A$, observe that $\judgetp{\Theta}{A}$,
  and that by definition of $\substextend{\Theta}{\Gamma}$,
  every solved variable in $\Theta$ is solved in $\Gamma$.
  Therefore $[\Theta]([\Gamma]A) = [\Gamma]A$,
  since $\unsolved{[\Gamma]A}$ contains no variables that $\Theta$ solves. \\

  To show that $[\Gamma]A = [\Gamma][\Theta]A$, we proceed by induction on $\typesize{\Gamma}{A}$. 
  
  \begin{itemize}
      \DerivationProofCase{}
                          {\alpha \in \Theta}
                          {\judgetp{\Theta}{\alpha}}

          Note that $[\Gamma]\alpha = \alpha = [\Theta]\alpha$, so $[\Gamma]\alpha = [\Gamma][\Theta]\alpha$. 

      \DerivationProofCase{}
                          {\judgetp{\Theta}{A} \\ \judgetp{\Theta}{B}}
                          {\judgetp{\Theta}{A \arr B}}

          By induction, $[\Gamma]A = [\Gamma][\Theta]A$. \\
          By induction, $[\Gamma]B = [\Gamma][\Theta]B$. \\
          Then
          \begin{displaymath}
          \begin{array}{lcll}
            [\Gamma](A \arr B) & = & [\Gamma]A \arr [\Gamma]B & \mbox{By definition of substitution} \\
                              & = & [\Gamma][\Theta]A \arr [\Gamma][\Theta]B & \mbox{By induction hypothesis (twice)} \\
                              & = & [\Gamma]([\Theta]A \arr [\Theta]B) & \mbox{By definition of substitution} \\
                              & = & [\Gamma][\Theta](A \arr B) & \mbox{By definition of substitution}
          \end{array}
          \end{displaymath}                  

      \DerivationProofCase{}
                          {\judgetp{\Theta, \alpha}{A}}
                          {\judgetp{\Theta}{\alltype{\alpha}{A}}}

          By inversion, we have $\judgetp{\Theta, \alpha}{A}$. \\
          By rule \substextendUU, $\substextend{\Theta, \alpha}{\Gamma, \alpha}$. \\
          By induction, $[\Gamma, \alpha]A = [\Gamma, \alpha][\Theta, \alpha]A$. \\
          By definition, $[\Gamma]A = [\Gamma][\Theta]A$.  \\
          Then 
          \begin{displaymath}
          \begin{array}{lcll}
            [\Gamma]\alltype{\alpha}{A} & = & \alltype{\alpha}{[\Gamma]A} & \mbox{By definition} \\
                              & = & \alltype{\alpha}{[\Gamma][\Theta]A} & \mbox{By conclusion above } \\
                              & = & [\Gamma](\alltype{\alpha}{[\Theta]A}) & \mbox{By definition} \\
                              & = & [\Gamma][\Theta](\alltype{\alpha}{A}) & \mbox{By definition} \\
                              & = & [\Gamma, \alpha][\Theta, \alpha](\alltype{\alpha}{A}) & \mbox{By definition} 
          \end{array}
          \end{displaymath}                  

      \DerivationProofCase{}
                          { }
                          {\judgetp{\underbrace{\Theta_0, \ahat, \Theta_1}_{\Theta}}{\ahat}}

           Note that $[\Theta]\ahat = \ahat$. \\
           Hence $[\Gamma][\Theta]\ahat = [\Gamma]\ahat$. 


      \DerivationProofCase{}
                          {}
                          {\judgetp{\Theta_0, \ahat=\tau, \Theta_1}{\ahat}}


          From $\substextend{\Theta}{\Gamma}$, 
          By a nested induction we get $\Gamma = \Gamma_0, \ahat=\tau', \Gamma_1$, 
                                   and $[\Gamma]\tau' = [\Gamma]\tau$. \\
          Note that $\typesize{\Theta}{\tau} < \typesize{\Theta}{\ahat}$. \\
          By induction, $[\Gamma]\tau = [\Gamma][\Theta]\tau$. \\
          Hence 
          \begin{displaymath}
            \begin{array}[b]{lcll}
              [\Gamma]\ahat & = & [\Gamma]\tau'        & \mbox{By definition} \\
                            & = & [\Gamma]\tau         & \mbox{From the extension judgment} \\
                            & = & [\Gamma][\Theta]\tau & \mbox{From the induction hypothesis} \\
                            & = & [\Gamma][\Theta]\ahat & \mbox{By definition}
\end{array}
            \qedhere
          \end{displaymath}
\end{itemize}
\end{proof}

\extensionequalitypreservation*
\begin{proof}
  By induction on the derivation of $\substextend{\Gamma}{\Delta}$. 

    \begin{itemize}
      \DerivationProofCase{\substextendId}
            { }
            { \substextend{\underbrace{\emptyctx}_{\Gamma}}{\underbrace{\emptyctx}_{\Delta}} }

            We have $[\Gamma]A = [\Gamma]B$, but $\Gamma = \Delta$, so  $[\Delta]A = [\Delta]B$.

      \DerivationProofCase{\substextendVV}
              {\substextend{\Gamma'}{\Delta'}}
              {\substextend{\Gamma', x:C}{\Delta', x:C}}

          We have $[\Gamma', x:C]A = [\Gamma', x:C]B$. \\
          By definition of substitution, $[\Gamma']A = [\Gamma']B$. \\
          By i.h., $[\Delta']A = [\Delta']B$. \\
          By definition of substitution, $[\Delta', x:C]A = [\Delta', x:C]B$. 

      \DerivationProofCase{\substextendUU}
            { \substextend{\Gamma'}{\Delta'} }
            { \substextend{\Gamma', \alpha}{\Delta', \alpha} }

          We have $[\Gamma', \alpha]A = [\Gamma', \alpha]B$. \\
          By definition of substitution, $[\Gamma']A = [\Gamma']B$. \\
          By i.h., $[\Delta']A = [\Delta']B$. \\
          By definition of substitution, $[\Delta', \alpha]A = [\Delta', \alpha]B$. 

      \DerivationProofCase{\substextendEE}
            { \substextend{\Gamma'}{\Delta'} }
            { \substextend{\Gamma', \ahat}{\Delta', \ahat} }
            
            Similar to the \substextendUU case.


      \DerivationProofCase{\substextendMonMon}
            { \substextend{\Gamma'}{\Delta'} }
            { \substextend{\Gamma', \MonnierComma{\ahat}}{\Delta', \MonnierComma{\ahat}} }

            Similar to the \substextendUU case.


      \DerivationProofCase{\substextendAdd}
            { \substextend{\Gamma}{\Delta'} }
            { \substextend{\Gamma}{\Delta', \ahat} }

          We have $[\Gamma]A = [\Gamma]B$. \\
          By i.h., $[\Delta']A = [\Delta']B$. \\
          By definition of substitution, $[\Delta', \ahat]A = [\Delta', \ahat]B$.

      \DerivationProofCase{\substextendAddSolved}
            { \substextend{\Gamma}{\Delta'} }
            { \substextend{\Gamma}{\Delta', \hypeq{\ahat}{\tau}} }

          We have $[\Gamma]A = [\Gamma]B$. \\
          By i.h., $[\Delta']A = [\Delta']B$. \\
          We implicitly assume that $\Delta$ is well-formed,
          so $\ahat \notin \dom{\Delta'}$. \\
          Since $\substextend{\Gamma}{\Delta'}$ and $\ahat \notin \dom{\Delta'}$,
          it follows that $\ahat \notin \dom{\Gamma}$. \\
          We have $\judgetp{\Gamma}{A}$ and $\judgetp{\Gamma}{B}$, so
          $\ahat \notin (\FV{A} \union \FV{B})$. \\
          Therefore, by definition of substitution, $[\Delta', \ahat=\tau]A = [\Delta', \ahat=\tau]B$.

      \DerivationProofCase{\substextendSolSol}
            { \substextend{\Gamma'}{\Delta'} \\ [\Delta']\tau = [\Delta']\tau'}
            { \substextend{\Gamma', \hypeq{\ahat}{\tau}}{\Delta', \hypeq{\ahat}{\tau'}} }

          We have $[\Gamma', \hypeq{\ahat}{\tau}]A = [\Gamma', \hypeq{\ahat}{\tau}]B$. \\
          
          By definition, $[\Gamma', \hypeq{\ahat}{\tau}]A = [\Gamma', \hypeq{\ahat}{\tau}]\tau$,
          but we implicitly assume that $\Gamma$ is well-formed, so $\ahat \notin \FV{\tau}$,
          so actually $[\Gamma', \hypeq{\ahat}{\tau}]A = [\Gamma']\tau$. \\
          Combined with similar reasoning for $B$, we get
          \[
               [\Gamma'][\tau/\ahat]A
               ~=~
               [\Gamma'][\tau/\ahat]B
          \]
          By i.h., $[\Delta'][\tau/\ahat]A = [\Delta'][\tau/\ahat]B$. \\
          By distributivity of substitution,
          $\big[[\Delta']\tau/\ahat\big][\Delta']A = \big[[\Delta']\tau/\ahat\big][\Delta']B$. \\
          Using the premise $[\Delta']\tau = [\Delta']\tau'$,
          we get $[[\Delta']\tau'/\ahat][\Delta']A = [[\Delta']\tau'/\ahat][\Delta']B$. \\
          By distributivity of substitution (in the other direction), $[\Delta'][\tau'/\ahat]A = [\Delta'][\tau'/\ahat]B$. \\
          It follows from the definition of substitution that
          $[\Delta', \hypeq{\ahat}{\tau'}]A = [\Delta', \hypeq{\ahat}{\tau'}]B$. 

      \DerivationProofCase{\substextendSolve}
            { \substextend{\Gamma'}{\Delta'} }
            { \substextend{\Gamma', \ahat}{\Delta', \hypeq{\ahat}{\tau}} }

          We have $[\Gamma', \ahat]A = [\Gamma', \ahat]B$. \\
          By definition of substitution, $[\Gamma']A = [\Gamma']B$. \\
          By i.h., $[\Delta'][\tau/\ahat]A = [\Delta'][\tau/\ahat]B$. \\
          It follows from the definition of substitution that
          $[\Delta', \hypeq{\ahat}{\tau}]A = [\Delta', \hypeq{\ahat}{\tau}]B$. 
    \qedhere
    \end{itemize}
\end{proof}


\substextendreflexivity*
\begin{proof}
  By induction on the structure of $\Gamma$. 

    \begin{itemize}
        \ProofCaseRule{$\Gamma = \cdot$} 
            Apply rule \substextendId. 

        \ProofCaseRule{$\Gamma = (\Gamma', \alpha)$}
            By i.h., $\substextend{\Gamma'}{\Gamma'}$.
            By rule \substextendUU, we get $\substextend{\Gamma', \alpha}{\Gamma', \alpha}$. 

        \ProofCaseRule{$\Gamma = (\Gamma', \ahat)$}
            By i.h., $\substextend{\Gamma'}{\Gamma'}$.
            By rule \substextendEE, we get $\substextend{\Gamma', \ahat}{\Gamma', \ahat}$. 

        \ProofCaseRule{$\Gamma = (\Gamma', \ahat=\tau)$}

            By i.h., $\substextend{\Gamma'}{\Gamma'}$. \\
            Clearly, $[\Gamma']\tau = [\Gamma']\tau$, so
            we can apply \substextendSolSol to get $\substextend{\Gamma', \ahat=\tau}{\Gamma', \ahat=\tau}$. 

        \ProofCaseRule{$\Gamma = (\Gamma', \MonnierComma{\ahat})$}
            By i.h., $\substextend{\Gamma'}{\Gamma'}$.
            By rule \substextendMonMon, we get $\substextend{\Gamma', \MonnierComma{\ahat}}{\Gamma', \MonnierComma{\ahat}}$.
\qedhere
    \end{itemize}
\end{proof}



\substextendtransitivity*
\begin{proof}
  By induction on the derivation of $\substextend{\Delta}{\Theta}$.

  \begin{itemize}
  \ProofCaseRule{\substextendId}
  
      In this case $\Theta = \Delta$. \\
      Hence $\substextend{\Gamma}{\Delta}$ suffices. 

  \DerivationProofCase{\substextendUU}
                      {\substextend{\Delta'}{\Theta'}}
                      {\substextend{\Delta', \alpha}{\Theta', \alpha}}

      We have $\Delta = (\Delta', \alpha)$ and $\Theta = (\Theta', \alpha)$. \\
      By inversion on $\substextend{\Gamma}{\Delta}$,
      we have $\Gamma = (\Gamma', \alpha)$ and $\substextend{\Gamma'}{\Delta'}$. \\
      By i.h., $\substextend{\Gamma'}{\Theta'}$. \\
      Applying rule \substextendUU gives $\substextend{\Gamma', \alpha}{\Theta', \alpha}$. 

  \DerivationProofCase{\substextendUU}
                      {\substextend{\Delta'}{\Theta'}}
                      {\substextend{\Delta', \ahat}{\Theta', \ahat}}

      We have $\Delta = (\Delta', \ahat)$ and $\Theta = (\Theta', \ahat)$. \\
      Either of two rules could have derived $\substextend{\Gamma}{\Delta}$:

      \begin{itemize}
          \DerivationProofCase{\substextendEE}
                              {\substextend{\Gamma'}{\Delta'}}
                              {\substextend{\Gamma', \ahat}{\Delta', \ahat}}

            Here we have $\Gamma = (\Gamma', \ahat)$ and $\substextend{\Gamma'}{\Delta'}$. \\
            By i.h., $\substextend{\Gamma'}{\Theta'}$. \\
            Applying rule \substextendEE gives $\substextend{\Gamma', \ahat}{\Theta', \ahat}$. 

          \DerivationProofCase{\substextendAdd}
                              {\substextend{\Gamma}{\Delta'}}
                              {\substextend{\Gamma}{\Delta', \ahat}}

            By i.h., $\substextend{\Gamma}{\Theta'}$. \\
            By rule \substextendAdd, we get $\substextend{\Gamma}{\Theta', \ahat}$. 
      \end{itemize}

  \DerivationProofCase{\substextendSolSol}
                      {\substextend{\Delta'}{\Theta'} \\ [\Theta']\tau_1 = [\Theta']\tau_2}
                      {\substextend{\Delta', \ahat=\tau_1}{\Theta', \ahat=\tau_2}}

      In this case $\Delta = (\Delta', \hypeq{\ahat}{\tau_1})$ and $\Theta = (\Theta', \hypeq{\ahat}{\tau_2})$. \\
      One of three rules must have derived $\substextend{\Gamma}{\Delta', \ahat=\tau}$:

          \begin{itemize}
              \DerivationProofCase{\substextendSolSol}
                                  {\substextend{\Gamma'}{\Delta'} \\ [\Delta']\tau_0 = [\Delta']\tau_1}
                                  {\substextend{\Gamma', \ahat=\tau_0}{\Delta', \ahat=\tau_1}}

                    Here, $\Gamma = (\Gamma', \ahat=\tau_0)$ and $\Delta = (\Delta', \ahat=\tau_1)$. \\
                    By i.h., we have $\substextend{\Gamma'}{\Theta'}$. \\
                    The premises of the respective $\extendssym$ derivations give us
                       $[\Delta']\tau_0 = [\Delta']\tau_1$ and $[\Theta']\tau_1 = [\Theta']\tau_2$. \\
                    We know that $\judgetp{\Gamma'}{\tau_0}$ and $\judgetp{\Delta'}{\tau_1}$ and $\judgetp{\Theta'}{\tau_2}$. \\ 
                    By extension weakening (\Lemmaref{lem:extension-weakening}), $\judgetp{\Theta'}{\tau_0}$. \\ 
                    By extension weakening (\Lemmaref{lem:extension-weakening}), $\judgetp{\Theta'}{\tau_1}$. \\
                    Since $[\Delta']\tau_0 = [\Delta']\tau_1$, we know that $[\Theta'][\Delta']\tau_0 = [\Theta'][\Delta']\tau_1$. \\ 
                    By \Lemmaref{lem:subst-extension-invariance}, $[\Theta'][\Delta']\tau_0 = [\Theta']\tau_0$. \\
                    By \Lemmaref{lem:subst-extension-invariance}, $[\Theta'][\Delta']\tau_1 = [\Theta']\tau_1$. \\
                    So $[\Theta']\tau_0 = [\Theta']\tau_1$. \\ 

                    Hence by transitivity of equality, $[\Theta']\tau_0 = [\Theta']\tau_1 = [\Theta']\tau_2$. \\
                    By rule \substextendSolSol,  $\substextend{\Gamma', \ahat=\tau}{\Theta', \ahat=\tau_2}$. 

              \DerivationProofCase{\substextendAddSolved}
                                  {\substextend{\Gamma}{\Delta'}}
                                  {\substextend{\Gamma}{\Delta', \ahat=\tau_1}}

                    By induction, we have $\substextend{\Gamma}{\Theta'}$. \\
                    By rule \substextendAddSolved, we get $\substextend{\Gamma}{\Theta', \ahat=\tau_2}$. 

              \DerivationProofCase{\substextendSolve}
                                  {\substextend{\Gamma'}{\Delta'} }
                                  {\substextend{\Gamma', \ahat}{\Delta', \ahat=\tau_1}}

                    We have $\Gamma = (\Gamma', \ahat)$. \\
                    By induction, $\substextend{\Gamma'}{\Theta'}$. \\
                    By rule \substextendSolve, we get $\substextend{\Gamma', \ahat}{\Theta', \ahat=\tau_2}$.
          \end{itemize}
          
  \DerivationProofCase{\substextendMonMon}
                      {\substextend{\Delta'}{\Theta'}}
                      {\substextend{\Delta', \MonnierComma{\ahat}}{\Theta', \MonnierComma{\ahat}}}

      In this case we know $\Delta = (\Delta', \MonnierComma{\ahat})$ and $\Theta = (\Theta', \MonnierComma{\ahat})$. \\
      Since $\Delta = (\Delta', \MonnierComma{\ahat})$, only \substextendMonMon could derive
      $\substextend{\Gamma}{\Delta}$,
      so by inversion,
        $\Gamma = (\Gamma', \MonnierComma{\ahat})$
        and $\substextend{\Gamma'}{\Delta'}$. \\
      By induction, we have $\substextend{\Gamma'}{\Theta'}$. \\
      Applying rule \substextendMonMon gives 
      $\substextend{\Gamma', \MonnierComma{\ahat}}{\Theta', \MonnierComma{\ahat}}$. 

  \DerivationProofCase{\substextendAdd}
                      {\substextend{\Delta}{\Theta'}}
                      {\substextend{\Delta}{\Theta', \ahat}}

      In this case, we have $\Theta = (\Theta', \ahat)$. \\
      By induction, we get $\substextend{\Gamma}{\Theta'}$. \\
      By rule \substextendAdd, we get $\substextend{\Gamma}{\Theta', \ahat}$. 

  \DerivationProofCase{\substextendAddSolved}
                      {\substextend{\Delta}{\Theta'}}
                      {\substextend{\Delta}{\Theta', \ahat=\tau}}

      In this case, we have $\Theta = (\Theta', \ahat=\tau)$. \\
      By induction, we get $\substextend{\Gamma}{\Theta'}$. \\
      By rule \substextendAddSolved, we get $\substextend{\Gamma}{\Theta', \ahat=\tau}$. 

  \DerivationProofCase{\substextendSolve}
                      {\substextend{\Delta'}{\Theta'}}
                      {\substextend{\Delta', \ahat}{\Theta', \ahat=\tau}}

      In this case, we have $\Delta = (\Delta', \ahat)$ and $\Theta = (\Theta', \ahat=\tau)$. \\
      One of two rules could have derived $\substextend{\Gamma}{\Delta', \ahat}$:

          \begin{itemize}
              \DerivationProofCase{\substextendEE}
                                  {\substextend{\Gamma'}{\Delta'}}
                                  {\substextend{\Gamma',\ahat}{\Delta',\ahat}}

                   In this case, we have $\Gamma = (\Gamma', \ahat)$
                     and $\substextend{\Gamma'}{\Delta'}$
                     and $\substextend{\Delta'}{\Theta'}$. \\
                   By induction, we have $\substextend{\Gamma'}{\Theta'}$. \\
                   By rule \substextendSolve, we get $\substextend{\Gamma', \ahat}{\Theta', \ahat=\tau}$. 

               \DerivationProofCase{\substextendAdd}
                                   {\substextend{\Gamma}{\Delta'}}
                                   {\substextend{\Gamma}{\Delta',\ahat}}

                   In this case, we have $\substextend{\Gamma}{\Delta'}$ and $\substextend{\Delta'}{\Theta'}$. \\
                   By induction, we have $\substextend{\Gamma}{\Theta'}$. \\
                   By rule \substextendSolve, we get $\substextend{\Gamma}{\Theta', \ahat=\tau}$. 
           \qedhere
\end{itemize}
\end{itemize}
\end{proof}


\rightsoftness*
\begin{proof}
  By induction on $\Theta$, applying rules \substextendAdd and \substextendAddSolved as needed.
\end{proof}






\evarinput*
\begin{proof}
  By induction on the given derivation.

  \begin{itemize}
      \ProofCasesRules{\substextendId, \substextendVV, \substextendUU, \substextendSolSol, \substextendMonMon}
          \\
          Impossible: the left-hand context cannot have the form $\Gamma, \ahat$.

      \DerivationProofCase{\substextendEE}
           {\substextend{\Gamma}{\Delta_0}}
           {\substextend{\Gamma, \ahat}{\underbrace{\Delta_0, \ahat}_{\Delta}}}

           Let $\Theta = \emptyctx$, which is vacuously soft.
           Therefore $\Delta = (\Delta_0, \ahat) = (\Delta_0, \ahat, \Theta)$;
           the subderivation is the rest of the result.

      \DerivationProofCase{\substextendSolve}
          { \substextend{\Gamma}{\Delta_0} }
          { \substextend{\Gamma, \ahat}{\underbrace{\Delta_0, \hypeq{\ahat}{\tau}}_{\Delta}} }

          Let $\Theta = \emptyctx$, which is vacuously soft.
          Therefore $\Delta = (\Delta_0, \ahat) = (\Delta_0, \hypeq{\ahat}{\tau}, \Theta)$;
          the subderivation is the rest of the result.

      \DerivationProofCase{\substextendAdd}
           {\substextend{\Gamma, \ahat}{\Delta_0}}
           {\substextend{\Gamma, \ahat}{\underbrace{\Delta_0, \bhat}_{\Delta}}}

           Suppose $\bhat = \ahat$. \\
           $~$~~We have $\substextend{\Gamma, \ahat}{\Delta_0}$.  By \Lemmaref{lem:declaration-preservation},
           $\ahat$ is declared in $\Delta_0$. \\
           $~$~~But then $(\Delta_0, \bhat) = (\Delta_0, \ahat)$ with multiple $\ahat$ declarations, \\
           $~$~~which violates the implicit assumption that $\Delta$ is well-formed.  Contradiction. \\
           Therefore $\bhat \neq \ahat$.

           By i.h., $\Delta' = (\Delta_0, \Delta_{\ahat}, \Theta')$ where $\substextend{\Gamma}{\Delta_0}$
           and $\Theta'$ is soft.
           
           Let $\Theta = (\Theta', \bhat)$.
           Therefore $(\Delta', \bhat) = (\Delta_0, \Delta_{\ahat}, \Theta', \bhat)$.
           As $\Theta'$ is soft, $(\Theta', \bhat)$ is soft.
           Since $\Delta = (\Delta', \bhat)$, this gives $\Delta = (\Delta_0, \Delta_{\ahat}, \Theta)$.

      \ProofCaseRule{\substextendAddSolved}
          Similar to the case for \substextendAdd.
\qedhere
  \end{itemize}
\end{proof}



\extensionorder*
\begin{proof}
  \begin{enumerate}[(i)]
  \item 
    By induction on the derivation of $\substextend{\Gamma_L, \alpha, \Gamma_R}{\Delta}$.

      \begin{itemize}
        \DerivationProofCase{\substextendId}
            { }
            { \substextend{\emptyctx}{\emptyctx} }

            This case is impossible since $(\Gamma_L, \alpha, \Gamma_R)$ cannot have the form $\cdot$.

       \ProofCasesRules{\substextendUU}

           We have two cases, depending on whether or not the rightmost variable is $\alpha$.

           \begin{itemize}
            \DerivationProofCase{\substextendUU}
                { \substextend{\Gamma}{\Delta'} }
                { \substextend{\Gamma, \alpha}{\Delta', \alpha} }

                Let $\Delta_L = \Delta'$, and let $\Delta_R = \cdot$ (which is soft). \\
                We have $\substextend{\Gamma}{\Delta'}$, which is
                $\substextend{\Gamma_L}{\Delta_L}$.
             
            \DerivationProofCase{\substextendUU}
                { \substextend{\Gamma_L, \alpha, \Gamma'_R}{\Delta'} }
                { \substextend{\Gamma_L, \alpha, \underbrace{\Gamma'_R, \beta}_{\Gamma_R}}
                  {\underbrace{\Delta', \beta}_{\Delta}} }

            By i.h., $\Delta' = (\Delta_L, \alpha, \Delta_R')$
              where $\substextend{\Gamma_L}{\Delta_L}$. \\
            Hence $\Delta = (\Delta_L, \alpha, \Delta_R', \beta)$. \\
            (Since $\beta \in \Gamma_R$, it cannot be the case that $\Gamma_R$ is soft.)
          \end{itemize}

    \DerivationProofCase{\substextendVV}
              {\substextend{\Gamma_L, \alpha, \Gamma'_R}{\Delta'}}
              {\substextend{\Gamma_L, \alpha, \underbrace{\Gamma'_R, x:A}_{\Gamma_R}}
                 {\underbrace{\Delta', x:A}_{\Delta}}}

            By i.h., $\Delta' = (\Delta_L, \alpha, \Delta_R')$
              where $\substextend{\Gamma_L}{\Delta_L}$. \\
            Hence $\Delta = (\Delta_L, \alpha, \Delta_R', x:A)$. \\
            (Since $x:A \in \Gamma_R$, it cannot be the case that $\Gamma_R$ is soft.)

        \DerivationProofCase{\substextendEE}
            { \substextend{\Gamma_L, \alpha, \Gamma'_R}{\Delta'} }
            { \substextend{\Gamma_L, \alpha, \underbrace{\Gamma'_R, \ahat}_{\Gamma_R}}
                {\underbrace{\Delta', \ahat}_{\Delta}} }

            By i.h., $\Delta' = (\Delta_L, \alpha, \Delta_R')$
              where $\substextend{\Gamma_L}{\Delta_L}$. \\
            Hence $\Delta = (\Delta_L, \alpha, \Delta_R', \ahat)$.  \\
            (If $\Gamma_R$ is soft, by i.h. $\Delta_R'$ is soft, so $\Delta_R = (\Delta_R', \ahat)$ is soft.)

        \DerivationProofCase{\substextendMonMon}
            { \substextend{\Gamma_L, \alpha, \Gamma'_R}{\Delta'} }
            { \substextend{\Gamma_L, \alpha,
                 \underbrace{\Gamma'_R, \MonnierComma{\bhat}}_{\Gamma'_R}}
                 {\underbrace{\Delta', \MonnierComma{\bhat}}_{\Delta}} }

            By i.h., $\Delta' = (\Delta_L, \alpha, \Delta_R')$
              where $\substextend{\Gamma_L}{\Delta_L}$. \\
            Hence $\Delta = (\Delta_L, \alpha, \Delta_R', \MonnierComma{\bhat})$.  \\
            (Since $\MonnierComma{\bhat} \in \Gamma_R$, it cannot be the case that $\Gamma_R$ is soft.)

        \DerivationProofCase{\substextendSolSol}
            { \substextend{\Gamma_L, \alpha, \Gamma'_R}{\Delta'} \\ [\Delta']\tau = [\Delta']\tau' }
            { \substextend{\Gamma_L, \alpha, \underbrace{\Gamma'_R, \hypeq{\ahat}{\tau}}_{\Gamma_R}}
                          {\underbrace{\Delta', \hypeq{\ahat}{\tau'}}_{\Delta'}} }

            By i.h., $\Delta' = (\Delta_L, \alpha, \Delta_R')$
              where $\substextend{\Gamma_L}{\Delta_L}$. \\
            Hence $\Delta = (\Delta_L, \alpha, \Delta_R', \hypeq{\ahat}{\tau'})$. \\
            (If $\Gamma_R$ is soft, by i.h. $\Delta_R'$ is soft, so $\Delta_R = (\Delta_R', \hypeq{\ahat}{\tau})$ is soft.)

        \DerivationProofCase{\substextendSolve}
            { \substextend{\Gamma_L, \alpha, \Gamma'_R}{\Delta'} }
            { \substextend{\Gamma_L, \alpha, \underbrace{\Gamma'_R, \ahat}_{\Gamma_R}}
                          {\underbrace{\Delta', \hypeq{\ahat}{\tau'}}_{\Delta}} }

            By i.h., $\Delta' = (\Delta_L, \alpha, \Delta_R')$
              where $\substextend{\Gamma_L}{\Delta_L}$. \\
            Therefore $\Delta = (\Delta_L, \alpha, \Delta_R, \hypeq{\ahat}{\tau})$. \\
            (If $\Gamma_R$ is soft, by i.h. $\Delta_R'$ is soft, so $\Delta_R = (\Delta_R', \hypeq{\ahat}{\tau})$ is soft.)

        \DerivationProofCase{\substextendAdd}
            { \substextend{\Gamma_L, \alpha, \Gamma_R}{\Delta'} }
            { \substextend{\Gamma_L, \alpha, \Gamma_R}{\underbrace{\Delta', \ahat}_{\Delta}} }

            By i.h., $\Delta' = (\Delta_L, \alpha, \Delta_R')$
              where $\substextend{\Gamma_L}{\Delta_L}$. \\
            Therefore $\Delta = (\Delta_L, \alpha, \Delta_R', \ahat)$. \\
            (If $\Gamma_R$ is soft, by i.h. $\Delta_R'$ is soft, so $\Delta_R = (\Delta_R', \ahat)$ is soft.)

        \DerivationProofCase{\substextendAddSolved}
            { \substextend{\Gamma_L, \alpha, \Gamma_R}{\Delta'} }
            { \substextend{\Gamma_L, \alpha, \Gamma_R}{\Delta', \hypeq{\ahat}{\tau}} }

            In this case, we know that  $\Delta = (\Delta', \hypeq{\ahat}{\tau})$. \\
            By i.h., $\Delta' = (\Delta_L, \alpha, \Delta_R')$
              where $\substextend{\Gamma_L}{\Delta_L}$. \\
            Hence $\Delta = (\Delta_L, \alpha, \Delta_R', \hypeq{\ahat}{\tau})$. \\
            (If $\Gamma_R$ is soft, by i.h. $\Delta_R'$ is soft, so $\Delta_R = (\Delta_R', \hypeq{\ahat}{\tau})$ is soft.)
      \end{itemize}

    \item Similar to the proof of (i), except that the \substextendMonMon and \substextendUU
      cases are swapped.

    \item Similar to (i), with $\Theta = \ahat$ in the \substextendEE case and
      $\Theta = (\hypeq{\ahat}{\tau})$ in the \substextendSolve case.

    \item Similar to (iii).

    \item Similar to (i), but using the equality premise of \substextendVV.
  \qedhere
  \end{enumerate}
\end{proof}


\extensionweakening*
\begin{proof}
  By a straightforward induction on $\judgetp{\Gamma}{A}$.

  In the \UvarWF case,
  we use \Lemmaref{lem:extension-order} (i).  In the \EvarWF case,
  use \Lemmaref{lem:extension-order} (iii).  In the \SolvedEvarWF case,
  use \Lemmaref{lem:extension-order} (iv).

  In the other cases, apply the i.h. to all subderivations, then apply the rule.
\end{proof}


\extensionsolve*
\begin{proof}
  By induction on $\Gamma_R$.

  \begin{itemize}
    \item Case $\Gamma_R = \cdot$: 
      \\
      By \Lemmaref{lem:substextend-reflexivity} (reflexivity), $\substextend{\Gamma_L}{\Gamma_L}$. \\
      Applying rule \substextendSolve gives $\substextend{\Gamma_L, \ahat}{\Gamma_L, \hypeq{\ahat}{\tau}}$. 

    \item Case $\Gamma_R = (\Gamma_R', x:A)$: 
      
      By i.h., $\substextend{\Gamma_L, \ahat, \Gamma_R'}{\Gamma_L, \hypeq{\ahat}{\tau}, \Gamma_R'}$. \\
      Applying rule \substextendVV gives
      $\substextend{\Gamma_L, \ahat, \Gamma_R', x:A}{\Gamma_L, \hypeq{\ahat}{\tau}, \Gamma_R', x:A}$. 

    \item Case $\Gamma_R = (\Gamma_R', \alpha)$:  By i.h.\ and rule \substextendUU.
    \item Case $\Gamma_R = (\Gamma_R', \bhat)$: By i.h.\ and rule \substextendAdd.
    \item Case $\Gamma_R = (\Gamma_R', \hypeq{\bhat}{\tau'})$: By i.h.\ and rule \substextendAddSolved.
    \item Case $\Gamma_R = (\Gamma_R', \MonnierComma{\bhat})$:  By i.h.\ and rule \substextendMonMon.
  \qedhere
  \end{itemize}
\end{proof}

\extensionaddsolve*
\begin{proof}
  By induction on $\Gamma_R$.
  The proof is exactly the same as the proof of \Lemmaref{lem:extension-solve},
  except that in the $\Gamma_R = \cdot$, we apply rule \substextendAddSolved instead of \substextendSolve.
\end{proof}


\extensionadd*
\begin{proof}  By induction on $\Gamma_R$.  The proof is exactly the same as the proof of
  \Lemmaref{lem:extension-solve}, except that in the $\Gamma_R = \cdot$ case, we apply
    rule \substextendAdd instead of \substextendSolve.
\end{proof}



\parallelevaradmissibility*
\begin{proof}
  By induction on $\Delta_R$.  As always, we assume that all contexts mentioned in the
  statement of the lemma are well-formed.  Hence, $\ahat \notin \dom{\Gamma_L} \union \dom{\Gamma_R}
  \union \dom{\Delta_L} \union \dom{\Delta_R}$.
  
  \begin{enumerate}[(i)]
  \item We proceed by cases of $\Delta_R$.  Observe that in all the extension rules, the right-hand context
    gets smaller, so as we enter subderivations of
    $\substextend{\Gamma_L, \Gamma_R}{\Delta_L, \Delta_R}$, the context $\Delta_R$ becomes smaller.
    
    The only tricky part of the proof is that to apply the i.h., we need $\substextend{\Gamma_L}{\Delta_L}$.
    So we need to make sure that as we drop items from the right of $\Gamma_R$ and $\Delta_R$, we don't
    go too far and start decomposing $\Gamma_L$ or $\Delta_L$!  It's easy to avoid decomposing $\Delta_L$:
    when $\Delta_R = \cdot$, we don't need to apply the i.h.\ anyway.  To avoid decomposing $\Gamma_L$,
    we need to reason by contradiction, using \Lemmaref{lem:declaration-preservation}.
    
        \begin{itemize}
        \ProofCaseRule{$\Delta_R = \cdot$} \\
            We have $\substextend{\Gamma_L}{\Delta_L}$.
            Applying \substextendEE to that derivation gives the result.
    
        \ProofCaseRule{$\Delta_R = (\Delta_R', \bhat)$}
            We have $\bhat \neq \ahat$ by the well-formedness assumption.

            The concluding rule of $\substextend{\Gamma_L, \Gamma_R}{\Delta_L, \Delta_R', \bhat}$
            must have been \substextendEE or \substextendAdd.  In both cases, the result follows by i.h.
            and applying \substextendEE or \substextendAdd.

            Note:  In \substextendAdd, the left-hand context doesn't change, so we clearly maintain
            $\substextend{\Gamma_L}{\Delta_L}$.  In \substextendEE, we can correctly apply the
            i.h.\ because $\Gamma_R \neq \cdot$.  Suppose, for
            a contradiction, that $\Gamma_R = \cdot$.  Then $\Gamma_L = (\Gamma_L', \bhat)$.
            It was given that $\substextend{\Gamma_L}{\Delta_L}$, that is, $\substextend{\Gamma_L', \bhat}{\Delta_L}$.
            By \Lemmaref{lem:declaration-preservation}, $\Delta_L$ has a declaration of $\bhat$.
            But then $\Delta = (\Delta_L, \Delta_R', \bhat)$ is not well-formed: contradiction.  Therefore
            $\Gamma_R \neq \cdot$.

        \ProofCaseRule{$\Delta_R = (\Delta_R', \hypeq{\bhat}{\tau})$}
            We have $\bhat \neq \ahat$ by the well-formedness assumption.

            The concluding rule must have been \substextendSolSol, \substextendSolve or \substextendAddSolved.
            In each case, apply the i.h.\ and then the corresponding rule.  (In \substextendSolSol and \substextendSolve,
            use \Lemmaref{lem:declaration-preservation} to show $\Gamma_R \neq \cdot$.)

        \ProofCaseRule{$\Delta_R = (\Delta_R', \alpha)$}
           The concluding rule must have been \substextendUU.  The result follows by i.h. and
           applying \substextendUU.

        \ProofCaseRule{$\Delta_R = (\Delta_R', \MonnierComma{\bhat})$}  Similar to the previous
           case, with rule \substextendMonMon.
        
        \ProofCaseRule{$\Delta_R = (\Delta_R', x : A)$}  Similar to the previous case,
           with rule \substextendVV.
        \end{itemize}

  \item Similar to part (i), except that when $\Delta_R = \cdot$, apply rule \substextendSolve.

  \item Similar to part (i), except that when $\Delta_R = \cdot$, apply rule \substextendSolSol,
     using the given equality to satisfy the second premise.
  \qedhere
  \end{enumerate}
\end{proof}


\parallelextensionsolve*
\begin{proof}
  By induction on $\Delta_R$.

  In the case where $\Delta_R = (\Delta_R', \hypeq{\ahat}{\tau'})$,
  we know that rule \substextendSolve must have concluded the derivation (we can use
  \Lemmaref{lem:declaration-preservation} to get a contradiction that rules out \substextendAddSolved);
  then we have a subderivation $\substextend{\Gamma_L}{\Delta_L}$, to which we can apply
  \substextendSolSol.
\end{proof}


\parallelextensionupdate*
\begin{proof}
  By induction on $\Delta_R$.  Similar to the proof of \Lemmaref{lem:parallel-extension-solve},
  but applying \substextendSolSol at the end.
\end{proof}


\subsection{Instantiation Extends}

\instantiationextension*
\begin{proof}
  By induction on the given instantiation derivation. 

    \begin{itemize}
      \DerivationProofCase{\InstLSolve}
              { \Gamma \entails \tau}     { \instjudg{\Gamma, \ahat, \Gamma'}
                          {\ahat}
                          {\tau}
                          {\Gamma, \hypeq{\ahat}{\tau}, \Gamma'}
               }

               By \Lemmaref{lem:extension-solve},
               $\substextend{\Gamma, \ahat, \Gamma'}{\Gamma, \hypeq{\ahat}{\tau}, \Gamma'}$. 

      \DerivationProofCase{\InstLReach}
              { }
              {\instjudg{\Gamma[\ahat][\bhat]}
                          {\ahat}
                          {\bhat}
                          {\Gamma[\ahat][\hypeq{\bhat}{\ahat}]}}

          $\Gamma[\ahat][\bhat] = \Gamma_0, \ahat, \Gamma_1, \bhat, \Gamma_2$ for some $\Gamma_0, \Gamma_1, \Gamma_2$. \\
          By the definition of well-formedness, $\judgetp{\Gamma_0, \ahat, \Gamma_1}{\ahat}$. \\
          Therefore, by \Lemmaref{lem:extension-solve}, $\substextend{\Gamma_0, \ahat, \Gamma_1, \bhat, \Gamma_2}{\Gamma_0, \ahat, \Gamma_1, \hypeq{\bhat}{\ahat}, \Gamma_2}$. 

      \DerivationProofCase{\InstLArr}
              {\instjudgr{\Gamma[\ahat_2, \ahat_1, \hypeq{\ahat}{\ahat_1 \arr \ahat_2}]}
                          {\ahat_1}
                          {A_1}
                          {\Gamma'} \\
               \instjudg{\Gamma'}
                          {\ahat_2}
                          {[\Gamma']A_2}
                          {\Delta}}
              {\instjudg{\Gamma[\ahat]}
                          {\ahat}
                          {A_1 \arr A_2}
                          {\Delta}}

          By \Lemmaref{lem:extension-add}, we can insert an (unsolved) $\ahat_2$, giving
            $\substextend{\Gamma[\ahat]}{\Gamma[\ahat_2, \ahat]}$. \\
          By \Lemmaref{lem:extension-add} again, $\substextend{\Gamma[\ahat_2, \ahat]}{\Gamma[\ahat_2, \ahat_1, \ahat]}$. \\
          By \Lemmaref{lem:extension-solve}, we can solve $\ahat$, giving
            $\substextend{\Gamma[\ahat_2, \ahat_1, \ahat]}{\Gamma[\ahat_2, \ahat_1, \hypeq{\ahat}{\ahat_1 \arr \ahat_2}]}$. \\
          Then by transitivity (\Lemmaref{lem:substextend-transitivity}), $\substextend{\Gamma[\ahat]}{\Gamma[\ahat_2, \ahat_1, \hypeq{\ahat}{\ahat_1 \arr \ahat_2}]}$. \\
          By i.h.\ on the first subderivation,
            $\substextend{\Gamma[\ahat_2, \ahat_1, \hypeq{\ahat}{\ahat_1 \arr \ahat_2}]}{\Gamma'}$. \\
          By i.h.\ on the second subderivation, $\substextend{\Gamma'}{\Delta}$. \\
          By transitivity (\Lemmaref{lem:substextend-transitivity}), $\substextend{\Gamma[\ahat_2, \ahat_1, \hypeq{\ahat}{\ahat_1 \arr \ahat_2}]}{\Delta}$. \\
          By transitivity (\Lemmaref{lem:substextend-transitivity}), $\substextend{\Gamma[\ahat]}{\Delta}$. 

      \DerivationProofCase{\InstLAllR}
            {\instjudg{\Gamma[\ahat], \beta}{\ahat}{B}{\Delta, \beta, \Delta'}}
            {\instjudg{\Gamma[\ahat]}{\ahat}{\alltype{\beta}{B}}{\Delta}}

          By induction, $\substextend{\Gamma[\ahat], \beta}{\Delta, \beta, \Delta'}$. \\
          By \Lemmaref{lem:extension-order} (i),
          we have $\substextend{\Gamma[\ahat]}{\Delta}$. 

      \DerivationProofCase{\InstRSolve}
              { \Gamma \entails \tau}
              { \instjudgr{\Gamma, \ahat, \Gamma'}
                          {\ahat}
                          {\tau}
                          {\Gamma, \hypeq{\ahat}{\tau}, \Gamma'}
               }

          By \Lemmaref{lem:extension-solve}, we can solve $\ahat$, giving
          $\substextend{\Gamma, \ahat, \Gamma'}{\Gamma, \hypeq{\ahat}{\tau}, \Gamma'}$. 

      \DerivationProofCase{\InstRReach}
              { }
              {\instjudgr{\Gamma[\ahat][\bhat]}
                         {\ahat}
                         {\bhat}
                         {\Gamma[\ahat][\hypeq{\bhat}{\ahat}]}}

          $\Gamma[\ahat][\bhat] = \Gamma_0, \ahat, \Gamma_1, \bhat, \Gamma_2$ for some $\Gamma_0, \Gamma_1, \Gamma_2$. \\
          By the definition of well-formedness, $\judgetp{\Gamma_0, \ahat, \Gamma_1}{\ahat}$. \\
          Hence by \Lemmaref{lem:extension-solve}, we can solve $\bhat$, giving
          $\substextend{\Gamma_0, \ahat, \Gamma_1, \bhat, \Gamma_2}{\Gamma_0, \ahat, \Gamma_1, \hypeq{\bhat}{\ahat}, \Gamma_2}$. 

      \DerivationProofCase{\InstRArr}
            {\instjudg{\Gamma[\ahat_2, \ahat_1, \hypeq{\ahat}{\ahat_1 \arr \ahat_2}]}
                      {\ahat_1}
                      {A_1}
                      {\Gamma'} \\
               \instjudgr{\Gamma'}
                         {\ahat_2}
                         {[\Gamma']A_2}
                         {\Delta}}
            {\instjudgr{\Gamma[\ahat]}
                       {\ahat}
                       {A_1 \arr A_2}
                       {\Delta}}

          Because the contexts here are the same as in \InstLArr, this is the same as the \InstLArr case.


      \DerivationProofCase{\InstRAllL}
            {\instjudgr{\Gamma[\ahat], \MonnierComma{\bhat}, \bhat}{\ahat}{[\bhat/\beta]B}{\Delta, \MonnierComma{\bhat}, \Delta'}}
            {\instjudgr{\Gamma[\ahat]}{\ahat}{\alltype{\beta}{B}}{\Delta}}

          By i.h., $\substextend{\Gamma[\ahat], \MonnierComma{\bhat}, \bhat}{\Delta, \MonnierComma{\bhat}, \Delta'}$. \\
          By \Lemmaref{lem:extension-order} (ii), $\substextend{\Gamma[\ahat]}{\Delta}$.
    \qedhere
    \end{itemize}
\end{proof}


\subsection{Subtyping Extends}

\subtypingextension*
 \begin{proof}
  By induction on the given derivation.

  For cases \SubVar, \SubUnit, \SubExvar, we have $\Delta = \Gamma$,
  so \Lemmaref{lem:substextend-reflexivity} suffices.
  
  \begin{itemize}
    \DerivationProofCase{\SubArr}
         {\subjudg{\Gamma}{B_1}{A_1}{\Theta}
          \\
          \subjudg{\Theta}{[\Omega]A_2}{[\Omega]B_2}{\Delta}
         }
         { \subjudg{\Gamma}{A_1 \arr A_2}{B_1 \arr B_2}{\Delta} }

         By IH on each subderivation, $\substextend{\Gamma}{\Theta}$
         and $\substextend{\Theta}{\Delta}$.

         By \Lemmaref{lem:substextend-transitivity} (transitivity), $\substextend{\Gamma}{\Delta}$, which was to be shown.

    \DerivationProofCase{\SubAllL}
          {\subjudg{\Gamma, \MonnierComma{\ahat}, \ahat}
                   {[\ahat/\alpha]A}
                   {B}
                   {\Delta, \MonnierComma{\ahat}, \Theta}}
          {\subjudg{\Gamma}{\alltype{\alpha}{A}}{B}{\Delta}}

          By IH, 
          $\substextend
                       {\Gamma, \MonnierComma{\ahat}, \ahat}
                       {\Delta, \MonnierComma{\ahat}, \Theta}$.
          
         By \Lemmaref{lem:extension-order} (ii) with $\Gamma_L = \Gamma$
         and $\Gamma_L' = \Delta$ and $\Gamma_R = \ahat$
         and $\Gamma_R' = \Theta$, we obtain
         \[
              \substextend{\Gamma}{\Delta}
         \]

    \DerivationProofCase{\SubAllR}
          {\subjudg{\Gamma, \beta}{A}{B}{\Delta, \beta, \Theta}}
          {\subjudg{\Gamma}{A}{\alltype{\beta}{B}}{\Delta}}

          By IH, we have
          $\substextend
                       {\Gamma, \beta}
                       {\Delta, \beta, \Theta}$.
          
          By \Lemmaref{lem:extension-order} (i), we obtain
          $
               \substextend{\Gamma}{\Delta}
          $, which was to be shown.
    


    \ProofCasesRules{\SubInstL, \SubInstR}
          In each of these rules, the premise has the same input and output contexts as the conclusion,
          so \Lemmaref{lem:instantiation-extension} suffices.
    \qedhere
  \end{itemize}
\end{proof}







\section{Decidability of Instantiation}
 

\leftunsolvednesspreservation*
\begin{proof}
  By induction on the given derivation.
  
    \begin{itemize}
        \DerivationProofCase{\InstLSolve}
                { \Gamma_0 \entails \tau}
                { \instjudg{\underbrace{\Gamma_0, \ahat, \Gamma_1}_\Gamma}
                            {\ahat}
                            {\tau}
                            {\Gamma_0, \hypeq{\ahat}{\tau}, \Gamma_1}
                 }

            Immediate, since to the left of $\ahat$, the contexts $\Delta$ and $\Gamma$ are the same.

        \DerivationProofCase{\InstLReach}
                { }
                {\instjudg{\Gamma[\ahat][\bhat]}
                            {\ahat}
                            {\bhat}
                            {\Gamma[\ahat][\hypeq{\bhat}{\ahat}]}}

            Immediate, since to the left of $\ahat$, the contexts $\Delta$ and $\Gamma$ are the same.

        \DerivationProofCase{\InstLArr}
                {\instjudgr{\Gamma[\ahat_2, \ahat_1, \hypeq{\ahat}{\ahat_1 \arr \ahat_2}]}
                            {\ahat_1}
                            {A_1}
                            {\Gamma'} \\
                 \instjudg{\Gamma'}
                            {\ahat_2}
                            {[\Gamma']A_2}
                            {\Delta}}
                {\instjudg{\Gamma[\ahat]}
                            {\ahat}
                            {A_1 \arr A_2}
                            {\Delta}}

            We have $\bhat \in \unsolved{\Gamma_0}$.  Therefore $\bhat \in \unsolved{\Gamma_0, \ahat_2}$. \\
            Clearly, $\ahat_2 \in \unsolved{\Gamma_0, \ahat_2}$. \\
            We have two subderivations:
\begin{align*}
                \instjudgr{\Gamma_0, \ahat_2, \ahat_1, \hypeq{\ahat}{\ahat_1 \arr \ahat_2}, \Gamma_1}{\ahat_1}{A_1}{\Gamma'}
                && \text{(1)}
                \\
                \instjudg{\Gamma'}{\ahat_2}{[\Gamma']A_2}{\Delta}
                && \text{(2)}
            \end{align*}
By induction on (1), $\bhat \in \unsolved{\Gamma'}$. \\
            Also by induction on (1), with $\ahat_2$ playing the role of $\bhat$, we get $\ahat_2 \in \unsolved{\Gamma'}$. \\
            Since $\bhat \in \Gamma_0$, it is declared to the left of $\ahat_2$
              in $\Gamma_0, \ahat_2, \ahat_1, \hypeq{\ahat}{\ahat_1 \arr \ahat_2}, \Gamma_1$. \\
            Hence by \Lemmaref{lem:declaration-order-preservation}, $\bhat$ is declared to the left of $\ahat_2$ in $\Gamma'$.
            That is, $\Gamma' = (\Gamma'_0, \ahat_2, \Gamma'_1)$, where $\bhat \in \unsolved{\Gamma'_0}$. \\
            By induction on (2), $\bhat \in \unsolved{\Delta}$. 


        \DerivationProofCase{\InstLAllR}
              {\instjudg{\Gamma_0, \ahat, \Gamma_1, \beta}{\ahat}{B}{\Delta, \beta, \Delta'}}
              {\instjudg{\Gamma_0, \ahat, \Gamma_1}{\ahat}{\alltype{\beta}{B}}{\Delta}}

            We have $\bhat \in \unsolved{\Gamma_0}$. \\
            By induction, $\bhat \in \unsolved{\Delta, \beta, \Delta'}$. \\
            Note that $\bhat$ is declared to the left of $\beta$ in $\Gamma_0, \ahat, \Gamma_1, \beta$. \\
            By \Lemmaref{lem:declaration-order-preservation},
              $\bhat$ is declared to the left of $\beta$ in $(\Delta, \beta, \Delta')$,
              that is, in $\Delta$. Since $\bhat \in \unsolved{\Delta, \beta, \Delta'}$,
              we have $\bhat \in \unsolved{\Delta}$. 


        \ProofCasesRules{\InstRSolve, \InstRReach}
           Similar to the \InstLSolve and \InstLReach cases.

        \DerivationProofCase{\InstRArr}
              {\instjudg{\Gamma[\ahat_2, \ahat_1, \hypeq{\ahat}{\ahat_1 \arr \ahat_2}]}
                        {\ahat_1}
                        {A_1}
                        {\Gamma'} \\
                 \instjudgr{\Gamma'}
                           {\ahat_2}
                           {[\Gamma']A_2}
                           {\Delta}}
              {\instjudgr{\Gamma[\ahat]}
                         {\ahat}
                         {A_1 \arr A_2}
                         {\Delta}}

            Similar to the \InstLArr case.


        \DerivationProofCase{\InstRAllL}
              {\instjudgr{\Gamma[\ahat], \MonnierComma{\chat}, \chat}{\ahat}{[\chat/\beta]B}{\Delta, \MonnierComma{\chat}, \Delta'}}
              {\instjudgr{\Gamma[\ahat]}{\ahat}{\alltype{\beta}{B}}{\Delta}}

            We have $\bhat \in \unsolved{\Gamma_0}$. \\
            By induction, $\bhat \in \unsolved{\Delta, \MonnierComma{\chat}, \Delta'}$. \\
            Note that $\bhat$ is declared to the left of $\MonnierComma{\chat}$ in $\Gamma_0, \ahat, \Gamma_1, \MonnierComma{\chat}, \chat$. \\
            By \Lemmaref{lem:declaration-order-preservation}, $\bhat$ is declared to the left of $\MonnierComma{\chat}$ in $\Delta, \MonnierComma{\chat}, \Delta'$. \\
            Hence $\bhat$ is declared in $\Delta$, and we know it is in $\unsolved{\Delta, \MonnierComma{\chat}, \Delta'}$,
            so $\bhat \in \unsolved{\Delta}$. 
\qedhere
    \end{itemize}
\end{proof}


\leftfreevariablepreservation*
\begin{proof}
  By induction on the given instantiation derivation.


    \begin{itemize}
        \DerivationProofCase{\InstLSolve}
                { \Gamma_0 \entails \tau}
                { \instjudg{\underbrace{\Gamma_0, \ahat, \Gamma_1}_\Gamma}
                            {\ahat}
                            {\tau}
                            {\underbrace{\Gamma_0, \hypeq{\ahat}{\tau}, \Gamma_1}_{\Delta}}
                 }

            We have $\ahat \notin \FV{[\Gamma]B}$.  Since $\Delta$ differs from $\Gamma$
            only in $\ahat$, it must be the case that $[\Gamma]B = [\Delta]B$.
            It is given that $\bhat \notin \FV{[\Gamma]B}$, so $\bhat \notin \FV{[\Delta]B}$.
            

        \DerivationProofCase{\InstLReach}
                { }
                {\instjudg{\underbrace{\Gamma'[\ahat][\chat]}_{\Gamma}}
                            {\ahat}
                            {\chat}
                            {\underbrace{\Gamma'[\ahat][\hypeq{\chat}{\ahat}]}_{\Delta}}}

            Since $\Delta$ differs from $\Gamma$ only
            in solving $\chat$ to $\ahat$, applying $\Delta$ to a type will not introduce a $\bhat$.
            We have $\bhat \notin \FV{[\Gamma]B}$, so $\bhat \notin \FV{[\Delta]B}$.
        
        \DerivationProofCase{\InstRSolve}
                { \Gamma_0 \entails \tau}
                { \instjudgr{\Gamma_0, \ahat, \Gamma_1}
                            {\ahat}
                            {\tau}
                            {\Gamma_0, \hypeq{\ahat}{\tau}, \Gamma_1}
                 }
             
             Similar to the \InstLSolve case.

        \DerivationProofCase{\InstRReach}
                { }
                {\instjudgr{\Gamma'[\ahat][\chat]}
                           {\ahat}
                           {\chat}
                           {\Gamma'[\ahat][\hypeq{\chat}{\ahat}]}}

             Similar to the \InstLReach case.
        
        \DerivationProofCase{\InstLArr}
                {\instjudgr{\overbrace{\Gamma_0, \ahat_2, \ahat_1, \hypeq{\ahat}{\ahat_1 \arr \ahat_2}, \Gamma_1}^{\Gamma'}}
                            {\ahat_1}
                            {A_1}
                            {\Delta} \\
                 \instjudg{\Delta}
                            {\ahat_2}
                            {[\Delta]A_2}
                            {\Delta}}
                {\instjudg{\underbrace{\Gamma_0, \ahat, \Gamma_1}_\Gamma}
                            {\ahat}
                            {A_1 \arr A_2}
                            {\Delta}}

            We have $\judgetp{\Gamma}{B}$
            and $\ahat \notin \FV{[\Gamma]B}$
and $\bhat \notin \FV{[\Gamma]B}$. \\
            By weakening, we get $\judgetp{\Gamma'}{B}$; since $\ahat \notin \FV{[\Gamma]B}$ and $\Gamma'$ only adds
            a solution for $\ahat$, it follows that $[\Gamma']B = [\Gamma]B$. \\
            Therefore $\ahat_1 \notin \FV{[\Gamma']B}$ and $\ahat_2 \notin \FV{[\Gamma']B}$ and $\bhat \notin \FV{[\Gamma']B}$. \\
            Since we have $\bhat \in \Gamma_0$, we also have $\bhat \in (\Gamma_0, \ahat_2)$. \\
            By induction on the first premise, $\bhat \notin \FV{[\Delta]B}$. \\
            Also by induction on the first premise, with $\ahat_2$ playing the role of $\bhat$, we have $\ahat_2 \notin \FV{[\Delta]B}$. \\
            Note that $\ahat_2 \in \unsolved{\Gamma_0, \ahat_2}$. \\
            By \Lemmaref{lem:left-unsolvedness-preservation}, $\ahat_2 \in \unsolved{\Delta}$. \\
            Therefore $\Delta$ has the form $(\Delta_0, \ahat_2, \Delta_1)$. \\
            Since $\bhat \neq \ahat_2$, we know that $\bhat$ is declared to the left of $\ahat_2$ in $\Gamma_0, \ahat_2$,
            so by \Lemmaref{lem:declaration-order-preservation}, $\bhat$ is declared to the left of $\ahat_2$ in $\Delta$.
            Hence $\bhat \in \Delta_0$. \\
            Furthermore, by \Lemmaref{lem:instantiation-extension}, we have $\substextend{\Gamma'}{\Delta}$. \\
            Then by \Lemmaref{lem:extension-weakening}, we have $\judgetp{\Delta}{B}$.
            Using induction on the second premise, $\bhat \notin \FV{[\Delta]B}$. 

        \DerivationProofCase{\InstLAllR}
              {\instjudg{\Gamma_0, \ahat, \Gamma_1, \gamma}{\ahat}{C}{\Delta, \gamma, \Delta'}}
              {\instjudg{\underbrace{\Gamma_0, \ahat, \Gamma_1}_{\Gamma}}{\ahat}{\alltype{\gamma}{C}}{\Delta}}

            We have $\judgetp{\Gamma}{B}$ and $\ahat \notin \FV{[\Gamma]B}$ and $\bhat \in \Gamma_0$ and $\bhat \notin \FV{[\Gamma]B}$. \\
            By weakening, $\judgetp{\Gamma, \gamma}{B}$; by the definition of substitution, $[\Gamma, \gamma]B = [\Gamma]B$. \\
            Substituting equals for equals, $\ahat \notin \FV{[\Gamma, \gamma]B}$ and $\bhat \notin \FV{[\Gamma, \gamma]B}$. \\
            By induction, $\bhat \notin \FV{[\Delta, \gamma, \Delta']B}$. \\
            Since $\bhat$ is declared to the left of $\gamma$ in $(\Gamma, \gamma)$, we can use
            \Lemmaref{lem:declaration-order-preservation} to show that
              $\bhat$ is declared to the left of $\gamma$ in $(\Delta, \gamma, \Delta')$,
              that is, in $\Delta$. \\
            We have $\judgetp{\Gamma}{B}$, so $\gamma \notin \FV{B}$.  Thus each free variable $u$ in $B$ is in $\Gamma$,
            to the left of $\gamma$ in $(\Gamma, \gamma)$. \\
            Therefore, by \Lemmaref{lem:declaration-order-preservation}, each free variable $u$ in $B$ is in $\Delta$. \\
            Therefore $[\Delta, \gamma, \Delta']B = [\Delta]B$. \\
            Earlier, we obtained $\bhat \notin \FV{[\Delta, \gamma, \Delta']B}$, so substituting equals for equals,
            $\bhat \notin \FV{[\Delta]B}$.

        \DerivationProofCase{\InstRArr}
              {\instjudg{\Gamma_0, \ahat_2, \ahat_1, \hypeq{\ahat}{\ahat_1 \arr \ahat_2}, \Gamma_1}
                        {\ahat_1}
                        {A_1}
                        {\Delta} \\
                 \instjudgr{\Gamma'}
                           {\ahat_2}
                           {[\Delta]A_2}
                           {\Delta}}
              {\instjudgr{\Gamma_0, \ahat, \Gamma_1}
                         {\ahat}
                         {A_1 \arr A_2}
                         {\Delta}}

            Similar to the \InstLArr case.


        \DerivationProofCase{\InstRAllL}
              {\instjudgr{\Gamma[\ahat], \MonnierComma{\chat}, \chat}{\ahat}{[\chat/\gamma]C}{\Delta, \MonnierComma{\chat}, \Delta'}}
              {\instjudgr{\Gamma[\ahat]}{\ahat}{\alltype{\gamma}{C}}{\Delta}}

             We have $\judgetp{\Gamma}{B}$ and $\ahat \notin \FV{[\Gamma]B}$ and $\bhat \in \Gamma_0$ and $\bhat \notin \FV{[\Gamma]B}$. \\
             By weakening, $\judgetp{\Gamma, \MonnierComma{\chat}, \chat}{B}$;
               by the definition of substitution, $[\Gamma, \MonnierComma{\chat}, \chat]B = [\Gamma]B$. \\
             Substituting equals for equals, $\ahat \notin \FV{[\Gamma, \MonnierComma{\chat}, \chat]B}$ and $\bhat \notin \FV{[\Gamma, \MonnierComma{\chat}, \chat]B}$. \\
             By induction, $\bhat \notin \FV{[\Delta, \MonnierComma{\chat}, \Delta']B}$. \\
             Note that $\bhat$ is declared to the left of $\MonnierComma{\chat}$ in $\Gamma, \MonnierComma{\chat}, \chat$. \\
             By \Lemmaref{lem:declaration-order-preservation}, $\bhat$ is declared to the left of $\MonnierComma{\chat}$ in $\Delta, \MonnierComma{\chat}, \Delta'$. \\
             So $\bhat$ is declared in $\Delta$. \\
             Now, note that each free variable $u$ in $B$ is in $\Gamma$, which is to the left of $\MonnierComma{\chat}$ in $\Gamma, \MonnierComma{\chat}, \chat$. \\
             Therefore, by \Lemmaref{lem:declaration-order-preservation}, each free variable $u$ in $B$ is in $\Delta$. \\
             Therefore $[\Delta, \MonnierComma{\chat}, \Delta']B = [\Delta]B$. \\
             Earlier, we obtained $\bhat \notin \FV{[\Delta, \MonnierComma{\chat}, \Delta']B}$, so substituting equals for equals,
             $\bhat \notin \FV{[\Delta]B}$.
    \qedhere
    \end{itemize}
\end{proof}



\instantiationsizepreservation*
\begin{proof}
  By induction on the given derivation.
  
  \begin{itemize}
      \DerivationProofCase{\InstLSolve}
              { \Gamma_0 \entails \tau}
              { \instjudg{\underbrace{\Gamma_0, \ahat, \Gamma_1}_\Gamma}
                          {\ahat}
                          {\tau}
                          {\Gamma_0, \hypeq{\ahat}{\tau}, \Gamma_1}
               }

              Since $\Delta$ differs from $\Gamma$ only in solving $\ahat$, and we know $\ahat \notin \FV{[\Gamma]B}$,
              we have $[\Delta]B = [\Gamma]B$; therefore $|[\Delta]B = [\Gamma]B|$.

      \DerivationProofCase{\InstLReach}
              { }
              {\instjudg{\Gamma[\ahat][\bhat]}
                          {\ahat}
                          {\bhat}
                          {\Gamma[\ahat][\hypeq{\bhat}{\ahat}]}
              }

              Here, $\Delta$ differs from $\Gamma$ only in solving $\bhat$ to $\ahat$.  However, $\ahat$ has the same
              size as $\bhat$, so even if $\bhat \in \FV{[\Gamma]B}$, we have $|[\Delta]B = [\Gamma]B|$.

      \DerivationProofCase{\InstLArr}
              {\instjudgr{\overbrace{\Gamma_0, \ahat_2, \ahat_1, \hypeq{\ahat}{\ahat_1 \arr \ahat_2}, \Gamma_1}^{\Gamma'}}
                          {\ahat_1}
                          {A_1}
                          {\Theta} \\
               \instjudg{\Theta}
                          {\ahat_2}
                          {[\Theta]A_2}
                          {\Delta}}
              {\instjudg{\underbrace{\Gamma_0, \ahat, \Gamma_1}_\Gamma}
                          {\ahat}
                          {A_1 \arr A_2}
                          {\Delta}}

      We have $\judgetp{\Gamma}{B}$ and $\ahat \notin \FV{[\Gamma]B}$.
      Since $\ahat_1, \ahat_2 \notin \dom{\Gamma}$, we have $\ahat, \ahat_1, \ahat_2 \notin \FV{[\Gamma]B}$.
      It follows that $[\Gamma']B = [\Gamma]B$. \\
      By weakening, $\judgetp{\Gamma'}{B}$. \\
      By induction on the first premise, $|[\Gamma']B| = |[\Theta]B|$. \\
      By \Lemmaref{lem:declaration-order-preservation},
        since $\ahat_2$ is declared to the left of $\ahat_1$ in $\Gamma'$,
        we have that $\ahat_2$ is declared to the left of $\ahat_1$ in $\Theta$. \\
      By \Lemmaref{lem:left-unsolvedness-preservation}, since $\ahat_2 \in \unsolved{\Gamma'}$, it is unsolved in $\Theta$:
      that is, $\Theta = (\Theta_0, \ahat_2, \Theta_1)$. \\
      By \Lemmaref{lem:instantiation-extension}, we have $\substextend{\Gamma'}{\Theta}$. \\
      By \Lemmaref{lem:extension-weakening}, $\judgetp{\Theta}{B}$. \\
      Since $\ahat_2 \notin \FV{[\Gamma']B}$,
        \Lemmaref{lem:left-free-variable-preservation} gives $\ahat_2 \notin \FV{[\Theta]B}$. \\
      By induction on the second premise, $|[\Theta]B| = |[\Delta]B|$,
        and by transitivity of equality, $|[\Gamma]B| = |[\Delta]B|$. 

      \DerivationProofCase{\InstLAllR}
            {\instjudg{\Gamma_0, \ahat, \Gamma_1, \beta}{\ahat}{A_0}{\Delta, \beta, \Delta'}}
            {\instjudg{\underbrace{\Gamma_0, \ahat, \Gamma_1}_{\Gamma}}{\ahat}{\alltype{\beta}{A_0}}{\Delta}}

      We have $\judgetp{\Gamma}{B}$ and $\ahat \notin \FV{[\Gamma]B}$. \\
      By weakening, $\judgetp{\Gamma, \beta}{B}$. \\
      From the definition of substitution, $[\Gamma]B = [\Gamma, \beta]B$.
      Hence $\ahat \notin \FV{[\Gamma, \beta]B}$. \\
      The input context of the premise is $(\Gamma_0, \ahat, \Gamma_1, \beta)$, which is $(\Gamma, \beta)$, so
      by induction, $|[\Gamma, \beta]B| = |[\Delta, \beta, \Delta']B|$. \\
      Suppose $u$ is a free variable in $B$.
      Then $u$ is declared in $\Gamma$, and so occurs before $\beta$ in $\Gamma, \beta$. \\
      By \Lemmaref{lem:declaration-order-preservation}, $u$ is declared before $\beta$ in $\Delta, \beta, \Delta'$. \\
      So every free variable $u$ in $B$ is declared in $\Delta$. \\
      Hence $[\Delta, \beta, \Delta']B = [\Delta]B$. \\
      We have $[\Gamma]B = [\Gamma, \beta]B$, so $|[\Gamma]B| = |[\Gamma, \beta]B|$;
      by transitivity of equality,  $|[\Gamma]B| = |[\Delta]B|$.

      \DerivationProofCase{\InstRSolve}
              { \Gamma_0 \entails \tau}
              { \instjudgr{\Gamma_0, \ahat, \Gamma_1}
                          {\ahat}
                          {\tau}
                          {\Gamma_0, \hypeq{\ahat}{\tau}, \Gamma_1}
              }

              Similar to the \InstLSolve case.

      \DerivationProofCase{\InstRReach}
              { }
              {\instjudgr{\Gamma[\ahat][\bhat]}
                         {\ahat}
                         {\bhat}
                         {\Gamma[\ahat][\hypeq{\bhat}{\ahat}]}}

              Similar to the \InstLReach case.


      \DerivationProofCase{\InstRArr}
            {\instjudg{\overbrace{\Gamma_0, \ahat_2, \ahat_1, \hypeq{\ahat}{\ahat_1 \arr \ahat_2}, \Gamma_1}^{\Gamma'}}
                      {\ahat_1}
                      {A_1}
                      {\Theta} \\
               \instjudgr{\Theta}
                         {\ahat_2}
                         {[\Theta]A_2}
                         {\Delta}}
            {\instjudgr{\underbrace{\Gamma_0, \ahat, \Gamma_1}_{\Gamma}}
                       {\ahat}
                       {A_1 \arr A_2}
                       {\Delta}}

           Similar to the \InstLArr case.

      \DerivationProofCase{\InstRAllL}
            {\instjudgr{\Gamma'[\ahat], \MonnierComma{\bhat}, \bhat}{\ahat}{[\bhat/\beta]A_0}{\Delta, \MonnierComma{\bhat}, \Delta'}}
            {\instjudgr{\Gamma'[\ahat]}{\ahat}{\alltype{\beta}{A_0}}{\Delta}}

          We have $\judgetp{\Gamma}{B}$ and $\ahat \notin \FV{[\Gamma]B}$. \\
          By weakening, $\judgetp{\Gamma, \MonnierComma{\bhat}, \bhat}{B}$. \\
          From the definition of substitution, $[\Gamma]B = [\Gamma, \MonnierComma{\bhat}, \bhat]B$.
          Hence $\ahat \notin \FV{[\Gamma, \MonnierComma{\bhat}, \bhat]B}$. \\
          By induction, $|[\Gamma, \MonnierComma{\bhat}, \bhat]B| = |[\Delta, \MonnierComma{\bhat}, \Delta']B|$. \\
          Suppose $u$ is a free variable in $B$. \\
          Then $u$ is declared in $\Gamma$, and so occurs before $\MonnierComma{\bhat}$ in $\Gamma, \MonnierComma{\bhat}, \bhat$. \\
          By \Lemmaref{lem:declaration-order-preservation}, $u$ is declared before $\MonnierComma{\bhat}$ in $\Delta, \MonnierComma{\bhat}, \Delta'$. \\
          So every free variable $u$ in $B$ is declared in $\Delta$. \\
          Hence $[\Delta, \MonnierComma{\bhat}, \Delta']B = [\Delta]B$. \\
          Since $[\Gamma]B = [\Gamma, \MonnierComma{\bhat}, \bhat]B$, we have
          $|[\Gamma]B| = |[\Gamma, \MonnierComma{\bhat}, \bhat]B|$;
          by transitivity of equality,
          $|[\Gamma]B| = |[\Delta]B|$.
          \qedhere
  \end{itemize}
\end{proof}



\instantiationdecidability*
\begin{proof}
  By induction on the derivation of $\judgetp{\Gamma}{A}$. 

  \begin{enumerate}[(1)]
  \item $\instjudg{\Gamma}{\ahat}{A}{\Delta}$ is decidable. 

      \begin{itemize}
       \DerivationProofCase{\UvarWF}
                           { }
                           {\judgetp{\underbrace{\Gamma_L, \ahat, \Gamma_R}_{\Gamma'[\alpha]}}{\alpha}}

          If $\alpha \in \Gamma_L$,
          then by \UvarWF we have $\judgetp{\Gamma_L}{\alpha}$,
          and by rule \InstLSolve we have a derivation. \\
          Otherwise no rule matches, and so no derivation exists. 
       
       \ProofCaseRule{\UnitWF}
          By rule \InstLSolve.

       \DerivationProofCase{\EvarWF}
                           { }
                           {\judgetp{\underbrace{\Gamma_L, \ahat, \Gamma_R}_{\Gamma}}{\bhat}}

           By inversion, we have $\bhat \in \Gamma$, and $[\Gamma]\bhat = \bhat$.
           Since $\ahat \notin \FV{[\Gamma]\bhat} = \FV{\bhat} = \{\bhat\}$,
           it follows that $\ahat \neq \bhat$:
           Either $\bhat \in \Gamma_L$ or $\bhat \in \Gamma_R$. \\
           If $\bhat \in \Gamma_L$, then we have a derivation by $\InstLSolve$. \\
           If $\bhat \in \Gamma_R$, then we have a derivation by $\InstLReach$. 

       \DerivationProofCase{\SolvedEvarWF}
                           { }
                           {\judgetp{\underbrace{\Gamma'[\hypeq{\bhat}{\tau}]}_{\Gamma}}{\bhat}}

           It is given that $[\Gamma]\bhat = \bhat$, so this case is impossible.

       \DerivationProofCase{\ArrowWF}
                           {\judgetp{\Gamma}{A_1} \\ \judgetp{\Gamma}{A_2}}
                           {\judgetp{\underbrace{\Gamma_L, \ahat, \Gamma_R}_\Gamma}{A_1 \arr A_2}}

           By assumption, $[\Gamma](A_1 \arr A_2) = A_1 \arr A_2$
           and $\ahat \notin \FV{[\Gamma](A_1 \arr A_2)}$. \\
           If $A_1 \arr A_2$ is a monotype and is well-formed under $\Gamma_L$, we can apply \InstLSolve. \\
           Otherwise, the only rule with a conclusion matching $A_1 \arr A_2$ is \InstLArr. \\
           First, consider whether $\instjudgr{\Gamma_L, \ahat_2, \ahat_1, \hypeq{\ahat}{\ahat_1 \arr \ahat_2}, \Gamma_R}{\ahat_1}{A}{-}$
           is decidable. \\[1ex]
           By definition of substitution,
             $[\Gamma](A_1 \arr A_2) = ([\Gamma]A_1) \arr ([\Gamma]A_2)$.
             Since $[\Gamma](A_1 \arr A_2) = A_1 \arr A_2$,
             we have $[\Gamma]A_1 = A_1$ and $[\Gamma]A_2 = A_2$. \\
           By weakening,
             $\judgetp{\Gamma_L, \ahat_2, \ahat_1, \hypeq{\ahat}{\ahat_1 \arr \ahat_2}, \Gamma_R}{A_1 \arr A_2}$. \\
           Since $\judgetp{\Gamma}{A_1}$ and $\judgetp{\Gamma}{A_2}$, we have
           $\ahat_1, \ahat_2 \notin \FV{A_1} \union \FV{A_2}$. \\
           Since $\ahat \notin \FV{A} \supseteq \FV{A_1}$, it follows that $[\Gamma']A_1 = A_1$. \\
           By i.h., either there exists $\Theta$ such that $\instjudgr{\Gamma_L, \ahat_2, \ahat_1, \hypeq{\ahat}{\ahat_1 \arr \ahat_2}, \Gamma_R}{\ahat_1}{A_1}{\Theta}$, or not. \\
           If not, then no derivation by \InstLArr exists. \\
           If so, then we have $\instjudg{\Gamma_L, \ahat_2, \ahat_1, \hypeq{\ahat}{\ahat_1 \arr \ahat_2}, \Gamma_R}{\ahat_1}{A_1}{\Theta}$. \\
           By \Lemmaref{lem:left-unsolvedness-preservation}, we know that $\ahat_2 \in \unsolved{\Theta}$. \\
           By \Lemmaref{lem:left-free-variable-preservation}, we know that $\ahat_2 \notin \FV{[\Theta]A_2}$. \\
           Clearly, $[\Theta]([\Theta]A_2) = [\Theta]A_2$. \\
           Hence by i.h., either there exists $\Delta$ such that $\instjudg{\Theta}{\ahat_2}{[\Theta]A_2}{\Delta}$, or not. \\
           If not, then no derivation by \InstLArr exists. \\
           If it does, then by rule \InstLArr, we have $\instjudg{\Gamma}{\ahat}{A}{\Delta}$.

      \DerivationProofCase{\ForallWF}
                          {\judgetp{\Gamma, \alpha}{A_0}}
                          {\judgetp{\Gamma}{\alltype{\alpha}{A_0}}}

           We have $\alltype{\alpha} A_0 = [\Gamma](\alltype{\alpha}A_0)$.
           By definition of substitution, $[\Gamma](\alltype{\alpha}A_0) = \alltype{\alpha}[\Gamma]A_0$, so $A_0 = [\Gamma]A_0$. \\
           By definition of substitution, $[\Gamma, \alpha]A_0 = [\Gamma]A_0$. \\
           We have $\ahat \notin \FV{[\Gamma](\alltype{\alpha} A_0)}$.  Therefore
           $\ahat \notin \FV{[\Gamma]A_0} = \FV{[\Gamma, \alpha]A_0}$. \\
           By i.h., either there exists $\Theta$
             such that $\instjudg{\Gamma, \alpha}{\ahat}{A_0}{\Theta}$, or not.
           \\[1ex]
           Suppose $\instjudg{\Gamma, \alpha}{\ahat}{A_0}{\Theta}$. \\
           By \Lemmaref{lem:instantiation-extension}, $\substextend{\Gamma}{\Theta}$; \\
           by \Lemmaref{lem:extension-order} (i), $\Theta = \Delta, \alpha, \Delta'$. \\
           Hence by rule \InstLAllR, $\instjudg{\Gamma}{\ahat}{\alltype{\alpha} A_0}{\Delta}$.
           \\[1ex]
           Suppose not. \\
           Then there is no derivation, since \InstLAllR is the only rule matching $\alltype{\alpha} A_0$. 
      \end{itemize}

  \item $\instjudgr{\Gamma}{\ahat}{A}{\Delta}$ is decidable. 
      \begin{itemize}
       \ProofCaseRule{\UvarWF}

                           Similar to the \UvarWF case in part (1), but applying rule \InstRSolve instead
                           of \InstLSolve.



       \ProofCaseRule{\UnitWF}
            Apply \InstRSolve.

       \DerivationProofCase{\EvarWF}
                           { }
                           {\judgetp{\underbrace{\Gamma_L, \ahat, \Gamma_R}_\Gamma}{\bhat}}

            Similar to the \EvarWF case in part (1), but applying
            \InstRSolve/\InstRReach instead of \InstLSolve/\InstLReach.


       \ProofCaseRule{\SolvedEvarWF} ~\\
           Impossible, for exactly the same reasons as in the \SolvedEvarWF case of part (1).

       \DerivationProofCase{\ArrowWF}
                           {\judgetp{\Gamma}{A_1} \\ \judgetp{\Gamma}{A_2}}
                           {\judgetp{\underbrace{\Gamma_L, \ahat, \Gamma_R}_\Gamma}{A_1 \arr A_2}}

              As the \ArrowWF case of part (1), except applying \InstRArr instead of \InstLArr.





      \DerivationProofCase{\ForallWF}
                          {\judgetp{\Gamma, \beta}{B}}
                          {\judgetp{\underbrace{\Gamma_L, \ahat, \Gamma_R}_\Gamma}{\alltype{\beta}{B}}}

           By assumption, $[\Gamma](\alltype{\beta}{B}) = \alltype{\beta}{B}$.
           With the definition of substitution, we get $[\Gamma]B = B$.
           Hence $[\Gamma]B = B$. \\
           Hence $[\bhat/\beta][\Gamma]B = [\bhat/\beta]B$.
           Since $\bhat$ is fresh, $[\bhat/\beta][\Gamma]B = [\Gamma][\bhat/\beta]B$. \\
           By definition of substitution, $[\Gamma, \MonnierComma{\bhat}, \bhat][\bhat/\beta]B = [\Gamma][\bhat/\beta]B$, which by transitivity of equality is $[\bhat/\beta]B$. \\
           We have $\ahat \notin \FV{[\Gamma](\alltype{\beta}{B})}$,
             so $\ahat \notin \FV{[\Gamma, \MonnierComma{\bhat}, \bhat][\bhat/\beta]B}$. \\
           Therefore, by induction, either $\instjudgr{\Gamma, \MonnierComma{\bhat}, \bhat}{\ahat}{[\bhat/\beta]B}{\Theta}$ or not.
           \\[1ex]
           Suppose $\instjudgr{\Gamma, \MonnierComma{\bhat}, \bhat}{\ahat}{[\bhat/\beta]B}{\Theta}$. \\
           By \Lemmaref{lem:instantiation-extension}, $\substextend{\Gamma, \MonnierComma{\bhat}, \bhat}{\Theta}$; \\
           by \Lemmaref{lem:extension-order} (ii), $\Theta = \Delta, \MonnierComma{\bhat}, \Delta'$. \\
           Hence by rule \InstRAllL, $\instjudgr{\Gamma}{\ahat}{\alltype{\beta}{B}}{\Delta}$.
           \\[1ex]
           Suppose not. \\
           Then there is no derivation, since \InstRAllL is the only rule matching $\alltype{\beta}{B}$. 
      \qedhere
      \end{itemize}
\end{enumerate}
\end{proof}



\section{Decidability of Algorithmic Subtyping}

\subsection{Lemmas for Decidability of Subtyping}

\monotypessolvevariables*
\begin{proof}  By induction on the given derivation. 

  \begin{itemize}
  \DerivationProofCase{\InstLSolve}
          { \Gamma_L \entails \tau}
          { \instjudg{\Gamma_L, \ahat, \Gamma_R}
                      {\ahat}
                      {\tau}
                      {\underbrace{\Gamma_L, \hypeq{\ahat}{\tau}, \Gamma_R}_{\Delta}}
           }

           It is evident that
           $|\unsolved{\Gamma_L, \ahat, \Gamma_R}| = |\unsolved{\Gamma_L, \hypeq{\ahat}{\tau}, \Gamma_R}| + 1$.

  \DerivationProofCase{\InstLReach}
          { }
          {\instjudg{\Gamma[\ahat][\bhat]}
                      {\ahat}
                      {\bhat}
                      {\Gamma[\ahat][\hypeq{\bhat}{\ahat}]}}

          Similar to the previous case.

  \DerivationProofCase{\InstLArr}
          {\instjudgr{\Gamma_0[\ahat_2, \ahat_1, \hypeq{\ahat}{\ahat_1 \arr \ahat_2}]}
                      {\ahat_1}
                      {\tau_1}
                      {\Theta} \\
           \instjudg{\Theta}
                      {\ahat_2}
                      {[\Theta]\tau_2}
                      {\Delta}}
          {\instjudg{\Gamma_0[\ahat]}
                      {\ahat}
                      {\tau_1 \arr \tau_2}
                      {\Delta}}

        \smallskip

        \begin{llproof}
            \eqPf{|\unsolved{\Gamma_0[\ahat_2, \ahat_1, \hypeq{\ahat}{\ahat_1 \arr \ahat_2}]}|}
                    {|\unsolved{\Gamma_0[\ahat]}| + 1}
                    {Immediate}
            \eqPf{|\unsolved{\Gamma_0[\ahat_2, \ahat_1, \hypeq{\ahat}{\ahat_1 \arr \ahat_2}]}|}
                   {|\unsolved{\Theta}| + 1}  {By i.h.}
            \eqPf{|\unsolved{\Gamma}|}{|\unsolved{\Theta}|}   {Subtracting $1$}
        \Hand    \continueeqPf{|\unsolved{\Delta}| + 1}   {By i.h.}
        \end{llproof}

  \DerivationProofCase{\InstLAllR}
        {\instjudg{\Gamma, \beta}{\ahat}{B}{\Delta, \beta, \Delta'}}
        {\instjudg{\Gamma}{\ahat}{\alltype{\beta}{B}}{\Delta}}

        This case is impossible, since a monotype cannot have the form $\alltype{\beta}{B}$.

  \ProofCasesRules{\InstRSolve, \InstRReach}
     Similar to the \InstLSolve and \InstLReach cases.

  \ProofCaseRule{\InstRArr}
     Similar to the \InstLArr case.


  \DerivationProofCase{\InstRAllL}
        {\instjudgr{\Gamma[\ahat], \beta}{\ahat}{B}{\Delta, \beta, \Delta'}}
        {\instjudgr{\Gamma[\ahat]}{\ahat}{\alltype{\beta}{B}}{\Delta}}

  This case is impossible, since a monotype cannot have the form $\alltype{\beta}{B}$.
  \qedhere
  \end{itemize}
\end{proof}


\monotypemonotonicity*
\begin{proof}  By induction on the given derivation.

  \begin{itemize}
    \ProofCasesRules{\SubVar, \SubExvar} ~\\
        In these rules, $\Delta = \Gamma$,
        so $\unsolved{\Delta} = \unsolved{\Gamma}$;
        therefore $|\unsolved{\Delta}| \leq |\unsolved{\Gamma}|$.

    \ProofCaseRule{\SubArr}
        We have an intermediate context $\Theta$.  
        
        By inversion, $\tau_1 = \tau_{11} \arr \tau_{12}$
        and $\tau_2 = \tau_{21} \arr \tau_{22}$.
        Therefore, we have monotypes in the first and second premises.

        By induction on the first premise,
        $|\unsolved{\Theta}| \leq |\unsolved{\Gamma}|$.
        By induction on the second premise,
        $|\unsolved{\Delta}| \leq |\unsolved{\Theta}|$.
        By transitivity of $\leq$, 
        $|\unsolved{\Delta}| \leq |\unsolved{\Gamma}|$,
        which was to be shown.

     \ProofCasesRules{\SubAllL, \SubAllR}
        We are given a derivation of subtyping on monotypes, so these
        cases are impossible.
     
     \ProofCasesRules{\SubInstL, \SubInstR
        }
        The input and output contexts in the premise
        exactly match the conclusion, so the result follows
        by \Lemmaref{lem:monotypes-solve-variables}.
        \qedhere
  \end{itemize}
\end{proof}


\typesizesubst*
\begin{proof}  By induction on $\typesize{\Gamma}{A}$.
  If $A = \unitty$ or $A = \alpha$, or $A = \ahat$ and $\ahat \in \unsolved{\Gamma}$
  then $[\Gamma]A = A$. 
  Therefore, $\typesize{\Gamma}{[\Gamma]A} = \typesize{\Gamma}{A}$.

  If $A = \ahat$ and $(\hypeq{\ahat}{\tau}) \in \Gamma$, then
  by induction hypothesis, $\typesize{\Gamma}{[\Gamma]\tau} \leq \typesize{\Gamma}{\tau}$.
  Of course $\typesize{\Gamma}{\tau} \leq \typesize{\Gamma}{\tau} + 1$.
  By definition of substitution, $[\Gamma]\tau = [\Gamma]\ahat$, so
  \[
      \typesize{\Gamma}{[\Gamma]\ahat} ~\leq~ \typesize{\Gamma}{\tau} + 1
   \]
  By the definition of type size, $\typesize{\Gamma}{\ahat} = \typesize{\Gamma}{\tau} + 1$, so
  \[
      \typesize{\Gamma}{[\Gamma]\ahat} ~\leq~ \typesize{\Gamma}{\ahat}
   \]
   which was to be shown.

  If $A = A_1 \arr A_2$, the result follows via the induction hypothesis (twice).
  
  If $A = \alltype{\alpha}{A_0}$, the result follows via the induction hypothesis.
\end{proof}




\monotypecontextinvariance*
\begin{proof}
  By induction on the derivation of $\subjudg{\Gamma}{\tau}{\tau'}{\Delta}$.

  \begin{itemize}
      \ProofCasesRules{\SubVar, \SubUnit, \SubExvar}

          In these rules, the output context is the same as the input context, so
          the result is immediate.

      \DerivationProofCase{\SubArr}
                {\subjudg{\Gamma}{\tau'_1}{\tau_1}{\Theta}
                  \\
                  \subjudg{\Theta}{[\Theta]\tau_2}{[\Theta]\tau'_2}{\Delta}}
                {\subjudg{\Gamma}{\tau_1 \arr \tau_2}{\tau'_1 \arr \tau'_2}{\Delta}}

          We have that $[\Gamma](\tau_1 \arr \tau_2) = \tau_1 \arr \tau_2$.
          By definition of substitution,
          $[\Gamma]\tau_1 = \tau_1$ and $[\Gamma]\tau_2 = \tau_2$.
          Similarly, $[\Gamma]\tau_1 = \tau'_1$ and $[\Gamma]\tau_2 = \tau'_2$.
          
          By i.h., $\Theta = \Gamma$. \\
          Since $\Theta$ is predicative, $[\Theta]\tau_2$ and $[\Theta]\tau'_2$ are monotypes. \\
          Substitution is idempotent: $[\Theta][\Theta]\tau_2 = [\Theta]\tau_2$
          and $[\Theta][\Theta]\tau'_2 = [\Theta]\tau'_2$. \\
          By i.h., $\Delta = \Theta$.  Hence $\Delta = \Gamma$.  

      \ProofCasesRules{\SubAllL, \SubAllR}
          Impossible, since $\tau$ and $\tau'$ are monotypes. 



      \DerivationProofCase{\SubInstL}
                {
                  \ahat \notin \FV{A}
                  \\
                  \instjudg{\Gamma_0[\ahat]}{\ahat}{A}{\Delta}
                }
                {\subjudg{\Gamma_0[\ahat]}{\ahat}{A}{\Delta}}

          By \Lemmaref{lem:monotypes-solve-variables},
          $|\unsolved{\Delta}| < |\unsolved{\Gamma_0[\ahat]}|$, but it is given that
          $|\unsolved{\Gamma_0[\ahat]}| = |\unsolved{\Delta}|$,
          so this case is impossible.

      \ProofCaseRule{\SubInstR}   Impossible, as for the \SubInstL case.
  \qedhere
  \end{itemize}
\end{proof}


\subsection{Decidability of Subtyping}

\subtypingdecidability*
\begin{proof}
  Let the judgment $\subjudg{\Gamma}{A}{B}{\Delta}$ be measured lexicographically by

  \begin{enumerate}[(S1)]
  \item the number of $\forall$ quantifiers in $A$ and $B$;
  \item $|\unsolved{\Gamma}|$, the number of unsolved existential variables in $\Gamma$;
  \item $\typesize{\Gamma}{A} + \typesize{\Gamma}{B}$.
  \end{enumerate}

  For each subtyping rule, we show that every premise is smaller than the conclusion.
The condition that $[\Gamma]A = A$ and $[\Gamma]B = B$ is easily satisfied at each inductive
  step, using the definition of substitution.

    \begin{itemize}
      \item Rules \SubVar, \SubUnit and \SubExvar have no premises.

      \DerivationProofCase{\SubArr}
          {\subjudg{\Gamma}{B_1}{A_1}{\Theta}
            \\
            \subjudg{\Theta}{[\Theta]A_2}{[\Theta]B_2}{\Delta}
          }
          {
            \subjudg{\Gamma}{A_1 \arr A_2}{B_1 \arr B_2}{\Delta}
          }

          If $A_2$ or $B_2$ has a quantifier, then the first premise is smaller by (S1).
          Otherwise,
          the first premise shares an input context with the conclusion, so it has the same (S2).
          The types $B_1$ and $A_1$ are subterms of the conclusion's types,
          so the first premise is smaller by (S3).
          
          If $B_1$ or $A_1$ has a quantifier, then the second premise is smaller by (S1).
          Otherwise,
          by \Lemmaref{lem:monotype-monotonicity} on the first premise,
          $|\unsolved{\Theta}| \leq |\unsolved{\Gamma}|$.

          \begin{itemize}
          \item 
            If $|\unsolved{\Theta}| < |\unsolved{\Gamma}|$,
            then the second premise is smaller by (S2).
          
          \item
            If $|\unsolved{\Theta}| = |\unsolved{\Gamma}|$, we have the same (S2).

            However, by \Lemmaref{lem:monotype-context-invariance},
            $\Theta = \Gamma$, so
            $\typesize{\Theta}{[\Theta]A_2} = \typesize{\Gamma}{[\Gamma]A_2}$,
            which by \Lemmaref{lem:typesize-subst} is less than or equal to
            $\typesize{\Gamma}{A_2}$.
            
            By the same logic, $\typesize{\Theta}{[\Theta]B_2} \leq \typesize{\Gamma}{B_2}$.
            
            Therefore,
            \[
                \typesize{\Theta}{[\Theta]A_2} ~+~ \typesize{\Theta}{[\Theta]B_2}
                ~~~\leq~~~
                \typesize{\Gamma}{(A_1 \arr A_2)} ~+~ \typesize{\Gamma}{(B_1 \arr B_2)}
             \]
             and the second premise is smaller by (S3).
          \end{itemize}


      \ProofCasesRules{\SubAllL, \SubAllR}
          In each of these rules, the premise has one less quantifier than the conclusion,
          so the premise is smaller by (S1).



      \ProofCasesRules{\SubInstL, \SubInstR}
          Follows from \Theoremref{thm:decidability-of-instantiation}.
      \qedhere
  \end{itemize}
\end{proof}



\section{Decidability of Typing}

\typingdecidable*
\begin{proof}
    For rules deriving judgments of the form
    \[
        \begin{array}[t]{lll}
               \synjudg{\Gamma}{e}{-}{-}
            \\ \chkjudg{\Gamma}{e}{B}{-}
            \\ \appjudg{\Gamma}{e}{A}{-}{-}
        \end{array}
    \]
    (where we write ``$-$'' for parts of the judgments that are outputs),
    the following induction measure on such judgments is adequate to prove decidability:
\[
      \left\langle
        e,~~~
        \begin{array}[c]{llll}
            \syn
          \\[0.3ex]
            \chk,
            &
\typesize{\Gamma}{B}
          \\[0.6ex]
            \app,
            &
\typesize{\Gamma}{A}
        \end{array}
      \right\rangle
\]
    where $\langle \dots \rangle$ denotes lexicographic order, and
    where (when comparing two judgments typing terms of the same size)
    the synthesis judgment (top line) is considered smaller than
    the checking judgment (second line), which in turn is considered smaller than
    the application judgment (bottom line).  That is,
    \[
               \syn ~~\bigprec~~ \chk ~~\bigprec~~ \app
    \]
Note that this measure only uses the input parts of the judgments, leading to a
    straightforward decidability argument.
    
    We will show that in each rule, every
    synthesis/checking/application premise is smaller than the
    conclusion.




    \begin{itemize}
      \ProofCaseRule{\Var}  No premises.

      \ProofCaseRule{\Sub}
          The first premise has the same subject term $e$ as the conclusion, but
          the judgment is smaller because the measure considers a synthesis judgment
          to be smaller than a checking judgment.

          The second premise is a subtyping judgment, which by \Theoremref{thm:subtyping-decidable}
          is decidable.

      \ProofCaseRule{\Anno}
      
          It is easy to show that the judgment $\judgetp{\Gamma}{A}$ is decidable. \\
          The second premise types $e$, but the conclusion types $(e : A)$,
          so the first part of the measure gets smaller.

      \ProofCaseRule{\UnitIntro}  No premises.

      \ProofCaseRule{\ArrIntro}  In the premise, the term is smaller.

      \ProofCaseRule{\ArrElim}  In both premises, the term is smaller.
      
      \ProofCaseRule{\AllIntro}  Both the premise and conclusion type $e$, and both are checking;
         however, $\typesize{\Gamma, \alpha}{A} < \typesize{\Gamma}{\alltype{\alpha}{A}}$,
         so the premise is smaller.
         


      \ProofCaseRule{\ArrApp}
         Both the premise and conclusion type $e$, but the premise is a
         checking judgment, so the premise is smaller.
      
      \ProofCaseRule{\SubstChk}  Both the premise and conclusion type $e$, and both are checking;
        however, since we can apply this rule only when $\Gamma$ has a solution for
        $\ahat$---that is, when $\Gamma = \Gamma_0[\hypeq{\ahat}{\tau}]$---we have that
        $\typesize{\Gamma}{[\Gamma]\ahat} < \typesize{\Gamma}{\ahat}$, making the last
        part of the measure smaller.

      \ProofCaseRule{\SubstApp}  Similar to \SubstChk.
      
      


      \ProofCaseRule{\AllApp}  Both the premise and conclusion type $e$, and both are application
        judgments; however, by the definition of $\typesize{\Gamma}{-}$, the size of
        the type in the premise $[\ahat/\alpha]A$ is smaller than $\alltype{\alpha}{A}$.

      \ProofCaseRule{\SolveApp}  Both the premise and conclusion type $e$, but we switch to
         checking in the premise, so the premise is smaller.
      
      \ProofCaseRule{\UnitIntroSyn}  No premises.

      \ProofCaseRule{\ArrIntroSyn}  In the premise, the term is smaller.
    \qedhere
    \end{itemize}
\end{proof}



\section{Soundness of Subtyping}

\subsection{Lemmas for Soundness}

\ctxappvarpreservation*
\begin{proof}  By mutual induction on $\Delta$ and $\Omega$.

  Suppose $(x : A) \in \Delta$.
  In the case where $\Delta = (\Delta', x : A)$ and $\Omega = (\Omega', x : A_\Omega)$,
  inversion on  $\substextend{\Delta}{\Omega}$ gives $[\Omega']A = [\Omega']A_\Omega$;
  by the definition of context application, $[\Omega', x : A_\Omega](\Delta', x : A) =
    [\Omega']\Delta', x : [\Omega']A_\Omega$, which contains $x : [\Omega']A_\Omega$,
    which is equal to $x : [\Omega']A$.  By well-formedness of $\Omega$, we know that
    $[\Omega']A = [\Omega]A$.

  Suppose $(x : A) \in \Omega$.  The reasoning is similar, because equality is symmetric.
\end{proof}

\substitutiontyping*
\begin{proof}
  By induction on $\typesize{\Gamma}{A}$ (the size of $A$ under $\Gamma$).
  
  \begin{itemize}
      \ProofCasesRules{\UvarWF, \UnitWF}
           Here $A = \alpha$ or $A = \unitty$, so applying $\Gamma$ to $A$ does not change it: $A = [\Gamma]A$.
           Since $\judgetp{\Gamma}{A}$, we have $\judgetp{\Gamma}{[\Gamma]A}$, which was to be shown.
      
      \ProofCaseRule{\EvarWF}
           In this case $A = \ahat$, but $\Gamma = \Gamma_0[\ahat]$, so applying $\Gamma$ to $A$ does not change it,
           and we proceed as in the \UnitWF case above.

      \ProofCaseRule{\SolvedEvarWF}
           In this case $A = \ahat$ and $\Gamma = \Gamma_L, \hypeq{\ahat}{\tau}, \Gamma_R$.
           Thus $[\Gamma]A = [\Gamma]\alpha = [\Gamma_L]\tau$.  We assume contexts are well-formed,
           so all free variables in $\tau$ are declared in $\Gamma_L$.  Consequently,
           $\typesize{\Gamma_L}{\tau} = \typesize{\Gamma}{\tau}$,
           which is less than $\typesize{\Gamma}{\ahat}$.  We can therefore apply the i.h. to $\tau$, yielding
           $\judgetp{\Gamma}{[\Gamma]\tau}$.  By the definition of substitution, $[\Gamma]\tau = [\Gamma]\ahat$,
           so we have $\judgetp{\Gamma}[\Gamma]\ahat$.

      \ProofCaseRule{\ArrowWF}
           In this case $A = A_1 \arr A_2$.
           By i.h., $\judgetp{\Gamma}{[\Gamma]A_1}$ and $\judgetp{\Gamma}{[\Gamma]A_2}$.
           By \ArrowWF, $\judgetp{\Gamma}{([\Gamma]A_1) \arr ([\Gamma]A_2)}$,
           which by the definition of substitution is
           $\judgetp{\Gamma}{[\Gamma](A_1 \arr A_2)}$.

      \ProofCaseRule{\ForallWF}
          In this case $A = \alltype{\alpha} A_0$.  By i.h., $\judgetp{\Gamma, \alpha}{[\Gamma, \alpha]A_0}$.
          By the definition of substitution, $[\Gamma, \alpha]A_0 = [\Gamma]A_0$,
          so by \ForallWF, $\judgetp{\Gamma}{\alltype{\alpha} [\Gamma]A_0}$,
          which by the definition of substitution is
          $\judgetp{\Gamma}{[\Gamma](\alltype{\alpha} A_0)}$.
  \qedhere
  \end{itemize}
\end{proof}



\completionwf*
\begin{proof}
    By induction on $\typesize{\Omega}{A}$, the size of $A$ under $\Omega$
    (\Definitionref{def:typesize}).
    
    We consider cases of the well-formedness rule concluding the derivation of
    $\judgetp{\Omega}{A}$.
    
    \begin{itemize}
        \DerivationProofCase{\UnitWF}
              { } 
              { \judgetp{\Omega}{\unitty} }

              \begin{llproof}
                \judgetpPf{[\Omega]\Omega} {\unitty}   {By \DeclUnitWF}
                \judgetpPf{[\Omega]\Omega} {[\Omega]\unitty}   {By definition of substitution}
              \end{llproof}

        \DerivationProofCase{\UvarWF}
              { }
              { \judgetp{\underbrace{\Omega'[\alpha]}_{\Omega}}{\alpha} }

              \begin{llproof}
                \substextendPf{\Omega}{\Omega}   {By \Lemmaref{lem:substextend-reflexivity}}
                \Pf{}{}{\alpha \in [\Omega]\Omega}  {By \Lemmaref{lem:ctxapp-uvar-preservation}}
                \judgetpPf{[\Omega]\Omega} {\alpha}   {By \DeclUvarWF}
                \judgetpPf{[\Omega]\Omega} {[\Omega]\alpha}   {By definition of substitution}
              \end{llproof}
        
        \DerivationProofCase{\SolvedEvarWF}
              { }
              { \judgetp{\underbrace{\Omega'[\hypeq{\ahat}{\tau}]}_{\Omega}}{\ahat} }

              \begin{llproof}
               \judgetpPf{\Omega}{\ahat}   {Given}
               \substextendPf{\Omega}{\Omega}    {By \Lemmaref{lem:substextend-reflexivity}}
               \judgetpPf{\Omega}{[\Omega]\ahat}   {By \Lemmaref{lem:substitution-typing}}
               \ltPf{\typesize{\Omega}{[\Omega]\ahat}}
                    {\typesize{\Omega}{\ahat}}     {Follows from definition of type size}
               \judgetpPf{[\Omega]\Omega}{[\Omega][\Omega]\ahat}   {By i.h.}
               \eqPf{[\Omega][\Omega]\ahat}{[\Omega]\ahat}   {By \Lemmaref{lem:subst-extension-invariance}}
               \judgetpPf{[\Omega]\Omega}{[\Omega]\ahat}   {Applying equality}
              \end{llproof}

        \DerivationProofCase{\EvarWF}
              { }
              { \judgetp{\underbrace{\Omega'[\ahat]}_{\Omega}}{\ahat} }
              
              Impossible: the grammar for $\Omega$ does not allow unsolved declarations.

        \DerivationProofCase{\ArrowWF}
              { \judgetp{\Omega}{A_1}
                \\
                \judgetp{\Omega}{A_2} }
              { \judgetp{\Omega}{A_1 \arr A_2} }
              
              \begin{llproof}
                \judgetpPf{\Omega}{A_1}    {Subderivation}
                \ltPf{\typesize{\Omega}{A_1}}
                   {\typesize{\Omega}{A_1 \arr A_2}} {Follows from definition of type size}
                \judgetpPf{[\Omega]\Omega}{[\Omega]A_1}   {By i.h.}
                \proofsep
                \judgetpPf{[\Omega]\Omega}{[\Omega]A_2}   {By similar reasoning on 2nd subderivation}
                \proofsep
                \judgetpPf{[\Omega]\Omega}{[\Omega]A_1 \arr [\Omega]A_2}   {By \DeclArrowWF}
                \judgetpPf{[\Omega]\Omega}{[\Omega](A_1 \arr A_2)}   {By definition of substitution}
              \end{llproof}

        \DerivationProofCase{\ForallWF}
              { \judgetp{\Omega, \alpha}{A_0} }
              { \judgetp{\Omega}{\alltype{\alpha}{A_0}} }
              
              \begin{llproof}
                \judgetpPf{\Omega, \alpha}{A_0}    {Subderivation}
                \LetPf{\Omega'}{(\Omega, \alpha)}  {}
                \ltPf{\typesize{\Omega'}{A_0}}
                   {\typesize{\Omega}{\alltype{\alpha}{A_0}}} {Follows from definition of type size}
                \judgetpPf{[\Omega'](\Omega, \alpha)}{[\Omega']A_0}   {By i.h.}
                \judgetpPf{[\Omega]\Omega, \alpha}{[\Omega']A_0}   {By definition of context application}
                \judgetpPf{[\Omega]\Omega, \alpha}{[\Omega]A_0}   {By definition of substitution}
                \judgetpPf{[\Omega]\Omega} {\alltype{\alpha} [\Omega]A_0}   {By \DeclForallWF}
                \judgetpPf{[\Omega]\Omega} {[\Omega](\alltype{\alpha} A_0)}   {By definition of substitution}
              \end{llproof}
              \qedhere
    \end{itemize}
\end{proof}



\substitutionstability*
\begin{proof}
  By induction on $\Omega_Z$.
If $\Omega_Z = \cdot$, the result is immediate.
  Otherwise, use the i.h.\ and the fact that
  $\judgetp{\Omega}{A}$ implies $\FV{A} \sect \dom{\Omega_Z} = \emptyset$.
\end{proof}

\contextpartitioning*
\begin{proof}
  By induction on the given derivation.
  
  \begin{itemize}
      \ProofCaseRule{\substextendId}   Impossible: $\Delta, \MonnierComma{\ahat}, \Theta$ cannot have the form $\cdot$.

      \ProofCaseRule{\substextendVV}  We have $\Omega_Z = (\Omega_Z', x : A)$ and $\Theta = (\Theta', x : A')$.
        By i.h., there is $\Psi'$ such that 
        $[\Omega, \MonnierComma{\ahat}, \Omega_Z'](\Delta, \MonnierComma{\ahat}, \Theta')
       =
       [\Omega]\Delta, \Psi'$.  Then by the definition of context application,
       $[\Omega, \MonnierComma{\ahat}, \Omega_Z', x : A](\Delta, \MonnierComma{\ahat}, \Theta', x : A')
      = [\Omega]\Delta, \Psi', x : [\Omega']A$.  Let $\Psi = (\Psi', x : [\Omega']A)$.

      \ProofCaseRule{\substextendUU}  Similar to the \substextendVV case, with $\Psi = (\Psi', \alpha)$.

      \ProofCasesRules{\substextendEE, \substextendSolve, \substextendMonMon, \substextendAdd, \substextendAddSolved}
        Broadly similar to the \substextendUU case, but since the rightmost context element is soft it disappears in context application,
        so we let $\Psi = \Psi'$.
  \qedhere
  \end{itemize}
\end{proof}







\completesstability*
\begin{proof}
  By induction on the derivation of $\substextend{\Gamma}{\Omega}$. 

  \begin{itemize}

    \DerivationProofCase{\substextendId}
          { }
          {\substextend{\cdot}{\cdot}}

          In this case, $\Omega = \Gamma = \cdot$. \\
          By definition, $[\cdot]\cdot = \cdot$, which gives us the conclusion. 

    \DerivationProofCase{\substextendVV}
          {\substextend{\Gamma'}{\Omega'}
            \\
            [\Omega']A_\Gamma = [\Omega']A}
         {\substextend{\Gamma', x : A_\Gamma}{\Omega', x : A}}

         \begin{llproof}
           \eqPf {[\Omega']\Gamma'} {[\Omega']\Omega'}   {By i.h.}
           \eqPf{[\Omega']A_\Gamma} {[\Omega']A}   {Premise}
           \proofsep
           \eqPf{[\Omega]\Gamma } { [\Omega', x : A](\Gamma', x : A_\Gamma)}   {Expanding $\Omega$ and $\Gamma$}
           \continueeqPf{ [\Omega']\Gamma', x : [\Omega']A_\Gamma}  {By definition of context application}
               \trailingjust{(using $[\Omega']A_\Gamma = [\Omega']A$)}
           \continueeqPf{ [\Omega']\Omega', x : [\Omega']A}  {By above equalities}
           \continueeqPf{ [\Omega]\Omega}   {By definition of context application}
         \end{llproof}
                  
    \DerivationProofCase{\substextendUU}
          {\substextend{\Gamma'}{\Omega'}}
          {\substextend{\Gamma', \alpha}{\Omega', \alpha}}

          \begin{llproof}
            \eqPf{[\Omega]\Gamma} {[\Omega', \alpha](\Gamma', \alpha)}   {Expanding $\Omega$ and $\Gamma$}
            \continueeqPf{[\Omega']\Gamma', \alpha}  {By definition of context application}
            \continueeqPf{[\Omega']\Omega', \alpha}  {By i.h.}
            \continueeqPf{[\Omega', \alpha](\Omega', \alpha)}  {By definition of context application}
            \continueeqPf{[\Omega]\Omega} {By $\Omega = (\Omega', \alpha)$}
          \end{llproof}

    \DerivationProofCase{\substextendMonMon}
          {\substextend{\Gamma'}{\Omega'}}
          {\substextend{\Gamma', \MonnierComma{\ahat}}{\Omega', \MonnierComma{\ahat}}}

          Similar to the \substextendUU case.



    
    \DerivationProofCase{\substextendAddSolved}
          {\substextend{\Gamma}{\Omega'}}
          {\substextend{\Gamma}{\Omega', \hypeq{\ahat}{\tau}}}

          \begin{llproof}
            \eqPf{[\Omega]\Gamma} {[\Omega', \hypeq{\ahat}{\tau}]\Gamma}  {Expanding $\Omega$}
            \continueeqPf { [\Omega']\Gamma}  {By $\ahat \notin \dom{\Gamma}$}
            \continueeqPf { [\Omega']\Omega'}  {By i.h.}
            \continueeqPf { [\Omega', \hypeq{\ahat}{\tau}](\Omega', \hypeq{\ahat}{\tau})}
                      {By definition of context application}
            \continueeqPf { [\Omega]\Omega}   {By $\Omega = (\Omega', \hypeq{\ahat}{\tau})$}
          \end{llproof}
                   
    \DerivationProofCase{\substextendSolSol}
          {\substextend{\Gamma'}{\Omega'} \\
           [\Omega']\tau_\Gamma = [\Omega']\tau}
          {\substextend{\Gamma', \hypeq{\ahat}{\tau_\Gamma}}{\Omega', \hypeq{\ahat}{\tau}}}

          \begin{llproof}
            \eqPf{[\Omega]\Gamma} {[\Omega', \hypeq{\ahat}{\tau}](\Gamma', \hypeq{\ahat}{\tau_\Gamma})} {Expanding $\Omega$ and $\Gamma$}
                      \continueeqPf { [\Omega']\Gamma'} {By definition of context application}
                      \continueeqPf { [\Omega']\Omega'} {By i.h.}
                      \continueeqPf { [\Omega', \hypeq{\ahat}{\tau}](\Omega', \hypeq{\ahat}{\tau})}  {By definition of context application}
                      \continueeqPf { [\Omega]\Omega} {By $\Omega = (\Omega', \hypeq{\ahat}{\tau})$}
           \end{llproof}


    \DerivationProofCase{\substextendSolve}
          {\substextend{\Gamma'}{\Omega'}}
          {\substextend{\Gamma', \ahat}{\Omega', \hypeq{\ahat}{\tau}}}

          \begin{llproof}
            \eqPf{[\Omega]\Gamma} {[\Omega', \hypeq{\ahat}{\tau}](\Gamma', \ahat)}
                {Expanding $\Omega$ and $\Gamma$}
                      \continueeqPf { [\Omega']\Gamma'} {By definition of context application}
                      \continueeqPf { [\Omega']\Omega'} {By i.h.}
                      \continueeqPf { [\Omega', \hypeq{\ahat}{\tau}](\Omega', \hypeq{\ahat}{\tau})}
                           {By definition of context application}
                      \continueeqPf { [\Omega]\Omega} {By $\Omega = (\Omega', \hypeq{\ahat}{\tau})$}
            \end{llproof}

    \DerivationProofCase{\substextendEE}
        { \substextend{\Gamma}{\Delta} }
        { \substextend{\Gamma, \ahat}{\Delta, \ahat} }
    
        Impossible: $\Omega$ cannot have the form $\Delta, \ahat$.

    \DerivationProofCase{\substextendAdd}
        { \substextend{\Gamma}{\Delta} }
        { \substextend{\Gamma}{\Delta, \ahat} }

        Impossible: $\Omega$ cannot have the form $\Delta, \ahat$.
  \qedhere
  \end{itemize}
\end{proof}



\finishingtypes*
\begin{proof}
  By \Lemmaref{lem:subst-extension-invariance},
  $[\Omega']A = [\Omega'][\Omega]A$. \\
  If $\FEV{C} = \emptyset$ then $[\Omega']C = C$. \\
  Since $\Omega$ is complete and $\judgetp{\Omega}{A}$, we have
  $\FEV{[\Omega]A} = \emptyset$.  Therefore $[\Omega'][\Omega]A = [\Omega]A$.
\end{proof}



\finishingcompletions*
\begin{proof}
  By induction on the given derivation of $\substextend{\Omega}{\Omega'}$.

  Only cases \substextendId, \substextendVV, \substextendUU, \substextendSolSol,
  \substextendMonMon and \substextendAddSolved are possible.
  In all of these cases, we use the i.h. and the definition of context application;
  in cases \substextendVV and \substextendSolSol, we also use the equality in the
  premise of the respective rule.
\end{proof}



\confluenceofcompleteness*
\begin{proof} ~\\
  \begin{llproof}
    \substextendPf{\Delta_1}{\Omega}    {Given}
    \eqPf{[\Omega]\Delta_1}{[\Omega]\Omega}  {By \Lemmaref{lem:completes-stability}}
    \substextendPf{\Delta_2}{\Omega}    {Given}
    \eqPf{[\Omega]\Delta_2}{[\Omega]\Omega}  {By \Lemmaref{lem:completes-stability}}
    \eqPf{[\Omega]\Delta_1}{[\Omega]\Delta_2}  {By transitivity of equality}
  \end{llproof}
\end{proof}
  


\subsection{Instantiation Soundness}


\instantiationsoundness*
\begin{proof}
  By induction on the given instantiation derivation.

    \begin{enumerate}[(1)]
      \item
      \begin{itemize}
      \DerivationProofCase{\InstLSolve}
              { \Gamma_0 \entails \tau}
              { \instjudg{\underbrace{\Gamma_0, \ahat, \Gamma_1}_\Gamma}
                          {\ahat}
                          {\tau}
                         {\underbrace{\Gamma_0, \hypeq{\ahat}{\tau}, \Gamma_1}_{\Delta}}
               }

              In this case $[\Delta]\ahat = [\Delta]\tau$.
              By reflexivity of subtyping (\Lemmaref{lem:declarative-reflexivity}),
              $\declsubjudg{[\Omega]\Delta}{[\Delta]\ahat}{[\Delta]\tau}$. 

      \DerivationProofCase{\InstLReach}
              { }
              {\instjudg{\Gamma[\ahat][\bhat]}
                          {\ahat}
                          {\bhat}
                          {\underbrace{\Gamma[\ahat][\hypeq{\bhat}{\ahat}]}_{\Delta}}}

              We have $\Delta = \Gamma[\ahat][\hypeq{\bhat}{\ahat}]$.  Therefore $[\Delta]\ahat = \ahat = [\Delta]\bhat$. \\
              By reflexivity of subtyping (\Lemmaref{lem:declarative-reflexivity}),
              $\declsubjudg{[\Omega]\Delta}{[\Delta]\ahat}{[\Delta]\bhat}$.

      \DerivationProofCase{\InstLArr}
              {\instjudgr{\overbrace{\Gamma[\ahat_2, \ahat_1, \hypeq{\ahat}{\ahat_1 \arr \ahat_2}]}^{\Gamma_1}}
                          {\ahat_1}
                          {A_1}
                          {\Gamma'} \\
               \instjudg{\Gamma'}
                          {\ahat_2}
                          {[\Gamma']A_2}
                          {\Delta}}
              {\instjudg{\Gamma[\ahat]}
                          {\ahat}
                          {A_1 \arr A_2}
                          {\Delta}}

                  \begin{llproof}
                    \eqPf{[\Gamma](A_1 \arr A_2)} {[\Gamma_1](A_1 \arr A_2)}  {$\ahat \notin \FV{A_1 \arr A_2}$}
                    \notinPf{\ahat_1, \ahat_2} {\FV{A_1} \union \FV{A_2}}  {$\ahat_1, \ahat_2$ fresh}
                    \instjudgPf{\Gamma'}
                          {\ahat_2}
                          {[\Gamma']A_2}
                          {\Delta}  {Subderivation}
                     \substextendPf{\Gamma'}{\Delta}  {By \Lemmaref{lem:instantiation-extension}}
                     \substextendPf{\Delta}{\Omega}   {Given}
                     \substextendPf{\Gamma'}{\Omega}    {By  \Lemmaref{lem:substextend-transitivity}}
                     \proofsep
                     \instjudgrPf{\Gamma_1}
                          {\ahat_1}
                          {A_1}
                          {\Gamma'}      {Subderivation}
                      \declsubjudgPf{[\Omega]\Delta}{[\Omega]A_1}{[\Omega]\ahat_1}   {By i.h. and \Lemmaref{lem:completes-confluence}}
                      \proofsep
                      \instjudgPf{\Gamma'}
                           {\ahat_2}
                           {[\Gamma']A_2}
                           {\Delta}  {Subderivation}
                      \declsubjudgPf{[\Omega]\Delta}{[\Omega][\Gamma']\ahat_2}{[\Omega][\Gamma']A_2}
                              {By i.h.}
                      \substextendPf{\Gamma'}{\Omega}  {Above}
                      \declsubjudgPf{[\Omega]\Delta}{[\Omega]\ahat_2}{[\Omega]A_2}
                              {By \Lemmaref{lem:subst-extension-invariance}}
                      \proofsep
                      \declsubjudgPf{[\Omega]\Delta}{[\Omega](\ahat_1 \arr \ahat_2)}{[\Omega]A_1 \arr [\Omega]A_2} 
                                     {By \DsubArr and definition of substitution}
                  \end{llproof}
                  
                  Since $(\hypeq{\ahat}{\ahat_1 \arr \ahat_2}) \in \Gamma_1$
                  and $\substextend{\Gamma_1}{\Delta}$, we know that $[\Omega]\ahat = [\Omega](\ahat_1 \arr \ahat_2)$. \\
                  Therefore $\declsubjudg{[\Omega]\Delta}{[\Omega]\ahat}{[\Omega](A_1 \arr A_2)}$.


      \DerivationProofCase{\InstLAllR}
            {\instjudg{\Gamma[\ahat], \beta}{\ahat}{B_0}{\Delta, \beta, \Delta'}}
            {\instjudg{\Gamma[\ahat]}{\ahat}{\alltype{\beta}{B_0}}{\Delta}}

                  We have $\substextend{\Delta}{\Omega}$
                  and $[\Gamma[\ahat]](\alltype{\beta}{B_0}) = \alltype{\beta}{B_0}$
                  and $\ahat \notin \FV{\alltype{\beta}{B_0}}$. \\
                  Hence $\ahat \notin \FV{B_0}$ and by definition, $[\Gamma[\ahat], \beta]B_0 = B_0$. \\
                  By \Lemmaref{lem:soln-completes}, $\substextend{\Delta, \beta, \Delta'}{\Omega, \beta, |\Delta'|}$. \\
                  By induction, $\declsubjudg{[\Omega, \beta, |\Delta'|](\Delta, \beta, \Delta')}{[\Omega, \beta, |\Delta'|]\ahat}{[\Omega, \beta, |\Delta'|]B_0}$. \\
                  Each free variable in $\ahat$ and $B_0$ is declared in $(\Omega, \beta)$, so $\Omega, \beta, |\Delta'|$
                  behaves as $[\Omega, \beta]$ on $\ahat$ and on $B_0$, yielding
                 $\declsubjudg{[\Omega, \beta, |\Delta'|](\Delta, \beta, \Delta')}{[\Omega, \beta]\ahat}{[\Omega, \beta]B_0}$. \\
                  By \Lemmaref{lem:context-partitioning} and thinning, 
                  $\declsubjudg{[\Omega, \beta](\Delta, \beta)}{[\Omega, \beta]\ahat}{[\Omega, \beta]B_0}$. \\
                  By the definition of context application,
                    $\declsubjudg{[\Omega]\Delta, \beta}{[\Omega, \beta]\ahat}{[\Omega, \beta]B_0}$. \\
                  By the definition of substitution, $\declsubjudg{[\Omega]\Delta, \beta}{[\Omega]\ahat}{[\Omega]B_0}$. \\
                  Since $\ahat$ is declared to the left of $\beta$, we have $\beta \notin \FV{[\Omega]\ahat}$. \\
                  Applying rule \DsubAllL gives
                    $\declsubjudg{[\Omega]\Delta}{[\Omega]\ahat}{\alltype{\beta}{[\Omega]B_0}}$.
      \end{itemize}

      \item
          \begin{itemize}
          \DerivationProofCase{\InstRSolve}
                  { \Gamma_0 \entails \tau}
                  { \instjudgr{\underbrace{\Gamma_0, \ahat, \Gamma_1}_\Gamma}
                              {\ahat}
                              {\tau}
                              {\underbrace{\Gamma_0, \hypeq{\ahat}{\tau}, \Gamma_1}_{\Gamma'}}
                   }

                Similar to the \InstLSolve case.

          \DerivationProofCase{\InstRReach}
                  { }
                  {\instjudgr{\Gamma[\ahat][\bhat]}
                             {\ahat}
                             {\bhat}
                             {\underbrace{\Gamma[\ahat][\hypeq{\bhat}{\ahat}]}_{\Gamma'}}}

              Similar to the \InstLReach case.

          \DerivationProofCase{\InstRArr}
                {\instjudg{\Gamma[\ahat_2, \ahat_1, \hypeq{\ahat}{\ahat_1 \arr \ahat_2}]}
                          {\ahat_1}
                          {A_1}
                          {\Gamma'} \\
                   \instjudgr{\Gamma'}
                             {\ahat_2}
                             {[\Gamma']A_2}
                             {\Delta}}
                {\instjudgr{\Gamma[\ahat]}
                           {\ahat}
                           {A_1 \arr A_2}
                           {\Delta}}

                 Similar to the \InstLArr case.




\DerivationProofCase{\InstRAllL}
                {\instjudgr{\Gamma[\ahat], \MonnierComma{\bhat}, \bhat}{\ahat}{[\bhat/\beta]B_0}{\Delta, \MonnierComma{\bhat}, \Delta'}}
                {\instjudgr{\Gamma[\ahat]}{\ahat}{\alltype{\beta}{B_0}}{\Delta}}

                \medskip

                \begin{llproof}
                  \eqPf{\big[\Gamma[\ahat]\big](\alltype{\beta}{B_0})} {\alltype{\beta}{B_0}} {Given}
                  \eqPf{\big[\Gamma[\ahat]\big]B_0} {B_0} {~}
                  \eqPf{\big[\Gamma[\ahat], \MonnierComma{\bhat}, \bhat\big][\bhat/\beta]B_0}
                       {[\bhat/\beta]B_0}   {~}
                  \proofsep
                  \substextendPf{\Delta}{\Omega}   {Given}
                  \substextendPf{\Delta, \MonnierComma{\bhat}, \Delta'}
                                {\Omega, \MonnierComma{\bhat}, |\Delta'|}
                                {By \Lemmaref{lem:soln-completes}}
                  \proofsep
                  \notinPf{\ahat} {\FV{\alltype{\beta}{B_0}}}  {Given}
                  \notinPf{\ahat} {\FV{B_0}}  {By definition of $\FV{-}$}
                  \proofsep
                  \decolumnizePf
                  \instjudgrPf{\Gamma[\ahat], \MonnierComma{\bhat}, \bhat}{\ahat}{[\bhat/\beta]B_0}{\Delta, \MonnierComma{\bhat}, \Delta'}  {Subderivation}
                   \declsubjudgPf{[\Omega, \MonnierComma{\bhat}, |\Delta'|](\Delta, \MonnierComma{\bhat}, \Delta')}
                                {[\Omega, \MonnierComma{\bhat}, |\Delta'|][\bhat/\beta]B_0}
                                {[\Omega, \MonnierComma{\bhat}, |\Delta'|]\ahat}
                                {By i.h.}
                   \substextendPf{\Gamma[\ahat], \MonnierComma{\bhat}, \bhat}
                                 {\Delta, \MonnierComma{\bhat}, \Delta'}
                                 {\hspace{-60pt}By \Lemmaref{lem:instantiation-extension}}
                \end{llproof} 

              By \Lemmaref{lem:declaration-order-preservation},
$\ahat$ is declared before $\MonnierComma{\bhat}$, that is, in $\Omega$. \\
              Thus, $\big[\Omega, \MonnierComma{\bhat}, |\Delta'|\big]\ahat = [\Omega]\ahat$. \\
              By \Lemmaref{lem:evar-input}, we know that $\Delta'$ is soft, so
              by \Lemmaref{lem:softness-goes-away},
              $[\Omega, \MonnierComma{\bhat}, |\Delta'|](\Delta, \MonnierComma{\bhat}, \Delta') = 
                         [\Omega, \MonnierComma{\bhat}](\Delta, \MonnierComma{\bhat}) = 
                         [\Omega]\Delta$. \\
              Applying these equalities to the derivation above gives
              \[
                 \declsubjudg{[\Omega]\Delta}{\big[\Omega, \MonnierComma{\bhat}, |\Delta'|\big][\bhat/\beta]B_0}{[\Omega]\ahat}
              \]
              By distributivity of substitution,
              \[
                  \declsubjudg{[\Omega]\Delta}
                          {\big[[\Omega, \MonnierComma{\bhat}, |\Delta'|]\bhat/\beta\big]\big[\Omega, \MonnierComma{\bhat}, |\Delta'|\big]B_0}{[\Omega]\ahat}
              \]
              Furthermore, $[\Omega, \MonnierComma{\bhat}, |\Delta'|]B_0 = [\Omega]B_0$, since $B_0$'s free variables are either $\beta$ or in $\Omega$, giving
              \[
                   \declsubjudg{[\Omega]\Delta}{\big[[\Omega, \MonnierComma{\bhat}, |\Delta'|]\bhat/\beta\big][\Omega]B_0}{[\Omega]\ahat}
              \]
              Now apply \DsubAllL and the definition of substitution to get
                $\declsubjudg{[\Omega]\Delta}{[\Omega](\alltype{\beta}{B_0})}{[\Omega]\ahat}$.
          \qedhere
      \end{itemize}
    \end{enumerate}
\end{proof}



\subsection{Soundness of Subtyping}


\soundness*
\begin{proof} By induction on the derivation of $\subjudg{\Gamma}{A}{B}{\Delta}$.

  \begin{itemize}
     \DerivationProofCase{\SubVar}
          { }
          {\subjudg{\underbrace{\Gamma'[\alpha]}_{\Gamma}}{\alpha}{\alpha}{\underbrace{\Gamma'[\alpha]}_{\Delta}}}
          
          \begin{llproof}
            \inPf{\alpha}{\Delta}  {$\Delta = \Gamma'[\alpha]$}
            \inPf{\alpha}{[\Omega]\Delta}  {Follows from definition of context application}
            \declsubjudgPf{[\Omega]\Delta} {\alpha} {\alpha} {By \DsubVar}
            \declsubjudgPf{[\Omega]\Delta} {[\Omega]\alpha} {[\Omega]\alpha} {By def. of substitution}
          \end{llproof}

     \ProofCaseRule{\SubUnit}
          Similar to the \SubVar case, applying rule \DsubUnit instead of \DsubVar.

     \DerivationProofCase{\SubExvar}
          { }
          {\subjudg{\Gamma_L, \ahat, \Gamma_R}{\ahat}{\ahat}{\Gamma_L, \ahat, \Gamma_R}}

          \begin{llproof}
               \Pf{}{}{[\Omega]\ahat\text{~defined}}  {Follows from definition of context application}
               \judgetpPf{[\Omega]\Delta} {[\Omega]\ahat}   {Assumption that $[\Omega]\Delta$ is well-formed}
               \declsubjudgPf{[\Omega]\Delta}{[\Omega]\ahat}{[\Omega]\ahat}   {By \Lemmaref{lem:declarative-reflexivity}}
          \end{llproof}

     \DerivationProofCase{\SubArr}
          {\subjudg{\Gamma}{B_1}{A_1}{\Theta} \\
            \subjudg{\Theta}{[\Theta]A_2}{[\Theta]B_2}{\Delta}}
          {\subjudg{\Gamma}{\underbrace{A_1 \arr A_2}_{A}}{\underbrace{B_1 \arr B_2}_{B}}{\Delta}}

          \begin{llproof}
            \subjudgPf{\Gamma}{B_1}{A_1}{\Theta} {Subderivation}
            \Pf{}{}{\substextend{\Delta}{\Omega}} {Given}
            \Pf{}{}{\substextend{\Theta}{\Omega}} {By \Lemmaref{lem:substextend-transitivity}} 
            \declsubjudgPf{[\Omega]\Theta} {[\Omega]B_1} {[\Omega]A_1}  {By i.h.}
            \declsubjudgPf{[\Omega]\Delta} {[\Omega]B_1} {[\Omega]A_1}  {By \Lemmaref{lem:completes-confluence}}
            \proofsep
            \subjudgPf{\Theta}{[\Theta]A_2}{[\Theta]B_2}{\Delta} {Subderivation}
            \declsubjudgPf{[\Omega]\Delta} {[\Omega][\Theta]A_2} {[\Omega][\Theta]B_2}  {By i.h.}
            \Pf{}{}{[\Omega][\Theta]A_2 = [\Omega]A_2}  {By \Lemmaref{lem:subst-extension-invariance}}
            \Pf{}{}{[\Omega][\Theta]B_2 = [\Omega]B_2}  {By \Lemmaref{lem:subst-extension-invariance}}
            \declsubjudgPf{[\Omega]\Delta} {[\Omega]A_2} {[\Omega]B_2}  {Above equations}
            \proofsep
            \declsubjudgPf{[\Omega]\Delta} {([\Omega]A_1) \arr ([\Omega]A_2)}
                                        {([\Omega]B_1) \arr ([\Omega]B_2)}  {By \DsubArr}
            \declsubjudgPf{[\Omega]\Delta} {[\Omega](A_1 \arr A_2)}
                                        {[\Omega](B_1 \arr B_2)}  {By def. of substitution}
          \end{llproof}

     \DerivationProofCase{\SubAllL}
          {\subjudg{\Gamma, \MonnierComma{\ahat}, \ahat}
                   {[\ahat/\alpha]A_0}
                   {B}
                   {\Delta, \MonnierComma{\ahat}, \Theta}}
          {\subjudg{\Gamma}{\alltype{\alpha}{A_0}}{B}{\Delta}}

          Let $\Omega' = (\Omega, \soln{\MonnierComma{\ahat}, \Theta})$.
          
          \begin{llproof}
            \subjudgPf{\Gamma, \MonnierComma{\ahat}, \ahat}
                   {[\ahat/\alpha]A_0}
                   {B}
                   {\Delta, \MonnierComma{\ahat}, \Theta} 
                       {Subderivation}
            \proofsep
            \substextendPf{\Delta}{\Omega}   {Given}
            \substextendPf{(\Delta, \MonnierComma{\ahat}, \Theta)}{\Omega'}   {By \Lemmaref{lem:soln-completes}}
            \proofsep
            \declsubjudgPf{[\Omega'](\Delta, \MonnierComma{\ahat}, \Theta)}
                          {[\Omega'][\ahat/\alpha]A_0}
                          {[\Omega']B}
                       {By i.h.}
            \declsubjudgPf{[\Omega'](\Delta, \MonnierComma{\ahat}, \Theta)}
                          {[\Omega'][\ahat/\alpha]A_0}
                          {[\Omega]B}
                       {By $[\Omega']B =[\Omega]B$ (\Lemmaref{lem:substitution-stability})}
            \declsubjudgPf{[\Omega'](\Delta, \MonnierComma{\ahat}, \Theta)}
                          {\big[[\Omega']\ahat/\alpha\big][\Omega']A_0}
                          {[\Omega]B}
                       {By distributivity of substitution}
            \proofsep
            \judgetpPf{\Gamma, \MonnierComma{\ahat}, \ahat} {\ahat} {By \EvarWF}
            \substextendPf{\Gamma, \MonnierComma{\ahat}, \ahat}  {\Delta, \MonnierComma{\ahat}, \Theta}  {By \Lemmaref{lem:subtyping-extension}}
            \judgetpPf{\Delta, \MonnierComma{\ahat}, \Theta} {\ahat}  {By \Lemmaref{lem:extension-weakening}}
            \substextendPf{(\Delta, \MonnierComma{\ahat}, \Theta)}{\Omega'}   {Above}
            \judgetpPf{[\Omega']\Omega'} {[\Omega']\ahat}   {By \Lemmaref{lem:completion-wf}}
            \judgetpPf{[\Omega'](\Delta, \MonnierComma{\ahat}, \Theta)} {[\Omega']\ahat}   {By \Lemmaref{lem:completes-stability}}
            \proofsep
            \declsubjudgPf{[\Omega'](\Delta, \MonnierComma{\ahat}, \Theta)}
                          {\alltype{\alpha}{[\Omega']A_0}}
                          {[\Omega]B}   {By \DsubAllL}
            \declsubjudgPf{[\Omega'](\Delta, \MonnierComma{\ahat}, \Theta)}
                          {\alltype{\alpha}{[\Omega, \alpha]A_0}}
                          {[\Omega]B}   {By \Lemmaref{lem:substitution-stability}}
            \declsubjudgPf{[\Omega]\Delta}
                          {\alltype{\alpha}{[\Omega, \alpha]A_0}}
                          {[\Omega]B}   {By \Lemmaref{lem:context-partitioning} and thinning}
            \declsubjudgPf{[\Omega]\Delta}
                          {\alltype{\alpha}{[\Omega]A_0}}
                          {[\Omega]B}   {By def.\ of substitution}
            \declsubjudgPf{[\Omega]\Delta}
                          {[\Omega](\alltype{\alpha}{A_0})}
                          {[\Omega]B}   {By def.\ of substitution}
          \end{llproof}

     \DerivationProofCase{\SubAllR}
           {\subjudg{\Gamma, \alpha}{A}{B_0}{\Delta, \alpha, \Theta}}
           {\subjudg{\Gamma}{A}{\alltype{\alpha}{B_0}}{\Delta}}

           \begin{llproof}
             \subjudgPf{\Gamma, \alpha}{A}{B_0}{\Delta, \alpha, \Theta}  {Subderivation}
             \LetPf{\Omega_Z}{\soln{\Theta}}   {}
             \LetPf{\Omega'}{(\Omega, \alpha, \Omega_Z)} {}
             \substextendPf{(\Delta, \alpha, \Theta)}{\Omega'}   {By \Lemmaref{lem:soln-completes}}
             \declsubjudgPf{[\Omega'](\Delta, \alpha, \Theta)} {[\Omega']A} {[\Omega']B_0}  {By i.h.}
             \declsubjudgPf{[\Omega, \alpha](\Delta, \alpha)} {[\Omega, \alpha]A} {[\Omega, \alpha]B_0}  {By \Lemmaref{lem:substitution-stability}}
             \declsubjudgPf{[\Omega, \alpha](\Delta, \alpha)} {[\Omega]A} {[\Omega]B_0}  {By def.\ of substitution}
             \declsubjudgPf{[\Omega]\Delta} {[\Omega]A} {\alltype{\alpha}{[\Omega]B_0}}  {By \DsubAllR}
             \declsubjudgPf{[\Omega]\Delta} {[\Omega]A} {[\Omega](\alltype{\alpha}{B_0})}  {By def.\ of substitution}
           \end{llproof}



     \DerivationProofCase{\SubInstL}
          {
            \ahat \notin \FV{B}
            \\
            \instjudg{\Gamma}{\ahat}{B}{\Delta}
          }
          {\subjudg{\underbrace{\Gamma}_{\Gamma_0[\ahat]}}{\ahat}{B}{\Delta}}

          \begin{llproof}
            \instjudgPf{\Gamma}{\ahat}{B}{\Delta}   {Subderivation}
            \declsubjudgPf{[\Omega]\Delta}{[\Omega]\ahat}{[\Omega]B}   {By \Theoremref{thm:instantiation-soundness}}
          \end{llproof}

     \ProofCaseRule{\SubInstR}   Similar to the case for \SubInstL.
     \qedhere
  \end{itemize}
\end{proof}



\soundnesspretty*
\begin{proof}
  By reflexivity (\Lemmaref{lem:substextend-reflexivity}), $\substextend{\Psi}{\Psi}$. \\ 
  Since $\Psi$ has no existential variables, it is a complete context $\Omega$. \\ 
  By \Theoremref{thm:subtyping-soundness}, $\declsubjudg{[\Psi]\Psi}{[\Psi]A}{[\Psi]B}$. \\ 
  Since $\Psi$ has no existential variables, $[\Psi]\Psi = \Psi$, and $[\Psi]A = A$, and $[\Psi]B = B$. \\
  Therefore $\declsubjudg{\Psi}{A}{B}$.
\end{proof}



\section{Typing Extension}


\typingextension*
\begin{proof}
  By induction on the given derivation.
  
  \begin{itemize}
       \ProofCasesRules{\Var, \UnitIntro, \UnitIntroSyn}

          Since $\Delta = \Gamma$, the result follows by \Lemmaref{lem:substextend-reflexivity}.

     \DerivationProofCase{\Sub}
          {\synjudg{\Gamma}{e}{B}{\Theta}
            \\
            \subjudg{\Theta}{[\Theta]B}{[\Theta]A}{\Delta}
          }
          {\chkjudg{\Gamma}{e}{A}{\Delta}}

          \begin{llproof}
            \substextendPf{\Gamma}{\Theta}   {By i.h.}
            \substextendPf{\Theta}{\Delta}   {By \Lemmaref{lem:subtyping-extension}}
\Hand        \substextendPf{\Gamma}{\Delta}  {By \Lemmaref{lem:substextend-transitivity}}
          \end{llproof}

     \DerivationProofCase{\Anno}
          {\judgetp{\Gamma}{A}
           \\
           \chkjudg{\Gamma}{e}{A}{\Delta}
          }
          {\synjudg{\Gamma}{(e : A)}{A}{\Delta}}
          
          \begin{llproof}
            \Hand \substextendPf{\Gamma}{\Delta}  {By i.h.}
          \end{llproof}


     \DerivationProofCase{\AllIntro}
           {\chkjudg{\Gamma, \alpha}{e}{A_0}{\Delta, \alpha, \Theta}
           }
           {\chkjudg{\Gamma}{e}{\alltype{\alpha}{A_0}}{\Delta}}

          \begin{llproof}
            \substextendPf{\Gamma, \alpha}{\Delta, \alpha, \Theta}  {By i.h.}
\Hand       \substextendPf{\Gamma}{\Delta}   {By \Lemmaref{lem:extension-order} (i)}
          \end{llproof}

     \DerivationProofCase{\AllApp}
            {\appjudg{\Gamma, \ahat}{e}{[\ahat/\alpha]A_0}{C}{\Delta}}
            {\appjudg{\Gamma}{e}{\alltype{\alpha}{A_0}}{C}{\Delta}}

          \begin{llproof}
             \substextendPf{\Gamma, \ahat}{\Delta}  {By i.h.}
             \substextendPf{\Gamma}{\Gamma, \ahat}  {By \substextendAdd}
\Hand        \substextendPf{\Gamma}{\Delta}  {By \Lemmaref{lem:substextend-transitivity}}
          \end{llproof}
            

     \DerivationProofCase{\ArrIntro}
          {\chkjudg{\Gamma, x : A_1}{e}{A_2}{\Delta, x : A_1, \Theta}
          }
          {\chkjudg{\Gamma}{\lam{x} e}{A_1 \arr A_2}{\Delta}}

          \begin{llproof}
            \substextendPf{\Gamma, x : A_1}{\Delta, x : A_1, \Theta}  {By i.h.}
\Hand       \substextendPf{\Gamma}{\Delta}   {By \Lemmaref{lem:extension-order} (v)}
          \end{llproof}

     \DerivationProofCase{\ArrElim}
           {\synjudg{\Gamma}{e_1}{B}{\Theta}
             \\
             \appjudg{\Theta}{e_2}{[\Theta]B}{A}{\Delta}
           }
           {\synjudg{\Gamma}{e_1\,e_2}{A}{\Delta}}

           By the i.h. on each premise, then \Lemmaref{lem:substextend-transitivity}.

      \DerivationProofCase{\ArrIntroSyn}
           { \chkjudg{\Gamma, \ahat, \bhat, x : \ahat}{e}{\bhat}{\Delta, x : \ahat, \Theta}
           }
           {{\synjudg{\Gamma}{\lam{x} e}{\ahat \arr \bhat}{\Delta}}}

          \begin{llproof}
            \substextendPf{\Gamma, \ahat, \bhat, x : \ahat}{\Delta, x : \ahat, \Theta}  {By i.h.}
            \substextendPf{\Gamma, \ahat, \bhat}{\Delta}   {By \Lemmaref{lem:extension-order} (v)}
            \substextendPf{\Gamma}{\Gamma, \ahat, \bhat}   {By \substextendAdd (twice)}
\Hand       \substextendPf{\Gamma}{\Delta}   {By \Lemmaref{lem:substextend-transitivity}}
          \end{llproof}
           
      \DerivationProofCase{\ArrApp}
            {\chkjudg{\Gamma}{e}{A}{\Delta}}
            {\appjudg{\Gamma}{e}{A \arr C}{C}{\Delta}}

            \begin{llproof}
\Hand              \substextendPf{\Gamma}{\Delta}   {By i.h.}
            \end{llproof}

      \DerivationProofCase{\SolveApp}
            {\chkjudg{\Gamma[\ahat_2, \ahat_1, \hypeq{\ahat}{\ahat_1 \arr \ahat_2}]}{e}{\ahat_1}{\Delta}}
            {\appjudg{\Gamma[\ahat]}{e}{\ahat}{\ahat_2}{\Delta}}

          \begin{llproof}
            \substextendPf{\Gamma[\ahat_2, \ahat_1, \hypeq{\ahat}{\ahat_1 \arr \ahat_2}]}{\Delta}  {By i.h.}
            \decolumnizePf
            \substextendPf{\Gamma[\ahat]}
                          {\Gamma[\ahat_2, \ahat_1, \hypeq{\ahat}{\ahat_1 \arr \ahat_2}]}
                          {By \Lemmaref{lem:extension-addsolve}}
                          \trailingjust{then \Lemmaref{lem:parallel-admissibility} (ii)}
\Hand       \substextendPf{\Gamma}{\Delta}   {By \Lemmaref{lem:substextend-transitivity}   \qedhere}
          \end{llproof}
            
  \end{itemize}
\end{proof}




\section{Soundness of Typing}



\typingsoundness*
\begin{proof}
  By induction on the given algorithmic typing derivation.

  \begin{itemize}
     \DerivationProofCase{\Var}
          {(x : A) \in \Gamma}
          {\synjudg{\Gamma}{x}{A}{\Gamma}}
          
          \begin{llproof}
            \inPf{(x : A)}{\Gamma}  {Premise}
            \inPf{(x : A)}{\Delta}  {By $\Gamma = \Delta$}
            \substextendPf{\Delta}{\Omega}  {Given}
            \inPf{(x : [\Omega]A)}{[\Omega]\Gamma}  {By \Lemmaref{lem:ctxapp-var-preservation}}
\Hand            \declsynjudgPf{[\Omega]\Gamma}{x}{[\Omega]A} {By \DeclVar}
          \end{llproof}

     \DerivationProofCase{\Sub}
          {\synjudg{\Gamma}{e}{A}{\Theta}
            \\
            \subjudg{\Theta}{[\Theta]A}{[\Theta]B}{\Delta}
          }
          {\chkjudg{\Gamma}{e}{B}{\Delta}}

          \begin{llproof}
            \synjudgPf{\Gamma}{e}{A}{\Theta}  {Subderivation}
            \subjudgPf{\Theta}{[\Theta]A}{[\Theta]B}{\Delta}  {Subderivation}
            \substextendPf{\Theta}{\Delta}  {By \Lemmaref{lem:typing-extension}}
            \substextendPf{\Delta}{\Omega}  {Given}
            \substextendPf{\Theta}{\Omega}  {By \Lemmaref{lem:substextend-transitivity}}
            \declsynjudgPf{[\Omega]\Theta} {e} {[\Omega]A}  {By i.h.}
            \eqPf{[\Omega]\Theta} {[\Omega]\Delta}   {By \Lemmaref{lem:completes-confluence}}
            \declsynjudgPf{[\Omega]\Delta} {e} {[\Omega]A}  {By above equalities}
            \proofsep
            \subjudgPf{\Theta}{[\Theta]A}{[\Theta]B}{\Delta}  {Subderivation}
            \declsubjudgPf{[\Omega]\Delta} {[\Omega][\Theta]A} {[\Omega][\Theta]B}  {By \Theoremref{thm:subtyping-soundness}}
            \eqPf{[\Omega][\Theta]A} {[\Omega]A}   {By \Lemmaref{lem:subst-extension-invariance}}
            \eqPf{[\Omega][\Theta]B} {[\Omega]B}   {By \Lemmaref{lem:subst-extension-invariance}}
            \declsubjudgPf{[\Omega]\Delta} {[\Omega]A} {[\Omega]B}  {By above equalities}
\Hand       \declchkjudgPf{[\Omega]\Delta} {e} {[\Omega]B}  {By \DeclSub}
          \end{llproof}

     \DerivationProofCase{\Anno}
          {\judgetp{\Gamma}{A}
           \\
           \chkjudg{\Gamma}{e_0}{A}{\Delta}
          }
          {\synjudg{\Gamma}{(e_0 : A)}{A}{\Delta}}
          
          \begin{llproof}
            \chkjudgPf{\Gamma}{e_0}{A}{\Delta}  {Subderivation}
            \declchkjudgPf{[\Omega]\Delta} {e_0} {[\Omega]A}  {By i.h.}
            \proofsep
            \judgetpPf{\Gamma}{A}  {Subderivation}
            \substextendPf{\Gamma}{\Delta}  {By \Lemmaref{lem:typing-extension}}
            \substextendPf{\Delta}{\Omega}  {Given}
            \substextendPf{\Gamma}{\Omega}  {By \Lemmaref{lem:substextend-transitivity}}
            \judgetpPf{\Omega} {A}   {By \Lemmaref{lem:extension-weakening}}
            \judgetpPf{[\Omega]\Omega} {[\Omega]A}   {By \Lemmaref{lem:completion-wf}}
            \eqPf{[\Omega]\Delta}{[\Omega]\Omega}  {By \Lemmaref{lem:completes-stability}}
            \judgetpPf{[\Omega]\Delta} {[\Omega]A}   {By above equality}
            \proofsep
            \declsynjudgPf{[\Omega]\Delta} {(e_0 : [\Omega]A)} {[\Omega]A}  {By \DeclAnno}
            \Pf{}{}{A\text{~contains no existential variables}}   {Assumption about source programs}
            \eqPf{[\Omega]A} {A}  {From definition of substitution}
\Hand     \declsynjudgPf{[\Omega]\Delta} {(e_0 : A)} {[\Omega]A}  {By above equality}
          \end{llproof}

     \DerivationProofCase{\UnitIntro}
          {}
          {\chkjudg{\Gamma}{\unitexp}{\unitty}{\underbrace{\Gamma}_{\Delta}}}

          \begin{llproof}
            \declchkjudgPf{[\Omega]\Delta}{\unitexp}{\unitty}   {By \DeclUnitIntro}
\Hand    \declchkjudgPf{[\Omega]\Delta} {\unitexp} {[\Omega]\unitty}  {By definition of substitution}
          \end{llproof}

     \DerivationProofCase{\ArrIntro}
          {\chkjudg{\Gamma, x : A_1}{e_0}{A_2}{\Delta, x : A_1, \Theta}
          }
          {\chkjudg{\Gamma}{\lam{x} e}{A_1 \arr A_2}{\Delta}}

          \begin{llproof}
            \substextendPf{\Delta}{\Omega}  {Given}
            \substextendPf{\Delta, x : A_1}{\Omega, x : [\Omega]A_1}  {By \substextendVV}
            \substextendPf{\Gamma, x : A_1}{\Delta, x : A_1, \Theta}  {By \Lemmaref{lem:typing-extension}}
            \Pf{}{}{\Theta\text{~is soft}}   {By \Lemmaref{lem:extension-order} (v)}
                                            \trailingjust{(with $\Gamma_R = \cdot$, which is soft)}
            \substextendPf{\underbrace{\Delta, x : A_1, \Theta}_{\Delta'}}{\underbrace{\Omega, x : [\Omega]A_1, \soln{\Theta}}_{\Omega'}}  {By \Lemmaref{lem:soln-completes}}
            \proofsep
            \chkjudgPf{\Gamma, x : A_1}{e_0}{A_2}{\Delta'}  {Subderivation}
            \proofsep
            \declchkjudgPf{[\Omega']\Delta'} {e_0} {[\Omega']A_2}  {By i.h.}
            \eqPf{[\Omega']A_2}{[\Omega]A_2}   {By \Lemmaref{lem:substitution-stability}}
            \declchkjudgPf{[\Omega']\Delta'} {e_0} {[\Omega]A_2} {By above equality}
            \proofsep
            \substextendPf{\underbrace{\Delta, x : A_1, \Theta}_{\Delta'}}{\underbrace{\Omega, x : [\Omega]A_1, \soln{\Theta}}_{\Omega'}}  {Above}
            \Pf{}{}{\Theta\text{~is soft}}   {Above}
            \eqPf{[\Omega']\Delta'} {[\Omega]\Delta, x : [\Omega]A_1}   {By \Lemmaref{lem:softness-goes-away}}
            \declchkjudgPf{[\Omega]\Delta, x : [\Omega]A_1} {e_0} {[\Omega]A_2} {By above equality}
            \decolumnizePf
            \declchkjudgPf{[\Omega]\Delta} {\lam{x} e_0} {([\Omega]A_1) \arr ([\Omega]A_2)}  {By \DeclArrIntro}
\Hand       \declchkjudgPf{[\Omega]\Delta} {\lam{x} e_0} {[\Omega](A_1 \arr A_2)}  {By definition of substitution}
          \end{llproof}

     \DerivationProofCase{\ArrElim}
          {\synjudg{\Gamma}{e_1}{A_1}{\Theta}
            \\
            \appjudg{\Theta}{e_2}{A_1}{A_2}{\Delta}
          }
          {\synjudg{\Gamma}{e_1\,e_2}{A_2}{\Delta}}

          \begin{llproof}
            \synjudgPf{\Gamma}{e_1}{A_1}{\Theta}  {Subderivation}
            \subjudgPf{\Theta}{A_1}{B}{\Delta}  {Subderivation}
            \substextendPf{\Theta}{\Delta}  {By \Lemmaref{lem:typing-extension}}
            \substextendPf{\Delta}{\Omega}  {Given}
            \substextendPf{\Theta}{\Omega}  {By \Lemmaref{lem:substextend-transitivity}}
            \declsynjudgPf{[\Omega]\Theta} {e_1} {[\Omega]A_1}  {By i.h.}
            \eqPf{[\Omega]\Theta} {[\Omega]\Delta}   {By \Lemmaref{lem:completes-confluence}}
            \declsynjudgPf{[\Omega]\Delta} {e_1} {[\Omega]A_1}  {By above equality}
            \proofsep
            \appjudgPf{\Theta}{e_2}{A_1}{A_2}{\Delta}  {Subderivation}
            \substextendPf{\Delta}{\Omega}  {Given}
            \declappjudgPf{[\Omega]\Delta}{e_2}{[\Omega]A_1}{[\Omega]A_2}  {By i.h.}
\Hand       \declsynjudgPf{[\Omega]\Delta} {e_1 e_2} {[\Omega]A_2}  {By \DeclArrElim}
          \end{llproof}



     \DerivationProofCase{\AllIntro}
           {\chkjudg{\Gamma, \alpha}{e}{A_0}{\Delta, \alpha, \Theta}
           }
           {\chkjudg{\Gamma}{e}{\alltype{\alpha}{A_0}}{\Delta}}

          (Similar to \ArrIntro, using a different subpart of \Lemmaref{lem:extension-order}
          and applying \DeclAllIntro; written out anyway.)

          \begin{llproof}
            \substextendPf{\Delta}{\Omega}  {Given}
            \substextendPf{\Delta, \alpha}{\Omega, \alpha}  {By \substextendUU}
            \substextendPf{\Gamma, \alpha}{\Delta, \alpha, \Theta}  {By \Lemmaref{lem:typing-extension}}
            \Pf{}{}{\Theta\text{~is soft}}   {By \Lemmaref{lem:extension-order} (i) (with $\Gamma_R = \cdot$, which is soft)}
            \substextendPf{\underbrace{\Delta, \alpha, \Theta}_{\Delta'}}{\underbrace{\Omega, \alpha, \soln{\Theta}}_{\Omega'}}  {By \Lemmaref{lem:soln-completes}}
            \proofsep
            \chkjudgPf{\Gamma, \alpha}{e}{A_0}{\Delta'}  {Subderivation}
            \proofsep
            \declchkjudgPf{[\Omega']\Delta'} {e} {[\Omega']A_0}  {By i.h.}
            \eqPf{[\Omega']A_0}{[\Omega]A_0}   {By \Lemmaref{lem:substitution-stability}}
            \declchkjudgPf{[\Omega']\Delta'} {e} {[\Omega]A_0} {By above equality}
            \proofsep
            \substextendPf{\underbrace{\Delta, \alpha, \Theta}_{\Delta'}}{\underbrace{\Omega, \alpha, \soln{\Theta}}_{\Omega'}}  {Above}
            \Pf{}{}{\Theta\text{~is soft}}   {Above}
            \eqPf{[\Omega']\Delta'} {[\Omega]\Delta, \alpha}   {By \Lemmaref{lem:softness-goes-away}}
            \declchkjudgPf{[\Omega]\Delta, \alpha} {e} {[\Omega]A_0} {By above equality}
            \proofsep
            \declchkjudgPf{[\Omega]\Delta} {e} {\alltype{\alpha} [\Omega]A_0}  {By \DeclAllIntro}
\Hand       \declchkjudgPf{[\Omega]\Delta} {e} {[\Omega](\alltype{\alpha} A_0)}  {By definition of substitution}
          \end{llproof}


     \DerivationProofCase{\AllApp}
            {\appjudg{\Gamma,\ahat}{e}{[\ahat/\alpha]A_0}{C}{\Delta}}
            {\appjudg{\Gamma}{e}{\alltype{\alpha}{A_0}}{C}{\Delta}}

            \begin{llproof}
              \appjudgPf{\Gamma, \ahat}{e}{[\ahat/\alpha]A_0}{C}{\Delta}  {Subderivation}
              \substextendPf{\Delta}{\Omega}   {Given}
              \declappjudgPf{[\Omega]\Delta}{e}{[\Omega][\ahat/\alpha]A_0}{[\Omega]C}  {By i.h.}
              \declappjudgPf{[\Omega]\Delta}{e}{\big[[\Omega]\ahat\,/\,\alpha\big]\,[\Omega]A_0}{[\Omega]C}  {By distributivity of substitution}
              \proofsep
              \substextendPf{\Gamma, \ahat} {\Delta}   {By \Lemmaref{lem:typing-extension}}
              \substextendPf{\Gamma, \ahat} {\Omega}   {By \Lemmaref{lem:substextend-transitivity}}
              \judgetpPf{\Gamma, \ahat} {\ahat}   {By \EvarWF}
              \judgetpPf{\Omega} {\ahat}    {By \Lemmaref{lem:extension-weakening}}
              \judgetpPf{[\Omega]\Omega} {[\Omega]\ahat}   {By \Lemmaref{lem:completion-wf}}
              \eqPf{[\Omega]\Omega} {[\Omega]\Delta}   {By \Lemmaref{lem:completes-stability}}
              \judgetpPf{[\Omega]\Delta} {[\Omega]\ahat}   {By above equality}
              \proofsep
              \declappjudgPf{[\Omega]\Delta}{e}{\alltype{\alpha} [\Omega]A_0}{[\Omega]C}   {By \DeclAllApp}
\Hand              \declappjudgPf{[\Omega]\Delta}{e}{[\Omega](\alltype{\alpha} A_0)}{[\Omega]C}   {By definition of substitution}
            \end{llproof}

     \DerivationProofCase{\ArrApp}
            {\chkjudg{\Gamma}{e}{B}{\Delta}}
            {\appjudg{\Gamma}{e}{B \arr C}{C}{\Delta}}
            
            \begin{llproof}
              \chkjudgPf{\Gamma}{e}{B}{\Delta}  {Subderivation}
              \substextendPf{\Delta}{\Omega}   {Given}
              \declchkjudgPf{[\Omega]\Delta}{e}{[\Omega]B}   {By i.h.}
              \declappjudgPf{[\Omega]\Delta}{e}{([\Omega]B) \arr ([\Omega]C)}{[\Omega]C}   {By \DeclArrApp}
\Hand              \declappjudgPf{[\Omega]\Delta}{e}{[\Omega](B \arr C)}{[\Omega]C}   {By definition of substitution}
            \end{llproof}




      \DerivationProofCase{\SolveApp}
            {\chkjudg{\Gamma_0[\ahat_2, \ahat_1, \hypeq{\ahat}{\ahat_1 \arr \ahat_2}]}{e}{\ahat_1}{\Delta}}
            {\appjudg{\underbrace{\Gamma_0[\ahat]}_{\Gamma}}{e}{\ahat}{\ahat_2}{\Delta}}

            \begin{llproof}
                \chkjudgPf{\overbrace{\Gamma_0[\ahat_2, \ahat_1, \hypeq{\ahat}{\ahat_1 \arr \ahat_2}]}^{\Gamma'}}{e}{\ahat_1}{\Delta}   {Subderivation}
                \substextendPf{\Delta}{\Omega}   {Given}
                \declchkjudgPf{[\Omega]\Delta}{e}{[\Omega]\ahat_1}   {By i.h.}
                \declappjudgPf{[\Omega]\Delta}{e}{([\Omega]\ahat_1) \arr ([\Omega]\ahat_2)}{[\Omega]\ahat_2}    {By \DeclArrApp}
                \decolumnizePf
                \substextendPf{\Gamma'}{\Delta}   {By \Lemmaref{lem:typing-extension}}
                \substextendPf{\Delta}{\Omega}   {Given}
                \substextendPf{\Gamma'}{\Omega}   {By \Lemmaref{lem:substextend-transitivity}}
                \eqPf{[\Gamma']\ahat}  {[\Gamma'](\ahat_1 \arr \ahat_2)}   {By definition of $[\Gamma'](-)$}
                \eqPf{[\Omega][\Gamma']\ahat}  {[\Omega][\Gamma'](\ahat_1 \arr \ahat_2)}   {Applying $\Omega$ to both sides}
                \eqPf{[\Omega]\ahat}  {[\Omega](\ahat_1 \arr \ahat_2)}   {By \Lemmaref{lem:subst-extension-invariance}, twice}
                \eqPf{~}  {([\Omega]\ahat_1) \arr ([\Omega]\ahat_2)}   {By definition of substitution}
                \proofsep
\Hand     \declappjudgPf{[\Omega]\Delta}{e}{[\Omega]\ahat}{[\Omega]\ahat_2}    {By above equality}
            \end{llproof}

\DerivationProofCase{\UnitIntroSyn}
                    { }
                    {{\synjudg{\Gamma}{\unitexp}{\unitty}{\underbrace{\Gamma}_{\Delta}}}}

                    \begin{llproof}
                      \Hand \declsynjudgPf{[\Omega]\Delta} {\unitexp} {[\Omega]\unitty}  {By \DeclUnitIntroSyn and definition of substitution}
                    \end{llproof}


               \DerivationProofCase{\ArrIntroSyn}
                    { \chkjudg{\Gamma, \ahat, \bhat, x : \ahat}{e_0}{\bhat}{\Delta, x : \ahat, \Theta}
                }
                    {{\synjudg{\Gamma}{\lam{x} e_0}{\ahat \arr \bhat}{\Delta}}}

                  \begin{llproof}
                    \substextendPf{\Gamma, \ahat, \bhat, x : \ahat}{\Delta, x : \ahat, \Theta}  {By \Lemmaref{lem:typing-extension}}
                    \Pf{}{}{\Theta\text{~is soft}}   {By \Lemmaref{lem:extension-order} (v) (with $\Gamma_R = \cdot$, which is soft)}
                    \substextendPf{\Gamma, \ahat, \bhat}{\Delta}   {\ditto}
                    \proofsep
                    \substextendPf{\Delta}{\Omega}  {Given}
                    \substextendPf{\Delta, x : \ahat}{\Omega, x : [\Omega]\ahat}  {By \substextendVV}
                    \substextendPf{\underbrace{\Delta, x : \ahat, \Theta}_{\Delta'}}{\underbrace{\Omega, x : [\Omega]\ahat, \soln{\Theta}}_{\Omega'}}  {By \Lemmaref{lem:soln-completes}}
                    \decolumnizePf
                    \chkjudgPf{\Gamma, \ahat, \bhat, x : \ahat}{e}{\bhat}{\Delta, x : \ahat, \Theta}  {Subderivation}
                    \proofsep
                    \declchkjudgPf{[\Omega']\Delta'}
                                 {e_0} {[\Omega']\bhat}  {By i.h.}
                    \eqPf{[\Omega']\bhat}{\big[\Omega, x : [\Omega]\ahat\big]\bhat}   {By \Lemmaref{lem:substitution-stability}}
                    \continueeqPf{[\Omega]\bhat}   {By definition of substitution}
                    \eqPf{[\Omega']\Delta'}{\big[\Omega, x : [\Omega]\ahat\big]\big(\Delta, x : \ahat\big)} 
                             {By \Lemmaref{lem:softness-goes-away}}
                    \continueeqPf{[\Omega]\Delta, x : [\Omega]\ahat}    {By definition of context substitution}
                    \declchkjudgPf{[\Omega]\Delta, x : [\Omega]\ahat}
                                 {e_0} {[\Omega]\bhat} {By above equalities}
                    \proofsep
                    \substextendPf{\Gamma, \ahat, \bhat}{\Delta}  {Above}
\substextendPf{\Gamma, \ahat, \bhat}{\Omega}   {By \Lemmaref{lem:substextend-transitivity}}
                    \judgetpPf{\Gamma, \ahat, \bhat} {\ahat}   {By \EvarWF}
                    \judgetpPf{\Omega} {\ahat}   {By \Lemmaref{lem:extension-weakening}}
                    \judgetpPf{[\Omega]\Delta}
                                {[\Omega]\ahat}   {By \Lemmaref{lem:completion-wf}}
                                \trailingjust{and \Lemmaref{lem:completes-stability}}
                    \proofsep
                    \judgetpPf{[\Omega]\Delta}
                                {[\Omega]\bhat}   {By similar reasoning}
                    \decolumnizePf
                    \judgetpPf{[\Omega]\Delta}
                                {([\Omega]\ahat) \arr ([\Omega]\bhat)}   {By \DeclArrowWF}
                    \Pf{}{}{\text{$[\Omega]\ahat$, $[\Omega]\bhat$ monotypes}}   {$\Omega$ predicative}
                    \proofsep
                    \declsynjudgPf{[\Omega]\Delta} {\lam{x} e_0} {([\Omega]\ahat) \arr ([\Omega]\bhat)}  {By \DeclArrIntroSyn}
        \Hand       \declsynjudgPf{[\Omega]\Delta} {\lam{x} e_0} {[\Omega](\ahat \arr \bhat)}  {By definition of substitution\qedhere}
                  \end{llproof}                    
      \end{itemize}
\end{proof}



\clearpage
\section{Completeness}

\subsection{Instantiation Completeness}

\instantiationcompletes*
\begin{proof}
  By mutual induction on the given declarative subtyping derivation.
  
  \begin{enumerate}[(1)]
  \item
    We have $\declsubjudg{[\Omega]\Gamma}{[\Omega]\ahat}{[\Omega]A}$. We now case-analyze the shape of $A$. 
    \begin{itemize}
      \ProofCaseRule{$A = \bhat$}

          It is given that $\ahat \notin \FV{\bhat}$, so $\ahat \neq \bhat$. \\
          Since $A = \bhat$, we have $\declsubjudg{[\Omega]\Gamma}{[\Omega]\ahat}{[\Omega]\bhat}$. \\ 
          Since $\Omega$ is predicative, $[\Omega]\ahat = \tau_1$ and $[\Omega]\bhat = \tau_2$,
          so we have $\declsubjudg{[\Omega]\Gamma}{\tau_1}{\tau_2}$. \\
          By \Lemmaref{lem:decl-monotype-equality},
             $\tau_1 = \tau_2$. \\ 
          We have $A = \bhat$ and $[\Gamma]A = A$, so $[\Gamma]\bhat = \bhat$.  Thus $\bhat \in \unsolved{\Gamma}$. \\ 
          Let $\Omega'$ be $\Omega$.  By \Lemmaref{lem:substextend-reflexivity}, $\substextend{\Omega}{\Omega}$. \\
          Now consider whether $\ahat$ is declared to the left of $\bhat$, or vice versa. 

          \begin{itemize}
            \ProofCaseRule{$\Gamma = (\Gamma_0, \ahat, \Gamma_1, \bhat, \Gamma_2)$}

                Let $\Delta$ be $\Gamma_0, \ahat, \Gamma_1, \hypeq{\bhat}{\ahat}, \Gamma_2$. \\ 
                By rule \InstLReach, $\instjudg{\Gamma}{\ahat}{\bhat}{\Delta}$. \\
                It remains to show that $\substextend{\Delta}{\Omega}$. \\ 
                We have $[\Omega]\ahat = [\Omega]\bhat$.
                Then by \Lemmaref{lem:parallel-extension-solve}, $\substextend{\Delta}{\Omega}$. 

            \ProofCaseRule{$(\Gamma = \Gamma_0, \bhat, \Gamma_1, \ahat, \Gamma_2)$}

                Let $\Delta$ be $\Gamma_0, \bhat, \Gamma_1, \hypeq{\ahat}{\bhat}, \Gamma_2$. \\ 
                By rule \InstLSolve, $\instjudg{\Gamma}{\ahat}{\bhat}{\Delta}$. \\
                It remains to show that $\substextend{\Delta}{\Omega}$. \\ 
                We have $[\Omega]\bhat = [\Omega]\ahat$.
                Then by \Lemmaref{lem:parallel-extension-solve}, $\substextend{\Delta}{\Omega}$. 
          \end{itemize}

      \ProofCaseRule{$A = \alpha$}

          Since $A = \alpha$, we have $\declsubjudg{[\Omega]\Gamma}{[\Omega]\ahat}{[\Omega]\alpha}$. \\
          Since $[\Omega]\alpha = \alpha$, we have $\declsubjudg{[\Omega]\Gamma}{[\Omega]\ahat}{\alpha}$. \\ 
          By inversion, \DsubVar was used, so $[\Omega]\ahat = \alpha$; therefore,
          since $\Omega$ is well-formed, $\alpha$ is declared to the left of $\ahat$ in $\Omega$. \\ 
          We have $\substextend{\Gamma}{\Omega}$. \\
          By \Lemmaref{lem:reverse-declaration-order-preservation}, we know that $\alpha$ is declared to the left of $\ahat$ in $\Gamma$; that is, $\Gamma = \Gamma_0[\alpha][\ahat]$. \\
          Let $\Delta = \Gamma_0[\alpha][\hypeq{\ahat}{\alpha}]$ and $\Omega' = \Omega$. \\
          By \InstLSolve, $\instjudg{\Gamma_0[\alpha][\ahat]}{\ahat}{\alpha}{\Delta}$. \\
          By \Lemmaref{lem:parallel-extension-solve}, $\substextend{\Gamma_0[\alpha][\ahat=\alpha]}{\Omega}$. 


      \ProofCaseRule{$A = A_1 \arr A_2$}

          By the definition of substitution, $[\Omega]A = ([\Omega]A_1) \arr ([\Omega]A_2)$. \\ 
          Therefore $\declsubjudg{[\Omega]\Gamma}{[\Omega]\ahat}{([\Omega]A_1) \arr ([\Omega]A_2)}$. \\ 
          Since we have an arrow as the supertype, only \DsubAllL or \DsubArr could have been used,
          and the subtype $[\Omega]\ahat$ must be either a quantifier or an arrow.
          But $\Omega$ is predicative, so $[\Omega]\ahat$ cannot be a quantifier.
          Therefore, it is an arrow: $[\Omega]\ahat = \tau_1 \arr \tau_2$, and \DsubArr concluded the derivation.
          Inverting \DsubArr gives
          $\declsubjudg{[\Omega]\Gamma}{[\Omega]A_2}{\tau_2}$ and 
          $\declsubjudg{[\Omega]\Gamma}{\tau_1}{[\Omega]A_1}$. \\

          Since $\ahat \in \unsolved{\Gamma}$, we know that $\Gamma$ has the form $\Gamma_0[\ahat]$. \\
          By \Lemmaref{lem:extension-add} twice, inserting unsolved variables $\ahat_2$ and $\ahat_1$ into
          the middle of the context extends it, that is: $\substextend{\Gamma_0[\ahat]}{\Gamma_0[\ahat_2, \ahat_1, \ahat]}$. \\
          Clearly, $\ahat_1 \arr \ahat_2$ is well-formed in $(\dots, \ahat_2, \ahat_1)$, so by 
          \Lemmaref{lem:extension-solve}, solving $\ahat$ extends the context:
          $\substextend{\Gamma_0[\ahat_2, \ahat_1, \ahat]}{\Gamma_0[\ahat_2, \ahat_1, \hypeq{\ahat}{\ahat_1 \arr \ahat_2}]}$.
          Then by \Lemmaref{lem:substextend-transitivity}, $\substextend{\Gamma_0[\ahat]}{\Gamma_0[\ahat_2, \ahat_1, \hypeq{\ahat}{\ahat_1 \arr \ahat_2}]}$.

          Since $\ahat \in \unsolved{\Gamma}$ and $\substextend{\Gamma}{\Omega}$, we know that
          $\Omega$ has the form $\Omega_0[\hypeq{\ahat}{\tau_0}]$.
          To show that we can extend this context,
          we apply \Lemmaref{lem:extension-addsolve} twice to introduce 
          $\hypeq{\ahat_2}{\tau_2}$ and  $\hypeq{\ahat_1}{\tau_1}$,
          and then \Lemmaref{lem:extension-solve} to overwrite $\tau_0$:
          \[
              \substextend{\underbrace{\Omega_0[\hypeq{\ahat}{\tau_0}]}_{\Omega}}
                          {\Omega_0[\hypeq{\ahat_2}{\tau_2}, \hypeq{\ahat_1}{\tau_1}, \hypeq{\ahat}{\ahat_1 \arr \ahat_2}]}
          \]
          We have $\substextend{\Gamma}{\Omega}$, that is,
          \[
                  \substextend{\Gamma_0[\ahat]}{\Omega_0[\hypeq{\ahat}{\tau_0}]}
          \]
          By \Lemmaref{lem:parallel-admissibility} (i) twice, inserting unsolved variables $\ahat_2$ and $\ahat_1$
          on both contexts in the above extension preserves extension:

          {\small
              \begin{llproof}
                      \substextendPf{\Gamma_0[\ahat_2, \ahat_1, \ahat]}
                                  {\Omega_0[\hypeq{\ahat_2}{\tau_2}, \hypeq{\ahat_1}{\tau_1}, \hypeq{\ahat}{\tau_0}]}
                                  {By \Lemmaref{lem:parallel-admissibility} (ii) twice}
                      \substextendPf{\underbrace{\Gamma_0[\ahat_2, \ahat_1, \hypeq{\ahat}{\ahat_1{\arr}\ahat_2}]}_{\Gamma_1}}
                                  {\underbrace{\Omega_0[\hypeq{\ahat_2}{\tau_2}, \hypeq{\ahat_1}{\tau_1}, \hypeq{\ahat}{\ahat_1 {\arr} \ahat_2}]}_{\Omega_1}}
                                  {By \Lemmaref{lem:parallel-extension-update}}
              \end{llproof}
          }

          Since $\ahat \notin \FV{A}$, it follows that $[\Gamma_1]A = [\Gamma]A = A$. \\ 
          Therefore $\ahat_1 \notin \FV{A_1}$ and $\ahat_1, \ahat_2 \notin \FV{A_2}$. \\ 
          By \Lemmaref{lem:finishing-completions} and \Lemmaref{lem:finishing-types}, $[\Omega_1]\Gamma_1 = [\Omega]\Gamma$ and $[\Omega_1]\ahat_1 = \tau_1$.  \\ 
          By i.h., there are $\Delta_2$ and $\Omega_2$ such that 
          $\instjudgr{\Gamma_1}{\ahat_1}{A_1}{\Delta_2}$ and $\substextend{\Delta_2}{\Omega_2}$ and $\substextend{\Omega_1}{\Omega_2}$. 

          Next, note that $[\Delta_2][\Delta_2]A_2 = [\Delta_2]A_2$. \\ 
          By \Lemmaref{lem:left-unsolvedness-preservation}, we know that $\ahat_2 \in \unsolved{\Delta_2}$. \\ 
          By \Lemmaref{lem:left-free-variable-preservation}, we know that $\ahat_2 \notin \FV{[\Delta_2]A_2}$. \\ 
          By \Lemmaref{lem:substextend-transitivity}, $\substextend{\Omega}{\Omega_2}$. \\
          We know $[\Omega_2]\Delta_2 = [\Omega]\Gamma$ because: 
          \begin{mathpar}
            \begin{array}{lcll}
              [\Omega_2]\Delta_2 
              & = & [\Omega_2]\Omega_2 & \text{By \Lemmaref{lem:completes-stability}} \\ 
              & = & [\Omega]\Omega & \text{By \Lemmaref{lem:finishing-completions}} \\
              & = & [\Omega]\Gamma & \text{By \Lemmaref{lem:completes-stability}}
            \end{array}
          \end{mathpar}
          By \Lemmaref{lem:finishing-types}, we know that $[\Omega_2]\ahat_2 = [\Omega_1]\ahat_2 = \tau_2$. \\ 
          By \Lemmaref{lem:finishing-types}, we know that $[\Omega_2]A_2 = [\Omega]A_2$. \\ 
          Hence we know that $\declsubjudg{[\Omega_2]\Delta_2}{[\Omega_2]\ahat_2}{[\Omega_2]A_2}$. \\ 
          By i.h., we have $\Delta$ and $\Omega'$ such that 
          $\instjudg{\Delta_2}{\ahat_2}{[\Delta_2]A_2}{\Delta}$ and $\substextend{\Omega_2}{\Omega'}$ and $\substextend{\Delta}{\Omega'}$.  \\
          By rule \InstLArr, $\instjudg{\Gamma}{\ahat}{A}{\Delta}$. \\ 
          By \Lemmaref{lem:substextend-transitivity}, $\substextend{\Omega}{\Omega'}$. 

      \ProofCaseRule{$A = \unitty$}

          We have $A = \unitty$, so $\declsubjudg{[\Omega]\Gamma}{[\Omega]\ahat}{[\Omega]\unitty}$. \\ 
          Since $[\Omega]\unitty = \unitty$, we have $\declsubjudg{[\Omega]\Gamma}{[\Omega]\ahat}{\unitty}$. \\ 
          The only declarative subtyping rules that can have $\unitty$ as the supertype in the conclusion are
            \DsubAllL and \DsubUnit.
          However, since $\Omega$ is predicative, $[\Omega]\ahat$ cannot be a quantifier, so \DsubAllL cannot
          have been used.  Hence \DsubUnit was used and $[\Omega]\ahat = \unitty$. \\ 
          Let $\Delta = \Gamma[\hypeq{\ahat}{\unitty}]$ and $\Omega' = \Omega$. \\
          By \InstLSolve, $\instjudg{\Gamma[\ahat]}{\ahat}{\unitty}{\Delta}$. \\
          By \Lemmaref{lem:parallel-extension-solve}, $\substextend{\Gamma[\ahat=\unitty]}{\Omega}$.

      \ProofCaseRule{$A = \alltype{\beta}{B}$}

          We have $\declsubjudg{[\Omega]\Gamma}{[\Omega]\ahat}{[\Omega](\alltype{\beta}{B})}$. \\
          By definition of substitution, $[\Omega](\alltype{\beta}{B}) = \alltype{\beta}{[\Omega]B}$, so
          we have $\declsubjudg{[\Omega]\Gamma}{[\Omega]\ahat}{\alltype{\beta}{[\Omega]B}}$.
          The only declarative subtyping rules that can have a quantifier as supertype are
             \DsubAllL and \DsubAllR.  However, since $\Omega$ is predicative, $[\Omega]\ahat$ cannot be
             a quantifier, so \DsubAllL cannot have been used.  Hence \DsubAllR was used,
             and we have a subderivation of
             $\declsubjudg{[\Omega]\Gamma, \beta}{[\Omega]\ahat}{[\Omega]B}$. \\
          Let $\Omega_1 = (\Omega, \beta)$ and $\Gamma_1 = (\Gamma, \beta)$. \\
          By \substextendUU, $\substextend{\Gamma_1}{\Omega_1}$. \\
          By the definition of substitution, $[\Omega_1]B = [\Omega]B$
          and $[\Omega_1]\ahat = [\Omega]\ahat$. \\
          Note that $[\Omega_1]\Gamma_1 = [\Omega]\Gamma, \beta$. \\
          Since $\ahat \in \unsolved{\Gamma}$, we have $\ahat \in \unsolved{\Gamma_1}$. \\
          Since $\ahat \notin \FV{A}$ and $A = \alltype{\beta}{B}$, we have $\ahat \notin \FV{B}$. \\
          By i.h., there are $\Omega_2$ and $\Delta_2$ such that
          $\instjudg{\Gamma, \beta}{\ahat}{B}{\Delta_2}$ and 
          $\substextend{\Delta_2}{\Omega_2}$ and $\substextend{\Omega_1}{\Omega_2}$. \\
          By \Lemmaref{lem:instantiation-extension}, $\substextend{\Gamma_1}{\Delta_2}$,
          that is, $\substextend{\Gamma, \beta}{\Delta_2}$. \\
          Therefore by \Lemmaref{lem:extension-order},
              $\Delta_2 = (\Delta', \beta, \Omega'')$ where $\substextend{\Gamma}{\Delta'}$. \\
          By equality, we know $\substextend{\Delta', \beta, \Delta''}{\Omega_2}$. \\
          By \Lemmaref{lem:extension-order},
             $\Omega_2 = (\Omega', \beta, \Omega'')$ where \Hand $\substextend{\Delta'}{\Omega'}$. \\
          We have $\substextend{\Omega_1}{\Omega_2}$, that is,
          $\substextend{\Omega, \beta}{\Omega', \beta, \Omega''}$,
          so \Lemmaref{lem:extension-order} gives \Hand $\substextend{\Omega}{\Omega'}$. \\
          By rule \InstLAllR, $\instjudg{\Gamma}{\ahat}{\alltype{\beta}{B}}{\Delta'}$. 
    \end{itemize}
        
  \item $\declsubjudg{[\Omega]\Gamma}{[\Omega]A}{[\Omega]\ahat}$

    These cases are mostly symmetric.  The one exception is the one connective that is not treated symmetrically
    in the declarative subtyping rules:
    \begin{itemize}
      \ProofCaseRule{$A = \alltype{\alpha}{B}$}

      Since $A = \alltype{\alpha}{B}$, we have
      $\declsubjudg{[\Omega]\Gamma}{[\Omega]\alltype{\beta}{B}}{[\Omega]\ahat}$. \\
      By symmetric reasoning to the previous case (the last case of part (1) above), \DsubAllL must have been
      used, with a subderivation of
      $\declsubjudg{[\Omega]\Gamma}{[\Omega]\ahat}{[\tau/\beta][\Omega]B}$. \\
      Since $\judgetp{[\Omega]\Gamma}{\tau}$, the type $\tau$ has no existential variables and is
      therefore invariant under substitution: $\tau = [\Omega]\tau$.
      Therefore $\big[\tau/\beta\big]\big[\Omega\big]B = \big[[\Omega]\tau/\beta\big]\big[\Omega\big]B$. \\
      By distributivity of substitution, this is $\big[\Omega\big][\tau/\beta]B$.
      Interposing $\bhat$, this is equal to $[\Omega][\tau/\bhat][\bhat/\beta]B$.
      Therefore
      $\declsubjudg{[\Omega]\Gamma}{[\Omega]\ahat}{[\Omega][\tau/\bhat][\bhat/\beta]B}$. \\
      Let $\Omega_1$ be $\Omega, \MonnierComma{\bhat}, \hypeq{\bhat}{\tau}$
      and let $\Gamma_1$ be $\Gamma, \MonnierComma{\bhat}, \bhat$.

      \begin{itemize}
          \item 
            By the definition of context application, $[\Omega_1]\Gamma_1 = [\Omega]\Gamma$.

          \item
            From the definition of substitution, $[\Omega_1]\ahat = [\Omega]\ahat$.

          \item
            It follows from the definition of substitution that $[\Omega][\tau/\bhat]C = [\Omega_1]C$ for all $C$.
            Therefore $[\Omega][\tau/\bhat][\bhat/\beta]B = [\Omega_1][\bhat/\beta]B$.
      \end{itemize}

      Applying these three equalities, $\declsubjudg{[\Omega_1]\Gamma_1}{[\Omega_1]\ahat}{[\Omega_1][\bhat/\beta]B}$. \\
      By the definition of substitution, $[\Gamma, \MonnierComma{\bhat}, \bhat]B = [\Gamma]B = B$, so $\ahat \notin \FV{[\Gamma_1]B}$. \\
      Since $\ahat \in \unsolved{\Gamma}$, we have $\ahat \in \unsolved{\Gamma_1}$. \\

      By i.h., there exist $\Delta_2$ and $\Omega_2$ such that
        $\instjudgr{\Gamma_1}{\ahat}{B}{\Delta_2}$
        and $\substextend{\Omega_1}{\Omega_2}$
        and $\substextend{\Delta_2}{\Omega_2}$. \\      
      By \Lemmaref{lem:instantiation-extension}, $\substextend{\Gamma_1}{\Delta_2}$,
        which is, $\substextend{\Gamma, \MonnierComma{\bhat}, \bhat}{\Delta_2}$. \\
      By \Lemmaref{lem:extension-order}, $\Delta_2 = (\Delta', \MonnierComma{\bhat}, \Delta'')$ and $\substextend{\Gamma}{\Delta'}$. \\
      By equality, $\substextend{\Delta', \MonnierComma{\bhat}, \Delta''}{\Omega_2}$. \\
      By \Lemmaref{lem:extension-order}, $\Omega_2 = (\Omega', \MonnierComma{\bhat}, \Omega'')$ and \Hand $\substextend{\Delta'}{\Omega'}$. \\
      By equality, $\substextend{\Omega, \MonnierComma{\bhat}, \hypeq{\bhat}{\tau}}{\Omega', \MonnierComma{\bhat}, \Omega''}$. \\
      \Hand By \Lemmaref{lem:extension-order}, $\substextend{\Omega}{\Omega'}$. \\
      By \InstRAllL, $\instjudgr{\Gamma}{\ahat}{\alltype{\beta}{B}}{\Delta'}$. 
    \qedhere
    \end{itemize}
  \end{enumerate}
\end{proof}



\clearpage

\subsection{Completeness of Subtyping}

\completingcompleteness*
\begin{proof}
  By induction on the derivation of $\declsubjudg{[\Omega]\Gamma}{[\Omega]A}{[\Omega]B}$. 
  
  \newcommand{\CASEhandled}[1]{\textcolor{dDkGreen}{\textbf{#1}}}
  \newcommand{\CASEhandledsymm}[1]{\textcolor{dBlue}{\textrm{#1}}}
  \newcommand{\CASEtobeproved}[1]{\textcolor{red}{\textbf{#1}}}
  \newcommand{\CASEimpossible}{\textcolor{dRed}{\textsl{impossible}}}
  \newcommand{\MAINLABEL}[1]{\mathsz{14pt}{#1}}
  \newcommand{\CaseBpoly}{\CASEhandled{1 (B poly)}}

  We distinguish cases of $[\Gamma]B$ and $[\Gamma]A$ that are
  \CASEimpossible, \CASEhandled{fully written out}, and
  \CASEhandledsymm{similar to fully-written-out cases}.

  \begin{displ}
    
    \begin{array}[t]{rccccccc}
      &\multicolumn{7}{c}{\MAINLABEL{[\Gamma]B}}
      \\[5pt]
      &
        & {\alltype{\beta} B'} &   \unitty  &   \alpha   &   \bhat   &    B_1 \arr B_2 &
      \\\cmidrule[1pt]{3-7}
      &\alltype{\alpha} A'
                &   \CaseBpoly
                &   \CASEhandled{2.Poly}
                &   \CASEhandled{2.Poly}
                &   \CASEhandled{2.Poly}
                &   \CASEhandled{2.Poly}
      \\\cmidrule{3-7}
      &\unitty
                &   \CaseBpoly
                &   \CASEhandled{2.Units}
                &   \CASEimpossible
                &   \CASEhandledsymm{2.BEx.Unit}
                &   \CASEimpossible
      \\\cmidrule{3-7}
      \MAINLABEL{[\Gamma]A}
      & \alpha
                &   \CaseBpoly
                &   \CASEimpossible
                &   \CASEhandled{2.Uvars}
                &   \CASEhandledsymm{2.BEx.Uvar}
                &   \CASEimpossible
      \\\cmidrule{3-7}
      & \ahat
                &   \CaseBpoly
                &   \CASEhandled{2.AEx.Unit}
                &   \CASEhandledsymm{2.AEx.Uvar}
                &
                    \begin{tabular}[c]{c}
                      \CASEhandled{2.AEx.SameEx}
                      \\
                      \CASEhandled{2.AEx.OtherEx}
                    \end{tabular}
&   \CASEhandled{2.AEx.Arrow}
      \\\cmidrule{3-7}
      & A_1 \arr A_2
                &   \CaseBpoly
                &   \CASEimpossible
                &   \CASEimpossible
                &   \CASEhandledsymm{2.BEx.Arrow}
                &   \CASEhandled{2.Arrows}
    \end{array}
  \end{displ}

  The impossibility of the ``\CASEimpossible'' entries follows from inspection of the
  declarative subtyping rules.

  \medskip

  We first split on $[\Gamma]B$.
  
  \begin{itemize}
  \item \textbf{Case \CaseBpoly: $[\Gamma]B$ polymorphic:}  $[\Gamma]B = \alltype{\beta} B'$:
    
    \begin{llproof}
      \eqPf{B} {\alltype{\beta} B_0}   {$\Gamma$ predicative}
      \eqPf{B'} {[\Gamma]B_0}   {$\Gamma$ predicative}
      \eqPf{[\Omega]B} {[\Omega](\alltype{\beta} B_0)}   {Applying $\Omega$ to both sides}
      \continueeqPf {\alltype{\beta} [\Omega]B_0}   {By definition of substitution}
      $\Dee \derives~$ \declsubjudgPf{[\Omega]\Gamma} {[\Omega]A} {[\Omega]B}   {Given}
      $\Dee \derives~$ \declsubjudgPf{[\Omega]\Gamma} {[\Omega]A} {\alltype{\beta} [\Omega]B_0}   {By above equality}
      $\Dee' \derives~$ \declsubjudgPf{[\Omega]\Gamma, \beta}
                    {[\Omega]A}
                    {[\Omega]B_0}
            {By \Lemmaref{lem:decl-invertibility}}
      \ltPf{\Dee'} {\Dee}   {\ditto}
      $\Dee' \derives~$ \declsubjudgPf{[\Omega, \beta](\Gamma, \beta)}
                    {[\Omega, \beta]A}
                    {[\Omega, \beta]B_0}
            {By definitions of substitution}
      \subjudgPf{\Gamma, \beta}{[\Gamma, \beta]A}{[\Gamma, \beta]B_0} {\Delta'}  {By i.h.}
      \substextendPf{\Delta'}{\Omega_0'}   {\ditto}
      \substextendPf{\Omega, \beta}{\Omega_0'}   {\ditto}
      \subjudgPf{\Gamma, \beta}{[\Gamma]A}{[\Gamma]B_0} {\Delta'}  {By definition of substitution}
      \proofsep
      \substextendPf{\Gamma, \beta} {\Delta'}  {By \Lemmaref{lem:instantiation-extension}}
      \eqPf{\Delta'}{\Delta, \beta, \Theta}  {By \Lemmaref{lem:extension-order} (i)}
      \substextendPf{\Gamma}{\Delta}  {\ditto}
      \substextendPf{\Delta, \beta, \Theta}{\Omega_0'}   {By $\substextend{\Delta'}{\Omega_0'}$ and above equality}
      \eqPf{\Omega_0'}{\Omega', \beta, \Omega_R}  {By \Lemmaref{lem:extension-order} (i)}
\Hand      \substextendPf{\Delta}{\Omega'}  {\ditto}
      \proofsep
      \subjudgPf{\Gamma, \beta}{[\Gamma]A}{[\Gamma]B_0} {\Delta, \beta, \Theta}  {By above equality}
\substextendPf{\Omega, \beta}{\Omega', \beta, \Omega_R}   {By above equality}
\Hand      \substextendPf{\Omega}{\Omega'}   {By \Lemmaref{lem:substextend-transitivity}}
      \proofsep
      \subjudgPf{\Gamma}{[\Gamma]A}{\alltype{\beta}{[\Gamma]B_0}}{\Delta}  {By \SubAllR}
\Hand      \subjudgPf{\Gamma}{[\Gamma]A}{\alltype{\beta} B'}{\Delta}  {By above equality}
    \end{llproof}


\clearpage
  \item \textbf{Cases 2.*: $[\Gamma]B$ not polymorphic:}

    We split on the form of $[\Gamma]A$.

    \begin{itemize}
    \item \textbf{Case \CASEhandled{2.Poly}: $[\Gamma]A$ is polymorphic:} $[\Gamma]A = \alltype{\alpha} A'$:

      \medskip
      
      \begin{llproof}
        \eqPf{A} {\alltype{\alpha} A_0}   {$\Gamma$ predicative}
        \eqPf{A'} {[\Gamma]A_0}   {$\Gamma$ predicative}
        \eqPf{[\Omega]A} {[\Omega](\alltype{\alpha} A_0)}   {Applying $\Omega$ to both sides}
        \eqPf{[\Omega]A} {\alltype{\alpha} [\Omega]A_0}   {By definition of substitution}
        \declsubjudgPf{[\Omega]\Gamma} {[\Omega]A} {[\Omega]B}   {Given}
        \declsubjudgPf{[\Omega]\Gamma} {\alltype{\alpha} [\Omega]A_0} {[\Omega]B}   {By above equality}
        \neqPf{[\Gamma]B} {(\alltype{\beta}{\cdots})}   {We are in the ``$[\Gamma]B$ not polymorphic'' subcase}
        \neqPf{B} {(\alltype{\beta}{\ldots})}   {$\Gamma$ predicative}
        \declsubjudgPf{[\Omega]\Gamma} {[\tau/\alpha][\Omega]A_0} {[\Omega]B}   {By inversion on \DsubAllL}
        \judgetpPf{[\Omega]\Gamma} {\tau}   {\ditto}
\proofsep
        \substextendPf{\Gamma}{\Omega}   {Given}
        \substextendPf{\Gamma, \MonnierComma{\ahat}}{\Omega, \MonnierComma{\ahat}}
                   {By \substextendMonMon}
        \substextendPf{\Gamma, \MonnierComma{\ahat}, \ahat}{\underbrace{\Omega, \MonnierComma{\ahat}, \hypeq{\ahat}{\tau}}_{\Omega_0}}
                   {By \substextendSolve}
        \proofsep
        \eqPf{[\Omega]\Gamma} {[\Omega_0](\Gamma, \MonnierComma{\ahat}, \ahat)}  {By definition of context application (lines 16, 13)}
        \decolumnizePf
        \proofsep
        \declsubjudgPf{[\Omega]\Gamma}  {[\tau/\alpha][\Omega]A_0}  {[\Omega]B}    {Above}
        \declsubjudgPf{[\Omega_0](\Gamma, \MonnierComma{\ahat}, \ahat)}
                      {[\tau/\alpha][\Omega]A_0}
                      {[\Omega]B}
                      {By above equality}
        \declsubjudgPf{[\Omega_0](\Gamma, \MonnierComma{\ahat}, \ahat)}
                      {\big[[\Omega_0]\ahat / \alpha\big][\Omega]A_0}
                      {[\Omega]B}
                      {By definition of substitution}
        \declsubjudgPf{[\Omega_0](\Gamma, \MonnierComma{\ahat}, \ahat)}
                      {\big[[\Omega_0]\ahat / \alpha\big][\Omega_0]A_0}
                      {[\Omega_0]B}
                      {By definition of substitution}
        \declsubjudgPf{[\Omega_0](\Gamma, \MonnierComma{\ahat}, \ahat)}
                      {[\Omega_0][\ahat / \alpha]A_0}
                      {[\Omega_0]B}
                      {By distributivity of substitution}
\proofsep
        \subjudgPf{\Gamma, \MonnierComma{\ahat}, \ahat} {[\Gamma, \MonnierComma{\ahat}, \ahat][\ahat / \alpha]A_0} {[\Gamma, \MonnierComma{\ahat}, \ahat]B} {\Delta_0}   {By i.h.}
        \substextendPf{\Delta_0} {\Omega''}    {\ditto}
        \substextendPf{\Omega_0} {\Omega''}    {\ditto}
        \subjudgPf{\Gamma, \MonnierComma{\ahat}, \ahat} {[\Gamma][\ahat / \alpha]A_0} {[\Gamma]B} {\Delta_0}   {By definition of substitution}
        \substextendPf{\Gamma, \MonnierComma{\ahat}, \ahat} {\Delta_0}   {By \Lemmaref{lem:subtyping-extension}}
        \eqPf{\Delta_0}{(\Delta, \MonnierComma{\ahat}, \Theta)}  {By \Lemmaref{lem:extension-order} (ii)}
        \substextendPf{\Gamma}{\Delta}   {\ditto}
        \eqPf{\Omega''}{(\Omega', \MonnierComma{\ahat}, \Omega_Z)}  {By \Lemmaref{lem:extension-order} (ii)}
\Hand        \substextendPf{\Delta}{\Omega'}   {\ditto}
        \substextendPf{\Omega_0} {\Omega''}    {Above}
        \substextendPf{\Omega, \MonnierComma{\ahat}, \hypeq{\ahat}{\tau}} {\Omega', \MonnierComma{\ahat}, \Omega_Z}    {By above equalities}
\Hand        \substextendPf{\Omega} {\Omega'}    {By \Lemmaref{lem:extension-order} (ii)}
        \proofsep
        \subjudgPf{\Gamma, \MonnierComma{\ahat}, \ahat}
                  {[\Gamma][\ahat/\alpha]A_0}
                  {[\Gamma]B}
                  {\Delta, \MonnierComma{\ahat}, \Theta}
                  {By above equality $\Delta_0 = (\Delta, \MonnierComma{\ahat}, \Theta)$}
\subjudgPf{\Gamma, \MonnierComma{\ahat}, \ahat}
                  {[\ahat/\alpha][\Gamma]A_0}
                  {[\Gamma]B}
                  {\Delta, \MonnierComma{\ahat}, \Theta}
                  {By def.\ of subst.\ ($[\Gamma]\ahat = \ahat$ and $[\Gamma]\alpha = \alpha$)}
\subjudgPf{\Gamma} {\alltype{\alpha}{[\Gamma]A_0}} {[\Gamma]B} {\Delta}   {By \SubAllL}
     \Hand      \subjudgPf{\Gamma} {\alltype{\alpha}{A'}} {[\Gamma]B} {\Delta}   {By above equality}
      \end{llproof}

      \smallskip

    \item \textbf{Case \CASEhandled{2.AEx}: $A$ is an existential variable} $[\Gamma]A = \ahat$:

      We split on the form of $[\Gamma]B$.
      \begin{itemize}
      \item \textbf{Case \CASEhandled{2.AEx.SameEx}: $[\Gamma]B$ is the same existential variable}  $[\Gamma]B = \ahat$:

        \smallskip

        \begin{llproof}
            \subjudgPf{\Gamma} {\ahat} {\ahat} {\Gamma}    {By \SubExvar}
\Hand         \subjudgPf{\Gamma} {[\Gamma]A} {[\Gamma]B} {\Gamma}    {By $[\Gamma]A = [\Gamma]B = \ahat$}
\Hand          \substextendPf{\Delta} {\Omega}   {$\Delta = \Gamma$}
\Hand          \substextendPf{\Omega} {\Omega'}   {By \Lemmaref{lem:substextend-reflexivity} and $\Omega' = \Omega$}
        \end{llproof}


      \item \textbf{Case \CASEhandled{2.AEx.OtherEx}: $[\Gamma]B$ is a different existential variable}  $[\Gamma]B = \bhat$ where $\bhat \neq \ahat$:

        Either $\ahat \in \FV{[\Gamma]\bhat}$, or $\ahat \notin \FV{[\Gamma]\bhat}$.

        \begin{itemize}
            \item $\ahat \in \FV{[\Gamma]\bhat}$: \\
                  We have $\ahat \subtermof [\Gamma]\bhat$.  \\
                  Therefore $\ahat = [\Gamma]\bhat$, or $\ahat \propersubtermof [\Gamma]\bhat$. \\
                  But we are in Case 2.AEx.\textbf{Other}Ex, so the former is impossible. \\
                  Therefore, $\ahat \propersubtermof [\Gamma]\bhat$. \\
                  Since $\Gamma$ is predicative, $[\Gamma]\bhat$ cannot have the form $\alltype{\beta} \cdots$,
                  so the only way that $\ahat$ can be a proper subterm of $[\Gamma]\bhat$ is if $[\Gamma]\bhat$
                  has the form $B_1 \arr B_2$ such that $\ahat$ is a subterm of $B_1$ or $B_2$, that is:
                  $\ahat \occursinsidearrow [\Gamma]\bhat$. \\
                  Then by a property of substitution, $[\Omega]\ahat \occursinsidearrow [\Omega][\Gamma]\bhat$. \\
                  By \Lemmaref{lem:subst-extension-invariance}, $[\Omega][\Gamma]\bhat = [\Omega]\bhat$, so
                   $[\Omega]\ahat \occursinsidearrow [\Omega]\bhat$. \\
                  We have $\declsubjudg{[\Omega]\Gamma}{[\Omega]\ahat}{[\Omega]\bhat}$,
                  and we know that $[\Omega]\ahat$ is a monotype, so we can use \Lemmaref{lem:occurrence} (ii)
                  to show that $[\Omega]\ahat \notoccursinsidearrow [\Omega]\bhat$, a contradiction.

            \item $\ahat \notin \FV{[\Gamma]\bhat}$:

                  \begin{llproof}
                    \instjudgPf{\Gamma} {\ahat} {[\Gamma]\bhat} {\Delta}   {By \Theoremref{thm:instantiation-completes} (1)}
    \Hand                \subjudgPf{\Gamma} {\ahat} {\bhat} {\Delta}    {By \SubInstL}
          \Hand     \substextendPf{\Delta}{\Omega'} {\ditto}
          \Hand     \substextendPf{\Omega}{\Omega'} {\ditto}
                  \end{llproof}
      \end{itemize}


      \item \textbf{Case \CASEhandled{2.AEx.Unit}:}  $[\Gamma]B = \unitty$:
        
        \begin{llproof}
          \substextendPf{\Gamma}{\Omega}   {Given}
          \eqPf{\unitty}{[\Omega]\unitty}   {By definition of substitution}
          \notinPf{\ahat} {\FV{\unitty}}   {By definition of $\FV{-}$}
          \declsubjudgPf{[\Omega]\Gamma}  {[\Omega]\ahat}  {[\Omega]\unitty}  {Given}
          \proofsep
          \instjudgPf{\Gamma}  {\ahat}  {\unitty}  {\Delta}  {By \Theoremref{thm:instantiation-completes} (1)}
\Hand          \substextendPf{\Omega}{\Omega'}    {\ditto}
\Hand          \substextendPf{\Delta}{\Omega'}    {\ditto}
          \proofsep
          \eqPf{\unitty}{[\Gamma]\unitty}   {By definition of substitution}
          \notinPf{\ahat} {\FV{\unitty}}   {By definition of $\FV{-}$}
          \proofsep
\Hand          \subjudgPf{\Gamma}  {\ahat}  {\unitty}  {\Delta}   {By \SubInstL}
        \end{llproof}


      \item \textbf{Case \CASEhandledsymm{2.AEx.Uvar}:}  $[\Gamma]B = \beta$:

          Similar to Case 2.AEx.Unit, using $\beta = [\Omega]\beta = [\Gamma]\beta$ and $\ahat \notin \FV{\beta}$.


      \item \textbf{Case \CASEhandled{2.AEx.Arrow}:}  $[\Gamma]B = B_1 \arr B_2$:
        
                   Since $[\Gamma]B$ is an arrow, it cannot be exactly $\ahat$.
                   
                   Suppose, for a contradiction, that $\ahat \in \FV{[\Gamma]B}$.
                   
                       \begin{llproof}
                         \subtermofPf {\ahat} {[\Gamma]B}   {$\ahat \in \FV{[\Gamma]B}$}
                         \subtermofPf {[\Omega]\ahat} {[\Omega][\Gamma]B}   {By a property of substitution}
                         \proofsep
                         \substextendPf {\Gamma}{\Omega}   {Given}
                         \eqPf {[\Omega][\Gamma]B} {[\Omega]B}   {By \Lemmaref{lem:subst-extension-invariance}}
                         \proofsep
                         \subtermofPf {[\Omega]\ahat} {[\Omega]B}   {By above equality}
                         \proofsep
                         \neqPf {[\Gamma]B} {\ahat}      {Given (2.AEx.Arrow)}
                         \neqPf {[\Omega][\Gamma]B} {[\Omega]\ahat}      {By a property of substitution}
                         \neqPf {[\Omega]B} {[\Omega]\ahat}      {By \Lemmaref{lem:subst-extension-invariance}}
                         \proofsep
                         \propersubtermofPf {[\Omega]\ahat} {[\Omega]B}   {Follows from $\subtermofsym$ and $\neq$}
                         \occursinsidearrowPf {[\Omega]\ahat} {[\Omega]B}   {$[\Omega]A$ has the form $\cdots \arr \cdots$}
                         \declsubjudgPf{[\Omega]\Gamma} {[\Omega]\ahat} {[\Omega]B}   {Given}
                         \Pf{}{}{[\Omega]B\text{~is a monotype}}  {$\Omega$ is predicative}
                         \notoccursinsidearrowPf {[\Omega]\ahat} {[\Omega]B}   {By \Lemmaref{lem:occurrence} (ii)}
                       \contraPf{\ahat \notin \FV{[\Gamma]B}}
                      \end{llproof}

                   \begin{llproof}
                     \instjudgPf{\Gamma} {\ahat} {[\Gamma]B} {\Delta}   {By \Theoremref{thm:instantiation-completes} (1)}
\Hand             \substextendPf{\Delta} {\Omega'}   {\ditto}
\Hand             \substextendPf{\Omega} {\Omega'}   {\ditto}
\Hand             \subjudgPf{\Gamma} {\ahat} {\underbrace{[\Gamma]B}_{B_1 \arr B_2}} {\Delta}   {By \SubInstL}
                   \end{llproof}
    \end{itemize}



    \item \textbf{Case \CASEhandledsymm{2.BEx}: $[\Gamma]A$ is not polymorphic and $[\Gamma]B$ is an existential variable}:
         $[\Gamma]B = \bhat$

         We split on the form of $[\Gamma]A$.

            \begin{itemize}
            \item \textbf{Case \CASEhandledsymm{2.BEx.Unit}} ($[\Gamma]A = \unitty$), \\
                  \textbf{Case \CASEhandledsymm{2.BEx.Uvar}} ($[\Gamma]A = \alpha$), \\
                  \textbf{Case \CASEhandledsymm{2.BEx.Arrow}} ($[\Gamma]A = A_1 \arr A_2$): \\
                Similar to Cases \CASEhandled{2.AEx.Unit}, \CASEhandled{2.AEx.Uvar} and \CASEhandled{2.AEx.Arrow},
                but using part (2) of \Theoremref{thm:instantiation-completes} instead of part (1),
                and applying \SubInstR instead of \SubInstL as the final step.
            \end{itemize}

    \item \textbf{Case \CASEhandled{2.Units}}: $[\Gamma]A = [\Gamma]B = \unitty$:

      \begin{llproof}
\Hand        \subjudgPf{\Gamma}{\unitty}{\unitty}{\Gamma}   {By \SubUnit}
          \substextendPf{\Gamma} {\Omega}   {Given}
\Hand          \substextendPf{\Delta} {\Omega}   {$\Delta = \Gamma$}
\Hand          \substextendPf{\Omega} {\Omega'}   {By \Lemmaref{lem:substextend-reflexivity} and $\Omega' = \Omega$}
      \end{llproof}
      


    \item \textbf{Case \CASEhandled{2.Uvars}}: $[\Gamma]A = [\Gamma]B = \alpha$:      

      \begin{llproof}
         \Pf{}{}{\alpha \in \Omega}  {By inversion on \DsubVar}
         \substextendPf{\Gamma}{\Omega}   {Given}
         \Pf{}{}{\alpha \in \Gamma} {By \Lemmaref{lem:extension-order}}
\Hand         \subjudgPf{\Gamma}{\alpha}{\alpha}{\Gamma}  {By \SubVar}
\Hand         \substextendPf{\Delta} {\Omega}   {$\Delta = \Gamma$}
\Hand         \substextendPf{\Omega} {\Omega'}   {By \Lemmaref{lem:substextend-reflexivity} and $\Omega' = \Omega$}        
      \end{llproof}
      


    \item \textbf{Case \CASEhandled{2.Arrows}}: $A = A_1 \arr A_2$ and $B = B_1 \arr B_2$:

      Only rule \DsubArr could have been used.

      \begin{llproof}
        \declsubjudgPf{[\Omega]\Gamma}  {[\Omega]B_1}  {[\Omega]A_1}    {Subderivation}
        \subjudgPf{\Gamma}  {[\Gamma]B_1}  {[\Gamma]A_1}  {\Theta}    {By i.h.}
        \substextendPf{\Theta} {\Omega_0}    {\ditto}
        \substextendPf{\Omega} {\Omega_0}    {\ditto}
        \substextendPf{\Gamma}{\Omega}   {Given}
        \substextendPf{\Gamma}{\Omega_0}   {By \Lemmaref{lem:substextend-transitivity}}
        \proofsep
        \substextendPf{\Theta}{\Omega_0}   {Above}
        \proofsep
        \eqPf{[\Omega]\Gamma}{[\Omega]\Theta}  {By \Lemmaref{lem:completes-confluence}}
        \proofsep
        \declsubjudgPf{[\Omega]\Gamma}  {[\Omega]A_2}  {[\Omega]B_2}    {Subderivation}
        \declsubjudgPf{[\Omega]\Theta}  {[\Omega]A_2}  {[\Omega]B_2}    {By above equality}
        \proofsep
        \eqPf{[\Omega]A_2} {[\Omega][\Gamma]A_2}  {By \Lemmaref{lem:subst-extension-invariance}}
        \eqPf{[\Omega]B_2} {[\Omega][\Gamma]B_2}  {By \Lemmaref{lem:subst-extension-invariance}}
        \proofsep
        \declsubjudgPf{[\Omega]\Theta}  {[\Omega][\Gamma]A_2}  {[\Omega][\Gamma]B_2}    {By above equalities}
        \subjudgPf{\Theta}  {[\Theta][\Gamma]A_2}  {[\Theta][\Gamma]B_2}  {\Delta}    {By i.h.}
\Hand   \substextendPf{\Delta} {\Omega'}    {\ditto}
        \substextendPf{\Omega_0} {\Omega'}    {\ditto}
        \decolumnizePf
        \subjudgPf{\Gamma}{([\Gamma]A_1) \arr ([\Gamma]A_2)}{([\Gamma]B_1) \arr ([\Gamma]B_2)}{\Delta} {By \SubArr}
\Hand   \subjudgPf{\Gamma}{[\Gamma](A_1 \arr A_2)}{[\Gamma](B_1 \arr B_2)}{\Delta} {By definition of substitution}
\Hand   \substextendPf{\Omega} {\Omega'}  {By \Lemmaref{lem:substextend-transitivity}}
      \end{llproof}\qedhere
    \end{itemize}
  \end{itemize}
\end{proof}



\completeness*
\begin{proof}
  Let $\Omega = \Psi$ and $\Gamma = \Psi$. \\
  By \Lemmaref{lem:substextend-reflexivity}, $\substextend{\Psi}{\Psi}$, so $\substextend{\Gamma}{\Omega}$. \\
  By \Lemmaref{lem:declarative-well-formed}, $\judgetp{\Psi}{A}$ and $\judgetp{\Psi}{B}$; since $\Gamma = \Psi$,
  we have
  $\judgetp{\Gamma}{A}$ and $\judgetp{\Gamma}{B}$.
  \\
  By \Theoremref{thm:completing-completeness},
  there exists $\Delta$ such that $\subjudg{\Gamma}{[\Gamma]A}{[\Gamma]B}{\Delta}$.
  \\
  Since $\Gamma = \Psi$ and $\Psi$ is a declarative context with no existentials, $[\Psi]C = C$ for all $C$,
  so we actually have $\subjudg{\Psi}{A}{B}{\Delta}$, which was to be shown.
\end{proof}




\clearpage
\section{Completeness of Typing}



\typingcompleteness*
\begin{proof}
  By induction on the given declarative derivation.

  \begin{itemize}
     \DerivationProofCase{\DeclVar}
          {(x : A) \in [\Omega]\Gamma}
          {\declsynjudg{[\Omega]\Gamma}{x}{A}}
          
          \begin{llproof}
            \inPf{(x : A)} {[\Omega]\Gamma}  {Premise}
            \substextendPf{\Gamma}{\Omega}   {Given}
            \inPf{(x : A')}{\Gamma\text{~where $[\Omega]A' = [\Omega]A$}}   {From definition of context application}
            \LetPf{\Delta}{\Gamma} {}
            \LetPf{\Omega'}{\Omega} {}
\Hand       \substextendPf{\Gamma}{\Omega}   {Given}
\Hand       \substextendPf{\Omega}{\Omega}   {By \Lemmaref{lem:substextend-reflexivity}}
\Hand       \synjudgPf{\Gamma}{x}{A'}{\Gamma}   {By \Var}
                   \eqPf{[\Omega]A'}{[\Omega]A}   {Above}
\Hand              \continueeqPf{A}   {$\FEV{A} = \emptyset$}
          \end{llproof}

     \DerivationProofCase{\DeclSub}
          {\declsynjudg{[\Omega]\Gamma}{e}{B}
            \\
            \declsubjudg{[\Omega]\Gamma}{B}{[\Omega]A}
          }
          {\declchkjudg{[\Omega]\Gamma}{e}{[\Omega]A}}

          \begin{llproof}
            \declsynjudgPf{[\Omega]\Gamma}{e}{B}   {Subderivation}
            \synjudgPf{\Gamma}{e}{B'}{\Theta}      {By i.h.}
            \eqPf{B}{[\Omega]B'}   {\ditto}
            \substextendPf{\Theta}{\Omega_0}   {\ditto}
            \substextendPf{\Omega}{\Omega_0}   {\ditto}
            \proofsep
            \substextendPf{\Gamma}{\Omega}   {Given}
            \substextendPf{\Gamma}{\Omega_0}   {By \Lemmaref{lem:substextend-transitivity}}
            \declsubjudgPf{[\Omega]\Gamma}{B}{[\Omega]A}  {Subderivation}
            \eqPf{[\Omega]\Gamma} {[\Omega]\Theta}   {By \Lemmaref{lem:completes-confluence}}
            \declsubjudgPf{[\Omega]\Theta}{B}{[\Omega]A}  {By above equalities}
            \substextendPf{\Theta}{\Omega_0}  {Above}
            \subjudgPf{\Theta}{[\Theta]B'}{[\Theta]A}{\Delta}  {By \Theoremref{thm:completing-completeness}}
            \substextendPf{\Delta}{\Omega'}   {\ditto}
            \substextendPf{\Omega_0}{\Omega'}   {\ditto}
\Hand   \substextendPf{\Delta}{\Omega'}   {By \Lemmaref{lem:substextend-transitivity}}
\Hand   \substextendPf{\Omega}{\Omega'}   {By \Lemmaref{lem:substextend-transitivity}}
            \proofsep
\Hand   \chkjudgPf{\Gamma}{e}{A}{\Delta}  {By \Sub}
          \end{llproof}

     \DerivationProofCase{\DeclAnno}
          {\judgetp{[\Omega]\Gamma}{A}
            \\
            \declchkjudg{[\Omega]\Gamma}{e_0}{A}
          }
          {\declsynjudg{[\Omega]\Gamma}{(e_0 : A)}{A}}

          \begin{llproof}
            \eqPf{A} {[\Omega]A}   {Source type annotations cannot contain evars}
            \continueeqPf           {[\Gamma]A}   {\ditto}
            \declchkjudgPf{[\Omega]\Gamma}{e_0}{A}   {Subderivation}
            \declchkjudgPf{[\Omega]\Gamma}{e_0}{[\Omega]A}   {By above equality}
            \chkjudgPf{\Gamma}{e_0}{[\Gamma]A}{\Delta}   {By i.h.}
\Hand       \substextendPf{\Delta}{\Omega}   {\ditto}
\Hand       \substextendPf{\Omega}{\Omega'}   {\ditto}
            \proofsep
            \judgetpPf{\Gamma}{A}   {Given}
            \proofsep
            \synjudgPf{\Gamma}{(e_0 : A)}{A}{\Delta}   {By \Anno}
            \eqPf{A} {[\Omega']A}   {Source type annotations cannot contain evars}
\Hand       \synjudgPf{\Gamma}{(e_0 : [\Omega']A)}{[\Omega']A}{\Delta}   {By above equality}
          \end{llproof}

     \DerivationProofCase{\DeclUnitIntro}
          {}
          {\declchkjudg{[\Omega]\Gamma}{\unitexp}{\unitty}}

          We have $[\Omega]A = \unitty$.  Either $[\Gamma]A = \unitty$ or $[\Gamma]A = \ahat \in \unsolved{\Gamma}$.

          In the former case:

          \begin{llproof}
            \LetPf{\Delta}{\Gamma} {}
            \LetPf{\Omega'}{\Omega} {}
\Hand       \substextendPf{\Gamma}{\Omega}   {Given}
\Hand       \substextendPf{\Omega}{\Omega'}   {By \Lemmaref{lem:substextend-reflexivity}}
             \chkjudgPf{\Gamma}{\unitexp}{\unitty}{\Gamma}   {By \UnitIntro}
\Hand       \chkjudgPf{\Gamma}{\unitexp}{[\Gamma]\unitty}{\Gamma}   {$\unitty = [\Gamma]\unitty$}
          \end{llproof}

          In the latter case:

          \begin{llproof}
             \synjudgPf{\Gamma}{\unitexp}{\unitty}{\Gamma}   {By \UnitIntroSyn}
             \declsubjudgPf{[\Omega]\Gamma}{\unitty}{\unitty}   {By \DsubUnit}
             \proofsep
             \eqPf{\unitty} {[\Omega]\unitty}   {By definition of substitution}
             \continueeqPf {[\Omega][\Gamma]\ahat}   {By $[\Omega]A = \unitty$}
             \continueeqPf {[\Omega]\ahat}   {By \Lemmaref{lem:subst-extension-invariance}}
             \proofsep
             \declsubjudgPf{[\Omega]\Gamma}{[\Omega]\unitty}{[\Omega]\ahat}   {By above equalities}
             \subjudgPf{\Gamma}{\unitty}{\ahat}{\Delta}   {By \Theoremref{thm:instantiation-completes} (1)}
             \eqPf{\unitty}{[\Gamma]\unitty}  {By definition of substitution}
             \eqPf{\ahat}{[\Gamma]\ahat}  {$\ahat \in \unsolved{\Gamma}$}
             \subjudgPf{\Gamma}{[\Gamma]\unitty}{[\Gamma]\ahat}{\Delta}   {By above equalities}
\Hand  \substextendPf{\Omega}{\Omega'} {\ditto}
\Hand  \substextendPf{\Delta}{\Omega'} {\ditto}
             \chkjudgPf{\Gamma}{\unitexp}{\ahat}{\Delta}   {By \Sub}
\Hand        \chkjudgPf{\Gamma}{\unitexp}{[\Gamma]A}{\Delta}   {By $[\Gamma]A = \ahat$}
          \end{llproof}


     \DerivationProofCase{\DeclAllIntro}
           {\declchkjudg{[\Omega]\Gamma, \alpha}{e}{A_0}
           }
           {\declchkjudg{[\Omega]\Gamma}{{e}}{\alltype{\alpha}{A_0}}}

           \begin{llproof}
             \eqPf{[\Omega]A}{\alltype{\alpha}{A_0}}   {Given}
             \continueeqPf{\alltype{\alpha}{[\Omega]A'}}  {By def. of subst. and predicativity of $\Omega$}
             \eqPf{A_0}{[\Omega]A'}   {Follows from above equality}
             \declchkjudgPf{[\Omega]\Gamma, \alpha}{e}{[\Omega]A'}  {Subderivation and above equality}
             \proofsep
             \substextendPf{\Gamma}{\Omega}  {Given}
             \substextendPf{\Gamma, \alpha}{\Omega, \alpha}  {By \substextendUU}
             \proofsep
             \eqPf{[\Omega]\Gamma, \alpha}{[\Omega, \alpha](\Gamma, \alpha)}  {By definition of context substitution}
             \declchkjudgPf{[\Omega, \alpha](\Gamma, \alpha)}{e}{[\Omega]A'}  {By above equality}
             \declchkjudgPf{[\Omega, \alpha](\Gamma, \alpha)}{e}{[\Omega, \alpha]A'}  {By definition of substitution}
           \end{llproof}

           \begin{llproof}
             \chkjudgPf{\Gamma, \alpha}{e}{[\Gamma, \alpha]A'}{\Delta'}  {By i.h.}
             \substextendPf{\Delta'}{\Omega_0'}   {\ditto}
             \substextendPf{\Omega, \alpha}{\Omega_0'}   {\ditto}
             \substextendPf{\Gamma, \alpha}{\Delta'}  {By \Lemmaref{lem:typing-extension}}
             \eqPf{\Delta'} {\Delta, \alpha, \Theta}   {By \Lemmaref{lem:extension-order} (i)}
             \substextendPf{\Delta, \alpha, \Theta}{\Omega_0'}   {By above equality}
             \eqPf{\Omega_0'} {\Omega', \alpha, \Omega_Z}   {By \Lemmaref{lem:extension-order} (i)}
\Hand        \substextendPf{\Delta} {\Omega'}   {\ditto}
\Hand        \substextendPf{\Omega}{\Omega'}  {By \Lemmaref{lem:extension-order} on $\substextend{\Omega, \alpha}{\Omega_0'}$}
             \proofsep
             \chkjudgPf{\Gamma, \alpha}{e}{[\Gamma, \alpha]A'}{\Delta, \alpha, \Theta}   {By above equality}
             \chkjudgPf{\Gamma, \alpha}{e}{[\Gamma]A'}{\Delta, \alpha, \Theta}   {By definition of substitution}
             \chkjudgPf{\Gamma}{e}{\alltype{\alpha}{[\Gamma]A'}}{\Delta}   {By \AllIntro}
\Hand        \chkjudgPf{\Gamma}{e}{[\Gamma](\alltype{\alpha}{A'})}{\Delta}   {By definition of substitution}
           \end{llproof}


     \DerivationProofCase{\DeclAllApp}
            {\judgetp{[\Omega]\Gamma}{\tau}
              \\ 
             \declappjudg{[\Omega]\Gamma}{e}{[\tau/\alpha]A_0}{C}}
            {\declappjudg{[\Omega]\Gamma}{e}{\underbrace{\alltype{\alpha}A_0}_{[\Omega]A}}{C}}

            \begin{llproof}
              \judgetpPf{[\Omega]\Gamma}{\tau}   {Subderivation}
              \proofsep
              \eqPf{[\Omega]A}{\alltype{\alpha} A_0}   {Given}
              \continueeqPf{\alltype{\alpha} [\Omega]A'}  {By def. of subst. and predicativity of $\Omega$}
              \declappjudgPf{[\Omega]\Gamma}{e}{[\tau/\alpha][\Omega]A'}{C}  {Subderivation and above equality}
              \substextendPf{\Gamma}{\Omega}   {Given}
              \substextendPf{\Gamma, \ahat}{\Omega, \hypeq{\ahat}{\tau}}  {By \substextendSolve}
              \proofsep
              \eqPf{[\Omega]\Gamma} {[\Omega, \hypeq{\ahat}{\tau}](\Gamma, \ahat)}  {By definition of context application}
              \declappjudgPf{[\Omega, \hypeq{\ahat}{\tau}](\Gamma, \ahat)}{e}{[\tau/\alpha][\Omega]A'}{C}  {By above equality}
              \declappjudgPf{[\Omega, \hypeq{\ahat}{\tau}](\Gamma, \ahat)}{e}{[\tau/\alpha][\Omega, \hypeq{\ahat}{\tau}]A'}{C}  {By def.\ of subst.}
              \eqPf{\big(\big[[\Omega]\tau/\alpha\big]\big[\Omega, \hypeq{\ahat}{\tau}\big]A'\big)} {\big([\Omega, \hypeq{\ahat}{\tau}][\ahat/\alpha]A'\big)}  {By distributivity of substitution}
              \eqPf{\tau}{[\Omega]\tau}   {$\FEV{\tau} = \emptyset$}
              \eqPf{\big(\big[\tau/\alpha\big]\big[\Omega, \hypeq{\ahat}{\tau}\big]A'\big)} {\big([\Omega, \hypeq{\ahat}{\tau}][\ahat/\alpha]A'\big)}  {By above equality}
              \declappjudgPf{[\Omega, \hypeq{\ahat}{\tau}](\Gamma, \ahat)}{e}{[\Omega, \hypeq{\ahat}{\tau}][\ahat/\alpha]A'}{C}  {By above equality}
            \end{llproof}

            \begin{llproof}
              \appjudgPf {\Gamma, \ahat} {e} {[\ahat/\alpha]A'} {C'} {\Delta}   {By i.h.}
\Hand         \eqPf{C}{[\Omega]C'}   {\ditto}
\Hand         \substextendPf{\Delta}{\Omega'}   {\ditto}
\Hand         \substextendPf{\Omega}{\Omega'}   {\ditto}
              \proofsep
\Hand         \appjudgPf{\Gamma}{e}{\alltype{\alpha} A'} {C'} {\Delta}   {By \AllApp}
            \end{llproof}

     \DerivationProofCase{\DeclArrIntro}
          {\declchkjudg{[\Omega]\Gamma, x : A_1'}{e_0}{A_2'}
          }
          {\declchkjudg{[\Omega]\Gamma}{\lam{x} e_0}{A_1' \arr A_2'}}

          We have $[\Omega]A = A_1' \arr A_2'$.  Either $[\Gamma]A = A_1 \arr A_2$
          where $A_1' = [\Omega]A_1$ and $A_2' = [\Omega]A_2$---or
          $[\Gamma]A = \ahat$ and $[\Omega]\ahat = A_1' \arr A_2'$.

          In the former case:

          \begin{llproof}
            \declchkjudgPf{[\Omega]\Gamma, x : A_1'}{e_0}{A_2'}   {Subderivation}
            \proofsep
            \eqPf{A_1'}  {[\Omega]A_1}   {Known in this subcase}
            \continueeqPf {[\Omega][\Gamma]A_1}   {By \Lemmaref{lem:subst-extension-invariance}}
            \eqPf{[\Omega]A_1'} {[\Omega][\Omega][\Gamma]A_1}   {Applying $\Omega$ on both sides}
            \continueeqPf {[\Omega][\Gamma]A_1}   {By idempotence of substitution}
            \proofsep
            \eqPf{[\Omega]\Gamma, x : A_1'}  {[\Omega, x : A_1'](\Gamma, x : [\Gamma]A_1)}   {By definition of context application}
            \proofsep
            \declchkjudgPf{[\Omega, x : A_1'](\Gamma, x : [\Gamma]A_1)}{e_0}{A_2'}   {By above equality}
            \proofsep
            \substextendPf{\Gamma}{\Omega}   {Given}
            \substextendPf{\Gamma, x : [\Gamma]A_1}{\Omega, x : A_1'}   {By \substextendVV}
            \proofsep
            \chkjudgPf{\Gamma, x : [\Gamma]A_1}{e_0}{A_2}{\Delta'}   {By i.h.}
            \substextendPf{\Delta'}{\Omega_0'}   {\ditto}
            \substextendPf{\Omega, x : A_1'}{\Omega_0'}   {\ditto}
            \eqPf{\Omega_0'} {\Omega', x : A_1', \Theta}   {By \Lemmaref{lem:extension-order} (v)}
\Hand       \substextendPf{\Omega} {\Omega'}  {\ditto}
            \proofsep
            \substextendPf{\Gamma, x : [\Gamma]A_1}{\Delta'}   {By \Lemmaref{lem:typing-extension}}
            \eqPf{\Delta'}{\Delta, x : \cdots, \Theta}   {By \Lemmaref{lem:extension-order} (v)}
            \substextendPf{\Delta, x : \cdots, \Theta}{\Omega', x : A_1', \Theta}  {By above equalities}
\Hand       \substextendPf{\Delta}{\Omega'}  {By \Lemmaref{lem:extension-order} (v)}
\decolumnizePf
            \chkjudgPf{\Gamma, x : [\Gamma]A_1}{e_0}{[\Gamma]A_2}{\Delta, \alpha, \Theta}   {By above equality}
            \chkjudgPf{\Gamma}{\lam{x} e_0}{([\Gamma]A_1) \arr ([\Gamma]A_2)}{\Delta}  {By \ArrIntro}
\Hand      \chkjudgPf{\Gamma}{\lam{x} e_0}{[\Gamma](A_1 \arr A_2)}{\Delta}  {By definition of substitution}
          \end{llproof}
          
          In the latter case:
          
          \begin{llproof}
            \eqPf{[\Omega]\ahat}{A_1' \arr A_2'}  {Known in this subcase}
            \declchkjudgPf{[\Omega]\Gamma, x : A_1'}{e_0}{A_2'}   {Subderivation}
            \substextendPf{\Gamma}{\Omega}   {Given}
            \substextendPf{\Gamma, \ahat, \bhat} {\Omega, \hypeq{\ahat}{A_1'}, \hypeq{\bhat}{A_2'}}
                      {By \substextendSolve twice}
            \eqPf{[\Omega]\ahat} {[\Omega]A_1'}   {By definition of substitution}
            \substextendPf{\Gamma, \ahat, \bhat, x : \ahat} {\Omega, \hypeq{\ahat}{A_1'}, \hypeq{\bhat}{A_2'}, x : A_1'}
                      {By \substextendVV}
            \eqPf{[\Omega]\Gamma, x : A_1'}
                   {\big[\Omega, \hypeq{\ahat}{A_1'}, \hypeq{\bhat}{A_2'}, x : A_1'\big]\big(\Gamma, \ahat, \bhat, x : \ahat\big)}
                   {By definition of context application}
            \LetPf{\Omega_0}{(\Omega, \hypeq{\ahat}{A_1'}, \hypeq{\bhat}{A_2'}, x : A_1')}   {}
            \declchkjudgPf{[\Omega_0](\Gamma, \ahat, \bhat, x : \ahat)}{e_0}{A_2'}   {By above equality}
            \chkjudgPf{\Gamma, \ahat, \bhat, x : \ahat} {e_0} {\bhat} {\Delta'}     {By i.h. with $\Omega_0$}
            \substextendPf{\Delta'} {\Omega_0'}     {\ditto}
            \substextendPf{\Omega_0} {\Omega_0'}     {\ditto}
          \end{llproof}

          \begin{llproof}
            \substextendPf{\Gamma, \ahat, \bhat, x : \ahat}{\Delta'}     {By \Lemmaref{lem:typing-extension}}
            \eqPf{\Delta'}{\Delta, x : \ahat, \Theta}   {By \Lemmaref{lem:extension-order} (v)}
            \substextendPf{\Delta, x : \ahat, \Theta} {\Omega_0'}   {By above equality}
            \eqPf{\Omega_0'} {\Omega'', x : \cdots, \Omega_Z}     {By \Lemmaref{lem:typing-extension}}
\Hand       \substextendPf{\Delta}{\Omega''}     {\ditto}
            \substextendPf{\Gamma, \ahat, \bhat}{\Delta}   {\ditto}
            \substextendPf{\Omega_0} {\underbrace{\Omega'', x : \cdots, \Omega_Z}_{\Omega_0'}}     {By above equality}
            \substextendPf{\Omega, \hypeq{\ahat}{A_1'}, \hypeq{\bhat}{A_2'}, x : A_1'}
                          {\Omega'', x : \cdots, \Omega_Z}
                          {By def.\ of $\Omega_0$}
            \eqPf{\Omega''}
                      {\Omega', \hypeq{\ahat}{\dots}, \dots}
                      {By \Lemmaref{lem:extension-order} (iii)}
\Hand       \substextendPf{\Omega}{\Omega'}  {\ditto}
            \decolumnizePf
            \chkjudgPf{\Gamma, \ahat, \bhat, x : \ahat} {e_0} {\bhat} {\Delta, x : \ahat, \Theta}     {By above equality}
            \chkjudgPf{\Gamma} {\lam{x} e_0} {\ahat \arr \bhat} {\Delta}     {By \ArrIntroSyn}
            \proofsep
            \eqPf{[\Gamma]\ahat}{\ahat}    {By definition of substitution}
            \eqPf{[\Gamma]\bhat}{\bhat}    {By definition of substitution}
            \chkjudgPf{\Gamma}{\lam{x} e_0}{([\Gamma]\ahat) \arr ([\Gamma]\bhat)}{\Delta} {By above equalities}
\Hand       \chkjudgPf{\Gamma}{\lam{x} e_0}{[\Gamma](\ahat \arr \bhat)}{\Delta} {By definition of substitution}
          \end{llproof}

     \DerivationProofCase{\DeclArrElim}
          {\declsynjudg{[\Omega]\Gamma}{e_1}{B}
            \\
            \declappjudg{[\Omega]\Gamma}{e_2}{B}{A}
          }
          {\declsynjudg{[\Omega]\Gamma}{e_1\,e_2}{A}}

          \begin{llproof}
            \declsynjudgPf{[\Omega]\Gamma}{e_1}{B}   {Subderivation}
            \substextendPf{\Gamma}{\Omega}  {Given}
            \synjudgPf{\Gamma} {e_1} {B'} {\Theta}   {By i.h.}
            \eqPf{B} {[\Omega]B'}  {\ditto}
            \substextendPf{\Theta} {\Omega_0'}  {\ditto}
            \substextendPf{\Omega} {\Omega_0'}  {\ditto}
            \proofsep
            \declappjudgPf{[\Omega]\Gamma}{e_2}{B}{A}  {Subderivation}
            \declappjudgPf{[\Omega]\Gamma}{e_2}{[\Omega]B'}{A}  {By above equality}
            \substextendPf{\Gamma}{\Omega_0'}  {By \Lemmaref{lem:substextend-transitivity}}
            \eqPf{[\Omega]\Gamma} {[\Omega]\Omega}   {By \Lemmaref{lem:completes-stability}}
            \continueeqPf {[\Omega_0']\Omega_0'}   {By \Lemmaref{lem:finishing-completions}}
            \continueeqPf {[\Omega_0']\Gamma}   {By \Lemmaref{lem:completes-stability}}
            \continueeqPf {[\Omega_0']\Theta}   {By \Lemmaref{lem:completes-confluence}}
            \declappjudgPf{[\Omega_0']\Theta}{e_2}{[\Omega]B'}{A}  {By above equality}
            \eqPf{[\Omega]B'}{[\Omega_0']B'}   {By \Lemmaref{lem:finishing-types}}
            \eqPf{[\Omega_0']B'}{[\Omega_0'][\Theta]B'}   {By \Lemmaref{lem:subst-extension-invariance}}
            \declappjudgPf{[\Omega_0']\Theta}{e_2}{[\Omega][\Theta]B'}{A}  {By above equalities}
            \proofsep
            \appjudgPf{\Theta}{e_2}{[\Theta]B'}{A'}{\Delta}  {By i.h. with $\Omega_0'$}
\Hand            \eqPf{A} {[\Omega]A'}  {\ditto}
\Hand            \substextendPf{\Delta}{\Omega'} {\ditto}
            \substextendPf{\Omega_0'}{\Omega'} {\ditto}
            \substextendPf{\Omega}{\Omega'} {By \Lemmaref{lem:substextend-transitivity}}
\Hand            \synjudgPf{\Gamma}{e_1\,e_2}{A'}{\Delta}  {By \ArrElim}
          \end{llproof}


\clearpage
      \DerivationProofCase{\DeclArrApp}
            {\declchkjudg{[\Omega]\Gamma}{e}{B}}
            {\declappjudg{[\Omega]\Gamma}{e}{\underbrace{B \arr C}_{[\Omega]A}}{C}}

          We have $[\Omega]A = B \arr C$.  Either $[\Gamma]A = B_0 \arr C_0$
          where $B = [\Omega]B_0$ and $C = [\Omega]C_0$---or
          $[\Gamma]A = \ahat$ where $\ahat \in \unsolved{\Gamma}$
          and $[\Omega]\ahat = B \arr C$.

          In the former case:

          \begin{llproof}
            \declchkjudgPf{[\Omega]\Gamma}{e}{B}  {Subderivation}
            \eqPf{B} {[\Omega]B_0}  {Known in this subcase}
            \proofsep
            \substextendPf{\Gamma}{\Omega}   {Given}
            \proofsep
            \chkjudgPf{\Gamma}{e}{[\Gamma]B_0}{\Delta}   {By i.h.}
            \appjudgPf{\Gamma}{e}{([\Gamma]B_0) \arr ([\Gamma]C_0)}{[\Gamma]C_0}{\Delta}  {By \ArrApp}
\Hand       \substextendPf{\Delta}{\Omega'}   {\ditto}
\Hand       \substextendPf{\Omega}{\Omega'}   {\ditto}
            \LetPf{C'} {[\Gamma]C_0}  {}
            \eqPf{C}{[\Omega]C_0}  {Known in this subcase}
            \continueeqPf {[\Omega][\Gamma]C_0}   {By \Lemmaref{lem:subst-extension-invariance}}
\Hand       \continueeqPf {[\Omega]C'}   {$[\Gamma]C_0 = C'$}
\Hand       \appjudgPf{\Gamma}{e}{[\Gamma](B_0 \arr C_0)}{[\Gamma]C_0}{\Delta}  {By definition of substitution}
          \end{llproof}
          
          In the latter case, $\ahat \in \unsolved{\Gamma}$,
          so the context $\Gamma$ must have the form $\Gamma_0[\ahat]$.
            
            \begin{llproof}
              \substextendPf{\Gamma}{\Omega}   {Given}
              \substextendPf{\Gamma_0[\ahat]}{\Omega}   {$\Gamma = \Gamma_0[\ahat]$}
              \eqPf{[\Omega]A}  {B \arr C}   {Above}
              \eqPf{[\Omega]\ahat}  {B \arr C}   {$A = \ahat$}
              \eqPf{\Omega} {\Omega_0[\hypeq{\ahat}{A_0}]
                             \AND   [\Omega]A_0 = B \arr C}    {Follows from $[\Omega]\ahat = B \arr C$}
              \LetPf{\Gamma'}{\Gamma_0[\ahat_2, \ahat_1, \hypeq{\ahat}{\ahat_1 \arr \ahat_2}]}   {}
              \LetPf{\Omega_0'}{\Omega_0[\hypeq{\ahat_2}{[\Omega]C}, \hypeq{\ahat_1}{[\Omega]B}, \hypeq{\ahat}{\ahat_1 \arr \ahat_2}]}   {}
              \decolumnizePf
              \substextendPf{\Gamma'}{\Omega_0'}    {By \Lemmaref{lem:parallel-admissibility} (ii) twice}
              \proofsep
              \declchkjudgPf{[\Omega]\Gamma}{e}{B}   {Subderivation}
              \proofsep
              \substextendPf{\Omega}{\Omega_0'}   {By \Lemmaref{lem:extension-addsolve}}
                       \trailingjust{then \Lemmaref{lem:parallel-admissibility} (iii)}
              \eqPf{[\Omega]\Gamma} {[\Omega]\Omega}   {By \Lemmaref{lem:completes-stability}}
              \continueeqPf {[\Omega_0']\Omega_0'}   {By \Lemmaref{lem:finishing-completions}}
              \continueeqPf {[\Omega_0']\Gamma'}   {By \Lemmaref{lem:completes-confluence}}
              \proofsep
              \eqPf{B} {[\Omega_0']\ahat_1}  {By definition of $\Omega_0'$}
\declchkjudgPf{[\Omega_0']\Gamma'}{e}{[\Omega_0']\ahat_1}   {By above equalities}
              \proofsep
              \chkjudgPf{\Gamma'}{e}{[\Gamma']\ahat_1}{\Delta}     {By i.h.}
\Hand         \substextendPf{\Delta}{\Omega'}   {\ditto}
              \substextendPf{\Omega_0'}{\Omega'}   {\ditto}              
\Hand         \substextendPf{\Omega}{\Omega'}   {By \Lemmaref{lem:substextend-transitivity}}
              \proofsep
              \eqPf {[\Gamma']\ahat_1} {\ahat_1}   {$\ahat_1 \in \unsolved{\Gamma'}$}
              \chkjudgPf{\Gamma'}{e}{\ahat_1}{\Delta}   {By above equality}
            \end{llproof}
            
            \begin{llproof}
              \appjudgPf{\Gamma} {e} {\ahat} {\ahat_2}  {\Delta}  {By \SolveApp}
              \LetPf{C'} {\ahat_2} {}
              \eqPf{C}{[\Omega_0']\ahat_2}  {By definition of $\Omega_0'$}
              \continueeqPf{[\Omega']\ahat_2}  {By \Lemmaref{lem:finishing-types}}
\Hand         \continueeqPf{[\Omega']C'}  {By above equality}
\Hand         \appjudgPf{\Gamma}{e}{[\Gamma]A}{C'}{\Delta}  {$\ahat = [\Gamma]A$ and $\ahat_2 = C'$}
            \end{llproof}




\DerivationProofCase{\DeclUnitIntroSyn}
                { }
                {\declsynjudg{[\Omega]\Gamma}{\unitexp}{\unitty}}

                \begin{llproof}
                  \eqPf {\unitty} {A}   {Given}
                  \synjudgPf{\Gamma} {\unitexp} {\unitty}  {\Gamma}   {By \UnitIntroSyn}
                  \LetPf{\Delta}{\Gamma}  {}
                  \LetPf{\Omega'}{\Omega}  {}
                  \substextendPf{\Gamma}{\Omega}  {Given}
\Hand             \substextendPf{\Delta}{\Omega}  {By above equality}
\Hand             \substextendPf{\Omega}{\Omega'}  {By \Lemmaref{lem:substextend-reflexivity}}
                  \LetPf{A'}{\unitty}  {}
\Hand             \synjudgPf{\Gamma} {\unitexp} {A'}  {\Delta}   {By above equalities}
\Hand             \eqPf {\unitty}  {[\Omega]A'}  {By definition of substitution}
                \end{llproof}

                \bigskip

          \DerivationProofCase{\DeclArrIntroSyn}
                {\judgetp{[\Omega]\Gamma}{\sigma \arr \tau}
                 \\
                 \declchkjudg{[\Omega]\Gamma, x : \sigma}{e_0}{\tau}
                }
                {\declsynjudg{[\Omega]\Gamma}{\lam{x} e_0}{\sigma \arr \tau}}

                \medskip

                \begin{llproof}
                  \eqPf{(\sigma \arr \tau)}{A}    {Given}
                  \declchkjudgPf{[\Omega]\Gamma, x : \sigma}{e_0}{\tau}   {Subderivation}
                  \proofsep
                  \LetPf{\Gamma'} {(\Gamma, \ahat, \bhat, x : \ahat)}  {}
                  \LetPf{\Omega_0} {(\Omega, \hypeq{\ahat}{\sigma}, \hypeq{\bhat}{\tau}, x : \sigma)}   {}
                  \proofsep
                  \substextendPf{\Gamma}{\Omega}   {Given}
                  \substextendPf{\Gamma'}{\Omega_0}   {By \substextendSolve twice, then \substextendVV}
                  \proofsep
                  \eqPf{[\Omega_0]\Gamma'} {\big([\Omega]\Gamma, x : \sigma\big)}   {By definition of context application}
                  \eqPf{\tau}{[\Omega_0]\bhat}{By definition of $\Omega_0$}
                  \declchkjudgPf{[\Omega_0]\Gamma'}{e_0}{[\Omega_0]\bhat}   {By above equalities}
                  \proofsep
                  \chkjudgPf{\Gamma'}{e_0}{\bhat} {\Delta'}   {By i.h.}
                  \substextendPf{\Delta'}{\Omega_0'}    {\ditto}
                  \substextendPf{\Omega_0}{\Omega_0'}    {\ditto}
                  \proofsep
                  \eqPf{\Delta'}{(\Delta, x : \ahat, \Theta)}  {By \Lemmaref{lem:extension-order} (v)}
                  \chkjudgPf{\Gamma, \ahat, \bhat, x : \ahat}{e_0}{\bhat} {\Delta, x : \ahat, \Theta}   {By above equalities}
\substextendPf{(\Delta, x : \ahat, \Theta)} {\Omega_0'}  {By above equality}
                  \eqPf{\Omega_0'}{\Omega', x : \sigma, \Omega_Z}   {By \Lemmaref{lem:extension-order} (v)}
\Hand             \substextendPf{\Delta} {\Omega'}  {\ditto}
                  \synjudgPf{\Gamma}{\lam{x} e_0}{\ahat \arr \bhat}{\Delta}   {By \ArrIntroSyn}
               \end{llproof}

               \begin{llproof}
                  \LetPf{A'} {(\ahat \arr \bhat)}  {}
\Hand              \synjudgPf{\Gamma}{\lam{x} e_0}{A'}{\Delta}   {By above equality}
                  \eqPf{\sigma \arr \tau}{([\Omega_0]\ahat) \arr ([\Omega_0]\bhat)}  {By definition of $\Omega_0$}
                  \eqPf{\sigma \arr \tau}{[\Omega_0](\ahat \arr \bhat)}  {By definition of substitution}
                  \eqPf{A} {[\Omega_0]A'}   {By above equalities}
\Hand            \eqPf{A} {[\Omega']A'}   {By \Lemmaref{lem:finishing-types}}
                  \proofsep
                  \substextendPf{\Gamma'}{\Delta'}   {By \Lemmaref{lem:typing-extension}}
\Hand             \substextendPf{\Omega}{\Omega'}  {By \Lemmaref{lem:substextend-transitivity}     \qedhere}
                \end{llproof}
\end{itemize}
\end{proof}







\ifnum\OPTIONLoudLabels=1
  \bibliographystyle{plainnatlocalcopy}
\else
  \bibliographystyle{plainnat}
\fi
\bibliography{local}


\end{document}
