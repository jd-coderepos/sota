\makeatletter
\@ifundefined{OPTIONAppendix}{\def\OPTIONAppendix{0}}{}
\makeatother
\def\OPTIONConf{0}
\ifnum\OPTIONAppendix=0
    \documentclass[a4paper]{article}
    \def\OPTIONLoudLabels{0}
\def\OPTIONArxiv{1}

 
    \usepackage{srcltx}
    \usepackage{fancyhdr}
    \usepackage{xspace}
    \usepackage{enumerate}
    \usepackage[hyperref,svgnames]{xcolor}
    \usepackage{graphics,pstricks,color}
    \usepackage{relsize}

    \usepackage{booktabs}

    \usepackage{phaistos}
    \newcommand{\alittlefishy}{\text{\PHtunny}}
    \newcommand{\alotfishy}{\flaming{\alittlefishy}}

    \renewcommand{\bfdefault}{b}
    \usepackage[hyphens]{url}


    \usepackage{hyperref} \usepackage{latexsym,stmaryrd}
    \usepackage{amsmath,amsthm,amssymb}
    \usepackage{thmtools,thm-restate}



    \usepackage{mathpartir}
    \usepackage{charter}
    \usepackage{euler}
    \usepackage{marvosym}
    \usepackage{fullpage}
    \usepackage[authoryear]{natbib}

    \bibpunct{(}{)}{;}{a}{}{,}

    \declaretheoremstyle[
      bodyfont=\sl
    ]{mytheoremstyle}

    \declaretheorem[style=mytheoremstyle]{theorem}
    \declaretheorem[style=mytheoremstyle, sibling=theorem]{property}
\declaretheorem[style=mytheoremstyle]{lemma}
    \declaretheorem[style=mytheoremstyle, sibling=lemma]{corollary}
    \declaretheorem[style=mytheoremstyle, sibling=lemma]{conjecture}
    \declaretheorem{example}
    \makeatletter
    \declaretheorem[style=mytheoremstyle, sibling=lemma, 
]{proposition}
    \makeatother
    \declaretheorem{remark}
    \declaretheorem[style=mytheoremstyle]{definition}
\else
\fi


\newcommand{\addmytocentry}[2]{\addtocontents{toc}{\protect\contentsline {subsubsection}{\protect\numberline {#1}#2}{\thepage}{xx}}}
\newcommand{\MyThmRestateHook}[3]{\relax
}

\newcommand{\lxbel}[1]{\label{#1}}

\makeatletter
\newcommand{\XLabel}[1]{\ifcsname IDEMPOTFLAG#1\endcsname \lxbel{PROOF#1}\LoudLabel{#1}\addmytocentry{\ref{PROOF#1}}{\hyperref[PROOF#1]{Proof of \thmt@thmname~(\thmt@optarg)}}\else \Label{#1}\addmytocentry{\ref{#1}}{\hyperref[#1]{\thmt@thmname~(\thmt@optarg)}}\expandafter\gdef\csname IDEMPOTFLAG#1\endcsname{d}\fi}
\makeatother


\newcounter{zzbackby}
\newcommand{\backby}[2]{\setcounter{zzbackby}{\uccode`#1}\addtocounter{zzbackby}{-#2}\char\value{zzbackby}}



\newcommand{\XAlph}[1]{\backby{\Alph{section}}{#1}}



\ifnum\OPTIONAppendix=0
        \usepackage[T1]{fontenc}

\definecolor{dHilite}{rgb}{0.9, 0.9, 0.6}
\definecolor{dRed}{rgb}{0.65, 0.0, 0.0}
\definecolor{dGreen}{rgb}{0.0, 0.65, 0.0}
\definecolor{dDkGreen}{rgb}{0.0, 0.35, 0.0}
\definecolor{dBlue}{rgb}{0.0, 0.0, 0.65}
\definecolor{dPurple}{rgb}{0.65, 0.0, 0.65}
\definecolor{dDigPurple}{rgb}{0.5, 0.0, 0.5}
\definecolor{dFaint}{rgb}{0.7, 0.7, 0.7}
\definecolor{dGray}{rgb}{0.5, 0.5, 0.5}
\definecolor{dDark}{rgb}{0.2, 0.2, 0.2}
\definecolor{dAlmostBlack}{rgb}{0.1, 0.1, 0.1}

\makeatletter
\def\url@MGstyle{\def\UrlFont{\tiny\huge\ttfamily}\Url@do
}
\makeatother
\newcommand{\marginPseudoURL}[1]{\tt #1}
\newcommand{\marginnote}[1]{\marginparwidth=40pt \marginpar{\raisebox{-2ex}{\parbox{40pt}{\raggedright\scriptsize #1}}}}

\makeatletter
\def\url@vttstyle{\@ifundefined{selectfont}{\def\UrlFont{\tt}}{\def\UrlFont{\normalfont\fontfamily{cmvtt}\selectfont}}}
\makeatother
\urlstyle{vtt}



\newcommand\textvtt[1]{{\normalfont\fontfamily{cmvtt}\selectfont #1}}

\newcommand{\LoudLabel}[1]{\idempotentlabel{#1}\ifnum\OPTIONLoudLabels=1\ifnum\OPTIONConf=1\marginnote{\tiny\textvtt{#1}}\else \marginnote{\textvtt{#1}}\fi \fi }

\newcommand{\idempotentlabel}[1]{\ifcsname IDEMPFLAG#1\endcsname \message{YYY ALREADY DEFINED: #1}
    \else \message{YYZ NOT ALREADY DEFINED: #1}
      \expandafter\gdef\csname IDEMPFLAG#1\endcsname{d}\label{#1}\fi}



\ifnum\OPTIONLoudLabels=1\newcommand{\Label}[1]{\LoudLabel{#1}}\newcommand{\FLabel}[1]{\idempotentlabel{#1}{\tt\scriptsize{#1}}}\else \newcommand{\Label}[1]{\idempotentlabel{#1}}\newcommand{\FLabel}[1]{\idempotentlabel{#1}}\fi

\newdimen\zzfontsz
\newcommand{\fontsz}[2]{\zzfontsz=#1{\fontsize{\zzfontsz}{1.2\zzfontsz}\selectfont{#2}}}

\newcommand{\texfontsz}[1]{\zzfontsz=#1\fontsize{\zzfontsz}{1.25\zzfontsz}\selectfont}

\newcommand{\mathsz}[2]{\text{\fontsz{#1}{}}}

\newcommand{\XX}{\text{\ding{55}}}
\newcommand{\redXX}{\text{\textcolor{dRed}{\XX}}}
\newcommand{\greencheck}{\text{\textcolor{dGreen}{\checkmark}}}

\newcommand{\fixme}[1]{\textcolor{red}{\texttt{FIXME: {#1}}}}

\newcommand{\flaming}[1]{\textcolor{red}{\fontsz{18pt}{\bf #1}}}
\newcommand{\flamingmath}[1]{\textcolor{red}{\fontsz{18pt}{\bf \ensuremath{#1}}}}

\newcommand{\semiflaming}[1]{\textcolor{dRed}{\sl #1}}

\newcommand{\mathcolor}[2]{\text{\textcolor{#1}{\ensuremath{#2}}}}

\newcommand{\smallblacktriangle}{\text{\textscale{0.7}{}}}

\newcommand{\setof}[1]{\left\{#1\right\}}
\newcommand{\comprehend}[2]{\setof{{#1} \;\middle|\; {#2}}}

\newcommand{\assign}{\ensuremath{\,{:=}\,}}

\newcommand{\arr}{\rightarrow}
\def\CompactJudgments{0}
\newcommand{\entails}{\mathrel{\ifnum\CompactJudgments=1\vdash \else \vdash\,\fi}}
\newcommand{\ctxoutsym}{\ifnum\CompactJudgments=1\dashv \else \,\dashv \fi}
\newcommand{\ctxout}[1]{\mathrel{\ctxoutsym}{#1}}

\newcommand{\J}{\mathcal{J}}
\newcommand{\judg}{\J}

\newcommand{\MonnierCommaSym}{{\smallblacktriangle}}
\newcommand{\MonnierComma}[1]{{\MonnierCommaSym}_{#1}}

\newcommand{\FV}[1]{\mathrm{FV}(#1)}
\newcommand{\xfev}{\mathsf{FEV}}
\newcommand{\fev}[1]{\xfev(#1)}
\newcommand{\FEV}[1]{\fev{#1}}
\newcommand{\xfsolev}{\mathsf{FSolEV}}
\newcommand{\fsolev}[2]{\xfsolev_{#1}(#2)}

\newcommand{\beeq}{=_{\beta\eta}}

\newcommand{\kindstar}{\star}
\newcommand{\kindnat}{\mathbb{N}}

\newcommand{\inversefn}[1]{{#1}^{-1}}

\newcommand{\emptysig}{\cdot}
\newcommand{\emptyctx}{\cdot}

\newcommand{\natzero}{\mathsf{zero}}
\newcommand{\xnatsucc}{\mathsf{succ}}
\newcommand{\natsucc}[1]{\xnatsucc\texttt{(}{#1}\texttt{)}}


\newcommand{\instantiate}[1]{{#1}\texttt{[-]}}
\newcommand{\quantify}[1]{\Lambda{#1}.\,}


\newcommand{\exvar}[1]{\widehat{#1}}
\newcommand{\exalpha}{\exvar{\alpha}}
\newcommand{\exbeta}{\exvar{\beta}}

\newcommand{\alln}[1]{\forall{#1}{:}\kindnat.\,}

\newcommand{\Match}[2]{{#1} \Rightarrow {#2}}
\newcommand{\matchor}{\ensuremath{\normalfont\,\texttt{|}\hspace{-5.35pt}\texttt{|}\,}}
\newcommand{\ind}[3]{\mathsf{ind}\texttt{(}\Match{\natzero}{#1}\texttt{,\;}\Match{\natsucc{#2}}{#3}
                     \texttt{)}}

\newcommand{\exunk}[2]{{#1} : {#2}}
\newcommand{\exsol}[3]{{#1} : {#2} \texttt{=} {#3}}
\newcommand{\exsolnokind}[2]{{#1} \texttt{=} {#2}}
\newcommand{\exsolwild}[2]{({#1} : {#2}\dots)}
\newcommand{\rexvar}[2]{\exsolnokind{#1}{#2}}

\newcommand{\tyname}[1]{\textsf{\normalfont #1}}
\newcommand{\unitexp}{\text{\normalfont \tt()}}
\newcommand{\unitty}{\tyname{1}}

\newcommand{\trientails}{\mathrel{{\rhd}\,}}
\newcommand{\trictxoutsym}{{\lhd}}
\newcommand{\trictxout}[1]{\mathrel{\trictxoutsym}{#1}}

\newcommand{\subtypingycolor}[1]{\textcolor{dDigPurple}{#1}}
\newcommand{\subtype}{\mathrel{\normalfont\texttt{\subtypingycolor{<:}}}}  \newcommand{\declsubtype}{\mathrel{\leq}}

\newcommand{\etal}{{et al.}}
\newcommand{\eg}{e.g.\ }
\newcommand{\ie}{i.e.\ }
\newcommand{\ala}{\`a la\ }
\newcommand{\Wlog}{w.l.o.g.\ }
\newcommand{\visavis}{vis-\`a-vis\ }
\newcommand{\Ocaml}{{OCaml}\xspace}
\newcommand{\OCaml}{\Ocaml}

\newcommand{\naive}{na\"ive\xspace}
\newcommand{\Backslash}{\char"5C}
\newcommand{\Lbrack}{\char"5B}
\newcommand{\Rbrack}{\char"5D}
\newcommand{\Lbrace}{\char"7B}
\newcommand{\Rbrace}{\char"7D}

\newcommand{\Appendixref}[1]{Appendix \ref{#1}}
\newcommand{\Figureref}[1]{Figure \ref{#1}}
\newcommand{\Figref}[1]{Fig.\ \ref{#1}}
\newcommand{\Sectionref}[1]{Section \ref{#1}}
\newcommand{\Secref}[1]{Sec.\ \ref{#1}}
\newcommand{\Chapterref}[1]{Chapter \ref{#1}}

\newcommand{\Listingref}[1]{Listing \ref{#1}}

\newcommand{\Theoremref}[1]{Theorem \ref{#1}}
\newcommand{\Thmref}[1]{Thm.\ \ref{#1}}
\newcommand{\Corollaryref}[1]{Corollary \ref{#1}}
\newcommand{\Corref}[1]{Cor.\ \ref{#1}}
\newcommand{\Lemmaref}[1]{Lemma \ref{#1} (\nameref{#1})}   \newcommand{\Lemref}[1]{\Lemma \ref{#1}}   \newcommand{\Conjectureref}[1]{Conjecture \ref{#1}}
\newcommand{\Propositionref}[1]{Proposition \ref{#1}}
\newcommand{\Propertyref}[1]{Property \ref{#1}}
\newcommand{\Remarkref}[1]{Remark \ref{#1}}
\newcommand{\Tableref}[1]{Table \ref{#1}}

\newcommand{\Definitionref}[1]{Definition \ref{#1}}
\newcommand{\Defnref}[1]{Def.\ \ref{#1}}

\newcommand{\ProofCaseRule}[1]{\item \textbf{Case }\textrm{{#1}}: ~ }
\newcommand{\ProofCaseThing}[1]{\ProofCaseRule{\ensuremath{#1}}}
\newcommand{\ProofCasesRules}[1]{\item \textbf{Cases }\textrm{{#1}}: ~ }
\newcommand{\ProofCaseRuleNoColon}[1]{\item \textbf{Case }\textrm{{#1}}}

\gdef\xxDerivationProofCaseColor{N}
\newcommand{\Begincolorcases}[1]{\gdef\xxDerivationProofCaseColor{#1}}
\newcommand{\Endcolorcases}{\gdef\xxDerivationProofCaseColor{N}}

\newcommand{\DerivationProofCase}[3]{\smallskip
     \item \parbox[t]{100ex}{\textbf{Case } \-0.8ex]
  }

\newcommand{\DoubleDerivationProofCase}[6]{\smallskip
     \item \parbox[t]{100ex}{\textbf{Case } \-0.8ex]
  }

\newcommand{\Dee}{\mathcal{D}}
\newcommand{\D}{\mathcal{\Dee}}

\newenvironment{displ}{\vspace{1pt} \begin{center} ~\!\!}{\end{center}}
\newenvironment{mathdispl}{\vspace{1pt} \begin{center} ~\!\!\end{center}}

\newenvironment{ctabular}{\renewcommand{\arraystretch}{1}\vspace{1pt}\begin{center} ~\!\!\begin{tabular}[t]{l}}{\end{tabular}\end{center}}

\newcommand{\arrayenv}[1]{\renewcommand{\arraystretch}{1} \begin{array}[t]{@{}c@{}}#1\end{array}}
\newcommand{\arrayenvc}[1]{\renewcommand{\arraystretch}{1} \begin{array}[c]{@{}c@{}}#1\end{array}}
\newcommand{\arrayenvcl}[1]{\renewcommand{\arraystretch}{1} \begin{array}[c]{@{}l@{}}#1\end{array}}
\newcommand{\arrayenvr}[1]{\renewcommand{\arraystretch}{1} \begin{array}[t]{@{}r@{}}#1\end{array}}
\newcommand{\arrayenvbr}[1]{\renewcommand{\arraystretch}{1} \begin{array}[b]{@{}r@{}}#1\end{array}}
\newcommand{\arrayenvl}[1]{\renewcommand{\arraystretch}{1} \begin{array}[t]{@{}l@{}}#1\end{array}}
\newcommand{\arrayenvb}[1]{\renewcommand{\arraystretch}{1}  \begin{array}[b]{@{}c@{}}#1\end{array}} 
\newcommand{\arrayenvbl}[1]{\renewcommand{\arraystretch}{1}  \begin{array}[b]{@{}l@{}}#1\end{array}}
\newcommand{\arrayenvblll}[1]{\renewcommand{\arraystretch}{1}  \begin{array}[b]{@{}lll@{}}#1\end{array}}
\newcommand{\pfarr}[1]{\begin{array}[b]{@{}l@{}}#1\end{array}}

\newcommand{\BeginProof}{\renewcommand{\arraystretch}{1.1} \begin{tabular}[b]{r@{}r @{} l  l}}
\newcommand{\EndProof}{\end{tabular} \renewcommand{\arraystretch}{\mydefaultarraystretch}}

\newcommand{\Hand}{\text{\Pointinghand~~~~}}

\newcommand{\Pf}[4] {& \, &  & #4 \\}
\newcommand{\Pfmrg}[3] {&\, &  & #3 \\}
\newcommand{\stepPf}[3] {\Pf{#1}{\,\step\,}{#2}{#3}}
\newcommand{\EPf}[3] {\Pf{#1}{\Entails}{#2}{#3}}
\newcommand{\LetPf}[3] {\Pf{\text{Let}\,~{#1}}{=\,}{#2\text{.}}{#3}}
\newcommand{\ForallPf}[3] {\Pf{\text{For all}\,~{#1}}{\in\,}{#2}{#3}}
\newcommand{\mkpf}[4] {\Pf{#2}{#1\,}{#3}{#4}}

\newcommand{\ePf}[3] {\mkpf{\entails}{#1}{#2}{#3}}
\newcommand{\eqPf}[3] {\mkpf{=}{#1}{#2}{#3}}
\newcommand{\continueeqPf}[2] {\mkpf{=}{~}{#1}{#2}}
\newcommand{\rightstarteqPf}[1] {\mkpf{~}{~}{#1}{~}}
\newcommand{\neqPf}[3] {\mkpf{\neq}{#1}{#2}{#3}}
\newcommand{\ltPf}[3] {\mkpf{<}{#1}{#2}{#3}}
\newcommand{\leqPf}[3] {\mkpf{\leq}{#1}{#2}{#3}}
\newcommand{\inPf}[3] {\mkpf{\in}{#1}{#2}{#3}}
\newcommand{\notinPf}[3] {\mkpf{\notin}{#1}{#2}{#3}}

\newcommand{\trailingjust}[1]{\Pf{}{}{}{~~{#1}}}
\newcommand{\derivesPf}[1]{}

\newcommand{\contraPf}[1] {\Pf{\Rightarrow\Leftarrow}{}{} {}\Pf{#1}{}{} {By contradiction}}

\newcommand{\NOTePf}[3] {\Pf{#1}{\not\entails\;}{#2}{#3}}
\newcommand{\proofsep}{\,\-10pt]
  \begin{enumerate}[(i)]
  \item  If  then .
  \item  If  then .
  \end{enumerate}
\end{restatable}

\begin{restatable}[Monotype Equality]{lemma}{declmonotypeequality}
  \XLabel{lem:decl-monotype-equality}
   If  then .
\end{restatable}













\begin{definition}[Contextual Size] \XLabel{def:typesize}
  The size of  with respect to a context , written , is defined by
  \begin{mathpar}
    \begin{array}{lcl}
      \typesize{\Gamma}{\alpha}  & = &   1
      \\
      \typesize{{\Gamma[\ahat]}}{\ahat}  & = &   1
      \\
      \typesize{{\Gamma[\hypeq{\ahat}{\tau}]}}{\ahat} & = &   1 + \typesize{\Gamma[\hypeq{\ahat}{\tau}]}{\tau} 
      \\
      \typesize{\Gamma}{\alltype{\alpha}{A}} & = &   1 + \typesize{\Gamma, \alpha}{A}
      \\
      \typesize{\Gamma}{A \arr B} & = &   1 + \typesize{\Gamma}{A} + \typesize{\Gamma}{B}
    \end{array}
  \end{mathpar}  
\end{definition}










\section{Type Assignment}



\begin{restatable}[Well-Formedness]{lemma}{declarativetypingwellformed}
\XLabel{lem:declarative-typing-well-formed}
~\\
  If 
  or 
  or  then
   (and in the last case, ).
\end{restatable}

\begin{restatable}[Completeness of Bidirectional Typing]{theorem}{completenessbidirectional}
  \XLabel{thm:completeness-bidirectional}
~\\
  If  then there exists  such that 
  and . 
\end{restatable}

\begin{restatable}[Subtyping Coercion]{lemma}{subtypingcoercion}
\XLabel{lem:subtyping-coercion}
  If  then there exists  which is -equal to the identity such that 
  . 
\end{restatable}

\begin{restatable}[Application Subtyping]{lemma}{applicationsubtyping}
\XLabel{lem:application-subtyping}
  If 
  then there exists  such that 
  and  by a smaller derivation.
\end{restatable}

\begin{restatable}[Soundness of Bidirectional Typing]{theorem}{soundnessbidirectional}
  \XLabel{thm:soundness-bidirectional}
  We have that:

  \begin{itemize}
  \item  If , then there is an  such that 
    and .
  \item  If , then there is an  such that 
    and .
  \end{itemize}
\end{restatable}



\section{Robustness of Typing}

\begin{restatable}[Type Substitution]{lemma}{typesubstitution}
\XLabel{lem:type-substitution}
~\\
  Assume .

  \begin{itemize}
  \item If\,  then\, .
  \item If\,  then\, .
  \item If\,  then\, .
  \end{itemize}   

  Moreover, the resulting derivation contains no more applications of typing rules than the given one.
  (Internal subtyping derivations, however, may grow.)
\end{restatable}

\bigskip

\begin{definition}[Context Subtyping]
\XLabel{def:context-subtyping}
    We define the judgment  with the following rules:
    \-3ex]
  \begin{enumerate}[(i)]
  \item If  then .
  \item If  then .
  \item If  then .
\end{enumerate}
\end{restatable}

\begin{restatable}[Inverse Substitution]{theorem}{termUnsubstitution}
\XLabel{thm:term-unsubstitution}
~\\
Assume .
\begin{enumerate}[(i)]
  \item If  then .
  \item If  then .
  \item If  then .
\end{enumerate}
\end{restatable}


\begin{restatable}[Annotation Removal]{theorem}{annotationRemoval}
\XLabel{thm:annotation-dropping}
  We have that:

  \begin{itemize}
    \item If\,  then .
    \item If\,  then .
    \item If\,  then .
    \item If\,  then  and .
    \item If\, 
       then 
      and . 
    \item If\, 
      then 
      and . 
    \item If 
      then .
  \end{itemize}
\end{restatable}


\begin{restatable}[Soundness of Eta]{theorem}{soundnessofeta}
~\\\Label{thm:soundness-of-eta}
  If\,  and , then .   
\end{restatable}



\section{Properties of Context Extension}




\subsection{Syntactic Properties}

\begin{restatable}[Declaration Preservation]{lemma}{declarationpreservation}
\XLabel{lem:declaration-preservation}
  If ,
  and  is a variable or marker  declared in , then  is declared in . 
\end{restatable}

\begin{restatable}[Declaration Order Preservation]{lemma}{declarationorderpreservation}
\XLabel{lem:declaration-order-preservation}
  If  
  and  is declared to the left of  in ,
  then  is declared to the left of  in . 
\end{restatable}

\begin{restatable}[Reverse Declaration Order Preservation]{lemma}{reversedeclarationorderpreservation}
\XLabel{lem:reverse-declaration-order-preservation}
  If  and  and  are both declared in 
  and  is declared to the left of  in ,
  then  is declared to the left of  in . 
\end{restatable}


\begin{restatable}[Substitution Extension Invariance]{lemma}{substitutionextensioninvariance}
\XLabel{lem:subst-extension-invariance}
   If  and 
   then
   
   and .
\end{restatable}

\begin{restatable}[Extension Equality Preservation]{lemma}{extensionequalitypreservation}
\XLabel{lem:extension-equality-preservation}
~\\
   If 
   and  
   and 
   and ,
   then 
   .   
\end{restatable}


\begin{restatable}[Reflexivity]{lemma}{substextendreflexivity}
\XLabel{lem:substextend-reflexivity}
  If  is well-formed, then . 
\end{restatable}



\begin{restatable}[Transitivity]{lemma}{substextendtransitivity}
\XLabel{lem:substextend-transitivity}
  If 
  and ,
  then .
\end{restatable}




\begin{definition}[Softness]  \XLabel{def:soft}
  A context  is \emph{soft} iff it consists only of  and 
  declarations.
\end{definition}

\begin{restatable}[Right Softness]{lemma}{rightsoftness}  \XLabel{lem:softness}
  If  and  is soft
  (and  is well-formed)
  then .
\end{restatable}




\begin{restatable}[Evar Input]{lemma}{evarinput}  \XLabel{lem:evar-input}
~\\
  If 
  then  where
  ,
  and  is either  or ,
  and  is soft.
\end{restatable}









\begin{restatable}[Extension Order]{lemma}{extensionorder} 
\XLabel{lem:extension-order}
~\-2ex]
  \begin{enumerate}[(i)]
      \item \emph{Synthesis:}  
        Given a context 
        and a term ,
        \\
        it is decidable whether there exist a type  and a context  such that \\
        .

      \item \emph{Checking:}
        Given a context ,
        a term ,
        and a type  such that ,
        \\
        it is decidable whether there is a context  such that \\
        .

      \item \emph{Application:}  
        Given a context ,
        a term ,
        and a type  such that ,
        \\
        it is decidable whether there exist a type  and a context  such that \\
        .
  \end{enumerate}
\end{restatable}



\section{Soundness of Subtyping}





\begin{definition}[Filling]
  The \emph{filling} of a context  solves all unsolved variables:

  \begin{displ}
        \begin{array}[t]{lcll}
            \soln{\cdot} & = &   \cdot
            \\
            \soln{\Gamma, x : A} & = &   \soln{\Gamma}, x : A
            \\
            \soln{\Gamma, \alpha} & = &   \soln{\Gamma}, \alpha
            \\
            \soln{\Gamma, \hypeq{\ahat}{\tau}} & = &   \soln{\Gamma}, \hypeq{\ahat}{\tau}
            \\
            \soln{\Gamma, \MonnierComma{\ahat}} & = &   \soln{\Gamma}, \MonnierComma{\ahat}
            \\
            \soln{\Gamma, \ahat} & = &   \soln{\Gamma}, \hypeq{\ahat}{\unitty}
        \end{array}
  \end{displ}
\end{definition}


\subsection{Lemmas for Soundness}

\begin{restatable}[Uvar Preservation]{lemma}{ctxappuvarpreservation}
\XLabel{lem:ctxapp-uvar-preservation} ~\\
  If 
  and 
  then
  .
\end{restatable}
\begin{proof}  By induction on , following the definition of context application.
\end{proof}


\begin{restatable}[Variable Preservation]{lemma}{ctxappvarpreservation}
\XLabel{lem:ctxapp-var-preservation} ~\\
  If  or 
  and 
  then
  .
\end{restatable}


\begin{restatable}[Substitution Typing]{lemma}{substitutiontyping}
\XLabel{lem:substitution-typing}
  If  then . 
\end{restatable}






\begin{restatable}[Substitution for Well-Formedness]{lemma}{completionwf}
\XLabel{lem:completion-wf}
  If 
  then
  .
\end{restatable}


\begin{restatable}[Substitution Stability]{lemma}{substitutionstability}   \XLabel{lem:substitution-stability}
~\\
  For any well-formed complete context ,
  if  then .
\end{restatable}

\begin{restatable}[Context Partitioning]{lemma}{contextpartitioning}   \XLabel{lem:context-partitioning}
  ~\\
  If 
  then
  there is a  such that
  . 
\end{restatable}

\begin{restatable}[Softness Goes Away]{lemma}{softnessgoesaway}  \XLabel{lem:softness-goes-away}
  ~\\
  If 
  where  and  is soft,
  then
  .
\end{restatable}
\begin{proof}
  By induction on , following the definition of .
\end{proof}








\begin{lemma}[Filling Completes]  \XLabel{lem:soln-completes}
  If 
  and  is well-formed,
  then
  .
\end{lemma}
\begin{proof}
  By induction on , following the definition of  and applying the rules
  for .
\end{proof}








\begin{restatable}[Stability of Complete Contexts]{lemma}{completesstability}
  \XLabel{lem:completes-stability}  ~\\
  If 
  then . 
\end{restatable}

\begin{restatable}[Finishing Types]{lemma}{finishingtypes}
  \XLabel{lem:finishing-types} ~\\
  If

and 
  then
  . 
\end{restatable} 

\begin{restatable}[Finishing Completions]{lemma}{finishingcompletions}
  \XLabel{lem:finishing-completions}  ~\\
  If 
  then
  . 
\end{restatable}


\begin{restatable}[Confluence of Completeness]{lemma}{confluenceofcompleteness}    \XLabel{lem:completes-confluence}  ~\\
  If 
  and 
  then
  .
\end{restatable}



\subsection{Instantiation Soundness}

\begin{restatable}[Instantiation Soundness]{theorem}{instantiationsoundness}  \XLabel{thm:instantiation-soundness}
   ~\\
    Given  and  and : \-2.5ex]
\begin{enumerate}[(i)]
      \item 
          If 
then
          . 
      
      \item
          If 
then
          . 

      \item
          If 
then
          .
    \end{enumerate}
\end{restatable}



\section{Completeness of Subtyping}





\subsection{Instantiation Completeness}

\begin{restatable}[Instantiation Completeness]{theorem}{instantiationcompletes}
   \XLabel{thm:instantiation-completes}
   ~\\
   Given  
   and  
   and   and :
   
       \begin{enumerate}[(1)]
           \item If 
                 \\
                 then there are ,  such that
                 
                 and 
                 and . 

           \item If 
                 \\
                 then there are ,  such that
                 
                 and 
                 and . 
      \end{enumerate}                                                    
\end{restatable}





\subsection{Completeness of Subtyping}

\begin{restatable}[Generalized Completeness of Subtyping]{theorem}{completingcompleteness}
  \XLabel{thm:completing-completeness}
  If 
  and 
  and 
  and 
  then
  there exist  and  such that 
   and 
   and 
  .
\end{restatable}


\begin{restatable}[Completeness of Subtyping]{corollary}{completeness}
  If 
  then
  there is a  such that
  . 
\end{restatable}





\section{Completeness of Typing}
\Label{sec:last-section-before-proofs}


\begin{restatable}[Completeness of Algorithmic Typing]{theorem}{typingcompleteness}
  \XLabel{thm:typing-completeness}
  Given  and :
  \
      \left\langle
        \#\forall(B),~~~
\Dee_1 + \Dee_2
      \right\rangle
    
                         \declsynjudg{\Psi}{[(e:A)/x]e'_1}{B}
                         \AND
                         \declappjudg{\Psi}{[(e:A)/x]e'_2}{B}{C}
                      -1.8em]
        \begin{itemize}
           \DerivationProofCase{\DeclAllApp}
                  {\judgetp{\Psi}{\tau} \\ 
                   \declappjudg{\Psi}{[(e:A)/x]e'}{[\tau/\alpha]B}{C}}
                  {\declappjudg{\Psi}{[(e:A)/x]e'}{\alltype{\alpha}{B}}{C} }

              Follows by i.h.\ (iii) and \DeclAllApp.

           \DerivationProofCase{\DeclArrApp}
                  {\declchkjudg{\Psi}{[(e:A)/x]e'}{B}}
                  {\declappjudg{\Psi}{[(e:A)/x]e'}{B \arr C}{C}}

              Follows by i.h.\ (i) and \DeclArrApp.
        \qedhere
        \end{itemize}
  \end{enumerate}
\end{proof}


\annotationRemoval*
\begin{proof}
  All of these follow directly from inversion and \Lemmaref{lem:term-subsumption}. The one exception is the
  third, which additionally requires a small induction on the application judgment. 
\end{proof}

\soundnessofeta*
\begin{proof}
  By induction on the derivation of . There are three
  non-impossible cases: 
  \begin{itemize}
       \DerivationProofCase{\DeclArrIntro}
          {\declchkjudg{\Psi, x : B}{e\;x}{C}
          }
          {\declchkjudg{\Psi}{\lam{x} e\;x}{B \arr C}}

        We have . \\ 
        By inversion on \DeclSub, we get  and . \\ 
        By inversion on \DeclArrElim, we get  and . \\ 
        By thinning, we know that . \\
        By \Lemmaref{lem:application-subtyping}, we get  so 
        and . \\ 
        By inversion, we know that  and . \\
        By \DsubArr, . \\ 
        Hence by \Lemmaref{lem:declarative-transitivity}, . \\
        Hence . \\
        By \DeclSub, . 


       \DerivationProofCase{\DeclAllIntro}
           {\declchkjudg{\Psi, \alpha}{\fun{x}{e\;x}}{B}
           }
           {\declchkjudg{\Psi}{\fun{x}{e\;x}}{\alltype{\alpha}{B}}}

         By induction, . \\ 
         By \DeclAllIntro, . 

       
       \DerivationProofCase{\DeclSub}
          {\declsynjudg{\Psi}{\fun{x}{e\;x}}{B}
            \\
            \declsubjudg{\Psi}{B}{A}
          }
          {\declchkjudg{\Psi}{\fun{x}{e\;x}}{A}}

          We have  and . \\ 
          By inversion on \DeclArrIntroSyn,  and .\\
          By inversion on \DeclSub, we get  and . \\ 
          By inversion on \DeclArrElim, we get  and . \\ 
          By thinning, we know that . \\
          By \Lemmaref{lem:application-subtyping},
          we get  such that 
          and . \\ 
          By inversion on \DeclSub,  and . \\
          By \DsubArr, . \\ 
          Hence by \Lemmaref{lem:declarative-transitivity}, . \\
          Hence . \\
          Hence by \Lemmaref{lem:declarative-transitivity}, . \\
          By \DeclSub, . 
  \qedhere
  \end{itemize}
\end{proof}

\section{Properties of Context Extension}

\subsection{Syntactic Properties}

\declarationpreservation* 
\begin{proof}
  By a routine induction on . 
\end{proof}


\declarationorderpreservation*
\begin{proof}
    By induction on the derivation of . 
    
    \begin{itemize}
        \DerivationProofCase{\substextendId}
            { }
            { \substextend{\emptyctx}{\emptyctx} }

           This case is impossible. 

        \DerivationProofCase{\substextendVV}
              {\substextend{\Gamma}{\Delta}}
              {\substextend{\Gamma, x:A}{\Delta, x:A}}

            There are two cases, depending on whether or not . 
            \begin{itemize}
            \item Case : \\
              Since  is declared to the left of ,  is declared in . \\
              By \Lemmaref{lem:declaration-preservation},  is declared in . \\
              Hence  is declared to the left of  in .

            \item Case : \\
              Then  is declared in , and  is declared to the left of  in .  \\
              By induction,  is declared to the left of  in . \\
              Hence  is declared to the left of  in . 
            \end{itemize}

        \DerivationProofCase{\substextendUU}
            { \substextend{\Gamma}{\Delta} }
            { \substextend{\Gamma, \alpha}{\Delta, \alpha} }

            This case is similar to the \substextendVV case. 

        \DerivationProofCase{\substextendEE}
            { \substextend{\Gamma}{\Delta} }
            { \substextend{\Gamma, \ahat}{\Delta, \ahat} }

            This case is similar to the \substextendVV case. 

        \DerivationProofCase{\substextendSolSol}
            { \substextend{\Gamma}{\Delta} \\ [\Delta]\tau = [\Delta]\tau'}
            { \substextend{\Gamma, \hypeq{\ahat}{\tau}}{\Delta, \hypeq{\ahat}{\tau'}} }

            This case is similar to the \substextendVV case. 

        \DerivationProofCase{\substextendMonMon}
            { \substextend{\Gamma}{\Delta} }
            { \substextend{\Gamma, \MonnierComma{\ahat}}{\Delta, \MonnierComma{\ahat}} }

            This case is similar to the \substextendVV case. 

        \DerivationProofCase{\substextendSolve}
            { \substextend{\Gamma}{\Delta} }
            { \substextend{\Gamma, \ahat}{\Delta, \hypeq{\ahat}{\tau}} }

            This case is similar to the \substextendVV case. 

        \DerivationProofCase{\substextendAdd}
            { \substextend{\Gamma}{\Delta} }
            { \substextend{\Gamma}{\Delta, \ahat} }

            By induction,  is declared to the left of  in . \\
            Therefore  is declared to the left of  in .

        \DerivationProofCase{\substextendAddSolved}
            { \substextend{\Gamma}{\Delta} }
            { \substextend{\Gamma}{\Delta, \hypeq{\ahat}{\tau}} }

            By induction,  is declared to the left of  in . \\
            Therefore  is declared to the left of  in .
    \qedhere
    \end{itemize}
\end{proof}

\reversedeclarationorderpreservation*
\begin{proof}
  It is given that  and  are declared in .  Either  is declared to the left of 
  in , or  is declared to the left of .  Suppose the latter (for a contradiction).
  By \Lemmaref{lem:declaration-order-preservation},  is declared to
  the left of  in .  But we know that  is declared to the left of  in :
  contradiction.   Therefore  is declared to the left of  in .
\end{proof}

\substitutionextensioninvariance*
\begin{proof}
  To show that , observe that ,
  and that by definition of ,
  every solved variable in  is solved in .
  Therefore ,
  since  contains no variables that  solves. \\

  To show that , we proceed by induction on . 
  
  \begin{itemize}
      \DerivationProofCase{}
                          {\alpha \in \Theta}
                          {\judgetp{\Theta}{\alpha}}

          Note that , so . 

      \DerivationProofCase{}
                          {\judgetp{\Theta}{A} \\ \judgetp{\Theta}{B}}
                          {\judgetp{\Theta}{A \arr B}}

          By induction, . \\
          By induction, . \\
          Then
                            

      \DerivationProofCase{}
                          {\judgetp{\Theta, \alpha}{A}}
                          {\judgetp{\Theta}{\alltype{\alpha}{A}}}

          By inversion, we have . \\
          By rule \substextendUU, . \\
          By induction, . \\
          By definition, .  \\
          Then 
                            

      \DerivationProofCase{}
                          { }
                          {\judgetp{\underbrace{\Theta_0, \ahat, \Theta_1}_{\Theta}}{\ahat}}

           Note that . \\
           Hence . 


      \DerivationProofCase{}
                          {}
                          {\judgetp{\Theta_0, \ahat=\tau, \Theta_1}{\ahat}}


          From , 
          By a nested induction we get , 
                                   and . \\
          Note that . \\
          By induction, . \\
          Hence 
          
\end{itemize}
\end{proof}

\extensionequalitypreservation*
\begin{proof}
  By induction on the derivation of . 

    \begin{itemize}
      \DerivationProofCase{\substextendId}
            { }
            { \substextend{\underbrace{\emptyctx}_{\Gamma}}{\underbrace{\emptyctx}_{\Delta}} }

            We have , but , so  .

      \DerivationProofCase{\substextendVV}
              {\substextend{\Gamma'}{\Delta'}}
              {\substextend{\Gamma', x:C}{\Delta', x:C}}

          We have . \\
          By definition of substitution, . \\
          By i.h., . \\
          By definition of substitution, . 

      \DerivationProofCase{\substextendUU}
            { \substextend{\Gamma'}{\Delta'} }
            { \substextend{\Gamma', \alpha}{\Delta', \alpha} }

          We have . \\
          By definition of substitution, . \\
          By i.h., . \\
          By definition of substitution, . 

      \DerivationProofCase{\substextendEE}
            { \substextend{\Gamma'}{\Delta'} }
            { \substextend{\Gamma', \ahat}{\Delta', \ahat} }
            
            Similar to the \substextendUU case.


      \DerivationProofCase{\substextendMonMon}
            { \substextend{\Gamma'}{\Delta'} }
            { \substextend{\Gamma', \MonnierComma{\ahat}}{\Delta', \MonnierComma{\ahat}} }

            Similar to the \substextendUU case.


      \DerivationProofCase{\substextendAdd}
            { \substextend{\Gamma}{\Delta'} }
            { \substextend{\Gamma}{\Delta', \ahat} }

          We have . \\
          By i.h., . \\
          By definition of substitution, .

      \DerivationProofCase{\substextendAddSolved}
            { \substextend{\Gamma}{\Delta'} }
            { \substextend{\Gamma}{\Delta', \hypeq{\ahat}{\tau}} }

          We have . \\
          By i.h., . \\
          We implicitly assume that  is well-formed,
          so . \\
          Since  and ,
          it follows that . \\
          We have  and , so
          . \\
          Therefore, by definition of substitution, .

      \DerivationProofCase{\substextendSolSol}
            { \substextend{\Gamma'}{\Delta'} \\ [\Delta']\tau = [\Delta']\tau'}
            { \substextend{\Gamma', \hypeq{\ahat}{\tau}}{\Delta', \hypeq{\ahat}{\tau'}} }

          We have . \\
          
          By definition, ,
          but we implicitly assume that  is well-formed, so ,
          so actually . \\
          Combined with similar reasoning for , we get
          
          By i.h., . \\
          By distributivity of substitution,
          . \\
          Using the premise ,
          we get . \\
          By distributivity of substitution (in the other direction), . \\
          It follows from the definition of substitution that
          . 

      \DerivationProofCase{\substextendSolve}
            { \substextend{\Gamma'}{\Delta'} }
            { \substextend{\Gamma', \ahat}{\Delta', \hypeq{\ahat}{\tau}} }

          We have . \\
          By definition of substitution, . \\
          By i.h., . \\
          It follows from the definition of substitution that
          . 
    \qedhere
    \end{itemize}
\end{proof}


\substextendreflexivity*
\begin{proof}
  By induction on the structure of . 

    \begin{itemize}
        \ProofCaseRule{} 
            Apply rule \substextendId. 

        \ProofCaseRule{}
            By i.h., .
            By rule \substextendUU, we get . 

        \ProofCaseRule{}
            By i.h., .
            By rule \substextendEE, we get . 

        \ProofCaseRule{}

            By i.h., . \\
            Clearly, , so
            we can apply \substextendSolSol to get . 

        \ProofCaseRule{}
            By i.h., .
            By rule \substextendMonMon, we get .
\qedhere
    \end{itemize}
\end{proof}



\substextendtransitivity*
\begin{proof}
  By induction on the derivation of .

  \begin{itemize}
  \ProofCaseRule{\substextendId}
  
      In this case . \\
      Hence  suffices. 

  \DerivationProofCase{\substextendUU}
                      {\substextend{\Delta'}{\Theta'}}
                      {\substextend{\Delta', \alpha}{\Theta', \alpha}}

      We have  and . \\
      By inversion on ,
      we have  and . \\
      By i.h., . \\
      Applying rule \substextendUU gives . 

  \DerivationProofCase{\substextendUU}
                      {\substextend{\Delta'}{\Theta'}}
                      {\substextend{\Delta', \ahat}{\Theta', \ahat}}

      We have  and . \\
      Either of two rules could have derived :

      \begin{itemize}
          \DerivationProofCase{\substextendEE}
                              {\substextend{\Gamma'}{\Delta'}}
                              {\substextend{\Gamma', \ahat}{\Delta', \ahat}}

            Here we have  and . \\
            By i.h., . \\
            Applying rule \substextendEE gives . 

          \DerivationProofCase{\substextendAdd}
                              {\substextend{\Gamma}{\Delta'}}
                              {\substextend{\Gamma}{\Delta', \ahat}}

            By i.h., . \\
            By rule \substextendAdd, we get . 
      \end{itemize}

  \DerivationProofCase{\substextendSolSol}
                      {\substextend{\Delta'}{\Theta'} \\ [\Theta']\tau_1 = [\Theta']\tau_2}
                      {\substextend{\Delta', \ahat=\tau_1}{\Theta', \ahat=\tau_2}}

      In this case  and . \\
      One of three rules must have derived :

          \begin{itemize}
              \DerivationProofCase{\substextendSolSol}
                                  {\substextend{\Gamma'}{\Delta'} \\ [\Delta']\tau_0 = [\Delta']\tau_1}
                                  {\substextend{\Gamma', \ahat=\tau_0}{\Delta', \ahat=\tau_1}}

                    Here,  and . \\
                    By i.h., we have . \\
                    The premises of the respective  derivations give us
                        and . \\
                    We know that  and  and . \\ 
                    By extension weakening (\Lemmaref{lem:extension-weakening}), . \\ 
                    By extension weakening (\Lemmaref{lem:extension-weakening}), . \\
                    Since , we know that . \\ 
                    By \Lemmaref{lem:subst-extension-invariance}, . \\
                    By \Lemmaref{lem:subst-extension-invariance}, . \\
                    So . \\ 

                    Hence by transitivity of equality, . \\
                    By rule \substextendSolSol,  . 

              \DerivationProofCase{\substextendAddSolved}
                                  {\substextend{\Gamma}{\Delta'}}
                                  {\substextend{\Gamma}{\Delta', \ahat=\tau_1}}

                    By induction, we have . \\
                    By rule \substextendAddSolved, we get . 

              \DerivationProofCase{\substextendSolve}
                                  {\substextend{\Gamma'}{\Delta'} }
                                  {\substextend{\Gamma', \ahat}{\Delta', \ahat=\tau_1}}

                    We have . \\
                    By induction, . \\
                    By rule \substextendSolve, we get .
          \end{itemize}
          
  \DerivationProofCase{\substextendMonMon}
                      {\substextend{\Delta'}{\Theta'}}
                      {\substextend{\Delta', \MonnierComma{\ahat}}{\Theta', \MonnierComma{\ahat}}}

      In this case we know  and . \\
      Since , only \substextendMonMon could derive
      ,
      so by inversion,
        
        and . \\
      By induction, we have . \\
      Applying rule \substextendMonMon gives 
      . 

  \DerivationProofCase{\substextendAdd}
                      {\substextend{\Delta}{\Theta'}}
                      {\substextend{\Delta}{\Theta', \ahat}}

      In this case, we have . \\
      By induction, we get . \\
      By rule \substextendAdd, we get . 

  \DerivationProofCase{\substextendAddSolved}
                      {\substextend{\Delta}{\Theta'}}
                      {\substextend{\Delta}{\Theta', \ahat=\tau}}

      In this case, we have . \\
      By induction, we get . \\
      By rule \substextendAddSolved, we get . 

  \DerivationProofCase{\substextendSolve}
                      {\substextend{\Delta'}{\Theta'}}
                      {\substextend{\Delta', \ahat}{\Theta', \ahat=\tau}}

      In this case, we have  and . \\
      One of two rules could have derived :

          \begin{itemize}
              \DerivationProofCase{\substextendEE}
                                  {\substextend{\Gamma'}{\Delta'}}
                                  {\substextend{\Gamma',\ahat}{\Delta',\ahat}}

                   In this case, we have 
                     and 
                     and . \\
                   By induction, we have . \\
                   By rule \substextendSolve, we get . 

               \DerivationProofCase{\substextendAdd}
                                   {\substextend{\Gamma}{\Delta'}}
                                   {\substextend{\Gamma}{\Delta',\ahat}}

                   In this case, we have  and . \\
                   By induction, we have . \\
                   By rule \substextendSolve, we get . 
           \qedhere
\end{itemize}
\end{itemize}
\end{proof}


\rightsoftness*
\begin{proof}
  By induction on , applying rules \substextendAdd and \substextendAddSolved as needed.
\end{proof}






\evarinput*
\begin{proof}
  By induction on the given derivation.

  \begin{itemize}
      \ProofCasesRules{\substextendId, \substextendVV, \substextendUU, \substextendSolSol, \substextendMonMon}
          \\
          Impossible: the left-hand context cannot have the form .

      \DerivationProofCase{\substextendEE}
           {\substextend{\Gamma}{\Delta_0}}
           {\substextend{\Gamma, \ahat}{\underbrace{\Delta_0, \ahat}_{\Delta}}}

           Let , which is vacuously soft.
           Therefore ;
           the subderivation is the rest of the result.

      \DerivationProofCase{\substextendSolve}
          { \substextend{\Gamma}{\Delta_0} }
          { \substextend{\Gamma, \ahat}{\underbrace{\Delta_0, \hypeq{\ahat}{\tau}}_{\Delta}} }

          Let , which is vacuously soft.
          Therefore ;
          the subderivation is the rest of the result.

      \DerivationProofCase{\substextendAdd}
           {\substextend{\Gamma, \ahat}{\Delta_0}}
           {\substextend{\Gamma, \ahat}{\underbrace{\Delta_0, \bhat}_{\Delta}}}

           Suppose . \\
           ~~We have .  By \Lemmaref{lem:declaration-preservation},
            is declared in . \\
           ~~But then  with multiple  declarations, \\
           ~~which violates the implicit assumption that  is well-formed.  Contradiction. \\
           Therefore .

           By i.h.,  where 
           and  is soft.
           
           Let .
           Therefore .
           As  is soft,  is soft.
           Since , this gives .

      \ProofCaseRule{\substextendAddSolved}
          Similar to the case for \substextendAdd.
\qedhere
  \end{itemize}
\end{proof}



\extensionorder*
\begin{proof}
  \begin{enumerate}[(i)]
  \item 
    By induction on the derivation of .

      \begin{itemize}
        \DerivationProofCase{\substextendId}
            { }
            { \substextend{\emptyctx}{\emptyctx} }

            This case is impossible since  cannot have the form .

       \ProofCasesRules{\substextendUU}

           We have two cases, depending on whether or not the rightmost variable is .

           \begin{itemize}
            \DerivationProofCase{\substextendUU}
                { \substextend{\Gamma}{\Delta'} }
                { \substextend{\Gamma, \alpha}{\Delta', \alpha} }

                Let , and let  (which is soft). \\
                We have , which is
                .
             
            \DerivationProofCase{\substextendUU}
                { \substextend{\Gamma_L, \alpha, \Gamma'_R}{\Delta'} }
                { \substextend{\Gamma_L, \alpha, \underbrace{\Gamma'_R, \beta}_{\Gamma_R}}
                  {\underbrace{\Delta', \beta}_{\Delta}} }

            By i.h., 
              where . \\
            Hence . \\
            (Since , it cannot be the case that  is soft.)
          \end{itemize}

    \DerivationProofCase{\substextendVV}
              {\substextend{\Gamma_L, \alpha, \Gamma'_R}{\Delta'}}
              {\substextend{\Gamma_L, \alpha, \underbrace{\Gamma'_R, x:A}_{\Gamma_R}}
                 {\underbrace{\Delta', x:A}_{\Delta}}}

            By i.h., 
              where . \\
            Hence . \\
            (Since , it cannot be the case that  is soft.)

        \DerivationProofCase{\substextendEE}
            { \substextend{\Gamma_L, \alpha, \Gamma'_R}{\Delta'} }
            { \substextend{\Gamma_L, \alpha, \underbrace{\Gamma'_R, \ahat}_{\Gamma_R}}
                {\underbrace{\Delta', \ahat}_{\Delta}} }

            By i.h., 
              where . \\
            Hence .  \\
            (If  is soft, by i.h.  is soft, so  is soft.)

        \DerivationProofCase{\substextendMonMon}
            { \substextend{\Gamma_L, \alpha, \Gamma'_R}{\Delta'} }
            { \substextend{\Gamma_L, \alpha,
                 \underbrace{\Gamma'_R, \MonnierComma{\bhat}}_{\Gamma'_R}}
                 {\underbrace{\Delta', \MonnierComma{\bhat}}_{\Delta}} }

            By i.h., 
              where . \\
            Hence .  \\
            (Since , it cannot be the case that  is soft.)

        \DerivationProofCase{\substextendSolSol}
            { \substextend{\Gamma_L, \alpha, \Gamma'_R}{\Delta'} \\ [\Delta']\tau = [\Delta']\tau' }
            { \substextend{\Gamma_L, \alpha, \underbrace{\Gamma'_R, \hypeq{\ahat}{\tau}}_{\Gamma_R}}
                          {\underbrace{\Delta', \hypeq{\ahat}{\tau'}}_{\Delta'}} }

            By i.h., 
              where . \\
            Hence . \\
            (If  is soft, by i.h.  is soft, so  is soft.)

        \DerivationProofCase{\substextendSolve}
            { \substextend{\Gamma_L, \alpha, \Gamma'_R}{\Delta'} }
            { \substextend{\Gamma_L, \alpha, \underbrace{\Gamma'_R, \ahat}_{\Gamma_R}}
                          {\underbrace{\Delta', \hypeq{\ahat}{\tau'}}_{\Delta}} }

            By i.h., 
              where . \\
            Therefore . \\
            (If  is soft, by i.h.  is soft, so  is soft.)

        \DerivationProofCase{\substextendAdd}
            { \substextend{\Gamma_L, \alpha, \Gamma_R}{\Delta'} }
            { \substextend{\Gamma_L, \alpha, \Gamma_R}{\underbrace{\Delta', \ahat}_{\Delta}} }

            By i.h., 
              where . \\
            Therefore . \\
            (If  is soft, by i.h.  is soft, so  is soft.)

        \DerivationProofCase{\substextendAddSolved}
            { \substextend{\Gamma_L, \alpha, \Gamma_R}{\Delta'} }
            { \substextend{\Gamma_L, \alpha, \Gamma_R}{\Delta', \hypeq{\ahat}{\tau}} }

            In this case, we know that  . \\
            By i.h., 
              where . \\
            Hence . \\
            (If  is soft, by i.h.  is soft, so  is soft.)
      \end{itemize}

    \item Similar to the proof of (i), except that the \substextendMonMon and \substextendUU
      cases are swapped.

    \item Similar to (i), with  in the \substextendEE case and
       in the \substextendSolve case.

    \item Similar to (iii).

    \item Similar to (i), but using the equality premise of \substextendVV.
  \qedhere
  \end{enumerate}
\end{proof}


\extensionweakening*
\begin{proof}
  By a straightforward induction on .

  In the \UvarWF case,
  we use \Lemmaref{lem:extension-order} (i).  In the \EvarWF case,
  use \Lemmaref{lem:extension-order} (iii).  In the \SolvedEvarWF case,
  use \Lemmaref{lem:extension-order} (iv).

  In the other cases, apply the i.h. to all subderivations, then apply the rule.
\end{proof}


\extensionsolve*
\begin{proof}
  By induction on .

  \begin{itemize}
    \item Case : 
      \\
      By \Lemmaref{lem:substextend-reflexivity} (reflexivity), . \\
      Applying rule \substextendSolve gives . 

    \item Case : 
      
      By i.h., . \\
      Applying rule \substextendVV gives
      . 

    \item Case :  By i.h.\ and rule \substextendUU.
    \item Case : By i.h.\ and rule \substextendAdd.
    \item Case : By i.h.\ and rule \substextendAddSolved.
    \item Case :  By i.h.\ and rule \substextendMonMon.
  \qedhere
  \end{itemize}
\end{proof}

\extensionaddsolve*
\begin{proof}
  By induction on .
  The proof is exactly the same as the proof of \Lemmaref{lem:extension-solve},
  except that in the , we apply rule \substextendAddSolved instead of \substextendSolve.
\end{proof}


\extensionadd*
\begin{proof}  By induction on .  The proof is exactly the same as the proof of
  \Lemmaref{lem:extension-solve}, except that in the  case, we apply
    rule \substextendAdd instead of \substextendSolve.
\end{proof}



\parallelevaradmissibility*
\begin{proof}
  By induction on .  As always, we assume that all contexts mentioned in the
  statement of the lemma are well-formed.  Hence, .
  
  \begin{enumerate}[(i)]
  \item We proceed by cases of .  Observe that in all the extension rules, the right-hand context
    gets smaller, so as we enter subderivations of
    , the context  becomes smaller.
    
    The only tricky part of the proof is that to apply the i.h., we need .
    So we need to make sure that as we drop items from the right of  and , we don't
    go too far and start decomposing  or !  It's easy to avoid decomposing :
    when , we don't need to apply the i.h.\ anyway.  To avoid decomposing ,
    we need to reason by contradiction, using \Lemmaref{lem:declaration-preservation}.
    
        \begin{itemize}
        \ProofCaseRule{} \\
            We have .
            Applying \substextendEE to that derivation gives the result.
    
        \ProofCaseRule{}
            We have  by the well-formedness assumption.

            The concluding rule of 
            must have been \substextendEE or \substextendAdd.  In both cases, the result follows by i.h.
            and applying \substextendEE or \substextendAdd.

            Note:  In \substextendAdd, the left-hand context doesn't change, so we clearly maintain
            .  In \substextendEE, we can correctly apply the
            i.h.\ because .  Suppose, for
            a contradiction, that .  Then .
            It was given that , that is, .
            By \Lemmaref{lem:declaration-preservation},  has a declaration of .
            But then  is not well-formed: contradiction.  Therefore
            .

        \ProofCaseRule{}
            We have  by the well-formedness assumption.

            The concluding rule must have been \substextendSolSol, \substextendSolve or \substextendAddSolved.
            In each case, apply the i.h.\ and then the corresponding rule.  (In \substextendSolSol and \substextendSolve,
            use \Lemmaref{lem:declaration-preservation} to show .)

        \ProofCaseRule{}
           The concluding rule must have been \substextendUU.  The result follows by i.h. and
           applying \substextendUU.

        \ProofCaseRule{}  Similar to the previous
           case, with rule \substextendMonMon.
        
        \ProofCaseRule{}  Similar to the previous case,
           with rule \substextendVV.
        \end{itemize}

  \item Similar to part (i), except that when , apply rule \substextendSolve.

  \item Similar to part (i), except that when , apply rule \substextendSolSol,
     using the given equality to satisfy the second premise.
  \qedhere
  \end{enumerate}
\end{proof}


\parallelextensionsolve*
\begin{proof}
  By induction on .

  In the case where ,
  we know that rule \substextendSolve must have concluded the derivation (we can use
  \Lemmaref{lem:declaration-preservation} to get a contradiction that rules out \substextendAddSolved);
  then we have a subderivation , to which we can apply
  \substextendSolSol.
\end{proof}


\parallelextensionupdate*
\begin{proof}
  By induction on .  Similar to the proof of \Lemmaref{lem:parallel-extension-solve},
  but applying \substextendSolSol at the end.
\end{proof}


\subsection{Instantiation Extends}

\instantiationextension*
\begin{proof}
  By induction on the given instantiation derivation. 

    \begin{itemize}
      \DerivationProofCase{\InstLSolve}
              { \Gamma \entails \tau}     { \instjudg{\Gamma, \ahat, \Gamma'}
                          {\ahat}
                          {\tau}
                          {\Gamma, \hypeq{\ahat}{\tau}, \Gamma'}
               }

               By \Lemmaref{lem:extension-solve},
               . 

      \DerivationProofCase{\InstLReach}
              { }
              {\instjudg{\Gamma[\ahat][\bhat]}
                          {\ahat}
                          {\bhat}
                          {\Gamma[\ahat][\hypeq{\bhat}{\ahat}]}}

           for some . \\
          By the definition of well-formedness, . \\
          Therefore, by \Lemmaref{lem:extension-solve}, . 

      \DerivationProofCase{\InstLArr}
              {\instjudgr{\Gamma[\ahat_2, \ahat_1, \hypeq{\ahat}{\ahat_1 \arr \ahat_2}]}
                          {\ahat_1}
                          {A_1}
                          {\Gamma'} \\
               \instjudg{\Gamma'}
                          {\ahat_2}
                          {[\Gamma']A_2}
                          {\Delta}}
              {\instjudg{\Gamma[\ahat]}
                          {\ahat}
                          {A_1 \arr A_2}
                          {\Delta}}

          By \Lemmaref{lem:extension-add}, we can insert an (unsolved) , giving
            . \\
          By \Lemmaref{lem:extension-add} again, . \\
          By \Lemmaref{lem:extension-solve}, we can solve , giving
            . \\
          Then by transitivity (\Lemmaref{lem:substextend-transitivity}), . \\
          By i.h.\ on the first subderivation,
            . \\
          By i.h.\ on the second subderivation, . \\
          By transitivity (\Lemmaref{lem:substextend-transitivity}), . \\
          By transitivity (\Lemmaref{lem:substextend-transitivity}), . 

      \DerivationProofCase{\InstLAllR}
            {\instjudg{\Gamma[\ahat], \beta}{\ahat}{B}{\Delta, \beta, \Delta'}}
            {\instjudg{\Gamma[\ahat]}{\ahat}{\alltype{\beta}{B}}{\Delta}}

          By induction, . \\
          By \Lemmaref{lem:extension-order} (i),
          we have . 

      \DerivationProofCase{\InstRSolve}
              { \Gamma \entails \tau}
              { \instjudgr{\Gamma, \ahat, \Gamma'}
                          {\ahat}
                          {\tau}
                          {\Gamma, \hypeq{\ahat}{\tau}, \Gamma'}
               }

          By \Lemmaref{lem:extension-solve}, we can solve , giving
          . 

      \DerivationProofCase{\InstRReach}
              { }
              {\instjudgr{\Gamma[\ahat][\bhat]}
                         {\ahat}
                         {\bhat}
                         {\Gamma[\ahat][\hypeq{\bhat}{\ahat}]}}

           for some . \\
          By the definition of well-formedness, . \\
          Hence by \Lemmaref{lem:extension-solve}, we can solve , giving
          . 

      \DerivationProofCase{\InstRArr}
            {\instjudg{\Gamma[\ahat_2, \ahat_1, \hypeq{\ahat}{\ahat_1 \arr \ahat_2}]}
                      {\ahat_1}
                      {A_1}
                      {\Gamma'} \\
               \instjudgr{\Gamma'}
                         {\ahat_2}
                         {[\Gamma']A_2}
                         {\Delta}}
            {\instjudgr{\Gamma[\ahat]}
                       {\ahat}
                       {A_1 \arr A_2}
                       {\Delta}}

          Because the contexts here are the same as in \InstLArr, this is the same as the \InstLArr case.


      \DerivationProofCase{\InstRAllL}
            {\instjudgr{\Gamma[\ahat], \MonnierComma{\bhat}, \bhat}{\ahat}{[\bhat/\beta]B}{\Delta, \MonnierComma{\bhat}, \Delta'}}
            {\instjudgr{\Gamma[\ahat]}{\ahat}{\alltype{\beta}{B}}{\Delta}}

          By i.h., . \\
          By \Lemmaref{lem:extension-order} (ii), .
    \qedhere
    \end{itemize}
\end{proof}


\subsection{Subtyping Extends}

\subtypingextension*
 \begin{proof}
  By induction on the given derivation.

  For cases \SubVar, \SubUnit, \SubExvar, we have ,
  so \Lemmaref{lem:substextend-reflexivity} suffices.
  
  \begin{itemize}
    \DerivationProofCase{\SubArr}
         {\subjudg{\Gamma}{B_1}{A_1}{\Theta}
          \\
          \subjudg{\Theta}{[\Omega]A_2}{[\Omega]B_2}{\Delta}
         }
         { \subjudg{\Gamma}{A_1 \arr A_2}{B_1 \arr B_2}{\Delta} }

         By IH on each subderivation, 
         and .

         By \Lemmaref{lem:substextend-transitivity} (transitivity), , which was to be shown.

    \DerivationProofCase{\SubAllL}
          {\subjudg{\Gamma, \MonnierComma{\ahat}, \ahat}
                   {[\ahat/\alpha]A}
                   {B}
                   {\Delta, \MonnierComma{\ahat}, \Theta}}
          {\subjudg{\Gamma}{\alltype{\alpha}{A}}{B}{\Delta}}

          By IH, 
          .
          
         By \Lemmaref{lem:extension-order} (ii) with 
         and  and 
         and , we obtain
         

    \DerivationProofCase{\SubAllR}
          {\subjudg{\Gamma, \beta}{A}{B}{\Delta, \beta, \Theta}}
          {\subjudg{\Gamma}{A}{\alltype{\beta}{B}}{\Delta}}

          By IH, we have
          .
          
          By \Lemmaref{lem:extension-order} (i), we obtain
          , which was to be shown.
    


    \ProofCasesRules{\SubInstL, \SubInstR}
          In each of these rules, the premise has the same input and output contexts as the conclusion,
          so \Lemmaref{lem:instantiation-extension} suffices.
    \qedhere
  \end{itemize}
\end{proof}







\section{Decidability of Instantiation}
 

\leftunsolvednesspreservation*
\begin{proof}
  By induction on the given derivation.
  
    \begin{itemize}
        \DerivationProofCase{\InstLSolve}
                { \Gamma_0 \entails \tau}
                { \instjudg{\underbrace{\Gamma_0, \ahat, \Gamma_1}_\Gamma}
                            {\ahat}
                            {\tau}
                            {\Gamma_0, \hypeq{\ahat}{\tau}, \Gamma_1}
                 }

            Immediate, since to the left of , the contexts  and  are the same.

        \DerivationProofCase{\InstLReach}
                { }
                {\instjudg{\Gamma[\ahat][\bhat]}
                            {\ahat}
                            {\bhat}
                            {\Gamma[\ahat][\hypeq{\bhat}{\ahat}]}}

            Immediate, since to the left of , the contexts  and  are the same.

        \DerivationProofCase{\InstLArr}
                {\instjudgr{\Gamma[\ahat_2, \ahat_1, \hypeq{\ahat}{\ahat_1 \arr \ahat_2}]}
                            {\ahat_1}
                            {A_1}
                            {\Gamma'} \\
                 \instjudg{\Gamma'}
                            {\ahat_2}
                            {[\Gamma']A_2}
                            {\Delta}}
                {\instjudg{\Gamma[\ahat]}
                            {\ahat}
                            {A_1 \arr A_2}
                            {\Delta}}

            We have .  Therefore . \\
            Clearly, . \\
            We have two subderivations:

By induction on (1), . \\
            Also by induction on (1), with  playing the role of , we get . \\
            Since , it is declared to the left of 
              in . \\
            Hence by \Lemmaref{lem:declaration-order-preservation},  is declared to the left of  in .
            That is, , where . \\
            By induction on (2), . 


        \DerivationProofCase{\InstLAllR}
              {\instjudg{\Gamma_0, \ahat, \Gamma_1, \beta}{\ahat}{B}{\Delta, \beta, \Delta'}}
              {\instjudg{\Gamma_0, \ahat, \Gamma_1}{\ahat}{\alltype{\beta}{B}}{\Delta}}

            We have . \\
            By induction, . \\
            Note that  is declared to the left of  in . \\
            By \Lemmaref{lem:declaration-order-preservation},
               is declared to the left of  in ,
              that is, in . Since ,
              we have . 


        \ProofCasesRules{\InstRSolve, \InstRReach}
           Similar to the \InstLSolve and \InstLReach cases.

        \DerivationProofCase{\InstRArr}
              {\instjudg{\Gamma[\ahat_2, \ahat_1, \hypeq{\ahat}{\ahat_1 \arr \ahat_2}]}
                        {\ahat_1}
                        {A_1}
                        {\Gamma'} \\
                 \instjudgr{\Gamma'}
                           {\ahat_2}
                           {[\Gamma']A_2}
                           {\Delta}}
              {\instjudgr{\Gamma[\ahat]}
                         {\ahat}
                         {A_1 \arr A_2}
                         {\Delta}}

            Similar to the \InstLArr case.


        \DerivationProofCase{\InstRAllL}
              {\instjudgr{\Gamma[\ahat], \MonnierComma{\chat}, \chat}{\ahat}{[\chat/\beta]B}{\Delta, \MonnierComma{\chat}, \Delta'}}
              {\instjudgr{\Gamma[\ahat]}{\ahat}{\alltype{\beta}{B}}{\Delta}}

            We have . \\
            By induction, . \\
            Note that  is declared to the left of  in . \\
            By \Lemmaref{lem:declaration-order-preservation},  is declared to the left of  in . \\
            Hence  is declared in , and we know it is in ,
            so . 
\qedhere
    \end{itemize}
\end{proof}


\leftfreevariablepreservation*
\begin{proof}
  By induction on the given instantiation derivation.


    \begin{itemize}
        \DerivationProofCase{\InstLSolve}
                { \Gamma_0 \entails \tau}
                { \instjudg{\underbrace{\Gamma_0, \ahat, \Gamma_1}_\Gamma}
                            {\ahat}
                            {\tau}
                            {\underbrace{\Gamma_0, \hypeq{\ahat}{\tau}, \Gamma_1}_{\Delta}}
                 }

            We have .  Since  differs from 
            only in , it must be the case that .
            It is given that , so .
            

        \DerivationProofCase{\InstLReach}
                { }
                {\instjudg{\underbrace{\Gamma'[\ahat][\chat]}_{\Gamma}}
                            {\ahat}
                            {\chat}
                            {\underbrace{\Gamma'[\ahat][\hypeq{\chat}{\ahat}]}_{\Delta}}}

            Since  differs from  only
            in solving  to , applying  to a type will not introduce a .
            We have , so .
        
        \DerivationProofCase{\InstRSolve}
                { \Gamma_0 \entails \tau}
                { \instjudgr{\Gamma_0, \ahat, \Gamma_1}
                            {\ahat}
                            {\tau}
                            {\Gamma_0, \hypeq{\ahat}{\tau}, \Gamma_1}
                 }
             
             Similar to the \InstLSolve case.

        \DerivationProofCase{\InstRReach}
                { }
                {\instjudgr{\Gamma'[\ahat][\chat]}
                           {\ahat}
                           {\chat}
                           {\Gamma'[\ahat][\hypeq{\chat}{\ahat}]}}

             Similar to the \InstLReach case.
        
        \DerivationProofCase{\InstLArr}
                {\instjudgr{\overbrace{\Gamma_0, \ahat_2, \ahat_1, \hypeq{\ahat}{\ahat_1 \arr \ahat_2}, \Gamma_1}^{\Gamma'}}
                            {\ahat_1}
                            {A_1}
                            {\Delta} \\
                 \instjudg{\Delta}
                            {\ahat_2}
                            {[\Delta]A_2}
                            {\Delta}}
                {\instjudg{\underbrace{\Gamma_0, \ahat, \Gamma_1}_\Gamma}
                            {\ahat}
                            {A_1 \arr A_2}
                            {\Delta}}

            We have 
            and 
and . \\
            By weakening, we get ; since  and  only adds
            a solution for , it follows that . \\
            Therefore  and  and . \\
            Since we have , we also have . \\
            By induction on the first premise, . \\
            Also by induction on the first premise, with  playing the role of , we have . \\
            Note that . \\
            By \Lemmaref{lem:left-unsolvedness-preservation}, . \\
            Therefore  has the form . \\
            Since , we know that  is declared to the left of  in ,
            so by \Lemmaref{lem:declaration-order-preservation},  is declared to the left of  in .
            Hence . \\
            Furthermore, by \Lemmaref{lem:instantiation-extension}, we have . \\
            Then by \Lemmaref{lem:extension-weakening}, we have .
            Using induction on the second premise, . 

        \DerivationProofCase{\InstLAllR}
              {\instjudg{\Gamma_0, \ahat, \Gamma_1, \gamma}{\ahat}{C}{\Delta, \gamma, \Delta'}}
              {\instjudg{\underbrace{\Gamma_0, \ahat, \Gamma_1}_{\Gamma}}{\ahat}{\alltype{\gamma}{C}}{\Delta}}

            We have  and  and  and . \\
            By weakening, ; by the definition of substitution, . \\
            Substituting equals for equals,  and . \\
            By induction, . \\
            Since  is declared to the left of  in , we can use
            \Lemmaref{lem:declaration-order-preservation} to show that
               is declared to the left of  in ,
              that is, in . \\
            We have , so .  Thus each free variable  in  is in ,
            to the left of  in . \\
            Therefore, by \Lemmaref{lem:declaration-order-preservation}, each free variable  in  is in . \\
            Therefore . \\
            Earlier, we obtained , so substituting equals for equals,
            .

        \DerivationProofCase{\InstRArr}
              {\instjudg{\Gamma_0, \ahat_2, \ahat_1, \hypeq{\ahat}{\ahat_1 \arr \ahat_2}, \Gamma_1}
                        {\ahat_1}
                        {A_1}
                        {\Delta} \\
                 \instjudgr{\Gamma'}
                           {\ahat_2}
                           {[\Delta]A_2}
                           {\Delta}}
              {\instjudgr{\Gamma_0, \ahat, \Gamma_1}
                         {\ahat}
                         {A_1 \arr A_2}
                         {\Delta}}

            Similar to the \InstLArr case.


        \DerivationProofCase{\InstRAllL}
              {\instjudgr{\Gamma[\ahat], \MonnierComma{\chat}, \chat}{\ahat}{[\chat/\gamma]C}{\Delta, \MonnierComma{\chat}, \Delta'}}
              {\instjudgr{\Gamma[\ahat]}{\ahat}{\alltype{\gamma}{C}}{\Delta}}

             We have  and  and  and . \\
             By weakening, ;
               by the definition of substitution, . \\
             Substituting equals for equals,  and . \\
             By induction, . \\
             Note that  is declared to the left of  in . \\
             By \Lemmaref{lem:declaration-order-preservation},  is declared to the left of  in . \\
             So  is declared in . \\
             Now, note that each free variable  in  is in , which is to the left of  in . \\
             Therefore, by \Lemmaref{lem:declaration-order-preservation}, each free variable  in  is in . \\
             Therefore . \\
             Earlier, we obtained , so substituting equals for equals,
             .
    \qedhere
    \end{itemize}
\end{proof}



\instantiationsizepreservation*
\begin{proof}
  By induction on the given derivation.
  
  \begin{itemize}
      \DerivationProofCase{\InstLSolve}
              { \Gamma_0 \entails \tau}
              { \instjudg{\underbrace{\Gamma_0, \ahat, \Gamma_1}_\Gamma}
                          {\ahat}
                          {\tau}
                          {\Gamma_0, \hypeq{\ahat}{\tau}, \Gamma_1}
               }

              Since  differs from  only in solving , and we know ,
              we have ; therefore .

      \DerivationProofCase{\InstLReach}
              { }
              {\instjudg{\Gamma[\ahat][\bhat]}
                          {\ahat}
                          {\bhat}
                          {\Gamma[\ahat][\hypeq{\bhat}{\ahat}]}
              }

              Here,  differs from  only in solving  to .  However,  has the same
              size as , so even if , we have .

      \DerivationProofCase{\InstLArr}
              {\instjudgr{\overbrace{\Gamma_0, \ahat_2, \ahat_1, \hypeq{\ahat}{\ahat_1 \arr \ahat_2}, \Gamma_1}^{\Gamma'}}
                          {\ahat_1}
                          {A_1}
                          {\Theta} \\
               \instjudg{\Theta}
                          {\ahat_2}
                          {[\Theta]A_2}
                          {\Delta}}
              {\instjudg{\underbrace{\Gamma_0, \ahat, \Gamma_1}_\Gamma}
                          {\ahat}
                          {A_1 \arr A_2}
                          {\Delta}}

      We have  and .
      Since , we have .
      It follows that . \\
      By weakening, . \\
      By induction on the first premise, . \\
      By \Lemmaref{lem:declaration-order-preservation},
        since  is declared to the left of  in ,
        we have that  is declared to the left of  in . \\
      By \Lemmaref{lem:left-unsolvedness-preservation}, since , it is unsolved in :
      that is, . \\
      By \Lemmaref{lem:instantiation-extension}, we have . \\
      By \Lemmaref{lem:extension-weakening}, . \\
      Since ,
        \Lemmaref{lem:left-free-variable-preservation} gives . \\
      By induction on the second premise, ,
        and by transitivity of equality, . 

      \DerivationProofCase{\InstLAllR}
            {\instjudg{\Gamma_0, \ahat, \Gamma_1, \beta}{\ahat}{A_0}{\Delta, \beta, \Delta'}}
            {\instjudg{\underbrace{\Gamma_0, \ahat, \Gamma_1}_{\Gamma}}{\ahat}{\alltype{\beta}{A_0}}{\Delta}}

      We have  and . \\
      By weakening, . \\
      From the definition of substitution, .
      Hence . \\
      The input context of the premise is , which is , so
      by induction, . \\
      Suppose  is a free variable in .
      Then  is declared in , and so occurs before  in . \\
      By \Lemmaref{lem:declaration-order-preservation},  is declared before  in . \\
      So every free variable  in  is declared in . \\
      Hence . \\
      We have , so ;
      by transitivity of equality,  .

      \DerivationProofCase{\InstRSolve}
              { \Gamma_0 \entails \tau}
              { \instjudgr{\Gamma_0, \ahat, \Gamma_1}
                          {\ahat}
                          {\tau}
                          {\Gamma_0, \hypeq{\ahat}{\tau}, \Gamma_1}
              }

              Similar to the \InstLSolve case.

      \DerivationProofCase{\InstRReach}
              { }
              {\instjudgr{\Gamma[\ahat][\bhat]}
                         {\ahat}
                         {\bhat}
                         {\Gamma[\ahat][\hypeq{\bhat}{\ahat}]}}

              Similar to the \InstLReach case.


      \DerivationProofCase{\InstRArr}
            {\instjudg{\overbrace{\Gamma_0, \ahat_2, \ahat_1, \hypeq{\ahat}{\ahat_1 \arr \ahat_2}, \Gamma_1}^{\Gamma'}}
                      {\ahat_1}
                      {A_1}
                      {\Theta} \\
               \instjudgr{\Theta}
                         {\ahat_2}
                         {[\Theta]A_2}
                         {\Delta}}
            {\instjudgr{\underbrace{\Gamma_0, \ahat, \Gamma_1}_{\Gamma}}
                       {\ahat}
                       {A_1 \arr A_2}
                       {\Delta}}

           Similar to the \InstLArr case.

      \DerivationProofCase{\InstRAllL}
            {\instjudgr{\Gamma'[\ahat], \MonnierComma{\bhat}, \bhat}{\ahat}{[\bhat/\beta]A_0}{\Delta, \MonnierComma{\bhat}, \Delta'}}
            {\instjudgr{\Gamma'[\ahat]}{\ahat}{\alltype{\beta}{A_0}}{\Delta}}

          We have  and . \\
          By weakening, . \\
          From the definition of substitution, .
          Hence . \\
          By induction, . \\
          Suppose  is a free variable in . \\
          Then  is declared in , and so occurs before  in . \\
          By \Lemmaref{lem:declaration-order-preservation},  is declared before  in . \\
          So every free variable  in  is declared in . \\
          Hence . \\
          Since , we have
          ;
          by transitivity of equality,
          .
          \qedhere
  \end{itemize}
\end{proof}



\instantiationdecidability*
\begin{proof}
  By induction on the derivation of . 

  \begin{enumerate}[(1)]
  \item  is decidable. 

      \begin{itemize}
       \DerivationProofCase{\UvarWF}
                           { }
                           {\judgetp{\underbrace{\Gamma_L, \ahat, \Gamma_R}_{\Gamma'[\alpha]}}{\alpha}}

          If ,
          then by \UvarWF we have ,
          and by rule \InstLSolve we have a derivation. \\
          Otherwise no rule matches, and so no derivation exists. 
       
       \ProofCaseRule{\UnitWF}
          By rule \InstLSolve.

       \DerivationProofCase{\EvarWF}
                           { }
                           {\judgetp{\underbrace{\Gamma_L, \ahat, \Gamma_R}_{\Gamma}}{\bhat}}

           By inversion, we have , and .
           Since ,
           it follows that :
           Either  or . \\
           If , then we have a derivation by . \\
           If , then we have a derivation by . 

       \DerivationProofCase{\SolvedEvarWF}
                           { }
                           {\judgetp{\underbrace{\Gamma'[\hypeq{\bhat}{\tau}]}_{\Gamma}}{\bhat}}

           It is given that , so this case is impossible.

       \DerivationProofCase{\ArrowWF}
                           {\judgetp{\Gamma}{A_1} \\ \judgetp{\Gamma}{A_2}}
                           {\judgetp{\underbrace{\Gamma_L, \ahat, \Gamma_R}_\Gamma}{A_1 \arr A_2}}

           By assumption, 
           and . \\
           If  is a monotype and is well-formed under , we can apply \InstLSolve. \\
           Otherwise, the only rule with a conclusion matching  is \InstLArr. \\
           First, consider whether 
           is decidable. \1ex]
           Suppose . \\
           By \Lemmaref{lem:instantiation-extension}, ; \\
           by \Lemmaref{lem:extension-order} (i), . \\
           Hence by rule \InstLAllR, .
           \1ex]
           Suppose . \\
           By \Lemmaref{lem:instantiation-extension}, ; \\
           by \Lemmaref{lem:extension-order} (ii), . \\
           Hence by rule \InstRAllL, .
           \
      \typesize{\Gamma}{[\Gamma]\ahat} ~\leq~ \typesize{\Gamma}{\tau} + 1
   
      \typesize{\Gamma}{[\Gamma]\ahat} ~\leq~ \typesize{\Gamma}{\ahat}
   
                \typesize{\Theta}{[\Theta]A_2} ~+~ \typesize{\Theta}{[\Theta]B_2}
                ~~~\leq~~~
                \typesize{\Gamma}{(A_1 \arr A_2)} ~+~ \typesize{\Gamma}{(B_1 \arr B_2)}
             
        \begin{array}[t]{lll}
               \synjudg{\Gamma}{e}{-}{-}
            \\ \chkjudg{\Gamma}{e}{B}{-}
            \\ \appjudg{\Gamma}{e}{A}{-}{-}
        \end{array}
    
      \left\langle
        e,~~~
        \begin{array}[c]{llll}
            \syn
          \0.6ex]
            \app,
            &
\typesize{\Gamma}{A}
        \end{array}
      \right\rangle

               \syn ~~\bigprec~~ \chk ~~\bigprec~~ \app
    
                 \declsubjudg{[\Omega]\Delta}{\big[\Omega, \MonnierComma{\bhat}, |\Delta'|\big][\bhat/\beta]B_0}{[\Omega]\ahat}
              
                  \declsubjudg{[\Omega]\Delta}
                          {\big[[\Omega, \MonnierComma{\bhat}, |\Delta'|]\bhat/\beta\big]\big[\Omega, \MonnierComma{\bhat}, |\Delta'|\big]B_0}{[\Omega]\ahat}
              
                   \declsubjudg{[\Omega]\Delta}{\big[[\Omega, \MonnierComma{\bhat}, |\Delta'|]\bhat/\beta\big][\Omega]B_0}{[\Omega]\ahat}
              
              \substextend{\underbrace{\Omega_0[\hypeq{\ahat}{\tau_0}]}_{\Omega}}
                          {\Omega_0[\hypeq{\ahat_2}{\tau_2}, \hypeq{\ahat_1}{\tau_1}, \hypeq{\ahat}{\ahat_1 \arr \ahat_2}]}
          
                  \substextend{\Gamma_0[\ahat]}{\Omega_0[\hypeq{\ahat}{\tau_0}]}
          5pt]
      &
        & {\alltype{\beta} B'} &   \unitty  &   \alpha   &   \bhat   &    B_1 \arr B_2 &
      \\\cmidrule[1pt]{3-7}
      &\alltype{\alpha} A'
                &   \CaseBpoly
                &   \CASEhandled{2.Poly}
                &   \CASEhandled{2.Poly}
                &   \CASEhandled{2.Poly}
                &   \CASEhandled{2.Poly}
      \\\cmidrule{3-7}
      &\unitty
                &   \CaseBpoly
                &   \CASEhandled{2.Units}
                &   \CASEimpossible
                &   \CASEhandledsymm{2.BEx.Unit}
                &   \CASEimpossible
      \\\cmidrule{3-7}
      \MAINLABEL{[\Gamma]A}
      & \alpha
                &   \CaseBpoly
                &   \CASEimpossible
                &   \CASEhandled{2.Uvars}
                &   \CASEhandledsymm{2.BEx.Uvar}
                &   \CASEimpossible
      \\\cmidrule{3-7}
      & \ahat
                &   \CaseBpoly
                &   \CASEhandled{2.AEx.Unit}
                &   \CASEhandledsymm{2.AEx.Uvar}
                &
                    \begin{tabular}[c]{c}
                      \CASEhandled{2.AEx.SameEx}
                      \\
                      \CASEhandled{2.AEx.OtherEx}
                    \end{tabular}
&   \CASEhandled{2.AEx.Arrow}
      \\\cmidrule{3-7}
      & A_1 \arr A_2
                &   \CaseBpoly
                &   \CASEimpossible
                &   \CASEimpossible
                &   \CASEhandledsymm{2.BEx.Arrow}
                &   \CASEhandled{2.Arrows}
    \end{array}
  \end{displ}

  The impossibility of the ``\CASEimpossible'' entries follows from inspection of the
  declarative subtyping rules.

  \medskip

  We first split on .
  
  \begin{itemize}
  \item \textbf{Case \CaseBpoly:  polymorphic:}  :
    
    \begin{llproof}
      \eqPf{B} {\alltype{\beta} B_0}   { predicative}
      \eqPf{B'} {[\Gamma]B_0}   { predicative}
      \eqPf{[\Omega]B} {[\Omega](\alltype{\beta} B_0)}   {Applying  to both sides}
      \continueeqPf {\alltype{\beta} [\Omega]B_0}   {By definition of substitution}
       \declsubjudgPf{[\Omega]\Gamma} {[\Omega]A} {[\Omega]B}   {Given}
       \declsubjudgPf{[\Omega]\Gamma} {[\Omega]A} {\alltype{\beta} [\Omega]B_0}   {By above equality}
       \declsubjudgPf{[\Omega]\Gamma, \beta}
                    {[\Omega]A}
                    {[\Omega]B_0}
            {By \Lemmaref{lem:decl-invertibility}}
      \ltPf{\Dee'} {\Dee}   {\ditto}
       \declsubjudgPf{[\Omega, \beta](\Gamma, \beta)}
                    {[\Omega, \beta]A}
                    {[\Omega, \beta]B_0}
            {By definitions of substitution}
      \subjudgPf{\Gamma, \beta}{[\Gamma, \beta]A}{[\Gamma, \beta]B_0} {\Delta'}  {By i.h.}
      \substextendPf{\Delta'}{\Omega_0'}   {\ditto}
      \substextendPf{\Omega, \beta}{\Omega_0'}   {\ditto}
      \subjudgPf{\Gamma, \beta}{[\Gamma]A}{[\Gamma]B_0} {\Delta'}  {By definition of substitution}
      \proofsep
      \substextendPf{\Gamma, \beta} {\Delta'}  {By \Lemmaref{lem:instantiation-extension}}
      \eqPf{\Delta'}{\Delta, \beta, \Theta}  {By \Lemmaref{lem:extension-order} (i)}
      \substextendPf{\Gamma}{\Delta}  {\ditto}
      \substextendPf{\Delta, \beta, \Theta}{\Omega_0'}   {By  and above equality}
      \eqPf{\Omega_0'}{\Omega', \beta, \Omega_R}  {By \Lemmaref{lem:extension-order} (i)}
\Hand      \substextendPf{\Delta}{\Omega'}  {\ditto}
      \proofsep
      \subjudgPf{\Gamma, \beta}{[\Gamma]A}{[\Gamma]B_0} {\Delta, \beta, \Theta}  {By above equality}
\substextendPf{\Omega, \beta}{\Omega', \beta, \Omega_R}   {By above equality}
\Hand      \substextendPf{\Omega}{\Omega'}   {By \Lemmaref{lem:substextend-transitivity}}
      \proofsep
      \subjudgPf{\Gamma}{[\Gamma]A}{\alltype{\beta}{[\Gamma]B_0}}{\Delta}  {By \SubAllR}
\Hand      \subjudgPf{\Gamma}{[\Gamma]A}{\alltype{\beta} B'}{\Delta}  {By above equality}
    \end{llproof}


\clearpage
  \item \textbf{Cases 2.*:  not polymorphic:}

    We split on the form of .

    \begin{itemize}
    \item \textbf{Case \CASEhandled{2.Poly}:  is polymorphic:} :

      \medskip
      
      \begin{llproof}
        \eqPf{A} {\alltype{\alpha} A_0}   { predicative}
        \eqPf{A'} {[\Gamma]A_0}   { predicative}
        \eqPf{[\Omega]A} {[\Omega](\alltype{\alpha} A_0)}   {Applying  to both sides}
        \eqPf{[\Omega]A} {\alltype{\alpha} [\Omega]A_0}   {By definition of substitution}
        \declsubjudgPf{[\Omega]\Gamma} {[\Omega]A} {[\Omega]B}   {Given}
        \declsubjudgPf{[\Omega]\Gamma} {\alltype{\alpha} [\Omega]A_0} {[\Omega]B}   {By above equality}
        \neqPf{[\Gamma]B} {(\alltype{\beta}{\cdots})}   {We are in the `` not polymorphic'' subcase}
        \neqPf{B} {(\alltype{\beta}{\ldots})}   { predicative}
        \declsubjudgPf{[\Omega]\Gamma} {[\tau/\alpha][\Omega]A_0} {[\Omega]B}   {By inversion on \DsubAllL}
        \judgetpPf{[\Omega]\Gamma} {\tau}   {\ditto}
\proofsep
        \substextendPf{\Gamma}{\Omega}   {Given}
        \substextendPf{\Gamma, \MonnierComma{\ahat}}{\Omega, \MonnierComma{\ahat}}
                   {By \substextendMonMon}
        \substextendPf{\Gamma, \MonnierComma{\ahat}, \ahat}{\underbrace{\Omega, \MonnierComma{\ahat}, \hypeq{\ahat}{\tau}}_{\Omega_0}}
                   {By \substextendSolve}
        \proofsep
        \eqPf{[\Omega]\Gamma} {[\Omega_0](\Gamma, \MonnierComma{\ahat}, \ahat)}  {By definition of context application (lines 16, 13)}
        \decolumnizePf
        \proofsep
        \declsubjudgPf{[\Omega]\Gamma}  {[\tau/\alpha][\Omega]A_0}  {[\Omega]B}    {Above}
        \declsubjudgPf{[\Omega_0](\Gamma, \MonnierComma{\ahat}, \ahat)}
                      {[\tau/\alpha][\Omega]A_0}
                      {[\Omega]B}
                      {By above equality}
        \declsubjudgPf{[\Omega_0](\Gamma, \MonnierComma{\ahat}, \ahat)}
                      {\big[[\Omega_0]\ahat / \alpha\big][\Omega]A_0}
                      {[\Omega]B}
                      {By definition of substitution}
        \declsubjudgPf{[\Omega_0](\Gamma, \MonnierComma{\ahat}, \ahat)}
                      {\big[[\Omega_0]\ahat / \alpha\big][\Omega_0]A_0}
                      {[\Omega_0]B}
                      {By definition of substitution}
        \declsubjudgPf{[\Omega_0](\Gamma, \MonnierComma{\ahat}, \ahat)}
                      {[\Omega_0][\ahat / \alpha]A_0}
                      {[\Omega_0]B}
                      {By distributivity of substitution}
\proofsep
        \subjudgPf{\Gamma, \MonnierComma{\ahat}, \ahat} {[\Gamma, \MonnierComma{\ahat}, \ahat][\ahat / \alpha]A_0} {[\Gamma, \MonnierComma{\ahat}, \ahat]B} {\Delta_0}   {By i.h.}
        \substextendPf{\Delta_0} {\Omega''}    {\ditto}
        \substextendPf{\Omega_0} {\Omega''}    {\ditto}
        \subjudgPf{\Gamma, \MonnierComma{\ahat}, \ahat} {[\Gamma][\ahat / \alpha]A_0} {[\Gamma]B} {\Delta_0}   {By definition of substitution}
        \substextendPf{\Gamma, \MonnierComma{\ahat}, \ahat} {\Delta_0}   {By \Lemmaref{lem:subtyping-extension}}
        \eqPf{\Delta_0}{(\Delta, \MonnierComma{\ahat}, \Theta)}  {By \Lemmaref{lem:extension-order} (ii)}
        \substextendPf{\Gamma}{\Delta}   {\ditto}
        \eqPf{\Omega''}{(\Omega', \MonnierComma{\ahat}, \Omega_Z)}  {By \Lemmaref{lem:extension-order} (ii)}
\Hand        \substextendPf{\Delta}{\Omega'}   {\ditto}
        \substextendPf{\Omega_0} {\Omega''}    {Above}
        \substextendPf{\Omega, \MonnierComma{\ahat}, \hypeq{\ahat}{\tau}} {\Omega', \MonnierComma{\ahat}, \Omega_Z}    {By above equalities}
\Hand        \substextendPf{\Omega} {\Omega'}    {By \Lemmaref{lem:extension-order} (ii)}
        \proofsep
        \subjudgPf{\Gamma, \MonnierComma{\ahat}, \ahat}
                  {[\Gamma][\ahat/\alpha]A_0}
                  {[\Gamma]B}
                  {\Delta, \MonnierComma{\ahat}, \Theta}
                  {By above equality }
\subjudgPf{\Gamma, \MonnierComma{\ahat}, \ahat}
                  {[\ahat/\alpha][\Gamma]A_0}
                  {[\Gamma]B}
                  {\Delta, \MonnierComma{\ahat}, \Theta}
                  {By def.\ of subst.\ ( and )}
\subjudgPf{\Gamma} {\alltype{\alpha}{[\Gamma]A_0}} {[\Gamma]B} {\Delta}   {By \SubAllL}
     \Hand      \subjudgPf{\Gamma} {\alltype{\alpha}{A'}} {[\Gamma]B} {\Delta}   {By above equality}
      \end{llproof}

      \smallskip

    \item \textbf{Case \CASEhandled{2.AEx}:  is an existential variable} :

      We split on the form of .
      \begin{itemize}
      \item \textbf{Case \CASEhandled{2.AEx.SameEx}:  is the same existential variable}  :

        \smallskip

        \begin{llproof}
            \subjudgPf{\Gamma} {\ahat} {\ahat} {\Gamma}    {By \SubExvar}
\Hand         \subjudgPf{\Gamma} {[\Gamma]A} {[\Gamma]B} {\Gamma}    {By }
\Hand          \substextendPf{\Delta} {\Omega}   {}
\Hand          \substextendPf{\Omega} {\Omega'}   {By \Lemmaref{lem:substextend-reflexivity} and }
        \end{llproof}


      \item \textbf{Case \CASEhandled{2.AEx.OtherEx}:  is a different existential variable}   where :

        Either , or .

        \begin{itemize}
            \item : \\
                  We have .  \\
                  Therefore , or . \\
                  But we are in Case 2.AEx.\textbf{Other}Ex, so the former is impossible. \\
                  Therefore, . \\
                  Since  is predicative,  cannot have the form ,
                  so the only way that  can be a proper subterm of  is if 
                  has the form  such that  is a subterm of  or , that is:
                  . \\
                  Then by a property of substitution, . \\
                  By \Lemmaref{lem:subst-extension-invariance}, , so
                   . \\
                  We have ,
                  and we know that  is a monotype, so we can use \Lemmaref{lem:occurrence} (ii)
                  to show that , a contradiction.

            \item :

                  \begin{llproof}
                    \instjudgPf{\Gamma} {\ahat} {[\Gamma]\bhat} {\Delta}   {By \Theoremref{thm:instantiation-completes} (1)}
    \Hand                \subjudgPf{\Gamma} {\ahat} {\bhat} {\Delta}    {By \SubInstL}
          \Hand     \substextendPf{\Delta}{\Omega'} {\ditto}
          \Hand     \substextendPf{\Omega}{\Omega'} {\ditto}
                  \end{llproof}
      \end{itemize}


      \item \textbf{Case \CASEhandled{2.AEx.Unit}:}  :
        
        \begin{llproof}
          \substextendPf{\Gamma}{\Omega}   {Given}
          \eqPf{\unitty}{[\Omega]\unitty}   {By definition of substitution}
          \notinPf{\ahat} {\FV{\unitty}}   {By definition of }
          \declsubjudgPf{[\Omega]\Gamma}  {[\Omega]\ahat}  {[\Omega]\unitty}  {Given}
          \proofsep
          \instjudgPf{\Gamma}  {\ahat}  {\unitty}  {\Delta}  {By \Theoremref{thm:instantiation-completes} (1)}
\Hand          \substextendPf{\Omega}{\Omega'}    {\ditto}
\Hand          \substextendPf{\Delta}{\Omega'}    {\ditto}
          \proofsep
          \eqPf{\unitty}{[\Gamma]\unitty}   {By definition of substitution}
          \notinPf{\ahat} {\FV{\unitty}}   {By definition of }
          \proofsep
\Hand          \subjudgPf{\Gamma}  {\ahat}  {\unitty}  {\Delta}   {By \SubInstL}
        \end{llproof}


      \item \textbf{Case \CASEhandledsymm{2.AEx.Uvar}:}  :

          Similar to Case 2.AEx.Unit, using  and .


      \item \textbf{Case \CASEhandled{2.AEx.Arrow}:}  :
        
                   Since  is an arrow, it cannot be exactly .
                   
                   Suppose, for a contradiction, that .
                   
                       \begin{llproof}
                         \subtermofPf {\ahat} {[\Gamma]B}   {}
                         \subtermofPf {[\Omega]\ahat} {[\Omega][\Gamma]B}   {By a property of substitution}
                         \proofsep
                         \substextendPf {\Gamma}{\Omega}   {Given}
                         \eqPf {[\Omega][\Gamma]B} {[\Omega]B}   {By \Lemmaref{lem:subst-extension-invariance}}
                         \proofsep
                         \subtermofPf {[\Omega]\ahat} {[\Omega]B}   {By above equality}
                         \proofsep
                         \neqPf {[\Gamma]B} {\ahat}      {Given (2.AEx.Arrow)}
                         \neqPf {[\Omega][\Gamma]B} {[\Omega]\ahat}      {By a property of substitution}
                         \neqPf {[\Omega]B} {[\Omega]\ahat}      {By \Lemmaref{lem:subst-extension-invariance}}
                         \proofsep
                         \propersubtermofPf {[\Omega]\ahat} {[\Omega]B}   {Follows from  and }
                         \occursinsidearrowPf {[\Omega]\ahat} {[\Omega]B}   { has the form }
                         \declsubjudgPf{[\Omega]\Gamma} {[\Omega]\ahat} {[\Omega]B}   {Given}
                         \Pf{}{}{[\Omega]B\text{~is a monotype}}  { is predicative}
                         \notoccursinsidearrowPf {[\Omega]\ahat} {[\Omega]B}   {By \Lemmaref{lem:occurrence} (ii)}
                       \contraPf{\ahat \notin \FV{[\Gamma]B}}
                      \end{llproof}

                   \begin{llproof}
                     \instjudgPf{\Gamma} {\ahat} {[\Gamma]B} {\Delta}   {By \Theoremref{thm:instantiation-completes} (1)}
\Hand             \substextendPf{\Delta} {\Omega'}   {\ditto}
\Hand             \substextendPf{\Omega} {\Omega'}   {\ditto}
\Hand             \subjudgPf{\Gamma} {\ahat} {\underbrace{[\Gamma]B}_{B_1 \arr B_2}} {\Delta}   {By \SubInstL}
                   \end{llproof}
    \end{itemize}



    \item \textbf{Case \CASEhandledsymm{2.BEx}:  is not polymorphic and  is an existential variable}:
         

         We split on the form of .

            \begin{itemize}
            \item \textbf{Case \CASEhandledsymm{2.BEx.Unit}} (), \\
                  \textbf{Case \CASEhandledsymm{2.BEx.Uvar}} (), \\
                  \textbf{Case \CASEhandledsymm{2.BEx.Arrow}} (): \\
                Similar to Cases \CASEhandled{2.AEx.Unit}, \CASEhandled{2.AEx.Uvar} and \CASEhandled{2.AEx.Arrow},
                but using part (2) of \Theoremref{thm:instantiation-completes} instead of part (1),
                and applying \SubInstR instead of \SubInstL as the final step.
            \end{itemize}

    \item \textbf{Case \CASEhandled{2.Units}}: :

      \begin{llproof}
\Hand        \subjudgPf{\Gamma}{\unitty}{\unitty}{\Gamma}   {By \SubUnit}
          \substextendPf{\Gamma} {\Omega}   {Given}
\Hand          \substextendPf{\Delta} {\Omega}   {}
\Hand          \substextendPf{\Omega} {\Omega'}   {By \Lemmaref{lem:substextend-reflexivity} and }
      \end{llproof}
      


    \item \textbf{Case \CASEhandled{2.Uvars}}: :      

      \begin{llproof}
         \Pf{}{}{\alpha \in \Omega}  {By inversion on \DsubVar}
         \substextendPf{\Gamma}{\Omega}   {Given}
         \Pf{}{}{\alpha \in \Gamma} {By \Lemmaref{lem:extension-order}}
\Hand         \subjudgPf{\Gamma}{\alpha}{\alpha}{\Gamma}  {By \SubVar}
\Hand         \substextendPf{\Delta} {\Omega}   {}
\Hand         \substextendPf{\Omega} {\Omega'}   {By \Lemmaref{lem:substextend-reflexivity} and }        
      \end{llproof}
      


    \item \textbf{Case \CASEhandled{2.Arrows}}:  and :

      Only rule \DsubArr could have been used.

      \begin{llproof}
        \declsubjudgPf{[\Omega]\Gamma}  {[\Omega]B_1}  {[\Omega]A_1}    {Subderivation}
        \subjudgPf{\Gamma}  {[\Gamma]B_1}  {[\Gamma]A_1}  {\Theta}    {By i.h.}
        \substextendPf{\Theta} {\Omega_0}    {\ditto}
        \substextendPf{\Omega} {\Omega_0}    {\ditto}
        \substextendPf{\Gamma}{\Omega}   {Given}
        \substextendPf{\Gamma}{\Omega_0}   {By \Lemmaref{lem:substextend-transitivity}}
        \proofsep
        \substextendPf{\Theta}{\Omega_0}   {Above}
        \proofsep
        \eqPf{[\Omega]\Gamma}{[\Omega]\Theta}  {By \Lemmaref{lem:completes-confluence}}
        \proofsep
        \declsubjudgPf{[\Omega]\Gamma}  {[\Omega]A_2}  {[\Omega]B_2}    {Subderivation}
        \declsubjudgPf{[\Omega]\Theta}  {[\Omega]A_2}  {[\Omega]B_2}    {By above equality}
        \proofsep
        \eqPf{[\Omega]A_2} {[\Omega][\Gamma]A_2}  {By \Lemmaref{lem:subst-extension-invariance}}
        \eqPf{[\Omega]B_2} {[\Omega][\Gamma]B_2}  {By \Lemmaref{lem:subst-extension-invariance}}
        \proofsep
        \declsubjudgPf{[\Omega]\Theta}  {[\Omega][\Gamma]A_2}  {[\Omega][\Gamma]B_2}    {By above equalities}
        \subjudgPf{\Theta}  {[\Theta][\Gamma]A_2}  {[\Theta][\Gamma]B_2}  {\Delta}    {By i.h.}
\Hand   \substextendPf{\Delta} {\Omega'}    {\ditto}
        \substextendPf{\Omega_0} {\Omega'}    {\ditto}
        \decolumnizePf
        \subjudgPf{\Gamma}{([\Gamma]A_1) \arr ([\Gamma]A_2)}{([\Gamma]B_1) \arr ([\Gamma]B_2)}{\Delta} {By \SubArr}
\Hand   \subjudgPf{\Gamma}{[\Gamma](A_1 \arr A_2)}{[\Gamma](B_1 \arr B_2)}{\Delta} {By definition of substitution}
\Hand   \substextendPf{\Omega} {\Omega'}  {By \Lemmaref{lem:substextend-transitivity}}
      \end{llproof}\qedhere
    \end{itemize}
  \end{itemize}
\end{proof}



\completeness*
\begin{proof}
  Let  and . \\
  By \Lemmaref{lem:substextend-reflexivity}, , so . \\
  By \Lemmaref{lem:declarative-well-formed},  and ; since ,
  we have
   and .
  \\
  By \Theoremref{thm:completing-completeness},
  there exists  such that .
  \\
  Since  and  is a declarative context with no existentials,  for all ,
  so we actually have , which was to be shown.
\end{proof}




\clearpage
\section{Completeness of Typing}



\typingcompleteness*
\begin{proof}
  By induction on the given declarative derivation.

  \begin{itemize}
     \DerivationProofCase{\DeclVar}
          {(x : A) \in [\Omega]\Gamma}
          {\declsynjudg{[\Omega]\Gamma}{x}{A}}
          
          \begin{llproof}
            \inPf{(x : A)} {[\Omega]\Gamma}  {Premise}
            \substextendPf{\Gamma}{\Omega}   {Given}
            \inPf{(x : A')}{\Gamma\text{~where }}   {From definition of context application}
            \LetPf{\Delta}{\Gamma} {}
            \LetPf{\Omega'}{\Omega} {}
\Hand       \substextendPf{\Gamma}{\Omega}   {Given}
\Hand       \substextendPf{\Omega}{\Omega}   {By \Lemmaref{lem:substextend-reflexivity}}
\Hand       \synjudgPf{\Gamma}{x}{A'}{\Gamma}   {By \Var}
                   \eqPf{[\Omega]A'}{[\Omega]A}   {Above}
\Hand              \continueeqPf{A}   {}
          \end{llproof}

     \DerivationProofCase{\DeclSub}
          {\declsynjudg{[\Omega]\Gamma}{e}{B}
            \\
            \declsubjudg{[\Omega]\Gamma}{B}{[\Omega]A}
          }
          {\declchkjudg{[\Omega]\Gamma}{e}{[\Omega]A}}

          \begin{llproof}
            \declsynjudgPf{[\Omega]\Gamma}{e}{B}   {Subderivation}
            \synjudgPf{\Gamma}{e}{B'}{\Theta}      {By i.h.}
            \eqPf{B}{[\Omega]B'}   {\ditto}
            \substextendPf{\Theta}{\Omega_0}   {\ditto}
            \substextendPf{\Omega}{\Omega_0}   {\ditto}
            \proofsep
            \substextendPf{\Gamma}{\Omega}   {Given}
            \substextendPf{\Gamma}{\Omega_0}   {By \Lemmaref{lem:substextend-transitivity}}
            \declsubjudgPf{[\Omega]\Gamma}{B}{[\Omega]A}  {Subderivation}
            \eqPf{[\Omega]\Gamma} {[\Omega]\Theta}   {By \Lemmaref{lem:completes-confluence}}
            \declsubjudgPf{[\Omega]\Theta}{B}{[\Omega]A}  {By above equalities}
            \substextendPf{\Theta}{\Omega_0}  {Above}
            \subjudgPf{\Theta}{[\Theta]B'}{[\Theta]A}{\Delta}  {By \Theoremref{thm:completing-completeness}}
            \substextendPf{\Delta}{\Omega'}   {\ditto}
            \substextendPf{\Omega_0}{\Omega'}   {\ditto}
\Hand   \substextendPf{\Delta}{\Omega'}   {By \Lemmaref{lem:substextend-transitivity}}
\Hand   \substextendPf{\Omega}{\Omega'}   {By \Lemmaref{lem:substextend-transitivity}}
            \proofsep
\Hand   \chkjudgPf{\Gamma}{e}{A}{\Delta}  {By \Sub}
          \end{llproof}

     \DerivationProofCase{\DeclAnno}
          {\judgetp{[\Omega]\Gamma}{A}
            \\
            \declchkjudg{[\Omega]\Gamma}{e_0}{A}
          }
          {\declsynjudg{[\Omega]\Gamma}{(e_0 : A)}{A}}

          \begin{llproof}
            \eqPf{A} {[\Omega]A}   {Source type annotations cannot contain evars}
            \continueeqPf           {[\Gamma]A}   {\ditto}
            \declchkjudgPf{[\Omega]\Gamma}{e_0}{A}   {Subderivation}
            \declchkjudgPf{[\Omega]\Gamma}{e_0}{[\Omega]A}   {By above equality}
            \chkjudgPf{\Gamma}{e_0}{[\Gamma]A}{\Delta}   {By i.h.}
\Hand       \substextendPf{\Delta}{\Omega}   {\ditto}
\Hand       \substextendPf{\Omega}{\Omega'}   {\ditto}
            \proofsep
            \judgetpPf{\Gamma}{A}   {Given}
            \proofsep
            \synjudgPf{\Gamma}{(e_0 : A)}{A}{\Delta}   {By \Anno}
            \eqPf{A} {[\Omega']A}   {Source type annotations cannot contain evars}
\Hand       \synjudgPf{\Gamma}{(e_0 : [\Omega']A)}{[\Omega']A}{\Delta}   {By above equality}
          \end{llproof}

     \DerivationProofCase{\DeclUnitIntro}
          {}
          {\declchkjudg{[\Omega]\Gamma}{\unitexp}{\unitty}}

          We have .  Either  or .

          In the former case:

          \begin{llproof}
            \LetPf{\Delta}{\Gamma} {}
            \LetPf{\Omega'}{\Omega} {}
\Hand       \substextendPf{\Gamma}{\Omega}   {Given}
\Hand       \substextendPf{\Omega}{\Omega'}   {By \Lemmaref{lem:substextend-reflexivity}}
             \chkjudgPf{\Gamma}{\unitexp}{\unitty}{\Gamma}   {By \UnitIntro}
\Hand       \chkjudgPf{\Gamma}{\unitexp}{[\Gamma]\unitty}{\Gamma}   {}
          \end{llproof}

          In the latter case:

          \begin{llproof}
             \synjudgPf{\Gamma}{\unitexp}{\unitty}{\Gamma}   {By \UnitIntroSyn}
             \declsubjudgPf{[\Omega]\Gamma}{\unitty}{\unitty}   {By \DsubUnit}
             \proofsep
             \eqPf{\unitty} {[\Omega]\unitty}   {By definition of substitution}
             \continueeqPf {[\Omega][\Gamma]\ahat}   {By }
             \continueeqPf {[\Omega]\ahat}   {By \Lemmaref{lem:subst-extension-invariance}}
             \proofsep
             \declsubjudgPf{[\Omega]\Gamma}{[\Omega]\unitty}{[\Omega]\ahat}   {By above equalities}
             \subjudgPf{\Gamma}{\unitty}{\ahat}{\Delta}   {By \Theoremref{thm:instantiation-completes} (1)}
             \eqPf{\unitty}{[\Gamma]\unitty}  {By definition of substitution}
             \eqPf{\ahat}{[\Gamma]\ahat}  {}
             \subjudgPf{\Gamma}{[\Gamma]\unitty}{[\Gamma]\ahat}{\Delta}   {By above equalities}
\Hand  \substextendPf{\Omega}{\Omega'} {\ditto}
\Hand  \substextendPf{\Delta}{\Omega'} {\ditto}
             \chkjudgPf{\Gamma}{\unitexp}{\ahat}{\Delta}   {By \Sub}
\Hand        \chkjudgPf{\Gamma}{\unitexp}{[\Gamma]A}{\Delta}   {By }
          \end{llproof}


     \DerivationProofCase{\DeclAllIntro}
           {\declchkjudg{[\Omega]\Gamma, \alpha}{e}{A_0}
           }
           {\declchkjudg{[\Omega]\Gamma}{{e}}{\alltype{\alpha}{A_0}}}

           \begin{llproof}
             \eqPf{[\Omega]A}{\alltype{\alpha}{A_0}}   {Given}
             \continueeqPf{\alltype{\alpha}{[\Omega]A'}}  {By def. of subst. and predicativity of }
             \eqPf{A_0}{[\Omega]A'}   {Follows from above equality}
             \declchkjudgPf{[\Omega]\Gamma, \alpha}{e}{[\Omega]A'}  {Subderivation and above equality}
             \proofsep
             \substextendPf{\Gamma}{\Omega}  {Given}
             \substextendPf{\Gamma, \alpha}{\Omega, \alpha}  {By \substextendUU}
             \proofsep
             \eqPf{[\Omega]\Gamma, \alpha}{[\Omega, \alpha](\Gamma, \alpha)}  {By definition of context substitution}
             \declchkjudgPf{[\Omega, \alpha](\Gamma, \alpha)}{e}{[\Omega]A'}  {By above equality}
             \declchkjudgPf{[\Omega, \alpha](\Gamma, \alpha)}{e}{[\Omega, \alpha]A'}  {By definition of substitution}
           \end{llproof}

           \begin{llproof}
             \chkjudgPf{\Gamma, \alpha}{e}{[\Gamma, \alpha]A'}{\Delta'}  {By i.h.}
             \substextendPf{\Delta'}{\Omega_0'}   {\ditto}
             \substextendPf{\Omega, \alpha}{\Omega_0'}   {\ditto}
             \substextendPf{\Gamma, \alpha}{\Delta'}  {By \Lemmaref{lem:typing-extension}}
             \eqPf{\Delta'} {\Delta, \alpha, \Theta}   {By \Lemmaref{lem:extension-order} (i)}
             \substextendPf{\Delta, \alpha, \Theta}{\Omega_0'}   {By above equality}
             \eqPf{\Omega_0'} {\Omega', \alpha, \Omega_Z}   {By \Lemmaref{lem:extension-order} (i)}
\Hand        \substextendPf{\Delta} {\Omega'}   {\ditto}
\Hand        \substextendPf{\Omega}{\Omega'}  {By \Lemmaref{lem:extension-order} on }
             \proofsep
             \chkjudgPf{\Gamma, \alpha}{e}{[\Gamma, \alpha]A'}{\Delta, \alpha, \Theta}   {By above equality}
             \chkjudgPf{\Gamma, \alpha}{e}{[\Gamma]A'}{\Delta, \alpha, \Theta}   {By definition of substitution}
             \chkjudgPf{\Gamma}{e}{\alltype{\alpha}{[\Gamma]A'}}{\Delta}   {By \AllIntro}
\Hand        \chkjudgPf{\Gamma}{e}{[\Gamma](\alltype{\alpha}{A'})}{\Delta}   {By definition of substitution}
           \end{llproof}


     \DerivationProofCase{\DeclAllApp}
            {\judgetp{[\Omega]\Gamma}{\tau}
              \\ 
             \declappjudg{[\Omega]\Gamma}{e}{[\tau/\alpha]A_0}{C}}
            {\declappjudg{[\Omega]\Gamma}{e}{\underbrace{\alltype{\alpha}A_0}_{[\Omega]A}}{C}}

            \begin{llproof}
              \judgetpPf{[\Omega]\Gamma}{\tau}   {Subderivation}
              \proofsep
              \eqPf{[\Omega]A}{\alltype{\alpha} A_0}   {Given}
              \continueeqPf{\alltype{\alpha} [\Omega]A'}  {By def. of subst. and predicativity of }
              \declappjudgPf{[\Omega]\Gamma}{e}{[\tau/\alpha][\Omega]A'}{C}  {Subderivation and above equality}
              \substextendPf{\Gamma}{\Omega}   {Given}
              \substextendPf{\Gamma, \ahat}{\Omega, \hypeq{\ahat}{\tau}}  {By \substextendSolve}
              \proofsep
              \eqPf{[\Omega]\Gamma} {[\Omega, \hypeq{\ahat}{\tau}](\Gamma, \ahat)}  {By definition of context application}
              \declappjudgPf{[\Omega, \hypeq{\ahat}{\tau}](\Gamma, \ahat)}{e}{[\tau/\alpha][\Omega]A'}{C}  {By above equality}
              \declappjudgPf{[\Omega, \hypeq{\ahat}{\tau}](\Gamma, \ahat)}{e}{[\tau/\alpha][\Omega, \hypeq{\ahat}{\tau}]A'}{C}  {By def.\ of subst.}
              \eqPf{\big(\big[[\Omega]\tau/\alpha\big]\big[\Omega, \hypeq{\ahat}{\tau}\big]A'\big)} {\big([\Omega, \hypeq{\ahat}{\tau}][\ahat/\alpha]A'\big)}  {By distributivity of substitution}
              \eqPf{\tau}{[\Omega]\tau}   {}
              \eqPf{\big(\big[\tau/\alpha\big]\big[\Omega, \hypeq{\ahat}{\tau}\big]A'\big)} {\big([\Omega, \hypeq{\ahat}{\tau}][\ahat/\alpha]A'\big)}  {By above equality}
              \declappjudgPf{[\Omega, \hypeq{\ahat}{\tau}](\Gamma, \ahat)}{e}{[\Omega, \hypeq{\ahat}{\tau}][\ahat/\alpha]A'}{C}  {By above equality}
            \end{llproof}

            \begin{llproof}
              \appjudgPf {\Gamma, \ahat} {e} {[\ahat/\alpha]A'} {C'} {\Delta}   {By i.h.}
\Hand         \eqPf{C}{[\Omega]C'}   {\ditto}
\Hand         \substextendPf{\Delta}{\Omega'}   {\ditto}
\Hand         \substextendPf{\Omega}{\Omega'}   {\ditto}
              \proofsep
\Hand         \appjudgPf{\Gamma}{e}{\alltype{\alpha} A'} {C'} {\Delta}   {By \AllApp}
            \end{llproof}

     \DerivationProofCase{\DeclArrIntro}
          {\declchkjudg{[\Omega]\Gamma, x : A_1'}{e_0}{A_2'}
          }
          {\declchkjudg{[\Omega]\Gamma}{\lam{x} e_0}{A_1' \arr A_2'}}

          We have .  Either 
          where  and ---or
           and .

          In the former case:

          \begin{llproof}
            \declchkjudgPf{[\Omega]\Gamma, x : A_1'}{e_0}{A_2'}   {Subderivation}
            \proofsep
            \eqPf{A_1'}  {[\Omega]A_1}   {Known in this subcase}
            \continueeqPf {[\Omega][\Gamma]A_1}   {By \Lemmaref{lem:subst-extension-invariance}}
            \eqPf{[\Omega]A_1'} {[\Omega][\Omega][\Gamma]A_1}   {Applying  on both sides}
            \continueeqPf {[\Omega][\Gamma]A_1}   {By idempotence of substitution}
            \proofsep
            \eqPf{[\Omega]\Gamma, x : A_1'}  {[\Omega, x : A_1'](\Gamma, x : [\Gamma]A_1)}   {By definition of context application}
            \proofsep
            \declchkjudgPf{[\Omega, x : A_1'](\Gamma, x : [\Gamma]A_1)}{e_0}{A_2'}   {By above equality}
            \proofsep
            \substextendPf{\Gamma}{\Omega}   {Given}
            \substextendPf{\Gamma, x : [\Gamma]A_1}{\Omega, x : A_1'}   {By \substextendVV}
            \proofsep
            \chkjudgPf{\Gamma, x : [\Gamma]A_1}{e_0}{A_2}{\Delta'}   {By i.h.}
            \substextendPf{\Delta'}{\Omega_0'}   {\ditto}
            \substextendPf{\Omega, x : A_1'}{\Omega_0'}   {\ditto}
            \eqPf{\Omega_0'} {\Omega', x : A_1', \Theta}   {By \Lemmaref{lem:extension-order} (v)}
\Hand       \substextendPf{\Omega} {\Omega'}  {\ditto}
            \proofsep
            \substextendPf{\Gamma, x : [\Gamma]A_1}{\Delta'}   {By \Lemmaref{lem:typing-extension}}
            \eqPf{\Delta'}{\Delta, x : \cdots, \Theta}   {By \Lemmaref{lem:extension-order} (v)}
            \substextendPf{\Delta, x : \cdots, \Theta}{\Omega', x : A_1', \Theta}  {By above equalities}
\Hand       \substextendPf{\Delta}{\Omega'}  {By \Lemmaref{lem:extension-order} (v)}
\decolumnizePf
            \chkjudgPf{\Gamma, x : [\Gamma]A_1}{e_0}{[\Gamma]A_2}{\Delta, \alpha, \Theta}   {By above equality}
            \chkjudgPf{\Gamma}{\lam{x} e_0}{([\Gamma]A_1) \arr ([\Gamma]A_2)}{\Delta}  {By \ArrIntro}
\Hand      \chkjudgPf{\Gamma}{\lam{x} e_0}{[\Gamma](A_1 \arr A_2)}{\Delta}  {By definition of substitution}
          \end{llproof}
          
          In the latter case:
          
          \begin{llproof}
            \eqPf{[\Omega]\ahat}{A_1' \arr A_2'}  {Known in this subcase}
            \declchkjudgPf{[\Omega]\Gamma, x : A_1'}{e_0}{A_2'}   {Subderivation}
            \substextendPf{\Gamma}{\Omega}   {Given}
            \substextendPf{\Gamma, \ahat, \bhat} {\Omega, \hypeq{\ahat}{A_1'}, \hypeq{\bhat}{A_2'}}
                      {By \substextendSolve twice}
            \eqPf{[\Omega]\ahat} {[\Omega]A_1'}   {By definition of substitution}
            \substextendPf{\Gamma, \ahat, \bhat, x : \ahat} {\Omega, \hypeq{\ahat}{A_1'}, \hypeq{\bhat}{A_2'}, x : A_1'}
                      {By \substextendVV}
            \eqPf{[\Omega]\Gamma, x : A_1'}
                   {\big[\Omega, \hypeq{\ahat}{A_1'}, \hypeq{\bhat}{A_2'}, x : A_1'\big]\big(\Gamma, \ahat, \bhat, x : \ahat\big)}
                   {By definition of context application}
            \LetPf{\Omega_0}{(\Omega, \hypeq{\ahat}{A_1'}, \hypeq{\bhat}{A_2'}, x : A_1')}   {}
            \declchkjudgPf{[\Omega_0](\Gamma, \ahat, \bhat, x : \ahat)}{e_0}{A_2'}   {By above equality}
            \chkjudgPf{\Gamma, \ahat, \bhat, x : \ahat} {e_0} {\bhat} {\Delta'}     {By i.h. with }
            \substextendPf{\Delta'} {\Omega_0'}     {\ditto}
            \substextendPf{\Omega_0} {\Omega_0'}     {\ditto}
          \end{llproof}

          \begin{llproof}
            \substextendPf{\Gamma, \ahat, \bhat, x : \ahat}{\Delta'}     {By \Lemmaref{lem:typing-extension}}
            \eqPf{\Delta'}{\Delta, x : \ahat, \Theta}   {By \Lemmaref{lem:extension-order} (v)}
            \substextendPf{\Delta, x : \ahat, \Theta} {\Omega_0'}   {By above equality}
            \eqPf{\Omega_0'} {\Omega'', x : \cdots, \Omega_Z}     {By \Lemmaref{lem:typing-extension}}
\Hand       \substextendPf{\Delta}{\Omega''}     {\ditto}
            \substextendPf{\Gamma, \ahat, \bhat}{\Delta}   {\ditto}
            \substextendPf{\Omega_0} {\underbrace{\Omega'', x : \cdots, \Omega_Z}_{\Omega_0'}}     {By above equality}
            \substextendPf{\Omega, \hypeq{\ahat}{A_1'}, \hypeq{\bhat}{A_2'}, x : A_1'}
                          {\Omega'', x : \cdots, \Omega_Z}
                          {By def.\ of }
            \eqPf{\Omega''}
                      {\Omega', \hypeq{\ahat}{\dots}, \dots}
                      {By \Lemmaref{lem:extension-order} (iii)}
\Hand       \substextendPf{\Omega}{\Omega'}  {\ditto}
            \decolumnizePf
            \chkjudgPf{\Gamma, \ahat, \bhat, x : \ahat} {e_0} {\bhat} {\Delta, x : \ahat, \Theta}     {By above equality}
            \chkjudgPf{\Gamma} {\lam{x} e_0} {\ahat \arr \bhat} {\Delta}     {By \ArrIntroSyn}
            \proofsep
            \eqPf{[\Gamma]\ahat}{\ahat}    {By definition of substitution}
            \eqPf{[\Gamma]\bhat}{\bhat}    {By definition of substitution}
            \chkjudgPf{\Gamma}{\lam{x} e_0}{([\Gamma]\ahat) \arr ([\Gamma]\bhat)}{\Delta} {By above equalities}
\Hand       \chkjudgPf{\Gamma}{\lam{x} e_0}{[\Gamma](\ahat \arr \bhat)}{\Delta} {By definition of substitution}
          \end{llproof}

     \DerivationProofCase{\DeclArrElim}
          {\declsynjudg{[\Omega]\Gamma}{e_1}{B}
            \\
            \declappjudg{[\Omega]\Gamma}{e_2}{B}{A}
          }
          {\declsynjudg{[\Omega]\Gamma}{e_1\,e_2}{A}}

          \begin{llproof}
            \declsynjudgPf{[\Omega]\Gamma}{e_1}{B}   {Subderivation}
            \substextendPf{\Gamma}{\Omega}  {Given}
            \synjudgPf{\Gamma} {e_1} {B'} {\Theta}   {By i.h.}
            \eqPf{B} {[\Omega]B'}  {\ditto}
            \substextendPf{\Theta} {\Omega_0'}  {\ditto}
            \substextendPf{\Omega} {\Omega_0'}  {\ditto}
            \proofsep
            \declappjudgPf{[\Omega]\Gamma}{e_2}{B}{A}  {Subderivation}
            \declappjudgPf{[\Omega]\Gamma}{e_2}{[\Omega]B'}{A}  {By above equality}
            \substextendPf{\Gamma}{\Omega_0'}  {By \Lemmaref{lem:substextend-transitivity}}
            \eqPf{[\Omega]\Gamma} {[\Omega]\Omega}   {By \Lemmaref{lem:completes-stability}}
            \continueeqPf {[\Omega_0']\Omega_0'}   {By \Lemmaref{lem:finishing-completions}}
            \continueeqPf {[\Omega_0']\Gamma}   {By \Lemmaref{lem:completes-stability}}
            \continueeqPf {[\Omega_0']\Theta}   {By \Lemmaref{lem:completes-confluence}}
            \declappjudgPf{[\Omega_0']\Theta}{e_2}{[\Omega]B'}{A}  {By above equality}
            \eqPf{[\Omega]B'}{[\Omega_0']B'}   {By \Lemmaref{lem:finishing-types}}
            \eqPf{[\Omega_0']B'}{[\Omega_0'][\Theta]B'}   {By \Lemmaref{lem:subst-extension-invariance}}
            \declappjudgPf{[\Omega_0']\Theta}{e_2}{[\Omega][\Theta]B'}{A}  {By above equalities}
            \proofsep
            \appjudgPf{\Theta}{e_2}{[\Theta]B'}{A'}{\Delta}  {By i.h. with }
\Hand            \eqPf{A} {[\Omega]A'}  {\ditto}
\Hand            \substextendPf{\Delta}{\Omega'} {\ditto}
            \substextendPf{\Omega_0'}{\Omega'} {\ditto}
            \substextendPf{\Omega}{\Omega'} {By \Lemmaref{lem:substextend-transitivity}}
\Hand            \synjudgPf{\Gamma}{e_1\,e_2}{A'}{\Delta}  {By \ArrElim}
          \end{llproof}


\clearpage
      \DerivationProofCase{\DeclArrApp}
            {\declchkjudg{[\Omega]\Gamma}{e}{B}}
            {\declappjudg{[\Omega]\Gamma}{e}{\underbrace{B \arr C}_{[\Omega]A}}{C}}

          We have .  Either 
          where  and ---or
           where 
          and .

          In the former case:

          \begin{llproof}
            \declchkjudgPf{[\Omega]\Gamma}{e}{B}  {Subderivation}
            \eqPf{B} {[\Omega]B_0}  {Known in this subcase}
            \proofsep
            \substextendPf{\Gamma}{\Omega}   {Given}
            \proofsep
            \chkjudgPf{\Gamma}{e}{[\Gamma]B_0}{\Delta}   {By i.h.}
            \appjudgPf{\Gamma}{e}{([\Gamma]B_0) \arr ([\Gamma]C_0)}{[\Gamma]C_0}{\Delta}  {By \ArrApp}
\Hand       \substextendPf{\Delta}{\Omega'}   {\ditto}
\Hand       \substextendPf{\Omega}{\Omega'}   {\ditto}
            \LetPf{C'} {[\Gamma]C_0}  {}
            \eqPf{C}{[\Omega]C_0}  {Known in this subcase}
            \continueeqPf {[\Omega][\Gamma]C_0}   {By \Lemmaref{lem:subst-extension-invariance}}
\Hand       \continueeqPf {[\Omega]C'}   {}
\Hand       \appjudgPf{\Gamma}{e}{[\Gamma](B_0 \arr C_0)}{[\Gamma]C_0}{\Delta}  {By definition of substitution}
          \end{llproof}
          
          In the latter case, ,
          so the context  must have the form .
            
            \begin{llproof}
              \substextendPf{\Gamma}{\Omega}   {Given}
              \substextendPf{\Gamma_0[\ahat]}{\Omega}   {}
              \eqPf{[\Omega]A}  {B \arr C}   {Above}
              \eqPf{[\Omega]\ahat}  {B \arr C}   {}
              \eqPf{\Omega} {\Omega_0[\hypeq{\ahat}{A_0}]
                             \AND   [\Omega]A_0 = B \arr C}    {Follows from }
              \LetPf{\Gamma'}{\Gamma_0[\ahat_2, \ahat_1, \hypeq{\ahat}{\ahat_1 \arr \ahat_2}]}   {}
              \LetPf{\Omega_0'}{\Omega_0[\hypeq{\ahat_2}{[\Omega]C}, \hypeq{\ahat_1}{[\Omega]B}, \hypeq{\ahat}{\ahat_1 \arr \ahat_2}]}   {}
              \decolumnizePf
              \substextendPf{\Gamma'}{\Omega_0'}    {By \Lemmaref{lem:parallel-admissibility} (ii) twice}
              \proofsep
              \declchkjudgPf{[\Omega]\Gamma}{e}{B}   {Subderivation}
              \proofsep
              \substextendPf{\Omega}{\Omega_0'}   {By \Lemmaref{lem:extension-addsolve}}
                       \trailingjust{then \Lemmaref{lem:parallel-admissibility} (iii)}
              \eqPf{[\Omega]\Gamma} {[\Omega]\Omega}   {By \Lemmaref{lem:completes-stability}}
              \continueeqPf {[\Omega_0']\Omega_0'}   {By \Lemmaref{lem:finishing-completions}}
              \continueeqPf {[\Omega_0']\Gamma'}   {By \Lemmaref{lem:completes-confluence}}
              \proofsep
              \eqPf{B} {[\Omega_0']\ahat_1}  {By definition of }
\declchkjudgPf{[\Omega_0']\Gamma'}{e}{[\Omega_0']\ahat_1}   {By above equalities}
              \proofsep
              \chkjudgPf{\Gamma'}{e}{[\Gamma']\ahat_1}{\Delta}     {By i.h.}
\Hand         \substextendPf{\Delta}{\Omega'}   {\ditto}
              \substextendPf{\Omega_0'}{\Omega'}   {\ditto}              
\Hand         \substextendPf{\Omega}{\Omega'}   {By \Lemmaref{lem:substextend-transitivity}}
              \proofsep
              \eqPf {[\Gamma']\ahat_1} {\ahat_1}   {}
              \chkjudgPf{\Gamma'}{e}{\ahat_1}{\Delta}   {By above equality}
            \end{llproof}
            
            \begin{llproof}
              \appjudgPf{\Gamma} {e} {\ahat} {\ahat_2}  {\Delta}  {By \SolveApp}
              \LetPf{C'} {\ahat_2} {}
              \eqPf{C}{[\Omega_0']\ahat_2}  {By definition of }
              \continueeqPf{[\Omega']\ahat_2}  {By \Lemmaref{lem:finishing-types}}
\Hand         \continueeqPf{[\Omega']C'}  {By above equality}
\Hand         \appjudgPf{\Gamma}{e}{[\Gamma]A}{C'}{\Delta}  { and }
            \end{llproof}




\DerivationProofCase{\DeclUnitIntroSyn}
                { }
                {\declsynjudg{[\Omega]\Gamma}{\unitexp}{\unitty}}

                \begin{llproof}
                  \eqPf {\unitty} {A}   {Given}
                  \synjudgPf{\Gamma} {\unitexp} {\unitty}  {\Gamma}   {By \UnitIntroSyn}
                  \LetPf{\Delta}{\Gamma}  {}
                  \LetPf{\Omega'}{\Omega}  {}
                  \substextendPf{\Gamma}{\Omega}  {Given}
\Hand             \substextendPf{\Delta}{\Omega}  {By above equality}
\Hand             \substextendPf{\Omega}{\Omega'}  {By \Lemmaref{lem:substextend-reflexivity}}
                  \LetPf{A'}{\unitty}  {}
\Hand             \synjudgPf{\Gamma} {\unitexp} {A'}  {\Delta}   {By above equalities}
\Hand             \eqPf {\unitty}  {[\Omega]A'}  {By definition of substitution}
                \end{llproof}

                \bigskip

          \DerivationProofCase{\DeclArrIntroSyn}
                {\judgetp{[\Omega]\Gamma}{\sigma \arr \tau}
                 \\
                 \declchkjudg{[\Omega]\Gamma, x : \sigma}{e_0}{\tau}
                }
                {\declsynjudg{[\Omega]\Gamma}{\lam{x} e_0}{\sigma \arr \tau}}

                \medskip

                \begin{llproof}
                  \eqPf{(\sigma \arr \tau)}{A}    {Given}
                  \declchkjudgPf{[\Omega]\Gamma, x : \sigma}{e_0}{\tau}   {Subderivation}
                  \proofsep
                  \LetPf{\Gamma'} {(\Gamma, \ahat, \bhat, x : \ahat)}  {}
                  \LetPf{\Omega_0} {(\Omega, \hypeq{\ahat}{\sigma}, \hypeq{\bhat}{\tau}, x : \sigma)}   {}
                  \proofsep
                  \substextendPf{\Gamma}{\Omega}   {Given}
                  \substextendPf{\Gamma'}{\Omega_0}   {By \substextendSolve twice, then \substextendVV}
                  \proofsep
                  \eqPf{[\Omega_0]\Gamma'} {\big([\Omega]\Gamma, x : \sigma\big)}   {By definition of context application}
                  \eqPf{\tau}{[\Omega_0]\bhat}{By definition of }
                  \declchkjudgPf{[\Omega_0]\Gamma'}{e_0}{[\Omega_0]\bhat}   {By above equalities}
                  \proofsep
                  \chkjudgPf{\Gamma'}{e_0}{\bhat} {\Delta'}   {By i.h.}
                  \substextendPf{\Delta'}{\Omega_0'}    {\ditto}
                  \substextendPf{\Omega_0}{\Omega_0'}    {\ditto}
                  \proofsep
                  \eqPf{\Delta'}{(\Delta, x : \ahat, \Theta)}  {By \Lemmaref{lem:extension-order} (v)}
                  \chkjudgPf{\Gamma, \ahat, \bhat, x : \ahat}{e_0}{\bhat} {\Delta, x : \ahat, \Theta}   {By above equalities}
\substextendPf{(\Delta, x : \ahat, \Theta)} {\Omega_0'}  {By above equality}
                  \eqPf{\Omega_0'}{\Omega', x : \sigma, \Omega_Z}   {By \Lemmaref{lem:extension-order} (v)}
\Hand             \substextendPf{\Delta} {\Omega'}  {\ditto}
                  \synjudgPf{\Gamma}{\lam{x} e_0}{\ahat \arr \bhat}{\Delta}   {By \ArrIntroSyn}
               \end{llproof}

               \begin{llproof}
                  \LetPf{A'} {(\ahat \arr \bhat)}  {}
\Hand              \synjudgPf{\Gamma}{\lam{x} e_0}{A'}{\Delta}   {By above equality}
                  \eqPf{\sigma \arr \tau}{([\Omega_0]\ahat) \arr ([\Omega_0]\bhat)}  {By definition of }
                  \eqPf{\sigma \arr \tau}{[\Omega_0](\ahat \arr \bhat)}  {By definition of substitution}
                  \eqPf{A} {[\Omega_0]A'}   {By above equalities}
\Hand            \eqPf{A} {[\Omega']A'}   {By \Lemmaref{lem:finishing-types}}
                  \proofsep
                  \substextendPf{\Gamma'}{\Delta'}   {By \Lemmaref{lem:typing-extension}}
\Hand             \substextendPf{\Omega}{\Omega'}  {By \Lemmaref{lem:substextend-transitivity}     \qedhere}
                \end{llproof}
\end{itemize}
\end{proof}







\ifnum\OPTIONLoudLabels=1
  \bibliographystyle{plainnatlocalcopy}
\else
  \bibliographystyle{plainnat}
\fi
\bibliography{local}


\end{document}
