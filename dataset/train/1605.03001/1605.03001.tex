The main goal of this section is to prove Lemma~\ref{lemma:algorithm:packing:vs:covering}. However, we first present some additional terminology and a subroutine that will be useful during the proof.

We use~ to denote the union of the vertices used on the holes in a flower~. Observe that for any -flower~ of order~, the structure~ consists of~ pairwise vertex-disjoint chordless paths~ that connect two nonadjacent neighbors of~, whose internal vertices belong to~. Throughout our proofs we will switch between the interpretation of a -flower as a set of intersecting holes in~, and as a collection of induced paths in~ between nonadjacent neighbors of~.

\begin{proposition} \label{proposition:twoflower}
There is a polynomial-time algorithm that, given a graph~ and a vertex~ such that~ is chordal, computes a -flower of order two or correctly determines that no such flower exists.
\end{proposition}
\begin{proof}
Observe that~ is a -flower of order two if and only if~ and~ are pairwise vertex-disjoint induced paths whose endpoints are nonadjacent neighbors of~, and whose internal vertices avoid~. To search for a flower of order two, we try all~ tuples~ for the endpoints of the paths~ and~ in the set~. For all tuples consisting of distinct vertices for which~ and~, we formulate an instance of the \textsc{Disjoint Paths} problem in~ in which we want to connect~ to~ for~ by pairwise vertex-disjoint paths. Any FPT-algorithm for the general \textsc{-Disjoint Paths} problem suffices for this purpose as the parameter is constant. We may also use the linear-time algorithm for \textsc{-Disjoint Paths in Chordal Graphs} due to Kammer and Tholey~\cite{KammerT09} since~ is chordal. If there are vertex-disjoint~-paths~ and~, then we can shortcut them to \emph{induced} paths, which gives a flower of order two. Conversely, if there is a flower of order two, then we will find such paths in the iteration where we guess the tuple containing their endpoints. Hence if the algorithm loops through all tuples without finding a solution to the \textsc{Disjoint Paths} problem, we may conclude that there is no flower of order two and terminate.
\end{proof}

We are now ready to prove Lemma~\ref{lemma:algorithm:packing:vs:covering}, which we re-state for completeness.

\begin{lemma*}
\erdosposa
\end{lemma*}
\begin{proof}
We first present an algorithm to compute a -flower~. Then we show how to compute a hitting set based on the structure of the flower, and argue that the constructed set is not too large. To initialize the procedure, we compute a clique tree~ of~ and root it at an arbitrary node.

\paragraph{Computing a flower.} The algorithm to compute the flower is a local search procedure that maintains a -flower~, initially empty, and iteratively improves this structure according to three rules that are described below.

\begin{enumerate}[(I)]
	\item If~ contains a hole~, then add~ to~. Equivalently, if there are two distinct nonadjacent vertices~ such that graph~ contains an induced~-path~, then add the hole~ to~.\label{improve:addpath}
	\item If the current flower~ contains a hole~ such that the graph~ contains a -flower~ of order two, then remove the hole~ from flower~ and add the holes~ and~ instead.\label{improve:addtwoflower}
	\item If the current flower~ contains a hole~ such that the path~ has endpoints~ and~, and there exists a~ such that:
	\begin{itemize}
		\item , and
		\item the graph~ contains an induced~-path~, i.e., there is an induced~ path~ in~ whose interior vertices avoid the closed neighborhood of~ and avoid all vertices used on other holes of the flower,
	\end{itemize}
	then replace~ in the flower by the hole consisting of~.\label{improve:shorten}
\end{enumerate}

\begin{figure}[t]
\begin{center}
\subfigure[Graph~.]{\label{fig:localsearch1}
\image{LocalSearch1}
}
\subfigure[Clique tree~.]{\label{fig:localsearch2}
\image{LocalSearch2}
}
\subfigure[Searching  path.]{\label{fig:localsearch3}
\image{LocalSearch3}
}
\caption{\ref{fig:localsearch1} A graph~ such that~ is chordal. A -flower of order two is highlighted by dotted edges. \ref{fig:localsearch2} A clique tree of~, visualized by drawing the contents bags inside the nodes. With respect to this clique tree, the -flower can be improved by Step~(\ref{improve:shorten}) by choosing~ and~. Note that~, while~. \ref{fig:localsearch3} The graph~ contains an induced -path~.} \label{fig:localsearch}
\end{center}
\end{figure}

Figure~\ref{fig:localsearch} illustrates these concepts. The applicability of each of the three improvement steps can be tested in polynomial time. For~(\ref{improve:addpath}) and~(\ref{improve:shorten}) this follows from the fact that induced paths can be found using breadth-first search; for~(\ref{improve:addtwoflower}) this follows from Proposition~\ref{proposition:twoflower}. The number of improvement steps of type~(\ref{improve:addpath}) and~(\ref{improve:addtwoflower}) is bounded by the maximum order of the flower, which is at most~. Step~(\ref{improve:shorten}) replaces a path in~ by one whose endpoints are closer together in~. As there are only~ pairs of possible endpoints, a path cannot be replaced more than~ times by a better path. Since the flower contains at most~ paths at any time, there can be no more than~ successive applications of Step~(\ref{improve:shorten}) before the algorithm either terminates, or applies a step that increases the order of the flower. The algorithm therefore terminates after a polynomial number of steps. It is easy to verify that the stated conditions ensure that the structure~ is a -flower at all times. Hence we can compute a -flower~ that is maximal with respect to the three improvement rules in polynomial time.

\paragraph{Computing a hitting set.} Based on a maximal flower~ and the associated set of paths~ in~, we compute a set~ that hits all holes. To every vertex of~ we associate an edge of the decomposition tree~. For~, define the \emph{cutpoint above~}, denoted~, as follows. Let~ be the first edge~ on the path from~ to the root for which~, or NIL if no such edge exists. The hitting set~ is constructed as follows.

\begin{enumerate}[(a)]
	\item For each path~ with~, add the endpoints of~ to~.\label{hittingset:endpoints}
	\item For each vertex~ with~ NIL, add~ to~.\label{hittingset:cutpoint}
\end{enumerate}

It is easy to carry out this construction in polynomial time.

\begin{claim} \label{claim:s:subset:flower}
The set~ is a subset of~.
\end{claim}
\begin{claimproof}
The endpoints of the paths~ that are added to~ in the first step clearly belong to the flower. The vertices added to~ in the second step belong to~. By definition of cutpoint we have~, hence~. As~ is a clique tree of~, vertex~ does not occur in any set~ and is not added to~.
\end{claimproof}

\begin{claim} \label{claim:saturated:edge}
If~ is a maximal -flower, then for any distinct~ with~, the path~ contains~ or~.
\end{claim}
\begin{claimproof}
Let~ be nonadjacent in~. Assume for a contradiction that~ for all~ on~. Let~, so~ for all edges~ on~. By applying Proposition~\ref{proposition:path:from:adhesions} to the chordal graph~ we find an induced -path in~. Such a path forms a hole together with~, since~. So Step~(\ref{improve:addpath}) can be applied to increase the order of the -flower,  contradicting the maximality of~.

Hence there is an edge~ on~ for which~. Observe that the trees~ and~ are vertex-disjoint as each bag forms a clique in~ and~. We conclude with a case distinction; see Figure~\ref{fig:pathcontainscutpoint} for an illustration.

\begin{figure}[t]
\begin{center}
\subfigure[Case~\ref{case:containcutpointsbelowt}.]{\label{fig:pathcontainscutpoint1}
\image{PathContainsCutpoint1}
}
\subfigure[Case~\ref{case:containcutpointlca}.]{\label{fig:pathcontainscutpoint2}
\image{PathContainsCutpoint2}
}
\caption{Illustration of the proof of Claim~\ref{claim:saturated:edge}. The figure shows a tree decomposition~ of~. The minimal path~ in~ is drawn with thick edges. \ref{fig:pathcontainscutpoint1} All bags containing~ are contained in the subtree of~ rooted at the highest bag containing~. \ref{fig:pathcontainscutpoint2} If the first two cases do not hold, then the bags containing~ are contained in a different subtree of the lowest common ancestor~ than the bags containing~.} \label{fig:pathcontainscutpoint}
\end{center}
\end{figure}

\begin{enumerate}
	\item \label{case:containcutpointsbelowt} If~ is contained in~, then some vertex of~ is an ancestor to all vertices of~. Hence~ is the path from~ to its first ancestor in~, so~. As~ shows that at least one edge on~ satisfies the defining condition for a cutpoint,~ is a well-defined. As~ is not higher than~, which lies on~, the edge~ lies on~.
	\item If~ is contained in~, then reversing the roles of~ and~ in the previous argument shows that~ lies on~.
	\item \label{case:containcutpointlca} If neither case holds, then~ and~ are contained in different child subtrees of~, implying that~ is the concatenation of a path~ from~ to its ancestor~, and a path~ from~ to its ancestor~. Since~ lies on~, it lies on~ or~. Suppose that~ lies on~. Then~ is well-defined, lies above~ and no higher than~, and therefore lies on~. The case when~ lies on~ is symmetric.\qedhere
\end{enumerate}
\end{claimproof}

\begin{claim} \label{claim:s:hittingset}
The graph~ is chordal.
\end{claim}
\begin{claimproof}
Assume for a contradiction that~ contains a hole~. Since~ is chordal,~ passes through~ so that~ is an induced -path in~ for nonadjacent~. By Step~(\ref{hittingset:endpoints}) we know that~ and~ are not endpoints of paths in~, with respect to the computed maximal flower~. Since paths in~ cannot contain members of~ as interior vertices either, we find~.

Since~ and~ are nonadjacent vertices of~, by Claim~\ref{claim:saturated:edge} we know~ contains~ or~. Let~ be a well-defined edge on~. As~, by Step~(\ref{hittingset:cutpoint}) of the construction of~ we know that~. As only one bag of~ contains~ by minimality of~, and only one bag of~ contains~, it follows that~. On the other hand, since~ connects~ to~ in~ and~ lies on~, by Observation~\ref{observation:stpaths:use:adhesion} we know that the -path~ in~ contains a vertex~. This shows that~ contains a vertex from~ in its interior, and hence has a chord; a contradiction to the assumption that~ is a hole.
\end{claimproof}

The claims so far established that we can compute a maximal -flower~ and a corresponding chordal deletion set~ that avoids~, in polynomial time. To complete the proof, it remains to show that the size of~ is bounded by twelve times the order of the computed flower. This is the most technical part. It follows from the next claim.

\begin{claim} \label{claim:twelve:from:path}
If~ is a hole of the maximal flower~ and~, then~ contains at most~ vertices from~.
\end{claim}
\begin{claimproof}
Assume for a contradiction that~ is a hole in~ such that~ contains at least~ vertices from the path~. Let~ be the endpoints of~. We identify two special vertices in~ that allow us to derive a contradiction. Define~ and note that~ may coincide with~ or~. Since~ is a clique in~ and~ is an induced path, we know~. Similarly,~ contains at most two vertices of each of the sets~ and~.

Let~ be the set containing the first two and last two vertices on~, together with~, , and~. Since~ and~, there are at least three vertices in~. Since they lie on the induced path~, these three vertices do not form a clique implying that there are two distinct vertices in~ that are not adjacent in~. Call these~ and~ and choose their labels such that when traversing the path~ from~ to~, we visit~ before visiting~. We will use the pair~ to derive a contradiction to the maximality of~ with respect to the three presented improvement steps.

Since~ and~ are internal vertices of~ and all paths in~ are vertex-disjoint,~ and~ are not endpoints of any path in~ and therefore Step~(\ref{hittingset:endpoints}) of the hitting set construction is not responsible for adding them to~. Consequently,~ and~ are in~ because there are vertices~ with well-defined cutpoints~ such that~ and~ that caused~ and~ to be added to~ in Step~(\ref{hittingset:cutpoint}); we will argue later that~ and~ are distinct.

Since no internal vertex of~ lies in~ or~, there is a unique tree~ in the forest~ that contains the internal nodes on~. To derive a contradiction we will argue for the following properties; note that~(\ref{twelve:twoflower}) contradicts the maximality of~ with respect to Step~(\ref{improve:addtwoflower}).

\begin{enumerate}[(i)]
	\item  and~ are distinct edges, hence~.\label{twelve:xy:distinct:cutpoint}
	\item  and~ both belong to~.\label{twelve:xy:cutpoints:in:subtree}
	\item The edges~ and~ are distinct from~ and~.\label{twelve:primetost:distinct:cutpoint}
	\item Neither one of~ is simultaneously adjacent to both~ and~.\label{twelve:prime:not:both:st}
	\item Neither one of~ is adjacent to~ or~.\label{twelve:prime:not:one:st}
	\item There are paths from~ to~ and from~ to~ that are pairwise vertex-disjoint and whose interior avoids~.\label{twelve:twopaths}
	\item The graph~ contains a -flower of order two.\label{twelve:twoflower}
\end{enumerate}

(\ref{twelve:xy:distinct:cutpoint}) If~, then~ are adjacent in~ because they occur in a common bag of a clique tree. This contradicts our choice of~ and~ as nonadjacent vertices. Since~ we must have~.

(\ref{twelve:xy:cutpoints:in:subtree}) Since~ is an edge of~ whose adhesion contains~, it follows that~ is an edge of the subtree~. Similarly,~ is an edge of~. It therefore suffices to prove that the subtrees~ and~ are entirely contained in~. This follows from Proposition~\ref{proposition:induced:paths:in:clique:tree}, since both~ and~ are vertices on the induced~-path~ in the chordal graph~ that are not among the first two or last two vertices.

(\ref{twelve:primetost:distinct:cutpoint}) Suppose that~ equals~ or~. Then~ is a vertex on~ that is in~. But this contradicts our choice of~ to be a vertex of~ above, since~ contains~ and~. The argument when~ equals~ or~ is symmetric.

(\ref{twelve:prime:not:both:st}) Assume for a contradiction that~ is adjacent to both~ and~. We make a case distinction on the structure of~ and~ in the clique tree~.

\begin{enumerate}
	\item If~ is contained in~, then let~ be the lowest node of~ for which~. Then tree~ as defined above is contained in~. As we assumed~ is adjacent to both~ and~, vertices~ and~ occur in a common bag, showing that~ is a proper ancestor to all nodes of~. Since~ is an edge on~, the edge~ lies above~ while all of~ lies below~. This implies that~ does not belong to~, contradicting~(\ref{twelve:xy:cutpoints:in:subtree}).
	
	\item If~ is contained in~, then reversing the roles of~ and~ in the previous argument yields a contradiction.
	
	\item If neither case holds, then~ and~ are contained in different child subtrees of~. It follows that~ is the concatenation of a path from~ to~ with a path from~ to~. Since~ is adjacent to both~ and~, we know that~ intersects both~ and~, and therefore contains~. It follows that~ is an ancestor of~, and therefore that the edge~ connects two ancestors of~. By Proposition~\ref{proposition:inducedpath:on:minimalpath} we know that~, which is a vertex on the induced -path~, occurs in a bag on~. Since~ and~ is an edge connecting two ancestors of~, while all paths from ancestors of~ to nodes in~ go through~, we have~. This contradicts the choice of~, as we chose it from a set that excludes~.
\end{enumerate}

The argument when~ is adjacent to both~ and~ is symmetric.

(\ref{twelve:prime:not:one:st}) Assume for a contradiction that~ is adjacent to~; by~(\ref{twelve:prime:not:both:st}) we then already know that~ is not adjacent to~. We distinguish two cases, depending on the relative position of~ and~.

\begin{enumerate}
	\item If~, then let~ be the lowest node of~ for which~. It follows that~. Now observe that since~ is adjacent to~ in~, the trees~ and~ share at least one node, and hence~ is not a proper descendant of~. Since~ is an edge on~, it follows that~ does not lie in~. But this contradicts~(\ref{twelve:xy:cutpoints:in:subtree}).
	\item If the previous case does not hold, then~ does not enter~ ``from below'', which implies that~ contains the edge from~ to its parent. We now distinguish two cases.
	\begin{enumerate}
		\item If~ is a proper ancestor of~, then since~ occurs in a common bag with~ we must have that~ is contained in the bag of the parent of~. We claim that, in this situation, Step~(\ref{improve:shorten}) is applicable to improve the structure of the flower, contradicting the maximality of~ with respect to the given rules. To see this, observe that the presence of~ in the parent bag of~, together with the fact that~ uses the edge from~ to its parent, implies that~ is strictly smaller than~; the latter quantity is exactly the number of edges on~. By Observation~\ref{observation:stpaths:use:adhesion}, the -path~ contains a vertex~ of~, where~ is the edge from~ to its parent in~. Consequently,~ is not~ itself since~ does not appear in the parent bag of~. Since both~ and~ appear in that parent bag, we know that~ is adjacent to~ and therefore that~ together with the subpath of~ from~ to~ is a~-path in~. Hence, there is an induced -path in~ on a vertex subset of~. As~ and~ is not adjacent to~, which is one of the endpoints of~, this shows that Step~(\ref{improve:shorten}) can be applied to improve the structure of the flower~; a contradiction.
		\item If~ is not a proper ancestor of~, then since~ and~ occur in a common bag we know that~. Since the cutpoint~ above~ lies on~ by definition, and is contained in~ (by~(\ref{twelve:xy:cutpoints:in:subtree})) which lies ``above''~ by the precondition to this case, it follows that~ is an edge between two ancestors of~. As the adjacent vertices~ and~ occur in a common bag and~ is not higher than~, path~ goes through~ and all bags on the path from~ to~ contain~. Since no bag in~ contains~ and~ it follows that~ is an edge connecting two ancestors of~, and must therefore be the first edge~ on the path from~ to the root for which~. But then~, which contradicts~(\ref{twelve:primetost:distinct:cutpoint}).
	\end{enumerate}
\end{enumerate}

\widefigure{
\subfigure[Clique tree.]{\label{fig:twoflower1}
\bigimage{TwoFlower1}
}
\subfigure[Intersecting paths.]{\label{fig:twoflower2}
\image{TwoFlower2}
}
\subfigure[-flower.]{\label{fig:twoflower3}
\image{TwoFlower3}
}}{
\caption{\ref{fig:twoflower1} Illustration for part~(\ref{twelve:twopaths}) of the proof of Claim~\ref{claim:twelve:from:path}. Example of how the different entities can be arranged in the clique tree~ of~. \ref{fig:twoflower2} The structure of a hypothetical intersection in the interiors of the paths~ and~. Path~ is shown by thick edges. \ref{fig:twoflower3} Illustration for part~(\ref{twelve:twoflower}) of Claim~\ref{claim:twelve:from:path}. Drawing of how the constructed paths~ and~ are combined with the original petal~ of the flower, to obtain a flower of order two (visualized by thick edges).} \label{fig:twoflower}
}

(\ref{twelve:twopaths}) We first construct the paths and then prove that they have interiors that avoid~ and are pairwise vertex-disjoint. Let~ and let~ be the corresponding vertex in~. Let~ be the path in~ from~ to the lower endpoint of~. Since~ by definition of~, we know that~ appears in the bag of the bottom endpoint of~ and so~ connects~ to~ in~. Since none of the edges of~ satisfied the condition for being the cutpoint above~, for each edge~ on~ the set~ is not empty. Define~. Applying Proposition~\ref{proposition:path:from:adhesions} to the chordal graph~ where~ plays the role of~ and~ plays the role of~, we establish that there is an induced -path in the graph~. The interior vertices on this path avoid~ (since the latter are endpoints), and therefore the interior avoids~ as demanded by~(\ref{twelve:twopaths}). As this holds for both choices of~, we obtain an induced~-path~ and an induced~-path~. It remains to prove that~ and~ are pairwise vertex-disjoint. By~(\ref{twelve:xy:distinct:cutpoint}) we know that~. As we chose~ and~ to be distinct vertices, and~ while~, it is clear that the endpoints of~ and~ are four distinct vertices. It remains to prove that the interior of~ is vertex-disjoint from the interior of~. If one of the two paths has no interior, then this is trivial. So assume for a contradiction that both paths have an interior, which includes a vertex common to both paths. Orient both paths from the prime to the non-prime endpoint.

Choose~ and~ such that the bottom endpoint~ of~ satisfies~. This is always possible; see Figure~\ref{fig:twoflower} for an illustration. Let~ depending on whether~ or~; let~. Since the two interiors of~ and~ intersect, there is a path~ in~ from the successor~ of~ on~, to the predecessor of~ on~, such that all vertices on~ belong to the interior of~ or~. So~ avoids~. Vertex~ occurs in a bag of~, since it is adjacent to~ and~ is a descendant of both endpoints of~. To reach a contradiction we show that~ does not occur in a bag in~. To that end, we first show that~ does not occur in a bag of~. This follows from the fact that~ occurs in the adhesion of~, that~ is not contained in~, and that if~ also occurred in a bag of~ then this would force~ to be in~ which contains~; but this would imply that~ and~ are adjacent, contradicting our choice of~ and~ in the beginning of the proof. So~ does not occur in~, implying that~ occurs in some bag outside~. Consider a path~ in~ from a bag in~ containing~, to a bag outside~ containing~; we established that~ contains the edge from the root of~ to its parent, which is~. By Observation~\ref{observation:stpaths:use:adhesion}, the path~ connecting~ to~ contains a vertex of~. By definition of cutpoint, we have that~. But~ avoids~; a contradiction. It follows that~ and~ are indeed vertex-disjoint, proving~(\ref{twelve:twopaths}).

(\ref{twelve:twoflower}) Consider the -path~ and the -path~, whose existence we proved in~(\ref{twelve:twopaths}). The interiors of these paths avoid~, the endpoints~ and~ belong to~, and the endpoints~ and~ belong to~. Consider the walk in~ that starts at~, traverses the -path~ until~, and then traverses~ to~. Since~ is not adjacent to~ by~(\ref{twelve:prime:not:one:st}) and~, this walk contains an induced -path between two neighbors of~, whose internal nodes avoid~. Moreover, the only vertices of~ used on this path belong to~. Similarly, there is an induced -path within the walk starting at~, traversing~ from~ to~, and then traversing~ from~ to~. Since we chose~ and~ such that~ occurs before~, the pieces of~ used by these two induced paths are vertex-disjoint. By~(\ref{twelve:twopaths}), the pieces of~ and~ used are also vertex-disjoint. Hence these two induced paths are vertex-disjoint and each connect two nonadjacent neighbors of~. Together with~ these form a -flower of order two in~. Since the only vertices of~ used on this order- flower are those of~ and~, it follows that there is a -flower of order two in~, establishing~(\ref{twelve:twoflower}). This contradicts the fact that~ is maximal with respect to Step~(\ref{improve:addtwoflower}) and completes the proof of Claim~\ref{claim:twelve:from:path}.
\end{claimproof}

Armed with these claims we finish the proof of Lemma~\ref{lemma:algorithm:packing:vs:covering}. Claim~\ref{claim:s:hittingset} shows that~ is a hitting set avoiding~ for the holes in~. All vertices of~ belong to~ by Claim~\ref{claim:s:subset:flower}, and therefore all lie on some path of the structure~. Since the number of paths equals the order of the flower, while~ contains at most~ vertices from each path by Claim~\ref{claim:twelve:from:path}, it follows that the size of~ is at most~ times the order of the flower. This concludes the proof of Lemma~\ref{lemma:algorithm:packing:vs:covering}.
\end{proof}

