



It follows easily that FZF terminates, and so it suffices to show that
the algorithm outputs YES if and only if the given history  is 2-atomic.
We reach this goal through a sequence of technical lemmas.
The first lemma captures the rationale behind dividing the input into
maximal chunks in Stage~1:

\begin{lemma}
  \label{lemma:subdivide}
  For every history ,  is 2-atomic if and only if for every chunk
  ,  is 2-atomic.
\begin{proof} Suppose that  is 2-atomic, and let  be a valid 2-atomic total order
on the operations in .
For any , let  be the total order over operations
in  such that  extends .
Then  is a valid 2-atomic total order on the operations in .


Conversely, suppose that for every chunk ,  is 2-atomic.
We will show how to construct a valid 2-atomic total order on the operations in .
For each , let  denote the minimum  for any zone 
corresponding to a cluster in , and let  denote the maximum  
for any such .
Next, for each , let  denote a valid 2-atomic total order on the
operations in .
Similarly, for each dangling cluster  in , define a valid 2-atomic total order  over operations in .
(The existence of  follows from assumptions stated in Section~\ref{sec:model}.)
Also let  and  denote  and , respectively, where  is the zone corresponding
to cluster .

Now define the relation  over chunks and dangling clusters as follows:
given elements , each either a chunk in  or a dangling cluster of ,
 denotes that .
It follows easily that  is a partial order.
Furthermore, since all chunks in  are maximal, it follows that all
such chunks are totally ordered by .
Finally, choose any total order that extends , and let  denote the concatenation
in that order of all  and  for each chunk  and 
each dangling cluster  of .

It follows easily that  is a 2-atomic total order on all the operations in .
It remains to show that  is valid (i.e., extends the ``precedes'' relation over operations in ).
Suppose for contradiction that  is not valid.
Then there exist distinct operations  as well as two elements ,
each either a chunk in  or a dangling cluster of ,
such that  belongs in ,  belongs in ,
 precedes  in , and yet  precedes  in .
(In this context, ``belongs'' means that an operation is either part of a cluster in some maximal chunk,
or is part of some dangling cluster.)
\newline\underline{Case~1:}  and  are both chunks.
Since  precedes  in  and since  totally orders all chunks in ,
it follows that  holds, and hence .
Next, note that  starts no later than , otherwise the zone for some cluster in  would extend to after ,
and similarly  finishes no earlier than , otherwise the zone for some cluster in  would extend to before .
Since , this implies that  starts before  finishes.
But that contradicts  preceding .
\newline\underline{Case~2:}  and  are both dangling clusters.
Since  precedes  in ,  is false, and so .
Next, note that since  and  are both backward clusters,
 is a point inside  and  is a point inside .
Since , this implies that  finishes no earlier than  starts.
But that contradicts  preceding  in .
\newline\underline{Case~3:}  is a chunk and  is a dangling cluster, and  precedes  in .
Since  precedes  in ,  is false, and so .
Furthermore, since  is a dangling cluster,  is not part of chunk , and so .
Next, note that  starts no later than , otherwise the zone for some cluster in  would extend to after .
Also, since  is a backward cluster,  is a point inside .
Since , this implies that  finishes after  starts.
But that contradicts  preceding  in .
\newline\underline{Case~4:}  is a dangling cluster and  is a chunk, and  precedes  in .
The proof is analogous to Case~3.
We show that , hence  finishes after  starts, which contradicts  preceding  in .
\end{proof}
\end{lemma}

\medskip
Next, we show the correctness of Stage~2 of FZF.
To simplify presentation, we introduce a definition:
letting  denote a valid total order on a subset  of the writes in , we say that 
the \emph{separation of a write  in } is the minimum number of writes
that separate  from any of its dictated reads (not including  itself) in any valid total order  that
extends  to both the writes in  and their dictated reads.
It follows that if  is viable, then the separation of every  in 
is less than two.

\begin{lemma}
\label{lemma:fzf-stage2-a}
For any chunk , if an iteration of the outer for loop occurs in Stage~2 for , then
any viable total order  over the writes of forward clusters in  is an element
of , where  and  are computed at the beginning of the iteration.
\begin{proof}
Suppose that  does exist, and note that the pattern of forward zones in 
cannot have the following property, denoted  in the remainder of the proof:
three zones overlap at one point, or one zone overlaps more than two others.
(If  has property  then one can show that in any valid total order  over the writes in ,
 the dictating write for one of the forward zones has separation at least two in , and hence  is not viable.)
As a result, each forward zone overlaps at most two others, forming a ``chain''
of forward zones resembling one of the three shown in Figure~\ref{fig:fzf}.
Note that since  is a maximal chunk, this chain has no ``breaks'' in it,
and so all but two of the zones overlap exactly two others.
We proceed by induction on the number of forward clusters in , denoted  in this proof.
If  then the set  contains all the possible total orders under consideration,
and so the lemma holds.
Now suppose that the lemma holds for some , and consider a chunk with  forward clusters.
Label the forward zones of these clusters as  in increasing order of their low endpoints,
and let  denote the corresponding dictating writes.
Note that  and .

\noindent\underline{Case~1:}  ends before  ends.
Since  does not have property , the chain of zones in this case initially resembles the
middle chunk in Figure~\ref{fig:fzf}, with , , and .
Next, note that if cluster  were removed from chunk , then by the induction hypothesis
any viable total order on  would be either 
or .
For chunk , this implies that  must order the writes for  in the same manner as either  or .
Now consider the possible positions of  in .
For each case, we must either show that  is identical to  or , or else derive a contradiction
by showing that  is not viable.
\newline\noindent\underline{Subcase~1a:}  is the second or later element in .
Then  (if  extends ) or  (if  extends ).
In the first case, since  and  both precede some dictated read of  in ,
 has separation at least two in , and so  is not viable.
In the second case, since  and  both precede some dictated read of  in ,
 has separation at least two in , and so  is not viable.
\newline\noindent\underline{Subcase~1b:}  is the first element in .
Then  (if  extends ) or  (if  extends ).
In the first case,  is identical to .
In the second case, since  precedes some dictated read of  in ,
 has separation at least two in , and so  is not viable.

\noindent\underline{Case~2:}  ends after  ends.
Since  does not have property , the chain of zones in this case initially resembles the
rightmost chunk in Figure~\ref{fig:fzf}, with , , and .
We proceed as in Case~1, but invoke the induction hypothesis on 
 instead of .
We deduce that  must order the writes for  in the same manner as either
 or .
Next, we consider the possible positions of  in .
\newline\noindent\underline{Subcase~2a:}  is the second or later element in .
Then  (if  extends ) or  (if  extends ).
In the first case, since  and  both precede some dictated read of  in ,
 has separation at least two in , and so  is not viable.
In the second case, since  and  both precede some dictated read of  in ,
 has separation at least two in , and so  is not viable.
\newline\noindent\underline{Subcase~2b:}  is the first element in .
Then  (if  extends ) or  (if  extends ).
In the first case,  is identical to .
In the second case, since  precedes some dictated read of  in ,
 has separation at least two in , and so  is not viable.
\end{proof}
\end{lemma}


\begin{lemma}
\label{lemma:fzf-stage2-b}
For any chunk , if an iteration of the outer for loop occurs in Stage~2 for , then
any viable total order  over all the writes of  is an element
of the set  computed in this iteration.
\begin{proof}
Suppose that  exists.
It follows from Lemma~\ref{lemma:fzf-stage2-a} that  must order the writes in 
consistently with either  or , which are computed at the beginning of the iteration,
otherwise  is not viable.
Next, note that in both  and , each write except the last one has separation one,
otherwise either  is not viable (if separation is higher than one) or  is not a
maximal chunk (if separation is zero).
Since  extends either  and , this implies that
the dictating writes of any backward zones in  cannot be placed in  between
two dictating writes of forward zones, otherwise the separation of one of the latter writes
becomes greater than one, and hence  is no longer viable.
In other words, the dictating writes of backward zones must be placed in  either before
or after all the writes of forward zones.
We use this observation in the case analysis below.

Let  denote number of backward clusters in .

\noindent\underline{Case~1: }.
Then  must be either  or , and indeed the algorithm includes both
orders in .

\noindent\underline{Case~2: }.
Let  denote the dictating write of the backward cluster, as in the algorithm.
Then  must be either append or pre-pend  to either  or .
Indeed the algorithm includes all four possible orders in .

\noindent\underline{Case~3: }.
Let  denote the dictating writes of the backward clusters, as in the algorithm,
and let  denote the corresponding backward zones.
Then in ,  must either precede or follow all the writes of forward clusters,
and similarly for .

We show first that  and  in  cannot both precede all the writes of forward clusters.
Let  be the dictating write of the forward cluster in  whose forward zone has the earliest low endpoint,
and let  denote this zone.
Suppose for contradiction that  places  and  before , in that order (without loss of generality).
Since  is a maximal chunk,  holds, which means that some operation in 
starts after some operation in  ends.
As a result, any valid 2-atomic total order  over  that extends  places some operation of  after .
( exists because we assume that  is viable and exists.)
Since we assume that  places  before ,  must place some dictated read  of  after .
In that case  is separated from  by both  and .
Thus,  has separation at least two in ,
which contradicts  being viable.

Next, we show that  and  in  cannot both succeed all the writes of forward clusters.
Let  be the dictating write of the forward cluster in  whose forward zone has the largest high endpoint,
and let  denote this forward zone.
Suppose for contradiction that  places  and  after , in that order (without loss of generality).
Since  is a maximal cluster,  holds, which means that some operation in 
ends before some dictated read  in  begins.
(Recall that every forward cluster has at least one dictated read.)
As a result, any valid 2-atomic total order  over  that extends  places some operation of  before .  
(Again,  exists because we assume that  is viable and exists.)
Since  is valid and 2-atomic, this implies that  places  before .
By an analogous argument,  places  before .
Since  and  both follow  in , this implies that  has separation at least two in ,
which contradicts  being viable.

Thus,  can only be one of four possible total orders:
, , , and .
Indeed the algorithm includes all four of these in .

\noindent\underline{Case~4: }.
Let  denote the dictating writes of the backward clusters.
Then in , each of these writes must either precede or follow all the writes of forward clusters.
This contradicts our observation from Case~3 that at most one of  can precede,
and at most one can follow, all the writes of forward clusters.
\end{proof}
\end{lemma}

\begin{lemma}
\label{lemma:fzf-stage2-c}
For any chunk , if an iteration of the outer for loop occurs in Stage~2 for ,
and this iteration outputs NO, then  is not 2-atomic.
Conversely, if the iteration for chunk  occurs and does not output NO, then  is 2-atomic.
\begin{proof}
Suppose that an iteration of the outer for loop occurs in Stage~2 for , computes the set , and outputs NO.
Since the algorithm outputs NO, none of the total orders in  are viable, and hence by 
Lemma~\ref{lemma:fzf-stage2-b} there is no viable total order  over the writes of all clusters in .
This implies that  is not 2-atomic.

Conversely, suppose that the iteration for chunk  occurs and does not output NO.
Then some total order  is viable.
Furthermore, it follows from the algorithm for Stage~2 that  is a total order
over the writes of all clusters in .
Since  is viable, this implies that  is 2-atomic.
\end{proof}
\end{lemma}

Finally, the following theorem asserts the overall correctness of FZF:

\begin{theorem}
\label{theorem:fzf-stage2}
For any input history , if  is 2-atomic then algorithm FZF outputs YES, otherwise it outputs NO.
\begin{proof}
Suppose first that FZF outputs YES.
Then the outer for loop iterates over all the chunks in Stage~2 without outputting NO, and so
by Lemma~\ref{lemma:fzf-stage2-c},  is 2-atomic for each maximal chunk .
Then by Lemma~\ref{lemma:subdivide},  itself is 2-atomic, as wanted.
Otherwise, suppose that FZF outputs NO.
Then this occurs in Stage~2, and so by Lemma~\ref{lemma:fzf-stage2-c} there is some chunk
 such that  is not 2-atomic.
This implies that  is not 2-atomic either, as wanted.
\end{proof}
\end{theorem}


\remove{
We first show that if FZF outputs YES, then  is 2-atomic.  We
establish this claim via a sequence of lemmas.

\begin{lemma}
  \label{lemma:spt-in-range}
  Every schedule point assigned by FZF to a zone is in range, i.e.,
  within the interval of the write in that zone.
\end{lemma}

\proof In stage 2, FZF assigns schedule points to forward zones.  When
it assigns , let  be the
write in .  Since  and
 is less than the minimum absolute distance between any two
operation endpoints, we conclude that , namely,
 is in range.  When it assigns ,
 is clearly in range.  Furthermore, in both cases, the
schedule point of a forward zone is  distance after
any operation endpoint.

In stage 3, we first observe that it is always possible to find a
schedule point for .  This is because, by the definition of
, the length of  that is not covered by any chain is at
least .  Even after avoiding being within distance
 to a chain on either side, there is still 
length to choose from.  And by the definition of the backward zone,
any point in a backward zone is an in-range choice for a schedule
point for that backward zone. 

In stage 4, when it assigns the schedule point of a backward zone to
the the high endpoint of the same zone, by the same reasoning as for
stage 3, we conclude that the schedule point is in range.  When it
assigns the schedule point to be , we first note
that, from the analysis above for stage 2,  is 
distance after any operation endpoint.  When  covers two backward
zones and FZF assigns to , let  be the write in .
We note that .  Therefore, .
Namely,  is in range.  A similar analysis can be done for
the case when  covers only one backward zone and FZF assigns the
schedule point to that backward zone.  \QED

\begin{lemma}
  \label{lemma:spt-distinct}
  All the schedule points assigned by FZF are distinct from each other.
\end{lemma}

\proof In stage 2, FZF assigns schedule points to either the low
endpoints of forward zones, or  less than the low
endpoints of forward zone.  By the assumption of distinct operation
endpoints, and by the definition of , all schedule points
are distinct.  In stage 3, the FZF algorithm states that the schedule
points assigned are distinct from assigned ones.  In stage 4, when FZF
assigns the schedule point of a backward zone to be , as this schedule point is outside of any chain, this
schedule point is distinct from any forward zone's schedule point.
Furthermore, since this new schedule point equals ,
it is within distance  of a
chain.  Therefore, it is different from any schedule points assigned
in stage 3.  In stage 4, when FZF assigns the schedule point of a
backward zone to be the high endpoint of that backward zone, the
schedule point is contained in the chain and is the same as the low
endpoint of an operation in that backward zone, and therefore, is
different from all other schedule points.  \QED

To facilitate the proof below, we define the \emph{extension} of a
forward zone  to be the closed interval from  to
.  The extension is undefined when the schedule point is not
assigned yet.  Note that the extension of a forward zone can either be
larger than the zone (when the schedule point is assigned to be less
than the low endpoint) or the same (when the schedule point is the
same as the low endpoint).

\begin{lemma}
  \label{lemma:bz-spt-uncovered}
  Any schedule point  assigned in stage 3 is not covered by the
  extension of any forward zone.
\end{lemma}

\proof In stage 2, FZF only may assign the schedule point of a forward
zone  to be .  Since in stage 3, FZF assigns
 to distance  from any chain, the claim
follows.  \QED

Recall that once the schedule point of a zone  is assigned, the
commit points induced by this schedule point is (1) for the write 
in : , (2) for every read  in : .  Therefore, we can view that a set of assigned
schedule points on a set of zones induce a total order on the
operations belonging to those zones.

\begin{figure}[tbp]
  \center{\scalebox{0.75}{\includegraphics{chain}}}
  \caption{A chain of forward zones.}
  \label{fig:chain}
\end{figure}

\begin{lemma}
  \label{lemma:fzf-yes-implies-h-2-atomic}
  If FZF outputs YES, then  is 2-atomic.
\end{lemma}

\proof It suffices to show that the total order induced by the
schedule points in the above manner form a 2-atomic total order.  We
have proven in Lemma~\ref{lemma:spt-in-range} that the schedule points
assigned by FZF are always in range.  We now show that, after each
stage, if we only consider the subhistory of  consisting of the
zones with assigned schedule points, then the subhistory is always
2-atomic.  Since all zones are assigned schedule points at the end of
stage 4,  is 2-atomic if FZF reaches the end of stage 4, at which
point FZF outputs YES.

Stage 1 does not assign any schedule points.  Stage 2 assigns schedule
points for forward zones.  Denote the forward zones in a chain to be
 to  in increasing order of their low endpoints.
Because of the overlapping patterns ruled out in stage 1, only 
can potentially contain the low endpoints of two other forward zones
(see Figure~\ref{fig:chain}).  If  only contains the low endpoint
of one other forward zone, then , for all , and .  If  contains the low endpoints
of two other forward zones, then  and
.  Therefore, FZF attempts a ``back stretching'' of
 and if successful, we have .

By Lemma~\ref{lemma:bz-spt-uncovered}, stage 3 assigns (to backward zones)
schedule points that are not covered by the extension of any forward zones.
Consequently, these schedule points do not increase the -value
of any forward zones, and the -value of the corresponding backward zones is only one.

Stage 4 assigns schedule points for backward zones that might cause
the -value of a forward zone to increase.  Consider the case
when a chain consisting of multiple forward zones covers only one
backward zone .  In this case, FZF tries two possible ways to
schedule .  The first is to schedule  before all other
forward zones' schedule points, but only if .  If
successful, this scheduling does not increase the -value of
any forward zones.  We require  because otherwise, if
, then the reads in  have to be scheduled after
, causing .  If the previous scheduling does not
work, then FZF checks if .  This is because the last
forward zone (in increasing order of low endpoints) in the chain has a
-value of only one, and only if  can be scheduled after
 can we ``squeeze''  in.  The analysis for the case where a
chain consisting of one forward zone covers only one backward zone is
largely similar and hence is omitted.

Now consider the case where a chain consisting of multiple forward
zones covers two backward zones.  By a similar analysis to the
previous paragraph, the two backward zones can only be scheduled one
near the low end of the chain and the other near the high end.
Furthermore, we have to make sure that the one scheduled near the low
end is not entirely to the high side of the one scheduled near the
high end, otherwise the one scheduled on the low end would have a
-value of more than two.

Therefore, if FZF reaches the end of stage 4 and outputs YES.  The
given history is 2-atomic.  \QED

The above is only half of the proof.  In what follows, we carry out
the other half of the proof, namely, if FZF outputs NO, then  is
not 2-atomic.  To this end, we consider each case where FZF outputs NO
and show that if such a case arises, then indeed  is not 2-atomic.

\begin{figure}[tbp]
  \center{\scalebox{0.75}{\includegraphics{overlap}}}
  \caption{Overlapping patterns: (a) three overlapping forward zones,
    (b) one forward zone containing the low endpoints of three other
    forward zones, (c) one forward zone containing the low endpoints
    of two other forward zones and its own low endpoint is contained
    in another forward zone.}
  \label{fig:overlap}
\end{figure}

\begin{lemma}
  \label{lemma:fzf-overlap-1}
  If the given history  has three forward zones that overlap at a
  common point in time, then  is not 2-atomic.
\end{lemma}

\proof Since we assume that all operation endpoints are unique, if
three forward zones overlap at a point, they overlap at an interval
(see Figure~\ref{fig:overlap}(a)).  Let  be a point in the
overlapping interval but not on the boundary.  Consider an arbitrary
in-range assignment of commit points for the operations in these three
forward zones.  Without loss of generality, suppose the commit points
for the writes are such that .  Note that, by
the definition of forward zones,  for all ,
and there exists a read  in the same zone as  such that .  Therefore, .
In other words, in any total order, 's -distance to 
is at least 3, implying that the history is not 2-atomic.  \QED

\begin{lemma}
  \label{lemma:fzf-overlap-2}
  If there exists a forward zone that contains the low endpoints of
  three other forward zones, then the given history is not 2-atomic.
\end{lemma}

\proof By Lemma~\ref{lemma:fzf-overlap-1}, we can safely assume that
no three forward zones overlap at the same point in time.  Let the
containing forward zone be , and the other three forward zones
be .  Let  be the write in  for all  and
 be the read in  with the maximum start time.  See
Figure~\ref{fig:overlap}(b).  By the definition of forward zones, we
have , for all .  Consider an arbitrary
in-range assignment of commit points to .  By the
definition of forward zones, if , then  is
distance 4 from .  Therefore, at least two of  are
.  But then between these two, the write with the earlier
commit point is at least distance 3 from its dictated read.
Therefore, the given history is not 2-atomic.  \QED

\begin{lemma}
  \label{lemma:fzf-overlap-3}
  If there exists a forward zone that contains the low endpoints of
  two other forward zones and its own low endpoint is contained in
  another forward zone, then the history is not 2-atomic.
\end{lemma}

\proof Again, by Lemma~\ref{lemma:fzf-overlap-1}, we assume that no
three forward zones overlap at the same point.  Let the four forward
zones be , numbered by increasing order of their low
endpoints.  See Figure~\ref{fig:overlap}(c).  By similar reasoning
as the above two lemmas, if , then  is
distance 3 from some its own dictated reads.  Therefore, one of
 is less than , say .  Then among the three
writes , the one with the minimum commit point is at least
distance 3 from some of its own dictated reads.  Therefore, the given
history is not 2-atomic.  \QED

The above three lemmas combined imply the following lemma.

\begin{lemma}
  \label{lemma:fzf-no-1}
  If FZF outputs NO in stage 1, then  is not 2-atomic.
\end{lemma}

\begin{lemma}
  \label{lemma:fzf-no-2}
  If FZF outputs NO in stage 2, then  is not 2-atomic.
\end{lemma}

\proof This is when FZF finds that  contains the low endpoints
of two other forward zones and tries to ``back-stretch'' .
Because of stage 1, the only interleaving pattern for these three
forward zones is as  in Figure~\ref{fig:fzf}(b).
Consider only these three forward zones.  If ,
then .  If we schedule  such that
, then .  But if we schedule
 such that , then .
Therefore,  is not 2-atomic.  \QED

We ignore stage 3 in this part of our analysis because the algorithm
never outputs NO in stage 3.

\begin{lemma}
  \label{lemma:fzf-no-3}
  If FZF outputs NO in stage 4, then  is not 2-atomic.
\end{lemma}

\proof FZF outputs NO in stage 4 under the following circumstances:
(1) chain  entirely covers at least three backward zones, (2) 
entirely covers two backward zones but the conditions stated in the
FZF algorithm are not satisfied, and (3)  entirely covers one
backward zone but the conditions stated in the algorithm are not
satisfied.  We consider these cases one by one.

Suppose  entirely covers three backward zones, in which case FZF
outputs NO.  Let the chain of forward zones be  to , in
increasing order of schedule low endpoints.  Consider an arbitrary
assignment of the schedule points for  to  in  and
for these three backward zones.  Then as explained in
Lemma~\ref{lemma:fzf-yes-implies-h-2-atomic}, the schedule points of
these three backward zones have to be either before  or after
.  Since at most one backward zone can be scheduled after 
(so as to keep ), at least two backward zones
have to be scheduled before , and the situation is similar to
three overlapping forward zones (see Lemma~\ref{lemma:fzf-overlap-1}),
implying that it is not possible to schedule these zones so that they
are 2-atomic.

Suppose  entirely covers two backward zones  and  and
suppose  consists of only one forward zone .  We prove by
contradiction, i.e., assume that FZF outputs YES, and we will show
that the conditions stated in the algorithm have to hold, leading to
the contradiction that FZF outputs NO.  Since FZF outputs YES, then
there exists an assignment of commit points to the operations in these
three zones such that their -values are all at most two.  Let
 be the write in ,  be the write for , and 
be the write for .  If , then , and if , then one of  will have
its -value at least 3.  Therefore,  is in the middle of
, and without loss of generality, assume .  Then we have .  And since  and , we have .  Similarly, since  and , we have
.  Lastly, since  and , we have .  Therefore, the
condition stated in the algorithm actually holds, implying that FZF
would have found these zones schedulable.  A contradiction.

The analysis for other cases is largely similar and is therefore
omitted.  \QED

Combining the above lemmas, we have

\begin{lemma}
  \label{lemma:???}
  If FZF outputs NO, then  is not 2-atomic.
\end{lemma}

We now reach the main theorem of this section.

\begin{theorem}
  \label{thm:???}
  FZF outputs YES iff  is 2-atomic.
\end{theorem}
}

