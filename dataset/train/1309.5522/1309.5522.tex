



It follows easily that FZF terminates, and so it suffices to show that
the algorithm outputs YES if and only if the given history $H$ is 2-atomic.
We reach this goal through a sequence of technical lemmas.
The first lemma captures the rationale behind dividing the input into
maximal chunks in Stage~1:

\begin{lemma}
  \label{lemma:subdivide}
  For every history $H$, $H$ is 2-atomic if and only if for every chunk
  $K \in \chunks(H)$, $H|K$ is 2-atomic.
\begin{proof} Suppose that $H$ is 2-atomic, and let $T$ be a valid 2-atomic total order
on the operations in $H$.
For any $K \in \chunks(H)$, let $T|K$ be the total order over operations
in $H|K$ such that $T$ extends $T|K$.
Then $T|K$ is a valid 2-atomic total order on the operations in $H|K$.


Conversely, suppose that for every chunk $K \in \chunks(H)$, $H|K$ is 2-atomic.
We will show how to construct a valid 2-atomic total order on the operations in $H$.
For each $K \in \chunks(H)$, let $K.l$ denote the minimum $Z.l$ for any zone $Z$
corresponding to a cluster in $K$, and let $K.h$ denote the maximum $Z.h$ 
for any such $Z$.
Next, for each $K \in \chunks(H)$, let $T_K$ denote a valid 2-atomic total order on the
operations in $H|K$.
Similarly, for each dangling cluster $D_j$ in $H$, define a valid 2-atomic total order $T_{D_j}$ over operations in $D_j$.
(The existence of $T_{D_j}$ follows from assumptions stated in Section~\ref{sec:model}.)
Also let $D.l$ and $D.h$ denote $Z.l$ and $Z.h$, respectively, where $Z$ is the zone corresponding
to cluster $D$.

Now define the relation $\leq_H$ over chunks and dangling clusters as follows:
given elements $X,Y$, each either a chunk in $\chunks(H)$ or a dangling cluster of $H$,
$X \leq_H Y$ denotes that $X.h < Y.l$.
It follows easily that $\leq_H$ is a partial order.
Furthermore, since all chunks in $\chunks(H)$ are maximal, it follows that all
such chunks are totally ordered by $\leq_H$.
Finally, choose any total order that extends $\leq_H$, and let $T$ denote the concatenation
in that order of all $T_{K_i}$ and $T_{D_j}$ for each chunk $K_i \in \chunks(H)$ and 
each dangling cluster $D_j$ of $H$.

It follows easily that $T$ is a 2-atomic total order on all the operations in $H$.
It remains to show that $T$ is valid (i.e., extends the ``precedes'' relation over operations in $H$).
Suppose for contradiction that $T$ is not valid.
Then there exist distinct operations $Op,Op'$ as well as two elements $X,Y$,
each either a chunk in $\chunks(H)$ or a dangling cluster of $H$,
such that $Op$ belongs in $X$, $Op'$ belongs in $Y$,
$T_X$ precedes $T_Y$ in $T$, and yet $Op'$ precedes $Op$ in $H$.
(In this context, ``belongs'' means that an operation is either part of a cluster in some maximal chunk,
or is part of some dangling cluster.)
\newline\underline{Case~1:} $X$ and $Y$ are both chunks.
Since $T_X$ precedes $T_Y$ in $T$ and since $T$ totally orders all chunks in $\chunks(H)$,
it follows that $X \leq_H Y$ holds, and hence $X.h < Y.l$.
Next, note that $Op$ starts no later than $X.h$, otherwise the zone for some cluster in $X$ would extend to after $X.h$,
and similarly $Op'$ finishes no earlier than $Y.l$, otherwise the zone for some cluster in $Y$ would extend to before $Y.l$.
Since $X.h < Y.l$, this implies that $Op$ starts before $Op'$ finishes.
But that contradicts $Op'$ preceding $Op$.
\newline\underline{Case~2:} $X$ and $Y$ are both dangling clusters.
Since $T_X$ precedes $T_Y$ in $T$, $Y \leq_H X$ is false, and so $Y.h \geq X.l$.
Next, note that since $X$ and $Y$ are both backward clusters,
$X.l$ is a point inside $Op$ and $Y.h$ is a point inside $Op'$.
Since $Y.h \geq X.l$, this implies that $Op'$ finishes no earlier than $Op$ starts.
But that contradicts $Op'$ preceding $Op$ in $H$.
\newline\underline{Case~3:} $X$ is a chunk and $Y$ is a dangling cluster, and $T_X$ precedes $T_Y$ in $T$.
Since $T_X$ precedes $T_Y$ in $T$, $Y \leq_H X$ is false, and so $Y.h \geq X.l$.
Furthermore, since $Y$ is a dangling cluster, $Y$ is not part of chunk $X$, and so $Y.h > X.h$.
Next, note that $Op$ starts no later than $X.h$, otherwise the zone for some cluster in $X$ would extend to after $X.h$.
Also, since $Y$ is a backward cluster, $Y.h$ is a point inside $Op'$.
Since $Y.h > X.h$, this implies that $Op'$ finishes after $Op$ starts.
But that contradicts $Op'$ preceding $Op$ in $H$.
\newline\underline{Case~4:} $X$ is a dangling cluster and $Y$ is a chunk, and $T_X$ precedes $T_Y$ in $T$.
The proof is analogous to Case~3.
We show that $Y.l > X.l$, hence $Op'$ finishes after $Op$ starts, which contradicts $Op'$ preceding $Op$ in $H$.
\end{proof}
\end{lemma}

\medskip
Next, we show the correctness of Stage~2 of FZF.
To simplify presentation, we introduce a definition:
letting $T$ denote a valid total order on a subset $S$ of the writes in $H$, we say that 
the \emph{separation of a write $w \in S$ in $T$} is the minimum number of writes
that separate $w$ from any of its dictated reads (not including $w$ itself) in any valid total order $T'$ that
extends $T$ to both the writes in $S$ and their dictated reads.
It follows that if $T$ is viable, then the separation of every $w \in S$ in $T$
is less than two.

\begin{lemma}
\label{lemma:fzf-stage2-a}
For any chunk $K \in \chunks(H)$, if an iteration of the outer for loop occurs in Stage~2 for $K$, then
any viable total order $T$ over the writes of forward clusters in $K$ is an element
of $\{T_F, T_F'\}$, where $T_F$ and $T_F'$ are computed at the beginning of the iteration.
\begin{proof}
Suppose that $T$ does exist, and note that the pattern of forward zones in $K$
cannot have the following property, denoted $P$ in the remainder of the proof:
three zones overlap at one point, or one zone overlaps more than two others.
(If $K$ has property $P$ then one can show that in any valid total order $T'$ over the writes in $K$,
 the dictating write for one of the forward zones has separation at least two in $T'$, and hence $T'$ is not viable.)
As a result, each forward zone overlaps at most two others, forming a ``chain''
of forward zones resembling one of the three shown in Figure~\ref{fig:fzf}.
Note that since $K$ is a maximal chunk, this chain has no ``breaks'' in it,
and so all but two of the zones overlap exactly two others.
We proceed by induction on the number of forward clusters in $K$, denoted $f$ in this proof.
If $f \leq 2$ then the set $\{T_F, T_F'\}$ contains all the possible total orders under consideration,
and so the lemma holds.
Now suppose that the lemma holds for some $f \geq 2$, and consider a chunk with $f+1$ forward clusters.
Label the forward zones of these clusters as $A, B, C, \ldots$ in increasing order of their low endpoints,
and let $w_A, w_B, w_C, \ldots$ denote the corresponding dictating writes.
Note that $T_F = w_A, w_B, w_C, \ldots$ and $T_F' = w_B, w_A, w_C, \ldots$.

\noindent\underline{Case~1:} $A$ ends before $B$ ends.
Since $K$ does not have property $P$, the chain of zones in this case initially resembles the
middle chunk in Figure~\ref{fig:fzf}, with $A = \FZ_2$, $B = \FZ_3$, and $C = \FZ_4$.
Next, note that if cluster $A$ were removed from chunk $K$, then by the induction hypothesis
any viable total order on $w_B,w_C,\ldots$ would be either $T_F'' = w_B,w_C,\ldots$
or $T_F''' = w_C,w_B,\ldots$.
For chunk $K$, this implies that $T$ must order the writes for $B, C, \ldots$ in the same manner as either $T_F''$ or $T_F'''$.
Now consider the possible positions of $w_A$ in $T$.
For each case, we must either show that $T$ is identical to $T_F$ or $T_F'$, or else derive a contradiction
by showing that $T$ is not viable.
\newline\noindent\underline{Subcase~1a:} $w_A$ is the second or later element in $T$.
Then $T = w_B, \ldots, w_A, \ldots$ (if $T$ extends $T_F''$) or $T = w_C, \ldots, w_A, \ldots$ (if $T$ extends $T_F'''$).
In the first case, since $w_A$ and $w_C$ both precede some dictated read of $w_B$ in $H$,
$w_B$ has separation at least two in $T$, and so $T$ is not viable.
In the second case, since $w_A$ and $w_B$ both precede some dictated read of $w_C$ in $H$,
$w_C$ has separation at least two in $T$, and so $T$ is not viable.
\newline\noindent\underline{Subcase~1b:} $w_A$ is the first element in $T$.
Then $T = w_A, w_B, w_C, \ldots$ (if $T$ extends $T_F''$) or $T = w_A, w_C, w_B, \ldots$ (if $T$ extends $T_F'''$).
In the first case, $T$ is identical to $T_F$.
In the second case, since $w_B$ precedes some dictated read of $w_A$ in $H$,
$w_A$ has separation at least two in $T$, and so $T$ is not viable.

\noindent\underline{Case~2:} $A$ ends after $B$ ends.
Since $K$ does not have property $P$, the chain of zones in this case initially resembles the
rightmost chunk in Figure~\ref{fig:fzf}, with $A = \FZ_5$, $B = \FZ_6$, and $C = \FZ_7$.
We proceed as in Case~1, but invoke the induction hypothesis on 
$A,C, \ldots$ instead of $B,C,\ldots$.
We deduce that $T$ must order the writes for $A,C, \ldots$ in the same manner as either
$T_F'' = w_A,w_C,\ldots$ or $T_F''' = w_C,w_A,\ldots$.
Next, we consider the possible positions of $w_B$ in $T$.
\newline\noindent\underline{Subcase~2a:} $w_B$ is the second or later element in $T$.
Then $T = w_A, \ldots, w_B, \ldots$ (if $T$ extends $T_F''$) or $T = w_C, \ldots, w_B, \ldots$ (if $T$ extends $T_F'''$).
In the first case, since $w_B$ and $w_C$ both precede some dictated read of $w_A$ in $H$,
$w_A$ has separation at least two in $T$, and so $T$ is not viable.
In the second case, since $w_A$ and $w_B$ both precede some dictated read of $w_C$ in $H$,
$w_C$ has separation at least two in $T$, and so $T$ is not viable.
\newline\noindent\underline{Subcase~2b:} $w_B$ is the first element in $T$.
Then $T = w_B, w_A, w_C, \ldots$ (if $T$ extends $T_F''$) or $T = w_B, w_C, w_A, \ldots$ (if $T$ extends $T_F'''$).
In the first case, $T$ is identical to $T_F'$.
In the second case, since $w_A$ precedes some dictated read of $w_B$ in $H$,
$w_B$ has separation at least two in $T$, and so $T$ is not viable.
\end{proof}
\end{lemma}


\begin{lemma}
\label{lemma:fzf-stage2-b}
For any chunk $K \in \chunks(H)$, if an iteration of the outer for loop occurs in Stage~2 for $K$, then
any viable total order $T$ over all the writes of $K$ is an element
of the set $S$ computed in this iteration.
\begin{proof}
Suppose that $T$ exists.
It follows from Lemma~\ref{lemma:fzf-stage2-a} that $T$ must order the writes in $K$
consistently with either $T_F$ or $T_F'$, which are computed at the beginning of the iteration,
otherwise $T$ is not viable.
Next, note that in both $T_F$ and $T_F'$, each write except the last one has separation one,
otherwise either $T$ is not viable (if separation is higher than one) or $K$ is not a
maximal chunk (if separation is zero).
Since $T$ extends either $T_F$ and $T_F'$, this implies that
the dictating writes of any backward zones in $K$ cannot be placed in $T$ between
two dictating writes of forward zones, otherwise the separation of one of the latter writes
becomes greater than one, and hence $T$ is no longer viable.
In other words, the dictating writes of backward zones must be placed in $T$ either before
or after all the writes of forward zones.
We use this observation in the case analysis below.

Let $B$ denote number of backward clusters in $K$.

\noindent\underline{Case~1: $B = 0$}.
Then $T$ must be either $T_F$ or $T_F'$, and indeed the algorithm includes both
orders in $S$.

\noindent\underline{Case~2: $B = 1$}.
Let $w$ denote the dictating write of the backward cluster, as in the algorithm.
Then $T$ must be either append or pre-pend $w$ to either $T_F$ or $T_F'$.
Indeed the algorithm includes all four possible orders in $S$.

\noindent\underline{Case~3: $B = 2$}.
Let $w_1,w_2$ denote the dictating writes of the backward clusters, as in the algorithm,
and let $B_1,B_2$ denote the corresponding backward zones.
Then in $T$, $w_1$ must either precede or follow all the writes of forward clusters,
and similarly for $w_2$.

We show first that $w_1$ and $w_2$ in $T$ cannot both precede all the writes of forward clusters.
Let $w_f$ be the dictating write of the forward cluster in $K$ whose forward zone has the earliest low endpoint,
and let $F$ denote this zone.
Suppose for contradiction that $T$ places $w_1$ and $w_2$ before $w_f$, in that order (without loss of generality).
Since $K$ is a maximal chunk, $B_1.l > F.l$ holds, which means that some operation in $B_1$
starts after some operation in $F$ ends.
As a result, any valid 2-atomic total order $T'$ over $H|K$ that extends $T$ places some operation of $B_1$ after $w_f$.
($T'$ exists because we assume that $T$ is viable and exists.)
Since we assume that $T$ places $w_1$ before $w_f$, $T'$ must place some dictated read $r_1$ of $w_1$ after $w_f$.
In that case $w_1$ is separated from $r_1$ by both $w_2$ and $w_f$.
Thus, $w_1$ has separation at least two in $T$,
which contradicts $T$ being viable.

Next, we show that $w_1$ and $w_2$ in $T$ cannot both succeed all the writes of forward clusters.
Let $w_f$ be the dictating write of the forward cluster in $K$ whose forward zone has the largest high endpoint,
and let $F$ denote this forward zone.
Suppose for contradiction that $T$ places $w_1$ and $w_2$ after $w_f$, in that order (without loss of generality).
Since $k$ is a maximal cluster, $B_1.h < F.h$ holds, which means that some operation in $B_1$
ends before some dictated read $r_f$ in $F$ begins.
(Recall that every forward cluster has at least one dictated read.)
As a result, any valid 2-atomic total order $T'$ over $H|K$ that extends $T$ places some operation of $B_1$ before $r_f$.  
(Again, $T'$ exists because we assume that $T$ is viable and exists.)
Since $T'$ is valid and 2-atomic, this implies that $T'$ places $w_1$ before $r_f$.
By an analogous argument, $T'$ places $w_2$ before $r_f$.
Since $w_1$ and $w_2$ both follow $w_f$ in $T$, this implies that $w_f$ has separation at least two in $T$,
which contradicts $T$ being viable.

Thus, $T$ can only be one of four possible total orders:
$w_1 T_F w_2$, $w_2 T_F w_1$, $w_1 T_F' w_2$, and $w_2 T_F' w_1$.
Indeed the algorithm includes all four of these in $S$.

\noindent\underline{Case~4: $B \geq 3$}.
Let $w_1,w_2,w_3, \ldots$ denote the dictating writes of the backward clusters.
Then in $T$, each of these writes must either precede or follow all the writes of forward clusters.
This contradicts our observation from Case~3 that at most one of $w_1,w_2,w_3, \ldots$ can precede,
and at most one can follow, all the writes of forward clusters.
\end{proof}
\end{lemma}

\begin{lemma}
\label{lemma:fzf-stage2-c}
For any chunk $K \in \chunks(H)$, if an iteration of the outer for loop occurs in Stage~2 for $K$,
and this iteration outputs NO, then $H|K$ is not 2-atomic.
Conversely, if the iteration for chunk $K$ occurs and does not output NO, then $H|K$ is 2-atomic.
\begin{proof}
Suppose that an iteration of the outer for loop occurs in Stage~2 for $K$, computes the set $S$, and outputs NO.
Since the algorithm outputs NO, none of the total orders in $S$ are viable, and hence by 
Lemma~\ref{lemma:fzf-stage2-b} there is no viable total order $T$ over the writes of all clusters in $K$.
This implies that $H|K$ is not 2-atomic.

Conversely, suppose that the iteration for chunk $K$ occurs and does not output NO.
Then some total order $T \in S$ is viable.
Furthermore, it follows from the algorithm for Stage~2 that $T$ is a total order
over the writes of all clusters in $K$.
Since $T$ is viable, this implies that $H|K$ is 2-atomic.
\end{proof}
\end{lemma}

Finally, the following theorem asserts the overall correctness of FZF:

\begin{theorem}
\label{theorem:fzf-stage2}
For any input history $H$, if $H$ is 2-atomic then algorithm FZF outputs YES, otherwise it outputs NO.
\begin{proof}
Suppose first that FZF outputs YES.
Then the outer for loop iterates over all the chunks in Stage~2 without outputting NO, and so
by Lemma~\ref{lemma:fzf-stage2-c}, $H|K$ is 2-atomic for each maximal chunk $K \in \chunks(H)$.
Then by Lemma~\ref{lemma:subdivide}, $H$ itself is 2-atomic, as wanted.
Otherwise, suppose that FZF outputs NO.
Then this occurs in Stage~2, and so by Lemma~\ref{lemma:fzf-stage2-c} there is some chunk
$K \in \chunks(H)$ such that $H|K$ is not 2-atomic.
This implies that $H$ is not 2-atomic either, as wanted.
\end{proof}
\end{theorem}


\remove{
We first show that if FZF outputs YES, then $H$ is 2-atomic.  We
establish this claim via a sequence of lemmas.

\begin{lemma}
  \label{lemma:spt-in-range}
  Every schedule point assigned by FZF to a zone is in range, i.e.,
  within the interval of the write in that zone.
\end{lemma}

\proof In stage 2, FZF assigns schedule points to forward zones.  When
it assigns $\FZ_2.\spt := \FZ_1.l - \epsilon/6$, let $w_2$ be the
write in $\FZ_2$.  Since $\FZ_2.\bp < \FZ_1.l < \FZ_2.l$ and
$\epsilon$ is less than the minimum absolute distance between any two
operation endpoints, we conclude that $w_2.s = \FZ_2.\bp < \FZ_1.l -
\epsilon/6 = \FZ_2.\spt < \FZ_2.l = \FZ_2.\zstart \leq w_2.f$, namely,
$\FZ_2.\spt$ is in range.  When it assigns $\FZ.\spt := \FZ.l$,
$\FZ.\spt$ is clearly in range.  Furthermore, in both cases, the
schedule point of a forward zone is $\geq 5\epsilon/6$ distance after
any operation endpoint.

In stage 3, we first observe that it is always possible to find a
schedule point for $\BZ$.  This is because, by the definition of
$\epsilon$, the length of $\BZ$ that is not covered by any chain is at
least $\epsilon$.  Even after avoiding being within distance
$\epsilon/3$ to a chain on either side, there is still $\epsilon/3$
length to choose from.  And by the definition of the backward zone,
any point in a backward zone is an in-range choice for a schedule
point for that backward zone. 

In stage 4, when it assigns the schedule point of a backward zone to
the the high endpoint of the same zone, by the same reasoning as for
stage 3, we conclude that the schedule point is in range.  When it
assigns the schedule point to be $t_1 - \epsilon/6$, we first note
that, from the analysis above for stage 2, $t_1$ is $\geq 5\epsilon/6$
distance after any operation endpoint.  When $C$ covers two backward
zones and FZF assigns to $\BZ_1$, let $w_1$ be the write in $\BZ_1$.
We note that $t_1 - \BZ_1.\bp \geq 5\epsilon/6$.  Therefore, $w_1.s =
\BZ_1.\bp < t_1 - 5\epsilon/6 < t_1 - \epsilon/6 = \BZ.\spt \leq \min
\{ \FZ.l: \FZ \subseteq C \} < \BZ.l = \BZ.\zstart \leq w_1.f$.
Namely, $\BZ_1.\bp$ is in range.  A similar analysis can be done for
the case when $C$ covers only one backward zone and FZF assigns the
schedule point to that backward zone.  \QED

\begin{lemma}
  \label{lemma:spt-distinct}
  All the schedule points assigned by FZF are distinct from each other.
\end{lemma}

\proof In stage 2, FZF assigns schedule points to either the low
endpoints of forward zones, or $\epsilon/6$ less than the low
endpoints of forward zone.  By the assumption of distinct operation
endpoints, and by the definition of $\epsilon$, all schedule points
are distinct.  In stage 3, the FZF algorithm states that the schedule
points assigned are distinct from assigned ones.  In stage 4, when FZF
assigns the schedule point of a backward zone to be $t_1 -
\epsilon/6$, as this schedule point is outside of any chain, this
schedule point is distinct from any forward zone's schedule point.
Furthermore, since this new schedule point equals $t_1 - \epsilon/6$,
it is within distance $\epsilon/6 + \epsilon/6 = \epsilon/3$ of a
chain.  Therefore, it is different from any schedule points assigned
in stage 3.  In stage 4, when FZF assigns the schedule point of a
backward zone to be the high endpoint of that backward zone, the
schedule point is contained in the chain and is the same as the low
endpoint of an operation in that backward zone, and therefore, is
different from all other schedule points.  \QED

To facilitate the proof below, we define the \emph{extension} of a
forward zone $\FZ$ to be the closed interval from $\FZ.\spt$ to
$\FZ.h$.  The extension is undefined when the schedule point is not
assigned yet.  Note that the extension of a forward zone can either be
larger than the zone (when the schedule point is assigned to be less
than the low endpoint) or the same (when the schedule point is the
same as the low endpoint).

\begin{lemma}
  \label{lemma:bz-spt-uncovered}
  Any schedule point $p$ assigned in stage 3 is not covered by the
  extension of any forward zone.
\end{lemma}

\proof In stage 2, FZF only may assign the schedule point of a forward
zone $\FZ$ to be $\FZ.l-\epsilon/6$.  Since in stage 3, FZF assigns
$\BZ.\spt$ to distance $> \epsilon/3$ from any chain, the claim
follows.  \QED

Recall that once the schedule point of a zone $Z$ is assigned, the
commit points induced by this schedule point is (1) for the write $w$
in $Z$: $w.c = Z.\spt$, (2) for every read $r$ in $Z$: $r.c =
\max(r.s, Z.\spt)$.  Therefore, we can view that a set of assigned
schedule points on a set of zones induce a total order on the
operations belonging to those zones.

\begin{figure}[tbp]
  \center{\scalebox{0.75}{\includegraphics{chain}}}
  \caption{A chain of forward zones.}
  \label{fig:chain}
\end{figure}

\begin{lemma}
  \label{lemma:fzf-yes-implies-h-2-atomic}
  If FZF outputs YES, then $H$ is 2-atomic.
\end{lemma}

\proof It suffices to show that the total order induced by the
schedule points in the above manner form a 2-atomic total order.  We
have proven in Lemma~\ref{lemma:spt-in-range} that the schedule points
assigned by FZF are always in range.  We now show that, after each
stage, if we only consider the subhistory of $H$ consisting of the
zones with assigned schedule points, then the subhistory is always
2-atomic.  Since all zones are assigned schedule points at the end of
stage 4, $H$ is 2-atomic if FZF reaches the end of stage 4, at which
point FZF outputs YES.

Stage 1 does not assign any schedule points.  Stage 2 assigns schedule
points for forward zones.  Denote the forward zones in a chain to be
$\FZ_1$ to $\FZ_k$ in increasing order of their low endpoints.
Because of the overlapping patterns ruled out in stage 1, only $\FZ_1$
can potentially contain the low endpoints of two other forward zones
(see Figure~\ref{fig:chain}).  If $\FZ_1$ only contains the low endpoint
of one other forward zone, then $\delta(\FZ_i) = 2$, for all $1 \leq i
< k$, and $\delta(\FZ_k) = 1$.  If $\FZ_1$ contains the low endpoints
of two other forward zones, then $\delta(\FZ_1)=3$ and
$\delta(\FZ_2)=1$.  Therefore, FZF attempts a ``back stretching'' of
$\FZ_2$ and if successful, we have $\delta(\FZ_1) = \delta(\FZ_2) =
2$.

By Lemma~\ref{lemma:bz-spt-uncovered}, stage 3 assigns (to backward zones)
schedule points that are not covered by the extension of any forward zones.
Consequently, these schedule points do not increase the $\delta$-value
of any forward zones, and the $\delta$-value of the corresponding backward zones is only one.

Stage 4 assigns schedule points for backward zones that might cause
the $\delta$-value of a forward zone to increase.  Consider the case
when a chain consisting of multiple forward zones covers only one
backward zone $\BZ$.  In this case, FZF tries two possible ways to
schedule $\BZ$.  The first is to schedule $\BZ$ before all other
forward zones' schedule points, but only if $\BZ.l < t_3$.  If
successful, this scheduling does not increase the $\delta$-value of
any forward zones.  We require $\BZ.l < t_3$ because otherwise, if
$\BZ.l > t_3$, then the reads in $\BZ$ have to be scheduled after
$t_3$, causing $\delta(\BZ) > 2$.  If the previous scheduling does not
work, then FZF checks if $\BZ.h > t_2$.  This is because the last
forward zone (in increasing order of low endpoints) in the chain has a
$\delta$-value of only one, and only if $\BZ$ can be scheduled after
$t_2$ can we ``squeeze'' $\BZ$ in.  The analysis for the case where a
chain consisting of one forward zone covers only one backward zone is
largely similar and hence is omitted.

Now consider the case where a chain consisting of multiple forward
zones covers two backward zones.  By a similar analysis to the
previous paragraph, the two backward zones can only be scheduled one
near the low end of the chain and the other near the high end.
Furthermore, we have to make sure that the one scheduled near the low
end is not entirely to the high side of the one scheduled near the
high end, otherwise the one scheduled on the low end would have a
$\delta$-value of more than two.

Therefore, if FZF reaches the end of stage 4 and outputs YES.  The
given history is 2-atomic.  \QED

The above is only half of the proof.  In what follows, we carry out
the other half of the proof, namely, if FZF outputs NO, then $H$ is
not 2-atomic.  To this end, we consider each case where FZF outputs NO
and show that if such a case arises, then indeed $H$ is not 2-atomic.

\begin{figure}[tbp]
  \center{\scalebox{0.75}{\includegraphics{overlap}}}
  \caption{Overlapping patterns: (a) three overlapping forward zones,
    (b) one forward zone containing the low endpoints of three other
    forward zones, (c) one forward zone containing the low endpoints
    of two other forward zones and its own low endpoint is contained
    in another forward zone.}
  \label{fig:overlap}
\end{figure}

\begin{lemma}
  \label{lemma:fzf-overlap-1}
  If the given history $H$ has three forward zones that overlap at a
  common point in time, then $H$ is not 2-atomic.
\end{lemma}

\proof Since we assume that all operation endpoints are unique, if
three forward zones overlap at a point, they overlap at an interval
(see Figure~\ref{fig:overlap}(a)).  Let $t$ be a point in the
overlapping interval but not on the boundary.  Consider an arbitrary
in-range assignment of commit points for the operations in these three
forward zones.  Without loss of generality, suppose the commit points
for the writes are such that $w_1.c < w_2.c < w_3.c$.  Note that, by
the definition of forward zones, $w_i.c \leq w_i.f < t$ for all $i$,
and there exists a read $r_1$ in the same zone as $w_1$ such that $t <
r_1.s \leq r_1.c$.  Therefore, $w_1.c < w_2.c < w_3.c < t < r_1.c$.
In other words, in any total order, $w_1$'s $\delta$-distance to $r_1$
is at least 3, implying that the history is not 2-atomic.  \QED

\begin{lemma}
  \label{lemma:fzf-overlap-2}
  If there exists a forward zone that contains the low endpoints of
  three other forward zones, then the given history is not 2-atomic.
\end{lemma}

\proof By Lemma~\ref{lemma:fzf-overlap-1}, we can safely assume that
no three forward zones overlap at the same point in time.  Let the
containing forward zone be $\FZ_1$, and the other three forward zones
be $\FZ_{2,3,4}$.  Let $w_i$ be the write in $\FZ_i$ for all $i$ and
$r_1$ be the read in $\FZ_1$ with the maximum start time.  See
Figure~\ref{fig:overlap}(b).  By the definition of forward zones, we
have $w_i.c < \FZ_1.h \leq r_1.c$, for all $i$.  Consider an arbitrary
in-range assignment of commit points to $w_{1,2,3,4}$.  By the
definition of forward zones, if $w_1.c < w_{2,3,4}.c$, then $w_1$ is
distance 4 from $r_1$.  Therefore, at least two of $w_{2,3,4}.c$ are
$< w_1.c$.  But then between these two, the write with the earlier
commit point is at least distance 3 from its dictated read.
Therefore, the given history is not 2-atomic.  \QED

\begin{lemma}
  \label{lemma:fzf-overlap-3}
  If there exists a forward zone that contains the low endpoints of
  two other forward zones and its own low endpoint is contained in
  another forward zone, then the history is not 2-atomic.
\end{lemma}

\proof Again, by Lemma~\ref{lemma:fzf-overlap-1}, we assume that no
three forward zones overlap at the same point.  Let the four forward
zones be $\FZ_{1,2,3,4}$, numbered by increasing order of their low
endpoints.  See Figure~\ref{fig:overlap}(c).  By similar reasoning
as the above two lemmas, if $w_2.c < w_{3,4}.c$, then $w_2$ is
distance 3 from some its own dictated reads.  Therefore, one of
$w_{3,4}.c$ is less than $w_1.c$, say $w_3$.  Then among the three
writes $w_{1,2,3}$, the one with the minimum commit point is at least
distance 3 from some of its own dictated reads.  Therefore, the given
history is not 2-atomic.  \QED

The above three lemmas combined imply the following lemma.

\begin{lemma}
  \label{lemma:fzf-no-1}
  If FZF outputs NO in stage 1, then $H$ is not 2-atomic.
\end{lemma}

\begin{lemma}
  \label{lemma:fzf-no-2}
  If FZF outputs NO in stage 2, then $H$ is not 2-atomic.
\end{lemma}

\proof This is when FZF finds that $\FZ_1$ contains the low endpoints
of two other forward zones and tries to ``back-stretch'' $\FZ_2$.
Because of stage 1, the only interleaving pattern for these three
forward zones is as $\FZ_{5,6,7}$ in Figure~\ref{fig:fzf}(b).
Consider only these three forward zones.  If $\FZ_2.\bp > \FZ_1.l$,
then $\FZ_1.\spt < \FZ_2.\spt$.  If we schedule $\FZ_3.\spt$ such that
$\FZ_3.\spt < \FZ_1.\spt$, then $\delta(\FZ_3)=3$.  But if we schedule
$\FZ_3$ such that $\FZ_1.\spt < \FZ_3.\spt$, then $\delta(\FZ_1) = 3$.
Therefore, $H$ is not 2-atomic.  \QED

We ignore stage 3 in this part of our analysis because the algorithm
never outputs NO in stage 3.

\begin{lemma}
  \label{lemma:fzf-no-3}
  If FZF outputs NO in stage 4, then $H$ is not 2-atomic.
\end{lemma}

\proof FZF outputs NO in stage 4 under the following circumstances:
(1) chain $C$ entirely covers at least three backward zones, (2) $C$
entirely covers two backward zones but the conditions stated in the
FZF algorithm are not satisfied, and (3) $C$ entirely covers one
backward zone but the conditions stated in the algorithm are not
satisfied.  We consider these cases one by one.

Suppose $C$ entirely covers three backward zones, in which case FZF
outputs NO.  Let the chain of forward zones be $\FZ_1$ to $\FZ_k$, in
increasing order of schedule low endpoints.  Consider an arbitrary
assignment of the schedule points for $\FZ_1$ to $\FZ_k$ in $C$ and
for these three backward zones.  Then as explained in
Lemma~\ref{lemma:fzf-yes-implies-h-2-atomic}, the schedule points of
these three backward zones have to be either before $t_1$ or after
$t_2$.  Since at most one backward zone can be scheduled after $t_2$
(so as to keep $\delta(\FZ_k) \leq 2$), at least two backward zones
have to be scheduled before $t_1$, and the situation is similar to
three overlapping forward zones (see Lemma~\ref{lemma:fzf-overlap-1}),
implying that it is not possible to schedule these zones so that they
are 2-atomic.

Suppose $C$ entirely covers two backward zones $\BZ_1$ and $\BZ_2$ and
suppose $C$ consists of only one forward zone $\FZ$.  We prove by
contradiction, i.e., assume that FZF outputs YES, and we will show
that the conditions stated in the algorithm have to hold, leading to
the contradiction that FZF outputs NO.  Since FZF outputs YES, then
there exists an assignment of commit points to the operations in these
three zones such that their $\delta$-values are all at most two.  Let
$w$ be the write in $\FZ$, $w_1$ be the write for $\BZ_1$, and $w_2$
be the write for $\BZ_2$.  If $w.c < w_{1,2}.c$, then $\delta(\FZ)
\geq 3$, and if $w_{1,2}.c < w.c$, then one of $\BZ_{1,2}$ will have
its $\delta$-value at least 3.  Therefore, $w.c$ is in the middle of
$w_{1,2}.c$, and without loss of generality, assume $w_1.c < w_2.c <
w_3.c$.  Then we have $\BZ_1.\bp \leq w_1.c < w_2.c \leq \FZ.l = C.l =
t_1$.  And since $\BZ_1 \subseteq C$ and $t_3 = C.h$, we have $\BZ_1.l
< t_3$.  Similarly, since $t_2 = t_1$ and $\BZ_2 \subseteq C$, we have
$t_2 < \BZ_2.h$.  Lastly, since $w_1.c < w_2.c$ and $\delta(\BZ_1)
\leq 2$, we have $\BZ.l < w_2.c \leq \BZ_2.h$.  Therefore, the
condition stated in the algorithm actually holds, implying that FZF
would have found these zones schedulable.  A contradiction.

The analysis for other cases is largely similar and is therefore
omitted.  \QED

Combining the above lemmas, we have

\begin{lemma}
  \label{lemma:???}
  If FZF outputs NO, then $H$ is not 2-atomic.
\end{lemma}

We now reach the main theorem of this section.

\begin{theorem}
  \label{thm:???}
  FZF outputs YES iff $H$ is 2-atomic.
\end{theorem}
}

