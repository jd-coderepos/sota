\newcommand{\sincx}{s_{0 \to n}^{x}}
\newcommand{\tincx}{t_{0 \to n}^{x}}
\newcommand{\sincy}{s_{0 \to n}^{y}}
\newcommand{\tincy}{t_{0 \to n}^{y}}
\newcommand{\sdeclt}{s_{n \to 0}^{<}}
\newcommand{\tdeclt}{t_{n \to 0}^{<}}
\newcommand{\sdecgt}{s_{n \to 0}^{>}}
\newcommand{\tdecgt}{t_{n \to 0}^{>}}

We reduce an input \subiso{} instance $(G,H)$
to a node-weighted variant of \dirmc{}, where every non-terminal vertex has a weight being a positive integer, and the goal is to find a cutset of total weight
not exceeding the budget $p$.
We fix some integer constant $M$; in fact it suffices to set $M=2$,
but it helps to think of it as a sufficiently large integer constant.
 We use three levels of weight: there will be \emph{light vertices}, of weight $1$, \emph{heavy vertices}, of various integer weights being multiples of $M$,
usually depending on the degree of the corresponding vertex of $H$, and \emph{undeletable vertices}, of weight $p+1$. Observe that it is easy to reduce the weighted variant to the original one by replacing every vertex of weight $w$ with $w$ unit-weight vertices. Thus, it suffices to show the reduction to the weighted variant, with only four terminal pairs.

\paragraph{Construction.}
Let us now describe the construction of the (weighted)
\dirmc{} instance $(G',\terms,p)$.
We start by setting budget $p = (6M+1)k$.
We also introduce eight terminals, arranged in four terminal pairs:
\begin{equation*}
\begin{split}
(\sincx,\tincx),\quad (\sincy,\tincy), \\ \quad (\sdeclt,\tdeclt), \quad (\sdecgt,\tdecgt).
\end{split}
\end{equation*}
 
For every $i \in V(H)$, we introduce a bidirected path on $2n+1$ vertices
$$z^i_0, \hat{z}^i_1, z^i_1, \hat{z}^i_2, z^i_2, \ldots, \hat{z}^i_n, z^i_n,$$
called henceforth the \emph{$z$-path for vertex $i$}, and denoted by $Z^i$.
Similarly, for every ordered pair $(i,j)$ where $ij \in E(H)$, we introduce two bidirected paths on $2n+1$ vertices
\begin{equation*}
\begin{split}
x^{i,j}_0, \hat{x}^{i,j}_1, x^{i,j}_1, \hat{x}^{i,j}_2, x^{i,j}_2, \ldots, \hat{x}^{i,j}_n, x^{i,j}_n;\\
y^{i,j}_0, \hat{y}^{i,j}_1, y^{i,j}_1, \hat{y}^{i,j}_2, y^{i,j}_2, \ldots, \hat{y}^{i,j}_n, y^{i,j}_n.
\end{split}
\end{equation*}
We call these paths \emph{the $x$-path and the $y$-path for the pair $(i,j)$}, and denote them by $X^{i,j}$ and $Y^{i,j}$.
All vertices $z^i_a$, $x^{i,j}_a$, $y^{i,j}_a$ are undeletable, while all vertices $\hat{z}^i_a$, $\hat{x}^{i,j}_a$, $\hat{y}^{i,j}_a$ are heavy:
vertex $\hat{z}^i_a$ has weight $M \cdot \deg_H(i)$ and the vertices
$\hat{x}^{i,j}_a$ and $\hat{y}^{i,j}_a$ have weight $M$ each.

Note that so far we have created $4k + \ell$ paths, each having $2n+1$ vertices.
Furthermore, if we are to delete one heavy vertex from each of these paths,
the total cost would be
$$M \cdot \left(4k + \sum_{i \in V(H)} \deg_H(i)\right) = 6kM.$$

For every pair $(i,j)$ with $ij \in E(H)$,
and every $0 \leq a \leq n$, we add arcs $(x^{i,j}_a,z^i_a)$ and $(z^i_a,y^{i,j}_a)$. Furthermore, we attach terminals to the paths as follows.
\begin{itemize}
\item for every pair $(i,j)$ with $ij \in E(H)$,
we add arcs $(\sincx,x^{i,j}_0)$ and $(y^{i,j}_n,\tincy)$;
\item for every $i \in V(H)$ we add arcs $(\sincy,z^i_0)$ and $(z^i_n,\tincx)$; \item for every pair $(i,j)$ with $ij \in E(H)$ and $i<j$ we add arcs $(\sdeclt, x^{i,j}_n)$ and $(y^{i,j}_0,\tdeclt)$;
\item for every pair $(i,j)$ with $ij \in E(H)$ and $i > j$
we add arcs $(\sdecgt, x^{i,j}_n)$ and $(y^{i,j}_0,\tdecgt)$.
\end{itemize}

\begin{figure*}[tb]
\begin{center}
\includegraphics{fig-dirmc}
\caption{Illustration of the reduction for \dirmc{}. Black vertices are light, gray are heavy, and white are undeletable.
The top figure illustates a $p$-grid, together with an intended solution marked by red circles.
Here, the vertex $p_{1,1}$ lies in the top-left corner of the grid, the first coordinate describes the row of the grid,
and the second one the column.
The bottom figure illustrates an $x$-, $z$-, and $y$-path for a pair $(i,j)$ with $ij \in E(H)$ and $i < j$.}
\label{fig:dirmc}
\end{center}
\end{figure*}

We refer to Figure~\ref{fig:dirmc} for an illustration.
Intuitively, the so-far constructed bidirectional paths and terminals require to delete at least one heavy vertex per 
bidirectional path; the connections between paths ensure that for every $i \in V(H)$, we need to chose
one index $1 \leq \phi(i) \leq n$ and delete vertices $\hat{z}^i_{\phi(i)}$, $\hat{x}^{i,j}_{\phi(i)}$, and $\hat{y}^{i,j}_{\phi(i)}$, that is, cut all paths corresponding to the vertex $i$ at the same place. The choice of the index $\phi(i)$ corresponds
to the choice of the image of $i$ in the sought homomorphism.

Let us now introduce gadgets that check the edge relations between the chosen vertices.
For every pair $(i,j)$ with $ij \in E(H)$, $i < j$ we introduce an acyclic $n \times n$ grid with vertices $p^{i,j}_{a,b}$ for $1 \leq a,b \leq n$
and arcs $(p^{i,j}_{a,b}, p^{i,j}_{a+1,b})$ for every $1 \leq a < n$ and $1 \leq b \leq n$, as well as 
$(p^{i,j}_{a,b}, p^{i,j}_{a,b+1})$ for every $1 \leq a \leq n$ and $1 \leq b < n$.
We call this grid \emph{$p$-grid for the pair $(i,j)$}, and denote it by $P^{i,j}$.
We set the vertex $p^{i,j}_{a,b}$ to be a light vertex if $v^i_a v^j_b \in E(G)$, and undeletable otherwise.
Finally, for every $1 \leq a \leq n$ we introduce arcs:
$$(x^{i,j}_a, p^{i,j}_{a,1}),\quad (p^{i,j}_{a,n}, y^{i,j}_{a-1}),\quad (x^{j,i}_a,p^{i,j}_{1,a}), \quad (p^{i,j}_{n,a}, y^{j,i}_{a-1}).$$
Intuitively, after deleting the aforementioned heavy vertices on the $x$-, $y$-, and $z$-paths, for fixed $ij \in E(H)$, $i < j$,
there is only one remaining path from $x^{i,j}_n$ (an out-neighbor of $\sdeclt$) to $y^{i,j}_0$ (an in-neighbor of $\tdeclt$),
passing through the $\phi(i)$-th row of the $p$-grid for the pair $(i,j)$,
and there is only one remaining path from $x^{j,i}_n$ (an out-neighbor of $\sdecgt$) to $y^{j,i}_0$ (an in-neighbor of $\tdecgt$),
passing through the $\phi(j)$-th column of the $p$-grid for the pair $(i,j)$.
We can kill both these paths with a single vertex $p^{i,j}_{\phi(i),\phi(j)}$, but only
the existence of the edge $v^i_{\phi(i)} v^j_{\phi(j)}$ ensures that this vertex is light, not undeletable.

This concludes the construction of the instance $(G',\terms,p)$.
We now formally show that the constructed instance is equivalent to the input
\subiso{} instance $(G,H)$.

\paragraph{From a homomorphism to a cutset.}
Let $\phi:V(H) \to [n]$ be such that $i \mapsto v^i_{\phi(i)} \in V_i$
is a homomorphism of $H$ into $G$.
Define
\begin{equation*}
\begin{split}
X &= \{\hat{x}^{i,j}_{\phi(i)}, \hat{y}^{i,j}_{\phi(i)} : (i,j) \in V(H) \times V(H) \textrm{\ s.t.\ }ij \in E(H)\}
\cup \\ &\cup \{\hat{z}^i_{\phi(i)} : i \in V(H) \}
\cup \{p^{i,j}_{\phi(i),\phi(j)} : (i,j) \in V(H) \times V(H) \textrm{\ s.t.\ }ij \in E(H), i < j\}.
\end{split}
\end{equation*}
The total weight of the vertices in $X$ equals:
$$2k \cdot 2 \cdot M + \sum_{i \in V(H)} M \cdot \deg_H(i) + k = (6M+1)k = p.$$
Note that the fact that 
$p^{i,j}_{\phi(i),\phi(j)}$ is light for every $ij \in E(H)$, $i < j$
follows from the assumption that the vertices $i \mapsto v^i_{\phi(i)}$ 
is a homomorphism.
Consequently, $X$ is of weight exactly $p$. We now show that it is a multicut in $(G',\terms)$.

We start with a simple observation about the structure of the graph $G'$. While
the $x$-, $y$-, and $z$-paths are bidirected, they --- together with the $p$-grids --- are arranged in a DAG-like fashion.
That is, there are directed arcs from $X^{i,j}$ to $Z^i$, from $Z^i$ to $Y^{i,j}$, from $X^{i,j}$ and $X^{j,i}$ to $P^{i,j}$, and from $P^{i,j}$ to $Y^{i,j}$ and $Y^{j,i}$,
but all cycles in $G'$ are contained in one $x$-, $y$-, or $z$-path. 
 
Consider first the terminal pair $(\sincx, \tincx)$. The out-neighbors of $\sincx$ are the endpoints $x^{i,j}_0$ for every pair $(i,j)$ with $ij \in E(H)$;
the only in-neighbors of $\tincx$ are the endpoints $z^i_n$ for every $i \in V(H)$.
Thus, by the previous observation, the only paths from $\sincx$ to $\tincx$ in the graph $G'$ start by going to some vertex $x^{i,j}_0$, traverse $X^{i,j}$ up to some vertex $x^{i,j}_a$, use the arc $(x^{i,j}_a,z^i_a)$
to fall to $Z^i$, and then traverse $Z^i$ to the endpoint $z^i_n$. 
(In particular, there are no paths from $\sincx$ to $\tincx$ that contain a vertex of some grid $P^{i,j}$.)
However, all such paths for $a \geq \phi(i)$ are cut
by the vertex $\hat{x}^{i,j}_{\phi(i)} \in X$, while all such paths for $a < \phi(i)$ are cut by the vertex $\hat{z}^i_{\phi(i)} \in X$. Consequently,
the terminal pair $(\sincx, \tincx)$ is separated in $H \setminus X$.

A similar argument holds for the pair $(\sincy,\tincy)$. By the same reasoning, the only paths between $\sincy$ and $\tincy$ in the graph $H$
are paths that start by going to some vertex $z^i_0$, traverse $Z^i$ up to some vertex $z^i_a$, use the arc
$(z^i_a,y^{i,j}_a)$ for some $ij \in E(H)$ to fall to $Y^{i,j}$, and then continue along this $y$-path to the vertex $y^{i,j}_n$.
However, all such paths for $a \geq \phi(i)$ are cut by the vertex $\hat{z}^{i}_{\phi(i)} \in X$, while all such paths for $a < \phi(i)$ are cut by the vertex $\hat{y}^{i,j}_{\phi(i)} \in X$.

Let us now focus on the terminal pair $(\sdeclt,\tdeclt)$. Observe that there are two types of paths from $\sdeclt$ to $\tdeclt$ in the graph $H$.
The first type consists of paths that starts by going to some vertex $x^{i,j}_n$ where $i < j$, traverse $X^{i,j}$ up to some vertex $x^{i,j}_a$,
use the arc $(x^{i,j}_a,z^i_a)$ to fall to $Z^i$, then traverse $Z^i$ up to some vertex $z^i_b$, use
the arc $(z^i_b,y^{i,j'}_b)$ for some $j' > i$ to fall to $Y^{i,j'}$, and finally traverse this $y$-path to the endpoint $y^{i,j'}_0$.
However, similarly as in the previous cases, the vertices $\hat{x}^{i,j}_{\phi(i)}, \hat{z}^i_{\phi(i)}, \hat{y}^{i,j'}_{\phi(i)} \in X$ cut all such paths.

The second type of paths use the $p$-grids in the following manner: the path starts by going to some vertex $x^{i,j}_n$ where $i < j$,
traverse $X^{i,j}$ up to some vertex $x^{i,j}_a$, use the arc $(x^{i,j}_a,p^{i,j}_{a,1})$ to fall to $P^{i,j}$,
traverse this $p$-grid up to a vertex $p^{i,j}_{b,n}$ where $b \geq a$, use the arc $(p^{i,j}_{b,n}, y^{i,j}_{b-1})$ to fall to $Y^{i,j}$,
and traverse this path to the endpoint $y^{i,j}_0$. These paths are cut by $X$ as follows: the paths where $a < \phi(i)$ are cut by $\hat{x}^{i,j}_{\phi(i)} \in X$,
the paths where $b > \phi(i)$ are cut by $\hat{y}^{i,j}_{\phi(i)}$, while the paths where $a=b=\phi(i)$ are cut by the vertex $p^{i,j}_{\phi(i),\phi(j)} \in X$;
note that the $\phi(i)$-th row of the grid is the only path from $p^{i,j}_{\phi(i),1}$ to $p^{i,j}_{\phi(i),n}$.
Please observe that the terminal $\tdeclt$ cannot be reached from $P^{i,j}$ by going to the other $y$-path reachable from this $p$-grid,
namely $Y^{j,i}$, as $Y^{j,i}$ has only outgoing arcs to the terminal $\tdecgt$ since $j > i$.

A similar argument holds for the pair $(\sdecgt,\tdecgt)$. The paths going through $X^{i,j}$, $i > j$,
$Z^i$, and $Y^{i,j'}$, $i > j'$, are cut by vertices
$\hat{x}^{i,j}_{\phi(i)}, \hat{z}^i_{\phi(i)}, \hat{y}^{i,j'}_{\phi(i)} \in X$.
The paths going through $X^{i,j}$, $i > j$, $P^{j,i}$, and $Y^{i,j}$,
are cut by the vertices $\hat{x}^{i,j}_{\phi(i)}, \hat{y}^{i,j}_{\phi(i)}, p^{j,i}_{\phi(j),\phi(i)}$. Again, it is essential that the other $y$-path
reachable from the $p$-grid for the pair $(j,i)$, namely $Y^{j,i}$, does not have outgoing arcs to the terminal $\tdecgt$, but only
to the terminal $\tdeclt$.

We infer that $X$ is a solution to the \dirmc{} instance $(G',\terms,p)$.

\paragraph{From a multicut to a homomorphism.}
Let $X$ be a solution to the \dirmc{} instance $(G',\terms,p)$. Our goal is to find a homomorphism of $H$ into $G$.

First, let us focus on heavy vertices in $X$. Observe that for every pair $(i,j)$, $ij \in E(H)$ the following three paths needs to be cut by $X$:
\begin{itemize}
\item
a path from $\sincx$ to $\tincx$ that traverses the entire path $X^{i,j}$ up to the vertex $x^{i,j}_n$, and uses the arc $(x^{i,j}_n,z^i_n)$ to reach
$\tincx$, 
\item a path from $\sincx$ to $\tincx$ that starts with using the arc $(x^{i,j}_0,z^i_0)$, and then traverses $Z^i$
up to the vertex $z^i_n$, and 
\item a path from $\sincy$ to $\tincy$ that starts with using the arc $(z^i_0,y^{i,j}_0)$, and then traverses $Y^{i,j}$
up to the vertex $y^{i,j}_n$.
\end{itemize}
We infer that $X$ needs to contain at least one heavy vertex on every $x$-, $y$-, and $z$-path in $H$. Recall that the total weight of exactly one heavy vertex from 
each of these paths is $6kM = p-k$. Thus, we have only $k$ slack in the budget constraint.

We say that a path $X^{i,j}$, $Y^{i,j}$, or $Z^i$ is \emph{normal}
if it contains exactly one vertex of $X$, and \emph{cheated}
otherwise.
We say that a pair $(i,j)$ for $ij \in E(H)$ is \emph{normal}
if each of the paths $X^{i,j}$, $Y^{i,j}$, $X^{j,i}$, $Y^{j,i}$, 
$Z^i$, and $Z^j$ is normal.
A pair $(i,j)$ is \emph{cheated} if it is not normal.
Note that $(i,j)$ is normal if and only if $(j,i)$ is normal.

For every $i \in V(H)$ we fix one $\phi(i) \in [n]$ such that
$\hat{z}^i_{\phi(i)} \in X$.

Fix now a normal pair $(i,j)$, $ij \in E(H)$.
Assume $i < j$; a symmetrical argument holds for $i > j$ but uses the terminal pair $(\sdecgt,\tdecgt)$
instead of $(\sdeclt,\tdeclt)$. Let $\hat{x}^{i,j}_a, \hat{z}^i_b, \hat{y}^{i,j}_c \in X$. 
Observe that $a \leq b$, as otherwise the path from $\sincx$ to $\tincx$ that traverses $X^{i,j}$ up to the vertex $x^{i,j}_{a-1}$, uses the arc $(x^{i,j}_{a-1},z^i_{a-1})$, and traverses $Z^i$ up to the endpoint $z^i_n$ is not cut by $X$, a contradiction.
A similar argument for the terminal pair $(\sincy,\tincy)$ implies that $b \leq c$.
However, if $a < b$, then the path from $\sdeclt$ to $\tdeclt$ that traverses $X^{i,j}$ up to the vertex $x^{i,j}_a$, uses the arc
$(x^{i,j}_a,z^i_a)$, traverses $Z^i$ up to the endpoint $z^i_0$, and finally uses the arc $(z^i_0,y^{i,j}_0)$, is not cut by $X$,
a contradiction. A similar argument implies gives a contradiction if $b < c$.

We infer that for every normal pair $(i,j)$ we have $\hat{z}^i_{\phi(i)} \in X$ and
$\hat{x}^{i,j}_{\phi(i)},\hat{y}^{i,j}_{\phi(i)} \in X$.

Fix now a normal pair $(i,j)$ with $i < j$. Observe that the following paths are not cut by the heavy vertices in $X$:
\begin{itemize}
\item a path from $\sdeclt$ to $\tdeclt$
that traverses $X^{i,j}$ up to the vertex $x^{i,j}_{\phi(i)}$, uses the arc $(x^{i,j}_{\phi(i)},p^{i,j}_{\phi(i),1})$,
traverses the $\phi(i)$-th row of $P^{i,j}$ up to the vertex $p^{i,j}_{\phi(i),n}$, uses the arc $(p^{i,j}_{\phi(i),n}, y^{i,j}_{\phi(i)-1})$, and traverses $Y^{i,j}$ up to the endpoint $y^{i,j}_0$;
\item a path from $\sdecgt$ to $\tdecgt$
that traverses $X^{j,i}$ up to the vertex $x^{j,i}_{\phi(j)}$, uses the arc $(x^{j,i}_{\phi(j)},p^{i,j}_{1,\phi(j)})$,
traverses the $\phi(j)$-th column of $P^{i,j}$ up to the vertex $p^{i,j}_{n,\phi(j)}$, uses the arc $(p^{i,j}_{n,\phi(j)}, y^{j,i}_{\phi(j)-1})$, and traverses $Y^{j,i}$ up to the endpoint $y^{j,i}_0$.
\end{itemize}
Consequently, $X$ needs to contain at least one light vertex in the $p$-grid
$P^{i,j}$.
Furthermore, if $X$ contains exactly one light vertex in $P^{i,j}$,
then, as the only vertex in common of the two aforementioned paths for a fixed choice of normal $(i,j)$, $i < j$, is the vertex $p^{i,j}_{\phi(i),\phi(j)}$,
we have that $p^{i,j}_{\phi(i),\phi(j)} \in X$ is a light vertex, and, by construction, $v^i_{\phi(i)} v^j_{\phi(j)} \in E(G)$.

It remains to show that every pair $(i,j)$, $ij \in E(H)$, is normal.
Indeed, if this is the case, then, as $p = 6kM + k$,
the total weight of exactly one heavy vertex on each path $X^{i,j}$, $Y^{i,j}$, and $Z^i$ is $6kM$,
and there are exactly $k$ grids $P^{i,j}$, every grid $P^{i,j}$ contains
exactly one vertex of $X$, and the argumentation from the previous section
shows that $i \mapsto v^i_{\phi(i)}$ is a homomorphism, concluding
the proof of Theorem~\ref{thm:dirmc-lb}.

Recall that $(i,j)$ is normal if and only if $(j,i)$ is normal.
Let $c$ be the number of cheated pairs $(i,j)$, $i < j$.
If $(i,j)$, $i < j$, is cheated, then there is a witness for it:
one of the paths $X^{i,j}$, $Y^{i,j}$, $X^{j,i}$, $Y^{j,i}$,
$Z^i$, or $Z^j$ is cheated, that is, contains more than one vertex of $X$.
However, note that a cheated path $X^{i,j}$, $Y^{i,j}$, $X^{j,i}$, or $Y^{j,i}$ is a witness only that $(i,j)$ and $(j,i)$ is cheated.
Let $c_{XY}$ be the number of cheated $x$- and $y$-paths.
Furthermore, a cheated path $Z^i$ is a witness that $(i,j)$ is cheated
for every $ij \in E(H)$: there are only $\deg_H(i)$ such pairs $(i,j)$.
We infer that
$$c \leq c_{XY} + \sum_{i \in V(H): Z^i\ \textrm{cheated}} \deg_H(i).$$
On the other hand, 
a cost of a second heavy vertex in $X$ on an $x$- or $y$-path is $M$,
while the cost of a second heavy vertex on $Z^i$ is $M \cdot \deg_H(i)$.
Furthermore, recall that if $(i,j)$, $i < j$, is normal, then $P^{i,j}$ contains at least one vertex of $X$.
Thus, the total weight of $X$ is at least
\begin{align*}
& 6kM + M\cdot c_{XY} + \sum_{i \in V(H): z^i\ \textrm{cheated}} M \cdot deg_H(i) + k - c \\
&\quad \geq 6kM + M\cdot c + k - c = p + (M-1)c.
\end{align*}
Consequently, if $M > 1$ and the weight of $X$ is at most $p$, we have $c=0$.
This finishes the proof of Theorem~\ref{thm:dirmc-lb}.

The resulting digraph can easily be shown to have directed pathwidth 2.
Clearly, it cannot have smaller directed pathwidth, as it contains two-vertex
cycles.
In the other direction, order the vertices of the resulting graphs as follows.
\begin{enumerate}
\item All source terminals.
\item The vertices $x^{i,j}_a$, sorted first by the pair $(i,j)$ lexicographically, and then by the subscript $a$.
\item The vertices $z^i_a$, sorted first by $i$, and then by the subscript $a$.
\item The vertices $p^{i,j}_{a,b}$, sorted first by the pair $(i,j)$ lexicographically, and then by the pair $(a,b)$ lexicographically.
\item The vertices $y^{i,j}_a$, sorted first by the pair $(i,j)$ lexicographically, and then by the subscript $a$.
\item All sink terminals.
\end{enumerate}
Observe now that we can 
construct a directed path decomposition of the resulting graph
by taking bags consisting of every two consecutive vertices in this order.

We infer that \dirmc{} is $W[1]$-hard already for integer-weighted instances,
parameterized by total solution weight, for instances with 4 terminals 
and directed pathwidth 2. This is in sharp contrast to the result
that it is FPT for DAGs~\cite{dags-alg}.
