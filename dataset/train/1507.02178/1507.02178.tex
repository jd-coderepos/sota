\newcommand{\sincx}{s_{0 \to n}^{x}}
\newcommand{\tincx}{t_{0 \to n}^{x}}
\newcommand{\sincy}{s_{0 \to n}^{y}}
\newcommand{\tincy}{t_{0 \to n}^{y}}
\newcommand{\sdeclt}{s_{n \to 0}^{<}}
\newcommand{\tdeclt}{t_{n \to 0}^{<}}
\newcommand{\sdecgt}{s_{n \to 0}^{>}}
\newcommand{\tdecgt}{t_{n \to 0}^{>}}

We reduce an input \subiso{} instance 
to a node-weighted variant of \dirmc{}, where every non-terminal vertex has a weight being a positive integer, and the goal is to find a cutset of total weight
not exceeding the budget .
We fix some integer constant ; in fact it suffices to set ,
but it helps to think of it as a sufficiently large integer constant.
 We use three levels of weight: there will be \emph{light vertices}, of weight , \emph{heavy vertices}, of various integer weights being multiples of ,
usually depending on the degree of the corresponding vertex of , and \emph{undeletable vertices}, of weight . Observe that it is easy to reduce the weighted variant to the original one by replacing every vertex of weight  with  unit-weight vertices. Thus, it suffices to show the reduction to the weighted variant, with only four terminal pairs.

\paragraph{Construction.}
Let us now describe the construction of the (weighted)
\dirmc{} instance .
We start by setting budget .
We also introduce eight terminals, arranged in four terminal pairs:

 
For every , we introduce a bidirected path on  vertices

called henceforth the \emph{-path for vertex }, and denoted by .
Similarly, for every ordered pair  where , we introduce two bidirected paths on  vertices

We call these paths \emph{the -path and the -path for the pair }, and denote them by  and .
All vertices , ,  are undeletable, while all vertices , ,  are heavy:
vertex  has weight  and the vertices
 and  have weight  each.

Note that so far we have created  paths, each having  vertices.
Furthermore, if we are to delete one heavy vertex from each of these paths,
the total cost would be


For every pair  with ,
and every , we add arcs  and . Furthermore, we attach terminals to the paths as follows.
\begin{itemize}
\item for every pair  with ,
we add arcs  and ;
\item for every  we add arcs  and ; \item for every pair  with  and  we add arcs  and ;
\item for every pair  with  and 
we add arcs  and .
\end{itemize}

\begin{figure*}[tb]
\begin{center}
\includegraphics{fig-dirmc}
\caption{Illustration of the reduction for \dirmc{}. Black vertices are light, gray are heavy, and white are undeletable.
The top figure illustates a -grid, together with an intended solution marked by red circles.
Here, the vertex  lies in the top-left corner of the grid, the first coordinate describes the row of the grid,
and the second one the column.
The bottom figure illustrates an -, -, and -path for a pair  with  and .}
\label{fig:dirmc}
\end{center}
\end{figure*}

We refer to Figure~\ref{fig:dirmc} for an illustration.
Intuitively, the so-far constructed bidirectional paths and terminals require to delete at least one heavy vertex per 
bidirectional path; the connections between paths ensure that for every , we need to chose
one index  and delete vertices , , and , that is, cut all paths corresponding to the vertex  at the same place. The choice of the index  corresponds
to the choice of the image of  in the sought homomorphism.

Let us now introduce gadgets that check the edge relations between the chosen vertices.
For every pair  with ,  we introduce an acyclic  grid with vertices  for 
and arcs  for every  and , as well as 
 for every  and .
We call this grid \emph{-grid for the pair }, and denote it by .
We set the vertex  to be a light vertex if , and undeletable otherwise.
Finally, for every  we introduce arcs:

Intuitively, after deleting the aforementioned heavy vertices on the -, -, and -paths, for fixed , ,
there is only one remaining path from  (an out-neighbor of ) to  (an in-neighbor of ),
passing through the -th row of the -grid for the pair ,
and there is only one remaining path from  (an out-neighbor of ) to  (an in-neighbor of ),
passing through the -th column of the -grid for the pair .
We can kill both these paths with a single vertex , but only
the existence of the edge  ensures that this vertex is light, not undeletable.

This concludes the construction of the instance .
We now formally show that the constructed instance is equivalent to the input
\subiso{} instance .

\paragraph{From a homomorphism to a cutset.}
Let  be such that 
is a homomorphism of  into .
Define

The total weight of the vertices in  equals:

Note that the fact that 
 is light for every , 
follows from the assumption that the vertices  
is a homomorphism.
Consequently,  is of weight exactly . We now show that it is a multicut in .

We start with a simple observation about the structure of the graph . While
the -, -, and -paths are bidirected, they --- together with the -grids --- are arranged in a DAG-like fashion.
That is, there are directed arcs from  to , from  to , from  and  to , and from  to  and ,
but all cycles in  are contained in one -, -, or -path. 
 
Consider first the terminal pair . The out-neighbors of  are the endpoints  for every pair  with ;
the only in-neighbors of  are the endpoints  for every .
Thus, by the previous observation, the only paths from  to  in the graph  start by going to some vertex , traverse  up to some vertex , use the arc 
to fall to , and then traverse  to the endpoint . 
(In particular, there are no paths from  to  that contain a vertex of some grid .)
However, all such paths for  are cut
by the vertex , while all such paths for  are cut by the vertex . Consequently,
the terminal pair  is separated in .

A similar argument holds for the pair . By the same reasoning, the only paths between  and  in the graph 
are paths that start by going to some vertex , traverse  up to some vertex , use the arc
 for some  to fall to , and then continue along this -path to the vertex .
However, all such paths for  are cut by the vertex , while all such paths for  are cut by the vertex .

Let us now focus on the terminal pair . Observe that there are two types of paths from  to  in the graph .
The first type consists of paths that starts by going to some vertex  where , traverse  up to some vertex ,
use the arc  to fall to , then traverse  up to some vertex , use
the arc  for some  to fall to , and finally traverse this -path to the endpoint .
However, similarly as in the previous cases, the vertices  cut all such paths.

The second type of paths use the -grids in the following manner: the path starts by going to some vertex  where ,
traverse  up to some vertex , use the arc  to fall to ,
traverse this -grid up to a vertex  where , use the arc  to fall to ,
and traverse this path to the endpoint . These paths are cut by  as follows: the paths where  are cut by ,
the paths where  are cut by , while the paths where  are cut by the vertex ;
note that the -th row of the grid is the only path from  to .
Please observe that the terminal  cannot be reached from  by going to the other -path reachable from this -grid,
namely , as  has only outgoing arcs to the terminal  since .

A similar argument holds for the pair . The paths going through , ,
, and , , are cut by vertices
.
The paths going through , , , and ,
are cut by the vertices . Again, it is essential that the other -path
reachable from the -grid for the pair , namely , does not have outgoing arcs to the terminal , but only
to the terminal .

We infer that  is a solution to the \dirmc{} instance .

\paragraph{From a multicut to a homomorphism.}
Let  be a solution to the \dirmc{} instance . Our goal is to find a homomorphism of  into .

First, let us focus on heavy vertices in . Observe that for every pair ,  the following three paths needs to be cut by :
\begin{itemize}
\item
a path from  to  that traverses the entire path  up to the vertex , and uses the arc  to reach
, 
\item a path from  to  that starts with using the arc , and then traverses 
up to the vertex , and 
\item a path from  to  that starts with using the arc , and then traverses 
up to the vertex .
\end{itemize}
We infer that  needs to contain at least one heavy vertex on every -, -, and -path in . Recall that the total weight of exactly one heavy vertex from 
each of these paths is . Thus, we have only  slack in the budget constraint.

We say that a path , , or  is \emph{normal}
if it contains exactly one vertex of , and \emph{cheated}
otherwise.
We say that a pair  for  is \emph{normal}
if each of the paths , , , , 
, and  is normal.
A pair  is \emph{cheated} if it is not normal.
Note that  is normal if and only if  is normal.

For every  we fix one  such that
.

Fix now a normal pair , .
Assume ; a symmetrical argument holds for  but uses the terminal pair 
instead of . Let . 
Observe that , as otherwise the path from  to  that traverses  up to the vertex , uses the arc , and traverses  up to the endpoint  is not cut by , a contradiction.
A similar argument for the terminal pair  implies that .
However, if , then the path from  to  that traverses  up to the vertex , uses the arc
, traverses  up to the endpoint , and finally uses the arc , is not cut by ,
a contradiction. A similar argument implies gives a contradiction if .

We infer that for every normal pair  we have  and
.

Fix now a normal pair  with . Observe that the following paths are not cut by the heavy vertices in :
\begin{itemize}
\item a path from  to 
that traverses  up to the vertex , uses the arc ,
traverses the -th row of  up to the vertex , uses the arc , and traverses  up to the endpoint ;
\item a path from  to 
that traverses  up to the vertex , uses the arc ,
traverses the -th column of  up to the vertex , uses the arc , and traverses  up to the endpoint .
\end{itemize}
Consequently,  needs to contain at least one light vertex in the -grid
.
Furthermore, if  contains exactly one light vertex in ,
then, as the only vertex in common of the two aforementioned paths for a fixed choice of normal , , is the vertex ,
we have that  is a light vertex, and, by construction, .

It remains to show that every pair , , is normal.
Indeed, if this is the case, then, as ,
the total weight of exactly one heavy vertex on each path , , and  is ,
and there are exactly  grids , every grid  contains
exactly one vertex of , and the argumentation from the previous section
shows that  is a homomorphism, concluding
the proof of Theorem~\ref{thm:dirmc-lb}.

Recall that  is normal if and only if  is normal.
Let  be the number of cheated pairs , .
If , , is cheated, then there is a witness for it:
one of the paths , , , ,
, or  is cheated, that is, contains more than one vertex of .
However, note that a cheated path , , , or  is a witness only that  and  is cheated.
Let  be the number of cheated - and -paths.
Furthermore, a cheated path  is a witness that  is cheated
for every : there are only  such pairs .
We infer that

On the other hand, 
a cost of a second heavy vertex in  on an - or -path is ,
while the cost of a second heavy vertex on  is .
Furthermore, recall that if , , is normal, then  contains at least one vertex of .
Thus, the total weight of  is at least

Consequently, if  and the weight of  is at most , we have .
This finishes the proof of Theorem~\ref{thm:dirmc-lb}.

The resulting digraph can easily be shown to have directed pathwidth 2.
Clearly, it cannot have smaller directed pathwidth, as it contains two-vertex
cycles.
In the other direction, order the vertices of the resulting graphs as follows.
\begin{enumerate}
\item All source terminals.
\item The vertices , sorted first by the pair  lexicographically, and then by the subscript .
\item The vertices , sorted first by , and then by the subscript .
\item The vertices , sorted first by the pair  lexicographically, and then by the pair  lexicographically.
\item The vertices , sorted first by the pair  lexicographically, and then by the subscript .
\item All sink terminals.
\end{enumerate}
Observe now that we can 
construct a directed path decomposition of the resulting graph
by taking bags consisting of every two consecutive vertices in this order.

We infer that \dirmc{} is -hard already for integer-weighted instances,
parameterized by total solution weight, for instances with 4 terminals 
and directed pathwidth 2. This is in sharp contrast to the result
that it is FPT for DAGs~\cite{dags-alg}.
