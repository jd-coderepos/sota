









In the previous sections we saw that \slrbsc\ parameterized by  is para-\NP-complete and is  \W[1]-hard parameterized by 
.  So there is no hope of an \FPT\ algorithm unless  or \FPT\ =\W[1], when parameterized by  and  respectively.  As a consequence, we consider combining different natural parameters with  to see if this helps to find \FPT\ algorithms. In fact, in this section, we describe a \FPT\ algorithm for \slrbsc\ parameterized by .  Since , this implies that \slrbsc\ parameterized by  is \FPT. This is one of our main technical/algorithmic contribution. 
Also, since  for any minimal solution family of an instance, it follows that \slrbsc\ parameterized by  belongs to \FPT. It is natural to ask whether the \srbsc\ problem, that is, where sets in the family are arbitrary subsets  of the universe and need not correspond to lines, is \FPT\ parameterized by . In fact, Theorem~\ref{MeCC_redn} states that the problem is \W[1]-hard even when each set is of size three and contains at least two red points. This shows that indeed restricting ourselves to sets corresponding to lines makes the problem tractable. 











We start by considering a simpler case, where the input instance is such that every line contains exactly  blue point. Later we will show how we can reduce our main problem to such instances. By the restrictions assumed on the input, no two blue points can be covered by the same line and any solution family must contain at least  lines. Thus,  or else, it is a \NO\ instance. Also, a minimal solution family will contain at most  lines. Hence, from now on we are only interested in the existence of minimal solution families. In fact, inferring from the above observations, a minimal solution family, in this special case, contains exactly  lines. Let  be the intersection graph that corresponds to a minimal solution . Recall, that in  vertices correspond to lines in  and there is an edge between two vertices  in  if the corresponding lines intersect either at a blue point or a red point. Next, we define notions of 
{\em good tuple} and {\em conformity} which will be useful in designing the \FPT\ algorithm for the special case. 
Essentially, a good tuple provides a numerical representation of connected components of . 

\begin{definition} \label{goodtuple}Given an instance  of \slrbsc\, we call a tuple 
  {\em good} if the following hold. 
\begin{enumerate}[(a)] 
 \item Integers   and ; Here  is the number of blue vertices in the instance.
 \item  is an -partition of ;
 \item For each ,  is an ordering for the blue points in part ;
 \item Integers , are such that . 
\end{enumerate}
\end{definition}


Below, we define the relevance of good tuples in the context of our problem.
\begin{definition}\label{conforming_tuple}
 We say that the minimal solution family  {\em conforms} with a good tuple 
if the following properties hold:
 \begin{enumerate}
\item The components  of  give the partition  on the blue points.
    \item For each component , , let . Let   be an ordering of blue points in . Furthermore assume that  covers the blue point .  has the property that for all   is connected. In other words for all ,  intersects with at least one of the lines from the set .  Notice that, by minimality of  
     , the point of intersection for such a pair of lines is a red point.  
    
\item  covers  red points.
  \item In each component ,  is the number of red points covered by the lines in that component. It follows that . In other words, the integers  form a combination of .
 \end{enumerate}
\end{definition}

The next lemma says that the existence of a minimal solution subfamily  results in a conforming good tuple. 
\begin{lem}
\label{lem:conformingoodtuplexists}
Let   be an input to  \slrbsc\ parameterized by , such that every line contains exactly  blue point. If there exists a solution subfamily  then there is a conforming good tuple. 
\end{lem}
\begin{proof}
Let  be a minimal solution family of size  that covers  red points. Let  have  components viz. , where . For each , let  denote the set of lines corresponding to the vertices of . ,  and . In this special case and by minimality of , . As  is connected, there is a sequence  for the lines in  such that for all  we have that 
     is connected. This means that, for all   intersects with at least one of the lines from the set . By minimality of , the point of intersection for such a pair of lines is a red point. For all , let  cover the blue point . Let . 
The tuple  is a good tuple and it also conforms with .   This completes the proof.  
\end{proof}



The idea of the algorithm is to generate all good tuples and then check whether there is a 
solution subfamily  that conforms to it. The next lemma states we can check for a conforming minimal solution family when we are given a good tuple. 
\begin{lem}\label{good_tuples}
 For a good tuple , 
we can verify in  time whether there is a minimal solution family  that conforms with this tuple.
\end{lem}

\begin{proof}
The algorithm essentially builds a search tree for each partition . For each part , we define a set of points  which is initially an empty set. 




For each , let  and let   be the ordering of blue points in . Our objective is to check whether there is a subfamily  such that it covers , and at most  red point. At any stage of the algorithm, we have a subfamily  covering  
 and at most  red points. In the next step we try to enlarge   in such a way 
that it also covers , but still covers at most  red points. In some sense we follow the ordering given by  to build . 

Initially, 
. At any point of the recursive algorithm we represent the problem to be solved by the following tuple: 
(, , (), ). 
We start the process by guessing the line in  that covers , say . That is, for every  such that 
 is contained in  we recursively check whether there is a solution to the tuple 
 (, , (),).  If any tuple returns \YES\ then we return that there is a subset  which covers , and at most  red points. 

Now suppose we are at an intermediate stage of the algorithm and the tuple we have is (, , (), ). Let  be the set of lines such that it contains  and a red point from . Clearly, . For every line , we  recursively check whether there is a solution to the tuple (, , (),).  If any tuple returns \YES\ then we return that there is a subset  which covers , and at most  red points. 

Let .  At each stage  drops by one and, except for the first step, the algorithm recursively solves at most  subproblems. This implies that the algorithm takes at most  time.  




Notice that the lines in the input instance are partitioned according to the blue points contained in it. Hence, the search corresponding to each part  is independent of those in other parts. In effect, we are searching for the components for  in the input instance, in parallel. If for each  we are successful in finding a minimal set of lines covering exactly the blue points of  while covering at most  red points, we conclude that a solution family  that conforms to the given tuple exists and hence the input instance is a \YES\ instance.

The time taken for the described procedure in each part is at most . Hence, the total time taken to check if there is a conforming minimal solution family  is at most 
 
This concludes the proof. 
\end{proof}


We are ready to describe our \FPT\ algorithm for this special case of \slrbsc\ parameterized by .
\begin{lem}\label{solution_1blue}
Let   be an input to  \slrbsc\ such that every line contains exactly  blue point. Then we can check whether there is a solution subfamily  to this instance in time 
 time.
\end{lem}
\begin{proof}
Lemma~\ref{lem:conformingoodtuplexists} implies that for the algorithm all we need to do is to enumerate all possible good tuples 
, and for each tuple, check whether there 
is a conforming minimal solution family. Later, we use the algorithm described in Lemma~\ref{good_tuples}. 
We first give an upper bound on the number of tuples  and how to enumerate them. 
\begin{enumerate}
\item There are  choices for  and  choices for .
\item There can be at most  choices for  which can be enumerated in   time. 
\item For each ,  is ordering for blue points in . Thus, if , then the number of  ordering 
tuples  is upper bounded by . 
Such orderings can be enumerated in  time.
 
\item For a fixed , there are 
at most  solutions for   
and this set of solutions can be enumerated in  time. Notice that if  then the time required for enumeration is . Otherwise, the required time is . As  and , the time required to enumerate the set of solutions is . 
\end{enumerate}
   Thus we can generate the set of tuples in time .Using Lemma~\ref{good_tuples}, for each tuple we check in at most  time whether there is a conforming solution family or not. If there is no tuple with a conforming solution family, we know that the input instance is a \NO\ instance. 
   The total time for this algorithm is . Again, if  then . Otherwise, . Either way, it is always true that . Thus, we can simply state the running time to be 
   .  
\end{proof}








We return to the general problem of \slrbsc\ parameterized by . Instances in this problem may have lines containing  or more blue points. 
We use the results and observations described above to arrive at an \FPT\ algorithm for \slrbsc\ parameterized by .

\begin{theorem}
\label{thm:colour_coding}
 \slrbsc\ parameterized by  is \FPT, with an algorithm that runs in  time.
\end{theorem}
\begin{proof}
Given an input  for \slrbsc\ parameterized by , we do some preprocessing to make the instance simpler. We exhaustively apply Reduction Rules~\ref{redn1}, \ref{red_vertices_bd} and~\ref{redn4}. After this, by Observation~\ref{blue_size}, the reduced equivalent instance has at most  blue points  if it is a \YES\ instance.

A minimal solution family can be broken down into two parts:  the set of lines containing at least  blue points, and the remaining set of lines which contain exactly  blue point.  Let us call these sets  and  respectively. We start with the following observation. 

\begin{observation}\label{2blue_lines}
 Let  be the set of lines that contain at least  blue points. There are at most  ways in which a solution family can intersect with .
\end{observation}
\begin{proof}
 Since , it follows from Observation~\ref{covering_lines_bd} that . For any solution family, there can be at most  lines containing at least  blue points. Since the number of subsets of  of size at most  is bounded by , the observation is true.
\end{proof}

 From Observation~\ref{2blue_lines}, there are  choices for the set of lines in . We branch on all these choices of . On each branch, we reduce the budget of  by the number of lines in  and the budget of  by . Also, we make some modifications on the input instance: we delete all other lines containing at least  blue points from the input instance. We delete all points of  covered by  and all lines passing through blue points covered by . Our modified input instance in this branch now satisfies the assumption of Lemma~\ref{solution_1blue} and we can find out in  time whether there is a minimal solution family  for this reduced instance. If there is, then  is a minimal solution for our original input instance and we correctly say \YES.
Thus the total running time of this algorithm is . 

It  may be noted here that for a special case where we can use any line in the plane as part of the solution, the second part of the algorithm becomes considerably simpler. Here for each blue point , we can use an arbitrary line containing only  and no red point.
 \end{proof} 
 





\begin{corollary}\label{sp_case_red}
\slrbsc\ parameterized by , where every line contains at most  red points, is \FPT. The running time of the \FPT\ algorithm is . The problem remains \FPT\ for all parameter sets  that contain  or . 
\end{corollary}

\begin{proof}
In this special case, any solution family can contain at most  red points. Hence we can safely assume that  and apply Theorem~\ref{thm:colour_coding}. 
\end{proof}
\subsection{Kernelization for \slrbsc\ parameterized by  and }


We give a polynomial parameter transformation from \SC\ parameterized by universe size , to \slrbsc\ parameterized by . Proposition \ref{SC_results}(ii) implies that on parameterizing by any subset of the parameters , we will also obtain a negative result for polynomial kernels.


\begin{thmk}\label{set_cover_redn}
 \slrbsc\ parameterized by  does not allow a polynomial kernel unless .
\end{thmk}
 
\begin{proof}
 
 Let  be a given instance of \SC. Let . We construct an instance  of \slrbsc\ as follows. We assign a blue point  for each element  and a red point  for each set . The red and blue points are placed such that no three points are collinear. We add a line between  and  if  in the \SC\ instance. Thus the \slrbsc\ instance  that we have constructed has ,  and . We set  and .
 
 \begin{claim}
  All the elements in  can be covered by  sets if and only if there exist  lines in  that contain all blue points but only  red points.
 \end{claim}
\begin{proof}
 Suppose  has a solution of size , say . The red points in the solution family for \slrbsc\ are   corresponding to . For each element , we arbitrarily assign a covering set  from . The solution family is the set of lines defined by the pairs 
 . This covers all blue points.

 Conversely, if  has a solution family  covering  red points and using at most  lines, the sets in  corresponding to the red points in  cover all the elements in .
\end{proof}
If , then the \SC\ instance is a trivial \YES\ instance. Hence, we can always assume that .
This completes the proof that \slrbsc\ parameterized by  cannot have a polynomial sized kernel unless .
\end{proof}


