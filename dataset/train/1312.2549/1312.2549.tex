

\documentclass[letterpaper,11pt]{article}

\usepackage{amsthm,amsmath,amssymb}
\usepackage{graphicx}
\usepackage[small]{caption}

\setlength\hoffset{0in}
\setlength\voffset{0in}
\setlength\oddsidemargin{0in}
\setlength\evensidemargin{0in}
\setlength\topmargin{0in}
\setlength\headheight{0in}
\setlength\headsep{0in}
\setlength\textwidth{6.57in}
\setlength\textheight{9.05in}

\newtheorem{theorem}{Theorem}
\newtheorem{lemma}{Lemma}
\newtheorem{proposition}{Proposition}
\newtheorem{conjecture}{Conjecture}

\def\etal{{et~al.}}
\def\ie{{i.e.}}
\def\eg{{e.g.}}

\def\F{{\mathcal F}}
\def\L{{\mathcal L}}
\def\S{{\mathcal S}}
\def\T{{\mathcal T}}

\newcommand{\eps}{\varepsilon}
\newcommand{\opt}{\mathrm{opt}}
\newcommand{\val}{\mathrm{val}}

\newcommand{\old}[1]{}
\newcommand{\later}[1]{}


\begin{document}

\title{On the approximability of covering points by lines\\and related
  problems\thanks{Supported in part by NSF grant DMS-1001667 awarded to the first author.
The work of the second author was performed during his sabbatical leave
in Fall 2012.}}

\author{Adrian Dumitrescu\thanks{Department of Computer Science,
University of Wisconsin--Milwaukee, USA\@.
Email:~\texttt{dumitres@uwm.edu}.}
\and
Minghui Jiang\thanks{Department of Computer Science,
Utah State University,
Logan, USA\@.
Email: \texttt{mjiang@cc.usu.edu}.}}

\maketitle
\thispagestyle{empty}


\begin{abstract}
Given a set  of  points in the plane, {\sc Covering Points by Lines}
is the problem of finding a minimum-cardinality set  of lines
such that every point  is incident to some line .
As a geometric variant of {\sc Set Cover}, {\sc Covering Points by Lines} 
is still NP-hard. 
Moreover, it has been proved to be APX-hard, and hence
does not admit any polynomial-time approximation scheme unless P  NP\@.
In contrast to the small constant approximation lower bound
implied by APX-hardness,
the current best approximation ratio for {\sc Covering Points by Lines} 
is still , namely the ratio achieved by the greedy
algorithm for {\sc Set Cover}.

In this paper,
we give a lower bound of 
on the approximation ratio of the greedy algorithm for {\sc Covering Points by Lines}.
We also study several related problems including
{\sc Maximum Point Coverage by Lines},
{\sc Minimum-Link Covering Tour},
{\sc Minimum-Link Spanning Tour},
and {\sc Min-Max-Turn Hamiltonian Tour}.
We show that all these problems are either APX-hard or at least NP-hard.
In particular, our proof of APX-hardness of {\sc Min-Max-Turn Hamiltonian Tour}
sheds light on the difficulty of {\sc Bounded-Turn-Minimum-Length Hamiltonian Tour},
a problem proposed by Aggarwal et~al.\ at SODA 1997.

\later{
\medskip
\medskip
\textbf{\small Keywords}:
Spanning tours,
Covering tours,
NP-hardness,
APX-hardness.
}

\end{abstract}

\newpage
\setcounter{page}{1}
\setcounter{footnote}{0}


\section{Introduction}

Given a set  of  elements and a family  of  subsets of ,
{\sc Set Cover} is the problem of finding a minimum-cardinality subfamily
 whose union is .
It is well-known that
{\sc Set Cover} can be approximated within
~\cite{Jo74,Lo75,Ch79},
where  is the th harmonic number,
by a simple greedy algorithm that
repeatedly selects a set that covers the most remaining elements;
a more refined analysis~\cite{Sl97} shows that the approximation ratio
of the greedy algorithm is in fact .
On the other hand, {\sc Set Cover} 
cannot be approximated in polynomial time
within  for some constant 
unless P  NP~\cite{RS97},
and
within 
for any 
unless NP  TIME~\cite{Fe98};
see also~\cite{AMS06,LY94}.

The first problem that we study in this paper
is a geometric variant of {\sc Set Cover}.
Given a set  of  points in the plane,
{\sc Covering Points by Lines}
is the problem of finding a minimum-cardinality set  of lines
such that every point  is in some line .
(Without loss of generality,
we can assume that  and that the lines in  are selected from
the set  of at most  lines
with at least two points of  in each line.)

As a restricted version of {\sc Set Cover},
{\sc Covering Points by Lines} may appear as a much easier problem.
Indeed, in terms of parameterized complexity, {\sc Set Cover} is clearly
W[2]-hard when the parameter is the number  of sets in the solution
(as easily seen by a reduction from the canonical W[2]-hard problem
-{\sc Dominating Set}),
while {\sc Covering Points by Lines} 
admits very simple FPT algorithms based on standard
techniques in parameterized complexity such as bounded search tree
and kernelization~\cite{LM05}; see also~\cite{GL06,WLC10}.
In terms of approximability, however,
the current best approximation ratio for {\sc Covering Points by Lines} is still
the same  upper bound for {\sc Set Cover}
achieved by the greedy algorithm.
No matching lower bounds are known for {\sc Covering Points by Lines},
although it has been proved to be NP-hard~\cite{MT82}
and even APX-hard~\cite{BHN01,KAR00};
the APX-hardness of the problem implies a constant lower bound
on the approximation ratio, in particular, the problem
does not admit any polynomial-time approximation scheme unless P  NP.

We first give an asymptotically tight lower bound
on the approximation ratio of the greedy algorithm:

\begin{theorem}\label{thm:greedy}
The approximation ratio of the greedy algorithm for {\sc Covering Points by Lines} is
.
\end{theorem}

We also prove that {\sc Covering Points by Lines} is APX-hard, unaware\footnote{We
  thank an anonymous source for bringing this to our attention.} of
the previous APX-hardness results of Brod\'en~\etal~\cite{BHN01} and
Kumar~\etal~\cite{KAR00}:
\begin{theorem}[Brod\'en \etal~\cite{BHN01} and Kumar \etal~\cite{KAR00}]
\label{thm:min}
{\sc Covering Points by Lines} is APX-hard~\textup{\cite{BHN01,KAR00}}.
This holds even if no four of the given points are collinear~\textup{\cite{BHN01}}.
\end{theorem}

A problem closely related to {\sc Set Cover} is the following.
Given a set  of  elements, a family  of  subsets of ,
and a number ,
{\sc Maximum Coverage} is the problem of finding a subfamily 
of  subsets whose union has the maximum cardinality.
In the setting of {\sc Covering Points by Lines},
given a set  of  points in the plane and a number ,
{\sc Maximum Point Coverage by Lines}
is the problem of finding  lines
that cover the maximum number of points in .
For the general {\sc Maximum Coverage} problem,
the greedy algorithm that repeatedly selects a set that covers the most
remaining elements achieves an approximation ratio of
~\cite[Section~3.9]{Ho97};
this is also the current best approximation ratio for 
{\sc Maximum Point Coverage by Lines}. 
On the other hand, {\sc Maximum Coverage} cannot be approximated better than
 for any  unless P  NP~\cite{Fe98},
while {\sc Maximum Point Coverage by Lines} is only known to be NP-hard as implied by
the NP-hardness of {\sc Covering Points by Lines}~\cite{MT82}.
We show that {\sc Maximum Point Coverage by Lines} is APX-hard too:

\begin{theorem}\label{thm:max}
{\sc Maximum Point Coverage by Lines} is APX-hard.
This holds even if no four of the given points are collinear.
\end{theorem}

Our proof of Theorem~\ref{thm:max} is based on the same construction
as in our proof of Theorem~\ref{thm:min}.
In retrospect, we note that
the construction in our proof is the exact dual of the construction
in~\cite{BHN01}: we cover points by lines; they cover lines by points.
For completeness, we include our proofs of Theorems~\ref{thm:min} and~\ref{thm:max}
in the appendix.

Instead of using lines,
we can cover the points using a polygonal chain of line segments.
Given a set  of  points in the plane,
a \emph{covering tour} is a closed chain of segments
that cover all  points in ,
and a \emph{spanning tour} is a covering tour in which
the endpoints of all segments are points in .
The problem {\sc Minimum-Link Covering Tour}
(respectively,
{\sc Minimum-Link Spanning Tour})
aims at finding a covering tour
(respectively,
\emph{spanning tour})
with the minimum number of links (\ie, segments).
Arkin~\etal~\cite{AMP03} proved that {\sc Minimum-Link Covering Tour}
is NP-hard; see also~\cite{ABDFMS05,Ji12} and the references therein.
Strengthening this result, our following theorem shows that
{\sc Minimum-Link Covering Tour} is in fact APX-hard:

\begin{theorem}\label{thm:covering}
{\sc Minimum-Link Covering Tour} is APX-hard.
This holds even if no four of the given points are collinear.
\end{theorem}

We also show that {\sc Minimum-Link Spanning Tour} is NP-hard:

\begin{theorem}\label{thm:spanning}
{\sc Minimum-Link Spanning Tour} is NP-hard.
This holds even if no four of the given points are collinear.
\end{theorem}

Given a set  of  points in the plane,
a \emph{Hamiltonian tour} is a closed polygonal chain of
\emph{exactly}  segments whose  endpoints along the chain are
a circular permutation of the  points in .
Note that every Hamiltonian tour is a spanning tour, but not vice versa,
although every spanning tour can be transformed into a Hamiltonian tour
by subdividing some segments into chains of shorter collinear segments.
\begin{figure}[htb]
\centering\includegraphics[scale=0.67]{f1.eps}
\caption{Left: turning angle at .
Right: an obtuse tour of  points.}
\label{fig:f1}
\end{figure}

When three points  are traversed in this order in a
Hamiltonian tour, 
the \emph{turning angle} at , denoted by ,
is equal to , where ;
see Figure~\ref{fig:f1}. Note that the turning angle belongs to , 
regardless of the direction of the turn (left or right). 
A~tour or path with each turning angle in  is called \emph{obtuse}.

In the {\sc Euclidean Traveling Salesman Problem} (ETSP),
given a set  of  points in the plane,
one seeks a shortest Hamiltonian tour that visits each point.
However, frequently other parameters are of interest,
such as in motion planning, where small turning angles are desired.
For example, an aircraft or a boat moving at high speed,
required to pass through a set of given locations,
cannot make sharp turns in its
motion\cite{ACKMS99,ART95,BCL94,Fr89,JC89,LFF07}. 
A rough approximation is provided by paths or tours that are obtuse.
However not all point sets admit obtuse tours or even obtuse paths.
For instance, some point sets require turning angles
at least  in any Hamiltonian path~\cite{FW97}.
Moreover, certain point sets (\eg, collinear)
require the maximum turning angle possible, namely ,
in any Hamiltonian tour.

Aggarwal~\etal~\cite{ACKMS99} have studied the following variant of angle-TSP,
which we refer to as {\sc Min-Sum-Turn Hamiltonian Tour}:
Given  points in the plane,
compute a Hamiltonian tour of the points
that minimizes the total turning angle.
The total turning angle of a tour is the sum of the turning
angles at each of the  points.
They proved that this problem is NP-hard and gave a polynomial-time algorithm
with approximation ratio .
They also suggested another natural variant of the basic angle-TSP problem,
where the maximum turning angle at a vertex is bounded and the goal is
to minimize the length measure.

Here we study the computational complexity
of the following two variants of the angle-TSP problem.
The first variant naturally presents itself, however it does not
appear to have been previously studied.
The second variant is one of the two proposed by Aggarwal~\etal~\cite{ACKMS99}.
\begin{itemize} \itemsep -1pt
\item[(I)]
{\sc Min-Max-Turn Hamiltonian Tour}:
Given  points in the plane,
compute a Hamiltonian tour that minimizes the maximum turning angle.
\item[(II)]
{\sc Bounded-Turn-Minimum-Length Hamiltonian Tour}:
Given  points in the plane and an angle ,
compute a Hamiltonian tour with each turning angle at most ,
if it exists, that has the minimum length.
\end{itemize}


We have the following two results for the two variants of angle-TSP:

\begin{theorem}\label{thm:hamiltonian}
{\sc Min-Max-Turn Hamiltonian Tour} is APX-hard.
\end{theorem}

\begin{theorem}\label{thm:hamiltonian2}
{\sc Bounded-Turn-Minimum-Length Hamiltonian Tour} is NP-hard.
\end{theorem}


\section{An  lower bound on the approximation ratio of
  the greedy algorithm} 

In this section we prove Theorem~\ref{thm:greedy}.
Our construction is inspired by a construction of Brimkov~\etal~\cite{BLWM12}
for the related problem {\sc Covering Segments by Points},
which is in turn inspired by a classic lower bound construction for {\sc Vertex Cover}.
This construction shows that there exist graphs with  vertices 
on which the greedy algorithm for {\sc Vertex Cover} achieves
a ratio of . 
\begin{figure}[htb]
\centering\includegraphics[scale=1.05]{greedy1.eps}
\caption{A lower bound construction for {\sc Vertex Cover} and 
{\sc Covering Segments by Points} 
( in this example).}
\label{fig:greedy1}
\end{figure}

Let  be a bipartite graph
where  is a set of  vertices,
 is a set of  vertices
partitioned into  subsets ,
and  is a set of  edges.
For ,
each subset  contains  vertices
which are connected to  vertices in ,
with each vertex in  connected to exactly  distinct vertices in .
Refer to Figure~\ref{fig:greedy1} for an illustration of the graph  with
.

Execute the greedy algorithm for {\sc Vertex Cover} on the bipartite graph .
In each step of the algorithm,
after a vertex of the maximum degree is selected,
the vertex and its incident edges are removed from the graph.
The crucial observation here is that before each selection,
the degree of each vertex in  is at most the number of subsets 
that are not empty,
while the degree of each vertex in a non-empty subset  is exactly .
Thus the vertex of maximum degree selected in each step
is always from a non-empty subset  with the maximum index .
A simple induction shows that the greedy algorithm always selects vertices
from , in this order, and stops when all vertices
in  are selected.
On the other hand, the set of vertices in  clearly covers all edges too.
Thus the approximation ratio of the greedy algorithm for Vertex Cover is
at least

which is , where  is the number
of vertices of .

We now relate {\sc Vertex Cover} to a geometric problem, 
{\sc Covering Segments by Points}:
Given a set  of  line segments in the plane, find a set  of
points of minimum size such that each segment in  contains at least one point in . 
To adapt the construction for {\sc Vertex Cover} to {\sc Covering Segments by Points},
Brimkov~\etal~\cite{BLWM12} place the vertices in  and  in two parallel
lines, with unit distance between consecutive vertices in each line,
and with the vertices in each subset  placed consecutively,
as illustrated in Figure~\ref{fig:greedy1}.
Each edge in  corresponds to
a line segment in  with the two vertices as the endpoints.
Without loss of generality,
each point in  is either a vertex in  or 
in one of the two parallel lines,
or the intersection of two or more segments in 
between the two parallel lines.
Observe that during the execution of the greedy algorithm,
each intersection between (but in neither of) the two parallel lines
is incident to at most one segment from the subset of segments
incident to the vertices in , ;
similar to the vertices in , these intersections are never selected by
the greedy algorithm.
Thus the greedy algorithm still selects the vertices in 
to cover the segments,
and its approximation ratio is still  by the same analysis.
\begin{figure}[htb]
\centering\includegraphics[scale=0.75]{greedy2.eps}
\caption{Adapting the lower bound construction for {\sc Covering Segments by Points} 
to {\sc Covering Lines by Points}.}
\label{fig:greedy2}
\end{figure}

We next adapt this construction further to the problem
{\sc Covering Lines by Points}:
given a set  of  lines in the plane,
find a set  of points of the minimum cardinality such that
each line in  contains at least one point in .
Since {\sc Covering Lines by Points} and 
our original problem {\sc Covering Points by Lines} 
are exact duals of each other, any lower bound we obtain for the
former is also a lower bound for the latter.

Refer to Figure~\ref{fig:greedy2}.
The straightforward part of the adaptation simply extends each segment
in  to a line in . This leads to more intersections, however,
above and below the two parallel lines.
As in the construction for {\sc Covering Segments by Points},
we place the vertices in  evenly in the top line,
with unit distance between consecutive points.
For the vertices in , however, we place them \emph{almost} evenly in
the bottom line, with \emph{near-unit} distance between consecutive
points (for convenience),
such that the following property is satisfied:
\begin{quote}
P1: Any intersection of the lines in ,
if it is not a vertex in  or  in one of the two parallel lines,
it is incident to exactly two lines in .
\end{quote}

To ensure this property, we place the vertices in  incrementally
as follows.
Let  be the subset of vertices in  that have been placed,
and let  be the subset of lines in  incident to .
Let  be the set of points that are intersections of the lines in 
but are not vertices in  or .
For each point  in , and for each vertex  in ,
mark the intersection of the bottom line and the line through  and .
Place the next vertex  in  in the bottom line to avoid such marks.

Due to the property P1, the greedy algorithm selects vertices in
 as before. Then, to cover the  lines incident to ,
it may select intersections not in the two parallel lines, but the number
of points it selects is at least 
since these lines are in general position.
Consequently, the same  lower bound follows and this
completes the proof of Theorem~\ref{thm:greedy}. 




\section{APX-hardness of {\sc Covering Points by Lines} and 
{\sc Maximum Point Coverage by Lines}}

In this section we prove
Theorems~\ref{thm:min} and~\ref{thm:max}.
Given a set  of  variables and a set  of  clauses,
where each variable has exactly  literals (in  different clauses)
and each clause is the disjunction of exactly  literals
(of  different variables),
{\sc E-Occ-Max-E-SAT} is the problem of finding an assignment of
the variables in  that satisfies the maximum number of clauses in .
Note that .
Berman and Karpinski~\cite{BK99} showed that
even the simplest version of this problem,
{\sc E-Occ-Max-E-SAT}, is APX-hard,
and moreover this holds even if
the three literals of each variable
are neither all positive nor all negative;
see also~\cite{BK03} for the current best approximation lower bounds for
the many variants of {\sc E-Occ-Max-E-SAT} and related problems.
We prove that both {\sc Covering Points by Lines} and {\sc Maximum Point Coverage
by Lines} are APX-hard by two gap-preserving reductions from {\sc E-Occ-Max-E-SAT}
(Lemmas~\ref{lem:max} and~\ref{lem:min}, respectively). 

Let  be an instance of {\sc E-Occ-Max-E-SAT},
where  is a set of  variables , ,
and  is a set of  clauses , .
We construct a set  of  points,
including four variable points 
for each variable , ,
and one clause point  for each clause , .
Assume that the three literals of each variable
are neither all positive nor all negative.
Then each variable has either two positive literals and one negative literal,
or two negative literals and one positive literal.
We place the point set  in the plane 
(an example appears in Figure~\ref{fig:line})
such that no line goes through more than two points in 
except in the following two cases:
\begin{enumerate} \itemsep -1pt
\item
If a variable  has two positive literals in  and ,
respectively,
and has one negative literal in ,
then  are collinear,
 are collinear,
and  are collinear.
\item
If a variable  has two negative literals in  and ,
respectively,
and has one positive literal in ,
then  are collinear,
 are collinear,
and  are collinear.
\end{enumerate}

\begin{figure}[htb]
\centering\includegraphics[scale=0.85]{line.eps}
\caption{The set  of  points
corresponding to the {\sc E-Occ-Max-E-SAT} instance  where
,
,
.}
\label{fig:line}
\end{figure}


For any set  of lines,
let  denote the subset of lines
incident to the four variable points 
of the variable .
For each variable ,
and for each pair of indices ,
let  denote the line through the two points  and .
We say that a set  of lines is \emph{canonical} if for each variable ,
 and
,
and moreover,
if , then  is
either 
or .
The following lemma is used by both reductions:

\begin{lemma}\label{lem:canonical}
Any set  of  lines that cover  points in 
can be transformed into a canonical set of at most  lines that
cover at least  (possibly different) points in . 
\end{lemma}
\begin{proof}
Consider an arbitrary function  that maps each point 
to a line  incident to ;
if  is not incident to any line in ,
then  is unmapped, \ie,  is undefined.
For each variable , let  denote the subset of
lines to which  maps the four variable points .
Clearly, .
Note that  but in general  is not necessarily
the same as , because a variable point of  may be incident to
multiple lines in  and is mapped by  to at most one of them.
In the following, we will transform  and update  accordingly,
until  for all variables 
and  is canonical.
Initially, every covered point is mapped to some line.
During the transformation, we maintain the invariant that
every mapped point is covered by some line
(but not necessarily every covered point is mapped to some line)
and the number of mapped points is non-decreasing.

Categorize each line  in one of three types according to :
if there are two variable points of the same variable mapped to ,
then  is of type ;
if there are no variable points mapped to ,
then  is of type ;
otherwise,  is of type .
Note that each type- line either has only one variable point mapped to it,
or has two variable points of different variables mapped to it.

In the first step,
we transform  until  for each variable .
If  for some variable ,
then  must include at least two lines of type .
Note that each type- line in 
has exactly one variable point of  and at most one other point
(either some clause point or a variable point of some other variable)
mapped to it.
As long as  includes two type- lines,
we replace them by a line of type 
(through the two variable points of 
previously mapped to the two type- lines)
and at most one other line of type  or 
(through the at most two other points, if any,
previously mapped to the two type- lines),
and then update the function  accordingly
(so that the points previously mapped to the two type- lines
are mapped to the lines that replace them).
This replacement reduces  by 
(because the other line, if any, does not have any variable points of 
mapped to it), and does not increase  for any~.
Repeat such replacement whenever applicable.
Eventually we have  for every variable .

In the second step, 
we transform  until no lines of type  are incident to variable
points. Consider any line  of type .
If  is incident to two clause points,
then by construction it is not incident to any variable point.
Otherwise  is incident to at most one clause point,
and hence can be rotated, if necessary, to avoid all variable points.
Note that the function  and the subsets 
are not changed during this step.

In the third step,
we transform  until it contains no lines of type .
Consider any line  of type  in ,
with a variable point of , say , mapped to it.
Since ,
there is at most one other line besides  in ,
with at most two other variable points of  mapped to it.
It follows that at least one of the four variable points of ,
say , is not mapped to any line in .
Replace  by the line , and update the function  accordingly
(first unmap the at most two points previously mapped to ,
including ,
then map both  and  to the line  of type ).

In the four step,
we transform  by considering two cases for each line of type :
\begin{enumerate} \itemsep -1pt

\item
A line  of type  is in  with .
Then  is the only line with variable points of  mapped to it,
and
.
If ,
we replace  by any line in
.

\item
Two lines  and  of type  are in  with .
Then  must be
either ,
or ,
or .
If the two lines are ,
we replace them by
either 
or , arbitrarily.
\end{enumerate}
After each replacement,
we update  accordingly.
This completes the transformation of  into canonical form.
\end{proof}

For the reduction to {\sc Maximum Point Coverage by Lines},
we have the following lemma about the construction:

\begin{lemma}\label{lem:max-iff}
There exists an assignment of the variables in 
that satisfies at least  clauses in 
if and only if
there exists a set of  lines
that cover at least  points in .
\end{lemma}

\begin{proof}
We first prove the direct implication.
Let 
be an assignment that satisfies at least  clauses in .
For each variable , ,
select two lines:
if ,
select the line through  and 
and the line through  and ;
if ,
select the line through  and 
and the line through  and .
By construction,
these  lines cover not only all  variable points,
but also the at least  clause points for the satisfied clauses.

We next prove the reverse implication.
Let  be a set of  lines that cover at least  points in .
We will construct an assignment of the variables in 
that satisfies at least  clauses in .
By Lemma~\ref{lem:canonical}, we can assume that  is canonical.
Consider any line .
If  is not incident to any variable point,
then it is incident to at most two clause points,
and can be replaced by a line through two variable points of some variable
while keeping  in canonical form.
Repeat such replacement whenever applicable.
Eventually  includes exactly  lines
incident to all  variable points
and at least  clause points.
Compose an assignment

by setting  to true if 
and to false if .
Then by construction  satisfies at least  clauses.
\end{proof}

For the reduction to {\sc Covering Points by Lines},
we have the following two lemmas analogous to Lemma~\ref{lem:max-iff}
about the construction:

\begin{lemma}\label{lem:min-direct}
If there exists an assignment of the variables in 
that satisfies at least  clauses in ,
then there exists a set of at most  lines
that cover all points in .
\end{lemma}

\begin{proof}
Let 
be an assignment that satisfies at least  clauses in .
For each variable , ,
select two lines:
if ,
select the line through  and 
and the line through  and ;
if ,
select the line through  and 
and the line through  and .
By construction,
these  lines cover not only all  variable points,
but also the at least  clause points for the satisfied clauses.
To cover the remaining at most  clause points
for the unsatisfied clauses,
we pair them up arbitrarily and use at most 
additional lines.
\end{proof}

\begin{lemma}\label{lem:min-reverse}
If there exists a set of at most  lines
that cover all points in ,
then there exists an assignment of the variables in 
that satisfies at least  clauses in .
\end{lemma}

\begin{proof}
Let  be a set of at most
 lines that cover all points in .
We will construct an assignment of the variables in 
that satisfies at least  clauses in .
By Lemma~\ref{lem:canonical}, we can assume that  is canonical.
Since all points are covered, this requires that  for each
, thus  includes exactly  lines incident to the
variable points. 
These  lines must cover at least  clause points
because the other at most  lines in  can
cover at most  clause points.
Compose an assignment

by setting  to true if 
and to false if .
Then  by construction  satisfies at least  clauses.
\end{proof}

The following lemma implies that {\sc Maximum Point Coverage By Lines} is APX-hard:

\begin{lemma}\label{lem:max}
For any , ,
if {\sc Maximum Point Coverage by Lines} admits a polynomial-time approximation algorithm
with ratio ,
then {\sc E-Occ-Max-E-SAT} admits a polynomial-time approximation algorithm
with ratio .
\end{lemma}

\begin{proof}
Let  be an instance of {\sc E-Occ-Max-E-SAT} with  variables
and  clauses, where .
Consider the following algorithm:
first construct a set  of points from  
(refer back to Figure~\ref{fig:line}) and set
 (the number of lines) to ; 
then run the -approximation algorithm for
{\sc Maximum Point Coverage by Lines}
on the instance  to obtain a set  of lines,
and finally compose an assignment

as in the reverse implication of Lemma~\ref{lem:max-iff}.
The algorithm can clearly be implemented in polynomial time.
It remains to analyze its approximation ratio.

Let  be the maximum number of points in 
that can be covered by any set of  lines.
Clearly, .
Let  be the maximum number of clauses in  that can be satisfied by
any assignment of .
Observe that 
since a random assignment of each variable independently to either true
or false with equal probability  satisfies each disjunctive clause
of two literals with probability .
Recall that .
Thus we have


Let  be the number of points in  covered by the lines in .
Let  be the number of clauses in  satisfied by the assignment .
Lemma~\ref{lem:max-iff} implies that

and by the reverse implication in Lemma~\ref{lem:max-iff} we have 
.
Thus
.
It then follows from~\eqref{eq:x*w*} that

The -approximation algorithm for {\sc Maximum Point Coverage by Lines}
guarantees the relative error bound
.
So we have

and hence , as desired.
\end{proof}

The following lemma, analogous to Lemma~\ref{lem:max},
implies that {\sc Covering Points by Lines} is APX-hard:

\begin{lemma}\label{lem:min}
For any , ,
if {\sc Covering Points by Lines} admits a polynomial-time approximation algorithm
with ratio ,
then {\sc E-Occ-Max-E-SAT} admits a polynomial-time approximation algorithm
with ratio .
\end{lemma}

\begin{proof}
Let  be an instance of {\sc E-Occ-Max-E-SAT} with  variables
and  clauses, where .
Let  be the maximum number of clauses in  that can be satisfied by
any assignment of .
We have 
since a random assignment of each variable independently to either true
or false with equal probability  satisfies each disjunctive clause
of two literals with probability .
Without loss of generality, we assume that ,
since otherwise the instance  would have size 
and would admit a straightforward brute-force algorithm
running in  time,
which is constant time for any fixed .

Under the assumption that ,
we have the following algorithm for {\sc E-Occ-Max-E-SAT}:
first construct a set  of points from 
(refer back to Figure~\ref{fig:line}),
then run the -approximation algorithm for {\sc Covering Points by Lines} on 
to obtain a set  of lines,
and finally compose an assignment

as in the proof of Lemma~\ref{lem:min-reverse}.
The algorithm can clearly be implemented in polynomial time.
It remains to analyze its approximation ratio.

Let  be the minimum cardinality of any set of lines that cover .
It is easy to see that .
Recall that  and .
Thus we have

Lemma~\ref{lem:min-direct} implies that
.
It follows that

and hence


Let  be the number of lines in .
Let  be the number of clauses in  that are satisfied by the
assignment .
Put .
Then

Note that
 is the smallest integer (there are two such integers)
satisfying the equation
.
If , then by (the contrapositive of) Lemma~\ref{lem:min-reverse}
we would have
,
which contradicts~\eqref{eq:k'w}.
So we must have , that is,

From \eqref{eq:w*k*} and \eqref{eq:w'k'}, we have

and hence

where the second inequality follows from~\eqref{eq:k*w*}.
The -approximation algorithm for {\sc Covering Points by Lines}
guarantees the relative error bound
.
Recall our assumption that
.
Consequently we have

and hence , as desired.
\end{proof}

\paragraph{Remark.} For simplicity, we did not attempt obtaining
the best multiplicative constant factors of  in the previous two
lemmas. Those expressions can be improved. 


\section{APX-hardness of {\sc Minimum-Link Covering Tour}}

In this section we prove Theorem~\ref{thm:covering}.
We show that {\sc Minimum-Link Covering Tour} is APX-hard
by a gap-preserving reduction from {\sc Covering Points by Lines}\footnote{Arkin~\etal~\cite{AMP03} proved the NP-hardness of
{\sc Minimum-Link Covering Tour} by a reduction from the same problem
{\sc Covering Points by Lines}, 
but since their reduction is not gap-preserving,
their proof does not immediately imply the APX-hardness of
{\sc Minimum-Link Covering Tour} even if {\sc Covering Points by
  Lines} was known to be APX-hard. 
It is quite likely, however, that their construction can be combined with
our construction in the proof of Theorem~\ref{thm:min}
to obtain a gap-preserving reduction
directly from {\sc E-Occ-Max-E-SAT} to {\sc Minimum-Link Covering Tour}.},
which was proved to be APX-hard in Theorem~\ref{thm:min}.

Let  be a set of  points for the problem {\sc Covering Points by Lines}.
We will construct a set  of  points for the problem
{\sc Minimum-Link Covering Tour},
such that  can be covered by  lines if and only if
 admits a covering tour with  segments.

\begin{figure}[htb]
\centering\includegraphics[scale=0.9]{triangle.eps}
\caption{The construction of  from .}
\label{fig:triangle}
\end{figure}

Refer to Figure~\ref{fig:triangle}.
By an affine transformation, we first transform 
into a set  of  points such that
(i)  is enclosed in a circle of some small radius , say, ;
(ii) the angle between any two lines  and ,
each incident to at least two points in ,
is at most some small angle , say, .
Now take an equilateral triangle  of side length  inscribed
in an equilateral triangle  of side length ,
where the three vertices of the smaller triangle are the midpoints of
the three edges of the larger triangle.
The point set  is the union of three rotated copies of  that we refer
to as the three clusters,
one cluster near each vertex of ,
such that the circle of radius  enclosing each cluster is centered
at the vertex,
and all lines through at least two points in the cluster are at angles at most
 from the edge of  that contains the vertex.

\begin{lemma}
There exists a set of  lines that cover all points in 
if and only if
there exists a covering tour with  segments for .
\end{lemma}
\begin{proof}
We first prove the direct implication.
Let  be a set of  lines that cover all points in .
Then by the affine transformation,
we have a set  of  lines that cover all points in ,
and the  lines in the three copies of 
corresponding to the three copies of  cover all points in .
These  lines can obviously be linked into a covering tour
with  segments,
where any three consecutive segments are from three different clusters,
and the turns between consecutive segments are near the vertices of .

We next prove the reverse implication.
Let  be a covering tour with  segments for ,
and let  be the set of at most  lines
supporting the  segments in .
A line is an \emph{intra-cluster} line if the points in  that are covered
by it, if any, are all from the same cluster;
it is an \emph{inter-cluster} line otherwise.
By construction,
each inter-cluster line covers exactly two points, from two different clusters.
Let  (respectively, ) be the number of
intra-cluster (respectively, inter-cluster) lines in ;
then .
Let  be the numbers of points in clusters near ,
respectively, that are covered by the inter-cluster lines;
then .
Since any two points in the same cluster can be covered by some intra-cluster
line, these  points can be covered by at most

intra-cluster lines (instead of  inter-cluster lines).
Since the sum of the three numbers  is even,
we have either two of them odd and one even, or all three of them even.
Thus we have
.
It follows that  can be covered by at most 
intra-cluster lines, and hence at least one of the three copies of 
can be covered by at most  lines.
The corresponding  lines obtained by reversing the affine transformation
cover all points in .
\end{proof}

From the above lemma, we can easily prove the following lemma similar to
Lemmas~\ref{lem:max} and~\ref{lem:min},
which implies the APX-hardness of {\sc Minimum-Link Covering Tour}:

\begin{lemma}\label{lem:covering}
For any ,
if {\sc Minimum-Link Covering Tour} admits a polynomial-time approximation algorithm
with ratio ,
then {\sc Covering Points by Lines} admits a polynomial-time approximation algorithm
with ratio .
\end{lemma}


\section{NP-hardness of {\sc Minimum-Link Spanning Tour}}

In this section we prove Theorem~\ref{thm:spanning}.
We show that {\sc Minimum-Link Spanning Tour} is NP-hard by a reduction from
a variant of the NP-hard problem
{\sc Hamiltonian Circuit in Cubic Graphs}~\cite{GJS74},
in which the input consists of not only a cubic graph 
but also some edge  of  that is required to be part of the Hamiltonian
circuit.
A simple Turing reduction shows that this variant is still NP-hard.

Let  be a cubic graph with  vertices and  edges, where .
Let  be the edge of  that is required to be part of the circuit.
We first obtain a graph  from  by removing the edge 
then adding two dummy vertices  and  with two new edges
 and .
Then there exists a Hamiltonian circuit in  containing the edge 
if and only if
there exists a Hamiltonian path in  from  to .
Observe that  has exactly  vertices and exactly  edges,
where ,
and moreover every vertex except  and  has degree .

We next construct a set  of  points,
one vertex point for each vertex,
and one edge point for each edge in .
The  vertex points are in some arbitrary convex position,
say, on a circle.
The  edge points are in the interior of the convex hull
of the  vertex points.
Moreover, for each edge ,
the edge point of  is in the line through the two vertex points
of  and  (\ie, in the interior of the segment ),
and is not in any other line containing more than two points in .

The reduction can clearly be implemented in polynomial time.
Then the following lemma establishes the NP-hardness
of {\sc Minimum-Link Spanning Tour}:

\begin{lemma}
There exists a Hamiltonian path in  from  to 
if and only if
there exists a spanning tour with  segments for .
\end{lemma}

\begin{proof}
We first prove the direct implication.
Let  be a Hamiltonian path in  from  to .
We will construct a spanning tour with  segments for .
Corresponding to the Hamiltonian path 
that visits all  vertices in  using  edges,
there is a polygonal chain of  segments
that connect the  vertex points in  in the same order,
which also cover  edge points.
The chain can be extended to visit the remaining  edge points
in any order with  segments,
and finally closed into a tour with another segment.
The total number of segments is .

We next prove the reverse implication.
Let  be a spanning tour with  segments for .
We will find a Hamiltonian path in  from  to .
We first transform ,
without increasing the number of segments,
into a canonical spanning tour that visits each vertex point exactly once.

\begin{figure}[htb]
\centering\includegraphics[scale=0.75]{shortcut.eps}
\caption{Transforming  into a canonical spanning tour that visits each
vertex point exactly once.}
\label{fig:shortcut}
\end{figure}

Suppose that a vertex point  is visited twice in .
Then there are two pairs of consecutive segments with turns at .
Refer to Figure~\ref{fig:shortcut}.
If, out of these four segments incident to ,
there is a segment connecting some point  directly to 
with no other point of  in the interior of the segment,
then we can take a shortcut (as in Figure~\ref{fig:shortcut} left)
in the pair of segments including  to skip a visit to .
Otherwise, each of the four segments must connect  to the vertex
point of a neighbor of the vertex of  in ,
going through the corresponding edge point.
Recall that every vertex in  has degree at most .
By the pigeonhole principle,
at least two of these four segments must be the same segment, say, .
If the two copies of  are consecutive
and form a turn at  in ,
then we can shorten both of them to skip a visit to .
Otherwise,
one copy of  must form a turn at 
with some other segment, say ,
and again we can take a shortcut (as in Figure~\ref{fig:shortcut} right)
to skip a visit of .
Observe that in both cases,
 remains a spanning tour after the transformation.

Now observe that every edge point is
either in the interior of some segment between two vertex points,
or at a turning point between two consecutive segments.
Associate a cost of  with each edge point.
For each edge point, charge its cost to the segments that contain it:
if it is in the interior of one segment, charge  to the segment;
if it is at a turning point between two consecutive segments,
charge  to each segment.
Observe that
\begin{itemize} \itemsep -1pt
\item
each segment between two vertex points is charged  if the segment
contains an edge point, and is charged  otherwise;
\item
each segment between two edge points is charged ;
\item
each segment between a vertex point and an edge point
(there must be at least two such segments along the tour
since there are more edge points than vertex points)
is charged exactly .
\end{itemize}
Since the number of edge points in  is 
and the number of segments in  is ,
we must have exactly  segments charged  each,
and exactly  segments charged  each,
so that .
It follows that (i)
there are exactly two segments between vertex points and edge points,
and (ii)
there is no segment connecting two vertex points and containing no edge point.
Condition (i) implies that the segments between vertex points
are consecutive in the tour.
Condition (ii) implies that these consecutive segments correspond to
a Hamiltonian path in .
Finally, this Hamiltonian path must have  and  as the two ends
because each of them has exactly one neighbor.
\end{proof}


\section{APX-hardness of {\sc Min-Max-Turn Hamiltonian Tour}
	and NP-hardness of {\sc Bounded-Turn-Minimum-Length Hamiltonian Tour}}

In this section we prove
Theorems~\ref{thm:hamiltonian} and~\ref{thm:hamiltonian2}.
We first show that {\sc Min-Max-Turn Hamiltonian Tour} is APX-hard
by a gap-preserving reduction from {\sc Covering Points by Lines},
which was proved to be APX-hard in Theorem~\ref{thm:min}.
Let  be a set of  points for the problem {\sc Covering Points by Lines}.
We will construct a set  of points for the problem
{\sc Min-Max-Turn Hamiltonian Tour}.
\begin{figure}[htb]
\centering\includegraphics[scale=0.6]{hamiltonian.eps}
\caption{The construction of the point set  for {\sc Min-Max-Turn Hamiltonian
Tour} ( in this example).}
\label{fig:hamiltonian}
\end{figure}

Let  be the set of at most  lines determined by ,
where each line in  goes through at least two points in 
(we assume without loss of generality that no line in  is vertical).
Let  be the minimum turning angle determined
by any three non-collinear points in ;
we will show later in Lemma~\ref{lem:lattice}
that  may be assumed to be polynomial in .
Let  for some suitable constant .

Refer to Figure~\ref{fig:hamiltonian}.
We first construct a simple closed curve  that is composed by
a half-circle and a polygonal chain
joined at their endpoints  and .
The polygonal chain consists of

segments, including  segments from each line in ,
 vertical segments, and  horizontal segment .
Observe that  is \emph{monotone} in the horizontal direction,
in the sense that the intersection of every vertical line
with the region enclosed by  is a single line segment.

\begin{figure}[htb]
\centering\includegraphics[scale=0.8]{corner.eps}
\caption{Smoothing the corner between two consecutive segments
into a circular arc tangent to both segments
( and  in this example).}
\label{fig:corner}
\end{figure}

Refer to Figure~\ref{fig:corner}.
We next transform  into a simple closed curve 
that is not only monotone but also \emph{smooth},
by smoothing the corner between every pair of consecutive segments
in the chain into a small circular arc tangent to both segments.
Then  is an alternating cycle of  segments
(including the horizontal segment )
and  circular arcs
(including the half-circle with diameter ).

Put .
The point set  consists of all  points in 
and  points from the curve .
Take  points (including the two endpoints)
from each of the  circular arcs,
which divide any such arc into  sub-arcs with the same central angle
at most .
Take  more points from each of the two segments adjacent
to the circular arc, near the shared endpoints, such that
the following \emph{-property} is satisfied:
the angle ,
where  is an endpoint of the arc,
 is any of the other  points in the arc,
and  is any of the  points in the segment containing ,
is at most .
Observe that the total number of points is . 

We have the following two lemmas about the construction:

\begin{lemma}\label{lem:hamiltonian-direct}
If there exists a set of  lines that cover all points in ,
for some ,
then there exists a Hamiltonian tour with maximum turning angle at most
 for .
\end{lemma}

\begin{proof}
Let 
be a set of  lines that cover all points in ;
without loss of generality, .
We will construct a Hamiltonian tour with maximum turning angle at most
 for .
Index the  points in  taken from the curve 
by the circular order of their locations along the curve:
.
Assign each point a color in :
 has color , .

The tour consists of  rounds.
In the th round, ,
the tour follows the half-circle from  to  and continues along the
chain from  to  until it reaches the line ,
then takes a shortcut along  from right to left
and continues along the chain from  to 
until it reaches the half-circle again;
the tour visits each point of color 
in or below the line  while following the curve,
and visits each point in  that is covered by the line 
(if the point was not visited in previous rounds)
while taking the shortcut.
In the last round, the tour follows the curve entirely
to visit each point of color ,
and points of other colors not visited in previous rounds due to the shortcuts.

Consider any three consecutive points  in the tour.
If the three points are all in some line  supporting a shortcut,
then obviously .
Otherwise, the three points must come from some sub-curve of 
consisting of a circular arc and an adjacent segment.
The number of sub-arcs of this arc that are between  and 
is at most ;
each such sub-arc contributes half of its central angle
to the turning angle at .
Taking into account the possibility that
the three points are not all in the arc
and using the -property,
we have
.
\end{proof}

\begin{lemma}\label{lem:hamiltonian-reverse}
If there exists a Hamiltonian tour with maximum turning angle at most
 for , for some ,
then there exists a set of  lines that cover all points in .
\end{lemma}
\begin{proof}
Let  be a Hamiltonian tour with maximum turning angle at most
 for .
We will find a set of  lines that cover all points in .

Break the tour  into rounds,
such that each round consists of
some points in the half-circle followed by
some points not in the half-circle (\ie, in the chain or in ).
When the angle  is sufficiently small
(hence  is even smaller),
in each round the tour
must visit some points in the half-circle in order of their -coordinates,
say, from left to right,
then visit points near some corners of the chain, from right to left.
While in the chain,
the tour may take shortcuts between non-consecutive corners,
but since the curve  is monotone,
it can take at most one \emph{crossing} shortcut from a corner on the  side
to a corner on the  side.
Only when taking such a crossing shortcut can the tour visit some points in .
Moreover,
since ,
the points in  that are visited during each crossing shortcut
must be collinear.

We next show that  has at most  rounds.
Index the  points in the half-circle from left to right
by numbers from  to ,
where  has index  and  has index .
Consider an arbitrary round.
Let  be the number of points in the half-circle
that are visited in this round.
Let  be the indices of these points, from left to right.
Then we must have  because otherwise
the turning angle at the point with index  would be greater than
.
Similarly, we must have ,
and  for each , .
Counting in pairs, we have ,
where .
It follows that the number of rounds in  is at most

It is easy to check that 
and hence .
Thus  has at most  rounds.
Finally,
since each round has at most one crossing shortcut
that can cover some collinear points in ,
all points in  can be covered by  lines.
\end{proof}

Unlike the reductions in previous sections,
our reduction to {\sc Min-Max-Turn Hamiltonian Tour} is numerically sensitive
because the construction depends on a small angle 
and has points placed precisely on circular arcs.
Even if we reduce from a restricted version of
{\sc Covering Points by Lines},
where the coordinates are polynomial in the number of points
(it can be checked that our proof for the APX-hardness of
{\sc Covering Points by Lines} fulfills this restriction),
it is still not immediately clear that the reduction is polynomial.
To clarify this,
we prove a property concerning lattice points in the next lemma,
which implies that  may be assumed to be polynomial in
the lattice size :

\begin{lemma}\label{lem:lattice}
Let , , and  be three non-collinear points in the 
section of the integer lattice, where . Then the turning
angle of the path  at point  is at least .
\end{lemma}
\begin{proof}
Let  denote the turning angle. We can assume that , since otherwise the inequality holds.
Since  are non-collinear lattice points,
the tangent function of the turning angle can be expressed as:

where
,
and  are nonnegative integers less or equal to .
Write  and .
We distinguish two cases depending on whether the product  is
smaller or larger than :

{\em Case 1}: . We have


{\em Case 2}: . We have


Since  was assumed,
in both cases it follows that , as required.
\end{proof}

Having polynomial representation of the small angles and rational points
on circular arcs is a non-trivial problem~\cite{CDR92,Bu98}.
Without delving too much into technical details such as Lemma~\ref{lem:lattice},
we claim that for any , ,
the construction can use integers polynomial in 
for the coordinates of all points in ,
such that the angle determined by any three points in  deviates by
a multiplicative factor at least  and at most .
Consequently,
the reduction is strongly polynomial,
and we have the following approximate versions of
Lemmas~\ref{lem:hamiltonian-direct} and~\ref{lem:hamiltonian-reverse}:
\begin{description}

\item[Lemma~\ref{lem:hamiltonian-direct} (Approximate Version).]
If there exists a set of  lines that cover all points in ,
for some ,
then there exists a Hamiltonian tour with maximum turning angle at most
 for .

\item[Lemma~\ref{lem:hamiltonian-reverse} (Approximate Version).]
If there exists a Hamiltonian tour with maximum turning angle at most
 for , for some ,
then there exists a set of  lines that cover all points in .

\end{description}

The following lemma shows that {\sc Min-Max-Turn Hamiltonian Tour}
is almost as hard to approximate as {\sc Covering Points by Lines}:

\begin{lemma}\label{lem:hamiltonian}
For any ,
if {\sc Min-Max-Turn Hamiltonian Tour}
admits a polynomial-time approximation algorithm with ratio ,
then {\sc Covering Points by Lines} admits a polynomial-time
approximation algorithm with ratio  for any .
\end{lemma}

\begin{proof}
Let  be a set of  points for {\sc Covering Points by Lines}.
Let  be the minimum number of lines
necessary for covering all points in .
Without loss of generality, we assume that , since otherwise
a brute-force algorithm can find  lines to cover 
in  time, which is polynomial time for any fixed .

Let  such that
.
Under the assumption that , we have the following algorithm:
first construct a point-set  from  (refer to Figure~\ref{fig:hamiltonian}),
and then run the -approximation algorithm for {\sc Min-Max-Turn Hamiltonian Tour}
on  to obtain a tour of maximum turning angle at most
 for some ,
and finally obtain a set  of at most  lines 
that cover  by Lemma~\ref{lem:hamiltonian-reverse} (approximate version).

By Lemma~\ref{lem:hamiltonian-direct} (approximate version),
the minimum value of the maximum turning angle of a Hamiltonian tour
for  is at most 
(\ie, there exists a Hamiltonian tour with maximum turning angle
bounded as such). 
The -approximation algorithm for {\sc Min-Max-Turn Hamiltonian Tour}
guarantees that

and hence
.
By the assumption that ,
we have 
.
\end{proof}

Since {\sc Covering Points by Lines} is APX-hard (Theorem~\ref{thm:min}),
the above lemma implies that {\sc Min-Max-Turn Hamiltonian Tour} is APX-hard too.
This completes the proof of Theorem~\ref{thm:hamiltonian}.

\medskip
Consider the decision versions of the two Hamiltonian Tour problems:
\begin{itemize} \itemsep -1pt
\item[(I)]
{\sc Min-Max-Turn Hamiltonian Tour} (Decision Problem):
Given  points in the plane and an angle ,
decide whether there exists a Hamiltonian tour with maximum turning angle
at most .
\item[(II)]
{\sc Bounded-Turn-Minimum-Length Hamiltonian Tour} (Decision Problem):
Given  points in the plane, an angle ,
and a positive number ,
decide whether there exists a Hamiltonian tour with maximum turning angle
at most  and length at most .
\end{itemize}
Observe that the decision problem of 
{\sc Min-Max-Turn Hamiltonian Tour}
is a special case of the decision problem of
{\sc Bounded-Turn-Minimum-Length Hamiltonian Tour},
with the parameter  set to some sufficiently large number,
say,  times the diameter of the point set.
Thus the APX-hardness (indeed NP-hardness suffices)
of {\sc Min-Max-Turn Hamiltonian Tour}
implies that {\sc Bounded-Turn-Minimum-Length Hamiltonian Tour} is NP-hard.
This completes the proof of Theorem~\ref{thm:hamiltonian2}.

\medskip
It is interesting to note that,
while the decision problem of {\sc Bounded-Turn-Minimum-Length Hamiltonian Tour}
has both an \emph{angle} constraint and a \emph{length} constraint,
our proof of its NP-hardness above
(via the reduction from the decision problem of {\sc Min-Max-Turn Hamiltonian Tour})
effectively only uses the angle constraint.
That is, the problem is already hard with the angle constraint alone.
On the other hand,
if the turning angle is unrestricted, \ie, if ,
then the problem {\sc Bounded-Turn-Minimum-Length Hamiltonian Tour}
is the same as the {\sc Euclidean Traveling Salesman Problem},
which is well known to be NP-hard with the length constraint alone.
Our proof
sheds light on a different aspect of the difficulty of the problem.


\section{Concluding remarks}

The obvious question left open by our work is whether 
{\sc Covering Points by Lines} admits an approximation algorithm with
constant ratio. 
Two other problems are finding approximation algorithms for 
{\sc Min-Max-Turn Hamiltonian Tour} and respectively, 
{\sc Bounded-Turn-Minimum-Length Hamiltonian Tour}.


\paragraph{Acknowledgment.}
The authors are grateful to an anonymous reviewer for suggesting the extension
to lines---in our Theorem~\ref{thm:greedy}---of the result
from~\cite{BLWM12} (Theorem~1, p.~1041) for line segments,
and to another anonymous reviewer for pertinent advice on polynomial
representation of points on circular arcs~\cite{CDR92,Bu98}.



\begin{thebibliography}{99}

\bibitem {ACKMS99}
A. Aggarwal, D. Coppersmith, S. Khanna, R. Motwani, and B. Schieber.
The angular-metric traveling salesman problem.
\emph{SIAM Journal on Computing},
29:697--711, 1999.
An extended abstract in \emph{Proceedings of the 8th Annual ACM-SIAM
  Symposium on Discrete Algorithms (SODA'97)}, 
pages 221--229, 1997.

\bibitem{ART95}
P. K. Agarwal, P. Raghavan, and H. Tamaki.
Motion planning for a steering-constrained robot through moderate obstacles.
In \emph{Proceedings of the 27th Annual ACM Symposium on Theory of
  Computing (STOC'95)}, 
pages 343--352, 1995.

\bibitem{AMS06}
N. Alon, D. Moshkovitz, and S. Safra.
Algorithmic construction of sets for -restrictions.
\emph{ACM Transactions on Algorithms},
2:153--177, 2006.

\bibitem{ABDFMS05}
E. M. Arkin, M. A. Bender, E. D. Demaine, S. P. Fekete,
J. S. B. Mitchell, and S. Sethia. 
Optimal covering tours with turn costs.
\emph{SIAM Journal on Computing},
35:531--566, 2005.

\bibitem{AMP03}
E. M. Arkin, J. S. B. Mitchell, and C. D. Piatko.
Minimum-link watchman tours.
\emph{Information Processing Letters},
86:203--207, 2003.

\bibitem{BK99}
P. Berman and M. Karpinski.
On some tighter inapproximability results.
\emph{DIMACS Technical Report} 99-23, 1999.

\bibitem{BK03}
P. Berman and M. Karpinski.
Improved approximation lower bounds on small occurrence optimization.
\emph{Electronic Colloquium on Computational Complexity},
TR03-008, 2003.

\bibitem{BCL94}
J. Boissonat, A. C{\'e}r{\'e}zo, and J. Leblond.
Shortest paths of bounded curvature in the plane.
\emph{Journal of Intelligent and Robotic Systems: Theory and Applications},
11:5--20, 1994.

\bibitem{BLWM12}
V. E. Brimkov, A. Leach, J. Wu, and M. Mastroianni.
Approximation algorithms for a geometric set cover problem.
\emph{Discrete Applied Mathematics},
160:1039--1052, 2012.

\bibitem{BHN01}
B. Brod\'en, M. Hammar, and B. J. Nilsson.
Guarding lines and 2-link polygons is APX-hard.
In \emph{Proceedings of the 13th Canadian Conference on Computational
  Geometry (CCCG'01)}, 
pages 45--48, 2001.

\bibitem{Bu98}
C. Burnikel.
Rational points on circles.
Technical Report MPI-I-98-1-023,
Max-Planck-Institut f{\"u}r Informatik,
1998.

\bibitem{CDR92}
J. Canny, B. Donald, E. K. Ressler.
A rational rotation method for robust geometric algorithms.
In
\emph{Proceedings of the 8th Annual Symposium on Computational Geometry (SOCG'92)},
pages 251--260, 1992.

\bibitem{Ch79}
V. Chv\'atal.
A greedy heuristic for the set-covering problem.
\emph{Mathematics of Operations Research},
4:233--235, 1979.

\bibitem{Fe98}
U. Feige.
A threshold of  for approximating set cover.
\emph{Journal of the ACM},
45:634--652, 1998.

\bibitem{FW97}
S. P. Fekete and G. J. Woeginger.
Angle-restricted tours in the plane.
\emph{Computational Geometry: Theory and Applications},
8:195--218, 1997.

\bibitem{Fr89}
T. Frachard.
Smooth trajectory planning for a car in a structured world.
In \emph{Proceedings of the IEEE International Conference on Robotics
  and Automation}, 
pages 318--323, 1989.

\bibitem{GJS74}
M. R. Garey, D. S. Johnson, and L. Stockmeyer.
Some simplified NP-complete problems.
In \emph{Proceedings of the 6th Annual ACM Symposium on Theory of
  Computing (STOC'74)}, 
pages 47--63, 1974.

\bibitem{GL06}
M. Grantson and C. Levcopoulos.
Covering a set of points with a minimum number of lines.
In \emph{Proceedings of the 22nd European Workshop on Computational Geometry},
pages 145--148, 2006.

\bibitem{Ho97}
D. S. Hochbaum.
\emph{Approximation Algorithms for NP-hard Problems},
PWS, 1997.

\bibitem{JC89}
P. Jacobs and J. Canny.
Planning smooth paths for mobile robots.
In \emph{Proceedings of the IEEE International Conference on Robotics and Automation},
pages 2--7, 1989.

\bibitem{Ji12}
M. Jiang.
On covering points with minimum turns.
In \emph{Proceedings of the 6th International Frontiers of Algorithmics
  Workshop and the 8th International Conference on Algorithmic Aspects
  of Information and Management (FAW-AAIM'12)}, 
LNCS 7285, pages 58--69, 2012.

\bibitem{Jo74}
D. S. Johnson.
Approximation algorithms for combinatorial problems.
\emph{Journal of Computer and System Sciences},
9:256--278, 1974.

\bibitem{KAR00}
V. S. A. Kumar, S. Arya, and H. Ramesh.
Hardness of set cover with intersection .
In \emph{Proceedings of the 27th International Colloquium on Automata,
  Languages and Programming (ICALP'00)}, 
pages 624--635, 2000.

\bibitem{LM05}
S. Langerman and P. Morin.
Covering things with things.
\emph{Discrete \& Computational Geometry},
33:717--729, 2005.

\bibitem{LFF07}
J. Le Ny, E. Frazzoli, and E. Feron.
The curvature-constrained traveling salesman problem for high point densities.
In \emph{Proceedings of the 46th IEEE Conference on Decision and Control},
pages 5985--5990, 2007.

\bibitem{Lo75}
L. Lovasz.
On the ratio of optimal integral and fractional covers.
\emph{Discrete Mathematics},
13:383--390, 1975.

\bibitem{LY94}
C. Lund and M. Yannakakis.
On the hardness of approximating minimization problems.
\emph{Journal of the ACM},
41:960--981, 1994.

\bibitem{MT82}
N. Megiddo and A. Tamir.
On the complexity of locating linear facilities in the plane.
\emph{Operation Research Letters},
1:194--197, 1982.

\bibitem{RS97}
R. Raz and S. Safra.
A sub-constant error-probability low-degree test, and a sub-constant
error-probability PCP characterization of NP. In
\emph{Proceedings of the 29th Annual ACM Symposium on Theory of Computing (STOC'97)},
pages 475--484, 1997.

\bibitem{Sl97}
P. Slav\'ik.
A tight analysis of the greedy algorithm for set cover.
\emph{Journal of Algorithms},
25:237--254, 1997.

\bibitem{WLC10}
J. Wang, W. Li, and J. Chen.
A parameterized algorithm for the hyperplane-cover problem.
\emph{Theoretical Computer Science},
411:4005--4009, 2010.

\end{thebibliography}


\end{document}
