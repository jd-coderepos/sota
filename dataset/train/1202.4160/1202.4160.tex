\documentclass[10pt]{article}

\usepackage{a4wide}


\usepackage{amsfonts,epsfig,rotating,amssymb,framed}

\topmargin=-36pt



\newtheorem{theorem}{Theorem}[section]
\newtheorem{example}[theorem]{Example}
\newtheorem{proposition}[theorem]{Proposition}
\newtheorem{corollary}[theorem]{Corollary}
\newtheorem{lemma}[theorem]{Lemma}
\newtheorem{fact}[theorem]{Fact}
\newtheorem{definition}[theorem]{Definition}
\newtheorem{remark}[theorem]{Remark}
\newtheorem{construction}[theorem]{Construction}

\newenvironment{proof}{\noindent{\bf Proof~}}{\null\hfill \par\medskip}
\newenvironment{proofsk}{\noindent{\bf Proof Sketch~}}{\null\hfill
\par\medskip}



\renewcommand{\topfraction}{1}
\renewcommand{\bottomfraction}{1}
\renewcommand{\textfraction}{0}
\setcounter{topnumber}{10}
\setcounter{bottomnumber}{10}
\setcounter{totalnumber}{10}



\newcommand{\score}{\mathit{score}}
\newcommand{\scoreof}[1]{\mathit{score}(#1)}
\newcommand{\scoresub}[2]{\mathit{score}_{#1}(#2)}

\newcommand{\IR}{\mathbb{R}}






\usepackage{mathrsfs}
\usepackage{colonequals}
\usepackage{paralist}
\usepackage{amsmath}
\usepackage{listings}
\usepackage{subfig}
\lstdefinelanguage{ mylng }{
	morekeywords={while,true}
}
\lstset{language=mylng, mathescape=true, breaklines=true,numbers=left,columns=fullflexible}

\newcommand{\q}[1]{``#1''}\newcommand{\fu}[1]{\mathcal{#1}}\newcommand{\mc}[1]{\mathsf{#1}}\newcommand{\ri}[1]{\mathscr{#1}}\newcommand{\co}[1]{\fu{C}({#1})}\newcommand{\lc}[1]{\alpha(#1)}\newcommand{\rc}[1]{\beta(#1)}\newcommand{\ema}[1]{\mathcal{#1}}\newcommand{\fe}[2]{\fu{F}(#1,#2)}\newcommand{\lv}[1]{l_{#1}}\newcommand{\rv}[1]{r_{#1}}\newcommand{\mv}[1]{m_{#1}}\newcommand{\lvv}{\lv{v}}
\newcommand{\rvv}{\rv{v}}
\newcommand{\mvv}{\mv{v}}
\newcommand{\svv}{s_v}
\newcommand{\cordered}{\fu{S}}\newcommand{\lvi}[2]{\lv{}^{#2}(#1)}\newcommand{\rvi}[2]{\rv{}^{#2}(#1)}\newcommand{\lvd}[1]{\fu{L}_v(#1)}\newcommand{\rvd}[1]{\fu{R}_v(#1)}
\newcommand{\dom}{\fu{D}(G)}\newcommand{\dist}[2]{dist(#1,#2)}\newcommand{\first}[1]{\fu{H}(#1)}\newcommand{\last}[1]{\fu{T}(#1)}



\makeatletter
\renewcommand{\p@enumii}{\theenumi.(}
\newcommand{\sref}[1]{\ref{#1})}
\newcommand{\ssref}[1]{(\ref{#1})}
\makeatother





\setcounter{secnumdepth}{4} 







\begin{document}

\title{Interval Routing Schemes for Circular-Arc Graphs}

\author{
Frank Gurski \\
University of D\"usseldorf\\
Institute of Computer Science\\
Algorithmics for Hard Problems Group\\
D-40225 D\"usseldorf\\
\texttt{frank.gurski@hhu.de}
\and 
Patrick Gwydion Poullie\thanks{Corresponding author.} \\
University of Zurich \\
Department of Informatics \\ Communication Systems Group\\
CH-8050 Z\"urich\\
\texttt{poullie@ifi.uzh.ch}
}


\maketitle





\begin{abstract}
Interval routing is a space efficient method to realize a distributed 
routing function.
In this paper we show that every circular-arc graph allows a shortest 
path strict 2-interval routing scheme, i.e., by introducing a global order
on the vertices and assigning at most two (strict) intervals in this order
to the ends of every edge allows to depict a routing function that 
implies exclusively shortest paths. Since circular-arc graphs do not allow shortest path 1-interval routing 
schemes in general, the result implies that the class of circular-arc 
graphs has strict compactness 2, which was a hitherto open question.
Additionally, we show that the constructed 2-interval routing scheme is 
a 1-interval routing scheme with at most one additional 
interval assigned at each vertex and we outline an algorithm 
to calculate the routing scheme for circular-arc graphs in 
 time, where  is the number of vertices.

\bigskip
\noindent
{\bf Keywords:} interval routing, compact routing, circular-arc graphs, cyclic permutations
\end{abstract}










\section{Introduction}











Routing is an essential task that a network of processors or computers must be able to 
perform. Interval routing is a space-efficient solution to this problem.
Sets of consecutive destination addresses that use the same output port are grouped into 
intervals and then assigned to this port.
In this way, the storage space required is greatly reduced compared to the straight 
forward approach, in which an output port is stored specifically for every destination address.
Of course, the advantage in space efficiency depends heavily on the number of intervals assigned 
to the output ports, which, in turn, depends on the address and network topology.
Interval routing was first introduced in \cite{Santoro82routingwithout,Santoro85labelling}; 
for a discussion of interval routing we refer the reader 
to \cite{Leeuwen87interval,FredericksonJ88,Bakker91linearinterval} 
and for a detailed survey to \cite{Gavoille00asurvey}.
When theoretical aspects of interval routing are discussed, the network is represented 
by a directed (symmetric) graph . 
As with many routing methods, the problem is that space efficiency and path optimality are 
conflicting goals \cite{65953}: as shown in \cite{Guevremont98worstcase}, for every , 
there exists an -vertex graph  such that for every shortest 
path interval routing scheme (for short IRS) for  the maximal number of intervals per directed edge 
is only bounded by . Also, it is NP-hard to determine the most space efficient 
shortest path IRS for a given graph \cite{Eilam2002,Flammini1997}.
Other worst case results can be found in \cite{savio99onthespace}.
On the other hand, there are many special graph classes \cite{FG98} including 
random graphs \cite{GP1998} that are known to allow shortest path IRSs 
with a constantly bounded number of intervals on all directed edges. 
If this number is a tight bound, it is denoted the compactness of the graph.
The compactness of undirected graphs is defined as the compactness of their directed 
symmetric version\footnote{The \emph{directed symmetric version of an undirected graph} 
 is the directed graph  with }
and the compactness of a graph class as the smallest , such that every graph in this 
class has compactness  at most , if such  exists.


In this paper we show that the class of circular-arc graphs has (strict) compactness 2 
by presenting an algorithm to construct a corresponding IRS. A different approach to realize 
space efficient, shortest path routing in circular-arc graphs can be found 
in \cite{f.dragan:new}. In \cite{GP08} an -bit distance
labeling scheme is developed that allows for each pair of vertices in circular-arc graphs to
compute their exact distance in  time.
In \cite{DYL06} an -bit routing labeling scheme is developed that allows 
to make a routing decision in   time for every vertex in an arbitrary circular-arc graph. 
The resulting routing path has at most two more edges than the respective shortest path.
Nevertheless, our result is interesting, since interval graphs and 
unit circular-arc graphs  are included in the class of circular-arc graphs and known to have 
compactness 1 \cite{FG98,NS96}, while for the compactness of circular-arc graphs hitherto 
only a lower bound of 2 was known (cf. the graph in Fig. \ref{fig:animals}) 
and the solutions from  \cite{FG98} and \cite{NS96} 
cannot be extended to the class of circular-arc graphs.
   
While interval graphs can be represented by intervals on a line, circular-arc graphs can be 
represented by arcs on a circle, that is to say, every circular-arc graph is the intersection 
graph of a set of arcs on a circle.
The first polynomial-time () algorithm 
to recognize circular-arc graphs and give a corresponding  circular-arc model can be found 
in \cite{firstCAG}. An algorithm with linear runtime for the same purpose is given 
in \cite{MCC03}. In circular-arc graphs maximum cliques can 
be computed much faster than in arbitrary graphs; an algorithm that determines a maximum 
clique in a circular-arc graph in  time or even in  time, 
if the circular-arc endpoints are given in sorted order, is presented 
in \cite{Bhattacharya1997336}. A discussion of circular-arc graphs can be found 
in \cite{Hsu95}. Circular-arc graphs have, amongst others, applications 
in cyclic scheduling, compiler design \cite{Gol04,tucker:493}, 
and genetics \cite{Rob76}.

This paper is organized as follows.
In Section \ref{chap:pre} we define cyclic permutations, circular-arc graphs, 
and  interval routing schemes.  
In Section \ref{main} we prove 
the main result of this paper, namely that every circular-arc graph allows a shortest path 
strict 2-interval routing scheme. 
To this end, we choose an arbitrary circular-arc graph  and 
fix a cyclic permutation   on , in Section \ref{constrCO}. 
In Section \ref{furtherDefs} we fix an arbitrary vertex  and partition  into the three sets , , and ; the vertices in these sets appear consecutively in  and are referred to as  the \emph{vertices to the right of }, the \emph{vertices face-to-face with }, and the \emph{vertices to the left of }, respectively.
In Section \ref{2-app} we show that the vertices in the 
three sets  , , and  can be assigned to the edges incident to , such that 
the conditions of a shortest path 
strict 2-interval routing scheme are fulfilled. 
In Section \ref{improving} we show that the upper bound of the number of intervals in a shortest path strict 2-interval routing scheme constructed in this way
is close to the number in an 1-IRS.
In Section \ref{implement} we outline  how to 
implement the implied IRS in  time.






\section{Preliminaries\label{chap:pre}}




\subsection{Cyclic Permutations}\label{cyclicOrders}



Cyclic permutations are relevant for the two main subjects of this paper, 
namely circular-arc graphs and non-linear interval routing schemes. 
Just like a permutation, a cyclic permutation implies an order on a set 
with the only difference being that there is no distinct first, second, ,
or -th element. We also define cyclic permutations on multisets, where
we assume that multiple appearances of an element are
distinguishable, i.e., 
a multiset  on  elements is identified with 
.







\begin{definition}[Cyclic Permutation]
A \emph{cyclic permutation} on a (multi-)set  with  elements is a function

such that  
.\footnote{If 
the condition holds for one element in , it holds for all elements in .}
We call  the \emph{successor of  (in )} and we write
.
\end{definition}

Every cyclic permutation is a bijective mapping.
Let  be a cyclic permutation on a (multi-)set .
One could imagine  as an arrangement of  on a clock face, for an -hour clock.
\emph{To go through  beginning at } means to consider the elements in 
 as they appear in , beginning at . Next we define subsets of  that appear 
consecutively in .


\begin{definition}[Ring-Interval, Ring-Sequence]\label{interval}
For a cyclic permutation  on a (multi-)set  and , we recursively define 
the \emph{ring-interval from  to  (in )} as

By  we indicate that the left endpoint is excluded, by   that the right endpoint is
excluded, and by  that both endpoints  are excluded.
Therefore we have 

The \emph{ring-sequence from  to  (in )}, denoted by 
, is the order in which elements of  appear when going through  beginning at .






\end{definition}



\begin{construction}\label{ringkombi}
Let  be a cyclic permutation on a (multi-)set  and .
If  and , we 
have .
\end{construction}


\subsection{Circular-Arc Graphs}\label{sub:CAG}


We assume that the reader is familiar with basic graph theoretical definitions.
The \emph{intersection graph} of a family of sets is the graph where the vertices are the sets, 
and the edges are the pairs of sets that intersect. Every graph is the intersection graph of 
some family of sets. A graph is an \emph{interval graph} if it is the intersection graph of a 
finite set of intervals (line segments) on a line and a \emph{unit interval graph} if these 
intervals have unit length.
A graph  is a \emph{circular-arc graph} if it is the intersection graph of a finite set of 
arcs on a circle; the latter we call an \emph{arc model of }.
Since an interval graph is a special case of a circular-arc graph, namely a circular-arc graph 
that can be represented with a set of arcs that do not cover the entire circle, we define the 
set of \emph{strict circular-arc graphs} as the set of circular-arc graphs that are not interval 
graphs. Every strict circular-arc graph is connected.
A \emph{unit circular-arc graph} is a circular-arc graph that has an arc model in which the 
arcs have unit length. For a survey on circular-arc graphs see \cite{Lin20095618} 
and for the definition of further special graph classes see \cite{BLS99}.
Fig. \ref{fig:animals} illustrates a strict circular-arc graph.


\begin{figure}[ht]
  \centering
  \subfloat[Arc model]{\label{fig:gull}\includegraphics[width=0.33\textwidth,trim=3cm 4cm 3cm 4cm]{gegenbeispielOhneMCs.pdf}}                
  \subfloat[Correspondence]{\includegraphics[width=0.33\textwidth,trim=3cm 4cm 3cm 4cm]{gegenbeispielEdges2.pdf}}
  \subfloat[Graph representation]{\label{fig:mouse}\includegraphics[width=0.33\textwidth,trim=3cm 4cm 3cm 4cm]{gegenbeispielEdgesOhneArcs2.pdf}}
   \vspace{-5pt}\caption{A (strict) circular-arc graph}
  \label{fig:animals}
\end{figure}


A difference between interval graphs and circular-arc graphs, that is worth mentioning, 
is that the maximal cliques of interval graphs can be associated to points of 
the \q{interval model} and therefore an interval graph can have no more maximal cliques 
than vertices. In contrast, circular-arc graphs may contain maximal cliques that do not 
correspond to points of  some arc model \cite{Lin20095618}.
In fact, just as in arbitrary graphs, the number of maximal cliques in circular-arc 
graphs can grow exponentially in the size of the graph \cite{firstCAG}. For
the special case that there is an arc model where no three arcs cover the whole
circle the number of maximal cliques is bounded by the
number of vertices of the graph \cite{CFZ08}. Also maximal cliques may
occur several times within some arc model, which is addressed by multisets in 
this paper.


The following definition and the subsequent corollary formalize important 
graph-theoretic properties of circular-arc graphs.

\begin{definition}[Clique-Cycle, Left Clique , and Right Clique ]\label{lcrc}
Let  be a circular-arc graph,  be an arc model of , and  be the 
circle of .
To each point  on  corresponds a clique that contains the vertices 
whose corresponding arcs contain .
Let multiset  contain all cliques that correspond 
to points on  and define  as the 
multiset of cliques we obtain, when we remove all cliques from  that 
are not maximal with respect to inclusion among the cliques in   and all 
but one cliques that are equal.
Pair , where  is the cyclic permutation on  
that is implied by ordering the elements in  as their corresponding points appear 
(clockwise) on , is called a \emph{clique-cycle} for . 
For every vertex , there
exist two cliques  such that a 
clique  contains  if and 
only if .
We call  the \emph{left clique of }, denoted by , and  
the \emph{right clique of }, denoted by .
\end{definition}

When circular-arc graphs are discussed in this paper, the argumentation is based 
on one of their clique-cycles deduced from some arc model.

A frequently used notation in our paper is that of a {\em dominating vertex}, which
is a vertex adjacent to all other vertices of the graph.  The set of all
dominating vertices of graph  is denoted by .
Since a dominating vertex  is adjacent to all the vertices, 
all the maximal cliques contain it, wherefore 's left and right clique form a cyclic interval that covers the entire clique cycle.


\begin{corollary}\label{remarkDoms}
Let  be a clique-cycle for a circular-arc graph  
and  a dominating vertex.
The left clique  and the right clique  of  are not unique, 
but can be chosen arbitrarily with 
 being the only constraint.
\end{corollary}




Consider the arc model of a circular-arc graph and two intersecting arcs .
The arcs  and  can intersect in the following two ways:
\begin{inparaenum}
\item
The intersection of  and  constitutes another arc.
\begin{inparaenum}
This is the case,
\item
if  is included in  or vice versa,
\item
if  and  are congruent, 
i.e., cover the exact same part of the cycle, or 
\item
if exactly one endpoint of  lies in  and vice versa.
\end{inparaenum}
\item
The intersection of  and  constitutes two arcs (the corresponding vertices will 
be defined as counter vertices in the subsequent definition).
This is the case, if  and  jointly cover the circle and meet at both ends.
\end{inparaenum}


\begin{definition}[Counter Vertex ]\label{reaches}
Let  be a clique-cycle for a circular-arc graph .
We call  a \emph{counter vertex} of , if  and  and  (or, equivalently, ).
We call  a \emph{pair of counter vertices} and denote the set of all counter 
vertices of  by .\footnote{Note that, .}
\end{definition}

The next corollary follows directly from the arc model of a circular-arc graph.


\begin{corollary}\label{adjacentToVorCoV}
Let  be a circular-arc graph and  be a vertex 
with . Then every vertex in  is adjacent 
to  or to every vertex in , in other words,  and 
any of its counter vertices constitute a dominating set.
\end{corollary}



\begin{definition}[Reaching to the Left/Right]\label{defi:reaches}
Let  be a clique-cycle for a circular-arc graph ,  two adjacent vertices with  and  a clique with .
We say, \emph{ reaches at least as 
far to the right as } if , 
and \emph{ reaches further to the right than } 
if .
Analogously, we define \emph{ reaches at least as 
far to the left as } by  
and \emph{ reaches further to the left than } 
by .
\end{definition}

Although the arcs corresponding to counter vertices 
cover very different areas of the circle, it is impossible to say 
which of the two arcs \q{reaches further to the left or right}, when the point of 
view is the middle of the circle.








\subsection{Interval Routing Schemes}\label{IRS}



We assume that the reader is familiar with basic concepts of interval routing and refer to \cite{Gavoille00asurvey} for an exhaustive introduction and survey.
If an IRS assigns at most  intervals to each (directed) edge and only implies shortest paths, we denote it by -IRS.
If this -IRS is furthermore strict, which means that for every vertex  no interval assigned to the outgoing edges of  contains 's number, it is denoted by -SIRS.
Definition \ref{fe} is needed for Definition \ref{defiIRS}, which is an alternative definition for shortest path strict interval routing schemes that is equivalent to definitions in literature but better suited for our purposes.
To apply Definition \ref{defiIRS} to an undirected graph (as for example a circular-arc graph), the graph is converted to its directed, symmetric version.
While directed edges are often called arcs, we choose the former term to avoid confusion with the arcs of a graph's arc model.























\begin{definition}[First Vertex ]\label{fe}
Let  be a directed graph and .
Vertex  is a \emph{first vertex from  to }, if there exists a shortest directed path 
, i.e., there is a shortest directed path from  to , that first traverses .
The set of first vertices from  to  is denoted by 

\end{definition}



\begin{definition}[Shortest Path Strict Interval Routing Scheme]\label{defiIRS}
~ Let  be a directed graph.
A \emph{shortest path strict interval routing scheme for } 
is a pair , where  is a cyclic permutation on , 
called the \emph{vertex order}, and , called the \emph{directed-edge-labeling}, maps 
every directed edge to a set of ring-intervals in  such that for every vertex ,
\begin{enumerate}

\item 
 maps the outgoing directed edges of  to a set of ring-intervals 
in , such that the intervals
assigned to different directed edges never intersect,

\item
for every vertex   one of these these ring-intervals 
contains  (vertex  must not appear in one of these intervals), and

\item
if  is contained in a ring-interval in , then  is a first vertex 
from  to .
\end{enumerate}

Let  be a vertex order.
We say that  \emph{ given  suffices a shortest path -SIRS}, if  
there exists a directed edge-labeling for  that maps every outgoing directed edge of  to at most 
 ring-intervals in , such that  satisfies the three constraints above.
\end{definition}



\begin{corollary}\label{coro:neu}
A directed graph  supports a 
shortest path -SIRS, if a vertex order  exists, such that every 
vertex   given  suffices a shortest path -SIRS.
\end{corollary}











\section{Main result\label{main}}


The compactness of the class of circular-arc graphs has not been determined until now, 
which is somehow surprising, since the class of circular-arc graphs is closely related 
to the class of interval graphs and unit circular-arc graphs and both classes are known 
to have strict compactness  \cite{NS96,FG98}. 
However, for the compactness of the class of circular-arc graphs only a lower bound of 2 was known.
In particular, circular-arc graphs exist that do not allow for optimal 1 interval routing schemes, as, for example, a wheel graph, i.e., a cycle together with a dominating vertex, with six outer vertices \cite{FG98}.
Also the circular-arc graph shown in Fig. \ref{fig:animals} does not allow optimal 1 interval routing schemes.
In this section we prove the main result of this paper, which is given by the following theorem and shows that the lower bound of 2 for the compactness of circular-arc graphs is indeed sharp.

\begin{theorem}[Main Theorem]\label{mainTheo}
The class of circular-arc graphs has strict compactness 2.
\end{theorem}



Since every non-strict circular-arc graph  is an interval graph and therefore has strict compactness 1 \cite{NS96},  we only need to consider strict circular-arc graphs in the proof.
For the rest of this section let  be a clique-cycle for an 
arbitrary strict circular-arc graph  and  be the directed symmetric 
version of . 

\paragraph*{Outline of the proof}
To show that there exists a shortest path -SIRS for  and thus 
Theorem \ref{mainTheo} holds, we proceed as follows. 
In Section \ref{constrCO} we construct 
a cyclic permutation  on .
In Section \ref{furtherDefs} we choose an arbitrary vertex  and partition the 
vertices in  in three disjoint ring-intervals  in  in 
dependency on . 
In order to show that  given  suffices a shortest path 2-SIRS, 
we show how to define an directed edge-labeling  for the outgoing directed edges of  that 
satisfies the corresponding  constraints given in Definition \ref{defiIRS}. We
consider each of the three ring-intervals  in 
Section  \ref{lvbisv} and \ref{mvbislv} and 
show how the vertices in the respective ring-interval can be sufficiently  mapped to 
by . 
Since we have chosen  arbitrarily, it follows that every vertex in  
suffices a shortest path -SIRS given  and therefore, by 
Corollary \ref{coro:neu},  supports a shortest path -SIRS.
Since  is an arbitrary strict circular-arc graph, this proves 
Theorem \ref{mainTheo}.


The following notations are straight forward but may be formalized for convenience.
Let  be a directed edge with , where  is a set of ring-intervals 
in , and  be a ring-interval in . \emph{Assigning  to directed edge } 
means to define .
Of course, when we start constructing , every directed edge is mapped to the empty set.
Let  and  be two ring-intervals that can be joined to one 
ring-interval  by Construction \ref{ringkombi}. If  and 
 are assigned to the same directed edge , we can redefine

and therefore save one ring-interval on directed edge .
This redefinition is called \emph{compressing the (two) ring-intervals on directed edge  
(to one ring-interval)}. Let  be a set of vertices. \emph{To distribute  (over the outgoing directed edges of )} means to partition 
 into ring-intervals and assign these to outgoing directed edges of  (sufficiently).
For convenience,
we sometimes refer to ring-intervals as \emph{intervals}.




\subsection{Definition of the Vertex Order}\label{constrCO}





In this section we consider a circular-arc graph  
together with a clique-cycle  and show
how to construct a cyclic permutation  on  
that serves as the given vertex order. The ordering is obtained by
sorting the arcs using their left cliques as primary sort key
and right cliques as secondary sort key, making the first
arc the successor of the last. A pseudocode for this 
purpose is presented in Listing \ref{fig:animals} and its idea is explained 
next.


By Corollary \ref{remarkDoms} the left and right clique of a dominating vertex are not unique.
Since it simplifies the proof, if all dominating vertices have the same left and thus also 
the same right clique, these cliques are unified in Line 1.
For a fixed  in Line 1, the generated cyclic permutation is fully deterministic, 
except for the ordering of true twins.\footnote{Two vertices in a graph are called 
\emph{true twins} if they are adjacent to the same set of vertices and to each other.}
The dummy vertex introduced in Line 4 is needed to close the cyclic order once all vertices 
are integrated in .
The loop in Line 6 runs once through all cliques in  (beginning at , that 
is arbitrarily chosen in Line 2) in the order defined by  (Line 13).
For every visited clique  the loop in Line 8 integrates every vertex, whose left 
clique is , in .
By Line 9, vertices with the same left clique are ordered with respect to 
their right clique. An example for the defined vertex ordering can be found in 
Fig. \ref{fig:animals}.











\begin{lstlisting}[caption={Definition of a cyclic order  on  that serves as the given vertex order.},captionpos=b,label=pseudoLabeling]
Fix  arbitrarily and choose  and  for all 
Choose  arbitrarily

    /* add dummy vertex  */

do
      
      while()
	    Choose  such that every vertex in  reaches at least as far to the right as .
	    
	    
	    
      
while() 

     /* remove dummy vertex  */
return 
\end{lstlisting}























The following observations are crucial for the rest of this paper.




\begin{remark}
\begin{enumerate}
\item\label{item:leftconsecutiv}
The vertices  are ordered primarily by their left clique, that is to say, vertices having 
the same left clique appear consecutively in .

\item 
Vertices with the same left clique are ordered in ascending 
order by their right clique.

\item\label{item:onemoreleft}
For two vertices  with , we know that  and  are adjacent and either
\begin{enumerate}

\item
 and  reaches at least as far to the right as , or

\item
, in other words,  reaches one clique further to the 
left than .\footnote{Because the relation \q{reaching to the right} is not defined, if  and  are counter vertices, we intuitively extend Definition \ref{defi:reaches} to this case by fixing .}

\end{enumerate}
\end{enumerate}
\end{remark}

The next definition is based on Assertion \ref{item:leftconsecutiv} of the preceding 
remark, that is to say, on the fact that, since vertices having the same left clique 
appear consecutively in , for a given clique , two 
vertices  exist, such that the vertices in  ring-interval  are 
exactly those with left clique .

\begin{definition}[Head Vertex  and Tail Vertex ]
Let  be a clique-cycle for some circular-arc graph ,  
given by Listing \ref{fig:animals}, , and  the unique 
vertices such that  ( and  are not necessarily distinct).
We call  the \emph{head vertex of } and  the \emph{tail vertex of } 
and denote them by  and , respectively.
\end{definition}



\begin{corollary}\label{coro:tail vertexheal}
Let  be a clique-cycle for a circular-arc graph ,  given 
by Listing \ref{pseudoLabeling}, and .
The successor (in ) of the tail vertex of  is the head vertex of the 
successor (in ) of , i.e.,
.
\end{corollary}







\subsection{Partition of the Vertex Order}\label{furtherDefs}





The cyclic permutation  on  allows to state further definitions.
We now fix  arbitrarily for the rest of the proof and show that  given  
suffices a shortest path -SIRS.
If  is a dominating vertex, we can define  
for every vertex  and have thereby shown that  given  suffices a shortest path -SIRS.
Therefore, we now assume that  is not a dominating vertex.
Fig. \ref{fig:tiger3} illustrates the subsequent Definition.

\begin{definition}[Left Vertex ]\label{de_l_v}
Let  be a clique-cycle for some circular-arc graph ,  
given by Listing \ref{fig:animals}, and  not a dominating vertex.
\begin{inparaenum}[(i)]
Let  be the union of the following two vertex sets.
\item The set of
vertices that are adjacent to  and reach further to the left than  
but are neither counter vertices of  nor dominating vertices.
\item
The set of vertices in .\footnote{We have  and .}
\end{inparaenum}
If , we define the \emph{left vertex of }, denoted by ,  
as the unique vertex in  such that

\end{definition}



\begin{figure}
\centering
\subfloat[The green vertices are in set  of Definition \ref{de_l_v}. 
The left vertex of , which is also in , is colored in light green 
and the middle vertex of  in red.]{\label{fig:tiger3}\includegraphics[trim=0cm 2cm 0cm 0cm, width=0.45\textwidth,]{EXAM010mod.pdf}}\quad
  \subfloat[The green vertices are to the left of , the red vertices are to the right of , and the black vertices are face-to-face with . 
]{\label{fig:gull3}\includegraphics[trim=0cm 2cm 0cm 0cm, width=0.45\textwidth,]{EXAM010mod2.pdf}}                
\vspace{-5pt}\caption[An example for the vertex partition defined in Section \ref{furtherDefs}]{Two examples for the Definitions made in this section
The fixed vertex  is colored blue. 
The yellow line connects the vertices as they appear in vertex order  as generated by Listing \ref{fig:animals}.
}
\label{fig:anotherExamplefurtherDefs}
\end{figure}




\begin{theorem}\label{no_vertex_further_than_lv}
Let  be a clique-cycle for a circular-arc graph ,  
given by Listing \ref{fig:animals},  not a dominating vertex,  the left 
vertex of , and  be a vertex that is adjacent to  and reaches further to the left than  but is neither 
a dominating vertex nor a counter vertex of . 
Then  reaches at least as far to the left as .
\end{theorem}



\begin{proof}
Assume  reaches further to the left than .
Then by Definition \ref{defi:reaches} we know that 
, which implies 
that 
Since  is primary ordered by the left cliques of the vertices, it follows 
 and therefore 
.
This contradicts (\ref{eq_l_v}) in Definition \ref{de_l_v}, since  reaches 
further to the left than  and is neither a 
dominating vertex nor a counter vertex of .
\end{proof}

In Definition \ref{leftrightface} we partition the vertices in  into three 
ring-intervals in .
The following argumentation gives the idea behind this partitioning and also proves 
Theorem \ref{Lvrl_v}. Consider we go through  beginning at .
Since the vertices in  are primary ordered by their left cliques, the left 
clique of the vertices first considered is  (or , if  is the 
tail vertex of clique ).
The next vertices traversed have left clique , followed by vertices with 
left clique , and so on.
We eventually reach the set of vertices with left clique .
The last vertex we come across in this set is  and fixed in Definition \ref{de_m_v}.
The vertices we came across so far (excluding , including ) will be 
defined as the \q{vertices to the right of } in Definition \ref{leftrightface} and are 
adjacent to , as  was contained in their left clique.

\begin{definition}[Middle Vertex ]\label{de_m_v}
Let  be a clique-cycle for a cir\-cular-arc graph ,  
given by Listing \ref{fig:animals}, and  not a dominating vertex.
We define the \emph{middle vertex of }, denoted by ,  
as the tail vertex of the right clique of .
\end{definition}






When we continue visiting the vertices after  in the same manner, the next vertex  we come across that is not a dominating vertex but adjacent to  has to be , as we could find a contradiction in the same manner as 
in the proof of Theorem \ref{no_vertex_further_than_lv}, if we had .
These vertices after  and before  are defined as the \q{vertices face-to-face with } 
in Definition \ref{leftrightface}. When we continue to go through , we eventually 
reach  again. These vertices from  to  (excluding ) are defined 
as the \q{vertices to the left of } in Definition \ref{leftrightface}, which is illustrated in Fig. \ref{fig:gull3}.


\begin{definition}[, , ]\label{leftrightface}
Let  be a clique-cycle for a circular-arc graph ,  
given by Listing \ref{fig:animals},  not a dominating vertex,  the left 
vertex of , if  exists, and  the middle vertex of . We define
\begin{enumerate}
\item
the \emph{vertices to the right of } 
as , and 



\item
the \emph{vertices face-to-face with }
\begin{enumerate}



\item
as , if  does not exist, or else

\item
as , and
\end{enumerate}
\item 
the \emph{vertices to the left of }\begin{enumerate}

\item 
as , if  exists, or else

\item
as .\footnote{
We have  from Definition \ref{de_l_v}.
}
\end{enumerate}
\end{enumerate}
\end{definition}



The next theorem follows from the argumentation between 
Theorem \ref{no_vertex_further_than_lv} and Definition \ref{leftrightface}.

\begin{theorem}\label{Lvrl_v}
Let  be a clique-cycle for a circular-arc graph ,  
given by Listing \ref{fig:animals}, and  but not a dominating vertex.
\begin{enumerate}

\item \label{Lvrl_v1}
Every vertex to the right of  is adjacent to .

\item
Every vertex face-to-face with  that is not a dominating vertex is not 
adjacent to .
\end{enumerate}
\end{theorem}




\subsection{Definition of the Directed Edge-labeling}\label{2-app}



Next we show that  given  suffices a shortest 
path -SIRS by constructing a mapping  from the set of outgoing directed edges of  
to at most two ring-intervals in , according to Definition \ref{defiIRS}.
We investigate the vertices to the left of  and the vertices 
to the right of  in Section \ref{lvbisv} and the vertices face-to-face with  
in Section \ref{mvbislv}.
For every vertex  in the respective interval, we determine a first vertex  from 
 to  such that  and every vertex that is hitherto assigned to directed edge  can 
be embraced by at most two ring-intervals in .
This is a simple task for the vertices in  and  and 
even for the vertices in  the approach is straight forward, if there is a 
dominating vertex or a counter vertex of .
In fact, it only gets tricky, if there are no dominating vertices and no 
counter vertices.






\subsubsection{Vertices to the Left and Right}\label{lvbisv}





By the first assertion of Theorem \ref{Lvrl_v} every vertex to the right of  is 
adjacent to , wherefore these vertices can be distributed by assigning  
to directed edge , for every vertex . 
Assume that we have .
In general, not every vertex in  is adjacent to . Although  is adjacent to all vertices in , it is generally not possible to assign 
 to directed edge , since  may contain vertices that are adjacent to  (other than ), and therefore must be assigned 
to \q{their own directed edge}.
Let  be a vertex that is adjacent to  and assume we start at  to go through 
 until we come across the next vertex that is adjacent to  or  itself.
Denote this vertex .
We can show that  is adjacent to every in , wherefore we 
can assign  to directed edge .
Therefore the vertices to the left of  can be distributed by assigning every vertex 
 to directed edge , if  is adjacent to , and else to directed edge , where  
is the first vertex that precedes  in  and is adjacent to  .
Theorem \ref{theototheleft} proves the outlined idea formally.
When distributing the vertices in sets  and  as just outlined, every 
outgoing directed edge of  (except for edges incident to dominating vertices face-to-face with , which are not yet labeled) gets one ring-interval assigned.




\begin{theorem}\label{theototheleft}
Let  be a clique-cycle for a circular-arc graph ,  
given by Listing \ref{fig:animals},  but not a dominating vertex, and  
the left vertex of  (then the vertices in  are the vertices to 
the left of ).
Let  be the sequence of vertices that we obtain, when we order the vertices in  that are adjacent to  as they appear in ring-sequence  and append .
For  and , vertex  is a first vertex 
from  to .
\end{theorem}



\begin{proof}
For some , , let .
Clearly the theorem holds if .
If ,  does not appear in  and therefore is not 
adjacent to . Since the vertices in  are primary ordered by their left clique 
and  appears before  in , it follows that the left 
clique of  is in . Since  is adjacent to ,  
is contained in every clique in .
This implies that  is contained in the left clique of  and thus  and  
are adjacent. Therefore  is a shortest path between  and  and  
is a first vertex from  to .
\end{proof}






\subsubsection{Vertices Face-to-Face}\label{mvbislv}





This section discusses the distribution of vertices face-to-face with .
It is probably the common case, that neither counter nor dominating vertices exist,
since only then path lengths are unbounded.
Thus, this case is discussed in the next paragraph and the 
subsequent paragraph discusses the remaining cases.


\paragraph{Neither Dominating nor Counter Vertices exist}\label{beideLeer}
In this case,  always reaches further to the left than  and there is at least 
one vertex adjacent to  that reaches further to the right.
Let  be a vertex that is not adjacent to ,  be the set of vertices adjacent to  
that reach farthest to the left and  be the set of vertices adjacent to  that reach 
farthest to the right ( and  might intersect).
As evident from the arc model of , every vertex in  or every vertex in  
is a first vertex from  to  (cf. Corollary \ref{b}).
Since we have , we now fix a vertex adjacent to  that reaches farthest to the right.

\begin{definition}[Right Vertex ]\label{de_r_v}
Let  be a clique-cycle for a strict cir\-cu\-lar-arc graph  without 
dominating vertices and without counter vertices,  given by Listing \ref{fig:animals}, 
,  the left vertex of ,  the middle vertex of , and  be the 
set of vertices that are adjacent to  and reach farthest to the right.
We define the \emph{right vertex of }, denoted by ,  as , if , or 
as , if , or else as an arbitrary vertex in .
\end{definition}
In order to show that  suffices a shortest path 2-SIRS given , we could choose  arbitrarily in  even if  or .
However, we will show that  given  suffices a 
shortest path -SIRS, if  or .
The next definition redefines the notation of the left and right vertex as a 
function, in order to allow 
recursive usage to easily determine a first vertex from  to every vertex.
The subsequent corollary is clear when illustrated.

\begin{definition}[,   and , ]\label{lvi}
Let  be a clique-cycle for a strict circular-arc graph  without 
dominating vertices and without counter vertices, cyclic permutation 
 given by Listing \ref{fig:animals}, 
and . We define  as the left vertex of  and  as the right vertex 
of .\footnote{It follows that  is the left vertex of the left vertex of  and  
is the right vertex of the right vertex of , etc.}
For , we define  as the smallest  such that  is adjacent to  and  
 as the smallest  such that  is adjacent to .
\end{definition}




\begin{corollary}\label{b}
Let  be a clique-cycle for a strict circular-arc 
graph  without dominating vertices and without counter 
vertices, cyclic permutation 
 given by Listing \ref{fig:animals}, 
and . The following three assertions hold for every 
vertex  that is not adjacent to .
\begin{enumerate}
\item\label{jans}
If
, then  is a first vertex from  to .
\item\label{jans2}
If
, then  is a first vertex from  to .
\item\label{jans3}
If
, then  and  are first vertices from  to .
\end{enumerate}
\end{corollary}






Since no vertex in  is adjacent to , the corollary implies that  or  is a 
first vertex from  to every vertex in .
Since sequence 
 \q{runs} through the clique-cycle as 
implied by  and sequence 
 \q{runs} through the clique-cycle as 
implied by , that is to say, in the other direction, 
there exists an  such that  and  \q{meet}, 
or more formally, are adjacent or equal. 
This number is fixed in the following definition. 



\begin{definition}[Apex Number]\label{apex}
Let  be a clique-cycle for a strict cir\-cular-arc graph  
without dominating vertices and without counter vertices, 
 given by Listing \ref{fig:animals}, and .
If we have  or 
 
we set , 
otherwise we define  as the smallest number greater than 1 
such that  and  are adjacent or equal.
We call integer  the \emph{apex number of }.
\end{definition}


The separate treatment for apex number  in Definition \ref{apex} ensures that the 
arcs of  and  intersect/meet adverse to  on the 
circle and therefore the arcs of , ,
and  to cover the entire circle.
In particular, if  and  intersect in 's arc they cannot 
cover the entire cycle, since they are not counter vertices, but the apex 
number would be defined as 1, if only the second part of the case distinction were in place.






By the definition, for apex number  vertices  and  are 
adjacent if and only if  and  are adjacent.
Furthermore, for  vertices  and  are not adjacent.

The following Theorem \ref{separator} shows that there 
exists a vertex  with , where  is 
the appex number of , such that  is a 
first vertex from  to every vertex in  and vertex  is a 
first vertex from  to every vertex in . 
Since Theorem \ref{separator} implies that the vertices in  can be distributed 
by assigning  to directed edge  and  to directed edge , the main theorem 
(Theorem \ref{mainTheo}) 
follows for the case in which neither dominating vertices nor counter vertices exist.











\begin{theorem}\label{separator}
Let  be a clique-cycle for a strict circular-arc graph  without 
dominating vertices and without counter vertices,  given by 
Listing \ref{fig:animals}, ,   the left 
vertex of ,  the middle vertex of , and  the right vertex of .
There exists a vertex  such that, for 
every vertex , we have , and 
for every vertex , 
we have .
\end{theorem}




\begin{proof}
Let  the apex number of .

If  the theorem holds for , since every vertex in 
 is adjacent to .

If  and , we have

and

wherefore, by Corollary \ref{b}, Theorem \ref{separator} holds for .



The following argumentation is partially illustrated in Fig. \ref{neu}.
\begin{figure}[t]
\begin{center}
\includegraphics[width=0.9\textwidth]{neu.pdf}
\end{center}
\caption{
This picture illustrates the proof of Theorem \ref{separator} and partially shows a 
circular-arc graph, where the lines represent arcs.
A line ends with a triangle, if it may reach not further to left or right, respectively.
The upper two lines with a dotted end illustrate that an arc, that has its right end in 
the interval depicted by the rotated 
curly bracket, may reach no further to the left than the triangle of the respective line.
We begin to search for  at the right end of the lower curly bracket.
The rotated square brackets on top mark the areas of the left cliques of 
the vertices in sets .
The case in which  looks similar.
}
\label{neu}
\end{figure}





If  and , let

and .
We have , because  is adjacent to a vertex 
in .
When  reaches further to the right than ,  might appear in a clique in  and, therefore, be adjacent to , which implies .
However, in this case, we also have . Therefore, we could only have , when  reaches so far to the right, such that it is adjacent to .
However, in this case  would reach further to the left than , because  is adjacent to  and  is \q{at most} adjacent to .
Because  is the left vertex of  this contradicts Theorem \ref{no_vertex_further_than_lv}.
Thus, Corollary \ref{b} implies 



Now we go through  beginning at 
,\footnote{This 
equality holds by Corollary \ref{coro:tail vertexheal}.} in other 
words, we begin at the first vertex after the \q{last} vertex in  , until we 
eventually come across a vertex  that either is  or adjacent 
to .\footnote{This vertex 
 may be . Also there may be vertices in  that are adjacent to 
, in other words, we may have \q{passed by} a vertex that is 
adjacent to  already.}
We define

i.e.  is the set of vertices after  and before  in .
Note that, by the choice of , no vertex in  is adjacent to .

Assume there is a vertex  that is not adjacent to .
When we go through  beginning at  we eventually come across  
(since  and  are adjacent, we have ).
Before we come across , we come across , since otherwise  and  would be adjacent.
We do not come  across  until we came across , since otherwise  and  would be adjacent.
It follows that  and  are not adjacent and, because  is adjacent to   and , it follows that  reaches further to the right than .
Because  and  are adjacent,  
must appear in .
It cannot appear in , because then  
is adjacent to ,
which contradicts Definition \ref{de_r_v}, since  reaches further to the right 
than  and  is the right vertex of .
Therefore, we have
.
Because we came across  before , we have  and, therefore,

Since the vertices in  are primary ordered by their left clique, Inclusion (\ref{wtf}) implies

Remember that we began at  
to go through  until we found a vertex  that is adjacent to .
Inclusion \ssref{wtfwtf} implies that we must  have come across  before .
But then we have .
Since
 
contradicts the initial assumption, every vertex in  is adjacent to .



Since by the choice of  no vertex in  is adjacent to , we have

which, by Corollary \ref{b}, implies

We define .
Since the ring-intervals  and  appear consecutively in , 
by Construction \ref{ringkombi}, we have

From \ssref{eqd}, \ssref{eqc}, and \ssref{bbt} it follows


Since the vertices in  are primary ordered by their left cliques, 
no vertex   is adjacent 
to  and therefore we have . We define

i.e.  is the set of vertices after  and before  in .
Note that 
We have

since otherwise  would reach further to the left than  although both are 
adjacent to  and  is the left vertex of .
This would contradict Theorem \ref{no_vertex_further_than_lv}.
It follows

which, by Corollary \ref{b}, implies 

Obviously we have 

By Construction \ref{ringkombi}, we have 

and therefore \ssref{eqa} and \ssref{eqb} imply 

The statement follows from \ssref{eqw1} and \ssref{eqw2}.
\end{proof}







In Section \ref{lvbisv} we assigned  to directed edge 
for some vertex  to the left of .
If  is assigned to directed edge  
additionally (as implied by Theorem \ref{separator}) 
the intervals on directed edge  can be compressed.
Therefore, if neither dominating vertices nor counter vertices exist, 
 given  suffices a shortest path 2-SIRS, 
but only  is assigned two ring-intervals.

Below Definition \ref{de_r_v} we stated that  given  suffices 
a shortest path 1-SIRS, if we can choose  or .
This is the case because then the ring-intervals assigned to directed edge  
can be compressed as well.

Theorem \ref{theo:1sirs} covers a further case where  given  suffices a shortest path 
1-SIRS.
Since Theorem \ref{theo:1sirs} is not essential for the proof of Theorem \ref{mainTheo}, 
we leave the proof to the interested reader.


\begin{theorem}\label{theo:1sirs}
Let  be a clique-cycle for a circular-arc graph ,  
given by Listing \ref{fig:animals},  not a dominating vertex,  the 
left vertex of , and  the middle vertex of .
If  is adjacent to every vertex face-to-face with  that  is not adjacent 
to,  suffices a shortest path 1-SIRS.
\end{theorem}




\paragraph{Dominating Vertices or Counter Vertices exist}\label{gyros}





In the foregoing section the existence of dominating and counter vertices was excluded and is therefore discussed in this section.
The vertices to be distributed are those face-to-face with , i.e., the vertices in .
By Theorem \ref{Lvrl_v}, the only vertices in  that are adjacent to  are dominating vertices;
this case is considered first.
Next, the case where dominating vertices not face-to-face with  or counter vertices of  exist is addressed. Finally, the case where a pair of counter vertices exists (but  has no counter vertex) is discussed. 



\subparagraph{Dominating Face-to-Face Vertices exist}


Listing \ref{fig:animals} ensures that all dominating vertices appear consecutively in , 
wherefore we can find two dominating vertices  and  such that . 
Let  such that .
Since by Theorem \ref{Lvrl_v},  are the only vertices in  that are adjacent to , we can assign  to directed edge ,  to directed edge , and for the remaining dominating vertices ,  to directed edge .
It follows that  given  suffices a shortest path 1-SIRS.


\subparagraph{Dominating Face-to-Face Vertices  do not exist}

If there are no dominating vertices in , Theorem \ref{Lvrl_v} implies that no vertex in  is adjacent to .
Therefore, by Corollary \ref{adjacentToVorCoV}, every vertex that
is a counter vertex of  is adjacent to every vertex in .
Let  and .
Then  is a first vertex 
from  to , wherefore we can assign ring-interval  to 
directed edge , which had hitherto one ring-interval assigned.
Thus,  given  suffices a shortest path 2-SIRS, wherein only directed edge  has two ring-intervals assigned.
By Theorem \ref{theo:1sirs},  given  suffices a 
shortest path 1-SIRS, if .





\subparagraph{Pairs of Counter Vertices Exist but  is in none of them}\label{counteratall}

Assume  does not contain dominating 
vertices and the considered vertex   has no counter vertices but 
there exists at least one pair  of counter vertices.
We have  and 
no vertex in  is adjacent to .
We make a distinction of cases for whether  is adjacent to one or to both counter vertices.


Assume  is adjacent to  as well as .
There is a clique  that contains , , and .
When we begin at  to go through ,
we eventually reach a clique  that contains only one of  or 
and not the other.
Without loss of generality, let  be contained in  and let  be the first clique that contains  again.
 is the left clique of  and 
contains  (we might have ), as
otherwise,  would reach further to the left than , 
which contradicts Theorem \ref{no_vertex_further_than_lv}.
Since  and  are counter vertices, all cliques we came across contained .
Since the vertices in  are primary ordered by their 
left cliques and we also must have traversed , which contains , every vertex in 
 appeared in at least one of the cliques we came across.
Therefore,  is adjacent to every vertex in .
Since  hitherto maps directed edge  to one ring-interval, we can assign  to directed edge . 
Thus,  given  suffices a shortest path 2-SIRS, wherein only directed edge  is 
actually mapped to two ring-intervals.




Now assume  is adjacent to one of the counter vertices and not to the other.
Without loss of generality, let this vertex be .
Let  be a vertex adjacent to  that, similar to Definition \ref{de_r_v}, reaches farthest to the right.
Since, by Theorem \ref{no_vertex_further_than_lv},  is one of the vertices 
adjacent to  that reaches farthest to the left,  as well as  are adjacent to .
In fact, we might have  or .
Let  be the middle vertex of  and  the set of vertices to the right of .
 is not empty, since otherwise  would be an interval graph.
Also  is contained in , if .
By Theorem \ref{Lvrl_v}, every vertex in  is adjacent to .
Consider we go through  beginning at  until we reach  or .
If we have , we consider it as reaching .
If we first reach , we exactly traversed the vertices in .
It follows that we have  and thus
assign  to directed edge .
Now assume we first reach . 
By Theorem \ref{Lvrl_v}, every vertex in  
is adjacent to , wherefore we can assign  to directed edge .
If , all vertices in  are distributed.
Otherwise let  be the vertices left to be distributed.
Since, by Theorem \ref{no_vertex_further_than_lv},  is a vertex adjacent to  
that reaches farthest to the left and  is adjacent to , 
for every vertex  either  or  is a shortest path.
Thus, we can assign  to directed edge  (and compress the two intervals on this edge),
wherefore all vertices are distributed and only directed edge  is assigned two ring intervals.

\section{Improving the Space Requirement}\label{improving}



This section shows that although circular-arc graphs do not allow shortest path 1-IRSs in general, 
the number of intervals in a shortest path 2-SIRS for a circular-arc graph is 
bounded close to the number of intervals in an 1-IRS.
Let  be the directed symmetric version of a strict circular-arc graph.
A shortest path 2-SIRS for  can be obtained by ordering the vertices of  
according to Listing \ref{fig:animals}
and then labeling the outgoing directed edges of every vertex according to 
Section \ref{lvbisv} and \ref{mvbislv}.
As pointed out in these sections, there is at most one outgoing edge per vertex that is assigned two intervals.
We have , since  is a strict circular-arc graph, and  only holds when  is a ring in which case our labeling yields a shortest path 1-SIRS.
Therefore, the number of intervals in the constructed 2-SIRS is less than , while already a 1-IRS for  permits up to  intervals.






\section{Implementation}\label{implement}


This section outlines how the interval routing scheme implied by the proof 
of Theorem \ref{mainTheo} can be implemented.
Let  be a circular-arc graph with  and  and 
 be a clique-cycle for  with 
maximal cliques ( does not need to contain sets of vertices; plain 
elements representing the cliques are sufficient).
We discuss how the vertex order  and the critical vertices 
 for every vertex  can be determined in  time.
We assume that every vertex  has a pointer  to 
its left clique and a pointer  to its right clique in .


We define \emph{broadness of a vertex } as the number of cliques 
in  that contain .
By numbering the elements in  as they appear in , beginning at 
an arbitrary element, we construct a bijective mapping 
, which allows to compute the 
broadness of a vertex in   time.
Sorting the vertices ascending by their broadness 
requires  time. We introduce a list  for 
every clique . Next the vertices are traversed in the sorted order and 
every vertex  is added to list .
By concatenating the lists as their respective cliques appear in  we obtain the vertex order
 in  time.
The left vertex  of a given vertex  can be determined as follows:
we inspect all vertices that are adjacent to  and store the vertex  that reaches 
farthest to the left (for two adjacent vertices , we can determine in constant time 
which one reaches further to the left, by considering  and ) 
and appears first in  (this can also be decided in constant time, by numbering 
the elements in each list consecutively). Vertex  must be .
It follows that the set of left vertices can be computed in  time.
The set of right vertices can be computed in similarly.
Determining the set of middle vertices is simpler and can be accomplished in 
.
For a given vertex ,  and  can be determined by traversing 
the sequences  and  synchronously; 
then the determination of   is a simple task and can be done in 
for a specific vertex and in  for all vertices.

Since the algorithm includes constant time operations on the elements in  
and  is linear within the number of vertices, the overall runtime 
is .
The algorithm can be extended to  handle the existence of dominating 
and counter vertices without exceeding this time bound.







\section{Conclusions}\label{conc}

We showed that the class of circular-arc graphs has strict compactness 2.
Throughout the proof special cases of strict circular-arc graphs that allow 
1-SIRSs were highlighted (not all cases were pointed out for the sake of brevity), which is in particular interesting, since the class of circular-arc graphs is a super-class of the class of interval and unit circular-arc graphs, which always allow 1-SIRSs.
We showed
that the constructed 2-SIRS requires a number of intervals that is 
closer to the maximal number of intervals in an 1-IRS than in a 2-IRS. 
An open question is, if strict circular-arc graphs exist that allow 1-SIRS but the vertex ordering generated by
Listing \ref{pseudoLabeling} does not allow these.











































\bibliographystyle{alpha}
\bibliography{bibfile}











\end{document}
