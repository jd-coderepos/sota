\documentclass[a4paper, 11pt]{article}


\usepackage{amsmath,amssymb,amsthm}



\usepackage[a4paper, margin=1.2in]{geometry}





\usepackage[lined,boxed]{algorithm2e}
\usepackage{graphicx}


\newtheorem{theorem}{Theorem}\newtheorem{proposition}[theorem]{Proposition}
\newtheorem{lemma}[theorem]{Lemma}
\newtheorem{corollary}[theorem]{Corollary}
\newtheorem{definition}[theorem]{Definition}

\newcommand{\cube}{\ensuremath{\{0,1\}}}

\newcommand{\SAT}{\textsc{SAT}}
\newcommand{\UNSAT}{\textsc{UNSAT}}

\newcommand{\NP}{\textsf{\textup{NP}}}
\renewcommand{\P}{\textsf{\textup{P}}}
 \newcommand{\us}{\ensuremath{{\overline{s}}}}
 \newcommand{\ls}{\ensuremath{{\underline{s}}}}
\newcommand{\sat}{\ensuremath{\textnormal{sat}}}
\newcommand{\poly}{\ensuremath{\textnormal{poly}}}
\newcommand{\F}{\ensuremath{\mathcal{F}}}
\newcommand{\M}{\ensuremath{\mathcal{M}}}
\newcommand{\MU}{\ensuremath{\mathcal{MU}}}

\newcommand{\MIXED}{\ensuremath{\mathcal{BGR}}}
\newcommand{\BLACK}{\ensuremath{\mathcal{B}}}
\newcommand{\GREEN}{\ensuremath{\mathcal{BG}}}
\newcommand{\occ}{\ensuremath{{\rm{occ}}}}
\newcommand{\OR}{\ensuremath{\veebar}}
\newcommand{\e}{\ensuremath{e}}
 \newcommand{\obda}{w.\,l.\,o.\,g.\ }
 \newcommand{\domi}[1]{{\small {\sc domi says}:  {\sf #1}}}
 \newcommand{\ignore}[1]{}




\title{Unsatisfiable CNF Formulas need many Conflicts
\thanks{Research is  supported by the SNF Grant 200021-118001/1}
}
\author{Dominik Scheder\\ 
  ETH Z\"urich\\
  \texttt{dscheder@inf.ethz.ch} \and
  Philipp Zumstein\\
  ETH Z\"urich\\
  \texttt{zuphilip@inf.ethz.ch}}




\begin{document}

\maketitle

\begin{abstract} A pair of clauses in a CNF formula constitutes a
  conflict if there is a variable that occurs positively in one clause
  and negatively in the other. A CNF formula without any conflicts is
  satisfiable. The Lov\'asz Local Lemma implies that a -CNF formula
is satisfiable if each clause conflicts with at most
   clauses. It does not, however, give any good bound
  on how many conflicts an unsatisfiable formula has globally. We show
  here that every unsatisfiable -CNF formula requires
   conflicts and there exist unsatisfiable -CNF
  formulas with  conflicts.
\end{abstract}

\section{Introduction}

A boolean formula in conjunctive normal form, a {\em CNF formula} for
short, is a conjunction (AND) of {\em clauses}, which are disjunctions
(OR) of literals. A {\em literal} is either a boolean variable  or
its negation .  We assume that a clause does neither contain
the same literal twice nor a variable and its negation. A CNF formula
where each clause contains exactly  literals is called a -CNF
formula.  Satisfiability, the problem of deciding whether a CNF
formula is satisfiable, plays a major role in computer science.  How
can a -CNF formula be unsatisfiable?  If  is large, each clause
is extremely easy to satisfy individually.  However, it can be that
there are conflicts between the clauses, making it impossible to
satisfy all of them simultaneously.  If a -CNF formula is
unsatisfiable,
then we expect that there are many conflicts.\\

To give a formal setup, we say two clauses {\em conflict} if there is
at least one variable that appears positively in one clause and
negatively in the other. For example, the two clauses  and
 conflict, as well as  and  do.  Any CNF formula without the empty clause and
without any conflicts is satisfiable.  For a formula  we define the
{\em conflict graph} , whose vertices are the clauses of ,
and two clauses are connected by an edge if they conflict. 
denotes the maximum degree of , and  the number of
conflicts in , i.e., the number of edges in .  In fact, any
-CNF formula is satisfiable unless  and  are
large. A quantitative result follows from the lopsided Lov\'asz Local
Lemma~\cite{ES1991,AS2000,LS2007}: A -CNF formula  is
satisfiable unless some clause conflicts with  or more
clauses, i.e., unless . Up to a constant
factor, this is tight: Consider the formula containing all 
clauses over the variables , the {\em complete
  -CNF formula} which we denote by .  It is
unsatisfiable, and
.\\

As its name suggests, the lopsided Lov\'asz Local Lemma implies a {\em
local} result.
Our goal is to obtain a {\em global} result:  is satisfiable unless the total number of
conflicts is {\em very} large. We define two functions

The abbreviations  and  stand for {\em local conflicts} and
{\em global conflicts}, respectively. From the above discussion,
, hence we know  up to a
constant factor. In contrast, it does not seem to be easy to prove
nontrivial upper and lower bounds on .  Certainly,  and .  Ignoring constant factors,  lies somewhere
between  and .  This leaves much space for improvement.
In~\cite{SZ2008}, we proved that  and . In this paper, we improve upon these
bounds.  Surprisingly,  is \emph{exponentially} smaller than
.

\begin{theorem}
  Any unsatisfiable -CNF formula contains
   conflicts. There are unsatisfiable
  -CNF formulas with  conflicts.
\label{main}
\end{theorem}

We obtain the lower bound by a more sophisticated application of the
idea we used in~\cite{SZ2008}. The upper bound follows from a
construction that is partially probabilistic, and inspired in parts by
Erd\H{o}s' construction in~\cite{Erdos1964} of sparse -uniform
hypergraphs that are not -colorable. To simplify notation, we view
formulas as sets of clauses, and clauses as sets of literals. Hence,
 denotes the number of clauses in . Still, we will sometimes
find it convenient to use the more traditional logic notation.

\subsection*{Related Work}

Let  be a CNF formula and  be a literal. We define . For a variable , we write  and call it the {\em degree} of .
We write . It is easy to see
that for a -CNF formula, . Define
 
By an
application of Hall's Theorem, Tovey~\cite{Tovey1984} showed that every
-CNF formula  with  is satisfiable, hence . Later, Kratochv\'il, Savick\'y and Tuza~\cite{KST1993} showed
that  and
. The upper bound was improved by
Savick\'{y} and Sgall~\cite{SS2000} to , by
Hoory and Szeider~\cite{HS2006} to , and recently by
Gebauer~\cite{Gebauer2009} to ,
closing the gap between lower and upper bound on  up to a
constant factor. Actually, we used the formulas constructed in~\cite{HS2006}
to prove the upper bound  in~\cite{SZ2008}.\\




\section{A First Attempt} \label{section-first-attempt}

We sketch a first attempt on proving a nontrivial lower bound on
. Though this attempt does not succeed, it leads us to other
interesting questions, results, and finally proof methods which can be
used to prove a lower bound on .  Let  be a -CNF formula and
 a variable. Every clause containing  conflicts with every
clause containing , thus . Furthermore,

where the  comes from the fact that each conflict might
be counted up to  times, if two clauses contain several
complementary literals.  Every unsatisfiable -CNF formula 
contains a variable  with .  If this
variable is {\em balanced}, i.e.,  and 
differ only in a polynomial factor in , then . Indeed, in the formulas constructed
in~\cite{Gebauer2009}, all variables are balanced. The same holds for
the complete -CNF formula .  It follows that when
trying to obtain an upper bound on  that is exponentially
smaller than , we should construct a very {\em unbalanced}
formula.  We ask the following question:

\begin{quotation}
  {\em Question:} Is there a number  such that for every
  unsatisfiable -CNF formula  there is a variable with
   and ?
\end{quotation}

The answer is a very
strong no:
In~\cite{SZ2008} we gave a simple inductive construction of a -CNF
formula  with  for every variable .
However, in this formula one has .
Allowing  to be a small exponential in , we have
the following result:

\begin{theorem}
 
  \begin{itemize}
  \item[(i)] For every ,  there is a
    constant  such that for all sufficiently large , there is an
    unsatisfiable -CNF formula  with  and , for all .
  \item[(ii)] Let  and . Then every -CNF formula  with
     and 
     is
    satisfiable.
  \end{itemize}
  \label{tradeoff-exponential}
\end{theorem}

Of course, we can interchange the
roles of  and , but it is convenient to assume that
 for every .
In the spirit of these
results, we might suspect that if  is unsatisfiable, then for some
variable , the product   is large.

\begin{quotation}
  {\em Question:} Is there a number  such that every unsatisfiable
  -CNF formula contains a variable  with 
  ?
\end{quotation}

Clearly,  for any such number . The complete
-CNF formula witnesses that  cannot be greater than , and it
is not at all easy to come up with an unsatisfiable -CNF formula
where  is exponentially smaller than
 for every . We cannot answer the above question, but we
suspect that the answer is yes. We prove an upper bound on the
possible value of :

\begin{theorem}
  There are unsatisfiable -CNF formulas with  for all variables .
\label{theorem-individual}
\end{theorem}



\section{Proofs}\label{section-proofs}

For a truth assignment  and a clause , we will write
 if  satisfies , and  if it does not.  Similarly, if  satisfies a formula , we
write . We begin by stating a version of the
Lopsided Lov\'asz Local Lemma formulated in terms of satisfiability.
See~\cite{SZ2008} for a derivation of this version.

\begin{lemma}[SAT version of the Lopsided Lov\'asz Local Lemma]
  Let  be a CNF formula not containing the empty clause.  Sample a
  truth assignment  by independently setting each variable 
  to {\em \texttt{true}} with some probability .  If
  for any clause , it holds that
 
 then  is satisfiable.
\label{corollary-llll}
\end{lemma}

It is not possible to apply
Lemma~\ref{corollary-llll} directly to a formula  which we want to prove
being satisfiable. Instead, we apply it to a formula  we obtain
from  in the following way:

\begin{definition}
  Let  be a CNF formula. A {\em truncation} of  is a CNF formula
   that is obtained from  by deleting some literals from some
  clauses.
\end{definition}

For example,  is a truncation of
.  A
truncation of a -CNF formula is not necessarily a -CNF formula
anymore. Any truth assignment satisfying a truncation  of  also
satisfies . In our proofs, we will often find it easier to apply
Lemma~\ref{corollary-llll} to a special truncation of  than to 
itself. We need a technical lemma on the binomial coefficient.
 
\begin{lemma}\label{lem:technical}
  Let  with . Then
  
\end{lemma}

\begin{proof}
  The upper bound is trivial and true for all . The lower bound
  follows like this.

where we used the fact that  for .
\end{proof}


\subsection{Proof of Theorem~\ref{tradeoff-exponential} and~\ref{theorem-individual}}

As we have argued in Section~\ref{section-first-attempt}, in order to
improve significantly upon the upper bound , we must
construct a formula that is very unbalanced, i.e.   is
exponentially larger than .
First, we will construct an unsatisfiable CNF formula with -clauses
and some smaller clauses.
In a second step, we expand all
clauses to size .


\begin{definition}
  Let  be a CNF formula with clauses of size at most .  For each
  -clause  with , construct a complete -CNF
  formula  over  new variables
  . We replace  by . Using distributivity, we expand it into a
  -CNF formula  called a \emph{-CNFification of }.
\end{definition}

For example, a -CNFification of  is . A truth
assignment satisfies  if and only if it satisfies its
-CNFification .

\begin{definition}
  Let . An -CNF formula is a
  formula consisting of -clauses containing only positive
  literals, and -clauses containing only negative literals.
\end{definition}

If  is an -CNF formula, we write ,
where  consists of the positive -clauses and  of
the negative -clauses.

\begin{proposition}
  Let  and let  be an -CNF
  formula.  Let  be the -CNFification of . Then
  \begin{itemize}
    \item[(i)] ,
    \item[(ii)] .
  \end{itemize}
\label{prop-lk}
\end{proposition}

\begin{proof}
  To prove (i), note that every edge in  runs between a
  positive -clause  and a negative -clause . Thus,
  .  In , this edge is replaced by
   edges, since  is replaced by  copies.
  This explains the term . Replacing  by
   many -clauses introduces at most  new
  conflicts. This explains the term , and proves (i).
  To prove , there are two cases. First, if  appears in ,
  then  and , thus . Second, if  appears in , but not
  in , then , and
  .
\end{proof}

We should explore for which values of  and  there are
unsatisfiable -CNF formulas. We can then use
Proposition~\ref{prop-lk} to derive upper bounds.




\begin{lemma}\label{lem:t}
  For any , there is a constant  such that for all
   and , there exists an unsatisfiable
  -CNF formula  with  and . \\
\end{lemma}

\begin{proof}
  We choose a set variables  of 
  variables.  There are  -clauses over  containing
  only negative literals. We form  by sampling 
  of them, uniformly with replacement, and similarly, we form  by
  sampling  purely positive -clauses,
  where  is some suitable constant determined later. Set .  We claim that with high probability,  is
  unsatisfiable.  Let  be any truth assignment. There are two
  cases.
  
  \emph{Case 1.}  sets at least  variables to
  \texttt{true}.  For a random negative clause ,
  
  The last inequality follows from Lemma~\ref{lem:technical}. Since we
  select the clauses of  independently of each other, we obtain
  
  provided we chose  large enough, i.e., .
  
  {\em Case 2}:  sets at most  variables to
  \texttt{true}. Now a similar calculation shows that 
  satisfies  with probability at most .

  In any case, . The expected
  number of satisfying assignments of  is at most  and with high probability  is unsatisfiable.
\end{proof}

The bound in Lemma~\ref{lem:t} is tight up to a polynomial factor in
:

\begin{lemma}
 Let  be an -CNF formula.
  If there is a  such that  and ,
  then  is satisfiable.
\end{lemma}

\begin{proof}
  Sample a truth assignment
   by setting each variable independently to \texttt{true}
  with probability . For a negative -clause , it holds
  that . Similarly, for a
  positive -clause , .  Hence the expected number of clauses in  that
  are unsatisfied by  is .
  Therefore, with positive probability  satisfies .
\end{proof}








\begin{proof}[Proof of Theorem~\ref{tradeoff-exponential}]
   Apply Lemma~\ref{lem:t} with  and .\\
  
   We fix some probability , and set every variable of  to \texttt{true} with
  probability , independent of each other. This gives a random
  truth assignment . We define a truncation  of  as
  follows: For each clause , if at least half the literals of
   are negative, we remove all positive literals from  and
  insert the truncated clause into , otherwise we insert  into
   without truncating it.  We write , where
   consists of purely negative clauses of size at least
  , and  consists of -clauses, each containing at
  least  positive literals.  A clause in  is
  unsatisfied with probability at most , and a clause
  in  with probability at most . This is because in the worst case, half of all
  literals are negative: Since , negative literals
  are more likely to be unsatisfied than positive ones.  Let  be any clause. A positive literal  causes conflicts
  between  and the  clauses
  of  containing . Similarly, a negative literal  causes conflicts with the at most  clauses of
   containing . Therefore
  
  since  and 
  .
  By Lemma~\ref{corollary-llll},  is satisfiable. 
\end{proof}

Part  of Theorem~\ref{tradeoff-exponential} can easily be improved
by defining a more careful truncation procedure: We remove all
positive literals from a clause  if  contains less than  of them, for some . Choosing  and 
optimally, we obtain a better result, but the calculations become
messy, and it offers no additional insight. The crucial part of the
proof is that by removing positive literals from a clause, we can use
the fact that  is small to bound the number of
clauses  that conflict with  and have a large probability of
being unsatisfied.  This is also the main idea in our proof of the
lower bound of Theorem~\ref{main}.
It should be pointed out that for , an -CNF
formula is just a monotone -CNF formula. The size of a smallest
unsatisfiable monotone -CNF formula is the same---up to a factor of
at most ---as the minimum number of hyperedges in a -uniform
hypergraph that is not -colorable. In 1963,
Erd\H{o}s~\cite{Erdos1963} raised the question what this number is,
and proved lower bound of  (this is easy, simple choose a
random -coloring). One year later, he~\cite{Erdos1964} gave a
probabilistic construction of a non--colorable -uniform
hypergraph using  hyperedges. For  and
, the statement and proof of Lemma~\ref{lem:t} are
basically the same in~\cite{Erdos1964}.


 \begin{proof}[Proof of Theorem~\ref{theorem-individual}]
   Combining Lemma~\ref{lem:t} and Proposition~\ref{prop-lk}, we
   conclude that for any  and ,
   there is an unsatisfiable -CNF formula  with
   
   for every variable . The constant  depends on , but not
   on  or . For fixed , the term  is minimized for .
   Choosing , we get  and .
 \end{proof}

 
 
\subsection{Poof of the Main Theorem}\label{section-lower-bound}


 \begin{proof}[Proof of the upper bound of Theorem~\ref{main}]
   As in the previous proof, Proposition~\ref{prop-lk} together
   with Lemma~\ref{lem:t} yield an unsatisfiable -CNF formula  
   with 
   
   For  and , we obtain .
 \end{proof}
 



\begin{proof}[Proof of the lower bound in Theorem~\ref{main}]
  Let  be an unsatisfiable -CNF and let  be the number of
  conflicts in . We will show that . In the proof,  denotes a variable and
   a positive or negative literal.  We assume  for all variables . We can do so since otherwise we
  just replace  by  and vice versa. This changes neither
  , nor satisfiability of . Also we can assume that
   and  are both at least , if 
  occurs in  at all. For , we define
  
  and set  to \texttt{true} with probability  independently
  of all other variables yielding a random assignment . Since
  , we have . We set . By definition, .  Let us list some
  properties of this distribution.  First, if  for
  some literal , then  is a negative literal , and
  . Second,
  if , then both
   and
   hold.
  We distinguish two types of clauses: {\em Bad} clauses, which
  contain at least one literal  with , and {\em
    good} clauses, which contain only literals  with .
Let  denote the set of bad clauses and  the set of good clauses.
  

  \begin{lemma}
   .
   \label{prop-bad}
  \end{lemma}
  \begin{proof}
    For each clause , let  be the literal in
     minimizing , breaking ties arbitrarily. This means
    . Since  is a
    bad clause, ,  is a negative literal
    , and .
    Thus
    
    Since clause  contains , it conflicts with all
     clauses containing , thus .  The factor  arises
    since we count each conflict possibly twice, once from each side.
    Combining this with (\ref{ineq-bad-2}) proves the lemma.
  \end{proof}


  We cannot directly apply Lemma~\ref{corollary-llll} to .
  Therefore we apply the below sparsification process to .
\begin{algorithm}[h]
\Titleofalgo{\ Sparsification Process}
\SetTitleSty{}{}
Let 
be the set of good clauses in .\\
\vspace{1mm}
 \While{}
 {
   \vspace{1mm}
   Let  be some clause maximizing 
   
   among all clauses .\\
   \vspace{1mm}
   \\
   
   \vspace{1mm}
 }
 \Return{}
\end{algorithm}


\begin{lemma}
  Let  be the result of the sparsification process. If  does
  not contain the empty clause, then  is satisfiable.
  \label{prop-F'}
\end{lemma}

\begin{proof}
  We will show that (\ref{ineq-sum}) applies to . Fix a clause . After the sparsification process, every literal 
  fulfills .  Therefore, the terms , for all good clauses  conflicting with , sum up to at
  most . By Lemma~\ref{prop-bad}, the terms  for all bad clauses  also sum up to at most
  .  Hence (\ref{ineq-sum}) holds, and by
  Lemma~\ref{corollary-llll},  is satisfiable, and clearly
   as well.  
\end{proof}

Contrary, if  is unsatisfiable, the sparsification process produces
the empty clause.  We will show that  is large. There is some  all whose literals are being deleted during the
sparsification process. Write , and
order the  such that . One checks that this implies that .  Fix any  and let
 be the first literal among  that is deleted
from . Let  denote what is left of  just before that
deletion, and consider the set  at this point of time.
Then . By the
definition of the process,
  
  
  Since  for all
  literals  in a good clause, it follows that ,
  for every .\\
  
  Let  be any sequence
  satisfying the  inequalities  for all , for example,
  the  are such a sequence. We want to make the  as
  small as possible: If (i)  and (ii)
  , we can
  decrease  until one of (i) and (ii) becomes an equality.
  The other  inequalities stay satisfied. In the end we get a
  sequence  satisfying  whenever . This
  sequence is non-decreasing: If , then
  , and , a
  contradiction. \\

  If all  are , then the \textsuperscript{th}
  inequality yields , and we are done. Otherwise,
  there is some . For
   both  and  are greater than
  , thus
  ,
  and . We define
  
  thus . By  we denote , the -fold iterated application of
  , with . We obtain  for .  By Part (v) of
  Proposition~\ref{prop-ell}, ,
  thus . Therefore , and the \textsuperscript{st} inequality reads
  as
  
  We obtain . How large is
  ? Define  By Part
  (v) of Proposition~\ref{prop-ell} (see appendix),  is finite.
  Since , we
  conclude that , thus .
  
  \begin{lemma}
    The sequence  converges to 
    .
    \label{prop-alpha}
  \end{lemma}
  
  The proof of this lemma is technical and not related to
  satisfiability. We prove it in the appendix. We conclude that .
\end{proof}

\section{Conclusion}

We want to give some hindsight why a sparsification procedure is
necessary in both lower bound proofs in this paper.  The probability
distribution we define is not a uniform one, but biased towards
setting  to \texttt{true} if .  Let
 be a clause containing . It conflicts with all clauses
containing . It could happen that in all those clauses, 
is the only literal with . In this case, each such
clause is unsatisfied with probability not much smaller than ,
and the sum (\ref{ineq-sum}) is greater than . By
removing  from these clauses, we reduce the number of clauses
conflicting with , making the sum (\ref{ineq-sum}) much smaller.
However, for other clauses , this sum might increase by removing
. We think that one will not be able to prove a tight lower bound
using just a smarter sparsification process. We state some open
problems and questions.

\begin{quotation}
{\em Question:} Does  exist?
\end{quotation}

If it does, it lies between  and . One way to prove
existence would be to define ``product'' taking a -CNF formula 
and an -CNF formula  to a -CNF formula 
that is unsatisfiable if  and  are, and . With  and  ruled out, there seems to be no obvious
guess for the value of the limit. What about ,
the geometric mean of  and ? 

\begin{quotation}
  {\em Question:} Is there an  such that every unsatisfiable
  -CNF formula contains a variable  with ?
\end{quotation}

Where do our methods fail to prove this? The part in the proof of the
lower bound of Theorem~\ref{main} that fails is Lemma~\ref{prop-bad}.
On the other hand, Lemma~\ref{prop-bad} proves more than we need for
Theorem~\ref{main}: It proves that , summed up
over {\em all} bad clauses gives at most . We only need
that the bad clauses conflicting with a specific clause sum up to at
most . Still, we do not see how to apply or extend our
methods to prove that such an  exists.\\

We discussed lower and upper bounds on the minimum of several parameters
of unsatisfiable -CNF formulas. The following table lists them
where bounds labeled with an asterisk are from this paper and
unlabeled bounds are not attributed to any specific paper. 

\begin{table}[htbp]
  \centering
  \begin{tabular}{lllll}
    parameter & notation & lower bound & upper bound\\\hline
    occurrences of a literal &  & 1 & 1\\
    occurrences of a variable &  & 
    ~\cite{KST1993} & ~\cite{Gebauer2009}\\
    local conflict number &  & ~\cite{KST1993} & \\
    conflicts caused by a variable \hspace{2mm} &  & 
    ~\cite{KST1993} & \\
    global conflict number &  &  & 
  \end{tabular}
\end{table}

\bibliographystyle{abbrv}
\bibliography{refs}


\newpage
\appendix

\section{Proof of Lemma~\ref{prop-alpha}}

\begin{proposition}
  Let  and  with . For , the following statements hold.
  \begin{itemize}
    
  \item[(i)]  attains its unique maximum at
    .
  \item[(ii)] , and  if and only if .
  \item[(iii)] For , .
  \item[(iv)] For  and ,
    .
  \item[(v)] For  and any , .
  \end{itemize}
  \label{prop-ell}
\end{proposition}

\begin{proof}
   follows from elementary calculus.  holds since
   is less than  for all . For ,
   follows from , and for greater , it follows from
   and induction on .  holds because each of the
   applications of  multiplies its argument with a factor
  that is at least  and at most .  Suppose  does not hold. Then by (iii) we get
  , and by , we have
  
  An elementary calculation shows that this does not hold for any 
  .
\end{proof}


To prove Lemma~\ref{prop-alpha}, we compute  (and show that the limit exists). Recall
the definition

where .  By Part  of
Proposition~\ref{prop-ell}, , for . We
generalize the definition of  by defining for ,

Further, we set .
 Let . We want to estimate
. This should be small if  is small.
For brevity, we write , . Clearly . We calculate

Where we used part  of Proposition~\ref{prop-ell}.  In fact, these
inequalities also hold if , when :

One checks that the inequalities even hold if . Note that .  Solving for
, the above inequalities yield

for all . The right inequality also holds for . Multiplying with , we see that it also holds if
. If , it is trivially true.
Hence this inequality is true for all .\\

Suppose  exists, for every fixed .
Inequality (\ref{ineq-sk}) also holds in the limit. Writing 
and  and dividing (\ref{ineq-sk}) by  gives

Letting  go to , we obtain , thus
. Observing that
 and  for all 
proves the Lemma. \\

The above argument shows that if  converges pointwise, then it
converges to a continuous function  on . We have to show
that  does in fact exist.  First
plug in  into the right inequality of (\ref{ineq-sk}) to
observe that for each fixed , the sequence  is bounded from above. Clearly it is bounded from below
by . Hence there exist  and similarly
. We write shorthand  and . Now (\ref{ineq-sk}) reads as 
. We claim that



For sequences , ,
 does not hold in
general, hence the claim is now completely trivial. We will proof that
.  This will
prove one claimed inequality. The other three inequalities can be
proven similarly.  Fix some small .  For all
sufficiently large , .  We have
 for infinitely many , thus
 for infinitely many .  Therefore .  By making 
arbitrarily small, the claimed inequality follows.\\

We can now apply our non-rigorous argument from above, this time
rigorously. Write , , and divide (\ref{ineq-us}) and
(\ref{ineq-ls}) by , send  to , and we obtain . Since , we obtain








\end{document}
