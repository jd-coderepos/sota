\documentclass[letter, 11pt]{article}
\usepackage{amsmath} \usepackage{amssymb}  \usepackage{amsthm}
\usepackage[dvips]{graphicx}
\usepackage{times}
\usepackage{multirow}
\textheight=9in
\textwidth=6.5in
\headheight=0mm \headsep=0mm
\topmargin=0mm 
\oddsidemargin=0mm \evensidemargin=0mm
\pagestyle{plain} 

\newcommand{\Order}{\mathrm{O}}
\newcommand{\Prob}{\mathrm{Pr}}
\newcommand{\Exp}{\mathrm{E}}
\newcommand{\order}{\mathrm{o}}
\newcommand{\poly}{\mathrm{poly}}
\newcommand{\INPUT}{\mathrm{INPUT}}
\newcommand{\defeq}{\stackrel{\mbox{\scriptsize{\normalfont\rmfamily def. }}}{=}}
\newcommand{\e}{\mathrm{e}}
\newcommand{\shortqed}{\hfill \mbox{} \smallskip}
\renewcommand{\Vec}[1]{\mbox{\boldmath }}
\newcommand{\1}{\mbox{1}\hspace{-0.25em}\mbox{l}}
\newcommand{\MDG}{\mathcal{G}} \newcommand{\ME}{\mathcal{E}} 
\newcommand{\numberT}{M} 
\newcommand{\sumDconf}{X}
\newcommand{\Li}{L}\newcommand{\dsig}{\Psi_\sigma} 
\newcommand{\MP}{\mathbf{P}}\newcommand{\MQ}{\mathbf{Q}}\newcommand{\MI}{\mathbf{I}}\newcommand{\Z}{Z}
\newcommand{\D}{D}
\newcommand{\I}{{\cal I}} \newcommand{\C}{{\cal C}} \newcommand{\dtv}{{\cal D}_{\rm tv}}
\newcommand{\dpw}{{\cal D}_{\rm pw}}
\newcommand{\tmix}{\tau}
\newcommand{\GCD}{{\rm GCD}}

\newtheorem{theorem}{Theorem}[section]
\newtheorem{lemma}[theorem]{Lemma}
\newtheorem{corollary}[theorem]{Corollary}
\newtheorem{proposition}[theorem]{Proposition}
\newtheorem{observation}[theorem]{Observation}

\newtheorem{definition}[theorem]{Definition}
\newtheorem{example}{Example}[section]
\newtheorem{algo}{Algorithm}
\allowdisplaybreaks[2]\title{Deterministic Random Walks for Rapidly Mixing Chains
}
\author{
 Takeharu Shiraga\footnote{
   Graduate School of Information Science and Electrical Engineering, 
   Kyushu University, Fukuoka, Japan\protect\\ 
{\ttfamily \{takeharu.shiraga,yamauchi,kijima,mak\}@inf.kyushu-u.ac.jp}} \and 
 Yukiko Yamauchi\footnotemark[1] \and 
 Shuji Kijima\footnotemark[1] \and
 Masafumi Yamashita\footnotemark[1]
}
\begin{document}
\makeatletter
\makeatother
\maketitle
\begin{abstract}
The rotor-router model is a deterministic process analogous to a
simple random walk on a graph.
 This paper is concerned with a generalized model, {\em
functional-router model},
   which imitates a Markov chain possibly containing irrational transition probabilities.
 We investigate the discrepancy of the number of tokens at a single vertex
    between the functional-router model and its corresponding Markov chain, and 
   give an upper bound in terms of the mixing time of the Markov chain. 


\smallskip
\noindent
{\bf Key words}: 
  rotor-router model, 
  Markov chain Monte Carlo, 
  mixing time. 
\end{abstract}


\section{Introduction}
The \textit{rotor-router model}, also known as the {\em Propp machine}, 
  is a deterministic process analogous to a random walk on a graph~\cite{PDDK96, CS06, KKM12}. 
 In the model\footnote{See Section~\ref{sec:rotor-router}, for the detail of the rotor-router model. }, 
   tokens distributed over vertices are deterministically served to neighboring vertices 
    by rotor-routers equipped on vertices, instead of traveling on the graph at random. 
Doerr et al.~\cite{CDST07, DF09} first called the rotor-router model {\em deterministic random walk}, 
  meaning a ``derandomized, hence {\em deterministic}, version of a {\em random walk}.'' 


\paragraph{Single vertex discrepancy for multiple-walk.}
Cooper and Spencer~\cite{CS06} investigated 
  the rotor-router model (with multiple tokens, in precise; {\em multiple-walk}) on , and  
   gave an analysis on the discrepancy on a single vertex: 
 they showed a bound that , 
  where 
    (resp.\ ) denotes 
   the number (resp.\ the expected number) of tokens on vertex  
   in a rotor-router model (resp.\ in the corresponding random walk) at time  
   on the condition that  for any , and 
    is a constant depending only on  but independent of the total number of tokens in the system. 
 Cooper et al.~\cite{CDST07} showed , and  
 Doerr and Friedrich~\cite{DF09} showed that 
   is about 7.29 or 7.83 depending on the routing rules. 
On the other hand, 
 Cooper et al.~\cite{CDFS10} gave 
  an example of  on infinite -regular trees, 
  the example implies that the discrepancy can get infinitely large as increasing the total number of tokens. 

Motivated by a derandomization of Markov chains, 
 Kijima et al.~\cite{KKM12} are concerned with multiple-walks 
  on general finite multidigraphs , and 
  gave a bound  
  in case that corresponding Markov chain is ergodic, reversible and lazy. 
 They also gave some examples of . 
Kajino et al.~\cite{KKM13} 
  sophisticated the approach by~\cite{KKM12}, and 
  gave a bound in terms of the second largest eigenvalue and eigenvectors of the corresponding Markov chain, 
  for an arbitrary irreducible finite Markov chain, 
    which may not be lazy, reversible nor aperiodic. 

 In the context of load balancing, 
  Rabani et al.~\cite{RSW98} are concerned with a deterministic algorithm 
  similar to the rotor-router model corresponding to Markov chains with {\em symmetric} transition matrices, and 
  gave a bound , where 
    denotes the maximum degree of the transition diagram and 
    denotes the second largest eigenvalue of the transition matrix.  

For some specific finite graphs, 
  such as hypercubes and tori, some bounds on the discrepancy in terms of logarithm of the size of transition diagram 
  are known. 
For -dimensional hypercube, 
  Kijima et al.~\cite{KKM12} gave a bound , and 
  Kajino et al.~\cite{KKM13} improved the bound to . 
Recently, 
  Akbari and Berenbrink~\cite{AB13} gave a bound , 
  using results by Friedrich et al.~\cite{FGS12}. 
Akbari and Berenbrink~\cite{AB13} also gave a bound  
  for constant dimensional tori.  
Those analyses highly depend on the structures of the specific graphs, and 
it is difficult to extend the technique to other combinatorial graphs.  
Kijima et al.~\cite{KKM12} gave rise to a question 
  if there is a deterministic random walk 
   for {\#}P-complete problems, such as - knapsack solutions, bipartite matchings, etc., 
  such that  is bounded by a polynomial in the input size.  

\paragraph{Other topics on deterministic random walk.}
As a highly related topic, 
 Holroyd and Propp~\cite{HP10} analyzed 
   ``hitting time'' of the rotor-router model 
   with a {\em single token} ({\em single-walk}) on finite simple graphs, and  
  gave a bound  
  where  denotes the frequency of visits of the token at vertex  in  steps, and 
   denotes the stationary distribution of the corresponding random walk. 
Friedrich and Sauerwald~\cite{FS10} 
   studied the cover time of a single-walk version of the rotor-router model 
   for several basic finite graphs such as tree, star, torus, hypercube and complete graph. 
Recently, Kosowski and Pajak~\cite{KP14} studied the cover time of a multiple tokens version of the rotor-router model. 

Holroyd and Propp~\cite{HP10} also proposed a generalized model called {\em stack walk}, 
   which is the first model of deterministic random walk for {\em irrational} transition probabilities, as far as we know. 
While Holroyd and Propp~\cite{HP10} indicated the existence of routers 
   approximating irrational transition probabilities well, 
 Angel et al.~\cite{AJJ10} gave a routing algorithm based on the ``shortest remaining time (SRT)'' rule. 
Shiraga et al.~\cite{shiraga}, that is a preliminary work of this paper, 
  independently proposed another model based on the van der Corput sequence, 
  motivated by irrational transition probabilities, too. 

As another topic on the rotor-router model, 
  the aggregation model has been investigated~\cite{LP05, Kleber05, LP08, LP09}. 
 For a random walk, 
  tokens in the Internal Diffusion-Limited Aggregation (IDLA) model on  
   asymptotically converge to the Euclidean ball~\cite{LBG92}, and  
 Jerison, Levine and Sheffield~\cite{JLS12} recently showed  
  the fluctuations from circularity are  after  steps.
 For the rotor-router model, 
  Levine and Peres~\cite{LP05, LP08, LP09} showed that 
  tokens in the rotor-router aggregation model also form the Euclidean ball, and  
  showed several bounds for the fluctuations. 
Kleber~\cite{Kleber05} gave some computational results. 

Doerr et al.~\cite{DFS08} showed that 
  information spreading by the rotor-router model is faster than the one by a random walk 
 on some specific graphs, 
  namely trees with the common depth and the common degree of inner vertices, and 
  random graphs with restricted connectivity.  
 Doerr et al.~\cite{DFKS09} 
   gave some computational results for this phenomena. 
 There is much other research on information spreading 
  by the rotor-router model on some graphs~\cite{ADHP09, Doerr09, DFS09a, DFS09b, DHL09, HF09}.

\paragraph{Our Results.}This paper is concerned with the {\em functional-router} model (of multiple-walk ver.), 
   which is a generalization of the rotor-router model. 
 While the rotor-router model is an analogy with a simple random walk on a graph, 
  the functional-router model imitates a Markov chain 
  possibly containing irrational transition probabilities. 
 In the functional-router model, 
  a configuration of  tokens over a finite set  
  is deterministically updated by functional-routers defined on vertices\footnote{
    See Section~\ref{sec:fr-model}, for the detail of the functional-router model.}. 
 Let  denote 
   the configuration at time , 
   i.e., . 
 For comparison, 
  let , and 
  let  for a transition matrix  corresponding to the functional-router model, 
 then 
   denotes the expected configuration of  tokens 
  independently according to  for  steps. 
 A main contribution of the paper is to show that  
    
   holds for any  at any time  
   in case that the corresponding transition matrix  is {\em ergodic} and {\em reversible}, 
  where 
   and  are respectively the maximum/minimum values of the stationary distribution  of , 
   is the {\em mixing rate} of , and 
   is the maximum degree of the transition diagram. 

An example of a random walk containing irrational transition probabilities is 
  the -random walk devised by Ikeda et al.~\cite{IKY09}, 
  which achieves an  hitting time and an  cover time {\em for any graphs}. 
 Another example should be the Markov chain Monte Carlo (MCMC), 
   such as Gibbs samplers for the Ising model (cf.\ \cite{Sinclair93, PW96}), 
   reversible Markov chains for queueing networks~(cf.~\cite{KM08}), etc.  


\paragraph{Organization}
 This paper is organized as follows.  
 In Section~\ref{sec:MCMC}, 
   we briefly review MCMC, 
    as a preliminary. 
 In Section~\ref{sec:modelresult}, 
  we describe the functional-router model and our main theorem.   
 In Section~\ref{sec:main}, 
   we prove the main theorem. In Section~\ref{sec:routingmodel}, 
   we present four particular functional-router models, and give detailed analyses on them. 
 In Section~\ref{sec:applications}, 
   we show some examples of the bounds 
    for some Markov chains over the combinatorial objects, 
   which are known to be rapidly mixing. 


\section{Preliminaries: Markov Chain Monte Carlo}\label{sec:MCMC}
As a preliminary step of explaining the functional-router model, this section briefly reviews the Markov chain Monte Carlo (MCMC). 
 See e.g., \cite{Sinclair93, LPW08, MT06} for details of MCMC. 

 Let  be a finite set, and 
  suppose that we wish to sample from  with a probability 
  proportional to a given positive vector ; 
  for example, we are concerned with {\em uniform} sampling of - knapsack solutions in Section~\ref{sec:knapsack}, 
   where  denotes the set of - knapsack solutions and  for each . 
The idea of a Markov chain Monte Carlo (MCMC) is 
   to sample from a limit distribution of a Markov chain 
   which is equal to the target distribution  
   where  is the normalizing constant. 

Let  be 
   a transition matrix of a Markov chain with the state space , 
  where  denotes the transition probability from  to  (). 
A transition matrix  is {\em irreducible} if  for any  and  in , and 
  is {\em aperiodic} if  holds for any , 
  where  denotes the  entry of , the -th power of . 
 An irreducible and aperiodic transition matrix is called {\em ergodic}. 
It is well-known for a ergodic , 
  there is a unique {\em stationary distribution} , 
   i.e., , 
  and the limit distribution is , 
   i.e.,  for any probability distribution  on . 

An ergodic Markov chain defined by a transition matrix  is {\em reversible} 
  if the {\em detailed balance equation} 

  holds for any . 
 When  satisfies the detailed balance equation, 
  it is not difficult to see that  holds, 
  meaning that  is the limit distribution (see e.g., ~\cite{LPW08}). 
Let  and  be a distribution on , 
then the {\em total variation distance}  between  and  is defined  by  

 Note that , since  and  are equal to one, respectively. 
The {\em mixing time} of a Markov chain is defined by 

 for any , 
  where   denotes the -th row vector of ; 
  i.e.,  denotes the distribution of 
   a Markov chain at time  
   stating from the initial state . 
In other words, 
   the distribution  of the Markov chain after  transition 
   satisfies , 
  meaning that we obtain an approximate sample from the target distribution. 

For convenience, let
  
 for , then 
 it is well-known that 
   satisfies a kind of {\em submultiplicativity}. 
 We will use the following proposition in our analysis in Section~\ref{sec:main}. 
 See Appendix~\ref{sec:RMC} for the proof (cf. \cite{LPW08,MT06}). 
\begin{proposition}\label{prop:dltimes}
  For any integers    and 
   , 

 holds for any  . 
\shortqed
\end{proposition} 
Since the submultiplicativity, , called {\em mixing rate}, is often used as a characterization of . 

\section{Model and Main Results}\label{sec:modelresult}A {\em functional-router model} is 
   a  deterministic process analogous to a multiple random walk. 
 Roughly speaking, 
  a router defined on each vertex  deterministically serves tokens to  
  at a rate of  in a functional-router model, 
  while tokens on a vertex  moves to a neighboring vertex  with probability  in a (multiple) random walk.

To get the idea, 
  let us start 
   with explaining the rotor-router model (see e.g., ~\cite{CS06, KKM12}), 
   which corresponds to a simple random walk on a graph. 
\subsection{Rotor-router model}\label{sec:rotor-router}
Let  be a simple undirected graph\footnote{
  In Section~\ref{sec:roter}, we are concerned with a model on multidigraphs. 
}, where . 
 Let  denote 
  the neighborhood of . 
 For convenience, 
  let . 
 Let  be an initial configuration of tokens, and  
 let  denote the configuration of tokens 
   at time~ in the rotor-router model. 
 A configuration  
   is updated by {\em rotor-routers} on vertices, as follows. 
 Without loss of generality, 
   we may assume that an ordering  is defined on  for each . 
 Then, a rotor-router  on  is defined by 

 for . 
 Let

  for , 
 where  denotes the number of tokens served from  to  in the update. 
 Then,  is defined by 

  for each . 

It is not difficult to see that 

  holds, which implies that  
  the ``outflow ratio'' 
    of tokens at  to  
 approaches asymptotically to  as  increasing. 
 Thus, the rotor-router hopefully approximates a distribution of tokens by a random walk. 

\subsection{Functional-router model}\label{sec:fr-model}Let  be 
   a transition matrix of a Markov chain with a state space , 
  where  denotes the transition probability from  to  (). 
 Note that  may be irrational\footnote{  
     e.g., , , , etc.~are allowed.
  }.
 In this paper, we assume that  is {\em ergodic} and {\em reversible} (see Section~\ref{sec:MCMC}). 
Let  
   denote an initial configuration of  tokens over , and 
 let  
   denote the {\em expected} configuration of tokens 
   independently according to  at time , 
   i.e.,  and . 

Let  be the transition digram of~, 
  meaning that . 
 Note that  may contain self-loop edges, and also 
 note that  holds. 
 Let  denote 
  the (out-)neighborhood\footnote{
 Since  is reversible, 
   if and only if , and then 
  we abuse  for in-neighborhood of . 
} of , 
  i.e., , and  
 let . 
 Note that  if . 

Let , and 
 let  denote the configuration of tokens 
   at time~ in the functional-router model. 
 A configuration  
   is updated by {\em functional-routers} 
    defined on each  
  to imitate . 
To be precise, let

for  and for any  satisfying , 
  for convenience. 
 Then, the functional router  on  is designed to minimize 

 for . 
See Section~\ref{sec:routingmodel} for some specific functional-routers. 
Let

  for , 
 where  denotes the number of tokens served from  to  in the update. 
 Then,  is defined by 

 for each . 

We in Section~\ref{sec:routingmodel} give some specific functional-routers, 
  in which the ``outflow ratio'' 
    from  to  
 approaches asymptotically to  as  increases, 
 meaning that the functional-router hopefully approximate a distribution of tokens by a random walk. 


Figure~\ref{fig:one} shows an example of the time evolution of a functional router model. 
 In the example,  and the initial configuration of tokens is . 
 According to the functional router  defined in the figure, 

and then the configuration of tokens is  at time 1.  
 In a similar way, 

  provides . 
\begin{figure}[t]
 \begin{center}
  \includegraphics[width=16.6cm]{functionala2.eps}
 \end{center}
 \caption{An example of a functional router model}
 \label{fig:one}
\end{figure}



\subsection{Main results}\label{subsec:main_results}
Our goal is to estimate the discrepancy  
for  and  for the functional router model described in Section \ref{sec:fr-model}. 
For convenience, let 

depending on a functional-router model , then the following is our main theorem. 
\begin{theorem}
\label{thm:mixupper-vertexds}
 Let  be a transition matrix of 
  a {\em reversible} and ergodic Markov chain with a state space , where  denotes the stationary distribution of  and  denotes the mixing time of  for any . 
Then, the discrepancy between  and  satisfies

for any ,  and , where  denotes the maximum degree of the transition diagram of , {\rm i.e.} . 
\end{theorem}
We remark that 

 holds, 
 since 

 holds by the definition. For instance, the {\em SRT router}, which we will introduce in Section~\ref{sec:greedy}, satisfies , and we obtain the following, from Theorem~\ref{thm:mixupper-vertexds}. 
\begin{theorem}
\label{thm:mixupper-vertexgreedy}
 Let  be a transition matrix of 
  a reversible and ergodic Markov chain with a state space , where  denotes the stationary distribution of  and  denotes the mixing rate of . 
For a SRT router model, the discrepancy between  and  satisfies

for any  and , where  denotes the maximum degree of the transition diagram of , {\rm i.e.} . 
\end{theorem}


See Section~\ref{sec:routingmodel} for detailed arguments on the bounds of  for some specific functional routers. 


\section{Analysis of the Point-wise Distance}\label{sec:main}
This section proves Theorem~\ref{thm:mixupper-vertexds}. 
 Our proof technique is similar to previous works~\cite{CS06, KKM12, RSW98}, in some part. 
To begin with, we establish the following key lemma. 
\begin{lemma}
\label{lemm:maindisc}
 Let  be a transition matrix of 
  a reversible and ergodic Markov chain with a state space , 
  and let  be the stationary distribution of . 
Then, 

holds for any  and for any .
\end{lemma}
\begin{proof}
Remark that

 holds where the last equality follows the assumption . 
It is not difficult to see that 

holds, thus we have

 While 
   
   in \eqref{eq:vchi-vmu2} 
   may not be  in general, 
  remark that 

 holds for any . 
Hence

 holds. 
Since  is reversible,  for any  and 
 holds by definition \eqref{eq:Zvut-vsum}. 
Thus, 

holds, and we obtain the claim.
\end{proof}
Now, we are concerned with {\em reversible} Markov chains, 
and show Theorem~\ref{thm:mixupper-vertexds}. 
\begin{proof}[Proof of Theorem~\ref{thm:mixupper-vertexds}]
By Lemma \ref{lemm:maindisc} and \eqref{def:dsigma}, we obtain that

  holds. 
 Since  is reversible, 
   holds for any  and  in  
 (see Proposition~\ref{prop:reversible} in Appendix~\ref{sec:RMC}). 
 Thus

  where the last equality follows 
  the fact that 
   , 
   by the definition \eqref{def:dtv} of the total variation distance. 
By Proposition \ref{prop:dltimes}, we obtain the following. 
\begin{lemma}
\label{lemm:dtsum}
For any  and for any , 

holds for any  . 
\end{lemma} 

\begin{proof}Let , for convenience. 
 Then,  is at most  for any , by the definition \eqref{def:dtv} of the total variation distance. 
 By Proposition \ref{prop:dltimes}, 

holds, and we obtain the claim.
\end{proof}

Now we obtain Theorem~\ref{thm:mixupper-vertexds} 
from \eqref{eq:mdisc3} and Lemma~\ref{lemm:dtsum}
\end{proof}

\section{Specific Functional-routers}\label{sec:routingmodel}This section shows some functional-router models, 
   namely {\em SRT router} in Section~\ref{sec:greedy}, 
    {\em billiard router} in Section~\ref{sec:billiard}, 
   {\em quasi-random router} in Section~\ref{sec:vander}, and  
   rotor-router on multigraph  in Section~\ref{sec:roter}. 
Using Theorem~\ref{thm:mixupper-vertexds}, we give upper bounds of  for them. 
\subsection{SRT router}\label{sec:greedy}This section introduces {\em SRT router}, 
  which is originally given by Holroyd and Propp~\cite{HP10} and Angel et al.~\cite{AJJ10} by the name of stack-walk. 
The SRT router  () on  is defined, as follows. 
Let 

Then, let  be  minimizing the value 

in all . 
If there are two or more such , then let  be arbitrary one of them. 

 Since ,  
  we can see that  holds 
  for any ,  and , 
  by an induction on . 
 The following theorem is 
  due to Angel et al. \cite{AJJ10} and Tijdeman~\cite{T80}.
\begin{theorem}\cite{T80, AJJ10}
\label{thm:upper-const}
 For any transition matrix , 

 holds for any  and any . 
\end{theorem}
Theorem \ref{thm:upper-const} was firstly given by Tijdeman \cite{T80}, 
   where he gave a slightly better bound 
  , in fact. 
 Angel et al.~\cite{AJJ10} rediscovered Theorem \ref{thm:upper-const} 
   in the context of deterministic random walk (see also~\cite{HP10}), 
  where they also showed a similar statement holds 
  even when the corresponding probability is time-inhomogeneous.  

Theorem~\ref{thm:upper-const} and (\ref{eq:dsigmabound}) imply that 

 holds for the SRT router model. 
 
 
\begin{proof}[Proof of Theorem~\ref{thm:mixupper-vertexgreedy}]
By Theorem~\ref{thm:mixupper-vertexds} and (\ref{bound:greedy}), 

holds, and we obtain the claim. 
\end{proof}

\subsection{Billiard router}\label{sec:billiard}{\em Billiard sequence} is known to be a balanced sequence (cf. \cite{SMK04}). 
This section presents a functional router based on the billiard sequence. 

The billiard sequence is given in a similar to the SRT router, but simpler. 
Let  be  minimizing the value 

in all , and if there are two or more such , then let  be arbitrary one of them. 
Then, the following theorem for the billiard sequence is known. 
\begin{lemma}\label{bound:billiardz}\cite{SMK04}
For any transition matrix , 

 holds for any , and for any  satisfying . 
\end{lemma}
Using Lemma~\ref{bound:billiardz}, we obtain an upper bound of  for the billiard sequence. 
\begin{lemma}\label{bound:billiard}
 holds for the billiard sequence. 
\end{lemma}
Thus, we obtain Theorem~\ref{thm:mixupper-vertexbi}
  by Theorem~\ref{thm:mixupper-vertexds} and Lemma~\ref{bound:billiard}. 
\begin{theorem}
\label{thm:mixupper-vertexbi}
 Let  be a transition matrix of 
  a reversible and ergodic Markov chain with a state space , where  denotes the stationary distribution of  and  denotes the mixing rate of . 
  For a billiard router model, the discrepancy between  and  satisfies

  for any  and , where  denotes the maximum degree of the transition diagram of , {\rm i.e.} . 
\end{theorem}
In fact, we obtain a better bound for the billiard router model as follows, by analyzing carefully. See Appendix~\ref{appendix:billiard} for the proof. 
\begin{theorem}
\label{thm:mixupper-vertexbi2}
 Let  be a transition matrix of 
  a reversible and ergodic Markov chain with a state space , where  denotes the stationary distribution of  and  denotes the mixing rate of . 
  For a billiard router model, the discrepancy between  and  satisfies

  for any  and , where  denotes the maximum degree of the transition diagram of , {\rm i.e.} . 
\end{theorem}

\subsection{Quasi-random router}\label{sec:vander}This section gives a router  
  based on the {\em van der Corput sequence}~\cite{JGV35,N78}, 
 which is a well-known low-discrepancy sequence. 

The van der Corput sequence  is defined as follows.  
Suppose  
  is represented in binary as  
  using  (). 
 Then, we define  

and . 
 For example, 
, 
, 
, 
, 
, 
, 
and so on. 
 Clearly,  holds for any (finite) . 


Now, given , we define  as follows.
 Without loss of generality, 
  we may assume that an ordering  is defined on  for . 
 Then, 
  we define the functional-router  on  
  such that  satisfies that 

  for , 
 where , for convenience. 

The following theorem is due to van der Corput \cite{JGV35}. 
\begin{theorem}\label{thm:upper-logM}\cite{JGV35}
 For any transition matrix , 

 holds for any  and any . 
\end{theorem}
More sophisticated bounds are found in~\cite{N78}. 
Carefully examining Theorem~\ref{thm:upper-logM}, 
  we obtain the following lemma. 
  See Appendix~\ref{appendix:vander} for the proof. 
\begin{lemma}\label{bound:vandercz}
For any transition matrix , 

 holds for any , and for any  satisfying . 
\end{lemma}
Lemma~\ref{bound:vandercz} suggests the following lemma. 
\begin{lemma}\label{bound:vanderc}
 holds for the van der Corput sequence. 
\end{lemma}

By Theorem~\ref{thm:mixupper-vertexds} and Lemma \ref{bound:vanderc}, we obtain the following. 

\begin{theorem}
\label{thm:mixupper-vertexvc}
 Let  be a transition matrix of 
  a reversible and ergodic Markov chain with a state space , where  denotes the stationary distribution of  and  denotes the mixing rate of . 
  For a quasi-random router model, the discrepancy between  and  satisfies

for any  and , where  denotes the maximum degree of the transition diagram of , {\rm i.e.}  and  denotes the total number of tokens on . 
\end{theorem}
Though the bound depends on ,  
    holds in terms of , 
   meaning that the discrepancy approaches asymptotically to zero 
     as increasing the number of tokens . 

\subsection{Rotor-router on multidigraph}\label{sec:roter}
The rotor-router model described in Section~\ref{sec:fr-model} 
   can be generally considered on digraphs with parallel edges (i.e., multidigraphs). 
 Kijima et al.~\cite{KKM12} and Kajino et al.~\cite{KKM13} are concerned with 
   the rotor-router model on finite multidigraphs. 
 Suppose that  is a transition matrix with {\em rational} entries. 
 For each , 
  let 
   be a common denominator (or the least common denominator) 
   of  for all , 
  meaning that  is integer for each . 
 We define a rotor-router 
    arbitrarily 
  satisfying that 

 for any  and . 
Then,  is defined by 


 For the rotor router on a multidigraph, we have , hence
  it is not difficult to observe the following. 
\begin{observation}\label{bound:rotorz}
For any transition matrix , 

 holds for any , and for any  satisfying . 
\end{observation}
Using Observation~\ref{bound:rotorz}, we obtain the following lemma. 
\begin{lemma}\label{bound:rotor}
  holds for the rotor-router model on a multidigraph, where . 
\end{lemma}

 By Theorem \ref{thm:mixupper-vertexds}, and the above lemma, 
  we obtain the following theorem. 
\begin{theorem}
\label{thm:mixupper-vertexrr}
 Let  be a transition matrix of 
  a reversible and ergodic Markov chain with a state space , where  denotes the stationary distribution of  and  denotes the mixing rate of . 
  For a rotor router model, the discrepancy between  and  satisfies

for any  and , where  denotes the maximum degree of the transition diagram of , {\rm i.e.} , and . 
\end{theorem}
Analyzing carefully, we obtain the following upper bound for the weighted rotor router model. See appendix~\ref{appendix:rotor} for the proof.  
\begin{theorem}
\label{thm:mixupper-vertexrr2}
 Let  be a transition matrix of 
  a reversible and ergodic Markov chain with a state space , where  denotes the stationary distribution of  and  denotes the mixing rate of . 
  For a rotor router model, the discrepancy between  and  satisfies

or any  and , where . 
\end{theorem}


\section{Bounds For Rapidly Mixing Chains}\label{sec:applications}
This section shows some examples of 
   bounds suggested by Theorems~\ref{thm:mixupper-vertexgreedy} and \ref{thm:mixupper-vertexbi2} 
  for some celebrated Markov chains known to be rapidly mixing,  
   namely ones for - knapsack solutions (Section~\ref{sec:knapsack}), 
  linear extensions (Section~\ref{sec:linear_extensions}), and 
  matchings (Section~\ref{sec:matching}). 

\subsection{- knapsack solutions}\label{sec:knapsack}
Given  and , 
  the set of  - knapsack solutions is defined by 
  . 
 We define a transition matrix 
    
  by

for , where . 
Note that the stationary distribution of  is uniform distribution since  is symmetric. 
The following theorem is due to Morris and Sinclair~\cite{MS04}. 
\begin{theorem}
\label{thm:knapsackmix}
\cite{MS04}
 The mixing time  of  is 

for any  and for any . 
\end{theorem}



 Thus, Theorem~\ref{thm:mixupper-vertexgreedy} (resp.\ Theorem~\ref{thm:mixupper-vertexbi2}) suggests the following. 
\begin{theorem}\label{thm:knapsackupper}
For the SRT-router model (as well as the billiard-router model) corresponding to , 
 the discrepancy between  and  satisfies

for any  and , where  is an arbitrary constant. 
\end{theorem}


 Let  , for simplicity, 
  then clearly  holds, 
  since  is ergodic (see Section~\ref{sec:MCMC}).  
 By the definition of the mixing time, 
   holds 
 where  denotes the mixing time of~, 
 meaning that  approximates the target distribution  well. 
 Thus, we hope for a deterministic random walk 
  that the ``distribution'' 
   
  approximates the target distribution  well. 
For convenience, 
  a {\em point-wise distance}  
   between  and  satisfying  
  is defined by 



\begin{corollary}
 For an arbitrary  , 
  let the total number of tokens  
   with some appropriate constants  and . 
 Then, 
  the pointwise distance between  and  satisfies

 for any  
   with an appropriate constant , 
 where  is the uniform distribution over~. 
\end{corollary}


\subsection{Linear extensions of a poset}\label{sec:linear_extensions}
Let , and  be a partial order. 
A linear extension of  is a total order  which respects , 
i.e., for all ,  implies . 
Let  denote the set of all linear extensions of . 
We define a relationship  () 
  for a pair of linear extensions  and   
  satisfying that , , and  for all , 
 i.e., 

holds. Then, we define a transition matrix  by

for , where  and . 
Note that  is ergodic and reversible, and 
 its stationary distribution is uniform on ~\cite{BD99}. 
The following theorem is due to Bubley and Dyer~\cite{BD99}. 
\begin{theorem}
\label{thm:linearmix}
\cite{BD99}
For , 
holds for any . 
\end{theorem}

 It is not difficult to see that the maximum degree  (including a self-loop) 
   of the transition diagram . 
 Thus, Theorem~\ref{thm:mixupper-vertexbi2} suggests the following\footnote{
   Theorem~\ref{thm:mixupper-vertexgreedy} also suggests that 
 
for the SRT-router model. 
  }. 
\begin{theorem}\label{thm:linearupper-SRT}
For the billiard-router model corresponding to , 
 the discrepancy between  and  satisfies

for any  and . 
\end{theorem}

\subsection{Matchings in a graph}\label{sec:matching}Counting all matchings in a graph, 
   related to the {\em Hosoya index}~\cite{Hosoya71}, 
  is known to be {\#}P-complete~\cite{Valiant79b}. 
 Jerrum and Sinclair~\cite{JS96} gave a rapidly mixing chain. 
 This section is concerned with a Markov chain for sampling from all matchings in a graph\footnote{ 
Remark that 
  counting all {\em perfect} matchings in a bipartite graph, 
   related to the {\em permanent}, 
  is also well-known {\#}P-complete problem, 
  and  
   Jerrum et al.~\cite{JSV04} gave a celebrated FPRAS 
   based on an MCMC method using annealing.
 To apply our bound to a Markov chain for sampling perfect matchings, 
  we need some assumptions on the input graph (see e.g.,~\cite{Sinclair93,JS96,JSV04}). 
}. 

Let  be an undirected graph, where  and . 
A matching in  is a subset  such that no edges in  share an endpoint. 
Let  denote the set of all possible matchings of . 
Let 
and let .
Then, for , we define  by 

The we define the transition matrix  by 

for any . 
Note that  is ergodic and reversible, and 
  its stationary distribution is uniform on ~\cite{JS96}. 
The following theorem is due to Jerrum and Sinclar~\cite{JS96}.  
\begin{theorem}
\cite{JS96}
\label{thm:matchmix}
For , 
holds for any . 
\end{theorem}

 It is not difficult to see that the maximum degree  (including a self-loop) 
   of the transition diagram . 
 Thus, Theorem~\ref{thm:mixupper-vertexbi2} suggests the following\footnote{
   Theorem~\ref{thm:mixupper-vertexgreedy} also suggests that 
 
for the SRT-router model.  }. 
\begin{theorem}\label{thm:matchingupper}
For the billiard-router model corresponding to , 
 the discrepancy between  and  satisfies

for any  and . 
\end{theorem}



\section{Concluding Remarks}
 This paper has been concerned with the functional-router model, 
   that is a generalization of the rotor-router model, and 
   gave an upper bound of  
   when its corresponding Markov chain is reversible. 
 We can also show a similar bound 
  for a version of functional-router model 
  with oblivious routers (see \cite{shiraga}). 
A bound of the point-wise distance 
   independent of  and/or independent of  
  is a future work. 
Development of deterministic approximation algorithms 
  based on deterministic random walks for {\#}P-hard problems is a challenge. 


\bibliographystyle{abbrv}
\begin{thebibliography}{99}
\bibitem{AB13} 
 H. Akbari and Petra Berenbrink, 
 Parallel rotor walks on finite graphs and applications in discrete load balancing, 
 Proceedings of the 25th ACM symposium on Parallelism in algorithms and architectures (SPAA 2013), 186--195. 

\bibitem{AJJ10} 
 O.~Angel, A.E.~Holroyd, J.~Martin, and J.~Propp, 
 Discrete low discrepancy sequences, 
 arXiv:0910.1077. 
\bibitem{ADHP09} 
 S.~Angelopoulos, B.~Doerr, A.~Huber, and K.~Panagiotou, 
 Tight bounds for quasirandom rumor spreading, 
 The Electronic Journal of Combinatorics, {\bf 16} (2009), {\#}R102.
\if0
\bibitem{BG08} 
 ** A.~Bandyopadhyay and D.~Gamarnik, 
 Counting without sampling: asymptotics of the log-partition function for certain statistical physics models, 
 Random Structures and Algorithms, {\bf 33} (2008), 452--479. 

\bibitem{BGKNT07} 
** M.~Bayati, D.~Gamarnik, D.~Katz, C.~Nair, and P.~Tetali, 
 Simple deterministic approximation algorithms for counting matchings, 
 Proceedings of the thirty-ninth annual ACM symposium on Theory of computing (STOC 2007), 
 122--127. 


\bibitem{BW91} 
 **G.~Brightwell and P.~Winkler, 
 Counting linear extensions, 
 Order, {\bf 8} (1991), 225--242. 
\fi
\bibitem{BD99} 
 R. Bubley and M. Dyer, 
 Faster random generation of linear extensions, 
 Discrete Mathematics, {\bf 201} (1999), 81--88. 

\bibitem{CDFS10} 
 J.~Cooper, B.~Doerr, T.~Friedrich, and J.~Spencer, 
 Deterministic random walks on regular trees, 
 Random Structures \& Algorithms, {\bf 37} (2010), 353--366. 

\bibitem{CDST07} 
 J.~Cooper, B.~Doerr, J.~Spencer, and G.~Tardos, 
 Deterministic random walks on the integers, 
 European Journal of Combinatorics, {\bf 28} (2007), 2072--2090.

\bibitem{CIKK11} 
 C.~Cooper, D.~Ilcinkas, R.~Klasing, and A.~Kosowski, 
 Derandomizing random walks in undirected graphs using locally fair exploration strategies, 
 Distributed Computing, {\bf 24} (2011), 91--99. 

\bibitem{CS06} 
 J.~Cooper and J.~Spencer, 
 Simulating a random walk with constant error, 
 Combinatorics, Probability and Computing, {\bf 15} (2006), 815--822. 

\bibitem{Doerr09} 
 B.~Doerr, 
 Introducing quasirandomness to computer science, 
 Efficient Algorithms, volume 5760 of Lecture Notes in Computer Science, Springer Verlag, (2009), 99--111.

\bibitem{DF09} 
 B.~Doerr and T.~Friedrich, 
 Deterministic random walks on the two-dimensional grid, 
 Combinatorics, Probability and Computing, {\bf 18} (2009), 123--144.

\bibitem{DFKS09} 
 B.~Doerr, T.~Friedrich, M.~K\"{u}nnemann, and T.~Sauerwald, 
 Quasirandom rumor spreading: An experimental analysis, 
 ACM Journal of Experimental Algorithmics, {\bf 16} (2011), 3.3:1--3.3:13.

\bibitem{DFS08} 
 B.~Doerr, T.~Friedrich, and T.~Sauerwald, 
 Quasirandom rumor spreading, 
 Proceedings of the 19th Annual ACM-SIAM Symposium on Discrete Algorithms (SODA 2008), 
 773--781.

\bibitem{DFS09a} 
 B.~Doerr, T.~Friedrich, and T.~Sauerwald, 
 Quasirandom rumor spreading: Expanders, push vs. pull, and robustness, 
 Proceedings of the 36th International Colloquium on Automata, Languages and Programming, 
 (ICALP 2009), 366--377.

\bibitem{DFS09b} 
 B.~Doerr, T.~Friedrich, and T.~Sauerwald, 
 Quasirandom rumor spreading on expanders, 
 Electronic Notes in Discrete Mathematics, {\bf 34} (2009), 243--247.

\bibitem{DHL09} 
 B.~Doerr, A.~Huber, and A.~Levavi, 
 Strong robustness of randomized rumor spreading protocols, 
 Proceedings of the 20th International Symposium on Algorithms and Computation 
 (ISAAC 2009), 812--821.

\bibitem{Dyer} 
 M.~Dyer, 
 Approximate counting by dynamic programming, 
 Proceedings of the 35th Annual ACM Symposium on Theory of Computing (STOC 2003), 693--699. 


\bibitem{FGS12} 
 T.~Friedrich, M.~Gairing, and T.~Sauerwald, 
 Quasirandom load balancing, 
 SIAM Journal on Computing, {\bf 41} (2012), 747--771.

\bibitem{FS10} 
 T. Friedrich and T. Sauerwald, 
 The cover time of deterministic random walks, 
 The Electronic Journal of Combinatorics, {\bf 17} (2010), R167. 



\bibitem{HP10} 
 A.E.~Holroyd and J.~Propp, 
 Rotor walks and Markov chains, 
 M.~Lladser, R.S.~Maier, M.~Mishna, A.~Rechnitzer, (eds.), 
 Algorithmic Probability and Combinatorics, 
 The American Mathematical Society, 2010, 105--126. 

\bibitem{Hosoya71} 
 H.~Hosoya, 
 Topological index. 
 A newly proposed quantity characterizing the topological nature of structural isomers of saturated hydrocarbons, 
 Bulletin of the Chemical Society of Japan, {\bf 44} (1971), 2332--2339. 

\bibitem{HF09} 
 A.~Huber and N.~Fountoulakis, 
 Quasirandom broadcasting on the complete graph is as fast as randomized broadcasting, 
 Electronic Notes in Discrete Mathematics, {\bf 34} (2009), 553--559.

\bibitem{IKY09} 
 S.~Ikeda, I.~Kubo, and M.~Yamashita, 
 The hitting and cover times of random walks on finite graphs using local degree information, 
 Theoretical Computer Science, {\bf 410} (2009), 94--100. 


\bibitem{JLS12}
 D.~Jerison, L.~Levine, and S.~Sheffield, 
 Logarithmic fluctuations for internal DLA, 
 Journal of the American Mathematical Society, {\bf 25} (2012), 271--301.

\bibitem{JS96}
 M.~Jerrum and A.~Sinclair, 
 Approximation algorithms for NP-hard problems, 
 D.S.~Hochbaum ed., 
 The Markov chain Monte Carlo method: an approach to approximate counting and integration, 
 PWS Publishing, 1996. 

\bibitem{JSV04}
  M.~Jerrum, A.~Sinclair, and E.~Vigoda, 
 A polynomial-time approximation algorithm for the permanent of a matrix with nonnegative entries, 
 Journal of the ACM, {\bf 51} (2004), 671--697. 

\bibitem{KKM13} 
 H.~Kajino, S.~Kijima, and K.~Makino, 
 Discrepancy analysis of deterministic random walks on finite irreducible digraphs, 
 discussion paper. 

\bibitem{KKM12} 
 S.~Kijima, K.~Koga, and K.~Makino, 
 Deterministic random walks on finite graphs, 
 Random Structures \& Algorithms, {\em to appear}.

\bibitem{KM08}
 S.~Kijima and T.~Matsui, 
 Approximation algorithm and perfect sampler for closed Jackson networks with single servers, 
 SIAM Journal on Computing, {\bf 38} (2008), 1484--1503. 
 
  \bibitem{KP14} 
A.~Kosowski and D.~Pajak, 
Does Adding More Agents Make a Difference? A Case Study of Cover Time for the Rotor-Router,
Proceedings of the 41st International Colloquium on Automata, Languages and Programming, 
(ICALP 2014), 544--555.

\bibitem{Kleber05} 
 M.~Kleber, 
 Goldbug variations, 
 The Mathematical Intelligencer, {\bf 27} (2005), 55--63.

\bibitem{LBG92} 
 G.F.~Lawler, M.~Bramson, and D.~Griffeath, 
 Internal diffusion limited aggregation, 
 The Annals of Probability, {\bf 20} (1992), 2117--2140.

\bibitem{LP05} 
 L.~Levine and Y.~Peres, 
 The rotor-router shape is spherical, 
 The Mathematical Intelligencer, {\bf 27} (2005), 9--11.

\bibitem{LP08} 
 L.~Levine and Y.~Peres, 
 Spherical asymptotics for the rotor-router model in , 
 Indiana University Mathematics Journal, {\bf 57} (2008), 431--450.

\bibitem{LP09} 
 L.~Levine and Y.~Peres, 
 Strong spherical asymptotics for rotor-router aggregation and the divisible sandpile, 
 Potential Analysis, {\bf 30} (2009), 1--27.

\bibitem{LPW08} 
 D.A.~Levine, Y.~Peres, and E.L.~Wilmer, 
 Markov Chain and Mixing Times, 
 American Mathematical Society, 2008. 

\bibitem{MS04} 
 B.~Morris and A.~Sinclair, 
 Random walks on truncated cubes and sampling 0-1 knapsack solutions, 
 SIAM Journal on Computing, {\bf 34} (2004), 195--226. 


\bibitem{MT06}
 R. Montenegro and P. Tetali, 
 Mathematical Aspects of Mixing Times in Markov Chains, 
NOW Publishers, 2006. 

\bibitem{N78}
 H. Niederreiter, 
 Quasi-Monte Calro methods and pseudo-random numbers, 
 Bull. Amer. Math .Soc., {\bf 84}(1978), 957--1042

\bibitem{PDDK96}
 V.~Priezzhev, D.~Dhar, A.~Dhar, and S.~Krishnamurthy, 
 Eulerian walkers as a model of self-organized criticality, 
 Physical Review Letters, {\bf 77} (1996), 5079--5082. 

\bibitem{PW96}
 J.~Propp and D.~Wilson, 
 Exact sampling with coupled Markov chains and applications to statistical mechanics, 
 Random Structures Algorithms, {\bf 9} (1996), 223--252. 

\bibitem{RSW98} 
 Y.~Rabani, A.~Sinclair, and R.~Wanka, 
 Local divergence of Markov chains and analysis of iterative load balancing schemes, 
 Proc. FOCS 1998, 694--705.

\bibitem{SMK04}
S.~Sano, N.~Miyoshi, R.~Kataoka, 
-Balanced words: A generalization of balanced words, 
Theoretical Computer Science, {\bf 314} (2004), 97--120. 

\bibitem{shiraga}
 T.~Shiraga, Y.~Yamauchi, S.~Kijima, and M.~Yamashita, 
 Deterministic random walks for irrational transition probabilities, 
 IPSJ SIG Technical Reports, 2012-AL-142(2), 2012. (in Japanese). 

\bibitem{SYKY14}
T.~Shiraga, Y.~Yamauchi, S.~Kijima, and M.~Yamashita, 
-discrepancy analysis of polynomial-time deterministic samplers emulating rapidly mixing chains, 
Lecture Notes in Computer Science, {\bf 8591} (COCOON 2014), 25--36. 



\bibitem{Sinclair93} 
 A.~Sinclair, 
 Algorithms for Random Generation \& Counting, A Markov chain approach, 
 Birkh\"{a}user, 1993. 


\bibitem{T80}
R. Tijdeman, The chairman assignment problem, 
Discrete Math. {\bf 32} (1980), 323--330.

\bibitem{Valiant79b} 
 L.G.~Valiant, 
 The complexity of enumeration and reliability problems, 
 SIAM Journal on Computing, {\bf 8} (1979), 410--421. 

\bibitem{JGV35} 
 J.~G.~van der Corput, 
 Verteilungsfunktionen, Proc. Akad. Amsterdam, {\bf 38} (1935), 813--821. 

\end{thebibliography}
\appendix
\section{Fundamental Properties of Markov Chain and Mixing Time}\label{sec:RMC}
\subsection{Proof of Proposition \ref{prop:dltimes}}
In this section, we show Proposition \ref{prop:dltimes} (see e.g., ~\cite{LPW08, MT06}). 
\paragraph{Proposition \ref{prop:dltimes}}
  For any integers    and 
   , 

 holds for any  . 

To begin with, we define 

Then, we show the  following. 
\begin{lemma}
\label{lemm:dbarp}
Let  be arbitrary probability distributions. 
Then, 

holds for any . 
\end{lemma} 
\begin{proof}
By \eqref{def:TV}, 

holds. Notice that 
  , since  and  are probability distributions. 
Thus, 

holds, where the second last equality follows , and we obtain the claim. 
\end{proof}

\begin{lemma}
\label{lemm:ddbarineq}

holds for any . 
\end{lemma} 
\begin{proof}
Let  denote the -th unit vector. 
By Lemma \ref{lemm:dbarp}, 

holds for any , and we obtain . 

By the definition of the total variation distance, 

holds for any . We obtain . 
\end{proof}

\begin{lemma}
\label{lemm:dbarineq}
Suppose a vector  satisfies , then

holds for any . 
\end{lemma} 
\begin{proof}
For convenience, let  be defined by  and . 
Then, 

holds, meaning that 

Since , 

holds by \eqref{eq:num}. By the definition of  and , 

holds. Hence

holds. 
Thus, by (\ref{eq:nueqpm}) and (\ref{eq:npsum}), 

holds, hence  and  are probabilistic distribution, respectively. 
Finally, by Lemma \ref{lemm:dbarp} and (\ref{eq:nupnum}), 

 holds, and we obtain the claim. 
\end{proof}

\begin{lemma}
\label{lemm:smp}

hold for any . 
\end{lemma} 
\begin{proof}
By Lemma \ref{lemm:dbarineq}, 

holds for any , and we get . 
Similarly, 

holds for any , and we get . 
\end{proof}

\begin{proof}[Proof of Proposition \ref{prop:dltimes}]
Using Lemma \ref{lemm:smp}, 

holds. 
By Lemma \ref{lemm:ddbarineq}, 

holds, and we obtain the claim. 
\end{proof}

\subsection{Supplemental proof of Theorem~\ref{thm:mixupper-vertexds}}
We show the following proposition, appearing in the proof of Theorem~\ref{thm:mixupper-vertexds}.  
\begin{proposition}
\label{prop:reversible}
If  is reversible, then 

holds for any  and for any . 
\end{proposition}
\begin{proof}
 We show the claim by an induction of . 
 For , the claim is clear by the definition of reversible. 
 Assuming that  holds for any , 
 we show that  holds for any  as follows;  

We obtain the claim.
\end{proof}

\section{Supplemental proofs in Section~\ref{sec:routingmodel}}
\subsection{Proof of Theorem~\ref{thm:mixupper-vertexbi2}}\label{appendix:billiard}
Remark that

holds using Proposition~\ref{prop:reversible} (cf. recall that the arguments on \eqref{eq:mdisc4} in Section~\ref{sec:main}). 
We also remark that

holds by Lemma~\ref{lemm:dtsum}. 
\begin{proof}[Proof of Theorem~\ref{thm:mixupper-vertexbi2}]
By (\ref{eq:mdisc1}) and Lemma~\ref{bound:billiardz}, we obtain 

By combining (\ref{eq:reverseP}) and (\ref{eq:mixbi}), we obtain

Since  is reversible, we obtain 

and hence

Thus, we obtain the claim by (\ref{eq:apbi1}), (\ref{eq:apbi2}) and (\ref{eq:apbi4}). 
\end{proof}

\subsection{Supplemental Proofs in Section~\ref{sec:vander}}\label{appendix:vander}
This section presents a proof of Theorem~\ref{thm:upper-logM} and Lemma~\ref{bound:vandercz}. 
To begin with, we remark two lemmas concerning the function  defined by~(\ref{eq:def-psi}). 
\begin{lemma}\label{lemmf2}
 For any  and any , 

 holds. 
\end{lemma}
\begin{proof}
 In case of , 
   the claim is easy 
  since
    and 
    holds. 
Suppose that , and that 
   
   is represented in binary as  
   using  (). 
 Then, 
 
  hold, respectively. 
 Let  and 
 let  (, for convenience, 
  then 

 and, we obtain the claim.
\end{proof}
\begin{lemma}\label{lemmf1}
 For any  and , 

holds. 
\end{lemma}
\begin{proof}
 By Lemma\ref{lemmf2}, 

  holds, where the last equality follows   by the definition. 
\end{proof}
Now, we define 

  for  satisfying . 
 For convenience, we define . 
It is not difficult to see that 

 holds for any . 
In general, we can show the following lemma, using Lemmas~\ref{lemmf2} and \ref{lemmf1}. 
\begin{lemma}\label{lemmfxy2}
For any  and for any 

 holds for any . 
\end{lemma}
\begin{proof}
 By Lemma~\ref{lemmf2}, 

  holds. 
 Since  holds for any , 

  holds for each . 
 The observation (\ref{eq:coloF}) implies that 

  holds. 
 Notice that (\ref{eq:lemmfxy2-b}) implies 

for any distinct . 
 Now, the claim is clear by (\ref{eq:lemmfxy2-a}) and (\ref{eq:lemmfxy2-b}). 
\end{proof}

 Note that Lemma~\ref{lemmfxy2} implies that 
  using appropriate  for . 

\begin{lemma}\label{lemmIvu1}
 Let , and let  satisfy . 
 Then, 

holds. 
\end{lemma}
\begin{proof}
 Lemma~\ref{lemmfxy2} and Equation~(\ref{eq:FPSI-explicit}) implies that 
  there exists  such that  and 

 where we assume 

  for convenience. 
 Then, it is clear that 
 
 where  means that . 
By (\ref{eq4}) and (\ref{eq5}), we obtain 

 respectively, and hence we obtain that 

 Since  and 
   (), 

 holds, and we obtain the claim. 
\end{proof}
Using Lemma~\ref{lemmIvu1}, 
  we obtain the following.  
\begin{lemma}\label{theoIvu3}
 Let , , and let  satisfy . 
 Then, 

  holds. \end{lemma}
\begin{proof}
 For simplicity, 
  let 

 for any . 
 Then, notice that 

  holds for any  satisfying . 
 Now, suppose  is represented as 
   in binary, 
  where . 
 Using Lemma~\ref{lemmIvu1}, we obtain that 

 In a similar way, 
  we also have 

 and we obtain the claim. 
\end{proof}
By Lemma~\ref{theoIvu3}, it is not difficult to see 
 Theorem~\ref{thm:upper-logM} and Lemma~\ref{bound:vandercz} holds. 

\subsection{Proof of Theorem~\ref{thm:mixupper-vertexrr2}}\label{appendix:rotor}
By (\ref{eq:mdisc1}) and Observation~\ref{bound:rotorz}, we obtain 

Thus, we obtain the claim by combining (\ref{eq:mixbi}) and (\ref{eq:appro1}).


\end{document}
