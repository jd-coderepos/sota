\documentclass[]{lipics-iclp}
\lipicsheading{year}{numbers}{city}




\theoremstyle{plain}\newtheorem{mainthm}[thm]{Main Theorem}
\theoremstyle{definition}\newtheorem{crucialdef}[thm]{Crucial Definition}
\def\eg{{\em e.g.}}
\def\cf{{\em cf.}}

\begin{document}

\title[Structured Interactive Scores]{Structured Interactive Musical Scores}

\author[lab1]{M. Toro}{Mauricio Toro-Berm\'{u}dez}
\address[lab1]{Universit\'{e} de Bordeaux 1, Laboratoire Bordelais de Recherche en Informatique, B\^{a}timent A30. 351, cours de la Lib\'{e}ration F-33405 Talence cedex, France.}  \urladdr{http://www.labri.fr/perso/mtoro/}  





\keywords{ntcc, ccp, interactive scores, temporal relations, faust, ntccrt, heterogeneous systems, automatic verification}
\subjclass{D. Software, D.1 Programming techniques, D.1.3 Concurrent programming, D.1.5 Logic programming.}




\begin{abstract}
  \noindent Interactive Scores is a formalism for the design and
performance of interactive scenarios that provides temporal
relations (TRs)  among the objects of the scenario.
We can model TRs among objects in Time Stream Petri nets, but it is difficult to represent 
global constraints. This can be done
explicitly in the Non-deterministic Timed Concurrent Constraint (ntcc) calculus.
We want to formalize a heterogeneous system that controls in one subsystem
the concurrent execution of the objects using ntcc, and audio and video processing in the other.
We also plan to develop an automatic verifier for ntcc.
\end{abstract}

\maketitle


\section*{Introduction}\label{S:one}

\textit{Interactive Scores (IS)} are currently used for the design and performance of Electroacoustic
music \cite{aadc08} and live spectacles \cite{bamcrs09} (e.g., interactive theater plays and interactive museums). 
 Both applications
are based on Petri nets \cite{artech2008}.
The main purpose of IS is to provide temporal
relations; for instance, precedence between two objects and relations between their durations. Recently, we extended IS to support conditional branching together with temporal relations \cite{tbdcb10}. It is now possible to represent loops and choices.


We can model temporal relations in \textit{Time Stream Petri nets (TSPN)} \cite{SSW95}, but it is difficult to represent 
global constraints involving (possibly) all the objects of the scenario. Instead, in \textit{Concurrent Constraint Programming} (\texttt{ccp}) \cite{ccp} there are agents that reason about partial information contained in a constraint \textit{store}; thus,  global constraints are inherent in \texttt{ccp}. However, there is not discrete time in \texttt{ccp}, which makes it difficult to represent  reactive systems.


There are some IS models
based on extensions of \texttt{ccp} with discrete time. An example is the \textit{Non-deterministic Timed Concurrent Constraint} (\texttt{ntcc}) model of IS \cite{ntcc, AADR06}. \texttt{Ntcc} is an extension of \texttt{ccp} for non-determinism, asynchrony and discrete time. 

Ntcc belongs to a family of formalisms called process calculi. Process calculi has been applied to the modeling of interactive multimedia systems
 \cite{toro2016faust, toro2016gelisp, 2016arXiv160202169T, toro2015ntccrt, is-chapter,tdcr14,ntccrt,cc-chapter,torophd,torobsc,Toro-Bermudez10,Toro15,ArandaAOPRTV09,tdcc12,toro-report09,tdc10,tdcb10,tororeport},  spatially-explicit ecological systems \cite{EPTCS2047, PT13,TPSK14,PTA13,mean-field-techreport}, data Structures \cite{PAT2016, MorenoPT17, RestrepoPT17}.


In the declarative view, \texttt{ntcc} processes can be interpreted as \textit{linear temporal logic} formulae \cite{Pnu77}. The \texttt{ntcc} model includes an inference system in this logic to verify properties of  \texttt{ntcc} models. This inference procedure was proved to be of exponential time complexity. Nevertheless, we believe practical automatic verification could be envisioned  for useful subsets of \texttt{ntcc} via model checking (see \cite{fv06}). At present, there is no such automatic verifier for \texttt{ntcc}.



Automated verification for \texttt{ntcc} will provide information about the correctness of the system to computer scientists, and will provide important properties about the scenario to its designers and users; for instance, reachability and liveness. We plan to augment \texttt{ntcc} models of IS with these features.
\section{Current and future work}

\textit{Functional AUdio STream (Faust)} \cite{faust} is a programming language
for signal processing with formal semantics and \textit{Ntccrt} \cite{ntccrt} is a real-time capable interpreter for \texttt{ntcc}.
We implemented a signal processing prototype where Faust and Ntccrt interact together.
In the future, we want to define formal semantics to describe a heterogeneous system that includes three subsystems: (i) one based on \texttt{ntcc} to control
discrete events from the user and to synchronize the objects of the scenario, (ii)
another one based on Faust to process audio and video, and finally (iii) one in charge
to load and play audio and video files. 

At the time of this writing, there are no formal semantics of a heterogeneous system that synchronizes concurrent objects, handles global constraints, and controls audio and video streams. Modeling this kind of systems will be useful in other domains such as \textit{machine musical improvisation} \cite{Assayag-using} and \textit{music video games}. An advantage over the existing implementations of these systems will be verification.

In the proof system of \texttt{ntcc}, we can prove properties like  
``10 time units (TUs) after the event $e_A$, during the next 4 TUs, the stream $B$ is the result of applying a \textit{gain filter} to the stream $A$''. However, real-time audio processing cannot be implemented in Ntccrt because it requires to simulate 44100 TUs per second to process a 44.1 kHz sound. If we replace some \texttt{ntcc} processes by Faust plugins, we can execute such system efficiently, but we cannot verify that the properties of the system hold. 

There are two open issues: (i) how to 
prove that a Faust plugin that replaces a \texttt{ntcc} process respect the temporal properties proved for the process, and (ii) whether an implementation of Interactive Scores in Ntccrt can be as efficient as the existing Petri nets implementation, or as one using synchronous languages such as \textit{Signal} \cite{signal}, although the performance results from Ntccrt are promising\footnote{We ran a prototype of a score with conditional branching in Ntccrt. The score contains 500 temporal objects. The average duration of each time-unit was 30 ms, which is compatible with real-time interaction.}.

\section*{Acknowledgement}
I wish to acknowledge fruitful discussions with my supervisors Myriam Desainte-Catherine and Camilo Rueda.
I also want to thank them for helping me writing this article and for guiding my current research.

\bibliographystyle{lipics}	
\begin{thebibliography}{{Tor}16a}
\providecommand{\url}[1]{\texttt{#1}}
\providecommand{\urlprefix}{URL }
\expandafter\ifx\csname urlstyle\endcsname\relax
  \providecommand{\doi}[1]{doi:\discretionary{}{}{}#1}\else
  \providecommand{\doi}{doi:\discretionary{}{}{}\begingroup
  \urlstyle{rm}\Url}\fi
\providecommand{\bibAnnoteFile}[1]{\IfFileExists{#1}{\begin{quotation}\noindent\textsc{Key:} #1\\
  \textsc{Annotation:}\ \input{#1}\end{quotation}}{}}
\providecommand{\bibAnnote}[2]{\begin{quotation}\noindent\textsc{Key:} #1\\
  \textsc{Annotation:}\ #2\end{quotation}}

\bibitem[All06]{AADR06}
Antoine Allombert, G\'erard Assayag, M.~Desainte-Catherine, and Camilo Rueda.
\newblock Concurrent constraint models for interactive scores.
\newblock In \emph{Proc. of SMC '06}. 2006.
\bibAnnoteFile{AADR06}
\newline\urlprefix\url{http://www.labri.fr/publications/is/2006/AADR06}

\bibitem[All08a]{aadc08}
Antoine Allombert, G.~Assayag, and M.~Desainte-Catherine.
\newblock Iscore: a system for writing interaction.
\newblock In \emph{Proc. of DIMEA '08}, pp. 360--367. ACM, New York, NY, USA,
  2008.
\bibAnnoteFile{aadc08}

\bibitem[All08b]{artech2008}
Antoine Allombert, Myriam Desainte-Catherine, J.~Larralde, and G{\'e}rard
  Assayag.
\newblock A system of interactive scores based on qualitative and quantitative
  temporal constraints.
\newblock In \emph{Proc. of Artech 2008}. 2008.
\bibAnnoteFile{artech2008}

\bibitem[All11]{is-chapter}
Antoine Allombert, Myriam Desainte-Catherine, and Mauricio Toro.
\newblock Modeling temporal constrains for a system of interactive score.
\newblock In G\'{e}rard Assayag and Charlotte Truchet (eds.), \emph{Constraint
  Programming in Music}, chap.~1, pp. 1--23. Wiley, 2011.
\bibAnnoteFile{is-chapter}

\bibitem[Ara09]{ArandaAOPRTV09}
Jes{\'{u}}s Aranda, G{\'{e}}rard Assayag, Carlos Olarte, Jorge~A. P{\'{e}}rez,
  Camilo Rueda, Mauricio Toro, and Frank~D. Valencia.
\newblock An overview of {FORCES:} an {INRIA} project on declarative formalisms
  for emergent systems.
\newblock In Patricia~M. Hill and David~Scott Warren (eds.), \emph{Logic
  Programming, 25th International Conference, {ICLP} 2009, Pasadena, CA, USA,
  July 14-17, 2009. Proceedings}, \emph{Lecture Notes in Computer Science},
  vol. 5649, pp. 509--513. Springer, 2009.
\newblock \doi{10.1007/978-3-642-02846-5_44}.
\bibAnnoteFile{ArandaAOPRTV09}
\newline\urlprefix\url{http://dx.doi.org/10.1007/978-3-642-02846-5_44}

\bibitem[Ass04]{Assayag-using}
G.~Assayag and Sholomo Dubnov.
\newblock Using factor oracles for machine improvisation.
\newblock \emph{Soft Comput.}, 8(9):604--610, 2004.
\newblock \doi{http://dx.doi.org/10.1007/s00500-004-0385-4}.
\bibAnnoteFile{Assayag-using}
\newline\urlprefix\url{http://mediatheque.ircam.fr/articles/textes/Assayag04a/}

\bibitem[Bal09]{bamcrs09}
P.~Baltazar, A.~Allombert, R.~Marczak, J.M. Couturier, M.~Roy, A.~S{\`e}des,
  and M.~Desainte-Catherine.
\newblock Virage : Une reflexion pluridisciplinaire autour du temps dans la
  creation numerique.
\newblock In \emph{in Proc. of JIM}. 2009.
\bibAnnoteFile{bamcrs09}

\bibitem[Fal06]{fv06}
Moreno Falaschi and Alicia Villanueva.
\newblock Automatic verification of timed concurrent constraint programs.
\newblock \emph{Theory Pract. Log. Program}, 6(4):265--300, 2006.
\bibAnnoteFile{fv06}

\bibitem[Gau87]{signal}
Thierry Gautier, Paul Le~Guernic, and L\"{o}ic Besnard.
\newblock Signal: A declarative language for synchronous programming of
  real-time systems.
\newblock In \emph{Proc. of FPCA '87}. 1987.
\bibAnnoteFile{signal}

\bibitem[Mor17]{MorenoPT17}
Juan David~Arcila Moreno, Santiago Passos, and Mauricio Toro.
\newblock On-line assembling mitochondrial {DNA} from de novo transcriptome.
\newblock \emph{CoRR}, abs/1706.02828, 2017.
\bibAnnoteFile{MorenoPT17}
\newline\urlprefix\url{http://arxiv.org/abs/1706.02828}

\bibitem[Nie02]{ntcc}
M.~Nielsen, C.~Palamidessi, and F.~Valencia.
\newblock Temporal concurrent constraint programming: Denotation, logic and
  applications.
\newblock \emph{Nordic Journal of Comp.}, 1, 2002.
\bibAnnoteFile{ntcc}
\newline\urlprefix\url{citeseer.ist.psu.edu/article/nielsen02temporal.html}

\bibitem[Ola11]{cc-chapter}
Carlos Olarte, Camilo Rueda, Gerardo Sarria, Mauricio Toro, and Frank Valencia.
\newblock {Concurrent Constraints Models of Music Interaction}.
\newblock In G{\'e}rard Assayag and Charlotte Truchet (eds.), \emph{{Constraint
  Programming in Music}}, chap.~6, pp. 133--153. Wiley, Hoboken, NJ, USA.,
  2011.
\bibAnnoteFile{cc-chapter}

\bibitem[Orl04]{faust}
Y.~Orlarey, D.~Fober, and S.~Letz.
\newblock Syntactical and semantical aspects of faust.
\newblock \emph{Soft Comput.}, 8(9):623--632, 2004.
\newblock \doi{http://dx.doi.org/10.1007/s00500-004-0388-1}.
\bibAnnoteFile{faust}
\newline\urlprefix\url{http://portal.acm.org/citation.cfm?id=1028287}

\bibitem[PF16]{PAT2016}
C.~Pati{\~n}o-Forero, M.~Agudelo-Toro, and M.~Toro.
\newblock {Planning system for deliveries in Medell\'in}.
\newblock \emph{ArXiv e-prints}, 2016.
\bibAnnoteFile{PAT2016}

\bibitem[Phi13a]{PT13}
Anna Philippou and Mauricio Toro.
\newblock {Process Ordering in a Process Calculus for Spatially-Explicit
  Ecological Models.}
\newblock In \emph{{Proceedings of MOKMASD'13}}, {LNCS 8368}, pp. 345--361.
  Springer, 2013.
\bibAnnoteFile{PT13}

\bibitem[Phi13b]{PTA13}
Anna Philippou, Mauricio Toro, and Margarita Antonaki.
\newblock {Simulation and Verification for a Process Calculus for
  Spatially-Explicit Ecological Models}.
\newblock \emph{Scientific Annals of Computer Science}, 23(1):119--167, 2013.
\bibAnnoteFile{PTA13}
\newline\urlprefix\url{http://www.cs.ucy.ac.cy/~annap/ptm2013.pdf}

\bibitem[Pnu77]{Pnu77}
A.~Pnueli.
\newblock The temporal logic of programs.
\newblock In \emph{In Proc. of the 18th IEEE Symposium on the Foundations of
  Computer Science (FOCS-77), pages 46-57. IEEE, IEEE Computer Society Press,
  1977.} 1977.
\bibAnnoteFile{Pnu77}

\bibitem[Res17]{RestrepoPT17}
Juan Manuel~Ciro Restrepo, Andr{\'{e}}s Felipe~Zapata Palacio, and Mauricio
  Toro.
\newblock Assembling sequences of {DNA} using an on-line algorithm based on
  debruijn graphs.
\newblock \emph{CoRR}, abs/1705.05105, 2017.
\bibAnnoteFile{RestrepoPT17}
\newline\urlprefix\url{http://arxiv.org/abs/1705.05105}

\bibitem[Sar92]{ccp}
Vijay~A. Saraswat.
\newblock \emph{Concurrent Constraint Programming}.
\newblock MIT Press, 1992.
\bibAnnoteFile{ccp}
\newline\urlprefix\url{http://books.google.com/books?id=8LKGpv3OCKcC&dq=Concurrent+Constraint+Programming&printsec=frontcover&source=bn&hl=en&sa=X&oi=book_result&resnum=4&ct=result}

\bibitem[Sen95]{SSW95}
Patrick Senac, Pierre~de Saqui-Sannes, and Roberto Willrich.
\newblock Hierarchical time stream petri net: A model for hypermedia systems.
\newblock In \emph{in Proc. of the 16th International Conference on Application
  and Theory of Petri Nets}, pp. 451--470. Springer-Verlag, London, UK, 1995.
\bibAnnoteFile{SSW95}

\bibitem[TB10]{tbdcb10}
Mauricio Toro-B., Myriam Desainte-Catherine, and P.~Baltazar.
\newblock A model for interactive scores with temporal constraints and
  conditional branching.
\newblock In \emph{Proc. of Journ{\'e}es d'informatique musical (JIM) '10 (to
  appear)}. 2010.
\bibAnnoteFile{tbdcb10}

\bibitem[Tor08]{tororeport}
Mauricio Toro.
\newblock Exploring the possibilities and limitations of concurrent programming
  for multimedia interaction and graphical representations to solve musical
  csp's.
\newblock Tech. Rep. 2008-3, Ircam, Paris.{(FRANCE)}, 2008.
\bibAnnoteFile{tororeport}
\newline\urlprefix\url{http://mediatheque.ircam.fr/articles/textes/ToroBermudez08a/}

\bibitem[Tor09a]{torobsc}
Mauricio Toro.
\newblock \emph{{Probabilistic Extension to the Factor Oracle Model for Music
  Improvisation}}.
\newblock Master's thesis, Pontificia Universidad Javeriana Cali, Colombia,
  2009.
\bibAnnoteFile{torobsc}

\bibitem[Tor09b]{toro-report09}
Mauricio Toro.
\newblock Towards a correct and efficient implementation of simulation and
  verification tools for probabilistic ntcc.
\newblock Tech. rep., Pontificia Universidad Javeriana, 2009.
\bibAnnoteFile{toro-report09}

\bibitem[Tor09c]{ntccrt}
Mauricio Toro, Carlos Ag{\'o}n, G{\'e}rard Assayag, and Camilo Rueda.
\newblock {Ntccrt: A concurrent constraint framework for real-time
  interaction}.
\newblock In \emph{{Proc. of ICMC '09}}. Montreal, Canada, 2009.
\bibAnnoteFile{ntccrt}

\bibitem[Tor10a]{Toro-Bermudez10}
Mauricio Toro.
\newblock Structured interactive musical scores.
\newblock In Manuel~V. Hermenegildo and Torsten Schaub (eds.), \emph{Technical
  Communications of the 26th International Conference on Logic Programming,
  {ICLP} 2010, July 16-19, 2010, Edinburgh, Scotland, {UK}}, \emph{LIPIcs},
  vol.~7, pp. 300--302. Schloss Dagstuhl - Leibniz-Zentrum fuer Informatik,
  2010.
\newblock \doi{10.4230/LIPIcs.ICLP.2010.300}.
\bibAnnoteFile{Toro-Bermudez10}
\newline\urlprefix\url{http://dx.doi.org/10.4230/LIPIcs.ICLP.2010.300}

\bibitem[Tor10b]{tdc10}
Mauricio Toro and Myriam Desainte-Catherine.
\newblock Concurrent constraint conditional branching interactive scores.
\newblock In \emph{Proc. of SMC '10}. Barcelona, Spain, 2010.
\bibAnnoteFile{tdc10}

\bibitem[Tor10c]{tdcb10}
Mauricio Toro, Myriam Desainte-Catherine, and P.~Baltazar.
\newblock A model for interactive scores with temporal constraints and
  conditional branching.
\newblock In \emph{Proc. of Journ{\'e}es d'Informatique Musical (JIM) '10}.
  2010.
\bibAnnoteFile{tdcb10}

\bibitem[Tor12a]{torophd}
Mauricio Toro.
\newblock \emph{{Structured Interactive Scores: From a simple structural
  description of a multimedia scenario to a real-time capable implementation
  with formal semantics }}.
\newblock Ph.D. thesis, Univerist{\'e} de Bordeaux 1, France, 2012.
\bibAnnoteFile{torophd}

\bibitem[Tor12b]{tdcc12}
Mauricio Toro, Myriam Desainte-Catherine, and Julien Castet.
\newblock An extension of interactive scores for multimedia scenarios with
  temporal relations for micro and macro controls.
\newblock In \emph{Proc. of Sound and Music Computing (SMC) '12}. Copenhagen,
  Denmark, 2012.
\bibAnnoteFile{tdcc12}

\bibitem[Tor14a]{tdcr14}
Mauricio Toro, Myriam Desainte-Catherine, and Camilo Rueda.
\newblock {Formal semantics for interactive music scores: a framework to
  design, specify properties and execute interactive scenarios}.
\newblock \emph{Journal of Mathematics and Music}, 8(1):93--112, 2014.
\newblock \doi{10.1080/17459737.2013.870610}.
\bibAnnoteFile{tdcr14}
\newline\urlprefix\url{http://dx.doi.org/10.1080/17459737.2013.870610}

\bibitem[Tor14b]{TPSK14}
Mauricio Toro, Anna Philippou, Christina Kassara, and Spyros Sfenthourakis.
\newblock Synchronous parallel composition in a process calculus for ecological
  models.
\newblock In Gabriel Ciobanu and Dominique M{\'{e}}ry (eds.), \emph{Proceedings
  of the 11th International Colloquium on Theoretical Aspects of Computing -
  {ICTAC} 2014, Bucharest, Romania, September 17-19}, \emph{Lecture Notes in
  Computer Science}, vol. 8687, pp. 424--441. Springer, 2014.
\newblock \doi{10.1007/978-3-319-10882-7_25}.
\bibAnnoteFile{TPSK14}
\newline\urlprefix\url{http://dx.doi.org/10.1007/978-3-319-10882-7_25}

\bibitem[Tor15a]{Toro15}
Mauricio Toro.
\newblock Structured interactive music scores.
\newblock \emph{CoRR}, abs/1508.05559, 2015.
\bibAnnoteFile{Toro15}
\newline\urlprefix\url{http://arxiv.org/abs/1508.05559}

\bibitem[Tor15b]{mean-field-techreport}
Mauricio Toro, Anna Philippou, Sair Arboleda, Carlos V\'{e}lez, and Mar\'{i}a
  Puerta.
\newblock {Mean-field semantics for a Process Calculus for Spatially-Explicit
  Ecological Models}.
\newblock Tech. rep., Department of Informatics and Systems, Universidad Eafit,
  2015.
\newblock Available at
  \url{http://blogs.eafit.edu.co/giditic-software/2015/10/01/mean-field/}.
\bibAnnoteFile{mean-field-techreport}

\bibitem[TOR15c]{toro2015ntccrt}
MAURICIO TORO, CAMILO RUEDA, CARLOS AG{\'O}N, and G{\'E}RARD ASSAYAG.
\newblock Ntccrt: A concurrent constraint framework for soft real-time music
  interaction.
\newblock \emph{Journal of Theoretical \& Applied Information Technology},
  82(1), 2015.
\bibAnnoteFile{toro2015ntccrt}

\bibitem[{Tor}16a]{2016arXiv160202169T}
M.~{Toro}.
\newblock {Probabilistic Extension to the Concurrent Constraint Factor Oracle
  Model for Music Improvisation}.
\newblock \emph{ArXiv e-prints}, 2016.
\bibAnnoteFile{2016arXiv160202169T}

\bibitem[TOR16b]{toro2016faust}
MAURICIO TORO, MYRIAM DESAINTE-CATHERINE, and JULIEN CASTET.
\newblock An extension of interactive scores for multimedia scenarios with
  temporal relations for micro and macro controls.
\newblock \emph{European Journal of Scientific Research}, 137(4):396--409,
  2016.
\bibAnnoteFile{toro2016faust}

\bibitem[Tor16c]{EPTCS2047}
Mauricio Toro, Anna Philippou, Sair Arboleda, Mar\'ia Puerta, and Carlos~M.
  V\'elez~S.
\newblock Mean-field semantics for a process calculus for spatially-explicit
  ecological models.
\newblock In C\'esar~A. Mu\~noz and Jorge~A. P\'erez (eds.), \emph{{\rm
  Proceedings of the Eleventh International Workshop on} Developments in
  Computational Models, {\rm Cali, Colombia, October 28, 2015}},
  \emph{Electronic Proceedings in Theoretical Computer Science}, vol. 204, pp.
  79--94. Open Publishing Association, 2016.
\newblock \doi{10.4204/EPTCS.204.7}.
\bibAnnoteFile{EPTCS2047}

\bibitem[TOR16d]{toro2016gelisp}
MAURICIO TORO, CAMILO RUEDA, CARLOS AG{\'O}N, and G{\'E}RARD ASSAYAG.
\newblock Gelisp: A framework to represent musical constraint satisfaction
  problems and search strategies.
\newblock \emph{Journal of Theoretical \& Applied Information Technology},
  86(2), 2016.
\bibAnnoteFile{toro2016gelisp}

\end{thebibliography}

\end{document}
