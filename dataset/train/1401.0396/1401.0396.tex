\documentclass{llncs}
\usepackage{amssymb,latexsym,pslatex}
\usepackage{graphics}
\usepackage{pspicture}
\usepackage{epsfig}
\usepackage{multirow}




\newtheorem{fact}[theorem]{Fact}




\newtheorem{mainresult}{Main Theorem}
\renewcommand{\themainresult}{\hspace{-0.4em}}

\newcommand{\IN}{{\cal N}}
\newcommand{\IR}{{\cal R}}
\newcommand{\braced}[1]{{\left\{#1\right\}}}
\newcommand{\LA}{{}}
\newcommand{\RA}{{}}
\newcommand{\fst}{\textit{fst}}
\newcommand{\lst}{\textit{lst}}

\title{Faster 3-Periodic Merging Networks}

\author{Marek Piotr\'ow}
\institute{Institute of Computer Science, University of Wroc\l aw,\\ 
ul.~Joliot-Curie~15, PL-50-383 Wroc\l aw, Poland\\
\email{Marek.Piotrow@ii.uni.wroc.pl}}



\begin{document}

\maketitle

\begin{abstract} We consider the problem of merging two sorted sequences on a comparator
network that is used repeatedly, that is, if the output is not sorted,
the network is applied again using the output as input. The challenging
task is to construct such networks of small depth. The first
constructions of merging networks with a constant period were given by
Kuty{\l}owski, Lory{\'s} and Oesterdikhoff \cite{klo}. They have given
-periodic network that merges two sorted sequences of  numbers in
time  and a similar network of period  that works in
.  We present a new family of such networks that are based
on Canfield and Williamson periodic sorter \cite{cw}. Our -periodic
merging networks work in time upper-bounded by . The construction
can be easily generalized to larger constant periods with decreasing
running time, for example, to -periodic ones that work in time 
upper-bounded by . Moreover, to obtain the facts we have introduced 
a new proof technique.
\end{abstract}

\medskip
\noindent\textbf{Keywords:} parallel merging, comparison networks, 
merging networks, periodic networks, comparators, oblivious merging.





\section{Introduction}  \label{intro}

Comparator networks are probably the simplest parallel model that is used to
solve such tasks as sorting, merging or selecting \cite{k}. Each network
represents a data-oblivious algorithm, which can be easily implemented in
hardware. Moreover, sorting networks can be applied in secure, multi-party
computation (SMC) protocols. They are also strongly connected with switching
networks \cite{l}. The most famous constructions of sorting networks are
Odd-Even and Bitonic networks of depth  due to Batcher
\cite{b} and AKS networks of depth  due to Ajtai, Komlos and
Szemeredi \cite{aks}. The long-standing disability to decrease a large
constant hidden behind the asymptotically optimal complexity of AKS networks
to a practical value \cite{s} has resulted in studying easier, sorting-related
problems, whose optimal networks have small constants.

A comparator network consists of a set of  registers, each of which can
contain an item from a totally ordered set, and a sequence of
comparator stages.  Each stage is a set of comparators that connect
disjoint pairs of registers and, therefore, can work in parallel (a
comparator is a simple device that takes a contents of two registers
and performs a compare-exchange operation on them: the minimum is put
into the first register and the maximum into the second one). Stages
are run one after another in synchronous manner, hence we can consider
the number of stages as the running time. The size of a network is 
defined to be the total number of comparators in all its stages.

A network  consisting of stages  is called -periodic
if  and for each , , stages  and  are
identical.  A periodic network is easy to implement, especially in hardware,
because one can use the first  stages in a cycle: if the output of -th
stage is not correct (sorted, for example), the sequence of  stages is run
again. We can also define a -periodic network just by giving the total
number of stages and a description of its first  stages. A challenging task
is to construct a family of small-periodic networks for sorting-related
problems with the running time equal to, or not much greater than that of
non-periodic networks.

Dowd et al.\ \cite{dpsr} gave the construction of -periodic sorting
networks of  registers with running time of .  Kuty{\l}owski et
al.\ \cite{klow} introduced a general method to convert a non-periodic sorting
network into a 5-periodic one, but the running time increases by a factor of
 during the conversion. For simpler problems such as merging or
correction there are constant-periodic networks that solve the corresponding
problem in asymptotically optimal logarithmic time \cite{klo,p}. In
particular, Kuty{\l}owski, Lory{\'s} and Oesterdikhoff \cite{klo} have given
-periodic network that merges two sorted sequences of  numbers in time
 and a similar network of period  that works in .
They have also sketched a construction of merging networks with periods larger
than 4 and running time decreasing asymptotically to .

In this paper, we introduce a new family of constant-periodic merging
networks that are based on the Canfield and Williamson -periodic sorter \cite{cw} by a certain periodification technique.
Our -periodic merging networks work in time upper-bounded by  and -periodic ones - in time upper-bounded by . The
construction can be easily generalized to larger constant periods with
decreasing running time.

The advantage of constant-periodic networks is that they have pretty
simple patterns of communication links, that is, each node (register)
of such a network can only be connected to a constant number of other
nodes. Such patterns are easier to implement, for example, in hardware.
Moreover, a node uses these links in a simple periodic manner and this
can save control login and simplify timing considerations.

\section{Periodic merging networks}

Our merging networks are based on the Canfield and Williamson \cite{cw}
-periodic sorters. We recall now the definition of their
networks: for each  let  denote a network of
 registers, where the stages are defined as follows (see also
Figures \ref{mergeCW} and \ref{anotherCW}):

\begin{figure}[ht]
\begin{center}
\begin{picture}(164,99)
\newcounter{reg}
\thicklines
\setcounter{reg}{28}
\multiput(0,3)(0,12){8}{
	\put(-4,7){\makebox(0,0)[br]{\footnotesize\arabic{reg}}} 
	\put(0,0){\line(1,0){164}} 
	\put(0,3){\line(1,0){164}} 
	\put(0,6){\line(1,0){164}} 
	\put(0,9){\line(1,0){164}} 
	\addtocounter{reg}{-4}}
\put(-4,2){\makebox(0,0)[br]{\footnotesize 31}}
\setlength{\arrowlength}{2pt}
\begin{picture}(164,99)
\put(20,96){\vector(0,-1){3}}
\put(20,90){\vector(0,-1){3}}
\put(20,84){\vector(0,-1){3}}
\put(20,78){\vector(0,-1){3}}
\put(20,72){\vector(0,-1){3}}
\put(20,66){\vector(0,-1){3}}
\put(20,60){\vector(0,-1){3}}
\put(20,54){\vector(0,-1){3}}
\put(20,48){\vector(0,-1){3}}
\put(20,42){\vector(0,-1){3}}
\put(20,36){\vector(0,-1){3}}
\put(20,30){\vector(0,-1){3}}
\put(20,24){\vector(0,-1){3}}
\put(20,18){\vector(0,-1){3}}
\put(20,12){\vector(0,-1){3}}
\put(20,6){\vector(0,-1){3}}
\put(40,93){\vector(0,-1){45}}
\put(44,87){\vector(0,-1){45}}
\put(48,81){\vector(0,-1){45}}
\put(52,75){\vector(0,-1){45}}
\put(56,69){\vector(0,-1){45}}
\put(60,63){\vector(0,-1){45}}
\put(64,57){\vector(0,-1){45}}
\put(68,51){\vector(0,-1){45}}
\put(88,93){\vector(0,-1){21}}
\put(92,87){\vector(0,-1){21}}
\put(96,81){\vector(0,-1){21}}
\put(100,75){\vector(0,-1){21}}
\put(88,69){\vector(0,-1){21}}
\put(92,63){\vector(0,-1){21}}
\put(96,57){\vector(0,-1){21}}
\put(100,51){\vector(0,-1){21}}
\put(88,45){\vector(0,-1){21}}
\put(92,39){\vector(0,-1){21}}
\put(96,33){\vector(0,-1){21}}
\put(100,27){\vector(0,-1){21}}
\put(120,93){\vector(0,-1){9}}
\put(124,87){\vector(0,-1){9}}
\put(120,81){\vector(0,-1){9}}
\put(124,75){\vector(0,-1){9}}
\put(120,69){\vector(0,-1){9}}
\put(124,63){\vector(0,-1){9}}
\put(120,57){\vector(0,-1){9}}
\put(124,51){\vector(0,-1){9}}
\put(120,45){\vector(0,-1){9}}
\put(124,39){\vector(0,-1){9}}
\put(120,33){\vector(0,-1){9}}
\put(124,27){\vector(0,-1){9}}
\put(120,21){\vector(0,-1){9}}
\put(124,15){\vector(0,-1){9}}
\put(144,93){\vector(0,-1){3}}
\put(144,87){\vector(0,-1){3}}
\put(144,81){\vector(0,-1){3}}
\put(144,75){\vector(0,-1){3}}
\put(144,69){\vector(0,-1){3}}
\put(144,63){\vector(0,-1){3}}
\put(144,57){\vector(0,-1){3}}
\put(144,51){\vector(0,-1){3}}
\put(144,45){\vector(0,-1){3}}
\put(144,39){\vector(0,-1){3}}
\put(144,33){\vector(0,-1){3}}
\put(144,27){\vector(0,-1){3}}
\put(144,21){\vector(0,-1){3}}
\put(144,15){\vector(0,-1){3}}
\put(144,9){\vector(0,-1){3}}
\thinlines
\put(30,0){\line(0,1){99}}
\put(78,0){\line(0,1){99}}
\put(110,0){\line(0,1){99}}
\put(134,0){\line(0,1){99}}
\end{picture}
 \end{picture}
\end{center}
\caption{The Canfield and Williamson -periodic sorter, where
  . Registers and comparators are represented by horizontal lines and 
  arrows, respectively. Stages are separated by vertical lines.}
\label{mergeCW}
\end{figure}

The merging and sorting properties of the networks are given in the
following proposition.

\begin{proposition}
(1) For each , if two sorted sequences of length
 are given in registers with odd and even indices, respectively, 
then  is a merging network.
(2) For each ,  is a -pass periodic sorting network. 
\end{proposition}

\begin{figure}[ht]
\begin{center}
\begin{picture}(210,160)
\includegraphics{cw32}
\end{picture}
\end{center}
\caption{Another view of  5-pass 5-periodic sorter. Registers and
  comparators are represented by dots and edges, respectively. Stages
  are separated by vertical lines.}
\label{anotherCW}
\end{figure}
We would like to implement a version of this network as a
constant-periodic comparator network. Consider first the most
challenging 3-periodic implementation. We start with the definition of a
temporally construction  which structure is similar to the
structure of . Then we transform it to 3-periodic network .
The idea is to replace each register  in  (except the first and
the last ones) with a sequence of  consecutive registers, move the
endpoints of long comparators one register further or closer depending
on the parity of  and insert between each pair of stages containing
long comparators a stage with short comparators joining the endpoints of
those long ones. The result is depicted in Fig. \ref {merge92}. In this
way, we obtain a network in which each register is used in at most
three consecutive stages. Therefore the network  can be packed into
the first 3 stages and used periodically to get the desired 3-periodic
merging network.
\begin{figure}[ht]
\begin{center}
\begin{picture}(320,100)
\includegraphics[scale=0.8]{merge92}
\end{picture}
\end{center}
\caption{ as an implementation of . Registers and comparators
  are represented by dots and edges, respectively. Stages are separated
  by vertical lines. Stages with short horizontal comparators are
  inserted between stages with long comparators.}
\label{merge92}
\end{figure}

Let  denote a comparator connecting registers  and . A
comparator  is {\em standard} if .  For an -register
network , where  denote
stages, and for an integer , we will use the
following notations:

where  denotes the set
.

Let us define formally the new family of merging networks.  For each
 we would like to transform the network  into a new
network . 
\begin{definition}\label{defMk}
  Let  be one less than the half of the number of
  registers in  and . The number of registers of 
  is defined to be . The stages of  are defined by
  the following equations, where :

\end{definition}

The network  is depicted in Figure \ref{merge3}.

\begin{figure}[ht]
\begin{center}
\setcounter{reg}{88}
\begin{picture}(204,279) \multiput(0,3)(0,12){23}{
	\put(-4,7){\makebox(0,0)[br]{\footnotesize\arabic{reg}}} 
	\put(0,0){\line(1,0){204}} 
	\put(0,3){\line(1,0){204}} 
	\put(0,6){\line(1,0){204}} 
	\put(0,9){\line(1,0){204}} 
	\addtocounter{reg}{-4}}
\put(-4,2){\makebox(0,0)[br]{\footnotesize 91}}
\setlength{\arrowlength}{2pt}
\begin{picture}(204,279)
\setlength{\arrowlength}{2pt}
\put(20,276){\vector(0,-1){3}}
\put(20,258){\vector(0,-1){3}}
\put(20,240){\vector(0,-1){3}}
\put(20,222){\vector(0,-1){3}}
\put(20,204){\vector(0,-1){3}}
\put(20,186){\vector(0,-1){3}}
\put(20,168){\vector(0,-1){3}}
\put(20,150){\vector(0,-1){3}}
\put(20,132){\vector(0,-1){3}}
\put(20,114){\vector(0,-1){3}}
\put(20,96){\vector(0,-1){3}}
\put(20,78){\vector(0,-1){3}}
\put(20,60){\vector(0,-1){3}}
\put(20,42){\vector(0,-1){3}}
\put(20,24){\vector(0,-1){3}}
\put(20,6){\vector(0,-1){3}}
\put(40,273){\vector(0,-1){141}}
\put(44,255){\vector(0,-1){141}}
\put(48,237){\vector(0,-1){141}}
\put(52,219){\vector(0,-1){141}}
\put(56,201){\vector(0,-1){141}}
\put(60,183){\vector(0,-1){141}}
\put(64,165){\vector(0,-1){141}}
\put(68,147){\vector(0,-1){141}}
\put(88,273){\vector(0,-1){3}}
\put(88,261){\vector(0,-1){3}}
\put(88,255){\vector(0,-1){3}}
\put(88,243){\vector(0,-1){3}}
\put(88,237){\vector(0,-1){3}}
\put(88,225){\vector(0,-1){3}}
\put(88,219){\vector(0,-1){3}}
\put(88,207){\vector(0,-1){3}}
\put(88,201){\vector(0,-1){3}}
\put(88,189){\vector(0,-1){3}}
\put(88,183){\vector(0,-1){3}}
\put(88,171){\vector(0,-1){3}}
\put(88,165){\vector(0,-1){3}}
\put(88,153){\vector(0,-1){3}}
\put(88,147){\vector(0,-1){3}}
\put(88,135){\vector(0,-1){3}}
\put(88,129){\vector(0,-1){3}}
\put(88,117){\vector(0,-1){3}}
\put(88,111){\vector(0,-1){3}}
\put(88,99){\vector(0,-1){3}}
\put(88,93){\vector(0,-1){3}}
\put(88,81){\vector(0,-1){3}}
\put(88,75){\vector(0,-1){3}}
\put(88,63){\vector(0,-1){3}}
\put(88,57){\vector(0,-1){3}}
\put(88,45){\vector(0,-1){3}}
\put(88,39){\vector(0,-1){3}}
\put(88,27){\vector(0,-1){3}}
\put(88,21){\vector(0,-1){3}}
\put(88,9){\vector(0,-1){3}}
\put(108,270){\vector(0,-1){63}}
\put(112,252){\vector(0,-1){63}}
\put(116,234){\vector(0,-1){63}}
\put(120,216){\vector(0,-1){63}}
\put(108,198){\vector(0,-1){63}}
\put(112,180){\vector(0,-1){63}}
\put(116,162){\vector(0,-1){63}}
\put(120,144){\vector(0,-1){63}}
\put(108,126){\vector(0,-1){63}}
\put(112,108){\vector(0,-1){63}}
\put(116,90){\vector(0,-1){63}}
\put(120,72){\vector(0,-1){63}}
\put(140,270){\vector(0,-1){3}}
\put(140,264){\vector(0,-1){3}}
\put(140,252){\vector(0,-1){3}}
\put(140,246){\vector(0,-1){3}}
\put(140,234){\vector(0,-1){3}}
\put(140,228){\vector(0,-1){3}}
\put(140,216){\vector(0,-1){3}}
\put(140,210){\vector(0,-1){3}}
\put(140,198){\vector(0,-1){3}}
\put(140,192){\vector(0,-1){3}}
\put(140,180){\vector(0,-1){3}}
\put(140,174){\vector(0,-1){3}}
\put(140,162){\vector(0,-1){3}}
\put(140,156){\vector(0,-1){3}}
\put(140,144){\vector(0,-1){3}}
\put(140,138){\vector(0,-1){3}}
\put(140,126){\vector(0,-1){3}}
\put(140,120){\vector(0,-1){3}}
\put(140,108){\vector(0,-1){3}}
\put(140,102){\vector(0,-1){3}}
\put(140,90){\vector(0,-1){3}}
\put(140,84){\vector(0,-1){3}}
\put(140,72){\vector(0,-1){3}}
\put(140,66){\vector(0,-1){3}}
\put(140,54){\vector(0,-1){3}}
\put(140,48){\vector(0,-1){3}}
\put(140,36){\vector(0,-1){3}}
\put(140,30){\vector(0,-1){3}}
\put(140,18){\vector(0,-1){3}}
\put(140,12){\vector(0,-1){3}}
\put(160,267){\vector(0,-1){21}}
\put(164,249){\vector(0,-1){21}}
\put(160,231){\vector(0,-1){21}}
\put(164,213){\vector(0,-1){21}}
\put(160,195){\vector(0,-1){21}}
\put(164,177){\vector(0,-1){21}}
\put(160,159){\vector(0,-1){21}}
\put(164,141){\vector(0,-1){21}}
\put(160,123){\vector(0,-1){21}}
\put(164,105){\vector(0,-1){21}}
\put(160,87){\vector(0,-1){21}}
\put(164,69){\vector(0,-1){21}}
\put(160,51){\vector(0,-1){21}}
\put(164,33){\vector(0,-1){21}}
\put(184,267){\vector(0,-1){3}}
\put(184,249){\vector(0,-1){3}}
\put(184,231){\vector(0,-1){3}}
\put(184,213){\vector(0,-1){3}}
\put(184,195){\vector(0,-1){3}}
\put(184,177){\vector(0,-1){3}}
\put(184,159){\vector(0,-1){3}}
\put(184,141){\vector(0,-1){3}}
\put(184,123){\vector(0,-1){3}}
\put(184,105){\vector(0,-1){3}}
\put(184,87){\vector(0,-1){3}}
\put(184,69){\vector(0,-1){3}}
\put(184,51){\vector(0,-1){3}}
\put(184,33){\vector(0,-1){3}}
\put(184,15){\vector(0,-1){3}}
\end{picture}
 \end{picture}
\end{center}
\caption{The traditional drawing of  network}
\label{merge3}
\end{figure}

\begin{fact}
 for . \qed
\end{fact}

Let  and  be
-input comparator networks such that for each , ,
.  Then
 is defined to be
,
where empty stages are added at the end of the network of smaller depth.

For any comparator network  and
, let us define a network  to be a {\em 
compact form} of , where , 
. Observe that  is correctly defined due to the delay
of .  Moreover, .

\begin{definition}
For  let  denote the compact form of  with the first
and the last registers deleted. That is, the network  
is using the set of registers numbered , where 
, and , .
\end{definition}
It is not necessary to delete the first and the last registers of  but 
this will simplify proofs a little bit in the next section. The network  
is given in Fig. \ref{merge3p}.
\begin{figure}[ht]
\begin{center}
\setcounter{reg}{88}
\begin{picture}(132,279)
\multiput(0,3)(0,18){15}{
	\put(-4,4){\makebox(0,0)[br]{\footnotesize\arabic{reg}}} 
	\addtocounter{reg}{-3}
	\put(-4,13){\makebox(0,0)[br]{\footnotesize\arabic{reg}}} 
	\put(0,0){\line(1,0){132}} 
	\put(0,3){\line(1,0){132}} 
	\put(0,6){\line(1,0){132}} 
	\put(0,9){\line(1,0){132}} 
	\put(0,12){\line(1,0){132}} 
	\put(0,15){\line(1,0){132}} 
	\addtocounter{reg}{-3}}
\put(-4,1){\makebox(0,0)[br]{\footnotesize 90}}
\setlength{\arrowlength}{2pt}
\begin{picture}(132,279)(0,3)
\setlength{\arrowlength}{2pt}
\put(20,270){\vector(0,-1){63}}
\put(24,267){\vector(0,-1){3}}
\put(24,258){\vector(0,-1){3}}
\put(24,252){\vector(0,-1){63}}
\put(28,249){\vector(0,-1){3}}
\put(28,240){\vector(0,-1){3}}
\put(28,234){\vector(0,-1){63}}
\put(32,231){\vector(0,-1){3}}
\put(32,222){\vector(0,-1){3}}
\put(32,216){\vector(0,-1){63}}
\put(36,213){\vector(0,-1){3}}
\put(20,204){\vector(0,-1){3}}
\put(20,198){\vector(0,-1){63}}
\put(36,195){\vector(0,-1){3}}
\put(24,186){\vector(0,-1){3}}
\put(24,180){\vector(0,-1){63}}
\put(36,177){\vector(0,-1){3}}
\put(28,168){\vector(0,-1){3}}
\put(28,162){\vector(0,-1){63}}
\put(36,159){\vector(0,-1){3}}
\put(32,150){\vector(0,-1){3}}
\put(32,144){\vector(0,-1){63}}
\put(36,141){\vector(0,-1){3}}
\put(20,132){\vector(0,-1){3}}
\put(20,126){\vector(0,-1){63}}
\put(36,123){\vector(0,-1){3}}
\put(24,114){\vector(0,-1){3}}
\put(24,108){\vector(0,-1){63}}
\put(36,105){\vector(0,-1){3}}
\put(28,96){\vector(0,-1){3}}
\put(28,90){\vector(0,-1){63}}
\put(36,87){\vector(0,-1){3}}
\put(32,78){\vector(0,-1){3}}
\put(32,72){\vector(0,-1){63}}
\put(36,69){\vector(0,-1){3}}
\put(20,60){\vector(0,-1){3}}
\put(20,51){\vector(0,-1){3}}
\put(20,42){\vector(0,-1){3}}
\put(20,33){\vector(0,-1){3}}
\put(20,24){\vector(0,-1){3}}
\put(20,15){\vector(0,-1){3}}
\put(56,273){\vector(0,-1){141}}
\put(60,270){\vector(0,-1){3}}
\put(60,264){\vector(0,-1){3}}
\put(60,255){\vector(0,-1){141}}
\put(64,252){\vector(0,-1){3}}
\put(64,246){\vector(0,-1){3}}
\put(64,237){\vector(0,-1){141}}
\put(68,234){\vector(0,-1){3}}
\put(68,228){\vector(0,-1){3}}
\put(68,219){\vector(0,-1){141}}
\put(72,216){\vector(0,-1){3}}
\put(72,210){\vector(0,-1){3}}
\put(72,201){\vector(0,-1){141}}
\put(76,198){\vector(0,-1){3}}
\put(76,192){\vector(0,-1){3}}
\put(76,183){\vector(0,-1){141}}
\put(80,180){\vector(0,-1){3}}
\put(80,174){\vector(0,-1){3}}
\put(80,165){\vector(0,-1){141}}
\put(84,162){\vector(0,-1){3}}
\put(84,156){\vector(0,-1){3}}
\put(84,147){\vector(0,-1){141}}
\put(88,144){\vector(0,-1){3}}
\put(88,138){\vector(0,-1){3}}
\put(56,126){\vector(0,-1){3}}
\put(56,120){\vector(0,-1){3}}
\put(56,108){\vector(0,-1){3}}
\put(56,102){\vector(0,-1){3}}
\put(56,90){\vector(0,-1){3}}
\put(56,84){\vector(0,-1){3}}
\put(56,72){\vector(0,-1){3}}
\put(56,66){\vector(0,-1){3}}
\put(56,54){\vector(0,-1){3}}
\put(56,48){\vector(0,-1){3}}
\put(56,36){\vector(0,-1){3}}
\put(56,30){\vector(0,-1){3}}
\put(56,18){\vector(0,-1){3}}
\put(56,12){\vector(0,-1){3}}
\put(108,273){\vector(0,-1){3}}
\put(108,267){\vector(0,-1){21}}
\put(112,261){\vector(0,-1){3}}
\put(112,255){\vector(0,-1){3}}
\put(112,249){\vector(0,-1){21}}
\put(108,243){\vector(0,-1){3}}
\put(108,237){\vector(0,-1){3}}
\put(108,231){\vector(0,-1){21}}
\put(112,225){\vector(0,-1){3}}
\put(112,219){\vector(0,-1){3}}
\put(112,213){\vector(0,-1){21}}
\put(108,207){\vector(0,-1){3}}
\put(108,201){\vector(0,-1){3}}
\put(108,195){\vector(0,-1){21}}
\put(112,189){\vector(0,-1){3}}
\put(112,183){\vector(0,-1){3}}
\put(112,177){\vector(0,-1){21}}
\put(108,171){\vector(0,-1){3}}
\put(108,165){\vector(0,-1){3}}
\put(108,159){\vector(0,-1){21}}
\put(112,153){\vector(0,-1){3}}
\put(112,147){\vector(0,-1){3}}
\put(112,141){\vector(0,-1){21}}
\put(108,135){\vector(0,-1){3}}
\put(108,129){\vector(0,-1){3}}
\put(108,123){\vector(0,-1){21}}
\put(112,117){\vector(0,-1){3}}
\put(112,111){\vector(0,-1){3}}
\put(112,105){\vector(0,-1){21}}
\put(108,99){\vector(0,-1){3}}
\put(108,93){\vector(0,-1){3}}
\put(108,87){\vector(0,-1){21}}
\put(112,81){\vector(0,-1){3}}
\put(112,75){\vector(0,-1){3}}
\put(112,69){\vector(0,-1){21}}
\put(108,63){\vector(0,-1){3}}
\put(108,57){\vector(0,-1){3}}
\put(108,51){\vector(0,-1){21}}
\put(112,45){\vector(0,-1){3}}
\put(112,39){\vector(0,-1){3}}
\put(112,33){\vector(0,-1){21}}
\put(108,27){\vector(0,-1){3}}
\put(108,21){\vector(0,-1){3}}
\put(108,9){\vector(0,-1){3}}
\end{picture}
 \end{picture}
\end{center}
\caption{The  network}
\label{merge3p}
\end{figure}
\begin{theorem} \label{3merger} There exists a family of 3-periodic
  comparator networks , , such that each  is a
  -pass merger of two sorted sequences given in odd and even
  registers, respectively. The running time of  is , where  is the number of registers in .
\end{theorem}

The proof is based on the observation that  merges  pairs of
sorted subsequences, one after another, in pipeline fashion. Details
are given in the next section. 

In a similar way, we can convert  into a 4-periodic merging
network. Assume that  is even. We replace each register (except the
first and the last ones) with a sequence of  consecutive
registers, move the endpoints of long comparators in such a way that
exactly two long comparators start or end at each new register and
insert after each pair of stages containing long comparators a stage
with short comparators joining the endpoints of those long
comparators. The result is depicted in Fig. \ref{merge4}.

\begin{figure}[ht]
\begin{center}
\setcounter{reg}{120}
\begin{picture}(264,381) \multiput(0,9)(0,12){31}{
	\put(-4,7){\makebox(0,0)[br]{\footnotesize\arabic{reg}}} 
	\put(0,0){\line(1,0){264}} 
	\put(0,3){\line(1,0){264}} 
	\put(0,6){\line(1,0){264}} 
	\put(0,9){\line(1,0){264}} 
	\addtocounter{reg}{-4}}
\put(0,3){\line(1,0){264}} 
\put(0,6){\line(1,0){264}} 
\put(-4,2){\makebox(0,0)[br]{\footnotesize 125}}
\setlength{\arrowlength}{2pt}
\begin{picture}(264,381)
\put(20,378){\vector(0,-1){3}}
\put(20,366){\vector(0,-1){3}}
\put(20,354){\vector(0,-1){3}}
\put(20,342){\vector(0,-1){3}}
\put(20,330){\vector(0,-1){3}}
\put(20,318){\vector(0,-1){3}}
\put(20,306){\vector(0,-1){3}}
\put(20,294){\vector(0,-1){3}}
\put(20,282){\vector(0,-1){3}}
\put(20,270){\vector(0,-1){3}}
\put(20,258){\vector(0,-1){3}}
\put(20,246){\vector(0,-1){3}}
\put(20,234){\vector(0,-1){3}}
\put(20,222){\vector(0,-1){3}}
\put(20,210){\vector(0,-1){3}}
\put(20,198){\vector(0,-1){3}}
\put(20,186){\vector(0,-1){3}}
\put(20,174){\vector(0,-1){3}}
\put(20,162){\vector(0,-1){3}}
\put(20,150){\vector(0,-1){3}}
\put(20,138){\vector(0,-1){3}}
\put(20,126){\vector(0,-1){3}}
\put(20,114){\vector(0,-1){3}}
\put(20,102){\vector(0,-1){3}}
\put(20,90){\vector(0,-1){3}}
\put(20,78){\vector(0,-1){3}}
\put(20,66){\vector(0,-1){3}}
\put(20,54){\vector(0,-1){3}}
\put(20,42){\vector(0,-1){3}}
\put(20,30){\vector(0,-1){3}}
\put(20,18){\vector(0,-1){3}}
\put(20,6){\vector(0,-1){3}}
\put(40,375){\vector(0,-1){189}}
\put(44,363){\vector(0,-1){189}}
\put(48,351){\vector(0,-1){189}}
\put(52,339){\vector(0,-1){189}}
\put(56,327){\vector(0,-1){189}}
\put(60,315){\vector(0,-1){189}}
\put(64,303){\vector(0,-1){189}}
\put(68,291){\vector(0,-1){189}}
\put(72,279){\vector(0,-1){189}}
\put(76,267){\vector(0,-1){189}}
\put(80,255){\vector(0,-1){189}}
\put(84,243){\vector(0,-1){189}}
\put(88,231){\vector(0,-1){189}}
\put(92,219){\vector(0,-1){189}}
\put(96,207){\vector(0,-1){189}}
\put(100,195){\vector(0,-1){189}}
\put(120,375){\vector(0,-1){93}}
\put(124,363){\vector(0,-1){93}}
\put(128,351){\vector(0,-1){93}}
\put(132,339){\vector(0,-1){93}}
\put(136,327){\vector(0,-1){93}}
\put(140,315){\vector(0,-1){93}}
\put(144,303){\vector(0,-1){93}}
\put(148,291){\vector(0,-1){93}}
\put(120,279){\vector(0,-1){93}}
\put(124,267){\vector(0,-1){93}}
\put(128,255){\vector(0,-1){93}}
\put(132,243){\vector(0,-1){93}}
\put(136,231){\vector(0,-1){93}}
\put(140,219){\vector(0,-1){93}}
\put(144,207){\vector(0,-1){93}}
\put(148,195){\vector(0,-1){93}}
\put(120,183){\vector(0,-1){93}}
\put(124,171){\vector(0,-1){93}}
\put(128,159){\vector(0,-1){93}}
\put(132,147){\vector(0,-1){93}}
\put(136,135){\vector(0,-1){93}}
\put(140,123){\vector(0,-1){93}}
\put(144,111){\vector(0,-1){93}}
\put(148,99){\vector(0,-1){93}}
\put(168,375){\vector(0,-1){3}}
\put(168,369){\vector(0,-1){3}}
\put(168,363){\vector(0,-1){3}}
\put(168,357){\vector(0,-1){3}}
\put(168,351){\vector(0,-1){3}}
\put(168,345){\vector(0,-1){3}}
\put(168,339){\vector(0,-1){3}}
\put(168,333){\vector(0,-1){3}}
\put(168,327){\vector(0,-1){3}}
\put(168,321){\vector(0,-1){3}}
\put(168,315){\vector(0,-1){3}}
\put(168,309){\vector(0,-1){3}}
\put(168,303){\vector(0,-1){3}}
\put(168,297){\vector(0,-1){3}}
\put(168,291){\vector(0,-1){3}}
\put(168,285){\vector(0,-1){3}}
\put(168,279){\vector(0,-1){3}}
\put(168,273){\vector(0,-1){3}}
\put(168,267){\vector(0,-1){3}}
\put(168,261){\vector(0,-1){3}}
\put(168,255){\vector(0,-1){3}}
\put(168,249){\vector(0,-1){3}}
\put(168,243){\vector(0,-1){3}}
\put(168,237){\vector(0,-1){3}}
\put(168,231){\vector(0,-1){3}}
\put(168,225){\vector(0,-1){3}}
\put(168,219){\vector(0,-1){3}}
\put(168,213){\vector(0,-1){3}}
\put(168,207){\vector(0,-1){3}}
\put(168,201){\vector(0,-1){3}}
\put(168,195){\vector(0,-1){3}}
\put(168,189){\vector(0,-1){3}}
\put(168,183){\vector(0,-1){3}}
\put(168,177){\vector(0,-1){3}}
\put(168,171){\vector(0,-1){3}}
\put(168,165){\vector(0,-1){3}}
\put(168,159){\vector(0,-1){3}}
\put(168,153){\vector(0,-1){3}}
\put(168,147){\vector(0,-1){3}}
\put(168,141){\vector(0,-1){3}}
\put(168,135){\vector(0,-1){3}}
\put(168,129){\vector(0,-1){3}}
\put(168,123){\vector(0,-1){3}}
\put(168,117){\vector(0,-1){3}}
\put(168,111){\vector(0,-1){3}}
\put(168,105){\vector(0,-1){3}}
\put(168,99){\vector(0,-1){3}}
\put(168,93){\vector(0,-1){3}}
\put(168,87){\vector(0,-1){3}}
\put(168,81){\vector(0,-1){3}}
\put(168,75){\vector(0,-1){3}}
\put(168,69){\vector(0,-1){3}}
\put(168,63){\vector(0,-1){3}}
\put(168,57){\vector(0,-1){3}}
\put(168,51){\vector(0,-1){3}}
\put(168,45){\vector(0,-1){3}}
\put(168,39){\vector(0,-1){3}}
\put(168,33){\vector(0,-1){3}}
\put(168,27){\vector(0,-1){3}}
\put(168,21){\vector(0,-1){3}}
\put(168,15){\vector(0,-1){3}}
\put(168,9){\vector(0,-1){3}}
\put(188,372){\vector(0,-1){39}}
\put(192,360){\vector(0,-1){39}}
\put(196,348){\vector(0,-1){39}}
\put(200,336){\vector(0,-1){39}}
\put(188,324){\vector(0,-1){39}}
\put(192,312){\vector(0,-1){39}}
\put(196,300){\vector(0,-1){39}}
\put(200,288){\vector(0,-1){39}}
\put(188,276){\vector(0,-1){39}}
\put(192,264){\vector(0,-1){39}}
\put(196,252){\vector(0,-1){39}}
\put(200,240){\vector(0,-1){39}}
\put(188,228){\vector(0,-1){39}}
\put(192,216){\vector(0,-1){39}}
\put(196,204){\vector(0,-1){39}}
\put(200,192){\vector(0,-1){39}}
\put(188,180){\vector(0,-1){39}}
\put(192,168){\vector(0,-1){39}}
\put(196,156){\vector(0,-1){39}}
\put(200,144){\vector(0,-1){39}}
\put(188,132){\vector(0,-1){39}}
\put(192,120){\vector(0,-1){39}}
\put(196,108){\vector(0,-1){39}}
\put(200,96){\vector(0,-1){39}}
\put(188,84){\vector(0,-1){39}}
\put(192,72){\vector(0,-1){39}}
\put(196,60){\vector(0,-1){39}}
\put(200,48){\vector(0,-1){39}}
\put(220,372){\vector(0,-1){15}}
\put(224,360){\vector(0,-1){15}}
\put(220,348){\vector(0,-1){15}}
\put(224,336){\vector(0,-1){15}}
\put(220,324){\vector(0,-1){15}}
\put(224,312){\vector(0,-1){15}}
\put(220,300){\vector(0,-1){15}}
\put(224,288){\vector(0,-1){15}}
\put(220,276){\vector(0,-1){15}}
\put(224,264){\vector(0,-1){15}}
\put(220,252){\vector(0,-1){15}}
\put(224,240){\vector(0,-1){15}}
\put(220,228){\vector(0,-1){15}}
\put(224,216){\vector(0,-1){15}}
\put(220,204){\vector(0,-1){15}}
\put(224,192){\vector(0,-1){15}}
\put(220,180){\vector(0,-1){15}}
\put(224,168){\vector(0,-1){15}}
\put(220,156){\vector(0,-1){15}}
\put(224,144){\vector(0,-1){15}}
\put(220,132){\vector(0,-1){15}}
\put(224,120){\vector(0,-1){15}}
\put(220,108){\vector(0,-1){15}}
\put(224,96){\vector(0,-1){15}}
\put(220,84){\vector(0,-1){15}}
\put(224,72){\vector(0,-1){15}}
\put(220,60){\vector(0,-1){15}}
\put(224,48){\vector(0,-1){15}}
\put(220,36){\vector(0,-1){15}}
\put(224,24){\vector(0,-1){15}}
\put(244,372){\vector(0,-1){3}}
\put(244,360){\vector(0,-1){3}}
\put(244,348){\vector(0,-1){3}}
\put(244,336){\vector(0,-1){3}}
\put(244,324){\vector(0,-1){3}}
\put(244,312){\vector(0,-1){3}}
\put(244,300){\vector(0,-1){3}}
\put(244,288){\vector(0,-1){3}}
\put(244,276){\vector(0,-1){3}}
\put(244,264){\vector(0,-1){3}}
\put(244,252){\vector(0,-1){3}}
\put(244,240){\vector(0,-1){3}}
\put(244,228){\vector(0,-1){3}}
\put(244,216){\vector(0,-1){3}}
\put(244,204){\vector(0,-1){3}}
\put(244,192){\vector(0,-1){3}}
\put(244,180){\vector(0,-1){3}}
\put(244,168){\vector(0,-1){3}}
\put(244,156){\vector(0,-1){3}}
\put(244,144){\vector(0,-1){3}}
\put(244,132){\vector(0,-1){3}}
\put(244,120){\vector(0,-1){3}}
\put(244,108){\vector(0,-1){3}}
\put(244,96){\vector(0,-1){3}}
\put(244,84){\vector(0,-1){3}}
\put(244,72){\vector(0,-1){3}}
\put(244,60){\vector(0,-1){3}}
\put(244,48){\vector(0,-1){3}}
\put(244,36){\vector(0,-1){3}}
\put(244,24){\vector(0,-1){3}}
\put(244,12){\vector(0,-1){3}}
\end{picture}
 \end{picture}
\end{center}
\caption{The  network}
\label{merge4}
\end{figure}



\section{Proof of Theorem \ref{3merger}}  

The first observation we would like to make is that we can consider
inputs consisting of 0's and 1's only. The well-known Zero-One Principle
states that any comparator network that sorts 0-1 input sequences
correctly sorts also arbitrary input sequences \cite{k}.  In the similar
way, we can prove that the same property holds also for merging:
\begin{proposition}
If a comparator network merges any two 0-1 sorted sequences, then it
correctly merges any two sorted sequences. \qed
\end{proposition} 

Therefore we can analyze computations of the network , , by
describing each state of registers as a 0-1 sequence , where  represents the content of register
. If  is an input sequence for  passes of ,
then by  we denote the content of registers after 
passes of , ,, that is,
 and . Since  consists of three stages ,
 and , we extend the notation to describe the output of each
stage:  and
, for
. For other values of  we assume that . We will use this superscript
notation for other equivalent representations of sequence
. 

Now let us fix some technical notations and definitions. A 0-1
sequence can be represented as a word over . A
non-decreasing (also called {\em sorted}) 0-1 sequence has a form of
 and can be equivalently represented by the number of ones (or
zeros) in it. For any  let  denote the number
of  in .  If  then , , denotes
the -th letter of  and ,  denotes the word . We say
that a 0-1 sequence  is {\em
  2-sorted} if both  and
 are sorted.

\subsection{Reduction to Analysis of Columns}

For any  let ,  (thus ). The set of registers  can be analyzed as an
 matrix with ,
, as columns. A content of all registers in the matrix,
that is , can be equivalently represented by the
sequence of contents of registers in , , \ldots, ,
that is .  Since  is an even
number, the following fact is obviously true.
\begin{fact} If  is 2-sorted then each ,
  , is sorted. \qed
\end{fact}

That is, the columns are sorted at the beginning of a computation of
 passes of . The first lemma we would like to prove is that
columns remain sorted after each stage of the computation. We start
with a following technical fact:

\begin{fact}\label{f33}
Let  and  be subsets of
 such that . Let  and . Then for any  such that  and 
are sorted, the output  has the following properties:
\begin{enumerate}
\item[(i)]  and  are sorted.
\item[(ii)] Let  and . Then  and .
\end{enumerate}
\end{fact}
\begin{proof}
To prove (i) we show only that  for
. If  then  since  is a non-decreasing function and both 
and  are sorted . If  then . For 
we have .

To prove (ii) let  and .
We consider two cases. If  then  and we
get  and . In this case no comparator from
 exchanges 0 with 1. To see this assume a.c. that a
comparator  exchanges  with
. Then  and  hold because of the
definitions of  and . It follows that ,
thus  --- a contradiction. If  then  and . In this case let us observe that a comparator
 exchanges  with  if and only
if . Therefore  and . \qed
\end{proof}
According to the definition of , it consists of three stages , where  (sets  are defined in Def.~\ref{defMk}). 
Using the notation from Fact \ref{f33}, the following fact is an easy
consequence of Definition \ref{defMk}.
\begin{fact}\label{f34} Let  and  denote the
  corresponding left and the right columns of registers, and
  , . Then
\begin{itemize}
\item[(i)]  and 
\item[(ii)]  and
  , for any 
\item[(iii)]  and , for any
  
\item[(iv)]  and
  
\item[(v)] if  then , for any 
\end{itemize} \qed
\end{fact}
\begin{lemma} \label{l35} If the initial content of registers is a
  2-sorted 0-1 sequence  then after each stage of multi-pass
  computation of  the content of each column ,
  , is sorted, that is, each  is of
  the form , , .
\end{lemma}
\begin{proof} By induction it suffices to prove that for each sequence
   with sorted columns , , the
  outputs ,  have also the columns sorted. Since
  each , as a mapping, is a composition of mapping , each of which, due to Facts 
  \ref{f33} and \ref{f34}, transforms sorted columns into sorted columns, the
  lemma follows. \qed
\end{proof}
From now on, instead of looking at 0-1 sequences with sorted columns, we
will analyze the computations of  on sequences of integers
, where , ,
denote the number of ones in a sorted column . Transformations of
0-1 sequences defined by sets ,  will be
represented by the following mappings:
\begin{definition} \label{defFun}
Let ,  for  and
. The functions ,  and  over
sequences of  reals are defined as follows. Let
 and .

\end{definition}
\begin{fact} \label{f6}
Let  be a 0-1 sequence with sorted columns , let  and . Let , 
  and , where
   and . Then
\begin{enumerate}
\item[(i)] 
\item[(ii)] , for any
  
\item[(iii)] , for any
  
\end{enumerate}
\end{fact}
\begin{proof} Generally, the fact follows from Fact \ref{f34} and the
  part (ii) of Fact \ref{f33} We prove only its parts (i) and (ii). Part
  (iii) can be proved in the similar way.

\textit{(i)~} Observe that 
  due to Fact \ref{f34} \textit{(ii)}. It follows that only the content
  of columns  and  can change, but they remain
  sorted (according to Lemma \ref{l35}). Using Fact
  \ref{f33} \textit{(ii)} we have: ,  and


Now let us consider the following three cases of values  and
: \\
\noindent
{\it Case  ~and~ .} Then 
   and 
  .\\
\noindent
{\it Case .} Then ,  and
. In this case:
   and 
  .\\
\noindent
{\it Case .} Then  and . In this case:
   and 
  .

\textit{(ii)~} We fix any  and observe
  that  due to Fact \ref{f34}
  \textit{(ii)}. It follows that only the content of columns 
  and  can change, but they remain sorted (according to
  Lemma \ref{l35}). Using Fact \ref{f33} \textit{(ii)} we have:


\end{proof}
\begin{definition} \label{defQ}
Let . Let ,  and  denote the following sets
of functions.
\
\end{definition}
Let us observe that each function in , , can only modify
a few positions in a given sequence of numbers. Moreover, different
functions in  can only modify disjoint sets of positions. For a
function  let us
define  The following facts
formalize our observations.
\begin{fact}
, , 
    , where .
\end{fact} \qed
\begin{fact}
For each pair of functions , , , we have
\begin{itemize}
\item[(i)] ;
\item[(ii)] for any  and 
                                                  

\end{itemize} 
\end{fact}  \qed
\begin{corollary} \label{color1}
Each set , , uniquely determines a mapping, in which
functions from  can be apply in any order. Moreover, if ,  and  then
.
\end{corollary}
We would like to prove that the result of applying , ,
to a sequence  of numbers of ones in
columns  is equivalent to applying the set of
comparators  to the content of registers, if each column is
sorted.
\begin{lemma} \label{l2}
Let  be a 0-1 sequence with sorted columns , let  and . Let , 
  and , where
   and . Then .
\end{lemma}
\begin{proof}
Recall that . For a set of comparators  let us define
 From Fact \ref{f34}(i--iv) it follows that
 and for   and . From Fact \ref{f34}(v) we get that  if . Thus we can observe a 1-1
correspondence between a function  in  and a set of
comparators  such that
 Then for each 
, as
the consequence of Corollary \ref{color1} and Fact \ref{f6}. \qed
\end{proof}
\begin{definition} \label{flat}
We say that a sequence of numbers 
is {\em flat} if . We say that
a sequence  is {\em 2-flat} if subsequences
 and  are flat. We
say that  is balanced if ,
for . For a balanced sequence  define 
 as .
\end{definition}
\begin{proposition} \label{p3}
Let , , , where  ( is as usual a
column in the matrix of registers), . Then
\begin{enumerate}
\item  is sorted if and only if columns of  are sorted  and 
 is flat;
\item  is 2-sorted if and only if columns of  are sorted and 
 is 2-flat;
\end{enumerate}
\end{proposition} \qed

Now we are ready to reduce the proof of Theorem \ref{3merger} to the
proof of following lemma.
\begin{lemma} \label{l3}
Let . If for each 2-flat sequence  of integers from  the result of
application  to  is a
flat sequence, then  is a -pass merger of two sorted sequences
given in odd and even registers, respectively.
\end{lemma}
\begin{proof}
Assume that for each 2-flat sequence  the result of application  to  is a flat sequence.  Let
 be a 2-sorted sequence and , where  ( is
as usual a column in the matrix of registers), . Then
 is 2-flat due to Proposition \ref{p3} and each 
, because the height of columns is . Recall 
that  and let
. Using Lemma \ref{l2} and easy
induction we get that the equality  is true for
. Since  is a flat sequence, the sequence
 is sorted. \qed
\end{proof}

\subsection{Analysis of Balanced Columns}

Due to Lemma \ref{l3} we can only analyze the results of periodic
application of the functions ,  and  to a sequence
of integers representing the numbers of ones in each register column. We
know also that an initial sequence is 2-flat. To simplify our analysis
further, we start it with initial values restricted to be balanced
2-flat sequences. Then we observe that the functions are monotone and
any 2-flat sequence can be bounded from below and above by balanced
2-flat sequences whose heights differ only by one.

\begin{lemma} \label{l4}
Let  and  be a balanced
sequence of numbers. Let  and let  be a
function from . Then  is
also balanced and .
\end{lemma}

\begin{proof}
Let  and  be as in Lemma and let .  The function 
can be either  or one of , , ,
according to Definition \ref{defQ}. Each of the functions can only
modify one or two pairs of positions of the form  in
 (see Definition \ref{defFun}). The other pairs are left
untouched, so the sum of their values cannot change. In case of 
the modified pair is  and . In case of  the pair is 
and .  Finally, if  then we have
two pairs  and . Then  and in case of the second pair
. \qed
\end{proof}

It follows from Lemma \ref{l4} that if we start the periodical
application of the functions ,  and  to a balanced
2-flat initial sequence then it remains balanced after each function
application and its height will not changed. Therefore, we can only
trace the values in the first half of generated sequences. If needed, a
value in the second half can be computed from the height and the
corresponding value in the first half. To get a better view on the
structure of generated sequences, we subtract half of the height from
each element of the initial sequence and proceed with such modified
sequences to the end. At the end the subtracted value is added to each
element of the final sequence. The following fact justifies the described
above procedure.

\begin{fact}\label{fct-10}
Let  be a function from . Then  is
monotone and for each  and  the following
equation is true

\end{fact}

\begin{proof}
The fact follows from the similar properties of  and 
functions: they are monotone and the equations:  and  are obviously true. Each
 in  is defined with the help of these
simple functions, thus  inherits the properties. \qed
\end{proof}

\begin{corollary}
Let , where , . Then  is monotone and for any  and 
 \qed
\end{corollary} 

\begin{definition} \label{reduce}
Let  be a balanced
sequence and . We call  the reduced
sequence of  and denote it by . For
a sequence  we define
-extended sequence  as

For any  and a function  that maps
each balanced sequence to a balanced one and preserves its height let
 denote a function on  such that
 for any
.
\end{definition}

Observe that for a balanced sequence  with height  the
sequence  is equal to
. Moreover, for any  and a sequence
 the sequence  is balanced
and its height is , thus . Note also that functions ,  and 
preserve the property of being balanced and the sequence height (see
Lemma \ref{l4}), so we can analyze a periodical application of their
reduced forms to a reduced balanced 2-flat input.

\begin{fact}
Let , where , . Let  be
balanced and  Let ,
. Then . \qed
\end{fact}

\begin{definition}\label{def-Cyc}
Let , ,  and , where . Moreover, let us define the following sequences of
functions:

where  denote concatenation of sequences and for  
\end{definition}

\begin{lemma}\label{l5}
Let  and . Then , where
 and  denotes the Cartesian product of a sequence of
functions.
\end{lemma}
\begin{proof}
Let ,  and . Let . By 
Def. \ref{reduce}, . Let . The sequence  is balanced and 
. To get the lemma we would like to prove that for 
 the equalities  hold. The proof is by case analysis 
of 
values of  and . In the following equations we use Definitions 
\ref{defFun}, \ref{defQ}, \ref{reduce} and \ref{def-Cyc}.
\begin{enumerate}
\item (Case:  and ). Then .
\item (Case:  and ). Let  be such that . Then .
\item (Case: ,  and ). Let  be such 
that . Then . Starting from the other side we get
 and we are done.
\item (Case: ,  and ). Let  be as in 
previous case. Then .
\item (Case:  and ). Let  be such 
that . Then  \\ . Starting from the other side we get
 and we are finally done. \qed
\end{enumerate}
\end{proof}

Instead of tracing individual values in reduced sequences after each
application of a function from
 we 
will
trace intervals in which the values should be and observe how the
lengths of intervals are decreasing during the computation. So let us
now define the intervals and show a fact about computations on them.
\begin{definition} \label{interval}
Let ,  for . Let  denote the
interval  and, in similar way, let , , 
and . Moreover, we
will write  for the Cartesian product , where each .
\end{definition}
\begin{fact} \label{fct-12}
The following inclusions are true:
\begin{enumerate}
\item  and 
  , for  and ;
\item  and 
  , for ;
\item  and ;
\item ;
\item , 
                              for  and .
\end{enumerate}
\end{fact}
\begin{proof}
The proof of each inclusion is a straightforward consequence of the
definitions of a given function and intervals. Therefore we check only
inclusions given in the first item. Let . If . then  since . Otherwise  must be in , but
  then  and  since .

To proof the second inclusion for  let us observe that if 
then . It follows that  and
. In case of  we only have
to check the positive values of . such that . But then
 and both . \qed
\end{proof}

Now we are ready to define sequences of intervals that are used to
describe states of computation after each periodic application of
functions ,  and  to a reduced
sequence of numbers of ones in columns.

\begin{definition}
Let . By  we denote the sequence
 and, in the similar way, ,  and .

Next, let ,  and let .

Finally, let ,  and let .
\end{definition}

Note that all sequences defined above are of length
 and their elements are interval
descriptors as defined in Definition \ref{interval}.

\begin{definition}
Let . Let  and
 be any sequences, where . For
 let  denote .
\end{definition}

\begin{definition}\label{def-X}
Let . Let  denote a state sequence after  stages and
be defined as:

\end{definition}

For example, to create  we take the first element of  and
the rest of elements from  obtaining the sequence  of length . In the
next lemma we claim that  really describes the state after the
first stage of computation, where input is a balanced 2-flat sequence.

\begin{lemma}\label{l6}
Let  and let  be a balanced
2-flat sequence of integers from . Let  and let . Then .
\end{lemma}
\begin{proof}
Recall that . Let  By
Definitions \ref{flat} and \ref{reduce}  and each
. Observe that each
. It follows
from the following sequence of inequalities: . Moreover, the sequence  is 2-flat, because 
 is 2-flat. That means that  and , 
where  and .
\begin{fact} \label{fct-13}
Either  and  or  
and .
\end{fact}
To prove the fact we consider three cases of the value of .

\noindent{\em Case :} In this case we only have to prove that 
. But it is true since . The last inequality holds, because 
 is 2-flat and both  and  are even.

\noindent{\em Case :} Then . Thus we have 
only to prove that . Similar to the previous case, we 
observe that .

\noindent{\em Case :} Then 
and from  we get . Since , we have . If , we are done. Otherwise  and we have to show that
. To this end let us notice that  and . It follows that  since . Thus  and this
concludes the proof of Fact \ref{fct-13}.

From Fact \ref{fct-13} and since  is 2-flat we can immediately 
get the following corollary.
\begin{corollary}
.
\end{corollary}
To finish the proof of the lemma we need one more fact:
\begin{fact}
.
\end{fact}
To prove this fact let us firstly represent  in the same form as 
 is.

where  is empty if ,  if 
 and  if 
. Looking now at both representations we can see 
that the output of  function should be in , the output of each 
 function should be in  and the output of  should be 
in . If  function is used, then its output should 
be in . The input to  is either from  or from . In 
both cases we get desired output according to Fact \ref{fct-12}.2. In the 
similar way, the input to each  function is either from  or from . But  by Fact \ref{fct-12}.1. From Fact \ref{fct-12}.3 we have 
. Finally, the input to  function is 
either from  or from . For this 
function the result follows from Fact \ref{fct-12}.4. \qed
\end{proof}

\begin{lemma}\label{l7}
For  and each  the following inclusion holds:

\end{lemma}
\begin{proof}
We have to prove that for  and  the following inclusions 
are true: , where  for  
and   for . The 
sequences , , are built of functions , 
,  and  introduced in Definition \ref{def-Cyc}. We 
consider these function one after another analysing which positions in state
sequences are modified by them and what values are in that positions before 
and after applying a function. In the following, we denote by  the -th 
element of a sequence .

The function  is used only in the definition of  and is
applied to position 1 of state sequences , where
. Thus it is enough to show the
inclusion . By Definition
\ref{def-X} the argument of  can be: 
for  or  for  or  for . The corresponding value of the next state sequence
is  for  or  for . Using Fact \ref{fct-12}, inclusions  and  are true and we are done.

In the sequence  we have several  functions, each
 is on the corresponding position  and it is applied to the
state sequence , where .
Similarly, in  we have several  functions, each
 is on the corresponding position  and it is applied to the 
state sequence , where .
Finally, in  we have  functions, each 
is on the corresponding position  and it is applied to the state sequence
, where . Assuming that
 also denotes , we can rewrite our proof goal for that
functions as the following fact.
\begin{fact}
For  and  the set  is
a subset of , where ,  for  and
 for .
\end{fact}
The sequences  are defined with the help of sequences , ,
 and , therefore we prove the fact by considering all possible
cases in the following table. In it we assume that ,
 and .
\setlength{\tabcolsep}{3pt}
\begin{center}
\begin{tabular}{||l|l||c|c||c||}
\hline\hline \multicolumn{1}{||c|}{Cases of} & 
      \multicolumn{1}{|c||}{Cases of} & Value of  & Value of  & Why  \\ 
        &   &   &   & 
      ? \\ \hline
\hline  &  &   &   & 
      \multirow{5}{*}{Fact \ref{fct-12}.1}  \\ 
\cline{1-4}  &  &  &  &\\ 
\cline{1-4}  &  &  &  &\\ 
\cline{1-4}  &  &  &  &\\ 
\cline{1-4}  &  & 0 & 0 & \\ 
\hline\hline 
\end{tabular}
\end{center}
The two remaining cases: (1)  and 
 and (2)  
and  are not possible, because, otherwise, (1) 
 should be even and , which cannot hold for 
any integers ,  and ; (2)  should be even and 
, which is not true for the same reason.

Now we consider the  function. It appears in the definition of 
 ( or , respectively) on the position 
 if  ( or , respectively). Thus, 
to prove the lemma, it suffices to show the following fact.
\begin{fact}
For  and  the set  is a subset of
, where  
 for  and
 for .
\end{fact}
As in the case of  functions we prove the fact by considering all 
possible cases in the following table. In it we assume that ,
 and .
\begin{center}
\begin{tabular}{||l|l||c|c||c||}
\hline\hline
      \multicolumn{1}{||c|}{Cases of} & \multicolumn{1}{|c||}{Cases of} & 
      Value of  & Value of  & Why  \\ 
        &   &   &   & 
      ? \\ \hline
\hline\multirow{2}{*}{} &  &  &   &
      \multirow{6}{*}{Fact \ref{fct-12}.3}  \\ 
\cline{2-4} &  &  &   & \\ 
\cline{1-4}  &  &  &  & \\ 
\cline{1-4} \multirow{2}{*}{} &  &  &  
& \\ 
\cline{2-4} &  &  &  & \\ 
\cline{1-4}  &  & 0 &  & \\ 
\hline\hline 
\end{tabular}
\end{center}
The remaining case  and 
 is not possible, because, otherwise , that is, , but  and in the 
consequence  - contradiction.

The last function we have to consider is , which appears in the
definition of all , , functions. In 
( and , respectively) a copy of  is on
positions  ((2,3), (5,6), \ldots and (1,2), (4,5), 
\ldots, respectively). Thus, to prove the lemma, it suffices to show the
following fact.
\begin{fact}
For  and  the set  is a subset of , where 
 for  and
 for .
\end{fact}
As in the case of previous functions we prove the fact by considering all
possible cases in the following table. In it we assume that ,
 and . To reduce the size of the table we also use
the following shortcuts: ,  and . Observe that
, therefore  and we can 
also apply Fact \ref{fct-12}.5.
\begin{center}
\begin{tabular}{|*{4}{|l}|*{4}{|c}||c||}
\hline\hline \multicolumn{2}{||c|}{Cases of } & 
      \multicolumn{2}{|c||}{Cases of } & 
      \multicolumn{2}{|c|}{Value of} & 
      \multicolumn{2}{|c||}{Value of}  & Why  \\ 
        &   &   &   & 
        &   &   &  & ? \\ \hline
\hline \multirow{2}{*}{} & \multirow{2}{*}{} &
       &  & 
     &  &  &  & \multirow{2}{*}{Fact \ref{fct-12}.4} \\
\cline{3-8} & &  &  & 
       &  &  &  &   \\ 
\hline \multirow{5}{*}{} & \multirow{2}{*}{} & 
\multirow{2}{*}{} &  & 
     &  &  &  &  \multirow{10}{*}{Fact \ref{fct-12}.5}  \\ 
\cline{4-8} &  &  &  &  &  &  &  &  \\ 
\cline{2-8} & \multirow{2}{*}{} &\multirow{2}{*}{}  
      &  &  &  &  &  &   \\ 
\cline{4-8} & & &  &  &  &  &  &   \\ 
\cline{2-8} &  &  &  & 
       &  &  &  &   \\ 
\cline{1-8} \multirow{2}{*}{} & \multirow{2}{*}{} 
      &  &  &  &  &  &  &   \\ 
\cline{3-8} & &  &  &  &  &  &  & \\ 
\cline{1-8} \multirow{3}{*}{} & \multirow{2}{*}{} & 
      \multirow{2}{*}{} &  &  &  &  &  &  \\ 
\cline{4-8} & & &  &  &  &  &  &  \\ 
\cline{2-8} &  &  &  &  &  &  &  &  \\ 
\hline\hline 
\end{tabular}
\end{center} \qed
\end{proof}

\begin{lemma}
Let  and let  be a balanced
2-flat sequence of integers from  and let
. Let , where , . Then 
 if  is even or  otherwise.
\end{lemma}
\begin{proof}
Since each  maps a balanced sequence to a balanced one, let , where the later
equality follows from Lemma \ref{l5}. Let also  and let 
for . Then  by Lemma \ref{l6}
and for  we get  by an easy
induction and Lemma \ref{l7}. Let  denote as usual the set of 
integers. By  we will denote the set 
. Looking at Definitions \ref{reduce} and 
\ref{def-Cyc} observe the 
following fact:
\begin{fact}
If  is even then all elements of sequences , , are integers. If  is odd then  all elements of sequences 
, , are in .
\end{fact}
Since  and  and , it follows that  if  is 
even and  , otherwise. Applying now 
the definition of -extended sequence to  and  
we get the desired conclusion of the lemma. \qed
\end{proof}

In this way, with respect to Lemma \ref{l3}, we have proved that the network 
 is able to merge in  stages two sorted sequences given in odd 
and even registers, provided that the numbers of ones in our matrix columns 
form a balanced sequence. If the sequence is not balanced,  additional 
stages are needed to get a sorted output.

\subsection{Analysis of General Columns}

In a general case we will use balanced sequences as lower and upper bounds on
the numbers of ones in our matrix columns and observe that , 
and  are monotone functions (see Fact \ref{fct-10}).

\begin{definition}\label{def-14}
Let  and let  be a
2-flat sequence of integers from  that is not balanced.
Since both  and 
 are 
flat sequences, let  (, respectively) be such that  
(, respectively) or let  () if 
 (, respectively) is a constant 
sequence. The defined below sequences  and  we will call 
lower and upper bounds of . If  then for   
If  then for   
\end{definition}
\begin{fact}\label{fct-19}
For  and any not balanced 2-flat sequence  of integers from  the sequences
 and  are balanced,  and .
\end{fact}
\begin{proof}
Let  and  be defined as in Definition \ref{def-14}. We will only 
consider the case . The proof of the other case is similar. Directly 
from the definition we get that  is balanced. To see that 
 is also balanced let us check for  whether the 
sum  is constant.

If  there is no otherwise case and we are done. If  then 
, because of the definition of  and  
and we are also done. Moreover . To prove that  we consider even and odd indices. For even indices 
from the definition we have: . For odd indices . If  we are done, otherwise, , because  is flat. \qed
\end{proof}

\begin{theorem}\label{thm-19}
Let  and let  be a 2-flat
sequence of integers from . Let , where ,
. Then  is a flat sequence.
\end{theorem}
\begin{proof}
For a a 2-flat sequence  of integers from  let 
 and  be its balanced lower and upper bounds, as defined 
in Definition \ref{def-14}. Let , ,  and for  let us 
define ,  and . Observe that 
, because of monotonicity of 
functions , ,  and Fact \ref{fct-19}. To prove that 
 is a flat sequence we need the following three 
technical facts.
\begin{fact}\label{fct-21}
Let . If  is even then  and 
for each  and . If  is odd then   and  for each  and . 
\end{fact}
\begin{proof}
Since both  and  are balanced, we can consider reduced 
forms of them and use Lemmas \ref{l6} and \ref{l7}. For the given range of 
's values that means that 
It follows that for a given range of 's values 
. 
From Fact \ref{fct-19} we know that  and from Lemma 
\ref{l4} that heights are preserved in sequences  and 
. Thus, from the definition of a reduced sequence, 
, , 
and . Since 
 and  are sequences of integers, for even  we get 

and ; for odd  we conclude that 
 and . Since , the fact follows.  \qed
\end{proof}
The second fact extends the first fact up to the last stage of our 
computation.
\begin{fact}\label{fct-22}
Let . If  is even then  and 
for each  and . If  is odd
then  and
 for each 
and .
\end{fact}
\begin{proof}
Consider first the sequence  and observe that for  the value of  is equal to . It
follows from Fact \ref{fct-21} that for even  all values from the left half
of  are equal to  and all values from the
right half of  are in .
For odd  all values from the left half of  are in
 and all values from the right half of
 are equal to . Since ,  
and  are built of functions ,  and  (cf. 
Definitions \ref{defFun} and \ref{defQ}) observe that each function ,  can only exchange values at positions from 
 that are from non-constant half of arguments (in case of 
 and  we can observe that for  and any  we have , ,  and 
, that is, the functions are identity mappings in stages ). The  functions can only exchange unequal values 
at neighbour positions moving the smaller value to the left.  \qed
\end{proof}
The last fact states that unequal values  described in the
previous two facts are getting sorted during the computation. Observe that if
 is odd (even, respectively) then we only have to trace the sorting process
in a left (right, respectively) region of indices

(, respectively),
where  and the values to be sorted differs at most by
one. We trace the positions of the smaller values  in the
left region and the greater values  in the right region. We
will call  a moving element. For  let us define  to be the stage, after which the length of the region extends from
 to  and a new element appears in it. Let  for odd  and
, otherwise, be the position of this new element and
 be its value. Finally, let  be the number of moving elements in the region after stage .
\begin{fact}\label{fct-23}
Using the above definitions, for , if  then for  we have  if  is 
odd and , otherwise.
\end{fact}
\begin{proof}
We prove the fact only for odd , that is, for the left region. The proof
for the right region is symmetric. We would like to show that if 
appears at position  after stage  then it moves in each of the
following stages one position to the left up to its final position . The
proof is by induction on  and . If  and  appears at
position 1 after stage  then  and  is already at its
final position. It never moves, because values at second position are , by Facts \ref{fct-21} and \ref{fct-22}. If  and  then the
basis  is obviously true. In the inductive step  we assume that 
 and that the fact is true for smaller 
values of . If  then also  
and, by the induction hypothesis, values at positions  
are all equal . That means that  is at its final position and we are 
done. Thus we left with the case: , that is, with . 

Consider the sequences  and . We know that . To prove that  we would like to show 
that  and . The 
later is a direct consequence of an observation that  if and 
only if . In our case . To prove the former, let us consider , . Then  and . By the induction 
hypothesis, . Setting  we 
get  and . Moreover, . That means that in the sequence  
 none of  elements  is at position  
and, consequently, . Since  
switches  with , this completes the proof of Fact \ref{fct-23}.  \qed
\end{proof}
Now we are ready to prove that  is a flat sequence. By
Fact \ref{fct-22}, if  is odd then , otherwise,
. The number of minority
elements in  has been denote by . If  is odd
and , , is a minority element , then, by
Fact \ref{fct-23}, . If  is even and , , is a minority element , then, by Fact
\ref{fct-23}, . In both cases this proves
that   is flat, which completes the proof of Theorem
\ref{thm-19}. \qed
\end{proof}

\subsection{Proof of Theorem \ref{3merger}}  

Theorem \ref{3merger} follows directly from Theorem \ref{thm-19} and Lemma
\ref{l3}. Let  and  be any 2-flat sequence of integers
from . By Theorem \ref{thm-19} the result of application
 to  is a flat sequence.
Then, by Lemma \ref {l3}, the network  is a -pass merger of two 
sorted sequences given in odd and even registers, respectively.

\section{Conclusions}

For each  we have shown a construction of a 3-periodic merging
comparator network of  registers and proved that it merge any
two sorted sequences (given in odd and even registers, respectively) in time
. A natural question remains whether it is the optimal merging
time for 3-periodic comparator networks.

\begin{thebibliography}{99} 

\bibitem{aks} {\sc M.~Ajtai, J.~Komlos and E.~Szemeredi}, {\em An  sorting network}, in Proc.\ 15th Annual ACM Symp.\ on Theory
  of Computing, 1983, pp.~1--9.

\bibitem{b} {\sc K.E.~Batcher}, {\em Sorting networks and their
  applications}, in Proc.\ AFIPS 1968 SJCC, Vol. 32, AFIPS Press,
  Montvale, NJ, pp.~307--314.

\bibitem{bw} {\sc E.~A.~Bender and S.~G.~Williamson}, {\em  Periodic
  Sorting Using Minimum Delay, Recursively Constructed Merging
  Networks}, The Electronic Journal of Combinatorics 5 (1998), pp.~1--21.

\bibitem{cw} {\sc E.~R.~Canfield and S.~G.~Williamson}, {\em A
  sequential sorting network analogous to the Batcher merge}, Linear and
  Multilinear Algebra, 29 (1991), pp.~43--51.

\bibitem{dpsr} {\sc M.~Dowd, Y.~Perl, M.~Saks and L.~Rudolph}, {\em The
  periodic balanced sorting network}, Journal of ACM, 36 (1989),
  pp.~738--757.

\bibitem{k} {\sc D.E.~Knuth}, {\em The Art of Computer Programming},
  Vol. 3, 2nd edition, Addison Wesley, Reading, MA, 1975.

\bibitem{klo} {\sc M.~Kuty{\l}owski, K.~Lory{\'s} and
  B.~Oesterdiekhoff}, {\em Periodic Merging Networks}, Theory of
  Computing Systems, 31.5 (1998), pp.~551--578.

\bibitem{klow} {\sc M.~Kuty{\l}owski, K.~Lory{\'s}, B.~Oesterdiekhoff
  and R.~Wanka}, {\em Periodification scheme: constructing sorting
  networks with constant period}, Journal of ACM, 47 (2000),
  pp.~944-967. 

\bibitem{l} {\sc F.T.~Leighton}, {\em Introduction to Parallel
  Algorithms and Architectures{\rm :} Arrays, Trees and Hypercubes},
  Morgan-Kaufmann, San Mateo, CA, 1992.
  
\bibitem{ll} {\sc  T.~Levi and A.~Litman}, {\em The Strongest Model of 
Computation Obeying 0-1 Principles}, Theory of Computing Systems, 48(2) 
(2011), pp.~374-388.

\bibitem {mpt} {\sc P.B.~Miltersen, M.~Paterson and J.~Tarui}, {\em The 
asymptotic complexity of merging networks}, Journal of the ACM, 43(1) (1996), 
pp.~147-165.

\bibitem{o} {\sc B.~Oesterdiekhoff}, {\em Periodic comparator networks},
  Theoretical Computer Science, 245 (2000), pp.~175-202.

\bibitem{p} {\sc M.~Piotr{\'o}w}, {\em Periodic, Random-Fault-Tolerant
  Correction Networks}, in Proc.\ 13th ACM Symposium on Parallel
  Algorithms and Architectures, ACM Press, New York, 2001.

\bibitem{ph} {\sc M.~Piotr{\'o}w}, {\em A note on Constructing Binary
  Heaps with Periodic Networks}, Information Processing Letters, 83
  (2002), pp.~129-134.

\bibitem{s} {\sc J.~Seiferas}, {\em Research note: Networks for sorting 
multitonic sequences}, Journal on Parallel Distrib. Comput., 65(12) (2005), 
pp.~1601-1606.

\end{thebibliography}


\end{document}
