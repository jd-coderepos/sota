\documentclass{LMCS} 






\usepackage{amssymb}
\usepackage{amsmath}
\usepackage{exscale}
\usepackage{enumerate}
\usepackage{stmaryrd}



\usepackage{ifthen}
\usepackage{graphicx}    \usepackage{hyperref}








\usepackage[all]{xy}









































\newcommand{\proofbegin}{\medskip\noindent\textit{Proof}: }
\newcommand{\proofsketchbegin}{\medskip\noindent\textit{Proof (Sketch)}: }
\newcommand{\proofend}{\hspace*{\fill}\vspace{\topsep}\par}



\newcommand{\problem}[3][9]{\begin{center}
                              \normalfont
                              \fbox{
                               \begin{tabular}[t]{rp{#1cm}}
                                \textit{Input}:&#2.\\
                                \textit{Problem}:&#3.
                               \end{tabular}}
                              \end{center}}     

\newcommand{\im}[1]{\item\hspace{#1cm}} 

\newenvironment{algorithm}{\begin{enumerate}
         \renewcommand{\labelenumi}{{\small\itshape\arabic{enumi}.}}
         \renewcommand{\itemsep}{0ex}}{\end{enumerate}}         

\newcommand{\FOR}{\textbf{for}}
\newcommand{\FORALL}{\textbf{for all}}
\newcommand{\TO}{\textbf{to}}
\newcommand{\DO}{\textbf{do}}
\newcommand{\OD}{\textbf{od}}
\newcommand{\IF}{\textbf{if}}
\newcommand{\FI}{\textbf{fi}}
\newcommand{\THEN}{\textbf{then}}
\newcommand{\ELSE}{\textbf{else}}
\newcommand{\WHILE}{\textbf{while}}
\newcommand{\REPEAT}{\textbf{repeat}}
\newcommand{\UNTIL}{\textbf{until}}
\newcommand{\OR}{\textbf{or}}
\newcommand{\AND}{\textbf{and}}
\newcommand{\PRINT}{\textbf{print}}
\newcommand{\ACCEPT}{\textbf{accept}}
\newcommand{\REJECT}{\textbf{reject}}



\newcommand{\new}[1]
        {\marginpar{\small\bfseries\ifthenelse{\equal{#1}{}}{!}{#1}}}



\newcommand{\restrict}{\upharpoonright}
\newcommand{\lmodels}{\rotatebox[origin=c]{180}{\ensuremath{\models}}}
\newcommand{\logequiv}{\lmodels \! \! \models}
\newcommand{\bigmid}{\;\big|\;}
\newcommand{\Bigmid}{\;\Big|\;}

\renewcommand{\phi}{\varphi}
\renewcommand{\epsilon}{\varepsilon}

\DeclareMathOperator{\free}{free}
\DeclareMathOperator{\tw}{tw}
\DeclareMathOperator{\ltw}{ltw}
\DeclareMathOperator{\rg}{rg}
\DeclareMathOperator{\TC}{TC}
\DeclareMathOperator{\Reach}{Reach}
\DeclareMathOperator{\sph}{-sph}
\DeclareMathOperator{\dist}{dist}
\DeclareMathOperator{\ind}{index}

\def\doi{3 (4:5) 2007}
\lmcsheading {\doi}
{1--18}
{}
{}
{Dec.~23, 2004}
{Nov.~\phantom{0}5, 2007}
{}   



\begin{document}

\title[Model Checking Synchronized Products]{Model Checking 
Synchronized Products of Infinite Transition Systems\rsuper *}

\author[S. W\"ohrle]{Stefan W\"ohrle}	\address{Informatik 7, RWTH Aachen, 52056 Aachen, Germany}
\email{\{woehrle,thomas\}@informatik.rwth-aachen.de}  

\author[w.~Thomas]{Wolfgang Thomas}	\address{\vskip -6 pt}	\keywords{Model checking, synchronized products, reachability, transitive closure 
logic}
\subjclass{F.4.1}
\titlecomment{{\lsuper *}A
preliminary version of the paper appeared in 19th IEEE Symposium on Logic
in Computer Science, Turku, July 2004 \cite{wt04}.}




\begin{abstract}
  \noindent 
Formal verification using the model checking paradigm
has to deal with two aspects: The system models are structured,
often as products of components, and the specification logic has to be expressive
enough to allow the formalization of reachability properties. The present 
paper is a study on what can be achieved for infinite transition systems 
under these premises. As models we  consider products of infinite transition 
systems with different synchronization constraints. We introduce
finitely synchronized transition systems,  i.e. product systems which contain 
only finitely many (parameterized) synchronized transitions, and show that the decidability of 
FO(R), first-order logic extended by reachability predicates, of the product 
system can be reduced to the decidability of FO(R) of the components.
This result is optimal in the following sense: (1) If we allow  semifinite synchronization, i.e. just in one component infinitely many 
transitions are synchronized, the FO(R)-theory of the product system is in general 
undecidable. (2) We cannot extend the expressive power of the logic 
under consideration. Already a weak extension of first-order logic with transitive 
closure, where we restrict the transitive closure operators to arity one and 
nesting depth two, is undecidable for an asynchronous (and hence finitely synchronized) 
product, namely for the infinite grid. 
\end{abstract}

\maketitle









\section{Introduction}

In the theory of algorithmic verification, a standard framework for 
modeling systems is given by finite transition systems (often in the form 
of Kripke structures). Much effort is presently spent on extending this framework 
to cover infinite transition systems, and to deal adequately with the 
internal structure of the systems under consideration, such as their 
composition from several components. The present paper is a study on 
the scope of algorithmic model checking over transition systems that 
are composed from infinite 
components as products with various constraints on the synchronization of 
their transitions. 
 





We consider transition graphs in the format 
where  is the set of states (or vertices) and  
the set of -labeled transitions. The direct product of two transition graphs has an -labeled transition from 
 to  if there are such transitions from  to  
and from  to . This is the case of complete synchronization. 
The other extreme is the asynchronous product, where a transition in one 
component does not affect the other components. A main result below deals 
with the ``intermediate'' case where the component graphs are infinite and 
in each component 
only finitely many transitions are used for synchronization. We call these product
structures ``finitely synchronized''. They arise whenever the local 
computations in the components involve infinite state-spaces but 
synchronization is restricted to a finite number of actions in each 
 component. 

We study the model checking problem for products of transition graphs 
with respect to several logics that are extensions of first-order logic FO.  
A basic requirement in verification is that reachability properties
should be expressible. There are numerous ways to extend FO by features
that allow to express reachability properties. We consider here four 
extensions that cover reachability relations, listed in the order 
of increasing expressiveness:

\begin{itemize}
\item Reachability logic FO(R), which is obtained from FO-logic by 
 adjoining transitive closure operators  over subsets  of 
 edge relations.
\item FO(Reg) as a generalization of FO(R) in which path labels have to match 
 a given regular expression.
\item Transitive closure logic over binary relations, which allows to proceed 
 from any definable relation (and not just from some edge relations) to 
 its transitive closure.
\item Monadic second-order logic MSO, which results from FO-logic by adjoining 
 variables and quantifiers for sets (and in which transitive closure over 
 binary relations can be expressed).
\end{itemize}

The purpose of this paper is to analyze for which types of products 
and for which of these logics 
the decidability of the model checking problem for a  
product can be inferred from the 
decidability of the corresponding model checking problem for the components. 
In other words, we analyze for which kinds of products the decidability 
of the -theory of the product can be derived from the 
decidability of the -theories of the components. 

Our first result is such a transfer result for the logic FO(R) over 
finitely synchronized products of transition graphs. For this, we 
use a technique of ``composition'' which resembles the method of 
Feferman and Vaught \cite{fv59} in first-order model theory 
(see 
\cite{ck73}, \cite{hodg93} for introductions and \cite{Ma04} for 
a comprehensive survey). The Feferman-Vaught method (applied to FO) 
allows 
to determine the FO-theory of a product structure (e.g., 
a direct product) from the FO-theories of the components and 
some additional information on the index structure. Our proof involves 
a more detailed semantic analysis of the components, thereby exploiting the 
assumption on finite synchronization. The result 
extends a theorem of Rabinovich \cite{rabino07} on propositional modal 
logic extended by the modality EF over asynchronous products.

We show that our result is optimal in two ways.

Firstly, the result does not extend to a case where we allow a slight 
liberalization of the constraint on finite synchronization: We consider 
``semi-finite synchronization'', 
in which all components except one can synchronize via finitely many 
transitions. In the presence of a single component with infinitely many 
synchronizing transitions we may obtain a structure with undecidable 
FO(R) model checking problem, whereas the problem is decidable for the 
components individually. 
 
Secondly, we investigate whether the logic FO(R) can be extended in the 
above mentioned preservation result. For a strong extension like MSO
it is clear that decidability of the component theories does not carry over 
to the theory of the product system. As is well-known, we may work  with 
the asynchronous product of the successor structure of the natural 
numbers, which is the infinite -grid. 
(Note that the asynchronous product is finitely synchronized
with an empty set of synchronizing transitions.) The grid has an undecidable 
monadic theory, whereas the component structures have decidable monadic theories. 

We clarify the situation for weaker extensions of FO(R), 
namely FO(Reg) and transitive closure logic. We show that 
asynchronous products do not preserve the decidability of the FO(Reg)-theory. 
For transitive closure logic this undecidability result can already be obtained
for a very simple example of an asynchronous product, namely the infinite grid 
as considered above.
Moreover, we show that this undecidability phenomenon 
only appears when the TC-operator is nested. For the fragment of 
transitive closure logic with unnested TC-operators interpreted over the 
infinite grid,  we obtain a reduction to Presburger arithmetic and hence 
the decidability of the corresponding theory. 

These undecidability results complement a theorem of Rabinovich \cite{rabino07}
where the corresponding fact is shown for propositional modal logic extended by the
modality EG over finite grids.

In our results the component structures are assumed to have a decidable 
theory in one of the logics considered above. Let us summarize some of the
relevant classes and their closure properties with respect to synchronization.

A fundamental result is that pushdown graphs have
a decidable monadic second-order theory \cite{ms85}. Since then
several extensions like prefix recognizable graphs \cite{ca96} or 
Caucal graphs \cite{cau02} have been considered, see \cite{tho03} for an overview. 
These classes form an increasing sequence in this order, and
all of them enjoy a decidable MSO-theory. None of these classes is closed under 
asynchronous products. 

Two other classes of infinite graphs we like to mention are the graphs of 
ground term rewriting systems \cite{col02}  for which the FO(R)-theory is decidable, 
and ground tree rewriting systems \cite{loe02} 
for which a temporal logic with reachability and recurrence operators is decidable. 
Both classes are closed under asynchronous products.

Classes which are closed under synchronized products are rational graphs
\cite{mo00}, graphs of Thue specifications \cite{pay00}, or graphs
of linear bounded machines \cite{kp99}. However for all these
classes already the FO-theory is undecidable and hence they 
are not suitable for model checking purposes.



The paper is organized as follows. In Section 2 we give the definition
of a synchronized product of a family of graphs or transition systems, 
recall the definition of transitive closure logic, and define FO(R) and FO(Reg).

In Section 3 we show the composition theorem for finitely synchronized 
products and reachability logic and prove that this result cannot be extended 
to FO(Reg) or semifinite synchronization in general.

In Section 4 we investigate transitive closure logic over the infinite grid.
We show that if we allow transitive closure operators of arity one without 
parameters but of nesting depth two the theory of the grid is undecidable.
On the other hand we show that if no nesting of transitive closure operators is
allowed, the respective theory is decidable even in presence of parameters
in the scope of the transitive closure operators. 

\section{Preliminaries}

Let  be a family of sets. We denote by 
 the Cartesian product of these sets.
Tuples  are usually denoted
by , and the th component of  as  .

Let  be a finite set of labels. A \emph{transition system} is a 
-labeled directed graph  where  
is the set of vertices of  and 
denotes the set of -labeled edges in .















\subsection{Synchronized Products}

For  let  be a 
-labeled graph. We assume that  is partitioned into
a set  of \emph{local} labels (or actions) and a set  of
\emph{synchronizing} labels, and to avoid notational complication we require 
the sets of local labels to be pairwise disjoint. An asynchronous transition 
labeled by  is applied only in the -th component of a
state  of the product graph while the other components stay 
fixed. For synchronizing transitions we distinguish explicitly between the 
components where a joint change of states is issued and the components where the 
state does not change. To describe the latter, define
 and . A \emph{synchronization constraint} is a set . If , a -labeled 
transtition induces a simultaneous change in the components  where 
while the states do not change in the other components.

Formally, the \emph{synchronized product} of  defined
by  is the graph  with vertex set , 
asynchronous transitions with labels 
defined by  if  and
 for , and synchronized transitions with labels  defined by  if  for every .
We denote the set of local transitions labels  of  
by , and the set  of all transition labels by .
A product is asynchronous if .

Note that we slightly deviate from the definition in \cite{arn94} 
since we require the sets of local labels and synchronizing labels to be 
disjoint, and implicitly assume an asynchronous behavior of local transitions.
 
Let   be a family of graphs and 
 be a synchronization
constraint. For  let . 
For  we write .
Define 

i.e.  if  and  agree on the synchronizing components.
The synchronized product  of  defined by  is called 
\emph{finitely synchronized} if , i.e. the number 
of equivalence classes of , is finite for every .
In the conference version \cite{wt04} of this paper, finitely synchronized products involve only finitely 
many individual synchronizing transitions, thus disallowing the label  in
the synchronization constraint. In the present treatment we allow finitely many parametrized 
synchronized transitions: The inclusion of constraints  with  means that 
in the -th component the transition  applies to arbitrary states of  and hence 
possibly infinitely many individual synchronizing transitions may be present in a finitely 
synchronized product\footnote{Thus, the proof of Theorem \ref{theo:compos} below involves more technicalities
than the corresponding proof in \cite{wt04}.}.

We collect some technical preparations in the subsequent Lemma \ref{lem:equiv-classes}. For this
we define for every  the eqivalence relation 

and restrict the relation  to the set of vertices of the synchronized product from
which an outgoing transition exists for every , i.e. to the set


\begin{lem}\label{lem:equiv-classes}
Let   be a family of graphs and 
 be a synchronization
constraint. 
\begin{enumerate}[(a)]
\item If  is finitely synchronized, then  is finite for every .
\item For every subset ,
if  and 
there exists a  such that 
and .
\item Let . If 
 , 
 and  then  and the path from 
  to  can be chosen such that no intermediate vertex is -equivalent to 
 .
\end{enumerate}
\end{lem}

\proof
(a) If  is finitely synchronized, then  is finite for every .
    If  then  refines  on . Therefore, 
    for every  the number of equivalence classes of  is bounded by
    .

(c) is a direct consequence of (b) which remains to be 
    shown. Let  and .
    Since transitions labeled with  symbols from  
    commute with transitions labeled by symbols from  
    we may w.l.o.g. assume that the path from  to  is of the form
    
    and  for  and  for .
    Hence by definition of  we have . Thus
    there is a path  in 
    and .
\qed
    

\subsection{First-Order Logic and Extensions}

We assume that the reader is familiar with first-order logic FO 
over graphs. We denote formulas by  to express that 
the free variables of  are among . If  is a graph and
 are the vertices assigned to the variables , we
denote by  or shortly by
 that the formula  is satisfied in 
under the respective variable assignment.

\emph{Transitive closure logic} FO(TC) is defined by extending 
FO with formulas of the type 

where  is a FO(TC)-formula,  are disjoint 
tuples of free variables of the same length ,   are tuples of 
variables of length  and . Note that in the notation 

the variables inside the square brackets are bound while the variables at the end of the formula
occur free.

Let  be a graph, let , , and  be the interpretations of the variables 
, , and  in . Let  be the relation on -tuples defined by 
,
and  be its transitive closure, i.e. 
iff there exists a sequence  such that
,  for , and
. The semantics of the FO(TC)-formula above is defined by


We call the variables  parameters for the transitive closure operator.
By  be denote the fragment of FO(TC) where the 
transitive closure operation is only allowed to define relations over tuples of 
length , i.e. the length of the tuples  in the
definition above is bounded by . For example, in 
we can only define binary relations using a transitive closure operator.
For finite models the arity hierarchy  is 
strict \cite{gro96}.

By  we denote the 
fragment of  where the nesting depth of transitive closure 
operations is bounded by . 

In transitive closure logic we can express that from a vertex
 a vertex  is reachable via a path with labels from some set 
 by 
 

We call the restriction of FO(TC) where the only transitive closure formulas allowed are
of the form  for   \emph{reachability logic} 
and denote it by FO(R).

 The expressive power of the reachability predicates in FO(R) is limited, 
 e.g. we cannot express that there is a path between vertex  and  in 
 the graph whose labels form a word in a given regular language. 

 We denote by FO(Reg) first-oder logic extended by reachability predicates 
  for regular expressions  over , where 
  if there is a path in  from  to 
 labeled by a word contained in the language described by .






\section{Synchronization and FO(R)}

In this section we show that synchronization preserves the decidability 
of the FO(R)-theory if (and only if) the product is finitely synchronized.
For this case we prove a composition theorem that reduces the evaluation
of a formula in the product graph to the evaluation of several formulas in
the component graphs and a Boolean combination of these truth values.
This result does not extend to the case of FO(Reg).

Furthermore we show that \emph{semifinite} synchronization of two components,
where in just  one of the components infinitely many edges are allowed to
be synchronized, does in general not preserve the decidability of the FO(R)-theory.

\begin{thm}\label{theo:compos}
Let  be a finitely synchronized product of a family  of graphs 
with decidable FO(R)-theories. Then the FO(R)-theory of  is also decidable, and for an
FO(R)-formula  we can effectively construct sets of formulas  and a Boolean 
formula  such that  iff  is true under an Boolean interpretation
defined by the truth values of the formulas in .
\end{thm}

\proof
Let  be a family of graphs whose signatures 
 are partitioned into local and synchronizing labels.  
Let  be a synchronization constraint
such that the product  of  is finitely synchronized with respect
to .

We show by induction that for every FO(R)-formula over  there are finite sets 
 of -formulas and a Boolean 
formula  over predicates  
such that 

where  is the Boolean interpretation defined by 

We start with the atomic formulas. For  let , for  
with  let  and  for , and
for  with  let . 
For every formula above let . Obviously (\ref{eq:theo1}) holds in all cases, so the remaining
``atomic'' formulas we have to take care of are of the form 
for . 

For this part of the proof we proceed by induction on the number of synchronizing transitions from 
 which appear in . We may assume that  comprises all local transition labels,
i.e. that ; otherwise in the following every occurrence of  has to be 
replaced by .

We first consider the case that there is only a single synchronizing 
transition . By the definition of finitely synchronized 
product 
we know that  is 
finite, and by Lemma   \ref{lem:equiv-classes} (c) that we have to pass through every equivalence class
at most once. Let . For  and  define

which expresses that on a path from  to  in component  exactly  vertices 
are passed from which a synchronized transition is possible.
For  we set 

and define . 
Setting 
 
ensures (\ref{eq:theo1}) for sets  which contain at most one synchronizing edge label .

Let now . By the induction hypothesis we may assume that for every subset  
there are families of formulas  and Boolean formulas 
 such that (\ref{eq:theo1}) holds, i.e.

Let  and , 
for  let  be a mapping 
and  a mapping .
The number of vertices in  which are passed on the path from vertex  to  is . The 
mapping  then determines the number of  equivalence classes which are passed on the path 
between consecutive vertices in  and  determines the order in which vertices from  
eqivalence classes appear.

Let  be an enumeration of all mappings which can be obtained by composing the mappings 
 and . We define for ,  with  as above 
and  the formula

The Boolean formula  is then defined to be 


We claim now that for every 


We first consider the direction from right to left. Let .
The case  has already been dealt with above. So assume that (\ref{eq:comp-to-show}) holds 
for every . Then  for
some , i.e there exits an  and mappings 
and  such that
for 

If we denote the the valuation of the variables  (respectively ) in  which make the formula above true by  
(respectively ) and their -tuple by  (respectively ) we obtain that 
 for  (here  denotes the
first component of ). Hence
 for  and
since 
also  for .
Hence we obtain .

For the direction from left to right suppose that 
By Lemma \ref{lem:equiv-classes} (c) we know that there is a path from  to  in  which
passes every  equivalence class ot most once. Let  be the
sequence of these vertices from  on the path. We now consider for  the path segments
between  and . Every such path segment can be further decomposed in  the following 
way: Let  be the first vertex in the segment which is contained in some  for . If there is no such  only local labels can appear on the path from  to 
. In this case choose . 

Then we choose  to be the last vertex
on the path from  to  such that ,
i.e.  for some  with .
This decomposition can be continued until  is reached. 

Figure \ref{fig:path-decomp} shows such a decomposition of a path from  to .
Every path segment from  to  is again partintioned as shown. For
sake of readability we mention only the set of synchronizing labels allowed on the intermediate 
paths and write  for .

\begin{figure}[htbp]
  \centering
  \fbox{
  \xymatrix@C=1pc{
   \bar u= \bar y_1 \ar@{~>}[rr]^{C'} && \bar y_2 \ar@{~>}[rr]^{C'} \ar@{--}[dl]
     && y_3 \ar@{~>}[rr]^{C'} \ar@{--}[drr] && \ar@{..}[r] & \ar@{~>}[rr]^{C'} && \bar y_r=\bar v \\
   & \bar y_2 \ar@{~>}[r]^{\Sigma^l} & \bar z_1 \ar@{~>}[r]^{C_1''}& \bar z_2 \ar@{~>}[r]^{C_2''}  
   & \ar@{..}[r] & \ar@{~>}[r]^>>>{C''_q} & \bar z_q = \bar y_3  }
  }
  \caption{Sample decomposition of a path}
  \label{fig:path-decomp}
\end{figure}



By Lemma \ref{lem:equiv-classes} (c)
we again know that the number of intermediate vertices  can be bounded by 
. By the induction hypothesis on subsets  
we know that for every pair of successive vertices  with 
there exists a conjunct of , i.e. some  such that
 
In particular we have  for every 
and all inermediate vertices . 

Combining these decomposition results we obtain that there exists some  bounded by  (the 
number of vertices ), a function  
which determines the number of intermediate vertices  between the  vertices, and a function 
 which determines
to which  an intermediate vertex  belongs and which conjunct of  is satisfied
by the interpretation induced by  and .  Thus we obtain that 
 for  and some  an hence
.

The finishes the proof for atomic formulas. Formulas composed by Boolean connectives and
existential quantification are now easy to handle.

The case of Boolean connectives may be solved in the standard way.
Let  and  be FO(R)-formulas and 
,  as well as , 
be given by the induction hypothesis.  Then, for  we can choose the
same  and the Boolean formula to be , and for
 we choose  and 
.

To finish the proof let . Let 
 and  be the formulas computed for . Let 
 be the set of all satisfying assignments for . For every 
 let . Then sets  for 
are constructed by adding for every  the formula 

Then we can define . 
\qed













For a complexity analysis of this algorithm, note that even in the special case in which the synchronization 
constraint does not contain , the number of formulas which have to be evaluated in the components 
cannot be bounded by an elementary function. This is due to the exponential increase of the sets  which result
from dealing with existential quantifiers. 













It is easy to see that Theorem \ref{theo:compos} also covers
FO(Reg)-formulas with regular expressions built from 
 for  using  and .
However, if we allow reachability predicates with regular expressions of the 
form  the decidability of the corresponding 
theory will be lost.












\begin{thm}
  Asynchronous products do not preserve the decidability of the 
 FO(Reg)-theory.
\end{thm}

\proof
We use a 2-PDA  (pushdown automaton 
with two stacks) that simulates 
a universal Turing machine (cf. \cite{hu79}). 
Formally a 2-PDA is a tuple 

where  is a finite set of states,  and  the input alphabet,
respectively  stack alphabet, 
 is the initial state,  is the final state, and 
 the transition relation. The configuration with 
state  and stack contents  (discarding the stack bottom symbols) 
is denoted by  (similarly a pair
 is a configuration of a standard PDA).  
We assume that Turing 
machines (as well as 2-PDA's) are normalized, i.e. that each state 
is reachable from the 
initial state , the only sink state is the final state  and there
are no incoming transitions to . 

Input words for the universal 2-PDA  are  
of the form w_2\#w_1w_2\mathcal Aw_1\ is written into the first stack 
(in reverse order) and then transferred into the second stack (with the 
first letter of  on top of the stack). With this configuration 
the second phase starts (and we call its initial state ), 
realizing the actual simulation of the universal 
Turing machine. It is well-known that the reachability problem for 
 (``Given w_2\#\mathcal A\mathcal A\mathcal A\delta=(q,a,\gamma_1,\gamma_2,\gamma_3,\gamma_4,p) \in \Delta\mathcal A_1\mathcal A_2\mathcal Br\mathcal A_1\mathcal A_2q'=q''\mathcal A(q,u,v)(q',u',v')w_1\ a first-order formula
w_2\#}(x,y)w_2\#}(x,y)[((q_0,\epsilon),(q_0,\epsilon)),((q,u_1),(q,u_2))]
 \mathcal B \models \exists z_1 \exists z_2 \exists z_3 \Big( \phi_{w_1\\mathcal Aw_1\.
Note that since  is normalized we can ensure that the initial configuration 

and all final configurations  are first-order definable.
\qed
We now turn to the proof that semifinite synchronization in general does not
preserve the decidability of the FO(R)-theory. We reduce the halting problem of 
deterministic Turing machines to the model checking problem for FO(R) for 
synchronized products of finite graphs and infinite graphs which are generated by  
ground tree rewriting systems (GTRS). The GTRS graphs we will construct are of finite out-degree 
and hence have a decidable FO(R)-theory \cite{loe02,loe03}.

The GTRS graph will encode computations of the Turing machine , but not all
of them are valid. We will use the synchronization with a finite graph to eliminate 
computations which are not valid.

Our construction of the GTRS graph encoding computations of  follows
ideas of \cite{loe03}. Before we start the proof we 
give a short definition of the Turing machine model we use and of ground 
tree rewriting systems. For a more detailed description we refer to  
\cite{hu79} and \cite{loe03}.

A \emph{deterministic Turing machine} is a tuple  where
 is a finite set of states,  is an alphabet containing a
designated blank symbol \textvisiblespace~,  is the initial state, 
is the halting state, and  is the transition function. A \emph{configuration} of  is a sequence
 where , 
and  denotes the symbol currently read by the head of the machine. We consider
two configurations to be equivalent if they differ only in heading or trailing blank 
symbols, and do not distinguish between equivalent configurations.

A \emph{ground tree rewriting system} is a tuple 
where  is a ranked alphabet,  is a set of labels for the 
rules,  is a finite set of rules, and  is a finite tree over .
We denote the set of all finite trees over  by . A \emph{rewriting rule}
 is of the form  with  and .
A rule  is applicable to a tree  if there is a subtree  of  equal to ,
and the result of an application of  to  is a tree  obtained from 
by replacing  with .  generates a -labeled graph 
whose vertices are the trees that can be obtained from  by applying 
rewriting rules from , with a -labeled edge between  and  if
 results from  by an application of a rule of the form  .

\begin{thm}\label{theo:semi-undec}
Semifinite synchronization does not preserve the decidability of 
the FO(R)-theory.
\end{thm}

\proof
Let  be a deterministic Turing machine.
We assume that , ,   and
encode a configuration  of  
by a tree 

Every transition of the Turing machine will be simulated by the
rewriting system in two steps, by first rewriting the right branch of
the configuration tree, and then rewriting the left branch.
The labels of the rewriting rules will indicate which letter from 
 has to be added  or removed  from the left branch of
the configuration tree, and  respectively  indicate
whether the halting state has been reached or not.

More precisely we define a GTRS 
where , , ,
 and

The set  is defined by adding for  and 
every  the rules


and for   and every  the rules

where  if  and  otherwise. Note that these rules can only be applied to the
right branch of a configuration tree. For the left branch we add for every  and
 the rules 

as well as 

and

By construction, a path through the graph  generated by  corresponds to a valid computation of 
started on the empty tape iff every transition with label  respectively  is followed by its 
counterpart labeled   respectively . Let  be the star graph with 
 many vertices where the center vertex  has  for every ,a,*) \in \{+,-\} \times 
\Gamma \times\{\bot, \top\}ww(\. 
If we define the synchronization constraint , the
synchronized product of  and  will contain exactly the valid computations of . To decide
whether  halts on the empty tape we thus have to check the truth of the formula 

in the semifinitely synchronized product of  and .
\qed

\section{Transitive Closure Logic over the Infinite Grid}

The infinite grid is the structure 
with two successor relations  and . It can be viewed 
as the asynchronous and hence finitely synchronized product of two copies
of the natural numbers with successor relation,  and 
, defined by the empty synchronization constraint.

We show in this section how to interpret the first-order theory of 
addition and multiplication of the natural numbers  
in FO(TC) (without parameters) over the infinite grid. 
FO(TC) allows only transitive closure operators of arity one and 
a nesting depth of two. 

It is well known that the FO-theory of addition and multiplication 
of  is undecidable. However, since FO(TC) can be interpreted
in MSO,  FO(TC) is decidable over . From these results we can 
conclude that the  FO(TC)-theory is not preserved by finitely
synchronized products and thus obtain that we cannot extend FO(R)
much without losing decidability for finitely synchronized products.

To interpret the theory of addition and multiplication in FO(TC)
over the infinite grid we first connect the transitive closure theories of
 and .

\begin{lem}\label{lem:grid<->N} Let .
\begin{enumerate}[(a)] 
\item For every -sentence  there is a 
-sentence  such that 
.
\item 
For every -sentence  there is a 
-sentence  such that 
.
\end{enumerate}
\end{lem}
\proof
For (a) there is almost nothing to show. It suffices to split every variable
 (interpreted as vertex of the grid) into coordinate variables  and 
(interpreted as natural numbers) and to replace the atomic formulas
 by  and  by  . 

For (b) we identify every  with .
To reduce the number of variables needed in a TC operator 
we represent a pair of variables  by a single variable 
 to be interpreted as a vertex of the grid. 

To finish the proof it suffices to show that the following operations are
  definable:
\begin{enumerate}[(i)]
\item  with  and ,
\item  with  and ,
\item  with 
\end{enumerate}

Then a  formula  

is equivalent to the  formula


where

and in  every occurrence of the symbol  is replaced by .

Let us now define the operations above:


Observe that if the formula  has no TC operators with parameters, then neither
 nor  has (in  only TC-formulas without parameters are introduced),
and that the nesting depth is not increased.
\qed

Let us now turn to the undecidability proof.

\begin{thm}
The -theory of the infinite grid is undecidable.
\end{thm}
\proof 
We define addition and multiplication in  over 
without the use of parameters. By Lemma \ref{lem:grid<->N} it 
is enough to  define these operations in  over 
. The definition of addition is straightforward.


To define multiplication note that ,
hence it suffices to define the square function.  To define  note that 
. The formula

defines all pairs of square numbers 

Hence  iff  for some .
Let 

Then  iff .
\qed

A similar technique was used in \cite{av03} to define multiplication in 
using a transitive closure operator of arity one.

The nesting of transitive closure operators in the previous proof is necessary. If
we disallow nesting, even in the presence of parameters in the transitive closure 
formulas, the theory of the infinite grid is decidable.

\begin{thm}
The  -theory of the infinite grid is decidable.
\end{thm}
\proof
We reduce the -theory of the infinite grid  
to Presburger arithmetic, the first-order theory of , in
the following sense: For every -formula 
one can construct  a Presburger formula  such that

In order to construct  it suffices to consider the case 
 
or for better readability 
 where  is a 
first-order formula. The second notation emphasizes that  serve as 
parameters in the transitive closure formula.

In a first step we rewrite  in a normal form, applying Hanf's Theorem for
first-order logic over graphs (see \cite{ha65,ef95,tho97a}).

For this purpose we recall some definitions. The -sphere 
around a vertex  is the set of grid vertices which are of distance 
less or equal to  from , where we allow to traverse the edges in
either direction. Invoking the distributive normal form and Hanf's Theorem,
there exists a suitable  such that  is equivalent 
to a disjunction of formulas  where each 
 describes the isomorphism type  of  for some tuple  of grid vertices. 
Let  be the set of all such types. Since  is finite
it suffices to consider only finitely many  tuples .

\emph{Remark.} In the general case, over an arbitrary graph instead of the 
infinite grid, Hanf's Theorem involves a statement on the number (up to a certain 
threshold) of spheres outside . This 
statement is superfluous here due to the regular structure of the infinite grid.
(For technical convenience we assume that  is included in the set of 
parameters, so every isomorphism type realizable in  outside 
 occurs an infinite number of times.)

Due to the special structure of the grid, which we depict as a diagram 
with the bottom row and left column as margins, open upwards and 
to the right, every formula 
 can be expressed by conditions on the vertices
 which fix their distances up to the radius  from the
left margin as well as the bottom margin, and their relative distances up to .

It is convenient to express  in terms of the 
components of the vertices, obtaining a formula 

The formula  is interpreted over  and equivalent to  in 
the sense of (\ref{eq:fotc-grid-dec-1}) above. It is a conjunction of statements

\begin{itemize}
\item  for  or 
\item  for 
\item 
\end{itemize}
where  and . 

We now have to evaluate formulas of the form 

for some .

In a first step we note that it is possible to add disjuncts to (\ref{eq:fotc-grid-dec-2})
such that vertices tied to occur in a -sphere around a parameter  for 
only need to appear as start vertex or as end vertex of any path described by (\ref{eq:fotc-grid-dec-2}).
Hence vertices tied to parameters can be handled without the use of TC, by an appropriate 
modification of the formula.

Let  be an initial segment of the grid encompassing the -spheres around parameters 
 for . Outside this initial segment, in a second step, it suffices to 
consider formulas (\ref{eq:fotc-grid-dec-2}) in which only type formulas 
which contain


and 
 
and
 
appear.

It is now possible to apply a finite saturation process to obtain a formula

which is equivalent to (\ref{eq:fotc-grid-dec-2}) and where TC and  commute, i.e.

 
The subformulas  in (\ref{eq:fotc-grid-dec-3}) have the same format as  
the subformulas  in (\ref{eq:fotc-grid-dec-2}) except that the center of 
the excluded -sphere around  may be shifted by a bounded distance 
from  or be missing, or  defines the complete relation 
outside  and the border stripes of width .
Thus it remains to consider two cases.

\noindent \emph{Case 1.}  If  contains a conjunct 
excluding some -sphere then the relation defined by  
is cofinite (w.r.t. the grid excluding  and border stripes of width , or a fixed line in one of the border stripes) 
and hence definable without the use of a transitive closure operator.

\noindent\emph{Case 2.} If   fixes relations of the form

for   and .
the formula   
expresses that there is a path from  to  consisting of steps of the form 
(\ref{eq:fotc-grid-dec-3}). The set of vertices  reachable in this way from  
can be represented as the union of paths in the finite initial segment  of the grid 
and finitely many sets of the form 
Here , the  range over boundary vertices of , 
and the  are from  (\ref{eq:fotc-grid-dec-4}). It follows that 
the relation defined by (\ref{eq:fotc-grid-dec-2}) is definable in Presburger arithmetic.
\qed






\section{Conclusion}













We have proved a result on compositional model checking for 
a logic including reachability predicates, and we have shown 
tight limitations for possible extensions of this result.

Let us mention some questions left open in this paper: 

\begin{enumerate}
\item The composition result (Theorem 3.1) 
should be generalized to infinite products. 
\item For an extension of Theorem 3.1, one can enrich FO(R) by an operator for 
``recurrent reachability'' (existence of an infinite path 
which visits a designated set infinitely often), or one 
can consider stronger logics like (fragments of) CTL.
\item Interesting subcases of Theorem 3.1 should be found where 
the mentioned blow-up of complexity can be avoided.
\item The distinction between products which are asynchronous, 
finitely synchronized, or synchronized should be refined, 
by allowing other means of coordination between component 
structures, also incorporating the special case of synchronization 
of parameterized systems composed from identical components.
\end{enumerate}

\section*{Acknowledgment}

We thank C. L\"oding for pointing us to GTRS-graphs to prove 
Theorem \ref{theo:semi-undec} and
the anonymous referees (both of the conference version and the 
journal version of this paper) for many helpful comments and 
pointers to related literature.




\bibliographystyle{alpha}


 
\bibliography{literatur}

\end{document}
