The set of transitions $T$ is equipped with a partial order $\le$ on the transitions. $t$ is enabled under a marking $M$, if $pre(t) \ge M$, if $cap(t) \ge M+post(t)$ and if 
 all $t’$ being enabled under $M$ we have $t’\le t$.

We first need to investigate the category \cPosets of partially ordered sets. In \cite{Cod07} this category has been examined.

\begin{definition}[Category \cPosets]
   The objects are partially orders sets, given by a set $P$ and a partial order $\le$ over $P$.
	 The morphisms if this category are order-preserving maps, that are maps $f:P_1 \to P_2$  preserving the order, so $x\le y$ implies $f(x) \le f(y)$.
\end{definition}
Composition  and identity are defined as for sets and are both order-preserving,  \cPosets is indeed a category \cite{Cod07}.

The relation to the category of sets can be given by two functors.
The free functor  $F:\cSets \to \cPosets$ is given by $F(M\nach{f}M')= (M,ID_M) \nach{f} /nach{f} (M',ID_{M'}$ where $ID_M$ is the identity relation of a set $M$.
The forgetful functor $V: \cPosets \to \cSets$ is defined by $V( (P,\le_P) \nach{g} (P',\le_{P'})) = P \nach{g} P'$.

\begin{lemma}[Adjunction to \cSets]

\end{lemma}

\begin{proof}
\end{proof}
So, we know that $F$ preserves  colimits ans $V$ preseves limits.


\begin{lemma}[Initial Object and Pushouts in \cPosets]
\label{l.poPosets}
\begin{enumerate}
	\item The initial object is $(\emptyset,\emptyset)$.
	\item Given the span $(P_1,\le_1) \von{f} (P_0,\le_0) \nach{g} (P_2,\le_2)$, then there exists  the pushout
	     $(P_1,\le_1) \nach{g'} (P_3,\le_3) \von{f'} (P_2,\le_2)$.
\end{enumerate}
\end{lemma}

\begin{proof}
\begin{enumerate}
	\item The initial object is $(\emptyset,\emptyset)$ as there is the empty order preserving 
	mapping to each partially orderes set in \cPosets.
	
	\item Given $(P_1,\le_1) \von{f} (P_0,\le_0) \nach{g} (P_2,\le_2)$, then 
	there is in $\cSets$ the span\\
	       $P_1\von{f} P_0 \nach{g} P_2$ and its pushout $P_1 \nach{\bar{g}} \bar{P_3} \von{\bar{f}} P_2$, see pushout $(PO)$ in \ref{eq.SetsPO}				
				 and the relation $R_3 \subseteq \bar{P_3} \times \bar{P_3} $ with 
				\begin{equation}
	         \label{eq.R3}
		       \begin{aligned}
					    (x_3,y_3) \in R_3 &\textrm{ if and only if } \\
					                      & \exists x_1,y_1 \in P_1: \bar{g}(x_1) =x_3 \und \bar{g}(y_1) =y_3 
																                                             \und x_1 \le_1 y_1 \\
					               \oder  & \exists x_2,y_2 \in P_2: \bar{f}(x_2) =x_3 \und \bar{f}(y_2) =y_3 
																                                             \und x_2 \le_2 y_2
				   \end{aligned}
         \end{equation}
				
				Since $R_3$ is not a partial order\footnote{Let $P_0=\{0,5\}$ and $P_1= \{0,3,5\}$ with $f$ the inclusion and $P_2=\{\bullet\}$,
								 then $3 \le_1 5$ yields $([3],[\bullet]) \in R_3$ and $0 \le_1 3$ yields $([\bullet],[3]) \in R_3$, but $[\bullet]=\{0,5\} \neq \{3\}=[3]$.
				        },
				we define the relation $\bar{R_3}$ to be the equivalence closure  of all
				symmetric pairs $\{(x_3,y_3) \mid (x_3,y_3), (y_3,x_3) \in R_3\}  \subseteq R_3$.  Then  we have the quotient
			   $P_3 =  \bar{P_3}_{\mid \bar{R_3}}$ with $g':= [ \_ ] \circ \bar{g}: P_1\to P_3$ and $f':= [ \_ ] \circ \bar{f}: P_2\to P_3$, where $[\_]: \bar{P_3} \to  \bar{P_3}_{\mid \bar{R_3}}=P_3$ is the natural function mapping each element of $\bar{P_3} $ to its equivalence class.\\
$\le_3$  is the transitive closure of \\
								 $\{(x_3,y_3) \mid $\\ \hspace*{5mm}
											$x_1 \le_1 y_1  \textrm{ for } g'(x_1) =x_3 \textrm{ and } g'(y_1) =y_3 $\\ \hspace*{3mm}
											or\\ \hspace*{5mm}
											$
											x_2 \le_2 y_2  \textrm{ for } f'(x_2) =x_3 \textrm{ and } f'(y_2) =y_3 
								\}$
\end{enumerate}
       $\le_3$ is a partial order, as it is reflexive, antisymmetric and transitive and $f'$ and $g'$ are order-preserving maps by construction.\\
So, in $\cPosets$ the category of partially ordered sets  $(P_1,\le_1) \nach{g'} (P_3,\le_3) \von{f'} (P_2,\le_2)$ is the pushout of $(P_1,\le_1) \von{f} (P_0,\le_0) \nach{g} (P_2,\le_2)$:
			
				
				
				\begin{equation}\label{eq.SetsPO}
					$$\xymatrix{
												P_0 \ar[r]^{f} \ar[d]_{g} \ar@{}[dr]|{(PO)}
											&  P_1 \ar[d]^{\bar{g}}  \ar@/^4mm/[dr]|{g':= [ \_ ] \circ \bar{g}} \ar@/^11mm/[ddrr]^{g''}\\
								P_2 \ar[r]_{\bar{f}}  \ar@/_8mm/[rr]|{f':= [ \_ ] \circ \bar{f}} \ar@/_11mm/[drrr]_{f''}
							&  \bar{P_3} \ar[r]|{[ \_ ]}  \ar[drr]|{\bar{h}}
							&  P_3    \ar[dr]^{h} \\
						&&&  P_4
					}
					$$
\end{equation}
Obviously $g' \circ f = f'\circ g$. \\
			For any partially ordered set $(P_4,\le_4)$ with
			$g'' \circ f= f''\circ g$ we have $\bar{h}: \bar{P_3} \to P_4$ in $\cSets$ due to the pushout $(PO)$ in Diagram \ref{eq.SetsPO}. So, we define $h: P_3 \to P_4$ with $h([x]) = \bar{h}(x)$. \\
To prove that $h$ is well-defined we show $h([x_3]) = h([y_3])$ with $x_3 \neq y_3$ 
			but 	$[y_3] = [x3]$.
			
			Since 	$[y_3] = [x3]$ and $x_3 \neq y_3$ there is $(x_3,y_3) \in \bar{R_3}$  and hence $(x_3,y_3) \in R_3$ and $(y_3,x_3) \in R_3$. Due to the definition of $R_3$ there are four cases:
			
			\begin{enumerate}
				\item $\exists x_1,y_1 \in P_1 : x_1 \le_1 y_1 \und \bar{g}(x_1) = x_3 \und \bar{g}(y_1) = y_3 $\\
				      $\und  \; \exists x_2,y_2 \in P_2 : y_2 \le_2 x_2 \und \bar{f}(x_2) = x_3 \und \bar{f}(y_2) = y_3 $:
							
							Then we have $g''(x_1) = \bar{h} \circ \bar{g} (x_1) =  \bar{h} \circ \bar{f} (x_2) = f''(x_2) $
							and $g''(y_1) = \bar{h} \circ \bar{g} (y_1) =  \bar{h} \circ \bar{f} (y_2) = f''(y_2) $.\\
							This yields
							$g''(x_1) \le_4 g''(y_1)$  and $g''(y_1) = f''(y_2) \ge_4 f''(x_2) = g''(x_1)$. Since $\le_4$ is a antisymmetric we have $g''(x_1)= g''(y_1)$.\\
							Hence, we have $h([x_3]) = \bar{h}(x_3) = \bar{h}\circ \bar{g} (x_1) = g''(x_1)= g''(y_1) =
							\bar{h}\circ \bar{g} (y_1) = \bar{h}(y_3) = h([y_3])$. \\
					\item $\exists x_1,y_1 \in P_1 : y_1 \le_1 x_1 \und \bar{g}(x_1) = x_3 \und \bar{g}(y_1) = y_3 $\\
				      $ \und  \; \exists x_2,y_2 \in P_2 : x_2 \le_2 y_2 \und \bar{f}(x_2) = x_3 \und \bar{f}(y_2) = y_3 $ analogously.\\
					\item $\exists x_1,y_1 \in P_1 : x_1 \le_1 y_1 \und \bar{g}(x_1) = x_3 \und \bar{g}(y_1) = y_3 $\\
				      $ \und \; \exists x'_1,y'_1 \in P_1 : y'_1 \le_1 x'_1 \und \bar{g}(x'_1) = x_3 \und \bar{g}(y'_1) = y_3 $:
							
							So, we have $\bar{g}(x_1) = x_3 = \bar{g}(x'_1)$ and $\bar{g}(y_1) = y_3 = \bar{g}(y'_1)$. and 
							   $x_1 \le_1 y_1$ and $ y'_1 \le_1 x'_1$.\\
This yields
							$g''(x_1) \le_4 g''(y_1)$  and $g''(y_1) = g''(y'_1) \le_4 g''(x'_1) = g''(x_1)$. Since $\le_4$ is a antisymmetric we have $g''(x_1)= g''(y_1)$.\\
							Hence, we have $h([x_3]) = \bar{h}(x_3) = \bar{h}\circ \bar{g} (x_1) = g''(x_1)= g''(y_1) =
							\bar{h}\circ \bar{g} (y_1) = \bar{h}(y_3) = h([y_3])$. \\
					\item $\exists x_2,y_2 \in P_2 : x_2 \le_2 y_2 \und \bar{f}(x_2) = x_3 \und \bar{f}(y_2) = y_3 $\\
				      $ \und \; \exists x'_2,y'_2 \in P_2 : y'_2 \le_2 x'_2 \und \bar{f}(x'_2) = x_3 \und \bar{f}(y'_2) = y_3 $
							analogously.
			\end{enumerate}
			Moreover, $h \circ g' = h \circ [\_] \circ  \bar{g} = \bar{h} \circ  \bar{g} = g''$ and 
			$h \circ f' = h \circ [\_] \circ  \bar{f} = \bar{h} \circ  \bar{f} = f''$.
			
\end{proof}

Next we introduce the subclass of monomorphisms $\M$. Monomorphisms in $\cPosets$ are the injective order preserving  maps \cite{Cod07} and order embeddings - those mappings that satisfy item \ref{d.M.i} in Def.~\ref{d.M} -are regular monomorphisms \cite{Cod07}.


\begin{definition}[Class $\M$]
\label{d.M}
The class $\M$ is given by the class of strict order embeddings,  that are order preserving mappings
$f:(P,\le_P) \to (P',\le_{P'})$ that additionally satisfy :

\begin{enumerate}
	\item \label{d.M.i} $x\le_P y$ if and only if $f(x) \le_{P'} f(y)$ for $x,y  \in P$ 
	\item \label{d.M.ii} for each $  z' \in P'$ with $f(x) \le_{P'} z'  \le_{P'} f(y)$  there exists some $z \in P$ with $f(z) = z'$ (and hence $x \le_P z \le_P y$).
\end{enumerate}
\end{definition}

Class $\M$ leads to pushouts that are constructed as in the category $\cSets$, hence the forgetful functor
$V:\cPosets\to \cSets$ preserves pushouts.


\begin{lemma}[$\M$-Pushouts in \cPosets]
Given $(P_1,\le_1) \von{f} (P_0,\le_0) \nach{g} (P_2,\le_2)$  with $f\in \M$ then there is the pushout
	     $(P_1,\le_1) \nach{g'} (P_3,\le_3) \von{f'} (P_2,\le_2)$,
			   such that in $\cSets$ $P_1 \nach{g'} P_3 \von{f'} P_2$  is the pushout of 
					        $P_1\von{f} P_0 \nach{g} P_2$.\\
				Moreover,  \M \ is stable under pushouts.
\end{lemma}


\begin{proof}
Obviously, the construction of $\bar{R_3}$ in the proof of Lemma \ref{l.poPosets} yields for $f \in \M$ that 
$\bar{R_3} =ID$ the identity relation. Hence , $\bar{P_3}= \bar{P_3}_{\mid \bar{R_3}}=P_3$.\\
Moreover, its is \M-stable:\\
For $f \in \M$ in Diagram \ref{eq.SetsPO} we know that $f'$ is injective, as pushouts in $\cSets$ preserve monomorphisms, i.e. injective mappings and it is order-preserving by construction. \\
$f'$ is an order embedding: \\
For $x_2, y_2 \in P_2$ and $f'(x_2) \le_3 f'(y_2)$ we have due to the construction of $\le_3$ four cases:

\begin{enumerate}
	\item There are $x_1,  y_1 \in P_1$ with $x_1 \le_1 y_1$ so that $g'(x_1) = f'(x_2)$ and $g'(y_1) = f'(y_2)$.
	      Due to the pushout construction there are  $x_0,  y_0 \in P_0$ with $x_0 \le_0 y_0$ 
				  so that $f(x_0) = x_1$ and $g(x_1) = x_2$ and  $f(y_0) = y_1$ and $g(y_1) = y_2$. Since $g$ is order preserving, we have $x_2 \le_2 y_2$.\\					
	\item There is $x_2 \le_2 y_2$.\\
	\item There is $z_3 \in P_3$ with $f'(x_2) \le_3 z_3 \le_3 f'(y_3)$, so that there are $x_1 \le_1 z_1$ with
	      $g'(x_1) = f'(x_2)$  and $g'(z_1) = z_3$ and $z_2 \le_2 y_2$ and $f'(z_2) =z_3$.\\
				Due to the pushout construction there are  $x_0,  z_0 \in P_0$ with $x_0 \le_0 z_0$ 
				  so that $f(x_0) = x_1$ and $g(x_1) = x_2$ and  $f(z_0) = z_1$ and $g(z_0) =z_2$.  Since $g$ is order preserving, we have $x_2 \le_2 z_2 \le y_2$.\\	
	\item There is $z_3 \in P_3$ with $f'(x_2) \le_3 z_3 \le_3 f'(y_3)$, so that there are $ z_1 \le_1 y_1$ with
	      $g'(y_1) = f'(y_2)$  and $g'(z_1) = z_3$ and $x_2 \le z_2$ and $f'(z_2) =z_3$ analogously.	
\end{enumerate}
$f'$ is a strict order embedding:\\ 
Let be  $x_2, y_2 \in P_2$ and $f'(x_2) \le_3 z_3 \le_3 f'(y_2)$ given for $z_3 \in P_3$.
Either $z_3 \in f'(P_2)$ and hence there is $f'(z_2)=z_3$ with $x_2 \le_2 \le_2 y_2$ or $z_3 \not \in f'(P_2)$.
Then there are $x_1,y_1, z_1,z'_1 \in P_1$ with $g'(x_1) = f'(x_2)$ and $g'(y_1) = f'(y_2)$ and  $g'(z_1) = z_3 = g(z_1)$ and $x_1 \le z_1$ and $z'_1 \le_1 y_1$. Due to the pushout construction there are  $x_0,  y_0 \in P_0$ with  $f(x_0) = x_1$ and $g(x_1) = x_2$ and  $f(y_0) = y_1$ and $g(y_1) = y_2$. Since $f$ is a strict order embedding we have 
additionally, $z_0,z'_0$ with $f(z_0) = z_1$ and $f(z'_0) = z'_1$ and $x_0 \le z_0 \le z'_0 \le y_0$. Due to pushout construction $g(z_0) = g(z'_0)$ and as $g$ is order preserving we have $x_2 = g(x_0) \le _2 g(z_0) \le_2 g(y_0) = y_2$
with $f'(g(z_0)) = z_3$.
\end{proof}

Next we investigat pullbacks in $\cPosets$.
\begin{lemma}[Pullbacks in \cPosets]
Given $(P_1,\le_1) \nach{g} (P_0,\le_0) \von{f} (P_2,\le_2)$  then there is the pullback
	     $(P_1,\le_1) \von{f'} (P_3,\le_3)  \nach{g'} (P_2,\le_2)$.
				Moreover,  \M \ is stable under pullbacks.
\end{lemma}
\begin{proof}
There is the pullback   $P_1 \von{f'} P_3 \nach{g'} P_2$ of $P_1 \nach{g} P_0 \von{f} P_2$   in $\cSets$.
$(P_1,\le_1) \von{f'} (P_3,\le_3)  \nach{g'} (P_2,\le_2)$ with $x_3 \le y_3$ if and only if $f'(x_3) \le_1 f'(y_3)$ and $g'(x_3) \le_1 g'(y_3)$ is pullback in \cPosets. Obviously, $f'$ and $g'$ are order-preserving mappings.

 \M-morphisms are monomorphisms and hence are preserved by pullbacks. 
\end{proof}


\begin{theorem}[$\cPosets$  is $\M$-Adhesive HLR Category.]
  \label{l.madHLR.posets}
\end{theorem}

\begin{proof}~

	\begin{enumerate}
		\item The class $\M$ in $\cPosets$  is  PO-PB compatible, since
		\begin{itemize}
			\item pushouts along \M-morphisms exist and \M \ is stable under pushouts,
		  \item pullbacks along \M-morphisms exist and \M \ is stable under pullbacks and
		  \item obviously, \M \ contains all identities and is closed under composition.
	\end{itemize}
	\item  In $\cPosets$ pushouts along \M-morphisms are $\M$-VK squares:
	  Let be given as : a  pushout (1) with $m \in \M$  and some commutative cube (2) with (1) in
the bottom and back faces being pullbacks in $\cPosets$.


 $
   \xymatrix@=3mm{
   			&\\
                       A  \ar[rr]|{m\in\M} \ar[dd]|{f}    
	         && B                   \ar[dd]|{g}     \\ 
	            & (1)  \\
		       C  \ar[rr]|{n}
		    && D
		 }
     $ \hfill
     $
    \xymatrix@=3mm{
                     &&&    A'	 \ar[ddd]|<<<<<{a}    \ar[dlll]|{f'}    \ar[drr]|{m'}     
		         &&  (2) \\
		                C'      \ar[ddd]|{c}                    \ar[drr]|{n'}
		   &&&&& B'      \ar[ddd]|{b}     \ar[dlll]|<<<<<{g'}          
		           &&                                                                \\
		          && D'      \ar[ddd]|<<<<<{d}                                        \\
                        &&& A                       \ar[dlll]|>>>>>{f}     \ar[drr]|{m}           \\
		                C                                      \ar[drr]|{n}
		   &&&&& B                       \ar[dlll]|{g}                       
		           &&                                                              \\
		          && D
		   } 	        	 
  $	
 
 \begin{description}
	 \item[$\Rightarrow$:] Let the top be a pushout in $\cPosets$. Pullbacks preserve  \M-morphisms, so  $m'\in \M$ and  
	      hence the top square is a pushout in $\cSets$ as well. As $\cSets$ is adhesive, the front faces are pullbacks in $\cSets$ as well.
				Since the construction of pullbacks coincides in $\cSets$ and $\cPosets$, the front faces are pullbacks in $\cPosets$.
	
   \item[$\Leftarrow$:] Let the front faces be pullbacks in $\cPosets$, and hence pullbacks in $\cSets$. Since $m \in \M$
	      (1) is pushout in $\cSets$ as well. So, $\cSets$ being adhesive, we have the top square being a pushout 
				in $\cSets$. Moreover, $M'\in \M$ as the back face is a pullback  preserving \M-morphisms. So, the top
				is a pushout along $\M$ is $\cPosets$.
 \end{description}

\end{enumerate}
    Hence, by Def. \label{d.madHLR}  $(\cPosets,\M)$ is an $\M$-adhesive HLR-category.
\end{proof}



\begin{definition}
   The category of  place/transition nets with  transition priorities $\cPTp$  is given by $N=(P,(T,\le_T), pre, post, m_0)$
	with $pre, post: V(T,\le_T) \to P^\oplus$ and
	morphisms $f_P,f_T: N_1 \to N_2$ where $f_P$ is a mapping and $f_T$ is an order-preserving map.
	
	A transition $t\in T$ is enabled under a marking $m$, if $pre(t) \ge m $ and if for all $t’\in T$ being enabled under $m$ we have $t'\le_T t$.
\end{definition}

\begin{lemma}[$(\cPTp,\M)$ is an $\M$-adhesive HLR-category]
  \label{l.madHLr_PTp}
	 with $\M$ the net morphisms where $f_p$ is strict injective and $f_T$
	is a strict order embedding. 
  \end{lemma}
	
	\begin{proof}
The proof applies the construction for weak adhesive HLR categories (see Theorem 1 in \cite{PEL08}):\\
	 We know that  $(\cSets,\M)$  with $\M$ being the injective mappings is an $\M$-adhesive
HLR category and that $ (\_)^\oplus: \cSets \to \cSets$ preserves pullbacks along injective morphisms.
 As shown above $(\cPosets,\M)$  with $\M$ being the strict order embeddings is an $\M$-adhesive  
HLR category and that $V: \cPosets \to \cSets$ preserves pushouts along $\M$-morphisms.
So, the category $cPTp$ is isomorphic to the comma category $ComCat(V,(\_)^\oplus;I)$ with I = {1,2}, where $V: \cPosets \to \cSets$ is the forgetful functor from partial ordered sets to sets and $(\_)^\oplus$ is the free commutative
monoid functor and hence  an $\M$-adhesive.  
HLR category.
	\end{proof}


\begin{lemma}[$(\cdPTip,\M)$ is an $\M$-adhesive HLR-category]
  \label{l.madHLr_PTp}
	 with $\M$ the net morphims where $f_p$ is strict injective and $f_T$
	is a strict order embedding. 
  \end{lemma}
	\begin{proof}
	Similar to the proof of Lemma 1 in \cite{Pad12} using $\cPTp$ instead of $ \cPT$ as the basis.\end{proof}