The set of transitions  is equipped with a partial order  on the transitions.  is enabled under a marking , if , if  and if 
 all  being enabled under  we have .

We first need to investigate the category \cPosets of partially ordered sets. In \cite{Cod07} this category has been examined.

\begin{definition}[Category \cPosets]
   The objects are partially orders sets, given by a set  and a partial order  over .
	 The morphisms if this category are order-preserving maps, that are maps   preserving the order, so  implies .
\end{definition}
Composition  and identity are defined as for sets and are both order-preserving,  \cPosets is indeed a category \cite{Cod07}.

The relation to the category of sets can be given by two functors.
The free functor   is given by  where  is the identity relation of a set .
The forgetful functor  is defined by .

\begin{lemma}[Adjunction to \cSets]

\end{lemma}

\begin{proof}
\end{proof}
So, we know that  preserves  colimits ans  preseves limits.


\begin{lemma}[Initial Object and Pushouts in \cPosets]
\label{l.poPosets}
\begin{enumerate}
	\item The initial object is .
	\item Given the span , then there exists  the pushout
	     .
\end{enumerate}
\end{lemma}

\begin{proof}
\begin{enumerate}
	\item The initial object is  as there is the empty order preserving 
	mapping to each partially orderes set in \cPosets.
	
	\item Given , then 
	there is in  the span\\
	        and its pushout , see pushout  in \ref{eq.SetsPO}				
				 and the relation  with 
				
				
				Since  is not a partial order\footnote{Let  and  with  the inclusion and ,
								 then  yields  and  yields , but .
				        },
				we define the relation  to be the equivalence closure  of all
				symmetric pairs .  Then  we have the quotient
			    with  and , where  is the natural function mapping each element of  to its equivalence class.\\
  is the transitive closure of \\
								 \\ \hspace*{5mm}
											\\ \hspace*{3mm}
											or\\ \hspace*{5mm}
											
\end{enumerate}
        is a partial order, as it is reflexive, antisymmetric and transitive and  and  are order-preserving maps by construction.\\
So, in  the category of partially ordered sets   is the pushout of :
			
				
				
				\xymatrix{
												P_0 \ar[r]^{f} \ar[d]_{g} \ar@{}[dr]|{(PO)}
											&  P_1 \ar[d]^{\bar{g}}  \ar@/^4mm/[dr]|{g':= [ \_ ] \circ \bar{g}} \ar@/^11mm/[ddrr]^{g''}\\
								P_2 \ar[r]_{\bar{f}}  \ar@/_8mm/[rr]|{f':= [ \_ ] \circ \bar{f}} \ar@/_11mm/[drrr]_{f''}
							&  \bar{P_3} \ar[r]|{[ \_ ]}  \ar[drr]|{\bar{h}}
							&  P_3    \ar[dr]^{h} \\
						&&&  P_4
					}
					
Obviously . \\
			For any partially ordered set  with
			 we have  in  due to the pushout  in Diagram \ref{eq.SetsPO}. So, we define  with . \\
To prove that  is well-defined we show  with  
			but 	.
			
			Since 	 and  there is   and hence  and . Due to the definition of  there are four cases:
			
			\begin{enumerate}
				\item \\
				      :
							
							Then we have 
							and .\\
							This yields
							  and . Since  is a antisymmetric we have .\\
							Hence, we have . \\
					\item \\
				       analogously.\\
					\item \\
				      :
							
							So, we have  and . and 
							    and .\\
This yields
							  and . Since  is a antisymmetric we have .\\
							Hence, we have . \\
					\item \\
				      
							analogously.
			\end{enumerate}
			Moreover,  and 
			.
			
\end{proof}

Next we introduce the subclass of monomorphisms . Monomorphisms in  are the injective order preserving  maps \cite{Cod07} and order embeddings - those mappings that satisfy item \ref{d.M.i} in Def.~\ref{d.M} -are regular monomorphisms \cite{Cod07}.


\begin{definition}[Class ]
\label{d.M}
The class  is given by the class of strict order embeddings,  that are order preserving mappings
 that additionally satisfy :

\begin{enumerate}
	\item \label{d.M.i}  if and only if  for  
	\item \label{d.M.ii} for each  with   there exists some  with  (and hence ).
\end{enumerate}
\end{definition}

Class  leads to pushouts that are constructed as in the category , hence the forgetful functor
 preserves pushouts.


\begin{lemma}[-Pushouts in \cPosets]
Given   with  then there is the pushout
	     ,
			   such that in    is the pushout of 
					        .\\
				Moreover,  \M \ is stable under pushouts.
\end{lemma}


\begin{proof}
Obviously, the construction of  in the proof of Lemma \ref{l.poPosets} yields for  that 
 the identity relation. Hence , .\\
Moreover, its is \M-stable:\\
For  in Diagram \ref{eq.SetsPO} we know that  is injective, as pushouts in  preserve monomorphisms, i.e. injective mappings and it is order-preserving by construction. \\
 is an order embedding: \\
For  and  we have due to the construction of  four cases:

\begin{enumerate}
	\item There are  with  so that  and .
	      Due to the pushout construction there are   with  
				  so that  and  and   and . Since  is order preserving, we have .\\					
	\item There is .\\
	\item There is  with , so that there are  with
	        and  and  and .\\
				Due to the pushout construction there are   with  
				  so that  and  and   and .  Since  is order preserving, we have .\\	
	\item There is  with , so that there are  with
	        and  and  and  analogously.	
\end{enumerate}
 is a strict order embedding:\\ 
Let be   and  given for .
Either  and hence there is  with  or .
Then there are  with  and  and   and  and . Due to the pushout construction there are   with   and  and   and . Since  is a strict order embedding we have 
additionally,  with  and  and . Due to pushout construction  and as  is order preserving we have 
with .
\end{proof}

Next we investigat pullbacks in .
\begin{lemma}[Pullbacks in \cPosets]
Given   then there is the pullback
	     .
				Moreover,  \M \ is stable under pullbacks.
\end{lemma}
\begin{proof}
There is the pullback    of    in .
 with  if and only if  and  is pullback in \cPosets. Obviously,  and  are order-preserving mappings.

 \M-morphisms are monomorphisms and hence are preserved by pullbacks. 
\end{proof}


\begin{theorem}[  is -Adhesive HLR Category.]
  \label{l.madHLR.posets}
\end{theorem}

\begin{proof}~

	\begin{enumerate}
		\item The class  in   is  PO-PB compatible, since
		\begin{itemize}
			\item pushouts along \M-morphisms exist and \M \ is stable under pushouts,
		  \item pullbacks along \M-morphisms exist and \M \ is stable under pullbacks and
		  \item obviously, \M \ contains all identities and is closed under composition.
	\end{itemize}
	\item  In  pushouts along \M-morphisms are -VK squares:
	  Let be given as : a  pushout (1) with   and some commutative cube (2) with (1) in
the bottom and back faces being pullbacks in .


  \hfill
     	
 
 \begin{description}
	 \item[:] Let the top be a pushout in . Pullbacks preserve  \M-morphisms, so   and  
	      hence the top square is a pushout in  as well. As  is adhesive, the front faces are pullbacks in  as well.
				Since the construction of pullbacks coincides in  and , the front faces are pullbacks in .
	
   \item[:] Let the front faces be pullbacks in , and hence pullbacks in . Since 
	      (1) is pushout in  as well. So,  being adhesive, we have the top square being a pushout 
				in . Moreover,  as the back face is a pullback  preserving \M-morphisms. So, the top
				is a pushout along  is .
 \end{description}

\end{enumerate}
    Hence, by Def. \label{d.madHLR}   is an -adhesive HLR-category.
\end{proof}



\begin{definition}
   The category of  place/transition nets with  transition priorities   is given by 
	with  and
	morphisms  where  is a mapping and  is an order-preserving map.
	
	A transition  is enabled under a marking , if  and if for all  being enabled under  we have .
\end{definition}

\begin{lemma}[ is an -adhesive HLR-category]
  \label{l.madHLr_PTp}
	 with  the net morphisms where  is strict injective and 
	is a strict order embedding. 
  \end{lemma}
	
	\begin{proof}
The proof applies the construction for weak adhesive HLR categories (see Theorem 1 in \cite{PEL08}):\\
	 We know that    with  being the injective mappings is an -adhesive
HLR category and that  preserves pullbacks along injective morphisms.
 As shown above   with  being the strict order embeddings is an -adhesive  
HLR category and that  preserves pushouts along -morphisms.
So, the category  is isomorphic to the comma category  with I = {1,2}, where  is the forgetful functor from partial ordered sets to sets and  is the free commutative
monoid functor and hence  an -adhesive.  
HLR category.
	\end{proof}


\begin{lemma}[ is an -adhesive HLR-category]
  \label{l.madHLr_PTp}
	 with  the net morphims where  is strict injective and 
	is a strict order embedding. 
  \end{lemma}
	\begin{proof}
	Similar to the proof of Lemma 1 in \cite{Pad12} using  instead of  as the basis.\end{proof}