
\documentclass[twocolumn]{autart}    

\usepackage{graphicx}

\usepackage[sort,compress]{cite}

\usepackage{latexsym}
\usepackage{psfrag}
\usepackage{amssymb,amscd,amssymb,amsmath,mathrsfs}\usepackage{overpic}
\usepackage{mdwlist}           \usepackage{ifpdf}
\usepackage{caption2} \usepackage{url} \usepackage{float}
\usepackage{color}



\makeatletter
\newcommand{\rmnum}[1]{\romannumeral #1}
\newcommand{\Rmnum}[1]{\expandafter\@slowromancap\romannumeral #1@}
\makeatother


\hyphenation{op-tical net-works semi-conduc-tor}


\begin{document}

\begin{frontmatter}

\title{Supervision Localization of Timed Discrete-Event Systems}

\thanks[footnoteinfo]{This work was supported
in part by the State Key Laboratory of Electrical Insulation and
Power Equipment (China), and by the Natural Sciences and Engineering
Research Council (Canada), Grant no. 7399.}

\author[XJTU]{Renyuan Zhang}\ead{r.yuan.zhang@gmail.com},    \author[UofT]{Kai Cai}\ead{kai.cai@scg.utoronto.ca},               \author[XJTU]{Yongmei Gan}\ead{ymgan@mail.xjtu.edu.cn},  \author[XJTU]{Zhaoan Wang}\ead{zawang@mail.xjtu.edu.cn},
\author[UofT]{W.M. Wonham}\ead{wonham@control.utoronto.ca}

\address[XJTU]{School of Electrical Engineering, Xi'an Jiaotong University, Xi'an, Shaanxi 710049, China}  \address[UofT]{Department of Electrical and Computer Engineering, University of Toronto, Toronto, ON M5S 3G4, Canada}    


\begin{keyword}                           Supervisory control; Supervisor localization; Timed discrete-event systems.               \end{keyword}                             

\begin{abstract}                          We study supervisor localization for real-time discrete-event
systems (DES) in the Brandin-Wonham framework of timed supervisory
control. We view a real-time DES as comprised of asynchronous agents
which are coupled through imposed logical and temporal
specifications; the essence of supervisor localization is the
decomposition of monolithic (global) control action into local
control strategies for these individual agents. This study extends
our previous work on supervisor localization for untimed DES, in
that monolithic timed control action typically includes not only
disabling action as in the untimed case, but also ``clock
preempting'' action which enforces prescribed temporal behavior. The
latter action is executed by a class of special events, called
``forcible'' events; accordingly, we localize monolithic preemptive
action with respect to these events. We demonstrate the new features
of timed supervisor localization with a manufacturing cell case
study, and discuss a distributed control implementation.
\end{abstract}

\end{frontmatter}

\section{Introduction} \label{sec:intro}

Recently we developed a top-down approach, called \emph{supervisor
localization} \cite{CaiWonham:2010a,CaiWonham:2010b} to the
distributed control of untimed discrete-event systems (DES) in the
Ramadge-Wonham (RW) supervisory control framework
\cite{RamadgeWonham:87,Wonham:2011a}. We view the plant to be
controlled as comprised of independent asynchronous agents which are
coupled implicitly through logical control specifications. To make
the agents smart and semi-autonomous, our localization algorithm
allocates \emph{external} supervisory control action to individual
agents as their \emph{internal} control strategies, while preserving
the optimality (maximal permissiveness) and nonblocking properties
of the overall monolithic (global) controlled behavior. Under the
localization scheme, each agent controls only its own events,
although it may very well need to observe events originating in
other (typically neighboring) agents.

In this paper we extend the supervisor localization theory to a
class of \emph{real-time} DES, and address distributed control
problems therein. Many time-critical applications can be modeled as
real-time DES, such as communication channels, sensor networks,
scheduling and resource management \cite{LeeLeuSon:07}. Typical
timing features include communication delays and operational hard
deadlines. The correctness and optimality of real-time DES depend
not only on the system's logical behavior, but also on the times at
which various actions are executed. Moreover, rapid advances in
embedded, mobile computation and communication technologies
\cite[Part~III]{LeeLeuSon:07} have enabled distributed
implementation of control algorithms.  These developments jointly
motivate this study of supervisor localization for real-time DES.

A variety of real-time DES models and approaches are available.
Notable works include Brave and Heymann's ``clock automata''
\cite{BraveHeymann:88}, Ostroff's ``timed transition models''
\cite{Ost:90}, Brandin and Wonham's timed DES (TDES)
\cite{BrandinWonham:94}, Cassandras's ``timed state automata''
\cite{Cassandras:93}, Wong-Toi and Hoffman's model based on ``timed
automata'' \cite{WongToiHoffman:91}, and Cofer and Garg's model
based on ``timed Petri nets'' \cite{CoferGarg:96}. We adopt Brandin
and Wonham's TDES (or BW model) as the framework for developing a
timed supervisor localization theory for two reasons. First, the BW
model is a direct extension from the RW framework (where our untimed
localization theory is based), retaining the central concepts of
controllability, and maximally permissive nonblocking supervision.
This feature facilitates developing a timed counterpart of
supervisor localization. Second, the BW model captures a variety of
timing issues in a useful range of real-time discrete-event control
problems \cite{BrandinWonham:94},\cite[Chapter~9]{Wonham:2011a}.
While it may be possible to develop supervisor localization in an
alternative framework, as a preliminary step into real-time
supervisor localization we choose the BW model for its close
relation with previous work.

The principal contribution of this paper is the development of a
timed supervisor localization theory in the BW TDES framework, which
extends the untimed counterpart in
\cite{CaiWonham:2010a,CaiWonham:2010b}.  In this timed localization,
a novel feature is ``event forcing'' as means of control, in
addition to the usual ``event disabling''. Specifically,
``forcible'' events are present in the BW model as events that can
be relied on, when subject to some temporal specification, to
``preempt the tick of the clock'', as explained further in
Section~\ref{sec:2}. Correspondingly, in localizing the monolithic
supervisor's control action, we localize not only its disabling
action as in the untimed case, but also its preemptive action with
respect to individual forcible events.  Central to the
latter are several new ideas: ``local preemptor'', ``preemption
consistency relation'', and ``preemption cover''. We will prove
that localized disabling and preemptive behaviors collectively
achieve the same global optimal and nonblocking controlled behavior
as the monolithic supervisor does.  The proof relies on
the new preemption concepts and also controllability for TDES.
Moreover, the derived local controllers typically have much smaller
state size than the monolithic supervisor, and hence their disabling
and preemptive logics are often more transparent. We demonstrate
this empirical result by a case study of a manufacturing cell taken
from \cite{BrandinWonham:94}.

The paper is organized as follows. Section~\ref{sec:2} provides a
review of the BW TDES framework. Section~\ref{sec:3}
formulates the timed supervisor localization problem, and
Section~\ref{sec:4} presents a constructive solution procedure.
Section~\ref{sec:5:exmp} studies a manufacturing cell example; and finally,
Section~\ref{sec:conclusion} draws conclusions.


\section{Preliminaries on Timed Discrete-Event Systems} \label{sec:2}

This section reviews the TDES model proposed by Brandin and Wonham
\cite{BrandinWonham:94},\cite[Chapter~9]{Wonham:2011a}. First
consider the untimed DES model

Here  is the finite set of {\it activities},  is
the finite set of {\it events},  is the (partial) {\it activity transition function},  is the {\it initial activity}, and  is the set
of {\it marker activities}. Let  denote the set of
natural numbers . We introduce \emph{time} into  by assigning to each event  a
{\it lower time bound}  and an {\it upper
time bound} , such that
; typically,  represents a delay
in communication or in control enforcement, while  is
often a hard deadline imposed by legal specification or physical
necessity. With these assigned time bounds, the event set
 is partitioned into two subsets:  ( denotes
\emph{disjoint union}) with  and ; here ``spe''
denotes ``prospective'', i.e.  will occur within some
prospective time (with a finite upper bound), while ``rem'' denotes
``remote'', i.e.  will occur at some \emph{indefinite} time
(with no upper bound), or possibly will never occur at all.

A distinguished event, written , is introduced which
represents ``tick of the global clock''. Attach to each event
 a (countdown) {\it timer} , whose default value  is set to be

When timer , it decreases by 1 (counting down) if event 
occurs; and when , event  must occur (resp.
may occur) if  (resp. if ). Note that while  is a global event, each timer
 is local (with respect to the event ). Also
define the {\it timer interval}  by
 \label{e4}
{\bf G} := (Q, \Sigma, \delta, q_0, Q_m),
 \label{eq:natpro}
\begin{split}
P(\epsilon) &= \epsilon, \ \ \epsilon \mbox{ is the empty string;} \\
P(\sigma) &= \left\{
  \begin{array}{ll}
    \epsilon, & \hbox{if ,} \\
    \sigma, & \hbox{if ;}
  \end{array}
\right.\\
P(s\sigma) &= P(s)P(\sigma),\ \ s \in \Sigma^*, \sigma \in \Sigma.
\end{split}
 \label{e6}
L({\bf G}) := \{s \in \Sigma^*|\delta(q_0,s)!\}
 \label{e7}
L_m({\bf G}) := \{s \in L({\bf G})| \delta(q_0, s) \in Q_m\} \subseteq L({\bf G}).
 \label{e8}
\Sigma_c := \Sigma_{hib}~\dot\cup~\{tick\}.
 \label{e8}
\Sigma_{u} := \Sigma - \Sigma_c =  \Sigma_{spe} \dot\cup
(\Sigma_{rem} - \Sigma_{hib}).
 \label{e11}
Elig_{\bf G}(s):=\{\sigma \in \Sigma|s\sigma \in L({\bf G})\}
 \label{e12}
Elig_F(s):= \{\sigma \in \Sigma|s\sigma \in \overline{F}\},
 \label{eq:control}
Elig_F(s)\supseteq
\left\{
   \begin{array}{lcl}
      Elig_{\bf G}(s)\cap(\Sigma_u \dot{\cup}\{tick\}) \\
      ~~~~~~~~~~~~~~~~~\text{if} ~~Elig_F(s)\cap \Sigma_{for} = \emptyset,\\
      Elig_{\bf G}(s)\cap\Sigma_u               \\
      ~~~~~~~~~~~~~~~~~\text{if} ~~Elig_F(s)\cap \Sigma_{for} \neq \emptyset.
   \end{array}
\right.
\label{e12a}
{\bf SUP} = (X, \Sigma, \xi, x_0, X_m)
 \label{e13}
L_m({\bf SUP}) = sup\mathcal{C}(E\cap L_m({\bf G})) \subseteq
L_m(\bf G)
 &s.tick \in L({\bf G}) ~\&~ s \in P^{-1}_{\alpha} L({\bf
LOC}^P_{\alpha})  ~\&~ \\
&s.tick \notin P^{-1}_{\alpha} L({\bf
LOC}^P_{\alpha})  \Rightarrow  s\alpha \in L({\bf G}) \cap
P^{-1}_{\alpha} L({\bf LOC}^P_{\alpha}).

s\sigma \in L({\bf G}) ~\&~ &s\in P^{-1}_{\alpha} L({\bf
LOC}^C_{\beta}) ~\&~ \\
 &s\sigma \notin P^{-1}_{\alpha} L({\bf
LOC}^C_{\beta})  \Rightarrow  \sigma = \beta.

   L({\bf LOC}) := \left(\mathop \bigcap\limits_{\alpha \in \Sigma_{for}}P_\alpha^{-1}L({\bf LOC}^P_{\alpha}) \right) \notag\\
   \cap \left(\mathop \bigcap\limits_{\beta \in \Sigma_{hib}}P_\beta^{-1}L({\bf LOC}^C_{\beta}) \right) \label{e20}\\
   L_m({\bf LOC}) := \left(\mathop \bigcap\limits_{\alpha \in \Sigma_{for}}P_\alpha^{-1}L_m({\bf LOC}^P_{\alpha})\right)\notag \\
   \cap \left(\mathop \bigcap\limits_{\beta \in \Sigma_{hib}}P_\beta^{-1}L_m({\bf LOC}^C_{\beta}) \right) \label{e21}

   L({\bf G}) \cap L({\bf LOC}) &= L({\bf SUP}), \\
   L_m({\bf G}) \cap L_m({\bf LOC}) &= L_m({\bf SUP}).
\label{e23}
E_{tick}(x) = 1~\text{iff}~\xi(x,tick)!.
 \label{e24}
\xi(x,\alpha)! ~\&~ \neg\xi(x,tick)! ~\&~  (\exists s\in
\Sigma^*) \notag\\
\Big( \xi(x_0,s)=x ~\&~\delta(q_0, s.tick)! \Big)
 \label{e25}
E_{tick}(x)\cdot F_\alpha(x') = 0 = E_{tick}(x')\cdot F_\alpha(x).
 \label{e26}
(\rmnum{1})~&(\forall i\in I, \forall x, x' \in X_i) (x,x') \in \mathcal{R}^P_{\alpha}, \notag\\
(\rmnum{2})~&(\forall i\in I, \forall \sigma \in \Sigma)\Big[(\exists x\in X_i)\xi(x,\sigma)! \Rightarrow \\
&\big((\exists j\in I)(\forall x'\in X_i)\xi(x',\sigma)!\Rightarrow
\xi(x',\sigma)\in X_j\big)\Big]. \notag
 \label{e277}
(\exists x\in X_i)&\xi(x,\sigma)\in X_j \mbox{\ \ \ and \ \ \ }\notag\\
&(\forall x'\in X_i)\big[\xi(x',\sigma)! \Rightarrow
\xi(x',\sigma)\in X_j\big].
\label{e27}
\Sigma_\alpha :=  \{\alpha, tick\}\dot\cup \{\sigma \in \Sigma -
\{\alpha, tick\} \ | \ (\exists i,j \in I) \notag\\
i \neq j \ \& \
\zeta_\alpha'(i,\sigma) = j\}.
\label{e28}
E_\beta(x) = 1~\text{iff}~\xi(x,\beta)!
\label{e29}
\neg\xi(x,\beta)! \ ~\&~\ (\exists s\in \Sigma^*)\left(\xi(x_0,s)=x
~\&~\delta(q_0, s\beta)! \right)
 \label{e33}
M(x) = 1~\text{iff}~ x \in X_m.
 \label{e34}
T(x) = 1~\text{iff}~ (\exists s\in \Sigma^*)\xi(x_0,s) = x  \&
\delta(q_0,s) \in Q_m
 \label{e30}
&(\rmnum{1})~E_\beta(x)\cdot D_\beta(x') = 0 = E_\beta(x')\cdot D_\beta(x), \notag\\
&(\rmnum{2})~T(x) = T(x') \Rightarrow M(x) = M(x').
 \label{e31}
(\rmnum{1})~&(\forall i\in I, \forall x, x' \in X_i) (x,x') \in \mathcal{R}^C_{\beta}, \notag\\
(\rmnum{2})~&(\forall i\in I, \forall \sigma \in \Sigma)\Big[(\exists x\in X_i)\xi(x,\sigma)! \Rightarrow \\
&\big((\exists j\in I)(\forall x'\in X_i)\xi(x',\sigma)! \Rightarrow
\xi(x',\sigma)\in X_j\big)\Big]. \notag
\label{e32}
\Sigma_\beta :=  \{\beta\}\dot\cup \{\sigma \in \Sigma - \{\beta\} \
|\ (\exists i,j \in I)\notag \\
i \neq j \ \&\ \zeta_\beta'(i,\sigma) = j\}.

   L({\bf G}) \cap L({\bf LOC}) &= L({\bf SUP}), \label{e38}\\
   L_m({\bf G}) \cap L_m({\bf LOC}) &= L_m({\bf SUP}). \label{e39}

{\bf MACH1}&\text{'s timed events}: \\
&(\alpha_{11},1,\infty)~~(\beta_{11},3,3)~~(\alpha_{12},1,\infty)~~(\beta_{12},2,2)\\
{\bf MACH2}&\text{'s timed events}: \\
&(\alpha_{21},1,\infty)~~(\beta_{21},1,1)~~(\alpha_{22},1,\infty)~~(\beta_{22},4,4)
 \label{eq:app1}
&(\forall \alpha \in \Sigma_{for})\ L_m({\bf SUP}) \subseteq
P_\alpha^{-1}L_m({\bf LOC}^P_{\alpha}),\\
\label{eq:app2}&(\forall \beta \in \Sigma_{hib})L_m({\bf SUP})
\subseteq P_\beta^{-1}L_m({\bf LOC}^C_{\beta}).
 \label{e40}
x_0 \in X_{i_0}~&\&~ \zeta_\alpha'(i_0,\sigma_0) = i_1, \notag \\
x_1 \in X_{i_1}~&\&~ \zeta_\alpha'(i_1,\sigma_1) = i_2, \notag\\
&\vdots \\
x_{h+1} \in X_{i_{h+1}}~&\&~ \zeta_\alpha'(i_h,\sigma_h) = i_{h+1}. \notag

L({\bf SUP}) &= \overline{L_m({\bf SUP})} \\
             &\subseteq \overline{L_m({\bf G}) \cap L_m({\bf LOC})} \\
&\subseteq L({\bf G}) \cap L({\bf LOC}).


(, \ref{e39}) Let ; by (\ref{e21}), for every , , i.e. . Write .
Then there exists ; thus  (defined
in (\ref{e33})), which also implies  (defined in
(\ref{e34})). On the other hand, since  (the
last equality has already been shown), we have .
That is, ; as in (\ref{e40}) above we derive
, and by the control cover definition
(\ref{e31}) it holds that . Since , i.e. , we have . Therefore by requirement (ii) of
the control consistency definition (\ref{e30}), , i.e. .


\end{document}
