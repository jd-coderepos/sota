



\documentclass[12pt]{article}

\usepackage[breaklinks]{hyperref}  \usepackage{graphicx}
\usepackage[active]{srcltx} \usepackage{sariel,wide}
\usepackage{amstext}
\usepackage{amsmath}
\usepackage{picins}


\DefineNamedColor{named}{OliveGreen}    {cmyk}{0.64,0,0.95,0.40}

\newcommand{\PntSet}{P}
\newcommand{\LinesX}[1]{\EuScript{L}\pth{#1}}

\newcommand{\linesPair}[1]{\EuScript{L}_{\cap #1}}

\newcommand{\pnt}{\mathsf{p}}
\newcommand{\pntA}{\mathsf{q}}
\newcommand{\pntB}{\mathsf{r}}

\newcommand{\lineA}{\ell}

\newcommand{\sepLP}[1]{\alpha_L\pth{#1}} 
\newcommand{\sepNum}[1]{\alpha\pth{#1}} 
\newcommand{\stabNum}[1]{\mathrm{s{t}b}\pth{#1}} 

\newcommand{\depthX}[1]{\mathrm{depth{}}\pth{#1}} 

\newcommand{\pairsX}[1]{\mathrm{E}\pth[]{#1}} 

\newcommand{\Term}[1]{\textsf{#1}}
\newcommand{\LP}{\Term{L{}P}\xspace}
\newcommand{\VC}{\Term{V{C}}\xspace}
\newcommand{\Family}{\EuScript{F}}

\newcommand{\ArrX}[1]{\mathop{\mathrm{\EuScript{A}}}\pth{#1}}

\newcommand{\Dim}{\tau}

\newcommand{\dCr}[2]{d_{#1}(#2)}
\newcommand{\diskCr}[2]{\mathsf{D}_{#1}(#2)}
\newcommand{\LineSet}{L}

\providecommand{\Holder}{H\"older\xspace}
\providecommand{\Bronnimann}{Br{\"o}nnimann\xspace}
\newcommand{\I}{\mathcal{I}}

\begin{document}

\title{Approximating Spanning Trees with Low Crossing Number}

\author{Sariel Har-Peled\SarielThanks{}}

\date{\today}

\maketitle

\begin{abstract}
    We present a linear programming based algorithm for computing a
    spanning tree  of a set  of  points in ,
    such that its crossing number is ,
    where  the minimum crossing number of any spanning tree of
    . This is the first guaranteed approximation algorithm
    for this problem.  We provide a similar approximation algorithm
    for the more general settings of building a spanning tree for a
    set system with bounded \VC dimension.

    Our approach is an alternative to the reweighting technique
    previously used in computing such spanning trees.
\end{abstract}




\section{Introduction}

The reweighting technique is a powerful tool in computer science
\cite{ahk-mwmma-06}. In Computational Geometry, it was introduced by
Chazelle and Welzl \cite{cw-qorss-89} who used it to compute spanning
paths with low crossing number in set systems with bounded \VC
dimension. Welzl \cite{w-stlcn-92} provided a tighter analysis for the
case of spanning tree of points in .  \Matousek \cite{m-ept-92}
used the reweighting technique to provide a powerful partition theorem
that proved to be very useful in building range searching
data-structures \cite{ae-rsir-98}.  Also, Clarkson \cite{c-apca-93}
provided an algorithm for polytope approximation that used the
reweighting technique.  \Bronnimann and Goodrich \cite{bg-aoscf-95}
realized that Clarkson's algorithm implies a general method for
solving hitting set and set cover problems in geometric settings.


Interestingly, Long \cite{l-upsda-01} had observed that set cover
problems in geometric settings can be solved by using \LP and taking a
random sample (guided by the \LP solution) that is an -net (a
similar observation was later made by \cite{ers-hsvcs-05}). In fact,
such packing/covering \LP{}s can be solved efficiently via reweighting
\cite{pst-faafp-91}. Thus, one can interpret the reweighting algorithm
for solving the geometric set cover problem as directly solving the
associated \LP.


\parpic[r]{\includegraphics{figs/grid_crossing}}

The result of Welzl \cite{w-stlcn-92}, mentioned above, is quite
intriguing. It shows that for a set  of  points in the
plane (resp., in ) one can find a spanning tree of the points,
such that any line (resp., hyperplane) crosses at most 
(resp., ) edges (i.e., segments) of the spanning tree.
To appreciate this result, consider the point set formed by the grid
. It is trivial in this case to come up with
a spanning tree with a crossing number  -- any spanning
tree of the grid points using only edges of the grid has this
property, see figure on the right. Surprisingly, the result of Welzl
\cite{w-stlcn-92} implies that any point set behaves like a grid point
set as far as the crossing number of the optimal spanning tree.


\bigskip In this work, we establish a connection between computing
spanning trees with low crossing number and \LP{}s (i.e., linear
programs); that is, we show that spanning trees with low crossing
number can be computed using \LP rounding.


\medskip
\noindent
\textbf{Approximate spanning tree with the lowest crossing number.}
Given a set system  of finite \VC dimension
, we show how to compute, in polynomial time\footnote{We make the standard assumption that solving a \LP of
   polynomial size takes polynomial time.}, a spanning tree of  with crossing number 
(assuming ), where  is the minimal crossing
number of any spanning tree of .  This is done by recursively
solving a \LP relaxation and rounding it.  See \secref{approximation}
for details.




Naturally, this algorithm also applies to the Euclidean case.
Specially, given a set  of  points in  one can
compute, in polynomial time, a spanning tree  such that every
hyperplane crosses at most  edges of , where  is
the minimum crossing number of any spanning tree of .  

Surprisingly, this is the first guaranteed approximation algorithm
known for this problem. In particular, achieving such an approximation
is mentioned as open in the Open Problems Project (see
\url{http://maven.smith.edu/~orourke/TOPP/P20.html#Problem.20}).




\medskip
\noindent
\textbf{Spanning trees in  with  crossings.}  We
also modify the analysis of our algorithm (but not the algorithm
itself) so that it yields worst case bound on the crossing number.
Specifically, we get a polynomial time algorithm that, for a given set
 of  points in , computes a spanning tree  of
 such that any hyperplane in  crosses at most
 edges of .  Our proof of the correctness of the
algorithm is self contained (except for a relatively easy lemma, see
\lemref{crossing:disk}), and uses \LP duality. We believe the new
proof provides a new insight into why such trees exist. In particular,
Chazelle and Welzl \cite{cw-qorss-89} and Welzl \cite{w-stlcn-92}
proofs of the existence of such spanning trees are simple but somewhat
``mysterious'' (at least for the author, but other people might not
see the mystery).

Here is a sketch of the resulting argument why such trees exist: In
the plane, it is sufficient to find spanning forest that span at least
 vertices of  (in connected components that are
not singletons) and has crossing number .  One can find such a
(fractional) spanning graph by doing \LP relaxation. The dual \LP then
asks (intuitively) to separate the given  points into singletons by
a set of lines of minimum cardinality (i.e., for any , there exists a selected line that crosses the segment
). It is not hard to show that any such set of lines need
to be of size . A somewhat more involved argument
(since we are dealing with a fractional solution of an \LP that has
some other constraints) implies that the dual \LP is feasible for  and its optimal solution is bounded from below. It follows
that the primal \LP is feasible. Now
, solving the primal \LP and using
a straightforward rounding implies that one can compute the required
spanning graph. Applying this recursively by selecting a vertex from
each connected component, and overlaying the resulting spanning graphs
together results in a connected graph of  with crossing
number .  See \secref{planar} for details.

Interestingly, while the above algorithm works for any point set in
, one can do slightly better in the planar case, and get a
deterministic rounding scheme, see \secref{deterministic} for details.



\medskip

\paragraph{Previous work.} Fekete \etal \cite{flm-msnmt-08} suggested
using \LP relaxation to compute a spanning tree with low crossing
number. Their \LP is considerably more elaborate than ours
(considering all cuts), and their iterated rounding scheme seems to
perform quite well in practice (although they are unable to provide a
theoretical guarantee on the performance). Furthermore, they prove
that computing the spanning tree with minimal crossing number is
\NPHard.

\paragraph{Organization.}
In \secref{approximation} we show the  approximation
algorithm for spanning trees with low crossing number, for the general
case of a set system with low \VC dimension.  In \secref{planar}, we
specialize this algorithm for the case of points and hyperplanes in
.  We discuss our results and some related open problems in
\secref{conclusions}.


\section{Approximating the spanning tree with optimal crossing number}
\seclab{approximation}

Consider a set system  of finite \VC
dimension . For more details on spaces with bounded \VC
dimension see \cite{pa-cg-95}. For our purposes, it is sufficient that
 is a set of subsets of  of cardinality bounded by
, and this holds for any set system induced by
a subset of .

For two distinct points , we will refer to
the set  as an \emphi{edge}, denoted by .  An edge  \emphi{crosses} a set  if
.

The \emphi{crossing number} of a set of edges  of  is the
maximum number of edges of  crossed by any set of .

\begin{example}
    As a concrete example of such a set system, consider a set 
    of  points in the plane, and let 
    
    The set system  in this case has \VC
    dimension , and a spanning tree  of  with crossing
    number , is a spanning tree of , drawn in the plane by
    straight segments, such that every line (i.e., the boundary of a
    halfplane) intersects at most  edges of .
\end{example}

\begin{lemma}
    Assume there exists a spanning tree  for  with
    crossing number . Then, for any subset 
    there exists a spanning tree with crossing number at most .

    \lemlab{inherent}
\end{lemma}

\begin{proof}
    Convert the spanning tree  of , with crossing number
    , into a closed cycle  visiting the points of , by
    doing an Euler tour of , using each edge of  twice. The new
    cycle  has crossing number .  Next, shortcut the cycle 
    such that it uses only elements of , by replacing each subpath
     (that uses inner vertices that are not in )
    connecting  by the edge .

    This results in a cycle that visits only the vertices of  and
    has crossing number , as such a shortcutting can only
    decrease the number of edges crossing a set . Indeed, consider a subpath , and observe that if  crosses a set , then
    one of the edges in the path must also cross this set. Thus,
    replacing a subpath (of a cycle) by an edge reduces the crossing
    number of the cycle.
\end{proof}

\bigskip

Let  denote the set of all edges of , and
consider the following \LP (parameterized by ).


Intuitively, this \LP tries to pick as many edges as possible (i.e.,
 indicates that we pick the edge ),
such that (i) no set is being crossed more than  times, and (ii)
every point of  participates in at least one edge that is
being picked.

\begin{remark}
    The above \LP might not be feasible, and in such a case
     is not defined. Naturally, one can modify
    this \LP to first compute the minimal value  for which it is
    feasible, and then solve the original \LP with this value of .

    In particular, in this case, we can choose almost any arbitrary
    target function to optimize the \LP for. We had chosen this one
    since the dual form is convenient to work with, see
    \secref{planar}.
\end{remark}

\begin{lemma}
    Consider a set system  with bounded \VC
    dimension, where , and let  be a
    parameter such that  is feasible. Then, one can
    compute (in polynomial time) a set of edges , such that
    , and the number of connected
    components of the graph  is (in expectation) at most
    .  The crossing number of  is  with high probability.

    \lemlab{spanning}
\end{lemma}
\begin{proof}
    We solve the \LP \EqrefQ{l:p:general} and compute . Next, for every , if
     then we add  to
    . Otherwise, if  then we pick the edge
     into  with probability .
    
    For a set , let  be its crossing number in
    .  We have that
    
    For a constant  sufficiently large, let ,
    and observe that . As
    such, by the Chernoff inequality, we have that
    
    Since  has \VC dimension , the number of sets one has to
    consider (i.e., the size of ) is 
    \cite{pa-cg-95}, which implies, by the above, that the crossing
    number of  is bounded by , with high probability.

    As for the number of connected components in the graph , observe that a point  is not adjacent to any
    edge of  with probability
    
    by the inequality (*) in \LP \EqrefQ{l:p:general}. Let  be the
    number of points of  that are singletons in the graph
    . By the above, we have that . As
    such, the expected number of connected components in 
    is
    
    \aftermathA
\end{proof}

\begin{theorem}
    Consider a set system  with bounded \VC
    dimension, where .  Let  be the minimum
    crossing number of any spanning tree of .  Then, one can
    compute, in polynomial time, a spanning tree  of  with
    a crossing number .

    \thmlab{low:crossing}
\end{theorem}
\begin{proof}
    We set . In the th iteration, we compute
    the minimal  for which  is
    feasible. Next, we compute a set of edges  over ,
    using \lemref{spanning}. If the number of connected components of
     is larger than , we
    repeat this iteration (we have constant probability to succeed by
    Markov's inequality). Next, from each connected component of
    , we pick one point into . We repeat
    this algorithm till we remain with a single point. This algorithm
    performs  iteration.  Now, the union  forms a spanning graph of , and we return any
    spanning tree  of .


    The crossing number of  is bounded by the total crossing
    numbers of the graphs . Now, the graph  has crossing number  by \lemref{spanning}. By
    \lemref{inherent}, , for all . As such, the
    crossing number of  is .
\end{proof}

\medskip

When  is a set of points in , we will be interested in
the spanning tree having the minimal number of crossings with any
hyperplane. In particular, the above result implies the following.

\begin{corollary}
    Let  be a set of  points in , and  be the
    minimum crossing number of any spanning tree of .  Then,
    one can compute, in polynomial time, a spanning tree  of
     with a crossing number .  Specifically, any hyperplane in  crosses
    at most  edges (i.e., segments) of .

    \corlab{low:crossing} 
\end{corollary}


\section{Spanning tree in \TPDF{}{Rd} with low crossing number}
\seclab{planar}

Let  be a set of  points in the plane in general position
(i.e., no three points are colinear). Let  denote
the set of all partitions of  into two non-empty sets, by a
line that does not contain any point of . For each such
partition, we select a representative line that realizes this
partition. We slightly abuse notations as refers to 
as a set of these lines.

We are interested in the question of finding a spanning tree  of
 such that each line of  crosses at most
 edges of .





\begin{defn}
    For a set of lines  in the plane, the crossing distance
    between two points is the number of lines of  crossed by
    the segment formed by these two points. Formally, for any two
    points , the \emphi{crossing distance}
    between them is , where 
    is the number of lines of  having  and  on
    opposite sides, and  is the number of lines that contain either
     or . It is easy to verify that
     complies with the triangle inequality (as
    such its a pseudo-metric).
    
    The \emphi{crossing disk} of radius  centered at a point
    , is the set of all vertices of the arrangement 
     in crossing distance at most  from . We
    denote this ``disk'' by .
\end{defn}

We need the following lemma due to Welzl \cite{w-stlcn-92}.
\begin{lemma}[\cite{w-stlcn-92}]
    Let  be a parameter,  be a set of lines (of
    size at least ) in the plane, and let  be a point in the
    plane not contained in any line of . Then
    .

    \lemlab{crossing:disk}
\end{lemma}

Here is the \LP \EqrefQ{l:p:general} specialized for this planar case,
and its dual \LP.

\medskip

\noindent\fbox{\begin{minipage}{0.45\linewidth}

\end{minipage}}~~
\fbox{\begin{minipage}{0.45\linewidth}
-0.75cm]
    & 
    \hspace{4cm}
    \forall \pnt \pntA \in
    \pairsX{\PntSet} 
    \hspace{-4cm}
    \
\end{minipage}}

\medskip

We will next show that  is feasible. This
would imply that one can find spanning graph of  with
crossing number of  that uses a constant fraction of the
vertices (i.e., \lemref{spanning}).

\begin{lemma}
    The \LP  is feasible for .

    \lemlab{feasible}
\end{lemma}
\begin{proof}
    Consider the dual \LP above and observe that it is always feasible
    (for example by setting  for all  and  for all ).
    Thus, if we show that  is bounded from below (and
    thus is finite), then the strong duality theorem would imply that
     is feasible and equal to .

    So consider a solution to this dual \LP, where all the values are
    rational numbers. Let  be the smallest integer such that if
    we scale all the values in the given \LP solution by  then they are
    integers. In particular, let , for all
    , and , for all
    .

    Let  be a set of lines, where we pick  copies
    of  into this set, for all . Formally,  copies of the same line
     (put into ) will be a collection of ,
    almost identical, copies of the line  slightly perturbed
    so that these  lines are in general position.  Thus,
     is a set of  lines in general position. Furthermore, the inequalities
    in the \LP implies that, for any segment , we have that  crosses
    
    lines of . 

    Observe that  for any pair . Otherwise, there would be a point
     in the plane such that 
    and . But the triangle
    inequality would imply that , which contradicts the above.
      
    By \lemref{crossing:disk}, for , the disk
     contains at least
     distinct vertices of . On
    the other hand, the total number of vertices in the arrangement
     is . We conclude that
    
    Now, by the Cauchy-Schwarz inequality and the above, we have that
    
    As  this implies that . The claim now follows.
\end{proof}



\begin{remark}
    The proof of \lemref{feasible} works also in higher dimensions,
    where we consider points and hyperplanes in . There, one
    has to use \Holder's inequality instead of the Cauchy-Schwarz
    inequality. Then, the \LP is feasible for .
    \remlab{easy}
\end{remark}


\begin{theorem}
    Given a set  of  points in , one can compute
    (in polynomial time) a spanning tree  of  with crossing
    number at most ; that is, any hyperplane in
     crosses at most  edges of .
\end{theorem}
\begin{proof}
    By \lemref{feasible} and \remref{easy},  is
    feasible, for . As such, by
    \lemref{spanning}, we can compute a set of edges  that engages
    a constant fraction of the points of , and it has
    crossing number .  Using the
    algorithm of \thmref{low:crossing} generates a spanning tree with
    crossing number . Namely,
    the resulting spanning tree has crossing number
    .
\end{proof}



\begin{remark}[Connection to separating/hitting and packing \LP{}s.]
    It is interesting to consider the \LP that just
    tries to separate all points of  from each other. It
    looks similar to our dual \LP while being simpler.
    
    \smallskip

    \centerline{\fbox{\begin{minipage}{0.45\linewidth}
              \vspace{-0.35cm}
              0.2cm]
                  & z_{\ell} \geq 0 & \hspace{-4cm} \forall \lineA \in \LinesX{\PntSet}\\
                  & z_{\pnt} \geq 0 & \forall \pnt \in \PntSet.
              
               \eqlab{second:l:p}
               \min& \sum_{\pnt \pntA \in
                  \pairsX{\PntSet} } \dist{ \pnt - \pntA} x_{\pnt
                  \pnt} &\\
     s.t. \quad & \sum_{\substack{\pnt \pntA \in \pairsX{\PntSet},
          \\
          \pnt\pntA
       \cap \ell \ne \emptyset}}  x_{\pnt\pntA} \leq t & \forall \ell \in
 \LinesX{\PntSet} \nonumber \\
  & \sum_{\pntA \in \PntSet, \pntA \ne \pnt}
               x_{\pnt\pntA} \geq 1 &
               \forall \pnt \in \PntSet
               \tag{*}\\
               & x_{\pnt\pntA} \geq 0 &
               \!\!\!\!\!\!\!\!\!\!\!\!\!\!\!\!\!\!\!
               \forall \pnt\pntA \in
               \pairsX{\PntSet}.\nonumber
           
       \end{minipage} 
    }

    \vspace{-0.2cm}

    \caption{The modified \LP}
    \figlab{modified:l:p}
\end{minipage} 

\medskip

\begin{proof}
    Instead of computing  we slightly modify the \LP so that it
    finds the ``shortest'' such solution.  The resulting modified
    \LP is depicted in 
    \figref{modified:l:p}.



    \parpic[r]{\includegraphics{figs/planarity}}

    Let  be the set of all the edges  in the solution
    to this \LP such that . We claim that this set
    of edges is planar. Indeed, if two such segments  and 
    intersect, then consider two opposing edges  and  of the
    quadrant formed by the convex hull of the endpoints of  and
    , see figure on the right.


    
    We have that , which
    implies that, for  sufficiently small, the solution
    , , ,  is feasible, as the total
    value of the edges attached to a vertex does not change, and the
    crossing number of any line does not increase by this change, as
    can be easily verified. But this implies that there is a feasible
    solution with a better target value (specifically, the target
    value goes down by ). A contradiction. 

    
    Thus  is a planar graph where each point of
     has at least one edge attached to it. Furthermore, the
    average degree in a planar graph is at most , which implies
    that at least half of the points of  have degree at most
     in . But each such point , must have an edge  attached to it, such that ,
    because of (*) in the \LP \EqrefQ{second:l:p}.
    
    Thus, scale the \LP solution by a factor of  and pick all the
    edges  with  into the set
    . We get a set that is adjacent to at least half of the points
    of , and its crossing number is at most .
\end{proof}    

\medskip

The planarity argument in the above proof of \lemref{planar} is
similar to the one used by Fekete \etal \cite{flm-msnmt-08} -- they
use it to argue that there is one heavy edge, while we use it to argue
that there are many heavy edges.



\begin{theorem}
    Let  be a set of  points in the plane.  One can
    compute, in deterministic polynomial time, a spanning tree  of
     with a crossing number ,
    where  is the minimum crossing number of any spanning tree of
    .

    \thmlab{low:crossing:2}
\end{theorem}

\section{Conclusions}
\seclab{conclusions}

We presented an approximation algorithm for computing a spanning tree
with low crossing number. The new algorithm relies on a natural \LP
relaxation of the problem and a straightforward rounding scheme.

Interestingly, our approach enables us to provide a direct proof to
the existence of such spanning trees in . This is, as far as we
know, the first algorithm for this problem that avoids using the
reweighting technique.  Intuitively, our algorithm (together with
previous results \cite{l-upsda-01}) suggests that reweighting in
geometric settings can sometimes be replaced by \LP rounding. This is
a significant feature, as \LP{}s are considerably more general and
flexible tool than reweighting. For example, using our algorithm, we
can add other constraints to the \LP; e.g., we can insist that some
certain cuts would have significantly lower crossing number than some
other cuts. In particular, it is not clear how one can incorporate
such considerations into a reweighting algorithm computing spanning
trees with low crossing number.

One interesting open problem, is to compute spanning trees with
\emph{relative} crossing number using the new \LP approach. Here,
given a point set  in , one would like to compute a
spanning tree  such that if a halfspace  contains  points
of  then it boundary plane  crosses (say)  edges of . Such a result is known in the plane
\cite{ahs-ahrcr-07}, but the problem is open in higher dimensions.




\section*{Acknowledgments}

The author would like to thank Chandra Chekuri, S{\'a}\si{ndor}
Fekete, Jirka \Matousek, and Emo Welzl for helpful discussions on the
problems studied in this paper.



\bibliographystyle{alpha} 
\bibliography{shortcuts,geometry}

\end{document}
