\documentclass[a4paper,twoside]{article}

\usepackage{epsfig}
\usepackage{subcaption}
\usepackage{calc}
\usepackage{amssymb}
\usepackage{amstext}
\usepackage{amsmath}
\usepackage{amsthm}
\usepackage{multicol}
\usepackage{pslatex}
\usepackage{natbib}
\let\bibhang\relax
\usepackage{example}
\usepackage[bottom]{footmisc}

\usepackage{mathtools}
\usepackage{xr}
\usepackage{svg}
\usepackage[ruled,linesnumbered]{algorithm2e}
\newtheorem{theorem}{Theorem}
\RestyleAlgo{ruled}
\usepackage{booktabs}
\usepackage{amsfonts}
\usepackage{multicol}
\usepackage{wrapfig}

\graphicspath{{./fig/}}
\theoremstyle{definition}
\newtheorem{definition}{Definition}
\usepackage{hyperref} 
\usepackage{SCITEPRESS}     



\begin{document}

\title{FInC Flow: Fast and Invertible  Convolutions for Normalizing Flows}



\author{\authorname{Aditya Kallappa\sup{1}, Sandeep Nagar\sup{1} and Girish Varma\sup{1}}
\affiliation{\sup{1}International Institute of Information Technology, Hyderabad, India}
\email{\{aditya.kallappa, sandeep.nagar\}@research.iiit.ac.in, girish.varma@iiit.ac.in}}


\keywords{Normalizing Flows, Deep Learning, Invertible Convolutions}

\abstract{Invertible convolutions have been an essential element for building expressive normalizing flow-based generative models since their introduction in Glow. Several attempts have been made to design invertible  convolutions that are efficient in training and sampling passes. Though these attempts have improved the expressivity and sampling efficiency, they severely lagged behind Glow which used only  convolutions in terms of sampling time. Also, many of the approaches mask a large number of parameters of the underlying convolution, resulting in lower expressivity on a fixed run-time budget. We propose a  convolutional layer and Deep Normalizing Flow architecture which i.)  has a fast parallel inversion algorithm with running time O ( is height and width of the input image and k is kernel size), ii.) masks the minimal amount of learnable parameters in a layer. iii.) gives better forward pass and sampling times comparable to other  convolution-based models on real-world benchmarks. We provide an implementation of the proposed parallel algorithm for sampling using our invertible convolutions on GPUs. Benchmarks on CIFAR-10, ImageNet, and CelebA datasets show comparable performance to previous works regarding bits per dimension while significantly improving the sampling time.}

\onecolumn \maketitle \normalsize \setcounter{footnote}{0} \vfill


\section{Introduction}



Normalizing flow is an important subclass of Deep Generative Models that offers distinctive benefits \citep{kobyzev2020normalizing}. In comparison to GANs \citep{gans} and VAEs \citep{kingma2018variational}, they are trained using a very intuitive Maximum Likelihood loss function. Images and the \emph{latent vector}, which is required to have a Gaussian distribution, correspond one-to-one in flow models. Despite these intriguing characteristics, GANs and VAEs are utilized more frequently. This is due to the need for the Normalizing Flows transformations to be invertible, which significantly restricts the neural network types employed. For deployment in a real-world scenario, the invertible transformations must be efficiently calculable in the forward and sample stages.


A significant breakthrough came with Glow \citep{kingma2018glow} which used  invertible convolutions to design normalizing flows. If it exists, the inverse function for a  convolution also happens to be a  convolution. Since computing  convolution has fast parallel algorithms for which running time does not depend on the spatial dimensions, they are also highly efficient in forward pass (i.e. computing \emph{latent vector} from an image) as well as the sampling passes (i.e. computing image from a sampled \emph{latent vector}). Extending Glow to use invertible  convolutions promises to improve the expressivity further, allowing it to model more complex datasets. However, this is a challenging problem since the inverse function for a  convolution, in general, is given by a  matrix where  (i.e. the spatial dimensions). Hence, while the forward pass can be fast, the trivial approach for the sampling pass will cost  operations per convolutional layer.


\begin{figure*}[!th]
    \centering
    \includegraphics[width=\linewidth]{fig/convolution_equation.png}
    \caption{Convolution of a  \emph{TL} (Top Left) padded image with a  filter viewed as a linear transform of vectorized input() by the convolution matrix . The \emph{TL} padding on the input results in the making matrix  lower triangular, and all diagonal values correspond to  of the filter. Each row of  can be used to find a pixel value. The rows or pixels with the same color can be inverted in parallel since all the other values required for computing them will already be available at a step of our inversion algorithm \ref{alg:inv_algo}.}
    \label{fig:Conv_equn}
\end{figure*}



CInC Flow \citep{nagar2021cinc} introduced a padded  convolution layer design and gave it the necessary and sufficient conditions to make it invertible. They showed that the convolution matrix is lower triangular by ensuring padding in only two sides of the input. Furthermore, all the diagonal entries of the convolution matrix are equal to a single weight parameter. By setting this parameter to 1, they ensured that the convolutions are invertible, and Jacobian is always 1.

We build on their work by proposing a parallel inversion algorithm for their convolution design. The parallel algorithm only uses  O sequential operations, unlike O operations used by most previous works. We also build a normalizing flow architecture, where channel-wise splitting is further used to parallelize operations.

\paragraph{Our Contributions.}
\begin{enumerate}
    \item We design a  invertible convolutional layer with a fast and parallel invertible sampling algorithm (see Sections \ref{sec:conv-design}, \ref{subsec:PIA}).
    \item We build a normalizing flow architecture based on the fast invertible convolution, which uses channel wise splitting to improve the parallelism further (see Sections \ref{section:fastflowunit}, \ref{subsec:arch}).
    \item We provide a fast GPU implementation of our parallel inversion algorithm and benchmark the sampling times of the model (see Section \ref{sec:results}). We show greatly improved sampling times due to our parallel inversion algorithm, while giving similar bits per dimensions as compared to other works.
\end{enumerate}
 
\begin{table*}[!t]
    \centering
    \begin{tabular}{l  l  l  l  l }
        \toprule
        Method    & \# of ops & \# params / CNN layer & \ Complexity of Jacobian & Inverse\\ 
        \midrule
        FInC Flow (our)  &     &     & 1 &     exact   \\
        Woodbury \citep{lu2020woodbury} &  &  & O & exact \\
        MaCow \citep{ma2019macow}     &     & ) &  O & exact\\ 
        Emerging \citep{hoogeboom2019emerging}  &   &  )  &  O  &    exact   \\
        CInC Flow \citep{nagar2021cinc}  &     &     & 1  &   exact      \\
        \midrule
        MintNet   \citep{song2019mintnet} &     &     &  O &    approx  \\ 
        SNF  \citep{keller2021self} &      &    &  approx &  approx \\ \bottomrule
    \end{tabular}
    \caption{Comparison of the learnable parameters. where  is input size,  is filter size which is constant,  is number of input/output channels.  is the number of latent dimensions. \# of ops: required number of operations for the inversion of convolutional layers. The complexity of Jacobian: Time complexity for calculating the Jacobian of a single convolution layer. For FInC Flow and CInC Flow, the Jacobian is 1, since the Convolution matrix is lower triangular with diagonal entries being 1.}
    \label{table:complexity}
\end{table*}

\section{Related Work}
\paragraph{Generative Modeling.}
The idea of generative modeling stems from training a generative model whose sample comes from the same distribution as the training data distribution. Most of the generative models can be grouped as Generative adversarial networks (GANs) \citep{goodfellow2014generative, brock2018large}, Energy-based models (EBMs) \citep{zhang2022generative, song2021maximum}, Variational autoencoders (VAEs) \citep{kingma2013auto, kingma2018variational, hazami2022efficient}, Autoregressive models \citep{oord2016wavenet, nash2020polygen}, Diffusion models \citep{ho2020denoising, song2021scorebased, song2019generative} and Flow-based models \citep{dinh2014nice, dinh2016density, hoogeboom2019emerging,kingma2018glow, ho2019flow++, ma2019macow, nagar2021cinc}. 

\paragraph{Normalizing Flows.}
Flows-based models construct complex distributions by transforming a probability density through a series of invertible mappings \citep{rezende2015variational}. At the end of these invertible mapping, we obtain a valid distribution; hence, this type of flow is referred to as a Normalizing Flow model. Flow models apply the rule for change of variables; the initial density ‘flows’ through the sequence of invertible mappings \citep{dinh2016density}. Flow-based models generalize a dataset distribution into a latent space \citep{kobyzev2020normalizing}.


\paragraph{Invertible kxk Convolutions.}
An invertible neural network requires the inverse of the network with fast and efficient computation of the Jacobian determinant \citep{song2019mintnet}. An invertible neural network can be used for generation and classification with more interpretability. \citep{kingma2018glow} proposed an invertible  convolution building on top of NICE \citep{dinh2014nice}  and RealNVP \citep{dinh2016density} consisting a series of flow step combined in a multi-scale architecture. Each flow step consists of actnorm followed by an invertible  convolution, followed by a coupling layer (see Sec \ref{section:fastflowunit}). Emerging \citep{hoogeboom2019emerging} presented method to generalized  convolution to invertible  convolutions. Emerging chains two specific auto regressive convolutions \citep{kingma2013auto} to form a single convolutional layer following the associativity of the convolution operation. Each of these autoregressive convolutions is chosen such that the resulting convolution matrix  is triangular with an inverse time of each of the convolutions is O. MintNet \citep{song2019mintnet} presented a method for designing invertible neural networks by combining building blocks with a set of composition rules. The inversion of the proposed blocks necessitates a sequence of dependent computations that increase the network's sampling time. SNF \citep{keller2021self} proposed a method to reduce the computation complexity of the Jacobian determinant by replacing the gradient term with a learned approximate inverse for each layer. This method avoids the determinant of Jacobian and makes it approximate, and requires an additional backward pass for inversion of convolution. MaCow \citep{ma2019macow} while many other papers make use of the invertibility of triangular matrix to reduce inversion time, MaCow outperforms all of them by performing the inverse in O by carefully masking  kernels at the top, left, bottom, right to achieve a full convolution, but this flow model use four autoregressive convolutions to make an effective standard convolution. Woodbury \citep{lu2020woodbury} this paper employs the \emph{Woodbury transformation} for invertible convolution, which is a generalized permutation layer that models dimension dependencies along the channel and spatial axes using the channel and spatial transformation. ButterflyFlow \citep{meng2022butterflyflow} introduced a new family of an invertible layer that works for special underlying structures and needs a sequence of layers for an effective invertible convolution.



\paragraph{Fast Algorithms for Invertible Convolutions.}
CInC Flow \citep{nagar2021cinc}, derive necessary and sufficient conditions on a padded CNN for it to be invertible and require a single CNN layer for every effective invertible CNN layer. The padded CNN can leverage the advantage of parallel computation for inversion, resulting in faster and more efficient computation of Jacobian determinants.

The distinguishing feature of our invertible convolutions as compared to previous works is that we have a parallel inversion algorithm that does only  operations where  is input size and  is kernel size. MaCow is the closest approach that takes twice the number of operations. Some of the approaches, like MintNet and SNF, do achieve a lesser number of operations. However, they are not proper normalizing flows as they compute only an approximate inverse. We use the convolution design from CInC Flow but give a parallel inversion algorithm for it. Furthermore, our FInC Flow \emph{Unit} is designed to efficiently parallelize the operations by splitting the convolution operations channel-wise. In Table \ref{table:complexity}, we compare our proposed flow model with the existing model in terms of the receptive fields/number of learnable parameters, complexity of computing the inverse of convolution layer for sampling.











%
 \section{Preliminaries}\label{sec:FastFlow}
\paragraph{Normalizing Flows.}
Formally, Normalizing Flows is a series of transformations of a known simple probability density into a much more complex probability density  using invertible and differentiable functions. These invertible function allows to write the probability of the output as a differentiable function of the model parameters. As a result, the models can be trained using backpropagation with the negative log likelihood loss function.

Let  be a random variable with tractable density . Let  be a differentiable and invertible function. If  then the density of  can be calculated as 
 

Note that  is a  matrix called the Jacobian. If  is transformed using a sequence of functions 's. That is
. Now probability density,  can be expressed as 

where .
The log-probability of  which will be used to model the complex image distribution is given by,

The functions  will be given by neural network layers and the above function can be computed during the forward pass of the neural network. The negative of this function called the negative log likelihood (NLL) is minimized when images in the dataset are given highest probabilities. Hence it gives a simple, interpretable loss function for training the model. 

\begin{figure*}[!ht]
    \centering
    \includegraphics[width=\textwidth]{fig/FInCFlow_model.pdf}
    \caption{(a) FInC Flow unit: to utilize the independence of convolution on channels the input channels are sliced into four equal parts and then padded (1. top-left, 2. top-right, 3. bottom-right, 4. bottom-left) to keep the input size and output size same. Next, parallelly convoluted each sliced channel with the corresponding masked kernel (masked corner of kernels: 1. bottom-right, 2. bottom-left, 3. top-left, 4. top-right). Finally, concatenate the output from each convolution. (b) We propose a FInC Flow architecture  \ref{section:fastflowunit} where each FInC Flow \emph{Step}  consists of an actnorm step, followed by an invertible 1 × 1 convolution, followed by coupling layer. (c) Flow is combined with a multi-scale architecture \ref{subsec:arch}.}
    \label{fastflow}
\end{figure*}

\paragraph{Invertible Convolutions.}

The convolution of an input  with shape  with a kernel  with shape  is  of shape  which is equal to

Notice that the dimensions of  and  are not necessarily the same. To ensure that the  and  are the same size, we apply padding to the input . For an input image  with shape , the  padding of  is the image  of shape  is defined as

The convolution operation is a linear transformation of the input. For the vectored (flattened) input , denoted by , the output vector  can be written as  . Matrix  is the called as the \emph{Convolution Matrix} and the dimensions of this matrix is . As long as matrix  is invertible, the convolutional layer can be included as a part of the Normalizing Flows. The common approach to building invertible convolutions is by making  upper triangular and ensuring invertibility by making diagonal entries to be nonzero.



\paragraph{Algorithms for Computing Inverse of Convolutions.}
For normalizing flows built using invertible convolutions, the sampling pass will involve computing the inverse of the convolution matrix. This involves solving a linear systems of equations .

For a general square matrix of size , the time complexity for inversion is . For a lower triangular matrix of size , the time complexity for inversion is  because of back-substitution method. Notice that the size of convolution matrix, \textbf{M}) is  (refer Figure \ref{fig:Conv_equn}) and also that row of the matrix has only  entries at the maximum, results in an inversion time of  which is used in many of the previous works like Emerging and CInC Flows. We show that this method can be parallelized for carefully designed convolutions giving a complexity of only .


\section{FInC Flow: Our Approach}
In this section we describe our approach including convolution layer design which has a fast parallel inversion algorithm with running time O. For more clarity, we refer to height of the image as , width as  and channels as  in this section.


\subsection{Convolution Design} \label{sec:conv-design}
As it is obvious from equation , the inverse timings depends on . Emerging \citep{behrmann2019invertible} masks almost half of the convolution kernel values to ensure  is a Lower Triangular Matrix. However, we follow the method followed in CInC Flow, where only a few values of the convolution kernel are masked.
For an input image  with shape , the top-left(\emph{TL}) i.e.,  padding of  is the image  of shape  is defined in equation \ref{eq:top_left_padding} and similarly for the top-right (\emph{TR}) as equation \ref{eq:top_right_padding}, bottom-left (\emph{BL}) as equation \ref{eq:bottom_left_padding}, bottom-right (\emph{BR}) as equation \ref{eq:bottom_right_padding}. 

Figure \ref{fig:Conv_equn} shows the convolution of a \emph{TL} padded  image with a  filter is equivalent to a matrix multiplication between convolution matrix  and vectored input .  We leverage this to find the inverse faster. We discuss this in more detail in the subsequent sections. Also padded input ,  and  are equivalent to  once they are flipped along corresponding dimension(s).  


\SetKwComment{Comment}{/* }{ */}

\begin{algorithm}[!t] 
    \caption{Fast Parallel Inversion Algorithm of \emph{TL} padded convolution block(PCB)}
    \label{alg:inv_algo}
    \KwIn{: Kernel of shape \\
    : output of the conv of shape }
    \KwResult{: inverse of the conv. with shape .}
    \textbf{Initialization:}  \;
    \For{}{
        \For {}{
            \Comment{The below lines of code executes parallelly on different threads on GPU for every index  of   on the th diagonal.} 
            
                \For{}{
                \For{}{
                \For{}{
                    \If{pixel  not out of bounds}{
                         
                      }
                }
                }
                } \Comment{synchronize all threads}
        } 
     }
\end{algorithm}

\subsection{Parallel Inversion Algorithm}\label{subsec:PIA}
We have presented our algorithm in Algorithm \ref{alg:inv_algo}. The algorithm can be understood using Figure \ref{fig:Conv_equn}. 


\begin{definition}[Diagonal Elements]
Two pixels  and  are said to be secondary diagonal elements if  . For brevity, we refer to these elements from here on simply as Diagonal Elements.
\end{definition}


Theorem \ref{th:ind_parallel} proves that every element of on the diagonal can be computed parallelly and Line 2 of the algorithm takes care of that. We initialize  to  in Line 1 and compute  in Line 8 which is given in Equation \ref{eq:inv_convolution_imp}. 
It is important that we wait for the threads to synchronize before we move to the next diagonal, as they are needed for computing the elements of the next diagonal. The \emph{not out of bounds} in Line 7 means we are remaining in the  convolution window and also we are not including pixel  while computing  as given in Equation \ref{eq:inv_convolution_imp}


\begin{theorem}
\label{th:ind_parallel}
The inverse of the pixels on the diagonals of a TL padded convolution can be computed independently and parallelly.
\end{theorem}
\begin{proof}
The  pixel value of the output  with shape  can be calculated as 
 
which means  is the dot product of  i.e., the corresponding row of matrix  and the vectored input .
Because it is a \emph{TL} padded convolution,  depends only on the values of  window of  pixels where  are the pixels that are on the top and left side of the pixel  including . Because all the diagonal values are , we have, 


where  are the pixels which are strictly top and left side of . Following the masking pattern of CInC Flow, we have  and  is a linear function which is given by weighted sum of the given pixels weighed by the filter values. So,



Let two pixels  and  be  on the same diagonal. This also means that only one of the following settings is true a)  and  or b)  or . Either way, we can conclude that computation of  is not dependent on  and vice versa following the result in Equation \ref{eq:inv_convolution}. Hence they can be computed independently.
Once  is computed, following the Equation \ref{eq:inv_convolution} and the above result, we can compute  and . Since, the sets of pixels  and  both include the elements of  and also , we can write


where  and  are kernel weights.

From Equations \ref{eq:parallel_1}  and \ref{eq:parallel_2}, we can conclude that  and  which are on the same diagonal can be calculated parallelly in a single step.
\end{proof}

\begin{theorem}
Algorithm \ref{alg:inv_algo} uses only  sequential operations.
\end{theorem}

\begin{proof}
We have proved in Theorem \ref{th:ind_parallel} that the inverse pixels on a single diagonal can be computed parallelly in one iteration of Algorithm \ref{alg:inv_algo}. Since there are  number of diagonals in a matrix and there are at maximum  entries in a row of the convolutional matrix, the number of sequential operations needed will be .
\end{proof}

Thus the running time of our algorithm is O where  

\begin{table*}[!t]
\caption{Comparison of the  bits per dimension (BPD), forward pass time (FT) and sampling time (ST) on standard benchmark datasets of various  convolution based Normalizing Flow models. FT and ST are presented in seconds.}
\centering
       \resizebox{\textwidth}{!}{

 \begin{tabular}{lcccccccccccc}
        \toprule
        \multicolumn{1}{l}{Model} & \multicolumn{3}{c}{MNIST} & \multicolumn{3}{c}{CIFAR-10} & \multicolumn{3}{c}{Imagenet-32x32} & \multicolumn{3}{c}{Imagenet-64x64}\\
        \cmidrule(lr){2-4} \cmidrule(lr){5-7} \cmidrule(lr){8-10} \cmidrule(lr){11-13}
           &   BPD &  FT & ST &  BPD &  FT & ST & BPD &  FT  &  ST   &   BPD  & FT & ST\\
        \midrule
Emerging &   -- &  0.16 & 0.62 &  3.34 & 0.49 & 17.19 &  4.09 &    0.73  &    25.79   &    3.81  &  1.71 &  137.04\\
        MaCow   &   -- &  --& -- &  3.16 & 1.49 & 3.23 &  -- &    --  &     --   &    3.69  &  2.91 &  8.05\\
CInC Flow & -- &  --& -- &  3.35 & 0.42 & 7.91 &  4.03 &   0.62  &     11.97   &    3.85  &  1.57 &  55.71\\
        MintNet & 0.98 &  0.16 & 17.29 &  3.32 & 2.09 & 230.17 &  4.06 &    2.08  &     230.44   &    --  &  -- &  --\\
FInC Flow (our) &   1.05 &  0.14 & 0.09 &  3.39 & 0.37 & 0.41 &  4.13 &    0.48  &  0.52   &    3.88  &  1.43 & 2.11 \\  
        \bottomrule
        \end{tabular}}

    \label{table:bpd_table}
\end{table*}

\subsection{FInC Flow Unit}\label{section:fastflowunit}

Figure \ref{fastflow}a visualizes our  convolution block. We call this block as FInC Flow \emph{Unit}. We use all the  padding techniques mentioned before to different channels of the image. For this purpose, we split the input into four equal parts along the channel axis. We do \emph{TL} padding to the first part, \emph{TR} to the second part, \emph{BL} to the third part and \emph{TR} to the fourth part. Then we use a masked filter on each of these parts to perform the convolution operation parallelly. We call each of this padded image along with it's corresponding kernel as Padded Convolution Block (PCB).


\subsection{Architecture}\label{subsec:arch}
Figure \ref{fastflow}c shows the complete architecture of our model. Our model architecture resembles the architecture of Glow. The multi-scale architecture involves a block of a Squeeze layer, FInC Flow \emph{Step} repeated  number of times and a Split layer. The whole block is repeated  number of times. A Squeeze layer follows this and finally FInC Flow \emph{Step} repeated  times.
At the end of each split layer, half of the channels are 'split' (taken away) and modeled as Gaussian distribution samples. These splited half channels are \emph{latent vectors}. The same is done for the output channels. These are denoted as  in Figure \ref{fastflow}(c). 
Each \emph{FInC Flow Step} consists of a \emph{FInC Flow Unit}, an Actnorm Layer, a  Convolutional Layer, followed by a coupling layer. \\
\textbf{Actnorm Layer:} Acts as an activation normalization layer similar to that of a batch normalization layer. Introduced in Glow, this layer performs the affine transformation using scale and bias parameters per channel. 
\\
\textbf{ Convolutional Layer:} This layer introduced in Glow does a  convolution for a given input. Its log determinant and inverse are very easy to compute. It also improves the effectiveness of coupling layers.
\\
\textbf{Coupling Layer:} RealNVP introduced a layer in which the input is split into two  half. The first half remains unchanged, and the second half is transformed and parameterized by the first half. The output is concatenation of first half and the affine transformation, by functions parameterized by the first, of second half. The inverse and log determinant of coupling layer are computed in a straightforward manner. Coupling layer consists of  convolution followed by a  and a modified  convolution used in Emerging.
\\
\textbf{Squeeze:} This layer takes features from spatial to channel dimension \citep{behrmann2019invertible}, i.e., it reduces the feature dimension by total four, two across the height dimension and two across the width dimension resulting in increases the channel dimension by four. As used by \citep{dinh2016density}, we use squeeze layer to reshape the feature maps to have smaller resolution but more channels.
\\
\textbf{Split:} Input is splited into two halves across the channel dimension. We retain the first half, and a function parameterized by first half transform the second half. The transformed second half is modeled as Gaussian samples, are the \emph{latent vectors}. We do not use the checkerboard pattern used in RealNVP \citep{dinh2016density} and many others to keep the architecture simple.

\SetKwComment{Comment}{/* }{ */}











































































 
\begin{table*}[!t]
    \centering
    \caption{CIFAR-10: comparison of learnable parameters and the sampling time. FInC Flow has less number of learnable parameters with the same receptive field and fast layers (all the times are averaged over ten loops for n = 100 sample images in seconds). ST = Sample time, FT = Forward Time. MaCow-FG is the fine-grained MaCow model and MaCow-org stands for MaCow model utilizes the original multi-scale architecture which is the same as Glow. MaCow and our method is closely similar in term of the convolutional design. So, here we show that our proposed method do fast sampling while maintaining the faster forward time .}
        \begin{tabular}{l  c  c  c c}
        \toprule
            Models    & Setting (K and L) & Learnable params (M = million)  & FT(n=100) & ST(n=100) \\ 
            \midrule
            MaCow-FG & [4, [12, 12], [12, 12], 12]  &  37.19M   & 0.88  & 2.64  \\
            MaCow-org &  [4, [12, 12], [12, 12], 12], [4, 4]  & 38.4M & 1.48  & 3.23 \\ 
FInC Flow (our) &  [28, 28, 28] & 39.46M  & \textbf{0.37}  & \textbf{0.41}   \\
            \bottomrule
        \end{tabular}
    
    \label{table:params}
\end{table*}
\vspace{2em}

\section{Results}\label{sec:results}
\paragraph{Bits Per Dimension (BPD):}
BPD is closely related to NLLLoss given in equation \ref{eq:nll}. BPD of  image is given by

Table \ref{table:bpd_table} shows the BPD comparative results of various models with our model. We present the results of MaCow-var which uses Variational Dequantization which was introduced in Flow++ \citep{ho2019flow++}. BPDs recorded are the reported numbers from the respective model papers.

\paragraph{Sampling Time:}\label{section:SamplingTimes}
\begin{figure*}[!ht]
    \centering
    \includegraphics[width=0.99\linewidth]{fig/timings_95.png}
    \caption{Sampling Times for four models - our, Emerging, CInC Flow,  MaCow. Each plot gives the  Confidence Interval (CI) time of the ten runs to sample  images. X-axis represents the sizes of the image sampled starting from  () all the way to . }
    \label{fig:timing}
\end{figure*}



\begin{figure*}[!ht]
    \centering
    \includegraphics[width=0.99\linewidth]{fig/celeba_reconstruct.pdf}
    \caption{Comparison of  (a) original and (b) reconstructed image samples for the  CelebA dataset after FInC Flow model for  epochs. From the images, we can conclude our model reconstruct original image.}
    \label{fig:celeba_reconstruct}
\end{figure*}
Table \ref{table:bpd_table} shows the comparative results of our model with other models. For MaCow, we use the official code released by the authors. We use the code for Emerging, which was implemented in PyTorch by the authors of SNF.. We have implemented CInC Flow in PyTorch and used it to generate results.
The FTs and STs are recorded by averaging ten runs on untrained models (including our model). 


\begin{figure}[!ht]
    \centering
    \includegraphics[width=0.99\linewidth]{fig/uncurated_samples.pdf}
    \caption{Uncurated generated samples images from our flow model.}
    \label{fig:uncurated_samples}
\end{figure}


In Figure \ref{fig:timing}, we plot the relationship between the input image size and inverse sampling time. As the input image size increase, our \emph{Parallel Inversion Algorithm} improve by utilizing the independence in the convolution matrix . If we input a single image (batch size = ), our model performs similarly to the CInC Flow and Emerging. MaCow is far slower because it does the masking of four kernels to maintain the receptive field. To do one convolution, it needs four convolutions to complete one standard convolution, making it slower. Emerging requires two consecutive autoregressive convolutions to have the same receptive field as standard convolution and solver compared to FInC Flow. For batch size =  and larger, FInC Flow beats the Emerging, MaCow , and CInC Flow by a big difference (see Figure \ref{fig:timing}) while maintaining the same receptive field. 


\paragraph{Scaling sampling time with spatial dimensions:}
Table \ref{table:params} shows the comparison among the invertible convolution-based models. To keep it fair, we restrict the total parameters across all the models to be close to  M. We note down the average sampling time (ST) to generate  images over ten runs while doubling the size of the sampled image from  all the way to  and also doubling our batch size from  all the way to . Our model outperforms all the other models in most, if not all, the settings. All the models were untrained and run on a single NVIDIA GTX 1080Ti GPU.
\paragraph{Image reconstruction and generation:}
In Figure \ref{fig:celeba_reconstruct}, we present the effectiveness of the  FInC Flow model in the reconstruction (sampling) of the images. First, we feed the input image to forward flow and get the \emph{latent vector} (). To reconstruct the images from the \emph{latent vector} (), give the  as input to the inverse flow. Figure \ref{fig:celeba_reconstruct} present the reconstructed face images for the CelebA dataset after training our model for  epochs. The model takes a random sample from the Gaussian distribution for the \emph{latent vector} to generate sample images. This \emph{latent vector} is used to generate images by going backward in the flow model. In Figure \ref{fig:uncurated_samples}, we present generated sample by our model on the MNIST, CIFAR-10, and ImageNet-64x64 dataset.





























 \section{Conclusion}

With a parallel inversion approach, we present a  invertible convolution for Normalizing flow models. We utilize it to develop a model with highly efficient sampling pass, normalizing flow architecture. We implement our parallel algorithm on GPU and presented benchmarking results, which show a significant enhancement in forward and sampling speeds when compared to alternative methods for  invertible convolution.




%
 



\bibliography{Example}
\bibliographystyle{Example}

\section*{Additional details}
Code to our implementation is available here:   \url{https://github.com/aditya-v-kallappa/FInCFlow}
\noindent
\paragraph{Datasets}
We train our model on standard benchmark datasets MNIST \citep{deng2012mnist}, CIFAR-10 \citep{krizhevsky2009learning}, and ImageNet \citep{russakovsky2015imagenet} sampled down to  and . We also train our model on CelebA \citep{liu2015deep} sampled down to . Figure \ref{fig:celeba_reconstruct}a present the CelebA- reconstructed samples and Figure \ref{fig:uncurated_samples}b for the CIFAR-10 and ImageNet-64x64 generated samples.

\paragraph{Hyperparameters}
To train our model, we use Adam optimizer \citep{kingma2014adam} with learning rate of 0.001 with an exponential decay of 0.99997 per epoch. For training on Imagenet, we also make sure that the gradients stay between  and  by clipping them. 

\paragraph{Masking}
To make sure the masked values in the Padded Conv Blocks, we ensure they are not affected by back propagation. To achieve this, we reset the gradients of the masked values to zero after every training iteration.

\paragraph{Cuda code details}
To run CUDA code, we use PyTorch-LTS 1.8.2 and cudatoolkit 10.2. While we can implement Algorithm \ref{alg:finc_flow_inverse} to find inverse of each padded block individually, we can take advantage of the fact that all 4 Padded Conv Blocks are equivalent after proper flipping of padded inputs and kernels. This is done by using Algorithm \ref{alg:finc_flow_inverse} on GPU.

\noindent
First we split  into 4 parts across channel dimension following the architecture shown in 2. This is given in Line 1. Then we flip  and kernels to match TL-padding which is given in Line 2. Then we concatenate them to get the final  and  respectively which are given in Lines 3 and 4. Then we apply Algorithm \ref{alg:finc_flow_inverse} to find the inverse. We do the reverse process of the above to get the correct . The steps are given in Lines 5, 6, 7, 8.

\noindent
We fix the number of threads of each grid of the GPU to be 1024, the number of  to be the batch size and  to be 4 which is the number of Padded Conv Blocks in a single FInC Flow Unit. This ensures that not only GPU inverts the whole FInC Flow Unit at once but also on a batch of images. 

\begin{algorithm}[!h]
    \caption{Fast Parallel Inversion Algorithm for FInC Flow \emph{Unit}}
    \label{alg:finc_flow_inverse}
    \KwIn{ - Convolution Kernels of different PCB,  -  Output of the FInC Flow \emph{Unit}}
    \KwResult{ -  Input to the FInC Flow Unit / Inverse of the FInC Flow Unit}
    \begin{enumerate}
        \item{ }
        \item{Flip  (inplace) appropriately to match \emph{TL} padding}
        \item{}
        \item{}
        \item{Apply Algorithm \ref{alg:inv_algo} with input   to get }
        \item{}
        \item{Flip  appropriately to get the correct output }
        \item{)}
    \end{enumerate}
\end{algorithm}


\paragraph{Running MaCow, Emerging, CInC Flow, SNF}
For MaCow and SNF, we use the official code released by the authors. Emerging was implemented in PyTorch by the authors of SNF. We make use of that. We have implemented CInC Flow on PyTorch to get the results.

\paragraph{Computing Run-time and Confidence Intervals}
We run the model (both forward and sampling)  times and ignore the  run as it includes the initialization time. We calculate the mean, standard deviation and  confidence interval and plot the numbers. 
\\
To calculate forward time, we pass  images, as for sampling times, we sample  images. We present these numbers in Table-\ref{table:params}.
\\
For sampling time comparison of different models shown in Figure \ref{fig:timing}, we set the total number of parameters for all the models to be close to  to make it a fair comparison. 

\paragraph{Hardware/Training Time}
Our hardware setup consists of Intel Xeon E5-2640 v4 processor providing 40 cores, 80 GB of DDR4 RAM, 4 Nvidia GeForce GTX 1080 Ti GPUs each with  GB of VRAM. We train our model on all GPUs using PyTorch's Data Parallel class. We implement early stopping mechanism for smaller datasets like MNIST, CIFAR-10. For others we train the model for a maximum epochs. To evaluate Forward Time and Sampling Time, we use only one of the GPUs.We evaluate our FInC Flow model on MNIST, CIFAR-10, Imagenet-32x32, and Imagenet-64x64 datasets for three metrics - (a) Loss expressed in Bits per Dimension (BPD), (b) Forward Pass Time (FT): Time taken for 100 images to be passed through the model and (c) Sampling Time(ST): Time taken by the model to generate 100 images.  To do this, we train our model with Adam Optimizer with a learning rate () of  and exponentially reduce the  by  after each epoch. We have used 4 NVIDIA GTX 1080 Ti GPUs to train our model. Evaluation(FT/ST) is done on a single GPU.
 





\end{document}
