\documentclass[microtype]{jloganal}

\usepackage{pinlabel}
\usepackage{latexsym}
\usepackage[all]{xy}





\title[A Coinductive Approach to Computing with Compact Sets]{A Coinductive Approach to Computing with Compact Sets}

\author[Ulrich Berger]{Ulrich Berger}
\givenname{Ulrich}
\surname{Berger}
\address{Department of Computer Science\\
Swansea University\\\newline
Swansea SA2 8PP, United Kingdom}
\email{U.Berger@swansea.ac.uk}
\urladdr{http://www-compsci.swan.ac.uk/~csulrich/} 

\author[Dieter Spreen]{Dieter Spreen}
\givenname{Dieter}
\surname{Spreen}
\address{Department of Mathematics\\
University of Siegen\\\newline 
57068 Siegen, Germany\\ and\\\newline
 Department of Decision Sciences\\
University of South Africa\\\newline 
Pretoria 0003, South Africa}
\email{spreen@math.uni-siegen.de}
\urladdr{http://www.uni-siegen.de/fb6/tcs/team/spreen/}

\keyword{program extraction}
\keyword{exact real number computation}
\keyword{computing with continuous objects}
\keyword{compact sets}
\subject{primary}{msc2010}{03B35}
\subject{primary}{msc2010}{68T15}
\subject{primary}{msc2010}{03C78}
\subject{secondary}{msc2010}{03B70}

\arxivreference{cs.LO/1510.08498}
\arxivpassword{xuet5}

\volumenumber{}
\issuenumber{}
\publicationyear{}
\papernumber{}
\startpage{}
\endpage{}
\doi{}
\MR{}
\Zbl{}
\received{}
\revised{}
\accepted{}
\published{}
\publishedonline{}
\proposed{}
\seconded{}
\corresponding{}
\editor{}
\version{}




\newtheorem{claim}{\it Claim}
\theoremstyle{plain}
\newtheorem{theorem}{Theorem}[section]
\newtheorem*{sdthm}{Soundness Theorem}
\newtheorem{lemma}[theorem]{Lemma}
\newtheorem{proposition}[theorem]{Proposition}
\newtheorem{corollary}[theorem]{Corollary}
\newtheorem{example}[theorem]{Example}
\theoremstyle{definition}
\newtheorem{definition}[theorem]{Definition}
\newcommand{\BB}{\mathbb{B}} 
\newcommand{\II}{\mathbb{I}}
\newcommand{\NN}{\mathbb{N}}
\newcommand{\ZZ}{\mathbb{Z}}
\newcommand{\QQ}{\mathbb{Q}}
\newcommand{\DD}{\mathbb{D}}
\newcommand{\RR}{\mathbb{R}}
\newcommand{\AAA}{\mathcal{A}}
\newcommand{\BBB}{\mathcal{B}}
\newcommand{\CCC}{\mathcal{C}}
\newcommand{\FFF}{\mathcal{F}}
\newcommand{\MMM}{\mathcal{M}}
\newcommand{\JJJ}{\mathcal{J}}
\newcommand{\OOO}{\mathcal{O}}
\newcommand{\PPP}{\mathcal{P}}
\newcommand{\QQQ}{\mathcal{Q}}
\newcommand{\TTT}{\mathcal{T}}
\newcommand{\AV}{\mathrm{AV}}
\newcommand{\av}[1]{\mathrm{av}_{#1}}
\newcommand{\ball}[3]{\mathrm{B}_{#1}(#2,#3)}
\newcommand{\cball}[3]{\mathrm{B}^\le_{#1}(#2,#3)}
\newcommand{\coco}{\mathrm{C}}
\newcommand{\compact}{\mathrm{K}}
\newcommand{\hdm}{\mu_{\mathrm H}}
\newcommand{\treehdm}{\delta_{\mathrm H}}
\newcommand{\hdr}{\eta_\mathrm{H}}
\newcommand{\mcr}{\eta_\mathrm{mc}}
\newcommand{\mds}[1]{\mathcal{#1}}
\newcommand{\SD}{\mathrm{SD}}
\newcommand{\val}[1]{[\![#1]\!]}
\newcommand{\set}[2]{\mbox{}}
\newcommand{\fun}[3]{\mbox{}}
\newcommand{\pfun}[3]{\mbox{}}
\newcommand{\pair}[1]{\langle #1 \rangle}
\newcommand{\card}[1]{\mathopen\parallel #1 \mathclose\parallel}
\def\dom{\mathop{\mathstrut\rm dom}}
\def\range{\mathop{\mathstrut\rm range}}
\def\int{\mathop{\mathstrut\rm int}\nolimits}
\newcommand{\cantor}{\mathbf{C}}
\newcommand{\fminus}{f_{-}}
\newcommand{\fplus}{f_{+}}
\newcommand{\palm}[1]{\langle #1 \rangle}
\newcommand{\PALM}{\mathrm{PALM}}
\newcommand{\id}{\mathrm{id}}
\newcommand{\rea}[2]{#1\,\mathbf{r}\,#2}
\newcommand{\powplus}{\mathcal{P}_{+}}
\newcommand{\branches}[1]{E_{#1}}
\newcommand{\subtree}[2]{#1_{#2}}
\newcommand{\powfin}{\mathcal{P}^+_{\mathrm{fin}}}
\newcommand{\cauchy}{\mathrm{A}}
\newcommand{\basic}{\mathrm{K}_0}
\newcommand{\prea}[1]{\mathbf{R}(#1)}
\newcommand{\reap}[1]{\widetilde{#1}}
\newcommand{\pleft}[1]{\mathbf{p}_0(#1)}
\newcommand{\pright}[1]{\mathbf{p}_1(#1)}
\newcommand{\pow}[1]{\mathcal{P}(#1)}
\newcommand{\powf}[1]{\mathcal{P}_{\mathrm{fin}}}




\begin{document}

\begin{abstract}
  Exact representations of real numbers such as the signed digit
  representation or more generally linear fractional representations or the
  infinite Gray code represent real numbers as infinite streams of digits. In
  earlier work by the first author it was shown how to extract certified
  algorithms working with the signed digit representations from constructive
  proofs. In this paper we lay the foundation for doing a similar thing with
  nonempty compact sets. It turns out that a representation by \emph{streams}
  of finitely many digits is impossible and instead \emph{trees} are needed.
\end{abstract}

\begin{asciiabstract}
Exact representations of real numbers such as the signed digit
  representation or more generally linear fractional representations or the
  infinite Gray code represent real numbers as infinite streams of digits. In
  earlier work by the first author it was shown how to extract certified
  algorithms working with the signed digit representations from constructive
  proofs. In this paper we lay the foundation for doing a similar thing with
  nonempty compact sets. It turns out that a representation by streams
  of finitely many digits is impossible and instead trees are needed.
  \end{asciiabstract}
  
  \maketitle
  


\section{Introduction}
\label{sec-intro}
Digital representations of real numbers have been widely studied in the 
literature. Probably best known is the signed digit representation as 
considered in 
Ciaffaglione and Di Gian\-antonio \cite{CiaffaglioneGianantonio06}, 
Escard\'{o} and Marcel-Romero \cite{EscardoMarcial-Romero07}
Berger and Hou~\cite{BergerHou08}
and many others, 
where a real number in  is represented by a stream of 
signed digits , a digit  representing the mapping 
. This has been generalized
to linear fractional representations studied in Edalat and 
S\"underhauf \cite{EdalatSuenderhauf98} as well as Edalat and 
Heckmann  \cite{EdalatHeckmann02}
where digits represent maps of the form .
A variant of the signed digit representation is the infinite
Gray code introduced by Tsuiki \cite{Tsuiki02} which represents real 
numbers in  by a binary stream
with possibly one undefined entry. The infinite Gray code has the 
remarkable property
that it is non-redundant, that is, every real has exactly one code.


Berger \cite{Berger11} showed how to use the method of program extraction 
from proofs 
(see eg Schwichtenberg and Wainer \cite{SchwichtenbergWainer12})
to extract certified algorithms 
working with the signed digit representations. 
In this paper we lay the foundation for doing a similar
thing with nonempty compact sets. 


In the first part of the paper 
(\fullref{sec-digit-space} to \fullref{sec-computable}) 
we develop our approach in the style of classical mathematics 
speaking explicitly about representations. 
In \fullref{sec-digit-space} we generalize the signed digit representation to 
\emph{digit spaces} , that is, we replace the interval  
by a complete bounded metric space  and the signed digits by a 
finite set  of contractions whose images cover .
In \fullref{sec-compact-hausdorff} and \fullref{sec-tree} we introduce the hyper-space
of nonempty compact sets and develop a representation of compact sets by non-wellfounded 
\emph{trees} of digits, after having shown that in most situations a representation by
\emph{streams} of digits is impossible.
In \fullref{sec-computable} we compare the notion of computability for compact sets 
generated by our tree representation with those studied by Brattka and Presser~\cite{BrattkaPresser03}.

In the second part
(\fullref{sec-coind} to \fullref{sec-equiv}) we recast the approach constructively 
in a representation free way.
We introduce a coinductive predicate on the powerset of  whose realizers 
are trees representing nonempty compact subsets of  generalizing the
coinductive approach to the signed digit representation studied by Berger \cite{Berger11}.
We sketch how this approach can be used to extract programs computing
with compact sets from constructive proofs and comment on the relation to 
iterated function systems as studied by Edalat \cite{Edalat97}. The Cantor set is considered as an example.

\section{Digit spaces}
\label{sec-digit-space}
A metric space  with metric  is called \emph{bounded} if there exists
a number , called \emph{bound} of , such that
 for all .  
A \emph{contraction} on  is a function  such that there exists 
a number , called \emph{contraction factor}, with  
 for all . 
\begin{definition}
\label{def-digit-space}
A \emph{digit space}  consists of
a bounded and complete nonempty metric space  and a 
finite set  of contractions on , called \emph{digits}, 
that cover , that is, , where . 
\end{definition}
Our running example of a digit space will be the \emph{signed digit space}
 where  with the 
usual metric and  with 
 and .
By a slight abuse of notation we will call both the elements of  and of 
 \emph{signed digits} or just \emph{digits}.

In the following we identify a finite sequence of digits 
 with the composition
 and a digit  with the singleton sequence . The set of all finite sequences of digits will be denoted by .
Moreover, we let  be a common contraction factor for all digits .

\begin{lemma}
\label{lem-digit-compact}
Every digit space is compact, that is, the underlying metric space is
compact.
\end{lemma}
\begin{proof}
Let  be a digit space. 
It suffices to show that  is totally bounded. So let .
Let  be so large that  where  is a common
contraction factor for each . Then for 
arbitrarily chosen , the set  
is a finite -covering, ie the  many balls  of radius  around  cover . 
\end{proof}
Conversely, every compact metric space is bounded, since the metric is continuous. Note that for any , the set of all elements  with  is dense in .

The purpose of a digit space is to provide representations for the
elements of  by infinite streams of digits. 
Let  be the set of all infinite sequences of elements of  
and set for  :
 
\begin{lemma}
\label{lem-digit-stream}
Let  be a digit space. Then

is a singleton for every . 
\end{lemma}
\begin{proof}
Since digits are continuous and  is compact, the set  
is compact as well. As  is Hausdorff, it is closed in particular. 
Moreover, it is nonempty, since  is nonempty. 
Since clearly ,
the family  has the finite 
intersection property. 
Therefore,  is not empty. 
On the other hand, 

which implies that 
, 
as . 
Consequently, . 
\end{proof}
\begin{definition}
\label{def-digit-stream-value}
Let  be a digit space and .
In view of Lemma~\ref{lem-digit-stream}, we let  
denote the uniquely determined element of 
.
\end{definition}

Note that for every infinite sequence  of digits and every digit ,  is an infinite sequence of digits as well.

\begin{lemma}\label{lem-valrec}
For every  and , .
\end{lemma}
\begin{proof}
We have that , for all , and hence that . Consequently, .
\end{proof}

The next technical lemma will be useful in what follows.

\begin{lemma}\label{lem-gensur}
Let . Then, for every , there is some  with  and .
\end{lemma}
\begin{proof}
Let  and let  such that . 
By the covering property of , there exist  and  such that .
Hence , in particular . 
By dependent choice, there is
some  with  and 
 for all , hence .
\end{proof} 

As is well known, 
 is a compact bounded metric space with metric


\begin{proposition}\label{prop-valcont}
\begin{enumerate}
\item \label{valcont1}  is onto and uniformly continuous.  
\item \label{valcont2} The metric topology in  is equivalent to the quotient topology induced by .
\end{enumerate}
\end{proposition}
\begin{proof}
(\ref{valcont1}) As a consequence of the preceding lemma,  is onto. For the verification of the remaining statement let  be distinct elements of . Then  and therefore 

By definition of , . It follows that

from which we obtain that  is uniformly continuous.

(\ref{valcont2}) We have to show that for any subset  of ,  is open in the metric topology if, and only if,  is open in the metric topology on . 

The `only-if'-part holds as  is continuous. For the other direction assume that  is open in  and . We need to show that  lies in the interior of .

Since , there is some  so that . Hence, . Since the latter set is open, there is some  with , which means that , for all  with .

\begin{claim}
\label{cl-qtop-1}


In words, there exists a natural number  such that for all sequences  and  in  the first  elements of which coincide, if  then .
\end{claim}

Since  is continuous, we have that  is closed and hence compact. Moreover,

Hence, there exist  with , for , so that

Set  and let  with . Then there is some  with . It follows that

Hence, . Thus, Claim~\ref{cl-qtop-1} is proven.

Let  be as in Claim~\ref{cl-qtop-1} and set 

as well as 

Then  and  are closed sets (since they are finite unions
of compact, hence closed sets) that cover . Furthermore, .
It follows that  is an open subset of . Therefore, it suffices
to show that .
Assume to the contrary that . Then there is some 
such that  and there exists some 
. 
With Lemma~\ref{lem-gensur} it moreover follows that there are 
 with prefix  such that 
 and . 
Thus . With Claim~\ref{cl-qtop-1} 
we therefore obtain  that , a contradiction.
\end{proof}

Note that the proof of Proposition~\ref{prop-valcont}(\ref{valcont1}) shows that 
 is even H\"older continuous of order . 
In particular if , then  is Lipschitz-continuous 
with Lipschitz constant .

Continuous images of compact sets are compact again. The next result is therefore a consequence of Proposition~\ref{prop-valcont}.
\begin{corollary}\label{cor-cant-x}
Let  be compact. Then  is a compact subset of .
\end{corollary}

Conversely, if  is compact, then  is closed, as  is Hausdorff. Hence,  is closed as well. Since  is compact, it follows that  is compact too. Moreover, , because  is onto.
\begin{lemma}\label{lem-x-cant}
Let  be a compact subset of  . Then there is a compact subset  of  with .
\end{lemma}


\section{Compact sets and the Hausdorff metric}
\label{sec-compact-hausdorff}

We start this section by deriving some facts about metric digit spaces 
needed in the sequel.

\begin{definition}
Let  be a metric space and . 
Then a subset  of  is an \emph{--chain} if 
, for all  such that .
\end{definition}

\begin{lemma}\label{lem-cbnd}
Let  be a digit space such that  is bounded with bound . 
Moreover, let  be a uniform contraction factor for all digits in . 
Then the size of every --chain in  is bounded by . 
\end{lemma}
\begin{proof}
Let  be a --chain in . Because of the covering property 
we have that . Now, 
let  with  and assume that there is some 
 so that . Then there are  
with  and . Hence


Since  is a bound of , we have that . On the other 
hand, , as  is a --chain. With 
(\ref{eq-bnd}) it thus follows that , a 
contradiction. Hence, , for each 
. Thus, . 
\end{proof}

For a metric space  we denote by  the set of 
nonempty compact subsets of . The \emph{Hausdorff metric}, , 
on  is defined by the formula

where 
. 

The subsequent properties are often useful. Let to this end

\begin{lemma}\label{lem-hausm}
For ,  and a contraction  the following statements hold:
\begin{enumerate}

\item\label{lem-hausm-1} 

\item\label{lem-hausm-2} 

\item\label{lem-hausm-3} 

\item\label{lem-hausm-4} 

\end{enumerate}
\end{lemma}

Note that  has the same bound  as . Moreover, it is well-known that  inherits completeness and compactness from .
However, we cannot expect  to
have a finite covering system of contractions as we show in the 
following.


\begin{lemma}\label{lem-cptchain}
Let  be a bounded metric space and . 
If  has an --chain  of size , then  has an 
--chain of size .
\end{lemma}
\begin{proof}
It suffices to show that the collection of all nonempty subsets of  is 
an --chain with respect to the Hausdorff metric. Let  be 
two different nonempty subsets of . We have to show that 
. Since ,  or 
 are not empty. Without restriction we consider the first 
case. Let  and . Then  and hence 
. 
It follows that . 
\end{proof}
\begin{corollary}\label{cor-lin}
Let  be a bounded metric space and  have a 
finite set of 
digits. Then there are constants  such that for 
every --chain  in , .
\end{corollary}
\proof
Let  be a bound of ,  be the set of digits of  
of size  and
 be a uniform contraction factor of the digits in .  
If  is a --chain in , it follows with Lemma~\ref{lem-cptchain} that  
has a --chain of size .  By Lemma~\ref{lem-cbnd} we therefore have that
 . Thus,

Now, let  be a --chain in  and . Then

It follows that  is a --chain as well. Therefore, 
by the above:



\begin{lemma}
\label{lem-convex}
Let  be a non-trivial bounded and convex subset of a normed linear space.
Then there cannot be finitely many contractions on  that 
cover .
\end{lemma}
\begin{proof}
Let  be different elements of . Hence 
for some . Let  and set 

as well as . Then 
is a --chain, since for  we 
have that:


Now, assume that  has a finite system of digits. 
By Corollary~\ref{cor-lin} there are constants  
(independent of ) such that:

On the other hand, , which is a contradiction for
large enough .
\end{proof}

\section{Representation of compact sets by trees} 
\label{sec-tree}
Since, according to Lemma~\ref{lem-convex}, it will in most cases be 
impossible to turn  into a digit space, we consider a 
representation of compact sets by \emph{trees} (instead of streams) 
of digits of the original digit space .
\begin{definition}
\label{def-tree}
Let  be a digit space. A \emph{digital tree} is a nonempty set 
 of finite sequences of digits that 
is downwards closed under the prefix ordering and has no maximal element, that is,
 and whenever
, then  and
 for some .
\end{definition}
Note that each such tree is finitely branching as  is finite.
Moreover, every element  can be continued to an 
infinite path  in , that is,  is such that , for , and 
 for all .
In the following we write   to mean that  is a 
path in  and by a path we always mean an infinite path.

Let  denote the set of digital trees with digits in  and for  and , let  be the finite initial subtree of  of height . Then: 

Every such initial subtree defines a map  from  into the powerset of  in the obvious way:


\begin{definition}
\label{def-val}
For every  we define its \emph{value} by

\end{definition}

\begin{lemma}
\label{lem-semT}
.
\end{lemma}
\begin{proof}
Observe that . Therefore, , for every . 
Conversely, let . Then there is some  with , for each . Since the set of all  with  is a finitely branching infinite tree, it follows with K\"onig's Lemma that there is a path  with , for all . Thus, .
\end{proof}

\xyoption{line}
\begin{corollary}\label{cor-tree-valrec}
Let  and let  be the set of digits 
such that  and  
(),
ie  is the th  immediate subtree of :

Then .
\end{corollary}
\begin{proof}
Apply Lemma~\ref{lem-valrec}.
\end{proof}
Note that this interpretation of a digital tree corresponds to the IFS-tree
of an iterated function system in Edalat \cite{Edalat97}.

Let  be a digital tree and  be an infinite path not lying in . Then there is a finite initial segment  of  that is not contained in , since as explained above, a path is identified with the sequence of its finite initial segments. Because  is closed under taking initial segments, for all infinite continuations  of  we have that  as well. Thus,  is open in the metric topology on .

\begin{lemma} \label{lem-treecl}
For any digital tree the set of its infinite paths is closed in .
\end{lemma}

Obviously, the set of infinite paths of a tree is nonempty. Conversely, let  be a nonempty closed subset of  and . Obviously,  is a digital tree. 


\begin{lemma}
Let  be a nonempty closed subset of . Then the sequences in  are exactly the paths of .
\end{lemma}
\begin{proof}
Clearly, every element of  is a path in . Conversely, if  is 
a path in , then its initial segment of length  is of the form 
 for some . Hence  has distance 
 from . It follows that  is in the closure of ,
hence in .
\end{proof}

As we have already seen in Corollary~\ref{cor-cant-x} and Lemma~\ref{lem-x-cant}, the compact subsets of  are exactly the images of the compact subsets in  under . Thus, we have the following result.

\begin{lemma}
\label{lem-tree-compact}
The nonempty compact subsets of a digit space  are exactly
the values of digital trees.
\end{lemma}

The metric defined on  in Section~\ref{sec-digit-space} can be transferred to . As we will see next, it coincides with the Hausdorff metric.

\begin{lemma}\label{lem-treehausd}
For ,

\end{lemma}
\begin{proof}
Without restriction let .  Then there exists 

Note that  as . It follows that . Let . Then there is some  such that , ie . Thus,  and hence , for all . This shows that , for all . Similarly, we obtain that , for all . It follows that .

Now, assume that . Then there is some  with . Let . Then there exists  with , from which we obtain that

and hence that

respectively, 

This shows that , where . In the same way we obtain that also . Thus, . Since , we have that . However, by definition of , ,  a contradiction. Thus, .
\end{proof}

\begin{proposition}\label{prop-conttreeval}
\begin{enumerate}
\item\label{conttreeval1}
 is onto and uniformly 
continuous.
\item\label{conttreeval2}
The topology on  induced by the Hausdorff metric is equivalent 
to the quotient topology induced by .
\end{enumerate}
\end{proposition}
\begin{proof}
(\ref{conttreeval1}) Ontoness is a consequence of Lemma~\ref{lem-tree-compact}. For the verification of uniform continuity let ,  and . With Inequality~(\ref{eq-unicont}) we have that

and hence that

Similarly, we obtain that also  and hence that  from which the uniform continuity of  follows.

(\ref{conttreeval2}) The statement follows by a straightforward adaption
of the proof of Proposition~\ref{prop-valcont}(\ref{valcont2}).
\end{proof}


\section{Computably compact sets}
\label{sec-computable}

The purpose of the present paper is to provide a logic-based approach to 
computing with continuous data. In this section we compare it with 
Weihrauch's Type-Two Theory of Effectivity~\cite{Weihrauch00}.

\begin{definition}[{\rm Brattka and Presser \cite{BrattkaPresser03}}]
\label{met-computable}
Let  be a metric space with dense subspace , say

the elements of which are called \emph{basic elements}.
Then  is called \emph{computable} if the two sets

are effectively enumerable, ie the function  is computable.
\end{definition}

Note that when we say that  is effectively enumerable, we mean that with respect to a canonical coding  of , the set

is computably enumerable. Similarly, when we say that  is computable, we mean that there is a computable function  such that for given ,  is the G\"odel number of a computable function  so that  is a Cauchy sequence of rationals converging to . In what follows we will work with  finite objects such as basic elements, finite sets of basic elements, digits or finite sequences of digits directly as in the above definition and leave it to the reader to make the statements precise, if wanted. Note that by doing so we identify a digit  with the letter . 

\begin{definition}
\label{def-dcomp}
Let  be metric spaces with countable dense subspaces  and , respectively. A uniformly continuous map  is \emph{computable} if it has a computable modulus of continuity and there is a procedure  which given  and  computes a basic element  with .
\end{definition}

As is easily verified, the set of all computable maps on  is closed under composition.

\begin{definition}\label{def-compdigsp}
Let  be a digit space such that the underlying metric space  has a countable dense subset  with respect to which it is computable.  is said to be a \emph{computable digit space} if, in addition, all digits  are computable.
\end{definition}

Let 


Besides , computable digit spaces possess other canonical dense subspaces. For  set
 
We want to show that  and  are \emph{effectively equivalent} in the sense that given  and  we can compute a sequence  such that , and that similarly there is a computable function  with:


To do so, some more requirements have to be satisfied.

\begin{definition}
\label{def-well-covering}
A digit space  is \emph{well-covering} if every element of  is
contained in the interior of  for some .
\end{definition}
\begin{lemma}
\label{lem-well-covering}
Let  be a well-covering digit space. Then there exists 
 such that for every  there exists  with
. 
\end{lemma}
\begin{proof}
Assume the contrary. Then for every  there exists  such
that  is not contained in  for any .
Let  be an accumulation point of the . Then clearly,  is not in the 
interior of  for any .
\end{proof}

Each  as in the preceding lemma will be called a \emph{well-covering number}. 

\begin{definition}
\label{def-deceffdense}
Let  be a computable digit space. We call 
\begin{enumerate}
\item\label{def-deceffdense-1}  \emph{decidable} if for ,  and  it can be decided whether ;

\item\label{def-deceffdense-2} \emph{constructively dense} if there is a procedure that given ,  and  computes a  such that .

\end{enumerate}
\end{definition}

\begin{lemma}
\label{lem-Q-QD}
Let  be a well-covering, decidable and constructively dense digit space. Then, for every  and  a sequence  of digits can effectively be found such that .
\end{lemma}
\proof
Let  be a well-covering number for  and set

Given  and , proceed as follows:
\begin{enumerate}
\item Let .  If , output  (empty sequence). Otherwise, set , increase  by 1 and go to (2).

\item Assume that  has been computed so far. Use the decidabilty of  to find some  with .
If , output . Otherwise, use computable density to find some  such that

increase  by 1 and go to (2).

\end{enumerate}

Now, if , we have that . Hence,

Otherwise, we have found  and  with

for . Then  and: 



\begin{lemma}
\label{lem-QD-Q}
Let  be computable. Then there is a procedure  which given  and  produces a basic element  so that .
\end{lemma}
\begin{proof}
Since , there is a procedure  which on input  computes a basic element  with . Now, define  by recursion on :

On input , if , output the result of procedure  on input . Otherwise, assume that  and that the result of  on input  and  is . Then output the result of applying  to input  and . 
\end{proof}

Summing up we obtain the following result.

\begin{proposition}\label{prop-baseeq}
Let  be a well-covering, decidable and constructively dense digit space. Then the topological bases  and  are effectively equivalent.
\end{proposition}

\begin{definition}
\label{def-computable-element}
Let  be a digit space. An element  is \emph{computable} if there is a computable infinite sequence  with . Denote the set of all computable elements of  by .
\end{definition}

Let  be computable and let this be witnessed by . With Lemma~\ref{lem-gensur} and (\ref{eq-mudelta}) we obtain that . Set 

and assume that  is computable. By Lemma~\ref{lem-QD-Q}, for any given , we can compute a basic element  with . It follows that . This shows that . The converse implication will be a consequence of Theorem~\ref{thm-a-sub-c} derived later in a constructive fashion. To this end a further condition is needed.

\begin{definition}
\label{def-continvdeff}
A computable digit space  \emph{has approximable choice} if for every
 there is an effective procedure 

such that for all :
\begin{enumerate}

\item\label{def-continvdeff-1} For all  and all , .

\item\label{def-continvdeff-2} One can compute  such that for all ,
if  then 
.

\item\label{def-continvdeff-3} For all  there is some  with 
.
\end{enumerate} 
\end{definition} 

Obviously,  every computable digit space with approximable choice is constructively dense.

\begin{proposition}
\label{prop-eqcomp-tte}
Let  be a well-covering and decidable computable digit space with approximable choice. Then .
\end{proposition}

For , a map  is a \emph{right inverse} of  if  is the identity on .

\begin {proposition}\label{prop-choice-inv}
A computable digit space  has approximable choice if, and only if, every  has a computable right inverse.
\end{proposition}
\begin{proof}
Assume that  has approximable choice and let  and . Because of density there is some  with , for all .

Use approximable choice to pick the function 
.  
For , pick  according to approximable choice, part (\ref{def-continvdeff-2}).
Let  such that , for . 
Without restriction let  be such that , for all . Set .
By approximable choice, part (\ref{def-continvdeff-3}), there is some  with
.
Because of the assumption on , we have that , 
for . Hence, . It follows that 
. Thus,  is a fast Cauchy sequence. 
Since  is complete, it converges to some . 
As  is continuous, we obtain that 


Now, let  with  as well as  with , for . Moreover,  such that , for . Without restriction assume that , for all . Finally, let  and  with . Then  and hence . It follows that  and thus .

For , we obtain that , ie  does not depend on the choice of the approximating sequence . Define . By what we have just shown,  is uniformly continuous with computable modulus of continuity. If , choose . Then . Thus,  is computable.

Conversely, let  be a right inverse of . For  let 

Since  is computable, we can compute for any given  and  a basic element  so that . Set . It remains to verify the conditions in Definition~\ref{def-continvdeff} of having approximable choice:

(\ref{def-continvdeff-1}) Let . Without restriction let . Then .

(\ref{def-continvdeff-2}) As  has a computable modulus of continuity, for given  we can compute a  such that for , if  then , from which it follows that
. 


(\ref{def-continvdeff-3}) is obvious: choose .
\end{proof}

In Type-Two Theory of Effectivity an element  is defined to be computable, if it is contained in . So, it follows that both computability notions coincide. In the present approach elements of  are represented by infinite streams of digits. In Type-Two Theory of Effectivity, similarly, an element  is represented by an infinite sequence  of basic elements with . The resulting representation is called the \emph{Cauchy representation} . As follows from the proofs leading to the preceding proposition, one can computably pass from an infinite stream  of digits to an infinite sequence of basic elements  so that , and vice versa. This means that there are computable translations between both representations as summarized by the next result.

\begin{theorem}
\label{thm-stream=cauchy}
Let  be a well-covering and decidable computable digit space with 
approximable choice. Then there are computable operators 
 and  
such that for  and , 

\end{theorem}

Let us next start an analogous investigation for compact sets. As has already been mentioned,  is a complete compact metric space. It has a canonical dense subset, the set , the elements of which we will call \emph{basic sets}. 
If  is computable the same holds for  (see Brattka~\cite{Brattka99}). 

So, every nonempty compact set is the limit of a fast Cauchy sequence of basic sets with respect to the Hausdorff metric. Brattka and Presser~\cite{BrattkaPresser03} call the resulting representation  of  the \emph{Hausdorff representation}.

\begin{proposition}\label{prop-treetohausdorff}
Let  be a computable digit space. Then there is a 
computable operator 
 such that for every 
,  with 

\end{proposition}
\begin{proof}
Let ,  and . Moreover, let the computable map  be as in Lemma~\ref{lem-QD-Q}. Then we have that 

for . Thus,

Now, set  where

Then we obtain that . In order to see that also , let . Hence, , for some . It follows that  and therefore . Thus, . 

On the other hand, if , then , for some . Since , we obtain that . Consequently, .
\end{proof}

As a consequence of Theorem~\ref{thm-ak-sub-ck}, also the converse holds: given an infinite word in  representing a nonempty compact set , one can compute a digital tree with value .

\begin{theorem}\label{thm-hausdorfftotree}
Let  be a well-covering and decidable computable digit space 
with approximable choice. Then there are computable operators 
 and 
 
such that for  and ,

\end{theorem}

It follows that we can effectively translate representations of compact 
sets in Type-Two Theory of Effectivity into representations in our setting, and vice versa.

\begin{definition}
\label{def-computable-compact}
Let  be a digit space. A set  is \emph{computable}
if it is the value of computable digital tree. Denote the set of computable compact sets by .
\end{definition}


In Type-Two Theory of Effectivity a compact set is \emph{computable} if it is contained in , where:


With the preceding theorem we obtain that both notions coincide.

\begin{corollary}\label{cor-tte-tree}
Let  be a well-covering and decidable computable digit space with approximable choice. Then .
\end{corollary} 



\section{Extracting digital trees from coinductive proofs}
\label{sec-coind}
In this section we recast the theory of digit spaces and their hyper-spaces
in a constructive setting with the aim to extract programs that provide
effective representations of certain objects or transformations 
between different representations.
One of the main results will be effective transformations between
the Cauchy--Hausdorff representation and the digital tree representation
of the hyper-space showing that the two representations are effectively
equivalent.
The method of program extraction will be based on a version of realizability,
and the main constructive definition and proof principle will be coinduction.
The advantage of the constructive approach lies in the fact that proofs can be
carried out in a representation-free way. Constructive logic and the Soundness
Theorem guarantee automatically that proofs are witnessed by effective and
provably correct transformations on the level of representations.


Regarding the theory of realizability and its applications to constructive
analysis we refer the reader to Schwichtenberg and Wainer~\cite{SchwichtenbergWainer12},
Berger and Seisenberger~\cite{SeisenBerger10} and Berger~\cite{Berger11}. Here, we only recall the main
facts.
We work in many-sorted first-order logic extended by the formation
of inductive and coinductive predicates.
Realizability assigns to each formula  a unary predicate 
to be thought of as the set of realizers of . Instead of 
one often writes  (`` realizes '').
The realizers  can be typed or untyped, but for the understanding of the 
following, details about the nature of realizers are irrelevant.
It suffices to think of them as being (idealized, but executable) functional
programs or (Oracle-)Turing machines.
The crucial clauses of realizability for the propositional connectives are:

Hence, an implication is realized by a function and a conjunction is realized
by a pair (accessed by left and right projections, ).

Quantifiers are treated uniformly in our version of realizability:

This reflects the fact we allow variables  to range over abstract 
mathematical objects without prescribed computational meaning. Therefore, the
usual interpretation of  to mean
 doesn't make sense since we would use
the abstract object  as an input to the program .

For atomic formulas , where  is a predicate and  are 
terms, realizability is defined in terms of a chosen predicate  
with on extra argument place, that is,

The choice of the predicates  allows us to fine tune the computational
content of proofs.

So far, we have covered first-order logic. Now we explain how inductive and 
coinductive definitions are realized. 
An inductively define predicate  is defined as the least fixed point of
a monotone predicate transformer , that is the formula
,
with free predicate variables  and , must be provable.
Then we have the closure axiom

as well as the induction schema

for every predicate  defined by some formula  
as .
Realizability for  is defined by defining  inductively by 
the operator 
.
This means we have the closure axiom 

as well as the induction schema:

Dually,  also gives rise to a coinductively defined predicate  
defined as the greatest fixed point of . Hence, we have the coclosure axiom

and the coinduction schema:

Realizability for  is defined by
defining  coinductively by the same operator  as above,
hence, the coclosure axiom 

and the coinduction schema:

The basis for program extraction from proofs is the Soundness Theorem.
\begin{sdthm}
 From a constructive proof of a formula 
from assumptions  one can extract a program 
 such that  is provable
from the assumptions .
\end{sdthm}
 
If one wants to apply this theorem to obtain a program realizing the
formula  one must provide terms  realizing the assumptions
. Then it follows that the term  realizes
.

That realizers do actually \emph{compute} witnesses is shown in 
Berger~\cite{Berger10} and Berger and Seisenberger~\cite{SeisenBerger10} by a 
\emph{Computational Adequacy Theorem} that relates the denotational 
definition of realizability with a lazy operational semantics. 

There is an important class of formulas where realizers do not matter:
We call a formula  \emph{non-computational} if
 
is provable.
Now, if we have proven  from assumptions , where 
 are non-computable, then we can extract a realizer 
of  that depends only on realizers of  and whose
correctness can be proven from the assumptions .
We can simplify the definition of realizability for formulas with 
non-computational parts: If  is non-computable, then:

Non-computational formulas can simplify program extraction drastically.
Therefore, it is important to have handy criteria for recognizing 
non-computable formulas.
First of all, realizability, and hence the question which formulas are 
non-computable, depends on how the predicates 
, defining realizability of atomic formulas , are
axiomatized. We call a predicate  non-computational if the 
axiom for  is:

Now, clearly,  is non-computable, if  is non-computable. Furthermore, 
 is non-computable and it is easy to see that the set of non-computable formulas is 
closed under implication, conjunction and universal and existential 
quantification.
Moreover, if  is a \emph{faithful} formula, that is,

then  is non-computable, provided  is non-computable. In particular, the negation 
of a faithful formula is non-computable.
Clearly, every non-computable formula is faithful and it is easy to see that
the set of faithful formulas is closed under conjunction, disjunction and 
existential quantification.

In our formalization and realizability interpretation of the theory of real
numbers and digit spaces, we regard the set  of real numbers as well as
the carrier set  of an arbitrary but fixed metric space  as sorts.
All arithmetic constants and functions we wish to talk about as well as the
metric  are admitted as constants or function symbols.  We
declare the predicates ,  and  on  as non-computational.
Furthermore, we admit all true non-computable statements about real numbers as well as
the axioms of a metric space (which are non-computable formulas) as axioms in our
formalization.

In order to be able to formalize a digit space  and the hyper-space
 we add to every sort  its powersort ,
equipped with a non-computational element-hood relation ,
as well as a
function space sort  to any two sorts  and ,
equipped with an application operation and operations such as composition
and the like. Furthermore, we add for every non-computable formula  the 
comprehension axiom

( may contain other free variables than ).
We will use the notation  for the element  of sort
 whose existence is postulated in the comprehension axiom above.
Hence, we can define the empty set ,
singletons  and the classical union of two sets
.

The comprehension axiom above is an example of an non-computable formula which we
wish to accept as true. In general, we may admit any non-computable formula
as axiom which is true in an intended model or provable in some
accepted theory (which may be classical).

Our first example of an inductive definition is the predicate  of
finite subsets of a predicate 

(Here  is a variable of sort  and  is a variable of sort ).
More formally,  is defined as the least fixed point of the operator:

Above, we may view  and  as a new constant and function
symbol, or else eliminate them with the usual technique (as in set theory).


The next example is a coinductive predicate that generalizes a corresponding
definition of a predicate  on the signed digit space 
introduced in Berger~\cite{Berger11} and Berger and Seisenberger~\cite{SeisenBerger10}.
\begin{definition}
\label{def-coco}
Let  be a digit space.
We define coinductively  as the largest subset of  such that 
for all :

\end{definition}


\begin{lemma}
\label{lem-coco-space}
.
\end{lemma} 
\begin{proof}
By definition, . For the proof of the converse inclusion 
it suffices to observe that because of Proposition~\ref{prop-valcont}(\ref{valcont1}) and Lemma~\ref{lem-valrec} the defining implication of  remains
correct if  is replaced by .
\end{proof}
Hence, classically, the set  is rather uninteresting, 
but, constructively, it is significant, since 
from a constructive proof that  one can extract a  
stream of signed digits  such that .
Furthermore, as shown by Berger~\cite{Berger11}, in the case of 
the signed digit , one can extract
from a constructive proof that  is closed under, say, 
multiplication, a program for multiplication with respect
to the signed digit representation.

In this paper we investigate whether what was done for the \emph{points} of
 can be done in a similar way for the \emph{nonempty compact subsets} of 
, (or, more generally, for the nonempty compact subsets of the 
underlying space  of a digit space ).
\begin{definition}
\label{def-coco-compact}
Given a digit space , we define, coinductively, the set 
 as the largest subset of  such that:

\end{definition}
\begin{lemma}
\label{lem-coco-compact}
.
\end{lemma}
\begin{proof}
By definition, .
The converse inclusion follows by coinduction, since
the implication in the above statement holds by Lemma~\ref{lem-tree-compact} and Corollary~\ref{cor-tree-valrec}.
\end{proof}

The proof of Lemma~\ref{lem-coco-compact} is classical because the set
 cannot be computed since, in general, one cannot decide whether 
.
The significance of the definition of  stems from the 
fact that the realizers of a statement  are 
exactly the digital trees representing , as we will show below. 
It follows from the definition of realizability in Berger~\cite{Berger11}
that the type  of realizers of a statement  
is defined by the recursive type equation
 
where  is the set of nonempty subsets of  and  is the 
cardinality of . Using 
constructive terminology one would call  the set of 
decidable inhabited subsets of ; since  is finite this set is 
finite as well and its cardinality exists constructively.
For example, if  has three elements , then:

One sees that  is, essentially, the set  of digital
trees. Indeed, every digital tree  can be identified with the pair
 where 
 and
. 
Note that  
(since  is a path in  exactly if  is a path in ,
and ).
What it means for a digital tree  to realize that 
, written ,
is defined coinductively as the largest subset  of
 such that if
, then:



\begin{theorem}
\label{thm-coco-compact-real}
The realizers of a statement  are exactly the 
digital trees representing , that is: 

In particular, from a constructive proof of  one 
can extract a digital tree representation of .
\end{theorem}
\begin{proof}
We first show by coinduction that if  then 
.
This means we have to show that the implication defining 
the relation  holds if that relation
is replaced by the relation .
Hence assume . 
For  set 
 which is compact and nonempty (since ).
Furthermore,  and
.
For the converse implication it suffices to show that any realizer
of a statement  has a value which is arbitrarily 
close to  in the Hausdorff metric. More precisely we show

by induction on , where  is a bound for  and  is a common 
contraction factor for the digits in . 
The case  is trivial. 
For the step we assume  and show 
.
By the assumption we have 
 such that
 and 
 for all .
By the induction hypothesis we have 
 for all .
Since all digits are contracting by the factor , it follows that
 for all 
. 
Since  and
, we conclude that
.
\end{proof}

\section{Extracting the Cantor Set}
\label{sec-cantor}
As an example, we prove that the Cantor set  
lies in  and extract a program
that computes a digital tree representation of it.
Of course, in this example, the digit space under discussion is the signed
digit space  which was introduced in Sect.~\ref{sec-digit-space}.
For convenience, we consider a scaled version of the Cantor set 
that fits better with
the signed digits. Therefore, we define the Cantor set 
as the fractal defined by the contractions: 

More precisely,  is the unique attractor of the iterated 
function system (IFS)  and can be defined explicitly
as

where .
Intuitively, this means that we start with the interval ,
remove the open middle third and repeat the process with the 
remaining parts\footnote{Traditionally, one would start with the interval 
and use the maps  and .}.
The only facts we will be using about the set  are that 
it is a nonempty subset of  and a fixed point of .

\begin{theorem}
\label{thm-cantor}
.
\end{theorem}
\begin{proof}
By a \emph{positive affine linear map (palm)} we mean a real function of 
the form  where  with . 
The set of all palms is a subgroup of the permutation group of 
, ie
palms are closed under composition and they are bijective with their 
inverses again being palms. 
The maps  and  as well as the signed digit maps 
 are examples of palms that map  into itself.
Note that for a palm  we have
 exactly if .
A \emph{rational palm} is a palm with rational coefficients . 
The rational palms form a subgroup of
all palms. The examples above are rational palms and in the following we will
be working exclusively with rational palms.

In order to show that  we show more generally,
by coinduction that the set

is a subset of . 
By the definition of  we have to show:

Since all elements of  are nonempty subsets of  
(since  is nonempty), this amounts to showing that
for every rational palm  with  we can find a set 
 and rational palms  () with 
 such that: 
 
We first consider the easy case that  for some
. In that case we can take  and 
, since 

and .


If we are not in that easy situation, we choose  such that
 and 
 (that such  do exist will be 
shown later). We set  and 

Using the fact that  is a fixed point of , ie

we obtain:

Furthermore, by the choice of , we have, as required,

and similarly, .


It remains to be shown that  above can always be found. 
First, note 

Assume  and let  and 
. Then . 
Since we are not in the easy case, we may assume . 
To determine , note that

Consider the case . Then we have  , since 
if  we would have  and therefore .
It follows that . This means that we can take
. 
In the case  we can take  since 
(if  we would have  and therefore ).
The calculation of  is symmetric: if , we can take , 
otherwise .
\end{proof}

\section{Equivalence with the Cauchy representation}
\label{sec-equiv}

Let  be a computable digit space.
We define the predicate  by:

A realizer of  is a fast Cauchy sequence in  converging to .

\begin{theorem}
\label{thm-c-sub-a}
.
\end{theorem}
\begin{proof}
Because of Lemma~\ref{lem-QD-Q} it suffices to show that

which will be done by induction on .
If , let  be any element in .
For , assume . Then there are 
 and  such that . By induction hypothesis,
there exists  such that . 
Set . Then, . 
\end{proof}
  

\begin{theorem}
\label{thm-a-sub-c}
Let  be a well-covering and decidable computable digit space with approximable choice.
Then .
\end{theorem}
\begin{proof}
We prove the statement by coinduction. Hence assume .
We have to find  and  such that  and .
Let  be a well-covering number.
Using ,  
pick  such that .
Pick  such that . 
Then . 

By Proposition~\ref{prop-choice-inv},  has a computable right inverse . 
Set . Since  has a computable modulus of continuity, we can, 
given , compute a number  so that for , 
if  then . 
Using the assumption  again, we find  such that
.
It follows that . By the computability 
of  we can moreover compute a basic element  with 
. 
Hence, , which shows that .
\end{proof}

Now we do for the hyper-space  what we did for  above.
We define the predicate  by:

A realizer of  is a fast Cauchy sequence of nonempty 
finite subsets of  converging to .

\begin{theorem}
\label{thm-ck-sub-ak}
.
\end{theorem}
\begin{proof}
We show 

by induction on .
If , we can take , for any .
For , assume . Then there are 
 and a family 
such that  and  for
all .
By induction hypothesis,
there exist  such that ,
for all . Set . 
Then it follows with Lemma~\ref{lem-hausm} that . 
\end{proof}


\begin{theorem}
\label{thm-ak-sub-ck}
Let  be a well-covering and decidable computable digit space with approximable choice. Then .
\end{theorem}
The proof will be based on a sequence of intermediate results.

We say that  has \emph{property (P)} if the 
following holds: For every  and every sequence 
 of basic sets
with  for all , one can compute from 
\begin{itemize}
\item[-] a decision procedure for a set  of digits in ,
\item[-] for every  a sequence of basic sets 
\end{itemize}
such that there exist  for each , with
 and  for all 
.

An equivalent way of stating the property (P) is to state constructively: 
if  holds, then there exists a decidable set 
 and  with  
for each . 

\begin{lemma}
\label{lem-p}
If  has property (P), then 
.

In terms of realizers this means: for every set  
and every sequence  of basic sets with  for all 
, one can compute from  a digital tree  such that .
\end{lemma}
\begin{proof}
Immediate, by coinduction.

Alternatively, one can directly define a decision procedure for a tree defining
a set  from a Cauchy-sequence of basic sets converging to .
To this end, assume that  for all  where the 
 are basic sets. We define a function 

where  is the set of basic sets and  is a new symbol
meaning intuitively ``not in the tree''. The definition of  is
by recursion on  and will be such that whenever 
, then  for all 
.
\begin{itemize}
\item[-] 
\item[-] If , then 
\item[-] If , then we use 
property (P), with  and , to compute 
and for every  a sequence of basic sets  such that 
there exist  for  with 
 and  for all .
\begin{itemize}
\item[-] If , then 
\item[-] If , then 
\end{itemize}
\end{itemize}
Finally we define .
Clearly,  is a digital tree that can be computed from the sequence
 (more precisely, a decision procedure for  can be
computed from ). We show that .

Let  be a path in , that is, 
 for all .
In order to show  it suffices to show
that  for all .
We show that , by induction on . 
.
, by induction hypothesis, and since,
by construction, .

Conversely, let . We define recursively ,
such that for all , , hence
 is of the form  and .
It follows then that  and .
For  there is nothing to define since 
 and , by assumption.
Now suppose  has been defined such that 
 and .
Let  such that . By the definition of ,
we have a set  such that  exactly if , 
as well as  for some 
 such that for , . 
Hence  for some . 
Set  and . It follows that 
 is of the form  and 
.
\end{proof}
\begin{lemma}
\label{lem-xink}
Let  be a compact metric space.
Let  and 
such that  for all .
\begin{itemize}
\item[(a)] For  the following are equivalent:
\begin{itemize}
\item[(i)] 
\item[(ii)] 
\item[(iii)] 
\end{itemize}
\item[(b)] Let  and 
such that  for all .
If for every  there exists  with ,
then .
\end{itemize}
\end{lemma}
\begin{proof}
(a) The implications from (i) to (ii) and from (ii) to (iii) are trivial.
Assume (iii) holds. For every  let  and 
such that  (using assumption (iii)) and also let
 such that  (using the hypothesis that
).
It follows .
Hence  is a limit of points in  and therefore in K, since  is closed.

(b) Let . From the assumptions it easily follows that condition (iii)
of part (a) holds. Therefore .
\end{proof}



\begin{lemma}
\label{lem-split}
Let  be a compact metric space with dense subset .  Let  be a
finite index set and  a nonempty closed subset of  for each . Assume that the  well-cover , ie there exists a rational 
 such that for each  there exists  with
.

Let  and  a family of basic sets such that 
 for all .

Then there exists  and for each  a set 
 and a family of basic sets 
such that:
\begin{itemize}
\item[-]  for all  and 
\item[-]  for all 
\item[-]  for all 
\item[-] 
\end{itemize}
Moreover, if  is a computable metric space such that for , 
 and  one can decide whether  
, then from  
and  one can compute the families  as well as
a decision procedure for .

\end{lemma}
\begin{proof}
Let  such that . 
Define, using decidabilty,

For  define  by recursion on :

\begin{claim}
\label{claim-split-cdn}
For all ,  and   there exists 
such that .
\end{claim}
Proof: Let .
Let  such that .
Let  such that .
Then , hence .

\begin{claim}
\label{claim-split-cdn-cauchy}
For all  and ,  is a basic set and 
.
\end{claim}
Proof: For the sets  to be basic, it suffices to show their nonemptyness, since,
by definition, they are finite subsets of . Nonemptyness follows by induction
on  using Claim~\ref{claim-split-cdn} and observing that  is nonempty,
by definition.
The inequality  follows from 
Claim~\ref{claim-split-cdn} and the definition of .

\begin{claim}
\label{claim-split-cdn-subxd}
, for all  and .
\end{claim}
Proof:
We show the stronger statement that 
, by induction on . (Here, .)
For the base case we have  
,
since .
For the step, let  and assume .
Let  with . Then 
. Hence ,
by induction hypothesis.
This ends the proof of Claim~\ref{claim-split-cdn-subxd}.

Since the space  is complete and, by 
Claim~\ref{claim-split-cdn-subxd}, for every ,  is a 
Cauchy sequence in , it has a limit, .
\begin{claim}
\label{claim-split-cdnk}
.
\end{claim}
Proof: From Claim~\ref{claim-split-cdn-cauchy} it follows that 
. But , since .

\begin{claim}
\label{claim-split-kdsubk}
.
\end{claim}
Proof: This follows from Lemma~\ref{lem-xink}~(b), since .


\begin{claim}
\label{claim-split-kdsubxd}
.
\end{claim}
Proof: Immediate, by Claim~\ref{claim-split-cdn-subxd} and Lemma~\ref{lem-xink}~(a).

\begin{claim}
\label{claim-split-union}
.
\end{claim}
Proof: By Claim~\ref{claim-split-kdsubk}, it suffices to show that
.
Let . Let  such that .
Then , since there exists  with 
, hence 
.
We show that .
By Lemma~\ref{lem-xink}~(a), it suffices to show that for all  there
exists  such that .
Since , let  such that 
.
We show that , by induction on :  and 
, therefore 
 (since 
). Hence .
For the induction step, we use the fact that , which
implies that , hence .
This ends the proof of Claim~\ref{claim-split-union} and the proof of the Lemma.
\end{proof}


\begin{proof}[Proof of \fullref{thm-ak-sub-ck}]
 By Lemma~\ref{lem-p}, it suffices
to show that  has property (P). 
Therefore, let  and  be a sequence of 
basic sets such that  for all . With
, for , clearly, the hypotheses of Lemma~\ref{lem-split}
are satisfied. Hence we obtain  and for each  a set 
 and a family of basic sets 
such that:
\begin{itemize}
\item[-]  for all  and 
\item[-]  for all 
\item[-]  for all 
\item[-] 
\end{itemize}
By Proposition~\ref{prop-choice-inv},  has a computable right inverse . Set . Then . Since  has a computable modulus of continuity, we can, given , compute a number  so that for , if  then . It follows that . By the computability of 
we can moreover, for any , compute a basic element  with .
Set . Then . It follows that . This completes the proof of Theorem~\ref{thm-ak-sub-ck}.
\end{proof}
 
 An important special case is if all digits have uniformly continuous inverses.
 
 \begin{definition}
 \label{def-unifinv}
 A digit space  is \emph{uniformly invertible} if for all  there exists (constructively)  such that for all  and , if , then .
 \end{definition}
 
Note that any continuous injection of a compact space into a 
Hausdorff space is a homeomorphism on its image. Hence any injective
digit of a digit space has a uniformly continuous inverse. So the
only extra condition of uniform invertibility beyond injectivity is that
the inverse digits have \emph{effective} moduli of uniform continuity.

\begin{lemma}
\label{lem-unifinv-uniffib}
Let  be a uniformly invertible constructively dense computable digit space. Then  has approximable choice. 
\end{lemma}
\begin{proof}
Because of Proposition~\ref{prop-choice-inv} it remains to show that for every  and  we can compute a basic element  such that . For given  set  and pick  according to uniform invertibility. Moreover, for given , use computable density to pick  with . Then .
 \end{proof}

\begin{corollary}
\label{cor-cauchy-coind}
Let   be a well-covering, decidable, uniformly invertible and constructively dense computable digit space. Then the following two statements hold:
\begin{enumerate}
\item\label{cor-cauchy-coind-1} 

\item\label{cor-cauchy-coind-2} 

\end{enumerate}
\end{corollary}



\section*{Acknowledgment}

The research leading to these results has received funding from the People Programme (Marie Curie Actions) of the European Union's Seventh Framework Programme FP7/2007-2013/ under REA grant agreement no.\ PIRSES-GA-2011-294962-COMPUTAL.

Research on this paper started on one of the many visits of the second author in Swansea. Main results were obtained while the first author was visiting the Department of Decision Sciences of the University of South Africa as part of the COMPUTAL project  and the second author was working in Pretoria as a Visiting Research Professor. 

Both authors are grateful to the Department of Decision Science in Pretoria for its hospitality and for having created such a splendid working atmosphere. The second author in addition also thanks the Computer Science Department of Swansea University for its great hospitality.

Thanks are also due to the anonymous referee for his careful reading of an earlier version of the manuscript.


 


\begin{thebibliography}{MRE07}

\bibitem{Berger10}
\textbf{U~Berger,}
\newblock \emph{Realisability for induction and coinduction with applications to
  constructive analysis,}
\newblock Journal of Universal Computer Science 16(18) (2010) 2535--2555, DOI 10.3217/jucs-016-18-2535

\bibitem{Berger11}
\textbf{U~Berger,}
\newblock \emph{From coinductive proofs to exact real arithmetic: theory and
  applications,}
\newblock Logical Methods in Computer Science 7(1) (2011) 1--24, DOI 10.2168/LMCS-7(1:8)2011

\bibitem{SeisenBerger10}
\textbf{U~Berger, M~Seisenberger,}
\newblock \emph{Proofs, programs, processes,}
\newblock in \textbf{F~Ferreira, B~L\"owe, E~Mayordomo, L\,M. Gomes, editors,}
  {\em Programs, Proofs, Processes, 6th Conference on Computability in Europe,
  CiE 2010, Ponta Delgada, Azores, Portugal, June 30 - July 4, 2010.
  Proceedings}, volume 6158 of \emph{Lecture~Notes~in~Computer~Science,} Springer-Verlag, Berlin (2010), pages
  39--48

\bibitem{BergerHou08}
\textbf{U~Berger, T~Hou,}
\newblock \emph{Coinduction for exact real number computation},
\newblock Theory of Computing Systems 43 (2008) 394--409, DOI 10.1007.s00224-007-9017-6
  
\bibitem{Brattka99}
\textbf{V~Brattka,}
\newblock \emph{Recursive and computable operations over topological structures,} Informatik-Berichte 255, FernUniversit\"at Hagen, Fachbereich Informatik, Hagen (1999)

\bibitem{BrattkaPresser03}
\textbf{V~Brattka, G~Presser,}
\newblock \emph{Computability on subsets of metric spaces,}
\newblock Theoretical Computer Science 305 (2003) 43--76

\bibitem{CiaffaglioneGianantonio06}
\textbf{A~Ciaffaglione, P~Di~Gianantonio,}
\newblock \emph{A certified, corecursive implementation of exact real numbers,}
\newblock Theoretical Computer Science 351 (2006) 39--51, DOI 10.1016/j.tcs.2005.09.061

\bibitem{Edalat97}
\textbf{A~Edalat,}
\newblock \emph{Power domains and iterated function systems,}
\newblock Information and Computation 3(4) (1997) 401--452

\bibitem{EdalatHeckmann02}
\textbf{A~Edalat, R~Heckmann,}
\newblock \emph{Computing with real numbers: I. {T}he {LFT} approach to real number
  computation; {II.} {A} domain framework for computational geometry,}
\newblock in \textbf{G~Barthe, P~Dybjer, L~Pinto, J~Saraiva, editors,} \emph{
  Applied Semantics -- Lecture Notes from the International Summer School,
  Caminha, Portugal}, Springer-Verlag, Berlin (2002), pages 193--267
  
\bibitem{EdalatPotts97}
\textbf{A~Edalat, P\,J~Potts,}
\newblock \emph{A new representation for exact real numbers,} Electronic Notes in Theoretical Computer Science 6 (1997) 119--132

\bibitem{EdalatSuenderhauf98}
\textbf{A~Edalat, P~S{\"u}nderhauf,}
\newblock \emph{A domain-theoretic approach to real number computation,}
\newblock Theoretical Computer Science 210 (1998) 73--98


\bibitem{EscardoMarcial-Romero07}
\textbf{J\,R Marcial-Romero, M~H\"otzel Escardo,}
\newblock \emph{Semantics of a sequential language for exact real-number computation,}
\newblock Theoretical Computer Science 379(1-2) (2007) 120--141, DOI 10.1016/j.tcs.2007.01.021

\bibitem{SchwichtenbergWainer12}
\textbf{H~Schwichtenberg, S\,S~Wainer.,}
\newblock {\em Proofs and computations,} Cambridge University Press, Cambridge (2012)

\bibitem{Tsuiki02}
\textbf{H~Tsuiki,}
\newblock \emph{Real number computation through Gray code embedding,}
\newblock Theoretical Computer Science 284(2) (2002) 467--485

\bibitem{Weihrauch00}
\textbf{K~Weihrauch,}
\newblock \emph{Computable analysis,} Springer-Verlag, Berlin (2000)

\end{thebibliography}



\end{document}
