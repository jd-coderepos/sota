\documentclass[copyright,creativecommons]{eptcs}


\providecommand{\event}{``Post-Proceedings of the 5th International Workshop on Logical Frameworks and Meta-languages: Theory and Practice, 2010''}

\usepackage{breakurl}             





\usepackage{latexsym}
\usepackage{amsmath}
\usepackage{amssymb}
\usepackage{amsfonts}
\usepackage{proof}
\usepackage{code}
\usepackage{cdsty}
\usepackage{url}
\usepackage{ifthen}
\usepackage{stmaryrd}
\usepackage{beluga}



\ifdefined\LONGVERSION
  \relax
\else
\newcommand{\LONGVERSION}[1]{}
\newcommand{\SHORTVERSION}[1]{#1}
\fi
\newcommand{\LONGSHORT}[2]{\LONGVERSION{#1}\SHORTVERSION{#2}}

\newtheorem{corollary}[theorem]{Corollary}



\def\squareforqed{\ensuremath{\Box}}
\def\qed{\ifmmode\squareforqed\else{\unskip\nobreak\hfil
\penalty50\hskip1em\null\nobreak\hfil\squareforqed
\parfillskip=0pt\finalhyphendemerits=0\endgraf}\fi}

\newenvironment{proof}[1][]{\noindent\ifthenelse{\equal{#1}{}}{{\it
      Proof.}}{{\it Proof #1.}}\hspace{2ex}}{\qed\bigskip}
\newenvironment{proof*}[1][]{\noindent\ifthenelse{\equal{#1}{}}{{\it
      Proof.}}{{\it Proof #1.}}\hspace{2ex}}{\bigskip}

\usepackage{srcltx}
\usepackage{listings}
\lstloadlanguages{ContextualML}
\lstset{language=ContextualML}  

\newdimen\zzlistingsize
\newdimen\zzlistingsizedefault
\zzlistingsizedefault=8pt
\zzlistingsize=\zzlistingsizedefault
\global\def\InsideComment{0}
\newcommand{\Lstbasicstyle}{\fontsize{\zzlistingsize}{1.05\zzlistingsize}\ttfamily}
\newcommand{\keywordFmt}{\fontsize{0.9\zzlistingsize}{1.0\zzlistingsize}\bf}
\newcommand{\smartkeywordFmt}{\if0\InsideComment\keywordFmt\fi}
\newcommand{\commentFmt}{\def\InsideComment{1}\fontsize{0.95\zzlistingsize}{1.0\zzlistingsize}\rmfamily\slshape}

\newcommand{\LST}{\setlistingsize{\zzlistingsizedefault}}

\newlength{\zzlstwidth}
\newcommand{\setlistingsize}[1]{\zzlistingsize=#1\settowidth{\zzlstwidth}{{\Lstbasicstyle~}}}
\setlistingsize{\zzlistingsizedefault}
\renewcommand{\bar}{\overline}



\newcommand{\bla}{\ensuremath{\mbox{
\begin{array}{l}
\infer{\Delta ; \Gamma \vdash M\;N : [N/x]B}{
\Delta ; \Gamma \vdash M : \Pi x{:}A.B  &
\Delta ; \Gamma \vdash N : A}
\quad \quad  
\infer{\Delta ; \Gamma \vdash M\;(\Psihat.N) : \msub{\Psihat.N/X}B}{
\Delta ; \Gamma \vdash M : \Pibox X{:}A[\Psi].B  &
\Delta ; \Psi \vdash N : A}
\end{array}
 
\begin{array}{@{}llcl@{}r@{}}
\mbox{Sorts} & s & \bnfas & \lfkind \mid \lftype \\
\mbox{Expressions} & E,F & \bnfas & \multicolumn 2 {l@{}} {
s \mid a \mid \fun E F
 \mid x_n \bnfalt X_n \bnfalt \lam{x}{E}
\bnfalt \app{F}{E} \bnfalt \clo{E}{\sigma} \bnfalt \cclo{E}{\theta} 
} \-0.75em]
\multicolumn{4}{@{}l@{}}{\mbox{Special cases of expressions:}} \0.5em]
\hline \0.5em]
\mbox{Meta-substitutions} 
& \theta & \bnfas & \Shift^n \bnfalt
\theta,M \bnfalt \cclo{\theta'}{\theta}
& (n \geq 0)
\
Constants are denoted by letter , their types/kinds are recorded in
a global well-formed signature .
We have two different de Bruijn indices  and  (), 
one for numbering bound variables and one for numbering meta-variables.  
 represents the
de Bruijn number  and stands for an ordinary bound variable, while
 represents the de Bruijn number  but stands for a
meta-variable. Due to the two kinds of substitutions, we also have two
kinds of closures; the closure of an expression with an ordinary substitution
 and the closure of an expression with a meta-substitution
. Following the treatment of meta-variables in
\cite{Nanevski:ICML05}, we describe the type of a meta-variable as
 which stands for a meta-variable of type  which
may refer to variables in . 

Meta-substitutions provide a term  for a meta-variable  of type
. Note that  does not denote a closed term, but a
term of type  in the context  and hence may refer to
variables from . In previous presentations where we use names
for variables, we hence wrote  to be able to rename the
variables in  appropriately. Because bound variables
are represented using de Bruijn indices in this paper, we simply write
 but keep in mind that  is not necessarily closed.

Our calculus also features closures on the level of substitutions and
meta-substitutions. For example, we allow the closure 
which will allow us to lazily treat ordinary substitution
composition and the closure  which will postpone
applying  to the ordinary substitution . Similarly,
the closure  for meta-substitutions allows us to
lazily compose meta-substitutions. We note the absence of a closure
. Applying an ordinary substitution  to a
meta-substitution  simply reduces to , since all
objects in the meta-substitution are closed objects and cannot be 
affected by . It is hence not meaningful to include a closure
. 
We also do not introduce a closure of a context  and a
meta-substitution . Instead we define  eagerly
by simply pushing the meta-substitution  to each
declaration as follows:  and .  The length of a context  is
denoted by  and likewise  for meta-contexts.


\begin{figure}[htbp]
\centering
1.25em]
\infer{\Delta; \Gamma, A \vdash x_1 : \esub {\shiftby 1} A}
      {\Delta; \Gamma \der A : \lftype}
\qquad
\infer{\Delta; \Gamma, B \vdash x_{n+1} : [\shift^1]A}
      {\Delta; \Gamma \vdash x_n : A & \Delta; \Gamma \der B : \lftype} 
\0.75em]
\infer{\Delta; \Gamma \vdash \lam{x}{M} : \Pi A. B}
      {\Delta; \Gamma, A \vdash  M : B \qquad
       \Delta; \Gamma, A \vdash B : \lftype}
\qquad
\infer{\Delta; \Gamma \vdash \app{E}{N} : [\shift^0, N]F}
       {\Delta; \Gamma \vdash E : \Pi A.F
        &
        \Delta; \Gamma \vdash  N : A}
\0.75em]
\infer{\Delta; \cclo{\Gamma}{\theta} \vdash \cclo{E}{\theta} : \lfkind}
{\Delta \vdash \theta : \Delta' & 
 \Delta' ; \Gamma \vdash E : \lfkind}
\qquad 
\infer{\Delta; \cclo{\Gamma}{\theta} \vdash \cclo{E}{\theta} : \cclo{F}{\theta}}
{\Delta \vdash \theta : \Delta' & 
 \Delta' ; \Gamma \vdash E : F}
\qquad
\infer{\Delta; \Gamma \vdash E : F_2}{
\Delta; \Gamma \vdash E : F_1 & \Delta; \Gamma \vdash F_1 \equiv F_2 : s
}
\0.5em]
\multicolumn{1}{l}{\mbox{Ordinary substitutions}} \0.75em]
\infer{\Delta; \Gamma \vdash \clo{\sigma}{\tau} : \Psi}
  {\Delta; \Gamma \vdash \tau : \Psi' & 
   \Delta; \Psi' \vdash \sigma : \Psi }
\qquad
\infer{\Delta; \cclo{\Gamma}{\theta} \vdash \cclo{\sigma}{\theta} :  \cclo{\Psi}{\theta}}
{\Delta \vdash \theta : \Delta' & 
  \Delta' ; \Gamma \vdash \sigma : \Psi  
}
\\\multicolumn{1}{l}{\mbox{Meta-substitutions}} \0.75em]
\qquad
\infer{\Delta \vdash \cclo{\theta'}{\theta} : \Delta'}{
\Delta   \vdash \theta  : \Delta_0 &
\Delta_0 \vdash \theta' : \Delta' 
}
\end{array}

\begin{array}{lclp{10cm}}
                & \vdash & \Delta \mctx & Meta-context  is well-typed \\
\Delta          & \vdash & \Psi \ctx    & Context  is well-typed
\\
\Delta; \Gamma & \vdash & E : F      & Expression  has ``type'' \\
\Delta; \Gamma & \vdash & \sigma : \Psi & Substitution  has
domain  and range \\
\Delta  & \vdash & \theta : \Delta' & Contextual Substitution  has
domain  and range 
\end{array}

\begin{array}[h]{l@{~}c@{~}l@{~}c@{~}l@{~}c@{~}lp{8cm}}
\Delta; \Gamma & \vdash & E_1 & \equiv & E_2 & : & F & Expressions 
and  are equal at ``type'' \\
 \Delta; \Gamma & \vdash & \sigma_1 & \equiv & \sigma_2 & : & \Psi &
 Substitutions  and  are equal at
domain  \\
\Delta  & \vdash & \theta_1 & \equiv & \theta_2 & : & \Delta' & 
Meta-substitutions  and  are equal at domain  
\end{array}

 \infer{\Delta; \Gamma \vd (\sigma, M) \equiv (\sigma', M')
   \hastype \Psi, A}{\Delta; \Gamma \vd M \equiv M' \hastype [\sigma]A
   & \Delta ; \Gamma \vd A \hastype \lftype 
   & \Delta; \Gamma \vd \sigma \equiv \sigma' \hastype \Psi}
\qquad
\infer{\Delta;\,\Gamma \vd E \equiv E' \hastype F'}{
  \Delta; \Gamma \vd E \equiv E' \hastype F & 
  \Delta; \Gamma \vd F \equiv F' \hastype s}

 \begin{array}[h]{c}
\abox{-Reduction}
\1.5em]
\abox{Substitution Propagation: Identity and Composition}
\1.5em]
\abox{Substitution Propagation: Constants}
\1.5em]
\abox{Substitution Propagation: Variable Lookup}
\1.5em]
\abox{Substitution Propagation: Pushing into Expression Constructions}
\1.5em]
\ru{\Delta; \Gamma \der \sigma : \Psi \qquad
    \Delta; \Psi,A \der M : B \qquad
    \Delta; \Psi,A \der B : \lftype 
  }{\Delta; \Gamma \der \esubp {\sigma} {\lambda M} \equiv
      \lambda {\esub {\esub {\shiftby 1} \sigma, x_1} M} :
      \fun {\esub \sigma A} {\esub {\esub {\shiftby 1} \sigma, x_1} B}
  } 
\1.5em]
\abox{Substitution Reductions: Pairing and Shifting\hspace{12cm}}
\1.5em]
\ru{\Delta \der \Gamma,\Gamma_1,\Gamma_2 \cxt \quad
    \length{\Gamma_1} = m \quad 
    \length{\Gamma_2} = n
  }{\Delta; \Gamma,\Gamma_1,\Gamma_2 \der \esub {\shiftby n}
    {\shiftby m} \equiv \shiftby{n+m} : \Gamma 
  }
\0.75em]
\ru{\Delta; \Gamma \der \sigma : \Psi
  }{\Delta; \Gamma \der \esub {\shiftby 0} \sigma \equiv \sigma : \Psi}
\qquad
\ru{\Delta; \Gamma \der \sigma : \Psi
  }{\Delta; \Gamma \der \esub \sigma {\shiftby 0}  \equiv \sigma : \Psi}
\
  
  \caption{Computational Laws I:  and substitutions}
  \label{fig:cl1}
\end{figure}



\begin{figure}
  \centering
0.75em]
\ru{\Delta; \Gamma \der E : F
  }{\Delta; \Gamma \der \msub {\Shiftby 0} E \equiv E : F}
\qquad
\ru{\Delta \der \theta : \Delta' \qquad
    \Delta' \der {\theta'} : \Delta'' \qquad
    \Delta''; \Gamma \der E : F
  }{\Delta; \msub {\msub \theta {\theta'}} \Gamma \der 
      \msub \theta {\msub {\theta'} E} \equiv \msub {\msub
        \theta {\theta'}} E 
      : \msub {\msub \theta {\theta'}} F
  } 
\0.75em]
\ru{\Delta \der \theta : \Delta' \quad 
    \Delta' \der \Gamma \cxt
  }{\Delta; \msub \theta \Gamma \der 
      \msub \theta \lftype \equiv \lftype : \lfkind} 
\quad
\ru{\Delta \der \theta : \Delta' \quad
    \Delta'; \Gamma \der a : K
  }{\Delta; \msub \theta \Gamma \der \msub \theta a \equiv a : \msub \theta K} 
\quad
\ru{\Delta \der \theta : \Delta' \quad
    \Delta'; \Gamma \der x_n : A 
  }{\Delta; \msub \theta \Gamma \der \msub \theta {x_n} \equiv x_n : \msub \theta A} 
\0.75em]
\ru{\Delta \der \theta : \Delta' \qquad
    \Delta'; \Gamma \der A : \lftype \qquad
    \Delta; \msub \theta \Gamma \der M : \msub \theta A 
  }{\Delta; \msub{\theta} \Gamma \der \msub {\theta, M} X_1 \equiv M : \msub \theta A}
\qquad
\ru{\Delta; \Gamma \der X_{n+1} : A
  }{\Delta; \Gamma \der X_{n+1} \equiv \msub{\Shiftby 1} X_n : A}
\0.75em]
\ru{\Delta \der \theta : \Delta' \quad
    \Delta'; \Gamma,A \der F : s
  }{\Delta; \msub \theta \Gamma \der \msubp \theta {\fun A F} \equiv
      \fun {\msub \theta A} {\msub \theta F}
      : s
  } 
\quad
\ru{\Delta \der \theta : \Delta' \quad
    \Delta'; \Gamma,A \der M : B \quad
    \Delta'; \Gamma,A \der B : \lftype 
  }{\Delta; \msub \theta \Gamma \der \msubp {\theta} {\lambda M} \equiv
      \lambda {\msub \theta M} :
      \fun {\msub \theta A} {\msub \theta B}
  } 
\
0.75em]
\ru{\Delta \der \theta : \Delta' \qquad
    \Delta' \der \Gamma,\Gamma' \cxt \qquad
    \length{\Gamma'} = n
  }{\Delta; \msub \theta \Gamma, \msub \theta {\Gamma'} \der
      \msub \theta {\shiftby n} \equiv \shiftby n : \msub \theta
      \Gamma
  }
\1.5em]  
\ru{\Delta \der \theta : \Delta' \quad
    \Delta'; \Gamma \der \tau : \Psi' \quad
    \Delta'; \Psi'  \der \sigma : \Psi
  }{\Delta; \msub \theta \Gamma \der \msub \theta {\esub \tau \sigma}
      \equiv \esub {\msub \theta \tau} {\msub \theta \sigma}
      : \msub \theta \Psi
  }
\quad
\ru{\Delta \der \theta : \Delta' \quad
    \Delta' \der \theta' : \Delta'' \quad
    \Delta''; \Gamma \der \sigma : \Psi
  }{\Delta;\msub{\msub{\theta}{\theta'}} \Gamma \der 
      \msub \theta {\msub{\theta'}\sigma} 
      \equiv \msub{\msub{\theta}{\theta'}} \sigma
      : \msub{\msub{\theta}{\theta'}} \Psi
  }
\end{array}

\begin{array}[h]{c}
\abox{Meta-Substitution Reductions: Pairing and Shifting}
\1.5em]
\ru{ \der \Delta,\Delta_1,\Delta_2 \mcxt \quad
    \length{\Delta_1} = m \quad 
    \length{\Delta_2} = n
  }{\Delta,\Delta_1,\Delta_2 \der \msub {\Shiftby n}
    {\Shiftby m} \equiv \Shiftby{n+m} : \Delta 
  }
\0.75em]
\ru{\Delta \der \theta : \Delta_0
  }{\Delta \der \msub {\Shiftby 0} \theta \equiv \theta : \Delta_0}
\quad
\ru{\Delta \der \theta : \Delta_0
  }{\Delta \der \msub \theta {\Shiftby 0}  \equiv \theta : \Delta_0}
\quad
\ru{\Delta_1 \der \theta_1 : \Delta_2 \quad
     \Delta_2 \der \theta_2 : \Delta_3 \quad
    \Delta_3 \der \theta_3 : \Delta_4 
  }{\Delta_1 \der \msub {\theta_1} \msub {\theta_2}
    {\theta_3} \equiv \msub {\msub {\theta_1}{\theta_2}} {\theta_3} : \Delta_4
   }
\end{array}

\begin{array}{c}
\multicolumn{1}{l}{\mbox{Step 1:\hspace{14cm}}}\0.75em] 
\multicolumn{1}{l}{\mbox{Step 2:}}\0.85em]
\multicolumn{1}{l}{\mbox{Step 3:}}\

where  
\vspace{-0.65cm}
-2.75em]
\cal{D} =    
  \end{array}
 &
\infer[\text{Transitivity}]{\Delta ; \Gamma \vdash [\sigma,M]\shift^1 \equiv \sigma :\Psi}{
  \infer[\text{Pairing}]{\Delta ; \Gamma \vdash [\sigma,M]\shift^1 = [\sigma]\shift^0 : \Psi}{
          \Delta ; \Gamma \vdash \sigma , M : \Psi,B} 
 & 
  \infer[\text{Category Laws}]{\Delta ; \Gamma \vdash [\sigma]\shift^0 \equiv \sigma : \Psi}{
         \Delta ; \Gamma \vdash \sigma : \Psi}
}  
\end{array}

\begin{array}{c}
\ru{\Delta; \Gamma \der M \hastype \fun A B 
  }{\Delta; \Gamma \der M \equiv 
      \lambda (\app {(\esub {\shiftby 1} M)} {x_1}) \hastype \fun A B
  }
\



\LONGVERSION{
\subsubsection{Compatibility and Equivalence Rules}
0.75em]
\infer{\Delta; \Gamma,\Gamma' \vd \shiftby n \equiv \shiftby n \hastype \Gamma}
{ \Delta \der \Gamma,\Gamma' \cxt & \length{\Gamma'} = n} 
\hspace{2em}
\infer{\Delta; \Gamma \vd (\sigma, M) \equiv (\sigma', M')
  \hastype \Psi, A}{\Delta; \Gamma \vd M \equiv M' \hastype [\sigma]A
  & \Delta; \Gamma \vd \sigma \equiv \sigma' \hastype \Psi}
\\
\mbox{etc. + equivalences}
\0.75em]
\infer{\Delta; \Gamma \vd \lam{}{M} \equiv \lam{}{N} \hastype\Pi A. B}{
 \Delta; \Gamma, A \vd M \equiv N \hastype B}
\0.75em]
\infer{\Delta; \Gamma \vd  \clo{M_1}{\sigma_1} \equiv
       \clo{M_2}{\sigma_2} \hastype \esub {\sigma_1} A
     }{\Delta; \Gamma \der \sigma_1 \equiv \sigma_2 : \Psi &
       \Delta; \Psi   \der M_1 \equiv M_2 : A
     }
\0.75em]
\multicolumn{1}{l}{\mbox{Equivalence}}\0.75em]
\multicolumn{1}{l}{\mbox{Type conversion}}\0.75em]
\abox{Family congruence}\1.5em]

\infer{\Delta; \Gamma \vd \Pi A_1. A_2 \equiv \Pi B_1. B_2 \hastype \lftype}{
       \Delta; \Gamma \vd A_1 \equiv B_1 \hastype \lftype & 
       \{\Delta; \Gamma \vd A_1 \hastype \lftype\} & 
       \Delta; \Gamma, A_1 \vd A_2 \equiv B_2 \hastype \lftype}
\0.75em]
\infer{\Delta; \Gamma \vd B \equiv A \hastype K}{
  \Delta; \Gamma \vd A \equiv B \hastype K} 
\hspace{2em}
\infer{\Delta; \Gamma \vd A \equiv C \hastype K}{
  \Delta; \Gamma \vd A \equiv B \hastype K & \Delta; \Gamma \vd B \equiv C \hastype K}
\0.75em]
\infer{\Delta; \Gamma \vd A \equiv B \hastype L}{
  \Delta; \Gamma \vd A \equiv B \hastype K & 
  \Delta; \Gamma \vd K \equiv L \hastype \lfkind}
\0.75em]
\infer{\Delta; \Gamma \vd \lftype \equiv \lftype \hastype \lfkind}{}
\0.75em]
\abox{Kind equivalence}\ 
} 



\subsection{Properties}
Next, we prove some standard properties about the presented type
assignment system. First, we show that contexts are indeed
well-formed. 
\begin{lemma}[Context well-formedness] \label{lem:cwf} \bla
\begin{enumerate}
  \item If  or  
        then .
  \item If  or  then .
\item If  then .
  \item If  or  then .
  \end{enumerate} 
The height of the output derivation is bounded by the height of the
input derivation, in all cases.
\end{lemma}
\begin{proof}
  By simultaneous induction over all judgments.
\end{proof}

The following inversion theorem for typing is standard for PTSs and
are necessary due to the type conversion rule which makes inversion a
non-obvious property.  It allows us to classify expressions into terms,
types, kinds, and the sort .  
We write  if there exists a sort 
such that .
\begin{theorem}[Inversion of typing] \label{thm:inv} \bla
\begin{enumerate}
\item There is no derivation of .
\item If  then .      
\item If  then .
\item If  then 
    and  and 
   either  or . \item If  then  with  and
  . \item If  then  with  and
   and
  .\item If  then there are  such
  that  and
   and \\ 
  and
  .
\item If  then there are  such
  that  and 
   \\and
  . 
\item If  then there are 
  such that  and  \\ and .
\item If  then there are
  
  such that  and 
   and 
   \\and 
  . 
\end{enumerate}
\end{theorem}
\begin{proof}
  By induction on the typing derivation, peeling off the type
  conversion steps and combining them with transitivity.
\end{proof}
Expression  is a \emph{kind} if  for some
, it is a \emph{type family} if  for
some kind  and some , and it is a \emph{term} if
 for some  with
.

The following inversion statement for meta-variables under a
substitution is crucial for the correctness of algorithmic equality
(Sec.~\ref{sec:aleq}) and bidirectional type checking (Sec.~\ref{sec:bidir}).
\begin{corollary} \label{cor:invmeta}If  then 
 with 
 and 
 and 
.  
\end{corollary}

\begin{theorem}[Syntactic Validity]\label{thm:synval} \bla
\begin{enumerate}
\item If  or  then 
  for some sort.
\item If  then 
  and . 
\item If  then 
  and . 
\item If  then 
  and . 
\end{enumerate}
\end{theorem}
\begin{proof}
  By simultaneous induction over all judgments.
\end{proof}





\section{Evaluation and Algorithmic Equality}
\label{sec:algo}

In this section, we define a weak head normalization strategy together
with algorithmic equality. The goal is to treat ordinary substitutions
and meta-substitutions lazily; in particular, we aim to postpone
shifting of substitutions until necessary.  
For the treatment of LF, an untyped algorithmic equality is
sufficient.  The design of the algorithm follows Coquand 
\cite{coquand:conversion} with refinements from joint work with the
first author \cite{abelCoquand:lfsigma}.  In this article, we only
show soundness of the algorithm; completeness can be proven using
techniques of the cited works.  However, an adaptation to de Bruijn
style and explicit substitutions is necessary; 
we leave the details to future work\LONGVERSION{, a sketch can be found in the appendix}.



We first characterize
our normal forms by defining normal and neutral expressions where
expressions include terms, types, and kinds. 
Normal forms are exactly the expressions we can type-check with a
bidirectional algorithm (see Section~\ref{sec:bidir}).  Note that type
checking normal forms is sufficient in practice, since the input to
the type checker, written by a user, is almost always in
-normal form (or can be turned into normal form by introducing
typed let-definitions).

Normal substitutions are
built out of normal expressions. However, it is worth keeping in mind
that our typing rules will ensure that they only contain terms and
not types, since we do not support type-level variables. Our normal
forms are only -normal, not necessarily -long. Only
meta-variables are associated with an ordinary normal substitution,
all other closures have been eliminated.   

0.75em]
\mbox{Normal substitutions} &
  \nu & ::= & \shiftby n \mid (\nu, V)
\end{array}

\begin{array}{llrl}
\mbox{Weak head normal forms} &
  W & ::= & \lftype
    \mid \dsub \rho \eta {\fun A B} 
    \mid \dsub \rho \eta {\lambda M}
    \mid H \\
\mbox{Neutral weak head normal forms} &
  H & ::= & a \mid x_n \mid \esub \rho {X_n} \mid \app H L 
\\
\mbox{Closures} &
  L & ::= & x_n \mid \dsub \rho \eta E 
\


Our weak head normal forms and closures combine substitutions and
meta-substitutions
and our whnf-reduction
strategy simultaneously treats substitutions and meta-substitutions. 
Instead of coupling expressions with two suspended substitutions, we
could have introduced a joint simultaneous substitutions and closures
built with them. The path taken in this paper builds on the individual
substitution operations instead of defining a new joint substitution
operation. To clarify the nature and the interplay of ordinary
substitutions and meta-substitutions it is helpful to
consider the typing rule of closures :





Intuitively, this means to obtain an expression  which makes actually
sense in  and , we first compute  and
subsequently apply the ordinary substitution  to
arrive at . 


\paradot{Shift propagation}\label{abbrev-sclo}
While we treat shifts in the environment as an explicit
operation---to avoid a traversal when lifting an environment under a
binder---, shifting a closure or a neutral weak head normal form can
be implemented inexpensively.
Let shifting  of a closure  be defined by
 and .  
It is extended to shifting of neutral weak head normal forms  by 
 and
 and .





\subsection{Weak head evaluation}

Our weak head evaluation strategy will postpone
propagation of substitutions into an expression until necessary. 
Treating substitutions lazily seems to be beneficial as also
supported by the experimental analysis on lazy vs eager reduction
strategies for substitutions by Nadathur and his collaborators
\cite{liangNadathurQi:jar05}. We present the algorithm for weak head
normalization in Figure \ref{fig:sub2env}.
We define a function  where  is either a variable  or
a proper closure . The function  is then
defined recursively on .  

\begin{figure}
  \centering
0.5em]
  \wmsub {\Shiftby m}{\Shiftby n}    & = & \Shiftby{m+n} \\
  \wmsub {(\eta, M)}{\Shiftby {n+1}} & = & \wmsub \eta {\Shiftby n} \\
  \wmsub {\eta}{(\theta, M)}         & = & (\wmsub \eta \theta, \msub
    \eta M) \\
  \wmsub {\eta}{\msub \theta {\theta'}} & = & \wmsub {(\wmsub \eta
    \theta)} {\theta'}  
\0.5em]
  \wsub {(\shiftEnv k \rho)} \eta \sigma & = & 
    \shiftEnvp k {\wsub \rho \eta \sigma} \\
  \wsub \rho \eta {\shiftby 0} & = & \rho \\  
  \wsub {\shiftby k} \eta {\shiftby n}   & = & \shiftby {k + n}  \\
  \wsub {(\rho, L)} \eta {\shiftby{n+1}} & = & \wsub \rho \eta {\shiftby n} \\
  \wsubp \rho \eta {\sigma, M} & = & (\wsub \rho \eta \sigma,
    \dsub \rho \eta M) \\   
  \wsubp \rho \eta {\esub \sigma \tau} & = & \wsub {(\wsub \rho
    \eta \sigma)} \eta \tau \\  
  \wsubp \rho \eta {\msub \theta \sigma} & = & \wsub \rho {(\wmsub
    \eta \theta)} \sigma 
\0.5em]
  \wmlookup  {\Shiftby n} m & = & X_{n+m} \\
  \wmlookupp {\eta, E}    1 & = & E \\
  \wmlookupp {\eta, E}{m+1} & = & \wmlookup \eta m
\0.5em]
  \wlookup  {\shiftby n} m & = & x_{n+m} \\
  \wlookupp {\rho, L}  1   & = & L \\
  \wlookupp {\rho, L}             {m+1} & = & \wlookup \rho m \\
  \wlookupp {\esub {\shiftby n} \rho} m & = & \shiftClosp n {\wlookup \rho m} 
\0.5em]
  \twhnf~x_m         & = & x_m \\
  \whnf  \rho \eta s & = & s \\
  \whnf  \rho \eta a & = & a\\
  \whnf  \rho \eta {x_m} & = & \twhnfp {\wlookup \rho m} \\  
  \whnf  \rho \mId {X_m} & = & \esub \rho {X_m} \\
  \whnf  \rho \eta {X_m} & = & \whnf \rho \mId (\wmlookup \eta m) \\
  \whnfp \rho \eta {\fun A E} & = & \dsubp \rho \eta {\fun A E} \\
  \whnfp \rho \eta {\lambda M} & = & \dsubp \rho \eta {\lambda M} \\
  \whnfp \rho \eta {\app M N}  & = & 
    \wapp{(\whnf \rho \eta M)}{\dsub \rho \eta N}\\
  \whnf \rho \eta {\esub \sigma M} & = & 
    \whnf {\wsub \rho \eta \sigma} \eta M\\
  \whnf \rho \eta {\msub \theta M} & = & \whnf \rho {\wmsub \eta \theta} M
\0.5em]
  \wapp {\dsubp \rho \eta {\lambda M}} L & = & \whnf {\rho, L} \eta
  M \\
  \wapp H L & = & \app H L 
\end{array}

\begin{array}{ccl}
W \alw W'  & & \mbox{weak head normal forms  are
    algorithmically equal} \\
  H \aln H'  & & \mbox{neutral weak head normal forms  are algorithmically equal} \\
\alreq k {k'} {\rho}{\rho'} & & \mbox{environments  are
    algorithmically equal under shifts by  resp.} \\
\end{array}

\mbox{Algorithmic equality of neutral weak head normal forms.\hspace{6.5cm}}\\
  \ru{}{a \aln a}
\quad
  \ru{}{x_m \aln x_m}
\quad
  \ru{\alreq 0 0 \rho {\rho'}
    }{\esub \rho {X_m} \aln \esub {\rho'} {X_m}}
\quad
  \ru{H \aln H' \quad 
      \twhnf~ L \alw \twhnf~ L' 
    }{\app H L \aln \app {H'} {L'}}
\0.75em]
  \ru{H \aln H'  
    }{H \alw H}
\qquad
  \ru{\whnf {\lift\rho} \eta M \alw \whnf {\lift{\rho'}} {\eta'}{M'}
    }{\dsubp \rho \eta {\lambda M} \alw 
      \dsubp {\rho'}{\eta'}{\lambda M'}}

\ru{\whnf {\lift\rho} \eta M \alw 
      \app{(\shiftNe 1 H)}{x_1}  
    }{\dsubp \rho \eta {\lambda M} \alw H}
\qquad
  \ru{\app{(\shiftNe 1 H)}{x_1}
      \alw \whnf {\lift\rho} \eta M 
    }{H \alw \dsubp \rho \eta {\lambda M}}

\mbox{Algorithmic equality of environments.\hspace{10cm}}\\
  \ru{k + n = k' + n'
    }{\alreq k {k'}{\shiftby n}{\shiftby {n'}}}
\qquad
  \ru{\alreq{k+n}{k'}\rho{\rho'}
    }{\alreq k{k'}{\shiftEnv n \rho}{\rho'}}
\qquad
  \ru{\alreq{k}{k'+n'}\rho{\rho'}
    }{\alreq k{k'}{\rho}{\shiftEnv {n'}{\rho'}}}
\0.75em]
\ru{\alreq k {k'} \rho {\shiftby{n'+1}} \qquad
      \twhnfp{\shiftClos k L} \alw x_{k'+n'+1}
    }{\alreq k {k'} {(\rho, L)}{\shiftby {n'}}}
\qquad
  \ru{\alreq k {k'} {\shiftby{n+1}} {\rho'} \qquad
      x_{k+n+1} \alw \twhnfp{\shiftClos {k'} {L'}}
    }{\alreq k {k'} {\shiftby n}{(\rho', L')}}

\begin{array}{ll}
  \Delta; \Gamma \der V \jchk s & \mbox{Type normal form  checks against
    sort } \\
  \Delta; \Gamma \der V \jchk L & \mbox{Normal form  checks against
    ``type'' closure } \\
  \Delta; \Gamma \der U \jinf L & \mbox{The type of neutral normal
    form  is inferred as closure } 
\
In these judgements,  is a list of type closures .  On
 we pose no restrictions; an entry  of 
is as before a list of type expressions  and a type expression .
 


\noindent
Inferring the type of neutral normal forms .
0.75em]
  \ru{\length{\Gamma'} = n
    }{\Delta; \Gamma, L, \Gamma' \der x_{n+1} \jinf 
\shiftClos {n+1} {L} 
      }
\qquad 
  \ru{\Delta = \Delta_1,\cdec \Psi A,\Delta_2 \qquad
      \length{\Delta_2} = n \qquad
\Delta; \Gamma \der \nu \jchk \dsub{\sid}{\Shiftby{n+1}}\Psi
    }{\Delta; \Gamma \der \esub \nu {X_{n+1}} \jinf 
        \dsub \nu {\Shiftby {n+1}} A} 

\ru{\twhnf~L = \dsubp \rho \eta {\fun A B} \quad
\Delta; \Gamma, \dsub \rho \eta A  \der V \jchk
         \dsub {\shiftby 1 \rho, x_1} \eta B
    }{\Delta; \Gamma \der \lambda V \jchk L}
\quad
\ru{\Delta; \Gamma \der U \jinf L \quad
      \twhnf~L \alw \twhnf~L'
    }{\Delta; \Gamma \der U \jchk L'}

\ru{}{\Delta; \Gamma \der \lftype \jchk \lfkind}
\quad
  \ru{\Delta; \Gamma \der V \jchk \lftype \quad
\Delta; \Gamma,  \dsub \sid \mId V \der V' \jchk s
    }{\Delta; \Gamma \der \fun V V' \jchk s} 
\quad
\ru{\Delta; \Gamma \der U \jinf L \quad
      \twhnf~L = \lftype
    }{\Delta; \Gamma \der U \jchk \lftype}

  \ru{\length\Gamma = n
    }{\Delta; \Gamma \der \shiftby n \jchk \cempty}
\qquad
  \ru{\Delta; \Gamma \der \nu \jchk \Psi \qquad
      \Delta; \Gamma \der V \jchk L
    }{\Delta; \Gamma \der (\nu, V) \jchk \Psi,L}







\section{Conclusion}
We have presented an explicit substitution calculus together with
algorithms for weak head normalization, definitional equality, and
bi-directional type checking where both ordinary variables and
meta-variables are modelled using de Bruijn indices and both kinds of
substitutions are handled lazily and simultaneously. 

We also have proven subject reduction and soundness of the
definitional equality algorithm.\LONGVERSION{
A sketch of the normalization proof,
which guarantees that the described algorithm is complete, can be
found in the appendix.}
Finally, we describe a bi-directional
type-checking algorithm which treats ordinary substitutions and
meta-substitutions at the same time. In the future, we plan to 
\SHORTVERSION{prove completeness of algorithmic equality and type
  checking and to }adapt
the presented explicit substitutions in the implementation of the
programming and reasoning environment Beluga.

\bibliographystyle{eptcs}


\begin{thebibliography}{10}
\providecommand{\bibitemstart}[1]{\bibitem{#1}}
\providecommand{\bibitemend}{}
\providecommand{\bibliographystart}{}
\providecommand{\bibliographyend}{}
\providecommand{\url}[1]{\texttt{#1}}
\providecommand{\urlprefix}{Available at }
\providecommand{\bibinfo}[2]{#2}
\bibliographystart

\bibitemstart{abadiCardelliCurienLevy:jfp91}
\bibinfo{author}{Mart\'{\i}n Abadi}, \bibinfo{author}{Luca Cardelli},
  \bibinfo{author}{Pierre-Louis Curien} \& \bibinfo{author}{Jean-Jacques
  L{\'e}vy} (\bibinfo{year}{1991}): \emph{\bibinfo{title}{Explicit
  Substitutions}}.
\newblock {\sl \bibinfo{journal}{Journal of Functional Programming}}
  \bibinfo{volume}{1}(\bibinfo{number}{4}), pp. \bibinfo{pages}{375--416}.
\bibitemend

\bibitemstart{abelCoquand:lfsigma}
\bibinfo{author}{Andreas Abel} \& \bibinfo{author}{Thierry Coquand}
  (\bibinfo{year}{2007}): \emph{\bibinfo{title}{Untyped Algorithmic Equality
  for {Martin-L\"of's} Logical Framework with Surjective Pairs}}.
\newblock {\sl \bibinfo{journal}{Fundamenta Informaticae}}
  \bibinfo{volume}{77}(\bibinfo{number}{4}), pp. \bibinfo{pages}{345--395}.
\newblock \bibinfo{note}{{TLCA'05} special issue.}
\bibitemend

\bibitemstart{adams:PhD}
\bibinfo{author}{Robin Adams} (\bibinfo{year}{2005}): \emph{\bibinfo{title}{A
  Modular Hierarchy of Logical Frameworks}}.
\newblock \bibinfo{type}{Ph.D. thesis}, \bibinfo{school}{University of
  Manchester}.
\bibitemend

\bibitemstart{boveDybjerNorell:tphols09}
\bibinfo{author}{Ana Bove}, \bibinfo{author}{Peter Dybjer} \&
  \bibinfo{author}{Ulf Norell} (\bibinfo{year}{2009}): \emph{\bibinfo{title}{A
  Brief Overview of {Agda} - A Functional Language with Dependent Types}}.
\newblock In: \bibinfo{editor}{Stefan Berghofer}, \bibinfo{editor}{Tobias
  Nipkow}, \bibinfo{editor}{Christian Urban} \& \bibinfo{editor}{Makarius
  Wenzel}, editors: {\sl \bibinfo{booktitle}{22nd International Conference on
  Theorem Proving in Higher Order Logics (TPHOLs'09)}}, {\sl
  \bibinfo{series}{Lecture Notes in Computer Science}} \bibinfo{volume}{5674},
  \bibinfo{publisher}{Springer-Verlag}, pp. \bibinfo{pages}{73--78}.
\newblock \urlprefix\url{http://dx.doi.org/10.1007/978-3-642-03359-9_6}.
\bibitemend

\bibitemstart{coquand:conversion}
\bibinfo{author}{Thierry Coquand} (\bibinfo{year}{1991}):
  \emph{\bibinfo{title}{An Algorithm for Testing Conversion in Type Theory}}.
\newblock In: \bibinfo{editor}{G.~Huet} \& \bibinfo{editor}{G.~Plotkin},
  editors: {\sl \bibinfo{booktitle}{Logical Frameworks}},
  \bibinfo{publisher}{Cambridge University Press}, pp.
  \bibinfo{pages}{255--279}.
\bibitemend

\bibitemstart{coquand:type}
\bibinfo{author}{Thierry Coquand} (\bibinfo{year}{1996}):
  \emph{\bibinfo{title}{An Algorithm for Type-Checking Dependent Types}}.
\newblock In: {\sl \bibinfo{booktitle}{Mathematics of Program Construction.
  Selected Papers from the Third International Conference on the Mathematics of
  Program Construction (July 17--21, 1995, Kloster Irsee, Germany)}}, {\sl
  \bibinfo{series}{Science of Computer Programming}}~\bibinfo{volume}{26},
  \bibinfo{publisher}{Elsevier}, pp. \bibinfo{pages}{167--177}.
\bibitemend

\bibitemstart{dowekHardinKirchner:infcomp00}
\bibinfo{author}{Gilles Dowek}, \bibinfo{author}{Th{\'e}r{\`e}se Hardin} \&
  \bibinfo{author}{Claude Kirchner} (\bibinfo{year}{2000}):
  \emph{\bibinfo{title}{Higher Order Unification via Explicit Substitutions}}.
\newblock {\sl \bibinfo{journal}{Information and Computation}}
  \bibinfo{volume}{157}(\bibinfo{number}{1-2}), pp. \bibinfo{pages}{183--235}.
\bibitemend

\bibitemstart{harperPfenning:equivalenceLF}
\bibinfo{author}{Robert Harper} \& \bibinfo{author}{Frank Pfenning}
  (\bibinfo{year}{2005}): \emph{\bibinfo{title}{On Equivalence and Canonical
  Forms in the {LF} Type Theory}}.
\newblock {\sl \bibinfo{journal}{ACM Transactions on Computational Logic}}
  \bibinfo{volume}{6}(\bibinfo{number}{1}), pp. \bibinfo{pages}{61--101}.
\bibitemend

\bibitemstart{liangNadathurQi:jar05}
\bibinfo{author}{Chuck Liang}, \bibinfo{author}{Gopalan Nadathur} \&
  \bibinfo{author}{Xiaochu Qi} (\bibinfo{year}{2005}):
  \emph{\bibinfo{title}{Choices in representation and reduction strategies for
  lambda terms in intensional contexts}}.
\newblock {\sl \bibinfo{journal}{Journal of Automated Reasoning}}
  \bibinfo{volume}{33}(\bibinfo{number}{2}), pp. \bibinfo{pages}{89--132}.
\bibitemend

\bibitemstart{Nadathur:TCS98}
\bibinfo{author}{Gopalan Nadathur} \& \bibinfo{author}{Debra~Sue Wilson}
  (\bibinfo{year}{1998}): \emph{\bibinfo{title}{A Notation for Lambda Terms: A
  Generalization of Environments}}.
\newblock {\sl \bibinfo{journal}{Theoretical Computer Science}}
  \bibinfo{volume}{198}(\bibinfo{number}{1-2}), pp. \bibinfo{pages}{49--98}.
\newblock \urlprefix\url{http://dx.doi.org/10.1016/S0304-3975(97)00184-9}.
\bibitemend

\bibitemstart{Nanevski:ICML05}
\bibinfo{author}{Aleksandar Nanevski}, \bibinfo{author}{Frank Pfenning} \&
  \bibinfo{author}{Brigitte Pientka} (\bibinfo{year}{2008}):
  \emph{\bibinfo{title}{Contextual modal type theory}}.
\newblock {\sl \bibinfo{journal}{ACM Transactions on Computational Logic}}
  \bibinfo{volume}{9}(\bibinfo{number}{3}), pp. \bibinfo{pages}{1--49}.
\bibitemend

\bibitemstart{norell:PhD}
\bibinfo{author}{Ulf Norell} (\bibinfo{year}{2007}):
  \emph{\bibinfo{title}{Towards a practical programming language based on
  dependent type theory}}.
\newblock \bibinfo{type}{Ph.D. thesis}, \bibinfo{school}{Department of Computer
  Science and Engineering, Chalmers University of Technology},
  \bibinfo{address}{G\"{o}teborg, Sweden}.
\bibitemend

\bibitemstart{Pfenning99cade}
\bibinfo{author}{Frank Pfenning} \& \bibinfo{author}{Carsten Sch{\"u}rmann}
  (\bibinfo{year}{1999}): \emph{\bibinfo{title}{System Description: {Twelf} ---
  A Meta-Logical Framework for Deductive Systems}}.
\newblock In: \bibinfo{editor}{H.~Ganzinger}, editor: {\sl
  \bibinfo{booktitle}{16th International Conference on
  Automated Deduction (CADE-16)}}, {\sl \bibinfo{series}{Lecture Notes in
  Artificial Intelligence}} \bibinfo{volume}{1632},
  \bibinfo{publisher}{Springer}, pp. \bibinfo{pages}{202--206}.
\bibitemend

\bibitemstart{Pientka03phd}
\bibinfo{author}{Brigitte Pientka} (\bibinfo{year}{2003}):
  \emph{\bibinfo{title}{Tabled higher-order logic programming}}.
\newblock \bibinfo{type}{Ph.D. thesis}, \bibinfo{school}{Department of Computer
  Science, Carnegie Mellon University}.
\newblock \bibinfo{note}{CMU-CS-03-185}.
\bibitemend

\bibitemstart{Pientka:PPDP08}
\bibinfo{author}{Brigitte Pientka} \& \bibinfo{author}{Joshua Dunfield}
  (\bibinfo{year}{2008}): \emph{\bibinfo{title}{Programming with proofs and
  explicit contexts}}.
\newblock In: {\sl \bibinfo{booktitle}{ACM SIGPLAN Symposium on Principles and
  Practice of Declarative Programming (PPDP'08)}}, \bibinfo{publisher}{ACM
  Press}, pp. \bibinfo{pages}{163--173}.
\bibitemend

\bibitemstart{pientkaDunfield:ijcar10}
\bibinfo{author}{Brigitte Pientka} \& \bibinfo{author}{Joshua Dunfield}
  (\bibinfo{year}{2010}): \emph{\bibinfo{title}{Beluga: {A} Framework for
  Programming and Reasoning with Deductive Systems (System Description)}}.
\newblock In: \bibinfo{editor}{J{\"u}rgen Giesl} \& \bibinfo{editor}{Reiner
  H{\"a}hnle}, editors: {\sl \bibinfo{booktitle}{5th International Joint
  Conference on Automated Reasoning (IJCAR'10)}}, \bibinfo{series}{Lecture
  Notes in Computer Science}, \bibinfo{publisher}{Springer-Verlag}.
\bibitemend

\bibitemstart{poswolskySchuermann:delphin}
\bibinfo{author}{Adam Poswolsky} \& \bibinfo{author}{Carsten Sch{\"u}rmann}
  (\bibinfo{year}{2009}): \emph{\bibinfo{title}{System Description: Delphin - A
  Functional Programming Language for Deductive Systems}}.
\newblock {\sl \bibinfo{journal}{Electronic Notes in Theoretical Computer
  Science}} \bibinfo{volume}{228}, pp. \bibinfo{pages}{113--120}.
\newblock \urlprefix\url{http://dx.doi.org/10.1016/j.entcs.2008.12.120}.
\bibitemend

\bibitemstart{Schuermann:ESOP08}
\bibinfo{author}{Adam Poswolsky} \& \bibinfo{author}{Carsten Sch{\"u}rmann}
  (\bibinfo{year}{2008}): \emph{\bibinfo{title}{Practical programming with
  higher-order encodings and dependent types}}.
\newblock In: {\sl \bibinfo{booktitle}{17th European
  Symposium on Programming (ESOP '08)}}, {\sl \bibinfo{series}{Lecture Notes in
  Computer Science}} \bibinfo{volume}{4960}, \bibinfo{publisher}{Springer},
  p.~\bibinfo{pages}{93}.
\bibitemend

\bibliographyend
\end{thebibliography}


\LONGVERSION{

\appendix
 
\input{normalization}







}

\end{document}
