\documentclass{LMCS}

\newcommand{\Omit}[1]{}



\usepackage{amsmath,amssymb,stmaryrd,amsthm,amscd,url}


\usepackage{hyperref}

\usepackage{diagrams}
\newarrow{Into} C--->
\newarrow{Onto} ----{>>}
\newarrow{Dashto} {}{dash}{}{dash}>

\newcommand{\e}{\varepsilon}
\renewcommand{\d}{\delta}
\newcommand{\fullversion}[1]{}

\newcommand{\tail}{\operatorname{tl}} 
\newcommand{\myomega}{\omega}
\newcommand{\cons}{*}\newcommand{\berger}{\e^{\text{Berger}}}
\newcommand{\bergerp}{\e^{\text{Berger}'}}
\newcommand{\eprod}{\e^{\prod}}
\newcommand{\eprodp}{\e^{\prod'}}
\newcommand{\eprodpp}{\e^{\prod''}}
\newcommand{\eberger}{\exists^{\text{Berger}}}
\newcommand{\ebergerp}{\exists^{\text{Berger}'}}
\newcommand{\evarepsilonprod}{\exists^{\prod}}
\newcommand{\aberger}{\forall^{\text{Berger}}}
\newcommand{\abergerp}{\forall^{\text{Berger}'}}
\newcommand{\avarepsilonprod}{\forall^{\prod}}
\newcommand{\cat}{}\newcommand{\eqdef}{\mathrel{:=}}\newcommand{\id}{\operatorname{id}}
\newcommand{\K}{\mathcal{K}}
\newcommand{\Z}{\mathcal{Z}}
\newtheorem{Example}[thm]{Example}
\newtheorem{Examples}[thm]{Examples}
\newenvironment{example}{\begin{Example}}{\end{Example}}
\newenvironment{examples}{\begin{Examples}}{\end{Examples}}
\newcommand{\myparagraph}{\paragraph}

\newcommand{\below}{\sqsubseteq}
\newcommand{\licsmathp}[1]{}

\newcommand{\licsmath}[1]{}
\newcommand{\varlicsmath}[1]{\begin{enumerate}
\item[] 
\end{enumerate}}
\newcommand{\licsmatht}[2]{\begin{enumerate}
\item[] \quad 
\item[] \quad 
\end{enumerate}}
\newcommand{\varlicsmatht}[2]{\begin{enumerate}
\item[] 
\item[] 
\end{enumerate}}
\newcommand{\licsmathtt}[3]{\begin{enumerate}
\item[] \quad 
\item[] \quad 
\item[] \quad 
\end{enumerate}}
\newcommand{\varlicsmathtt}[3]{\begin{enumerate}
\item[]  
\item[]  
\item[]  
\end{enumerate}}
\newcommand{\licsmathttt}[4]{\begin{enumerate}
\item[] \, 
\item[] \, 
\item[] \, 
\item[] \, 
\end{enumerate}}
\newcommand{\licsmathtttt}[5]{\begin{enumerate}
\item[] \quad 
\item[] \quad 
\item[] \quad 
\item[] \quad 
\item[] \quad 
\end{enumerate}}
\newcommand{\licsmathttttt}[6]{\begin{enumerate}
\item[] \quad 
\item[] \quad 
\item[] \quad 
\item[] \quad 
\item[] \quad 
\item[] \quad 
\end{enumerate}}
\newcommand{\varlicsmathtttt}[5]{\begin{enumerate}
\item[] \hspace{-3ex} 
\item[] \hspace{-3ex} 
\item[] \hspace{-3ex} 
\item[] \hspace{-3ex} 
\item[] \hspace{-3ex} 
\end{enumerate}}
\newcommand{\varlicsmathttttt}[6]{\begin{enumerate}
\item[] \hspace{-3ex} 
\item[] \hspace{-3ex} 
\item[] \hspace{-3ex} 
\item[] \hspace{-3ex} 
\item[] \hspace{-3ex} 
\item[] \hspace{-3ex} 
\end{enumerate}}



\newcommand{\bnf}{\mathrel{::=}}
\newcommand{\lift}[1]{#1_{\bot}}
\newcommand{\N}{\mathbb{N}}
\newcommand{\R}{\mathbb{R}}
\newcommand{\I}{\mathbb{I}}
\newcommand{\Bool}{2}\newcommand{\Sierp}{\mathcal{S}}
\newcommand{\pN}{\mathcal{N}}
\newcommand{\pT}{\mathcal{T}}
\newcommand{\pU}{\mathcal{U}}
\newcommand{\pI}{\mathcal{I}}
\newcommand{\pM}{\mathcal{M}}
\newcommand{\pR}{\mathcal{R}}
\newcommand{\pBool}{\mathcal{B}}
\newcommand{\If}{\,\mathrel{\operatorname{if}}}
\newcommand{\Then}{\mathrel{\operatorname{then}}}
\newcommand{\Else}{\mathrel{\operatorname{else}}}
\newcommand{\True}{1}\newcommand{\False}{0}\newcommand{\domain}[1]{{\D_{#1}}}
\newcommand{\total}[1]{{\T_{#1}}}
\newcommand{\totaleq}[1]{\sim_{#1}}
\newcommand{\quo}[1]{\qq_{#1}}
\newcommand{\qq}{\rho}
\newcommand{\D}{D}
\newcommand{\E}{E}
\newcommand{\C}{C}
\newcommand{\T}{T}
\newcommand{\kk}[1]{{\C_{#1}}}
\newcommand{\Scott}{\operatorname{Scott}}
\newcommand{\tproduct}{\sigma \times \tau}
\newcommand{\tfunction}{\sigma \to \tau}
\newcommand{\rh}[1]{\rho_{#1}}
\newcommand{\comp}{\circ}
\newcommand{\siff}{\iff}\newcommand{\Search}{\operatorname{\mathcal{S}}}
\newcommand{\g}{\operatorname{\text{\boldmath{}}}}
\newcommand{\f}{\operatorname{\text{\boldmath{}}}}
\newcommand{\x}{\operatorname{\text{\boldmath{}}}}
\newcommand{\sK}{\operatorname{\text{\boldmath{}}}}
\newcommand{\sF}{\operatorname{\text{\boldmath{}}}}
\newcommand{\sU}{\operatorname{\text{\boldmath{}}}}
\newcommand{\sV}{\operatorname{\text{\boldmath{}}}}
\newcommand{\sQ}{\operatorname{\text{\boldmath{}}}}
\newcommand{\sP}{\operatorname{\text{\boldmath{}}}}
\newcommand{\sE}{\operatorname{\text{\boldmath{}}}}

\def\doi{4 (3:3) 2008}
\lmcsheading {\doi}
{1--37}
{}
{}
{Jan.~27, 2008}
{Aug.~27, 2008}
{}   

\begin{document}

\author[M.~Escard\'o]{Mart\'\i n Escard\'o}
\address{School of Computer Science, University of Birmingham, B15 2TT, UK}
\email{m.escardo@cs.bham.ac.uk}

\title[Exhaustible sets]{Exhaustible sets in higher-type computation}

\keywords{Higher-type recursion, continuous functional, PCF, domain
  theory, Scott domain, semantics, topology, compactly generated
  space, functional programming, Haskell}

\subjclass{F.4.1, F.3.2}
\amsclass{03D65, 68Q55, 06B35, 54D50.} 

\begin{abstract}
  We say that a set is \emph{exhaustible} if it admits algorithmic
  universal quantification for continuous predicates in finite time,
  and \emph{searchable} if there is an algorithm that, given any
  continuous predicate, either selects an element for which the
  predicate holds or else tells there is no example.  The Cantor space
  of infinite sequences of binary digits is known to be searchable.
  Searchable sets are exhaustible, and we show that the converse also
  holds for sets of hereditarily total elements in the hierarchy of
  continuous functionals; moreover, a selection functional can be
  constructed uniformly from a quantification functional.  We prove
  that searchable sets are closed under intersections with decidable
  sets, and under the formation of computable images and of finite and
  countably infinite products.  This is related to the fact,
  established here, that exhaustible sets are topologically compact.
  We obtain a complete description of exhaustible total sets by
  developing a computational version of a topological Arzela--Ascoli
  type characterization of compact subsets of function spaces. We also
  show that, in the non-empty case, they are precisely the computable
  images of the Cantor space. The emphasis of this paper is on the
  theory of exhaustible and searchable sets, but we also briefly
  sketch applications. 
\end{abstract}

\maketitle

\section{Introduction}

\noindent
A wealth of computational problems of interest have the following
form:
\begin{quote}
  \em Given a set  and a property  of elements of , decide
  whether or not \emph{all} elements of  satisfy~.
\end{quote}
For  fixed in advance, this is equivalent to the emptiness problem
for~. One is often interested in suitable restrictions on the
possible syntactical forms of the predicate~ that guarantee that
this problem is decidable (or, less ambitiously, that the
non-emptiness problem is semi-decidable) uniformly in the syntactical
form of~. 
In this work, on the other hand, the emphasis is on the set~ rather
than the predicate~, and we study the case in which  is
infinite. Moreover,  is not assumed to be given syntactically or
via any other kind of intensional information: we only use information
about the input-output relation determined by~ considered as a
boolean-valued function.
In the absence of intensional information, continuity of~ plays a
fundamental role, where  is continuous iff for any  in the
domain of , the boolean value  depends only on a finite
amount of information about~.  We work in the realm of higher-type
computation with continuous functionals, using Ershov--Scott domains
to model partial functionals, and Kleene--Kreisel spaces to model
total functionals~\cite{normann:computer,normann:recursion}.

\pagebreak[3]
We say that the set  is \emph{exhaustible} if the above problem can
be algorithmically solved for any continuous  defined on ,
uniformly in~. The uniform dependency on  is formulated by
giving the algorithm the type , where 
is a domain, , and  is the domain of booleans.
The main question investigated in this work is what kinds of infinite
sets are exhaustible.

Clearly, finite sets of computable elements are exhaustible. What may
be rather unclear is whether there are \emph{infinite} examples.
Intuitively, there can be none: how could one possibly check
infinitely many cases in finite time? This intuition is correct
when~ is a set of natural numbers: it is a theorem that, in this
case,  is exhaustible if and only if it is finite.  This can be
proved by reduction to the halting problem, but there is also a purely
topological argument (Remark~\ref{introduction}).  However, it turns
out that there is a rich supply of infinite exhaustible sets. A first
example, the Cantor space of infinite sequences of binary digits, goes
back to the 1950's, or even earlier, with the work of Brouwer, as
discussed in the related-work paragraph below.

We say that  is \emph{searchable} if there is an algorithm that,
given any continuous predicate~, either selects some  such
that  holds, or else reports that there isn't any. It is easy to
see that searchable sets are exhaustible. We show that, for sets of
total elements, the converse also holds and hence the two notions
coincide.  Moreover, a selection functional can be constructed
uniformly from a quantification functional
(Section~\ref{characterization}).

We develop tools for systematically building exhaustible and
searchable sets, and some characterizations, including the following:
they are closed under intersections with decidable sets, under the
formation of computable images and of finite and countably infinite
products (Section~\ref{building}).  In the case of exhaustibility, the
last claim is restricted to sets of total elements, and is open beyond
this case.  The non-empty exhaustible sets of total elements are
precisely the computable images of the Cantor space (Section~\ref{characterization}). We also formulate and
prove an Arzela--Ascoli type characterization of exhaustible sets of
total elements of function types (Section~\ref{arzela:ascoli}).

The above closure properties and characterizations resemble those of
compactness in topology. This is no accident: we show that exhaustible
sets of total elements are indeed compact, in the Kleene--Kreisel
topology (Section~\ref{criteria}).  This plays a crucial role in the
correctness proofs of some of the algorithms, and, indeed, in their
very construction.  Thus, the specifications of all of our algorithms
can be understood without much background, but an understanding of the
working of some of them requires some amount of topology.  We have
organized the presentation so that the algorithms occurring earlier
are motivated by topology but don't rely on knowledge of topology for
their formulation or correctness proofs.

In Section~\ref{background} we include background material that can be
consulted on demand and in Section~\ref{definitions} we define the
central notions investigated in this work. In Section~\ref{technical}
we include technical remarks, further work, announcement of results,
applications, and research directions.  In the concluding
Section~\ref{conclusion} we review the role topology plays in our
investigation of exhaustible and searchable sets.

\pagebreak[4]
\myparagraph{Related work.}  Brouwer's Fan functional gives the
modulus of uniform continuity of a discrete-valued continuous
functional on the Cantor space.  According to personal communication
by Dag Normann, computability of the Fan functional was known in the
late 1950's. This immediately gives rise to the exhaustibility of the
Cantor space. A number of authors have considered the definability of
the Fan functional in various formal systems.
Normann~\cite{normann:computer} cites Tait (1958, unpublished), Gandy
(around 1982, unpublished) and Berger~\cite{berger:thesis} (1990).
Tait showed that the Fan functional is not definable from Kleene's
schemes S1--S9 interpreted over \emph{total} functionals. Berger
observed that, for \emph{partial functionals}, PCF definability
coincides with S1--S9 definability, and showed that the Fan functional
is PCF definable. In order to do that, he first explicitly defined a
selection functional for the Cantor space.  Then Hyland informed the
community that Gandy was aware of the PCF/S1--S9 definability of the
Fan functional for the partial interpretation of Kleene's schemes,
but Gandy's construction seems to be lost.



\pagebreak[3] \myparagraph{Acknowledgements.} I have benefited from
stimulating discussions with, and questions by, Andrej Bauer, Ulrich
Berger, Dan Ghica, Achim Jung, John Longley, Paulo Oliva, Matthias
Schr\"oder, and Alex Simpson. I also thank Dag Normann for having
answered many questions regarding the history and technical
ramifications of the subject of higher-type computation, and for
sending me a copy of Tait's unpublished manuscript --- but the reader
should consult his paper~\cite{normann:computer} for a more accurate
and detailed account.

\section{Background} \label{background}

The material developed here can consulted on demand, except for
Section~\ref{domains}, which introduces and briefly discusses our
model of computation. Some readers will be more familiar with domain
theory and Ershov--Scott continuous functionals (and PCF or functional
programming) via denotational semantics, and others with the
Kleene--Kreisel continuous functionals via higher-type computability
theory, and we consider these two models and their
relationship~\cite{normann:computer}.  Alternatively, we could have
worked with Weihrauch's model of computation via
representations~\cite{weihrauch:analysis}, which generalizes Kleene's
approach via associates~\cite{normann:recursion}.  Even better, we
could have worked with the QCB model of computation, which subsumes
both domain theory and representation theory in a natural
way~\cite{MR2328287,MR1948051}. We adopt the Ershov--Scott and
Kleene--Kreisel approaches as they have played a wide
role~\cite{normann:computer}. A presentation based on QCB spaces would
have been not only more general but also cleaner in several ways, but
less familiar and perhaps more technically demanding.  Our objective
in this paper is to address the essential issues without getting
distracted by an excessive amount of generality.

\subsection{Domains of computation} \label{domains}

We work on a cartesian closed category of computable maps of
effectively given domains that contains the flat domains of booleans,

and of natural numbers, 
 
such as~\cite{egli:constable} or~\cite{smyth:effectively} among other
possibilities. Notice that these categories are closed under countable
cartesian powers. We don't need to, and we don't, explicitly refer to
effective presentations, and in particular to numberings of finite
elements or abstract bases etc.\ to formulate computability results.
We instead start from well known computable functions and use the fact
that computable functions are closed under definition by lambda
abstraction, application, least fixed points etc.  Moreover, we don't
invoke non-sequential functions such as Platek's
parallel-or~\cite{scott:lcf} or Plotkin's
parallel-exists~\cite{plotkin:lcf} in order to construct new
computable functions. Our algorithms can thus be directly understood
as functional programs in e.g.\ PCF~\cite{plotkin:lcf},
FPC~\cite{plotkin:domains} (PCF extended with recursive types,
interpreted as solutions of domain equations) or practical versions of
FPC such as Haskell~\cite{bird,haskell:hutton}, as done
in~\cite{escardo:lics07}.

At some point we need further assumptions on our domains of
computation to be able to formulate and prove certain results. Some of
those results can be formulated, and perhaps also be proved, for
domains with totality in the sense of Berger~\cite{berger:total}.  We
consider the particular case consisting of the smallest collection of
domains containing  and  and closed under finite
products, countable powers and exponentials (=function spaces).

\subsection{Higher-type computation} \label{higher:background}

The remainder of this section is not needed until
Theorem~\ref{uniform:cont}.  As discussed in
e.g.~\cite{normann:computer,MR2143877,longley:ubiquitous}, there are
many approaches to higher-type computation.  Kleene defined the total
functionals directly, but it has been found more convenient to work
with the larger collection of partial functionals and isolate the
total ones within them, as done by Kreisel. The approaches are
equivalent, and such total functionals are often referred to as
\emph{Kleene--Kreisel functionals} or \emph{continuous functionals}.
It turns out that, as discussed by Normann~\cite{normann:computer},
this coincides with another approach that also arises programming
language semantics: equivalence classes of total functionals on
Ershov--Scott domains.  We work with both total functionals on domains
and a characterization of the Kleene--Kreisel functionals, due to
Hyland, in terms of compactly generated spaces.


\myparagraph{Types.}  The \emph{simple types} are defined by induction
as
\licsmath{\sigma,\tau \bnf o \mid \iota \mid \tproduct \mid \tfunction,}
where  and  are ground types for booleans and natural
numbers respectively.  The subset of \emph{pure types} is defined by
\licsmath{\sigma \bnf \iota \mid \sigma \to \iota.}  As usual, we'll
occasionally reduce statements about simple types to statements about
pure types.


\myparagraph{Partial functionals.}  
For each type , define a domain~ of
\emph{partial functionals} of type  by induction as follows:
\licsmathtt{\domain{o} = \pBool, \quad \domain{\iota} = \pN,}
{\domain{\tproduct} = \domain{\sigma} \times
  \domain{\tau},}{\domain{\tfunction} = (\domain{\sigma} \to
  \domain{\tau}) = \domain{\tau}^{\domain{\sigma}}} where the products
and exponentials are calculated in the cartesian closed category of
continuous maps of Scott domains, where a \emph{Scott domain} is an
algebraic, bounded complete, and directed complete
poset~\cite{abramsky:jung}.

\myparagraph{Total functionals.} 
For each type , define a set  of \emph{total
  functionals} and a relation~ on~
as follows, where  ranges over the ground types~ and~:
1ex]
\total{\tproduct} = \total{\sigma} \times \total{\tau},
&  & 
(x,x') \totaleq{\tproduct} (y,y') \siff \text{}, \
Then the set  can be recovered from the
relation~ as 

and the relation can be recovered from the set as

See e.g.~\cite{berger:total} and~\cite{plotkin:totality}. In
particular,  is an equivalence relation
on~.

\myparagraph{Computability.}  Plotkin~\cite{plotkin:lcf} characterized
the computable partial functionals as those that are PCF-definable
from parallel-or and parallel-exists~\cite{plotkin:lcf}.  All
computable functionals we construct from Section~\ref{characterization} onwards
are defined in PCF without parallel extensions. This characterization
of computability includes, in particular, total functionals. An
interesting fact, which we don't need to invoke, is that every
\emph{total} functional definable in PCF with parallel extensions is
equivalent to one definable in PCF without parallel
extensions~\cite{normann:totality}.

\subsection{Kleene--Kreisel functionals} \label{kk:background}

For each type~, define by induction a set~ of
\emph{Kleene--Kreisel functionals} of type~ and a surjection
 as follows, so
that
\licsmath{\kk{\sigma} \cong \total{\sigma}/\totaleq{\sigma}.}
For ground types and product types, define

For function types, consider the diagram \
\kk{\tfunction} & = & \{ \phi \colon \kk{\sigma} \to \kk{\tau}
  \mid \exists f \in \total{\tfunction}.\text{ commutes}\}, \\
\rh{\tfunction}(f) & = & \text{the unique  such that 
    commutes.}
\text{ iff  and }.N(K,V) = \{ f \in Y^X \mid f(K) \subseteq
V\},
  Y^X \times X  & \to & Y \\
  (f,x) & \mapsto & f(x)

  \bar{f} \colon Z & \to&  Y^X \\
  z & \mapsto & (x \mapsto f(z,x))

\e_{\N_\infty}(p)(i) = \exists n \le i.\, p(0^n 1^\omega).

p_{n,x_n}(\alpha)& = & p(x_0x_1\dots x_{n-1} x_{n}\alpha) \\
& = & p(\Pi(\e)(p)(0) * \Pi(\e)(p)(1) * \operatorname{\dots} * \Pi(\e)(p)(n-1) * x_n * \alpha).

\beta^{(k)}_i = \beta_{k+i}.
 \overline{\beta}(n) = \langle\beta_0, ... , \beta_{n-1}\rangle  p_{n,x,\e}(\alpha) = p(\overline{\Pi(\e)(p)}(n) * x * \alpha).
p_x(\alpha) & = & p(x * \alpha), \\
x_{\e,p} & = & \Pi(\e)(p)(0) = \e_0(\lambda x.p_x(\Pi(\e')(p_x))), \\
p_\e & = & p_{x_{\e,p}} = p_{0,x_{\e,p},\e}.

\overline{\Pi(\e)(p)}(n+1) 
& = & \Pi(\e)(p)(0) * \langle\Pi(\e)(p)(1), ... , \Pi(\e)(p)(n)\rangle \\
& = & x_{\e,p} * \langle\Pi(\e')(p_\e)(0), ... , \Pi(\e')(p_\e)(n-1)\rangle \\
& = &  x_{\e,p} * \overline{\Pi(\e')(p_\e)}(n).

p_{n+1,x,\e}(\alpha) 
& = & p(\overline{\Pi(\e)(p)}(n+1) * x * \alpha) \\
& = & p(x_{\e,p} * \overline{\Pi(\e')(p_\e)}(n) * x * \alpha) \\
& = & p_\e(\overline{\Pi(\e')(p_\e)}(n) * x * \alpha)\\
& = & (p_\e)_{n,x,\e'}(\alpha).

\Pi(\e)(p)(n+1) 
& = & \e_{n+1}(\lambda x.p_{n+1,x,\e}(\Pi(\e^{(n+2)})(p_{n+1,x,\e}))) \\
& = & \e'_n (\lambda x.(p_\e)_{n,x,\e'}(\Pi(\e'^{(n+1})((p_\e)_{n,x,\e})))  \\
& = & \Pi(\e')(p_\e)(n).
 \Pi(\e)(p) = \Pi(\e)(p)(0) * (\Pi(\e)(p))' = x_{\e,p} *
\Pi(\e')(p_\e).
  \Solve(p,\prod_i K_i) & \iff & \textstyle{\text{ solves  over  whenever  is a}} \
We need to show that if  is defined on  then
 holds.  The set of continuous predicates 
defined on  can be defined as follows by bar induction:
\begin{enumerate}
\item if  then  is defined on , and
\item if  is defined on  for all  then
 is defined on .
\end{enumerate}
Therefore, it suffices to show that for all ,
\begin{enumerate}
\item[(i)] if  then , and
\item[(ii)] if  for all  then .
\end{enumerate}
\noindent\emph{Proof of (i):} Let  and assume that .  Then , by monotonicity of . Let  be a selection
function for~. Then .  Since
 is defined on~, it follows that . Hence . If  for some 
then  and hence .

\medskip
\noindent
\emph{Proof of (ii):} Assume the bar induction hypothesis
\begin{quote}
()   for all .
\end{quote}
We need to show that if  is a selection function for~ then:
\begin{enumerate}  
\item[(ii)(a)] .
\item[(ii)(b)] If  for some , then .
\end{enumerate}
\noindent
\emph{Proof of (ii)(a):} We show that  by
induction on~.

\medskip
\noindent
\emph{Base case for (ii)(a):} .  Since  is a selection
function for , it follows from () that
 solves  over  for all . Then  is defined on , and
since  is a selection function for , it follows that
.

\medskip
\noindent
\emph{Induction step for (ii)(a):}  by
Claim~(2), by () and by the fact that  (base case).

\medskip
\noindent
\emph{Proof of (ii)(b):} Assume  for some .  Then  and
, and, by (), we have
. Since  is a selection
function for , it follows that  and .  Then  because
 is a selection function for . But , by Claim~(3).
\end{proof}

\pagebreak[3]
\begin{examples}
  \leavevmode
  \begin{enumerate}
  \item The Cantor space  is
    searchable.  A selection functional is given by  where  is a selection
    functional for the finite set .
  \item If  is a sequence of finite sets that are
    finitely enumerable uniformly in , then  is searchable, again using the product functional.
  \end{enumerate}
  If a product  is searchable, then each set  is
  searchable uniformly in , by Proposition~\ref{image} as it is the
  computable image of  under the -th projection.  \qed
\end{examples}

\begin{rem}
  Berger's selection algorithm   for
the Cantor space, mentioned in the introduction, can
be written as

If one defines  by
, as in the proof of Lemma~\ref{implicit},
then the above definition is equivalent to

Our product algorithm is inspired by this idea. \qed
\end{rem}

\medskip From now on, we rely on Section~\ref{higher:background} for
the definition of totality.  
By Lemma~\ref{implicit} above and by Theorem~\ref{ex:main} below, a
non-empty set of total elements is exhaustible iff it is searchable,
and hence the above theorem shows that non-empty, exhaustible sets of
total elements are closed under countable products. For
Sierpinski-valued, rather than boolean-valued, universal
quantification functionals, a countable-product algorithm is given
in~\cite{escardo:barbados}, but we don't know how to approach
countable products of boolean-valued quantifiers without the detour
via selection functionals at the time of writing.

We now derive a uniform continuity principle from
Theorem~\ref{searchable:tychonoff}, motivated by topological theorems
that assert that, in certain contexts, continuous functions are
uniformly continuous on compact sets. Define
 
Then  iff .

\begin{thm} \label{uniform:cont} If 
  is defined on a product  of searchable sets, then there
  is a number  such that for all ,
  \,\,\licsmath{\alpha =_n \alpha' \implies f(\alpha)=f(\alpha').}
\end{thm}
\begin{proof}
  Let  be the unique
  total function such that
 iff . 
Then .  If we define 
  then  and hence . So, by continuity of , there
  is  such that  We cannot conclude that
   for all 
  because there is no reason why the predicate  should be defined on~.  To overcome this difficulty, let  and define
   so that
   is defined on~ and
  above~. By monotonicity,  Now the predicate
   is defined on  and hence  for all .  But if  then ,
  and so , as required.
\end{proof}


\medskip
The following is an immediate consequence of this and
Theorem~\ref{searchable:tychonoff}:
\begin{cor} \label{fan}
  The functional  defined by

is computable uniformly in any sequence of selection functionals for
the sets~, and is defined on any  that is defined
on . Moreover, if the sets  consist of total
elements of a domain , then the fan functional is total.
\end{cor}
This holds, in particular, if  and each  is a finite
subset of  defined uniformly in , which is the case that has
been considered in higher-type computability theory regarding the fan
functional (see e.g.~\cite{gandy77}).
Here we have generalized this to
arbitrary higher types .
A consequence of the exhaustibility of the Cantor space is that:
\begin{cor}\label{decidable:equivalence}
  The total elements of the function space 
  have decidable equivalence.
\end{cor}
\begin{proof}
  The algorithm  given by  
  does the job.
\end{proof}

This can be generalized as follows, where we now rely on
Section~\ref{kk:background} for the definition of the Kleene--Kreisel
spaces~.  
\begin{defi} \label{discrete:compact}
The \emph{discrete} and \emph{compact}
types are inductively defined as
\newcommand{\discrete}{\mathtt{discrete}}
\newcommand{\compact}{\mathtt{compact}}

The reason for this terminology is that the space  is
discrete if  is discrete, and it is compact if  is
compact, as observed in~\cite{escardo:barbados}. \qed
\end{defi}
\pagebreak[3]
\begin{thm} \label{thm:discrete:compact}
\leavevmode
\begin{enumerate}
\item If  is discrete, then  is computably enumerable. 
\item The total elements of a domain of compact type form a searchable set.
\item The total elements of a domain of discrete type have decidable equivalence.
\end{enumerate}
\end{thm}
\begin{proof}
  By induction on the definitions of discrete and compact type.  The
  first condition holds by the Kleene--Kreisel density theorem, which
  gives a computable dense sequence of~, and by the fact
  that  is discrete.  For the second condition, use
  Theorem~\ref{searchable:tychonoff} with the aid of the first
  condition, and, for the third one, use the argument of
  Corollary~\ref{decidable:equivalence}.
\end{proof}

We conclude this section with a natural notion that plays a
fundamental role in our investigation of exhaustible and searchable
sets and their relationship. Let  and 
for types~ and~.
\begin{defi} \label{def:entire} We say that a set  is \emph{entire} if it consists of total elements and is
  closed under total equivalence. \qed
\end{defi}
Notice that if  is total then it is defined on every entire set.
If  is not total and  is not entire, but if  is defined
on~, then  for all  in , because if  then  and  are bounded above and hence so are 
and , which then must be equal as they are non-bottom by
definition.  But if  and  for  outside , it
doesn't follow that  (consider e.g.\  for  and ).

\medskip
The following closure properties of entire sets are easily verified:
\begin{enumerate}
\item If  and  are entire, so is 
\item If  is a sequence of entire subsets of , then
 is an entire subset of .
\end{enumerate}
\begin{defi}
  The image of an entire set by a total function doesn't need to be
  entire, but it consists of total elements, and hence its closure
  under total equivalence is entire. We refer to this as its
  \emph{entire image}. (Thus, entire images are defined for total
  functions and entire sets only.) \qed
\end{defi}


\begin{prop} \label{image:bis} Exhaustible and searchable sets are
  closed under the formation of computable entire images.
\end{prop}
\begin{proof}
For given  and  exhaustible,
consider the quantification functional
\licsmath{\forall_{f(K)}(q)=\forall x \in K.q(f(x)).}
defined in the proof of Proposition~\ref{image}.
If  is total and  is
entire with entire image , then we can take
. To verify this, let  be defined on
. Then  is defined on , and hence if
 for all , then . If, on the
other hand,  for some , then  for
some . But then , and so
, which concludes the verification.
The argument for searchable sets is similar. 
\end{proof}

\begin{defi} \label{semidecidable} \leavevmode Let  by the Sierpinski domain and 
  be entire.
  \begin{enumerate}
  \item  is \emph{decidable} if there is
    a total computable map  such that, for
    all total ,  iff .

  \item  is \emph{semi-decidable} if
    there is a computable map  such that,
    for all total ,  iff .

  \item  is \emph{co-semi-decidable} if its complement in
     semi-decidable. \qed
  \end{enumerate}
\end{defi}
Notice that the functions  is not uniquely determined by ,
because its behaviour is specified on a subset of~, but that
 is uniquely determined by~. Notice also 
 is decidable if and only if it is decidable on~ in the sense of
Definition~\ref{decidableonk} with .

\section{Compactness of exhaustible sets} \label{criteria}

A notion analogous to exhaustibility, with the Sierpinski
domain~ playing the role of the boolean
domain~, is considered in~\cite{escardo:barbados}.  A crucial
fact, formulated here as Lemma~\ref{crucial:generalized}, is that the
(now unique) quantification functional  is continuous iff the set~ is compact in the Scott
topology of~.  Hence, because computable functionals are
continuous, Sierpinski-exhaustible sets are compact, and so Sierpinski
exhaustibility is seen as articulating an algorithmic version of the
topological notion of compactness. The computational idea is that,
given any \emph{semi-decidable} property of , one can
\emph{semi-decide} whether it holds for all elements of~. Closure
properties analogous to the above are established for Sierpinski
exhaustibility in~\cite{escardo:barbados}. 

The present investigation can be seen as a natural follow-up of that
work that arises by asking what changes if one moves from
semi-decision problems to decision problems.  One significant change
is that continuity of a quantification functional  doesn't entail the compactness of  in
the Scott topology any longer:

\pagebreak[3]
\begin{examples} \leavevmode
\label{counter:example} 
\begin{enumerate}
\item \label{counter:example:1} \emph{There are exhaustible sets that
    fail to be compact in the Scott topology.}

  \medskip
  \noindent
  By~\cite{scott:datatypes,plotkin:tomega}, any second-countable
  ~space, e.g.\ the real line~, can be embedded into the
  domain  under the Scott topology. But  is a
  connected space, which is equivalent to saying that every continuous
  boolean-valued map defined on it is constant. Hence a predicate  is defined on  iff it is constant on~.
  Therefore  is trivially exhaustible: .
  But it is not compact.

  Notice also that any space embedded into the total elements of
   must be totally disconnected, and hence any
  embedding of  into  must assign non-total
  elements of  to some real numbers. One may suspect
  that if such embeddings are ruled out, this problem would disappear.
  But this is not the case, as the next example shows.

\item \label{counter:example:2} \emph{There are exhaustible sets of
    total elements that fail to be Scott compact.}

  \medskip
  \noindent
  In fact, there is a trivial and pervasive counter-example. Let  be total.  Then the total equivalence
  class  of , as is well known and easy to verify, doesn't have
  minimal elements, and hence cannot be compact in the Scott topology.
  But it is exhaustible with .  \qed
\end{enumerate}
\end{examples}

One may feel somewhat cheated by the second counter-example, because
although the set  is not Scott compact, it is generated by the
singleton , which is Scott compact, and because we took
 to be~ (cf.\ the proof of
Proposition~\ref{image:bis}).
Lemma~\ref{lemma:criterion}(\ref{lemma:criterion:3}) below shows that
any counter-example is generated by a Scott compact set in a similar
fashion.  In any case, although exhaustible sets do fail to be compact
in the Scott topology, if they consist of total elements then they are
compact in the Kleene--Kreisel topology.  In order to formulate and
prove this, we need some definitions. We now rely on
Section~\ref{kk:background}.
\begin{defi} 
  Let  be a type, , ,
   and . 
\begin{enumerate}
\item By the \emph{shadow} of a set  we mean its
  -image in~. 
  Similarly, by the shadow of an element  we mean its
  -image  in .

\item A set  is called \emph{Kleene--Kreisel compact}
  if its shadow is compact. \qed
\end{enumerate}
\end{defi}
Recall that the \emph{Cantor space} is the set  of
maximal elements of .
\begin{rem}
  Sometimes, for example for the implementation of the product
  functional defined in Section~\ref{building} in the language PCF,
  which lacks countable powers, one works with the Cantor space within
  the function space . The Cantor space is homeomorphic to
  the subspace of total strict functions ,
  where  is strict if .  It is also
  homeomorphic to the quotient of the set of \emph{all} total elements
  of~. But notice that the set of \emph{maximal} elements
  of  is \emph{not} homeomorphic to the Cantor
  space.  This is because the two non-strict elements  and  are finite (or order compact), and
  hence isolated in the relative Scott topology (meaning that the two
  corresponding singletons are open), and hence the maximal elements
  have a topology strictly finer than that of the Cantor space, as
  there are no isolated points in the Cantor space. As is well known
  in topology, no compact Hausdorff topology can have another compact
  Hausdorff topology as a strict refinement. \qed
\end{rem}

Every (computationally) exhaustible set is topologically exhaustible
in the sense of the following definition, because computable maps are
continuous.
\begin{defi} \label{topologically:exhaustible} \leavevmode
We say that a set  is \emph{topologically
    exhaustible} if there is a continuous map  satisfying the conditions of
  Definition~\ref{exhaustible:def}. \qed
\end{defi}
\noindent The following is our main tool in the constructions and
proofs of correctness and termination of algorithms developed in
Sections~\ref{characterization}--\ref{arzela:ascoli}.  Its proof relies on
Sections~\ref{compactly} and~\ref{hyland}.  \pagebreak[3]
\begin{lem} \label{lemma:criterion} \leavevmode 
  \begin{enumerate}
  \item \label{lemma:criterion:1} Any topologically
    exhaustible set of total elements is Kleene--Kreisel compact.

  \item \label{lemma:criterion:2} Any non-empty, Kleene--Kreisel
    compact entire set is an entire continuous image of the Cantor
    space and hence is topologically exhaustible.



  \item \label{lemma:criterion:3} Any Kleene--Kreisel compact entire
    set has a Scott compact subset with the same shadow.
  \end{enumerate}
\end{lem}
\pagebreak[3]
\begin{proof}
  (\ref{lemma:criterion:1}): Let  be exhaustible. By
  Lemma~\ref{zero:compact} and the fact that clopen sets are closed
  under finite unions, to establish compactness of , it is
  enough to consider a directed clopen cover . By
  Lemma~\ref{clopen:extension}, for every  there is
  a total  with
\licsmath{(\dagger) \quad \text{ and .}}
Define predicates  by
    
  Then , the set  is directed, and . Because , we have that
   and hence . So, by
  continuity of , there is  with
  , and hence with  by
  monotonicity.  Let . Then  by specification
  of~ and the fact that  is total and hence defined on
  .  But then , for otherwise  would
  entail .  This shows that , and
  so  is compact.

\medskip

(\ref{lemma:criterion:2}): By e.g.\ \cite{escardo:lawson:simpson}, any
compact subset of  is countably based (even though  is not). But
any non-empty compact Hausdorff countably based space is a continuous
image of the Cantor space. Hence there is a continuous map
 with image  for any entire set . Then the entire image of the Cantor space under any
representative  is~.

(\ref{lemma:criterion:3}): This follows from the argument given in
(\ref{lemma:criterion:2}), because the Cantor space is Scott compact.
\end{proof}

\begin{rem} \label{introduction}
In particular, this gives a topological view of the computational fact
stated in the introduction that exhaustible sets of natural numbers
must be finite: all compact sets are finite in a discrete space. \qed
\end{rem}
Kleene--Kreisel compactness can be expressed as a finite-subcover
condition for the Scott topology as follows: An entire set  is Kleene--Kreisel compact if and only if every cover of
 by Scott open sets that are closed under total equivalence has a
finite subcover.  This is a straightforward consequence of the fact
that the Kleene--Kreisel topology is the quotient by total equivalence
of the relative Scott topology on the total elements.








We also remark that there is a natural topology on , coarser than
the Scott topology, in which \emph{all} exhaustible sets are compact.
Part of the argument of
Lemma~\ref{lemma:criterion}(\ref{lemma:criterion:1}) shows that any
exhaustible set  is compact in the coarsest topology on  such
that all predicates  defined on  are
continuous.  This is generated by directed unions of basic open sets
of the form  with  as above, because such sets are
closed under finite unions and intersections.  This construction is
analogous to the zero-dimensional reflection of a topology, and
happens to coincide with it in the case considered in
Lemma~\ref{lemma:criterion}(\ref{lemma:criterion:1}), modulo
quotienting, and can also be compared with the weak topology in
functional analysis.


\section{Searchability  of exhaustible sets} 
\label{characterization}

We already know that every searchable set is exhaustible
(Lemma~\ref{implicit}).  This implication is uniform, in the sense
that there is a computable functional  that transforms selection functionals
into quantification functionals, namely . We now establish the converse for non-empty
entire sets, and some additional results. The fact that exhaustible
entire sets are Kleene--Kreisel compact, established in the previous
section, plays a fundamental role in the construction of the
algorithms
\begin{defi} \label{total:retract} We say that a set  is a \emph{retract up to total
    equivalence} if there is a function  such that
\begin{enumerate}
\item  for all total ,
\item  for all . \qed
\end{enumerate}
\end{defi}
In this case,  is total, all elements of  are total, and
 for all total~. Notice that  is
a retract up to total equivalence iff it is a total function and its
Kleene--Kreisel shadow  is a retract in the
usual topological sense, where .
\begin{defi}
  We say that two entire sets  and  are \emph{homeomorphic up to total
    equivalence} if there are total functions  and  such that  and  for
  all  and . \qed
\end{defi}
This is equivalent to saying that the shadow functions  and
 restrict to a (true) homeomorphism between the shadows
of~ and~. In this case,  is the entire -image of~, and
 is the entire -image of~.

\begin{thm} \label{ex:main} If  is a
  non-empty, exhaustible entire set then, uniformly in any
  quantification functional for :
  \begin{enumerate}
  \item  is searchable.
  \item  is a computable entire image of the Cantor space.
  \item  is computably homeomorphic to some entire exhaustible
    subset of the Baire domain~, up to total equivalence.
  \item  is a computable retract up to total equivalence.
  \item   is co-semi-decidable.
  \end{enumerate}  
\end{thm}
In particular, after the theorem is proved, one can w.l.o.g.\ work
with total predicates rather than predicates defined on~, as for any
predicate  defined on~ one can
uniformly find a total predicate that agrees with  on , by
composition with the retraction.

\pagebreak[3] 
\begin{proof}
  We proceed by cases, of increasing generality, on the type of~.
  The case  is trivial and is implicitly used in the
  case , which in turn is used in the next
  case . The general case  is
  reduced to this last case via retracts using Lemma~\ref{pure}.

\medskip

\emph{(i)} Case : We can define

This construction defines  uniformly in
. We could now easily show that  satisfies the other
conditions of the theorem, but this won't be required for our proof,
as this will follow in later cases.

\medskip

For future use, notice that if  is entire and
exhaustible, then the supremum of the finite set~ (which is zero
if  is empty and the largest element of  otherwise) can be
computed uniformly in any quantification functional for~ as
 
Hence, the finite enumeration  of
the elements of , in ascending order, 
for , is
uniformly computable as

We stop when we find  such that , and we include
 iff .

\medskip

\emph{(ii)}  Case .
We first argue that we can find some , uniformly in
, by the following algorithm defined by course-of-values
induction on :

Recall that we defined 

in the paragraph preceding Theorem~\ref{uniform:cont}.  By
construction, for every  there is  with , and in particular  is total.  Because the shadow of
 is compact, it is closed, and because  is entire, , as required.  Then we can define, using
Proposition~\ref{prop:intersec} and the above algorithm to construct
 in both cases,

Again,
this construction defines  uniformly in .

\medskip

We now show that  is a computable retract up to total equivalence,
uniformly in any quantification functional for~.  Define  by course-of-values induction
on :

Because the shadow of  is closed, the finite prefixes of its
members form a tree whose infinite paths correspond to the elements
of~. 
The above algorithm follows the infinite path  through the
tree, either for ever (always following the first case) or until the
path exits the tree (reaching the second case). If and when 
exits the tree, we replace the remainder of  by the left-most
infinite branch of the subtree at which  exits the tree. Then
 clearly satisfies the required conditions.  

\medskip

A semi-decision procedure for the complement of~ is given by

using the fact that apartness of total elements of  is
semi-decidable.
(This is a computational version of the topological fact
that retracts of Hausdorff spaces are closed.) 

\medskip

We now show that  is a computable entire image of the Cantor space.
For any , the set  is
exhaustible by Proposition~\ref{image} as evaluation at  is
computable. It is enough to show that  is an entire image of the
Cantor space by a computable map , because then  has  as its entire image
since  is contained in that product. But this is straightforward:
at each stage  of the computation of , look at the next
 digits of the input~,
compute the natural
number  represented by this finite sequence, and let
.

\medskip

\pagebreak[3] \emph{(iii)} Case  where
 for an arbitrary type~: In order to reduce this
to case~\emph{(ii)}, we invoke the Kleene--Kreisel density
theorem, to get a computable sequence  such that the shadow
sequence  is dense in .  Define


We will define a total function  in the other direction,

such that

exhibits  as a retract of  up to total equivalence.
Thus, for , one can recover the behaviour of 
at total elements from its behaviour on the dense sequence~.
Because this implies that  is the entire image of , and
because  is searchable by case~\emph{(ii)}, it will follow that
 is searchable and an entire image of the Cantor space, because 
preserves total equivalence on~.

For  and , define

Here we regard  as potentially coding the action of some 
on the set of elements .  The set  is decidable on
 uniformly in  and , and hence  is uniformly
exhaustible by Proposition~\ref{prop:intersec}.  \pagebreak[3]
\begin{lem} \label{c} If  is a
  Kleene--Kreisel compact entire set, then for all total  and all total  there is  such that  for all .
\end{lem}
\begin{proof}
  For any  the set  is clopen, where we
  write . By density of , the set  has at most one element, where  denotes
  the shadow of . Hence if 
  then .  Because
   is Hausdorff and because each 
  is compact and  is open, there is  such that already
  . So for all 
  one has , and hence for all  one has .
\end{proof}
By Proposition~\ref{image:bis}, the entire -image  of  is exhaustible.  Let 
be defined as in case~\emph{(ii)}, and define  by

where
 is the least number such that 
By exhaustibility of~, this can be found uniformly
  in , and hence  is computable uniformly in~.

\pagebreak[3]
\begin{proof}[Proof of correctness of .]
~

\emph{(a)~ is total and maps  into .} 
Let  be total.  Then , by
construction of , and hence there is  with , and so with  for any~. Let 
be total and  be the least number such that  for all
. Then  for all , and hence .  Therefore , and hence  as  is entire, and in
particular  is total.  By construction , and
hence, because  exhibits  as a retract up to total equivalence
and  is entire, the -image of  is .

\medskip
\emph{(b)~If  is total then .} 
Because .

\medskip
\emph{(c)~If  then .} 
Continuing from the proof of (a), for  we have  by construction of , and hence for any  such
that  we have  and so  by density, which shows
that , as required.
\end{proof}

\medskip

A semi-decision procedure for the complement of~ is
given as in case \emph{(ii)},

because 

for total functions  and  since  is entire and  is dense.

Because  and  are total, they induce computable Kleene--Kreisel
functionals  and  where . If  is the shadow of~, then the restriction of  to 
followed by the co-restriction to its image is a homeomorphism : abstractly because any continuous bijection of compact
Hausdorff spaces is a homeomorphism, and concretely because the
bi-restriction of  is a continuous inverse.  Hence any
exhaustible subset of  is computably homeomorphic to the
shadow of some exhaustible subset of the Baire domain~,
up to total equivalence.

\medskip

\pagebreak[3] \emph{(iv)} General case. We derive this from the case
\emph{(iii)}. By Lemma~\ref{pure}, for any  there are
 and computable  and  such  is a retraction up to total
equivalence and  is the entire image of .  Let  be a non-empty, exhaustible entire set, and let  be
the entire -image of . Then  is the entire image of ,
and, because  is entire, a predicate 
defined on  holds for all  if and only if 
holds for all .  Hence  is exhaustible with
. By case \emph{(iii)} above,
 is searchable. Therefore  is searchable by
Proposition~\ref{image}. Similarly, the other properties we need to
establish are closed under the formation of retracts and hence are
inherited from case~\emph{(iii)}.  

\medskip This concludes the proof of Theorem~\ref{ex:main}.
\end{proof}



\section{Ascoli--Arzela type characterizations of exhaustible sets} \label{arzela:ascoli}


We reformulate a theorem of Gale's~\cite{gale} that characterizes
compact subsets of function spaces (Theorem~\ref{gale:modified}).
This suggests a characterization of exhaustible entire sets
(Theorem~\ref{exhaustible:arzela}), whose topological version is
developed first (Theorem~\ref{exhaustible:arzela:topological}).  The
main idea is to replace a condition in Gale's theorem by a continuity
condition (Section~\ref{arzela:compact}), and then further replace it
by a computability condition (Section~\ref{arzela}). This method of
transforming topological theorems into computational theorems is the
main thrust of the paper~\cite{escardo:barbados}, which develops many
instances of computational manifestations of topological theorems.

\subsection{Topological version}
\label{arzela:compact}



The Heine--Borel theorem characterizes the compact subsets of
Euclidean space~ as those that are closed and bounded. The
Arzela--Ascoli theorem generalizes this to subsets of , where
 is a compact metric space and  is the set of continuous
functions endowed with the metric defined by  A set  is compact if
and only if it is closed, bounded and equi-continuous.
Equi-continuity of  means that the functions  are
simultaneously continuous, in the sense that for every  and
every , there is  such that  for all  and all . The
Heine--Borel theorem is the particular case in which  is the
discrete space , for equi-continuity holds
automatically for any subset of  in this case.  The above metric
on  induces the compact-open topology. More general
Arzela--Ascoli type theorems characterize compact subsets of
spaces~ of continuous functions under the compact-open topology,
for a variety of spaces~ and~, with a number of generalizations
or versions of the notion of equi-continuity, notably even continuity
in the sense of Kelley~\cite{kelley}.

Among a multitude of generalizations of the Arzela--Ascoli theorem,
that of Gale~\cite[Theorem 1]{gale} proves to be relevant concerning
exhaustibility of entire sets:
\pagebreak[3]
\begin{quote}
\em
  If  and  are Hausdorff -spaces with  regular, 
a set  is compact 
  if and only if
  \begin{enumerate}
  \item  is closed,
  \item the set  is compact for every ,
  \item the set  is open for every
    closed set  and for every open set .
  \end{enumerate}
\end{quote}
Gale didn't assume  to be a -space and formulated this for the
compact-open topology, but his theorem holds for the exponential
topology if we require  to be a -space.  Regarding compactness,
we have already mentioned that a Hausdorff space has the same compact
sets as its -coreflection, and that the exponential topology is the
-reflection of the compact-open topology. Although there are more
closed sets in the exponential topology, Gale's argument works with
closedness of  in the exponential topology. This follows
from the general considerations of Kelley~\cite[Chapter 7]{kelley}.

The last condition is a version of equi-continuity.  Because  is
not assumed to be compact, the set  cannot be globally bounded in
any sense, but it is pointwise bounded in the sense of the second
condition.  This gives a characterization of compact subsets of
Kleene--Kreisel spaces of the form  and in particular of
Kleene--Kreisel spaces of pure type, because  is regular. However,
as discussed in Section~\ref{hyland}, Matthias Schr\"oder has recently
shown that  is not regular, and this justifies the
restriction of our characterizations of exhaustible entire sets to
particular kinds of types in Section~\ref{arzela}.

Notice that when , this amounts to the well known
characterization of compact subsets~ of the Baire space 
as finitely branching trees. The equi-continuity condition, as in the
case of the Heine--Borel theorem, is superfluous, because any set is
equi-continuous in this case as the topology of the exponent is
discrete. Condition (1) says that the elements of  are the paths of
a tree, and (2) says that the tree is finitely branching, because the
compact subsets of the base space are finite.

\medskip

Lemma~\ref{crucial:generalized} and the remarks preceding it allow one
to consider continuity of functions involving points of a
-space~, open sets and closed sets (using the function space
 and representing open sets and closed sets by their
characteristic functions), and compact sets (using the function space
 and representing compact sets by their universal
quantification functionals). We now reformulate Gale's theorem by
expressing condition (3) as a continuous version of a slight
strengthening of condition~(2). 
\begin{thm} \label{gale:modified}
  If  and  are Hausdorff -spaces with  regular, a set  is compact if and only if
  \begin{enumerate}
  \item  is closed, and
  \item  is compact, continuously in  and ,
    for any closed set  and any .
  \end{enumerate}
\end{thm}
\noindent The dependence of  in the parameters  and  is
given by the functional

defined by

where we write

\begin{proof}
  (): The set  is closed because  is Hausdorff.
  The set  is compact because  is closed.  Because the
  evaluation map is continuous and because  is the
  continuous image of  under evaluation at , it is
  compact. To see that  is continuous, let .  Then
   for all    for
  all    or  for all .
  Hence 
  where  is defined by
   iff  or .  Because the
  functional  is continuous as  is compact, and because
  the category of -spaces is cartesian closed and the above is a
  -definition from continuous maps,  is
  continuous.

\medskip

 (): It suffices to show that Gale's conditions (1)-(3)
 hold.  Condition (1) is the same as ours, and Gale~(2) follows from
 our condition (2) with . To prove Gale~(3), let  be closed and  be open. Then the set 
 
 is open because  is continuous, and
 
 which shows that the set  is the
 same as  and hence is open.
\end{proof}


\begin{defi}
We say \emph{topologically decidable} etc.\ taking
  the continuous versions of Definitions~\ref{decidableonk}
  and~\ref{semidecidable}. \qed
\end{defi}

We now formulate and prove an analogue of this theorem, which replaces
(i)~the Sierpinski space~ by the boolean domain~,
(ii)~Hausdorff -spaces by Scott domains, (iii)~compact subsets by
topologically exhaustible entire subsets, (iv)~closed subsets by
topologically decidable sets (cf.\ Definitions~\ref{exhaustible:def}
and~\ref{topologically:exhaustible}).  We again apply Gale's theorem,
exploiting Hyland's characterization of the Kleene--Kreisel spaces as
-spaces.  The proof follows the same pattern as that of
Theorem~\ref{gale:modified}, but there are a number of additional
steps. Firstly, using Gale's theorem, we get continuous maps defined
on Kleene--Kreisel spaces.  These are extended to continuous maps on
domains using the Kleene--Kreisel density theorem and Scott's
injectivity theorem, as in Lemma~\ref{clopen:extension}.  (In
Theorem~\ref{exhaustible:arzela}, such an extension will be instead
defined by an algorithm, but still relying on the density theorem.)
Secondly, the set  in condition (2) is closed in
Theorem~\ref{gale:modified} but is neither open nor closed in
Theorem~\ref{exhaustible:arzela:topological}, although it has clopen
shadow, because the Sierpinski space has been replaced by the boolean
domain. To overcome this difficulty, we rely on the following version
of Gale's theorem:

\begin{rem} \label{F:specialcase} An inspection of the proof of Gale's
  theorem shows that it also holds if, in condition (3), the set 
  ranges over subbasic closed sets in the compact-open topology:
\begin{enumerate}
\item[.] the set  is open
  for every compact set , every closed set , and every open set .
\end{enumerate}
In one direction this is clear: if condition (3) holds for all closed
, then it holds for~. For the other direction, notice
that condition (3) is used only in the ``Lemma'' \cite[page 305]{gale}
for  of this form (the sets  in the second last line of that
page, and the set  of page 306). \qed
\end{rem}
Let  and  for an arbitrary type
, and recall the concepts and notation introduced in
Definitions~\ref{semidecidable} and~\ref{topologically:exhaustible}.
\pagebreak[3]
\begin{thm} \label{exhaustible:arzela:topological} An entire set
   is topologically exhaustible if and only
  if the following two conditions hold:
\pagebreak[3]
  \begin{enumerate}
  \item[1.]  is topologically co-semi-decidable. 
  \item[2.] The set  is topologically exhaustible for
    any  that is topologically decidable on , and any 
    total, continuously in~ and~.
  \end{enumerate}
\end{thm}
Here the dependence of  in  and  is
to be given by a functional

such that



\begin{proof}
  : (1): By Lemma~\ref{lemma:criterion}, the shadow
   of  is compact and hence closed.
  Hence the map  that sends  to~
  and  to~ is continuous. By composition with
  the quotient map , where , we get a map . Because  is dense in  and  is densely injective, the domain
   under the Scott topology is injective over
  dense embeddings, which means that this map extends to a continuous
  map .  By construction, this exhibits  as
  a topologically co-semi-decidable subset of .

  (2): Define . 
  The result then follows from the fact that the category of Scott
  domains under the Scott topology is cartesian closed, and hence
  functions that are -definable from continuous maps are
  themselves continuous. 

  : We apply Gale's theorem to show that the shadow
   is compact. Then it is topologically exhaustible by
  Lemma~\ref{lemma:criterion}. 

  Gale~(1): If  is topologically co-semi-decidable, then, by
  definition, we have a continuous function 
  that maps  to  and  to .  Hence
   is closed in  because it is the inverse image of the closed
  set  restricted to~.  Because  is entire, it is
  closed under total equivalence by definition, and hence, because
   is a quotient map,  is closed.

  Gale~(2): The assumption gives that for any  total, 
  is exhaustible, considering . Because  is entire
  and  is total, . Hence by
  Lemma~\ref{lemma:criterion},  is compact in .

  \pagebreak[3] Gale~(3): Let  be a subbasic open
  set of the form  with  compact and  (necessarily) clopen. Then the set 
  is entire and Kleene--Kreisel compact, and hence, by
  Lemma~\ref{lemma:criterion}, it is topologically exhaustible. Also,
   is a topologically decidable subset of . So the predicate
   defined by  is continuous and defined on~, and 
  for .  Now define  by  Then
   is continuous and
  
  Hence the set  is open. Therefore
  its shadow  is open,
  because it is closed under total equivalence and because  is a
  quotient map. \end{proof}



\subsection{Computational version}
\label{arzela}



At this stage of our investigation, such a characterization is
available only for certain types, which include pure types, and for
entire sets (for the reasons explained in
Section~\ref{arzela:compact}). Let  and 
for an arbitrary type .  We establish the computational
version of Theorem~\ref{exhaustible:arzela:topological}.
\pagebreak[3]
\begin{thm} \label{exhaustible:arzela}
  An entire set  is exhaustible if and only if 
  the following two conditions hold:
  \begin{enumerate}
  \item[1.]  is co-semi-decidable.
  \item[2.] The set  is exhaustible for any 
    decidable on , and any  total, uniformly in~
    and~.
  \end{enumerate}
  Moreover, the equivalence is uniform. 
\end{thm}

\medskip

A few remarks are in order before embarking into the proof.
The claim holds, with the same proof, if
conditions (1) and (2) are replaced by any of the following
conditions, respectively:
\begin{enumerate}
\item[.]  is topologically co-semi-decidable.
\item[.]  has closed shadow.
\item[.] The shadow of  is closed in the topology of
  pointwise convergence.
\item[.]  has compact shadow.
\item[.]  The set  is exhaustible, uniformly
    in ,  and  total.
\end{enumerate}
Recall (proof of Theorem~\ref{ex:main}) that
we defined


In the formulation of the theorem, the fact that
conditions~(1) and~(2) uniformly imply the exhaustibility of  is in
principle given by a computable functional of type

{\footnotesize
}

However, the computational information given by condition (1) is not
used in the construction of the conclusion (although the topological
information is used in its correctness proof). Moreover, the
information given by condition (2) is not fully used in the
construction. Replacing it by () we get

Additionally the pair  is really coding a finite sequence,
and, as we have seen, exhaustible sets of natural numbers are uniformly
equivalent to finite enumerations of natural numbers. Hence the
above can be written as

Therefore the above characterization reduces the type level of
 by two.

\medskip

The last step of the proof of this theorem mimics topological proofs
of Arzela--Ascoli type theorems (which we haven't included): to show
that  is compact under assumptions such as those of
Gale's theorem (Section~\ref{arzela:compact}), one first concludes
that  is compact by the Tychonoff theorem, then
shows that the relative topology of  is the topology of pointwise
convergence, and that it is pointwise closed, and hence concludes that
it is homeomorphically embedded into the product as a closed subset,
and therefore that it must be compact. In the proof below, we have
replaced the Tychonoff theorem by its countable computational version
given by Theorem~\ref{searchable:tychonoff}, using a dense sequence of
the exponent. The first steps of the proof are needed in order to
make this replacement possible, and they are modifications of the
constructions developed in Section~\ref{characterization}.




\begin{proof}
() (1): Theorem~\ref{ex:main}.

\medskip

(2): Define .

\medskip

(): By Theorem~\ref{exhaustible:arzela:topological}, the
set  is topologically exhaustible, and hence is Kleene--Kreisel
compact by Lemma~\ref{lemma:criterion}. This compactness conclusion is
our only use of Theorem~\ref{exhaustible:arzela:topological} in this
proof. We apply this to establish the correctness of the algorithms
defined below.

Define  by , as
in the proof of Theorem~\ref{ex:main}, where  is a
computable dense sequence, and let .  Because
 is decidable on , the set  is
exhaustible by Proposition~\ref{prop:intersec}, and  is
exhaustible uniformly in ,  and  by
Proposition~\ref{image} applied to evaluation at~.
Now modify the definition of 
given in Theorem~\ref{ex:main} as follows:

Then  is computable, and satisfies

Hence it also satisfies

This shows that  for  as defined in Theorem~\ref{ex:main}.
But notice that, although the second and third equations hold, the
algorithm is not the same as in Theorem~\ref{ex:main}.  In fact, the
second and third equations don't establish computability of ,
because exhaustibility of  and  are not known at this stage of
the proof. In any case, the last equation shows that  exhibits 
as a retract up to total equivalence, using the fact that , being
the continuous -image of , is topologically exhaustible and
hence is Kleene--Kreisel compact, as in Theorem~\ref{ex:main}

Similarly, modify the definition of  in Theorem~\ref{ex:main} as follows:

where
 is the least number such that 
Because this condition is equivalent to

such a number exists by Lemma~\ref{c} and the compactness of the
  shadow of~. By uniform exhaustibility of the
  set~, this can be found uniformly in 
  and , and hence  is computable. Moreover, although the
  definition of  is not the same, as before, we again have 
  for  defined as in Theorem~\ref{ex:main}.

  Finally, because  for , the set 
  is exhaustible uniformly in  total, and hence the set  is searchable uniformly in .  In fact, each  is searchable
  uniformly in , by Theorem~\ref{ex:main}, and hence  is
  searchable by Theorem~\ref{searchable:tychonoff}. Now  and hence the entire -image of  is , and hence  is
  searchable by Proposition~\ref{image:bis}. In turn  is the entire
  -image of  and hence is also searchable.  Therefore it is
  exhaustible.
\end{proof}

Notice that the proof actually concludes that  is searchable, and
hence we could have formulated the theorem as: An entire set  is searchable iff  is co-semi-decidable and
the set  is exhaustible for any  decidable on ,
and any  total, uniformly in~ and~.  But this
strengthening of the theorem follows from the given formulation and
the results of Section~\ref{characterization}. However, we could have
included the above theorem, with the stronger formulation, before
Section~\ref{characterization} and then derived the results of that
section as a corollaries. But we feel that the developments of both
sections become more mathematically transparent with the current
organization of the technical material.

\section{Technical remarks, further work, applications and directions} \label{technical}

We now discuss some technical aspects of the above development,
announce some results that we intend to report elsewhere, and discuss
potential applications and directions for future work in this field.

\subsection{Analysis of the selection functional given by the product functional}

\newcommand{\eval}{\operatorname{eval}}

Using course of values induction, one easily sees that a functional
\licsmath{\Pi \colon ((\D \to
  \pBool) \to \D)^\omega \to ((\D^\omega \to \pBool) \to \D^\omega)}
satisfies the equation of Definition~\ref{product:functional} if and
only if it satisfies the equation

and where  and .  Now
define a selection function  for  by

and a selection function  for the Cantor space  by

Then  satisfies the equation

An interesting aspect of this selection function for the Cantor space
is that it doesn't perform case analysis on the value of~, and so,
in some sense, it doesn't work by trial and error.

In order to understand this, first notice that the above recursive
definition of  makes sense if the domain of booleans is replaced
by any domain  with an element :

We consider the case in which~ is the domain of possibly
non-well-founded -branching trees with leaves labelled by .
We define this as the canonical solution of the domain equation
 where the sum is lifted.  Thus, a
tree is either , or else a leaf , or else an unlabelled root
followed by a forest of countably many trees.  Denote the
canonical isomorphism by

Then  and the forest 
gives a general formula for solving  with 
ranging over~.  In fact, for any given , define an evaluation function  by

Equivalently,  is the unique homomorphism
from the initial algebra  to the
algebra .  Hence the
solution  of the equation  is given by evaluating
the general solution  at :
 We illustrate this with finite
forests.  Any  defined on 
is uniformly continuous, and hence of the form
 for some~ and for  defined by this equation.  Now consider
the domain  of -branching trees,
 and denote the canonical isomorphism by

To make sense of the above definition of  for this choice of ,
define  for  and . We tabulate some forests, which grow doubly exponentially, but
only exponentially if auxiliary variables are used to denote common
subtrees (corresponding to the variable  in the recursive
definition of~):

where

In order to find  such that  holds, we substitute  for  in the above equations,
compute , and check whether 
holds. If it does, then we have found a solution (in fact the largest
in the lexicographic order), and otherwise we conclude that there is
no solution.  Thus, the forest  gives a closed formula for
solving the equation , and telling whether there is a
solution, composed only from  and the constant~.  To solve
, just replace  by  in the formula.

\subsection{Solution of equations with exhaustible domain}

By definition, a set  is searchable iff for every
predicate  defined on  one can find , uniformly in , such that if the equation  has a
solution , then  is a solution.  We first observe that
this is equivalent to requiring that for every function  defined on  and any total  one can find , uniformly in  and , such that if the equation  has a
solution , then  is a solution. For one direction,
consider the predicate , and, for the other,
consider the natural inclusion of  into  (which is the
identity under our notation). Clearly, this generalizes from  to
any domain  with  discrete in the sense of
Definition~\ref{discrete:compact}. But notice that in this case the
equation has to be written in the form .

\pagebreak[4]
It is natural to ask whether this generalizes to functions  with  arbitrary. But it is known that, in general, if an
equation has more than one solution, it is typically not possible to
algorithmically find some solution~\cite{beeson}.  We announce the
following result: \pagebreak[3]
\begin{quote} \em Let  and  for types  and , let  be an
  exhaustible entire set,  be total and 
  be total.
\begin{enumerate}
\item If the equation  has a solution , unique
  up to total equivalence, then some  is computable,
  uniformly in ,  and any universal quantification functional
  for~.
\item It is semi-decidable whether  doesn't have a
  solution , with the same uniformity condition.
\end{enumerate}
\end{quote}
In order to establish this, we prove the following generalization of
Lemma~\ref{c}: Let  be a sequence
of entire sets such that  and that  is the equivalence class of some total~. Then one can find a
computable total function , uniformly in any sequence of
universal quantification functionals for~.

This is not very useful in computable analysis via representations,
because typically uniqueness, when it holds, is only up to
equivalence of representations rather than total equivalence. But we
do have a corresponding result for equations involving real numbers.
In light of the following, it is natural to ask whether there is a
further corresponding result for real valued functions of real
variables.

\subsection{An exhaustible set of analytic functions}
An application of the exhaustibility of the Cantor space to the
computation of definite integrals and function maxima has been given
by Simpson~\cite{simpson:integration}. A generalization of this is
developed by Scriven~\cite{scriven}. We consider computation with real
numbers via admissible Baire-space
representations~\cite{weihrauch:analysis} and domain
representations~\cite{blanck}.  For any  and
any sequence , the Taylor series 
converges to a number in the interval .
We announce the following example of a searchable, and hence
exhaustible, set:
\begin{quote}
  \em For any real number , the set  of analytic
  functions 
  
has a searchable set of representatives, uniformly in .
\end{quote}
In our proof of this, we argue that any  can be computed
uniformly in its Taylor coefficients and use the fact that
  has a searchable set of representatives.
This can be used to deduce that:
\pagebreak[3]
  \begin{enumerate}
  \item The Taylor coefficients of any  can be computed
        uniformly in~.

  \item The distance function  defined by
     is computable (cf.\
    Bishop's notion of locatedness~\cite{MR36:4930,bishop:bridges}).

  \item For any , it is semi-decidable, uniformly in ,
        whether . 
  \end{enumerate}

\subsection{Peano's theorem}

This celebrated theorem asserts that certain differential equations
have solutions, but without indicating what the solutions might look
like.  Its proofs are typically based on the Arzela--Ascoli theorem,
and proceed by applying Euler's algorithm to produce a sequence of
approximate solutions. In general, however, this sequence is not
convergent, but, by an application of compactness, there is a
convergent subsequence, although no specific example is exhibited by
this argument, which is then easily seen to produce a solution of the
equation.  It is therefore natural to ask whether our tools could be
applied to compute unique solutions of such differential equations
under suitable assumptions.  Here the goal is not to obtain a usable
algorithm, but rather to understand the classical proof from a
computational perspective in connection with the notion of
exhaustibility and its interaction with the notion of compactness and
with the Arzela--Ascoli theorem.

\subsection{Uncountable products of searchable sets}

It is natural to ask whether the countable product
theorem~\ref{searchable:tychonoff} can be generalized to uncountable
index sets.  This question is pertinent in view of well known
constructive versions of the Tychonoff theorem in locale
theory~\cite{MR641111} and formal topology~\cite{MR1150923}, which
don't restrict the cardinality of the index set.  However, this seems
unlikely in the realm of Kleene--Kreisel higher type computability
theory.  Consider the case in which the index set is the Cantor space.
By the classical Tychonoff theorem, the product of -many
copies of~ is compact.  This product could be written
as~. But this notation in higher-ype
computation is interpreted as a function space, and in the category of
Kleene--Kreisel spaces one has , because
the base is discrete and the exponent is compact (cf.\
Theorem~\ref{thm:discrete:compact}).  This phenomenon in fact also
takes place in the categories of locales~\cite{hyland:functionspaces}
and topological spaces~\cite{escardo:barbados}. In the Tychonoff
theorem for locales or spaces, the indices form a set or equivalently a
discrete space. But a discrete Kleene--Kreisel space is countable (and
more generally a discrete QCB space is countable).

\subsection{Totality of the product functional and bar recursion}

\newcommand{\CBR}{\operatorname{CBR}}
\newcommand{\MBR}{\operatorname{MBR}} In
Theorem~\ref{searchable:tychonoff} we constructed a computable
functional  such that 
whenever~ is a selection functional for a set  and  is
defined on . It is natural to ask whether the
functional~ is actually total. Paulo Oliva has shown that this is
indeed the case (personal communication). Moreover, he has observed
that if the type of booleans is replaced by the type of natural
numbers, our recursive definition of  still makes sense and that
it also gives rise to a total functional, which he calls 
(course-of-values bar recursion). He additionally proved that 
is primitively recursively inter-definable with the modified bar
recursion functional  defined in~\cite{berger:oliva:mbr}. We are
currently investigating together the ramifications of these
observations.

\subsection{Alternative notions of exhaustibility}

If one is interested only in total functionals and sets of total
elements, it is natural formulate the following alternative notion of
exhaustibility: A set  is \emph{entirely exhaustible}
if it is entire and there is a total computable functional  such that for every total  one has  iff  for all . Because, as we have seen, non-empty, exhaustible entire sets
are computable retracts, it follows that any exhaustible entire set is
entirely exhaustible. The converse fails (but see the next paragraph),
because e.g.\ any dense subset of the Cantor space is entirely
exhaustible using Berger's algorithm and the fact that any total
predicate is uniquely determined, up to total equivalence, by its
behaviour on a dense of set of total elements.

Moreover, when one is only interested in total functions and total
elements, it is perhaps more natural to work with Kleene--Kreisel
spaces directly, without the detour via domains, e.g.\ defined as
-spaces.  QCB spaces are a natural and general setting for such
considerations~\cite{MR2328287,MR1948051}.  One might say that a
subset  of a space~ is \emph{totally exhaustible} if there is a
computable functional  such
that for every , we have that
 iff  for all .  When e.g.\
, total exhaustibility of  doesn't entail
compactness of~, again considering the example of a dense subset of
the Cantor space. But Matthias Schr\"oeder (personal communication in
2006) proved that if  is a QCB space which is the sequential
coreflection of a zero-dimensional Hausdorff space, then any totally
exhaustible \emph{closed} set  is compact. This includes
the case in which  is a Kleene--Kreisel space. Using this and the
above observations, one can show that, as far as higher-type
computation with total continuous functionals is concerned, the
notions of exhaustibility and total exhaustibility agree for closed
sets.

\subsection{A unified type system for total and partial computation}

As we have already discussed, Kleene--Kreisel spaces and Ershov--Scott
domains live together in the cartesian closed category of compactly
generated spaces, and in fact in the subcategory of QCB spaces.
Additionally, the inclusions of -spaces and of Ershov--Scott
domains into these categories preserve the cartesian-closed
structure~\cite{escardo:lawson:simpson,MR2328287}.  Hence total and
partial higher-type functionals coexist in the same cartesian closed
category.  One can envisage a higher-type system that simultaneously
incorporates, but explicitly distinguishes, total and partial objects,
and corresponding PCF-style formal systems. Among the formation rules
one can have two types for the natural numbers, with and without
, and it would make sense to stipulate that  is
a partial type whenever  is any type and  is a partial
type, and that  is a total type when both 
and  are total types. In its simplest form, such a language
could include G\"odel's system  for total types and PCF for partial
types. Such a formalism would have simplified, and made more
transparent, much of the development concerning exhaustible sets of
total elements, where we could have benefited from functionals that
take total inputs and produce potentially partial outputs.  In
particular, all the technical considerations of total equivalence and
shadows could have been avoided in this way, making the development
more transparent. Such functionals are actually total, but their
construction uses modes of definition that belong to the realm of
partial computation. The system- fragment could be further extended
with total computable functionals such as bar recursion and some of
those developed here, once one has shown they are indeed total.

\subsection{Time complexity of exhaustive search}

In the paper~\cite{escardo:lics07}, we report some surprisingly fast
experimental results, which serve to counteract an impression that
might be gained from the technical development that the algorithms
presented would be essentially intractable and of purely theoretical
interest. Moreover, that paper formulates run-time conjectures that
provide examples of questions that one would like to be able to treat
rigorously and that are potentially useful as target problems for work
in higher-type complexity theory.  The conjectures express the run
time in terms of the modulus of uniform continuity of the input
predicate on the exhaustible set, and hence topology seems to play a
role in higher-type complexity too.

It might be possible to apply our search algorithms to practical
problems, e.g.\ in real analysis and in program verification.  But it
is more likely that, in order to obtain feasible algorithms, such
applications will need to rely on the development of particular
algorithms for particular kinds of infinite search tasks, perhaps
inspired or guided by the general algorithms we have developed, but in
any case needing new insights and techniques.  In fact, this is
already the case for finite search problems, as is well known.  But
the fast examples reported in~\cite{escardo:lics07} do highlight that
the task of obtaining particular search algorithms that are efficient
for particular kinds of infinite search problems of interest is a
direction of research that deserves attention and is likely to be
fruitful, and that a study of feasible infinite search problems cries
to be carried out.

\subsection{A fast product functional}

We have just discussed that one should look for efficient search
algorithms for specialized problems.  But it is still interesting to
ask how fast a general infinite search algorithm can be.  We don't
know the answer, but we report an algorithm that outperforms all the
algorithms applied for the experimental results
of~\cite{escardo:lics07}, and whose theoretical run-time behaviour
remains to be investigated.

\newcommand{\Root}{\operatorname{root}}
\newcommand{\Left}{\operatorname{left}}
\newcommand{\Right}{\operatorname{right}}
\newcommand{\branch}{\operatorname{branch}} 

We regard an infinite sequence~ as an infinite binarily
branching tree with the elements of the sequence organized in a
breadth-first manner: the root is , and the left and right
branches of the node  are  and .  With this
in mind, define functions
\begin{quote}
  , \,\, , \,\, 
\end{quote}
by


Then, for  and ,


Our experimentally faster product algorithm is then recursively defined by

where



The idea is that treating sequences as trees reduces some linear
factors to logarithmic factors (very much like in the well-known
heap-sort algorithm).


\subsection{Operational perspective}
An advantage of the proof of Theorem~\ref{searchable:tychonoff}
sketched in~\cite{escardo:lics07} is that it can be directly
interpreted in the operational
setting~\cite{escardo:barbados,escardo:ho}.  The proofs of the other
results of Section~\ref{building} are also easily seen to work in the
above operational setting.  But a development of operational
counter-parts for those of later sections is left as an open problem.
This requires an operational reworking of the topological
Section~\ref{criteria}, which seems challenging.

\section{Concluding remark on the role of topology} 
\label{conclusion}

The algorithms developed in this work have purely computational
specifications, which allow them to be applied without knowledge of
specialized mathematical techniques in the theory of computation.
However, the correctness proofs of some of the algorithms crucially
rely on topological techniques. In this sense, this work is a genuine
application of topology to computation: theorems formulated in the
language of computation, proofs developed in the language of topology.

But there is another sense in which topology proves to play a crucial
role.  Compact sets in topology are advertised as sets that behave, in
many important respects, as if they were finite.  Then exhaustively
searchable sets \emph{ought} to be compact.  And compact sets are
known to be closed under continuous images and under finite and
infinite products.  Moreover, for countably based Hausdorff spaces,
they are the continuous images of the Cantor space.  Hence searchable
sets \emph{ought} to have corresponding closure properties and
characterization, which is what this work establishes, among other
things, \emph{motivated} by these considerations.  Thus, in a more
abstract level, topology is applied as a paradigm for discovering
unforeseen notions, algorithms and theorems in computability theory.




\nocite{smyth:topology}
\bibliographystyle{plain}
\bibliography{exhaustive-journal}

\vskip-50 pt







\end{document}
