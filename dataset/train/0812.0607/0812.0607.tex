\documentclass[10pt, conference, compsocconf]{IEEEtran}
\pdfoutput=1

\usepackage{graphicx}
\usepackage{hyperref}
\usepackage{cite}
\usepackage{wrapfig}
\usepackage{amsfonts}

\def\Real{\ensuremath{\Bbb R}}
\def\Z{\ensuremath{\Bbb Z}}
\def\symdiff{\mathbin{\triangle}}

\newtheorem{theorem}{Theorem}
\newtheorem{proposition}[theorem]{Proposition}
\newtheorem{lemma}[theorem]{Lemma}

\begin{document}

\title{Planar Voronoi Diagrams for  Sums of Convex Functions,\\ Smoothed Distance, and Dilation} 

\author{\IEEEauthorblockN{Matthew Dickerson}
\IEEEauthorblockA{Department of Computer Science\\
Middlebury College\\
Middlebury, Vermont, USA\\
Email: dickerso@middlebury.edu}
\and
\IEEEauthorblockN{David Eppstein}
\IEEEauthorblockA{Computer Science Department\\
University of California, Irvine\\
Irvine, California, USA\\
Email: eppstein@uci.edu}
\and
\IEEEauthorblockN{Kevin A. Wortman}
\IEEEauthorblockA{Department of Computer Science\\
California State University, Fullerton\\
Fullerton, California, USA\\
Email: kwortman@fullerton.edu}
}

\maketitle

\begin{abstract}
We study Voronoi diagrams for distance functions that add together two convex functions, each taking as its argument the difference between Cartesian coordinates of two planar points. When the functions do not grow too quickly, then the Voronoi diagram has linear complexity and can be constructed in near-linear randomized expected time. Additionally, the level sets of the distances from the sites  form a family of pseudocircles in the plane, all cells in the Voronoi diagram are connected, and the set of bisectors separating any one cell in the diagram from each of the others forms an arrangement of pseudolines in the plane. We apply these results to the smoothed distance or biotope transform metric, a geometric analogue of the Jaccard distance whose Voronoi diagrams can be used to determine the dilation of a star network with a given hub. For sufficiently closely spaced points in the plane, the Voronoi diagram of smoothed distance has linear complexity and can be computed efficiently. We also experiment with a variant of Lloyd's algorithm, adapted to smoothed distance, to find uniformly spaced point samples with exponentially decreasing density around a given point.
\end{abstract}

\begin{IEEEkeywords}
biotope transform metric;
convex function; dilation; Lloyd's algorithm;
pseudocircle; pseudoline; randomized algorithms
smoothed distance; Voronoi diagram.
\end{IEEEkeywords}

\section{Introduction}

Any bivariate function  and finite set of point sites  in the plane give rise to a \emph{minimization diagram}  in which the cell for site  consists of points  such that the value of the translated function  is less than or equal to the value of any of the other translates of . A familiar example is given by Euclidean Voronoi diagrams, the minimization diagrams of translates of the convex functions  (or, equivalently, ) that measure the (squared) distance of  from the origin.

Minimization diagrams have many applications, and typically have quadratic complexity~\cite{HalSha-DCG-94}, but in some important cases their complexity is much smaller.
In a Euclidean Voronoi diagram, each cell is a convex polygon, so the Voronoi diagram for  sites partitions the plane into  connected regions and has  vertices and edges. These combinatorial facts form the basis of efficient algorithms for constructing Euclidean Voronoi diagrams and using them in other geometric algorithms. It is natural, therefore, to ask: Which other minimization diagrams have connected cells and a linear number of number of features?

A partial answer was provided by Chew and Drysdale~\cite{CheDry-SoCG-85}:  Voronoi diagrams for \emph{convex distance functions} have connected cells and linear complexity. A convex function  is a convex distance function if, for any positive scalar  and any point , . Not every convex function has this property: it implies that the \emph{level sets}  are all similar, and that the function is linear on rays from the origin, neither of which are true for all convex functions. Another answer is given by the \emph{abstract Voronoi diagrams}~\cite{Kle-89}  defined by a family of bisector curves that are required to intersect each other finitely many times and form simply connected cells. Bregman Voronoi diagrams~\cite{NieBoiNoc-SODA-07} fall into this class: they have linear bisectors and convex-polygon cells. Abstract Voronoi diagrams may be constructed efficiently~\cite{Kle-89,KleMehMei-CGTA-93,MehMeiODu-DCG-91} but it is unclear how to tell whether a given convex function has minimization diagrams that form abstract Voronoi diagrams.

In this paper we study minimization diagrams for another class of convex functions, different from the convex distance functions. The functions we study take the form , where  and  are triply-differentiable and  and  for all  and . For example, squared Euclidean distance has this form with  and . More generally,  (for ) satisfies these requirements,\footnote{The divergence of the double derivative of  at the origin when , and its non-differentiability there when , do not present any serious difficulties to our theory.} and its minimization diagrams are the Voronoi diagrams for  distance, known to have linear complexity~\cite{CheDry-SoCG-85}. We show in Section~\ref{sec:mdcf} that the minimization diagrams of arbitrary functions in this class have analogous properties:
\begin{itemize}
\item Any two \emph{level sets}  are simple closed curves that intersect in at most two points; if they intersect at two points, they cross properly at these points. That is, these sets, which are defined analogously to Euclidean circles, form a family of \emph{pseudocircles}~\cite{KedLivPac-DCG-86,LinOrt-BAG-05}.
\item Any \emph{bisector}  forms either an axis-parallel line or a simple curve that is both -monotone and -monotone;  and  vary unimodally along the bisector .
\item Any two bisectors  and  intersect in at most one point; if they intersect, they do so at a proper crossing. That is, if we fix  and let  vary, the bisectors  form a \emph{weak pseudoline arrangement}~\cite{Epp-GD-04,EppFalOvc-07,FraOss-GD-03}, and the correspondence between  and  can be viewed as a form of duality for these pseudolines.
\item Each cell of the minimization diagram is simply connected, as it is a single cell in a pseudoline arrangement. Thus, the minimization diagram has linear complexity and, like Voronoi diagrams for convex distance functions and abstract Voronoi diagrams,\footnote{We do not show that these diagrams satisfy all requirements of an abstract Voronoi diagram, as we do not bound the number of crossings between bisectors for unrelated pairs of points.} can be constructed in randomized expected time  (Theorem~\ref{thm:voronoi}).
\end{itemize}
In Section~\ref{sec:bad-examples} we provide examples showing that our restrictions are necessary: minimization diagrams for more general convex functions do not in general have these properties.

Our motivation for studying this type of minimization diagram comes from \emph{smoothed distance}. Given a fixed point  in the plane, the \emph{smoothed distance} or \emph{biotope transform metric} ~\cite{Cla-UCI-08,DezDez-09} is a geometric analogue of the Jaccard similarity measure for clustering binary data.  A maximal set of points with a given minimum smoothed distance will be distributed around  with exponentially decreasing density~\cite{Cla-UCI-08}, but as we show in Section~\ref{sec:lloyd}, the point spacing may be improved by using smoothed distance in Lloyd's algorithm~\cite{DuFabGun-SR-99,Llo-ITIT-82}, a continuous variant of -means clustering that repeatedly moves each site to the centroid of its Voronoi cell.

Smoothed distance is a monotonic function of the \emph{dilation} , a measure of the quality of a star network as an approximation to the Euclidean distances among a set of points. For the maximum dilation pair of points  from a set  for a fixed center ,  is among the  nearest neighbors to  either in Euclidean distance or in the sequence of points sorted by distance from , and this result forms the basis of an efficient algorithm for finding the center  that minimizes the maximum dilation~\cite{EppWor-CGTA-07} . However, this can be simplified using smoothed distance: the maximum-dilation pair must be neighbors in the smoothed distance Voronoi diagram (Proposition~\ref{prop:dilation-neighbors}).

Neither smoothed distance nor dilation are translates of convex functions. However, in Section~\ref{sec:xform} we use complex logarithms to transform the plane and we perform a suitable monotonic transformation of smoothed distance, showing that Voronoi diagrams for smoothed distance are equivalent to minimization diagrams for the convex function

As we require, this function is the sum of univariate functions  and , with . It is not true that  for all , but it is true for , so smoothed distance Voronoi diagrams are well behaved whenever each Voronoi cell spans an angle of at most  on each side of its site with respect to~ (Theorem~\ref{thm:smoothed}).

\section{Dilation, Smoothed Distance, and Logarithmic Transformation}
\label{sec:xform}

Two important measures of similarity between sets  and  are the Hamming distance  (where  denotes the symmetric difference of sets) and the Jaccard distance~\cite{Jac-BSV-01}, which weights the Hamming distance by the size of the union of the sets:

A monotone transformation of this distance lies in the interval :

The same formula defining the (modified) Jaccard distance in terms of the Hamming distance can be used to derive a new metric from any given metric space  and fixed point . Define the \emph{-smoothed distance} or \emph{biotope transform metric} by the formula


This is a metric on , as can be shown using the tight span~\cite{Isb-CMH-64,Dre-AiM-84}, the minimal -like metric completion of a metric space.
Any four points  may be embedded isometrically into a metric space formed by an axis-aligned rectangle of the  plane (possibly a degenerate rectangle), together with four line segments (possibly of length zero) connecting the corners of this rectangle to the four points. When the four line segments all have length zero, the four points are in cyclic order (, , , ) on the rectangle's corners, and the rectangle has aspect ratio , then the triangle inequality for  is satisfied exactly: . Increasing the length of the line segments connecting the four points to the rectangle or placing the points in a different cyclic order only strengthens the triangle inequality.

For the points within a -ball of radius  around a point , the factor  in the definition of  ranges from  to , varying by a factor of approximately  within this range as  approaches zero. Thus, closely spaced sets of points in the smoothed distance have distances that are approximately similar to their unsmoothed distances. As a metric space on , the smoothed distance  is  topologically equivalent to the metric induced by  on the same space: both distances have the same open sets and neighborhood structures.

Smoothed distance  for the Euclidean plane is invariant with respect to rotations and scaling centered at . These transformations may be expressed by representing point  as the complex number , with : if  is any complex number, then the product  may be interpreted geometrically as rotating the point  by the angle  and scaling the rotated point by . With this interpretation,
\begin{wrapfigure}{r}{0.23\textwidth}
\centering\includegraphics[width=0.23\textwidth]{polarcircles}
\caption{Concentric circles for smoothed distance in the Euclidean plane, with , , and radii 0.5, 0.6, 0.7, 0.8, and 0.9}
\label{fig:polarcircles}
\end{wrapfigure}
.

As Figure~\ref{fig:polarcircles}
shows, smoothed-distance circles with smaller radius closely resemble Euclidean circles, while larger-radius smoothed-distance circles are not even convex.

Smoothed distance is closely related to \emph{dilation}, a measure of the quality of a graph as an approximation to a metric space~\cite{Epp-HCG-00}. The dilation of vertices  and  is the ratio of their distances in the graph and in the ambient space; the dilation of the graph is the largest dilation of any pair of vertices. For a star graph  with center  and all other vertices as leaves~\cite{EppWor-CGTA-07}, the dilation of pair  is

and the dilation of the whole star graph is


Because dilation is a monotonic transformation of smoothed distance, the point  that is nearest to a query point  in terms of smoothed distance is also the point that has the greatest dilation with . The Voronoi diagram for smoothed distance partitions the plane into regions such that the region containing a query point  corresponds to the point  for which a path from  will have to make the greatest detour by passing through  instead of connecting directly. The adjacency relations between cells in this Voronoi diagram are also meaningful for dilation:

\begin{proposition}
\label{prop:dilation-neighbors}
The points  and  defining the dilation of a star graph have adjacent cells in the Voronoi diagram for smoothed distance, using the star's leaves as Voronoi sites and its hub as the point~.
\end{proposition}

\begin{proof}
Form the Voronoi diagram for  and then add  to form the Voronoi diagram of . Let  be the Voronoi region of  prior to adding .
 must contain , because otherwise some other point than  would define the greatest dilation with respect to . After adding , part of  becomes incorporated into the Voronoi region for , while the rest of  (in particular the point  itself) remains in the region of . Thus, these two regions meet.
\end{proof}

Thus, we would like to understand and compute Voronoi diagrams for smoothed distance. Smoothed distance is neither convex nor translation-invariant, but we can transform it into an equivalent distance that is, by interpreting polar coordinates in the plane for which we are computing smoothed distance as Cartesian coordinates in a transformed plane. Equivalently, using complex numbers (with ), if complex number  has polar coordinates , then  has Cartesian coordinates . Define a distance  on the transformed complex number plane by the formula

This logarithmic transformation replaces the invariance of  under complex multiplication by invariance of  under complex addition (that is, translation of the plane):

Figure~\ref{fig:XYcircles} shows concentric circles for~.

\begin{figure}[t]
\centering\includegraphics[height=1.25in]{XYcircles}
\caption{Concentric circles for the logarithmically transformed distance  with radii 0.5, 0.75, 0.9, 0.99, and 0.999.}
\label{fig:XYcircles}
\end{figure}

We may calculate  directly as a formula of  and :
it equals , where , , and .
Thus, , , and

Therefore, the smoothed distance is

It is convenient to monotonically transform this formula:

and

Therefore, if we define

it follows that

Thus, we have represented smoothed distance as a monotonic transform of  translates of , where

Graphs of these two functions are depicted in Figure~\ref{fig:components}.

\begin{figure}[t]
\centering\includegraphics[width=1.5in]{Ycomponent}
\qquad\includegraphics[width=1.5in]{Xcomponent}
\caption{Graphs of the functions  and  used to represent the transformed value of the smoothed distance.}
\label{fig:components}
\end{figure}

\section{Minimization Diagrams of Convex Functions}
\label{sec:mdcf}

The transformation in Section~\ref{sec:xform} from a Voronoi diagram of smoothed distance to a minimization diagram of translates of  motivates the more general study of minimization diagrams of this type of function.
A univariate function  is \emph{convex} if the set of points 

on or above  is a convex subset of the plane.  In higher dimensions, a multivariate function  is convex if the set

is a convex subset of . 
A convex univariate function is \emph{strictly convex} if there is no interval within which it is linear. For a doubly-differentiable univariate function , convexity is equivalent to the inequality  and strict convexity is equivalent to the inequality . A curve in the -plane is \emph{-monotone} if every line parallel to the  axis intersects it in at most one point; intuitively, these curves are the graphs of functions from  to . Symmetrically, a curve is \emph{-monotone} if every line parallel to the  axis intersects it in at most one point. These concepts should be distinguished from that of a function being \emph{monotonically increasing} or \emph{strictly monotonically increasing}: if , and  is monotonically increasing, then ; if   is strictly monotonically increasing, then . In a minimization diagram of translates of a function , a \emph{bisector}  of two distinct points  and  is the locus of points with equal values of  and : . The minimization diagram of the points  is formed by partitioning the plane into two cells along the bisector.

\begin{proposition}
\label{prop:monotone-bisectors}
Let , where  and  are strictly convex. Then any bisector  is either an axis-parallel line or an -monotone and -monotone curve.
\end{proposition}

\begin{proof}
If  and  have equal -coordinates, then  whenever ; since this condition is independent of , the bisector must be the union of one or more lines parallel to the  axis. Otherwise, any line  parallel to the  axis intersects the bisector  in the points  such that . By strict convexity, the function  is strictly monotonically increasing, so there is at most one such point and the bisector is -monotone.
Symmetrically, if  and  have the same -coordinate, the bisector must be the union of one or more lines parallel to the  axis, and otherwise it is -monotone. The only curves that are simultaneously -monotone or a union of -parallel lines, and -monotone or a union of -parallel lines, are the ones listed in the proposition: a single axis-parallel line, or a doubly-monotone curve.
\end{proof}

In particular, each bisector must be a \emph{pseudoline} (the image of a line under a homeomorphism of the plane\footnote{This definition is from~\cite{Sho-VKF-91}; see~\cite{EppFalOvc-07} for a comparison with other common definitions of pseudolines.}), because it is either itself a line or a monotonic curve that partitions the plane into two cells. In the next sequence of lemmas, we show that the level sets of the translates of  are \emph{pseudocircles} (simple closed curves that cross each other at most twice) by comparing their curvatures at tangent points of the same slope. For convenience, when the argument of the functions  and  is specified in the context, the notations , , , , etc., refer to the values , , , , etc.

\begin{lemma}
Let  where  and  are convex and differentiable.
Let   be any point other than the global minimum of , let , and let  be the level set .
Then the slope of the tangent to  at  is .
\end{lemma}

\begin{proof}
Since  is not the minimum,  is a simple closed curve.
Therefore there is a unique tangent line which is characterized by the fact that, at points  on the line at a small distance  from , the difference between  and  vanishes to first order. A point  at a distance proportional to  on the line of slope  can be found by adding  to ; at this point,  is , so the difference vanishes to first order as desired.
\end{proof}

Next, suppose we move  units along the curve  from ; to within
first order, the point we move to is found by adding
 to . When we do so, 
changes by a factor of
,
and  changes by a factor of , so the slope 
changes by a factor of . We may view the term  appearing in this expression as a local measure of the
curvature of ; it is not rotation-invariant but can be used to compare
the curvatures of two curves at points of equal tangent slope. Larger
values of this term mean a smaller radius of curvature and lower values mean a greater
radius. To show that the radius of curvature at a given slope increases as  increases, we will examine the behavior of this function as we increase  by a small quantity  while moving  along a curve that keeps the slope of the tangent to  fixed.

\begin{lemma}
Let  where  and  are convex and twice-differentiable.
At any point  for which , define the curve  through  to consist of points with the same value of  as at . Then the tangent line to  at  has slope .
\end{lemma}

\begin{proof}
Move  by a distance proportional to some small  along this line, by adding the vector  to . As  moves,  and  both change by factors of . Both factors are equal to within first order, so the slope  does not change to first order and we have identified the correct tangent line to .
\end{proof}

\begin{lemma}
\label{lem:curvature}
Let  where  and  are convex and triply-differentiable, and where  and  for all  and . Then, along any curve , the radius of curvature of the curves  at the points where these curves are crossed by  is a monotonically increasing function of .
\end{lemma}

\begin{proof}
We assume without loss of generality that , , , and  are all positive at , for otherwise we may achieve these assumptions by translating the plane, reflecting it across one or both of the coordinate axes, or adding a constant to , without changing the truth of the lemma.
As in the previous lemma, we move at a distance proportional to  along , to within first order, by adding  to . And as in the previous lemma, this motion causes  and  to both change by a factor of ; this same factor also describes the change in . Thus, to find the direction of change to
the value  that we are using to compare curvatures for a given tangent slope, it only remains to evaluate the change to  and . The double derivative , and therefore the term  appearing in the numerator of our local curvature function, changes by a factor of
; if , this is smaller than the factor of  describing the change to the denominator of our local curvature function. The double derivative , and therefore the term  appearing in the numerator of our local curvature function, changes by a factor of ; again, if , this is smaller than the factor of  describing the change to the denominator of our local curvature function. Thus, if the assumptions of the lemma are met,  decreases and the radius of curvature increases as  increases along .
\end{proof}

Due to the convexity of , and the strict convexity of  and , the level sets  are themselves convex; in particular, they are simple closed curves in the plane and, as the following proposition shows, they form a family of \emph{pseudocircles} in the plane.

\begin{proposition}
\label{prop:pseudocircles}
Let  be a convex function such that  and  are triply-differentiable, , and , and let  and  be numbers and  and  be points such that . Then the level sets  intersect in at most two points; if they intersect at two points, they cross properly at these points.
\end{proposition}

\begin{proof}
We suppose that a pair that forms more than two points of intersection or that has two non-crossing intersections exists, and proceed to derive a contradiction. If there are exactly two points of intersection, one of which is not a crossing, then (because both are simple closed curves) both must be points of tangency; we may increase the radius of the inner level set slightly and form two level sets that cross four times, so we may assume without loss of generality that there are three or more points of intersection. If , then again we may change one of the radii slightly while preserving the property of having more than two points of intersection, for this change of radii cannot remove any crossings and can be chosen to turn at least half of the points of tangency into pairs of crossings.
And if some of the points of intersection are tangencies, we may increase or decrease  by a small amount and replace at least half of the tangencies by crossings without changing the property that there are more than two intersection points. Thus, we may assume that the two level sets intersect more than two times at proper crossings.

Now, let  be the set of real numbers  such that  and  have more than two proper crossings; that is, we consider translating  directly away from  for as far as we can while preserving the overly large number of crossings between the two curves. Because both level sets are bounded,  is itself bounded; let  and let .
In order for the translated level sets to have more than two crossings for  but only to have two crossings for , the sets  and  must be tangent. However, this point of tangency cannot be one at which the two level sets cross, because our assumptions imply that both curves have curvature that varies continuously along the curves. It cannot be a tangency in which the curve with the smaller value of  lies inside the curve with the larger value of , because then for  the tangency would become two crossings and  would not be equal to . It cannot be a tangency in which the two curves meet externally, because then by convexity for  the curves would have only two crossing points near the tangency and again  would not be equal to . And it cannot be a tangency in which the curve with the smaller value of  lies outside the curve with the larger value of , because that would violate Lemma~\ref{lem:curvature}. However, these exhaust the possible ways the two curves can be tangent. This contradiction completes the proof.
\end{proof}

The next proposition states that (with the same assumptions as Lemma~\ref{lem:curvature} and Proposition~\ref{prop:pseudocircles}) the bisectors  and  act like pseudolines: they meet in at most one point, and if they meet they cross properly.

\begin{proposition}
\label{prop:pseudolines}
Let  be a convex function such that  and  are triply-differentiable, , and . Then any two bisectors  and  defined from  have at most one point of intersection. If they intersect, they cross properly.
\end{proposition}

\begin{proof}
We show that  and  meet in at most one point by showing that any two points  and  in the plane are intersected by at most one bisector . If there is a bisector  that contains both of these points, then  and  are equidistant from  and from ; that is,  and  both belong to the level sets  and . These level sets are rotated by  from the level sets of Proposition~\ref{prop:pseudocircles} but that rotation does not affect the conclusion of the proposition: they have at most two points of intersection. One of these intersection points is  and the other is ; there can be no third intersection to form another bisector with  through  and . If  and  met in two points, it would violate the uniqueness of bisectors through pairs of points, so such a double intersection cannot happen.

To complete the proof of the proposition, we must show that if two bisectors intersect, they meet in a proper crossing. But if two bisectors  and  met at a point of tangency without crossing, then a small translation of  either towards or away from  would transform this tangency into a pair of crossings, violating the first part of the proposition.
\end{proof}

In other words, if we fix  and let  vary, the family of bisectors  forms a weak pseudoline arrangement. However, the proposition applies only to pairs of bisectors that share one of their defining points. These results are not quite enough to show that minimization diagrams for  are abstract Voronoi diagrams in the sense of Klein~\cite{Kle-89}, because abstract Voronoi diagrams require all bisectors to have a constant number of intersection points. However, the same general results as for abstract Voronoi diagrams follow in this case.

\begin{theorem}
\label{thm:voronoi}
Let  be a convex function such that  and  are triply-differentiable, , and . Then any minimization diagram for , defined by a finite set  of  point sites, subdivides the plane into  simply-connected regions, one per site. The diagram can be constructed in time  using a primitive that finds minimization diagrams for three sites.
\end{theorem}

\begin{proof}
We may assume without loss of generality that the minima of  and  occur at , for otherwise we may add an appropriate constant to  and  without changing the combinatorial description of the minimization diagram. Thus, the cell for any point  contains  itself. In any weak arrangement of pseudolines, any nonempty intersection of halfspaces defined by the pseudolines forms a single cell of the arrangement (e.g. see Theorem~11.4.11 of~\cite{EppFalOvc-07}).
It follows by Proposition~\ref{prop:pseudolines} that the cell of  in the minimization diagram is a single cell in a weak arrangement of pseudolines and is therefore simply connected. Thus, the minimization diagram for  and  consists of  simply connected cells.

We construct the diagram by a standard randomized incremental algorithm in which we add points to  one at a time according to a uniformly random permutation, and maintain a history DAG describing the cells of the diagram in past states of the construction. We also maintain the sequence of bisectors surrounding each cell of the diagram, in a balanced binary tree data structure. To add a point , we use the history DAG to locate the cell of the diagram that contains . We then form a list  of cells known to overlap with the cell of ; initially this list includes only the cell containing . To build the cell for , we repeatedly remove the cell for a site  from , and split this cell along the bisector . The points where this bisector crosses the boundary of the cell for  may be found by a binary search of the boundary, in which each step consists of finding the vertex of the minimization for , , and a third site determining one of the boundary segments, and comparing the  and  coordinates of this vertex to those of the other vertices on the segment. After the part to be removed from the cell of  is determined, the boundary segments in the binary search tree for  are removed and the associated cells are added to  if necessary.

The time to locate  is  in expectation by a standard analysis of history DAGs.
Each new feature of the minimization diagram takes  time to construct, using the binary search trees, and there are in expectation  new features for the th added site. Thus, the total expected time for the construction is .
\end{proof}

\section{Two Bad Examples}
\label{sec:bad-examples}

The assumptions that we make on the form of  as a sum of two univariate convex functions, and on the triple derivatives of these functions, may seem technical and unnecessary. However, in this section we provide examples showing that without the assumption on the form of  beyond convexity, its minimization diagrams may have quadratic complexity, and without assumption on the triple derivatives of  and , the level sets for  may not be pseudocircles.

\begin{wrapfigure}{r}{0.25\textwidth}
\centering\includegraphics[height=1.5in]{expx2}
\caption{The level sets for  (shown for function values , , , , , , and ) do not form pseudocircles.}
\label{fig:expx2}
\end{wrapfigure}

Figure~\ref{fig:expx2} shows an example in which  and  grow so quickly that  and  for sufficiently large .
Specifically, , while ; it follows that  for . One level set has been translated so that it crosses the outer level set four times, so these level sets are not pseudo\-circles.
More generally, whenever  over some interval of values of , the level sets of  do not form pseudocircles: the radius of curvature at one of the tangents with slope  shrinks rather than growing for increasing values of~.

Figures~\ref{fig:diamonds} and~\ref{fig:quadvor} provide a sketch of a construction for a convex function that has minimization diagrams with quadratically growing complexity. The function grows most slowly on a horizontal line through the origin, and more quickly along any other horizontal line, so that for any sites  and  that are not on a horizontal line, the Voronoi cell for  dominates that for  for points far enough away from both sites on a horizontal line through . In particular the vertical line of cells to the right of the figure generates a sequence of cells with horizontal boundaries that extends, despite interruptions, through the whole figure. A more widely spaced sequence of sites at the bottom of the figure interrupts these cells, splitting them into linearly many pieces.

\begin{figure}[hb]
\centering\includegraphics[width=3in]{diamonds}
\caption{Level sets for a convex function the minimization diagrams of which have quadratic complexity.}
\label{fig:diamonds}
\end{figure}

\begin{figure}[hb]
\centering\includegraphics[width=3.25in]{quadvor}
\caption{Sketch of the structure for a minimization diagram of a convex function with quadratic complexity.}
\label{fig:quadvor}
\end{figure}

\section{Voronoi Diagrams for Smoothed Distance}

So, now that we've transformed the smoothed distance Voronoi diagram into the form of a minimization diagram for a function , and now that we know conditions on  and  that ensure that the minimization diagram to have linear complexity and be constructable efficiently, do the specific  and  arising in this application meet these conditions? The answer is: it depends.

First, consider the function , with derivatives
,
, and
.
For positive ,  and therefore  are negative while  is positive, so .
Because the function is symmetric () the same holds for negative .
And at zero,  while  is nonzero. Therefore,  meets the conditions of Theorem~\ref{thm:voronoi}.

Next, consider the function . Its derivatives are
,
, and
.
Then , while .
The ratio of these two values, , is less than one
when  and greater than one when . Thus,  satisfies the requirement that  only for  in the interval . This implies that we may only apply Theorem~\ref{thm:voronoi} to smoothed distance when each Voronoi cell consists of points spanning at most a right angle with  and the cell's site.

\begin{theorem}
\label{thm:smoothed}
Let  be a point set such that, in the Voronoi diagram for -smoothed distance, every point  within the Voronoi cell for a site  forms an angle  that is at most .  Then each cell in the Voronoi diagram is connected, and the diagram may be constructed in randomized expected time .
\end{theorem}

\begin{proof}
To prove this, in outline, we replace the point set  by its logarithmically transformed image ; each point in  corresponds to infinitely many transformed points, all with the same  coordinate and with  coordinates that differ by integer multiples of . Under this transformation, each point in the transformed plane is associated with a Voronoi region that the exponential function maps back to the correct Voronoi region for the associated input point in the original plane. Then, instead of using the functions  and  in the transformed plane, as defined above, we replace the function  by a modified function that has the same values within the interval  but that obeys the inequality  for larger values as well. This replacement allows us to apply Theorem~\ref{thm:voronoi} to the transformed input, and
we show that, with the assumptions stated in the theorem, it produces the same cell decomposition as the one we wish to compute. We use an efficient randomized incremental algorithm to compute the smoothed Voronoi diagram, similar to the algorithm used in Theorem~\ref{thm:voronoi}. 

Smoothed distance may be replaced by the translates of a convex function by logarithmically mapping the sites and monotonically transforming the distance values.
Thus, Voronoi diagrams for smoothed distance are closely related to minimization diagrams for this convex function. The specific relation is this: if we view the points in the plane as complex numbers,
with , and map the finite set  of sites to the infinite vertically-periodic set , then the exponential function forms a covering map from the minimization diagram of  with respect to  to the Voronoi diagram for smoothed distance of . The cells in the minimization diagram are mapped many-to-one to cells in the Voronoi diagram, and edges and features in the minimization diagram are mapped to edges and features in the Voronoi diagram, etc. Although it is problematic to perform geometric algorithms on infinite point sets such as , we may use our analysis to show that cells in the minimization diagram for  are simply connected, while performing the randomized incremental algorithm described in the proof of Theorem~\ref{thm:voronoi} directly on the finite point set .

There is a difficulty with this approach, however: our proof that the cells are simply connected does not apply, because the convex function into which we have transformed smoothed distance does not meets the requirement of Theorem~\ref{thm:voronoi} that the function  satisfies the inequality , at least for large values of . And although the assumptions of Theorem~\ref{thm:smoothed} imply that only small values of  need be considered in the final Voronoi diagram, larger values may be needed at early stages of our randomized incremental construction.

Therefore, rather than computing the Voronoi diagram for smoothed distance  itself, we compute
the Voronoi diagram for a (non-metric) distance function .
Here , matching the monotonically-transformed formula for smoothed distance. However, we set  to be a function that equals  for  but that is a quadratic polynomial for values of  outside this range; the two quadratic polynomials (for positive and negative ) are chosen to have a value, first derivative, and second derivative that matches  at  and . However, being quadratic polynomials, they have third derivative equal to zero, and therefore satisfy the requirement that  for the range of values of  in which they define the value of . The fact that the overall  function has a discontinuous third derivative at  and at  does not cause any problems for our analysis.

With this modified distance function, Theorem~\ref{thm:voronoi} can be used to show that the cells of the minimization diagram for  with respect to the points of  are simply connected, and therefore that their images, the cells of the Voronoi diagram for  with respect to the points of , are also connected. We may then apply a randomized incremental algorithm of the type described in Theorem~\ref{thm:voronoi} to construct the Voronoi diagram for  with respect to the points of . At intermediate stages of this construction, the cells of the diagram may have self-adjacencies (corresponding to boundaries between two different images in  of a point in ) but these need not be treated any differently than any other of the bisectors in the diagram, and must vanish before the algorithm finishes.

It remains to show that each cell in the diagram for the modified distance function is equal to the corresponding cell in the diagram for the true smoothed distance . Observe that, for any site , the cell for  for distance  is contained within the boundaries of the cell for  for distance : by assumption, these boundaries are all within the region of the plane in which  and (the corresponding monotonic transformation of)  coincide, so on those boundary points  is accurately represented by  while the distances from  to other points as measured by  may be underestimates of the true value as measured by . Thus, the cells for the Voronoi diagram for  are subsets of the corresponding cells for the Voronoi diagram of . However, since the cells for the Voronoi diagram for  nevertheless cover the plane, both sets of cells coincide and the computed Voronoi diagram is correct.
\end{proof}

\section{Lloyd's Algorithm}
\label{sec:lloyd}

Evenly spaced points in Euclidean and related metric spaces have applications ranging from coding theory~\cite{Llo-ITIT-82} and color quantization~\cite{Hec-SIGGRAPH-82} to dithering (spatial halftoning) and stippling for image rendering~\cite{MacIseAnd-CG-08,Uli-87}. However, random points  typically have uneven spacing, and metric -nets (maximal point sets such that no two points are closer than  to each other), while more uniform, may still have varying density. A common method for improving the spacing of Euclidean point sets is \emph{Lloyd's algorithm}~\cite{Llo-ITIT-82}, a variant of the -means clustering algorithm that repeatedly computes a Voronoi diagram and replaces each point by the centroid of its Voronoi cell. Its output is a \emph{centroidal Voronoi diagram}~\cite{DuFabGun-SR-99}, a well-spaced collection of points that form the centroids of their Voronoi cells.

Clarkson~\cite{Cla-UCI-08} suggested using metric -nets for -smoothed distance to generate well-spaced point sets with a distribution centered at  that decreases exponentially with distance from . For this application, one must restrict the points to an annulus centered on ; otherwise, an -net would not have a finite number of points. As an alternative means of generating exponentially-distributed and well-spaced points in this annulus, we experimented with a variant of Lloyd's algorithm that uses smoothed distance in place of Euclidean distance in its calculations.
Specifically, rather than computing a Euclidean Voronoi diagram of the given points, we computed a Voronoi diagram for the smoothed distance. And rather than moving each point to the centroid of its cell  (the point  minimizing ), we move each point to the point minimizing . Finally, in order to make the measure  of area used in the definition of the area integral transform scale-invariantly to match the symmetries of the smoothed distance, we chose a measure that is uniform not in the Euclidean plane in which the smoothed distance is defined, but rather the uniform measure in the transformed plane that has the Euclidean polar coordinates  as its Cartesian coordinates.

\begin{figure}[t]
\centering
\includegraphics[width=1.5in]{Lloyd-128-0}\qquad
\includegraphics[width=1.5in]{Lloyd-128-1}\0.1in]
\includegraphics[width=1.5in]{Lloyd-128-8}\qquad
\includegraphics[width=1.5in]{Lloyd-128-16}
\caption{Lloyd's algorithm on an annulus with an 18:1 ratio of inner to outer radius, for 128 exponentially-distributed random points. Top: initial configuration and one iteration. Center: after two and four iterations. Bottom: after 4, 8, and 16 iterations. The large black dots are the sites at each iteration; the smaller white spots represent the smoothed-distance Voronoi centroids to which these sites will be moved at the next iteration.}
\label{fig:lloyd}
\end{figure}


For simplicity, our implementation performs its calculations in the Euclidean plane, rasterized as a bitmap image. We compute the Voronoi diagram by finding the nearest site to each pixel of the rasterized annulus, and approximate  by , where the sum is over the pixels in the Voronoi region  and the  term weights each pixel by its measure in the transformed plane. The results of one run of our implementation are depicted in Figure~\ref{fig:lloyd}. We found that the iteration quickly (within two iterations) smoothed out any gross variation in the spacing of the given points, and then more slowly converged to a more ideal shape for each Voronoi cell. It was necessary to choose an initial set of points that was exponentially distributed around ; we placed each point by choosing  uniformly within the interval  (where  and  are the inner and outer radius of the annulus) and  uniformly in , and then selecting the point   in the Euclidean plane. We found that if instead we selected points with uniform Euclidean measure in the annulus, too many points were placed far from  and too few were placed close to ; Lloyd's algorithm was slow in correcting this imbalance.

\section{Conclusions}

We have identified a general condition on convex functions that causes their minimization diagrams to have linear complexity, applied this condition to Voronoi diagrams of smoothed distance and to finding the minimum dilation pair of leaves in a star network, and experimented with using a smoothing algorithm based on these Voronoi diagrams to generate evenly-spaced points exponentially distributed around a given center point.

Several directions for further research remain open:

\begin{itemize}
\item If we translate the convex function  in three dimensions rather than two by adding independent constants to the values for each point site, when does the resulting planar minimization diagram still have linear complexity? For instance, additively weighting  in this way results in a power diagram. For any convex , the bisectors of an additively weighted minimization diagram are still monotone curves, but we can no longer guarantee that they form pseudoline arrangements. Nevertheless it might be possible that they form minimization diagrams with connected cells.
\item Is it possible to characterize the functions that can be monotonically transformed into the form  with  and  both convex? The functions that have this form already are exactly the convex functions for which every axis-parallel rectangle has equal sums on its two pairs of opposite corners. However, if a function does not already have this form it may not be clear how to transform it into this form, as we did for smoothed distance.
\item Does our condition  and  characterize the convex functions  and  such that the translated level sets of  form pseudocircles? It does when : If , then level sets of  have a curvature at slope 1 that grows tighter for larger circles. However the situation is less clear when  and  may differ.
\item In particular, does the convex function  coming from smoothed distance and dilation have level sets that are pseudocircles when ?
\item If not, is there some other natural distance function  on the nonzero complex numbers, satisfying scale invariance , that has simply connected Voronoi regions for all sets of two or more points in general position? Note that the most obvious choice, the Euclidean distance between  and , does not work: in general there will be a single Voronoi region containing the origin, which will not be simply connected. The replacement function used in the proof of Theorem~\ref{thm:smoothed} does not seem very natural.
\item If we define a maximization diagram in which the sites are point pairs  and the function to be maximized is the dilation  of  and  with respect to a query point , our previous results~\cite{EppWor-CGTA-07} imply that this diagram has  cells. Are these cells simply connected?
\item Can we characterize the convex functions whose level sets are pseudocircles? For instance, as well as the functions  studied here, this is also true of convex distance functions.
\item If a convex function has level sets that are pseudocircles, do its minimization diagrams automatically have simply connected cells, or is an additional condition required for this to be true?
\item To what extent can this theory be generalized to minimization diagrams in higher dimensions?
\end{itemize}

\section*{Acknowledgements}

This work was supported in part by NSF grant
0830403 and by the Office of Naval Research under grant
N00014-08-1-1015. We thank Elena Mumford for helpful comments on a draft of this paper.

\bibliographystyle{IEEEtran}
\bibliography{convex-min-diagrams}

\end{document}
