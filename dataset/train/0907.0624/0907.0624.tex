\documentclass{beatcs}



\newtheorem{proposition}{Proposition}
\newtheorem{lemma}[proposition]{Lemma}
\newtheorem{example}[proposition]{Example}
\newtheorem{sta}[proposition]{Statement}

\title{A note on syndeticity, recognizable sets and Cobham's theorem}
\author{Michel Rigo, Laurent Waxweiler\thanks{University of Li\`ege,
    Department of Mathematics, Grande Traverse 12 (B 37), B-4000
    Li\`ege, Belgium. \texttt{M.Rigo@ulg.ac.be}}}

\date{}

\begin{document}
\maketitle

\begin{abstract}
    In this note, we give an alternative proof of the following
    result.  Let  be two multiplicatively independent
    integers. If an infinite set of integers is both - and
    -recognizable, then it is syndetic. Notice that this result is
    needed in the classical proof of the celebrated Cobham's theorem.
    Therefore the aim of this paper is to complete \cite{Pe} and
    \cite{AS} to obtain an accessible proof of Cobham's theorem.
\end{abstract}

\section{Introduction}
Cobham's theorem is related to numeration systems and can be
considered as a classical result in formal languages theory. It is
formulated as follows. Let  be two multiplicatively
independent integers (i.e., the only integers satisfying 
are ).  If a subset  of integers is
both - and -recognizable then it is a finite union of arithmetic
progressions (i.e.,  is an {\it ultimately periodic} set). Recall
that  is said to be {\it -recognizable} if the
language  of the -ary representations (without leading
zeroes) of the elements in  is a regular language accepted by a
finite automaton (see for instance \cite[Chap. 5]{Ei}).  This famous
result has been widely studied from various points of view (we give
here just a few references): extension to non-standard numeration
systems \cite{Du,Ha} or to the framework of -regular sequences
\cite{bell}, study of the multidimensional case (known as
Cobham-Semenov's theorem) \cite{BHMV,PB}, alternative proofs using the
formalism of the first order logic \cite{Bes,MV}, \ldots.

The original proof due to Cobham is widely considered as rather
difficult \cite{Co}. In his book, S.~Eilenberg proposed as a challenge
to find an easier proof \cite{Ei}. The major improvements in the
simplification of the proof of Cobham's theorem were made by G. Hansel
in \cite{Ha1} where he makes use of the notion of syndeticity and
sketches the key-points leading to the result. Recall that an infinite
set of integers  is said to be {\it syndetic} if
there exists  such that for all , .
(Notice that Hansel's ideas about syndeticity also hold in a wider
framework than -ary numeration systems \cite{Ha2}.)

Afterwards, a great work of presentation relying on the main ideas
found in \cite{Ha1} was made by several authors \cite{AS,Pe}.
Unfortunately, in these last two documents a same mistake can be found
(Statement \ref{lem:faux} below is not correct and
Example \ref{exa:c} is a counter-example). In this note, our modest
contribution is to correct this error using as simple arguments as
possible. In the spirit, we are naturally close to \cite{Co} and
\cite{Ha1} but new ideas appear in our reasoning.  Finally, we hope
that this erratum added to \cite{Pe} or \cite{AS} will now give a
complete presentation of the proof of Cobham's theorem.

Let us set  as the alphabet of the -ary
digits. In \cite{AS,Pe}, the following result is presented.
\begin{sta}\label{lem:faux}
  If an infinite -recognizable set  is such
  that  is right dense, i.e., for all 
  there exists  such that , then
   is syndetic.
\end{sta}

\begin{example}\label{exa:c}
  As stated above, Statement \ref{lem:faux} is not correct.  An easy
  counter-example is given by the following set  of integers
  
  Indeed, this set is
  -recognizable : , and trivially
  right dense but not syndetic.
\end{example}
In the literature, Statement \ref{lem:faux} is generally presented to
obtain the following proposition.

\begin{proposition}\label{pro:1}\cite[Prop. 5]{Ha1}
  Let  be two multiplicatively independent integers. If an
  infinite set of integers if both - and -recognizable, then it
  is syndetic.
\end{proposition}
In substance, this latter result can naturally be found in Cobham's
work (see \cite[Lemma 3]{Co}). In this note, our aim is to give an
alternative proof of Proposition~\ref{pro:1} not using
Statement~\ref{lem:faux}. Our approach relies on five easy lemmas.

\section{Proof of the result}

We assume that the reader has some basic knowledge in automata theory
(see for instance \cite{Ei}). If  is a set of
integers, we define a mapping (or a right-infinite word)
 such that  if
and only if . If  is a finite word,  denotes its length.

This first lemma will be useful in the proof of Lemma~\ref{lem:3} and
\ref{lem:4}.
\begin{lemma}\label{lem:1}
  Let  be a DFA (Deterministic
  Finite Automaton) with  as transition
  function. For any state , the set
  
  is
  such that  is ultimately periodic, i.e., there
  exist  and  such that for all ,
  .
\end{lemma}

\begin{proof}
  For any state , we define a mapping 
  Since
   is finite, there exist  and  such that
   and . Obviously, for any
  , . Consequently
  for all ,
  
  In other words,  is ultimately
  periodic:  if . To conclude the
  proof, observe that  where
  .
\end{proof}

\begin{lemma}\label{lem:2}
    Let  be arbitrary integers
    such that  and  be two multiplicatively independent
    integers.  Then there exist integers  such that
    .
\end{lemma}

\begin{proof}
  It is enough to find integers  satisfying
  
  This is a direct consequence of
  Kronecker's theorem (because  and  are still
  multiplicatively independent hence  is
  irrational) \cite{HW}.
\end{proof}

\begin{lemma}\label{lem:3}
    Let  and  be an infinite
    -recognizable set. Then there exist integers  such
    that for all , the set
     is nonempty. Moreover, the
    integer  can be chosen arbitrarily large.
\end{lemma}

\begin{proof}
    Let  be a DFA recognizing
    . Since  is infinite, there exists  arbitrarily
    large such that  is prefix of an infinite number of
    elements in . Let . By Lemma
    \ref{lem:1}, there exist  and  such that
     for all .
  
  For any , the interval  contains all the
  integers having a -ary representation of the form 
  with . Since the set  is
  infinite, there exists a word  such that  is the
  -ary representation of an element in  with .
  Take .  Consequently, the interval  contains
  an element belonging to . The conclusion follows from the
  periodicity of :
  , for all .
\end{proof}

Recall that a state  is said to be {\it accessible} (resp. {\it
  coaccessible}) if there exists a word  such that
 (resp. ). The {\it trimmed}
minimal automaton of a language  is obtained by taking only states
which are accessible and coaccessible.

\begin{lemma}\label{lem:4}
  Let  and  be an infinite
  -recognizable set such that
   is the trimmed minimal
  automaton of . If there exists a state  such that
   is infinite, then there exist integers
   such that for all , the set
   is empty.
\end{lemma}

\begin{proof}
  Let  be a state such that  is infinite.
  Without loss of generality, we may assume that  and there
  exists  such that . (Indeed, if
   is infinite then the same property
  holds for some other state .) We use the same reasoning as in the
  previous proof. Thanks to Lemma~\ref{lem:1}, there exist  and  such that 
  for all .  Since  is infinite,
  there exists  such that no word  of length  is
  such that . In other words, if  then
   and the interval  does
  not contain any element of . Once again, the conclusion follows
  from the periodicity of .
\end{proof}

The last lemma is a simple consequence of the three previous ones.

\begin{lemma}\label{lem:5}
  Let  be two multiplicatively independent integers and
   be an infinite - and -recognizable set
  of integers. If  is trimmed
  minimal automaton of , then for any state , the
  set  is cofinite.
\end{lemma}

\begin{proof}
    Assume to the contrary that  is infinite.
    By Lemma~\ref{lem:4}, there exist  such that for
    all ,  is
    empty.
    
    By Lemma~\ref{lem:3}, there also exist  such that for
    all ,  is nonempty
    and .
  
  
  
  To obtain a contradiction, simply observe that as a consequence of
  Lemma~\ref{lem:2}, there exist  such that
  .
\end{proof}

We now have at our disposal all the necessary material to conclude
this short note.
\begin{proof}[Proof of Proposition \ref{pro:1}]
  Assume that . Let  be the
  trimmed minimal automaton of . For all , we write
  .  Thanks to Lemma~\ref{lem:5},
   is cofinite. This means that for all , there
  exists  such that for all ,  belongs to .
  Clearly,  depends only on the state  and there are a
  finite number of such states. Let . Consequently, for
  any , there exists a word  of length  such that
  . In other words, for any , there
  exist  such that . We conclude that
  any interval of length  contains at least an element belonging
  to .
\end{proof}

\begin{thebibliography}{99}
  
\bibitem{AS} J.-P. Allouche, J. Shallit, {\it Automatic sequences,
    Theory, Applications, Generalizations}, Cambridge University
  Press, Cambridge, (2004).
  
\bibitem{bell} J.~P. Bell, A generalization of Cobham's theorem for
  regular sequences, Preprint (2005).
  
\bibitem{Bes} A. B\`es, An extension of the Cobham-Semenov theorem,
  {\it J. Symbolic Logic} {\bf 65} (2000), 201--211.
  
\bibitem{BHMV} V. Bruy\`ere, G. Hansel, C. Michaux, R. Villemaire,
  Logic and -recognizable sets of integers, {\it Bull. Belg. Math.
    Soc.} {\bf 1} (1994), 191--238.
  
\bibitem{Co} A. Cobham, On the base dependence of sets of numbers
  recognizable by finite automata, {\it Math. Syst. Theory} {\bf 3}
  (1969), 186--192.

\bibitem{Du} F. Durand, A theorem of Cobham for non primitive
  substitutions, {\it Acta Arith.} {\bf 104} (2002), 225--241.
  
\bibitem{Ei} S. Eilenberg, {\it Automata, Languages and Machines},
  vol. A, Academic Press (1974).

\bibitem{Ha1} G. Hansel, \`A propos d'un th\'eor\`eme de Cobham, In D.
  Perrin Ed., {\it Actes de la f\^ete des mots}, 55--59, Greco de
  programmation, CNRS, Rouen, (1982).
  
\bibitem{Ha2} G. Hansel, Syst\`emes de num\'eration ind\'ependants et
  synd\'eticit\'e, {\it Theoret. Comput. Sci.} {\bf 204} (1998),
  119--130.
  
\bibitem{Ha} G. Hansel, T. Safer, Vers un th\'eor\`eme de Cobham pour
  les entiers de Gauss, {\it Bull. Belg. Math. Soc.  Simon Stevin}
  {\bf 10} (2003), 723--735.
  
\bibitem{HW} G.H. Hardy, E.M. Wright, Introduction to the Theory of
  Numbers, Oxford Univ. Press, (1985).
  
\bibitem{MV} C. Michaux, R. Villemaire, Presburger arithmetic and
  recognizability of sets of natural numbers by automata: new proofs
  of Cobham's and Semenov's theorems, {\it Ann. Pure Appl. Logic} {\bf
    77} (1996), 251--277.
  
\bibitem{Pe} D. Perrin, Finite Automata, J. Van Leeuwen Ed.
  {\it Handbook of Theoret. Comput. Sci.}, vol. B, 1--57,
  Elsevier--MIT Press, (1990).
  
\bibitem{PB} F. Point and V. Bruy\`ere, On the Cobham-Semenov theorem,
  {\it Theory Comput. Syst.} {\bf 30} (1997), 197--220.
  
\end{thebibliography}

\end{document}
