\newcommand{\simplify}{\textsc{Simplify}}
\newcommand{\E}{\text{\bf E}}
\newcommand{\X}{\mathbf{X}}
\newcommand{\qavg}{\hat{q}}
\newcommand{\alphavg}{\hat{\alpha}}

Recall the setup described in Section~\ref{subsec:dep-round}, and properties (A1)-(A4) in particular. For the -median application, we will actually need a weighted generalization of this, as mentioned briefly in Section~\ref{subsubsec:depround-kmedian}. The basic change is that we now have positive weights , and want to preserve the \emph{weighted} sum , instead of  as in (A2). Such preservation may not be possible (no matter what the rounding), so we will leave at most one variable un-rounded at the end, as specified by property (A0') next. Let us describe our main problem, and then discuss the results we obtain. Our \textbf{main problem},  
given  and positive weights , is to efficiently sample a vector  from a distribution on  which satisfies the following properties:
\begin{description}
\item[(A0')] ``almost-integrality": all but at most one of the  lies in  (with the remaining at most one element lying in ); 
\item[(A1')] ;
\item[(A2')] ;
\item[(A3')] , , and 
;
\item[(A4': informal)] if the weights  are ``not too far apart", there is near-independence for subsets of  that are of cardinality at most , 
analogously to (A4). 
\end{description}

In this section we will give an -time algorithm (called {\sc DepRound}) for sampling from such a distribution; see Section~\ref{sec:depround}.
In particular, this shows that such distributions exist. This algorithm is a generalization of the unweighted version given in \cite{srin:level-sets}, with a specific (random) ordering of operations, leading to the added property (A4') of near-independence. 
Our main theorem is Theorem~\ref{thm:limited-dep-general}. 
It basically says, in the notation of (\ref{eqn:desired-goal}), that we can achieve  when 
. Thus, we obtain near-independence up to fairly large sizes . This bound is further improved in 
Section~\ref{sec:dep-round-special}. Let us also remark about the possible (sole) index  that is left unrounded, as in (A0'). Three simple ways to round this are to round down, round up, or round randomly; these can be chosen in a problem-specific manner. None of these three fits the -median application perfectly; a different probabilistic handling of this index  is done in Section~\ref{sec:bipoint-main-case}. 

Note that {\sc DepRound} could be described more simply as a form of pipage rounding (with an added, crucial, element of processing the variables in random order), with flow adjusted proportionally to the weights. However, in order to facilitate the analysis of (A4'), we give the following, less compact description of the algorithm.
Also note that we will apply this method to -median in Section~\ref{sec:bipoint-main-case}, but here it is described as a general-purpose rounding procedure, independent of -median; this is because we believe that this method is of independent interest since it goes much beyond negative correlation. Thus, the reader is asked to note that the notation defined in this section is largely separate from that of the other sections.

\subsection{The {\sc Simplify} subroutine}
\label{sec:simplify}
As in \cite{srin:level-sets}, our main subroutine is a procedure called . It takes as input two fractional values , and corresponding positive weights . The subroutine outputs a random pair of values , with the following properties:

\begin{description}
\item[(B0)] , and at least one of the two variables is integral (0 or 1);
\item[(B1)]  and ;
\item[(B2)] ; and
\item[(B3)]
, and .
\end{description}

Now define  as follows. 
There are four cases:
\begin{enumerate}[\bf\text{Case} I:]
\item . With probability  set . With remaining probability set .
\item . With probability  set . With remaining probability set .
\item . With probability  set . With remaining probability set .
\item . With probability  set . With remaining probability set .
\end{enumerate}

If we set , then set .

If we set , then set .

If we set , then set .

If we set , then set .

\begin{lemma}\label{lemma:b-props}
 outputs  with properties (B0), (B1), (B2), and (B3).
\end{lemma}
This is straightforward to show; we provide a partial proof in Appendix \ref{apdx:depround}.

\subsection{Main algorithm: {\sc DepRound}} \label{sec:depround}
We now describe the full dependent rounding algorithm, which we denote {\sc DepRound}. 


\begin{algorithm}[h]
\caption{}
\begin{algorithmic}[1]
\STATE 
\STATE Let  be a random permutation.
\WHILE{ contains at least two fractional elements} 
\STATE Let  and  be the two left-most fractional elements in .\label{step:pick_ij}
\STATE 
\ENDWHILE
\STATE \textbf{Return} .
\end{algorithmic} 
\label{algo:A}
\end{algorithm}
Define  to be the value of  after the step  of the rounding process, with , and  if the algorithm halts after  steps. 
We say  is \emph{fixed} during step  if  is fractional but  is integral, since this implies  will not change in any future steps.

It is not hard to follow the proof of \cite{srin:level-sets} and validate properties (A0'), (A1'), (A2'), and (A3'); we present a proof sketch below. 
\begin{lemma}
{\sc DepRound} samples a vector in  time; this vector satisfies properties (A0'), (A1'), (A2'), and (A3').
\end{lemma}
\begin{proof}
By definition, {\sc DepRound} ends only when there is at most one fractional variable remaining, and by definition of , the remaining variables are either 0 or 1, showing (A0'). Since at least one variable is fixed in each constant-time step, the algorithm takes at most  steps, and thus runs in linear time. (The random permutation  may also be generated in linear time.)

(B1) implies that for each  and , . By induction, this implies (A1'). (B2) implies that , thus implying (A2'). Finally, let  and  be the variables chosen in line \ref{step:pick_ij} during step . If , then (B3) implies  and  (all other terms in the product are constant). If only one or neither of  are in , then the same hold with equality. By induction, these imply (A3').
\end{proof}
Note that the above properties hold under any ordering . The additional element of using  to process the indices in random order is needed only for the new property (A4').

\subsection{Limited dependence} 
In this section we will prove the limited dependence property (A4').  Consider a subset  of  indices, with corresponding ``target'' bits .  For each , define  if , or  if . We are interested in the value of , which is essentially equal to the joint probability . Note these are not exactly equivalent, because of the one fractional variable. This variable must be handled in a domain-specific way to ensure this property holds, as we will do when applying it to -median.

From property (A1'), we have . Similarly, , define  such that . That is, let  be either  or  if  is 1 or 0, respectively. If we independently rounded each variable,  would be exactly  (but (A2') would be violated). We will show that when set  is not too large, the product is still very close to  in expectation (i.e., that the variables  are near-independent). 

During a run of {\sc DepRound}, we say that two variables  and  are \emph{co-rounded} if they are both changed in the same step. We will first show that if two variables are far apart in , then they are unlikely to be co-rounded. Then we will show that a group of variables is near-independent if the probability of any two of them being co-rounded is small. Finally, we will show that for a small enough set  variables are very likely to be far apart, and thus very likely to remain independent.

\subsubsection{Distant variables are seldom co-rounded}
Recall that {\sc DepRound} always calls  on the two left-most fractional variables in , fixing at least one of them (to 0 or 1). Thus, once a variable \emph{survives} (i.e., remains fractional after) one step it will continue to be included in all subsequent  steps until it gets fixed (or becomes the last remaining fractional variable). We want to upper bound the probability that a variable survives for too long. We first show that in any two consecutive steps involving , there is a minimum probability that  gets fixed (if the weights are not too different).
\begin{lemma}\label{lemma:pair-prob}
Let  and  be the minimum and maximum weights. Suppose . 
Suppose  is co-rounded with variable  in step , and if it survives it will be co-rounded with variable  in step . Let  and . 
Then  will be fixed in one of these two steps with probability at least .
\end{lemma}
\begin{proof}
What is the probability that  is fixed in the first step? It depends on which of the four cases occur. In case I, it is . In case II it is 1. In case IV it is . In these three cases,  is fixed with probability at least . However, in the remaining case III,  will be fixed with probability 0. 

Given that case III occurs in the first step, what is the probability that  gets fixed in the second step? It is at least the probability that one of cases I, II or IV occur in the second step, times  (by the same reasoning as above). When case III occurs in the first step,  gets set randomly to one of two values:  or , with probability  or , respectively.  These values differ by exactly . 
Now, in the second step, case III only occurs if  lies in the open interval . But the distance between any two numbers in this interval is strictly less than . Therefore, the two possible values of  cannot both lie in the interval required for case III, so with probability at least , the second step will be a case other than III. 
\end{proof} 

\smallskip \noindent \textbf{Important remark on notation:} In the following paragraph, by overloading notation, we fix the random permutation  to be some arbitrary but fixed . Several pieces of notation such as , and, most importantly, 
, are functions of this . All statements and proofs \emph{until the end of the proof of Lemma~\ref{lemma:sandwich1}}, are \emph{conditional on the random permutation equaling } (this is sometimes stated explicitly, sometimes not). 

\smallskip
Lemma~\ref{lemma:pair-prob} implies that the probability of a variable's survival decays exponentially with the number of steps survived. Now, given a permutation , let  be the bijection such that . 
Let 
 
be the set of the indices not in . Now partition  into sequences , using elements of  as dividers. Formally, for , let  be the maximal sequence  satisfying , letting  and . Note that if  and  are directly adjacent in , then  will be empty, as seen in the example below. 


For , let  be the ``bad'' event that {\sc DepRound} co-rounds  and . For  to occur, it is necessary that  be co-rounded with all variables inbetween as well. For example, in the above sequence,  means that  must be co-rounded with , (surviving each round), and finally . The next lemma bounds  in terms of the set .

\begin{lemma}\label{lemma:zbound}
Let . If , then , we have , where .
\end{lemma}

\begin{proof}
Assume  (else the lemma is trivially true).
Let  be the event that  is co-rounded with  (i.e. is not fixed earlier in the algorithm).
For , let  be the event that  is corounded with  \emph{and} survives. To apply Lemma 2.3, we first express the probability of  in terms of consecutive pairs of events.



Note that  implies . So .
Observe that event  is equivalent to the event that  is co-rounded with . Also, observe that if  occurs during step  (for ), then  is the first step involving , so the input value to  is . These observations together with Lemma 2.3 imply that . Then (\ref{eq:pair_prod}) is bounded by  as defined in the lemma.
\end{proof}

\subsubsection{Seldom co-rounded variables are near-independent}

The following lemmas show that if the probability of variables in  being co-rounded is low, then . For notational convenience, define .

\begin{lemma}\label{lemma:sandwich1} Let .
Let  denote a set of events which consists of exactly one of  or  for each , where  is some attainable value of . Then, conditioned on a fixed permutation , the following holds for all : 

\end{lemma}
\begin{proof}
Recall  is a fixed permutation; all probabilities and expectations in this proof are conditioned on . This proof formalizes the idea that if  doesn't occur, then  and  remain independent. If it does occur, the effect on the expected value is limited by .

After step  of {\sc DepRound}, we may consider the remainder of the algorithm as simply a recursive call on vector  (using the same permutation ). Let  be the first such call where  is one of the two left-most fractional variables to be co-rounded. Then there is at most one fractional variable to the left of , say , and all variables to the right still have their initial values from .

The key observation is that while events in  do affect the identity and initial value of , they do not further influence the outcome of . The only way  could be further influenced is if  contains the event , where  is with some probability the variable . However, for each ,  either contains  (which means  was fixed earlier and cannot be ), or it lacks .

This means that all the properties of {\sc DepRound} shown so far (which hold for a fixed ) still hold for  when conditioned on . Namely, we have  (for all  to the right of and including ), , and -- as we will claim by induction --  (\ref{eq:sandwich-ih}) for .
The bounds derived below handle the problematic compound event  explicitly by assuming that when  occurs,  always attains its worst-case value (1 for the upper bound, or 0 for the lower bound).

As a base case, consider the singleton set . As just described, we have  , so , and (\ref{eq:sandwich-ih}) follows from .
We now proceed by induction on , counting backward from . Let . Let  be the complement of event .  For some , assume that (\ref{eq:sandwich-ih}) holds for  with set . Then, using the independence properties just mentioned, we have that  is

Similarly, we have that  is

But also  so we use the better of the two lower bounds.
\end{proof} 

\smallskip \noindent \textbf{Remark.} From now on, we will no longer take  as fixed, and hence the  (which are functions of the random permutation) will be viewed as random variables.

\smallskip
\begin{lemma} \label{lemma:sandwich}
If , we have 
 
\end{lemma}
\begin{proof}
Apply Lemma \ref{lemma:sandwich1} with  and . Recall . Take the expectation over all permutations , and then factor out the constant .
\end{proof}
\subsubsection{Small subsets are spread out}

The following lemma gives a very useful combinatorial characterization of the distribution of .
\begin{lemma}\label{lemma:bijection}
Let  be a sequence of nonnegative integers which sum to , picked uniformly at random from all such possible sequences. Then the distribution of  is equal to the distribution of . Both distributions are symmetric.
\end{lemma}
\begin{proof}Consider the mapping  from permutations on  to binary strings of length , in which we replace each index in  with a 1, and the rest with a 0. Also define  to be the following standard combinatorial bijection from binary strings to arrangements of balls in boxes: given a binary string , first add a 1 to the beginning and end of the string; then, viewing the space between each nearest pair of 1's as a `box', and the zeros between each pair as `balls' in that box, let  be the sequence which counts the number of balls in each box, from left to right. For an arbitrary permutation , and the corresponding sets , we see that  corresponds to the number of zeros between the 'th and the 'th 1 in , and thus to the number of balls in the corresponding box in . Therefore we have that .
 

Now recall that {\sc DepRound} chooses a uniformly random permutation . Notice that  only maps to binary strings of length  with exactly  ones. Furthermore, for each such binary string, there are exactly  permutations which map to it. Thus,  is uniformly distributed over all  such binary strings.  provides an exact bijection between binary strings of length  with  ones, and sequences  as defined in the lemma. This implies that  -- and thus  -- is uniformly distributed over all such possible sequences . 

Furthermore, by definition of , all permutations of a sequence  would be equally likely. Therefore the distribution of , and thus , is symmetric.
\end{proof}

\smallskip \noindent \textbf{Remark.} Note that the above distribution over balls-in-boxes is such that each possible arrangement is equally likely. This is not to be confused with distributions obtained by randomly and independently throwing the balls into the boxes. 

\begin{lemma}\label{lemma:exjc}
 Consider a subset  of size , and let . Then for any constant ,

\end{lemma}
\begin{proof}
From Lemma \ref{lemma:bijection} the distribution of  is symmetric. Thus, when considering the distribution of a function of the sizes , we may w.l.o.g.\ assume that . Notice since  are all disjoint, we have ; also, .

Now for a quick exercise in counting. From the previous proof,  maps permutations uniformly to -digit binary strings with  1's. For a given permutation , we observe that  iff the binary string  has exactly  zeros before the 'th 1 (i.e., there are  total balls in the first  boxes). How many of the  possible strings have this property? It is the number of ways to put  1's in the first  digits, a  in the 'th digit, and  1's in the remaining  digits. Thus,

where  denotes the falling factorial. Now we combine (\ref{eq:ejc}) and (\ref{eq:prjkm}), and relax the bound by allowing  to go up to infinity. The resulting series converges when .

A quick proof of the series' convergence is to start with , and take the 'th derivative of both sides, with respect to .
\end{proof}

\begin{lemma}\label{lemma:exprod}
Let , and . Let  of size . If , 

\end{lemma}

\begin{proof} First, recall by definition that  is either  or , so . Then

In the first line we used . In (\ref{step:sqrt-bound}) we used . In (\ref{step:lemma-exjc}) we applied Lemma \ref{lemma:exjc} and then used .
\end{proof}
Now we can complete the bound given in Lemma \ref{lemma:sandwich}. The upper bound follows by expanding the binomial, bounding the expected value of each term, and then refactoring. The lower bound follows by the Weierstrass product inequality.



Thus we are led to our main theorem on dependent rounding (which in turn is improved upon, with further work, in Section~\ref{sec:dep-round-special}): 

\begin{theorem}\label{thm:limited-dep-general}
Let  be the vector returned by running {\sc DepRound} with probabilities  and positive weights . Let  and  be disjoint subsets of . Define , , , and . Then if , we have

\end{theorem}
\begin{proof}
For all , set ; for all , set . Then apply (\ref{eq:exp-up}) and (\ref{eq:exp-down}) to Lemma \ref{lemma:sandwich}. The theorem follows by recognizing that

\end{proof}
Note that . Thus we see that Theorem \ref{thm:limited-dep-general} allows us to bound the dependence among groups of variables as large as  when .
\subsection{Improvements and Special Cases}
\label{sec:dep-round-special}
In this section we present several refinements of Theorem \ref{thm:limited-dep-general}. The proofs all follow the same outline as that of the main result; we describe only the places where they differ.  We reuse the same definitions unless stated otherwise.

In our -median application we will have that all  are uniform. In this case, if the maximum ratio of weights is sufficiently small, we can tighten the bound to show a weaker dependency on .
\begin{theorem}\label{cor:dep-uniform-p}
Suppose  and . Then if , we have  

\end{theorem}

\begin{proof}
The improvement comes from strengthening the result of Lemma \ref{lemma:pair-prob}: Suppose  is co-rounded with variable  in step  and then (if it survives) variable  in step , where . Then we can show  will be fixed in one of these two steps with probability at least .


First assume that during both steps Case III occurs. 
By requirements for Case III, we have  and . 
Suppose  is fixed to 0 in step . Then 
 
Else suppose  is fixed to 1. Then


In either outcome, we have a contradiction. Therefore in at least one of the two steps, a case other than III must occur. As shown in the proof of Lemma \ref{lemma:pair-prob}, in the other 3 cases  will be fixed with probability at least .

This stronger bound carries through the remaining lemmas in a straightforward way. Following the proof for Lemma \ref{lemma:zbound}, starting from (\ref{eq:pair_prod}), we get

So Lemmas \ref{lemma:zbound}, \ref{lemma:sandwich1} and \ref{lemma:sandwich} will now hold with the new definition . Then in Lemma \ref{lemma:exprod}, we get

The theorem follows as before.
\end{proof}

In the unweighted case (where all ), we can similarly tighten the bound. We can also refine the bound to be in terms of a sort of average of the probabilities instead of just the most extreme. 
\begin{theorem}
\label{thm:limited-dep-unweighted}
Let  be the vector returned by running {\sc DepRound} with probabilities  and unit weights . Let . Let  and  be disjoint subsets of .  Let  for , and let  for ; let
. Define
 
Then,

Furthermore, if  is an integer, then  has no fractional elements.
\end{theorem}

\begin{proof}
Uniform weights allow us to strengthen Lemma \ref{lemma:pair-prob} even further; in particular, we no longer need to consider pairs of steps. Suppose  is co-rounded with  during step . Since all , Cases II and III cannot occur. Thus,  will be fixed with probability at least  (as in proof of Lemma \ref{lemma:pair-prob}).

If we follow the proof of Lemma \ref{lemma:zbound}, but without splitting events into pairs, we can show

So Lemmas \ref{lemma:zbound}, \ref{lemma:sandwich1} and \ref{lemma:sandwich} will now hold with the new definition . Now (as in Lemma \ref{lemma:exprod}), we wish to upper bound . Recall the expectation here is conditioned on the random permutation . We may decompose  into 3 independent components. First, recall  is the binary string corresponding to  with  1's representing the locations of indices in . Second, let  be the permutation representing the ordering of  over the 1's in . Third, let  be the permutation representing the ordering of  over the 0's in . Then  is uniquely defined by the tuple (, , ) and vice versa. So we can think of  as being generated by choosing each element of the tuple uniformly at random. 
Thus, the value of  depends only on ; the sizes  depend only on ; and the elements of  (conditioned on a particular set of sizes) depend only on . This shows that some of the variables are independent, so we may separate the terms. Here we are explicit over which random variable we take each expectation:

The following lemma is basically a restatement of Maclaurin's inequality for symmetric polynomials:
\begin{lemma}\label{lemma:maclaurin}
Given a vector of positive reals , with average value , let  be a subset chosen uniformly at random from all such subsets of size . Then .
\end{lemma}
The first expectation in (\ref{eq:exp_decompose}) is over a product of  random terms from . The second expectation is a product over  (which as a function of  is fixed for each term) random terms from . So both expectations may be bounded by Lemma \ref{lemma:maclaurin}:

The theorem follows as before.
\end{proof}

\subsubsection{An alternative lower bound}
All the lower bounds given thus far become negative for  larger than . We now derive an alternative lower bound which remains positive even for larger values of . We will do this for the uniform weight case for simplicity, but it may be adapted in a straightforward manner to the weighted case. 

\begin{theorem}\label{thm:alt-lower-bound}
Suppose . Let  be an integer which satisfies  and . Then

\end{theorem}
\begin{proof}
Start with the lower bound given by Lemma \ref{lemma:sandwich}. As shown in the proof of Theorem \ref{thm:limited-dep-unweighted}, if all , we may use . To lower bound the expression, we will focus on the event that sets  all have at least  elements, and use the trivial bound of 0 if this event does not occur. This event is useful because it implies that  are all far away in .

To calculate this probability, recall that in Lemma \ref{lemma:bijection}, we showed that the distribution of  is equivalent to the uniform distribution over unique arrangements of  identical balls into  boxes. Note there are  such arrangements. How many of these arrangements have at least  balls in the middle  boxes? To count these arrangements, we suppose that there are already exactly  balls in each of the middle  boxes and then count how many ways there are to add the remaining  balls to  boxes, which is . So,

\end{proof}

We show that if , and  is sufficiently large, then Theorem \ref{thm:alt-lower-bound} gives a nontrivial bound for  and a tight bound (close to ) for some .

First suppose  and set , for some . Assume . Then we have  and , so  is valid. These two inequalities also imply that the bound is positive.



Now suppose for some  that  and set . 
Assume , which implies . 
For simplicity of the argument, observe that
.
Then we have 
 
and 
, so  is valid. Then

This implies the bound is at least .


