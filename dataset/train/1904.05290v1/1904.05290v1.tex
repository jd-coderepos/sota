\documentclass[10pt,twocolumn,letterpaper]{article}

\usepackage{cvpr}
\usepackage{times}
\usepackage{epsfig}
\usepackage{graphicx}
\usepackage{amsmath}
\usepackage{amssymb}
\usepackage{balance}

\usepackage[format=plain,labelformat=simple,labelsep=period,font=small,skip=4pt,compatibility=false]{caption}
\usepackage[font=footnotesize,skip=2pt]{subcaption}
\newcommand{\myparagraph}[1]{\medbreak\noindent\textbf{#1}}

\usepackage{pifont}
\newcommand{\cmark}{\ding{51}}
\newcommand{\xmark}{\ding{55}}

\usepackage{booktabs}
\usepackage{multirow}
\usepackage{tabularx}
\usepackage{visinfdefs}
\usepackage[super]{nth}

\usepackage{etoolbox}
\usepackage[binary-units]{siunitx}
\sisetup{detect-all=true}
\sisetup{quotient-mode = fraction}
\sisetup{fraction-function = \nicefrac}
\robustify\bfseries
\DeclareSIUnit{\inch}{inch}


\usepackage{tikz}
\newcommand*\circled[1]{\tikz[baseline=(char.base)]{
        \node[shape=circle,draw,fill=white,minimum size=3mm,inner sep=0pt] (char)
        {\vphantom{1g}\emph{#1}};}}
        




\usepackage[breaklinks=true,bookmarks=false,colorlinks,pagebackref=true]{hyperref}
\usepackage[capitalize]{cleveref}
\crefname{section}{Sec.}{Section}


\newcommand{\supptitle}{Iterative Residual Refinement for Joint Optical Flow and Occlusion Estimation \\ {\large -- Supplementary Material --}}
	\newcommand{\suppauthor}{Junhwa Hur \qquad\qquad Stefan Roth \\
Department of Computer Science, TU Darmstadt}

\cvprfinalcopy \def\cvprPaperID{****} \def\httilde{\mbox{\tt\raisebox{-.5ex}{\symbol{126}}}}

\ifcvprfinal\pagestyle{empty}\fi



\usepackage{fancyhdr}
\usepackage{setspace}
\renewcommand{\headrulewidth}{0pt}
\renewcommand{\footrulewidth}{0pt}
\fancyhf{}
\lfoot{{\footnotesize\begin{spacing}{.5}\parbox{\linewidth}{\vspace{2.5em}To appear in Proceedings of the \emph{ IEEE/CVF Conference on Computer Vision and Pattern Recognition (CVPR)}, Long Beach, CA, USA, June 2019.\\\hrule\vspace{\baselineskip}
\copyright~2019 IEEE. Personal use of this material is permitted. Permission from IEEE must be obtained for all other uses, in any current or future media, including reprinting/republishing this material for advertising or promotional purposes, creating new collective works, for resale or redistribution to servers or lists, or reuse of any copyrighted component of this work in other works.
}\end{spacing}}}


\begin{document}
\title{Iterative Residual Refinement for Joint Optical Flow and Occlusion Estimation}

\author{Junhwa Hur \qquad\qquad Stefan Roth \\
Department of Computer Science, TU Darmstadt}

\maketitle
\thispagestyle{fancy} 

\begin{abstract}
	Deep learning approaches to optical flow estimation have seen rapid progress over the recent years.
One common trait of many networks is that they refine an initial flow estimate either through multiple stages or across the levels of a coarse-to-fine representation.
While leading to more accurate results, the downside of this is an increased number of parameters.
Taking inspiration from both classical energy minimization approaches as well as residual networks, we propose an \emph{iterative residual refinement} (IRR) scheme based on \emph{weight sharing} that can be combined with several backbone networks.
It reduces the number of parameters, improves the accuracy, or even achieves both.
Moreover, we show that integrating occlusion prediction and bi-directional flow estimation into our IRR scheme can further boost the accuracy.
Our full network achieves state-of-the-art results for both optical flow and occlusion estimation across several standard datasets. \end{abstract}



\section{Introduction}
\label{sec:introduction}



Akin to many areas of computer vision, deep learning has had a significant impact on optical flow estimation.
But in contrast to, \eg, object detection \cite{Girshick:2016:RBC} or human pose estimation \cite{Tompson:2014:JTC}, the accuracy of deep learning-based flow methods on public benchmarks \cite{Butler:2012:NOS,Geiger:2012:AWR,Menze:2015:OSF} had initially not surpassed that of classical approaches.
Still, the efficient test-time inference has led to their widespread adoption as a sub-module in applications requiring to process temporal information, including video object segmentation \cite{Cheng:2017:SFJ}, video recognition \cite{Gadde:2017:SVC,Nilsson:2018:SVS,Zhu:2017:DFF}, and video style transfer \cite{Chen:2017:COV}.

FlowNet \cite{Dosovitskiy:2015:FLO} pioneered the use of convolutional neural networks (CNNs) for estimating optical flow and relied on a -- by now standard -- encoder-decoder architecture with skip connections, similar to semantic segmentation \cite{Long:2015:FCN}, among others.
Since the flow accuracy remained behind that of classical methods based on energy minimization, later work has focused on designing more powerful CNN architectures for optical flow.
FlowNet2 \cite{Ilg:2017:FN2} remedied the accuracy limitations of FlowNet and started to outperform classical approaches.
Its main principle is to stack multiple FlowNet-family networks \cite{Dosovitskiy:2015:FLO}, such that later stages effectively refine the output from the previous ones.
However, one of the side effects of this stacking is the linearly and strongly increasing number of parameters, being a burden for the adoption in other applications.
Also, stacked networks require training the stages sequentially rather than jointly, resulting in a complex training procedure in practice.

\begin{figure}[t]
\centering 
\includegraphics[width=0.9\linewidth]{fig/exp/param_plot_v5.pdf}
\caption{\textbf{Accuracy / network size tradeoff of CNNs for optical flow:}
Combining our iterative residual refinement (IRR), as well as bi-directional (Bi) and occlusion estimation (Occ) with PWC-Net \cite{Sun:2017:PWC} in comparison to previous work.
Our full model (IRR-PWC), combining all three components, yields significant accuracy gains over \cite{Sun:2017:PWC} while having many fewer parameters.}
\label{fig:param_plot}
\vspace{-0.5em}
\end{figure}

More recently, SpyNet \cite{Ranjan:2017:OFE}, PWC-Net \cite{Sun:2017:PWC}, and LiteFlowNet \cite{Hui:2018:LFN} proposed lightweight networks that still achieve competitive accuracy (\cf\cref{fig:param_plot}).
SpyNet adopts coarse-to-fine estimation in the network design, a well-known principle in classical approaches.
It residually updates the flow across the levels of a spatial pyramid with individual trainable weights and demonstrates better accuracy than FlowNet but with far fewer model parameters.
LiteFlowNet and PWC-Net further combine the coarse-to-fine strategy with multiple ideas from both classical methods and recent deep learning approaches. 
Particularly PWC-Net outperformed all published methods on the common public benchmarks \cite{Butler:2012:NOS,Geiger:2012:AWR,Menze:2015:OSF}.

\begin{figure*}[t]
\centering
\subcaptionbox{FlowNet2-like \cite{Ilg:2017:FN2} stack of FlowNet networks.\label{fig:intro_a}}{\includegraphics[width=0.435\textwidth]{fig/network/flownet_stack_v2.pdf}} \quad
\subcaptionbox{Iterative residual refinement version of \emph{(\subref{fig:intro_a})}.\label{fig:intro_b}}{\includegraphics[width=0.49\textwidth]{fig/network/flownet_iterres_v2.pdf}} \
\mv{f}_\text{fw}^i = D\Big(E(\mv{I}_1), w\big(E(\mv{I}_2), \mv{f}_\text{fw}^{i-1}\big)\Big) + \mv{f}_\text{fw}^{i-1}, 
\label{eq:flownet_baseline}
1mm]
\subcaptionbox{Our IRR version of PWC-Net.\label{fig:pwcnet_shared}}{\includegraphics[width=0.93\linewidth]{fig/network/pwcnet_iter_v4.pdf}}
\caption{\textbf{Our IRR version of PWC-Net}, which uses only one single shared decoder over the pyramid levels, see text for details.}
\label{fig:pwcnet}
\vspace{-0.8em}
\end{figure}

\myparagraph{IRR with PWC-Net.}
Based on the classical coarse-to-fine principle, PWC-Net \cite{Sun:2017:PWC} and SpyNet \cite{Ranjan:2017:OFE} both use multiple repetitive modules for the same purpose but with \emph{separate weights}.
\cref{fig:pwcnet_standard} shows a 3-level PWC-Net (for ease of visualization, originally 7-level) that incrementally updates the estimation across the pyramid levels with individual decoders for each level.
Adopting our IRR scheme here to address the second scenario, we can substitute the multiple decoders with only one shared decoder that iteratively refines the output over all the pyramid levels, \cf\cref{fig:pwcnet_shared}.
We set the number of iterations equal to the number of pyramid levels, keeping the original pipeline but with fewer parameters and a more compact representation:

\label{eq:pwc_net}
\mv{f}_\text{fw}^i = D\Big(\mv{P}^i(\mv{I}_1), c\big(\mv{P}^i(\mv{I}_1), w\big(\mv{P}^i(\mv{I}_2), \hat{\mv{f}}_\text{fw}^{i-1}\big)\big), \hat{\mv{f}}_\text{fw}^{i-1}\Big) + \hat{\mv{f}}_\text{fw}^{i-1}  

\label{eq:pwc_net_flow_up}
\hat{\mv{f}}_\text{fw}^{i-1} = 2 \cdot \uparrow\!(\mv{f}_\text{fw}^{i-1}),

where  is the feature map at pyramid level ,  calculates a cost volume, and  performs  bilinear upsampling to twice the resolution of the previous flow field.
As the dimension increases, we also scale the flow magnitude accordingly (\Eq\ref{eq:pwc_net_flow_up}).

One important change from the original PWC-Net \cite{Sun:2017:PWC}, which estimates flow for each level on the original scale, is that we estimate flow for each level at its native spatial resolution.
This enables us to use only one shared decoder and yet make it possible to handle different resolutions across all levels.
When calculating the loss, we revert back to the original scale to use the same loss function.

In addition, we add a  convolution layer after the input feature map  to make the number of feature maps input to the decoder  be equal across the pyramid levels. 
This enables us to use one single shared decoder with a fixed number of input channels across the pyramid.

\myparagraph{Occlusion estimation.}
It is widely reported that jointly localizing occlusions and estimating optical flow can benefit each other \cite{Alvarez:2007:SDO,Ballester:2012:ATL,Hur:2017:MFE,Ince:2008:OOF,Sun:2014:LLJ,Unger:2012:JME,Xiao:2006:BFO}.
Toward leveraging this in the setting of CNNs, we attach an additional decoder estimating occlusion  in the first frame at the end of the encoder, in parallel to the flow decoder as shown in \cref{fig:intro_d}, similar to \cite{Janai:2018:ULM, Neoral:2018:COO}. 
The occlusion decoder has the same configuration as the flow decoder, but the number of output channels is  (instead of  for flow). 
The input to the occlusion decoder is the same as to the flow decoder.

\subsection{Joint optical flow and occlusion estimation}

\begin{figure*}[t]
\centering
\begin{minipage}{0.73\textwidth}
	\centering
	\includegraphics[width=\linewidth]{fig/network/flownet_iterres_biocc_v3.pdf}
	\caption{\textbf{Joint optical flow and occlusion estimation: bi-directional estimation, bilateral refinement, and upsampling layer} (in the FlowNet setting): We estimate flow in both temporal directions and occlusion maps in both frames by switching the order of inputs to the decoder. Bilateral refinement and the upsampling layer further improve the accuracy of flow and occlusion. Modules with the same color share their weights.
	\Cf supplemental material for the corresponding PWC-Net variant.}
	\label{fig:flownet_iterres_biocc}	
\end{minipage} \quad 
\begin{minipage}{0.245\textwidth}
	\centering
	\subcaptionbox{Ground-truth occlusion map.\label{fig:occ_compare_a}}{\includegraphics[width=0.98\linewidth]{fig/network/occ_orig.png}} \
    

l^i_\text{flow} = \tfrac{1}{2} \sum \big( \lVert \mv{f}^i_\text{fw} - \mv{f}_{\text{fw},\text{GT}} \rVert_2 + \lVert \mv{f}^i_\text{bw} - \mv{f}_{\text{bw},\text{GT}} \rVert_2 \big),
\label{eq:loss_flow}

\begin{split}
l^i_\text{occ} = - \tfrac{1}{2} \sum \big( w^i_1 o^i_1 \log o_{1,\text{GT}} & + \bar{w}^i_1 ( 1\!-\!o^i_1 ) \log (1\!-\!o_{1,\text{GT}})  \\ 
 + w^i_2 o^i_2 \log o_{2,\text{GT}} & + \bar{w}^i_2 ( 1\!-\!o^i_2 ) \log (1\!-\!o_{2,\text{GT}}) 
 \big).
\label{eq:loss_occ}
\end{split}

l_\text{FlowNet} = \frac{1}{N}\sum_{i=1}^N \sum_{s=s_0}^S \alpha_s (l^{i,s}_\text{flow} + \lambda \cdot l^{i,s}_\text{occ}), \label{eq:loss_final_flownet}

l_\text{PWC-Net} = \frac{1}{N}\sum_{i=1}^N \alpha_i (l^{i}_\text{flow} + \lambda \cdot l^{i}_\text{occ}). \label{eq:loss_final_pwc}
2mm]
\begin{subfigure}{\linewidth}
  \centering
  \includegraphics[width=0.7\linewidth]{fig/network/pwc_upsample.pdf}
  \caption{Upsampling optical flow and occlusion using the upsampling layer, \ie pyramid levels .}
  \label{fig:irr_pwc_upsample}
\end{subfigure}
\caption{\textbf{IRR-PWC}: Our PWC-Net variant with joint optical flow and occlusion estimation based on bi-directional estimation, bilateral refinement, and the occlusion upsampling layer. \emph{(\subref{fig:irr_pwc_estimator})} Our IRR-PWC model jointly estimates optical flow and occlusion up to a quarter resolution of the input image (\ie up to the \nth{5} level), the same as the original PWC-Net. \emph{(\subref{fig:irr_pwc_upsample})} Then, we use our occlusion upsampling layer to upscale the outputs back to the original resolution while improving accuracy.}
\label{fig:irr_pwc_all}
\end{figure}
 

\section{Details on the Occlusion Upsampling Layer}
\label{sec:occ_upsampling_layer}

In the occlusion upsampling layer shown in \cref{fig:occ_up} in the main paper, the \emph{residual blocks} \cite{Lim:2017:EDR} are fed a set of feature maps as input and output residual occlusion estimates to refine the upscaled occlusion map from the previous level.
\cref{fig:residual_blocks_all} shows the details of the \emph{residual blocks}. 
As shown in \cref{fig:residual_blocks}, the subnetwork consists of  residual blocks (\ie  \emph{ResBlock}s) with  convolution layers. 
One \emph{ResBlock} consists of \emph{Conv+ReLu+Conv+Mult} operations as shown in \cref{fig:one_residual_block}, \cf \cite{Lim:2017:EDR}.
This sequence of  \emph{ResBlock}s with one convolution layer afterwards estimates the residuals over one convolution output of the input feature maps, and the final convolution layer of the \emph{residual blocks} outputs the residual occlusion.
The number of channels for all convolution layers here is , except for the final convolution layer, which has only  channel for the occlusion output. 

We use weight sharing also on the upsampling layers between bi-directional estimations and between pyramid levels or iteration steps. 
Furthermore, the \emph{ResBlock}s in \cref{fig:residual_blocks} also share their weights, which is different from \cite{Lim:2017:EDR}, where they are not shared.
With this efficient weight-sharing scheme, the occlusion upsampling layer improves the occlusion accuracy by  on the training domain (\ie the FlyingChairsOcc dataset) and  across datasets (\ie Sintel) with only adding 0.031 \si{\mega} parameters.

\begin{figure}[t]
\centering
\begin{subfigure}{\linewidth}
  \includegraphics[width=\linewidth]{fig/network/resblocks.pdf}
  \caption{Residual blocks subnetwork.}
  \label{fig:residual_blocks}  
\end{subfigure}\1mm]
\begin{subfigure}{\linewidth}
  \centering
  \includegraphics[width=0.5\linewidth]{fig/network/resblock.pdf}
  \caption{One residual block.}
  \label{fig:one_residual_block}
\end{subfigure}
\caption{\textbf{Residual blocks in the upsampling layer}: \emph{(\subref{fig:residual_blocks})} The \emph{residual blocks} consist of  weight-shared \emph{ResBlock}s with  convolution layers. \emph{(\subref{fig:one_residual_block})} One \emph{ResBlock} consists of \emph{Conv+ReLu+Conv+Mult} operations \cite{Lim:2017:EDR}. }
\label{fig:residual_blocks_all}
\vspace{-0.5em}
\end{figure}


 

{
\begin{figure*}[!b]
\centering
\footnotesize
\setlength\tabcolsep{0.3pt}
\renewcommand{\arraystretch}{0.2}
\begin{tabular}{>{\centering\arraybackslash}m{.25\textwidth} >{\centering\arraybackslash}m{.25\textwidth} >{\centering\arraybackslash}m{.25\textwidth} >{\centering\arraybackslash}m{.25\textwidth}}
	
	\begin{tikzpicture}
    \node[inner sep=0] (img) {\includegraphics[width=\linewidth]{fig/exp_res/occ_ablation/bamboo_1_img.png}};    
    \node[anchor=north west] at (img.north west){\circled{a}};
	\end{tikzpicture}&
	\begin{tikzpicture}
    \node[inner sep=0] (img) {\includegraphics[width=\linewidth]{fig/exp_res/occ_ablation/bamboo_2_gt.png}};    
    \node[anchor=north west] at (img.north west){\circled{b}};
	\end{tikzpicture}&
	\begin{tikzpicture}
    \node[inner sep=0] (img) {\includegraphics[width=\linewidth]{fig/exp_res/occ_ablation/bamboo_3_before.png}};    
    \node[anchor=north west] at (img.north west){\circled{c}};
	\end{tikzpicture}&
	\begin{tikzpicture}
    \node[inner sep=0] (img) {\includegraphics[width=\linewidth]{fig/exp_res/occ_ablation/bamboo_4_after.png}};    
    \node[anchor=north west] at (img.north west){\circled{d}};
	\end{tikzpicture} \\ \\
	
	\begin{tikzpicture}
    \node[inner sep=0] (img) {\includegraphics[width=\linewidth]{fig/exp_res/occ_ablation/bandage_1_img.png}};    
    \node[anchor=north west] at (img.north west){\circled{a}};
	\end{tikzpicture}&
	\begin{tikzpicture}
    \node[inner sep=0] (img) {\includegraphics[width=\linewidth]{fig/exp_res/occ_ablation/bandage_2_gt.png}};    
    \node[anchor=north west] at (img.north west){\circled{b}};
	\end{tikzpicture}&
	\begin{tikzpicture}
    \node[inner sep=0] (img) {\includegraphics[width=\linewidth]{fig/exp_res/occ_ablation/bandage_3_before.png}};    
    \node[anchor=north west] at (img.north west){\circled{c}};
	\end{tikzpicture}&
	\begin{tikzpicture}
    \node[inner sep=0] (img) {\includegraphics[width=\linewidth]{fig/exp_res/occ_ablation/bandage_4_after.png}};    
    \node[anchor=north west] at (img.north west){\circled{d}};
	\end{tikzpicture} \\ \\
	
	\begin{tikzpicture}
    \node[inner sep=0] (img) {\includegraphics[width=\linewidth]{fig/exp_res/occ_ablation/cave_1_img.png}};    
    \node[anchor=north west] at (img.north west){\circled{a}};
	\end{tikzpicture}&
	\begin{tikzpicture}
    \node[inner sep=0] (img) {\includegraphics[width=\linewidth]{fig/exp_res/occ_ablation/cave_2_gt.png}};    
    \node[anchor=north west] at (img.north west){\circled{b}};
	\end{tikzpicture}&
	\begin{tikzpicture}
    \node[inner sep=0] (img) {\includegraphics[width=\linewidth]{fig/exp_res/occ_ablation/cave_3_before.png}};    
    \node[anchor=north west] at (img.north west){\circled{c}};
	\end{tikzpicture}&
	\begin{tikzpicture}
    \node[inner sep=0] (img) {\includegraphics[width=\linewidth]{fig/exp_res/occ_ablation/cave_4_after.png}};    
    \node[anchor=north west] at (img.north west){\circled{d}};
	\end{tikzpicture} \\ \\
	
	\begin{tikzpicture}
    \node[inner sep=0] (img) {\includegraphics[width=\linewidth]{fig/exp_res/occ_ablation/temple_1_img.png}};    
    \node[anchor=north west] at (img.north west){\circled{a}};
	\end{tikzpicture}&
	\begin{tikzpicture}
    \node[inner sep=0] (img) {\includegraphics[width=\linewidth]{fig/exp_res/occ_ablation/temple_2_gt.png}};    
    \node[anchor=north west] at (img.north west){\circled{b}};
	\end{tikzpicture}&
	\begin{tikzpicture}
    \node[inner sep=0] (img) {\includegraphics[width=\linewidth]{fig/exp_res/occ_ablation/temple_3_before.png}};    
    \node[anchor=north west] at (img.north west){\circled{c}};
	\end{tikzpicture}&
	\begin{tikzpicture}
    \node[inner sep=0] (img) {\includegraphics[width=\linewidth]{fig/exp_res/occ_ablation/temple_4_after.png}};    
    \node[anchor=north west] at (img.north west){\circled{d}};
	\end{tikzpicture} \\ \\
	
\end{tabular}
\caption {\color{black} \textbf{Qualitative examples of using the occlusion upsampling layer}: \emph{(a)} overlapped input images, \emph{(b)} occlusion ground truth, \emph{(c)} without using the occlusion upsampling layer, and \emph{(d)} with using the occlusion upsampling layer. The occlusion upsampling layer makes occlusion estimates much sharper along motion boundaries and detects additional thinly-shaped occlusions.}
\label{fig:occ_ablation}
\end{figure*}
}



\section{Additional Qualitative Examples}
\label{sec:qualitative}

\paragraph{Occlusion upsampling layer.} \cref{fig:occ_ablation} provides qualitative examples of occlusion estimation and demonstrates the advantage of using the occlusion upsampling layer. 
The models used here are trained on the FlyingChairsOcc dataset only (no fine-tuning on the FlyingThings3D-subset dataset or Sintel) and tested on Sintel Train Clean.
The occlusion upsampling layer enhances the occlusion estimates to be much sharper along motion boundaries and refines coarse estimates. 
Also, the upsampling layer further detects thinly-shaped occlusions that were not detected at the quarter resolution. 
Unlike optical flow, where a quarter resolution estimate is largely sufficient, we can see from these qualitative examples that estimating occlusions up to the original resolution is very critical for yielding high accuracy.


\myparagraph{Ablation study on PWC-Net.}
In addition to \cref{fig:pwc_ablation} in the main paper, we here give more qualitative examples for the ablation study.
In \cref{fig:pwc_ablation_more}, all models are also trained on the FlyingChairsOcc dataset and tested on Sintel Train Clean.
Our proposed schemes significantly improve the accuracy over the baseline model (\ie PWC-Net \cite{Sun:2017:PWC}), yielding better generalization across datasets.


{
\begin{figure*}[ht]
\centering
\footnotesize
\color{black}
\setlength\tabcolsep{0.3pt}
\renewcommand{\arraystretch}{0.2}
\begin{tabular}{>{\centering\arraybackslash}m{.20\textwidth} >{\centering\arraybackslash}m{.20\textwidth} >{\centering\arraybackslash}m{.20\textwidth} >{\centering\arraybackslash}m{.20\textwidth} >{\centering\arraybackslash}m{.20\textwidth}}
	
	\begin{tikzpicture}
    \node[inner sep=0] (img) {\includegraphics[width=\linewidth]{fig/exp_res/pwc_ablation/ambush_1_img.png}};    
    \node[anchor=north west] at (img.north west){\circled{a}};
	\end{tikzpicture}&
	\begin{tikzpicture}
    \node[inner sep=0] (img) {\includegraphics[width=\linewidth]{fig/exp_res/pwc_ablation/ambush_2_baseline.png}};    
    \node[anchor=north west] at (img.north west){\circled{b}};
	\end{tikzpicture}&
	\begin{tikzpicture}
    \node[inner sep=0] (img) {\includegraphics[width=\linewidth]{fig/exp_res/pwc_ablation/ambush_3_bi.png}};    
    \node[anchor=north west] at (img.north west){\circled{c}};
	\end{tikzpicture}&
	\begin{tikzpicture}
    \node[inner sep=0] (img) {\includegraphics[width=\linewidth]{fig/exp_res/pwc_ablation/ambush_4_occ.png}};    
    \node[anchor=north west] at (img.north west){\circled{d}};
	\end{tikzpicture}&
	\begin{tikzpicture}
    \node[inner sep=0] (img) {\includegraphics[width=\linewidth]{fig/exp_res/pwc_ablation/ambush_5_biocc.png}};    
    \node[anchor=north west] at (img.north west){\circled{e}};
	\end{tikzpicture}\\	
	\begin{tikzpicture}
    \node[inner sep=0] (img) {\includegraphics[width=\linewidth]{fig/exp_res/pwc_ablation/ambush_6_gt.png}};    
    \node[anchor=north west] at (img.north west){\circled{f}};
	\end{tikzpicture}&
	\begin{tikzpicture}
    \node[inner sep=0] (img) {\includegraphics[width=\linewidth]{fig/exp_res/pwc_ablation/ambush_7_irr.png}};    
    \node[anchor=north west] at (img.north west){\circled{g}};
	\end{tikzpicture}&
	\begin{tikzpicture}
    \node[inner sep=0] (img) {\includegraphics[width=\linewidth]{fig/exp_res/pwc_ablation/ambush_8_occirr.png}};    
    \node[anchor=north west] at (img.north west){\circled{h}};
	\end{tikzpicture}&
	\begin{tikzpicture}
    \node[inner sep=0] (img) {\includegraphics[width=\linewidth]{fig/exp_res/pwc_ablation/ambush_9_bioccirr.png}};    
    \node[anchor=north west] at (img.north west){\circled{i}};
	\end{tikzpicture}&
	\begin{tikzpicture}
    \node[inner sep=0] (img) {\includegraphics[width=\linewidth]{fig/exp_res/pwc_ablation/ambush_10_fullmodel.png}};    
    \node[anchor=north west] at (img.north west){\circled{j}};
	\end{tikzpicture}\\  \\


	\begin{tikzpicture}
    \node[inner sep=0] (img) {\includegraphics[width=\linewidth]{fig/exp_res/pwc_ablation/bamboo_1_img.png}};    
    \node[anchor=north west] at (img.north west){\circled{a}};
	\end{tikzpicture}&
	\begin{tikzpicture}
    \node[inner sep=0] (img) {\includegraphics[width=\linewidth]{fig/exp_res/pwc_ablation/bamboo_2_baseline.png}};    
    \node[anchor=north west] at (img.north west){\circled{b}};
	\end{tikzpicture}&
	\begin{tikzpicture}
    \node[inner sep=0] (img) {\includegraphics[width=\linewidth]{fig/exp_res/pwc_ablation/bamboo_3_bi.png}};    
    \node[anchor=north west] at (img.north west){\circled{c}};
	\end{tikzpicture}&
	\begin{tikzpicture}
    \node[inner sep=0] (img) {\includegraphics[width=\linewidth]{fig/exp_res/pwc_ablation/bamboo_4_occ.png}};    
    \node[anchor=north west] at (img.north west){\circled{d}};
	\end{tikzpicture}&
	\begin{tikzpicture}
    \node[inner sep=0] (img) {\includegraphics[width=\linewidth]{fig/exp_res/pwc_ablation/bamboo_5_biocc.png}};    
    \node[anchor=north west] at (img.north west){\circled{e}};
	\end{tikzpicture}\\	
	\begin{tikzpicture}
    \node[inner sep=0] (img) {\includegraphics[width=\linewidth]{fig/exp_res/pwc_ablation/bamboo_6_gt.png}};    
    \node[anchor=north west] at (img.north west){\circled{f}};
	\end{tikzpicture}&
	\begin{tikzpicture}
    \node[inner sep=0] (img) {\includegraphics[width=\linewidth]{fig/exp_res/pwc_ablation/bamboo_7_irr.png}};    
    \node[anchor=north west] at (img.north west){\circled{g}};
	\end{tikzpicture}&
	\begin{tikzpicture}
    \node[inner sep=0] (img) {\includegraphics[width=\linewidth]{fig/exp_res/pwc_ablation/bamboo_8_occirr.png}};    
    \node[anchor=north west] at (img.north west){\circled{h}};
	\end{tikzpicture}&
	\begin{tikzpicture}
    \node[inner sep=0] (img) {\includegraphics[width=\linewidth]{fig/exp_res/pwc_ablation/bamboo_9_bioccirr.png}};    
    \node[anchor=north west] at (img.north west){\circled{i}};
	\end{tikzpicture}&
	\begin{tikzpicture}
    \node[inner sep=0] (img) {\includegraphics[width=\linewidth]{fig/exp_res/pwc_ablation/bamboo_10_fullmodel.png}};    
    \node[anchor=north west] at (img.north west){\circled{j}};
	\end{tikzpicture}\\  \\
	
	\begin{tikzpicture}
    \node[inner sep=0] (img) {\includegraphics[width=\linewidth]{fig/exp_res/pwc_ablation/cave_1_img.png}};    
    \node[anchor=north west] at (img.north west){\circled{a}};
	\end{tikzpicture}&
	\begin{tikzpicture}
    \node[inner sep=0] (img) {\includegraphics[width=\linewidth]{fig/exp_res/pwc_ablation/cave_2_baseline.png}};    
    \node[anchor=north west] at (img.north west){\circled{b}};
	\end{tikzpicture}&
	\begin{tikzpicture}
    \node[inner sep=0] (img) {\includegraphics[width=\linewidth]{fig/exp_res/pwc_ablation/cave_3_bi.png}};    
    \node[anchor=north west] at (img.north west){\circled{c}};
	\end{tikzpicture}&
	\begin{tikzpicture}
    \node[inner sep=0] (img) {\includegraphics[width=\linewidth]{fig/exp_res/pwc_ablation/cave_4_occ.png}};    
    \node[anchor=north west] at (img.north west){\circled{d}};
	\end{tikzpicture}&
	\begin{tikzpicture}
    \node[inner sep=0] (img) {\includegraphics[width=\linewidth]{fig/exp_res/pwc_ablation/cave_5_biocc.png}};    
    \node[anchor=north west] at (img.north west){\circled{e}};
	\end{tikzpicture}\\	
	\begin{tikzpicture}
    \node[inner sep=0] (img) {\includegraphics[width=\linewidth]{fig/exp_res/pwc_ablation/cave_6_gt.png}};    
    \node[anchor=north west] at (img.north west){\circled{f}};
	\end{tikzpicture}&
	\begin{tikzpicture}
    \node[inner sep=0] (img) {\includegraphics[width=\linewidth]{fig/exp_res/pwc_ablation/cave_7_irr.png}};    
    \node[anchor=north west] at (img.north west){\circled{g}};
	\end{tikzpicture}&
	\begin{tikzpicture}
    \node[inner sep=0] (img) {\includegraphics[width=\linewidth]{fig/exp_res/pwc_ablation/cave_8_occirr.png}};    
    \node[anchor=north west] at (img.north west){\circled{h}};
	\end{tikzpicture}&
	\begin{tikzpicture}
    \node[inner sep=0] (img) {\includegraphics[width=\linewidth]{fig/exp_res/pwc_ablation/cave_9_bioccirr.png}};    
    \node[anchor=north west] at (img.north west){\circled{i}};
	\end{tikzpicture}&
	\begin{tikzpicture}
    \node[inner sep=0] (img) {\includegraphics[width=\linewidth]{fig/exp_res/pwc_ablation/cave_10_fullmodel.png}};    
    \node[anchor=north west] at (img.north west){\circled{j}};
	\end{tikzpicture}\\ \\
	
	\begin{tikzpicture}
    \node[inner sep=0] (img) {\includegraphics[width=\linewidth]{fig/exp_res/pwc_ablation/temple_1_img.png}};    
    \node[anchor=north west] at (img.north west){\circled{a}};
	\end{tikzpicture}&
	\begin{tikzpicture}
    \node[inner sep=0] (img) {\includegraphics[width=\linewidth]{fig/exp_res/pwc_ablation/temple_2_baseline.png}};    
    \node[anchor=north west] at (img.north west){\circled{b}};
	\end{tikzpicture}&
	\begin{tikzpicture}
    \node[inner sep=0] (img) {\includegraphics[width=\linewidth]{fig/exp_res/pwc_ablation/temple_3_bi.png}};    
    \node[anchor=north west] at (img.north west){\circled{c}};
	\end{tikzpicture}&
	\begin{tikzpicture}
    \node[inner sep=0] (img) {\includegraphics[width=\linewidth]{fig/exp_res/pwc_ablation/temple_4_occ.png}};    
    \node[anchor=north west] at (img.north west){\circled{d}};
	\end{tikzpicture}&
	\begin{tikzpicture}
    \node[inner sep=0] (img) {\includegraphics[width=\linewidth]{fig/exp_res/pwc_ablation/temple_5_biocc.png}};    
    \node[anchor=north west] at (img.north west){\circled{e}};
	\end{tikzpicture}\\	
	\begin{tikzpicture}
    \node[inner sep=0] (img) {\includegraphics[width=\linewidth]{fig/exp_res/pwc_ablation/temple_6_gt.png}};    
    \node[anchor=north west] at (img.north west){\circled{f}};
	\end{tikzpicture}&
	\begin{tikzpicture}
    \node[inner sep=0] (img) {\includegraphics[width=\linewidth]{fig/exp_res/pwc_ablation/temple_7_irr.png}};    
    \node[anchor=north west] at (img.north west){\circled{g}};
	\end{tikzpicture}&
	\begin{tikzpicture}
    \node[inner sep=0] (img) {\includegraphics[width=\linewidth]{fig/exp_res/pwc_ablation/temple_8_occirr.png}};    
    \node[anchor=north west] at (img.north west){\circled{h}};
	\end{tikzpicture}&
	\begin{tikzpicture}
    \node[inner sep=0] (img) {\includegraphics[width=\linewidth]{fig/exp_res/pwc_ablation/temple_9_bioccirr.png}};    
    \node[anchor=north west] at (img.north west){\circled{i}};
	\end{tikzpicture}&
	\begin{tikzpicture}
    \node[inner sep=0] (img) {\includegraphics[width=\linewidth]{fig/exp_res/pwc_ablation/temple_10_fullmodel.png}};    
    \node[anchor=north west] at (img.north west){\circled{j}};
	\end{tikzpicture}\\ \\
\end{tabular}
\caption {\textbf{More qualitative examples from the ablation study on PWC-Net}: \emph{(a)} overlapped input images, \emph{(b)} the original PWC-Net \cite{Sun:2017:PWC}, \emph{(c)} PWC-Net with Bi, \emph{(d)} PWC-Net with Occ, \emph{(e)} PWC-Net with Bi-Occ, \emph{(f)} optical flow ground truth, \emph{(g)} PWC-Net with IRR, \emph{(h)} PWC-Net with Occ-IRR, \emph{(i)} PWC-Net with Bi-Occ-IRR, and \emph{(j)} our full model (i.e. IRR-PWC). Our full model significantly improves flow estimation over the original PWC-Net with fewer missing details and clearer motion boundaries. Note that there are gradual improvements when combining several of the proposed components.}
\label{fig:pwc_ablation_more}
\end{figure*}
}









 

\section{Qualitative Comparison}
\label{sec:comparison}

\subsection{Occlusion estimation}

\Cref{fig:occ_compare_supp} demonstrates a qualitative comparison with the state of the art on occlusion estimation. 
Qualitatively, MirrorFlow \cite{Hur:2017:MFE} misses many occlusions in general, and FlowNet-CSSR-ft-sd \cite{Ilg:2018:OMD} is able to detect fine details of occlusion.
In contrast, our method tries not to miss occlusions, which results in a better recall rate but somewhat lower precision than those of FlowNet-CSSR-ft-sd \cite{Ilg:2018:OMD}.
Overall, our method demonstrates better F1-score than FlowNet-CSSR-ft-sd \cite{Ilg:2018:OMD}, achieving state-of-the-art results on the evaluation dataset (\ie Sintel Train Clean and Final).
Note that FlowNet-CSSR-ft-sd \cite{Ilg:2018:OMD} is additionally trained on the ChairsSDHom dataset \cite{Ilg:2017:FN2} for handling small-displacement motion, which is related to thinly-shaped occlusions.
Our approach is not trained further.

{
\begin{figure*}[!t]
\centering
\footnotesize
\setlength\tabcolsep{0.3pt}
\renewcommand{\arraystretch}{0.2}
\begin{tabular}{>{\centering\arraybackslash}m{.20\textwidth} >{\centering\arraybackslash}m{.20\textwidth} >{\centering\arraybackslash}m{.20\textwidth} >{\centering\arraybackslash}m{.20\textwidth} >{\centering\arraybackslash}m{.20\textwidth}}
	
	\begin{tikzpicture}
    \node[inner sep=0] (img) {\includegraphics[width=\linewidth]{fig/exp_res/occ_comparison/bamboo_1_img.png}};    
    \node[anchor=north west] at (img.north west){\circled{a}};
	\end{tikzpicture}&
	\begin{tikzpicture}
    \node[inner sep=0] (img) {\includegraphics[width=\linewidth]{fig/exp_res/occ_comparison/bamboo_2_gt_orig.png}};    
    \node[anchor=north west] at (img.north west){\circled{b}};
	\end{tikzpicture}&
	\begin{tikzpicture}
    \node[inner sep=0] (img) {\includegraphics[width=\linewidth]{fig/exp_res/occ_comparison/bamboo_3_mirrorflow_vis.png}};    
    \node[anchor=north west] at (img.north west){\circled{c}};
    \node[anchor=north east] at (img.north east){\tiny \color{white} \textbf{F-score: 0.281}};
	\end{tikzpicture}&
	\begin{tikzpicture}
    \node[inner sep=0] (img) {\includegraphics[width=\linewidth]{fig/exp_res/occ_comparison/bamboo_4_flownet_vis.png}};    
    \node[anchor=north west] at (img.north west){\circled{d}};
    \node[anchor=north east] at (img.north east){\tiny \color{white}\textbf{F-score: 0.579}};
	\end{tikzpicture}&
	\begin{tikzpicture}
	\node[inner sep=0] (img) {\includegraphics[width=\linewidth]{fig/exp_res/occ_comparison/bamboo_5_ours_vis.png}};    
    \node[anchor=north west] at (img.north west){\circled{e}};
    \node[anchor=north east] at (img.north east){\tiny \color{white} \textbf{F-score: 0.541}};
	\end{tikzpicture} \\ \\

	\begin{tikzpicture}
    \node[inner sep=0] (img) {\includegraphics[width=\linewidth]{fig/exp_res/occ_comparison/cave_1_img.png}};    
    \node[anchor=north west] at (img.north west){\circled{a}};
	\end{tikzpicture}&
	\begin{tikzpicture}
    \node[inner sep=0] (img) {\includegraphics[width=\linewidth]{fig/exp_res/occ_comparison/cave_2_gt_orig.png}};    
    \node[anchor=north west] at (img.north west){\circled{b}};
	\end{tikzpicture}&
	\begin{tikzpicture}
    \node[inner sep=0] (img) {\includegraphics[width=\linewidth]{fig/exp_res/occ_comparison/cave_3_mirrorflow_vis.png}};    
    \node[anchor=north west] at (img.north west){\circled{c}};
    \node[anchor=north east] at (img.north east){\tiny \color{white} \textbf{F-score: 0.408}};
	\end{tikzpicture}&
	\begin{tikzpicture}
    \node[inner sep=0] (img) {\includegraphics[width=\linewidth]{fig/exp_res/occ_comparison/cave_4_flownet_vis.png}};    
    \node[anchor=north west] at (img.north west){\circled{d}};
    \node[anchor=north east] at (img.north east){\tiny \color{white} \textbf{F-score: 0.688}};
	\end{tikzpicture}&
	\begin{tikzpicture}
	\node[inner sep=0] (img) {\includegraphics[width=\linewidth]{fig/exp_res/occ_comparison/cave_5_ours_vis.png}};    
    \node[anchor=north west] at (img.north west){\circled{e}};
    \node[anchor=north east] at (img.north east){\tiny \color{white} \textbf{F-score: 0.708}};
	\end{tikzpicture} \\ \\
	
	\begin{tikzpicture}
    \node[inner sep=0] (img) {\includegraphics[width=\linewidth]{fig/exp_res/occ_comparison/market_1_img.png}};    
    \node[anchor=north west] at (img.north west){\circled{a}};
	\end{tikzpicture}&
	\begin{tikzpicture}
    \node[inner sep=0] (img) {\includegraphics[width=\linewidth]{fig/exp_res/occ_comparison/market_2_gt_orig.png}};    
    \node[anchor=north west] at (img.north west){\circled{b}};
	\end{tikzpicture}&
	\begin{tikzpicture}
    \node[inner sep=0] (img) {\includegraphics[width=\linewidth]{fig/exp_res/occ_comparison/market_3_mirrorflow_vis.png}};    
    \node[anchor=north west] at (img.north west){\circled{c}};
    \node[anchor=north east] at (img.north east){\tiny \color{white} \textbf{F-score: 0.072}};
	\end{tikzpicture}&
	\begin{tikzpicture}
    \node[inner sep=0] (img) {\includegraphics[width=\linewidth]{fig/exp_res/occ_comparison/market_4_flownet_vis.png}};    
    \node[anchor=north west] at (img.north west){\circled{d}};
    \node[anchor=north east] at (img.north east){\tiny \color{white} \textbf{F-score: 0.307}};
	\end{tikzpicture}&
	\begin{tikzpicture}
	\node[inner sep=0] (img) {\includegraphics[width=\linewidth]{fig/exp_res/occ_comparison/market_5_ours_vis.png}};    
    \node[anchor=north west] at (img.north west){\circled{e}};
    \node[anchor=north east] at (img.north east){\tiny \color{white} \textbf{F-score: 0.392}};
	\end{tikzpicture} \\ \\	
	
	
\end{tabular}
\caption {\textbf{Qualitative comparison of occlusion estimation with the state of the art}: \emph{(a)} overlapped input images, \emph{(b)} occlusion ground truth, \emph{(c)} MirrorFlow \cite{Hur:2017:MFE}, \emph{(d)} FlowNet-CSSR-ft-sd \cite{Ilg:2018:OMD}, and \emph{(e)} ours. In the result image of each method, blue pixels denote \textbf{\color{blue}false positives}, red pixels denote \textbf{\color{red}false negatives}, and white ones denote true positives (\ie correctly estimated occlusion). We include the F-score of each method in the top-right corner. Our model yields a better F-score on the second and the third scene than FlowNet-CSSR-ft-sd \cite{Ilg:2018:OMD}.}
\label{fig:occ_compare_supp}
\end{figure*}
}


\subsection{Bi-directional flows and occlusion maps}

MirrorFlow \cite{Hur:2017:MFE} is one of the most recent related works that estimates bi-directional flow and occlusion maps.
\cref{fig:bidirection_comparison} provides a qualitative comparison with MirrorFlow \cite{Hur:2017:MFE} on the Sintel and KITTI 2015 datasets.
In this comparison, we use our model fine-tuned on the training set of each dataset and display qualitative examples from the validation split.
Comparing to MirrorFlow \cite{Hur:2017:MFE}, our model demonstrates far fewer artifacts and fewer missing details for both flow and occlusion estimation. 
Although there is no ground truth for backward flow nor an occlusion map for the second image available for supervision, our bi-directional model is able to estimate the backward flow and the second occlusion map well while only using the ground truth of forward flow and the occlusion map for the first image (latter is only available on Sintel) during fine-tuning.


{
\begin{figure*}[!b]
\centering
\scriptsize
\setlength\tabcolsep{0.3pt}
\renewcommand{\arraystretch}{0.2}
\begin{tabular}{>{\centering\arraybackslash}m{.20\textwidth} >{\centering\arraybackslash}m{.20\textwidth} >{\centering\arraybackslash}m{.20\textwidth} >{\centering\arraybackslash}m{.20\textwidth} >{\centering\arraybackslash}m{.20\textwidth}}


	\begin{tikzpicture}
    \node[inner sep=0] (img) {\includegraphics[width=\linewidth]{fig/exp_res/bidirection_comparison/sintel/bamboo_2-flow_gt_vis.png}};    
    \node[anchor=north west] at (img.north west){\circled{a}};
	\end{tikzpicture}&
	\begin{tikzpicture}
    \node[inner sep=0] (img) {\includegraphics[width=\linewidth]{fig/exp_res/bidirection_comparison/sintel/bamboo_2-ours-flow_vis.png}};    
    \node[anchor=north west] at (img.north west){\circled{b}};
	\end{tikzpicture}&
	\begin{tikzpicture}
    \node[inner sep=0] (img) {\includegraphics[width=\linewidth]{fig/exp_res/bidirection_comparison/sintel/bamboo_2-ours-flow_b_vis.png}};    
    \node[anchor=north west] at (img.north west){\circled{c}};
	\end{tikzpicture}&
	\begin{tikzpicture}
    \node[inner sep=0] (img) {\includegraphics[width=\linewidth]{fig/exp_res/bidirection_comparison/sintel/bamboo_2-ours-occ_vis.png}};    
    \node[anchor=north west] at (img.north west){\circled{d}};
	\end{tikzpicture}&
	\begin{tikzpicture}
	\node[inner sep=0] (img) {\includegraphics[width=\linewidth]{fig/exp_res/bidirection_comparison/sintel/bamboo_2-ours-occ_b_vis.png}};    
    \node[anchor=north west] at (img.north west){\circled{e}};
	\end{tikzpicture} \\	
	
	\begin{tikzpicture}
    \node[inner sep=0] (img) {\includegraphics[width=\linewidth]{fig/exp_res/bidirection_comparison/sintel/bamboo_2-occ_gt_vis.png}};    
    \node[anchor=north west] at (img.north west){\circled{f}};
	\end{tikzpicture}&
	\begin{tikzpicture}
    \node[inner sep=0] (img) {\includegraphics[width=\linewidth]{fig/exp_res/bidirection_comparison/sintel/bamboo_2-mf-flow_vis.png}};    
    \node[anchor=north west] at (img.north west){\circled{g}};
	\end{tikzpicture}&
	\begin{tikzpicture}
    \node[inner sep=0] (img) {\includegraphics[width=\linewidth]{fig/exp_res/bidirection_comparison/sintel/bamboo_2-mf-flow_b_vis.png}};    
    \node[anchor=north west] at (img.north west){\circled{h}};
	\end{tikzpicture}&
	\begin{tikzpicture}
    \node[inner sep=0] (img) {\includegraphics[width=\linewidth]{fig/exp_res/bidirection_comparison/sintel/bamboo_2-mf-occ_vis.png}};    
    \node[anchor=north west] at (img.north west){\circled{i}};
	\end{tikzpicture}&
	\begin{tikzpicture}
	\node[inner sep=0] (img) {\includegraphics[width=\linewidth]{fig/exp_res/bidirection_comparison/sintel/bamboo_2-mf-occ_b_vis.png}};    
    \node[anchor=north west] at (img.north west){\circled{j}};
	\end{tikzpicture} \\ \\


	\begin{tikzpicture}
    \node[inner sep=0] (img) {\includegraphics[width=\linewidth]{fig/exp_res/bidirection_comparison/sintel/temple_2-flow_gt_vis.png}};    
    \node[anchor=north west] at (img.north west){\circled{a}};
	\end{tikzpicture}&
	\begin{tikzpicture}
    \node[inner sep=0] (img) {\includegraphics[width=\linewidth]{fig/exp_res/bidirection_comparison/sintel/temple_2-ours-flow_vis.png}};    
    \node[anchor=north west] at (img.north west){\circled{b}};
	\end{tikzpicture}&
	\begin{tikzpicture}
    \node[inner sep=0] (img) {\includegraphics[width=\linewidth]{fig/exp_res/bidirection_comparison/sintel/temple_2-ours-flow_b_vis.png}};    
    \node[anchor=north west] at (img.north west){\circled{c}};
	\end{tikzpicture}&
	\begin{tikzpicture}
    \node[inner sep=0] (img) {\includegraphics[width=\linewidth]{fig/exp_res/bidirection_comparison/sintel/temple_2-ours-occ_vis.png}};    
    \node[anchor=north west] at (img.north west){\circled{d}};
	\end{tikzpicture}&
	\begin{tikzpicture}
	\node[inner sep=0] (img) {\includegraphics[width=\linewidth]{fig/exp_res/bidirection_comparison/sintel/temple_2-ours-occ_b_vis.png}};    
    \node[anchor=north west] at (img.north west){\circled{e}};
	\end{tikzpicture} \\	

	\begin{tikzpicture}
    \node[inner sep=0] (img) {\includegraphics[width=\linewidth]{fig/exp_res/bidirection_comparison/sintel/temple_2-occ_gt_vis.png}};    
    \node[anchor=north west] at (img.north west){\circled{f}};
	\end{tikzpicture}&
	\begin{tikzpicture}
    \node[inner sep=0] (img) {\includegraphics[width=\linewidth]{fig/exp_res/bidirection_comparison/sintel/temple_2-mf-flow_vis.png}};    
    \node[anchor=north west] at (img.north west){\circled{g}};
	\end{tikzpicture}&
	\begin{tikzpicture}
    \node[inner sep=0] (img) {\includegraphics[width=\linewidth]{fig/exp_res/bidirection_comparison/sintel/temple_2-mf-flow_b_vis.png}};    
    \node[anchor=north west] at (img.north west){\circled{h}};
	\end{tikzpicture}&
	\begin{tikzpicture}
    \node[inner sep=0] (img) {\includegraphics[width=\linewidth]{fig/exp_res/bidirection_comparison/sintel/temple_2-mf-occ_vis.png}};    
    \node[anchor=north west] at (img.north west){\circled{i}};
	\end{tikzpicture}&
	\begin{tikzpicture}
	\node[inner sep=0] (img) {\includegraphics[width=\linewidth]{fig/exp_res/bidirection_comparison/sintel/temple_2-mf-occ_b_vis.png}};    
    \node[anchor=north west] at (img.north west){\circled{j}};
	\end{tikzpicture} \\ \\
		
	\begin{tikzpicture}
    \node[inner sep=0] (img) {\includegraphics[width=\linewidth]{fig/exp_res/bidirection_comparison/kitti/000002_10_flow_gt_vis.png}};    
    \node[anchor=north west] at (img.north west){\circled{a}};
	\end{tikzpicture}&
	\begin{tikzpicture}
    \node[inner sep=0] (img) {\includegraphics[width=\linewidth]{fig/exp_res/bidirection_comparison/kitti/000002-ours-flow_vis.png}};    
    \node[anchor=north west] at (img.north west){\circled{b}};
	\end{tikzpicture}&
	\begin{tikzpicture}
    \node[inner sep=0] (img) {\includegraphics[width=\linewidth]{fig/exp_res/bidirection_comparison/kitti/000002-ours-flow_b_vis.png}};    
    \node[anchor=north west] at (img.north west){\circled{c}};
	\end{tikzpicture}&
	\begin{tikzpicture}
    \node[inner sep=0] (img) {\includegraphics[width=\linewidth]{fig/exp_res/bidirection_comparison/kitti/000002-ours-occ_vis.png}};    
    \node[anchor=north west] at (img.north west){\circled{d}};
	\end{tikzpicture}&
	\begin{tikzpicture}
	\node[inner sep=0] (img) {\includegraphics[width=\linewidth]{fig/exp_res/bidirection_comparison/kitti/000002-ours-occ_b_vis.png}};    
    \node[anchor=north west] at (img.north west){\circled{e}};
	\end{tikzpicture} \\	
	&
	\begin{tikzpicture}
    \node[inner sep=0] (img) {\includegraphics[width=\linewidth]{fig/exp_res/bidirection_comparison/kitti/000002-mf-flow_vis.png}};    
    \node[anchor=north west] at (img.north west){\circled{g}};
	\end{tikzpicture}&
	\begin{tikzpicture}
    \node[inner sep=0] (img) {\includegraphics[width=\linewidth]{fig/exp_res/bidirection_comparison/kitti/000002-mf-flow_b_vis.png}};    
    \node[anchor=north west] at (img.north west){\circled{h}};
	\end{tikzpicture}&
	\begin{tikzpicture}
    \node[inner sep=0] (img) {\includegraphics[width=\linewidth]{fig/exp_res/bidirection_comparison/kitti/000002-mf-occ_vis.png}};    
    \node[anchor=north west] at (img.north west){\circled{i}};
	\end{tikzpicture}&
	\begin{tikzpicture}
	\node[inner sep=0] (img) {\includegraphics[width=\linewidth]{fig/exp_res/bidirection_comparison/kitti/000002-mf-occ_b_vis.png}};    
    \node[anchor=north west] at (img.north west){\circled{j}};
	\end{tikzpicture} \\ \\
	
	\begin{tikzpicture}
    \node[inner sep=0] (img) {\includegraphics[width=\linewidth]{fig/exp_res/bidirection_comparison/kitti/000116_10_flow_gt_vis.png}};    
    \node[anchor=north west] at (img.north west){\circled{a}};
	\end{tikzpicture}&
	\begin{tikzpicture}
    \node[inner sep=0] (img) {\includegraphics[width=\linewidth]{fig/exp_res/bidirection_comparison/kitti/000116-ours-flow_vis.png}};    
    \node[anchor=north west] at (img.north west){\circled{b}};
	\end{tikzpicture}&
	\begin{tikzpicture}
    \node[inner sep=0] (img) {\includegraphics[width=\linewidth]{fig/exp_res/bidirection_comparison/kitti/000116-ours-flow_b_vis.png}};    
    \node[anchor=north west] at (img.north west){\circled{c}};
	\end{tikzpicture}&
	\begin{tikzpicture}
    \node[inner sep=0] (img) {\includegraphics[width=\linewidth]{fig/exp_res/bidirection_comparison/kitti/000116-ours-occ_vis.png}};    
    \node[anchor=north west] at (img.north west){\circled{d}};
	\end{tikzpicture}&
	\begin{tikzpicture}
	\node[inner sep=0] (img) {\includegraphics[width=\linewidth]{fig/exp_res/bidirection_comparison/kitti/000116-ours-occ_b_vis.png}};    
    \node[anchor=north west] at (img.north west){\circled{e}};
	\end{tikzpicture} \\	
	&
	\begin{tikzpicture}
    \node[inner sep=0] (img) {\includegraphics[width=\linewidth]{fig/exp_res/bidirection_comparison/kitti/000116-mf-flow_vis.png}};    
    \node[anchor=north west] at (img.north west){\circled{g}};
	\end{tikzpicture}&
	\begin{tikzpicture}
    \node[inner sep=0] (img) {\includegraphics[width=\linewidth]{fig/exp_res/bidirection_comparison/kitti/000116-mf-flow_b_vis.png}};    
    \node[anchor=north west] at (img.north west){\circled{h}};
	\end{tikzpicture}&
	\begin{tikzpicture}
    \node[inner sep=0] (img) {\includegraphics[width=\linewidth]{fig/exp_res/bidirection_comparison/kitti/000116-mf-occ_vis.png}};    
    \node[anchor=north west] at (img.north west){\circled{i}};
	\end{tikzpicture}&
	\begin{tikzpicture}
	\node[inner sep=0] (img) {\includegraphics[width=\linewidth]{fig/exp_res/bidirection_comparison/kitti/000116-mf-occ_b_vis.png}};    
    \node[anchor=north west] at (img.north west){\circled{j}};
	\end{tikzpicture} \\ \\
	
	
\end{tabular}
\caption {\textbf{Qualitative comparison of the bi-drectional optical flows and occlusion maps in both views with MirrorFlow \cite{Hur:2017:MFE}}: All results are overlayed on the corresponding image, either the first frame or the second frame. \emph{(a)} Ground truth optical flow, \emph{(b)} our forward flow, \emph{(c)} our backward flow, \emph{(d)} our occlusion map for the first frame, \emph{(e)} our occlusion map for the second frame, \emph{(f)} ground truth occlusion map, \emph{(g)} forward flow of MirrorFlow, \emph{(h)} backward flow of MirrorFlow, \emph{(i)} occlusion map of MirrorFlow for the first frame, \emph{(j)} occlusion map of MirrorFlow for the second frame. Note that KITTI has only sparse ground truth for optical flow and does not provide ground truth for occlusion.}
\label{fig:bidirection_comparison}
\end{figure*}
}
 
\end{document}
