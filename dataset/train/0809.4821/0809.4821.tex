\documentclass{elsart}
\usepackage[normalem]{ulem}\usepackage{graphicx}
\usepackage{latexsym,amssymb,amsmath,amssymb,xspace,theorem}
\usepackage{multicol}
\usepackage{t1enc}
\usepackage{epic,eepic}
\usepackage{algorithmic}
\usepackage{subfigure}
\theoremstyle{plain} \theoremheaderfont{\scshape}
\newtheorem{inv}{Invariant}
\newtheorem{Thm}{\bf Theorem}
\newtheorem{Lem}[Thm]{\bf Lemma}
\newtheorem{Clm}{Claim}[Thm]

\newtheorem{claimsansnum}{Claim}
\newtheorem{Conj}[Thm]{{\bf Conjecture}}

\newtheorem{Prop}[Thm]{\bf Proposition}
\newtheorem{Cor}[Thm]{ \bf Corollary}
{\theorembodyfont{\rmfamily}
 \newtheorem{Def}[Thm]{\bf Definition}
 \newtheorem{Rem}[Thm]{\bf Remark}
 \newtheorem{Question}[Thm]{\bf Question}
 \newtheorem{Problem}[Thm]{\bf Problem}
}

\renewcommand{\theClm}{\arabic{Clm}}
\renewcommand{\theConj}{\arabic{Conj}}
\renewcommand{\theQuestion}{\arabic{Question}}
\renewcommand{\theProblem}{\arabic{Problem}}
\renewcommand{\theThm}{\arabic{Thm}}


\newenvironment{Prf}{{\bf \noindent Proof } }{\hfill\\}
\newenvironment{PrfClaim}{{\bf Proof }}{{\hfill\tiny{\\}}}

\newcommand{\ignore}[1]{}
\newcommand{\ligne}{\vspace{0.5cm}}
\newcommand{\trou}{\vspace{5mm} \noindent}

\newcommand{\cqfd}{\unskip\kern 6pt\penalty 500
\raise -2pt\hbox{\vrule\vbox to 10pt{\hrule width 4pt
\vfill\hrule}\vrule}\par}

\newcommand{\dm}{minimum edge-colouring \xspace}
\newcommand{\dms}{minimum edge-colourings \xspace}
\newcommand{\dmv}{minimum edge-colouring}
\newcommand{\sct}{}


\newcommand{\ppm}{}
\newcommand{\stp}{}
\newcommand{\np}{}
\newcommand{\npprime}{}
\newcommand{\npsecond}{}
\newcommand{\stprime}{}
\newcommand{\cubthree}{cubic -edge colourable graph \xspace}
\newcommand{\cubthrees}{cubic -edge colourable graphs \xspace}
\newcommand{\cubthreev}{cubic -edge colourable graph}
\newcommand{\cubthreesv}{cubic -edge colourable graphs}
\newcommand{\threenc}{,  and  \xspace}
\newcommand{\otherthreenc}{,  and  \xspace}
\newcommand{\TroisPNICsv}{three compatible normal odd partitions}
\newcommand{\TroisPNIC}{three compatible normal odd partitions \xspace}
\newcommand{\Ajoute}[1]{{{\uwave{#1}}} }\newcommand{\Enleve}[1]{{\xout{#1}} }



\input{epsf.sty}
\begin{document}
\begin{frontmatter}




\title{On Fan Raspaud conjecture}
\author{J.L. Fouquet and J.M. Vanherpe}


\address{L.I.F.O., Facult\'e des Sciences, B.P. 6759 \\
Universit\'e d'Orl\'eans, 45067 Orl\'eans Cedex 2, FR}
\begin{abstract}
A conjecture of Fan and Raspaud \cite{FanRas} asserts that every
bridgeless cubic graph contains three perfect matchings with empty
intersection. Kaiser and Raspaud \cite{KaiRas} suggested a possible
approach to this problem based on the concept of a balanced join in
an embedded graph. We give here some new results concerning this
conjecture and prove that a minimum counterexample must have at
least  vertices.
\end{abstract}
\begin{keyword}
Cubic graph;  Edge-partition;
\end{keyword}
\end{frontmatter}




\section{Introduction}
Fan and Raspaud \cite{FanRas} conjectured that any bridgeless cubic
graph can be provided with three perfect matchings with empty
intersection (we shall say also {\em non intersecting perfect
matchings}).
\begin{Conj}\cite{FanRas} \label{Conjecture:FanRaspaud} Every
bridgeless cubic graph contains perfect matching , , 
such that

\end{Conj}

This conjecture seems to be originated independently by Jackson.
Goddyn \cite{God} indeed mentioned this problem proposed by Jackson
for graphs (regular graphs with an even number of vertices
such that all odd cuts have size at least , as defined by Seymour
\cite{Sey}) in the proceedings of a joint summer research conference
on graphs minors which dates back 1991.

\begin{Conj}\cite{God} \label{Conjecture:Jackson}There exists  such
that any r-graph contains  perfect matchings with empty
intersection.
\end{Conj}
Seymour \cite{Sey} conjectured that:
\begin{Conj}\cite{Sey} \label{Conjecture:Seymour}If  then any
r-graph has a perfect matching whose deletion yields an (r-1)-graph.
\end{Conj}

Hence Seymour's conjecture leads to a  specialized form of Jackson's
conjecture when dealing with cubic bridgeless graphs and the Fan
Raspaud conjecture appears as a refinement of Jackson's conjecture.

 A {\em join} in a graph  is a set  such that the degree of every vertex in G has the same parity
as its degree in the graph . A perfect matching being a
particular join in a cubic graph Kaiser and Raspaud conjectured in
\cite{KaiRas}

\begin{Conj}\cite{KaiRas} \label{Conjecture:KaiserRaspaud} Every
bridgeless cubic graph admits two perfect matching ,  and
a join  such that

\end{Conj}


The {\em oddness} of a cubic graph  is the minimum number of odd
circuits in a 2-factor of . Conjecture
\ref{Conjecture:FanRaspaud} being obviously true for cubic graphs
with chromatic index , we shall be concerned here by bridgeless
cubic graphs with chromatic index . Hence any 2-factor of such a
graph has at least two odd cycles. The class of  bridgeless cubic
graphs with oddness two is, in some sense, the "easiest" class to
manage with in order to tackle some well known conjecture.  In
\cite{KaiRas} Kaiser and Raspaud proved that Conjecture
\ref{Conjecture:KaiserRaspaud} holds true for bridgeless cubic graph
of oddness  two. Their proof is based on the notion of {\em balanced
join} in the multigraph obtained in contracting the cycles of a two
factor. Using an equivalent formulation of this notion in the next
section, we shall see that we can get some new results on Conjecture
\ref{Conjecture:FanRaspaud} with the help of this technique.





For basic graph-theoretic terms, we refer the reader to Bondy and
Murty \cite{BonMur}.





\section{Preliminary results}
Let  be a perfect matching of a  cubic graph and let  be the 2-factor .  is
a {\em balanced} matching  whenever there is a perfect matching
 such that . That means that each odd cycle of
 is incident to at least one edge in  and the
subpaths determined by the ends of  on the cycles of  incident to  have odd lengths.

In the following example,  is the perfect matching (thick edges)
of the Petersen graph. Taking any edge ( by example) of this
perfect matching we are led to a balanced matching since the two
cycles of length  give rise to two paths of length  (we have
"opened" these paths closed to  and ). Remark that given a
perfect matching  of a bridgeless cubic graph,  is obviously a
balanced matching.


\begin{figure}[t]
\centering \epsfsize=0.35 \hsize \noindent \epsfbox{petersen.eps}
\caption{A balanced matching} \label{Figure:Petersen}
\end{figure}

Kaiser and Raspaud \cite{KaiRas} introduced this notion via the
notion of {\em balanced join} in  the context of a combinatorial
representation of graphs embedded on surfaces. They remarked that a
natural approach to the Fan Raspaud conjecture would require finding
two disjoint balanced joins and hence two balanced matchings for
some perfect matching . In fact  Conjecture
\ref{Conjecture:FanRaspaud} and balanced matching are related by the
following lemma

\begin{Lem}\label{Lemma:FondamentalDisjointsBalanced}
A bridgeless cubic graph contains  non intersecting perfect
matching if and only if there is a perfect matching  and two
balanced disjoint balanced matchings.
\end{Lem}
\begin{Prf} Assume that , ,  are three perfect
matchings of  such that . Let
,  and . Since ,  and  are two balanced matchings with
empty intersection.

Conversely, assume that  is a perfect matching and that  and
 are two balanced matchings with empty intersection. Let
,  be a perfect matching such that 
 and  be a perfect
matching such that . We have  and the three perfect matchings ,  and 
have an empty intersection.
\end{Prf}

 The following theorem is a corollary of Edmond's Matching Polyhedron
Theorem \cite{Edm}. A simple proof is given by Seymour in
\cite{Sey}.

\begin{Thm} \label{Theorem:EdmondsSeymour} Let  be an -graph. Then there is an integer 
and a family  of perfect matchings such that each edge
of  is contained in precisely  members of .
\end{Thm}

\begin{Lem}\label{Lemma:TwoEdgesAvoiding} Let  be a bridgeless
cubic graph and let  and  be two edges of . Then
there exists a perfect matching avoiding these two edges.
\end{Lem}
\begin{Prf}  Remark that a bridgeless cubic graph is a -graph as defined by Seymour.
Applying Theorem \ref{Theorem:EdmondsSeymour}, let  be a
set of perfect matching such that each edge of  is contained in
precisely  members of  (for some fixed integer ).

Assume first that  and  have a common end vertex (say ).
Then  is incident to a third edge . Any perfect matching
using  avoids  and .

When  and  have no common end then, let  and  be the
two edges incident with . Assume that any perfect matching using
 or  contains also the edge . Then  is contained in
 members of , impossible. Hence some perfect
matchings using  or  must avoid , as claimed.
\end{Prf}

It can be pointed out that Lemma \ref{Lemma:TwoEdgesAvoiding} is not
extendable, so easily, to a larger set of edges. Indeed, a corollary
of Theorem \ref{Theorem:EdmondsSeymour} asserts that 
(the family of perfect matching considered) intersects each edge
cut in exactly one edge. Hence for such a edge cut, there is no
perfect matching in  avoiding this set.

Let  be an odd cycle and let  a set of three
distinct vertices of . We shall say that  is a {\em balanced
triple} when the three subpaths of  determined by  have odd
lengths.

\begin{figure}[t]
\centering \epsfsize=0.25 \hsize \noindent
\epsfbox{balancedtriple.eps} \caption{A balanced triple}
\label{Figure:BalancedTriple}
\end{figure}

Let  be an odd cyle of length at least .
Assume that its vertex set is coloured with three colours , 
and  such that , 
denoting the set of vertices coloured with , . Then we
shall say that  is {\em good odd cycle}.

\begin{Lem}\label{Lemma:DisjointBalancedTriples}  Any good odd cycle 
contains two disjoint balanced triples  and  intersecting
each colour exactly once.
\end{Lem}
\begin{Prf} We
shall prove this lemma by induction on .

Assume first that  has length . Then  and  have
exactly two vertices while  must have  vertices. We can
distinguish, up to isomorphism,  subcases

\begin{enumerate}
  \item   and
   then  and  are  two
  disjoint balanced triples.
  \item   and
   then  and  are  two
  disjoint balanced triples.
  \item  and
   then  and  are  two
  disjoint balanced triples.
  \item  and
   then  and  are  two
  disjoint balanced triples.
  \item  and
   then  and  are  two
  disjoint balanced triples.
  \item  and
   then  and  are  two
  disjoint balanced triples.
  \item  and
   then  and  are  two
  disjoint balanced triples.
  \item  and
   then  and  are  two
  disjoint balanced triples.
  \item  and
   then  and  are  two
  disjoint balanced triples.
\end{enumerate}

Assume that  is a good odd cycle of length at least  and
assume that the property holds for any good odd cycle of length
.

\begin{Clm} \label{Claim:Claim1DisjointBalancedTriples}
If  has  two consecutive vertices  and  ( being
taken modulo ) in the same set  ( or ) such that
, then the property holds.
\end{Clm}
\begin{PrfClaim} Assume that  has two consecutive
vertices  and   in the same set  ( or )
such that , then delete  and  and add
the edge . We get hence a good odd cycle  of
length .  has two disjoint balanced triples  and 
by induction hypothesis and we can check that these two triples are
also balanced in  since the edge  is replaced by
the path  in .
\end{PrfClaim}

\begin{Clm} \label{Claim:Claim2DisjointBalancedTriples}
If  has  two consecutive vertices  and  ( being
taken modulo ) one of them being in  while the other is in
 (), then the property holds as
soon as  and .
\end{Clm}
\begin{PrfClaim} Use the same trick as in the proof of Claim
\ref{Claim:Claim1DisjointBalancedTriples}
\end{PrfClaim}

If , we can suppose, by Claim
\ref{Claim:Claim1DisjointBalancedTriples} that no two vertices of
 are consecutive on . When ,  (its succesor
in the natural ordering) is in  or . By Claim
\ref{Claim:Claim2DisjointBalancedTriples}, the vertices in 
have at most two successors in  and at most two successors in
. Hence we must have  and ,
impossible. If  then we must have   since
 has length . In that case we certainly have two consecutive
vertices with distinct colours and we can apply the above claim
\ref{Claim:Claim1DisjointBalancedTriples}.
\end{Prf}

Let  be an even cycle and let  a set of two distinct
vertices of . We shall say that  is a {\em balanced pair} when
the two subpaths of  determined by  have odd lengths.

Let  be an even cyle of length at least
. Assume that its vertex set is coloured with three colours ,
 and . Let   be  the set of vertices coloured with ,
. Assume that   or  for at most one colour,
then we shall say that  is {\em good even cycle}.

\begin{Lem}\label{Lemma:DisjointBalancedPairs}  Any good even cycle 
contains two disjoint balanced pairs  and  intersecting
  exactly once each as soon as  has at least two vertices
().
\end{Lem}
\begin{Prf}
We prove the lemma for . Assume that  and . Assume that  is a vertex in  and let  be the
first vertex in ,  be the last vertex in  when
running on   in the sens given by . If  or   and  are two
distinct balanced pairs intersecting  exactly once each. Assume
that  and . Since  contains another vertex 
(). Let  be the first vertex in  when running
from  to  (. Then  and
 are two disjoint balanced pairs intersecting
 exactly once each.
\end{Prf}

\begin{Lem}\label{Lemma:DisjointBalancedPairsSpecialEvenCycle}  Let
 be an even cycle of length  and let  and  be
two vertices. Assume that the vertices of  are
partitioned into  and  with  and .
Then there are at least two disjoint balanced pairs intersecting 
and  exactly once each.
\end{Lem}
\begin{Prf}
Let us colour alternately the vertices of  in red and blue. If
 contains at least two red (or blue) vertices  and  and 
two blue (or red respectively) vertices  and  then
 and  are two disjoint balanced pairs. If
 contains a red vertex  and a blue vertex  and,
symmetrically,  contains a red vertex  and a blue vertex 
then  and  are two disjoint balanced. It is
clear that at least one of the above cases must happens and the
result follows.
\end{Prf}


\section{Applications}

From now on, we consider that our graphs are cubic, connected and
bridgeless (multi-edges are allowed). Moreover we suppose that they
are not edge colourable. Hence these graphs have perfect
matchings and any 2-factor have a non null even number of odd
cycles. If  and ,  is the
length of a shortest path between these two sets.

\subsection{Graphs with small oddness}
\begin{Thm} \label{Theorem:Distance3Oddness2}Let  be a cubic graph of oddness two. Assume that 
has a perfect matching  where the factor  of  is such that  and  are
the only odd cycles and . Then  has three
perfect matchings with an empty intersection.
\end{Thm}
\begin{Prf}

If  let  be an edge joining  and  ( and ).  is a balanced matching.
Let  be a perfect matching such that . There is
certainly a perfect matching  avoiding  (see Theorem
\ref{Theorem:EdmondsSeymour}). Hence ,  and  are three
perfect matchings with an empty intersection.

It can be noticed that . Indeed, Let
 be a shortest path joining  to , then the cycle of  containing  cannot be
disjoint from  or , impossible.

Assume thus now that  and let  be a
shortest path joining  to  (with  and ). Then  is a balanced matching.
Let  be a perfect matching such that . From
Lemma \ref{Lemma:TwoEdgesAvoiding} there is a perfect matching 
avoiding these two edges of . Hence ,  and  are
three non intersecting perfect matchings
\end{Prf}

A graph  is {\em near-bipartite} whenever there is an edge  of
 such that  is bipartite.

\begin{Thm} \label{Theorem:3cyclesOddness2}Let  be a cubic graph of oddness two. Assume that 
has a perfect matching  where the factor  of
 has only  cycles  (odds) and  (even) such
that the subgraph of  induced by  is a near-bipartite graph.
Then  has three perfect matchings with an empty intersection.
\end{Thm}
\begin{Prf}
From Theorem \ref{Theorem:Distance3Oddness2}, we can suppose that
. That means that the neighbors of  are
contained in  as well as those of . Let us colour the
vertices of  with two colours red and blue alternately along
. Assume that  and  are two vertices of  with
distinct colours such that  is a neighbor of  and  is a
neighbor of . Let  and  be the two edges of  so
determined by  and . Then  is a balanced
matching. Let  be a perfect matching such that  and  be a perfect matching avoiding  (Lemma
\ref{Lemma:TwoEdgesAvoiding}). Then  and  are  non
intersecting perfect matchings.

It remains thus to assume that the neighbors of  and  have
the same colour (say red).  being bridgeless, we have an odd
number (at least ) of edges in  joining  and  (
and  respectively). The remaining vertices of  are matched
by edges of , but we have at least  blue vertices more than
red vertices in  to be matched and hence at least three pairs
of blue vertices must be matched. Let  such that 
is bipartite, if  then  must have odd length,
impossible. Hence  is the only chord of  whose ends have the
same colour, impossible.

\end{Prf}
\begin{Thm} \label{Theorem:Oddness4Distance1}
Assume that  is a cubic graph having a perfect matching  where
the factor  of 
is such that , ,  and  are the only odd cycles.
Assume moreover that  as well as . Then
 has three perfect matchings with an empty intersection.
\end{Thm}
\begin{Prf}
Let  be an edge joining  to  and  be an
edge joining  to .  is a balanced
matching. Let  be a perfect matching such that . By Lemma \ref{Lemma:TwoEdgesAvoiding}, there is a perfect
matching  avoiding these two edges. Hence the three perfect
matchings ,  and  are non intersecting.
\end{Prf}

\begin{Thm} \label{Theorem:4OddChordlessCycles}
Assume that  has a perfect matching  where the factor
 has only  chordless cycles . Then  has three perfect matchings with an
empty intersection.
\end{Thm}
\begin{Prf}
By the connectivity of , every vertex of three cycles of
 (say  and ) are joined to  while no
other edge exists. Otherwise the result holds by Theorem
\ref{Theorem:Oddness4Distance1}.

Each cycle of  has length at least  and, hence 
has length at least . We can colour each vertex  with
,  or  following the fact the edge of  incident with 
has its other end on ,  or . From lemma
\ref{Lemma:DisjointBalancedTriples}, there is two balanced triples
 and  intersecting  each colour. These two balanced triples
determine two disjoint balanced matchings. Hence, the result
holds from Lemma \ref{Lemma:FondamentalDisjointsBalanced}.
\end{Prf}

\subsection{Good Rings, Good stars}

A {\em good path of index } is a set  of  disjoint
cycles  such that

\begin{itemize}
  \item  and  are the only odd cycles of 
  \item  is joined to  ()  by an
  edge  (called {\em jonction edge of index })
  \item the two jonction edges incident to an even cycle
  determine two odd paths on this cycle
\end{itemize}



A {\em good ring} is a set  of disjoint odd cycles  and even cycles such that
\begin{itemize}
  \item  is joined to  ( is taken modulo ) by a
  good path  of index  whose even cycles are in 
  \item the good paths involved in  are pairwise disjoint.
\end{itemize}


A {\em good star} ({\em centered in }) is a set  of four
disjoint cycles  such that
\begin{itemize}
  \item  (the center) is chordless and has length at least 
  \item  is joined to each other cycle by at
  least two edges and has no neighbor outside of 
  \item there is no edge between ,  and 
\end{itemize}

\begin{Thm} \label{Theorem:GoodRingGoodStars}
Assume that  has a perfect matching  where the factor
 of  can be partitioned into good rings, good stars
and even cycles. Then  has three perfect matchings with an empty
intersection.
\end{Thm}
\begin{Prf}
Let  be the set of good rings of  and
 be the set of good stars.

Let , and let  be its set of
odd cycles.  Let us us say that a junction edge of  has an even
index whenever this edge is a junction edge of index  with 
even. A junction edge of odd index is defined in the same way. Let
 be the set of junction edge of even index of  and  the
set of junction edge of odd index. We let  and .

For each star , assume that each vertex of the
center is coloured with the name of the odd cycle of  to whom
this vertex is adjacent.  Let  and  be two disjoint
balanced triples (Lemma \ref{Lemma:DisjointBalancedTriples}) of the
center of . Let  and  be the sets of three edges
joining the center of  to the other cycles of , determined by
 and . Let  and
.




It is an easy task to check that  and  are two
disjoint balanced matchings. Hence, the result holds from Lemma
\ref{Lemma:FondamentalDisjointsBalanced}.
\end{Prf}

A particular case of the above result is given by E.
M\`{a}\v{c}ajov\'{a} and M. \v{S}koviera. The length of a ring is
the number of jonction edges. A ring of length  is merely a set
of two odd cycles joined by two edges.
\begin{Cor}\cite{MacSko}Assume that  has a perfect matching  where the odd cycles of the factor
 can be arranged into rings of length . Then  has
three perfect matchings with an empty intersection.
\end{Cor}

It can be pointed out that this technique of rings of length  was
used in \cite{Fou85} for the  flow problem when dealing with
graphs of small order and graphs with low genus. This technique has
been developped independently  by Steffen in \cite{Ste96}.



\section{On graphs with at most  vertices}

Determining the structure of a minimal counterexample to a
conjecture is one of the most typical methods in Graph Theory. In
this section we investigate some basic structures of minimal
counterexamples to Conjecture \ref{Conjecture:FanRaspaud}.

The {\em girth} of a graph is the length of shortest cycle.
M\`{a}\v{c}ajov\'{a} and  \v{S}koviera \cite{MacSko} proved that the
girth of a minimal counterexample is at least .
\begin{Lem}\cite{MacSko} \label{Lemma:Girth5}  If  is a smallest bridgeless
cubic graph with no  non-intersecting perfect matchings, then the
girth of  is at least 
\end{Lem}


\begin{figure}[t]
\begin{center}
\centering\epsfsize=0.30 \hsize\epsfbox {G8.eps} \caption{}
 \label{Figure:G8}
\end{center}
\end{figure}




\begin{Lem}\label{Lemma:Graph8}  If  is a smallest bridgeless
cubic graph with no  non-intersecting perfect matchings, then 
does not contain a subgraph isomorphic to   (see Figure
\ref{Figure:G8}).
\end{Lem}
\begin{Prf}
Assume that  contains . Let  and  be the
vertices of  adjacent to, respectively  and . Let
 be the graph obtained in deleting  and joining  to
 and  to . It is an easy task to verify that  has
chromatic index  if and only if  itself has chromatic index
.   We do not know whether this graph is connected or not but
each component is smaller than  and contains thus 
non-intersecting perfect matchings leading to  non-intersecting
perfect matchings for . Let   and  these perfect
matchings. Our goal is to construct  non-intersecting perfect
matchings for  ,  and  from those of . We have
thus to delete the edge  and  from ,  and
 whenever they belong to these sets and add some edges of 
in order to obtain the perfect matchings for .

Let us now consider the number of edges in  which are
contained in  or in  or in .

When none of , or  contain
 nor  we set ,  and .

Assume that the edges  and  both belong to some  (), say . In this case 
cannot contain one of those edges. Thus we write
, and
.

Finally assume w.l.o.g that . When  we set ,  and . On the last hand, if one of
the sets  or  (say ) contain
the edge , we write , and
.

In all cases, since  we have .

\end{Prf}

\begin{Lem}\label{Lemma:P-vExcluded}  If  is a smallest bridgeless
cubic graph with no  non-intersecting perfect matchings, then 
does not contain a subgraph isomorphic to the Petersen graph with
one vertex deleted.
\end{Lem}
\begin{Prf}
Let  be a graph isomorphic to the Petersen graph whose vertex set
is  and such that  and  are
the two odd cycles of the -factor associated to the perfect
matching . Assume that  is a subgraph
of . Let ,  and  be respectively the neighbors of
,  and  in . Let  be the graph whose vertex set is
 where  is a new vertex and whose
edge set is . Since  is smaller
than ,  contains  non-intersecting perfect matchings
, , .

For  we can associate to  two perfect matchings
of  , namely  and , as follows (observe that exactly
one of the edges ,  or  belongs to )~:
\begin{description}
\item When  we set  \\and .
\item When  we set  \\ and .
\item When  we set  \\ and .
\end{description}
But now, if on one hand  contains one of the edges in
  for some  and  for  distinct from  and , ,a contradiction. If, on
the other hand, each of ,  and  (for , , 
distinct members of ) contains exactly one edge of
 we also have , a contradiction.
\end{Prf}
\begin{Thm} \label{Theorem:MinimumCounterExample32}
If  is a smallest bridgless cubic graph with no 
non-intersecting perfect matchings, then  has at least 
vertices
\end{Thm}
\begin{Prf}
Assume to the contrary that  is a counterexample with at most
 vertices. We can obviously suppose that  is connected. Let
 be a perfect matching and let  be the factor of
. Assume that the number of odd cycles of  is the
oddness of .  Since  has girth at least  by Lemma
\ref{Lemma:Girth5}, the oddness of  is ,  or .
\begin{Clm} \label{Claim:Claim1MinimummCounterExample32}
 has oddness  or .
\end{Clm}
\begin{PrfClaim} Assume  that  has oddness . We have
 and each cycle  () is chordless and has length . Each cycle  is
joined to at least two other cycles of . Otherwise, if
 is joined to only one cycle  (), these two
cycles would form a connected component of  and   would not be
connected, impossible. It is an easy task to see that we can thus
partition   into good rings and the results comes from
Theorem \ref{Theorem:GoodRingGoodStars}.
\end{PrfClaim}

Assume now that  has oddness . Hence  contains 
odd cycles  and . Since these cycles have length
at least ,  contains eventually an even cycle .
From Lemmas \ref{Lemma:Girth5} and  \ref{Lemma:Graph8}
 if  exists,   is a chordless cycle of length  or  has length
 (with at most one chord) or . When  has length ,
 and  are chordless cycles of length . When
 has length ,  and  are chordless cycles
of length  or  of them have length  while the last one has
length .




Theorem \ref{Theorem:Oddness4Distance1} says that we are done as
soon as we can  find two edges allowing to arrange by pairs
 and   (say for example  joined to  and
 to )  and Theorem \ref{Theorem:GoodRingGoodStars} says
that we are done whenever these  odd cycles induce a good star.
That means that the subgraph  induced  by the four odd cycles is
of one of the two following types:

\begin{itemize}
  \item [Type 1]One odd cycle (say ) has all its neighbors in  and the 
other odd cycles induce a connected subgraph
  \item [Type 2] One cycle (say
) is joined to the other by at least one edge while the others
are not adjacent.
\end{itemize}

\begin{Clm} \label{Claim:Claim2_1_MinimumCounterExample32}
 has length at least .
\end{Clm}
\begin{PrfClaim}
Assume that , the girth of  being at least  (Lemma
\ref{Lemma:Girth5}) we can suppose that  has no chord.  is
not of type , otherwise  having its neighbors in ,
 is connected to the remaining part of  with one edge only,
impossible since  is bridgeless. Assume thus that  is of type
. Then, there are  edges between  and . Since there
are at least  edges going out  and  that means
that there are at least  edges between  and the other odd
cycles. Hence,  must have length  and can not be adjacent to
.  is then partitioned into a good star and an even cycle
and the result comes from Theorem \ref{Theorem:GoodRingGoodStars}.
\end{PrfClaim}

\begin{Clm} \label{Claim:Claim2_2_MinimumCounterExample32}
If  has length  then it has no chord.
\end{Clm}
\begin{PrfClaim} If  has a chord then there are at most 
edges joining  to . If  is of type  then  has at
least  neighbors in . Hence there is at most one edge
between  and , impossible. If  is of type , then the
the three cycles ,  and  have at least  neighbors
in , impossible since  has at most  vertices.
\end{PrfClaim}


\begin{Clm} \label{Claim:Claim3MinimumCounterExample32}
If  exists then  is not of type 1.
\end{Clm}
\begin{PrfClaim}
If  is of type 1, then  has its neighbors (at least ) in
 and there are  or  edges between  and .









Whenever there are  edges between  and ,  has length
 and  have length  (as well as ). In that
case w.l.o.g., we can consider that  is joined by exactly one
edge to  and joined by  edges to . The las t neighbor
of  cannote be on , otherwise the  neighbors of 
are on  and  must have length , impossible. Hence,
 is joined to  by exactly one edge and  is joined to
 by  edges. Let us colour each vertex  of  with
 or  when  is adjacent to  (). From Lemma
\ref{Lemma:DisjointBalancedPairs}, we can find  disjoint balanced
pairs on   and  with  and 
coloured with ,  and  coloured with . These two pairs
determine two disjoint set of edges  and  in
 and allow us to construct two disjoint balanced matchings
 and  in choosing two distinct edges 
and  between  and . The result follows from Lemma
\ref{Lemma:FondamentalDisjointsBalanced}




Whenever there are  edges between  and ,  has length
 or , any two cycles in  are joined by at
least two edges and  each of them is joined to  by exactly one
edge. Let  be the three vertices of  which are the neighbors
of . Let  be the neighbors of  on
. When  has length  this cycle induces a  chord .
In that case, Lemma
\ref{Lemma:DisjointBalancedPairsSpecialEvenCycle} says that we can
find  disjoint balanced pairs  and  with
  and  . These two pairs determine two
disjoint set of edges  and  in  and allow
us to construct two disjoint balanced matchings 
and  in choosing two suitable distinct edges  and
 joining two of the cycles in . When  has
no chord, we can apply the same technique in choosing  and  in
.

The result follows from Lemma
\ref{Lemma:FondamentalDisjointsBalanced}.

\end{PrfClaim}

\begin{Clm} \label{Claim:Claim4_1_MinimumCounterExample32}
if   is  of type 2 then  has  vertices.
\end{Clm}
\begin{PrfClaim} When  has length , this cycle has no chord. Otherwise, we
have at most  edges between  and . Hence  and
 are joined to  with at least  edges, impossible since
 hat at most  vertices. Assume thus that  is a chordless
cycle of length  then there are  edges going out  and at most  of them are incident to . Hence
there are  edges between  and , 
edges between  and  and henceforth no
edge between  and . One cycle in  has
exactly one neighbor in  (say ) or two of them (say 
and ) have this property .


It is an easy task to find a balanced triple  on  where
 is a neighbor of ,  a neighbor of  and  a
neighbor of . This balanced triple determine a balanced
matching . We can construct a balanced matching 
disjoint from  in choosing two edges  end  connecting  to  whose ends are adjacent on  (since  or
 edges are involved between these two sets) and an edge  between  and . The result follows from Lemma
\ref{Lemma:FondamentalDisjointsBalanced}

\end{PrfClaim}

\begin{Clm} \label{Claim:Claim4_2_MinimumCounterExample32}
If  exists then  is  not of type 2.
\end{Clm}
\begin{PrfClaim} From claim \ref{Claim:Claim4_1_MinimumCounterExample32}, it remains to assume that
  has length . Then  is joined to
 by at least  edges.  has then no neighbor in  and
 is partitioned into a good star centered on  and an even
cycle as soon as ,  and  have two neighbors at least
in . In that case, the result follows from Theorem
\ref{Theorem:GoodRingGoodStars}.

Assume thus that  has only one neighbor in  (and then 
neighbors in ). Assume that  has more neighbors in 
than . Hence  has at least  neighbors in . Let us
colour each vertex  of  with  or  when  is
adjacent to  or . With that colouring  is a good
even cycle. We can find  disjoint balanced pairs intersecting the
colour  exactly once each. Let  and  the two
pairs of edges of  so determined. We can complete these two pairs
with a third edge  ( respectively) connecting  to 
or  to ,following the cases, in such a way that
 and  are two disjoint balanced
matchings. The result follows from Lemma
\ref{Lemma:FondamentalDisjointsBalanced}



\end{PrfClaim}

\begin{Clm} \label{Claim:Claim5MinimumCounterExample32}
The oddness of  is at most .
\end{Clm}
\begin{PrfClaim} In view of the previous claims, it remains to consider the case were  is reduced to a
set of four odd cycles . Once again, Theorem
\ref{Theorem:Oddness4Distance1}, says that, up to the name of
cycles,  and  are joined to the last cycle  and
have no other neighboring cycle. That means that , , 
have length  and  has length . These  cycles are
chordless and the result comes from Theorem
\ref{Theorem:4OddChordlessCycles}.
\end{PrfClaim}




Hence, we can assume that  contains only two odd cycles
 and . Since we consider graphs with at most  vertices
and since the even cycles of  have length at least ,
 contains only one even cycle  or two even cycles
 and  or three even cycles  and . From
Theorem \ref{Theorem:Distance3Oddness2},  and  are at
distance at least . That means that the only neighbors of these
two cycles are vertices of the remaining even cycles.

It will be convenient, in the sequel, to consider  that the vertices
of the even cycles are coloured alternately in red and blue.

\begin{Clm} \label{Claim:Claim6MinimumCounterExample32}
If  and  are  joined to an even cycle in ,
then their neighbors in that even cycle have the same colour
\end{Clm}
\begin{PrfClaim}
 Assume
that  is joined to a blue vertex of an even cycle of  by the edge  and  is joined to a red vertex of this same
cycle  by the edge .  is then a balanced
matching. Let  be the perfect matching of  such that   and let  be a perfect matching avoiding  and
 (Lemma \ref{Lemma:TwoEdgesAvoiding}). then ,  and 
are two non intersecting perfect matchings, a contradiction.
\end{PrfClaim}

Hence, for any even cycle of   joined to the two cycles
 and , we can consider that, after a possible permutation
of colours for some even cycle,  the vertices adjacent to  or
 have the same colour (say red).

\begin{Clm} \label{Claim:Claim7MinimumCounterExample32}
 contains  even cycles
\end{Clm}
\begin{PrfClaim}
Assume that  contains  even cycles  and
. We certainly have, up to isomorphism,  and  with
length  and  of length  or  while the lengths of 
and  are bounded above by . In view of Claim
\ref{Claim:Claim6MinimumCounterExample32}  has at most
 neighbors in  and in  and at most  neighbors in
. Since  and  have at least  neighbors, that
means that all the red vertices of  are
adjacent to some vertex in  or . It is then easy to see
that two even cycles are joined  by two distinct edges ( and )
whose ends are blue and each of them is connected to both  and
 (say  and  connecting   and  and  connecting
). Then  and  are two disjoint
balanced matchings and the result follows.

 Assume now that
. Since  and  have at least
 neighbors each in  ,  must have  red vertices.
Hence  and  have length  and  has length . The
 blue vertices of  are matched by  edges of . For any
chord of , we can find a red vertex in each path determined by
this chord on , one being adjacent to  and the other to
. Let  be the three edges so determined.  is a balanced
matching. By systematic inspection we can check that it is
always possible to find two disjoint balanced matchings so
constructed. The result follows from Lemma
\ref{Lemma:FondamentalDisjointsBalanced}
\end{PrfClaim}

We shall say that  is a graph of {\em type 3} when
\begin{itemize}
  \item [Type 3] contains two even cycles  and , the
neighborhood of  is contained in , the neighborhood of
 is contained in , and  and  are joined by 
or  edges.
\end{itemize}


\begin{Clm} \label{Claim:Claim8MinimumCounterExample32}
 and  have length  or  or one of them has length
. In the latter case  is a graph of type 
\end{Clm}
\begin{PrfClaim}
 being connected and  bridgeless,  and  are joined to
the remaining cycles of  by an odd number of edges (at
least ).

Assume that  has length at least , then there at least 
vertices involved in . Hence,  contains
exactly one even cycle. From Claim
\ref{Claim:Claim7MinimumCounterExample32} this is  impossible.

Assume that  has length  , then if  is connected to the
remaining part of  with  edges, that means that  has 
chords. Since  has girth at least ,  induces a subgraph
isomorphic to the Petersen graph where a vertex is deleted. This is
impossible in view of Lemma \ref{Lemma:P-vExcluded}.

Hence  is connected to the even cycles of  with 
edges. If  has only one cycle , then, in view of
claim this cycle must has length at least , impossible. We can
thus assume that  contains two cycles  and .
Since  contains at least  vertices,  and
 have length . If  and  have both some neighbors
in , there are at most  such vertices in view of Claim
\ref{Claim:Claim6MinimumCounterExample32}. In that case, the
remaining (at least )neighbors are in , impossible since
this forces  to have length at least .

Hence  has all its neighbors in  and  all its
neighbors in . The perfect matching  forces  and 
to be connected with an odd number ( or ) of edges and  is
a graph of type 3, as claimed.
\end{PrfClaim}





From now on, we have  

\begin{Clm} \label{Claim:Claim9MinimumCounterExample32}
 is not a graph of type 
\end{Clm}
\begin{PrfClaim}

Let  and  two edges joining  to  with  and
 in . Whatever is the colour of  and  we can choose
two distinct vertices  and  in the neighboring vertices of
 on  such that  and  have distinct colours as well
as  and . Let  be the edge joining  to  and  the
edge joining  to . In the same way, we can find two distinct
vertices  and  in the neighboring vertices of  on 
with the same property relatively to  and  leading to the
edges  and .

We can check that  and  are two
disjoint balanced matchings. The result follows from Lemma
\ref{Lemma:FondamentalDisjointsBalanced}

\end{PrfClaim}

\begin{Clm} \label{Claim:Claim10MinimumCounterExample32}
One of  or  has its neighborhood included in  or

\end{Clm}
\begin{PrfClaim}
If   and  have neighbors in  and  each, then,
from Claim \ref{Claim:Claim6MinimumCounterExample32}, there are at
least  vertices involved in . Hence  and
 have length  and  contains exactly 
vertices. The  red vertices of  are adjacent to
 or  and the blue vertices are connected together.

Let  be a chord for  and  be a chord for  (whenever
these two chord exist). We can find two red vertices in 
separated by , one being adjacent to  by an edge  while
the other is adjacent to  by an edge . Let  be
the  set of three edges so constructed.In the same way we get
 when considering .  and  are two
disjoint balanced matchings. The result follows from Lemma
\ref{Lemma:FondamentalDisjointsBalanced}

Assume thus that  has no chord. That means that we can find two
distinct edges  and  connecting  to . Let  be an
edge connecting  to ,  be an edge connecting  to
,  an edge connecting  to  and  an edge
connecting  to . Then  and 
are two disjoint balanced matchings. The result follows from
Lemma \ref{Lemma:FondamentalDisjointsBalanced}
\end{PrfClaim}

We can assume now that  has its neighbors contained in .
Since  is not of type  by Claim
\ref{Claim:Claim9MinimumCounterExample32},  has some neighbor
in .  must have length  at least from Claim
\ref{Claim:Claim6MinimumCounterExample32}. This forces  to have
length , 
 length  and  and  lengths . Moreover, there is one
 edge exactly between  and  and  or  edges between
  and . It is then an easy task to find  and
  with  and  connecting  and ,  and  connecting
  and ,  and  connecting  and  such that
  and  are two disjoint balanced matchings. The result follows from
Lemma \ref{Lemma:FondamentalDisjointsBalanced}
\end{Prf}

\section{Conclusion}

A Fano colouring of  is any assignment of points of the Fano
plane  (see, e.g., \cite{MacSko}) to edges of  such
that the three edges incident with each vertex of  are mapped to
three distinct collinear points of  . The following
conjecture appears in \cite{MacSko}

\begin{Conj}\cite{MacSko} \label{Conjecture:MacajovaSkoviera} Every
bridgeless cubic graph admits a  Fano colouring which uses at most
four lines.
\end{Conj}

In fact, M\`{a}\v{c}ajov\'{a} and  \v{S}koviera  proved in
\cite{MacSko} that conjecture \ref{Conjecture:FanRaspaud} and
Conjecture \ref{Conjecture:MacajovaSkoviera} are equivalent. Hence,
our results can be immediately translated in terms of the
M\`{a}\v{c}ajov\'{a} and \v{S}koviera conjecture.





\bibliographystyle{plain}


\bibliography{Bibliographie}


\end{document}
