\newcommand{\reading}{\mathit{reading}}
\newcommand{\latest}{\mathit{latest}}
\renewcommand{\index}{\mathit{index}}
\newcommand{\slot}{\var{slot}}
\newcommand{\profiveone}{
\assh{Inv1\defeq \lambda h.~}\\
\assh{\inpar{
\Let ph_1,ph_2,P_1,Q_1,R_1,S_1,T_1 \In~\\
P_1 = \lambda h.~ \r{reading}{ph_1}(h) \In\\
Q_1 = \lambda h.~ \r{index[\neg ph_1]}{ph_2}(h) \In\\
R_1 = \lambda h.~ \w{slot[\neg ph_1][\neg ph_2]}{\_}(h) \In\\
S_1 = \lambda h.~ \w{index[\neg ph_1]}{\neg ph_2}(h) \In\\
T_1 = \lambda h.~ \w{latest}{\neg ph_1}(h) \In\\
~~~~[P_1;Q_1;R_1;S_1;T_1]^*(h) \land \None!\reading(h)
}}
}

\newcommand{\profivetwo}{
\assh{Inv2\defeq \lambda h.~}\\
\assh{\inpar{
\Let ph_1',ph_2',P_2,Q_2,R_2,S_2. \In~\\
P_2 = \lambda h.~ (?latest,ph_1')(h) \In\\
Q_2 = \lambda h.~ (!reading,ph_1')(h) \In\\
R_2 = \lambda h.~ (?index[ph_1'],ph_2')(h) \In\\
S_2 = \lambda h.~ (?slot[ph_1'][ph_2'],\_)(h) \In\\
~~~~[P_2;Q_2;R_2;S_2]^*(h) \land \None!\index(h)\\	
~~~~~~~\land \None!\latest(h) \land \None!\slot(h)
}}
}

\newcommand{\MergedHist}{\mathrm{MergedHist}}
\newcommand{\Reader}{\mathrm{Reader}}
\newcommand{\Writer}{\mathrm{Writer}}


\newcommand{\n}[1]{\overline{#1}}
\newcommand{\call}[3]{#3_{#1^{#2}}}

\newcommand{\FigureExV}{
\begin{figure}[h]
\centering
$\begin{array}{@{~}c@{~}}
  \Assign{reading}{\false}, \Assign{latest}{\false}, \Assign{slot[2][2]}{\set{0}}, \Assign{index[2]}{\set{\false}}
\\
\begin{array}{@{}l@{~}||@{~}l@{}}
\inarr{
\profiveone\\ \\
\qquad \qquad \qquad \qquad \Writer\\
\CWhile{\true}\\
\quad \assh{Inv1(h_1)}\\
\quad \Assign{fwp}{\neg reading}\\
\quad \Assign{fwindex}{\neg index[fwp]}\\ 
\quad \assh{Inv1'\defeq[[P_1;Q_1;R_1;S_1;T_1]^*;}\\
\quad \assh{~~~(?reading,ph_1);(?index[\neg ph_1],ph_2)](h_1)}\\
\quad \assh{\land fwp=\neg ph_1 \land fwindex=\neg ph_2}\\
\quad \assh{\land \None!reading(h_1)}\\ \\
\CS \\ \\
\quad Write\ to\ slot[fwp][fwindex]\\
\quad \Assign{index[fwp]}{fwindex}\\
\quad \Assign{latest}{fwp}\\
\quad \assh{[P_1;Q_1;R_1;S_1;T_1]^+(h_1)}\\
\COd
}
& 
\inarr{
\profivetwo\\ \\
\qquad \qquad \qquad \qquad \Reader\\
\CWhile{\true}\\
\quad \assh{Inv2(h_2)}\\
\quad \Assign{srp}{latest}\\
\quad \Assign{reading}{srp}\\
\quad \Assign{srindex}{index[srp]}\\
\quad \assh{Inv2'\defeq[[P_2;Q_2;R_2;S_2]^*;}\\
\quad \assh{~~(?latest,ph_1');(!reading,ph_1');(?index[ph_1'],ph_2')](h_2)}\\
\quad \assh{\land srp=ph_1' \land srindex=ph_2'}\\
\quad \assh{\land \None!index(h_2) \land \None!latest(h_2)}\\
\quad \assh{\land \None!slot(h_2)}\\ \\
\CS\\ \\
\quad Read\ from\ slot[srp][srindex]\\
\quad \assh{[P_2;Q_2;R_2;S_2]^+(h_2)}\\
\COd
}
\end{array}
\end{array}$
 \caption{Simpson's 4 slot algorithm }
 \label{fig:ex5}
\end{figure}
}

\subsection{Example: Simpson's 4 slot algorithm \cite{Simpson4slot}}
This is a wait-free algorithm for concurrent access of a location from a single reader and a single writer processes. This location is 
simulated using a 2$\times$2 array variable $\slot$. If the
reader is reading the data and the writer wants to write new data at the same time then 
instead of waiting for the reader to complete, the writer writes to a different index of $\slot$ and 
indicates the reader to read from this location in the subsequent read. 
Boolean variables $reading$ and $latest$ represent two indices ($\false$ as 0 and $\true$ as 1) to denote the row and column 
indices of $\slot$ and $\index$ variable. Variable $\index$ is a two element boolean array such that for any $s\in \set{\true,\false}$, 
$\slot[s][index[s]]$ has the latest data written by the writer in the row $\slot[s]$. Variables $fwp$ and $fwindex$ are local to the writer process. 
Variables $srp$ and $srindex$ are local to the reader process. The algorithm with inline assertions 
on history and local variables are shown in Figure \ref{fig:ex5}. In order to simulate different invocations of the writer and the reader 
the respective program is enclosed within the loop. We are interested in proving following two properties of this algorithm.

\FigureExV	


\begin{description}
 \item [Interference Freedom]
We want to show that at the entry point of critical sections (for the writer and the reader) $fwp\neq srp$, 
i.e. the reader and the writer use different rows of the $\slot$ variable, and 
$fwp=srp \limp fwindex\neq srindex$, i.e. if both use the same row of the $\slot$ variable then 
they read from and write to different column of that row. 
Therefore, we want to prove 
\[
\begin{array}{l}
 Inv1'(h_1) \land Inv2'(h_2) \land \compat(\false,\false,\set{0},\set{\false},h_1,h_2) \implies \\
~~~fwp\neq srp \lor fwp=srp \limp fwindex\neq srindex
\end{array}
\]
Where $Inv1'$ and $Inv2'$ are the assertions just before the program point where writer is going to write the data and reader is going to read the data.
\begin{proof}
In any compatible merged history $h$ of $h_1$ and $h_2$ the placement of the last write $\w{reading}{ph_1'}$ of $h_2$ has 
two choices with respect to the last read $\r{reading}{ph_1}$ of $h_1$.
\begin{itemize}
 \item \textit{Last write $\w{reading}{ph_1'}$ of $h_2$ is placed \ul{before} the last read $\r{reading}{ph_1}$ of $h_1$:} As writer does not 
write to $reading$ hence $ph_1=ph_1'$, or $\neg fwp=srp$ (from assertions $fwp=\neg ph_1$ and $srp=ph_1'$) and therefore $fwp\neq srp$. Hence proved.

 \item \textit{Last write $\w{reading}{ph_1'}$ of $h_2$ is placed \ul{after} the last read $\r{reading}{ph_1}$ of $h_1$:} This, along with the assumption $fwp=srp$ implies
$\neg ph_1 = ph_1'$. Therefore in $Inv2'$, $\r{index[ph_1']}{ph_2'}$ is same as $\r{index[\neg ph_1]}{ph_2'}$. According to $Inv1'$ 
$\r{index[\neg ph_1]}{ph_2}$ is in $h_1$. We want to establish the relation between $ph_2$ and $ph_2'$.

$Inv1'$ implies that \ul{$S_1\before \r{reading}{ph_1}$} hence there is no write to variable $\index$ beyond $\r{reading}{ph_1}$ in the merged history. 
Hence the value at $index[\neg ph_1]$ is same for both processes, i.e. $ph_2=ph_2'$. From $Inv1'$ and $Inv2'$ we get $fwindex=\neg ph_2$ and $srindex=ph_2$
which implies $fwindex\neq srindex$. Hence proved. 
\end{itemize}
\end{proof}
From the above proof, we can see that the only ordering important in proving this property is $S_1\before P_1$ where $P_1$ is from later iteration than that of $S_1$. Now
we prove another property of interest and find out the program orderings used in that proof.

\item [Consistent Reads]
Second property of interest is related to the order of reads. It specifies that the values read by the reader form a stuttering sequence of values written by the writer, i.e. if the writer writes the sequence 1,2,3,4,5 in subsequent invocations of write then the reader cannot observe 1,4,3,5 as a sequence read. It must observe the sequence which preserve the order of writes and possibly interspersed with the repetition of the same data. First, we define some notations used in this section.
Let $\MergedHist(\Reader^R,\Writer^W)$ be the set of all compatible merged histories consisting of $R$ invocations of the reader process and $W$ invocations of the writer process. 
Let $\Reader(n)$ and $\Writer(n)$ be the $n^{th}$ invocations of the reader and the writer respectively. Let $\Data{k}$ be the data written by the writer in the $k^{th}$ invocation. 
Let $D(w)$ be a sequence $\Data{1}.\Data{2}.\cdots.\Data{w}$ of values written by $w$ consecutive iterations of the writer. For $s\in \set{\true,\false}$, $\n{s}$ denote the negation of $s$.
Let $\call{Reader}{r}{elem}$ be the element $elem\in \set{\r{\any}{\any},\w{\any}{\any}}$ in the merged history from $r^{th}$ invocation of the $\Reader$. Similarly $\call{Writer}{w}{elem}$ denote the same for for the $w^{th}$ iteration of the $\Writer$.

\paragraph{\textbf{Stuttering sequence}}
 Let $\Seq{r,w}$ be a sequence of length $r$, constructed from the elements of $D(w)$. $\Seq{r,w}$ is a stuttering sequence of $D(w)$ if for any index $i$ of the sequence $\Seq{r,w}$ such the $\Seq{r,w}[i-1]=\Data{k_1}$ and $\Seq{r,w}[i]=\Data{k_2}$ then $k_1 \le k_2$. 
 



\paragraph{\textbf{Some interesting properties of the Reader and the Writer processes}}
Only the writer writes to $latest$ and reads from $reading$ variable. Further, only the reader writes to $reading$ and reads from $latest$. Also, the value written to $reading$ by the reader in any invocation is same as the value read from $latest$. The value written to $latest$ by the writer in any invocation is negation of the value that it reads from $reading$. 


Following lemma characterizes the sequence of values written to $\reading$ in a segment of the merged history. This characterization is then used in the proof of Lemma \ref{finalaim}. 
\begin{lemma}\label{one}
For all $R$, $W$, $h\in \MergedHist(\Reader^R,\Writer^W)$, $r\le R, w\le W$, $s\in \set{\true,\false}$, if $\Reader(r)$ reads the value of $\latest$ written by $\Writer(w)$ as $s$ then in the sequence of values written to variable $\reading$ between $P_1$ of $\Writer(w)$ and $P_2$ of $\Reader(r)$, $s$ is never followed by $\n{s}$.
\end{lemma}
 \begin{figure}[h]
\begin{tabular}{@{}c@{~}}
$\xymatrix @R=1pc @C=0.5pc {
\ar@{--}[d]\\
A'':\call{Writer}{n'}{\w{latest}{\n{s}}} \ar[dr]_{}             &  \\
 & B'':\call{Reader}{r'}{\r{latest}{\n{s}}} \ar[d]_{} \\
 & B': \call{Reader}{r'}{\w{reading}{\n{s}}} \ar[dl]_{}\\
A':\call{Writer}{n+1}{\r{reading}{\n{s}}} \ar[d]_{}\\
A:\call{Writer}{n+1}{\w{latest}{s}} \ar[dr]_{}\\
 &B:\call{Reader}{r}{\r{latest}{s}}\\
}$
\end{tabular}
\caption{Merged history for Lemma \ref{one} }
\label{lem:fig1}
\end{figure}

\begin{proof}
We prove it using induction on the iteration number of the writer process. \\
\textbf{Base case, $w=0$:} If $\Reader(r)$ reads the initial value of $\latest$ (in 0$^{th}$ iteration of the writer process), say $s$, then all the invocations of the reader before $\Reader(r)$ also see this initial value of $\latest$ as $s$ and therefore the sequence is made of only this value. Hence the base case satisfies this property.\\
\textbf{Induction Hypothesis, $w\le n$:} For all $w\le n$, if $\Reader(r)$ reads the value of $\latest$ as $s$, written by $\Writer(w)$ then the sequence of values written to $\reading$ between 
$P_1$ of $\Writer(w)$ and $P_2$ of $\Reader(r)$ satisfies this property.\\
\textbf{Induction Step, $w=n+1$:} Consider the merged history of Figure \ref{lem:fig1} where $\Reader(r)$ reads the value of $\latest$(denoted by $B$) from $\Writer(n+1)$ (denoted by $A$). We want to prove this property for the sequence of values written to $\reading$ between $A'$ and $B$ where $A'$ denotes $P_1$ of $\Writer(n+1)$ in Figure \ref{lem:fig1}. \ul{$P_1\before T_1$} implies that $A'$ appears before $A$ in the merged history. Let $\Writer(n+1)$ read the value of $\reading$ from $\Reader(r')$ (denoted by $B'$). \ul{$P_2\before Q_2$} implies that $P_2$ of $\Reader(r')$ appears before $B'$ in the merged history. Let $\Reader(r')$ at $P_2$ read the value of $\latest$ from $\Writer(n')$($A''$). It is clear that $n'<n+1$ hence from the induction hypothesis we know that in the sequence of values written to $\reading$ between $A''$ and $B''$, $\n{s}$ is never followed by $s$. We use this knowledge to characterize the values written to $\latest$ between $B''$ and $A'$. From the writer process we know that the value written to $\latest$ is negation of the value read from $\reading$ variable. Therefore in the sequence of values written to $\latest$ between $B''$ and $A'$, $s$ is never followed by $\n{s}$. Further, we also know that the reader process writes the same value to $\reading$ as is read from the  variable $\latest$. Therefore in the sequence of values written to $\reading$ between $A'$ and $B$, $s$ is never followed by $\n{s}$. Hence proved.
\end{proof}
 For any reader we establish the relation between its read of variable $\latest$ and the read of data. More formally we want to say that,
\begin{lemma}\label{finalaim}
 For all $R$, $W$, $h\in \MergedHist(\Reader^R,\Writer^W)$, $r\le R, w\le W$, $s\in \set{\true,\false}$, if $\Reader(r)$ reads the value of $latest$ as $s$ written by $\Writer(w)$ 
 and reads the value of $\index[s]$ written by $\Writer(w')$ then $w'\ge w$ and $\Reader(r)$ reads the data $\Data{w'}$ and 
 all the invocations of the writer from $w$ to $w'$ write the data in the row $s$ of the $\slot$ variable.
\end{lemma}
\begin{proof}
Let us assume that $\Reader(r)$ reads the value of $\latest$ as $s$ written by $\Writer(w)$ which means that $P_2$ of $\Reader(r)$ appears after $T_1$ of $\Writer(w)$ and no write to $\latest$ appears in between these two points in the merged history. As $\Writer(w)$ also writes to $\index[s]$ and \ul{$S_1\before T_1$} and \ul{$P_2\before R_2$}, hence $\Reader(r)$ reads the value of $\index[s]$ either from $\Writer(w)$ or from $\Writer(w')$ such that $w'\ge w$. We prove the following two properties,
\begin{itemize}
 \item \textbf{$\mathbf{\Reader(r)}$ reads the data written by $\Writer(w')$ in $\slot[s][k]$ where $k$ is the value written to $\index[s]$ by $\Writer(w')$} \\
\emph{Proof:} If $\Reader(r)$ reads $\index[s]$ as $k\in \set{\true,\false}$ from $\Writer(w')$ then \ul{$R_1\before S_1$} implies that $\Writer(w')$ has also written the data in $\slot[s][k]$. We want to show that 
no subsequent invocation of the writer writes in $\slot[s][k]$ before $S_2$ of $\Reader(r)$. From the assumption that $R_2$ of $\Reader(r)$ reads the value of $\index[s]$ from $\Writer(w')$ implies that there is no write to $\index[s]$ between $S_1$ of $\Writer(w')$ and $R_2$ of $\Reader(r)$. \ul{$S_1 \before P_1$}, where $P_1$ is from later iterations than 
that of $S_1$, in $Inv1'$ implies that the writer can be invoked at most once after the iteration $w'$ and before $R_2$ of $\Reader(r)$. If invoked more than once it results in the write of $\index[s]$ to appear after $S_1$ of $\Writer(w')$ and before $R_2$ of $\Reader(r)$, contradicting our assumption. Further, if the next invocation after $w'$ happens then it writes the data to column $\n{k}$ of row $s$ in $\slot$ variable still satisfying the property. If any further invocation of the writer happens after $R_2$ and before $S_2$ of $\Reader(r)$ then because of \ul{$Q_2\before R_2$} it observes the 
value of $\reading$ as $s$ and therefore writes the data to $\n{s}$ row of the $\slot$ variable. $Inv2'$ implies that \ul{$S_2$ of $\Reader(r) \before Q_2$} of subsequent invocations of the reader and therefore no write to $\reading$ exists between $R_2$ and $S_2$ of $\Reader(r)$. This results in observing the same value of $\reading$, $s$, and subsequently writing the data to row $\n{s}$ by all invocations of the writer between $R_2$ and $S_2$ of $\Reader(r)$.

\item \textbf{All the invocations of the writer from $w$ to $w'$ write the data in the row $s$ of the $\slot$ variable}\\
\emph{Proof:} We prove an alternate equivalent property; All invocations of the writer from $w$ to $w'$ read the value of $\reading$ as $\n{s}$. This follows from the property that the 
writer writes in that row of the $\slot$ variable that is obtained by negating the value of the variable $\reading$. From the assumption, $\Writer(w')$ writes to $\index[s]$ and therefore it also reads the value of $\reading$ as $\n{s}$ and the same holds for $\Writer(w)$ as well. We need to show that this property holds for the rest of the invocations in between $w$ and $w'$. We prove it by contradiction by assuming that there exists a $w''$ such that $w<w''<w'$ and $P_1$ of $\Writer(w'')$ reads the value of $\reading$ as $s$. It is clear that $P_1$ of $\Writer(w'')$ appears after $P_1$ of $\Writer(w)$ and before $P_1$ of $\Writer(w')$. From the earlier argument we know that the value of $\reading$ visible at $P_1$ of $\Writer(w)$ is $\n{s}$. Hence, $P_1$ of $\Writer(w'')$ cannot read from the value of $\reading$ visible at $P_1$ of $\Writer(w)$. Therefore, it must read the value of $\reading$ as $s$ from the sequence of values written to $\reading$ between $P_1$ of $\Writer(w)$ and $P_2$ of $\Reader(r)$. 
Following Lemma \ref{one}, $s$ can never be followed by $\n{s}$ in the sequence of values written to $\reading$ in between $P_1$ of $\Writer(w)$ and $P_2$ of $\Reader(r)$. 
It further implies that $P_1$ of $\Writer(w')$ cannot read the value of $\reading$ as $\n{s}$, a contradiction to our assumption. Therefore all invocations of the writer from $w$ to $w'$ read the value of $\reading$ as $\n{s}$. Hence proved.
\end{itemize}

\end{proof}
   
Now we use Lemma \ref{finalaim} to prove the following theorem.
\begin{theorem}[Stuttering Sequence]
Stutter(n)$\defeq$ For all $n,w \in \mathbb{N},h\in \MergedHist(\Reader^n,\Writer^w)$ the sequence
$\Seq{n,w}$, constructed from the elements of $D(w)$, is a stuttering sequence of $D(w)$.
\end{theorem}
\begin{proof}
 \textbf{Base case, $n=0$}: With no history corresponding to the Reader in the merged history, the empty sequence trivially forms a stuttering sequence.\\
 \textbf{Induction Hypothesis}: For all $r \le n$ Stutter(r) holds true.\\
 \textbf{Induction Step, $r=n+1$}: 


Let $\Reader(n)$ read the value of $\latest$ from $\Writer(w_a)$.
From Lemma \ref{finalaim}, we know that $Reader(n)$
reads the value $\Data{w'}$ such that $w_a \le w' \le w$ and all the invocations of the writer from $w_a$ to $w'$ write the data in the same row $s$ of the $\slot$ variable.
Further, the value of $index[s]$ visible at $R_2$ of $\Reader(n)$ is from $S_1$ of $\Writer(w')$. We know that the write to $latest$ from consecutive iterations are totally ordered and 
the same holds for consecutive reads from $latest$ as well.  
Therefore, $\Reader(n+1)$ reads the value of $latest$ from some $\Writer(w_b)$ such that $w_b \ge w_a$. There are two possibilities based on whether this value is $s$ or $\n{s}$.
\begin{itemize}
\item $\Reader(n+1)$ reads the value of $latest$ as $s$ from the $\Writer(w_b)$ such that $w_b \ge w_a$: From our assumption we know that $\Reader(n)$ sees the value of $index[s]$ from 
$\Writer(w')$ therefore the value of $index[s]$ visible at $R_2$ of $\Reader(n+1)$ will be from some $w''$ such that $w\ge w''\ge w'$. 
Following Lemma \ref{finalaim},  the data read by 
$\Reader(n+1)$ from $slot[s][index[s]]$ will be from $\Writer(w'')$ and $w''\ge w'$ implies that the resulting sequence $\Seq{n+1,w}$ is still a stuttering sequence.

\item  $\Reader(n+1)$ reads the value of $latest$ as $\n{s}$ from $\Writer(w_b)$ such that $w_b \ge w_a$: From Lemma \ref{finalaim}, we know that all the invocations of the writer from 
$w_a$ to $w'$ write the data in the same row $s$ of $\slot$ and because $\Writer(w_b)$ writes the data in row $\n{s}$ of $\slot$ therefore $w_b > w'$. Following Lemma \ref{finalaim}, $Reader(n+1)$ reads the data from some $\Writer(w'')$ such that $w''\ge w_b$. Combining these two we get $w''>w'$ such that $\Reader(n)$ reads $\Data{w'}$ and $\Reader(n+1)$ reads $\Data{w''}$. Therefore the 
resulting sequence $\Seq{n+1,s}$ is still a stuttering sequence.
\end{itemize}



\end{proof}



\end{description}
\paragraph{\textbf{Simpson's 4 slot algorithm under PSO memory model}}
Following per-thread instruction orderings are used in the proofs of the interference freedom and the consistent reads properties.
\begin{itemize}
 \item $P_1\before T_1$(from the proof of Lemma \ref{one}), $S_1 \before T_1$, $R_1\before S_1$
 \item $P_2\before Q_2$(from the proof of Lemma \ref{one}), $P_2\before R_2$, $Q_2\before R_2$
 \item $S_1\before P_1$, where $P_1$ is from iteration later than that of $S_1$
 \item $S_2 \before Q_2$, where $Q_2$ is from iteration later than that of $S_2$.
\end{itemize}
Out of all these orderings, $P_1\before T_1$, $P_2\before Q_2$ and $P_2\before R_2$ are data dependent orders which are respected by the PSO memory model. 
$S_2\before Q_2$ where $Q_2$ is from iteration later than that of $S_2$ is also respected by PSO memory model because $S_2$ corresponds to read and $Q_2$ 
corresponds to write instruction. Therefore, we have to enforce only $R_1\before S_1$, $S_1\before T_1$, $Q_2\before R_2$ and $S_1\before P_1$ where $P_1$ is 
from iteration later than that of $S_1$. Following the semantics of $\fence$ it is sufficient to put two fence instructions in the writer; one between $R_1$ and $S_1$ 
and another between $S_1$ and $T_1$. Further, we need one fence instruction between $Q_2$ and $R_2$ in the reader as well. 
