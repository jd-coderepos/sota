\newcommand{\reading}{\mathit{reading}}
\newcommand{\latest}{\mathit{latest}}
\renewcommand{\index}{\mathit{index}}
\newcommand{\slot}{\var{slot}}
\newcommand{\profiveone}{
\assh{Inv1\defeq \lambda h.~}\\
\assh{\inpar{
\Let ph_1,ph_2,P_1,Q_1,R_1,S_1,T_1 \In~\\
P_1 = \lambda h.~ \r{reading}{ph_1}(h) \In\\
Q_1 = \lambda h.~ \r{index[\neg ph_1]}{ph_2}(h) \In\\
R_1 = \lambda h.~ \w{slot[\neg ph_1][\neg ph_2]}{\_}(h) \In\\
S_1 = \lambda h.~ \w{index[\neg ph_1]}{\neg ph_2}(h) \In\\
T_1 = \lambda h.~ \w{latest}{\neg ph_1}(h) \In\\
~~~~[P_1;Q_1;R_1;S_1;T_1]^*(h) \land \None!\reading(h)
}}
}

\newcommand{\profivetwo}{
\assh{Inv2\defeq \lambda h.~}\\
\assh{\inpar{
\Let ph_1',ph_2',P_2,Q_2,R_2,S_2. \In~\\
P_2 = \lambda h.~ (?latest,ph_1')(h) \In\\
Q_2 = \lambda h.~ (!reading,ph_1')(h) \In\\
R_2 = \lambda h.~ (?index[ph_1'],ph_2')(h) \In\\
S_2 = \lambda h.~ (?slot[ph_1'][ph_2'],\_)(h) \In\\
~~~~[P_2;Q_2;R_2;S_2]^*(h) \land \None!\index(h)\\	
~~~~~~~\land \None!\latest(h) \land \None!\slot(h)
}}
}

\newcommand{\MergedHist}{\mathrm{MergedHist}}
\newcommand{\Reader}{\mathrm{Reader}}
\newcommand{\Writer}{\mathrm{Writer}}


\newcommand{\n}[1]{\overline{#1}}
\newcommand{\call}[3]{#3_{#1^{#2}}}

\newcommand{\FigureExV}{
\begin{figure}[h]
\centering

 \caption{Simpson's 4 slot algorithm }
 \label{fig:ex5}
\end{figure}
}

\subsection{Example: Simpson's 4 slot algorithm \cite{Simpson4slot}}
This is a wait-free algorithm for concurrent access of a location from a single reader and a single writer processes. This location is 
simulated using a 22 array variable . If the
reader is reading the data and the writer wants to write new data at the same time then 
instead of waiting for the reader to complete, the writer writes to a different index of  and 
indicates the reader to read from this location in the subsequent read. 
Boolean variables  and  represent two indices ( as 0 and  as 1) to denote the row and column 
indices of  and  variable. Variable  is a two element boolean array such that for any , 
 has the latest data written by the writer in the row . Variables  and  are local to the writer process. 
Variables  and  are local to the reader process. The algorithm with inline assertions 
on history and local variables are shown in Figure \ref{fig:ex5}. In order to simulate different invocations of the writer and the reader 
the respective program is enclosed within the loop. We are interested in proving following two properties of this algorithm.

\FigureExV	


\begin{description}
 \item [Interference Freedom]
We want to show that at the entry point of critical sections (for the writer and the reader) , 
i.e. the reader and the writer use different rows of the  variable, and 
, i.e. if both use the same row of the  variable then 
they read from and write to different column of that row. 
Therefore, we want to prove 

Where  and  are the assertions just before the program point where writer is going to write the data and reader is going to read the data.
\begin{proof}
In any compatible merged history  of  and  the placement of the last write  of  has 
two choices with respect to the last read  of .
\begin{itemize}
 \item \textit{Last write  of  is placed \ul{before} the last read  of :} As writer does not 
write to  hence , or  (from assertions  and ) and therefore . Hence proved.

 \item \textit{Last write  of  is placed \ul{after} the last read  of :} This, along with the assumption  implies
. Therefore in ,  is same as . According to  
 is in . We want to establish the relation between  and .

 implies that \ul{} hence there is no write to variable  beyond  in the merged history. 
Hence the value at  is same for both processes, i.e. . From  and  we get  and 
which implies . Hence proved. 
\end{itemize}
\end{proof}
From the above proof, we can see that the only ordering important in proving this property is  where  is from later iteration than that of . Now
we prove another property of interest and find out the program orderings used in that proof.

\item [Consistent Reads]
Second property of interest is related to the order of reads. It specifies that the values read by the reader form a stuttering sequence of values written by the writer, i.e. if the writer writes the sequence 1,2,3,4,5 in subsequent invocations of write then the reader cannot observe 1,4,3,5 as a sequence read. It must observe the sequence which preserve the order of writes and possibly interspersed with the repetition of the same data. First, we define some notations used in this section.
Let  be the set of all compatible merged histories consisting of  invocations of the reader process and  invocations of the writer process. 
Let  and  be the  invocations of the reader and the writer respectively. Let  be the data written by the writer in the  invocation. 
Let  be a sequence  of values written by  consecutive iterations of the writer. For ,  denote the negation of .
Let  be the element  in the merged history from  invocation of the . Similarly  denote the same for for the  iteration of the .

\paragraph{\textbf{Stuttering sequence}}
 Let  be a sequence of length , constructed from the elements of .  is a stuttering sequence of  if for any index  of the sequence  such the  and  then . 
 



\paragraph{\textbf{Some interesting properties of the Reader and the Writer processes}}
Only the writer writes to  and reads from  variable. Further, only the reader writes to  and reads from . Also, the value written to  by the reader in any invocation is same as the value read from . The value written to  by the writer in any invocation is negation of the value that it reads from . 


Following lemma characterizes the sequence of values written to  in a segment of the merged history. This characterization is then used in the proof of Lemma \ref{finalaim}. 
\begin{lemma}\label{one}
For all , , , , , if  reads the value of  written by  as  then in the sequence of values written to variable  between  of  and  of ,  is never followed by .
\end{lemma}
 \begin{figure}[h]
\begin{tabular}{@{}c@{~}}

\end{tabular}
\caption{Merged history for Lemma \ref{one} }
\label{lem:fig1}
\end{figure}

\begin{proof}
We prove it using induction on the iteration number of the writer process. \\
\textbf{Base case, :} If  reads the initial value of  (in 0 iteration of the writer process), say , then all the invocations of the reader before  also see this initial value of  as  and therefore the sequence is made of only this value. Hence the base case satisfies this property.\\
\textbf{Induction Hypothesis, :} For all , if  reads the value of  as , written by  then the sequence of values written to  between 
 of  and  of  satisfies this property.\\
\textbf{Induction Step, :} Consider the merged history of Figure \ref{lem:fig1} where  reads the value of (denoted by ) from  (denoted by ). We want to prove this property for the sequence of values written to  between  and  where  denotes  of  in Figure \ref{lem:fig1}. \ul{} implies that  appears before  in the merged history. Let  read the value of  from  (denoted by ). \ul{} implies that  of  appears before  in the merged history. Let  at  read the value of  from (). It is clear that  hence from the induction hypothesis we know that in the sequence of values written to  between  and ,  is never followed by . We use this knowledge to characterize the values written to  between  and . From the writer process we know that the value written to  is negation of the value read from  variable. Therefore in the sequence of values written to  between  and ,  is never followed by . Further, we also know that the reader process writes the same value to  as is read from the  variable . Therefore in the sequence of values written to  between  and ,  is never followed by . Hence proved.
\end{proof}
 For any reader we establish the relation between its read of variable  and the read of data. More formally we want to say that,
\begin{lemma}\label{finalaim}
 For all , , , , , if  reads the value of  as  written by  
 and reads the value of  written by  then  and  reads the data  and 
 all the invocations of the writer from  to  write the data in the row  of the  variable.
\end{lemma}
\begin{proof}
Let us assume that  reads the value of  as  written by  which means that  of  appears after  of  and no write to  appears in between these two points in the merged history. As  also writes to  and \ul{} and \ul{}, hence  reads the value of  either from  or from  such that . We prove the following two properties,
\begin{itemize}
 \item \textbf{ reads the data written by  in  where  is the value written to  by } \\
\emph{Proof:} If  reads  as  from  then \ul{} implies that  has also written the data in . We want to show that 
no subsequent invocation of the writer writes in  before  of . From the assumption that  of  reads the value of  from  implies that there is no write to  between  of  and  of . \ul{}, where  is from later iterations than 
that of , in  implies that the writer can be invoked at most once after the iteration  and before  of . If invoked more than once it results in the write of  to appear after  of  and before  of , contradicting our assumption. Further, if the next invocation after  happens then it writes the data to column  of row  in  variable still satisfying the property. If any further invocation of the writer happens after  and before  of  then because of \ul{} it observes the 
value of  as  and therefore writes the data to  row of the  variable.  implies that \ul{ of } of subsequent invocations of the reader and therefore no write to  exists between  and  of . This results in observing the same value of , , and subsequently writing the data to row  by all invocations of the writer between  and  of .

\item \textbf{All the invocations of the writer from  to  write the data in the row  of the  variable}\\
\emph{Proof:} We prove an alternate equivalent property; All invocations of the writer from  to  read the value of  as . This follows from the property that the 
writer writes in that row of the  variable that is obtained by negating the value of the variable . From the assumption,  writes to  and therefore it also reads the value of  as  and the same holds for  as well. We need to show that this property holds for the rest of the invocations in between  and . We prove it by contradiction by assuming that there exists a  such that  and  of  reads the value of  as . It is clear that  of  appears after  of  and before  of . From the earlier argument we know that the value of  visible at  of  is . Hence,  of  cannot read from the value of  visible at  of . Therefore, it must read the value of  as  from the sequence of values written to  between  of  and  of . 
Following Lemma \ref{one},  can never be followed by  in the sequence of values written to  in between  of  and  of . 
It further implies that  of  cannot read the value of  as , a contradiction to our assumption. Therefore all invocations of the writer from  to  read the value of  as . Hence proved.
\end{itemize}

\end{proof}
   
Now we use Lemma \ref{finalaim} to prove the following theorem.
\begin{theorem}[Stuttering Sequence]
Stutter(n) For all  the sequence
, constructed from the elements of , is a stuttering sequence of .
\end{theorem}
\begin{proof}
 \textbf{Base case, }: With no history corresponding to the Reader in the merged history, the empty sequence trivially forms a stuttering sequence.\\
 \textbf{Induction Hypothesis}: For all  Stutter(r) holds true.\\
 \textbf{Induction Step, }: 


Let  read the value of  from .
From Lemma \ref{finalaim}, we know that 
reads the value  such that  and all the invocations of the writer from  to  write the data in the same row  of the  variable.
Further, the value of  visible at  of  is from  of . We know that the write to  from consecutive iterations are totally ordered and 
the same holds for consecutive reads from  as well.  
Therefore,  reads the value of  from some  such that . There are two possibilities based on whether this value is  or .
\begin{itemize}
\item  reads the value of  as  from the  such that : From our assumption we know that  sees the value of  from 
 therefore the value of  visible at  of  will be from some  such that . 
Following Lemma \ref{finalaim},  the data read by 
 from  will be from  and  implies that the resulting sequence  is still a stuttering sequence.

\item   reads the value of  as  from  such that : From Lemma \ref{finalaim}, we know that all the invocations of the writer from 
 to  write the data in the same row  of  and because  writes the data in row  of  therefore . Following Lemma \ref{finalaim},  reads the data from some  such that . Combining these two we get  such that  reads  and  reads . Therefore the 
resulting sequence  is still a stuttering sequence.
\end{itemize}



\end{proof}



\end{description}
\paragraph{\textbf{Simpson's 4 slot algorithm under PSO memory model}}
Following per-thread instruction orderings are used in the proofs of the interference freedom and the consistent reads properties.
\begin{itemize}
 \item (from the proof of Lemma \ref{one}), , 
 \item (from the proof of Lemma \ref{one}), , 
 \item , where  is from iteration later than that of 
 \item , where  is from iteration later than that of .
\end{itemize}
Out of all these orderings, ,  and  are data dependent orders which are respected by the PSO memory model. 
 where  is from iteration later than that of  is also respected by PSO memory model because  corresponds to read and  
corresponds to write instruction. Therefore, we have to enforce only , ,  and  where  is 
from iteration later than that of . Following the semantics of  it is sufficient to put two fence instructions in the writer; one between  and  
and another between  and . Further, we need one fence instruction between  and  in the reader as well. 
