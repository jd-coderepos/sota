\date{}

\documentclass[11pt, letterpaper]{article}
\usepackage{amssymb}
\usepackage{fullpage}
\usepackage[noadjust]{cite}
\usepackage[noend]{algpseudocode}
\usepackage{times}

\usepackage{amsfonts,amsmath,color,float,graphicx,verbatim,url, amsthm}

\usepackage{algorithm}

\usepackage{enumerate}

\newtheorem{theorem}{Theorem}
\newtheorem{lemma}[theorem]{Lemma}
\newtheorem{proposition}[theorem]{Proposition}
\newtheorem{corollary}[theorem]{Corollary}
\newtheorem{observation}[theorem]{Observation}
\newtheorem{property}[theorem]{Property}
\newtheorem{example}[theorem]{Example}
\newtheorem{counterexample}[theorem]{Counterexample}
\newtheorem{openproblem}[theorem]{Open Problem}
\newtheorem{assumption}[theorem]{Assumption}
\newtheorem{conjecture}[theorem]{Conjecture}
\newtheorem{definition}[theorem]{Definition}
\newtheorem{fact}[theorem]{Fact}
\newtheorem{remark}[theorem]{Remark}

\usepackage{graphicx}  

 \usepackage{amsfonts,amsmath} 

 \usepackage{hyperref}
 
 \usepackage{varwidth}


 \usepackage{url}
 \urlstyle{tt}   
 
\newcommand{\stream}{{\sigma}}
\newcommand{\length}{m}
\newcommand{\uni}{\mathcal{U}}
\newcommand{\cV}{\mathcal{V}}
\newcommand{\cP}{\mathcal{P}}
\newcommand{\eps}{\epsilon}
\newcommand{\eat}[1]{}

 

\begin{document}

\title{Stream Verification}

\author{Justin Thaler\thanks{Yahoo Labs}}

\maketitle 

\begin{abstract}
We survey models and algorithms for stream verification. 
\end{abstract}

\section{Problem Definition}

Stream verification is concerned with the following setting. 
A computationally limited client wants to compute some property of a massive input, but lacks the resources to store even a small fraction of the input, and hence cannot perform the desired computation locally. The client therefore accesses a powerful but untrusted service provider (e.g., a commercial cloud computing service), who not only performs the requested computation, but also proves that the answer is correct.
An array of closely related models have been introduced to capture this scenario. The following section provides a unified presentation of these models, emphasizing their common features before delineating their differences. 

\medskip
\noindent \textbf{Stream Verification Model.}
Let  be a data stream, where each  comes from a data universe  of size , and let  be a function mapping data streams to a finite range . 
A stream verification protocol for  involves two parties: a \emph{prover} , and a (randomized) \emph{verifier} . The protocol consists of two stages:  a stream observation stage and a proof verification stage. 
 
In the stream observation stage,  processes the stream , subject to the 
standard constraints of the data-stream model, i.e., sequential access to  and limited memory. 
In the proof verification stage,  and  exchange a sequence of one or more messages, and afterward  outputs a value .  is allowed to output a special symbol  indicating a rejection of 's claims.  Formally,  constitutes a stream verification protocol if the following two properties are satisfied.

\begin{itemize}
\item \textbf{Completeness}: There is some prover strategy  such that, for all streams , the probability that  outputs  after interacting with  is at least . 
\item \textbf{Soundness}: For all streams  and all prover strategies , the probability that  outputs a value not in  after interacting with  is at most .
\end{itemize}

Here, the probabilities are taken over 's internal randomness. The constants  and  are not essential and are chosen by convention. The parameter
 is referred to as the \emph{soundness error} of the protocol.

\medskip
\noindent \textbf{Costs.}
There are five primary costs in any stream verification protocol: (1) 's space usage, (2) the total communication cost, (3) 's runtime, (4) 's runtime, and (5) the number of messages exchanged. 

\medskip
\noindent \textbf{Differences Between Models.}
There are three primary differences between the various models of stream verification that have been put forth in the literature. The first is whether the soundness condition is required to hold against all cheating provers (such protocols are called \emph{information-theoretically} or \emph{statistically} sound), or only against cheating provers that run in polynomial time (such protocols are called \emph{computationally} sound). The second is the amount and format of the interaction allowed between  and .  The third is the temporal relationship between the stream observation and proof verification stage --- in particular, several models permit  and  to exchange messages before and during the stream observation stage, and sometimes permit the prover's messages to depend on parts of the data stream that  has not yet seen. 
In general, more permissive models allow a larger class of problems to be solved efficiently, but may yield protocols that are less realistic.

\medskip
\noindent \textbf{Summary of Models.}
The \emph{annotated data streaming} (ADS) model, introduced by Chakrabarti et al. \cite{icalp}, is non-interactive:  is permitted to send a single message to , with no communication allowed in the reverse direction. Technically, this model permits the contents of 's message to be interleaved with the stream, in which case each bit of 's message may be viewed as an ``annotation'' associated with a particular stream update. However, for most ADS protocols that have developed, 's message does not need to be interleaved with the stream.
Chakrabarti et al. \cite{icalp} distinguish between two kinds of ADS protocols: \emph{prescient} protocols, in which the annotation sent at any given time can depend on parts of the data stream that  has not yet seen, and \emph{online} protocols, which disallow this kind of dependence.

\emph{Streaming interactive proofs} (SIPs) extend the ADS model to allow the prover and verifier to exchange many messages. This model was introduced by Cormode et al. \cite{vldb}.

The \emph{Arthur-Merlin streaming model} was introduced by Gur and Raz \cite{gur}: this model is equivalent to a restricted class of SIPs, in which  is only allowed to send a single message to  (which must consist entirely of random coin tosses, in analogy with the classical complexity class ), before receiving 's reply. 

The model of \emph{streaming delegation} was introduced by Chung et al. \cite{chung}, and corresponds to SIPs that only satisfy computational, rather than information-theoretic, soundness. 


\eat{
\begin{table}
\centering
\begin{tabular}{|c|c|c|c|}
\hline
Model Name & Soundness Type & Number of & Message \\
 & Type & Messages Allowed & Format Restrictions\\
\hline
Annotated Data Streams \cite{icalp} & Information-Theoretic & 1 & None\\
\hline
Streaming Interactive Proofs \cite{vldb} & Information-Theoretic & Arbitrary & None\\
\hline
Arthur-Merlin Streaming \cite{gur} & Information-Theoretic & 2 & 's message to \\
& & &  must consist only of coin tosses\\
\hline
Streaming Delegation \cite{chung} & Computational & 2 & None\\
\hline
\end{tabular}
\caption{Model Summary.}
\end{table}
}

\section{Key Results}
Obtaining exact answers even for basic problems in the
standard data streaming model is impossible using  space. In contrast, stream verification
protocols with  space and communication costs have been developed for
(exactly solving) a wide variety of problems. Many of these protocols have adapted powerful algebraic techniques 
originally developed in the classical literature on interactive proofs, particularly the \emph{sum-check} protocol of Lund et al. \cite{lfkn}.
All of the protocols described in Sections \ref{sec:ads}-\ref{sec:cs} apply even to streams in the strict turnstile update model, where universe items can be deleted as well as inserted. The protocols in Section \ref{sec:perupdate} do not support deletions. Unless stated otherwise, all protocols described in this survey are online, i.e., the honest prover's message at any given time does not depend on parts of the data stream that  has not yet seen.

\subsection{Annotated Data Streams}
\label{sec:ads}
Chakrabarti et al. \cite{icalp} showed that prescient ADS protocols can be exponentially more powerful than online ones for some problems. 
For example, there is a prescient ADS protocol with logarithmic space and communication costs for computing the median of a sequence of numbers:  sends  the claimed median  at the start of the stream, and while observing the stream,  checks that , and , which can be done using an -bit counter. Meanwhile, Chakrabarti et al. \cite{icalp} proved that any online protocol for  with communication
cost  and space cost  requires , and gave an online ADS protocol achieving these communication--space tradeoffs up to logarithmic factors.

Chakrabarti et al. \cite{icalp} also gave online ADS protocols achieving identical tradeoffs between space and communication costs for problems including \textsc{Frequency Moments} and \textsc{Frequent Items}, and used a lower bound  due to Klauck \cite{Klauck03} on the \emph{Merlin-Arthur communication complexity} of the \textsc{Set-Disjointness} function to show that these tradeoffs are optimal for these problems, even among prescient protocols. 
Subsequent work \cite{esa, semistreaming} gave similarly optimal online ADS protocols for several more problems, including maximum matching and counting triangles in graphs, and matrix-vector multiplication. Chakrabarti et al. \cite{soda} gave optimized protocols for streams whose
length  is much smaller than the universe size .



\subsection{Streaming Interactive Proofs}
Cormode et al. \cite{vldb} showed that several general protocols from the classical literature on interactive proofs can be simulated in the SIP model. 
In particular, this includes a
powerful, general-purpose protocol due to Goldwasser, Kalai, and Rothblum \cite{gkr} (henceforth, the GKR protocol). Given any problem 
computed by an arithmetic or Boolean circuit of polynomial size and polylogarithmic depth,
the GKR protocol requires only polylogarithmic space and communication while using polylogarithmic
rounds of verifier--prover interaction. This yields SIPs for exactly solving many basic streaming problems with
polylogarithmic space and communication costs, including \textsc{Frequency Moments}, \textsc{Frequent Items}, and \textsc{Graph Connectivity}.
Cormode et al. \cite{vldb} also gave optimized protocols for specific problems, including \textsc{Frequency Moments} (see the Detailed Example below).

Gur and Raz \cite{gur} gave an Arthur-Merlin streaming protocol for the \textsc{Distinct Elements} problem 
with communication cost  space cost  for any  satisfying , where the  notation hides factors that are polylogarithmic in . Klauck and Prakash \cite{prakashnew}
extended this protocol to give an SIP for \textsc{Distinct Elements} with polylogarithmic space and communication costs and logarithmically
many rounds of prover--verifier interaction. 

Chakrabarti et al. \cite{suresh} gave \emph{constant-round} online SIPs with logarithmic space and communication costs for many problems, including \textsc{Index}, \textsc{Range-Counting}, and \textsc{Nearest-Neighbor Search}. These protocols are exponentially more efficient than what can be achieved by constant-round online SIPs in which 's messages to the prover are independent of the input (such as is required in the Arthur-Merlin streaming model of \cite{gur}). 
For classical interactive proofs where the verifier is not restricted to be streaming, allowing 's messages to depend on the input does not yield analogous efficiency improvements: Goldwasser and Sipser \cite{goldwassersipser} showed that any interactive proof can be simulated with a polynomial blowup in all costs by an interactive proof in which 's messages to  consist entirely of random coin tosses.  





 
\subsection{Computationally Sound Protocols}
\label{sec:cs}
Computationally sound protocols may achieve properties that are unattainable in the information-theoretic setting. For example, they
typically achieve reusability, allowing the verifier to use the same randomness to answer many queries. In
contrast, most SIPs only support ``one-shot'' queries, because they require the verifier to reveal secret
randomness to the prover over the course of the protocol.

Chung et al. \cite{chung} combined the GKR protocol with fully homomorphic encryption (FHE) to give reusable
two-message streaming delegation protocols with polylogarithmic space and communication costs for any problem in the complexity class \textsc{NC}. They
 also gave reusable four-message protocols with polylogarithimic space and communication costs for any problem in the complexity class \textsc{P}. 
 
 Papamanthou et al. \cite{eurocryptpaper} gave
improved streaming delegation protocols for a class of low-complexity queries including point queries and range search: these
protocols avoid the use of FHE, and allow the prover to answer such queries in polylogarithmic time. 
(In contrast, protocols based on the GKR protocol \cite{chung, vldb} require the prover to spend time 
quasilinear in the size of the data stream after receiving
a query, even if the answer itself can be computed in sublinear time.) The protocols of \cite{eurocryptpaper} are based on \emph{hash trees}.


\subsection{Models Allowing Linear Total Communication}
\label{sec:perupdate}
The literature on stream verification contains two models that depart from those described above in that they allow  and  to communicate on each stream update. In particular, they allow a linear amount of total communication, but aim to bound the maximum amount of communication exchanged (and processing time spent) during any stream update.

\medskip
\noindent \textbf{A Statistically Sound Model.} 
Klauck and Prakash \cite{prakash} studied a variant of online SIPs, in which  and  are allowed to exchange a constant number of messages during each stream update, as long as at most a constant number of machine words are exchanged during each update. They gave protocols in this model for \textsc{Median}, for determining if a matrix is full-rank, and for approximating the longest-increasing subsequence of the stream.

\medskip
\noindent \textbf{A Computationally Sound Model.} Schr{\"o}der and Schr{\"o}der \cite{ccs} introduced a model requiring only computational soundness, which they called the \emph{verifiable data streaming} (VDS) model. This model is targeted at settings in which there are many parties who may wish to verify answers returned by the prover. In the VDS model,  is allowed to send a short message to  on each stream update, with no communication allowed in the reverse direction. VDS protocols are required to be reusable and publicly verifiable, in the sense that any party who knows 's public key can verify any response by the prover. The VDS model does not permit  to update the public key on every stream update, as this would require coordination between all parties wishing to verify information returned by the prover. 

There is a trivial VDS protocol for the \textsc{Index} problem, requiring polylogarithmic communication on each stream update. Recall that in the \textsc{Index} problem, the data stream specifies all entries of a vector , followed by an index , and the goal to output . In the trivial VDS protocol, for every stream update ,  sends to  a digital signature of the tuple . After the stream observation phase,  sends to  the claimed value of  along with a valid digital signature of the tuple . The protocol is secure assuming the prover cannot forge valid signatures: under this assumption, the only way  can efficiently produce a valid signature of the tuple  is if  had previously sent the signature to .

Schr{\"o}der and Schr{\"o}der \cite{ccs} gave a VDS protocol that supports {\sc Index} queries in a more general setting in which the stream not only contains entries of the vector , but also contains \textsc{Update} operations. Here, an \textsc{Update} operation changes the value of a designated entry  of  to a new value . However, in the protocol of \cite{ccs}, each  operation requires bidirectional communication (one message from  to , followed by one from  to ), as well as an update to 's public key. 

Krupp et al. \cite{krupp} gave related VDS protocols that reduces costs by logarithmic factors.


\subsection{Implementations}
Implementations of the GKR protocol were provided by Cormode et al. in \cite{itcs} and Thaler in \cite{crypto}. Cormode et al. \cite{itcs} also provided 
optimized implementations of several ADS protocols from \cite{icalp, esa}. Thaler et al. \cite{hotcloud} provided parallelized implementations using Graphics Processing Units. 

Qian \cite{sadimp} refined and implemented the streaming delegation protocol for point queries and range search of Papamanthou et al. \cite{eurocryptpaper}.

Implementations of VDS protocols are described by Schr{\"o}der and Simkin \cite{fc} and Krupp et al.  \cite{krupp}.


\section{Detailed Example}
As described in \cite{vldb}, the sum-check protocol can be directly applied to give an SIP for the th frequency moment problem with  rounds of prover--verifier iteration,
and  space and communication costs.
The sum-check protocol is described in Figure \ref{fig}.

\begin{figure}
\fbox{
\small
\begin{varwidth}{15cm}
\noindent \textbf{Input:}  is given oracle access to a -variate polynomial  over finite field  and an .\\
\noindent \textbf{Goal:} Determine whether . \\
\begin{itemize}
\item In the first round,  computes the univariate polynomial  
and sends  to .  checks that  is a univariate polynomial of degree at most , and that ,
rejecting if not.

\item  chooses a random element , and sends  to .

\item In the th round, for ,  sends to  the univariate polynomial
 
 checks that  is a univariate polynomial of degree at most ,
and that , rejecting if not. 

\item  chooses a random element , and sends  to .

\item In round ,  sends to  the univariate polynomial
  checks that  is a univariate polynomial of degree at most
, rejecting if not. 

\item  chooses a random element  and evaluates  with
a single oracle query to .  checks that
, rejecting if not.

\item If  has not yet rejected,  halts and accepts.
\end{itemize}
\end{varwidth}
}
\caption{Description of the sum-check protocol.  denotes the degree of  in the th variable.} 
\label{fig}
\end{figure}

\medskip
\noindent \textbf{Properties and Costs of the Sum-check Protocol.}
The sum-check protocol satisfies perfect completeness, and has soundness error ,
where  denotes maximum degree of  in any variable (see \cite{lfkn} for a proof). 
There is one round of prover--verifier interaction in the sum-check protocol for each
of the  variables of , and the total communication  field elements.

Note that as described in Figure \ref{fig}, the sum-check protocol assumes that the verifier has oracle access to . However, this will not be the case in applications,
as  will ultimately be a polynomial that depends on the input data stream.

\medskip \noindent \textbf{The SIP for Frequency Moments.}
In the th frequency moments problem, the goal is to output , where 
is the number of times item  appears in the data stream . 
For a vector , let 
, where  and .  is the unique multilinear polynomial that maps 

to  and all other values in  to 0, and it is referred to
as the \emph{multilinear extension} of . 

For each , associate  with a vector  in the natural way, and
let  be a finite field with .
Define the polynomial  via

Note that  is the unique multilinear polynomial satisfying the property that  for all
.

The th frequency moment of  is equal to .
Hence, in order to compute the th frequency moment of , 
it suffices to apply the sum-check protocol to the polynomial . 
This requires  rounds of prover--verifier interaction, and since the total degree of  is , 
the total communication cost is  field elements.
which require  total bits to specify. 

At the end of the sum-check protocol,  must compute  for randomly chosen . It suffices for  to evaluate , since . The following lemma establishes
that  can evaluate  with a single pass over , while storing  field elements.

\begin{lemma}
 can compute  with a single streaming pass over , while storing  field elements.
\end{lemma}
\begin{proof}
Given any stream update , let  denote the binary vector associated with . 
It follows from Equation \eqref{eq} that .
Thus,  can compute  incrementally from the raw stream by initializing , and processing each update  via:

 only needs to store  and , which is  field elements in total.
\end{proof}

\nocite{aw, prakash, ccs,allspice, bestorder}
\section{Open Problems}
\begin{itemize}
\item Two-message online SIP protocols with logarithmic space and communication costs are known for several functions, including the \textsc{Index} function (cf. \cite{suresh}). It is also known that \emph{existing techniques} cannot yield 2- or 3-message online SIPs of polylogarithmic cost for the \textsc{Median} or \textsc{Frequency Moments} problems.
However, the following is open: exhibit an explicit function  that cannot be computed by any online two-message SIP with communication and space costs
both bounded above by , for some . See \cite{suresh} for details.
Candidate functions satisfying this property include \textsc{Set-Disjointness} and \textsc{Inner-Product-Mod-2}.
\item For several functions , it is known that
any online ADS protocol for  with communication
cost  and space cost  requires . This lower bound is tight
in many cases, such as for the \textsc{Index} function (cf. \cite{icalp}). However,
the following is open: exhibit an explicit function that cannot be computed by any online ADS protocol with communication and space costs
 both bounded above by , for some . 
\item Chakrabarti et al. \cite{soda} and Thaler \cite{semistreaming} have also identified explicit problems which are just as hard for online ADS protocols as they are in the standard data streaming model, in the sense that the sum of 's space cost and the communication cost in any online ADS protocol must be at least as large as the space complexity of a standard streaming algorithm for the problem, up to constant or logarithmic factors. However, it is open to exhibit an explicit function  that is just as hard in for \emph{prescient} ADS protocols as it is in the standard streaming model. 
 
 Formally, the following is open: exhibit an explicit function  such that, for any constant ,  cannot be computed by any prescient ADS protocol with communication and space cost both bounded above by . Here,  denotes the minimum space complexity of any streaming algorithm that, for any input stream , outputs  with probability at least . Candidate functions include \textsc{Graph Connectivity} and \textsc{Graph Bipartiteness}. 
\end{itemize}

For all three problems, functions satisfying the relevant properties are known to exist by counting arguments. But as stated above, identifying explicit functions satisfying the properties remains open.














 \bibliographystyle{plain}
 \bibliography{mydatabase}




\end{document}
