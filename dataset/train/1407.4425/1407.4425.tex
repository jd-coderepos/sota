\documentclass{LMCS}

\def\dOi{10(3:19)2014}
\lmcsheading {\dOi}
{1--51}
{}
{}
{Mar.~14, 2012}
{Sep.~11, 2014}
{}

\ACMCCS{[{\bf Theory of computation}]: Semantics and
  reasoning---Program semantics---Categorical semantics} 

\keywords{corecursive algebra, corecursive monads, Bloom monads,
  iteration theory}

\usepackage{hyperref}
\usepackage{url}

\usepackage{amsmath, amsthm, amscd, amsfonts, amssymb}

\usepackage[all]{xy}
\SelectTips{cm}{}

\usepackage[final,layout=footnote]{fixme}



\newcommand{\takeout}[1]{\empty}




\theoremstyle{plain}
\newtheorem{theorem}{Theorem}[section]
\newtheorem{algorithm}[theorem]{Algorithm}
\newtheorem{axiom}[theorem]{Axiom}
\newtheorem{case}[theorem]{Case}
\newtheorem{conclusion}[theorem]{Conclusion}
\newtheorem{condition}[theorem]{Condition}
\newtheorem{criterion}[theorem]{Criterion}
\theoremstyle{definition}
\newtheorem{exercise}[theorem]{Exercise}
\newtheorem{note}[theorem]{Note}
\newtheorem{problem}[theorem]{Problem}
\newtheorem{solution}[theorem]{Solution}
\newtheorem{summary}[theorem]{Summary}
\newtheorem{construction}[theorem]{Construction}
\numberwithin{equation}{section}

\def\card{\mathrm{card}\,}
\def\emp{\emptyset}
\def\o{\cdot}
\def\colim{\mathop{\mathrm{colim}}}


\begin{document}
\FXRegisterAuthor{sm}{asm}{SM}\title[Corecursive Algebras, Corecursive Monads and Bloom Monads]{Corecursive Algebras, Corecursive Monads \\and Bloom Monads}




\author[J.~Ad\'amek]{Ji\v{r}\'\i\ Ad\'amek\rsuper a}
\address{{\lsuper a}Institut f\"ur Theoretische Informatik, Technische Universit\"at Braunschweig, Germany}
\email{adamek@iti.cs.tu-bs.de}

\author[M.~Haddadi]{Mahdie Haddadi\rsuper b}
\address{{\lsuper b}Department of Mathematics, Statistics and Computer Science, Semnan University, Semnan, Iran}
\email{mahdiehaddadi7@gmail.com}

\author[S.~Milius]{Stefan Milius\rsuper c}
\address{{\lsuper c}Lehrstuhl f\"ur Theoretische Informatik, Friedrich-Alexander-Universit\"at Erlangen-N\"urnberg, Germany}
\email{mail@stefan-milius.eu}




\begin{abstract}
An algebra is called corecursive if from every coalgebra a unique coalgebra-to-algebra homomorphism exists into it. We prove that free corecursive algebras are obtained as coproducts of the terminal coalgebra (considered as an algebra) and free algebras. The monad of free corecursive algebras is proved to be the free corecursive monad, where the concept of corecursive monad is a generalization of Elgot's iterative monads, analogous to corecursive algebras generalizing completely iterative algebras. We also characterize the Eilenberg-Moore algebras for the free corecursive monad and call them Bloom algebras.
\end{abstract}

\maketitle



\section{Introduction}

The study of structured recursive definitions is fundamental in many
areas of computer science. This study can use algebraic methods
extended by suitable recursion concepts. One such example are
completely iterative algebras: these are algebras in which recursive
equations with parameters have unique solutions, see \cite{m_cia}. In
the present paper we study corecursive algebras.  These are
-algebras for a given endofunctor  in which recursive equations
without parameters have unique solutions. Equivalently, for every
coalgebra there exists a unique coalgebra-to-algebra morphism. The
dual concept, recursive coalgebra, was introduced by G.~Osius in
\cite{g}, and for endofunctors weakly preserving pullbacks P.~Taylor
proved that this is equivalent to being parametrically recursive, see
\cite{t}. In the dual situation, since weak preservation of pushouts
is rare, the concepts of corecursive algebra and completely iterative
algebra usually do not coincide. The former was studied by
V.~Capretta, T.~Uustalu and V.~Vene \cite{cuv2}, and various
counter-examples demonstrating e.g. the difference of the two concepts
for algebras can be found there. In the present paper we contribute to
the development of the mathematical theory of corecursive
algebras. The goal is to eventually arrive at a useful body of results
and constructions for these algebras. A major ingredient of any theory
of algebraic structures is the study of how to freely endow an object
with the structure of interest. So the main focus of the present paper
are corecursive -algebras freely generated by an object . Let
 denote the free -algebra on  and let  be the terminal
-coalgebra (which, due to Lambek's Lemma, can be regarded as an
algebra). We prove that the coproduct of these two
algebras  is the free corecursive algebra on
. Here  is the coproduct in the category of
-algebras. For example for the endofunctor  the
algebra  consists of all (finite and infinite) binary trees with
finitely many leaves labelled in .

We also introduce the concept of a corecursive monad.  This is a
weakening of completely iterative monads of C.~Elgot, S.~Bloom and R.~Tindell \cite{ebt} 
analogous to corecursive algebras as a weakening of completely
iterative ones.  The monad  of free corecursive algebras
is proved to be corecursive, indeed, this is the free corecursive
monad generated by . For endofunctors of {\bf Set} we also prove
the converse: whenever  generates a free corecursive monad, then it
has free corecursive algebras (and the free monad is then given by the
corresponding adjunction).

We characterize the Eilenberg-Moore algebras for the free corecursive monad:  these are -algebras in which  every recursive equation without parameters has a solution (not necessarily unique), and which allow a functorial choice of solutions. We call these algebras \emph{Bloom algebras}; they are analogous to the complete Elgot algebras of \cite{amv3} where the corresponding monad was the free completely iterative monad on .

We also study the finitary versions of our concepts. An algebra  is
called finitary corecursive if all coalgebras on finitely presentable
objects have a unique coalgebra-to-algebra morphism in to . And
finitary corecursive monads are defined analogously. Every finitary
endofunctor  is proved to generate a free finitary corecursive
monad . We form the free strict functor  on
 and obtain a monad  on the category of finitary
functors given by
 
The Eilenberg-Moore algebras for  are called Bloom
monads. They correspond to iteration theories of Bloom and \'Esik:
recall from \cite{amv_what} that the latter are precisely the
Eilenberg-Moore algebras for the free-iteration-theory monad. Bloom
monads are monads  equipped with an operation 
assigning to every finitary non-parametric equation morphism a
solution in free algebras for . This operation satisfies
precisely the equational properties that non-parametric iteration in
Domain Theory satisfies. We list some of those equational
properties. The question whether our list is complete is open.

This paper is a revised and extended version of the conference paper
\cite{ahm11}. Here we added all technical details and proofs, and our
discussion of the equational properties of Bloom monads and their
properties in new.

\subsection*{Acknowledgments}
We are grateful to Zolt\'an \'Esik for a substantial contribution to the discussion of equations in Bloom monads. And to Paul Levy who suggested that  Proposition \ref{Bloom cat=slice cat} holds.






\section{Corecursive Algebras}
The following definition is the dual of the concept introduced by G. Osius in \cite{g} and studied by P. Taylor \cite{t,t2}. We assume throughout the paper that  a category  and an endofunctor  are given. We denote by  the category of algebras  and homomorphisms, and by  the category of coalgebras  and homomorphisms. A coalgebra-to-algebra morphism from the latter to the former  is a morphism  such that .



\begin{defi}\label{def of cor.alg}
An algebra  is called {\it corecursive} if for every coalgebra  there exists a unique coalgebra-to-algebra homomorphism . That is, the square

commutes. We call  an {\it equation morphism} and  its {\it solution}.
\end{defi}
\begin{rem}
 For an endofunctor on {\bf Set}, we can view  as a system of recursive equations using variables from the set , and  is the solution of the system. We illustrate this on classical -algebras. These are the algebras for the polynomial set functor

where  is the arity of . For every set  (of recursion variables) and every system of mutually recursive equations

one for every , where  has arity  and , we get the corresponding coalgebra

The square~(\ref{eq:sol}) tells us that the substitution of  for  makes the formal equations  identities in :

\end{rem}
\begin{exa}\label{exam of core}\hfill
  \begin{enumerate}
  \item  In~\cite{cuv2} this concept of corecursive algebras is studied and compared with a number of related concepts. A concrete example of a corecursive algebra from that paper, for the endofunctor  on {\bf Set}, is the set  of all streams. The operation  is given by  having head  and continuing by the merge of  and .

\takeout{
(2) In a category with finite products consider algebras on one binary operation, i.e., . An algebra  is corecursive iff it has a unique idempotent (global) element. That is, a unique  with .

Indeed, this is necessary since  is the solution of . And it is sufficient: given an equation morphism , the morphism  is
easily seen to be the unique solution of  in .
}

\item If  has a terminal coalgebra , then by Lambek's Lemma  is invertible and the resulting algebra  is corecursive. In fact, this is the initial corecursive algebra, that is, for every corecursive algebra   a unique algebra homomorphism  from  exists, see the dual of~\cite[Proposition~2]{cuv2}. There also the converse is proved (dual of Proposition 7), that is, if the initial corecursive algebra exists, then it is a terminal coalgebra (via the inverse of the algebra structure).

\item The trivial terminal algebra , where  is the terminal object in , is clearly corecursive.

\item If  is a corecursive algebra, then so is , see~\cite[Proposition~6]{cuv2}. We generalize this in Lemma \ref{generalized} below.

\item Combining (3) and (4) we conclude that the terminal -chain

consists of corecursive algebras. Indeed, the continuation to 
for all ordinals (with  for all limit
ordinals) also yields corecursive algebras. This follows from the
following.
\end{enumerate}
\end{exa}

\begin{prop}\label{lim of core is core}
Let  be a complete category. Then corecursive algebras are closed under limits in . Thus, limits of corecursive algebras are formed on the level of .
\end{prop}

\begin{proof}
It is easy to verify that limits in  are formed on the level of . Let us prove that the product of corecursive algebras is  corecursive. The proof for general limits is analogous.

Let  be  the product of corecursive algebras , with projections . For every coalgebra  we have the unique coalgebra-to-algebra morphism , for all , and the morphism  is a coalgebra-to-algebra morphism. Indeed, for every , the diagram

commutes, except perhaps for the left hand inner square; but this
suffices to establish the desired commutativity of the left hand
square.  Since all  are corecursive, the uniqueness of
 follows from the observation that there is a one-one
correspondence between solutions  of  in  and
families of solutions  of  in , for all
.
\end{proof}

In the following we write  and  for the injections of a coproduct.

\begin{lem}\label{generalized}
Let  be an algebra and   a morphism. Then  is a corecursive algebra if and only if the algebra

is corecursive.
\end{lem}

\begin{proof}
Let  be a corecursive algebra and  be an equation morphism. Then there is a unique solution:

Now inspection of the following commutative diagram shows that  is a solution of  in .

Indeed, commutativity of the middle rectangle follows from Diagram (\ref{1}), the lower triangle on the right is trivial and the upper triangle is the definition of .
To show the uniqueness of the solution, suppose that  is a solution for , so we have the following commutative diagram:

Since the solution  in  is unique,  and hence


Conversely, let  be a corecursive algebra and  be an equation morphism. So there exists a unique solution  of  in , and hence we have the above commutative diagram with  in lieu  of . That is  is a solution of  in the algebra  . To show uniqueness suppose that  is a solution of , that is . Then we have Diagrams (\ref{1}) and (\ref{*}) with the morphism  in lieu of . So, by uniqueness of solution in the corecursive algebra , we have   , and hence .
\end{proof}

\begin{exa}
  \label{ex:binary}
Binary algebras: For , every algebra  (given by the binary operation ``" on a set ) which is corecursive has a unique {\it idempotent} . This is the solution of the recursive equation 
expressed by the isomorphism . Moreover the idempotent is {\it completely factorizable}, where
the set of all completely factorizable elements is defined to be the largest subset of  such that every element  in it can be factorized as , with  completely factorizable. The corecursiveness of  implies  that no other element but  is completely factorizable: consider the system of recursive equations

for all finite binary words . Every completely factorizable element  provides a solution  with . Since solutions are unique, .

Conversely, every binary algebra  with an idempotent  which is the only completely factorizable element is corecursive. Indeed, given a morphism , the constant map  with value  is a coalgebra-to-algebra morphism. Conversely, if  is a coalgebra-to-algebra morphism, then for every  the element  is clearly completely factorizable. Therefore, .
\end{exa}

\begin{rem}\label{cia}
Recall the concept of {\it completely iterative algebra} ({\it cia} for short) from \cite{m_cia}: it is an algebra  such that for every ``flat equation" morphism  there exists a unique solution, i.e. a unique morphism  such that the square
 
commutes. This is obviously stronger than corecursiveness because every coalgebra  yields a flat equation morphism . Then solutions are determined uniquely. Thus, for example, in the category of complete metric spaces with distance less than one and nonexpanding functions, all algebras for contracting endofunctors (in the sense of P. America and J. Rutten \cite{america+rutten}) are corecursive, because, as proved in \cite{m_cia}, they are cia's. Here is a concrete example:  equipped with the metric taking  of the maximum of the two distances is contracting. Thus every binary algebra whose operation is contracting is corecursive.
\end{rem}

\begin{exa}
The endofunctor  has many corecursive algebras that are not cia's. For example the algebra  of all binary trees with finitely many leaves. The operation is tree-tupling and the only completely factorizable tree is the complete binary tree . Thus,  is corecursive. However, if  denotes the root-only tree, then the system of recursive equations

does not have a solution in  (because the tree corresponding to  has infinitely many leaves). Thus  is not a cia.
\end{exa}



\begin{lem}\label{homos are solution prese}
Every homomorphism  in  with  and  corecursive preserves solutions. That is,  given a coalgebra  with a solution  in the domain algebra, then  is the solution in the codomain one.
\end{lem}

\begin{proof}
This follows from the diagram

\end{proof}


\noindent We thus consider corecursive algebras as a full subcategory  of . We obtain a forgetful functor


In Section~\ref{sec:4} we prove that this forgetful functor has a left adjoint, that is, free corecursive algebras exist, if and only if a terminal coalgebra  exists and every object  generates a free algebra  (i.e., the forgetful functor  has a left adjoint). Our result holds for example for all set functors, and for them the formula for the free corecursive algebra is , where  is the coproduct in .

Recall from \cite{GU} that given an infinite cardinal number , a functor is called {\it -accessible} if it preserves -filtered colimits. An object  whose hom-functor  is -accessible is called {\it -presentable}. A category  is {\it locally -presentable} if it has
\begin{enumerate}[label=
\xymatrix{
X\ar[r]^{e^\dagger}\ar[d]_e&C&A_t\ar[l]_{k_t}\ar@{<-}`u[l]`[ll]_-s[ll]\\
HX\ar[r]_{He^\dagger}&HC\ar[u]_c&HA_t\ar[u]_{a_t}\ar[l]^{Hk_t}\ar@{<-}`d[l]`[ll]^-{Hs}[ll]\\
}

k_t\cdot(a_t\cdot Hs\cdot e)&=c\cdot Hk_t\cdot Hs\cdot e\\
&=c\cdot He^\dagger \cdot e\\
&=e^\dagger \\
&=k_t\cdot s.
u\cdot s=u\cdot a_t\cdot Hs\cdot e=a_{t'}\cdot Hu\cdot Hs\cdot e = a_{t'} \cdot H(u\cdot s) \cdot e.
\xymatrix{
X_i\ar[r]^{x_i}\ar[d]_{e_i}&X\ar[r]^{e^\dagger}\ar[d]^e&C\\
HX_i\ar[r]_{Hx_i}&HX\ar[r]_{He^\dagger}&HC\ar[u]_c
}
\dagger:\mathsf{Coalg}\, H\rightarrow {\mathcal A}/A.
\xymatrix{
X\ar[r]^{e}\ar[d]_{h}&HX\ar[d]^{Hh}\\
X'\ar[r]_{f}&HX'
}

\xymatrix{
X\ar[rr]^h\ar[dr]_{e^\dagger}&&X'\ar[ld]^{f^\dagger}\\
&A
}
{\alph*}]
  \item Every corecursive algebra is a Bloom algebra. Indeed,
    functoriality easily follows from the uniqueness of solutions due
    to the diagram
    

  \item A unary algebra  () is a Bloom algebra iff
     has a fixpoint, i.\,e., a morphism  with . More precisely:
    \begin{enumerate}[label=(\arabic*)]
    \item Given a fixpoint, then  is a Bloom algebra
      where  for the unique morphism .
    \item Given a Bloom algebra , then  is a fixpoint of .
    \end{enumerate}
  \item Let  have finite products. An algebra  for  is a Bloom algebra if and only
    if it has an idempotent global element, that is 
    satisfying  (recall that ). More
    precisely:

    \begin{enumerate}[label=(\arabic*)]
    \item Given an idempotent , we have a Bloom algebra , where  is the constant function with value .

    \item Given a Bloom algebra , there exists an idempotent  such that  is the constant function with value .
    \end{enumerate}
    Compare this with Example \ref{ex:binary}. In particular every group, considered as a binary algebra in {\bf Set}, is thus a Bloom algebra in a unique sense. But no nontrivial group is corecursive.

  \item Every continuous algebra is a Bloom algebra if we define
     to be the least solution of . More detailed, let
     be a locally continuous endofunctor of the category
     of complete ordered sets (i.\,e., partially ordered
    sets with a least element  and with joins of
    -chains). For every -algebra  and every equation
    morphism , we can define in 
    a function  as a join of the sequence
     defined by  and
    . Then the least
    solution of  is  and
     is a Bloom algebra. Example~(b) demonstrates that
    this need not be corecursive.

  \item Every product of Bloom algebras is a Bloom algebra. We define  as in the proof of Proposition \ref{lim of core is core}. More generally: every limit of Bloom algebras is a Bloom algebra.

  \item Every complete Elgot algebra in the sense of
    \cite{amv_classes} is a Bloom algebra.
  \end{enumerate}
\end{exa}

\begin{defi}\label{preserv solu}
By a {\it homomorphism} of Bloom algebras from  to  is meant an algebra homomorphism  preserving solutions, that is, for every coalgebra  the triangle

commutes. We denote by  the corresponding category of Bloom algebras.
\end{defi}

\begin{prop}\label{Bloom cat=slice cat}
Let  be a terminal coalgebra for . The category of Bloom algebras for  is isomorphic to the slice category .
\end{prop}

\begin{proof}
Let us, for a coalgebra , denote the unique coalgebra homomorphism from  to  by .
We shall define two functors between  and  and show that they are mutually  inverse.

(a) From Bloom algebras to the slice category : given a Bloom algebra  we form the solution  which clearly is an object  of . For a homomorphism  of Bloom algebras, we clearly have a morphism  of , since  is solution preserving. This defines a functor from  to .

(b) From  to Bloom algebras: Suppose we are given an object  in , that is, an algebra homomorphism:

We define for every   its dagger as . This is functorial; indeed, for every coalgebra homomorphism  we have  by unicity of the universal property of the terminal coalgebra , thus

 In addition, every morphism  of  is a homomorphism of Bloom algebras :

That this gives a functor from  to  is immediate.

(c) The two functors above  are mutually inverse. Indeed, it suffices to show that we have  a bijection on the level of objects, since both functors    are  the identity maps on morphisms. So for  in  we form first  as in (b) and then  as in (a) and we have , as  is the identity (being the unique coalgebra homomorphism from  to itself). Finally, given a Bloom algebra  we first form  as in (a) and then  as in (b). Then we have

This completes the proof.
\end{proof}

\begin{rem}\label{final B=final coalg}
  Being an algebra homomorphism and preserving solutions are
  independent concepts: neither of them implies the other one. To see
  this, consider for  an algebra  with a
  binary operation  such that  and  are idempotent. We turn
   into a Bloom algebra by taking  for
  every . Then there are two homomorphisms
  from the one-point binary algebra (which clearly is corecursive) to
  , yet only one of them is solution preserving. Thus, there exist
  homomorphisms which are not solution preserving. 

  Conversely, there
  exist solution preserving morphisms which are not homomorphisms. To
  see this, let us assume that we have  for all  in
  . There are two different structures of Bloom algebras on ,
  () and (). The map on  which swaps  and  is a
  solution preserving map between the two Bloom algebras, but not a
  homomorphism.

\end{rem}
 \begin{prop}\label{3x}
 An initial Bloom algebra is precisely a terminal coalgebra.
   \end{prop}
More precisely, the statement in Example \ref{ex:binary} generalizes from corecursive algebras to Bloom algebras.
Indeed, the proof in \cite{cuv2} can be used again.
\begin{lem}\label{A is B then B is B}
If  is a Bloom algebra and  is a homomorphism of algebras, then there is a unique structure of a Bloom algebra on  such that  is a solution preserving morphism. We call it, \emph{the Bloom algebra induced by} .
\end{lem}

\begin{proof}
For every coalgebra  we define


and verify that  is a solution of  by the following commutative diagram
 
Functoriality is easily checked too:
let  be a coalgebra homomorphism. Then the following equations hold:

Finally,  is clearly solution preserving.

The unicity of the Bloom algebra structure given by  is clear.
\end{proof}

\begin{rem}
 We are going to characterize the left adjoint of the forgetful functor

In other words, we characterize the free Bloom algebras: they are
coproducts  of the terminal coalgebra and free algebras. For that we first attend to the existence of those ingredients.
\end{rem}

\begin{lem}\label{free B-> final coalg}
Let  be a complete category. If  has a free Bloom algebra on an object  with , then  has a terminal coalgebra.
\end{lem}
\begin{proof}
The free Bloom algebra  on , is weakly initial in . To see this, choose a morphism . For every Bloom algebra  the solution  extends to a homomorphism  of Bloom algebras.


Since  is complete by Example \ref{ex 3}(e), we can use Freyd's Adjoint Functor Theorem.\smnote{Should be Freyd's Initial Object Theorem}
The existence of a weakly initial object implies that  has an initial object. Now apply Proposition \ref{3x}.
\end{proof}

\takeout{
\begin{exa}\label{set functor}
Every set functor with a free Bloom algebra has a terminal coalgebra. The condition {\bf Set} is here automatically satisfied except when  is constantly .
\end{exa}
}

\begin{construction}\label{free chain} Free-Algebra Chain.
Recall from \cite{a} that if  is cocomplete, we can define a chain constructing the free -algebra on  as follows:

We mean the essentially unique chain  with

and for limit ordinals 

whose connecting morphisms  are defined by

and for limit ordinals 


This chain is called the {\it free-algebra chain}. If it \emph{converges} at some ordinal , that is, if  is an isomorphism, then  is a free algebra on . More detailed: this isomorphism turns  into a coproduct

and thus  is an algebra via the left-hand coproduct injection, and the right-hand one  yields the universal arrow.
\end{construction}

\begin{cor}\label{c-access}
Every accessible endofunctor of a cocomplete category has free algebras.
\end{cor}
Indeed, if the given functor is -accessible, the free-algebra chain converges at .

\begin{defi}(See \cite{at})
We say that in a given category monomorphisms are {\it constructive} provided that
\begin{enumerate}[label=\xymatrix{b\equiv H(HA+Y)\ar[r]^-{H[a,\eta]}& HA\ar[r]^-{\mathsf{inl}}&HA+Y}e^\ddag=\mathsf{inl}\cdot He^\dag\cdot e, \ \ \ {\rm for\ all\ } e:X\rightarrow HX.h:A\rightarrow HA+Y\ \ \ {\rm with}\ \ \ h\cdot \eta=\mathsf{inr}.
\xymatrix{
H(HA+Y)\ar[d]_{H[a,\eta]}\ar[r]^-{H[a,\eta]}&HA\ar[r]^-{\mathsf{inl}}\ar[rd]^a&HA+Y\ar[d]^{[a,\eta]}\\
HA\ar[rr]_a&&A
}
[a,\eta]\cdot e^\ddag=a\cdot He^\dag\cdot e=e^\dag. ([a,\eta]\cdot h)\cdot \eta=[a,\eta]\cdot \mathsf{inr}=\eta.[a,\eta]\cdot h=id_A.h\cdot a=b\cdot Hh=\mathsf{inl}\cdot H([a,\eta]\cdot h)=\mathsf{inl}.h\cdot [a,\eta]=[h\cdot a,h\cdot\eta]=[\mathsf{inl},\mathsf{inr}]=id.m_0\equiv \eta:Y\rightarrow A\xymatrix{ m_{i+1}\equiv  HV_i+Y\ar[rr]^-{Hm_i+id}&& HA+Y\ar[r]^-{[a,\eta]}& A}.{\alph*}]
\item   is cocomplete and has a set 
  of -presentable objects (that is, objects whose
  hom-functors preserve -filtered colimits) such that every
  object is a -filtered colimit of objects in , and

\item  preserves -filtered colimits.
\end{enumerate}

From this it follows that  is locally
-presentable, where  (see \cite{ap}). Thus, this category  has a terminal object, .
We know from Corollary \ref{c-access} that the free algebra 
exists. And  exists since the category of algebras for an
accessible functor on a locally presentable category is itself locally
presentable and therefore comcomplete. 


\begin{rem}
For concrete examples of  see Example~\ref{some exams} below.
\end{rem}

\begin{prop}\label{lim of B is B}
Let  be a complete category. Then so is  and limits of Bloom algebras are formed on the level of .
\end{prop}
\begin{proof}
This is completely analogous to the proof of Proposition \ref{lim of core is core}. The verification that the function  is functorial is trivial.
\end{proof}

\begin{cor}\label{cor:mono}
  For a complete category , the monomorphisms in  are precisely those homomorphisms carried by monomorphisms in . 
\end{cor}
\begin{proof}
  To see this use that in any category with pullbacks a morhism  is a monomorphism iff its kernel pair consists of two identity morphisms. 
\end{proof}

\section{Free Corecursive Algebras}
\label{sec:4}

For accessible functors  we prove that free corecursive algebras  exist and, in the case where  preserves monomorphisms, they coincide with the free Bloom algebras . Moreover an iterative construction of these free algebras (closely related to the free algebra chain in \ref{free chain}) is presented.

We first prove that the category of corecursive algebras is strongly epireflective in the category of Bloom algebras. That is, the full embedding is a right adjoint, and the components of the unit of the adjunction are strong epimorphisms.


\begin{prop}\label{cor reflected of B}
For every accessible endofunctor of  a locally presentable category, corecursive algebras form a strongly epireflective subcategory of the category of Bloom algebras. In particular, every Bloom subalgebra of a corecursive algebra in  is corecursive.
\end{prop}

\begin{proof}
Let  be an infinite cardinal such that  is a
locally -presentable category and  preserves
-filtered colimits. Since  can be chosen
arbitrarily large, we can assume that  is uncountable. The
category  is locally -presentable. The
proof is analogous to that of Proposition \ref{dirlim of cor is cor}
(the only difference is that in the proof of the uniqueness of
 we simply observe that since 's are supposed to be
solution-preserving, we have , where  is the
dagger of  in ). Consequently,  is a
complete, well-powered, and cowellpowered category, and it has (strong
epi-mono) factorization of morphisms, see \cite{ar}. The full
subcategory of corecursive algebras is closed under products by
Proposition \ref{lim of core is core}. Thus, the proof will be
completed when we prove that the subcategory of  corecursive algebras
is closed in  under subalgebras, then it is
strongly epireflective (see \cite[Theorem~16.8]{ahs}).

Let  be a monomorphism in
 with  corecursive. From Corollary~\ref{cor:mono} we have that  is a monomorphism in . It is our task to prove that for every coalgebra  the morphism  is the unique solution in . This follows from the fact that . Since  is unique in  and  is a monomorphism in , the proof is concluded.
\end{proof}

\begin{cor}\smnote{added cor. to satisfy referee}
Every accessible endofunctor of a locally presentable category has free corecursive algebras.
\end{cor}
Indeed, since the functors  and  have left adjoints by Corollary \ref{free B exists} and Proposition \ref{cor reflected of B}, their composite has a left adjoint.

\begin{rem}
We believe  that in the generality of the above corollary, the free corecursive algebras are  (as in Corollary \ref{free B exists}). But we can only prove this in the case where  preserves monomorphisms and monomorphisms are constructive. We are going to apply the following transfinite construction of free corecursive algebras closely related to the free algebra construction of \ref{free chain}
\end{rem}

\begin{construction}\label{free core.chain}Free-Corecursive-Algebra Chain.

Let  be cocomplete and  have a terminal coalgebra . We define an essentially unique chain  by

and for limit ordinals 

The connecting morphisms  are defined by


and for limit ordinals 

We say that the chain {\it converges at } if the connecting morphism  is an isomorphism, thus . Then  is an algebra (via ) connected to  via .
\end{construction}

\begin{prop}\label{cor chain converges U=T+FY}
Let  be a cocomplete and wellpowered category with constructive mono\-morphisms, and let  preserve monomorphisms and have a terminal coalgebra . If the corecursive chain for  converges in  steps, then .
\end{prop}

\begin{rem}
(a) More detailed, we prove that a free algebra  exists and the algebra  (obtained from ) is a coproduct of  and  in .

(b) The proposition is valid for every fixpoint of , not necessarily a terminal coalgebra: given any isomorphism  and forming the corresponding chain with  and , then  whenever it converges, it yields a coproduct of  and the free algebra on  in .
\end{rem}
\begin{proof}(1) A free algebra  exists. To prove this, we define a natural transformation  from the free-algebra chain to the corecursive chain (delayed on finite ordinals by one step, recall that  for all infinite ordinals). Put 
and 
The first naturality square

 clearly commutes, and the -th square implies the next one easily. Therefore, the limit ordinals  define  automatically. Since  preserves monomorphisms, we see by transfinite induction that all 's are monic. Consequently, we obtain a transfinite chain of subobjects of 
 
Since  has only a set of subobjects, there exist ordinals  and  with  such that above monomorphisms with  and  represent the same subobject. Similar reasoning as in point~(2) \smnote{clarified reasoning as required by referee}of the proof of Proposition~\ref{free B then free alg} then shows that  is an isomorphism using the commutative triangle

We thus proved that the algebra  is free on  with respect to . Shortly .

(2) Analogously,  is an algebra with respect to  (since  is invertible). We will prove that this algebra is the coproduct of  and  with injections  and 
in . Indeed,  is an algebra homomorphism:  and we have , therefore the following square commutes:


Also  is a homomorphism: the following diagram


commutes and yields (by inverting ) the square


To verify the universal property, let  be an algebra and  and  be homomorphisms. We prove that there exists a unique homomorphism  with  and . Put .

{\it Existence}: Define a cocone  of the chain from
Construction~\ref{free core.chain} by
 
The first naturality triangle commutes because  is a homomorphism, thus, :

The further verification of the naturality, , is now an easy transfinite induction. We obtain the desired homomorphism .
Indeed, since  (by the above rule for ), the square

commutes.

The first equation  follows
from  and . For the second one
 observe that
both sides are algebra homomorphisms. Thus, it is sufficient to prove
that the universal arrow  merges them. Recall that , thus we need to prove

This follows from , thus the outside of the diagram below commutes as desired:


{\it Uniqueness}: Consider an algebra homomorphism
 with  and
, we prove 
by transfinite induction on . The case  yields . The initial step is clear:
Assuming , we are going to prove that the triangle

commutes. The left-hand component with domain  commutes because  is a homomorphism, that is, :

The right-hand one with domain  follows from : we have

where the last equation follows from the commutative diagram below expressing :

Finally, for a limit ordinal  we easily derive  by extending with the colimit injections , , and using that they are jointly epic: 
\end{proof}

\begin{thm}\label{free cor exists}
Let  be a locally presentable category with constructive monomorphisms. Every accessible endofunctor preserving monomorphisms has free corecursive algebras .
\end{thm}

\begin{proof}
From Corollary \ref{free B exists} we know that  is a free Bloom algebra, thus, it is sufficient to prove that this algebra is corecursive. For that, we use Proposition \ref{cor reflected of B} and find a corecursive algebra such that  is its subalgebra; this will finish the proof.

The endofunctor  is also accessible. Thus, it also has a terminal
coalgebra (see the proof of Corollary~\ref{free B exists}). We denote it by
. The components of the inverse of its coalgebra structure
 are denoted by
 and , respectively. As proved in \cite{m_cia} the algebra  is a cia for , cf. Remark \ref{cia}. We are going to prove that  is a subalgebra of this -algebra .

Since  is accessible and preserves monomorphisms, the terminal chain
of  converges (and yields a terminal coalgebra ), as proved in \cite{at2}. That is, if we define an chain  on objects by

with  for limit ordinals  (and on morphisms by
 unique,  and
 forming a limit cone), then there exists an ordinal
 such that  is invertible. We then get 

We can choose  arbitrarily large, thus we can assume that  is a cardinal such that  is -accessible.

Since monomorphisms are constructive,  also preserves monomorphisms and is also -accessible. Denote by  its terminal chain. Then this chain converges and yields a terminal coalgebra . Since we again can choose an arbitrary large ordinal for the convergence of , we can assume that this is the above cardinal , thus,  and . We conclude that  is a (canonical) subalgebra of : define a natural transformation  by  and . The constructivity of monomorphisms implies that  is monic, thus, we see by easy transfinite induction that 's are monomorphisms for all . And  is a coalgebra homomorphism because the -th naturality square of ( yields :


We now define a cocone , for , of
the chain from Construction~\ref{free core.chain} by

and

We need to verify compatibility, , from which the limit steps follow automatically. For , use that  is a coalgebra homomorphism:

The isolated  step is easy because if the compatibility holds for , then the diagram

commutes. It is obvious (by transfinite induction) that all 's
are monomorphisms. It remains to verify that
 is an algebra
homomorphism. Indeed, since  preserves -filtered colimits,
the corecursive chain  converges after  steps, so that , by Proposition \ref{cor chain converges U=T+FY}. We have following commutative square:


\vspace*{-20pt}
\end{proof}

\begin{exa}\label{some exams}
Free corecursive algebras  that are obtained as .
\begin{enumerate}
\item For  we have

Indeed, the terminal object  is , and

with colimit .

\item More generally, let  preserve countable coproducts and have a terminal coalgebra . Then 

\item For the endofunctor  of {\bf Set} (of binary algebras) we have the free corecursive algebra

Indeed, the corecursive chain  yields

which we represent by the complete binary tree . Then

is represented by  and all singleton trees labelled in , and

is represented by all binary trees with leaves of depth at most  labelled in .
We conclude that

Consequently, the corecursive chain yields

which is the above set of trees. Observe that this description of 
corresponds well with the fact that  is a free binary algebra with
an additional idempotent (see Example \ref{ex:binary}): the unique
idempotent of  is the complete binary tree. And  is generated
by this tree and all finite trees in .

\item More generally, let  be a signature. Then -algebras are precisely the algebras for the polynomial
endofunctor

Recall that the terminal coalgebra is the coalgebra of all {\it -trees}, that is, trees labelled in  so that every node with a label of arity  has precisely  children. And  is the algebra of all finite -trees, where members of  are considered to have arity . Then  is the set of all -trees with no leaf of depth greater than  having a label from . (That is, all leaves on level  or more are labelled by a nullary symbol in .) Consequently the free corecursive algebra  is


\item For the finite power set functor , J. Worrell
\cite{w} described the terminal coalgebra  as the coalgebra of all finitely branching,
non-ordered, strongly extensional  trees. Recall that a (non-ordered)
tree is called {\it strongly extensional} if for every node, the subtrees rooted at distinct children of the given node are not tree bisimilar. The free corecursive algebra is

Indeed, this is analogous to (3) since

Thus  is the above set of trees. The
corecursive chain converges in  steps  because 
preserves -colimits.
\end{enumerate}
\end{exa}
\begin{exa}
Now we illustrate the need of  for infinite ordinals . The
functor  with
 has the following free corecursive algebras, where
 is the is the complete countably branching tree:

To see this, represent  by the single tree  and obtain

 But  is not an algebra because it does not contain the following tree
 
 This tree is in .

 We can identify, for every ordinal , the set  with the set of all countably branching trees of type  where type  means the tree is , type  means that all maximal subtrees have type , and for limit ordinals , type  means type  for some . Then
 
 It is easy to verify that a tree has countable type if and only if on every infinite path there exists a node whose subtree is . Thus,  is the above free corecursive algebra.
 \end{exa}

For the proof of Theorem~\ref{equivalence 1}, the main result of this section, we are going to use the following technical lemma concerning fixpoints and coproducts of algebras.\smnote{explanation added due to referee comment}

\begin{lem}\label{fixed point lemma}
Let  be a fixpoint and . If the algebras  and  have a coproduct  with injections  and  in , then we have a coproduct  in  with injections

\end{lem}

\begin{proof}
For the algebra  with structure

we observe that  and   are homomorphisms.
Indeed, for the former we have the commutative diagram

whose middle left-hand part commutes because  is a homomorphism. For the latter one consider the diagram

The left-hand triangle commutes because  and because  is a homomorphism, that is, . Therefore, we have a unique homomorphism  with

and

This is the inverse of . Indeed, we prove  by using the universal property of coproducts. Firstly,  is a homomorphism:

Therefore,  is an endomorphism of . And we have both

and

The remaining identity  translates into  and . The former equality follows from  and the fact that  is a homomorphism

the latter one follows from (\ref{eq 4.1}) and (\ref{eq 4.2}) since .
\end{proof}

\begin{rem}\label{remark pre fixed point}\hfill
\begin{enumerate}[label=\xymatrix{
m_{i+1}\equiv HU_i+Y\ar[rr]^-{Hm_i+id}&&HC+Y\ar[r]^-{[c,d]}&C.
}
\xymatrix{
  T
  \ar[r]^-{\tau}
  \ar[ddr]_-{m_0=\overline{\mathsf{inl}}}
  &
  HT
  \ar[rr]^-{\mathsf{inl}}
  \ar[d]_-{H\overline{\mathsf{inl}}}
  &&
  HT+Y
  \ar[ld]^-{H\overline{\mathsf{inl}}+Y}
  \ar `d[dd] [ddll]^-{m_1}
  \\
  &
  HC\ar[r]^-{\mathsf{inl}}\ar[d]_-c&HC+Y\ar[ld]^-{[c,d]}
  \\
  &
  C && }
\xymatrix{
H\overline W_i\ar[r]^-{\mathsf{inr}} &H\overline W_i+Y\ar[r]^-{\overline w_{i+1,i}}& \overline{W_i}
}m_0=id_1\ \ \ \text{ and} \ \ \ m_{i+1}=\mathsf{inl}\cdot Hm_i\xymatrix{ p_0 = m_\lambda &{\rm and}& p_{i+1}\equiv HU_i+Y\ar[rr]^-{Hp_i+id}&&H\overline W_\lambda+Y\ar[r]^-{\overline w_{\lambda+1,\lambda}}&\overline W_\lambda.}p_\gamma:T\oplus FY=U_\gamma\rightarrow \overline W_\lambda\delta_Y:HMY\rightarrow MY\eta_Y:Y\rightarrow MY\mu_Y:(MMY,\delta_{MY})\rightarrow (MY,\delta_Y)\label{mu is homo}\vcenter{
\xymatrix{
HMMY\ar[r]^-{\delta_{MY}}\ar[d]_{H\mu_Y}&MMY\ar[d]^{\mu_Y}&MY\ar[l]_-{\eta_Y}\ar[ld]^{id_{MY}}\\
HMY\ar[r]_-{\delta_Y}&MY
}}
\mathbb{M}=(M,\mu,\eta)
\xymatrix@1{
\mathsf{Alg}_C\,H \ar@<-5pt>[r] 
\ar@{}[r]|-\perp
& 
\mathcal A
\ar@<-5pt>[l]
}.
MY={\mathbb N}\times Y+1MY=\text{binary trees with finitely many leaves, all of which are labelled in ,}MY=\text{finitely branching trees with finitely many leaves labelled in }\mu_Y\cdot\delta_{MY}=\delta_Y\cdot H\mu_Y\delta_Y=\mu_Y\cdot\delta_{MY}\cdot H\eta_{MY}.f,g:(A,a,\dagger)\rightarrow (B,b,\ddagger)k:B\rightarrow C\ \ \ \text{with}\ \ \ k\cdot f=k\cdot gs:C\rightarrow B\ \ \ \text{with}\ \ \ k\cdot s=id_At:B\rightarrow A\ \ \ \text{with}\ \ \ s\cdot k=f\cdot t\ \ \ \text{and}\ \ \ id_B=g\cdot th:(B,b,\ddagger)\rightarrow (D,d,+)h'\cdot e^*=h'\cdot k\cdot e^\ddagger=h\cdot e^\ddagger=e^+.HX=\{M\subseteq X\ |\ \text{card  or }\}Hf(M)=\left \{ \begin{array}{ll}
f[M]& \text{if  is monic when restricted to }\\
\emptyset& \text{otherwise}
\end{array} \right. \card H(2^\alpha)\leq1+\Sigma_{\beta\in \mathbb{B},\beta\leq 2^\alpha} (2^\alpha)^\beta\leq\Sigma_{\beta<\alpha} (2^\alpha)^\beta\leq\alpha\cdot(2^\alpha)^\omega=2^\alpha.\mathbb{S}=(S,\eta,\mu,S',\sigma,\mu')\label{5.1}\vcenter{
\xymatrix@C+1pc{
X\ar[r]^-{e^\dagger}\ar[d]_e&SY\\
S(X+Y)\ar[r]_-{S[e^\dagger,\eta_Y]}&SSY\ar[u]_{\mu_Y}
}}
TY=\text{all binary trees with leaves labelled in .}
 \xymatrix{
 X\ar[r]^e\ar@{-->}[rd]_-{e_0}& SX\\
                       &S'X\ar[u]_{\sigma_X}
 }
 
\xymatrix{
  X 
  \ar[r]^-{e^\dagger}
  \ar[d]_e
  &
  SY
  \\
  SX
  \ar[r]_-{Se^\dagger}
  &
  SSY
  \ar[u]_{\mu_Y}
}
SY=\text{all finitely branching trees with leaves labelled in },RY=\text{all rational, finitely branching trees with leaves labelled in ,}MY=HMY+Y\label{4}\vcenter{
\xymatrix{
Y\ar[r]^{\eta_Y}\ar[rd]^{\mathsf{inr}}\ar[rdd]_{\eta_Y}&MY\ar[d]^{\overline{\mathsf{inr}}}&&HMY\ar[ll]_-{\delta_Y}\ar[d]^{H\overline{\mathsf{inr}}}\\
&HMY+Y\ar[d]^{[\delta_Y,\eta_Y]}&&H(HMY+Y)\ar[d]^{H[\delta_Y,\eta_Y]}\ar[ll]^-{\mathsf{inl}\cdot H[\delta_Y,\eta_Y]}\\
&MY&&HMY\ar[ll]^-{\delta_Y}
}}

\xymatrix{
X\ar[r]^{e^\dagger}\ar[d]_e&A\\
MX\ar[r]_{Me^\dagger}&MA\ar[u]_{\overline a}
}
 \xymatrix@C+2pc{\overline e\equiv MX\ar[r]^-{[\delta_X,\eta_X]^{-1}}&HMX+X\ar[r]^-{[HMX,e_0]}&HMX.}\label{5}\vcenter{
\xymatrix{
MX\ar[r]^{s}\ar@/_/[d]_{[\delta_{X},\eta_X]^{-1}}&A\\
HMX+X\ar@/_/[u]_{[\delta_{X},\eta_X]}\ar[d]_{[HMX,e_0]}\\
HMX\ar[r]_{Hs}&HA\ar[uu]_{a\ \ \ \ \ \ \ }
}}

s\cdot\delta_{X}&=a\cdot Hs\label{6}\\
s\cdot\eta_X&=a\cdot Hs\cdot e_0\label{7}

\xymatrix{
X\ar[d]^{\eta_X}\ar[rr]^{\eta_X}&&MX\ar[d]^{M\eta_X}\\
MX\ar[d]_{s}\ar[rr]^{\eta_{MX}}&&MMX\ar[d]^{Ms}\\
A\ar[rr]_{\eta_A}&&MA
}

\overline a\cdot Ms\cdot M\eta_X=s. &\label{8}
\label{5.2}\vcenter{
\xymatrix@+1pc{
X\ar[r]^{\eta_X}\ar[ddd]_e\ar[rdd]_{e_0}\ar[rd]^{\mathsf{inr}}&MX\ar[d]^-{[\delta_X,\eta_X]^{-1}}\ar[rr]^s
&&A \ar@{<-} `u[l] `[lll]_{e^\dagger} [lll]\\
&HMX+X\ar[d]|(.4){[HMX,e_0]}&HA\ar[ur]^{a}\\
&HMX\ar[ld]^{\delta_{MX}}\ar[ru]_{Hs}\ar[r]_{HMe^\dagger}&HMA\ar[u]^{H\overline a}\ar[dr]_{\delta_{MA}}\\
MX\ar[rrr]_{Me^\dagger}&&&MA\ar[uuu]_{\overline a}
}}

s\cdot \eta_X = \overline a \cdot Me^\dagger \cdot \eta_X = \overline a \cdot \eta_A \cdot e^\dagger = e^\dagger.
{\alph*}]
\item An {\it ideal monad morphism} from an ideal monad
   to an ideal monad
   is a pair consisting of a monad
  morphism  and
  a natural transformation  with
  .

\item Given a functor , a natural transformation  is called {\it ideal}  if it factors through .

\item By a {\it free corecursive monad} on an endofunctor  is meant a corecursive monad  together with an ideal natural transformation  with the following universal property:
For every ideal natural transformation , where  is a corecursive
monad, there exists a unique ideal monad morphism  such that the triangle below commutes:

\end{enumerate}
\end{defi}

\begin{rem}
Let  denote the category of corecursive monads and ideal monad morphisms. We have a forgetful functor  to , the category of all endofunctors of , assigning to every corecursive monad  its ideal . A free corecursive monad on  is precisely a universal arrow from  to the above forgetful functor.
\end{rem}


\begin{exa}\label{exam ideal monad}
If  has free corecursive algebras, then we have the corecursive monad  of Proposition \ref{M is ideal monad}. And the natural transformation  (see Notation \ref{4.13}) is obviously ideal. We prove that  has the universal property:
\end{exa}

\begin{thm}\label{M is free core monad}
If an endofunctor  has free corecursive algebras, then the corresponding monad  is the free corecursive monad on .
\end{thm}

\begin{rem}
The proof is analogous to the corresponding theorem for free completely iterative monads, see \cite[Theorem~4.3]{m_cia}.
\end{rem}
\begin{proof}

For every corecursive monad  and every ideal natural transformation

we are going to find an ideal monad morphism
 with , and prove that it is unique.

\noindent\label{eq_5.4}
\rho_A\equiv HSA\stackrel{\lambda_{SA}}\longrightarrow SSA\stackrel{\mu_A^S}\longrightarrow SA.

\xymatrix{
X\ar[rr]^{e^\dagger}\ar[d]_{e}&&SA\\
HX\ar[r]^{He^\dagger}\ar[d]_{\lambda_X}&HSA\ar[ru]^{\rho_A}\ar[rd]_{\lambda_{SA}}&\\
SX\ar[rr]_{Se^{\dagger}}&&SSA\ar[uu]^{\mu_A^S}
}
b\ There exists a unique homomorphism  of -algebras such that . Now we show that  is a natural transformation. Consider , then  is a homomorphism:

The outside of the following diagram

commutes by the  universal property of .\medskip

\noindent
\xymatrix{
M\ar[rr]^{\hat\lambda}&&S&&MM\ar[r]^{\hat\lambda M}\ar[d]_{\mu}&SM\ar[r]^{S\hat\lambda}&SS\ar[d]^{\mu^S}\\
&Id\ar[ul]^{\eta}\ar[ur]_{\eta^S}& &&M\ar[rr]_{\hat\lambda}&&S
}

\xymatrix{
MMA\ar[dd]_{\mu^M_A}\ar[rr]^{\hat\lambda_{MA}}&&SMA\ar[rr]^{S\hat\lambda_A}&&SSA\ar[dd]^{\mu_A^S}\\
&MA\ar[lu]^{\eta_{MA}}\ar[ru]_{\eta^S_{MA}}\ar[rr]_{\hat\lambda_A}&&SA\ar[ur]_{\eta^S_{SA}}\\
MA\ar@{=}[ur]\ar[rrrr]_{\hat\lambda_A}&&&&SA\ar@{=}[lu]
}
d\ Now we have to show that  which follows from the commutativity of the diagram below:

The left-hand upper part and the central one commute because  is an ideal natural transformation. The right-hand upper part commutes by the definition of . The lower right-hand triangle commutes by the definition of , see (\ref{eq_5.4}) and the lowest part commutes by the monad laws of .\medskip

\noindent
\xymatrix{
MA\ar[rrr]^{\hat\lambda}&&&SA\\
&&SSA\ar[ur]^{\mu^S_A}\\
HMA\ar[uu]_{\delta}\ar[r]_{H\hat\lambda_A}&HSA\ar[r]_{\lambda'_{SA}}\ar[ur]^{\lambda_{SA}}&S'SA\ar[u]^{\sigma_{SA}}\ar[r]_{\mu'^S}&S'A\ar[uu]_{\sigma}
}
f\ It remains to prove that  is
unique. Let  be
an ideal monad morphism with . It is
sufficient to prove that  is a
homomorphism of -algebras w.r.t. the structure  above and
, then  of (b)
above. From that we derive  since  is
a monomorphism:

 The equation  follows from  preserving the units of the monad. And the fact that  is a homomorphism follows from the following diagram:

For the upper triangle see Remark~\ref{4.15} (and recall that ), the right-hand square is the preservation of the monad multiplication, and for the left-hand one we use  and the naturality of :


\vspace*{-20pt}
\end{proof}

\begin{exa}\hfill
\begin{enumerate}
\item The functor  generates the free corecursive monad   see Example \ref{some exams}(1).
This is also the free completely iterative monad, since the functor  has the terminal coalgebra .

\item The polynomial functor  of a signature  generates the free corecursive monad

 See Example \ref{some exams}(4).
\end{enumerate}
\end{exa}

\noindent Are there any other free corecursive monads than the monads  of free corecursive algebras?
Not for endofunctors of {\bf Set}:

\begin{prop}
If a set functor generates a free corecursive monad, then it has free corecursive algebras.
\end{prop}

\begin{proof}

\sloppypar Let  generate a free corecursive monad , and let  be the universal arrow. Following Theorem \ref{equivalence 1} we need to prove the existence of (a) arbitrary large pre-fixpoints and (b) a corecursive fixpoint.

The main technical statement is that the ideal  is naturally
isomorphic to . This proof is analogous to the same proof
concerning free completely iterative monads, see Sections 5 and 6 in
\cite{aamv}. We therefore omit it.

Ad (a). Since  for every set , we see that  is
a pre-fixpoint of cardinality at least .

Ad (b). The isomorphism 
defines a corecursive algebra for . To prove this, consider an
arbitrary equation morphism  and form the equation
morphism . Then solutions of
 w.r.t  (in ) are in bijective
correspondence with solutions  in the algebra . This is easy
to prove, the details are as in the of proof of Theorem 6.1 of
\cite{m_cia}.
\end{proof}

\section{Hyper-Extensive Categories}
Recall that for the more general case of recursion with parameters the equational properties of  are captured by iteration theories of S.~Bloom and Z.~\'{E}sik \cite{be}. Recently \emph{functoriality} was ``added" to these properties; functoriality states that for two equation morphisms with parameters  and  (cf.~Definition~\ref{def ideal monad}) we have

Being an implication, this is not equational if one takes, as in \cite{be}, the category of signatures as the base category. Instead, in \cite{amv_em2} the presheaf category

where  is the category of finite sets and functions, was
suggested as a base category. Equivalently, this is the category of
all finitary endofunctors on .  Then functoriality is an
equational property in the sense of Kelly and Power~\cite{kp93}, and the functorial iteration theories are called {\it Elgot Theories} in \cite{amv_em2}. It follows from the results
in~\cite{be} that all equational properties of  in Domain
Theory are precisely captured by the concept of iteration theory. \smnote{I tried to clarify with the following sentence (also below Theorem).} More precisely, every equation that holds for a parametrized fixpoint operator  given by least fixpoints in a category of domains follows from the axioms of iteration theories (see e.g.~Simpson and Plotkin~\cite{sp00}). 

We have proved in~\cite{amv_em2} that Elgot theories are monadic over sets in context:
\begin{thm}[\cite{amv_em2}]\label{eilenberg-moore alg for M=elgot theo}
 Form the monad  on  by assigning to every set in context  the free iterative theory on  of C.~Elgot \cite{elgot}. Then the Eilenberg-Moore algebras for   are precisely the  Elgot theories.
\end{thm}
This result implies, using the results of Kelly and Power~\cite{kp93}, that Elgot theories are equational over sets in context, and we gave one axiomatization (that includes functoriality) in~\cite{amv_em2}. 


\begin{exa}
  The polynomial set functor  of Example~\ref{some exams}(4) defines a set in context (that we also denote by ) by a domain restriction to . The corresponding iterative theory  is given by
  
  This is a subtheory of the theory  of all trees labelled in , which is the free continuous theory (see Example~\ref{ex 3}(c)).
\end{exa}

It was proved by Bloom and \'Esik in~\cite{be} that the equational properties of the operation  (of solving recursive equations) of the above theory  are precisely the equational properties that  has in an impressive number of applications of iteration. Thus, the axiomatization of these properties in~\cite{be} can be understood as the summary of equational properties that  is expected to have in applications. 

For every finitary set functor  there exists a signature  such that  is a quotient of  (see~\cite{at}). Therefore,  is a quotient theory of the theory  of rational trees. Thus, the equational properties of  in all free Elgot theories  for finitary set functors  are determined by those of  in rational trees.

In the present section we provide the first steps to an analogous result for iteration without parameters. We introduce  \emph{finitary corecursive monads} as the analogy of Elgot's iterative theories, and we prove that every finitary endofunctor on {\bf Set} generates a free finitary corecursive monad. Let  be the monad on  given by

Then we prove that the Eilenberg-Moore algebras for  are precisely the {\it Bloom theories}, i.\,e., theories with an operation  satisfying the equational properties that hold in non-parametric iteration. We list some of these properties. It is an open problem whether our list is complete.

In lieu of {\bf Set} we work, more generally, in a locally finitely presentable category. Thus in lieu of theories we work with finitary monads (in analogy to iterative monads of \cite{elgot}). For most of the results we need to assume the category we work with is hyper-extensive. We now start by recalling this concept from \cite{abmv_how}.

\begin{defi}(See \cite{abmv_how})
A locally finitely presentable category  is called {\it hyper-extensive} if every object is a coproduct of connected objects, i.\,e., objects  such that  preserves coproducts.
\end{defi}

\begin{rem}
In \cite{abmv_how} the definition is different, but Theorem 2.7 of \cite{abmv_how} states that the present formulation is equivalent. Every hyper-extensive category is extensive, i.~e., coproducts are
\begin{enumerate}[label={\alph*},resume]
\item given pairwise disjoint monics , , if each  is coproduct injection then so is .
\end{enumerate}
For locally finitely presentable categories (a)--(c) are equivalent to hyper-extensivity.
\end{rem}

\begin{exa}
Sets, posets, graphs, and every presheaf category are hyper-extensive. Given a signature  the category of -algebras is hyper-extensive iff all arities are 1.
\end{exa}

\begin{rem}
In a hyper-extensive category a monad  is
{\it ideal} (see Definition \ref{def ideal monad}) iff that 
is a coproduct  with injections  and
 and the multiplication has a
restriction 
Thus, in this setting ``ideal'' is a property not an additional structure of a monad. 
\end{rem}

\begin{nota}
 denotes the category of ideal monads and ideal monad morphisms, i.e., morphisms  for which a restriction to the ideals exist: we have 

for a natural transformation .
\end{nota}

\begin{rem}
We shall prove below that every corecursive monad  on a hyper-extensive category has solutions for all, not only ideal, equation morphisms. For that we need to specify an element of  which then serves for defining solutions of non-ideal equations such as . In the following definition  denotes the terminal object of  and  the initial one. The unique morphism from  to  is denoted by . Analogously .
\end{rem}

\begin{defi}
A {\it strict endofunctor} is an endofunctor  together with a morphism . A monad is {\it strict} if its underlying endofunctor is. A natural transformation  between strict functors is called strict if  preserves .

Every strict endofunctor has a special global element in every : it is the composite of  and . We denote it again by .
\end{defi}

\begin{rem}
We now recall from \cite{abmv_how} the concept of a strict solution and the fact that every equation morphism has a unique strict solution. In \cite{abmv_how} equation morphisms with parameters  (see Definition \ref{def ideal monad}) were considered, here we restrict ourselves to .
\end{rem}


\begin{defi}{\rm (See \cite{abmv_how})}\label{def_derived subobjects}
(a) For every equation morphism  we denote by 
the intersection of the \emph{derived subobjects}  obtained by forming recursively pullbacks as follows:

\end{defi}

\begin{rem}
 represents those variables for which solutions of  have ``difficulties" assigning a value. For example, if  represents the iterative equation  or the system 
then . We resolve the difficulties by assigning the value  to such variables:
\end{rem}

\begin{defi}(See \cite{abmv_how})
Let  be an equation morphism. A solution  is called {\it strict} if its restriction to  factorizes through :


\end{defi}

\begin{thm}{\rm(}See \cite{abmv_how}{\rm)}\label{the_ equations have strict solutions}
Let  be a strict, corecursive monad on a hyper-extensive category. Then every equation morphism  has a unique strict solution , for every object .

\end{thm}

In \cite{abmv_how} we proved this for completely iterative monads, the proof for the corecursive monads is the same.

\section{Finitary Bloom algebras and monads}

In this section we investigate the variant of iteration in which only equation morphisms  with finitely presentable objects  (of variables) are considered.


Throughout this section we assume that the base category is locally finitely presentable and hyper-extensive. And a finitary endofunctor  is given.

\begin{defi}\hfill
\begin{enumerate}[label=\xymatrix{
X\ar[rd]_{e^\dagger}\ar[rr]^h&&X'\ar[ld]^{(e')^\dagger}\\
&A
}{\alph*}]
\item  Every finitary corecursive algebra is a finitary Bloom algebra: the functoriality follows from the uniqueness of solutions.

\item Lemmas~\ref{homos are solution prese} and~\ref{A is B then B is B} hold also for finitary
  Bloom algebras. 
\end{enumerate}
\end{rem}

\begin{exa}
Consider unary algebras in , that is, . 
\begin{enumerate}[label={\alph*}]
\item Recall from \cite{amv_atwork} the concept of an iterative algebra: it is an algebra  such that every equation morphism  with  finitely presentable has a unique solution. That is, the algebra  for  is finitary corecursive. Every iterative algebra is obviously finitary corecursive (for ). An example of a finitary corecursive algebra that is not iterative is the algebra of all binary trees with finitely many leaves, all of which are labelled in , see Example \ref{some exams}(3).

\item Recall further from \cite{amv_atwork} that the
  category  of all coalgebras on finitely
  presentable objects of  is filtered, and the filtered
  colimit of the forgetful functor to ,  carries the structure
  of a coalgebra . This structure is an isomorphism, and
  its inverse  is the initial iterative
  algebra. Consequently,  is a finitary Bloom algebra as
  well. Indeed:
\end{enumerate}
\end{rem}

\begin{prop}\label{R is i.f.c.a}
 is the initial finitary corecursive algebra.
\end{prop}

\begin{proof}
We know that  is finitary corecursive because it is even iterative. Let  be a finitary Bloom algebra. Given a (solution-preserving) homomorphism , for every  in  the triangle

commutes, where  denotes the solution in . As proved in \cite{amv_atwork}, these morphisms  form the colimit cocone of  (as a colimit of the forgetful functor of ). Thus, the above triangles determine, since  is functorial, a unique morphism  which is solution-preserving. It remains to prove that  is a homomorphism. For that recall from \cite{amv_atwork} that the algebra structure  is defined as the inverse of the unique isomorphism  with 
Thus in order to prove  we use that  which follows from the fact that  are collectively epic: .
\end{proof}

\begin{exa}\hfill
\begin{enumerate}[label=H_\Sigma X=\Sigma_0+\Sigma_1\times X+\Sigma_2\times X^2+\cdotsR_\Sigma=\text{all rational -trees}.\xymatrix{
R\ar[r]^-{\mathsf{inl}}\ar[rd]_-f&R\oplus FY\ar[d]_-{[f,\overline{g}]}&FY\ar[l]_-{\mathsf{inr}}\ar[ld]_-{\overline{g}}&Y\ar[l]_-{\eta_Y}\ar[lld]^-g\\
&B
}{\alph*}]
\item Let  be a full subcategory of  representing all finitely presentable objects. The functor category  is equivalent to the category of all finitary endofunctors of .

\item  denotes the non-full subcategory of all strict finitary endofunctors (and strict natural transformations). The embedding  has a left adjoint  where .

\item We denote by  the monad on  given by free finitary corecursive monads: 
More precisely  is the monad obtained by the composite adjoint situation

where  is the forgetful functor of the category
 of all strict finitary corecursive
monads.
\end{enumerate}
\end{nota}

\begin{defi}
A \emph{Bloom monad} on  is an Eilenberg-Moore algebra for the monad .
\end{defi}

\begin{rem}
Following Theorem \ref{eilenberg-moore alg for M=elgot theo}, this concept is, for , completely analogous to iteration theories of Bloom and \'Esik: whereas iteration theories formalizes equational properties of parametrized iteration, Bloom monads on  formalize equational properties of non-parametrized iteration. But what are Bloom monads?

\begin{enumerate}
\item Every Bloom monad is a finitary monad on . Indeed, let
 be the free-monad on : to every
finitary endofunctor  it assigns the free monad  on . It is
well-known that the Eilenberg-Moore algebras for  are
precisely the finitary monads on ,
see~\cite{lack}.

For every  in  we have the unique monad morphism  given by the universal property of . These morphisms form components of a monad morphism  (over ). Thus, every Eilenberg-Moore algebra for  is automatically one for , too.

\item Every Bloom monad  comes equipped with an operation  assigning to every equation morphism  with  finitely presentable and every object  a morphism  which is a solution:

Indeed, the Eilenberg-Moore structure

is a monad morphism, and we have also the universal arrow (see Example \ref{exam ideal monad})

which, due to the unit law of , fulfils

For every equation morphism  the unique strict solution (see Theorem \ref{the_ equations have strict solutions}) of  w.r.t.  is denoted by . Then

is a (canonical) solution of  w.r.t. . Indeed, the following diagram, where  denotes the multiplication of , commutes: 

The left-hand part commutes since  is a solution of . The right-hand part (with ) commutes because  is a monad morphism. And to prove the lower part, we just need to verify

which follows easily from  and
the naturality of .
\end{enumerate}
\end{rem}

\begin{rem}\label{re equational prop}
(a) The operation  above satisfies all the equational laws that the formation of strict solutions in all finitary corecursive monads satisfies. This follows from the fact that the Bloom monad  is by definition a quotient algebra (for ) of the finitary corecursive monad .

(b) For the base category  we can say more. Since  is
finitary, there exists a finitary signature  such that  is
a quotient of  (see~\cite{at}). Let  be the extension by a nullary operation . Then  is a quotient of . Indeed, there exists a signature  such that for suitable natural transformations  we have a coequalizer

see \cite{amm12}. The functor  of free  finitary corecursive monads is a left adjoint, thus, it preserves coequalizers. Therefore  is a quotient of .

Consequently, a Bloom monad in  is a finitary monad with a solution operation  satisfying all the equational laws that the operation of unique strict solutions for the rational-tree monads  satisfies.
\end{rem}

In the following example we list some equational properties of 
in Bloom monads. It is an open problem whether this list is complete
in the sense that every equational property of  holding in all
Bloom monads can be derived from the properties stated.

\begin{exa}
Equational properties of  in Bloom monads. We use the terminology of the monograph \cite{be}.
\begin{enumerate}[label=e^\dag=\mu_Y\cdot Se^\dag\cdot e\xymatrix{
X\ar[r]^e\ar[d]_-h&SX\ar[d]^-{Sh}\\
\overline{X}\ar[r]_-{\overline{e}}&S\overline X
}e^\dag={\overline e}^\dagger\cdot h.\mu_Y\cdot S(\overline{e}^\dag\cdot h)\cdot e=\mu_Y\cdot S\overline e^\dag\cdot \overline e\cdot h=\overline e^\dag\cdot h
\xymatrix{
  \cdots\  
  \ar@{>->}[r]^{i_3} 
  &
  X_2
  \ar@{>->}[r]^{i_2} 
  \ar[d]^{h_2}
  &
  X_1
  \ar@{>->}[r]^{i_1} 
  \ar[d]^{h_1}
  &
  X
  \ar[d]^{h = h_0}
  \\
  \cdots\ 
  \ar@{>->}[r]_{\overline i_3} 
  &
  \overline X_2
  \ar@{>->}[r]_{\overline i_2}
  &
  \overline X_1
  \ar@{>->}[r]_{\overline i_1}
  &
  \overline X 
}

\xymatrix{
  X_1
  \ar[ddd]_{e_1}
  \ar@{>->}[rrr]^-{i_1}
  \ar[rd]^{h_1}
  &&&
  X
  \ar[ddd]^{e}
  \ar[ld]_h
  \\
  &
  \overline X_1
  \ar@{>->}[r]^-{\overline i_1}
  \ar[d]_{\overline e_1}
  &
  \overline X
  \ar[d]^{\overline e}
  \\
  &
  \overline X
  \ar@{>->}[r]_-{\eta_X}
  &
  S\overline X
  \\
  X
  \ar@{>->}[rrr]_-{\eta_X}
  \ar[ru]^h
  &&&
  SX
  \ar[lu]_{Sh}
}
(\overline e^\dag\cdot h)\cdot i_\infty=\overline e^\dag \cdot \overline i_\infty\cdot h_\infty=\bot\cdot !\cdot i_\infty=\bot\cdot !\hat h=\mu_Z\cdot Sh:SY\to SZ\xymatrix{
X\ar[r]^-{e_Z^\dag}\ar[d]_{e^\dag_Y}&SZ\\
SY\ar[ru]_-{\hat h}
}\hat h\cdot e^\dag_Y\cdot i_\infty=\hat h\cdot \bot\cdot !=\bot\cdot !\xymatrix{
X\ar[r]^{e_Y^\dag}\ar[dd]_-e&SY\ar[r]^-{\hat h}&SZ\\
&SSY\ar[u]_-{\mu_Y}\ar[rd]^-{S\hat h}\\
SX\ar[ru]^-{Se_Y^\dag}\ar[rr]_-{S(\hat h\cdot e^\dag_Y)}&&SSZ\ar[uu]_-{\mu_Z}
}e^\dag=(\hat e\cdot e)^\dag\xymatrix{
X\ar[rr]^-{e^\dag}\ar[d]_-e&&SY\\
SX\ar[r]^{Se^\dag}\ar[d]_-{Se}&SSY\ar[ru]^-{\mu_Y}\\
SSX\ar[r]^{SSe^\dag}\ar[d]_-{\mu_X}&SSY\ar[u]_-{S\mu_Y}\ar[rd]^-{\mu_{SY}}\\
SX\ar[rr]_-{Se^\dag}&&SSY\ar[uuu]_-{\mu_Y}
}\xymatrix{
X_2\ar[r]^-{e_2}\ar[d]_-{i_2}&X_1\ar[r]^-{e_1}\ar[d]_-{i_1}&X\ar@{=}[r]\ar[d]_-{\eta_X}&X=X_0\ar[dd]^-{\eta_X}\\
X_1\ar[r]^-{e_1}\ar[d]_-{i_1}&X\ar[d]_-{\eta_X}\ar[r]^-e&SX\ar[d]_{\eta_{SX}}\\
X\ar[r]_-e&SX\ar[r]_-{Se}&SSX\ar[r]_-{\mu_X}&SX
}\hat g\cdot f:X\to SX\qquad\text{and}\qquad\hat f\cdot g:Z\to SZ.(\hat g\cdot f)^\dag=\widehat{(\hat f\cdot g)^\dag}\cdot f:X\to SY\xymatrix{
X\ar[r]^-f\ar[d]_-f&SZ\ar[r]^{S(\hat f\cdot g )^\dag}\ar@{=}[dl]\ar[d]^-{Sg}&SSY\ar[r]^\mu&SY\\
SZ\ar[d]_-{Sg}\ar@{=}[ur]&SSX\ar[d]^-{SSf}\\
SSX\ar[d]_-{\mu}\ar[r]_-{SSf}\ar@{=}[ur]&SSSZ\ar[d]^-{S\mu}\\
SX\ar[r]_-{Sf}&SSZ\ar[r]_-{SS(\hat f\cdot g )^\dag}&SSSY\ar[r]_-{S\mu}\ar[uuu]_-{S\mu}&SSY\ar[uuu]_-\mu
}\xymatrix{
\ar@{}[d]|{\objectstyle\cdots}&X_3\ar@{>->}[r]^-{i_3}\ar[d]_-{f_3}&X_2\ar@{>->}[r]^-{i_2}\ar[d]_-{f_2}&X_1\ar@{>->}[r]^-{i_1}\ar[d]_-{f_1}&X=X_0\ar[d]_-{f}\\
Z_3\ar@{>->}[r]_-{j_3}&Z_2\ar@{>->}[r]_-{j_2}&Z_1\ar@{>->}[r]_-{j_1}&Z\ar[r]_-{\eta_Z}&SZ
}

\xymatrix{
\ar@{}[d]|{\objectstyle\cdots}&Z_3\ar@{>->}[r]^-{j_3}\ar[d]_-{g_3}&Z_2\ar@{>->}[r]^-{j_2}\ar[d]_-{g_2}&Z_1\ar@{>->}[r]^-{j_1}\ar[d]_-{g_1}&Z=Z_0\ar[d]_-{g}\\
X_3\ar@{>->}[r]_-{i_3}&X_2\ar@{>->}[r]_-{i_2}&X_1\ar@{>->}[r]_-{i_1}&X\ar[r]_-{\eta_X}&SX
}

\xymatrix{
&&X_4\ar[d]_-{f_4}\ar@{>->}[r]^-{i_4}&X_3\ar@{>->}[r]^-{i_3}\ar[d]_-{f_3}&X_2\ar@{>->}[r]^-{i_2}\ar[d]_-{f_2}&
X_1\ar@{>->}[r]^-{i_1}\ar[d]_-{f_1}&X\ar[d]^-f\\
&\cdots&Z_3\ar@{>->}[r]_-{j_3}\ar[d]_-{g_3}&Z_2\ar@{>->}[r]_-{j_2}\ar[d]_-{g_2}&Z_1\ar@{>->}[r]_-{j_1}\ar[d]_-{g_1}&Z\ar[d]_-{g}
\ar[r]_-{\eta_Z}&SZ\ar[d]^-{Sg}\\
X_4\ar@{>->}[r]_-{i_4}\ar@{=}[d]&X_3\ar@{=}[d]\ar@{>->}[r]_-{i_3}&X_2\ar@{=}[d]\ar@{>->}[r]_-{i_2}&X_1\ar@{=}[d]\ar@{>->}[r]_-{i_1}&X\ar@{=}[d]
\ar[r]_-{\eta_X}&SX\ar[r]_-{S\eta_X}&SSX\ar[d]^-{\mu_X}
\\
X_4\ar@{>->}[r]_-{i_4}&X_3\ar@{>->}[r]_-{i_3}&X_2\ar@{>->}[r]_-{i_2}&X_1\ar@{>->}[r]_-{i_1}&X\ar[rr]_-{\eta_X}&&SX\ar@{<-}`r[u]`[uuu]_-{\hat g\cdot f}[uuu]
}
j_1\cdot j_2\cdot \cdots\cdot j_{2n}:Z_{2n}\rightarrowtail Z.\xymatrix{
X_\infty=X_{n+1}\ar[r]^-{f_n}\ar[dd]_-{i_\infty}&Z_\infty=Z_{n}\ar[r]^-{!}\ar[d]_-{j_\infty}&1\ar[d]_-\bot\ar[rdd]^-\bot
\ar@{<-}`u[l]`[ll]_-{!}[ll]
\\
&Z\ar[r]_-{(\hat f \cdot g)^\dag}\ar[d]_-{\eta_Z}&SZ\ar[d]_-{\eta_{SZ}}\ar@{=}[rd]\\
X\ar[r]_-f&SZ\ar[r]_-{S(\hat f \cdot g)^\dag}&SSZ\ar[r]_-{\mu_Z}&SZ
}
This completes the proof.
\end{enumerate}
\end{exa}


\section{Conclusions and Open problems}

For coalgebras, recursivity caN be defined by the existence of unique algebra-to-coalgebra homomorphisms (no parameters are used). Or, equivalently, assuming the given endofunctor preserves weak pullbacks, by the unique solutions of all recursive systems with parameters. In contrast, in the dual situation we need to study non-equivalent variations. The present paper is dedicated to corecursive algebras , where corecursivity means that  every recursive system of equations represented by a coalgebra has a unique solution in . The formulation above is strictly weaker than the concept of a completely iterative algebra, where every parametrized recursive system of equations has a unique solution. For example, if we consider the endofunctor  of one binary operation in {\bf Set}, the algebra of all binary trees with finitely many leaves is corecursive, but not completely iterative.

The main result of our paper is the description of the free
corecursive algebra on  as the coproduct  of the
terminal coalgebra  and the free algebra  in the category of all
algebras. The above example of binary trees is the free corecursive
algebra  on one generator. Our description is true for all
accessible ( bounded)  endofunctors on {\bf Set} and, more generally, for
all endofunctors on {\bf Set} having free corecursive algebras. For
accessible monos-preserving endofunctors on more general base categories (posets, groups,
monoids etc.) the above description of the free corecursive algebras
also holds.

We introduce the concept of a corecursive monad, a weakening of
completely iterative monad. We prove that the assignment  is the free corecursive monad on the given accessible
endofunctor. And we characterize the Eilenberg-Moore algebras for this
monad. We call them Bloom algebras in honor of Stephen Bloom. They
play the analogous role that Elgot algebras, studied in \cite{amv3},
play for iterative monads: solutions of recursive equations are not
required to be unique, but have to satisfy some ``basic''
properties. In the case of Bloom algebras, the only property needed is
functoriality.

We further treat finitary equations: If we consider systems of
recursive equations as coalgebras , then finite
systems of recursive equation are represented by coalgebras in which
 is a finite set (or more generally, a finitely presentable
object). We can speak about finitary corecursive algebras as those in
which these finite systems have unique solutions. We prove that if 
is the initial iterative algebra, then  is a free
finitary corecursive algebra.

Another question is: what is the analogy of the notion of an iteration
monad of S.~Bloom and Z.~\'{E}sik \cite{be} in the realm of
corecursive algebras? We do not know the answer. But at least we can
formulate the question precisely. The idea of iteration monads is to
collect all ``equational" properties that the operation  of solving recursive systems  has in trees for a
signature. This can be understood as forming the monad of free
iterative theories (or monads) on the category
 of sets in context, and characterizing
monadic algebras: these are, as proved in \cite{amv_em2},
precisely the iteration theories of S. Bloom and Z. \'{E}sik that are
functorial. So the open problem we state is this: form the monad of
free finitary corecursive theories on ,
what are its monadic algebras? We called them Bloom monads, and listed
some of their properties.

\bibliography{ourpapers}
\bibliographystyle{plain}
\vspace{-40 pt}
\end{document}
