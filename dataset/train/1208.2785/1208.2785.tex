\noindent In this section, we show bounds on .

 Let  be a set of  points in . Assume that all points in 
have distinct co-ordinates, i.e. if  and  are two points in , then  for all , . This assumption can be easily removed by slightly perturbing the input point set such that the co-ordinates are distinct. It can be seen that an -net for the
perturbed set acts as an -net for the original set also(see section 4
in \cite{AAH09}). 

\subsection{Strong centerpoints in }
\noindent Let  be the family of axis-parallel boxes in .
\begin{thm}\label{ddim}
 
\end{thm}
\begin{figure}
\begin{center}
\scalebox{0.60}{\begin{picture}(0,0)\includegraphics{r1.pstex}\end{picture}\setlength{\unitlength}{4144sp}\begingroup\makeatletter\ifx\SetFigFont\undefined \gdef\SetFigFont#1#2#3#4#5{\reset@font\fontsize{#1}{#2pt}\fontfamily{#3}\fontseries{#4}\fontshape{#5}\selectfont}\fi\endgroup \begin{picture}(7632,4441)(346,-3815)
\put(1666,254){\makebox(0,0)[lb]{\smash{{\SetFigFont{12}{14.4}{\rmdefault}{\mddefault}{\updefault}}}}}
\put(2746,254){\makebox(0,0)[lb]{\smash{{\SetFigFont{12}{14.4}{\rmdefault}{\mddefault}{\updefault}}}}}
\put(361,-556){\makebox(0,0)[lb]{\smash{{\SetFigFont{12}{14.4}{\rmdefault}{\mddefault}{\updefault}}}}}
\put(361,-1546){\makebox(0,0)[lb]{\smash{{\SetFigFont{12}{14.4}{\rmdefault}{\mddefault}{\updefault}}}}}
\put(1261,-1141){\makebox(0,0)[lb]{\smash{{\SetFigFont{12}{14.4}{\rmdefault}{\mddefault}{\updefault}}}}}
\put(2071,-1141){\makebox(0,0)[lb]{\smash{{\SetFigFont{12}{14.4}{\rmdefault}{\mddefault}{\updefault}}}}}
\put(1261,-1951){\makebox(0,0)[lb]{\smash{{\SetFigFont{12}{14.4}{\rmdefault}{\mddefault}{\updefault}}}}}
\put(2116,-1951){\makebox(0,0)[lb]{\smash{{\SetFigFont{12}{14.4}{\rmdefault}{\mddefault}{\updefault}}}}}
\put(2881,-1951){\makebox(0,0)[lb]{\smash{{\SetFigFont{12}{14.4}{\rmdefault}{\mddefault}{\updefault}}}}}
\put(2161,-241){\makebox(0,0)[lb]{\smash{{\SetFigFont{12}{14.4}{\rmdefault}{\mddefault}{\updefault}}}}}
\put(2971,-241){\makebox(0,0)[lb]{\smash{{\SetFigFont{12}{14.4}{\rmdefault}{\mddefault}{\updefault}}}}}
\put(2881,-1141){\makebox(0,0)[lb]{\smash{{\SetFigFont{12}{14.4}{\rmdefault}{\mddefault}{\updefault}}}}}
\put(1306,-241){\makebox(0,0)[lb]{\smash{{\SetFigFont{12}{14.4}{\rmdefault}{\mddefault}{\updefault}}}}}
\end{picture} }
\caption{Bounds for  in }

\label{rect1}
\end{center}
\end{figure}
\begin{proof}
Let  and , , be two axis-parallel hyperplanes orthogonal to the  dimension that divide  into three slabs. Let  be the subset of  contained in the positive hyperspace defined by    and  be the subset of  contained in the negative hyperspace defined by   .  and  are placed such that . The hyperplanes  and , , partition  into 
axis-parallel -dimensional boxes. Indexing the partition along each
dimension, these boxes are denoted as ,
where  (see figure~\ref{rect1}(a) for the upper bound construction in ). Let . We claim that . 

Let . Since none of the points in   is counted in any
 or , . This implies that
. 

Let  be any point in . We claim that
 is a -net. Any -dimensional
box that does not contain  has to avoid some  or  containing
 points. Hence it contains at most  points.

For the lower bound, place  subsets of  points such that each axis has two subsets at unit distance on either side of the origin. The lower bound construction for  in  is shown in figure~\ref{rect1}(b). Let  be an -net. Without loss of generality, assume that  is chosen from the subset placed at coordinates .
Now the -dimensional axis-parallel box defined by  avoids
 but contains all the
remaining  subsets thereby containing  points.



\end{proof}



\noindent

\subsection{Upper Bounds on }
\noindent Let  be a set of  points and  be the family of axis-parallel rectangles in . We prove upper bounds for .
\begin{lem}\label{halfhalf}
There exists a point  with coordinates
 such that the
halfspaces  and  contain at least  points of .
\end{lem}

\begin{proof} Divide P into two horizontal and two vertical slabs such that each slab contains  points (see figure~\ref{fig1}).

\begin{itemize}
 \item Case 1: . Any point  has the desired
property.
\item Case 2: . In this case,  and . Then the point  with the
smallest perpendicular distance to the horizontal line has the desired property.
\end{itemize}


\begin{figure}
\begin{center}
\includegraphics[scale=0.35]{sub1.eps}
\caption{Bisection of  by vertical and horizontal lines}
\label{fig1}
\end{center}
\end{figure}
\end{proof}
\vspace{0.2in}
\begin{lem}\label{lemma2RU}
	

\end{lem}

\begin{figure}
\begin{center}
\begin{picture}(0,0)\includegraphics{r2ub.pstex}\end{picture}\setlength{\unitlength}{4144sp}\begingroup\makeatletter\ifx\SetFigFont\undefined \gdef\SetFigFont#1#2#3#4#5{\reset@font\fontsize{#1}{#2pt}\fontfamily{#3}\fontseries{#4}\fontshape{#5}\selectfont}\fi\endgroup \begin{picture}(5157,2551)(166,-2555)
\put(2071,-151){\makebox(0,0)[lb]{\smash{{\SetFigFont{8}{9.6}{\rmdefault}{\mddefault}{\updefault}}}}}
\put(1351,-2491){\makebox(0,0)[lb]{\smash{{\SetFigFont{8}{9.6}{\rmdefault}{\mddefault}{\updefault}(a)}}}}
\put(1441,-151){\makebox(0,0)[lb]{\smash{{\SetFigFont{8}{9.6}{\rmdefault}{\mddefault}{\updefault}}}}}
\put(181,-961){\makebox(0,0)[lb]{\smash{{\SetFigFont{8}{9.6}{\rmdefault}{\mddefault}{\updefault}}}}}
\put(181,-511){\makebox(0,0)[lb]{\smash{{\SetFigFont{8}{9.6}{\rmdefault}{\mddefault}{\updefault}}}}}
\put(181,-1636){\makebox(0,0)[lb]{\smash{{\SetFigFont{8}{9.6}{\rmdefault}{\mddefault}{\updefault}}}}}
\put(4141,-2491){\makebox(0,0)[lb]{\smash{{\SetFigFont{8}{9.6}{\rmdefault}{\mddefault}{\updefault}(b)}}}}
\put(631,-151){\makebox(0,0)[lb]{\smash{{\SetFigFont{8}{9.6}{\rmdefault}{\mddefault}{\updefault}}}}}
\end{picture} \caption{Upper bound for }
\label{fig2RU}
\end{center}
\end{figure}

\begin{proof}
 Divide  into three horizontal slabs containing , 
and  points respectively. Similarly, divide  into three
vertical slabs in the same proportion to get a grid with nine rectangular
regions (figure~\ref{fig2RU}(a)).
\begin{itemize}
 \item Case 1: : Let  be any point in E. Now 
is a -net since any axis-parallel rectangle that avoids  will
avoid an extreme slab which contains  points.
\item Case 2: : Let  be the regions  respectively and let  denote  for all , . Since , we have
. Therefore either  or
. Without loss of generality
assume that
. Using lemma~\ref{halfhalf}, choose a point  such that the halfspaces  and  contain at least half of the points in . Similarly, choose a point  such that the halfspaces
 and  contain at least half of the points in . We claim that  is a -net. Any axis-parallel rectangle that does not take points from all the three rows and columns contains at most  points of . So assume that  is an axis-parallel rectangle that takes points from all the rows and columns. To avoid ,  must avoid at least half of
the points from
 as well as 
(figure~\ref{fig2RU}(b)). So it must avoid at least
 points. Therefore any axis-parallel rectangle
that avoids  contains
at most  points.
\end{itemize}


\end{proof}

\vspace{0.5in}
\noindent Now we discuss two general recursive constructions for
. 
\begin{figure}
\begin{center}
\scalebox{0.5}{\begin{picture}(0,0)\includegraphics{gen3.pstex}\end{picture}\setlength{\unitlength}{4144sp}\begingroup\makeatletter\ifx\SetFigFont\undefined \gdef\SetFigFont#1#2#3#4#5{\reset@font\fontsize{#1}{#2pt}\fontfamily{#3}\fontseries{#4}\fontshape{#5}\selectfont}\fi\endgroup \begin{picture}(9657,5443)(301,-5750)
\put(7606,-5551){\makebox(0,0)[lb]{\smash{{\SetFigFont{12}{14.4}{\rmdefault}{\mddefault}{\updefault}(b)}}}}
\put(3421,-466){\makebox(0,0)[lb]{\smash{{\SetFigFont{14}{16.8}{\rmdefault}{\mddefault}{\updefault}}}}}
\put(3511,-1861){\makebox(0,0)[lb]{\smash{{\SetFigFont{14}{16.8}{\rmdefault}{\mddefault}{\updefault}}}}}
\put(1171,-2086){\makebox(0,0)[lb]{\smash{{\SetFigFont{14}{16.8}{\rmdefault}{\mddefault}{\updefault}}}}}
\put(1171,-466){\makebox(0,0)[lb]{\smash{{\SetFigFont{14}{16.8}{\rmdefault}{\mddefault}{\updefault}}}}}
\put(316,-2446){\makebox(0,0)[lb]{\smash{{\SetFigFont{14}{16.8}{\rmdefault}{\mddefault}{\itdefault}h}}}}
\put(2881,-2311){\makebox(0,0)[lb]{\smash{{\SetFigFont{14}{16.8}{\rmdefault}{\mddefault}{\updefault}q}}}}
\put(2791,-5101){\makebox(0,0)[lb]{\smash{{\SetFigFont{14}{16.8}{\rmdefault}{\mddefault}{\itdefault}v}}}}
\put(2611,-5686){\makebox(0,0)[lb]{\smash{{\SetFigFont{12}{14.4}{\rmdefault}{\mddefault}{\updefault}(a)}}}}
\end{picture} }
\caption{General constructions for axis-parallel rectangles}
\label{genfig}
\end{center}
\end{figure}
\begin{thm}\label{onept}

for  and 
\end{thm}
\begin{proof}
Construct an -net  for  as described in 
Theorem~\ref{ddim}. Let  and  be vertical and horizontal lines through  that divide  into two vertical and two horizontal slabs respectively. Let  be a parameter that will be fixed later. If either of the two vertical slabs contains at least  points of
 then construct an -net for the points in the slab containing at least  points
and -net for the points in the other slab. If both the vertical 
slabs contain less than  points, then construct an
-net for the points in each of the two vertical slabs. Repeat the same
construction for the horizontal slabs also. We have thus added at most
 points to our -net .


Since , any
axis-parallel rectangle that avoids  is contained in one of the 
(vertical or horizontal) slabs and this slab has at most  points.

First consider the case where there is a vertical or horizontal slab with at least  points.
 After adding an -net to
,  any axis-parallel rectangle that avoids  contains at most
 points of  from this slab. Similarly,
the other slab has at most  points of  and any
axis-parallel rectangle that avoids  contains at most
 points of  from this slab. Thus an
axis-parallel rectangle that avoids  and is contained in  one of these slabs
has at most  points(see figure~\ref{genfig} (a)). In the
case where both the slabs have less than  points, any
axis-parallel rectangle that avoids  has at most  points. Thus any axis-parallel rectangle that avoids  has at most  points. Setting  so that , we get .
\end{proof}
\begin{thm}\label{grid}
 for  and .
\end{thm}
\begin{proof}
Divide  into  horizontal slabs and  vertical slabs to get a grid with each horizontal slab containing  points and each vertical slab containing
 points. Let the horizontal slabs be denoted as . For a horizontal slab , there are  vertical lines of the grid intersecting it. For each of these lines, find two points(if present) of  in  that has the
least perpendicular distance from that line on either side. Repeat this for all horizontal slabs  (see figure~\ref{genfig}~(b)). Add these (at most)
 points  to the -net .

 We claim that any
axis-parallel rectangle that avoids these points has at most  points. If an axis-parallel rectangle intersects at most one vertical slab or at most two horizontal slabs, then it contains at most  points. Let
 be an axis-parallel rectangle that intersects at least two vertical slabs and at least three horizontal slabs. Let  be the horizontal slabs intersected by . 
 also intersects at least one 
vertical line and avoids the nearest points to these vertical lines in all , . Therefore  cannot take  points from any such . Thus  can only take points from  and  and hence contains at most  points of .  
\vspace{0.1in}


Now add an -net for points in each horizontal slab and an
-net for points in each vertical slab and add these -net points
to . Now  and the result follows.

\end{proof} 

\begin{lem}
; ;
; 
\end{lem}
\begin{proof}
The results follow from the fact that  and Theorem~\ref{onept} with  for
;  for
;  for ;  for
.

\end{proof}
\begin{lem}
; ;

\end{lem}
\begin{proof}
The results follow from the fact that  and Theorem~\ref{grid} with  for
; with  for ;
 for .

\end{proof}
\subsection{Lower Bounds on }
In this subsection, we call an axis-parallel rectangle  an -big rectangle if . Let  be an -net and . We call   as free if .
\begin{lem}\label{lemma2RL}
 
\end{lem}
\begin{figure}
\begin{center}
\includegraphics[scale=0.5]{r2lb_new.eps}
\caption{Lower bound for }
\label{fig2RL}
\end{center}
\end{figure}
\begin{proof}
 Divide the  points into three equal subsets of 
points each and place each subset uniformly
on the bold segments as shown in figure~\ref{fig2RL}. Let  be an
-net of size two. The different cases of choosing
 from this set are shown in figure~\ref{fig2RL}. Let  denote the fraction of
points from a subset which lie on one side of the point in  as shown in the
corresponding figure for each case. Let  be a function that represents the
maximum fraction of points of  
that an axis-parallel rectangle contains without
containing any of the points in . In each case, we compute the values of 
 and  that minimize  by considering all the rectangles that avoids the -net points.

\begin{enumerate}[(a)]
 \item .  is minimized when
, which results in .
 \item .  is minimized when
, which results in .
 \item .  is minimized when
, which results in .
 \item .  is minimized when
, which results in .
\end{enumerate}
So there always exists an axis-parallel rectangle that avoids all the points in the -net
and contains at least  points.

\end{proof}

\begin{figure}
\begin{center}
\scalebox{0.25}{\begin{picture}(0,0)\includegraphics{r3lb_new.pstex}\end{picture}\setlength{\unitlength}{4144sp}\begingroup\makeatletter\ifx\SetFigFont\undefined \gdef\SetFigFont#1#2#3#4#5{\reset@font\fontsize{#1}{#2pt}\fontfamily{#3}\fontseries{#4}\fontshape{#5}\selectfont}\fi\endgroup \begin{picture}(17746,6365)(256,-7325)
\put(2071,-1411){\makebox(0,0)[lb]{\smash{{\SetFigFont{20}{24.0}{\rmdefault}{\mddefault}{\updefault}}}}}
\put(5581,-3211){\makebox(0,0)[lb]{\smash{{\SetFigFont{20}{24.0}{\rmdefault}{\mddefault}{\updefault}}}}}
\put(3781,-1411){\makebox(0,0)[lb]{\smash{{\SetFigFont{20}{24.0}{\rmdefault}{\mddefault}{\updefault}}}}}
\put(4771,-1861){\makebox(0,0)[lb]{\smash{{\SetFigFont{20}{24.0}{\rmdefault}{\mddefault}{\updefault}A}}}}
\put(5311,-5191){\makebox(0,0)[lb]{\smash{{\SetFigFont{20}{24.0}{\rmdefault}{\mddefault}{\updefault}B}}}}
\put(631,-5191){\makebox(0,0)[lb]{\smash{{\SetFigFont{20}{24.0}{\rmdefault}{\mddefault}{\updefault}C}}}}
\put(991,-1951){\makebox(0,0)[lb]{\smash{{\SetFigFont{20}{24.0}{\rmdefault}{\mddefault}{\updefault}D}}}}
\put(5581,-4201){\makebox(0,0)[lb]{\smash{{\SetFigFont{20}{24.0}{\rmdefault}{\mddefault}{\updefault}}}}}
\put(3781,-6001){\makebox(0,0)[lb]{\smash{{\SetFigFont{20}{24.0}{\rmdefault}{\mddefault}{\updefault}}}}}
\put(2071,-6001){\makebox(0,0)[lb]{\smash{{\SetFigFont{20}{24.0}{\rmdefault}{\mddefault}{\updefault}}}}}
\put(361,-4201){\makebox(0,0)[lb]{\smash{{\SetFigFont{20}{24.0}{\rmdefault}{\mddefault}{\updefault}}}}}
\put(271,-3301){\makebox(0,0)[lb]{\smash{{\SetFigFont{20}{24.0}{\rmdefault}{\mddefault}{\updefault}}}}}
\put(2881,-7261){\makebox(0,0)[lb]{\smash{{\SetFigFont{20}{24.0}{\rmdefault}{\mddefault}{\updefault}(a)}}}}
\put(9361,-7261){\makebox(0,0)[lb]{\smash{{\SetFigFont{20}{24.0}{\rmdefault}{\mddefault}{\updefault}(b)}}}}
\put(15121,-7261){\makebox(0,0)[lb]{\smash{{\SetFigFont{20}{24.0}{\rmdefault}{\mddefault}{\updefault}(c)}}}}
\put(8011,-1861){\makebox(0,0)[lb]{\smash{{\SetFigFont{20}{24.0}{\rmdefault}{\mddefault}{\updefault}}}}}
\put(11431,-4651){\makebox(0,0)[lb]{\smash{{\SetFigFont{20}{24.0}{\rmdefault}{\mddefault}{\updefault}}}}}
\put(13411,-2401){\makebox(0,0)[lb]{\smash{{\SetFigFont{20}{24.0}{\rmdefault}{\mddefault}{\updefault}}}}}
\put(17731,-4201){\makebox(0,0)[lb]{\smash{{\SetFigFont{20}{24.0}{\rmdefault}{\mddefault}{\updefault}}}}}
\put(15931,-6001){\makebox(0,0)[lb]{\smash{{\SetFigFont{20}{24.0}{\rmdefault}{\mddefault}{\updefault}}}}}
\put(8551,-6001){\makebox(0,0)[lb]{\smash{{\SetFigFont{20}{24.0}{\rmdefault}{\mddefault}{\updefault}}}}}
\end{picture} }
\caption{Lower bound for }
\label{fig3RL}
\end{center}
\end{figure}
\vspace{0.5in}
\begin{lem}\label{lemma3RL}
  
\end{lem}

\begin{proof}
Let  be arranged into 20 equal subsets. Each quadrant has a group of five subsets, we will denote these groups as 
 as shown in figure~\ref{fig3RL}.



Clearly, all the subsets of one of the groups will be free. Assume that all the subsets
in  are free. The three points in  can be chosen in two ways.
\begin{itemize}
 \item \emph{Case 1: One point is chosen from each of the other three groups:}
Any set of eight consecutive subsets is contained in a -big axis-parallel
rectangle. Since subsets in  are free, all the three subsets near  in  and 
cannot be free. This means a point has to be chosen from one of the subsets
 or . Now if a
point is chosen from subset  then the remaining subsets in  are free and there exists
a -big axis-parallel rectangle containing all those subsets and  from .
Therefore  has to be free. Symmetrically,  also has to be free. Clearly two points have to be selected from the subsets  and 
to avoid the -big axis-parallel
rectangle with consecutive subsets from  and  ( and ). If the subset  is free then there is a -big axis-parallel rectangle containing the free subsets 
.
 Therefore the three points have to be chosen from the subsets
. Now there exists a -big axis-parallel rectangle as shown in figure~\ref{fig3RL}(b) that avoids these subsets.
\item \emph{Case 2: Two points are chosen from a group and the third point is chosen
from the diagonally opposite group:}
Assume that two points are chosen from 
and one point from . Clearly, a point has to be chosen from the subset  to avoid the -big
axis-parallel rectangles with consecutive subsets from  and  ( and ).
 The other subsets in  are free. The next two points
chosen have to be from the subsets  and  to avoid the -big axis-parallel
rectangles containing consecutive subsets from  () and  (). Now
there exists a -big axis-parallel rectangle as shown in figure~\ref{fig3RL}(c)
which avoids the subsets .
\end{itemize}

\end{proof}

\begin{lem}\label{lemma4RL}
 
\end{lem}

\begin{proof}
Let  be arranged as ten equal subsets as shown in figure~\ref{fig4RL}. The eight
subsets in between the topmost and bottommost subsets are arranged as two
quadrilaterals. Let  be an -net of size four. We
claim that there always exists a -big
axis-parallel rectangle that avoids . 

It is easy to see that there can be at most two free subsets in each of the
quadrilaterals since there is a -big axis-parallel rectangle that contains
any three free subsets from a quadrilateral. Hence all the four points of  have to be chosen from the  subsets forming the two quadrilaterals(two points from each quadrilateral). Therefore the top and bottom
subsets are free. Both these subsets form a -big axis-parallel
rectangle with the black subsets in the quadrilateral(see figure~\ref{fig4RL}). Therefore two of the points in  have to be chosen from the black subsets. The
two gray subsets in the lower(resp. upper) half form a -big axis-parallel
rectangle with the bottommost(resp. topmost) subset. Hence the remaining two points in  have to be chosen from the gray subsets(one point from the top gray subsets and one from the bottom gray subsets). Thus one gray subset in the
lower half is free which forms a -big axis-parallel rectangle that avoids  as shown
in figure~\ref{fig4RL}.

\end{proof}
\begin{figure}
\begin{minipage}[b]{160pt}


\begin{center}
\includegraphics[scale=0.5]{r4lb.eps}
\caption{Lower bound for }
\label{fig4RL}
\end{center}

\end{minipage}
\begin{minipage}[b]{200pt}
 \begin{center}
 \includegraphics[scale=0.5]{r5lb.eps}
\caption{Lower bound for }
\label{fig5lb}
\end{center}
\end{minipage}

\end{figure}

\begin{lem}
  
\end{lem}
\begin{proof}
Let  be
arranged as eight equal subsets, four in the outer layer and four in the
inner layer (see figure~\ref{fig5lb}). Let  be an -net of size five. We claim that there always exists a -big axis-parallel rectangle that avoids .  

It is clear that we have to choose a point each from at least three of the inner
layer subsets, otherwise there can be an axis-parallel rectangle that contains
all the points of two free subsets. Also, a point each has to be chosen from
at least two of the outer subsets to avoid the axis-parallel rectangles
that contains all the points of two consecutive subsets in the outer layer. 
Moreover, these two points are to be chosen from diagonally opposite subsets. Let the -net points be chosen from the five black subsets as shown in figure~\ref{fig5lb}. Now there exists a -big axis-parallel rectangle containing the free subset in the
inner layer and the nearest free subset in the outer layer.


\end{proof}

\begin{lem}
; ; ;
; 
\end{lem}

\begin{proof}
 The results follow from theorem~\ref{genlower} with  for
;
 for ;  for
;  for ;
 for . 
\end{proof}

\begin{lem}
 , for 
\end{lem}
\begin{proof}
We use mathematical induction to prove this.
The lemma holds for  and  since  (by lemma~\ref{lemma2RL}) and
 (by lemma~\ref{lemma3RL}). Now assume the lemma holds for ,
i.e.,  or .
Thus, , 
i.e., .
Therefore,  (from theorem~\ref{genlower}).
Hence the result follows for all values of .


\end{proof}

\begin{table}[t]\label{rectable}
\begin{centering}

\begin{tabular}{|c|c|c|c|c|c|c|c|c|c|c|}
\hline
&&&&&&&&&&\\
\hline
LB&3/4&5/9&2/5&3/10&1/4&1/5&5/29&2/13&3/22&1/8\\
\hline
UB&3/4&5/8&9/16&1/2&15/32&15/32&3/7&2/5&5/13&3/8\\
\hline
\end{tabular}
\caption{Summary of lower and upper bounds for .
}
\end{centering}
\end{table} 

\noindent The summary of lower and upper bounds for  is given in table 1.
