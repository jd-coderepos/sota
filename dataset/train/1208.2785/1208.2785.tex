\noindent In this section, we show bounds on $\epsilon_i^{\mathcal{R}}$.

 Let $P$ be a set of $n$ points in $\mathbb{R}^d$. Assume that all points in $P$
have distinct co-ordinates, i.e. if $p=(p_1,p_2,...,p_d)$ and $q=(q_1,q_2,....,q_d)$ are two points in $P$, then $p_i \neq q_i$ for all $i$, $1\leq i \leq d$. This assumption can be easily removed by slightly perturbing the input point set such that the co-ordinates are distinct. It can be seen that an $\epsilon$-net for the
perturbed set acts as an $\epsilon$-net for the original set also(see section 4
in \cite{AAH09}). 

\subsection{Strong centerpoints in $\mathbb{R}^d$}
\noindent Let $\mathcal{R}$ be the family of axis-parallel boxes in $\mathbb{R}^d$.
\begin{thm}\label{ddim}
 $\epsilon_1^{\mathcal{R}} = \frac{2d-1}{2d}$
\end{thm}
\begin{figure}
\begin{center}
\scalebox{0.60}{\begin{picture}(0,0)\includegraphics{r1.pstex}\end{picture}\setlength{\unitlength}{4144sp}\begingroup\makeatletter\ifx\SetFigFont\undefined \gdef\SetFigFont#1#2#3#4#5{\reset@font\fontsize{#1}{#2pt}\fontfamily{#3}\fontseries{#4}\fontshape{#5}\selectfont}\fi\endgroup \begin{picture}(7632,4441)(346,-3815)
\put(1666,254){\makebox(0,0)[lb]{\smash{{\SetFigFont{12}{14.4}{\rmdefault}{\mddefault}{\updefault}$\mathcal{H}_1^-$}}}}
\put(2746,254){\makebox(0,0)[lb]{\smash{{\SetFigFont{12}{14.4}{\rmdefault}{\mddefault}{\updefault}$\mathcal{H}_1^+$}}}}
\put(361,-556){\makebox(0,0)[lb]{\smash{{\SetFigFont{12}{14.4}{\rmdefault}{\mddefault}{\updefault}$\mathcal{H}_2^+$}}}}
\put(361,-1546){\makebox(0,0)[lb]{\smash{{\SetFigFont{12}{14.4}{\rmdefault}{\mddefault}{\updefault}$\mathcal{H}_2^-$}}}}
\put(1261,-1141){\makebox(0,0)[lb]{\smash{{\SetFigFont{12}{14.4}{\rmdefault}{\mddefault}{\updefault}$R_{21}$}}}}
\put(2071,-1141){\makebox(0,0)[lb]{\smash{{\SetFigFont{12}{14.4}{\rmdefault}{\mddefault}{\updefault}$R_{22}$}}}}
\put(1261,-1951){\makebox(0,0)[lb]{\smash{{\SetFigFont{12}{14.4}{\rmdefault}{\mddefault}{\updefault}$R_{11}$}}}}
\put(2116,-1951){\makebox(0,0)[lb]{\smash{{\SetFigFont{12}{14.4}{\rmdefault}{\mddefault}{\updefault}$R_{12}$}}}}
\put(2881,-1951){\makebox(0,0)[lb]{\smash{{\SetFigFont{12}{14.4}{\rmdefault}{\mddefault}{\updefault}$R_{13}$}}}}
\put(2161,-241){\makebox(0,0)[lb]{\smash{{\SetFigFont{12}{14.4}{\rmdefault}{\mddefault}{\updefault}$R_{32}$}}}}
\put(2971,-241){\makebox(0,0)[lb]{\smash{{\SetFigFont{12}{14.4}{\rmdefault}{\mddefault}{\updefault}$R_{33}$}}}}
\put(2881,-1141){\makebox(0,0)[lb]{\smash{{\SetFigFont{12}{14.4}{\rmdefault}{\mddefault}{\updefault}$R_{23}$}}}}
\put(1306,-241){\makebox(0,0)[lb]{\smash{{\SetFigFont{12}{14.4}{\rmdefault}{\mddefault}{\updefault}$R_{31}$}}}}
\end{picture} }
\caption{Bounds for $\epsilon_1^\mathcal{R}$ in $\mathbb{R}^2$}

\label{rect1}
\end{center}
\end{figure}
\begin{proof}
Let $\mathcal{H}_i^+$ and $\mathcal{H}_i^-$, $1\leq i \leq d$, be two axis-parallel hyperplanes orthogonal to the $i^{th}$ dimension that divide $P$ into three slabs. Let $\mathcal{P}_i^+ $ be the subset of $P$ contained in the positive hyperspace defined by   $\mathcal{H}_i^+$ and $\mathcal{P}_i^- $ be the subset of $P$ contained in the negative hyperspace defined by   $\mathcal{H}_i^-$. $\mathcal{H}_i^+$ and $\mathcal{H}_i^-$ are placed such that $|\mathcal{P}_i^+| = |\mathcal{P}_i^-| = \frac{n}{2d}-1$. The hyperplanes $\mathcal{H}_i^+$ and $\mathcal{H}_i^-$, $1\leq i \leq d$, partition $\mathbb{R}^d$ into $3^d$
axis-parallel $d$-dimensional boxes. Indexing the partition along each
dimension, these boxes are denoted as $R_{x_1 x_2 ... x_d}$,
where $x_i \in \{1,2,3\}$ (see figure~\ref{rect1}(a) for the upper bound construction in $\mathbb{R}^2$). Let $P_{x_1 x_2 ... x_d} =R_{x_1 x_2 ... x_d} \cap P$. We claim that $P_{22...2} \ne \emptyset$. 

Let $\mathcal{K} = \sum\limits_{i=1}^d (|\mathcal{P}_i^-| + |\mathcal{P}_i^+|) =
n-2d$. Since none of the points in  $P_{22...2}$ is counted in any
$\mathcal{P}_i^+$ or $\mathcal{P}_i^-$, $\mathcal{K} \ge n - |P_{22...2}|$. This implies that
$|P_{22...2}| \ge 2d$. 

Let $p$ be any point in $P_{22...2}$. We claim that
$\{p\}$ is a $\frac{2d-1}{2d}$-net. Any $d$-dimensional
box that does not contain $p$ has to avoid some $\mathcal{P}_i(\mathcal{P}_i^+$ or $\mathcal{P}_i^-)$ containing
$\frac{n}{2d}-1$ points. Hence it contains at most $\frac{2d-1}{2d} n$ points.

For the lower bound, place $2d $ subsets of $\frac{n}{2d}$ points such that each axis has two subsets at unit distance on either side of the origin. The lower bound construction for $\epsilon_1^\mathcal{R}$ in $\mathbb{R}^2$ is shown in figure~\ref{rect1}(b). Let $\{q\}$ be an $\epsilon$-net. Without loss of generality, assume that $q$ is chosen from the subset placed at coordinates $(1,0,0,...0)$.
Now the $d$-dimensional axis-parallel box defined by $x \le 0.5$ avoids
$q$ but contains all the
remaining $2d-1$ subsets thereby containing $\frac{2d-1}{2d}n$ points.



\end{proof}



\noindent

\subsection{Upper Bounds on $\epsilon_i^{\mathcal{R}}$}
\noindent Let $P$ be a set of $n$ points and $\mathcal{R}$ be the family of axis-parallel rectangles in $\mathbb{R}^2$. We prove upper bounds for $\epsilon_i^{\mathcal{R}}$.
\begin{lem}\label{halfhalf}
There exists a point $p\in P$ with coordinates
$(x\textquotesingle,y\textquotesingle)$ such that the
halfspaces $x \geq x\textquotesingle$ and $y \geq y\textquotesingle $ contain at least $\frac{n}{2}$ points of $P$.
\end{lem}

\begin{proof} Divide P into two horizontal and two vertical slabs such that each slab contains $\frac{n}{2}$ points (see figure~\ref{fig1}).

\begin{itemize}
 \item Case 1: $ D\cap P \neq \emptyset$. Any point $p\in D\cap P$ has the desired
property.
\item Case 2: $ D\cap P  = \emptyset$. In this case, $|A\cap P|=|C\cap
P|=\frac{n}{2}$ and $B\cap P=\emptyset$. Then the point $p \in A\cap P$ with the
smallest perpendicular distance to the horizontal line has the desired property.
\end{itemize}


\begin{figure}
\begin{center}
\includegraphics[scale=0.35]{sub1.eps}
\caption{Bisection of $P$ by vertical and horizontal lines}
\label{fig1}
\end{center}
\end{figure}
\end{proof}
\vspace{0.2in}
\begin{lem}\label{lemma2RU}
$\epsilon_{2}^{\mathcal{R}} \leq {5\over8}$	

\end{lem}

\begin{figure}
\begin{center}
\begin{picture}(0,0)\includegraphics{r2ub.pstex}\end{picture}\setlength{\unitlength}{4144sp}\begingroup\makeatletter\ifx\SetFigFont\undefined \gdef\SetFigFont#1#2#3#4#5{\reset@font\fontsize{#1}{#2pt}\fontfamily{#3}\fontseries{#4}\fontshape{#5}\selectfont}\fi\endgroup \begin{picture}(5157,2551)(166,-2555)
\put(2071,-151){\makebox(0,0)[lb]{\smash{{\SetFigFont{8}{9.6}{\rmdefault}{\mddefault}{\updefault}$\frac{3n}{8}$}}}}
\put(1351,-2491){\makebox(0,0)[lb]{\smash{{\SetFigFont{8}{9.6}{\rmdefault}{\mddefault}{\updefault}(a)}}}}
\put(1441,-151){\makebox(0,0)[lb]{\smash{{\SetFigFont{8}{9.6}{\rmdefault}{\mddefault}{\updefault}$\frac{n}{4}$}}}}
\put(181,-961){\makebox(0,0)[lb]{\smash{{\SetFigFont{8}{9.6}{\rmdefault}{\mddefault}{\updefault}$\frac{n}{4}$}}}}
\put(181,-511){\makebox(0,0)[lb]{\smash{{\SetFigFont{8}{9.6}{\rmdefault}{\mddefault}{\updefault}$\frac{3n}{8}$}}}}
\put(181,-1636){\makebox(0,0)[lb]{\smash{{\SetFigFont{8}{9.6}{\rmdefault}{\mddefault}{\updefault}$\frac{3n}{8}$}}}}
\put(4141,-2491){\makebox(0,0)[lb]{\smash{{\SetFigFont{8}{9.6}{\rmdefault}{\mddefault}{\updefault}(b)}}}}
\put(631,-151){\makebox(0,0)[lb]{\smash{{\SetFigFont{8}{9.6}{\rmdefault}{\mddefault}{\updefault}$\frac{3n}{8}$}}}}
\end{picture} \caption{Upper bound for $\epsilon^{\cal{R}}_2$}
\label{fig2RU}
\end{center}
\end{figure}

\begin{proof}
 Divide $P$ into three horizontal slabs containing $\frac{3n}{8}$, $\frac{n}{4}$
and $\frac{3n}{8}$ points respectively. Similarly, divide $P$ into three
vertical slabs in the same proportion to get a grid with nine rectangular
regions (figure~\ref{fig2RU}(a)).
\begin{itemize}
 \item Case 1: $E \cap P \ne \emptyset$: Let $x$ be any point in E. Now $\{x\}$
is a $\frac{5}{8}$-net since any axis-parallel rectangle that avoids ${x}$ will
avoid an extreme slab which contains $\frac{3}{8}n$ points.
\item Case 2: $E \cap P = \emptyset$: Let $R_1, R_2, R_3, R_4$ be the regions $A \cup B \cup D \cup E,B\cup C \cup E\cup F, D\cup E\cup G\cup H, E\cup F\cup H\cup I$ respectively and let $P_i$ denote $R_i \cap P$ for all $i$, $1\leq i \leq 4$. Since $  E \cap P = \emptyset$, we have
$\vert P_1 \vert + \vert P_2 \vert +\vert P_3 \vert + \vert P_4 \vert = n + | (B \cup D \cup F \cup H) \cap P | =
\frac{3n}{2}$. Therefore either $(\vert P_1 \vert + \vert P_4 \vert) \geq \frac{3n}{4}$ or
$(\vert P_2 \vert + \vert P_3 \vert) \geq \frac{3n}{4}$. Without loss of generality
assume that
$(\vert P_2 \vert + \vert P_3 \vert) \geq \frac{3n}{4}$. Using lemma~\ref{halfhalf}, choose a point $p=(p_x,p_y) \in P_2$ such that the halfspaces $x \ge p_x$ and $y \ge p_y$ contain at least half of the points in $P_2$. Similarly, choose a point $q=(q_x,q_y) \in P_3$ such that the halfspaces
$x \leq q_x$ and $y \leq q_y$ contain at least half of the points in $P_3$. We claim that $\{p,q\}$ is a $\frac{5}{8}$-net. Any axis-parallel rectangle that does not take points from all the three rows and columns contains at most $\frac{5n}{8} $ points of $P$. So assume that $R$ is an axis-parallel rectangle that takes points from all the rows and columns. To avoid $\{p,q\}$, $R$ must avoid at least half of
the points from
$P_2$ as well as $P_3$
(figure~\ref{fig2RU}(b)). So it must avoid at least
$\frac{\vert P_2 \vert + \vert P_3 \vert}{2}=\frac{3n}{8}$ points. Therefore any axis-parallel rectangle
that avoids $\{p,q\}$ contains
at most $\frac{5n}{8}$ points.
\end{itemize}


\end{proof}

\vspace{0.5in}
\noindent Now we discuss two general recursive constructions for
$\epsilon_i^\mathcal{R}$. 
\begin{figure}
\begin{center}
\scalebox{0.5}{\begin{picture}(0,0)\includegraphics{gen3.pstex}\end{picture}\setlength{\unitlength}{4144sp}\begingroup\makeatletter\ifx\SetFigFont\undefined \gdef\SetFigFont#1#2#3#4#5{\reset@font\fontsize{#1}{#2pt}\fontfamily{#3}\fontseries{#4}\fontshape{#5}\selectfont}\fi\endgroup \begin{picture}(9657,5443)(301,-5750)
\put(7606,-5551){\makebox(0,0)[lb]{\smash{{\SetFigFont{12}{14.4}{\rmdefault}{\mddefault}{\updefault}(b)}}}}
\put(3421,-466){\makebox(0,0)[lb]{\smash{{\SetFigFont{14}{16.8}{\rmdefault}{\mddefault}{\updefault}$>\delta n$}}}}
\put(3511,-1861){\makebox(0,0)[lb]{\smash{{\SetFigFont{14}{16.8}{\rmdefault}{\mddefault}{\updefault}$\frac{3}{4}\epsilon_x$}}}}
\put(1171,-2086){\makebox(0,0)[lb]{\smash{{\SetFigFont{14}{16.8}{\rmdefault}{\mddefault}{\updefault}$(1-\delta)\epsilon_y$}}}}
\put(1171,-466){\makebox(0,0)[lb]{\smash{{\SetFigFont{14}{16.8}{\rmdefault}{\mddefault}{\updefault}$\leq(1-\delta)n$}}}}
\put(316,-2446){\makebox(0,0)[lb]{\smash{{\SetFigFont{14}{16.8}{\rmdefault}{\mddefault}{\itdefault}h}}}}
\put(2881,-2311){\makebox(0,0)[lb]{\smash{{\SetFigFont{14}{16.8}{\rmdefault}{\mddefault}{\updefault}q}}}}
\put(2791,-5101){\makebox(0,0)[lb]{\smash{{\SetFigFont{14}{16.8}{\rmdefault}{\mddefault}{\itdefault}v}}}}
\put(2611,-5686){\makebox(0,0)[lb]{\smash{{\SetFigFont{12}{14.4}{\rmdefault}{\mddefault}{\updefault}(a)}}}}
\end{picture} }
\caption{General constructions for axis-parallel rectangles}
\label{genfig}
\end{center}
\end{figure}
\begin{thm}\label{onept}
$\epsilon_{2(x+y)+1}^\mathcal{R} \leq
\max({{3\over{4}}\epsilon_x^\mathcal{R}},{{\epsilon_y^\mathcal{R}
\epsilon_z^\mathcal{R}}\over{\epsilon_y^\mathcal{R}+\epsilon_z^\mathcal{R}}})$
for $x,y \geq 0, x\geq y$ and $z=\lfloor{x+y\over2}\rfloor $
\end{thm}
\begin{proof}
Construct an $\epsilon_1^\mathcal{R}$-net $\{q\}$ for $P$ as described in 
Theorem~\ref{ddim}. Let $v$ and $h$ be vertical and horizontal lines through $q$ that divide $P$ into two vertical and two horizontal slabs respectively. Let $\delta \in [0,1]$ be a parameter that will be fixed later. If either of the two vertical slabs contains at least $\delta n$ points of
$P$ then construct an $\epsilon_x^\mathcal{R}$-net for the points in the slab containing at least $\delta n$ points
and $\epsilon_y^\mathcal{R} $-net for the points in the other slab. If both the vertical 
slabs contain less than $\delta n$ points, then construct an
$\epsilon_z^\mathcal{R}$-net for the points in each of the two vertical slabs. Repeat the same
construction for the horizontal slabs also. We have thus added at most
$2(x+y)+1$ points to our $\epsilon$-net $Q$.


Since $q\in Q$, any
axis-parallel rectangle that avoids $Q$ is contained in one of the 
(vertical or horizontal) slabs and this slab has at most ${3n\over4}$ points.

First consider the case where there is a vertical or horizontal slab with at least $\delta n$ points.
 After adding an $\epsilon_x^\mathcal{R}$-net to
$Q$,  any axis-parallel rectangle that avoids $Q$ contains at most
${3\over4}\epsilon_x^\mathcal{R}n$ points of $P$ from this slab. Similarly,
the other slab has at most $(1-\delta) n$ points of $P$ and any
axis-parallel rectangle that avoids $Q$ contains at most
$(1-\delta)\epsilon_y^\mathcal{R}n$ points of $P$ from this slab. Thus an
axis-parallel rectangle that avoids $Q$ and is contained in  one of these slabs
has at most $\max({3\over4}\epsilon_x^\mathcal{R}n,
(1-\delta)\epsilon_y^\mathcal{R}n )$ points(see figure~\ref{genfig} (a)). In the
case where both the slabs have less than $\delta n$ points, any
axis-parallel rectangle that avoids $Q$ has at most $\delta
\epsilon_z^\mathcal{R}n$ points. Thus any axis-parallel rectangle that avoids $Q$ has at most $max({3\over4}\epsilon_x^\mathcal{R}n, (1-\delta)\epsilon_y^\mathcal{R}n,
\delta \epsilon_z^\mathcal{R}n)$ points. Setting $\delta=
{{\epsilon_y^\mathcal{R}}\over{\epsilon_y^\mathcal{R}
+\epsilon_z^\mathcal{R}}}$ so that $(1-\delta)\epsilon_y^\mathcal{R}=\delta \epsilon_z^\mathcal{R}$, we get $\epsilon_{2(x+y)+1}^\mathcal{R} \leq
\max({{3\over{4}}\epsilon_x^\mathcal{R}},{{\epsilon_y^\mathcal{R}
\epsilon_z^\mathcal{R}}\over{\epsilon_y^\mathcal{R}+\epsilon_z^\mathcal{R}}})$.
\end{proof}
\begin{thm}\label{grid}
$\epsilon_{2(x-2)(y-1)+jx+ky}^\mathcal{R} \leq
\max({{{2\epsilon_j^{\mathcal{R}}} \over{x}},{{\epsilon_{k}^{\mathcal{R}}} \over
{y}}})$ for $x,y \geq 2$ and $j,k\geq 0$.
\end{thm}
\begin{proof}
Divide $P$ into $x$ horizontal slabs and $y$ vertical slabs to get a grid with each horizontal slab containing $n\over x$ points and each vertical slab containing
$n\over y$ points. Let the horizontal slabs be denoted as $H_1,H_2,\dots,H_x$. For a horizontal slab $H_i$, there are $y-1$ vertical lines of the grid intersecting it. For each of these lines, find two points(if present) of $P$ in $H_i$ that has the
least perpendicular distance from that line on either side. Repeat this for all horizontal slabs $H_i, 2\leq i \leq x-1$ (see figure~\ref{genfig}~(b)). Add these (at most)
$2(x-2)(y-1)$ points  to the $\epsilon^\mathcal{R}$-net $Q$.

 We claim that any
axis-parallel rectangle that avoids these points has at most $max({2n\over
x},{n\over y})$ points. If an axis-parallel rectangle intersects at most one vertical slab or at most two horizontal slabs, then it contains at most $max({2n\over
x},{n\over y})$ points. Let
$R$ be an axis-parallel rectangle that intersects at least two vertical slabs and at least three horizontal slabs. Let $H_i,H_{i+1},\dots,H_m$ be the horizontal slabs intersected by $R$. 
$R$ also intersects at least one 
vertical line and avoids the nearest points to these vertical lines in all $H_l$, $i+1 \leq l \leq m-1$. Therefore $R$ cannot take  points from any such $H_l$. Thus $R$ can only take points from $H_i$ and $H_m$ and hence contains at most ${2n\over
x}$ points of $P$.  
\vspace{0.1in}


Now add an $\epsilon_j^\mathcal{R}$-net for points in each horizontal slab and an
$\epsilon_k^\mathcal{R}$-net for points in each vertical slab and add these $\epsilon$-net points
to $Q$. Now $|Q|\leq 2(x-2)(y-1)+jx+ky$ and the result follows.

\end{proof} 

\begin{lem}
$\epsilon_3^\mathcal{R}\leq {9\over16}$; $\epsilon_5^\mathcal{R}\leq {15\over32}$;
$\epsilon_7^\mathcal{R}\leq{3\over 7}$; $\epsilon_9^\mathcal{R}\leq{5\over13}$
\end{lem}
\begin{proof}
The results follow from the fact that $\epsilon_0^\mathcal{R}=1$ and Theorem~\ref{onept} with $x=1, y=0$ for
$\epsilon_3^\mathcal{R}$; $x=2, y=0$ for
$\epsilon_5^\mathcal{R}$; $x=3, y=0$ for $\epsilon_7^\mathcal{R}$; $x=4,y=0$ for
$\epsilon_9^\mathcal{R}$.

\end{proof}
\begin{lem}
$\epsilon_4^\mathcal{R}\leq{1\over2}$; $\epsilon_8^\mathcal{R}\leq {2\over5}$;
$\epsilon_{10}^\mathcal{R}\leq{3\over8}$
\end{lem}
\begin{proof}
The results follow from the fact that $\epsilon_0^\mathcal{R}=1$ and Theorem~\ref{grid} with $x=4, y=2,j=k=0$ for
$\epsilon_4^\mathcal{R}$; with $x=5,y=2,j=0,k=1$ for $\epsilon_8^\mathcal{R}$;
$x=4,y=2,j=1,k=1$ for $\epsilon_{10}^\mathcal{R}$.

\end{proof}
\subsection{Lower Bounds on $\epsilon_i^{\mathcal{R}}$}
In this subsection, we call an axis-parallel rectangle $R$ an $\alpha$-big rectangle if $\vert P \cap R \vert \geq \alpha \vert P \vert$. Let $Q$ be an $\epsilon$-net and $P_1 \subset P$. We call $P_1$  as free if $P_1 \cap Q = \emptyset$.
\begin{lem}\label{lemma2RL}
 $\epsilon^{\cal{R}}_2 \geq \frac{5}{9}$
\end{lem}
\begin{figure}
\begin{center}
\includegraphics[scale=0.5]{r2lb_new.eps}
\caption{Lower bound for $\epsilon^{\cal{R}}_2$}
\label{fig2RL}
\end{center}
\end{figure}
\begin{proof}
 Divide the $n$ points into three equal subsets of $\frac{n}{3}$
points each and place each subset uniformly
on the bold segments as shown in figure~\ref{fig2RL}. Let $Q$ be an
$\epsilon$-net of size two. The different cases of choosing
$Q$ from this set are shown in figure~\ref{fig2RL}. Let $x, y$ denote the fraction of
points from a subset which lie on one side of the point in $Q$ as shown in the
corresponding figure for each case. Let $f$ be a function that represents the
maximum fraction of points of $P$ 
that an axis-parallel rectangle contains without
containing any of the points in $Q$. In each case, we compute the values of 
$x$ and $ y $ that minimize $f$ by considering all the rectangles that avoids the $\epsilon$-net points.

\begin{enumerate}[(a)]
 \item $f=max(\frac{2x}{3}+\frac{1}{3}, 1-\frac{2(x+y)}{3},
\frac{2(1-x)}{3}+\frac{y}{3})$. $f$ is minimized when
$x=y=\frac{1}{3}$, which results in $f=\frac{5}{9}$.
 \item $f=max(\frac{2y}{3}+\frac{1}{3}, 1-\frac{2x+y}{3},
\frac{2x}{3}+\frac{1-y}{3})$. $f$ is minimized when
$x=\frac{1}{2}, y=\frac{1}{3}$, which results in $f=\frac{5}{9}$.
 \item $f=max(1-\frac{x+y}{3}, \frac{2x}{3}+\frac{1-y}{3},
\frac{2y}{3}+\frac{1-x}{3},\frac{2(x+y)-1}{3})$. $f$ is minimized when
$x=y=\frac{2}{3}$, which results in $f=\frac{5}{9}$.
 \item $f=max(\frac{1}{3}+\frac{2x}{3}, \frac{1}{3}+\frac{2y}{3},
1-\frac{2(x+y)}{3})$. $f$ is minimized when
$x=y=\frac{1}{3}$, which results in $f=\frac{5}{9}$.
\end{enumerate}
So there always exists an axis-parallel rectangle that avoids all the points in the $\epsilon$-net
and contains at least $\frac{5n}{9}$ points.

\end{proof}

\begin{figure}
\begin{center}
\scalebox{0.25}{\begin{picture}(0,0)\includegraphics{r3lb_new.pstex}\end{picture}\setlength{\unitlength}{4144sp}\begingroup\makeatletter\ifx\SetFigFont\undefined \gdef\SetFigFont#1#2#3#4#5{\reset@font\fontsize{#1}{#2pt}\fontfamily{#3}\fontseries{#4}\fontshape{#5}\selectfont}\fi\endgroup \begin{picture}(17746,6365)(256,-7325)
\put(2071,-1411){\makebox(0,0)[lb]{\smash{{\SetFigFont{20}{24.0}{\rmdefault}{\mddefault}{\updefault}$d_5$}}}}
\put(5581,-3211){\makebox(0,0)[lb]{\smash{{\SetFigFont{20}{24.0}{\rmdefault}{\mddefault}{\updefault}$a_5$}}}}
\put(3781,-1411){\makebox(0,0)[lb]{\smash{{\SetFigFont{20}{24.0}{\rmdefault}{\mddefault}{\updefault}$a_1$}}}}
\put(4771,-1861){\makebox(0,0)[lb]{\smash{{\SetFigFont{20}{24.0}{\rmdefault}{\mddefault}{\updefault}A}}}}
\put(5311,-5191){\makebox(0,0)[lb]{\smash{{\SetFigFont{20}{24.0}{\rmdefault}{\mddefault}{\updefault}B}}}}
\put(631,-5191){\makebox(0,0)[lb]{\smash{{\SetFigFont{20}{24.0}{\rmdefault}{\mddefault}{\updefault}C}}}}
\put(991,-1951){\makebox(0,0)[lb]{\smash{{\SetFigFont{20}{24.0}{\rmdefault}{\mddefault}{\updefault}D}}}}
\put(5581,-4201){\makebox(0,0)[lb]{\smash{{\SetFigFont{20}{24.0}{\rmdefault}{\mddefault}{\updefault}$b_1$}}}}
\put(3781,-6001){\makebox(0,0)[lb]{\smash{{\SetFigFont{20}{24.0}{\rmdefault}{\mddefault}{\updefault}$b_5$}}}}
\put(2071,-6001){\makebox(0,0)[lb]{\smash{{\SetFigFont{20}{24.0}{\rmdefault}{\mddefault}{\updefault}$c_1$}}}}
\put(361,-4201){\makebox(0,0)[lb]{\smash{{\SetFigFont{20}{24.0}{\rmdefault}{\mddefault}{\updefault}$c_5$}}}}
\put(271,-3301){\makebox(0,0)[lb]{\smash{{\SetFigFont{20}{24.0}{\rmdefault}{\mddefault}{\updefault}$d_1$}}}}
\put(2881,-7261){\makebox(0,0)[lb]{\smash{{\SetFigFont{20}{24.0}{\rmdefault}{\mddefault}{\updefault}(a)}}}}
\put(9361,-7261){\makebox(0,0)[lb]{\smash{{\SetFigFont{20}{24.0}{\rmdefault}{\mddefault}{\updefault}(b)}}}}
\put(15121,-7261){\makebox(0,0)[lb]{\smash{{\SetFigFont{20}{24.0}{\rmdefault}{\mddefault}{\updefault}(c)}}}}
\put(8011,-1861){\makebox(0,0)[lb]{\smash{{\SetFigFont{20}{24.0}{\rmdefault}{\mddefault}{\updefault}$d_4$}}}}
\put(11431,-4651){\makebox(0,0)[lb]{\smash{{\SetFigFont{20}{24.0}{\rmdefault}{\mddefault}{\updefault}$b_2$}}}}
\put(13411,-2401){\makebox(0,0)[lb]{\smash{{\SetFigFont{20}{24.0}{\rmdefault}{\mddefault}{\updefault}$d_3$}}}}
\put(17731,-4201){\makebox(0,0)[lb]{\smash{{\SetFigFont{20}{24.0}{\rmdefault}{\mddefault}{\updefault}$b_1$}}}}
\put(15931,-6001){\makebox(0,0)[lb]{\smash{{\SetFigFont{20}{24.0}{\rmdefault}{\mddefault}{\updefault}$b_5$}}}}
\put(8551,-6001){\makebox(0,0)[lb]{\smash{{\SetFigFont{20}{24.0}{\rmdefault}{\mddefault}{\updefault}$c_1$}}}}
\end{picture} }
\caption{Lower bound for $\epsilon^{\cal{R}}_3$}
\label{fig3RL}
\end{center}
\end{figure}
\vspace{0.5in}
\begin{lem}\label{lemma3RL}
 $\epsilon^{\cal{R}}_3 \geq \frac{2}{5}$ 
\end{lem}

\begin{proof}
Let $P$ be arranged into 20 equal subsets. Each quadrant has a group of five subsets, we will denote these groups as 
$A,B,C,D$ as shown in figure~\ref{fig3RL}.



Clearly, all the subsets of one of the groups will be free. Assume that all the subsets
in $A$ are free. The three points in $Q$ can be chosen in two ways.
\begin{itemize}
 \item \emph{Case 1: One point is chosen from each of the other three groups:}
Any set of eight consecutive subsets is contained in a $\frac{2}{5}$-big axis-parallel
rectangle. Since subsets in $A$ are free, all the three subsets near $A$ in $B$ and $D$
cannot be free. This means a point has to be chosen from one of the subsets
$b_1, b_2$ or $b_3$. Now if a
point is chosen from subset $b_1$ then the remaining subsets in $B$ are free and there exists
a $\frac{2}{5}$-big axis-parallel rectangle containing all those subsets and $a_1,a_2,a_3,a_4$ from $A$.
Therefore $b_1$ has to be free. Symmetrically, $d_5$ also has to be free. Clearly two points have to be selected from the subsets $b_2$ and $d_4$
to avoid the $\frac{2}{5}$-big axis-parallel
rectangle with consecutive subsets from $D,B$ and $A$ ($d_5,a_1,\dots,a_5,b_1,b_2$ and $d_4,d_5,a_1,\dots,a_5,b_1$). If the subset $c_1$ is free then there is a $\frac{2}{5}$-big axis-parallel rectangle containing the free subsets 
$d_5,a_1,a_2,a_3,b_3,b_4,b_5,c_1$.
 Therefore the three points have to be chosen from the subsets
$b_2,c_1,d_4$. Now there exists a $\frac{2}{5}$-big axis-parallel rectangle as shown in figure~\ref{fig3RL}(b) that avoids these subsets.
\item \emph{Case 2: Two points are chosen from a group and the third point is chosen
from the diagonally opposite group:}
Assume that two points are chosen from $B$
and one point from $D$. Clearly, a point has to be chosen from the subset $d_3$ to avoid the $\frac{2}{5}$-big
axis-parallel rectangles with consecutive subsets from $D,A$ and $D,C$ ($d_3,d_4,d_5,a_1,\dots,a_5$ and $c_1,\dots,c_5,d_1,d_2,d_3$).
 The other subsets in $D$ are free. The next two points
chosen have to be from the subsets $b_1$ and $b_5$ to avoid the $\frac{2}{5}$-big axis-parallel
rectangles containing consecutive subsets from $B,C,D$ ($b_5,c_1,\dots,c_5,d_1,d_2$) and $D,A,B$ ($d_4,d_5,a_1,\dots,a_5,b_1$). Now
there exists a $\frac{2}{5}$-big axis-parallel rectangle as shown in figure~\ref{fig3RL}(c)
which avoids the subsets $b_1,b_5,d_3$.
\end{itemize}

\end{proof}

\begin{lem}\label{lemma4RL}
 $\epsilon^{\cal{R}}_4 \geq \frac{3}{10}$
\end{lem}

\begin{proof}
Let $P$ be arranged as ten equal subsets as shown in figure~\ref{fig4RL}. The eight
subsets in between the topmost and bottommost subsets are arranged as two
quadrilaterals. Let $Q$ be an $\epsilon$-net of size four. We
claim that there always exists a $\frac{3}{10}$-big
axis-parallel rectangle that avoids $Q$. 

It is easy to see that there can be at most two free subsets in each of the
quadrilaterals since there is a $\frac{3}{10}$-big axis-parallel rectangle that contains
any three free subsets from a quadrilateral. Hence all the four points of $Q$ have to be chosen from the  subsets forming the two quadrilaterals(two points from each quadrilateral). Therefore the top and bottom
subsets are free. Both these subsets form a $\frac{3}{10}$-big axis-parallel
rectangle with the black subsets in the quadrilateral(see figure~\ref{fig4RL}). Therefore two of the points in $Q$ have to be chosen from the black subsets. The
two gray subsets in the lower(resp. upper) half form a $\frac{3}{10}$-big axis-parallel
rectangle with the bottommost(resp. topmost) subset. Hence the remaining two points in $Q$ have to be chosen from the gray subsets(one point from the top gray subsets and one from the bottom gray subsets). Thus one gray subset in the
lower half is free which forms a $\frac{3}{10}$-big axis-parallel rectangle that avoids $Q$ as shown
in figure~\ref{fig4RL}.

\end{proof}
\begin{figure}
\begin{minipage}[b]{160pt}


\begin{center}
\includegraphics[scale=0.5]{r4lb.eps}
\caption{Lower bound for $\epsilon^{\cal R}_4$}
\label{fig4RL}
\end{center}

\end{minipage}
\begin{minipage}[b]{200pt}
 \begin{center}
 \includegraphics[scale=0.5]{r5lb.eps}
\caption{Lower bound for $\epsilon^{\cal R}_5$}
\label{fig5lb}
\end{center}
\end{minipage}

\end{figure}

\begin{lem}
 $\epsilon^{\cal{R}}_5 \geq \frac{1}{4}$ 
\end{lem}
\begin{proof}
Let $P$ be
arranged as eight equal subsets, four in the outer layer and four in the
inner layer (see figure~\ref{fig5lb}). Let $Q$ be an $\epsilon$-net of size five. We claim that there always exists a $\frac{1}{4}$-big axis-parallel rectangle that avoids $Q$.  

It is clear that we have to choose a point each from at least three of the inner
layer subsets, otherwise there can be an axis-parallel rectangle that contains
all the points of two free subsets. Also, a point each has to be chosen from
at least two of the outer subsets to avoid the axis-parallel rectangles
that contains all the points of two consecutive subsets in the outer layer. 
Moreover, these two points are to be chosen from diagonally opposite subsets. Let the $\epsilon$-net points be chosen from the five black subsets as shown in figure~\ref{fig5lb}. Now there exists a $\frac{1}{4}$-big axis-parallel rectangle containing the free subset in the
inner layer and the nearest free subset in the outer layer.


\end{proof}

\begin{lem}
$\epsilon_6^{\mathcal{R}} \ge \frac{1}{5}$; $\epsilon_7^{\mathcal{R}} \ge
\frac{5}{29}$; $\epsilon_8^{\mathcal{R}} \ge \frac{2}{13}$;
$\epsilon_9^{\mathcal{R}} \ge \frac{3}{22}$; $\epsilon_{10}^{\mathcal{R}} \ge
\frac{1}{8}$
\end{lem}

\begin{proof}
 The results follow from theorem~\ref{genlower} with $j=k=3$ for
$\epsilon_6^{\mathcal{R}}$;
$j=2, k=5$ for $\epsilon_7^{\mathcal{R}}$; $j=3, k=5$ for
$\epsilon_8^{\mathcal{R}}$; $j=4, k=5$ for $\epsilon_9^{\mathcal{R}}$;
$j=5, k=5$ for $\epsilon_{10}^{\mathcal{R}}$. 
\end{proof}

\begin{lem}
 $\epsilon_i^{\mathcal{R}} \ge \frac{10}{9i}$, for $i \ge 2$
\end{lem}
\begin{proof}
We use mathematical induction to prove this.
The lemma holds for $i=2$ and $i=3$ since $\epsilon_2^{\mathcal{R}} \ge
\frac{5}{9} \ge \frac{10}{9 \times 2}$ (by lemma~\ref{lemma2RL}) and
$\epsilon_3^{\mathcal{R}} \ge \frac{2}{5} \ge \frac{10}{9 \times 3}$ (by lemma~\ref{lemma3RL}). Now assume the lemma holds for $i=k$,
i.e., $\epsilon_k^{\mathcal{R}} \ge \frac{10}{9k}$ or $\frac{9}{10}k \ge
\frac{1}{\epsilon_k^{\mathcal{R}}}$.
Thus, $\frac{9}{10}k + \frac{9}{5} \ge
\frac{1}{\epsilon_k^{\mathcal{R}}} + \frac{1}{\epsilon_2^{\mathcal{R}}}$, 
i.e., $\frac{9}{10} (k+2) \ge \frac{\epsilon_k^{\mathcal{R}} +
\epsilon_2^{\mathcal{R}}}{\epsilon_k^{\mathcal{R}} \epsilon_2^{\mathcal{R}}}$.
Therefore, $\epsilon_{k+2}^{\mathcal{R}} \ge \frac{\epsilon_k^{\mathcal{R}}
\epsilon_2^{\mathcal{R}}}{\epsilon_k^{\mathcal{R}} + \epsilon_2^{\mathcal{R}}} 
\ge \frac{10}{9(k+2)}$ (from theorem~\ref{genlower}).
Hence the result follows for all values of $i$.


\end{proof}

\begin{table}[t]\label{rectable}
\begin{centering}

\begin{tabular}{|c|c|c|c|c|c|c|c|c|c|c|}
\hline
&$\epsilon_{1}$&$\epsilon_{2}$&$\epsilon_{3}$&$\epsilon_{4}$&$\epsilon_{5}
$&$\epsilon_{6}$&$\epsilon_{7}$&$\epsilon_{8}$&$\epsilon_{9}$&$\epsilon_{10}$\\
\hline
LB&3/4&5/9&2/5&3/10&1/4&1/5&5/29&2/13&3/22&1/8\\
\hline
UB&3/4&5/8&9/16&1/2&15/32&15/32&3/7&2/5&5/13&3/8\\
\hline
\end{tabular}
\caption{Summary of lower and upper bounds for $\epsilon_{i}^{\mathcal{R}}$.
}
\end{centering}
\end{table} 

\noindent The summary of lower and upper bounds for $\epsilon_{i}^{\mathcal{R}}$ is given in table 1.
