\label{sec:TraceMetaTheory}

The Trace Meta-Theory (TMT) \cite{TMTmanuscript} provides a set of definitions about how to design a trace for a specific domain of observation.

A \textbf{trace} may be interpreted as a sequence of communication actions
that may take place between an \textbf{observer} and an
\textbf{observed process}. It consists of
finite unbounded sequences trace events. There
is also the \textbf{tracer} that means the generator of trace. According
to \cite{deransart2008trace}, the TMT focuses particularly on providing semantics to 
tracers and the produced traces as independent as
possible from those of the processes or from the ways the tracers produce them.

There are two concepts of trace \cite{deransart2008trace} (cf. the Figure~\ref{fig:chromepipiline} and the Section~\ref{sec:conttrace}). The first one is the \textbf{virtual trace}, it
represents a sequence of events showing the evolution of a virtual state which contains
all that one can or wants to know about the observed process. The second one
is called \textbf{actual trace}, it represents the generated trace in the form of some encoding of the current virtual state. Finally, there is the idea of \textbf{full trace} if the parameters chosen to be observed about the process
represent the totality of useful knowledge regarding it (explicitly or implicitly).


\subsection{Components of Trace Generation and Use}

The Figure~\ref{fig:componentsPD} shows the different components related to a unique trace.
We distinguish  5 components, in this order.\begin{enumerate}
\item Observed process

The observed process is assumed more or less abstracted in such a way that his behavior can be described by a virtual trace, that is to say, a sequence of (partial) states. A formal description of the process, if possible, can be considered as a formal semantics, which can be used to describe the actual trace extraction.

\begin{figure} [h]
\centering
\scalebox{0.3}{\includegraphics{img/virtualactualtrace}}
\caption[Virtual and Actual Trace]
{Virtual and Actual Trace.}
\label{fig:chromepipiline}
\end{figure} 


\item Extractor

This component is the extraction function of the actual trace from the virtual trace. From a theoretical point of view, we can see it as a specific component, but in practice it corresponds to the tracer whose realization, in the case of a programming language, usually requires modifying the code of the process.
\item Filter

The role of the filter component, or {\em driver} \cite{DLads05}, is to select a useful sub-trace. This component requires a specific study. It is assumed here that it operates on the actual trace (that produced by the tracer). The fact of making it as a proper component corresponds to the specific approach adopted here, which implies that the extracted actual trace is full. The filtering depends on the specific application, implying that the full trace already contains all the information potentially needed for various uses.
\item Rebuilder

The reconstruction component performs the reverse operation of the extraction, at least for a subpart of the trace, and then reconstructs a sequence of partial virtual states. If the trace is faithful (i.e. no information is lost by the driver) \cite{TMTmanuscript}, this ensures that the virtual trace reconstruction is possible. In this case also, the separation between two components (rebuilder and analyzer) is essentially theoretical; these two components may be in practice very entangled.
\item Analyzer

The component using a trace may be a trace analyzer or any application. 
\end{enumerate}

\begin{figure}  \centering
\scalebox{0.6}{\includegraphics{img/components2.eps}}
\caption[Components of the TMT, Composants pour la production et l'utilisation d'une trace]
{Components of the TMT}
\label{fig:componentsPD}  
\end{figure} 


\vspace{4mm}
With these components it may be associated three main specification steps, as illustrated on the Figure~\ref{fig:componentsso}.

\begin{figure} \centering
\scalebox{0.7}{\includegraphics{img/componentsSO2.eps}}
\caption[Concepts formels li\'es \`a la production et l'utilisation d'une trace]
{Formal concepts related to the generation and use of a trace}
\label{fig:componentsso}  
\end{figure} 

\begin {enumerate}[-]
\item Observational Semantics (OS)

The OS describes formally the observed process (or a family of processes) and the actual trace extraction. This aspect will be studied deeper in the Section~\ref{sec:conttrace}. The intention here is to express the OS using simple fluent calculus.
\item Querying

Due to the separation in several components, the actual trace may be expressed in any language. We suggest using XML. This allows to use standard querying techniques defined for XML. This aspect will not be developed here, but we chose to express the trace in XML and give in the Appendix~\ref{app:chrtraceschema} the corresponding XML schema.
\item Interpretative Semantics (IS)

The interpretation of a trace, i.e. the capacity of reconstructing the sequence of virtual states from an actual trace, is formally described by the Interpretative Semantics. In the TMT no particular application is defined; its objective is just to make sure that the original observed semantics of the process has been fully communicated to the application, independently of what the application does.
\end{enumerate}

\subsection{Contiguous Full Traces}
\label{sec:conttrace}

We introduce here the two traces which may be associated to a single process equipped with a tracer. We recall here the definitions used in \cite{TMTmanuscript}.




\subsubsection{Virtual Trace}

A full virtual trace is defined on a domain of states. Given ${\cal P}$  a finite set of (names) of parameters $ p_i $  defined on the domains ${\cal P}_i$.
The ${\cal P}_i$ are domains of objects of any kind. They may also have relations (functional or otherwise) between them and they can be infinite in size.

A domain of states ${\cal S}$  is defined on the Cartesian product of the parameter domains: $S \subseteq {\cal P}_{1}\times ...\times {\cal P}_{n}$.


\begin{definition} [Contiguous Full Virtual Trace]

A contiguous full virtual trace  is a sequence of trace events of the form {\bf $e_t: (t, r_t, s_t),\ t \ge 1$}, where:
\begin{itemize}
  \item $t$: is the {\bf chrono}, specific time of the trace. It is an integer increased by one unit in each successive event. To point a particular value of the chrono, we will talk about {\em moment} of the trace.
   \item $r_t$: an identifier of {\bf action} characterizing the type of actions undertaken to make the transition from state $s_{t-1}$ to state $s_t$.
   \item $s_t$: is an element of the state domain. $s_t = p_{1, t}, ..., p_{n, t}$ is the current state reached
at moment $t$, and the $p_{i, t}$ are values of the {\bf parameters} $p_i$ at moment $t$.
$s_t$ is the {\em current full virtual state}.
\end{itemize}
\end{definition}
A finite virtual trace over $t\ (t > 0)$ events will be denoted $T^v_t = <s_0, \overline{e_t}>$, where $s_0$ is the initial full virtual state and $\overline{e_t}$ represents the sequence $e_1 \ldots e_i \ldots e_t $.



The full virtual trace is {\em contiguous} insofar as all the moments in the interval $[1..t]$ are present in the trace $T^v_t = <s_0, \overline{e_t}>$.


\subsubsection{Actual Trace}

The full virtual trace represents what we want or what is possible to observe of a process. It describes the development stages of this process in the form of the evolution of a state which contains the observables. As the current virtual state of a process can be fully represented in this trace, one cannot expect neither to produce it nor to communicate it efficiently. In practice we will perform a kind of ``compression'' of the information conveyed by the virtual states and their evolution, transmitted or communicable to the process observers, and one shall ensure that these processes are able to ``decompress'' it. This actually communicated information is the {\em actual trace}.

An actual full trace is defined on an actual state domain. Let ${\cal A}$ be a finite set of (names of) attributes $a_i$ defined on domains of attributes ${\cal A}_i$.
The attributes may have relationships (they are not necessarily independent) and they can be infinite in size.

An actual state domain ${\cal A}$ is defined on the Cartesian product of attributes domains: ${\cal A} \subseteq {\cal A}_{1}\times ...\times {\cal A}_{n}$. 

\begin{definition} [Contiguous Full Actual Trace]

An actual trace is a sequence of trace events of the form {\bf $w_t: (t, a_t),\ t \geq 1$}, where:

\noindent
$t$ is the chrono and $a_t\ \in\ {\cal A}$ denotes a finite sequence of attributes values. $a_t$ is the current actual state. The number of attributes of a trace event is bound by $ n $. Each state $a_t$ contains at most $n$ attributes whose number depends exclusively on the type of action which produced it.
\end{definition}
An actual trace with $t\ (t > 0)$  events is denoted $T^w_t = <s_0, \overline{w_t}>$, where $s_0$ is the initial virtual state common to both traces and $\overline{w_t}$  represents the sequence $w_1, \ldots w_i, \ldots, w_t $.




\subsection{Generic Trace and Composition}

We study here the methodology of generic full trace development for a multi-layer based application.

\subsubsection{Generic Trace of a Familly of Observed Processes}

Consider again the Figure~\ref{fig:tmtprocess} in the introduction.
It illustrates the fact that different implementations of CHR can be abstracted by a unique simpler model.
This common model is used to specify the unique virtual and actual traces of these implementations.
This illustrates the way we will proceed to get a generic trace of CHR: starting from an abstract theoretical, general but sufficiently refined, semantics of CHR which is (almost) the same implemented in all CHR platforms.


\subsubsection{Composition of Traces}
\label{sec:compos}

Now we consider the case of an application written in CHR. It may be for example a particular constraints solver like CLP(FD). In this case there exists already a generic trace called {\em GenTra4CP} \cite{oadimpac}. This trace is generic for most of the CLP(FD) existing constraints solvers. Therefore a tracer of CLP(FD) solver implemented in CHR should also produce this trace. But we may be interested in refining the trace considering that there are two layers: the layer of the application (CLP(FD)) and the layer of the language in which it is implemented (CHR). The most refined trace will then be the trace in the GenTra4CP format extended with elements of the generic full trace of CHR alone. The generic full trace of CLP(FD) on CHR is an extension of the application trace taking into account details of lower layers. 

This is illustrated by the Figure~\ref{fig:componentsaplchr} in the case of two layers: an application (like CLP(FD) for example) implemented in CHR. This method can be generalized to applications with several layers of software. The Figure~\ref{fig:chromepipiline} shows in fact at least 4 layers.

\begin{figure} \centering
\scalebox{0.6}{\includegraphics{img/componentsapl-chr}}
\caption[Combination of Generic Full Traces for a two Layers Application]
{Composition of Generic Full Traces for a two Layers Application}
\label{fig:componentsaplchr}
\end{figure} 

In our components based approach it means that we may define separately and independently specific generic full traces for each layer, and, so in this case for the application (APL) and the under-layer of CHR. The generic full trace APL on CHR is a kind of {\em composition} of traces and will be obtained by some merging of both generic full traces into a unique one. The result may not exactly be a union of all actions, parameters and attributes, but it is not our purpose here to study more deeply this aspect. For more details see \cite{TMTmanuscript}.






\subsection{Observational Semantics}

The Observational Semantics (OS) is a description of a possibly unbounded
data flow without explicit reference to the operational semantics of the
process which produced it \cite{deransart2007observational}. The
OS may be considered as a abstract model of process, in the
case of a single observed process or it can be an abstraction of the
semantics to several processes. It is defined as a Labelled
Transition Systems (LTS) \cite{gilmore2005trends}.





\subsubsection{Representation of the Observational Semantics}

The Observational Semantics has two parts: a state transition function and a trace extraction function.

The first part is a formal model on the way successive events of the virtual trace are related. 
It is a virtual trace semantics in the sense that, given a full virtual trace \newline $T^v_t = <s_0, \overline{e_t}>$, it explains the sequence of events $e_t$ by a transition function\footnote{It is fact a relation since the transitions may be nondeterministic.} recursively applied from an initial virtual state.



\vspace{1mm}
The second part, the function of extraction, produces what is actually ``broadcasted'' outside from the observed process. This function has as arguments the current state and the type of action, and produces the attributes of the actual trace.

\begin{definition} [Observational Semantics (OS)]

An Observational Semantics is defined by the tuple $< S, R_O, A, E, T, S_0 >$, where
\begin{itemize}
\item $S$ is a  {\em virtual state domain}, where each state is described by a set of parameters.
\item $R_O$ is a finite set of action types, set of identifiers used as labels for transitions.
\item $A$ is a {\em actual state domain}, where each state is described by a set of attributes.
\item $E$ is the local extraction function of the actual state $a$, performed by transition of action type $r$ issued from state $s$, $E : R$ {\tt x} $S \rightarrow A$, which satisfies by definition: $E(r,s)= a$ ($a \in A$, set of actual states). More precisely, the set of attributes $a_t$ of the event $t$ of the actual trace is derived from the current state at moment $t-1$ of the virtual trace and the transition labelled by the action type $r_t$, i.e.
\begin{quote}
$E(r_t, s_{t-1})= a_t$
\end{quote}
\item $T$ state transition function $T : R$ {\tt x} $\rightarrow S$, i.e.
\begin{quote}
$T(r_t, s_{t-1})= s_t$
\end{quote}
\item $S_0 \subseteq S$, set of initial states.
\end{itemize}
\end{definition}

\vspace{2mm}

The OS may be represented by ``rules'', one for each action, describing the transition and the actual trace event extraction corresponding to the action. A rule has 4 items.

\vspace{0,5mm}
\begin{itemize}
\item {\bf AType}: an action identifier $r \in R_O$
\item {\bf ACond}: \{ some auxiliary computations on the current virtual state and condition for executing the action corresponding to the transition: a first-order logic formula using predicates on the parameters \}
\item {\bf VSEffect}: \{ the effect of the action $r$ on the current state $s$, resulting in a new state $s'$ , and some auxiliary computations relative to the attributes of the trace event \}
\item {\bf Etrace}: \{ the attributes of the trace event produced by the action $r$: $a$, new extracted actual trace event \} 
\end{itemize}



\vspace{5mm}
{\bf Example 2.1: the Fibonacci Function}
\label{ex:fibo}

Idealized (biologically unrealistic) rabbit population.

\vspace{2mm}
The OS $< S, I_f, R_O, A, E, T, S_0 >$, describes the deterministic transition function $T_l$.

\vspace{2mm}
$S$: ${\cal N}_+^*$ (positive integers list), $s_t$ is the complete evolution of the population from moment $0$ until moment $t+1$: $s_t = [popu_0,\ldots, popu_t, popu_{t+1}]$ 

$R_O$: \{$mg$\} (monthly growing)  

$A$: ${\cal N}_+$, $a_t$ is the population at moment $t+1$ ($popu(t+1)$).

$E$: $E(mg,s) = plast(s)+last(s)$. There is one rule only to describe $E$.

$T_l$: $T(mg,s) = s \ o \ [plast(s)+last(s)]$ (respectively before last and last elements of the list s$s$, $o$ denote lists concatenation). The new virtual state $t$ is the previous state to which the sum of the two last elements is appended.


$S_0$: $s_0 = [1,1]$.



\vspace{1mm}
{\bf AType}: $mg$

{\bf ACond}: \{ $true$\}

{\bf VSEffect}:  \{ $v \leftarrow plast(s)+last(s) \wedge \ s' \leftarrow s\  o \  [v]$ \}

{\bf Etrace}:  \{$v$\} 

\vspace{4mm}
{\bf Traces:}
\ \ 
\newline $T^v_5 = <[1,1],[(1,mg,[1,1,2]), (2,mg,[1,1,2,3]), \ldots,$

$ \ \ \ \ \ \ \ \ \ \ \ \ \ \ \ \ \ \ \ \ \ \ \ \  (4,mg,[1,1,2,3,5,8]),(5,mg,[1,1,2,3,5,8,13])]>$
\newline $T^w_5 = <[1,1],[(1,2),(2,3),(3,5),(4,8),(5,13)]> $



\subsubsection{Simple Fluent Calculus}\label{sec:simplefluentcalculus}
\label{FC}


The Fluent Calculus (FC) is a logic-based representation language
for knowledge about actions, change, and causality \cite{thielscher1999situation}. As an
extension of the classical Situation Calculus \cite{reiter1991frame}, Fluent
Calculus provides a general framework for the development of axiomatic semantics
for dynamic domains.

\vspace{1mm}
The simple fluent calculus (SFC) has the following appealing qualities:
\begin{enumerate} [-]
\item Simplification of the description, since the notions of virtual state and actual trace are ``naturally'' embedded in the fluent calculus (the situation corresponding to a state can be viewed as a representation of the actual trace).

\item Reasoning on transitions may be simpler, as it supports handling partial virtual states (with appropriate axiomatisation). Formal proofs become simpler in the SFC  (less deductions, at least for ``direct closed'' effects, see \cite{thielscher1999situation,framepb03}). It follows that properties like ``faithfulness'' \cite{deransart2007observational} should be easier to prove formally. The Symmetry of ``state axioms'' allows forward and backward reasoning. With the simplicity of the representation of state changes, this should make such proofs simpler.

\item The description in SFC makes such specification potentially executable in Flux. There is some limits to the executability, related to the partial axiomatisation of the observational semantics and executability of formal specification in general. However such framework may facilitate some simulations.

\end{enumerate}

The Fluent Calculus is a sorted logic language with four standard sorts: FLUENT,
STATE, ACTION, and SIT (which stands for situation). A fluent describes a single state property
that may change by the means of the actions of some agent. A state is a collection of fluents.
Adopted from Situation Calculus, the standard sort SIT describes sequences of actions.

The pre-defined constant $\emptyset : STATE$ stands for the empty state. Each
term of sort FLUENT is also an (atomic) STATE, and the function $\circ : STATE
\times STATE \mapsto STATE$, written in infix notation, represents the composition
of two states. The following abbreviation $Holds(f, z)$ is used to express that
fluent $f$ holds in state $z$:
\begin{equation}
Holds(f, z) =_{def} (\exists z') f \circ z' = z
\end{equation}

The behavior of function ``$\circ$'' is governed by the
foundational axioms of Fluent Calculus, which essentially characterize states as sets of fluents.

\begin{equation}
( z_1 \circ z_2 ) \circ z_3 = z_1 \circ ( z_2 \circ z_3 )
\end{equation}

\begin{equation}
z_1 \circ z_2 = z_2 \circ z_1
\end{equation}

\begin{equation}
z \circ z = z
\end{equation}

\begin{equation}
z_1 \circ \emptyset = z
\end{equation}

\begin{equation}
Holds(f, f_1 \circ z) \supset f_1 \vee Holds(f, z)
\end{equation}

States can be updated by adding and/or removing one or more fluents. Addition of
a sub-state $z$ to a state $z_1$ is simply expressed as $z_2 = z_1 \circ z$, and
removal is defined by

\begin{equation}
z_2 = z_1 - z =_{def} (Holds(f, z_2) \equiv Holds(f,
z_1) \land \neg Holds(f, z))
\end{equation} 

The standard predicate $Poss : ACTION \times STATE$ in Fluent Calculus is used to
axiomatize the conditions under which an action is possible in a state, i.e., the situations
in which the \textit{pre-condition} of this actions is satisfied.

The pre-defined constant $S_0 : SIT$ is the initial (i.e., before the execution
of any action) situation. The function $Do : ACTION \times SIT \mapsto SIT$ 
denotes the addition of an action to a situation. The standard function $State
: SIT \mapsto STATE$ is used to denote the state, i.e., the fluents that hold in
a situation, after a sequence of actions. This allows to extend macro Holds and 
predicate $Poss$ to SITUATION arguments as follows.

\begin{equation}
Holds(f, s) =_{def} Holds(f, State(s))
\end{equation}  
\begin{equation}
Poss(a, s) =_{def} Poss(a, State(s))
\end{equation} 

In a Fluent Calculus Axiomatization, beyond the definition of the domain sorts, functions
and predicates, we can define a set of axioms that must follow three pre-defined
axiom schemas: the precondition axioms, the state update axioms and the state constraint
axioms.
 
\begin{definition}[Pure State Formula]
A Pure State Formula is a First Order formula
$\Pi(z)$

\begin{itemize}
    \item There is only one free state variable $z$
	\item It is composed of atomic formulas in the form:
	\subitem $Holds(\phi, z)$, where $\phi$ is of the sort FLUENT
	\subitem atoms which do not use any reserved predicate of Fluent Calculus 
\end{itemize}

\end{definition}

\begin{definition}[Precondition Axiom]
A precondition axiom follows the schema: $Poss(A(\vec{x}), z) \equiv
\Pi(\vec{x}, z)$, where $\Pi(\vec{x}, z)$ is a Pure State Formula.
\end{definition}

This kind of axiom states that the execution of the action $A$ with the
parameters $\vec{x}$ is possible in the state $z$ if and only if $\Pi_A(\vec{x},
z)$ is true.



\begin{definition}[State Update Axiom]
A state update axiom follows the schema:

$Poss(A(\vec{x}), State(s)) \land \Pi(\vec{x}, State(s)) \supset \Gamma(State(Do(A(\vec{x}), s)), State(s))$


where

$\Gamma(State(Do(A(\vec{x}), s)), State(s)) = State(Do(A(\vec{x}), s)) = State(s) \circ \vartheta^{+} - \vartheta^{-}$


where

$\vartheta^{+}$ and $\vartheta^{-}$ are partial states.
\end{definition}


\subsubsection{Observational Semantics in Simple Fluent Calculus}
\label{sec:obssemSfc}

\vspace{2mm}
A virtual state of the observed process corresponds to a state in SFC described by a set of fluents (this correspondence must be explicitly specified).

Each type of action in the OS is an action name in the SFC. A particular action is denoted $R$ in the following.

Actual states are elements of the Cartesian product of attribute domains (this domain must be explicitly specified).

Transition and extraction function (or relation) are described using both fundamental following schemes (fundamental axioms of the Fluent calculus)

\begin{enumerate}
\item \textit{Pre-Condition Axioms}:
 
$\ \ \ Poss(R, \vec{x}, z)  \equiv \Pi(\vec{x}, z)$




\item \textit{State Update Axioms}: 

$\ \ \ Poss(R, \vec{x}, State(s)) \land \Pi(\vec{x}.\vec{y}, State(s)) \supset $

$\ \ \ \ \ \ \ \ \ \ \ \ \ \ \ \ \ \Gamma_R(State(Do(R, w(\vec{x}.\vec{y}, State(s)),s)), State(s))$


where $w(\vec{x}.\vec{y},State(s))$ is an actual trace event associated with the transition, and derived from the current state (using local variables too).
\end{enumerate}

There are as many pre-conditions and state update axioms as there are action types $R$ in the OS. $\Gamma_R$ may be a disjunction. It defines the new virtual state and the corresponding extracted actual trace event attributes $w(\vec{x}.\vec{y},State(s))$.

\vspace{1mm}
Nota: a situation $s$ contains the sequence of actions in the OS executed to reach the current virtual state $z=State(s)$, and also the sequence of extracted actual trace events such that $T^w = E(T^v)$.

\vspace{1mm}
An actual trace $T^w$ is the sequence of $w_i$, with chrono, found in the situation 

$\ \ \ s = Do(R_n, \vec{x}_n, w_n, Do(R_{n-1}, \vec{x}_{n-1}, w_{n-1}, ..., S_0)...)$


It may be computed according to the following axioms:

$\ \ \ Extraction(0,S) = S$

$\ \ \ Extraction(n+1, Do(R,\vec{x}, w, s)) = ((n+1).w).Extraction(n, s)$


\vspace{4mm}
{\bf Example 2.2: OS for Fibonacci}
\label{ssec:fibSFC}

The virtual state contains only one fluent $Fib/1$ of type $List(Int) -> Fluent$, and there is only one type of action $Mg$. The actual state contains just 2 attributes, respectively of type $String$ and $Int$. A vector is represented by a sequence (Prolog list syntax).









\vspace{4mm}



$
S_0 = Fib([1,1])
$ \\

$
Poss(Mg, [l, pl], z)  \equiv Holds(Fib([l,pl|x]), z) 
$ \\

$
Poss(Mg, [l, pl], State(s)) \land v = l+pl \supset $

$\ \ \ \ State(Do(Mg, [Mg, v], s)) = State(s) \circ Fib([v,l,pl|x]) - Fib([l,pl|x])
$

\subsection{Trace Query and Analysis Tools}





Trace query and analysis tools are considered here as separate components, as illustrated in the Figure~\ref{fig:componentsPD}. Although they are part of the CHROME-REF project, they are not studied more deeply here. Instead, we propose a first repesentation of the actual CHR trace using XML. The XML schema is given in the Appendix~\ref{app:chrtraceschema}, and an example of produced trace in the XML format in the Appendix~\ref{app:leqexample}.



