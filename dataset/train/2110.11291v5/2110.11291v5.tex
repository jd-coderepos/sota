\documentclass{article}


\usepackage{iclr2022_conference,times}
\iclrfinalcopy

\usepackage[utf8]{inputenc} \usepackage[T1]{fontenc}    \usepackage{hyperref}       \usepackage{url}            \usepackage{booktabs}       \usepackage{amsfonts}       \usepackage{nicefrac}       \usepackage{microtype}      \usepackage{xcolor}         





\usepackage{amsmath,amsfonts,bm}

\newcommand{\figleft}{{\em (Left)}}
\newcommand{\figcenter}{{\em (Center)}}
\newcommand{\figright}{{\em (Right)}}
\newcommand{\figtop}{{\em (Top)}}
\newcommand{\figbottom}{{\em (Bottom)}}
\newcommand{\captiona}{{\em (a)}}
\newcommand{\captionb}{{\em (b)}}
\newcommand{\captionc}{{\em (c)}}
\newcommand{\captiond}{{\em (d)}}

\newcommand{\newterm}[1]{{\bf #1}}


\def\figref#1{figure~\ref{#1}}
\def\Figref#1{Figure~\ref{#1}}
\def\twofigref#1#2{figures \ref{#1} and \ref{#2}}
\def\quadfigref#1#2#3#4{figures \ref{#1}, \ref{#2}, \ref{#3} and \ref{#4}}
\def\secref#1{section~\ref{#1}}
\def\Secref#1{Section~\ref{#1}}
\def\twosecrefs#1#2{sections \ref{#1} and \ref{#2}}
\def\secrefs#1#2#3{sections \ref{#1}, \ref{#2} and \ref{#3}}
\def\eqref#1{(\ref{#1})}
\def\Eqref#1{Equation~\ref{#1}}
\def\plaineqref#1{\ref{#1}}
\def\chapref#1{chapter~\ref{#1}}
\def\Chapref#1{Chapter~\ref{#1}}
\def\rangechapref#1#2{chapters\ref{#1}--\ref{#2}}
\def\algref#1{algorithm~\ref{#1}}
\def\Algref#1{Algorithm~\ref{#1}}
\def\twoalgref#1#2{algorithms \ref{#1} and \ref{#2}}
\def\Twoalgref#1#2{Algorithms \ref{#1} and \ref{#2}}
\def\partref#1{part~\ref{#1}}
\def\Partref#1{Part~\ref{#1}}
\def\twopartref#1#2{parts \ref{#1} and \ref{#2}}

\def\ceil#1{\lceil #1 \rceil}
\def\floor#1{\lfloor #1 \rfloor}
\def\1{\bm{1}}
\newcommand{\train}{\mathcal{D}}
\newcommand{\valid}{\mathcal{D_{\mathrm{valid}}}}
\newcommand{\test}{\mathcal{D_{\mathrm{test}}}}

\def\eps{{\epsilon}}


\def\reta{{\textnormal{}}}
\def\ra{{\textnormal{a}}}
\def\rb{{\textnormal{b}}}
\def\rc{{\textnormal{c}}}
\def\rd{{\textnormal{d}}}
\def\re{{\textnormal{e}}}
\def\rf{{\textnormal{f}}}
\def\rg{{\textnormal{g}}}
\def\rh{{\textnormal{h}}}
\def\ri{{\textnormal{i}}}
\def\rj{{\textnormal{j}}}
\def\rk{{\textnormal{k}}}
\def\rl{{\textnormal{l}}}
\def\rn{{\textnormal{n}}}
\def\ro{{\textnormal{o}}}
\def\rp{{\textnormal{p}}}
\def\rq{{\textnormal{q}}}
\def\rr{{\textnormal{r}}}
\def\rs{{\textnormal{s}}}
\def\rt{{\textnormal{t}}}
\def\ru{{\textnormal{u}}}
\def\rv{{\textnormal{v}}}
\def\rw{{\textnormal{w}}}
\def\rx{{\textnormal{x}}}
\def\ry{{\textnormal{y}}}
\def\rz{{\textnormal{z}}}

\def\rvepsilon{{\mathbf{\epsilon}}}
\def\rvtheta{{\mathbf{\theta}}}
\def\rva{{\mathbf{a}}}
\def\rvb{{\mathbf{b}}}
\def\rvc{{\mathbf{c}}}
\def\rvd{{\mathbf{d}}}
\def\rve{{\mathbf{e}}}
\def\rvf{{\mathbf{f}}}
\def\rvg{{\mathbf{g}}}
\def\rvh{{\mathbf{h}}}
\def\rvu{{\mathbf{i}}}
\def\rvj{{\mathbf{j}}}
\def\rvk{{\mathbf{k}}}
\def\rvl{{\mathbf{l}}}
\def\rvm{{\mathbf{m}}}
\def\rvn{{\mathbf{n}}}
\def\rvo{{\mathbf{o}}}
\def\rvp{{\mathbf{p}}}
\def\rvq{{\mathbf{q}}}
\def\rvr{{\mathbf{r}}}
\def\rvs{{\mathbf{s}}}
\def\rvt{{\mathbf{t}}}
\def\rvu{{\mathbf{u}}}
\def\rvv{{\mathbf{v}}}
\def\rvw{{\mathbf{w}}}
\def\rvx{{\mathbf{x}}}
\def\rvX{{\mathbf{X}}}
\def\rvy{{\mathbf{y}}}
\def\rvY{{\mathbf{Y}}}
\def\rvz{{\mathbf{z}}}
\def\rvZ{{\mathbf{Z}}}

\def\erva{{\textnormal{a}}}
\def\ervb{{\textnormal{b}}}
\def\ervc{{\textnormal{c}}}
\def\ervd{{\textnormal{d}}}
\def\erve{{\textnormal{e}}}
\def\ervf{{\textnormal{f}}}
\def\ervg{{\textnormal{g}}}
\def\ervh{{\textnormal{h}}}
\def\ervi{{\textnormal{i}}}
\def\ervj{{\textnormal{j}}}
\def\ervk{{\textnormal{k}}}
\def\ervl{{\textnormal{l}}}
\def\ervm{{\textnormal{m}}}
\def\ervn{{\textnormal{n}}}
\def\ervo{{\textnormal{o}}}
\def\ervp{{\textnormal{p}}}
\def\ervq{{\textnormal{q}}}
\def\ervr{{\textnormal{r}}}
\def\ervs{{\textnormal{s}}}
\def\ervt{{\textnormal{t}}}
\def\ervu{{\textnormal{u}}}
\def\ervv{{\textnormal{v}}}
\def\ervw{{\textnormal{w}}}
\def\ervx{{\textnormal{x}}}
\def\ervy{{\textnormal{y}}}
\def\ervz{{\textnormal{z}}}

\def\rmA{{\mathbf{A}}}
\def\rmB{{\mathbf{B}}}
\def\rmC{{\mathbf{C}}}
\def\rmD{{\mathbf{D}}}
\def\rmE{{\mathbf{E}}}
\def\rmF{{\mathbf{F}}}
\def\rmG{{\mathbf{G}}}
\def\rmH{{\mathbf{H}}}
\def\rmI{{\mathbf{I}}}
\def\rmJ{{\mathbf{J}}}
\def\rmK{{\mathbf{K}}}
\def\rmL{{\mathbf{L}}}
\def\rmM{{\mathbf{M}}}
\def\rmN{{\mathbf{N}}}
\def\rmO{{\mathbf{O}}}
\def\rmP{{\mathbf{P}}}
\def\rmQ{{\mathbf{Q}}}
\def\rmR{{\mathbf{R}}}
\def\rmS{{\mathbf{S}}}
\def\rmT{{\mathbf{T}}}
\def\rmU{{\mathbf{U}}}
\def\rmV{{\mathbf{V}}}
\def\rmW{{\mathbf{W}}}
\def\rmX{{\mathbf{X}}}
\def\rmY{{\mathbf{Y}}}
\def\rmZ{{\mathbf{Z}}}

\def\ermA{{\textnormal{A}}}
\def\ermB{{\textnormal{B}}}
\def\ermC{{\textnormal{C}}}
\def\ermD{{\textnormal{D}}}
\def\ermE{{\textnormal{E}}}
\def\ermF{{\textnormal{F}}}
\def\ermG{{\textnormal{G}}}
\def\ermH{{\textnormal{H}}}
\def\ermI{{\textnormal{I}}}
\def\ermJ{{\textnormal{J}}}
\def\ermK{{\textnormal{K}}}
\def\ermL{{\textnormal{L}}}
\def\ermM{{\textnormal{M}}}
\def\ermN{{\textnormal{N}}}
\def\ermO{{\textnormal{O}}}
\def\ermP{{\textnormal{P}}}
\def\ermQ{{\textnormal{Q}}}
\def\ermR{{\textnormal{R}}}
\def\ermS{{\textnormal{S}}}
\def\ermT{{\textnormal{T}}}
\def\ermU{{\textnormal{U}}}
\def\ermV{{\textnormal{V}}}
\def\ermW{{\textnormal{W}}}
\def\ermX{{\textnormal{X}}}
\def\ermY{{\textnormal{Y}}}
\def\ermZ{{\textnormal{Z}}}

\def\vzero{{\bm{0}}}
\def\vone{{\bm{1}}}
\def\vmu{{\bm{\mu}}}
\def\vtheta{{\bm{\theta}}}
\def\va{{\bm{a}}}
\def\vb{{\bm{b}}}
\def\vc{{\bm{c}}}
\def\vd{{\bm{d}}}
\def\ve{{\bm{e}}}
\def\vf{{\bm{f}}}
\def\vg{{\bm{g}}}
\def\vh{{\bm{h}}}
\def\vi{{\bm{i}}}
\def\vj{{\bm{j}}}
\def\vk{{\bm{k}}}
\def\vl{{\bm{l}}}
\def\vm{{\bm{m}}}
\def\vn{{\bm{n}}}
\def\vo{{\bm{o}}}
\def\vp{{\bm{p}}}
\def\vq{{\bm{q}}}
\def\vr{{\bm{r}}}
\def\vs{{\bm{s}}}
\def\vt{{\bm{t}}}
\def\vu{{\bm{u}}}
\def\vv{{\bm{v}}}
\def\vw{{\bm{w}}}
\def\vx{{\bm{x}}}
\def\vy{{\bm{y}}}
\def\vz{{\bm{z}}}

\def\evalpha{{\alpha}}
\def\evbeta{{\beta}}
\def\evepsilon{{\epsilon}}
\def\evlambda{{\lambda}}
\def\evomega{{\omega}}
\def\evmu{{\mu}}
\def\evpsi{{\psi}}
\def\evsigma{{\sigma}}
\def\evtheta{{\theta}}
\def\eva{{a}}
\def\evb{{b}}
\def\evc{{c}}
\def\evd{{d}}
\def\eve{{e}}
\def\evf{{f}}
\def\evg{{g}}
\def\evh{{h}}
\def\evi{{i}}
\def\evj{{j}}
\def\evk{{k}}
\def\evl{{l}}
\def\evm{{m}}
\def\evn{{n}}
\def\evo{{o}}
\def\evp{{p}}
\def\evq{{q}}
\def\evr{{r}}
\def\evs{{s}}
\def\evt{{t}}
\def\evu{{u}}
\def\evv{{v}}
\def\evw{{w}}
\def\evx{{x}}
\def\evy{{y}}
\def\evz{{z}}

\def\mA{{\bm{A}}}
\def\mB{{\bm{B}}}
\def\mC{{\bm{C}}}
\def\mD{{\bm{D}}}
\def\mE{{\bm{E}}}
\def\mF{{\bm{F}}}
\def\mG{{\bm{G}}}
\def\mH{{\bm{H}}}
\def\mI{{\bm{I}}}
\def\mJ{{\bm{J}}}
\def\mK{{\bm{K}}}
\def\mL{{\bm{L}}}
\def\mM{{\bm{M}}}
\def\mN{{\bm{N}}}
\def\mO{{\bm{O}}}
\def\mP{{\bm{P}}}
\def\mQ{{\bm{Q}}}
\def\mR{{\bm{R}}}
\def\mS{{\bm{S}}}
\def\mT{{\bm{T}}}
\def\mU{{\bm{U}}}
\def\mV{{\bm{V}}}
\def\mW{{\bm{W}}}
\def\mX{{\bm{X}}}
\def\mY{{\bm{Y}}}
\def\mZ{{\bm{Z}}}
\def\mBeta{{\bm{\beta}}}
\def\mPhi{{\bm{\Phi}}}
\def\mLambda{{\bm{\Lambda}}}
\def\mSigma{{\bm{\Sigma}}}

\DeclareMathAlphabet{\mathsfit}{\encodingdefault}{\sfdefault}{m}{sl}
\SetMathAlphabet{\mathsfit}{bold}{\encodingdefault}{\sfdefault}{bx}{n}
\newcommand{\tens}[1]{\bm{\mathsfit{#1}}}
\def\tA{{\tens{A}}}
\def\tB{{\tens{B}}}
\def\tC{{\tens{C}}}
\def\tD{{\tens{D}}}
\def\tE{{\tens{E}}}
\def\tF{{\tens{F}}}
\def\tG{{\tens{G}}}
\def\tH{{\tens{H}}}
\def\tI{{\tens{I}}}
\def\tJ{{\tens{J}}}
\def\tK{{\tens{K}}}
\def\tL{{\tens{L}}}
\def\tM{{\tens{M}}}
\def\tN{{\tens{N}}}
\def\tO{{\tens{O}}}
\def\tP{{\tens{P}}}
\def\tQ{{\tens{Q}}}
\def\tR{{\tens{R}}}
\def\tS{{\tens{S}}}
\def\tT{{\tens{T}}}
\def\tU{{\tens{U}}}
\def\tV{{\tens{V}}}
\def\tW{{\tens{W}}}
\def\tX{{\tens{X}}}
\def\tY{{\tens{Y}}}
\def\tZ{{\tens{Z}}}


\def\gA{{\mathcal{A}}}
\def\gB{{\mathcal{B}}}
\def\gC{{\mathcal{C}}}
\def\gD{{\mathcal{D}}}
\def\gE{{\mathcal{E}}}
\def\gF{{\mathcal{F}}}
\def\gG{{\mathcal{G}}}
\def\gH{{\mathcal{H}}}
\def\gI{{\mathcal{I}}}
\def\gJ{{\mathcal{J}}}
\def\gK{{\mathcal{K}}}
\def\gL{{\mathcal{L}}}
\def\gM{{\mathcal{M}}}
\def\gN{{\mathcal{N}}}
\def\gO{{\mathcal{O}}}
\def\gP{{\mathcal{P}}}
\def\gQ{{\mathcal{Q}}}
\def\gR{{\mathcal{R}}}
\def\gS{{\mathcal{S}}}
\def\gT{{\mathcal{T}}}
\def\gU{{\mathcal{U}}}
\def\gV{{\mathcal{V}}}
\def\gW{{\mathcal{W}}}
\def\gX{{\mathcal{X}}}
\def\gY{{\mathcal{Y}}}
\def\gZ{{\mathcal{Z}}}

\def\sA{{\mathbb{A}}}
\def\sB{{\mathbb{B}}}
\def\sC{{\mathbb{C}}}
\def\sD{{\mathbb{D}}}
\def\sF{{\mathbb{F}}}
\def\sG{{\mathbb{G}}}
\def\sH{{\mathbb{H}}}
\def\sI{{\mathbb{I}}}
\def\sJ{{\mathbb{J}}}
\def\sK{{\mathbb{K}}}
\def\sL{{\mathbb{L}}}
\def\sM{{\mathbb{M}}}
\def\sN{{\mathbb{N}}}
\def\sO{{\mathbb{O}}}
\def\sP{{\mathbb{P}}}
\def\sQ{{\mathbb{Q}}}
\def\sR{{\mathbb{R}}}
\def\sS{{\mathbb{S}}}
\def\sT{{\mathbb{T}}}
\def\sU{{\mathbb{U}}}
\def\sV{{\mathbb{V}}}
\def\sW{{\mathbb{W}}}
\def\sX{{\mathbb{X}}}
\def\sY{{\mathbb{Y}}}
\def\sZ{{\mathbb{Z}}}

\def\emLambda{{\Lambda}}
\def\emA{{A}}
\def\emB{{B}}
\def\emC{{C}}
\def\emD{{D}}
\def\emE{{E}}
\def\emF{{F}}
\def\emG{{G}}
\def\emH{{H}}
\def\emI{{I}}
\def\emJ{{J}}
\def\emK{{K}}
\def\emL{{L}}
\def\emM{{M}}
\def\emN{{N}}
\def\emO{{O}}
\def\emP{{P}}
\def\emQ{{Q}}
\def\emR{{R}}
\def\emS{{S}}
\def\emT{{T}}
\def\emU{{U}}
\def\emV{{V}}
\def\emW{{W}}
\def\emX{{X}}
\def\emY{{Y}}
\def\emZ{{Z}}
\def\emSigma{{\Sigma}}

\newcommand{\etens}[1]{\mathsfit{#1}}
\def\etLambda{{\etens{\Lambda}}}
\def\etA{{\etens{A}}}
\def\etB{{\etens{B}}}
\def\etC{{\etens{C}}}
\def\etD{{\etens{D}}}
\def\etE{{\etens{E}}}
\def\etF{{\etens{F}}}
\def\etG{{\etens{G}}}
\def\etH{{\etens{H}}}
\def\etI{{\etens{I}}}
\def\etJ{{\etens{J}}}
\def\etK{{\etens{K}}}
\def\etL{{\etens{L}}}
\def\etM{{\etens{M}}}
\def\etN{{\etens{N}}}
\def\etO{{\etens{O}}}
\def\etP{{\etens{P}}}
\def\etQ{{\etens{Q}}}
\def\etR{{\etens{R}}}
\def\etS{{\etens{S}}}
\def\etT{{\etens{T}}}
\def\etU{{\etens{U}}}
\def\etV{{\etens{V}}}
\def\etW{{\etens{W}}}
\def\etX{{\etens{X}}}
\def\etY{{\etens{Y}}}
\def\etZ{{\etens{Z}}}

\newcommand{\pdata}{p_{\rm{data}}}
\newcommand{\ptrain}{\hat{p}_{\rm{data}}}
\newcommand{\Ptrain}{\hat{P}_{\rm{data}}}
\newcommand{\pmodel}{p_{\rm{model}}}
\newcommand{\Pmodel}{P_{\rm{model}}}
\newcommand{\ptildemodel}{\tilde{p}_{\rm{model}}}
\newcommand{\pencode}{p_{\rm{encoder}}}
\newcommand{\pdecode}{p_{\rm{decoder}}}
\newcommand{\precons}{p_{\rm{reconstruct}}}

\newcommand{\laplace}{\mathrm{Laplace}} 

\newcommand{\E}{\mathbb{E}}
\newcommand{\Ls}{\mathcal{L}}
\newcommand{\R}{\mathbb{R}}
\newcommand{\emp}{\tilde{p}}
\newcommand{\lr}{\alpha}
\newcommand{\reg}{\lambda}
\newcommand{\rect}{\mathrm{rectifier}}
\newcommand{\softmax}{\mathrm{softmax}}
\newcommand{\sigmoid}{\sigma}
\newcommand{\softplus}{\zeta}
\newcommand{\KL}{D_{\mathrm{KL}}}
\newcommand{\Var}{\mathrm{Var}}
\newcommand{\standarderror}{\mathrm{SE}}
\newcommand{\Cov}{\mathrm{Cov}}
\newcommand{\Proj}{\mathrm{Proj}}
\newcommand{\Alg}{\mathrm{Alg}}
\newcommand{\vectorize}{\mathrm{vec}}

\newcommand{\normlzero}{L^0}
\newcommand{\normlone}{L^1}
\newcommand{\normltwo}{L^2}
\newcommand{\normlp}{L^p}
\newcommand{\normmax}{L^\infty}

\newcommand{\parents}{Pa} 

\DeclareMathOperator*{\argmax}{arg\,max}
\DeclareMathOperator*{\argmin}{arg\,min}

\DeclareMathOperator{\sign}{sign}
\DeclareMathOperator{\Tr}{Tr}
\let\ab\allowbreak
 
\def\comma{{ \text{ ,} }}
\def\period{{ \text{ .} }}

\def\intT{{ \int_{t_0}^{t_1} }}
\def\rintT{{ \int^{t_0}_{t_1} }}
\def\wt{{ \mathbf{W}_t }}
\def\ws{{ \mathbf{W}_s }}
\def\dwt{{ \mathrm{d} \wt }}
\def\dws{{ \mathrm{d} \ws }}


\def\hvx{{ \bar{\vx} }}
\def\hvu{{ \bar{\vu} }}
\def\hrvx{{ \bar{\rvx} }}
\def\hrvu{{ \bar{\rvu} }}
\def\hrvxt{{ \bar{\rvx}_t }}
\def\hrvut{{ \bar{\rvu}_t }}


\def\hWt{{ \overline{\mathbf{W}}_t }}
\def\drvx{{\delta \rvx}}
\def\drvu{{\delta \rvu}}
\def\drvxt{{\delta \rvx_t}}
\def\drvut{{\delta \rvu_t}}
\def\Lx{{\ell_{\hvx} }}
\def\Lu{{\ell_{\hvu} }}
\def\Lxx{{{\ell}_{\hvx \hvx}}}
\def\Luu{{{\ell}_{\hvu \hvu}}}
\def\Lux{{{\ell}_{\hvu \hvx}}}
\def\Lxu{{{\ell}_{\hvx \hvu}}}
\def\Vx{{V_{\rvx} }}
\def\Vxx{{V_{\rvx\rvx} }}


\def\hS{{\bar{\sigma}}}
\def\hSu{{\bar{\sigma}_\rvu}}
\def\hSx{{\bar{\sigma}_\rvx}}
\def\hST{{\bar{\sigma}^\T}}
\def\hSuT{{\bar{\sigma}^\T_\rvu}}
\def\hSxT{{\bar{\sigma}^\T_\rvx}}
\def\hF{{\bar{F}}}
\def\hFu{{\bar{F}_\rvu}}
\def\hFx{{\bar{F}_\rvx}}
\def\Fu{{{F}_\hvu}}
\def\Fx{{{F}_\hvx}}
\def\FxT{{{F}_\hvx^\T}}
\def\FuT{{{F}_\hvu^\T}}

\def\Gu{{{G}_\rvu}}
\def\Gx{{{G}_\rvx}}
\def\hG{{\bar{G}}}
\def\hGu{{\bar{G}_\rvu}}
\def\hGx{{\bar{G}_\rvx}}
\def\hGT{{\bar{G}^\T}}
\def\hGuT{{\bar{G}^\T_\rvu}}
\def\hGxT{{\bar{G}^\T_\rvx}}

\def\Su{{{\sigma}_\rvu}}
\def\Sx{{{\sigma}_\rvx}}
\def\SxT{{{\sigma}_\rvx^\T}}
\def\SuT{{{\sigma}_\rvu^\T}}

\def\QxT{{Q^\T_{\hvx}}}
\def\QuT{{Q^\T_{\hvu}}}
\def\Qx{{Q_{\hvx}}}
\def\Qu{{Q_{\hvu}}}
\def\Qxu{{Q_{\hvx \hvu}}}
\def\Qxx{{Q_{\hvx \hvx}}}
\def\Quu{{Q_{\hvu \hvu}}}
\def\Qux{{Q_{\hvu \hvx}}}
\def\srd{{\circ\rd}}
\newcommand{\sfracdiff}[2]{\frac{\srd #1}{\rd  #2}}
\def\htheta{{\hat{\theta}}}


\newcommand{\norm}[1]{\lVert#1\rVert}

\DeclareMathOperator{\Linear}{Linear}
\DeclareMathOperator{\ReLU}{ReLU}
\DeclareMathOperator{\Tanh}{Tanh}
\DeclareMathOperator{\Sigmoid}{Sigmoid}
\DeclareMathOperator{\CurvApprox}{CurvApprox}
\DeclareMathOperator{\EKFACDDP}{EKFAC-DDP}
\DeclareMathOperator{\KFACDDP}{KFAC-DDP}
\DeclareMathOperator{\diag}{diag}

\def\ttranspose{{t \text{ } \transpose}}
\def\vxr{{\vx_r}}

\def\kt{{\vk}}
\def\Kt{{\mK}}
\def\Gt{{\mathbf{G}_t}}

\def\KtT{{\mathbf{K}_t^\transpose}}
\def\GtT{{\mathbf{G}_t^\transpose}}


\def\It{{\mathbf{I}_t}}
\def\Lt{{\mathbf{L}_t}}
\def\Ht{{\mathbf{H}_t}}

\def\kut{{\tilde{\mathbf{k}}_t}}
\def\Kut{{\tilde{\mathbf{K}}_t}}
\def\Gut{{\tilde{\mathbf{G}}_t}}

\def\Ivt{{\tilde{\mathbf{I}}_t}}
\def\Lvt{{\tilde{\mathbf{L}}_t}}
\def\Hvt{{\tilde{\mathbf{H}}_t}}

\def\QuuC{{\tilde{Q}_{\vu \vu}}}
\def\QvvC{{\tilde{Q}_{\vv \vv}}}
\def\QuutC{{\tilde{Q}^t_{\vu \vu}}}
\def\QvvtC{{\tilde{Q}^t_{\vv \vv}}}
\def\QuuCInv{{\tilde{Q}^{-1}_{\vu \vu}}}
\def\QvvCInv{{\tilde{Q}^{-1}_{\vv \vv}}}
\def\QA{{Q_{\vu \vx_r}}}
\def\QB{{Q_{\vv \vx}}}
\def\QAt{{Q^t_{\vu \vx_r}}}
\def\QBt{{Q^t_{\vv \vx}}}

\def\Axx{{\E[\vx_{\vu}\vx_{\vu}^\transpose]}}
\def\Axy{{\E[\vx_{\vu}\vx_{\vv}^\transpose]}}
\def\Ayy{{\E[\vx_{\vv}\vx_{\vv}^\transpose]}}
\def\Ayx{{\E[\vx_{\vv}\vx_{\vu}^\transpose]}}
\def\Bxx{{\E[\vg_{\vu}\vg_{\vu}^\transpose]}}
\def\Bxy{{\E[\vg_{\vu}\vg_{\vv}^\transpose]}}
\def\Byy{{\E[\vg_{\vv}\vg_{\vv}^\transpose]}}
\def\Byx{{\E[\vg_{\vv}\vg_{\vu}^\transpose]}}

\def\Auu{{A_{\vu\vu}}}
\def\Auv{{A_{\vu\vv}}}
\def\Avu{{A_{\vv\vu}}}
\def\Avv{{A_{\vv\vv}}}
\def\Buu{{B_{\vu\vu}}}
\def\Buv{{B_{\vu\vv}}}
\def\Bvu{{B_{\vv\vu}}}
\def\Bvv{{B_{\vv\vv}}}
\def\AuvT{{A^\transpose_{\vu\vv}}}
\def\BuvT{{B^\transpose_{\vu\vv}}}
\def\AvuT{{A^\transpose_{\vv\vu}}}
\def\BvuT{{B^\transpose_{\vv\vu}}}
\def\AuuInv{{A^{-1}_{\vu\vu}}}
\def\AvvInv{{A^{-1}_{\vv\vv}}}
\def\BuuInv{{B^{-1}_{\vu\vu}}}
\def\BvvInv{{B^{-1}_{\vv\vv}}}
\def\AuuCInv{{\tilde{A}^{-1}_{\vu\vu}}}
\def\BuuCInv{{\tilde{B}^{-1}_{\vu\vu}}}
\def\AvvCInv{{\tilde{A}^{-1}_{\vv\vv}}}
\def\BvvCInv{{\tilde{B}^{-1}_{\vv\vv}}}
\def\AvvInvT{{A^{-\transpose}_{\vv\vv}}}
\def\AuuCInvT{{\tilde{A}^{-\transpose}_{\vu\vu}}}

\def\VxN{{\mathbf{V}_\mX^T}}
\def\VxxN{{\mathbf{V}_{\mX\mX}^T}}

\def\VX{{\mathbf{V}_{\mX}}}
\def\VXX{{\mathbf{V}_{\mX\mX}}}

\def\QQXt{{\mathbf{Q}^t_\mX}}
\def\QQut{{\mathbf{Q}^t_\vu}}
\def\QQXXt{{\mathbf{Q}^t_{\mX\mX}}}
\def\QQuut{{\mathbf{Q}^t_{\vu\vu}}}
\def\QQuXt{{\mathbf{Q}^t_{\vu\mX}}}
\def\QQuuInvt{{[\mathbf{Q}^{t}_{\vu\vu}]^{-1}}}
\def\VXt{{\mathbf{V}^t_{\mX}}}
\def\VXXt{{\mathbf{V}^t_{\mX\mX}}}


\def\xxi{{\vx_t^{(i)}}}

\def\dvx{{\delta\vx}}
\def\dvu{{\delta\vu}}
\def\dvv{{\delta\vv}}
\def\vxTraj{{\bar{\vx}}}
\def\vuTraj{{\bar{\vu}}}

\def\Inv{{-1}}

\def\ConvT{{\text{ } \hat{*} \text{ }}}
\def\eqConv{{\text{ }\text{ } \stackrel{\mathclap{\tiny\mbox{conv}}}{=}\text{ }\text{ } }}

\def\Vh{{V_{\vh}}}
\def\Vht{{V^{t}_{\vh}}}
\def\Vhht{{V^{t}_{\vh\vh}}}



\def\fx{{{f}_{\vx}}}
\def\fu{{{f}_{\vu}}}
\def\fxt{{{f}^t_{\vx}}}
\def\fut{{{f}^t_{\vu}}}
\def\fuT{{{f}_{\vu}^\transpose}}
\def\fxT{{{f}_{\vx}^\transpose}}
\def\futT{{{{f}^t_{\vu}}^\transpose}}
\def\fxtT{{{{f}^t_{\vx}}^\transpose}}
\def\fxx{{{f}_{\vx \vx}}}
\def\fuu{{{f}_{\vu \vu}}}
\def\fux{{{f}_{\vu \vx}}}
\def\fxu{{{f}_{\vx \vu}}}
\def\fxxt{{{f}^t_{\vx \vx}}}
\def\fuut{{{f}^t_{\vu \vu}}}
\def\fuxt{{{f}^t_{\vu \vx}}}
\def\fxut{{{f}^t_{\vx \vu}}}

\def\gutT{{{g}^{t \text{ }\transpose}_{\vu}}}
\def\gxtT{{{g}^{t \text{ }\transpose}_{\vx}}}
\def\gut{{{g}^{t}_{\vu}}}
\def\gxt{{{g}^{t}_{\vx}}}
\def\sht{{{\sigma}^{t}_{\vh}}}
\def\shtT{{{\sigma}^{t \text{ }\transpose}_{\vh}}}

\def\lxt{{{\ell}^t_{\vx}}}
\def\lut{{{\ell}^t_{\vu}}}
\def\lxxt{{{\ell}^t_{\vx \vx}}}
\def\luut{{{\ell}^t_{\vu \vu}}}
\def\luxt{{{\ell}^t_{\vu \vx}}}
\def\lxut{{{\ell}^t_{\vx \vu}}}


\def\lx{{{\ell}_{\vx}}}
\def\lu{{{\ell}_{\vu}}}
\def\lxx{{{\ell}_{\vx \vx}}}
\def\luu{{{\ell}_{\vu \vu}}}
\def\lux{{{\ell}_{\vu \vx}}}
\def\lxu{{{\ell}_{\vx \vu}}}

\def\gx{{{g}_{\vx}}}
\def\gu{{{g}_{\vu}}}
\def\gxx{{{g}_{\vx \vx}}}
\def\guu{{{g}_{\vu \vu}}}
\def\gux{{{g}_{\vu \vx}}}
\def\gxu{{{g}_{\vx \vu}}}

\def\gxxt{{{g}^t_{\vx \vx}}}
\def\guut{{{g}^t_{\vu \vu}}}
\def\guxt{{{g}^t_{\vu \vx}}}
\def\gxut{{{g}^t_{\vx \vu}}}


\def\Dxu{{\nabla_{\vu \vx}}}
\def\llxu{{{\ell}(\vx^{(i)},\vu)}}
\def\ffxu{{{f}(\vx^{(i)},\vu)}}



\def\VH{{\mathbf{V}^t_{\mH}}}
\def\VhN{{\mathbf{V}_\mH^T}}
\def\VhhN{{\mathbf{V}_{\mH\mH}^T}}
\def\VHNxt{{\mathbf{V}^\prime_{\mH^\prime}}}
\def\VHHNxt{{\mathbf{V}^t_{\mH\mH}}}


\def\QQX{{\mathbf{Q}_\mX^t}}
\def\QQu{{\mathbf{Q}_\vu^t}}
\def\QQXX{{\mathbf{Q}^t_{\mX\mX}}}
\def\QQuu{{\mathbf{Q}^t_{\vu\vu}}}
\def\QQuX{{\mathbf{Q}^t_{\vu\mX}}}
\def\QQuuInv{{\mathbf{Q}^{-1}_{\vu\vu}}}
\def\QQuuInvApprox{{\mathbf{\widetilde{Q}}^{-1}_{\vu\vu}}}
\def\QQuXT{{{\mathbf{Q}^{t\transpose}_{\vu\mX}}}}
\def\VXNxt{{\mathbf{V}^\prime_{\mX^\prime}}}
\def\VXXNxt{{\mathbf{V}^\prime_{\mX^\prime\mX^\prime}}}

\def\QQXprevT{{\mathbf{Q}_\mX^{T-1}}}
\def\QQuprevT{{\mathbf{Q}_\vu^{T-1}}}
\def\QQXXprevT{{\mathbf{Q}^{T-1}_{\mX\mX}}}
\def\QQuuprevT{{\mathbf{Q}^{T-1}_{\vu\vu}}}
\def\QQuXprevT{{\mathbf{Q}^{T-1}_{\vu\mX}}}

\def\VX{{\mathbf{V}_{\mX}}}
\def\VXX{{\mathbf{V}_{\mX\mX}}}

\def\QQXt{{\mathbf{Q}^t_\mX}}
\def\QQut{{\mathbf{Q}^t_\vu}}
\def\QQXXt{{\mathbf{Q}^t_{\mX\mX}}}
\def\QQuut{{\mathbf{Q}^t_{\vu\vu}}}
\def\QQuXt{{\mathbf{Q}^t_{\vu\mX}}}
\def\QQuuInvt{{(\mathbf{Q}^{t}_{\vu\vu})^{-1}}}
\def\VXt{{\mathbf{V}^t_{\mX}}}
\def\VXXt{{\mathbf{V}^t_{\mX\mX}}}

\def\idxFu{{(B_{n_{t^\prime}}^{(i)},:)}}
\def\idxFuu{{(B_{n_{t^\prime}}^{(i)},:,:)}}
\def\idxFx{{(B_{n_{t^\prime}}^{(i)},B_{n_{t}}^{(j)})}}
\def\idxFux{{(B_{n_{t^\prime}}^{(i)},:,B_{n_{t}}^{(j)})}}
\def\idxFxx{{(B_{n_{t^\prime}}^{(i)},B_{n_{t}}^{(j)},B_{n_{t}}^{(k)})}}


\def\cspace{{\mathbb{R}^{m}}}
\def\Cspace{{\mathbb{R}^{\bar{m}}}}
\def\Xspace{{\mathbb{R}^{n}}}
\def\xyspace{{\mathbb{R}^{n+d}}}
\def\yspace{{\mathbb{R}^{d}}}
\def\expdnn{{\mathbb{E}_{\substack{ (\rvx_t, \rvy) \sim \mu_t \\ \text{subject to (\ref{eq:fc-dnn-dynamics})}}}}}

\def\transpose{{\mathsf{T}}}


\def\gradsampleeq{{
    \nabla_\vtheta l ( f ( \rvx^{(i)}, \vtheta ), \rvy^{(i)} )
}}
\def\gradsample{{ g^{i}(\vtheta) }}
\def\gradmb{{ g^{mb}(\vtheta) }}
\def\gradfull{{ g^{}(\vtheta) }}
\def\gradmbt{{   g^{mb}(\vtheta_{t}) }}
\def\gradfullt{{ g^{}(\vtheta_{t}) }}
\def\diffusionmatrixx{{ \mathbf{\Sigma}_{\mathcal{D}} }}
\def\diffusionmatrix{{ \mathbf{\Sigma}_{\mathcal{D}}(\vtheta) }}
\def\diffusionmatrixt{{ \mathbf{\Sigma}_{\mathcal{D}}(\vtheta_{t}) }}
\def\diffusionmatrixthaft{{ \mathbf{\Sigma}^{\frac{1}{2}}_{\mathcal{D}}(\vtheta_{t}) }}
\def\diffusionmatrixthaftt{{ \mathbf{\Sigma}^{\frac{1}{2}}_{\mathcal{D}} }}
\def\diffusionmatrixthafttt{{ \tilde{\mathbf{\Sigma}}^{\frac{1}{2}}_{\mathcal{D}} }}
\def\diffusionmatrixttt{{ \tilde{\mathbf{\Sigma}}_{\mathcal{D}} }}
\def\diffusionmatrixbatch{{ \tilde{\mathbf{\Sigma}}_{\mathcal{B}} }}
\def\absB{{ \mid \mathcal{B} \mid }}
\def\absD{{ \mid \mathcal{D} \mid }}
\def\drift{{ b(\vtheta_{t}) }}
\def\diffusion{{ \sigma(\vtheta_{t}) }}
\def\dt{{ \mathrm{d} t }}
\def\ds{{ \mathrm{d} s }}

\def\dLat{{ \mathrm{d} L_t^{\alpha} }}
\def\tt{{ \vtheta_{t} }}

\def\pss{{ \rho^{\mathrm{ss}} }}
\def\psss{{ \rho^{\mathrm{ss}}(\vtheta) }}

\def\Dk{{ \mathrm{D}^k }}
\def\gmbupdate{{ \vtheta^{\prime} - \eta g^{mb}(\vtheta^{\prime}) }}
\def\gmbupdatee{{ \vtheta - \eta g^{mb}(\vtheta) }}
\def\deltaa{{ \delta\left\{ \vtheta - \left[\gmbupdate\right] \right\} }}
\def\thprime{{ \vtheta^{\prime} }}
\def\dthprime{{ \mathrm{d} \thprime }}
\def\dth{{ \mathrm{d} \vtheta }}
\def\expB{{   \mathbb{E}_{\mathcal{B}} }}
\def\expPss{{ \mathbb{E}_{\pss} }}
\def\smallint{ \int}
\def\smallsum{\begingroup\textstyle \sum \endgroup}

\def\expBof#1{ \expB \left[ {#1} \right]}
\def\expPssof#1{ \expPss \left[ {#1} \right]}
\def\Phigmd{{ \Phi \left(\gmbupdatee\right) }}

\def\itoarg{{\left( X_t, t \right) }}



\def\thetap{{ \vtheta^{\prime} }}
\def\LocalEntropy{{ - \log \int_{\thetap \in \cspace} \exp (-\Phi (\thetap)-\frac{\gamma}{2}\|\vtheta-\thetap\|_{2}^{2}) \mathrm{d} \thetap }}
\def\LocalEntropyy{{ - \log \int_{\thetap} \exp \left(-\Phi (\thetap)-\frac{\gamma}{2}\|\vtheta-\thetap\|_{2}^{2}\right) \mathrm{d} \thetap }}
\def\ExpOf#1{ \mathbb{E} \left[ {#1} \right]}
\def\ExpOfXi#1{ \mathbb{E}_\xi \left[ {#1} \right]}
\def\ExpOfGibbs#1{ \mathbb{E}_{ \mathcal{P}_\gamma} \left[ {#1} \right]}
\def\fracK {{\frac{1}{k}}}

\def\metanext{{ \vtheta_{\text{adapt}}^{n+1} }}
\def\metacurr{{ \vtheta_{\text{adapt}}^{n}   }}


\def\calA{{\cal A}}
\def\calB{{\cal B}}
\def\calC{{\cal C}}
\def\calD{{\cal D}}
\def\calE{{\cal E}}
\def\calF{{\cal F}}
\def\calG{{\cal G}}
\def\calH{{\cal H}}
\def\calI{{\cal I}}
\def\calJ{{\cal J}}
\def\calK{{\cal K}}
\def\calL{{\cal L}}
\def\calM{{\cal M}}
\def\calN{{\cal N}}
\def\calO{{\cal O}}
\def\calP{{\cal P}}
\def\calQ{{\cal Q}}
\def\calR{{\cal R}}
\def\calS{{\cal S}}
\def\calT{{\cal T}}
\def\calU{{\cal U}}
\def\calV{{\cal V}}
\def\calW{{\cal W}}
\def\calX{{\cal X}}
\def\calY{{\cal Y}}
\def\calZ{{\cal Z}}

\newcommand{\nicefracpartial}[2]{\nicefrac{\partial #1}{\partial  #2}}
\newcommand{\nicefracdiff}[2]{\nicefrac{\rd #1}{\rd  #2}}
\newcommand{\fracpartial}[2]{\frac{\partial #1}{\partial  #2}}
\newcommand{\fracdiff}[2]{\frac{\rd #1}{\rd  #2}}
\newcommand{\br}[1]{\left[#1\right]}
\newcommand{\pr}[1]{\left(#1\right)}
\newcommand{\T}{\mathsf{T}}
\newcommand*\bvec[1]{\begin{bmatrix}#1\end{bmatrix}}
\newcommand{\bmat}[4]{\begin{bmatrix} #1 & #2 \\ #3 & #4\end{bmatrix}}
\def\vec{{\mathrm{vec}}}
\def\Hess{{\mathrm{Hess}}}
 \newcommand{\eq}[1]{{Eq.~(#1)}}
\newcommand{\eg}{{\ignorespaces\emph{e.g.}}{ }}
\newcommand{\ie}{{\ignorespaces\emph{i.e.}}{ }}
\newcommand{\wlg}{{\ignorespaces\emph{w.l.o.g.}}{ }}
\newcommand{\cf}{{\ignorespaces\emph{c.f.}}{ }}
\newcommand{\tmp}{{\ignorespaces(x)}{ }}
\newcommand{\resp}{{\ignorespaces\emph{resp.}}{ }}

\usepackage{mathtools}
\usepackage{tabularx}
\usepackage{framed}

\usepackage{amssymb}
\usepackage{amsthm}
\usepackage{multirow}

\newtheorem{theorem}{Theorem}
\newtheorem{lemma}[theorem]{Lemma}
\newtheorem{proposition}[theorem]{Proposition}
\newtheorem{corollary}[theorem]{Corollary}
\newtheorem{definition}[theorem]{Definition}
\newtheorem{remark}[theorem]{Remark}



\newcommand\numberthis{\addtocounter{equation}{1}\tag{\theequation}}
\usepackage{cancel}
\usepackage{lipsum}


\newcommand{\todo}[1]{{\tiny \color{red} TODO: #1}}






\usepackage{capt-of}
\usepackage{wrapfig}

\usepackage{xcolor}
\usepackage{color,soul}
\colorlet{color1}{green!50!black}
\colorlet{color2}{orange!95!black}
\colorlet{color3}{red!80!black}
\colorlet{color4}{red!65!black}
\colorlet{color5}{blue!75!green}
\colorlet{blueee}{blue!50!black}

\definecolor{label1}{HTML}{99292A}
\definecolor{label2}{HTML}{D89A3C}
\definecolor{label3}{HTML}{417481}
\colorlet{label22}{label2!80!black}

\definecolor{amaranth}{rgb}{0.9, 0.17, 0.31}

\newcommand{\markgreen}[1]{{\ignorespaces\color{color1} #1}}

\newcommand{\markblue}[1]{{\ignorespaces\color{color5} #1}}
\newcommand{\markred}[1]{\ignorespaces{\color{color3} #1}}

\newcommand{\markgray}[1]{\ignorespaces{\color{gray} #1}}

\newcommand{\markaa}[1]{\ignorespaces{\color{label1} #1}}
\newcommand{\markbb}[1]{\ignorespaces{\color{label22} #1}}
\newcommand{\markcc}[1]{\ignorespaces{\color{label3} #1}}


\let\svthefootnote\thefootnote
\usepackage{footnote}


\usepackage{pifont}
\usepackage{ifsym}
\newcommand{\cmark}{{\ding{51}}}\newcommand{\xmark}{{\ding{55}}}




\let\oldsqrt\sqrt
\def\sqrt{\mathpalette\DHLhksqrt}
\def\DHLhksqrt#1#2{\setbox0=\hbox{}\dimen0=\ht0
\advance\dimen0-0.2\ht0
\setbox2=\hbox{\vrule height\ht0 depth -\dimen0}{\box0\lower0.4pt\box2}}

\newcommand{\specialcell}[2][c]{\begin{tabular}[#1]{@{}c@{}}#2\end{tabular}}

\newcommand{\specialcellr}[2][r]{\begin{tabular}[#1]{@{}r@{}}#2\end{tabular}}

\newcommand{\specialcelll}[2][l]{\begin{tabular}[#1]{@{}l@{}}#2\end{tabular}}

\usepackage{enumitem}
\usepackage[position=top]{subfig}
\usepackage{arydshln}

\usepackage{algorithm}
\usepackage{algorithmic}
\newcommand*\mystrut[1]{\vrule width0pt height0pt depth#1\relax}

\usepackage{booktabs}       

\usepackage{enumitem}
\usepackage{tikz}
\newcommand*\numcircledmod[1]{\raisebox{.5pt}{\textcircled{\raisebox{-.9pt} {#1}}}}
\newcommand*\circled[1]{\tikz[baseline=(char.base)]{
            \node[shape=circle,draw,inner sep=2pt] (char) {#1};}}

\usepackage{empheq}
\newcommand*\widefbox[1]{\fbox{\hspace{1em}#1\hspace{1em}}}
\usepackage{tablefootnote}

\usepackage{textcomp}
\usepackage{colortbl}

\newcommand{\corcmidrule}[1][1pt]{\
    \rd \rvX_t = f(t, \rvX_t) \dt + g(t) \dwt, \quad \rvX_0 \sim \pdata,
    \label{eq:fsde}
  
\rd \rvX_t = [f - g^2~\nabla_\vx \log \pp{t}{\eqref{eq:fsde}}(\rvX_t) ] \dt + g~\dwt,
\quad \rvX_T \sim \pp{T}{\eqref{eq:fsde}},
\label{eq:rsde}

    &{\log \pp{0}{SGM}(\vx_0) \ge }\calL^{\text{}}_{\text{SGM}}(\vx_0; \theta)
    =          \E\br{\log p_T(\rvX_T)} - \int_0^T \E \br{
       \frac{1}{2}g^2\norm{\rvs_t}^2 + \nabla_\vx \cdot \pr{g^2\rvs_t - f}
    } \dt, \numberthis \label{eq:sgm-nll} \\
    &\qquad= \E\br{\log p_T(\rvX_T)} - \int_0^T \E \br{
       \frac{1}{2}g^2\norm{\rvs_t - \condscore}^2 - \frac{1}{2}\norm{g\condscore}^2 - \nabla_\vx \cdot f
    } \dt,

\rd \rvX_t = [f - g^2~\rvs(t,\rvX_t; \theta) ] \dt + g~\dwt,
\quad \rvX_T \sim \prior.
\label{eq:sample-sgm}

    \min_{\QQ \in \calP(\pdata, \prior)}
    \KL(\QQ~||~\PP),
    \label{eq:sb}

        \begin{cases}
        \fracpartial{\Psi}{t} = - \nabla_\vx \Psi^\T f {-} \frac{1}{2} \Tr(g^2\nabla^2_{\vx}\Psi) \
    Then, the solution to the optimization \eqref{eq:sb}
    can be expressed by the path measure of the following
    forward \eqref{eq:fsb}, or equivalently backward \eqref{eq:bsb}, SDE:
    
        \rd \rvX_t &= [f + g^2~\nabla_\vx \log {\Psi}(t,\rvX_t) ] \dt + g~\dwt,
        \quad \rvX_0 \sim \pdata,
        \label{eq:fsb}
        \\
        \rd \rvX_t &= [f - g^2~\nabla_\vx \log \widehat{\Psi}(t,\rvX_t) ] \dt + g~\dwt,
        \quad \rvX_T \sim \prior,
        \label{eq:bsb}
    
    where  and 
    are the optimal forward and backward drifts for SB.
\end{theorem}
Similar to the forward/backward processes in SGM,
the stochastic processes of SB in \eqref{eq:fsb} and \eqref{eq:bsb}
are also equivalent in the sense that
.
In fact, its marginal density obeys a factorization principle:
.


To construct the generative pipeline from \eqref{eq:bsb},
one requires solving the PDEs in \eqref{eq:sb-pde}
to obtain .
Unfortunately, these PDEs are hard to solve even for low-dimensional systems \citep{renardy2006introduction};
let alone for generative applications.
Indeed, previous works either have to replace the original Schr{\"o}dinger Bridge ()
with multiple stages, ,
so that each segment admits an analytic solution \citep{wang2021deep},
or consider the following half-bridge ( \textit{vs.} ) optimization \citep{de2021diffusion,vargas2021solving},

which can be solved with IPF algorithm \citep{kullback1968probability}
starting from .
In the following section,
we will present a scalable computational framework for
solving the optimality PDEs in \eqref{eq:sb-pde}
and show that it paves an elegant way connecting the optimality principle of SB \eqref{eq:sb-pde} to
the parameterized log-likelihood of SGM \eqref{eq:sgm-nll}.


 
\vspace{-1pt}
\section{Approach} \label{sec:3}
\vspace{-1pt}


\begin{figure}
  \vskip -0.2in
  \centering
  \includegraphics[width=\textwidth]{fig/sb3-crop.pdf}
  \caption{
    Schematic diagram of the our stochastic optimal control interpretation,
    and how it connects the objective of SGM \eqref{eq:sgm-nll}
    and optimality of SB \eqref{eq:sb-pde} through Forward-Backward SDEs theory.
  }
  \label{fig:2}
  \vskip -0.05in
\end{figure}


We motivate our approach starting from some control-theoretic observation (see Fig.~\ref{fig:2}).
Notice that both SGM and SB consist of forward and backward SDEs with similar structures.
From the stochastic control perspective, these SDEs belong to the class of
\emph{control-affine} SDEs with {additive} noise:

It is clear that the
control-affine SDE \eqref{eq:control-affine} includes all SDEs (\ref{eq:fsde},\ref{eq:rsde},\ref{eq:sample-sgm},\ref{eq:sb-sde}) appearing in \ref{sec:2}
by considering  and different interpretations of the \emph{control} variables .
This implies that
the optimization processes of
both SGM and SB can be aligned through the lens of \textit{{stochastic optimal control}} (SOC).
Indeed,
both problems can be interpreted as seeking some time-varying control policy,
either the score function  in SGM or  in SB,
that minimizes some objectives, \eqref{eq:sgm-nll} \textit{vs.} \eqref{eq:sb},
while subjected to some control-affine SDEs, (\ref{eq:fsde},\ref{eq:rsde}) \textit{vs.} \eqref{eq:sb-sde}.
In what follows, we will show that
a specific mathematical methodology in nonlinear SOC literature
--
called \textit{Forward-Backward SDEs} theory (FBSDEs; see \citet{ma1999forward}) --
links the optimality condition of SB~\eqref{eq:sb-pde}
to the log-likelihood objectives of SGM~\eqref{eq:sgm-nll}.
All proofs are left to Appendix~\ref{app:a}.

\vspace{-1pt}
\subsection{Forward-Backward SDEs (FBSDEs) Representation for SB} \label{sec:3.1}
\vspace{-1pt}

The theory of FBSDEs establishes an innate connection between
different classes of PDEs and forward-backward SDEs.
Below we introduce the following connection related to our problem.{
\begin{lemma}[Nonlinear Feynman-Kac;\footnote{
  Lemma~\ref{lemma:non-fc} can be viewed as the nonlinear extension of the celebrated Feynman-Kac formula \citep{karatzas2012brownian},
  which characterizes the connection between linear PDEs and forward SDEs.
} \citet{exarchos2018stochastic}]\label{lemma:non-fc}Consider the coupled SDEs
  [left={\empheqlbrace}]{align}
        \rd\rvX_t &=  f(t,\rvX_t) \dt + G(t,\rvX_t) \dwt, \qquad\qquad\quad\text{ }\text{ } \rvX_0 = \vx_0 \label{eq:fbsde-f} \\
        \rd\rvY_t &=  - h(t, \rvX_t, \rvY_t, \rvZ_t) \dt + \rvZ(t,\rvX_t)^\T \dwt, \quad \rvY_T = \varphi(\rvX_T) \label{eq:fbsde-b}
  
where the functions , , , and  satisfy proper regularity conditions\footnote{{
  \citet{yong1999stochastic,kobylanski2000backward} require
  , , , and  to be continuous,  and  to be uniformly Lipschitz in ,
  and  to satisfy quadratic growth condition in .
  \label{ft:cond}}
}
so that there exists a pair of unique strong solutions satisfying \eqref{eq:fbsde}.
Now, consider the following second-order parabolic PDE and suppose  is once continuously differentiable in  and twice in , \ie ,

then the solution to \eqref{eq:fbsde} coincides with the solution to \eqref{eq:hjb} along paths generated by the forward SDE \eqref{eq:fbsde-f} almost surely, i.e.,
the following stochastic representation (known as the nonlinear Feynman-Kac relation) is valid:

\end{lemma}}
Lemma~\ref{lemma:non-fc} states that
solutions to a certain class of nonlinear (via the function  in \eqref{eq:hjb}) PDEs
can be represented by solutions to a set of forward-backward SDEs \eqref{eq:fbsde}
through the transformation \eqref{eq:hjb-fbsde},
  and this relation can be extended to the viscosity case (\citet{pardoux1992backward}; see also {Appendix~\ref{app:a}}).
  Note that
   is the solution to the backward SDE \eqref{eq:fbsde-b} whose
  randomness is driven by the forward SDE \eqref{eq:fbsde-f}.
  Indeed, it is clear from \eqref{eq:hjb-fbsde} that  (hence also )
  is a time-varying function of  .
  Since the  appearing in the nonlinear Feynman-Kac relation \eqref{eq:hjb-fbsde}
  takes the random vector  as its argument,
   shall also be understood as a random variable.
  Finally, it is known that the original (deterministic) PDE solution  can be recovered by taking conditional expectation, \ie
  Since it is often computationally favorable to solve SDEs rather than PDEs,
Lemma~\ref{lemma:non-fc} has been widely used as a scalable method for solving high-dimensional PDEs \citep{han2018solving,pereira2019neural}.
Take SOC applications for instance,
their PDE optimality condition
can be characterized by \eqref{eq:hjb-fbsde} under proper conditions,
with the optimal control given in the form of .
Hence,
one can adopt Lemma~\ref{lemma:non-fc} to solve the underlying FBSDEs, rather than the original PDE optimality, for the optimal control.
Despite seemingly attractive,
whether these principles can be extended to SB,
whose optimality conditions are given by \textit{two coupled PDEs} in \eqref{eq:sb-pde},
remains unclear.
Below we derive a similar FBSDEs representation for SB.
\begin{theorem}[FBSDEs to SB optimality \eqref{eq:sb-pde}] \label{thm:3}
  Consider the following set of coupled SDEs,
  [left={\empheqlbrace}]{align}
      \rd \rvX_t &= \pr{f + g \rvZ_t} \dt + g \dwt \label{eq:fsde-is} \\
      \rd \rvY_t &= \frac{1}{2} \rvZ_t^\T\rvZ_t \dt + \rvZ_t^\T \dwt \label{eq:psi-bsde-is} \\
      \rd \widehat{\rvY}_t &= \pr{\frac{1}{2} \widehat{\rvZ}_t^\T\widehat{\rvZ}_t + \nabla_\vx \cdot (g\widehat{\rvZ}_t -f) + \widehat{\rvZ}_t^\T\rvZ_t } \dt + \widehat{\rvZ}_t^\T \dwt \label{eq:psi-hat-bsde-is}
  
  where  and  satisfy the same regularity conditions in Lemma~\ref{lemma:non-fc} (see Footnote~\ref{ft:cond}), and
  the boundary conditions are given by  and
  .
  Suppose ,
  then the nonlinear Feynman-Kac relations between the FBSDEs \eqref{eq:psi-hat-fbsde-is} and PDEs \eqref{eq:sb-pde} are given by
  
  Furthermore, 
  obey the following relation:
  
\end{theorem}
The FBSDEs for SB \eqref{eq:psi-hat-fbsde-is}
share a similar forward-backward structure as in \eqref{eq:fbsde},
where \eqref{eq:fsde-is} and (\ref{eq:psi-bsde-is},\ref{eq:psi-hat-bsde-is})
respectively represent the forward and backward SDEs.
One can verify that the forward SDE \eqref{eq:fsde-is} coincides with the \textit{optimal} forward SDE~\eqref{eq:fsb} with the substitution .
In other words,
these FBSDEs provide a \textit{local} representation of  and 
evaluated on the optimal path governed by \eqref{eq:fsb}.
  Since  and  can be understood as the forward/backward policies,
  in a similar spirit of policy-based methods \citep{pereira2020feynman,schulman2015trust},
  that guide the SDE processes of SB,
  they sufficiently characterize the SB model.
  Hence, our next step is to derive a proper training objective to optimize these policies.



\vspace{-2pt}
\subsection{Log-likelihood Computation of SB} \label{sec:3.2}
\vspace{-2pt}


\def\LSGM{{\calL^{\text{}}_{\text{\normalfont{SGM}}}}}
\def\LSB{{\calL^{\text{}}_{\text{\normalfont{SB}}}}}
\def\LSBB{{\widetilde{\calL}^{\text{}}_{\text{\normalfont{SB}}}}}

Theorem~\ref{thm:3} has an important implication:
It suggests that
given a path sampled from the forward SDE~\eqref{eq:fsde-is},
the solutions to the backward SDEs (\ref{eq:psi-bsde-is},\ref{eq:psi-hat-bsde-is}) at 
provide an unbiased estimation of the log-likelihood of the data point ,
\ie ,
where  is sampled from \eqref{eq:fsde-is}.
We now state our main result, which makes this observation formal:\begin{theorem}[Log-likelihood of SB model] \label{thm:4}
    Given the solution satisfying the FBSDE system in \eqref{eq:psi-hat-fbsde-is},
    the log-likelihood of the SB model , at a data point , can be expressed as
  
  where the expectation is taken over the forward SDE \eqref{eq:fsde-is} with the initial condition .
\end{theorem}
  Similar to \eqref{eq:sgm-nll}, Theorem~\ref{thm:4} suggests a parameterized lower bound to the log-likelihoods, \ie  where  shares the same expression in \eqref{eq:sb-nll} except that
   and  are approximated with some parameterized models (\eg DNNs).
  Note that  is \emph{intractable} in practice for any nontrivial .
  Hence, we use the divergence-based objective in \eqref{eq:sb-nll} as our training objective of both policies.

\textbf{Connection to score-based models.}
Recall Fig.~\ref{fig:2} and compare the parameterized log-likelihoods of SB \eqref{eq:sb-nll} to SGM \eqref{eq:sgm-nll};
one can verify that  collapses to  when
.
From the SB perspective, this occurs only when .
Since no effort is required in the forward process to reach ,
the optimal forward control , by definition, degenerates;
thereby making the backward control 
collapses to the score function.
However, in any case when ,
{for instance when the diffusion SDEs are improperly designed,}
the forward policy  steers the diffusion process
back to ,
while its backward counterpart  compensates the reversed process accordingly.
From this view,
SB
alleviates the problematic design in SGM by
enlarging the class of diffusion processes to accept \textit{nonlinear} drifts and
providing an optimization principle on learning these processes.
Moreover, Theorem~\ref{thm:4}
generalizes the log-likelihood training from SGM to SB.


\textbf{Connection to flow-based models.}
Interestingly,
the log-likelihood computation in Theorem~\ref{thm:4},
where we use a path  sampled from a data point  to parameterize its log-likelihood,
resembles modern training of (deterministic) flow-based models \citep{grathwohl2018ffjord}, which have recently been shown to admit a close relation to SGM \citep{song2020score,gong2021interpreting}.
The connection is built on the concept of \textit{probability flow}
-- which suggests that
the marginal density of an SDE can be evaluated through an
ordinary differential equation (ODE).
Below, we provide a similar flow representation for SB,
further strengthening their connection to modern generative models.
\begin{corollary}[Probability flow for SB] \label{coro:5}
  The following ODE
  characterizes the probability flow of the optimal processes of SB \eqref{eq:sb-sde}
  in the sense that .
  
\end{corollary}
One can verify (see Remark~\ref{remark:app-5} in \ref{app:a}) that computing the log-likelihood of this ODE model \eqref{eq-sb-prob}
using flow-based training techniques indeed recovers the training objective of SB derived in \eqref{eq:sb-nll}.

{
\vspace{-2pt}
\subsection{Practical Implementation} \label{sec:3.3}
\vspace{-2pt}

In this section, we detail the implementation of our FBSDE-inspired SB model, named \textbf{SB-FBSDE}.

\textbf{Training process.}
We treat the log-likelihood in \eqref{eq:sb-nll} as our training objective,
where the divergence can be can be estimated efficiently following \citet{hutchinson1989stochastic}.
This immediately distinguishes SB-FBSDE from prior SB models \citep{de2021diffusion,vargas2021solving}, which instead rely on regression-based objectives.\footnote{
  In fact, their regression targets may be recovered from \eqref{eq:sb-nll-bad} under proper transformation; see Appendix~\ref{app:d2}.
}
For low-dimensional datasets, we simply perform joint optimization, , to train the parameterized policies  and .
For higher-dimensional (\eg image) datasets, however, it can be prohibitively expensive to keep the entire computational graph.
In these cases, we follow \citet{de2021diffusion} by caching the sampled trajectories in a reply buffer and refreshing them
in a lower frequency basis (around 1500 iterations).
Although this implies that the gradient path w.r.t.  will be discarded,
we can leverage the symmetric structure of SB
and re-derive the log-likelihood for the sampled noise, \ie , based on the backward trajectories sampled from~\eqref{eq:bsb}.
We leave the derivation to Theorem~\ref{thm:9} in \ref{app:a} due to space constraint.
This results in an alternate training between the following two objectives after dropping all unrelated terms,
  
Our training process is summarized in Alg.~\ref{alg:train}.
While the joint training scheme in Alg.~\ref{alg:train2} resembles recent {diffusion} flow-based models \citep{zhang2021diffusion},
the alternate training in Alg.~\ref{alg:train3} relates to the classical IPF \citep{de2021diffusion}, despite differing in the underlying objectives.
Empirically, the joint training scheme can converge faster yet at the cost of introducing memory complexity.
We highlight these flexible training procedures arising from the unified viewpoint provided in Theorem~\ref{thm:4}.
Hereafter, we refer to
each cycle, \ie  training steps, in Alg.~\ref{alg:train3} as a \textit{training stage} of SB-FBSDE.

\begin{figure}[t]
\vskip -0.15in
\begin{minipage}[t]{0.48\textwidth}
\begin{algorithm}[H]
\small
     \caption{\small Likelihood training of SB-FBSDE}
     \label{alg:train}
  \begin{algorithmic}
   \STATE {\bfseries Input:}
      boundary distributions  and ,  \\
       parameterized policies  and 
   \REPEAT
     \IF{ memory resource is affordable }
       \STATE {\bfseries run} Algorithm~\ref{alg:train2}.
     \ELSE
       \STATE {\bfseries run} Algorithm~\ref{alg:train3}.
     \ENDIF
   \UNTIL{ converges }
  \end{algorithmic}
\end{algorithm}
\vskip -0.25in
\begin{algorithm}[H]
  \small
     \caption{\small Joint (diffusion flow-based) training}
     \label{alg:train2}
  \begin{algorithmic}
      \FOR{ {\bfseries to}  }
         \STATE Sample  from \eqref{eq:fsde-is} where  (computational graph retained).
         \STATE Compute  with \eqref{eq:sb-nll}.
         \STATE Update  with .
      \ENDFOR
  \end{algorithmic}
\end{algorithm}
\end{minipage}
\hfill
\begin{minipage}[t]{0.50\textwidth}
\begin{algorithm}[H]
\small
     \caption{\small Alternate (IPF-based) training}
     \label{alg:train3}
  \begin{algorithmic}
   \STATE {\bfseries Input:}
      Caching frequency 
     \FOR{ {\bfseries to}  }
       \IF{  == 0 }
         \STATE Sample  from \eqref{eq:fsde-is} where  (computational graph discarded).
       \ENDIF
       \STATE Compute  with \eqref{eq:high-dim-loss}.
       \STATE Update  with gradient .
     \ENDFOR
     \FOR{ {\bfseries to}  }
       \IF{  == 0 }
         \STATE Sample  from \eqref{eq:bsb} where  (computational graph discarded).
       \ENDIF
       \STATE Compute  with \eqref{eq:high-dim-loss2}.
       \STATE Update  with gradient .
     \ENDFOR
  \end{algorithmic}
\end{algorithm}
\end{minipage}
\vskip -0.15in
\end{figure}
}

\textbf{Generative process.}
While the generative processes for SB can be performed as simply as
propagating \eqref{eq:bsb} given the trained policy ,
it has been constantly observed
that adopting Langevin sampling to the generative process greatly improves performance \citep{song2020score}.
This procedure, often referred to as the \textit{Langevin corrector}, requires knowing
the score function .
For SB, we can estimate its value by recalling (see \ref{sec:2.2}) that
.
This results in the following predictor-corrector sampling procedure (see Alg.~\ref{alg:sample} in Appendix~\ref{Appendix:Exp_detals} for more details).

where  and  is the pre-specified noise scales (see \eqref{eq:noise-scale} in Appendix~\ref{Appendix:Exp_detals}).

\vspace{-1pt}

\textbf{Limitations \& efficiency.}
The main computational burden of our method comes from the computation of the divergence and maintaining two distinct networks.
Despite it typically increases the memory by 22.5 times compared to SGM,
we empirically observe that the divergence-based training converges much faster per iteration than standard regression.
As a result, SB-FBSDE admits comparable training time (+6.8\% in our CIFAR10 experiment) compared to SGM, yet with a substantially fewer sampling time (-80\%) due to adopting nonlinear SDEs.




 
\section{Experiments} \label{sec:4}


\begin{figure}[H]
  \vskip -0.28in
  \begin{minipage}{\textwidth}
    \centering
    \captionsetup[subfloat]{captionskip=1pt}
    \subfloat[GMM]{\includegraphics[width=0.5\textwidth]{fig/gmm.pdf}}
    \hfill
    \subfloat[Checkerboard]{\includegraphics[width=0.5\textwidth]{fig/check.pdf}}
    \vskip -0.1in
    \caption{
      Validation of our SB-FBSDE model on two synthetic toy datasets that represent continuous and discontinuous distributions.
      \textit{Upper}:
      \markblue{Generation ()} process with the
      \markblue{backward vector field }.
      \textit{Bottom}:
      \markred{Diffusion ()} process with the
      \markred{forward vector field }.
    }\label{fig:toy}\end{minipage}
  \vskip 0.05in
  \begin{minipage}{\textwidth}
    \centering
    \includegraphics[width=\textwidth]{fig/datasets_samples_new.pdf}
    \vskip -0.1in
    \caption{
        Uncurated samples from our SB-FBSDE models trained on MNIST (left), resized CelebA (middle) and CIFAR10 (right).
        More images can be found in Appendix~\ref{appendix:addtional_fig}.
    }
    \label{fig:all_datasets}
  \end{minipage}
  \vskip -0.1in
\end{figure}


\textbf{Setups.}
We testify SB-FBSDE on
two toy datasets and three image datasets, \ie
MNIST, CelebA,\footnote{
  We follow a similar setup of prior SB models \citep{de2021diffusion} and resize the image size to 32.
} and CIFAR10.
 is set to a zero-mean Gaussian whose variance varies for each task and can be computed according to \citet{song2020improved}.
We parameterize  and  with residual-based networks for toy datasets and consider
Unet \citep{ronneberger2015u} and NCSN++ \citep{song2020score} respectively for MNIST/CelebA and CIFAR10.
All networks adopt position encoding and are trained with AdamW \citep{loshchilov2017decoupled} on a TITAN RTX.
We adopt VE-SDE (\ie ; see \citet{song2020score})
as our SDE backbone,
which implies that in order to achieve reasonable performance,
SB must \textit{learn} a proper data-to-noise diffusion process.
On all datasets,
we set the horizon 1.0
and solve the SDEs via the Euler-Maruyama method.
The interval  is discretized into 200 steps for CIFAR10 and 100 steps for all other datasets,
which are much fewer than the ones in SGM (1000 steps).
Other details are left in Appendix~\ref{Appendix:Exp_detals}.

\textbf{Toy datasets.}
We first validate our joint optimization (\ie Alg~\ref{alg:train2}) on generating a mixture of Gaussian and checkerboard (adopted from \citet{grathwohl2018ffjord})
as the representatives of continuous and discontinuous distributions.
Figure~\ref{fig:toy}
shows how the learned policies, \ie  and ,
construct the vector fields that progressively transport
samples back-and-forth between  and .
The vector fields can be highly nonlinear and dissimilar to each other.
This resembles neither SGMs, whose forward vector field must obey linear structure,
nor flow-based models, whose vector fields are simply with opposite directions.
We highlight this as a distinct feature arising from SB models.



\textbf{Image datasets.}
Next, we validate our alternate training (\ie Alg~\ref{alg:train3}) on high-dimensional image generation.
The generated images for MNIST, CelebA, and CIFAR10 are presented in Fig.~\ref{fig:all_datasets},
which clearly suggest that our SB-FBSDE is able to synthesize high-fidelity images.
More uncurated images can be founded in Appendix~\ref{appendix:addtional_fig}.
Regarding the quantitative evaluation, Table~\ref{table:NLL_FID} summarizes
the negative log-likelihood (NLL; measured in bits/dim) and the
Fr\'echet Inception Distance score (FID; \citet{heusel2017gans}) on CIFAR10.
For our SB-FBSDE, we compute the NLL on the test set using Corollary~\ref{coro:5},
in a similar vein to SGMs and flow-based models,
and report the FID over 50k samples w.r.t the training set.
Notably, our SB-FBSDE achieves 2.98 bits/dim and 3.18 FID score on CIFAR10,
which is comparable to the top existing methods from other model classes (\eg SGMs) and
outperforms prior Optimal Transport (OT) methods \citep{wang2021deep,tanaka2019discriminator}
by a large margin in terms of the sample quality.
More importantly, it enables log-likelihood computations that are otherwise infeasible in prior OT methods.
We note that the quantitative comparisons on MNIST and CelebA are omitted
as the scores on these two datasets are not widely reported and different pre-processing (\eg resizing of CelebA) can lead to values that are not directly comparable.


\begin{figure}[t]
  \vskip -0.15in
  \begin{minipage}{\textwidth}
    \centering
    \captionsetup{type=table}
    \caption{
      CIFAR10 evaluation using negative log-likelihood (NLL; bits/dim) on the test set and sample quality (FID score) w.r.t. the training set.
      Our \textbf{SB-FBSDE} outperforms other optimal transport baselines by a large margin
      and is comparable to existing generative models.
    }
      \vskip -0.05in
      \begin{tabular}{llcccccccc}
        \toprule
        {Model Class} & {Method}  & {NLL } & {FID } \\

        \midrule

        \multirow{4}{*}{\specialcelll[l]{Optimal Transport }}
        & \textbf{SB-FBSDE (ours)}             & {2.96} &  3.01 \1pt]
        & Multi-stage SB \citep{wang2021deep}  &  -     & 12.32 \1pt]

        \midrule

        \multirow{5}{*}{\specialcelll[l]{SGMs}}
        & SDE (deep, sub-VP; \citet{song2020score})                  & 2.99         & 2.92  \1pt]
        & VDM \citep{kingma2021variational}             &\textbf{2.49} & 4.00  \1pt]







        \bottomrule
      \end{tabular} \label{table:NLL_FID}
  \end{minipage}
\end{figure}






\begin{figure}[t]
  \vskip -0.1in
  \begin{minipage}{0.43\textwidth}
    \centering
    \includegraphics[height=2.25cm]{fig/mnist-kl.pdf}
    \vskip -0.1in
    \caption{
      Validation of our SB-FBSDE on \textit{learning} forward diffusions
      that are closer (in KL sense) to  compared to SGM.
    }
    \label{fig:mnist-kl}
  \end{minipage}
  \hfill
  \begin{minipage}{0.55\textwidth}
    \centering
    \includegraphics[height=2.25cm]{fig/FID-both.pdf}
    \vskip -0.1in
    \caption{
        Ablation analysis where we show that adding Langevin corrector to SB-FBSDE
        uniformly improves the FID scores on both CelebA and CIFAR10 training.
    }
    \label{fig:fid}
  \end{minipage}
\vskip -0.1in
\end{figure}


\textbf{Validity of SB forward diffusion.}
Our theoretical analysis in \ref{sec:3.2}
suggests that the forward policy  plays
an essential role in governing samples towards .
Here, we validate this conjecture by computing the KL divergence between
the terminal distribution induced by ,
\ie ,
and the designated prior .
We refer readers to Appendix~\ref{Appendix:Exp_detals} for the actual computation.
Figure~\ref{fig:mnist-kl} reports these values over MNIST training.
For both degenerate () and linear () base drifts,
our SB-FBSDE generates terminal distributions that are much closer to .
Note that the values of SGM remain unchanged throughout training since SGM relies on \textit{pre-specified} diffusion.
This is in contrast to our SB-FBSDE whose forward policy 
gradually shortens the KL gap to ,
thereby providing a better forward diffusion for training the backward policy.




\textbf{Effect of Langevin corrector.}
In practice,
we observe that the Langevin corrector greatly affects the generative performance.
As shown in Fig.~\ref{fig:fid},
including these corrector steps uniformly improves the sample quality (FID) on both CelebA and CIFAR10 throughout training.
Since the SDEs are often solved via the Euler-Maruyama method,
their propagation can be subjected to discretization errors accumulated over time.
These Langevin steps thereby help
re-distributing the samples at each time step  towards the desired density .
We emphasize
this improvement
as the benefit gained from
applying modern generative training techniques based on
the solid connection between SB and SGM.














 

\vspace{-5pt}
\section{Conclusion}
\vspace{-5pt}
In this work, we present a novel computational framework, grounded on Forward-Backward SDEs theory,
for computing the log-likelihood of Schr{\"o}dinger Bridge (SB) --
a recently emerging model that adopts entropy-regularized optimal transport for generative modeling.
Our findings
provide new theoretical insights by generalizing previous theoretical results for Score-based Generative Model,
and facilitate applications of modern generative training for SB.
We validate our method on various image generative tasks, \eg MNIST, CelebA, and CIFAR10,
showing encouraging results in synthesizing high-fidelity samples
while retaining the rigorous mathematical framework.


\newpage

\section*{Acknowledgments}
The authors would like to thank Ioannis Exarchos and Oswin So for their generous involvement and helpful supports during the rebuttal. The authors would also like to thank Marcus A Pereira and Ethan N Evans for their participation and kind discussion in the early stage of project exploration.
This research was supported by the ARO Award \# W911NF2010151.

\section*{Author Contributions} \label{sec:author}
The original idea of solving the PDE optimality of SB with FBSDEs theory was initiated by Tianrong.
Later, Guan derived the main theories (\ie Theorem~\ref{thm:3},~\ref{thm:4},~\ref{thm:9}, and Corollary~\ref{coro:5})
presented in Section~\ref{sec:3.1}, \ref{sec:3.2} and Appendix~\ref{app:a} with few helps from Tianrong.
Tianrong designed the practical algorithms (\eg stage-wise optimization and Langevin-corrector) in Section~\ref{sec:3.3} and
conducted most experiments with few helps from Guan.
Guan wrote the main paper except for Section~\ref{sec:4},
which were written by both Tianrong and Guan.
Both Guan and Tianrong contributed to code development.


\section*{Reproducibility Statement}

Our training algorithms are detailed in Alg.~\ref{alg:train}, \ref{alg:train2}, and \ref{alg:train3}, with the training objectives given in the same section (see (\ref{eq:sb-nll}, \ref{eq:high-dim-loss}, \ref{eq:high-dim-loss2})). Other implementation details (\eg data pre-processing) are left in Appendix~\ref{Appendix:Exp_detals}. This shall provide sufficient information for readers of interests to reproduce our results. As we strongly believe in the merit of open sourcing, we intend to release our implementation upon publication. On the theoretical side, all proofs are left to Appendix~\ref{app:a} due to space constraint. We provide the assumptions in the same section.





\bibliographystyle{iclr2022_conference}
\bibliography{reference.bib}

\newpage
\appendix


\def\dWt{{\dwt}}


\section{Introduction of Schr{\"o}dinger Bridge}\label{app:d1}

    In this subsection, we provide a brief review for Schr{\"o}dinger Bridge (SB) and some reasonings for Theorem~\ref{thm:1}.
    The SB problem, at its classical form, considers the following optimization \citep{dai1991stochastic,pavon1991free},
    
    where .
    The optimization \eqref{eq:app-sb} characterizes a standard stochastic optimal control (SOC) programming with energy
    (\ie ) minimization except with an additional terminal boundary condition.
    The optimality conditions to \eqref{eq:app-sb} are given by
      [left={\empheqlbrace}]{align}
          \fracpartial{\psi}{t} &= - \frac{1}{2} \norm{\nabla_\vx\psi}^2 - \epsilon~\Delta\psi, \label{eq:app-sb2a} \\
          \fracpartial{p^*}{t} &= \nabla_\vx \cdot \pr{p^* \nabla_\vx \psi} + \epsilon~\Delta p^*, \label{eq:app-sb2b}
      
    where  is known as the \textit{value} function in SOC literature
    and  is the associated optimal marginal density.  denotes the Laplace operator.
    Equations \eqref{eq:app-sb2a} and \eqref{eq:app-sb2b}
    are respectively the Kolmogorov’s backward and forward PDEs,
    also known as Hamilton-Jacobi-Bellman and Fokker-Planck equations.
    The SB system can be obtained by applying the Hopf-Cole \citep{hopf1950partial,cole1951quasi} transformation ,
        3pt]
            \fracpartial{\widehat{\Psi}}{t} = \epsilon~\Delta\widehat{\Psi}
            \end{cases}
            \text{s.t. } \Psi(0,\cdot) \widehat{\Psi}(0,\cdot) = p_0,~\Psi(T,\cdot) \widehat{\Psi}(T,\cdot) = p_T.
            \label{eq:app-sb-pde2}
        
            \begin{split}
                &\qquad\quad\min_{\rvu}
                \E\br{\int_0^T \frac{1}{2} \norm{\rvu(t, \rvX_t)}^2} \\
                \text{s.t. }&
                \begin{cases}
                    \rd \rvX_t = [f(t, \rvX_t) + g(t)~\rvu(t, \rvX_t)] \dt + \sqrt{2\epsilon}g(t)~\dwt \\
                    \rvX_0\sim p_0(\rvX), \quad \rvX_T\sim p_T(\rvX)
                \end{cases},
            \end{split} \label{eq:app-sb3}
        
            \begin{cases}
            \fracpartial{\Psi}{t} = - \nabla_\vx \Psi^\T f {-} \epsilon \Tr(g^2\nabla^2_{\vx}\Psi) \
        with the same boundary conditions .
        The optimal control to \eqref{eq:app-sb3} is thereby given by
        
    \end{theorem}
    \begin{proof}
        See Section III and Theorem 2 inf \citet{caluya2021wasserstein}.
    \end{proof}

    Theorem~\ref{thm:10} is particularly attractive to us since its SDE corresponds exactly to the one appearing in score-based generative models.
    One can recover Theorem~\ref{thm:1} by
    \begin{enumerate}[leftmargin=20pt]
        \item[\textit{(i)}] Following \citet{pavon1991free}, we know that the objective in \eqref{eq:app-sb3} is equivalent to  by an application of Girsanov’s Theorem.
        \item[\textit{(ii)}] Equation \eqref{eq:app-sb-pde3} is exactly \eqref{eq:sb-pde} with . Furthermore, substituting the optimal control \eqref{eq:app-sb-opt} to the stochastic process in \eqref{eq:app-sb3} yields the optimal forward SDE in \eqref{eq:fsb}.
        \item[\textit{(iii)}] Finally, reversing the SDE \eqref{eq:fsb} from forward to backward following \citet{anderson1982reverse},
        
        and recalling the factorization principle,
        ,
        from Equation (4.15) in \citet{chen2021stochastic} yield the optimal backward SDE in \eqref{eq:bsb}.
    \end{enumerate}




\section{Proofs and Remarks in Section~\ref{sec:3}} \label{app:a}

In this section, we provide proofs for all of our theorems.
We following the same notation by denoting  as the marginal density driven by some  process  until the time step .
Gradient and Hessian of a function , where , will respectively be denoted as
 and .
Divergence and Laplace operators will respectively be denoted as
 and .
Note that .
For notational brevity, we will only keep the subscript  for multivariate functions.
Finally,
 denotes the trace of a square matrix .

We first restate the celebrated It{\^o} lemma, which is known as the extension of the chain rule of ordinary calculus to the stochastic setting. It relies on the fact that  and  are of the same scale and
keeps the expansion up to .
\begin{lemma}[It{\^o} formula; \citet{ito1951stochastic}] \label{lemma:ito}
  Let  and let  be the stochastic process satisfying
  
  Then, the stochastic process  is also an It{\^o} process satisfying
  
\end{lemma}

Next, the following lemma will be useful in proving Theorem~\ref{thm:3}.
\begin{lemma} \label{lemma:sbp}
  The following equality holds at any point  such that .
  
\end{lemma}
\begin{proof}
  
\end{proof}


{
\paragraph*{Assumptions}
Before stating our proofs, we provide the assumptions used throughout the paper.
These assumptions are adopted from stochastic analysis for SGM \citep{song2021maximum,yong1999stochastic,anderson1982reverse}, SB \citep{caluya2021wasserstein}, and FBSDE \citep{exarchos2018stochastic,gorodetsky2015efficient}.
\begin{enumerate}[label=(\roman*)]
  \item  with finite second-order moment. \label{assum:i}

  \item  and  are continuous functions, and  is uniformly lower-bounded w.r.t. .

  \item , we have
        , and 
        Lipschitz and at most linear growth w.r.t. .

  \item . , and  are continuous functions.  satisfies quadratic growth w.r.t.  uniformly in .


  \item  as . \label{assum:v}

\end{enumerate}
Assumptions (i) (ii) (iii) are standard conditions in stochastic analysis to ensure the existence-uniqueness of the SDEs; hence also appear in SGM analysis \citep{song2021maximum}.
Assumption (iv) allows applications of It{\^o} formula and properly defines the backward SDE in FBSDE theory.
Finally, assumption (v) assures the exponential limiting behavior when performing integration by parts.
}

Now, let us begin the proofs of Theorem~\ref{thm:3}, \ref{thm:4}, and Corollary~\ref{coro:5}.
\begin{reptheorem}{thm:3}[FBSDEs to SB optimality \eqref{eq:sb-pde}]
  Consider the following set of coupled SDEs,
  [left={\empheqlbrace}]{align}
      \rd \rvX_t &= \pr{f + g \rvZ_t} \dt + g \dwt  \\
      \rd \rvY_t &= \frac{1}{2} \rvZ_t^\T\rvZ_t \dt + \rvZ_t^\T \dwt  \\
      \rd \widehat{\rvY}_t &= \pr{\frac{1}{2} \widehat{\rvZ}_t^\T\widehat{\rvZ}_t + \nabla_\vx \cdot (g\widehat{\rvZ}_t -f) + \widehat{\rvZ}_t^\T\rvZ_t } \dt + \widehat{\rvZ}_t^\T \dwt
  
  where  and  satisfy the same regularity conditions in Lemma~\ref{lemma:non-fc} (see Footnote~\ref{ft:cond}), and
  the boundary conditions are given by  and
  .
  Suppose ,
  then nonlinear Feynman-Kac relations between the FBSDEs \eqref{eq:psi-hat-fbsde-is} and PDEs \eqref{eq:sb-pde} are given by
  
  Furthermore, 
  obey the following relation:
  
\end{reptheorem}
\begin{proof}
  Similar to how the original nonlinear Feynman-Kac (\ie Lemma~\ref{lemma:non-fc})
  can be carried out by an application of It{\^o} lemma \citep{ma1999forward}.
  We can apply It{\^o} lemma \ref{lemma:ito} to the stochastic process  w.r.t.
  the optimal forward SDE \eqref{eq:fsb}.
        
  From the PDE dynamics \eqref{eq:sb-pde}, we know that
        
  The \markgreen{first term} in the RHS can be readily canceled out with the related -term in \eqref{eq:ito-psi}. The \markaa{second term} can also be canceled out using the fact that
  .
  Hence, we are left with
        
  Likewise, applying It{\^o} lemma to , where  follows the SDE in \eqref{eq:fsb},
        
  but now noticing that the dynamics of  become
        
  Only the \markbb{first term} in the RHS will be canceled out in \eqref{eq:ito-psi2}. Hence, we are left with
        
  Notice that the trace terms above can be simplified to
        
    where the last equality follows by Lemma~\ref{lemma:sbp}. Substituting this result back to \eqref{eq:ito-psi22}, we get
        
  Finally, by rewriting \eqref{eq:thm3-Y} and \eqref{eq:ito-psi222} with the nonlinear Feynman-Kac in \eqref{eq:app-non-fc} yields
        [box=\widefbox]{align*}
          \rd \rvX_t &= \pr{f + g \rvZ_t} \dt + g \dwt  \\
          \rd \rvY_t &= \frac{1}{2} \rvZ_t^\T\rvZ_t \dt + \rvZ_t^\T \dwt  \\
          \rd \widehat{\rvY}_t &= \pr{\frac{1}{2} \widehat{\rvZ}_t^\T\widehat{\rvZ}_t + \nabla_\vx \cdot (g\widehat{\rvZ}_t -f) + \widehat{\rvZ}_t^\T\rvZ_t } \dt + \widehat{\rvZ}_t^\T \dwt
        
  This concludes the proof.
\end{proof}

\begin{remark}[Viscosity solutions]\normalfont
  These FBSDE results can be extended to viscosity solutions in the case when the classical solution does not exist \citep{pardoux1992backward}. For the completeness, one shall understand them in the sense of  uniformly in  over a compact set. Here  is the classical solution to \eqref{eq:hjb} with  converge uniformly toward  over the compact set. We refer readers of interests to \citet{exarchos2018stochastic,negyesi2021one}, and their references therein.
\end{remark}

\begin{reptheorem}{thm:4}[Log-likelihood for SB models]
    Given the solution satisfying the FBSDE system in \eqref{eq:psi-hat-fbsde-is},
    the log-likelihood of the SB model , at a data point , can be expressed as
  
  where the expectation is taken over the forward SDE \eqref{eq:fsde-is} with the initial condition .
\end{reptheorem}
\begin{proof}
    
    which recovers \eqref{eq:app-sb-nll}. Finally, notice that with integration by part, we have
    
    where we adopt common practice and assume the limiting behavior of ; in other words,
     as .
    With \eqref{eq:app-coro-proof2}, we can rewrite the related parts in \eqref{eq:coro-proof1} as
    
    Hence, we also recover \eqref{eq:app-sb-nll-bad}.
\end{proof}




 
\textbf{Corollary~\ref{coro:5}} (Probability flow for SB)\textbf{.}
  \textit{
  The following ODE
  characterizes the probability flow of the optimal processes of SB \eqref{eq:sb-sde}
  in the sense that .}
  
\begin{proof}
  The probability ODE flow \citep{song2020score,maoutsa2020interacting} suggests that
  the equivalent ODE model for the SDE \eqref{eq:fsde} is given by
  
  We can adopt this result to the SDEs of SB \eqref{eq:fsb} by considering
   and .
  This yields
  
  Applying the the factorization principle \citep{chen2021stochastic} with
   concludes the proof.
\end{proof}


\begin{remark}[Connection between SB-FBSDE and flow-based models]\label{remark:app-5}\normalfont
  To demonstrate how applying flow-based training techniques to the probability ODE flow of SB \eqref{eq:app-sb-prob2} recovers the same log-likelihood objective in \eqref{eq:app-sb-nll},
  recall that given an ODE  with ,
  flow-based models compute the change in log-density using the instantaneous change of variables formula \citep{chen2018neural}:
  
  which implies that the log-likelihood of  can be computed as
  
  Now, consider the probability ODE flow of SB in \eqref{eq:app-ode},
  
  Substituting this vector field  to \eqref{eq:cnf} yields
  
  where (*) is due to integration by parts (recall \eqref{eq:app-coro-proof2}) and (**) again uses the factorization principle .
  One can verify that \eqref{eq:sb-nll-ode} indeed recovers \eqref{eq:app-sb-nll}.
\end{remark}


\begin{theorem}[FBSDE computation for  in SB models] \label{thm:9}
  With the same regularity conditions in Theorem~\ref{thm:3},
  the following FBSDEs also satisfy the nonlinear Feynman-Kac relations in \eqref{eq:app-non-fc}.
  [left={\empheqlbrace}]{align}
      \rd \rvX_t &= \pr{f - g \widehat{\rvZ}_t} \dt + g \dwt \label{eq:psi-hat-fbsde-is3a} \\
      \rd \rvY_t &= -\pr{\frac{1}{2} {\rvZ}_t^\T{\rvZ}_t + \nabla_\vx \cdot (g{\rvZ}_t +f) + {\rvZ}_t^\T\widehat{\rvZ}_t } \dt + {\rvZ}_t^\T \dwt \\
      \rd \widehat{\rvY}_t &= -\frac{1}{2} \widehat{\rvZ}_t^\T\widehat{\rvZ}_t \dt + \widehat{\rvZ}_t^\T \dwt
  
  Given a backward trajectory sampled from \eqref{eq:psi-hat-fbsde-is3a},
  where  and ,
  the log-likelihood of  is given by
  . In particular,
  
\end{theorem}
\begin{proof}
  Due to the symmetric structure of SB, we can consider a new time coordinate
  
  Under this transformation, the base reference  appearing in \eqref{eq:sb} is equivalent to
  
  The corresponding PDE optimality becomes
  3pt]
      \fracpartial{\widehat{\Phi}}{s} = \nabla_\vx \cdot (\widehat{\Phi} f) {+} \frac{1}{2} \Tr(g^2\nabla^2_{\vx}\widehat{\Phi})
      \end{cases}
      \text{s.t. } \Phi(0,\cdot) \widehat{\Phi}(0,\cdot) = \prior,~\Phi(T,\cdot) \widehat{\Phi}(T,\cdot) = \pdata,
  
   \label{eq:app-sb-sde}
    \Phi(s, \rvX_s) = \widehat{\Psi}(T-t, \rvX_{T-t}) \qquad \text{and} \qquad
    \widehat{\Phi}(s, \rvX_s) = {\Psi}(T-t, \rvX_{T-t}).
    \label{eq:phi-to-psi}
  
  
  
    ({\rvY}^\prime_s, \widehat{\rvY}^\prime_s, \rvZ^\prime_s, \widehat{\rvZ}^\prime_s)
    = (\widehat{\rvY}^\prime_{T-t}, {\rvY}^\prime_{T-t}, \widehat{\rvZ}^\prime_{T-t}, \rvZ^\prime_{T-t}).
    \label{eq:phi-to-psi2}
  
        &\LSB(\vx_T) \\
        =& \E \br{\rvY_T + \widehat{\rvY}_T | \rvX_T=\vx_T} \\
        =& \E\br{{\rvY}_0 - \int_0^T \pr{\frac{1}{2} \norm{\widehat{\rvZ}_t}^2}\dt +
         \widehat{\rvY}_0 - \int_0^T \pr{\frac{1}{2} \norm{{\rvZ}_t}^2 + \nabla \cdot (g{\rvZ}_t +f) + {\rvZ}_t^\T\widehat{\rvZ}_t } \dt \Big| \rvX_T=\vx_T} \\
        =& \E \br{\rvY_0 + \widehat{\rvY}_0 | \rvX_T=\vx_T}
         - \int_0^T \E\br{ \frac{1}{2} \norm{\widehat{\rvZ}_t}^2 + \frac{1}{2} \norm{{\rvZ}_t}^2 + \nabla \cdot (g{\rvZ}_t +f) + {\rvZ}_t^\T\widehat{\rvZ}_t \Big| \rvX_T=\vx_T} \dt  \\
        =& \E [\log p_0(\rvX_0)]
         - \int_0^T {\E\br{ \frac{1}{2} \norm{\widehat{\rvZ}_t}^2 + \frac{1}{2} \norm{{\rvZ}_t}^2 + \nabla \cdot (g{\rvZ}_t +f) + {\rvZ}_t^\T\widehat{\rvZ}_t}} \dt. \numberthis \label{eq:coro-proof2}
    
        \rvX_{k+1}\sim\calN(F_k(\rvX_k), 2\gamma_{k+1}\mI), \quad \text{and} \quad
        \rvX_{k}\sim\calN(B_{k+1}(\rvX_{k+1}), 2\gamma_{k+1}\mI),
    
   \label{eq:dsm-loss}
         & \norm{B_{k+1}(\rvX_{k+1}) - (\rvX_{k+1} + F_k(\rvX_k) - F_{k}(\rvX_{k+1})}^2 \\
        =& \norm{\pr{\rvX_{k+1} + \gamma_{k+1}b_{k+1}(\rvX_{k+1})}
                 - \pr{\rvX_{k+1} + \rvX_k + \gamma_{k+1}f_k(\rvX_k) - \rvX_{k+1} - \gamma_{k+1}f_k(\rvX_{k+1})}}^2\\
        =& \norm{
            \underbrace{\gamma_{k+1}f_k(\rvX_{k+1})}_{\numcircledmod{1}}
          + \underbrace{\gamma_{k+1}b_{k+1}(\rvX_{k+1})}_{\numcircledmod{2}}
          - \underbrace{\pr{\rvX_k + \gamma_{k+1}f_k(\rvX_k) - \rvX_{k+1}}}_{\numcircledmod{3}}
          }^2, \numberthis
    
        \rd \rvX_t = f(t, \rvX_t) \dt + \sqrt{2\gamma}~\dwt,
    
        \rd \rvX_t = f(t, \rvX_t) \dt + g(t) \dwt,
        \label{eq:sgm-dyn}
    
        \rvX_{k+1} = \rvX_k + \gamma_{k+1} f(k, \rvX_k) + \sqrt{2\gamma_{k+1}}~\epsilon,
        \label{eq:dsb-dyn}
    1pt]
        & DOT \citep{tanaka2019discriminator}  &  -     & 15.78 \1pt]
        & DGflow \citep{ansari2020refining}    &  -     &  9.63 \1pt]
        & ScoreFlow \citep{song2021maximum}             &2.74 & 5.7  \1pt]
        & LSGM\citep{vahdat2021score}                   & 3.43         &\textbf{2.10} \1pt]
        & NVAE \citep{vahdat2020nvae}                   & 2.91         & 23.49 \1pt]

        \midrule

        \multirow{3}{*}{\specialcelll[l]{Flows}}
        & FFJORD \citep{grathwohl2018ffjord}            &  3.40        & - \1pt]
        & ANF  \citep{huang2020augmented}               &  3.05        & - \1pt]
        & StyleGAN2-ADA  \citep{karras2020training}     &   -          &  2.92 \1pt]

        \bottomrule
      \end{tabular} \label{table:app-NLL_FID}
  \end{minipage}
\end{figure}

\begin{figure}[H]
  \vskip -0.1in
  \begin{minipage}{\textwidth}
    \captionsetup{type=table}
    \caption{Training Hyper-parameters}
    \label{table:hyperparam}
      \centering
      \vskip -0.1in
      \centering
      \begin{tabular}{r|rrrr}
        \toprule
        Dataset & learning rate & time steps & batch size & variance of  \\
        \midrule
        Toy                             &2e-4           &100                 & 400 & 1.0\1pt]
        CelebA                          &2e-4           &100                 & 200 & 900.0\1pt]
        Mnist     &reduced Unet (1.95M) &  reduced Unet (1.95M)    \1pt]
        CIFAR10   & Unet (39.63M)   & NCSN++    (62.69M)           \\
        \bottomrule
      \end{tabular}
  \end{minipage}
\end{figure}

\textbf{Training.}
We use Exponential Moving Average (EMA) with the decay rate of 0.99.
Table~\ref{table:hyperparam} details the hyper-parameters used for each dataset.
{As mentioned in \citet{de2021diffusion},
the alternate training scheme may substantially accelerate the convergence
under proper initialization.
Specifically, when  is initialized with degenerate outputs (\eg by zeroing out its last layer),
training  at the first  steps can be made in a similar SGM fashion
{since  now admits analytical expression}.}
As for the proceeding stages, we resume to use (\ref{eq:high-dim-loss}, \ref{eq:high-dim-loss2}) since 
no longer have trivial outputs.


\textbf{Data pre-processing.}
MNIST is padded from 2828 to 3232 to prevent degenerate feature maps through Unet.
CelebA is resized to 33232 to accelerate training.
Both CelebA and CIFAR10 are augmented with random horizontal flips to enhance the diversity.


\begin{wrapfigure}[14]{r}{0.57\textwidth}
  \vspace{-10pt}
  \begin{minipage}{0.57\textwidth}
  \begin{algorithm}[H]
    \small
       \caption{\small Generative Process of SB-FBSDE}
       \label{alg:sample}
    \begin{algorithmic}
     \STATE {\bfseries Input:}
        , policies  and  \\
       \STATE Sample .
       \FOR{ {\bfseries to}  }
         \STATE Sample .
         \STATE Predict .
         \FOR{ {\bfseries to}  }
           \STATE Sample .
           \STATE Compute .
           \STATE Compute  with \eqref{eq:noise-scale}.
           \STATE Correct .
         \ENDFOR
         \STATE Propagate .
       \ENDFOR
       \STATE {\bfseries return} 
    \end{algorithmic}
  \end{algorithm}
  \end{minipage}
\end{wrapfigure}
\textbf{Sampling.}
The sampling procedure is summarized in Alg.~\ref{alg:sample}.
Given some pre-defined signal-to-noise ratio  (we set 0.05 for all experiments), the Langevin noise scale  at each time step  and each corrector step  is computed by



\begin{figure}[t]
  \vskip -0.1in
  \begin{minipage}{\textwidth}
      \centering
      \includegraphics[height=0.15\textwidth]{fig/toy_arch.png}
      \vskip -0.1in
      \caption{
          Network architecture for toy datasets.
      }
      \label{fig:toy_arch}
  \end{minipage}
\end{figure}



\textbf{Network architectures.}
Table~\ref{table:network_arch} summarizes the network architecture used for each dataset.
For toy datasets, we parameterize  and  with the architectures shown in Fig.~\ref{fig:toy_arch}.
Specifically, \textit{FCBlock} represents a fully connected layer followed by a swish nonlinear activation \citep{ramachandran2017searching}.
As for MNIST, we consider a smaller version of Unet \citep{ho2020denoising} by reducing
the numbers of residual block, attention heads, and channels respectively to 1, 2, and 32.
Unet and NCSN++ respectively correspond to the architectures appeared in \citet{ho2020denoising} and \citet{song2020score}.

\textbf{Remarks on Table~\ref{table:NLL_FID}.}
We note that the values of our SB-FBSDE reported in Table~\ref{table:NLL_FID} are computed \textit{without} the Langevin corrector due to the computational constraint.
For all other experiments, we adopt the Langevin corrector
as it generally improves the performance (see Fig.~\ref{fig:fid}).
This implies that our results on CIFAR10, despite already being encouraging, may be further improved with the Langevin corrector.


\textbf{Remarks on Fig.~\ref{fig:mnist-kl}.}
To estimating ,
we first compute the pixel-wise first and second moments given the generated samples  at the end of the forward diffusion.
After fitting a diagonal Gaussian to ,
we can apply the analytic formula for computing the KL divergence between two multivariate Gaussians.

\textbf{Remarks on Fig.~\ref{fig:fid}.}
To accelerate the sampling process with the Langevin corrector,
for this experiment
we consider a reduced Unet (see Table~\ref{table:network_arch}) for CelebA.
The FID scores on both datasets are computed with 10k samples.
We stress, however, that the performance improvement using the Langevin corrector remains consistent across other (larger) architectures and if one increases the FID samples.







\section{Additional Experiments}
\label{appendix:addtional_fig}


\paragraph{Comparison to \citet{de2021diffusion} under same setup.}


To demonstrate the superior performance of our model, we conduct experiments with the exact same setup implemented in \citet{de2021diffusion}. Specifically, we adopt the same network architecture (reduced U-net), image pre-processing (center-cropping 140 pixel and resizing to 32  32), step sizes (=50), and horizon (0.5 second) for fair comparison.
Comparing our Fig.~\ref{fig:DSB_compare-b} to \citet{de2021diffusion} (see their Fig.~6), it is clear that images generated by our model have higher diversity (\eg color skin, facing angle, background, etc) and better visual quality. We conjecture that our performance difference may come from \textit{(i)} the (in)sensitivity to numerical discretization between our divergence objectives and their mean-matching regression, and \textit{(ii)} the foundational differences in how diffusion coefficients are designed.


\begin{figure}[H]\centering
  \subfloat[Ground Truth]{\includegraphics[width=5cm]{fig/ground_truth2.png} \label{fig:DSB_compare-a}}
  \qquad
  \subfloat[SB-FBSDE Generated Image]{\includegraphics[width=5cm]{fig/small_net_sample_stage9.png} \label{fig:DSB_compare-b}}\caption{Comparison between images generated by ground truth and SB-FBSDE on reduced CelebA. Our SB-FBSDE is trained under the same data pre-processing, network architecture and stepsizes implemented in \citet{de2021diffusion}. }\label{fig:DSB_compare}\end{figure}

\begin{figure}[H]
  \vskip -0.3in
  \centering
  \subfloat[SGM/50k]{\includegraphics[width=5cm]{fig/sample_stage50000.png} \label{fig:DSM-vs-SB-a} }\qquad
  \subfloat[SGM/50k + SB/b/5k]
  {\includegraphics[width=5cm]{fig/sample_stage1.png} \label{fig:DSM-vs-SB-b} }\qquad
  \subfloat[SGM/50k + SB/f/5k + SB/b/5k]
  {\includegraphics[width=5cm]{fig/sample_stage2.png} \label{fig:DSM-vs-SB-c} }\caption{
      Qualitative results at the different stages of training.
      \textit{(a)} Results after 50k training iterations using SGM's regression loss.
      \textit{(b)} Refine the results of Fig.~\ref{fig:DSM-vs-SB-a} by training the backward policy using \eqref{eq:high-dim-loss} with 5k iterations.
      \textit{(c)} Refine the results of Fig.~\ref{fig:DSM-vs-SB-a} with a full SB-FBSDE stage using (\ref{eq:high-dim-loss},\ref{eq:high-dim-loss2}).
  }\label{fig:DSM-vs-SB}\end{figure}

\paragraph{SGM regression training + SB divergence-based training.}

Table~\ref{table:SB-refine} reports the FID (using 10k samples, without corrector steps) at different stages of CIFAR10 training. We first train the backward policy with SGM's regression loss for a sufficient long iterations (50k) until the FID roughly converges. Then, we switch to our alternate training (Alg.~\ref{alg:train3}) using the divergence-based objectives.
Crucially, with only 5k iterations of our divergence-based training, we drop the FID dramatically down to 13.35 from 33.68. With a full stage of training (last column), the FID decreases even lower to 11.85.
The qualitative results are provided in Fig.~\ref{fig:DSM-vs-SB}. Comparing Fig.~\ref{fig:DSM-vs-SB-a} (corresponds to ``SGM/50k'' in Table~\ref{table:SB-refine}) and Fig.~\ref{fig:DSM-vs-SB-b} (corresponds to ``SGM/50k + SB/b/5k'' in Table~\ref{table:SB-refine}), it can be seen that the visible flaw and noise have been substantially improved.

\vspace{20pt}

\textbf{Additional Figures}





\begin{figure}[t]
  \vskip 0.1in
  \begin{minipage}{\textwidth}
      \centering
      \captionsetup{type=table}
      \caption{SGM regression training + SB divergence-based training. We denote ``SGM/50k'' as ``training 50k steps using SGM loss'', and ``SB/\{f,b\}/5k'' as ``training forward/backward policy with 5k steps using our divergence loss'', and etc.}
      \vskip -0.1in
      \centering
      \begin{tabular}{r|cccccc}
        \toprule
        & initialization & SGM/10k & SGM/20k & SGM/50k &
        \specialcell[c]{SGM/50k \\ + SB/b/5k} &
        \specialcell[c]{SGM/50k \\ + SB/f/5k + SB/b/5k} \\
        \midrule
         FID & 448 & 41.37 & 35.47 & 33.68 & 13.35 & 11.85 \\
        \bottomrule
      \end{tabular}
      \label{table:SB-refine}
  \end{minipage}
\end{figure}




\begin{figure}[h]
\begin{center}
\includegraphics[width=0.9\textwidth]{fig/mnist-large.png}
\vskip -0.05in
\caption{
    Uncurated samples generated by our SB-FBSDE on MNIST.
}
\end{center}
\end{figure}

\newpage

\begin{figure}[h]
\begin{center}
\includegraphics[width=0.9\textwidth]{fig/celebA-large.png}
\vskip -0.05in
\caption{
    Uncurated samples generated by our SB-FBSDE on resized CelebA.
}
\end{center}
\end{figure}

\newpage

\begin{figure}[h]
\begin{center}
\includegraphics[width=0.9\textwidth]{fig/cifar10-large.png}
\vskip -0.05in
\caption{
    Uncurated samples generated by our SB-FBSDE on CIFAR10.
}
\end{center}
\end{figure}
 


\end{document}
