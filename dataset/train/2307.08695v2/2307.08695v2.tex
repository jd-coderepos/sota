\documentclass[10pt,twocolumn,letterpaper]{article}

\usepackage{iccv}
\usepackage{times}
\usepackage{epsfig}
\usepackage{graphicx}
\usepackage{amsmath}
\usepackage{amssymb}
\usepackage{multirow}
\usepackage[title]{appendix}
\usepackage{booktabs}
\usepackage{utfsym}
\usepackage{amsmath}
\usepackage{amssymb}
\usepackage{caption}
\usepackage{subcaption}
\usepackage{enumitem}
\usepackage{url}
\usepackage[accsupp]{axessibility}


\usepackage[export]{adjustbox}
\usepackage{etoolbox}
\usepackage[normalem]{ulem}
\usepackage[symbol]{footmisc}
\usepackage{stmaryrd}
\usepackage{placeins}
\usepackage{adjustbox}
\usepackage{dcolumn}
\usepackage[square,sort,comma,numbers]{natbib}
\usepackage{multibib}






\usepackage[pagebackref=true,breaklinks=true,letterpaper=true,colorlinks,bookmarks=false]{hyperref}

\iccvfinalcopy 

\def\iccvPaperID{1187} \def\httilde{\mbox{\tt\raisebox{-.5ex}{\symbol{126}}}}


\def\framework{Neural Video Depth Stabilizer}
\def\fw{neural video depth stabilizer}
\def\sx{NVDS}
\def\network{Depth Stabilizer}
\def\SbN{Stabilization Network}
\def\Sbn{Stabilization network}
\def\sbn{stabilization network}
\def\data{VDW}
\def\sota{state-of-the-art}
\def\reffig{Fig.}
\def\reftab{Table}
\def\refequ{Eq.}
\def\refsec{Sec.}

\newcommand{\sm}[1]{{\color[rgb]{0.9,0.1,0.1}{#1}}}
\newcommand{\wyr}[1]{{\color{cyan} #1}}
\newcommand{\jz}[1]{{\color{blue} #1}}

\ificcvfinal\pagestyle{empty}\fi
\DeclareMathSymbol{@}{\mathord}{letters}{"3B}
\begin{document}

\title{Neural Video Depth Stabilizer}





\author{Yiran Wang\hspace{0.1in} 
        Min Shi\hspace{0.1in} 
        Jiaqi Li\hspace{0.1in}
        Zihao Huang\hspace{0.1in} \\
        Zhiguo Cao\hspace{0.1in}
        Jianming Zhang\hspace{0.1in}
        Ke Xian\footnotemark[1]~\hspace{0.1in}
        Guosheng Lin\\
School of Artificial Intelligence and Automation, Huazhong University of Science and Technology\\
Adobe Research\hspace{0.2in} 
S-Lab, Nanyang Technological University\\
{\tt\small \{wangyiran,min\_shi,lijiaqi\_mail,zihaohuang,zgcao\}@hust.edu.cn}\\
{\tt\small jianmzha@adobe.com, \{ke.xian,gslin\}@ntu.edu.sg}\\
{\small{\url{https://github.com/RaymondWang987/NVDS}}}
\vspace{-2mm}
}

\maketitle
\ificcvfinal\thispagestyle{empty}\fi

\renewcommand{\thefootnote}{\fnsymbol{footnote}} \footnotetext[1]{Corresponding author.}
\begin{abstract} 
   Video depth estimation aims to infer temporally consistent depth. Some methods achieve temporal consistency by finetuning a single-image depth model during test time using geometry and re-projection constraints, which is inefficient and not robust. An alternative approach is to learn how to enforce temporal consistency from data, but this requires well-designed models and sufficient video depth data. To address these challenges, we propose a plug-and-play framework called \framework{} (\sx{}) that stabilizes inconsistent depth estimations and can be applied to different single-image depth models without extra effort. We also introduce a large-scale dataset, Video Depth in the Wild (VDW), which consists of 14,203 videos with over two million frames, making it the largest natural-scene video depth dataset to our knowledge. We evaluate our method on the \data{} dataset as well as two public benchmarks and demonstrate significant improvements in consistency, accuracy, and efficiency compared to previous approaches. Our work serves as a solid baseline and provides a data foundation for learning-based video depth models. We will release our dataset and code for future research.
\end{abstract}

\section{Introduction}
\label{sec:introduction}

\begin{figure}[!t]
\centering
\includegraphics[width=0.495\textwidth,trim=25 15 30 10,clip]{figures/fig1_0309.pdf}
\vspace{-30pt}
\caption{\textbf{(a) Performance and efficiency comparisons}. Circle area represents inference time. Smaller circles mean faster speed.
The X-axis represents  on Sintel~\cite{sintel} dataset for spatial accuracy.  The Y-axis represents consistent metric . Lower  means better temporal consistency. Our framework outperforms prior arts by large margins.  \textbf{ (b) Dataset comparisons}. 
Larger circles mean larger amounts of frames. We present \data{} dataset, the largest video depth dataset in the wild with diverse scenes.}
\label{fig:fig1}
\end{figure}

Monocular video depth estimation is a prerequisite for various video applications, \textit{e.g.}, bokeh rendering~\cite{bokehme,videobokeh,luo2023defocus}, 2D-to-3D video conversion~\cite{n1}, and novel view synthesis~\cite{diudiu1,diudiu2}. 
An ideal video depth model should output depth results with both spatial accuracy and temporal consistency.
Although the spatial accuracy has been greatly improved by recent advances in single-image depth models~\cite{dpt,midas,zhuzhu1,newcrfs,ljq} and datasets~\cite{mega,kexian2020,kexian2018}, how to obtain temporal consistency, \textit{i.e.}, removing flickers in the predicted depth sequences, is still an open question. The prevailing video depth approaches~\cite{CVD,rcvd,dycvd} require test-time training (TTT). During inference, a single-image depth model is finetuned on the testing video with geometry constraints and pose estimation. These TTT-based methods have two main issues: limited robustness and heavy computation overhead. Due to the heavy reliance on camera poses, \textit{e.g.}, CVD~\cite{CVD} shows erroneous predictions and robust-CVD~\cite{rcvd} produces obvious artifacts for many videos when camera poses~\cite{colmapsfm,rcvd} are inaccurate. Moreover, test-time training is extremely time-consuming. CVD~\cite{CVD} takes  minutes for  frames on four NVIDIA Tesla M40 GPUs.

This motivates us to build a learning-based model that learns to enforce temporal consistency from video depth data. However, like all the deep-learning models, learning-based paradigm requires proper model design and sufficient training data. Previous learning-based methods~\cite{deepv2d,fmnet,ST-CLSTM,MM21} show worse performance than the TTT-based ones. Video depth data is also limited in scale and diversity. 



To address the two aforementioned challenges, we first propose a flexible learning-based framework termed \framework{} (\sx{}), which can be directly applied to different single-image depth models. \sx{} contains a depth predictor and a \sbn{}. The depth predictor can be any off-the-shelf single-image depth model. Different from the previous learning-based methods~\cite{MM21,fmnet,ST-CLSTM,deepv2d} that function as stand-alone models, \sx{} is a plug-and-play refiner for different depth predictors. Specifically, the \sbn{} processes initial flickering disparity estimated by the depth predictor and outputs temporally consistent results. 
Therefore, our framework can benefit from the cutting-edge depth models without extra effort. As for the design of \sbn{}, inspired by attention~\cite{transformer} in other video tasks~\cite{stt,ctrans,vivit,cffm}, we adopt a cross-attention module in our framework. Each frame can attend relevant information from adjacent frames for temporal consistency. We also design a bidirectional inference strategy to further improve the consistency. As shown in \reffig{}~\ref{fig:fig1}(a), our \sx{} outperforms the previous approaches in terms of consistency, accuracy, and efficiency significantly.

Moreover, we collect a large-scale natural-scene video depth dataset, Video Depth in the Wild (\data{}), to support the training of robust learning-based models. 
Current video depth datasets are mostly closed-domain~\cite{kitti,tum,nyu,scannet,irs}. A few in-the-wild datasets~\cite{sintel,tata,wsvd} are still limited in quantity, diversity, and quality, \textit{e.g.}, Sintel~\cite{sintel} only contains  animated videos. In contrast, our \data{} dataset contains  stereo videos of over  hours and  frames from four different data sources, including movies, animations, documentaries, and web videos. We adopt a rigorous data annotation pipeline to obtain high-quality disparity ground truth for these data.
As shown in \reffig{}~\ref{fig:fig1}(b), to the best of our knowledge, \data{} is the largest in-the-wild video depth dataset with diverse scenes. 



We conduct evaluations on the \data{} and two public benchmarks: Sintel~\cite{sintel} and NYUDV2~\cite{nyu}. Our method achieves \sota{} in both the accuracy and the consistency.
We also fit three different depth predictors~\cite{dpt,midas,newcrfs} into our framework and evaluate them on NYUDV2~\cite{nyu}. The results demonstrate the flexibility and effectiveness of our plug-and-play manner. 
Our main contributions can be summarized as follows:
\begin{itemize}[leftmargin=*]


    \item We propose a plug-and-play and bidirectional learning-based framework termed \framework{} (\sx{}), which can be directly adapted to different single-image depth predictors to remove flickers.

    \item We propose \data{} dataset, which is currently the largest video depth dataset in the wild with the most diverse video scenes.

\end{itemize}

\section{Related Work}


\noindent\textbf{Consistent Video Depth Estimation.} In addition to predicting spatial-accurate depth, the core task of consistent video depth estimation is to achieve temporal consistency, \textit{i.e.}, removing the flickering effects between consecutive frames. Current video depth estimation approaches can be categorized into test-time training (TTT) ones and learning-based ones.
TTT-based methods train an off-the-shelf single-image depth estimation model on testing videos during inference with geometry~\cite{CVD,rcvd,dycvd} and pose~\cite{colmapmvs,colmapsfm,rcvd} constraints. The test-time training can be time-consuming. For example, as illustrated by CVD~\cite{CVD}, their method takes  minutes on  NVIDIA Tesla M40 GPUs to process a video of  frames. Besides, TTT-based approaches are not robust on in-the-wild videos as they heavily rely on camera poses, which are not reliable for natural scenes.
In contrast, the learning-based approaches train video depth models on video depth datasets by spatial and temporal supervision. ST-CLSTM~\cite{ST-CLSTM} adopts long short-term memory (LSTM) to model temporal relations. FMNet~\cite{fmnet} restores the depth of masked frames by the unmasked ones with convolutional self-attention~\cite{ctrans}. Cao \textit{et al.} adopt a spatial-temporal propagation
network trained by knowledge distillation~\cite{kd0,kd1}. However, those methods are independent and cannot refine the results from single-image depth models for consistency. Their performance on consistency and accuracy is also limited. For example, as shown by~\cite{fmnet}, ST-CLSTM~\cite{stt} only exploits sub-sequences of several frames and produces obvious flickers in the outputs. In this paper, we propose a novel framework called \framework{} (\sx{}), which can be directly adapted to any off-the-shelf single-image depth models in a plug-and-play manner. 

\begin{figure*}[!t]
\centering
\includegraphics[width=0.97\textwidth,trim=0 30 0 30,clip]{figures/pipeline_0307.pdf}
\vspace{-10pt}
\caption{
\textbf{Overview of the neural video depth stabilizer.}
Our framework consists of a depth predictor and a \sbn{}. The depth predictor can be any single-image depth model which produces initial flickering disparity maps. Then, the \sbn{} refines the flickering disparity maps into temporally consistent ones. The \sbn{} functions in a sliding window manner: the frame to be predicted fetches information from adjacent frames for stabilization.
During inference, our \sx{} framework can be directly adapted to any off-the-shelf depth predictors in a plug-and-play manner. We also devise bidirectional inference to further improve consistency.}
\label{fig:pipeline}
\end{figure*}

\noindent\textbf{Video Depth Datasets} According to the scenes of samples, existing video depth datasets can be categorized into closed-domain datasets and natural-scene datasets. Closed-domain datasets only contain samples in certain scenes, \textit{e.g.,} indoor scenes~\cite{nyu,scannet,irs}, office scenes~\cite{tum}, and autonomous driving~\cite{kitti}. To enhance the diversity of samples, natural-scene datasets are proposed, which use computer-rendered videos~\cite{sintel,tata} or crawl stereoscopic videos from YouTube~\cite{wsvd}. However, the scene diversity and scale of these datasets are still very limited for training robust video depth estimation models that can predict consistent depth in the wild. For instance, WSVD~\cite{wsvd}, which shares a few similar data annotation steps with the proposed \data{} dataset, only contains  YouTube videos with varied quality and insufficient diversity. Sintel~\cite{sintel} only contains  animated videos. To better train and benchmark video depth models, we propose our \data{} dataset with  videos from  different data sources. To the best of our knowledge, our \data{} dataset is currently the largest video depth dataset in the wild with the most diverse scenes.


\section{\framework{}}



As shown in \reffig{}~\ref{fig:pipeline}, the proposed \framework{} (\sx{}) consists of a depth predictor and a \sbn{}. The depth predictor predicts the initial flickering disparity for each frame. The \sbn{} converts the disparity maps into temporally consistent ones. Our \sx{} framework can coordinate with any off-the-shelf single-image depth models as depth predictors. We also devise a bidirectional inference strategy to further enhance the temporal consistency during testing. 

\subsection{\SbN{}}
The \sbn{} takes RGB frames along with initial disparity maps as inputs. A backbone~\cite{segformer} encodes the input sequences into depth-aware features. The next step is to build inter-frame correlations. We use a cross-attention module to refine the depth-aware features with temporal information from relevant frames.
Finally, the refined features are fed into a decoder which restores disparity maps with temporal consistency.

\noindent \textbf{Depth-aware Feature Encoding.}
\Sbn{} works in a sliding-window manner: each frame refers to a few previous frames, which are denoted as reference frames, to stabilize the depth. We denote the frame to be predicted as the target frame. Each sliding window consists of four frames.

Due to the varied scale and shift of disparity maps produced by different depth predictors, the initial disparity maps within a sliding window  should be normalized into :


Then, the normalized disparity maps are concatenated with the RGB frames to form a RGB-D sequence. We use a transformer backbone~\cite{segformer} to encode the RGB-D sequence into depth-aware feature maps. 






\noindent \textbf{Cross-attention Module.}
With the depth-aware features, the subsequent phase entails the establishment of inter-frame correlations. We leverage a cross-attention module to build temporal and spatial dependencies across pertinent video frames.
Specifically, in the cross-attention module, the target frame selectively attends the relevant features in the reference frames to facilitate depth stabilization. Pixels in the target frame feature maps serve as the query in the cross-attention operation~\cite{transformer}, while the keys and values are generated from the reference frames.

Computational cost can become prohibitively high when employing cross-attention for each position in depth-aware features. Hence, we utilize a patch merging strategy~\cite{VIT} to down-sample the target feature map. Besides, we also restrict the cross-attention into a local window, whereby each token in the target features can only attend a local window in the reference frames. 
Let  denote the depth-aware feature of the target frame, while  and  represent the features for the three reference frames.  is partitioned into  patches with no overlaps; each patch is merged into one token  , where  is the dimension. For each , we conduct a local window pooling on  and  and stack the pooling results into . Then, the cross-attention is computed as:

where , , and  are learnable linear projections. The cross-attention layer is incorporated into a standard transformer block~\cite{transformer} with residual connection and multi-layer perceptron (MLP). We denote the resulting target feature map refined by the cross-attention module as . 

Ultimately, a depth decoder with feature fusion modules~\cite{FFM1,FFM2} integrates the depth-aware feature of the target frame () with the cross-attention refined feature , and predicts the consistent disparity map for target frame. 







\subsection{Training the \SbN{}}
In the training phase, only the \sbn{} is optimized. The depth predictor is the freezed pre-trained DPT-L~\cite{dpt}. For the \sbn{}, we apply spatial and temporal loss that supervises the depth accuracy and temporal consistency respectively. The training loss can be formulated by:

where  and  denote the spatial loss of frame  and  respectively.  denotes the temporal loss between frame  and .

We adopt the widely-used affinity invariant loss and gradient matching loss~\cite{midas,dpt} as the spatial loss . As for the temporal loss, we adopt the optical flow based warping loss~\cite{MM21,fmnet} to supervise temporal consistency:

where  is the predicted disparity  warped by the optical flow . In our implementation, we adopt the GMFlow~\cite{gmflow} for optical flow.  is the mask calculated as~\cite{MM21,fmnet} and  denotes pixel numbers. See supplementary for more details on loss functions. 


\subsection{Bidirectional Inference}
\label{sec:biinfer}
Expanding the temporal receptive range can be beneficial for consistency, \textit{e.g.}, adding more former or latter reference frames. However, directly training the \sbn{} with bidirectional reference frames will introduce large training burdens. To remedy this, we only train the \sbn{} with the former three reference frames. To further enlarge the temporal receptive field and enhance consistency, we introduce a bidirectional inference strategy.


Unlike the training phase, during inference, both the former and latter frames will be used as the reference frames. An additional sliding window is added, where the reference frames are the subsequent three frames of the target.
Let us define the stabilizing process as a function , where  and  denotes the target RGB-D frame and the reference frames set. When denoting the RGB-D sequence as ,  represents frame numbers of a certain video, using this additional sliding window for stabilization can be formulated as:

where  denotes the target frame. Likewise, using the original sliding window for stabilization can be denoted by:

We ensemble the bidirectional results for a larger temporal receptive field as:

 denotes the final disparity prediction of the  frame (target frame). This bidirectional manner can further improve the temporal consistency as demonstrated in~\refsec{}~\ref{sec:abl}.



Note that, the cross-attention module is shared by the two sliding windows for inference. Besides, the initial disparity maps and depth-aware features are pre-computed. Hence, the bidirectional inference only increases the inference time by  compared with single-direction inference and brings no extra computation for the training process. 

In addition to bidirectional inference, the depth predictor can be reconfigurable during inference. For example, simply using a more advanced model NeWCRFs~\cite{newcrfs} as the depth predictor can obtain performance gain without extra training, as shown in \reftab{}~\ref{tab:blab}. As the depth accuracy can be inherited from \sota{} depth predictor, our neural video depth stabilizer (\sx{}) framework can focus on the learning of depth stabilization and combine the depth accuracy with temporal consistency. 


\subsection{Implementation Details}

\noindent \textbf{Model Architecture.} We use the DPT-L~\cite{dpt}, Midas-v2~\cite{midas}, and NeWCRFs~\cite{newcrfs} as the single-image depth predictors during inference, while only the disparity maps from DPT-L are used during training. Midas-v2~\cite{midas} is the same depth model as TTT-based methods~\cite{CVD,rcvd,dycvd} for fair comparisons. Mit-b5~\cite{segformer} is adopted as the backbone to encode depth-aware features. For each target frame, we use three reference frames with inter-frame intervals .

\noindent \textbf{Training Recipe.} All frames are resized so that the shorter side equals , and then randomly cropped to  for training. In each epoch, we randomly sample  input sequences. Note that the sampled frames in each epoch does not overlap. We use Adam optimizer to train the model for  epochs with a batchsize of . The initial learning rate is set to  and decreases by  for every five epochs. When finetuning our model on NYUDV2~\cite{nyu}, we use a learning rate of  for only one epoch. In all experiments, the temporal loss weight  is set to . 











\begin{table}
\begin{center}
\addtolength{\tabcolsep}{-4.2pt}
\resizebox{\columnwidth}{!}{
\begin{tabular}{llccccccc}
\toprule
 Type &  Dataset & Videos & Frames() & Indoor & Ourdoor & Dynamic & Resolution \\
\midrule
\multirow{4}{*}{Closed} & NYUDV2~\cite{nyu} &  &  &  &  &  &  \\
& KITTI~\cite{kitti} &  &  &  &  &  &  \\
\multirow{2}{*}{Domains} & TUM~\cite{tum} &  &  &  &  &  &  \\
& IRS~\cite{irs} &  &  &  &  &  &  \\
& ScanNet~\cite{scannet} &  &  &  &  &  &  \\
\midrule
\multirow{4}{*}{Natural} & Midas~\cite{midas} &  &  &  &  &  &  \\
 & Sintel~\cite{sintel} &  &  &  &  &  &  \\
 \multirow{2}{*}{Scenes} & TartanAir~\cite{tata} &  &  &  &  &  &  \\
 &WSVD~\cite{wsvd} &  &  &  &  &  &  \\
& Ours &  &  &  &  &  &  \\
\bottomrule
\end{tabular}
}
\end{center}
\vspace{-12pt}
\caption{\textbf{Comparisons of video depth datasets.}  The 3D Movies dataset of Midas~\cite{midas} is not released
and only contains 75k images but not videos. TartanAir~\cite{tata} only has some limited dynamic scenes (\textit{e.g.}, fish in the ocean sequence). However, most videos in TartanAir~\cite{tata} lack major dynamic objects (\textit{e.g.}, pedestrians). For example, models
trained on TartanAir cannot predict satisfactory results
in scenes with moving people as such scenes are rare. Our \data{} dataset shows advantages in diversity and quantity. }

\label{tab:dacp}
\end{table}

\section{\data{} Dataset}
As mentioned in \refsec{}~\ref{sec:introduction}, current video depth datasets are limited in both diversity and volume. To compensate for the data shortage and boost the performance of learning-based video depth models, we elaborate a large-scale natural-scene dataset, Video Depth in the Wild (VDW). To our best knowledge, our \data{} dataset is currently the largest video depth dataset with the most diverse video scenes.

\noindent\textbf{Dataset Construction.} We collect stereo videos from four data sources: movies, animations, documentaries, and web videos. A total of  movies, animations, and documentaries in Blu-ray format are collected. We also crawl  web stereo videos from YouTube with the keywords such as ``stereoscopic'' and ``stereo''. To balance the realism and diversity, only  movies, animations, and documentaries are retained. For instance, ``Seven Wonders of the Solar System'' is removed as it contains many virtual scenes. The disparity ground truth is generated with two main steps: sky segmentation and optical flow estimation. A model ensemble manner is adopted to remove errors and noises in the sky masks, which can improve the quality of the ground truth and the performance of the trained models, especially on sky regions as shown in \reffig{}~\ref{fig:qpqp}. A \sota{} optical flow model GMFlow~\cite{gmflow} is used to generate the disparity ground truth. Finally, a rigorous data cleaning procedure is conducted to filter the videos that are not qualified for our dataset. \reffig{}~\ref{fig:datagt} shows some examples of our \data{} dataset. 

\begin{figure}[!t]
\begin{center}
   \includegraphics[width=0.47\textwidth,trim=10 0 10 0,clip]{figures/gtv2.pdf}
\end{center}
\vspace{-12pt}
   \caption{
   \textbf{Examples of our \data{} dataset.}
   Four rows are from web videos, documentaries, animations, and movies, respectively. Sky regions and invalid pixels are masked out.}
\label{fig:datagt}
\end{figure}

\begin{figure}[!t]
\begin{center}
   \includegraphics[width=0.49\textwidth,trim=10 10 10 10,clip]{suppfig/vdw.pdf}
\end{center}
\vspace{-12pt}
   \caption{\textbf{Objects presented in our \data{} dataset.} We conduct semantic segmentation with Mask2Former~\cite{mask2former} trained on ADE20k~\cite{ade20k}. Refer to our supplementary for more detailed construction process and data statistics.}
\label{fig:wccc}
\end{figure}


\begin{table*}[!t]
\begin{center}
    \resizebox{\textwidth}{!}{
    \begin{tabular}{llcccccccccccc}
    \toprule
    \multirow{2}{*}{Type} & \multirow{2}{*}{Method} & \multirow{2}{*}{Time()}  &
    \multicolumn{3}{c}{\data{}} &
    \multicolumn{5}{c}{Sintel} &
    \multicolumn{3}{c}{NYUDV2} \\
    \cmidrule{4-6} \cmidrule{8-10} \cmidrule{12-14}
& & &  &  &
         & & 
         &  &  & & 
         &  &  \\
    \midrule
    \multirow{1}{*}{Single} & Midas~\cite{midas}    &&&  & &&&  &  && &  & \\
    \multirow{1}{*}{Image} & DPT~\cite{dpt}        &&& & &&& & && & &  \\
\midrule
    \multirow{2}{*}{Test-time} & CVD~\cite{CVD}    &&&  & &&&  &  && &  &  \\
    \multirow{2}{*}{Training} & Robust-CVD~\cite{rcvd}        & && & &&& & && & & \\
    & Zhang \textit{et al.}~\cite{dycvd} &&& & &&& & && & &  \\
    \midrule
    \multirow{6}{*}{Learning} & ST-CLSTM~\cite{ST-CLSTM}    &&&  & &&&  &  && &  & \\
    \multirow{6}{*}{Based} & Cao \textit{et al.}~\cite{MM21}        &&& & &&& & && & & \\
    & FMNet~\cite{fmnet}        & && & &&& & && & &\\
    & DeepV2D~\cite{deepv2d}& && & &&& & && & &\\
    & WSVD~\cite{wsvd} &&& & &&& & && & &\\
    & Ours(Midas) &&& & &&& & && & &\\
    & Ours(DPT) &&& & &&& & && & &\\
    \bottomrule
    \end{tabular}
}
\end{center}
\vspace{-12pt}
\caption{
\textbf{Comparisons with the state-of-the-art approaches.} We report the total time of processing eight  frames by different methods on one NVIDIA RTX A6000 GPU. Best performance is in boldface. Second best is underlined.}
\label{tab:bigone}
\end{table*}


\noindent\textbf{Dataset Statistics.}
\label{sec:ds}
\data{} dataset contains  videos with a total of  frames. The total data collection and processing time takes over six months and about  man-hours. To verify the diversity of scenes and entities in our dataset, we conduct semantic segmentation by Mask2Former~\cite{mask2former} trained on ADE20k~\cite{ade20k}. All the  categories are covered in our dataset, and each category can be found in at least  videos. \reffig{}~\ref{fig:wccc} shows the word cloud of the 150 categories.
We randomly choose  videos with  frames as the test set. The testing videos adopt different data sources from the training data, \textit{i.e.}, different movies, web videos, or animations.
Our \data{} not only alleviates the data shortage for learning-based approaches, but also serves as a comprehensive benchmark for video depth. 

\noindent\textbf{Comparisons with Other Datasets.} 
\label{sec:dcom}
As shown in \reftab{}~\ref{tab:dacp}, the proposed \data{} dataset has significantly larger numbers of video scenes. Compared with the closed-domain datasets~\cite{nyu,kitti,scannet,tum,irs}, the videos of \data{} are not restricted to a certain scene, which is more helpful to train a robust video depth model. For the natural-scene datasets, our dataset has more than ten times the number of videos as the previous largest dataset WSVD~\cite{wsvd}. Although WSVD~\cite{wsvd} has  frames, the scenes (video numbers) are limited. Midas~\cite{midas} also proposes their 3D Movies dataset with in-the-wild images and disparity. Compared with the 3D Movies dataset of Midas~\cite{midas}, VDW differs in two main aspects: (1) accessibility; and (2) dataset scale and format. Their 3D Movies dataset~\cite{midas} is not released and only contains  images. In contrast, VDW contains 14,203 videos with  frames.
It is also worth noticing that our \data{} dataset has higher resolution and a rigorous data annotation and cleaning pipeline.  We only collect videos with resolutions over  and crop all our videos to  to remove black bars and subtitles. See supplementary for more statistics and construction process.




























\begin{figure*}[!t]
\begin{center}
   \includegraphics[width=0.97\textwidth,trim=0 0 0 0,clip]{figures/qiepian_0308.pdf}
\end{center}
\vspace{-10pt}
   \caption{
   \textbf{Qualitative comparisons.}
   DeepV2D~\cite{deepv2d} and Robust-CVD~\cite{rcvd} show obvious artifacts in those videos. We draw the scanline slice over time; fewer zigzagging pattern means better consistency. Compared with the other video depth methods, our \sx{} is more robust on natural scenes and achieves better spatial accuracy and temporal consistency.}
\label{fig:qpqp}
\end{figure*}


\section{Experiments}
To prove the effectiveness of our framework \framework{} (\sx{}), we conduct experiments on  different datasets, which contain videos for real-world and synthetic, static and dynamic, indoor and outdoor.
\subsection{Datasets and Evaluation Protocol}
\label{sec:dataset}
\noindent \textbf{\data{} Dataset}. We use the proposed \data{} as the training data for its diversity and quantity on natural scenes. We also evaluate the previous video depth approaches on the test split of \data{}, serving as a new video depth benchmark.

\noindent \textbf{Sintel Dataset}. Following~\cite{rcvd,dycvd}, we use the final version of Sintel~\cite{sintel} to demonstrate the generalization ability of our \sx{}. We conduct zero-shot evaluations on Sintel~\cite{sintel}. All learning-based methods are not finetuned on Sintel dataset.


\noindent \textbf{NYUDV2 Dataset}. Except for natural scenes, a closed-domain NYUDV2~\cite{nyu} is adopted for evaluation. We pretrain the \sbn{} on \data{} and finetune the model on NYUDV2~\cite{nyu} dataset. Besides, we also test our \sx{} on DAVIS~\cite{davis} for qualitative comparisons.









\noindent \textbf{Evaluation Metrics.}
We evaluate both the depth accuracy and temporal consistency of different methods. For the temporal consistency metric, we adopt the optical flow based warping metric () following FMNet~\cite{fmnet}, which can be computed as:

We report the average  of all the videos in the testing sets. As for the depth metrics, we adopt the commonly-applied  and .
 


\begin{table}
\begin{center}
\resizebox{\columnwidth}{!}{
\begin{tabular}{lcccccc}
\toprule
Method &  &  &  &  & \\
\midrule
SC-DepthV1~\cite{scd1} &  &  &  &  & \\
SC-DepthV2~\cite{scd2} &  &  &  &  & \\
ST-CLSTM~\cite{ST-CLSTM} &  &  &  &  & \\
Cao \textit{et al.}~\cite{MM21} &  &  &  &  & \\
FMNet~\cite{fmnet} &  &  &  &  & \\
DeepV2D~\cite{deepv2d} &  &  &  &  & \\
Ours-scratch(DPT) &  &  &  &  & \\
\bottomrule
\end{tabular}
}
\end{center}
\vspace{-12pt}
\caption{
\textbf{Comparisons of the learning-based approaches on NYUDV2~\cite{nyu} dataset}.
All the compared methods use NYUDV2~\cite{nyu} as the training and evaluation data. Our \fw{} trained from scratch also achieves better performance than all the other methods.}
\label{tab:nyu}
\end{table}




\subsection{Comparisons with Other Video Depth Methods}

\noindent \textbf{Comparisons with the TTT-based methods.}
First focus on the test-time training (TTT) approaches~\cite{CVD,rcvd,dycvd}. As shown in \reftab{}~\ref{tab:bigone}, our learning-based framework outperforms TTT-based approaches by large margins in terms of inference speed, accuracy and consistency. Our \sx{} shows at least  and  improvements for  and  than Robust-CVD~\cite{rcvd} on \data{}, Sintel~\cite{sintel}, and NYUDV2~\cite{nyu}. Our learning-based approach is over one hundred times faster than Robust-CVD~\cite{rcvd}. Our strong performance demonstrates that learning-based frameworks are capable of attaining great performance with much higher efficiency than TTT-based methods~\cite{CVD,rcvd,dycvd}.

It is also worth-noticing that the TTT-based approaches are not robust for natural scenes. CVD~\cite{CVD} and Zhang \textit{et al.}~\cite{dycvd} fail on some videos on \data{} and Sintel~\cite{sintel} dataset due to erroneous pose estimation results. Hence, some of their results are not reported in \reftab{}~\ref{tab:bigone}. Refer to supplementary for more details. Although Robust-CVD~\cite{rcvd} can produce results for all testing videos by jointly optimizing the camera poses and depth, it is still not robust for many videos and produces obvious artifacts as shown in \reffig{}~\ref{fig:qpqp}.

\noindent \textbf{Comparisons with the learning-based methods.} The proposed \fw{} also attains better accuracy and consistency than previous learning-based approaches~\cite{ST-CLSTM,fmnet,MM21,wsvd} on all the three datasets, including natural scenes and closed domain. As shown in \reftab{}~\ref{tab:bigone}, on our \data{} and Sintel with natural scenes, the proposed \sx{} shows obvious advantages: improving  and  by over  and  compared with previous learning-based methods. Note that, our \sx{} can benefit from stronger single-image models and obtain better performance, which will be discussed in \reftab{}~\ref{tab:blab}.  

To better compare \sx{} with previous learning-based method, we only use NYUDV2~\cite{nyu} as training and evaluation data for comparisons. As shown in \reftab{}~\ref{tab:nyu}, the proposed \sx{} improves the FMNet~\cite{fmnet} by  and  in terms of  and  . We also achieve better performance than DeepV2D~\cite{deepv2d}, which is the previous \sota{} structure-from-motion-based methods but can only deal with completely static scenes. The results demonstrate that using our architecture alone can also obtain better video depth performance. 



\begin{table}
    \centering
\resizebox{0.48\columnwidth}{!}{
    \begin{subtable}[t]{0.53\linewidth}
        \addtolength{\tabcolsep}{-4pt}
        \begin{tabular}{lcc}
            \toprule
            Dataset &  &  \\
            \midrule
            NYUDV2 &  & \\
            IRS+TartanAir &  & \\
            \data{}(Ours) &  &  \\
            \bottomrule
        \end{tabular}
        \caption{Different Training Data}
    \end{subtable}
    }
    \resizebox{0.48\columnwidth}{!}{
    \begin{subtable}[t]{0.53\linewidth}
         \addtolength{\tabcolsep}{-4pt}
        \begin{tabular}{lcc}
       
            \toprule
            Setting &  &  \\
            \midrule
          Scratch(DPT) &  & \\
          Pretrain(Midas) &  &  \\
          Pretrain(DPT) &  &  \\
            \bottomrule
        \end{tabular}
        \caption{Pretraining and Finetuning}
    \end{subtable}
    }
    \vspace{-6pt}
    \caption{
    \textbf{Influence of different training data}. (a) Training with different datasets. We conduct zero-shot evaluations on Sintel~\cite{sintel} with different training data for our \sx{}. (b) Pretraining and finetuning. Pretraining on our \data{} can further improve the results on the closed-domain NYUDV2~\cite{nyu}, compared with training from scratch.}
    \label{tab:sjjyx}
\end{table}

\begin{table}
\begin{center}
    \resizebox{\columnwidth}{!}{
    \begin{tabular}{lccccccc}
    \toprule
    \multirow{2}{*}{} & 
    \multicolumn{3}{c}{Initial} &
    \multicolumn{4}{c}{Ours} \\
    \cmidrule{2-4} \cmidrule{6-8}
&  &  &
         & & 
         &  &  \\
    \midrule
    Midas~\cite{midas}    &&  & &&&  &  \\
    DPT~\cite{dpt}        && & &&& & \\
    NeWCRFs~\cite{newcrfs} && & &&& & \\
    \bottomrule
    \end{tabular}
}
\end{center}
\vspace{-12pt}
\caption{
\textbf{Comparisons of different depth predictors on the NYUDV2~\cite{nyu} dataset.} Our framework is compatible with different depth predictors in a plug-and-play manner.}
\label{tab:blab}
\end{table}



\noindent \textbf{Qualitative Comparisons.}
We show some qualitative comparisons on natural-scene videos in \reffig{}~\ref{fig:qpqp}. We draw the scanline slice over time. Fewer zigzagging pattern means better consistency. The initial estimation of DPT~\cite{dpt} in the sixth row contains flickers and blurs, which are eliminated with the proposed \sx{}, as shown in the last row. Although the TTT-based Robust-CVD~\cite{rcvd} shows competitive performances on the indoor NYUDV2~\cite{nyu} dataset, it is not robust on the natural scenes. As can be observed in the fourth row, Robust-CVD produces obvious artifacts due to erroneous pose estimation. 

One can also observe that we produce much sharper estimation at the edges, especially on the skylines, which can be down to our rigorous annotation pipeline for \data{}, \textit{e.g.}, the ensemble strategy for sky segmentation.

\noindent \textbf{Influence of Training Data.}
The quality and diversity of data can greatly influence the learning-based video depth models. Our \data{} dataset offers hundreds of times more data and scenes compared to previous works, which can be used to train robust learning-based models in the wild. To better show the difference, we compare our dataset with the existing datasets under zero-shot cross-dataset setting. As shown in \reftab{}~\ref{tab:sjjyx} (a), we train our \sx{} with existing video depth datasets~\cite{nyu,irs,tata} and evaluate the model on Sintel~\cite{sintel} dataset. With both quantity and diversity, using \data{} as the training data yields the best accuracy and consistency. Our VDW
dataset is far more diverse for training robust video depth models, compared with large closed-domain dataset NYUDV2~\cite{nyu} or synthetic natural-scene dataset like IRS~\cite{irs} and TartanAir~\cite{tata}. 

Moreover, although the proposed \data{} is designed for natural scenes, it can also boost the performance on closed domains by serving as pretraining data. As in \reftab{}~\ref{tab:sjjyx} (b), the \data{}-pretrained model outperforms the model that is trained from scratch, even with weaker single-image model (Midas~\cite{midas}). This suggests that \data{} can also benefit some closed-domain scenarios. 



















\begin{table}
    \centering
    \resizebox{0.49\columnwidth}{!}{
    \begin{subtable}[t]{0.6\linewidth}
        \begin{tabular}{lcc}
            \toprule
            Method &  &  \\
            \midrule
            DPT~\cite{dpt} &  & \\
            Pre-window &  & \\
            Post-window &  & \\
            Bidirectional &  & \\
            \bottomrule
        \end{tabular}
        \caption{DPT Initialization}
    \end{subtable}
    }
    \resizebox{0.49\columnwidth}{!}{
    \begin{subtable}[t]{0.6\linewidth}
        \begin{tabular}{lcc}
            \toprule
            Method &  &  \\
            \midrule
            Midas~\cite{midas} &  & \\
            Pre-window &  & \\
            Post-window &  & \\
            Bidirectional &  & \\
            \bottomrule
        \end{tabular}
        \caption{Midas Initialization}
    \end{subtable}
    }
    \vspace{-6pt}
    \caption{\textbf{Ablation of bidirectional inference on \data{}.} Bidirectional inference with larger temporal receptive fields can further improve the consistency.} 
    \label{tab:sxsx}
\end{table}

\begin{table}[!t]
\addtolength{\tabcolsep}{-4pt}
\begin{center}
\resizebox{\columnwidth}{!}{
\begin{tabular}{lcccc}
\toprule
  & DPT-L~\cite{dpt} & NeWCRFs~\cite{newcrfs}&Midas-v2~\cite{midas} & Stabilization Network \\
\midrule
FLOPs () &  & & & \\
Params () &  &  & &  \\
\bottomrule
\end{tabular}
}
\end{center}
\vspace{-12pt}
\caption{\textbf{Comparisons of FlOPs and model parameters.} We evaluate the efficiency of our stabilization network and different depth predictors~\cite{midas,dpt,newcrfs}. The FlOPs are evaluated on a  video with four frames.}\label{tab:flopszw}
\end{table}

\subsection{Model Efficiency Comparisons}
\label{sec:spd}
To evaluate the efficiency, we compare the inference time on a  video with eight frames. The inference is conducted on one NVIDIA RTX A6000 GPU. As shown in \reftab{}~\ref{tab:bigone}, the proposed \sx{} reduces the inference time by hundreds of times compared to the TTT-based approaches CVD~\cite{CVD}, Robust-CVD~\cite{rcvd}, and Zhang \textit{et al.}~\cite{dycvd}. The learning-based method DeepV2D~\cite{deepv2d} alternately estimates depth and camera poses, which is time-consuming. WSVD~\cite{wsvd} is also slow because they need to compute optical flow~\cite{flownet2} between consecutive frames while inference. 

We also evaluate the efficiency of the proposed stabilization network and different depth predictors. Model parameters and FLOPs are reported in \reftab{}~\ref{tab:flopszw}. The FlOPs are evaluated on a  video with four frames. Our stabilization network only introduces limited computation overhead compared with the depth predictors~\cite{midas,dpt,newcrfs}.


\begin{figure*}[!t]
\begin{center}
   \includegraphics[width=0.92\textwidth,trim=0 0 0 0,clip]{suppfig/qiepian-nyu.pdf}
\end{center}
\vspace{-12pt}
   \caption{\textbf{Visual results on NYUDV2~\cite{nyu} dataset.} We compare NVDS with three different depth predictors~\cite{midas,dpt,newcrfs}.}
\label{fig:nyudp}
\end{figure*}





\subsection{Ablation Studies}
\label{sec:abl}
Here we verify the effectiveness of the proposed method. We first ablate the plug-and-play manner with different single-image depth models. Besides, We also discuss the bidirectional inference, the temporal loss, the reference frames, and baselines without the stabilization network. 






\noindent \textbf{Plug-and-play Manner.} As shown in \reftab{}~\ref{tab:blab}, we directly adapt our \sx{} to three different \sota{} single-image depth models DPT~\cite{dpt}, Midas~\cite{midas}, and NeWCRFs~\cite{newcrfs}. For NeWCRFs~\cite{newcrfs}, we adopt their official checkpoint on NYUDV2~\cite{nyu}. By post-processing their initial flickering disparity maps, our \sx{} achieves better temporal consistency and spatial accuracy. With higher initial depth accuracy, the spatial performance of our \sx{} is also improved. The experiment demonstrates the effectiveness of our plug-and-play manner. Visual comparisons with those three depth predictors are shown in \reffig{}~\ref{fig:nyudp}. Depth maps and scanline slice prove our accuracy and consistency.


\begin{table}[!t]
    \centering
\resizebox{0.48\columnwidth}{!}{
    \begin{subtable}[t]{0.53\linewidth}
        \begin{tabular}{lcc}
            \toprule
            Method &  &  \\
            \midrule
            DPT~\cite{dpt} &  & \\
             &  & \\
             &  & \\
            \bottomrule
        \end{tabular}
        \caption{Temporal Loss}
    \end{subtable}
    }
    \resizebox{0.48\columnwidth}{!}{
    \begin{subtable}[t]{0.53\linewidth}
        \begin{tabular}{lcc}
            \toprule
            Method &  &  \\
            \midrule
            &  & \\
            &  & \\
             &  & \\
            \bottomrule
        \end{tabular}
        \caption{Inter-frame Intervals}
    \end{subtable}
    }
    \vspace{-6pt}
    \caption{\textbf{Temporal loss and inter-frame intervals} . We randomly split  videos for training and  videos for testing from our \data{} dataset in these two experiments.}
    \label{tab:clipop}
\end{table}

\noindent \textbf{Bidirectional Inference.} As shown in \reftab{}~\ref{tab:sxsx}, whether using DPT~\cite{dpt} or Midas\cite{midas} as the single-image depth predictor, our \sx{} can already enforce the temporal consistency with previous or post sliding window of target frame. The bidirectional inference can further improve the consistency with larger bidirectional temporal receptive fields.

\noindent \textbf{Temporal Loss.} As in \reftab{}~\ref{tab:clipop} (a), without the temporal loss as explicit supervision, our \sbn{} can enforce temporal consistency. Adding the temporal loss can further remove flickers and improve temporal consistency.

\noindent \textbf{Reference Frame Intervals.} We denote the inter-frame intervals as . As shown in \reftab{}~\ref{tab:clipop} (b),  attains the best performance in our experiments.

\noindent \textbf{Reference Frame Numbers.} As shown in \reftab{}~\ref{tab:nostab} (b), using three reference frames (=3) for a target frame achieves the best results. 
More reference frames (=4) increase computational costs but bring no improvement, which can be caused by the temporal redundancy of videos.



\noindent \textbf{Baselines without Stabilization Network.} With DPT as the depth predictor, we train and evaluate two baselines without the stabilization network on the same subset as \reftab{}~\ref{tab:clipop}. We use the same temporal window as NVDS. The first baseline (Single-frame) can only process each frame independently. Temporal window and loss  are used for consistency. The second baseline (Multi-frame) uses neighboring frames concatenated by channels to predict depth of the target frame.
Training and inference strategies are kept the same as NVDS. As shown in the \reftab{}~\ref{tab:nostab} (a), temporal flickers cannot be solved by simply adding temporal windows and training loss on those baselines. Proper designs are needed for inter-frame correlations. Our stabilization network improves consistency () significantly.










\begin{table}[!t]
    \centering
    
\hspace{-3pt}
    \resizebox{0.45\columnwidth}{!}{
    \begin{subtable}[t]{0.60\linewidth}
        \centering
        \addtolength{\tabcolsep}{-5pt}
        \begin{tabular}{lccc}
            \toprule
           DPT w/ & Single-frame & Multi-frame & \textbf{Ours} \\
           \midrule
            &  &  &  \\
            & & &\\
           \bottomrule
        \end{tabular}
        \caption{ Stabilization Network}
    \end{subtable}
    }
    \hspace{+18pt}
    \resizebox{0.45\columnwidth}{!}{
    \begin{subtable}[t]{0.6\linewidth}
         \addtolength{\tabcolsep}{-4.5pt}
         \begin{tabular}{lcccc}
            \toprule
            & n=1 & n=2 & \textbf{n=3} & n=4 \\
            \midrule
             &  & &  & \\
             &  &  &  &  \\
           \bottomrule
           
        \end{tabular}
        \caption{Reference Frame Numbers}
        
    \end{subtable}
    }
    \caption{\textbf{Baselines without NVDS stabilization network and reference frames numbers} . The experiment is conducted on the same VDW subset as \reftab{}~\ref{tab:clipop}.}
    \label{tab:nostab}
    
\end{table}



\section{Conclusion}
In this paper, we propose a \framework{} framework and a large-scale natural-scene \data{} dataset for video depth estimation. Different from previous learning-based video depth models that function as stand-alone models, our \framework{} learns to stabilize the flickering results from the estimations of single-image depth models. In this way, \framework{} can focus on the learning of temporal consistency, while inheriting the depth accuracy from the cutting-edge depth predictors without further tuning. We also elaborate on the \data{} dataset to alleviate the data shortage. To our best knowledge, it is currently the largest video depth dataset in the wild. We hope our work can serve as a solid baseline and provide a data foundation for the learning-based video depth models.

\noindent \textbf{Limitations and future work.} 
Currently, we only offer one implementation for the \sx{} framework. In future work, we will consider using more mechanisms in the \sbn{} and adding more implementations for different applications, \textit{e.g.}, the lightweight models.

\noindent \textbf{Acknowledgments.}
This work was funded by Adobe Research. Meanwhile, our work is also supported under the RIE2020 Industry Alignment Fund - Industry Collaboration Projects (lAF-ICP) Funding Initiative, as well as cash and in-kind contribution from the industry partner(s).






\newpage

{\small
\bibliographystyle{ieee_fullname}
\bibliography{main}
}

\newpage





\end{document}
