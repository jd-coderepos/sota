\documentclass[11pt]{article}

\usepackage{amssymb,amsmath,amsthm}
\usepackage{enumerate}
\usepackage{version}
\usepackage{fullpage}
\usepackage{url}
\usepackage{graphicx}
\usepackage{dsfont}
\usepackage{caption}

\newtheorem{theorem}{Theorem}[section]
\newtheorem{problem}[theorem]{Problem}
\newtheorem{proposition}[theorem]{Proposition}
\newtheorem{observation}[theorem]{Observation}
\newtheorem{conjecture}[theorem]{Conjecture}
\newtheorem{definition}[theorem]{Definition}
\newtheorem{example}[theorem]{Example}
\newtheorem{remark}[theorem]{Remark}
\newtheorem{corollary}[theorem]{Corollary}
\newtheorem{claim}[theorem]{Claim}
\newtheorem{lemma}[theorem]{Lemma}
\newtheorem{polyrule}{Rule}[section]

\def\eg{{\em e.g.}}
\def\etal{{\em et al.}}
\def\ie{{\em i.e.}~}
\def\cf{{\em cf.}}

\newenvironment{proofclaim}{
	\noindent \emph{Proof.}
}{\hfill  \\
}

\newcommand{\BIT}{{\sc Dense Betweenness}}
\newcommand{\FAST}{\textsc{Feedback Arc Set in Tournaments}}
\newcommand{\rBIT}{{\sc -Dense Betweenness}}
\newcommand{\FASHT}{\textsc{-Dense Feedback Arc Set}}
\newcommand{\rFAST}{\textsc{-Dense Transitive Feedback Arc Set}}

\newcommand{\kBIT}{{\sc -dBIT}}
\newcommand{\kFAST}{\textsc{FAST}}
\newcommand{\kFASHT}{\textsc{-dAST}}

\usepackage{mathrsfs}
\def\baselinestretch{1.1}

\title{Linear vertex-kernels for several dense ranking -CSPs}

\author{Anthony Perez\thanks{LIFO - Universit\'e d'Orl\'eans}}

\date{}

\begin{document}

\maketitle

\begin{abstract}

\noindent A {\sc ranking -constraint satisfaction} problem (ranking -CSP for short) consists of a ground set of vertices , an arity 
, a parameter  and a \emph{constraint system} , where  is a function which maps 
rankings of -sized sets  to . The objective is to 
decide if there exists a ranking  of the vertices satisfying all but at most  constraints (\ie ). Famous 
ranking -CSPs include \FAST{} and \BIT{}~\cite{ALS09,KS10}. We consider such problems from the kernelization viewpoint.
We prove that so-called \emph{-simply characterized} ranking -CSPs admit linear vertex-kernels whenever they admit constant-factor approximation algorithms. This implies that 
\rBIT{} and \rFAST{}~\cite{KS10}, two natural generalizations of the previously mentioned problems, admit linear vertex-kernels. Moreover, we introduce another generalization of \FAST{}, which does not fit the aforementioned framework. Based on techniques from~\cite{CFR06} we obtain a -approximation, and then provide a linear vertex-kernel. 
\end{abstract}

\section*{Introduction}
\label{sec:intro}

Parameterized complexity is a powerful theoretical framework to cope with NP-Hard problems. The aim is to identify some \emph{parameter} , 
independent from the instance size , which captures the exponential growth of the complexity to solve 
the problem at hand. A parameterized problem is said to be \emph{fixed parameter tractable} (FPT for short) whenever it can be solved in  time, 
where  is any computable function~\cite{DF99,Nie06}. In this extended abstract, we focus on \emph{kernelization}, which is one of the most efficient technique to design parameterized algorithms~\cite{Bod09}. A \emph{kernelization algorithm} (or kernel for short) 
for a parameterized problem  is a \emph{polynomial-time} algorithm that given an 
instance  of  outputs an \emph{equivalent} instance  of  
such that  and . The function  is said to be 
the \emph{size} of the kernel, and  admits a \emph{polynomial kernel} 
whenever  is a polynomial. A well-known result states that a (decidable) parameterized problem 
is FPT if and only if it admits a kernel~\cite{Nie06}. Observe that this theoretical result provides kernels of \emph{super-polynomial} size. 
Recently, several results gave evidence that 
there exist parameterized problems that \emph{do not} admit \emph{polynomial} kernels~(under some complexity-theoretic assumption, see \emph{e.g.}~\cite{BDFH09,BJK11}). Hence, determining the existence of \emph{polynomial} kernels is of important interest from both 
the theoretical and practical viewpoints.  \\

We mainly study 
ranking -CSPs from the kernelization viewpoint. 
A ranking -CSP consists of a ground set of vertices , an arity 
 and a \emph{constraint system} , where  is a function which maps rankings (\ie orderings) of -sized sets  to .~The objective is to find a ranking   of the vertices 
that minimizes the number of unsatisfied constraints (\ie ). We study the decision version of these problems, where the instance comes together with some parameter  and the aim is to decide if there exists a ranking of the vertices satisfying \emph{all 
but at most } constraints.~We focus on problems where a constraint  
can be represented by a set of \emph{selected vertices}, which determines the conditions that 
a ranking must verify in order to satisfy . Moreover, we consider such problems 
on \emph{dense instances}, where \emph{every} set of  vertices is a constraint. 
A lot of well-studied problems can be expressed in such terms~\cite{ALS09, KS11}. 
For instance, \FAST{} fits this 
framework with , any arc  being satisfied by a ranking  (\ie ) iff . 
Such problems can be equivalenty stated in terms of  
modification problems: can we \emph{edit} at most  constraints to obtain an instance that admits a ranking satisfying all its constraints? \\



\noindent \textbf{Related results.} While a lot of kernelization results are known for \emph{graph} modification 
problems~\cite{BP11,KW09,Tho10,vBMN10}, fewer results exist regarding directed graph and hypergraph modification problems. 
An example of polynomial kernel for a directed graph modification problem is the 
quadratic vertex-kernel for \textsc{Transitivity Editing}~\cite{WKNU09}.~Regarding dense ranking -CSPs, \FAST{} and \BIT{} are NP-Complete~\cite{AA07,Alo06,CTY07} but fixed parameter tractable \cite{ALS09,KS10}, and both admit a linear vertex-kernel~\cite{PPT11}. Concerning ranking -CSPs, Karpinski and Schudy~\cite{KS11} recently showed PTASs and subexponential parameterized algorithms for so-called (\emph{weakly})-fragile ranking -CSPs. A constraint is \emph{fragile} (resp. \emph{weakly-fragile}) if whenever it is satisfied by one ranking then 
making one single move (resp. making one of the following moves: swapping the first two vertices, the last two vertices 
or making a cyclic move) makes it unsatisfied. These problems contain in particular \FAST{} and \BIT{}. \\

\noindent \textbf{Our results.} Following this line of research, we provide several linear  vertex-kernels for dense ranking -CSPs. 
More precisely, we introduce a particular type of 
ranking -CSPs, called \emph{-simply characterized}, and prove that such problems admit linear vertex-kernels whenever they admit constant-factor approximation algorithms (Section~\ref{sec:general}). Surprisingly, our kernels mainly 
use a modification of the classical sunflower reduction rule, which usually provides polynomial kernels~\cite{ALS09,BP11,GHPP12}. This result improves the size of the kernel for \BIT{}~\cite{PPT11}. Moreover, it implies linear vertex-kernels for \rBIT{} and \rFAST{}, two natural generalizations of \FAST{} and \BIT{}. Both problems were left open by Karpinski and Schudy~\cite{KS11}. 
Finally, we introduce a different generalization of \FAST{}, which allows more freedom on the satisfiability of a given constraint. 
Based on ideas used for \FAST{}~\cite{CFR06}, we prove that this problem admits a -approximation algorithm, and then obtain a linear vertex-kernel (Section~\ref{subsec:rfasht}). 

\section{Preliminaries}
\label{sec:prelim}

A ranking -CSP consists of a ground set of vertices , an arity , a parameter  and a \emph{constraint system} , where  is a function which maps rankings (\ie orderings) of -sized sets  to . In a slight abuse of notation, we refer to a set of vertices , , as a \emph{constraint} (when we are actually referring to  applied to rankings of ). A constraint  is \emph{non-trivial} whenever there exists a ranking  such that . In the following, we always mean non-trivial constraints when speaking of constraints. A constraint  is \emph{satisfied} by a ranking  whenever , in which case  is said to be \emph{consistent} w.r.t.  (we forget the mention \emph{w.r.t. } whenever the context is clear). Otherwise, we say that  is inconsistent. 
Similarly, a ranking  is \emph{consistent} with the constraint system  if it does not contain any inconsistent constraint, and \emph{inconsistent} otherwise. The objective of a ranking -CSP is to find a ranking of the vertices 
with \emph{at most } inconsistent constraints. 
We consider ranking -CSPs where a constraint  can 
be represented by a subset  of \emph{selected vertices}, that determines  the conditions that a ranking must verify in order to satisfy . 
An equivalent formulation of these problems is the 
following: is it possible to \emph{edit} at most  constraints so that there exists 
a ranking consistent with the new constraint system? By \emph{editing a constraint}, 
we mean that we modify its set of selected vertices. \\

We consider \emph{dense} instances, where \emph{every} subset of  vertices of  is a constraint. Let  be an instance of any ranking -CSP. Given a set of vertices , we define the instance \emph{induced by } (and denote it ) as the constraint system  restricted to -sized subsets of . A set of vertices  is a \emph{conflict} if there does not exist any ranking consistent with the instance induced by .  
We mainly study the following problems.\\

\fbox{\begin{minipage}{0.9\textwidth}
	\rBIT{}:\\
	\textbf{Input}: A set of vertices , a constraint system , where a constraint  contains two \emph{selected vertices}  and , , and is satisfied by a ranking  iff  or  holds for , . \\
	\textbf{Parameter}: . \\
	\textbf{Output}: A ranking  of  that satisfies all but at most  constraints. \end{minipage}}	
\0.3cm]

We also consider another generalization of the \FAST{} problem, namely \rFAST{} ({\sc -FAST})~\cite{KS11}, where a constraint  corresponds to an acyclic tournament and is satisfied by a ranking  if 
	 is the transitive ranking of the corresponding tournament.

\begin{figure}[h]
\label{fig:constraints}

	\centerline{\includegraphics[scale=3]{constraints.pdf}}
	\caption{Illustration of  {\sc -BIT},  {\sc -TFAST} and  {\sc -FAST} for . The square vertices represent the 
	selected vertices of the constraints, that are consistent with .}

\end{figure} 

\newpage

\noindent \textbf{Ordered instances.} In the following, we consider instances whose vertices are ordered under some fixed ranking  (\ie instances of the form ). Given any constraint , with  for ,  denotes the set of vertices . A constraint  is \emph{unconsecutive} if 
, and \emph{consecutive} otherwise. Given , 
 denotes the instance  ordered under . Finally, given a ranking  over  and an inconsistent constraint , we say that we \emph{edit  w.r.t. } whenever we edit its selected vertices so that it becomes 
satisfied by .

\section{Simple characterization and sunflower}
\label{sec:general}

In this section, we describe the general framework of our kernelization algorithms, using a modification of the \emph{sunflower rule} together with the notion of \emph{simple characterization}. We first define the notion of sunflower, which has been widely used to obtain polynomial kernels for modification problems~\cite{Abu10,ALS09,BP11,GHPP12}.
An \emph{edition} is a set of constraints  such that one can obtain a consistent instance by editing constraints in .  

\begin{definition}
	A \emph{sunflower}  is a set of conflicts  pairwise intersecting in \emph{exactly one} constraint , called the \emph{center} of .
\end{definition} 

\begin{lemma}
\label{lem:flower}
	Let  be an instance of any ranking -CSP, and  be the center of a sunflower  . Any edition of size at most  has to edit .
\end{lemma}

\begin{proof}
	Let  be any edition of size at most , and assume that  does not edit . This means that  must contain one constraint for every conflict , . Since , we conclude that  contains more than  constraints, a contradiction.
 \end{proof}

Observe that the sunflower rule cannot be applied directly on ranking -CSPs, , since it may be the case that there exist several ways to edit the center of a given sunflower. In order to deal with this issue, we introduce the notion of \emph{simple characterization} for ranking -CSPs. 
Roughly speaking, a ranking -CSP is \emph{-simply characterized} if for any ordered instance, any set of  vertices which involve \emph{exactly one} inconsistent constraint is a conflict. We first give a formal definition of this notion, and next describe how the sunflower rule can be modified for such problems. 

\begin{definition}[Simple characterization]
\label{def:simplecharac}
	Let  be a ranking -CSP,  be any ordered instance of , and . The ranking -CSP  is \emph{-simply characterized} iff any set  of  vertices such that  contains exactly one inconsistent constraint is a conflict. 
	
\end{definition}



\begin{definition}[Simple sunflower]
\label{def:simplesunflower}
	 Let  be an ordered instance of a -simply 
	 characterized ranking -CSP. A sunflower  of 
	  is \emph{simple} if its center is the only inconsistent constraint in , .
	 
\end{definition}

\begin{polyrule}
\label{rule:correct}
	Let  be a -simply  characterized ranking -CSP. Let  be an ordered instance of  
	and , , be a simple sunflower of center . 
	Edit  w.r.t.  and decrease  by .
\end{polyrule}

\begin{lemma}
\label{lem:correct}
	Rule~\ref{rule:correct} is sound. 

\end{lemma}

\begin{proof}
	Let  be any edition of size at most : by Lemma~\ref{lem:flower},  must contain . Since  and , there exists  such that  is the only constraint edited by  in . Assume that  was not edited w.r.t. : since no other constraint has been edited in ,  still contains exactly one inconsistent constraint (namely ). Since  is -simply characterized, it follows that  defines a conflict, contradicting the fact that  is an edition.
 \end{proof}

The main problem that remains is to compute such a sunflower in polynomial time. The following result will allow us to do so, providing that  contains sufficiently many vertices (w.r.t. parameter ). 

\begin{lemma}
	\label{lem:correctSC}
	Let  be a -simply characterized ranking -CSP, and  be an ordered instance of  with at most  inconsistent constraints. If , there exists a simple sunflower , , that can be found in polynomial time.
\end{lemma}

\begin{proof}
	Let  be any inconsistent constraint of . Since  contains at 
	most  inconsistent constraints, there are at most  disjoint sets , , such that  and 
	 contains more than one inconsistent constraint. It follows that there exist at least  disjoint sets  of size  such that:   contains  vertices and   contains \emph{exactly one} inconsistent constraint, . Since  is -simply  characterized,   defines a conflict for every . It follows that  is a simple sunflower of center .\end{proof}

In order to obtain the ranking necessary to apply Lemma~\ref{lem:correctSC}, we rely on the existence of a constant-factor approximation algorithms for the problem at hand. 

\begin{theorem}
\label{thm:kernelSC}
	Let  be a -simply  characterized ranking -CSP that admits a -factor approximation algorithm for some constant . Then  admits a kernel with at most  vertices.
\end{theorem}

\begin{proof}
	Let  be an instance of . We start by computing a ranking  containing  inconsistent constraints using the -factor approximation algorithm. Observe that we can assume that , since otherwise we simply return a small trivial \textsc{Yes}-instance. Similarly, we can assume that , since otherwise we return a small trivial \textsc{No}-instance. We now consider  and assume that : by Lemma~\ref{lem:correctSC}, it follows that there exists a simple sunflower that can be found in polynomial time, and hence Rule~\ref{rule:correct} can be applied. Since conditions of Lemma~\ref{lem:correctSC} still hold after an application of Rule~\ref{rule:correct}, repeating this process on  implies that every inconsistent constraint must be edited. Since , we return a small trivial \textsc{No}-instance in such a case. This means that , implying the result.
 \end{proof}

\section{Simple characterization of several ranking -CSPs}
\label{sec:ranking}

\subsection{{\sc \BIT{}}} 

As a first consequence of Theorem~\ref{thm:kernelSC}, we improve the size of the linear vertex-kernel for 
BIT from ~\cite{PPT11} to  for any . The result directly follows 
from the fact that BIT admits a PTAS~\cite{KS11} and is -simply characterized~\cite{PPT11}. 

\begin{corollary}
\label{thm:kernelbit}
	\BIT{} admits a kernel with at most  vertices.
\end{corollary}

\subsection{{\sc \rBIT{}} ()} 
\label{subsec:rbit}

We now consider the {\sc -BIT} problem with constraints of arity . The main difference with the case  lies in the fact that there is no longer a \emph{unique} way to rank the vertices of a consistent instance in order to satisfy all constraints. In particular, this means that the problem is not 
-simply characterized. 
However, as we shall see in Lemma~\ref{lem:rBITdoubleconflict}, {\sc -BIT} is -simply characterized. To see this, we first need the following result. 

\begin{lemma}
\label{claim:rBITconflict}
	Let  be an ordered instance of -BIT, and  be a set of  vertices such that , . Assume that  contains exactly one inconsistent constraint . Then  is a conflict if and only if:
	\begin{enumerate}[(i)]
		 \item  is unconsecutive or, 
		 \item   and  with  (resp.  and  with ). 
	\end{enumerate}
\end{lemma}

\begin{proof}
	Assume first that the vertices of  are unconsecutive (\ie there exists  such that  is equal to ). 
	
	\begin{itemize}
	
		\item \emph{Case 1.} Assume that a single move on a selected vertex is sufficient to make  consistent. To be more precise, we assume that we have  for some ,  (the case , 
		 and  being similar). Since all constraints induced by  but  are consistent w.r.t. , we know that  with   (which is well-defined since ).

In order to satisfy , any consistent ranking  must rank  between  and , while  implies that  must be between  and . Since no ranking can satisfy both these properties, it follows that there does not exist any ranking consistent with .

	\item \emph{Case 2.} Next, assume that the two selected vertices must be moved to obtain a consistent ranking for . In other words, we have  for some , . Since all constraints but  are consistent, we know that . In 
	order to satisfy , any 
	consistent ranking must rank  between  and , while 
	 implies that  must be before (or after) both 
	 and , which cannot be. 

\end{itemize}

	Assume now that the vertices of the inconsistent constraint  are . Moreover, assume that a single move on a selected vertex is sufficient to make  consistent (see Figure~\ref{fig:two} for an illustration). 
	
	\begin{itemize}
	
	\item \emph{Case 1.} If we have  with , then swapping  and  will yield a consistent ranking. Indeed, observe that since all constraints but  are consistent, it follows that the only constraint where  is a selected vertex is , and that  and  are the selected vertices of the other constraints. Hence swapping  and  will not modify the consistency of any constraint but . This is the only case where we can get a consistent ranking. 
	
	\item \emph{Case 2.} Assume now that  with . Then we have  and . Observe here that in order to satisfy  and , any ranking  must rank  between  and  and between  and . W.l.o.g., this means 
	that we must have , which is inconsistent 
	with . It follows that  is a conflict. 
	
	\end{itemize}
	
	\begin{figure}
	\label{fig:two}
	
		\centerline{\includegraphics[scale=3]{simply.pdf}}
		\caption{An illustration of some cases for . The gray boxes represent the inconsistent constraint .   with ;   with  and  
		 is unconsecutive. In the former case  is not a conflict, while  is a conflict in the latter ones. }
	
	\end{figure}
	
	\begin{itemize}

	\item \emph{Case 3.} Assume next that we have , , and . We know that , with . Once again, this means that any consistent ranking  
	must rank  between  and  in order to satisfy , and  between  and 
	 in order to satisfy ) (recall that ). W.l.o.g., this means that we must have 
	, which is inconsistent with . Hence  is a conflict. 
	
	\item \emph{Case 4.} Finally, assume that the two selected vertices must be moved to obtain a consistent ranking for . In other words, we have  with . We have . This implies that any consistent ranking must rank  between 
 and  (to satisfy ) and  and  between  and 
 (to satisfy ). Since no ranking can satisfy both these 
properties, it follows that  is a conflict. 
	\end{itemize}
	
	The cases where the vertices of the inconsistent constraint are  is similar to the previous one, the only case yielding a consistent ranking being when  and .
\end{proof}

\noindent \textbf{Compatible constraints.}~ 
Given an ordered instance   of -BIT, we say that an inconsistent constraint , , is \emph{right-} (resp. \emph{left-}) \emph{compatible} whenever  with  (resp.  with ), and \emph{right-} (resp. \emph{left-})\emph{incompatible} otherwise. The intuition behind this notion is the following: for any vertex  lying before (resp. after)  in  such that  is the only inconsistent constraint in , the set  does not define a conflict (see Figure~\ref{fig:compatible}). The following result directly follows by definition of a compatible constraint. 

\begin{observation}
\label{obs:comp}
	Any right- (resp. left-)compatible constraint is left- (resp. right-)incompatible. 

\end{observation}

\begin{figure}[t]

	\centerline{\includegraphics[scale=2.25]{compatible.pdf}}
	\caption{Illustration of the notion of left-compatible constraints (only  is inconsistent). By definition, 
	 is not selected in any constraint in , and  and  are selected in every constraint but . Hence swapping  and  yields a consistent ranking. \label{fig:compatible}}
\end{figure}

\begin{lemma}
\label{lem:rBITdoubleconflict}
	The {\sc -BIT} problem is -simply characterized.
\end{lemma}

\begin{proof} 	Observe that compatible constraints correspond to the cases of 
Lemma~\ref{claim:rBITconflict} that fail to define a conflict. 
Let  be an ordered instance of {\sc -BIT} and  be a set of  vertices such that 
	 for . Assume that  contains exactly 
	one inconsistent constraint . We need to prove that  is a conflict. By Lemma~\ref{claim:rBITconflict}, the result holds if  is neither right nor left-compatible. So we assume w.l.o.g. that  is right-compatible. By Lemma~\ref{claim:rBITconflict} we can also assume that the vertices of  are consecutive and are the first of the ranking, since  otherwise  is a conflict by Lemma~\ref{claim:rBITconflict} and we are done (recall that  is left-incompatible by Observation~\ref{obs:comp}). Hence we may assume that  and  for . Moreover, the constraints  and  (with ) have as selected vertices  and . 
In order to be consistent with  and , any ranking  must rank  
between  and , which is inconsistent with the last constraint (which forces  to be between  
and ). 
\end{proof}
\vspace{0.2cm}

\begin{corollary}
\label{thm:kernelrBIT}
	-BIT admits a kernel with at most  vertices.
\end{corollary}

\subsection{{\sc \rFAST{}}} 
\label{subsec:rfast}

Karpinski and Schudy~\cite{KS10} considered a particular generalization of the \FAST{} problem, where every constraint  is satisfied by a \emph{one particular} ranking  and no other (\ie  corresponds to an acyclic tournament). 
We show that the {\sc -TFAST}  problem admits a linear vertex-kernel as a particular case of \emph{fragile} ranking -CSP~\cite{KS11}. We say that a ranking 
-CSP is \emph{strongly fragile} whenever a constraint is satisfied by \emph{one particular ranking} and no other. 

\begin{lemma}
\label{lem:lcrfast}
	Let  be any strongly fragile ranking -CSP, . Then  is -simply  
	characterized. 
\end{lemma}

\begin{proof}
	Let  be an ordered instance of , and  be a 
	set of  vertices such that  contains exactly one inconsistent constraint . We need to prove that  is a conflict. Assume for a contradiction 
	that this is not the case, \ie that there exists a ranking  consistent 
	with . In particular, there exist two vertices  such that 
	 and . Let  be any constraint of  
	such that  (observe that  is well-defined since ). 
	Since  was consistent in  and since  
	is strongly fragile,  is inconsistent in : a contradiction.
 \end{proof}

\begin{corollary}
	Any strongly fragile ranking -CSP admits a kernel with at most   
	vertices. 
\end{corollary}

\subsection{\sc{-Dense Feedback Arc Set} ()}
\label{subsec:rfasht}

As mentioned previously, the {\sc -TFAST}  problem deals with constraints that are given by a transitive tournament and are thus satisfied by \emph{one particular ranking} and no other. To allow more freedom on the satisfiability of a constraint, we consider a different generalization of this problem, namely {\sc -FAST}. Recall that, in this problem, any constraint  contains a selected vertex  and is satisfied by a ranking  if  for any . Observe that {\sc -FAST} is not (weakly-)fragile, since swapping the first two vertices of any consistent ranking yields a consistent ranking. Hence, we cannot directly apply the PTASs from~\cite{KS11}. 
\noindent However, the results needed to obtain a -approximation for \textsc{Feedback Arc Set in Tournaments}~\cite{CFR06} can be generalized to the \FASHT{} problem.

\subsubsection{Approximation algorithm}

\begin{definition}
\label{def:indegree}
	Let  be any instance of {\sc -FAST}, and . The \emph{in-degree} of  is the number of constraints where  is selected.
\end{definition}

\noindent \textbf{Algorithm [{\sc Inc-Degree}]}~ Order the vertices of  according to their increasing in-degrees.

\begin{theorem}
\label{thm:approx}
\textsc{Inc-Degree} is a -approximation for {\sc -FAST}.
\end{theorem}

We prove Theorem~\ref{thm:approx} by proving a series of Lemmata. 
For the sake of simplicity, we let  in the remaining of this Section. For any vertex  of , we call \emph{left constraint} (resp. \emph{above constraint}) any constraint containing  and vertices before (resp. after or both before and after) , and let  (resp. ) be the set of left constraints (resp. above constraints) of . Moreover, we set . Finally, given an ordered instance 
, we define  as the set of inconsistent constraints of , and let . 
We need to define a distance function between two rankings  and :



where  denotes the indicative function. This distance gives the number of constraints which are consistent in exactly one out of the two rankings, and thus generalizes the Kendall-Tau distance between two rankings~\cite{CFR06}.

\begin{lemma}
\label{lem:approxone}
Let  be any ranking. The following holds:

\end{lemma}

\begin{proof}
Let  be any vertex. We set: 


These numbers respectively represent the left constraints where  is the selected vertex, those where  is not the selected vertex and finally the above constraints where  is the selected vertex. Observe that in the last two cases, the considered constraints are inconsistent. 
By definition, we have that . Moreover, we have . Now, observe that:

To see this, observe that any inconsistent constraint  with , , will be counted exactly twice in the sum:   is counted in  and  in , and these are the only vertices for which  will be taken into account.

We conclude the proof by showing that . By the previous observations, we have :

 \end{proof}

In the following, we denote by  the ranking returned by \textsc{Inc-Degree}, and by  the ranking returned by any optimal solution.

\begin{lemma}
\label{lem:approxtwo}
Let  be any ranking. The following holds:

\end{lemma}

\begin{proof}
Observe that this result is trivial if  orders the vertices by increasing in-degrees. So assume this is not the case. This means that there exists  such that  and , . We build a new ranking  from  by swapping  and . We now prove the following (the result will follow by induction): 

To see this, notice that for every vertex , we have , and hence the only terms non equal to  are obtained in  and . This gives :

There are now three main cases to study. Observe that we have , ,  et , and hence ,  and . 

\begin{enumerate}[(i)]
	\item  : in this case, we obtain  since .
	\item  : here, we have  = . One can see that such a term is equal to  if , and to  if .
	\item  : there we have , which in turn is equal to :



We now have to study several cases to be able to conclude (observe that  and ):
\begin{enumerate}[(1)]
	\item  : then the term is equal to .
	\item  : then the term is equal to  . If , the term is equal to , which is positive. If , the whole sum is again equal to . Finally, if , then the term is  which is again positive.
	\item  : the first term is . If , then the sum is  which is non negative. Otherwise, the sum is , which is equal to , positive by hypothesis.
\end{enumerate}
\end{enumerate}
 \end{proof}

\begin{lemma}
\label{lem:approxthree}
Let  be two rankings. The following holds:

\end{lemma}

\begin{proof}
We consider the set of constraints which are inconsistent in  but not in  and denote it by . Similarly, we consider . First observe that:


Moreover, we know by definition that . We need the following result. 

\begin{claim}
\label{claim:tau}
	The following holds: . 

\end{claim}

\begin{proofclaim}
	Let  be any vertex. First, observe that since we are considering a dense instance, we have (assuming w.l.o.g. that ): 
	
	
	
	We now count the number of constraints whose consistency may have been modified by the change of position of  in  and . Let  be the set . Moreover, we define the following sets: 
	
\begin{minipage}{0.4 \linewidth}
	
	\end{minipage}
	\begin{minipage}{0.6 \linewidth}
		\centerline{\includegraphics[scale=1.75]{kendall.pdf}}
		\captionof{figure}{,  and  defined according to .}
		\label{fig:kendall}
	\end{minipage}
	\vspace{0.4cm}
	
	Let  be the set of constraints of  whose consistency have been modified between the two rankings due to the change of position of . Observe that any constraint  in  must contain . Moreover, cannot contain  vertices in , since otherwise its consistency is not modified by moving  (which would be the leftmost vertex of  in both  and ). Since we are on a dense instance, we know that: 
	
	
	
	A similar counting argument can be applied when  (observe that the consistency of any constraint is not modified by  when ). Altogether, we thus have: 
	
	
	
	To conclude the proof, we show that . To see this, notice that any constraint that belongs to  must contain two vertices  and  such that (w.l.o.g.)  and , with the additional property that  and . Indeed, by definition of \rFAST{}, the leftmost vertex of  in  cannot be the leftmost vertex of  in , since otherwise we would have . In particular, this means that  will be counted in (at least) . 
\end{proofclaim}

This concludes the proof of Lemma~\ref{lem:approxthree}.
 \end{proof}

We are now ready to prove the main result of this section :\\

\emph{Proof of Theorem~\ref{thm:approx}.} By the previous Lemmata, we have the following :
			
			Hence we have , which implies the result. 
\hfill \qed


 
\subsubsection{Kernelization algorithm}

In order to design our kernelization algorithm for this problem, we need to study the topology of conflicts that contain exactly one inconsistent constraint. As we shall see, the configuration for \FASHT{} is slightly different to the ones previously observed. In particular, as we shall see, the problem is not -simply characterized. However, the addition of a new reduction rule will allow us to conclude as in the other cases.  

\begin{lemma}
\label{lem:fashtconflict}
	Let  be an ordered instance of {\sc -FAST}, and  be a set of  vertices such that  for every . Assume  contains exactly one inconsistent constraint . Then  is a conflict iff  is unconsecutive or .
\end{lemma}

\begin{proof}
	Assume first that  is unconsecutive (\ie there exists  such that ). Since any constraint different from  is consistent w.r.t. , it follows that  for any . Together with the fact that  for , this implies that any consistent ranking  must verify  and  which cannot be.
	We now deal with the case where  is induced by . Once again, this means that , , and  for any , which is inconsistent with any ranking. 
	Finally, assume that  is induced by . Now, by swapping  
	the vertices  and  (notice that there is no constraint where  is 
	the selected vertex), we 
	obtain a consistent ranking, implying that  is not a conflict in this case. 
 \end{proof}
	
\begin{corollary}
\label{coro:fashtlr}
	There does not exist  such that {\sc -FAST} is -simply characterized. 
\end{corollary}

\begin{proof}
	Let  be an instance of {\sc -FAST} and , . 
	Let  be any set of vertices ordered 
under some ranking  such 
that  for . Assume that  is the only inconsistent constraint of . By Lemma~\ref{lem:fashtconflict}, we know that  is not a conflict, implying that -FAST is not -simply 
characterized. 
 \end{proof}

We need the following rule, 
which implies that the last vertex of any ordered instance of {\sc -FAST} belongs 
to an inconsistent constraint. 

\begin{polyrule}
\label{rule:uselessvertex}
	Let  be any vertex which is selected in every constraint containing it. Remove 
	 and any constraint containing it from . 
\end{polyrule}

\begin{lemma}
\label{lem:uselessvertex}
	Rule~\ref{rule:uselessvertex} is sound and can be applied in polynomial time.
\end{lemma}

\begin{proof}
	First, observe that any edition of size at most  for the original instance will yield an edition for the reduced one. In the other direction, assume that  admits an edition  of size at most , and let  be the consistent ranking obtained after editing the constraints of . Since adding  to the end of  does not introduce any inconsistent constraint,  is also an edition for the original instance.
 \end{proof}

Observe that a given instance can contain at most one such vertex.~We thus iteratively apply this rule until no vertex selected in every constraint containing it remains. 
Given any constraint  of an ordered instance ,  denotes the set containing  and all vertices lying before  in .~Observe that by Lemma~\ref{lem:fashtconflict}, any set  of  vertices such that  contains exactly one inconsistent constraint is a conflict iff . 

\begin{polyrule}
\label{rule:correctFASHT}
	Let  be an ordered instance of {\sc -FAST}  and , , 
	be a simple sunflower of center . Edit  w.r.t.  and decrease  by .

\end{polyrule}

\begin{lemma}
\label{lem:correctFASHT}
	Rule~\ref{rule:correctFASHT} is sound. 

\end{lemma}

\begin{proof}
	Let  be any edition of size at most : by Lemma~\ref{lem:flower},  must contain . Since , there exists  such that  is the only constraint edited by  in . Assume that  was not edited w.r.t. : since no other constraint has been edited in ,  still contains exactly one inconsistent constraint (namely ). Observe now that, by definition of a simple conflict,  holds for any simple sunflower . Hence Lemma~\ref{lem:fashtconflict} implies that  is 
	a conflict, contradicting the fact that  is an edition. 
 \end{proof}

Here again, we can prove the existence of a simple sunflower under a certain cardinality assumption. 

\begin{lemma}
	\label{lem:correctrFAST}
	Let  be an ordered instance of {\sc -FAST}  with at most  inconsistent constraints, and  be an inconsistent constraint with . Then  is the center of a simple sunflower , .
\end{lemma}

\begin{proof}
Since there at most  vertices  such that  contains more than one inconsistent constraint, there exist  vertices  such that  contains exactly one inconsistent constraint, . Since  for , Lemma~\ref{lem:fashtconflict} implies that  is a conflict for . It follows that  is the center of the simple sunflower .
 \end{proof}

\begin{theorem}
\label{thm:fasht}
	{\sc -FAST} admits a kernel with at most  vertices.
\end{theorem}

\begin{proof}
	Let  be an instance of {\sc -FAST} . We start by running the constant-factor approximation (Theorem~\ref{thm:approx}) on , obtaining a ranking  of  with at most  inconsistent constraints. Notice that we may assume  and , since otherwise we return a small trivial \textsc{Yes}- (resp. \textsc{No}-)instance. Assume that  and let  be any inconsistent constraint containing  (thus ). Observe that  is well-defined since  is reduced by Rule~\ref{rule:uselessvertex} (and hence  must belong to an inconsistent constraint). 
	By Lemma~\ref{lem:correctrFAST}, it follows that we can find a simple sunflower ,  in polynomial time. We thus apply Rule~\ref{rule:correctFASHT} and edit  w.r.t. . We now apply Rule~\ref{rule:uselessvertex} and repeat this process until we either do not find a large enough simple sunflower or . In the former case, Lemma~\ref{lem:correctrFAST} implies that , while in the latter case we return a small trivial \textsc{No}-instance.  
 \end{proof}

\section{Conclusion}

In this paper, we considered the kernelization of several dense ranking -CSPs. 
In particular, we proved that any ranking -CSP that can be \emph{simply characterized} and that 
admits a constant factor-approximation algorithm also admits a linear vertex-kernel. Interestingly, our kernelization algorithm makes use of a modification of the classical 
sunflower rule, which usually provides \emph{polynomial} kernels.  
As a main consequence of this result, we improved the size of the kernel for \BIT{}, and described the first polynomial kernels for {\sc -BIT} and {\sc -TFAST}. 
A natural question is thus whether these techniques be applied 
to dense ranking -CSP for (weakly-)fragile constraints~\cite{KS11}? Such problems are known to be fixed-
parameter tractable~\cite{KS10}, but the existence of a polynomial kernel remains an 
open problem. We also considered another generalization of \FAST{}, namely \FASHT{}. We obtained a -approximation algorithm as well as a linear vertex-kernel for this problem. Investigating analogy with \FAST{}, it would be interesting to determine whether this problem admits a PTAS. \\

{\small

\noindent \textbf{Acknowledgments.} Research supported by the AGAPE project  (ANR-09-BLAN-0159). The author would like to thank Christophe Paul, St\'ephan Thomass\'e,  Mathieu Liedloff and Mathieu Chapelle for helpful discussions and comments. 

\bibliographystyle{plain}
\bibliography{ranking}
}

\end{document}