\documentclass[10pt,journal,letterpaper]{IEEEtran}

\usepackage[T1]{fontenc}
\usepackage{ae,aecompl}
\usepackage{epsfig}
\usepackage{amssymb}
\usepackage{amsfonts}
\usepackage{url}
\usepackage[dvips]{color}
\usepackage[usenames,dvipsnames]{xcolor}
\usepackage{amsmath}   

\hyphenation{op-tical net-works semi-conduc-tor meth-od}





\newcommand{\ie}{{\it i.e.}}
\newcommand{\eg}{{\it e.g.}}
\newcommand{\argmin}[1]{\mathop{\mbox{argmin}}_{#1}}
\newcommand{\argmax}[1]{\mathop{\mbox{argmax}}_{#1}}
\newcommand{\pdfarray}[2]{{ \left\{\begin{array}{ll} {#1}, & {#2} \\ 0, & o.w. \end{array}\right. }}
\newcommand{\Order}[1]{\mathcal{O}\left( {#1} \right)}


\definecolor{darkgreen}{rgb}{0,0.5,0}

\begin{document}



\title{Breathfinding: A Wireless Network that Monitors and Locates Breathing in a Home}


\author{Neal Patwari, Lara Brewer, Quinn Tate, Ossi Kaltiokallio, and Maurizio Bocca\thanks{ N.~Patwari and M.~Bocca are with the Department
of Electrical and Computer Engineering, University of Utah, Salt Lake City, USA.  L.~Brewer is with the Department of Anesthesiology, University of Utah Health Sciences Center, Salt Lake City, USA.  Q.~Tate is with the School of Medicine, University of Utah, Salt Lake City, USA.  Ossi Kaltiokallio is with the Department of Automation and Systems Technology, Aalto University School of Electrical Engineering, 
Helsinki, Finland.  This material is based upon work supported by the National Science Foundation under Grant Nos. \#0748206 and \#1035565.  Contact email: npatwari@ece.utah.edu.}}




\maketitle
\begin{abstract}
This paper explores using RSS measurements on many links in a wireless network to estimate the breathing rate of a person, and the location where the breathing is occurring, in a home, while the person is sitting, laying down, standing, or sleeping.  The main challenge in breathing rate estimation is that \emph{motion interference}, \ie, movements other than a person's breathing, generally cause larger changes in RSS than inhalation and exhalation.  We develop a method to estimate breathing rate despite motion interference, and demonstrate its performance during multiple short (3-7 minute) tests and during a longer 66 minute test.  Further, for the same experiments, we show the location of the breathing person can be estimated, to within about 2 m average error in a 56 m apartment. Being able to locate a breathing person who is not otherwise moving, without calibration, is important for applications in search and rescue, health care, and security.
\end{abstract}





\section{Introduction} \label{S:Introduction}
This paper explores using standard wireless devices which measure only received signal strength (RSS) to monitor and localize the breathing of a person, who does not wear any device, in a residential environment.  In comparison, existing breathing monitoring methods either require sensors in contact with the body, or require specialized and expensive radar devices \cite{rivera2006multi,park2006single}.  The core of the idea of non-contact breathing monitoring via RSS is that some links' RSS values, on a particular channel, are highly sensitive to motion near the link line (the line between transmitter and receiver), and that those links can be used, in the absence of other motion, to estimate the person's breathing rate, and in addition, to estimate that a breathing person is near that link line.  From many links' RSS data, and from measurements on multiple frequency channels, we can estimate breathing rate and perform breathing localization. One of the core challenges is that these same links also 
experience rapid change due to other movement, for example, if a person's limb moves.  Our methods are robust to these normal occasional movements because it identifies when a sudden movement occurs, subtracts its effects out, and can still then estimate breathing rate in periods containing that motion.  Our work is the first, to our knowledge, to use RSS measurements to locate where that breathing is occurring.

A significant body of related research has shown that people moving people can be located in a building using a static wireless network that measures links' RSS values \cite{woyach06,zhang2007rf,patwari08b,zhang2009dynamic,kanso09b,wilson09a,chen11sequential}.  This can be done through walls \cite{wilson10see,zhao11noise,zheng2012through}, using a variety of statistics of the measured RSS.  However, state-of-the-art RSS-based non-cooperative localization methods require either: 1) \emph{motion}, that is, a person to be moving at least once during the course of the time in which measurements are collected \cite{wilson10see, youssef07,kaltiokallio2011}, or 2) \emph{empty-area calibration}, measurements of the RSS from the period of time when the area was cleared of any people \cite{patwari08b,wilson09a}.  Even systems sophisticated enough to adaptively learn to distinguish the statistics of RSS during crossing vs.~no crossing require some periods of both to occur \cite{zheng2012through}.  In an search and 
rescue operation or a building fire, people may be unconscious and thus motionless by the time emergency responders arrive, and thus a non-cooperative user localization system may not be able to locate them.  Although some work on robotic RF sensor networks has demonstrated the ability to image concrete structures \cite{mostofi2011compressive}, work has not yet shown that a motionless person could be located without prior empty calibration.  In this paper, we show a method that would locate a stationary but breathing person without an empty-area calibration, although we note that we do not test it through walls or in rubble.

As another application, using standard wireless networks for contact-free breathing monitoring may be useful for health care and wellbeing.  One may be able to use a wireless network deployed in a bedroom as a baby breathing monitor, or as part of an in-home sleep apnea diagnostic system.  
Elder care systems which track and monitor an elderly person's location and activities for abnormal changes could use breathing rates as additional context information.  As of yet, we have not developed algorithms to identify or distinguish the breathing rates of two people in the same room.  However, in these applications, breathing monitoring is most critical when the person is alone, and thus no other person is present to notice their apnea or their stoppage of breathing.  

Finally, in general, breathing rate provides a measure of a person's context, and context-aware computing systems could benefit.  Perhaps ``smart'' homes could also be ``empathetic'' and respond appropriately when your at-rest breathing rate is abnormally high.  A scary movie could adapt to the viewer's measured level of stress.








In the area of contact-free breathing monitoring, most systems use specialized radars.  Doppler radar devices can be used \cite{park2006single}, but slow and small breathing movements mean that the Doppler shift is very low unless the center frequency is very high.  As a result, commercial Doppler radar breathing monitoring products are expensive and out of reach for many applications. For example, the Kai Medical ``Continuous'' system \cite{kaimedical}, although not yet FDA approved, is said to be priced at \d^4d^2SClT_lR_lF_lL=C S(S-1)Tnlr_l[n]S=33C=415202526T=428\times\bullet(x,y)S=12C=5NLf_{min}f_{max}ij=\sqrt{-1}y_l[n]\bar{r}_lly_l[n]\bar{r}_l = \frac{1}{N} \sum_{n=i-N+1}^{i} r_l[n]Nlt\grave{r}_l[n]\acute{r}_l[n]Qr_l[m]n\grave{\sigma}^2_l[n]\acute{\sigma}^2_l[n]r_l[m]n\epsilon>0\epsilon=0.5n\tau_l[n]ln\tau_{RMS}[n] \ge \gamma\gammaiNi-N+1i\tilde{r}_l[n]r_l[m]mb_p \le m < b_fb_pnb_fn\tilde{r}_l[n]y_l[n]Nf_{min}f_{max}SCTQ\gammay_l[n]y_l[n]ly_l[n]n=213n=175\gamma=0.8y_l[n]\bullet\square\hat{f} = 0.10\hat{f}_2\pm 1\gamma = 0.5\cdot\vdotsf_{min}f_{min} = 6\hat{f}\cdotlil\hat{f}v = [v_1, \ldots, v_L]^Txx_kkvxW\etaW(l,k)z_{T_l}z_{R_l}lp_kk\lambda_eP_{l}lP_{l}\delta_p\sigma_x^2\delta\lambda_exvG(k,m)G_{k,m} = \sigma_x^2 e^{-\| z_k - z_m \|/\delta}\sigma_x^2x\deltax\Pi\Order{LP}P\hat{x}lv_l\hat{x}v_l\hat{x}\boldsymbol{\times}\boldsymbol{+}\hat{x}\hat{x}SS<12SSS=6S\{1,3,5,7\}\{0,2,4,6\}\{0,1,2,3\}\{0,2,4,6\}\{1,3,5,7\}\{0,2,4,6,8\}\{0,1,2,3\}\{0,1, \ldots, 5\}\{0,1, \ldots, 7\}\{0,1, \ldots, 9\}\{0,1, \ldots, 11\}S=12C=1C=5\{0,2,4,6\}C=5C=1C\{0, \ldots, 11\}\{0,2,4,6\}$, are both shown. }
    \label{F:nap_rmsdm_subset_channels}
\end{figure}


\section{Conclusion} \label{S:Conclusion}

This paper demonstrates key advances for the use of RSS in a wireless network to monitor the breathing of a person in the deployment area.  First, we introduce a new method for mean removal that makes the method robust to small movements, a key requirement to be able to perform RSS-based breathing monitoring in real-world residential scenarios.  Second, we show that the link PSD at the breathing rate can be used to estimate the location of the breathing person.  We test both advances using a multi-channel network deployed in an apartment building.  The results show the possibilities for accurate and reliable breathing monitoring and localization of a breathing person for a variety of applications.  

The area of wireless network breathing monitoring has many unanswered questions.  What are good statistical and radio propagation models that explain a link's ability to measure breathing as a function of the person's position?  Can adaptive algorithms and protocols be used to reduce the sampling requirements, and thus reduce the traffic and energy required for breathing monitoring?  How should breathing be measured with other wireless hardware, such as 802.11 devices?  What algorithms should be used to monitor and track the breathing of multiple people in the same deployment area?  What WLAN protocols could be used to prevent an adversary from surreptitiously using someone's wireless network to eavesdrop on their breathing?  Each of these questions may result in interesting and useful research directions.

\section*{Acknowledgements}
The authors would like to thank Brad Mager, who helped with the experiments.



\begin{thebibliography}{10}
\providecommand{\url}[1]{#1}
\def\UrlFont{\rmfamily}
\providecommand{\newblock}{\relax}
\providecommand{\bibinfo}[2]{#2}
\providecommand\BIBentrySTDinterwordspacing{\spaceskip=0pt\relax}
\providecommand\BIBentryALTinterwordstretchfactor{4}
\providecommand\BIBentryALTinterwordspacing{\spaceskip=\fontdimen2\font plus
\BIBentryALTinterwordstretchfactor\fontdimen3\font minus
  \fontdimen4\font\relax}
\providecommand\BIBforeignlanguage[2]{{\expandafter\ifx\csname l@#1\endcsname\relax
\typeout{** WARNING: IEEEtran.bst: No hyphenation pattern has been}\typeout{** loaded for the language `#1'. Using the pattern for}\typeout{** the default language instead.}\else
\language=\csname l@#1\endcsname
\fi
#2}}

\bibitem{rivera2006multi}
N.~Rivera, S.~Venkatesh, C.~Anderson, and R.~Buehrer, ``Multi-target estimation
  of heart and respiration rates using ultra wideband sensors,'' in \emph{14th
  European Signal Processing Conference}, 2006.

\bibitem{park2006single}
B.-K. Park, S.~Yamada, O.~Boric-Lubecke, and V.~Lubecke, ``Single-channel
  receiver limitations in doppler radar measurements of periodic motion,'' in
  \emph{IEEE Radio and Wireless Symposium}, Jan. 2006, pp. 99--102.

\bibitem{woyach06}
K.~Woyach, D.~Puccinelli, and M.~Haenggi, ``Sensorless sensing in wireless
  networks: Implementation and measurements,'' in \emph{WiNMee 2006}, April
  2006.

\bibitem{zhang2007rf}
D.~Zhang, J.~Ma, Q.~Chen, and L.~M. Ni, ``{An {RF}-based system for tracking
  transceiver-free objects},'' in \emph{IEEE PerCom'07}, 2007, pp. 135--144.

\bibitem{patwari08b}
N.~Patwari and P.~Agrawal, ``Effects of correlated shadowing: Connectivity,
  localization, and {RF} tomography,'' in \emph{IEEE/ACM Int'l Conf.~on
  Information Processing in Sensor Networks (IPSN'08)}, April 2008, pp. 82--93.

\bibitem{zhang2009dynamic}
D.~Zhang, J.~Ma, Q.~Chen, and L.~M. Ni, ``Dynamic clustering for tracking
  multiple transceiver-free objects,'' in \emph{IEEE PerCom'09}, 2009, pp.
  1--8.

\bibitem{kanso09b}
M.~A. Kanso and M.~G. Rabbat, ``Compressed {RF} tomography for wireless sensor
  networks: Centralized and decentralized approaches,'' in \emph{5th IEEE Intl.
  Conf. on Distributed Computing in Sensor Systems (DCOSS-09)}, Marina Del Rey,
  CA, June 2009.

\bibitem{wilson09a}
J.~Wilson and N.~Patwari, ``Radio tomographic imaging with wireless networks,''
  \emph{IEEE Trans.~Mobile Computing}, vol.~9, no.~5, pp. 621--632, May 2010,
  appeared online 8 January 2010.

\bibitem{chen11sequential}
X.~Chen, A.~Edelstein, Y.~Li, M.~Coates, M.~Rabbat, and M.~Aidong, ``Sequential
  {M}onte {C}arlo for simultaneous passive device-free tracking and sensor
  localization using received signal strength measurements,'' in \emph{ACM/IEEE
  Information Processing in Sensor Networks (IPSN)}, April 2011.

\bibitem{wilson10see}
J.~{Wilson} and N.~Patwari, ``See through walls: motion tracking using
  variance-based radio tomography networks,'' \emph{IEEE Trans.~Mobile
  Computing}, vol.~10, no.~5, pp. 612--621, May 2011, appeared online 23
  September 2010.

\bibitem{zhao11noise}
Y.~Zhao and N.~Patwari, ``Noise reduction for variance-based device-free
  localization and tracking,'' in \emph{8th IEEE Conference on Sensor, Mesh and
  Ad Hoc Communications and Networks (SECON'11)}, June 2011.

\bibitem{zheng2012through}
Y.~Zheng and A.~Men, ``Through-wall tracking with radio tomography networks
  using foreground detection,'' in \emph{Proc. Wireless Communications and
  Networking Conference (WCNC 2012)}, April 2012, pp. 3278--3283.

\bibitem{youssef07}
M.~Youssef, M.~Mah, and A.~Agrawala, ``Challenges: device-free passive
  localization for wireless environments,'' in \emph{MobiCom '07: ACM Int'l
  Conf.~Mobile Computing and Networking}, 2007, pp. 222--229.

\bibitem{kaltiokallio2011}
O.~Kaltiokallio and M.~Bocca, ``Real-time intrusion detection and tracking in
  indoor environment through distributed rssi processing,'' in \emph{2011 IEEE
  17th Intl. Conf. Embedded and Real-Time Computing Systems and Applications
  (RTCSA)}, vol.~1, Aug. 2011, pp. 61 --70.

\bibitem{mostofi2011compressive}
Y.~Mostofi, ``Compressive cooperative sensing and mapping in mobile networks,''
  \emph{IEEE Trans.~Mobile Computing}, vol.~10, no.~12, pp. 1769--1784, Dec.
  2011.

\bibitem{kaimedical}
\BIBentryALTinterwordspacing
{Kai Medical, Inc.} Kai continuous. [Online]. Available:
  \url{http://www.kaimedical.com/en2/kaicontinuous.php}
\BIBentrySTDinterwordspacing

\bibitem{rappaport2001com}
T.~Rappaport, \emph{Wireless Communications: Principles and Practice},
  2nd~ed.\hskip 1em plus 0.5em minus 0.4em\relax Upper Saddle River, NJ, USA:
  Prentice Hall PTR, 2001.

\bibitem{stutzman}
W.~L. Stutzman and G.~A. Theile, \emph{Antenna Theory and Design}.\hskip 1em
  plus 0.5em minus 0.4em\relax John Wiley \& Sons, 1981.

\bibitem{patwari11breathing}
\BIBentryALTinterwordspacing
N.~Patwari, J.~Wilson, S.~{Ananthanarayanan P.R.}, S.~K. Kasera, and
  D.~Westenskow, ``Monitoring breathing via signal strength in wireless
  networks,'' {arXiv.org}, Tech. Rep. arXiv:1109.3898v1 [cs.NI], Sept. 2011.
  [Online]. Available: \url{http://arxiv.org/abs/1109.3898}
\BIBentrySTDinterwordspacing

\bibitem{murray86}
J.~F. Murray, \emph{The Normal Lung: The Basis for Diagnosis and Treatment of
  Pulmonary Disease}, 2nd~ed.\hskip 1em plus 0.5em minus 0.4em\relax W.B.
  Saunders Co., 1986.

\bibitem{tidongle}
\BIBentryALTinterwordspacing
{Texas Instruments}. {A USB-enabled system-on-chip solution for 2.4 GHz IEEE
  802.15.4 and ZigBee applications}. [Online]. Available:
  \url{http://www.ti.com/lit/ds/symlink/cc2531.pdf}
\BIBentrySTDinterwordspacing

\bibitem{kaltiokallio2012follow}
O.~Kaltiokallio, M.~Bocca, and N.~Patwari, ``Follow {@grandma}: long-term
  device-free localization for residential monitoring,'' in \emph{7th IEEE
  International Workshop on Practical Issues in Building Sensor Network
  Applications (SenseApp 2012)}, October 2012.

\bibitem{kay1993detection}
S.~M. Kay, \emph{Fundamentals of statistical signal processing, Volume II:
  Detection theory}.\hskip 1em plus 0.5em minus 0.4em\relax New Jersey:
  Prentice Hall, 1993.

\bibitem{cook2011major}
T.~Cook, N.~Woodall, J.~Harper, and J.~Benger, ``{Major complications of airway
  management in the {UK}: Results of the fourth national audit project of the
  {R}oyal {C}ollege of {A}naesthetists and the {D}ifficult {A}irway {S}ociety.
  Part 2: Intensive care and emergency departments},'' \emph{British Journal of
  Anaesthesia}, vol. 106, no.~5, pp. 632--642, 2011.

\bibitem{white1985metabolic}
D.~P. White, J.~V. Weil, and C.~W. Zwillich, ``Metabolic rate and breathing
  during sleep,'' \emph{J. Appl. Physiol.}, vol.~59, pp. 384--391, 1985.

\bibitem{katayose2009metabolic}
Y.~Katayose, M.~Tasaki, H.~Ogata, Y.~Nakata, K.~Tokuyama, and M.~Satoh,
  ``Metabolic rate and fuel utilization during sleep assessed by whole-body
  indirect calorimetry,'' \emph{Metabolism Clinical and Experimental}, vol.~58,
  pp. 920--926, 2009.

\bibitem{douglas1982respiration}
N.~J. Douglas, D.~P. White, C.~K. Pickett, J.~V. Weil, and C.~W. Zwillich,
  ``Respiration during sleep in normal man,'' \emph{Thorax}, vol.~37, pp.
  840--844, 1982.

\end{thebibliography}





\end{document}
