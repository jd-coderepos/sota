
\pdfoutput=1
\documentclass[11pt,letter]{article}
\usepackage{algorithmic}
\usepackage[ruled]{algorithm}
\usepackage{amsmath}
\usepackage{amssymb}
\usepackage{amsthm}
\usepackage{cite}
\usepackage{fullpage}
\usepackage{listings}
\usepackage{graphicx}
\usepackage{palatino}
\usepackage{url}

\long\def\omit#1{\relax}

\newtheorem{theorem}{Theorem}[section]
\newtheorem{lemma}[theorem]{Lemma}
\newtheorem{definition}[theorem]{Definition}
\newtheorem{corollary}[theorem]{Corollary}

\begin{document}

\lstset{language=Python}

\title{Optimal Angular Resolution for Face-Symmetric Drawings}

\author{David Eppstein and Kevin A. Wortman \\
Department of Computer Science \\
Universitiy of California, Irvine \\
\emph{\{eppstein, kwortman\}@ics.uci.edu}}

\maketitle

\begin{abstract}
Let  be a graph that may be drawn in the plane in such a way that all internal faces are centrally symmetric convex polygons. We show how to find a drawing of this type that maximizes the angular resolution of the drawing, the minimum angle between any two incident edges, in polynomial time, by reducing the problem to one of finding parametric shortest paths in an auxiliary graph.  The running time is at most , where  is a parameter of the input graph that is at most  but is more typically proportional to .
\end{abstract}

\section{Introduction}
Angular resolution, the minimum angle between any two edges at the
same vertex, has been recognized as an important aesthetic criterion
in graph drawing since its introduction by Malitz and Papakostas in
1992~\cite{MalPap-STOC-92}. Much past work on angular resolution in
graph drawing has focused on bounding the resolution by some function
of the vertex degree or other related quantities, rather than on exact optimization; however, recently,
Eppstein and Carlson~\cite{EppCar-GD-06} showed that, when drawing
trees in such a way that all faces form (infinite) convex polygons,
the optimal angular resolution may be found by a simple linear time
algorithm. The resulting drawings have the convenient property that
the lengths of the tree edges may be chosen arbitrarily, keeping fixed
the angles selected by the optimization algorithm, and no crossing can
occur; therefore, one may choose edge lengths either to achieve other
aesthetic goals such as good vertex spacing or to convey additional
information about the tree.

In this paper, we consider similar problems of calculating edge angles
with optimal angular resolution for a more complicated class of graph
drawings: planar straight line drawings in which each internal face of
the drawing is a centrally-symmetric convex polygon. In previous
work~\cite{Epp-GD-04}, we investigated drawings of this type, which we called \emph{face-symmetric drawings}.  We characterized them as the duals of weak pseudoline arrangements in the
plane, and described an algorithm that finds a face-symmetric drawing
(if one exists) in linear time, based on an SPQR-tree decomposition of
the graph. Graphs with face-symmetric drawings are automatically
\emph{partial cubes} (or, equivalently, media~\cite{EppFalOvc-07}),
graphs in which the vertices may be labeled by bitvectors in such a
way that graph distance equals Hamming distance;
see~\cite{EppFalOvc-07} for the many applications of graphs of this
type.

\begin{figure}[t]
\centering
\includegraphics[width=3in]{h220}
\qquad
\includegraphics[width=3in]{c220}
\caption{A hyperbolic line arrangement (left), and its dual
squaregraph (right), drawn by our previous un-optimized algorithm with
all faces rhombi.}
\label{fig:old-220}
\end{figure}

Figure~\ref{fig:old-220} depicts an example, a hyperbolic line
arrangement with no three mutually intersecting lines, the
intersection graph of which requires five colors, as constructed by
Ageev~\cite{Age-DM-96}, and the planar dual graph of the arrangement.
The dual of a hyperbolic line arrangement with no three mutually
intersecting lines, such as this one, is a \emph{squaregraph}, a type
of planar median graph in which each internal face is a quadrilateral
and each internal vertex is surrounded by four or more
faces~\cite{CheDraVax-SODA-02, bendelt_chepoi_eppstein}. Any squaregraph may be
drawn in such a way that its faces are all rhombi, by our previous
algorithm~\cite{Epp-GD-04}, and the drawing produced in this way is
shown in the figure. However, note that some rhombi have such sharp
angles that they appear only as line segments in the figure, making
them difficult to view. Other rhombi, even those near the edge of the
figure, are drawn with overly wide angles, making them very legible
but detracting from the legibility of other nearby rhombi. Thus, we
are led to the problem of spreading out the angles more uniformly
across the drawing, in such a way as to optimize its angular
resolution, while preserving the property that all faces are rhombi.

As we show, this problem of optimizing the angular resolution of a
face-symmetric drawing may be solved in polynomial time, by
translating it into a problem of finding parametric shortest paths in
an auxiliary network. In the parametric shortest path problem, each
edge of a network is given a length that is a linear function of a
parameter ; substituting different values of  into
these functions gives different real weights on the edges and
therefore different shortest path problems~\cite{Gus-JACM-83}. The
number of different shortest paths formed in this way, as  varies, may
be superpolynomial~\cite{Car-84}, but fortunately for our problem we
do not need to find all shortest paths for all values of . Rather, the
optimal angular resolution we seek may be determined as the maximum
value of  for which the associated network has no negative
cycles, and the drawing itself may be constructed using distances from
the source in this network at the critical value of . The
linear functions of our auxiliary network have a special
structure---each is either constant or a constant minus
---that allows us to apply an algorithm of Karp and
Orlin~\cite{KarOrl-DAM-81} for solving this variant of the parametric
shortest path problem, and thereby to optimize in polynomial time the
angular resolution of our drawings.

\section{Drawings with symmetric faces}
\label{section:drawings}

The algorithm of~\cite{Epp-GD-04} can be used to find a non-optimal face-symmetric
drawing, when one exists, in linear time; therefore, in the algorithms
here we will assume that such a drawing has already been given. We
summarize here the relevant properties of the drawings produced in
this way.

In a face-symmetric drawing, any two opposite edges of any face must
be parallel and of equal length. The transitive closure of this
relation of being opposite on the same face partitions the edges of
the drawing into equivalence classes, which we call \emph{zones}; any
zone consists of a set of edges that all have equal angles and equal
lengths. (Note, however, that edges in different zones may also have
equal angles and equal lengths.) The collection of line segments
connecting opposite pairs of edge midpoints within each face forms a
collection of curves which can be extended to infinity to form a
\emph{weak pseudoline arrangement}; each zone is formed by the drawing
edges that cross one of the curves of this arrangement. This
construction is depicted in Figure~\ref{fig:wpla}.

\begin{figure}[t]
\centering\includegraphics[scale=.6]{wpla}
\caption{Converting a face-symmetric drawing into a weak pseudoline
arrangement, by drawing line segments connecting opposite pairs of
edge midpoints within each face. From~\cite{Epp-GD-04}.}
\label{fig:wpla}
\end{figure}

Thus, the positions of all the vertices of the drawing are determined,
up to translation of the whole drawing, by a choice of a vector for each zone that specifies the direction and length of the zone's edges. For, if an arbitrarily chosen base vertex has its
placement fixed at the origin, the position of any other vertex 
must be the sum of the vectors corresponding to the weak pseudolines
that separate  from the base vertex. All vertex positions may be
computed by performing a depth first search of the graph, setting the
position of each newly visited vertex  to be the position of 's
parent in the depth first search tree plus the vector for the zone
containing the edge connecting  to its parent.

The algorithm from~\cite{Epp-GD-04} determines the vectors for each
zone as follows. Associate a unit vector with each end of each of the
pseudolines dual to the drawing, in such a way that these unit vectors
are equally spaced around the unit circle in the same cyclic order as
the order in which the pseudoline ends extend to infinity. The vector
associated with each zone is then simply the difference of the two
unit vectors associated with the corresponding pseudoline's ends,
normalized so that it is itself a unit vector.

As shown in \cite{Epp-GD-04}, this choice of a vector for each zone
leads to a face-symmetric drawing without crossings. Additionally, it
has two more properties, both of which are important and will be preserved by our optimization algorithm. First, if the planar embedding
chosen for the given graph has any symmetries, they will be reflected
in the dual pseudoline arrangement, in the choice of zone vectors, and
therefore in the resulting drawing.

Second, although it may not be possible to draw the given graph in
such a way that the outer face is convex, its concavities are all
mild. This can be measured by defining the \emph{winding number} of
any point  on the boundary of the drawing, with respect to any
other point  also on the boundary, as the sum of the turning angles
between consecutive edges along a path counterclockwise around the
boundary from  to . For any simple polygonal boundary, the
winding number from  to  is  minus the turning angle from
 to . In a convex polygon, all winding numbers would lie in the
range ; in the drawings produced by this method, they
instead lie in the range . That is, intuitively, the
sides of any concavity may be parallel but may not turn back towards
each other. It follows from this property that the vectors of each
zone may be scaled independently of each other, preserving only their
relative angles, and the resulting drawing will remain
planar~\cite{Epp-GD-04}. In particular, the step in the algorithm
of~\cite{Epp-GD-04} in which the zone vectors are normalized to unit
length leads to a planar drawing. A similar constraint on winding
numbers was used in the algorithm of~\cite{EppCar-GD-06} for finding
the optimal angular resolution of a tree drawing, and again led to the
ability to adjust edge lengths arbitrarily while preserving the
planarity of the drawing. Indeed, tree drawings may be seen as a very
special case of face-symmetric drawings in which there are no internal
faces to be symmetric.

Thus, we may formalize the problem to be solved, as follows: given a
face-symmetric drawing of a graph , described as a partition of the
edges of  into zones and a zone vector for each zone, we wish to
find a new set of zone vectors to use for a new drawing of ,
preserving the relative orientation of any two edges that meet at a
vertex of , maintaining the constraint that all winding numbers lie
in the range , and maximizing the angular resolution of
the resulting drawing.

\section{Algorithmic results}
\label{section:algorithm}

In this section we describe a polynomial time algorithm for the problem at hand.  As described above, the core of our algorithm is a routine that maps a planar face-symmetric drawing  to an auxiliary graph , such that the auxiliary graph may be used as input to a parametric shortest paths algorithm of Karp and Orlin \cite{KarOrl-DAM-81}, and the output of that algorithm describes a drawing  isomorphic to  and with maximal angular resolution.  The algorithm of \cite{KarOrl-DAM-81} runs in polynomial time, so what remains to be seen is the correctness and running time of our mapping routine.

Table \ref{table:relationships} summarizes the intuitive relationships between concepts in the drawings  and  and their representation in the auxiliary graph .  As shown in the table, edge zones correspond to auxiliary graph vertices; angular resolution corresponds to the parametric variable ; the change in angle of zones between  and  corresponds to lengths of shortest paths in ; and constraints on the angles are represented by edges in .  Figure \ref{figure:example_input} shows a small example input graph  with five zones, and Figure \ref{figure:example_aux} shows the corresponding auxiliary graph .

\begin{table}
\centering
\begin{tabular}{|p{3in}|p{3in}|}
\hline
\bf{Concept in drawing} & \bf{Representation in auxiliary graph} \\ \hline \hline
Zone  & Vertex  \\ \hline
Angular resolution & Parametric variable  \\ \hline
 is feasible angular resolution & No negative cycles when  \\ \hline
Difference between angles of vectors for zone  in unoptimized and optimized drawings & Shortest path distance  from  to  \\ \hline
Angle between  and  is  & Edge with weight  \\ \hline
Interior faces are convex & Edge with weight  \\ \hline
Exterior boundary is mildly convex & Opposing edges with weight   and  \\ \hline
\end{tabular}
\caption{Relationships between drawing concepts and auxiliary graph components.}
\label{table:relationships}
\end{table}

\begin{figure}[t]
\centering
\vspace{.5in}
\includegraphics[width=4in]{example} \hspace{.5in} \includegraphics[width=1.5in]{dial}
\caption{Example input graph , using degrees rather than radians.}
\label{figure:example_input}
\end{figure}

\begin{figure}[t]
\centering
\includegraphics[width=6in]{aux_fixed}
\caption{Auxiliary graph  for the input shown in Figure \ref{figure:example_input}, again using degrees.  Only the lowest-weight edge between any pair of vertices is shown.}
\label{figure:example_aux}
\end{figure}

We now define our notation formally and show that our reduction is correct.

\begin{definition}
Let
\begin{itemize}
\item  be the planar face-symmetric graph and its drawing given as input;
\item  be the zones of ;
\item  be the number of zones in ;
\item  be a weighted, directed, auxiliary graph, such that every zone  corresponds to a vertex  in , and every edge of  has weight , where  and  are per-edge constants, and  is a graph-wide variable;
\item  be a special start vertex in ;
\item  have edges  and  for every pair of vertices ;
\item  be the largest value of  such that  contains no negative cycles;
\item  be the weight of the shortest path from  to  in  when ;
\item  be the angle assigned to edges of zone  in , in radians counterclockwise from the -axis;
\item  be the counterclockwise turning angle from  to  in ;
\item for every pair of distinct zones , such that  and there exists a face in  with some edge from  and some edge from ,  contains the following edges:
   \begin{itemize}
   \item an edge  with weight ,
   \item if a corresponding angle in  is interior, an edge  with weights , and
   \item if a corresponding angle in  is exterior, opposing edges with weights  and ;
   \end{itemize}
\item  be the output drawing of ;
\item  be the angle assigned to edges in zone  in the output drawing , in radians counterclockwise from the -axis;
\item and  be the angle between  and  in , analogous to  in .
\end{itemize}
\end{definition}

\begin{lemma}
\label{lemma:path_lengths}
If  contains any edge from  to  with weight , then .
\end{lemma}

\begin{proof}
The shortest path to  in , followed by the edge , is a path to  with total weight .  Since  is defined for ,   has no negative cycle; so the shortest path to  has weight .
\end{proof}

\begin{lemma}
\label{lemma:angles}
 has angular resolution , and every interior face of  is non-concave.
\end{lemma}

\begin{proof}
Let  and  be any pair of edges in  that form an angle on some interior or exterior face.  If  has the lesser absolute angle measure, then by construction there exists an edge  with weight , where  and  are the zones for  and , respectively.  Thus by Lemma \ref{lemma:path_lengths},

\noindent so every corner angle in  has measure no less than .

Now suppose  is interior.  Then by construction there also exists an opposing edge  with weight .  So again by Lemma \ref{lemma:path_lengths},

\noindent which implies that  contains no concave face.
\end{proof}

\begin{lemma}
\label{lemma:winding_numbers}
The winding number of any  and  on the boundary of  is in the range .
\end{lemma}

\begin{proof}
As in Lemma \ref{lemma:angles}, let  and  be edges in  such that .  If  and  form an angle on the boundary of , then the corresponding vertices  and  in  have opposing edges with weights  and .  Then by algebraic manipulations symmetric to those in Lemma \ref{lemma:angles},

\end{proof}

\begin{theorem}
The output graph  is a planar face symmetric drawing isomorphic to the input drawing , such that all boundary winding numbers lie in the range , and the angular resolution of  is maximal for any such drawing.  Given ,  may be generated in  time.
\end{theorem}

\begin{proof}
Identifying the zones of  and constructing  may be done naively in  time.  The value  and corresponding distances  for every vertex  in  may be reported by invoking the algorithm of Karp and Orlin.  This algorithm runs in  time, where  is the number of vertices in the input; our graph  has  vertices, so this step takes  time.  The  distances define the zone angles , allowing the drawing  to be output in linear time as described in Section \ref{section:drawings}.

Our method of choosing new angles for the zone vectors implies that  is isomorphic to  and every interior face of  is centrally symmetric.  By Lemmas \ref{lemma:angles} and \ref{lemma:winding_numbers}, every interior face of  is convex, and the exterior boundary of  satisfies the convexity constraint.

Finally we must show that  has maximal angular resolution.  Let  be any correct drawing of the graph , and let  have angular resolution .  Then  has no negative cycles for : for, suppose we replace every weight  with , where  is the angle assigned to zone  in .  Any cycle contains an equal number of  and  terms for each  in the cycle, so our transformation does not change total cycle weights, and hence preserves negative cycles.   Each edge has nonnegative weight, so no negative cycles exist in .  We use the Karp-Orlin algorithm to find the largest value  for which  contains no negative cycle, and by Lemma \ref{lemma:angles}, the output graph  has angular resolution .  Thus the angular resolution of  is greater than or equal to that of any correct drawing .
\end{proof}

\section{Experimental results}

We implemented a simplified version of the algorithm described in Section \ref{section:algorithm} in the Python language.  Our simpler algorithm performs a numerical binary search for  rather than an implementation of the Karp-Orlin parametric search algorithm.  We make binary search decisions by generating the auxiliary graph as described in Section \ref{section:algorithm}, substituting in the value of  to obtain real-valued edge weights, then checking for negative cycles with a conventional Bellman-Ford shortest paths computation.  This algorithm runs in  time, where  is the number of pseudolines (zones) in the drawing and  is the number of bits of numerical precision used.  Wall clock time ranged from a matter of seconds for small  to roughly two minutes for  and .

Figure \ref{figure:cd15} shows a sample of input and output for our implementation.  The input is a correct, though suboptimal, planar face-symmetric drawing of a graph with 15 pseudolines generated by the algorithm presented in ~\cite{Epp-GD-04}.  The angular resolution of the output is visibly improved.  Figure \ref{figure:cd220} shows the output of the algorithm when the 220-pseudoline graph shown in Figure \ref{fig:old-220} is used as input.  That unoptimized drawing has angular resolution  radians.  The left output drawing was produced by our optimization algorithm using all correctness constraints, and has angular resolution  radians.  The right drawing is the result of removing the constraint that concavities have opening angles of at least  radians, which may in general result in a nonplanar drawing, but in this case yields a planar drawing with an even greater angular resolution of  radians.

\begin{figure}[t]
\centering
\includegraphics[width=3in]{cd15a} \includegraphics[width=3in]{cd15b}
\caption{Example input (left) and output (right) with 15 pseudolines.}
\label{figure:cd15}
\end{figure}

\begin{figure}[t]
\centering
\includegraphics[width=3.1in]{cd220b} \includegraphics[width=3.1in]{cd220c}
\caption{Output of our implementation for the graph shown in Figure \ref{fig:old-220}, safely optimized (left) and unsafely optimized (right).}
\label{figure:cd220}
\end{figure}

\section{Conclusion}

We have described two algorithms for generating face-symmetric drawings with optimal angular resolution.  The first algorithm runs in strictly cubic time and uses a subsidiary parametric shortest paths algorithm as a black-box subroutine.  The second algorithm runs in pseudopolynomial time and relies on the simpler Bellman-Ford shortest paths algorithm.  We implemented the latter algorithm and found that it generates output that is both numerically and visibly improved over unoptimized drawings.

Finally we offer the following possible directions for future research:
\begin{itemize}
\item We choose the edge length for each zone arbitrarily; can choosing them more carefully lead to improved legibility?
\item Can this approach be applied to related types of drawings, such as the projections of high dimensional grid embeddings also studied in \cite{Epp-GD-04}?
\item Our algorithm respects a fixed embedding.  What about allowing for different embeddings, e.g. flipping parts of the graph at articulation vertices?  Is it still possible to optimize angular resolution efficiently in this more general problem?
\item What about other angle optimization criteria, such as those defined by all angles in the drawing, rather than merely the sharpest angle?
\end{itemize}

\bibliography{oarpdsf}
\bibliographystyle{abbrv}

\end{document}
