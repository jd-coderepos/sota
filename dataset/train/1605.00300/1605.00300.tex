\documentclass{llncs}

\usepackage{times}
\usepackage{graphicx}
\usepackage{amsfonts}
\usepackage{amsmath}
\usepackage{algorithm}
\usepackage{algpseudocode}
\usepackage{cite}
\usepackage{xspace}
\usepackage[pdfpagelabels=false]{hyperref}
\usepackage[caption=false,font=footnotesize]{subfig}
\usepackage{flushend}
\usepackage{epstopdf}
\usepackage{array}
\usepackage{multirow}
\usepackage{wasysym}

\def\baselinestretch{0.94}
\setlength{\abovecaptionskip}{1ex}
 \setlength{\belowcaptionskip}{1ex}
 \setlength{\floatsep}{1ex}
 \setlength{\textfloatsep}{1ex}
\newcommand{\bibfont}{\small}

\newcommand{\sysname}{{CheapSMC}\xspace}
\DeclareMathOperator*{\argmax}{arg\,max}
\DeclareMathOperator*{\argmin}{arg\,min}

\begin{document}

\title{\sysname: A Framework to Minimize SMC Cost in Cloud}

\author{Erman Pattuk\inst{1} \and Murat Kantarcioglu\inst{1} \and Huseyin Ulusoy\inst{1} \and Bradley Malin\inst{2}}
\authorrunning{Pattuk et al.}

\institute{The Univ. of Texas at Dallas \\
\email{erman.pattuk,muratk,huseyin.ulusoy@utdallas.edu}
\and
Vanderbilt University \\
\email{b.malin@vanderbilt.edu}}

\maketitle

\begin{abstract}
Secure multi-party computation (SMC) techniques are increasingly becoming more efficient and practical thanks to many recent novel improvements. The recent work have shown that different protocols that are implemented using different sharing mechanisms (e.g., boolean, arithmetic sharings, etc.) may have different computational and communication costs. Although there are some works that automatically mix protocols of different sharing schemes to fasten execution, none of them provide a generic optimization framework to find the cheapest mixed-protocol SMC execution for cloud deployment.

In this work, we propose a generic SMC optimization framework \sysname that can use any mixed-protocol SMC circuit evaluation tool as a black-box to find the cheapest SMC cloud deployment option. To find the cheapest SMC protocol, \sysname runs one time benchmarks for the target cloud service and gathers performance statistics for basic circuit components. Using these performance statistics, optimization layer of \sysname runs multiple heuristics to find the cheapest mix-protocol circuit evaluation. 
Later on, the optimized circuit is passed to a mix-protocol SMC tool for actual executable generation. Our empirical results gathered by running different cases studies show that significant cost savings could be achieved using our optimization framework.
\end{abstract}

\section{Introduction}
\label{sec:intro}

Over the last couple of years, many two party secure multi-party computation (SMC)  protocols have been  proposed to address different secure computation needs ranging from privacy-preserving face recognition (e.g.,~\cite{cite:sadeghi2010efficient}) to secure biometric identification (e.g.,~\cite{cite:evans2011efficient}). 
In addition, many generic two party circuit evaluation platforms (e.g.~\cite{cite:henecka2010tasty, cite:bogdanov2008sharemind}) have been developed to improve the efficiency of the existing secure protocols.  Most of these platforms (e.g.,~\cite{cite:malkhi2004fairplay}) also provide  high level programming language support that can automatically generate circuits from programs written in C like languages. These recent advances made two party SMC protocols ever more practical and created the possibility of wide spread deployment in practice. 

Still, there are other problems that need to be solved to make these platforms truly practical. One important challenge that received little attention is the performance optimization. Recent work \cite{cite:demmler2015aby} showed that different two party SMC protocols may have different computational and communication cost profiles. For example,  arithmetic sharing based circuit evaluation protocols may be better for certain tasks compared to Yao's garbled circuit evaluation protocols. On the other hand, boolean secret sharing based circuit evaluation techniques could be best in some situations. It has been shown~\cite{cite:demmler2015aby} that  combination of these three different techniques could perform much better than using any single one of them.  This raises the following research question: \textit{how to find the best combination of the different two party SMC techniques for a given task?}

Unfortunately, most of the existing work do not consider the problem of finding the optimal combination of different sharing based protocols for a given task and require the user to manually choose the specific sub-protocols. Although there were some recent work that tries to optimize and automate the selection process (e.g.,~\cite{cite:kerschbaum2014automatic}), that work had limited scope with respect to the cost dimensions that the user can choose to optimize. For example, if one party leverages a cloud infrastructure for running the protocol, the network communication may significantly impact the overall monetary cost paid by the parties. Therefore, we need an optimization framework that can automatically consider communication, computation and monetary costs when finding the optimal two party SMC protocol composition.

Our goal in this work is to build an optimization framework where the given two party SMC task could be automatically optimized under the given cost constraints; and the optimal (or near optimal) combination of different sharing based sub-protocols (e.g., arithmetic, boolean, Yao's secret sharing protocols, etc.) are selected automatically.  Our goal could be seen similar to other automatic task optimization frameworks seen in other systems. In our optimization framework, we especially focus on the cloud setting since the cloud computing is being widely adopted by organizations~\cite{cite:forbes} due to its flexibility and low initial cost of ownership. In the cloud setting, in addition to minimizing the overall run time of the system, we may need to balance the network traffic, and computation time to achieve overall lowest monetary cost. This makes the problem even more challenging since we may need to consider both communication and computation costs in the optimal mix-protocol circuit generation. 

\textbf{Overview of the \sysname.} The aim of our system is to make it easier for the parties to implement and execute SMC protocols, while minimizing the monetary cost of the SMC execution in the cloud. In order to ease the implementation phase and make our system easily extensible to available SMC tools, we divided \sysname into three main parts. The \emph{Programming API} acts as the front-end for the users that allows them to implement their SMC protocol using our C++ library. In the end, this layer is responsible for representing the user protocol as a circuit of atomic operations. Note that our system can be further extended by designing a custom language (e.g., SFDL of \emph{Fairplay}~\cite{cite:malkhi2004fairplay}), which uses our \emph{Programming API} in the background. In this work, we do not
focus on such integration and instead focus on the optimization aspects.

The \emph{Optimization Module} is responsible for assigning secret sharing schemes (e.g., Arithmetic, Boolean, Yao sharing as in the \emph{ABY} framework~\cite{cite:demmler2015aby}; Additively homomorphic, Yao sharing as in the \emph{Tasty} framework~\cite{cite:henecka2010tasty}, etc.) to the nodes in the circuit, such that the total cost of executing this protocol in cloud is minimized. Since finding an optimal solution to this optimization problem is NP-Hard, the optimization module provides heuristics to find near optimal solutions.

Finally, the \emph{SMC Layer} actually implements the \emph{optimized} circuit given some existing SMC tool (e.g., \emph{ABY}, \emph{Sharemind}~\cite{cite:bogdanov2008sharemind}). Since the proposal for new and more efficient SMC tools is a popular research topic, the design of \sysname does not focus on a single SMC tool that could limit its usefulness. Instead, we leverage a given SMC tool as a black-box, provided that it is a mixed-protocol SMC tool (i.e., allows implementation using different sharing schemes). 

\sysname proposes several atomic operations (e.g., addition, multiplication, binary xor, binary and, etc.) that covers various application scenarios. In Section~\ref{sec:case}, we show the results of applying our system to several different case studies, while further applications can simply be done using our C++ library. Moreover, the process of optimizing the circuit is totally decoupled from the circuit generation and other layers, so that proposing a new heuristic and actually implementing it can be achieved with minimal effort.

\textbf{Our Contributions.} They can be summarized as follows:
\begin{itemize}
	\item We formally define the problem of minimizing the monetary cost of running SMC applications in the cloud setting given for circuit based two party SMC protocols. We particularly focus on the monetary cost of running virtual machines (VMs) and transferring data over the network in the optimization objectives. Our profiling results in Section~\ref{sec:cost} show that the network transfer cost may significantly impact the overall cost. Therefore, making protocol selections based on just optimizing the running time may not be ideal (i.e., optimizing performance only).
	\item We propose an easily extensible system, \sysname, which uses existing SMC tools as a black-box, and propose two different novel heuristics in addition two existing baseline methods to address the optimization problem. Our heuristics specially designed for the circuit based SMC protocols and can be used with any circuit based SMC tool.
	\item We profile the cost of \sysname using Amazon EC2 cloud service. We investigate two different scenarios (i.e., \emph{Inter-Region} and \emph{Intra-Region} VM placement) and four different VM models. For each combination, we derive the average cost of executing atomic operations that highly depends on the VM model and the scenario.
	\item We apply our system to two applications (e.g., biometric matching, matrix multiplication) and evaluate the monetary cost reductions achieved by our platform. We compare our heuristics with existing optimization heuristic of Kerschbaum et al.~\cite{cite:kerschbaum2014automatic}, and pure garbled circuit~\cite{cite:yao1982protocols}. For the Inter-Region scenario, our heuristics gives up to  and  cheaper results compared to pure garbled circuit execution and Kerschbaum et al., respectively. 
\end{itemize}


\section{Background}
\label{sec:background}

\emph{Secret Sharing} is a well-known cryptographic primitive that allows a party to partition its secret input and share those partitions (called a \emph{share}) with other parties.~\cite{cite:shamir1979share}. Due to the nature of the secret sharing, none of the parties (except the party that owns the secret input) can learn anything about the secret input by just looking at a given share. A number of shares should be combined to \emph{reconstruct} the partitioned secret. 

In the remaining of the paper, we will heavily use several different two party secret sharing schemes, such as \emph{Arithmetic}, \emph{Boolean}, \emph{Yao}. We will briefly discuss them here.

\paragraph{Arithmetic Secret Sharing} In this secret sharing scheme, the secret is assumed to an integer  in a ring . The first party, who owns the secret , generates a random number  and subtracts it from the secret to get .  is the first party's share, while  sent to the second party is her share. This type of secret sharing is also called the \emph{Additive Secret Sharing}~\cite{cite:cramer2000general}. 

\paragraph{Boolean Secret Sharing} The key idea of this secret sharing scheme is similar to the previous one,  except the granularity of the sharing. In order to share a secret using \emph{Boolean} sharing, first the secret should be represented in its binary form, then each bit is shared separately in mod . Given the Boolean shares of a secret, the parties can execute any operation using the Goldreich-Micali-Widgerson (GMW) protocol~\cite{cite:goldwasser1987play}. 

\paragraph{Yao Secret Sharing} It is also called the Yao's \emph{Garbled Circuit} protocol, and allows parties to execute any boolean circuit over any number of secret inputs~\cite{cite:yao1986generate}. The key idea of the garbled circuit protocol is that one party, the \emph{garbler}, randomly generates two encryption keys  for each wire  of the circuit. She then encrypts all combinations of the outputs using the generated keys and sends the encryption keys for the corresponding input wires of the party, along with the garbled circuit. The other party, the \emph{evaluator}, decrypts the wires using the keys at hand, and recurses through the garbled circuit by using the decrypted keys. An interested reader is referred to the original paper for the details~\cite{cite:yao1986generate}. 

\paragraph{Conversion Between Different Schemes} Since most SMC tools provide multiple sharing schemes in their implementation, they also provide conversion protocols from one scheme to another. However the certain steps of the conversion protocol may vary in different SMC tools. The most recent SMC tool, the \emph{ABY} Framework, allows conversion amongst Arithmetic, Boolean, and Yao sharing schemes~\cite{cite:demmler2015aby}. The \emph{Sharemind} protocol shows a different method on converting shares between Arithmetic and Yao sharing~\cite{cite:bogdanov2008sharemind}. \emph{Tasty} framework allows conversion between additively homomorphic sharing and Yao sharing~\cite{cite:henecka2010tasty} An interested reader is referred to those works for conversion protocols. 

\section{The Optimal Partitioning Problem}
\label{sec:problem}

\subsection{Problem Definition}
\label{sec:problem:definition}

Let  be the set of provided secret sharing mechanisms. Then given a variable  in some domain , let  represent the secret sharing of  using . 

Next, we define the set of operations  such that each operation  takes a set of parameters that are secretly shared in  and outputs a single variable secretly shared in . Note that the number of inputs that an operation takes is fixed, regardless of the secret sharing scheme. An operation  is \textbf{supported} in , if there exists an execution protocol that takes the input parameters to  and outputs a result secretly shared in . Let  represent the secret sharing mechanisms that support operation . 

Since an operation can be executed in the cloud environment, one should approximate the monetary cost of performing the protocol execution in a certain setup. In order to achieve this, we focus on the processing and network transfer costs of executing a single operation in the pay-as-you-go cloud model. In this computing model, a customer of a cloud provider service is charged a constant amount per unit time for using a particular type of virtual machine (VM), while the prices vary depending on the processing capabilities of the VM. On the other hand, the monetary cost of transferring a single byte in and out of the VM is fixed based on the VM specifications. The unit cost of network transfer vary as the network capacity of the VM changes.

Under such circumstances, we define the processing and network transfer costs of executing an operation  in the secret sharing scheme  as  and , respectively. Furthermore, we define the processing and network transfer cost of converting a variable that is secretly shared in  to  as  and , respectively. Note that defined costs may vary based on VM specifications. 

Given the set of operations , the parties in the computation (i.e., the server and the client) implement a circuit  that is represented as a directed acyclic graph (DAG) and consists of  nodes . Each node represents a single operation and takes input from other nodes, whereas the number of inputs is decided by the operation. To be more concrete, let  represent the operation that is assigned to the node , while  is the set of nodes that supply input to \footnote{Note that this circuit representation of a computation can be provided by the programming language.}. Furthermore, let  be the secret sharing scheme assigned to . Then the monetary cost of executing a node  is simply:


Using the above definitions, the optimal partitioning problem that we investigate in this paper can be formally defined as follows:

\begin{definition}
	Given the processing and network transfer cost for a set of operations  using a set of secret sharing schemes , the cost of conversion between different secret sharing schemes, and a circuit  of  nodes, where each node  is assigned an operation , assign a secret sharing scheme to each node  such that the total monetary cost of executing the circuit with the assigned secret sharing schemes is minimal.


\end{definition}

The partitioning problem is NP-Hard, and the reduction can be simply done from the Integer Programming problem.

\subsection{Assumptions}
\label{sec:problem:assumptions}

We assume that the protocol between the two parties (i.e., the server and the client) is represented as a DAG tree. This can easily be assured by the secure multi-party implementation interface, or by some interpreter that allows programmers to implement the protocol using a custom programming language. The protocols implemented by that interface (or custom programming language) can be converted to the circuit representation. A similar assumption is also made by Kerschbaum et al., where the protocol written in custom language is transformed into a set of static operations~\cite{cite:kerschbaum2014automatic}. 


Finally, we assume that overall secure circuit evaluation is performed sequentially. In order to decrease the running time and reduce the processing cost, one may execute the operations specified by the non-dependent nodes in the circuit in parallel. Under such circumstances, our monetary cost model gives an upper bound on the expected monetary cost. Nevertheless, the cost model can be augmented to reflect this condition by considering the number of concurrent threads. However, the processing cost of executing operations under a secret sharing scheme would not decrease linearly as more threads are used, since some threads may overlap and block each other from uninterrupted execution. Due to such complications, the unit cost calculations may not be accurate.  For this reason, we will focus on the sequential protocol execution scenario in this work; and leave the multi-threaded implementations as a future work.


\vspace{-0.3cm}
\section{The Details of \sysname}
\label{sec:system}

\subsection{Architecture}
\label{sec:system:arch}

\sysname has three main parts: (i) The programming interface (API), (ii) the optimization module, and (iii) the SMC layer. Each \sysname part is responsible for a different task: Programming interface morphs user program into the circuit representation; the optimization module assigns secret sharing schemes to each node in the circuit using some heuristic; the SMC layer generates the executables using state-of-the-art cryptographic primitives and techniques. The user of our system is expected to insert two inputs:  specifications of the protocol that is going to be executed securely, and the unit costs for the operations and secret sharing schemes supported by the SMC framework. As we discuss later, we provide a benchmark suite to automatically learn these unit costs for any target cloud service to help the user. 

\textbf{User Inputs.} The user is assumed to know the secure protocol for the application that \sysname is used for. One input to our system is this protocol specification that can be either implemented using our C++ library or via a possible custom programming language, whose compiler turns the user input to an output compatible with our programming API. Next, the user has to input the unit monetary cost of the operations for the hardware specifications that secure executables will work on. We have implemented a set of benchmark applications that can be easily executed by a user to find the unit costs of each single operation. We would like to stress that this is a one-time operation per the tested cloud environment.  
 
\textbf{Programming API.} We implemented an extensible library in C++ programming language that allows a user of our system to implement a secure protocol (e.g., set intersection, biometric matching, etc.). We provided several operations in the API that will cover a variety of applications. Currently, the set of operations  include addition (), subtraction (), multiplication (), greater (), equality (), multiplexer (), binary xor (), and binary and (). There are also two additional operations input () and output () that allows the programmer to specify secret inputs to the circuit, and to learn the outputs of the protocol execution. 

As a proof of concept, we provided the interface as a C++ library that can be easily used to generate cost-optimal SMC executables. Additionally, this interface can be bound with the compiler of a custom programming language. Using this custom programming language, the user of our system can type the protocol specifications. Then the compiler can simply generate the C++ program that in turn uses \sysname programming interface. In any scenario, the programming interface morphs the protocol to the circuit representation discussed in Section~\ref{sec:problem}. 

\textbf{Optimizer.} Given the circuit representation of the user protocol and the secret sharing schemes that are provided by the SMC layer, the optimizer module applies one of the heuristics (cf. Section~\ref{sec:system:opt}) to assign secret sharing schemes to each node in the circuit. As we discussed previously, this module is responsible for finding the assignment that minimizes the monetary cost of executing the protocol securely. 

Due to the NP-Hard optimal partitioning problem, finding the optimal assignment may be impractical even for a slightly large circuit. Hence, we proposed several heuristics (cf. Section~\ref{sec:system:opt}) that tries to find a reasonable solution. In the background, \sysname applies each heuristic and chooses the one that gives the best result. 

\textbf{SMC Layer.} Once the secret sharing schemes are assigned to each node in the circuit, \sysname gives the circuit to the SMC layer to generate the SMC executables. There are several related works that provide mixed-protocol SMC tools, such as \emph{ABY} framework by Demmler et al~\cite{cite:demmler2015aby}, \emph{TASTY} framework by Henecka et al.~\cite{cite:henecka2010tasty}, \emph{Sharemind} framework by Bogdanov et al.~\cite{cite:bogdanov2008sharemind}, etc. We discuss all those tools in Section~\ref{sec:related}. The SMC layer in our \sysname benefits from such existing tools, and is responsible for automatically implementing the optimized circuit using the selected tool. We designed our system to enable easy integration with any possible SMC tool. Note that since the selected SMC tool provides the low-level implementation of the cryptographic primitives (e.g., oblivious transfer~\cite{cite:naor2001efficient, cite:rabin2005exchange}, multiplication triplets~\cite{cite:beaver1992efficient}, sharing and reconstructing secret inputs, etc.), we focus on optimizing user protocol using sharing assignments.

One clear connection between the SMC layer and the Optimizer module is the dependency on the selected SMC tool. Since each SMC tool may provide a different set of secret sharing mechanisms, the optimizer should perform the assignment such that the optimal circuit can actually be realized and executed by the SMC layer. For instance the \emph{ABY} framework provides three different secret sharing schemes: arithmetic sharing, boolean sharing, and garbled circuit sharing. On the other hand, the \emph{TASTY} framework allows additively homomorphic sharing and garbled circuit sharing. In any case, the user is responsible for running our sample applications only once for the tested environment and input the statistics data, while the SMC layer and the Optimizer exchanges information about the available secret sharing schemes based on selected SMC tool.

\subsection{Optimization Heuristics}
\label{sec:system:opt}
In our work, we developed two new heuristics to solve the optimal partitioning problem in addition to two existing heuristics. Below, we provide the details of these heuristics.

\textbf{Bottom-up Heuristic.} The key idea in this heuristic is to assign \emph{optimal} secret sharing scheme to the nodes in their topological order in the circuit. When a node  is to be processed, the heuristic first assigns sharing schemes to the nodes that provide input to . Based on the values assigned to the \emph{children}, this heuristic selects the scheme that minimizes the expected monetary cost for . 

\textbf{Top-Down Heuristic.} In this technique, we process the nodes in the circuit in the opposite manner compared to the previous heuristic, and assign secret sharing schemes to the higher level nodes first and iterate down to the lower levels. The idea in this heuristic is to assign the scheme that minimizes the cost of the current node given that the schemes for the nodes that it is input to are already known. Assume that the secret sharing scheme for the node  is set to  previously. Now when assigning the secret sharing scheme for the node , this heuristic takes into account that the result of the node  should be converted to . The \emph{optimal} decision is made with this consideration in mind. 

\textbf{Fixed Secret Sharing.} In this optimization heuristic, each node in the circuit is assigned the same secret sharing scheme. However, in some SMC tools, certain secret sharing schemes may not necessarily support each single operation (e.g., Arithmetic sharing in \emph{ABY} framework does not support , , and ). In such a case, this heuristic selects one scheme that supports each \sysname operation. One common secret sharing scheme that is included in almost all SMC tools and supports each operation is garbled circuit sharing that implements Yao's garbled circuit protocol~\cite{cite:yao1982protocols}. Using this heuristic, we can measure the monetary cost of executing the user protocol by a single secret sharing scheme (e.g., pure SMC by garbled circuit sharing).

\textbf{Hill-Climbing.} This heuristic is based on the technique of Kerschbaum et al.~\cite{cite:kerschbaum2014automatic}. The basic idea is to start by assigning a common secret sharing scheme (e.g., garbled circuit sharing) to each node in the circuit. Next, we check if the total cost can be reduced by changing the current secret sharing scheme of a node. This loop continues till the total cost cannot be improved by any further assignment. 

\vspace{-0.6cm}
\begin{algorithm}[hbt]
\footnotesize
	\begin{algorithmic}[1]
		\State \textbf{Known:} , , , 
		\State \textbf{Known:} ~and~, 
		\State \textbf{Known:} ~and~, 
		\State \textbf{Known:} 

		\Function{OptimizeCircuit}{, }
		\ForAll{} 
			\State  
		\EndFor

		\Repeat
			\State 
			\ForAll{}
				\State 
				\If{}
					\State 
					\State 
				\EndIf
			\EndFor
		\Until{}
		\EndFunction
	\end{algorithmic}
	
	\caption{Hill-Climbing heuristic that assigns secret sharing schemes to the nodes in the circuit}
	\label{alg:heuristic:hill}
\end{algorithm}

\vspace{-0.45cm}
Algorithm~\ref{alg:heuristic:hill} details the hill-climbing heuristic. It takes a circuit  and an initial secret sharing scheme  that supports each operation in the circuit as inputs (step 5). First, this given scheme is assigned to each node in the circuit (steps 6-8). Then, for each node in the circuit, it checks if the total cost of this node can be reduced by changing its current secret sharing scheme to another one(step 12). If successful, the change is processed and a flag is set to true to signal another round of loop iteration (steps 13-16). This iteration continue until no improvement can be made (step 18).

\subsection{Prototype Implementation}
\label{sec:system:proto}

As a proof of concept, we implemented a prototype of \sysname using C++ programming language. As discussed before, using the library of  our programming interface, a user can implement protocols that will be compiled into SMC executables. For the SMC layer, we used the ABY framework by Demmler et al.~\cite{cite:demmler2015aby}. The set of sharing schemes that are provided by \emph{ABY} are Arithmetic, Boolean, and garbled circuit sharing (that we refer to as the Yao sharing). Only two operations (i.e.,  and ) are supported for the Arithmetic sharing, which are implemented using multiplications triplets by Beaver et al.~\cite{cite:beaver1992efficient}. Boolean sharing support each \sysname operation, while the low-level implementations by ABY are performed using Goldreich-Micali-Wigderson (GMW) protocol~\cite{cite:goldwasser1987play}. Finally, Yao sharing support each operation in \sysname and executes the operations using Yao's garbled circuit protocol~\cite{cite:yao1982protocols}. Further information about the \emph{ABY} framework can be found in ~\cite{cite:demmler2015aby}.


\vspace{-0.5cm}
\section{Cost Profiling}
\label{sec:cost}

\subsection{Experiment Setup}
\label{sec:cost:setup}

\textbf{Virtual Machines.} We used four different virtual machine models in Amazon EC2 cloud service. Table~\ref{table:cost:ec2_types} shows an overview of the VM specifications as of May 2015. We focused on selecting two different VM \emph{types}. The first two, m3.medium and m3.large, are \emph{memory optimized} models that provide better memory throughput. On the other hand, the last two VM models are \emph{compute optimized} that give faster processing throughput. As can be seen from table~\ref{table:cost:ec2_types}, the number of virtual central processing units (vCPU) and the memory sizes differ in the selected instances to test whether a faster (and more expensive) VM model provides any monetary gains.

\vspace{-0.4cm}
\begin{table}[hbt]
	\scriptsize
	\centering
	\begin{tabular}{|c|c|c|c|}

	\hline
	Model & CPU Type (Intel Xeon) &  vCPU & Memory (GB) \\
	\hline
	m3.medium	& E5-2670 v2 & 1 & 3.75 \\
	m3.large		& E5-2670 v2 & 2 & 7.5 \\
	c4.large		& E5-2666 v3 & 2 & 3.75 \\
	c4.xlarge		& E5-2666 v3 & 4 & 7.5 \\
	\hline
	\end{tabular}

	\caption{The specifications of four different Amazon EC2 VM models as of May 2015.}
	\label{table:cost:ec2_types}
\end{table}

\vspace{-0.9cm}
\textbf{Unit Costs.} We selected two Amazon EC2 regions to initiate our VMs: North Virginia and Tokyo. As of May 2015, customers of Amazon EC2 service are charged hourly, whereas the unit cost of running a VM varies with respect to its specifications. Table~\ref{table:cost:ec2_cost} gives the unit costs of running a VM for an hour in North Virginia or Tokyo Amazon EC2 regions. The network transfer cost also varies with respect to the Amazon EC2 region. The unit network transfer cost in North Virginia for incoming and outgoing network is 0.01/GB\, respectively. For the Tokyo EC2 region, the network costs are 0.01/GB\ for outgoing traffic.

\vspace{-0.3cm}
\begin{table}[hbt]
	\scriptsize
	\centering
	\begin{tabular}{|c|c|c|}
	\cline{2-3}
	\multicolumn{1}{c}{} & \multicolumn{2}{|c|}{Unit Cost ()} \\
	\cline{2-3}
	\multicolumn{1}{c|}{} & N. Virginia & Tokyo \\
	\hline
	m3.medium	& 7		& 10.1 \\
	m3.large		& 14		& 20.3 \\
	c4.large		& 11.6	& 14.7 \\	
	c4.xlarge		& 23.2	& 29.4 \\
	\hline
	\end{tabular}

	\caption{The average cost of running an Amazon EC2 VM model for an hour in two different AWS EC2 regions.}
	\label{table:cost:ec2_cost}
\end{table}

\vspace{-1.0cm}
\textbf{Scenarios.} We consider and perform cost measurements for two different scenarios. In the first scenario that we refer to as the \emph{Intra-Region} scenario, two parties in the computation initiate their VMs in the same Amazon EC2 region, North Virginia. Since Amazon does not charge network transfer between two VMs in the same region, the network cost for each gate is simply zero. On the other hand, the computation cost is simply running two VMs (one for the server, one for the client) for a given specification. 

In the second scenario that we refer to as the \emph{Inter-Region} scenario, one VM is located in North Virginia, while the other is in Tokyo. The processing cost is the cost of running two VMs in two different regions at the same time. We take the unit network transmission cost of transferring as 0.065/GBo_i1000o_i321000o_i10\cent10^{-10}\mathtt{Add}\mathtt{Mul}\mathtt{Add}\mathtt{Mul}\mathtt{Add}\mathtt{Add}\mathtt{Mul}\mathtt{Add}\mathtt{Mul}\mathtt{Xor}\mathtt{Eq}\mathtt{Ge}\mathtt{Add}\mathtt{Sub}\mathtt{Mul}\mathtt{And}\mathtt{Xor}\mathtt{Mux}\mathtt{Eq}\mathtt{Ge}\mathtt{A2B}\mathtt{A2Y}\mathtt{B2A}\mathtt{B2Y}\mathtt{Y2A}\mathtt{Y2B}\cent10^{-10}\cent10^{-6}4\mathtt{Mul}\mathtt{Add}\mathtt{Add}\mathtt{Sub}\mathtt{Mul}\mathtt{And}\mathtt{Xor}\mathtt{Mux}\mathtt{Eq}\mathtt{Ge}\mathtt{A2B}\mathtt{A2Y}\mathtt{B2A}\mathtt{B2Y}\mathtt{Y2A}\mathtt{Y2B}\cent10^{-6}\cent10^{-6}\mathtt{Add}\mathtt{Mul}\mathtt{Mul}\mathtt{Mul}\mathtt{Add}\mathtt{Sub}\mathtt{Mul}\mathtt{And}\mathtt{Xor}\mathtt{Mux}\mathtt{Eq}\mathtt{Ge}\mathtt{A2B}\mathtt{A2Y}\mathtt{B2A}\mathtt{B2Y}\mathtt{Y2A}\mathtt{Y2B}\cent10^{-6}t3053215\%ms\cent10^{-3}\cent10^{-3}\cent10^{-3}0030\%80\%ABnnO(n^3)AB5x5ms3360\%\cent10^{-3}\cent10^{-3}\cent10^{-3}96\%30\%$ less than using pure garbled circuit and Hill-Climbing, respectively. Moreover, we conclude that purchasing faster and more expensive VM model from Amazon EC2 does not necessarily reduce the total monetary cost of executing SMC protocols. In general, compute optimized VMs result in more expenses, while memory optimized ones produce cheaper SMC executions.

\bibliographystyle{abbrv}
\bibliography{references}

\end{document}
