\documentclass[11pt,letterpaper]{article}

\usepackage{amsmath,amsfonts}
\usepackage{amssymb}
\usepackage{amsthm}
\usepackage{graphicx}
\usepackage{verbatim}
\usepackage{mathrsfs}
\usepackage{algorithm,algorithmic}
\usepackage{url}
\usepackage{ogonek}
\usepackage{enumerate}

\usepackage{times}
\usepackage{fullpage}
\addtolength{\hoffset}{0.06in}
\addtolength{\textwidth}{-0.08in}
\addtolength{\voffset}{0.01in}
\addtolength{\textheight}{-0.01in}

\newtheorem{lemma}{Lemma}
\newtheorem{thm}{Theorem}
\newtheorem{cor}{Corollary}
\newtheorem{defn}{Definition}
\newtheorem{obs}{Observation}
\newtheorem{claim}{Claim}

\newcommand{\E}{\mathrm{E}}
\newcommand{\st}{\mbox{-} }

\renewcommand{\algorithmicrequire}{\textbf{Input:}}
\renewcommand{\algorithmicensure}{\textbf{Output:}}

\algsetup{
	linenodelimiter=\mbox{ }
}

\urlstyle{same}

\bibliographystyle{abbrv}

\title{Improving Christofides' Algorithm for the - Path TSP}

\author{ 
Hyung-Chan An
\thanks{
{\tt anhc@cs.cornell.edu}.
Dept. of Computer Science, Cornell University, Ithaca, NY 14853.
Research supported in part by NSF under grants no. CCF-1017688 and CCF-0729102, and the Korea Foundation for Advanced Studies.
Part of this research was conducted while the author was a visiting student at CSAIL, MIT.
}
\and
Robert Kleinberg
\thanks{
{\tt rdk@cs.cornell.edu}.
Dept. of Computer Science, Cornell University, Ithaca, NY 14853.
Supported by NSF grants CCF-0643934 and CCF-0729102, AFOSR grant FA9550-09-1-0100, a Microsoft Research New
Faculty Fellowship, a Google Research Grant, and an Alfred P. Sloan Foundation Fellowship.
}
\and
David B. Shmoys
\thanks{
{\tt shmoys@cs.cornell.edu}.
School of ORIE and Dept. of Computer Science, Cornell University, Ithaca, NY 14853.
Research supported in part by NSF under grants no. CCF-0832782 and CCF-1017688.
Part of this research was conducted while the author was a visiting professor at Sloan School of Management, MIT.
}
}

\date{}

\begin{document}

\maketitle 
\def\thepage {} \thispagestyle{empty}

\iffalse
\begin{quote}
\begin{center}
{\bf Abstract}
\end{center}
\fi

\begin{abstract}
We present a deterministic -approximation algorithm for the \st path TSP for an arbitrary metric. Given a symmetric metric cost on  vertices including two prespecified endpoints, the problem is to find a shortest Hamiltonian path between the two endpoints; Hoogeveen showed that the natural variant of Christofides' algorithm is a -approximation algorithm for this problem, and this asymptotically tight bound in fact has been the best approximation ratio known until now. We modify this algorithm so that it chooses the initial spanning tree based on an optimal solution to the Held-Karp relaxation rather than a minimum spanning tree; we prove this simple but crucial modification leads to an improved approximation ratio, surpassing the 20-year-old barrier set by the natural Christofides' algorithm variant. Our algorithm also proves an upper bound of  on the integrality gap of the path-variant Held-Karp relaxation. The techniques devised in this paper can be applied to other optimization problems as well: these applications include improved approximation algorithms and improved LP integrality gap upper bounds for the prize-collecting \st path problem and the unit-weight graphical metric \st path TSP.
\end{abstract}
\newpage
\pagenumbering {arabic}
\normalsize

\section{Introduction}

After 35 years, Christofides' -approximation algorithm~\cite{C} still provides the best performance guarantee known for the metric traveling salesman problem (TSP), and improving upon this bound is a fundamental open question in combinatorial optimization. For the path variant of the metric TSP in which the aim is to find a shortest Hamiltonian path between given endpoints  and , Hoogeveen~\cite{H} showed that the natural variant of Christofides' algorithm yields an approximation ratio of  that is asymptotically tight, and this has been the best approximation algorithm known for this \st path variant for the past 20 years. Recently, there has been progress for the special case of metrics derived as shortest paths in unit-weight (undirected) graphs: Oveis Gharan, Saberi, and Singh~\cite{OSS} gave a -approximation algorithm for the TSP, which can be extended to yield an analogous result of a -approximation algorithm for the \st path TSP in the same special case (see Appendix~\ref{ap:c53}). M\"omke and Svensson~\cite{MS} gave a -approximation algorithm for the same special case of the TSP, as well as a -approximation algorithm for the \st path TSP in the same case (where the results of Appendix~\ref{ap:c53} and M\"omke \& Svensson~\cite{MS} were obtained independently). We note the techniques devised in these results for the unit-weight graphical metric case proved useful in both path and ordinary (circuit) variants. The main result of this paper is to provide the first improvement for the \emph{general} metric case of the \st path TSP: more specifically, we give a deterministic -approximation algorithm for the metric \st path TSP for an arbitrary metric, breaking the  barrier. It remains an open question whether these techniques can be extended to yield a comparable improvement (over the  barrier) for the general-metric ordinary (circuit) TSP.

Our analysis gives the first constant upper bound on the integrality gap of the path-variant Held-Karp relaxation as well, showing it to be at most the golden ratio, . We will also demonstrate how the techniques devised in the present paper can be applied to other problems, such as the prize-collecting \st path problem and the unit-weight graphical metric \st path TSP, to obtain better approximation ratios and better LP integrality gap upper bounds than the current best known.

Proposed by Held and Karp \cite{HK} originally for the circuit problem, the Held-Karp relaxation~\cite{HK} is a standard LP relaxation to (the variants of) TSP, and has been successfully used by many algorithms \cite{BGSW, G:pc, AGMOS, AKS, OSS, MS, M}. In the LP-based design of an approximation algorithm, one important measure of the strength of a particular LP relaxation is its integrality gap, i.e., the worst-case ratio between the integral and fractional optimal values; however, there exists a significant gap between currently known lower and upper bounds on the integrality gap of the Held-Karp relaxation. For the circuit case, the best upper bound known of  is constructively proven by the analyses of Christofides' algorithm due to Wolsey~\cite{W} and Shmoys \& Williamson~\cite{SW}; yet, the best lower bound known is , achieved by the family of graphs depicted in Figure~\ref{f:ig}(a) under the unit-weight graphical metric~\cite{G:ineq}. For the path problem, Hoogeveen~\cite{H} shows the natural variant of Christofides' algorithm is a -approximation algorithm, but the analysis compares the output solution value to the optimal (integral) solution; therefore it is unclear whether the algorithm yields an integrality gap bound for the Held-Karp relaxation formulated for the path problem. The analysis of the present algorithm, in contrast, reveals an upper bound of  on its integrality gap, matching the approximation ratio; we also show in Appendix~\ref{ap:c53} that an LP-based analysis of Christofides' algorithm proves a weaker upper bound of . We observe that the family of graphs in Figure~\mbox{\ref{f:ig}(b)} establishes the integrality gap lower bound of  under the unit-weight graphical metric. Note that this lower bound is strictly greater than the known upper bound of  on the integrality gap of the circuit-variant Held-Karp relaxation under the unit-weight graphical metric; this suggests that the lack of a performance guarantee known for the \st path TSP matching the  for other TSP variants has a true structural cause.

\begin{figure}
\center
\includegraphics[width=310pt]{f1}
\caption{Examples establishing the integrality gap lower bounds for the circuit- and path-variant Held-Karp relaxations.}
\label{f:ig}
\end{figure}
\begin{figure}
\center
\includegraphics[width=280pt]{f2}
\caption{Example showing  is asymptotically tight~\cite{H}: a minimum spanning tree is marked with thick edges.}
\label{f:53ex}
\end{figure}

A feasible solution to the path-variant Held-Karp relaxation is in the spanning tree polytope; thus, given a feasible Held-Karp solution, there exists a probability distribution over spanning trees whose marginal edge probabilities coincide with the Held-Karp solution. The present algorithm first computes an optimal solution to the Held-Karp relaxation, and samples a spanning tree from a probability distribution whose marginal is given by the Held-Karp solution. Then it augments this tree with a minimum -join, where  is the set of vertices with ``wrong'' parity of degree, to obtain an Eulerian path visiting every vertex; this Eulerian path can be shortcut into an \st Hamiltonian path of no greater cost. Our analysis of this algorithm shows that the expected cost of the Eulerian path is at most  times the Held-Karp optimum; the analysis relies only on the marginal probabilities, and therefore holds for \emph{any} arbitrary distribution with the given marginals. We note that this flexibility enables a simple derandomization: a feasible Held-Karp solution can be efficiently decomposed into a convex combination of polynomially many spanning trees (see Gr{\"o}tschel, Lov{\'a}sz, and Schrijver~\cite{GLS}) and trying every spanning tree in this convex combination yields a simple deterministic algorithm. We also note that our algorithm differs from Christofides' in only one crucial respect: rather than taking a single tree and augmenting it with a -join, we try out polynomially many trees and then take the one whose augmentation yields the lowest-cost path. The example in Figure~\ref{f:53ex} due to Hoogeveen~\cite{H} shows that this simple modification of the original algorithm is crucial to achieving the improved approximation ratio: if one only tries augmenting the minimum spanning tree, the approximation ratio remains no better than .

As the expected cost of the sampled spanning tree is equal to the Held-Karp optimum, the rest of the analysis focuses on bounding the cost of the minimum -join by providing a low-cost \emph{fractional -join dominator} that serves as an upper bound on the cost of the minimum -join. First we show that the Held-Karp solution and the spanning tree, while being costly fractional -join dominators themselves, are complementary: a certain linear combination of them is a fractional -join dominator whose expected cost is no greater than  times the Held-Karp optimum, thereby recovering the same  performance guarantee provided by Hoogeveen's analysis of Christofides' algorithm. Based on this beginning analysis, we present progressively better ways of constructing a low-cost fractional -join dominator. In all of these approaches, we perturb the coefficients of the tree and the Held-Karp solution to reduce the cost of their linear combination, at the expense of potentially violating some constraints of the fractional -join dominator linear program, and then we add a low-cost correction to repair the violated constraints. To construct this correction vector and to bound its cost, we show that the only potentially violated constraints correspond to \emph{narrow cuts} having a layered structure, as illustrated in Figure~\ref{f:layer}. The layered structure allows us to choose disjoint sets of representative edges for each cut and to correct the violated constraints using a sum of vectors each supported on the representative edge set of the corresponding narrow cut. We show that this idea leads to a slight improvement upon , using the fact that the representative edge sets, while being mutually disjoint, occupy a large portion of each cut and that each narrow cut constraint has only a small probability of being violated. After that, we present a tighter analysis with a similar construction. Finally, pushing the performance guarantee towards the golden ratio requires relaxing the disjointness of the representatives to a notion of ``fractional disjointness''. We define this relaxed disjointness, construct the requisite fractionally disjoint vectors via the analysis of an auxiliary flow network, and prove the performance guarantee of . We note that neither the fractional -join dominator nor the narrow cuts are actually computed by the algorithm: these progressive analyses all analyze the same single algorithm while different fractional -join dominators are considered in each analysis. That is, it might be possible to obtain a better performance guarantee for the same algorithm by providing a better construction of a fractional -join dominator. The narrow cuts are purely for the purpose of analysis in Section~\ref{s:improv} and never determined by the algorithm; however, their algorithmic use is explored in Section~\ref{s:appl}.

Section~\ref{s:appl} demonstrates how the present results can be applied to other problems to obtain better approximation algorithms than the current best known. We first consider the metric prize-collecting \st path problem. In a prize-collecting problem, we are given ``prize'' values defined on vertices, and the objective function becomes the sum of the ``regular'' solution cost and the total ``missed'' prize of the vertices that are not included in the solution. For example, the prize-collecting \st path problem finds a (not necessarily spanning) \st path that minimizes the sum of the path cost and the total prize of the vertices not on the path. Chaudhuri, Godfrey, Rao, and Talwar~\cite{CGRT} give a primal-dual -approximation algorithm for this problem. Prize-collecting TSP, the circuit version of this problem, has been introduced in Balas~\cite{B}; Bienstock, Goemans, Simchi-Levi, and Williamson~\cite{BGSW} give a LP-rounding -approximation algorithm, and Goemans \& Williamson~\cite{GW} show a primal-dual -approximation algorithm. For both problems, Archer, Bateni, Hajiaghayi, and Karloff~\cite{ABHK} give improvement on approximation ratios: using the path-variant Christofides' algorithm as a black box, Archer et al. give a -approximation algorithm for the prize-collecting \st path problem; a -approximation algorithm is given for the prize-collecting TSP, using Christofides' algorithm as a black box again. For the prize-collecting (circuit) TSP, Goemans~\cite{G:pc} combines Bienstock et al.~\cite{BGSW} and Goemans \& Williamson~\cite{GW} to obtain a -approximation algorithm, the current best known.

As the analysis of Archer et al.~\cite{ABHK} treats Christofides' algorithm as a black box, replacing this with the present algorithm readily gives an improvement. Furthermore, we will show that, since the present analysis produces the performance guarantee in terms of the Held-Karp optimum, Goemans' analysis~\cite{G:pc} can be extended to the prize-collecting \st path problem. One obstacle is that the parsimonious property~\cite{GB} used in Bienstock et al. does not immediately apply to the path case; however, we prove that a modification to the graph and the Held-Karp solution allows us to utilize this property. This yields a -approximation algorithm for the prize-collecting \st path problem; the same upper bound is established on the integrality gap of the LP relaxation used.

Secondly, we study the \emph{unit-weight} graphical metric \st path TSP to present a -approximation algorithm. As discussed above, there has been progress for this special case in both the ordinary (circuit) TSP and the \st path TSP. In Appendix~\ref{ap:c53}, we show how the results of Oveis Gharan et al.~\cite{OSS} extend to the path case. Most recently, Mucha~\cite{M} gave an improved analysis of M\"omke \& Svensson's algorithm~\cite{MS} to prove the performance guarantee of  for the circuit case and  for the path case, for any . We observe that the critical case of this analysis is when the Held-Karp optimum is small, and we show how to obtain an algorithm that yields a better performance guarantee on this critical case, based on the main results of this paper. In particular, we devise an algorithm that works on narrow cuts to be run in parallel with the present algorithm; this illustrates that the narrow cuts are a useful algorithmic tool as well, not only an analytic tool. Our algorithm establishes an upper bound on the integrality gap of the path-variant Held-Karp relaxation under the unit-weight graphical metric, which does not match the performance guarantee but smaller than .

\section{Preliminaries}\label{s:pre}

In this section, we introduce some definitions and notation to be used throughout this paper.

Let  be the input complete graph with metric cost function . \emph{Endpoints}  are given as a part of the input; we call the other vertices \emph{internal points}.

For  such that ,  denotes the set of edges between  and : i.e., . Let  denote the set of edges within : .

For nonempty , let  denote the cut defined by , and  be the edge set in the cut: .  is called an \st \emph{cut} if ; we call  \emph{nonseparating} otherwise.

For  and ,  is a shorthand for ;  is . The incidence vector  of  is a -vector defined as follows:

For two vectors , let  denote the vector defined as:

\begin{defn}[\cite{HK}]
\label{d:hkpath}
The \emph{path-variant Held-Karp relaxation} is defined as follows:
\end{defn}

This linear program can be solved in polynomial time via the ellipsoid method using a min-cut algorithm to solve the separation problem \cite{GLS}. The following observation gives an equivalent formulation of \eqref{e:hkpath1}.

\begin{obs}
\label{o:hkequiv}
Following is an equivalent formulation of \eqref{e:hkpath1}:
\end{obs}

\begin{defn}
For  and ,  is a \emph{-join} if the set of odd-degree vertices in  is .
\end{defn}

Edmonds and Johnson \cite{EJ} give a polyhedral characterization of -joins: let  be the convex hull of the incidence vectors of the -joins on ;  is exactly characterized by

We call a feasible solution to \eqref{e:tjoind} a \emph{fractional -join dominator}.

Lastly, the polytope defined by the path-variant Held-Karp relaxation is contained in the spanning tree polytope of the same graph; thus, given a feasible solution  to the path-variant Held-Karp relaxation, there exist spanning trees  and  such that  and , where  is bounded by a polynomial. This follows from Gr{\"o}tschel, Lov{\'a}sz, and Schrijver~\cite{GLS}.

\section{Improving upon }\label{s:improv}

We present the algorithm for the metric \st path TSP and its analysis in this section.

\subsection{Algorithm}

Given a complete graph  with cost function  and the endpoints , the algorithm first computes an optimal solution  to the path-variant Held-Karp relaxation. Then it decomposes  into a convex combination  of polynomially many spanning trees  with coefficients ; a spanning tree  is sampled among these spanning trees 's from the probability distribution given by 's. This decomposition can be performed in polynomial time, as noted in Section~\ref{s:pre}. Let  be the set of the vertices with the ``wrong'' parity of degree in : i.e.,  is the set of odd-degree internal points and even-degree endpoints in . The algorithm finds a minimum -join  and an \st Eulerian path of the multigraph . This Eulerian path is shortcut to obtain a Hamiltonian path  between  and ;  is the output of the algorithm.

We note that this algorithm can be derandomized by trying each  instead of sampling . Observe that  implies that the derandomized algorithm is a deterministic -approximation algorithm.

In the rest of this section, we prove the following theorem.

\begin{thm}\label{t:main}
The present algorithm returns a Hamiltonian path between  and  whose expected cost is no more than . Therefore, there exists a deterministic -approximation algorithm for the \st path TSP.
\end{thm}

\begin{cor}\label{c:ig}
The integrality gap of the path-variant Held-Karp relaxation is at most .
\end{cor}

\subsection{Proof of -approximation}

In this subsection, we present a simple proof that the present algorithm is a (expected) -approximation algorithm. Improved analyses are presented in later subsections based on this simple proof.

We can understand the well-known -approximation algorithm for the circuit TSP and Christofides' -approximation algorithm as respectively using the minimum spanning tree and half the Held-Karp solution~\cite{W, SW} as a fractional -join dominator. Let us consider whether  and  can be used to bound the cost of a minimum -join in our case.

It can be seen from \eqref{e:hkpath1} that  is a fractional -join dominator for . If it were not for the \st cuts, the same could be shown for . However, an \st cut may have capacity as low as 1, making it hard to establish the feasibility of  for any .

 also is a fractional -join dominator for ; in this case, however, \st cuts do have some slack. Suppose that an \st cut  is odd with respect to : i.e.,  is odd. Since  contains exactly one of  and ,  contains an even number of vertices that have odd degree in .  is given as the sum of the degrees of the vertices in  minus twice the number of edges within , and is therefore even. This shows  and hence  for  does not violate \eqref{e:tjoind} as far as \st cuts are concerned. It is the nonseparating cuts that render it difficult to show the feasibility of  for .

Given the difficulties in these two cases are complementary, it is natural to consider  as a candidate for a fractional -join dominator; Theorem~\ref{t:a53} elaborates this observation.

\begin{thm}\label{t:a53}
.
\end{thm}
\begin{proof}
Let  for some parameters  to be chosen later. We examine a sufficient condition on  and  for  to be a feasible solution to \eqref{e:tjoind}.

It is obvious that .

Consider an odd cut  with respect to : i.e.,  is odd. We have  from the connectedness of . Suppose that  is an -cut; then  is even as previously argued. Thus,

Suppose that  is nonseparating; then we have  from the Held-Karp feasibility, and hence

Therefore, if  and  then  is feasible.
Now we bound the expected cost of :
where the second line holds since  is a fractional -join dominator. Choose  and .
\end{proof}

\subsection{First improvement upon }

We demonstrate in this subsection that the above analysis can be slightly improved.

Recall that the lower bound on the nonseparating cut capacities of  was given as  in the previous analysis; consider perturbing  and  by small amount while maintaining . In particular, if we decrease  by  and increase  by , we decrease the expected cost of  by , without changing ; that is, if we can fix the possible deficiencies of  in \st cuts with small cost, this perturbation will lead to an improvement in the performance guarantee.

Note that \st cuts with large capacities are not a problem:  and thus, if  is large enough, the bound remains greater than one after a small perturbation.

On the other hand, cuts with  are also not a concern. , and  from the connectedness of ; hence  is identically 1 and  is always even. Formulation~\eqref{e:tjoind} constrains the capacities of only the cuts that are odd with respect to , so the capacity of this particular cut  will never be constrained. In fact, for an \st cut , 

We will begin with  for perturbed  and , and ensure that  is a fractional -join dominator by adding small fractions of the deficient odd \st cuts. Yet, a cut being odd with small probability as shown by \eqref{e:prob} does not directly connect to its edge being added with small probability, since an edge belongs to many \st cuts. We address this issue by showing that the \st cuts of small capacities are ``almost'' disjoint.

First, consider the \st cuts  whose capacities are not large enough for  to be readily as large as 1; the following definition captures this idea. Let .

\begin{defn}
For some , an \st cut  is called \emph{-narrow} if .
\end{defn}

The following lemma shows that -narrow cuts do not cross.

\begin{lemma}\label{l:noncross}
Let . For  and , if both  and  are -narrow, then  or .
\end{lemma}
\begin{proof}
Suppose not. Then both  and  are nonempty andon the other hand,contradicting \eqref{e:l:noncross:1}.
\end{proof}

Lemma \ref{l:noncross} shows that the -narrow cuts constitute a layered structure, as illustrated in Figure~\ref{f:layer}:
\begin{figure}
\center
\includegraphics[width=350pt]{f3}
\caption{-narrow cuts of a feasible Held-Karp solution ().  is marked with thick edges.}
\label{f:layer}
\end{figure}

\begin{cor}\label{c:layer}
There exists a partition  of  such that\begin{enumerate}
\item , , and
\item , where .
\end{enumerate}
\end{cor}

Let  denote  and  denote . .

Now we show that -narrow cuts are almost disjoint: for each -narrow cut , we can choose  that occupies a large portion of  and mutually disjoint.

\begin{defn}
.
\end{defn}

\begin{lemma}\label{l:largedisjoint}
For each -narrow cut , .
\end{lemma}
\begin{proof}
The lemma holds trivially for . Suppose . We haveand
From \eqref{e:1} and \eqref{e:2},on the other hand,Thus,
\end{proof}

It is obvious that 's are disjoint and . For each -narrow cut , we define  as

\begin{thm}\label{t:qi}
.
\end{thm}
\begin{proof}
Letfor ,  and . We claim  is a fractional -join dominator. It is obvious that , and we have argued that  for nonseparating . Suppose  is an \st cut with  odd. If  is not -narrow, thenIf  is -narrow, then

Thus  is a fractional -join dominator. Now it remains to bound the expected cost of . Let . From \eqref{e:prob},where the last line follows from the disjointness of . Note that .
\end{proof}

\subsection{A tighter analysis}

In the previous analysis, we separately bounded the probability that a -narrow cut is odd, the deficit of the cut, and ; moreover, we used  instead of  from Lemma~\ref{l:largedisjoint}. These observations lead to some improvement, as shown in the following theorem.

\begin{thm}\label{t:iint}
.
\end{thm}
\begin{proof}
Let  denote the lower bound of  given by Lemma~\ref{l:largedisjoint}.

Letwhere  and  are to be chosen later; . As in the previous subsection,  and  denote the -narrow cuts and their layered structure. Assume  and .

A similar argument as in Theorem~\ref{t:qi} proves that  is a fractional -join dominator; it can also be shown that

Let . We haveand the unique solution toisSince  for  and ,  is maximized at ; hence, from \eqref{e:iint:1},
Choose ,  and we obtain
\end{proof}

\subsection{Proof of -approximation}

In this final subsection, we show that , proving Theorem~\ref{t:main} and Corollary~\ref{c:ig}.

In the previous analyses, 's serve as ``representatives'' of -narrow cuts. These representatives are useful since they have large weights while being disjoint. We improve the performance guarantee by introducing a new set of representatives that are ``fractionally disjoint''. Note that the three key properties of  used in the proof of Theorem~\ref{t:iint} are:\begin{enumerate}
\item  for all ;
\item ; and
\item  for all .
\end{enumerate}
 chosen in the previous analyses also satisfies that, for any given ,  for at most one . However, this was not a useful property in the analysis; Lemma~\ref{l:fd} states that, by relaxing the definition of disjointness, we can choose  that have larger weights. The definitions of ,  and  are unchanged.

\begin{lemma}
\label{l:fd}
There exists a set of vectors  satisfying:\begin{enumerate}
\item  for all ;
\item ; and
\item  for all .
\end{enumerate}
\end{lemma}

This lemma is proven later; based on it, Lemma~\ref{l:fd} proves the desired performance guarantee.

\begin{lemma}\label{l:fin}
.
\end{lemma}
\begin{proof}
Letwhere  and  are parameters to be chosen later, satisfying

By following the same argument as in Theorem~\ref{t:iint}, we can easily show that  is a fractional -join dominator; the only slight difference is when  is -narrow and  is odd, where we havefrom the first and the third properties of Lemma~\ref{l:fd}. Hence,  is a fractional -join dominator.

Now it remains to bound .From the second property of Lemma~\ref{l:fd},
We choose  and .
\end{proof}

\begin{proof}[Proof of Lemma~\ref{l:fd}]
Consider an auxiliary flow network illustrated in Figure~\ref{f:fd}, consisting of the source , sink , a node  for each -narrow cut , and a node  for each edge  in one or more -narrow cuts. The network has arcs of:\begin{enumerate}
\item capacity 1 from  to  for every -narrow cut ;
\item capacity  from  to  for every , for all ;
\item capacity  from  to  for every .
\end{enumerate} Let  be this capacity function.

\begin{figure}
\center
\includegraphics[width=320pt]{f4}
\caption{A feasible Held-Karp solution () and its corresponding flow network.}
\label{f:fd}
\end{figure}

Let  be an arbitrary cut on this flow network, where . We claim the cut capacity of  is at least .

Suppose there exists a -narrow cut  and  such that  and ; the cut capacity is then . So assume from now that (abusing the notation) every edge in any -narrow cut in  is also in . Let  for some . The cut capacity is then at leastif , the claim holds; the claim also holds for  since . Suppose  (see Figure~\ref{f:fd2}).proving the claim.

\begin{figure}
\center
\includegraphics[width=280pt]{f5}
\caption{Schematic diagram: , , , , and .}
\label{f:fd2}
\end{figure}

Thus the maximum flow on this flow network is of value at least . Consider a maximum flow; this flow saturates all the edges from  to , since the cut capacity of  is . Now, for each -narrow cut , define  as the flow from  to  if , and 0 otherwise. Then the first property is satisfied from the definition of flow; the second property is satisfied from the capacity constraints on  to ; lastly, the third property is satisfied since every edge from  to  is saturated.
\end{proof}

\section{Application to other problems}\label{s:appl}

In this section, we exhibit how the present results can be applied to other problems to obtain approximation algorithms with better performance guarantees than the current best known and improved LP integrality gap upper bounds.

\subsection{Prize-collecting \st path problem}\label{ss:prize}

We discuss the prize-collecting \st path problem in this subsection.

\begin{defn}[Metric prize-collecting \st path problem]
Given a complete graph  with , metric edge cost function , and vertex prize , the metric prize-collecting \st path problem is to find a simple \st path  that minimizes the sum of the path cost and the total prize ``missed'', i.e., .
\end{defn}

The \st path TSP can be considered as a special case of the prize-collecting \st path problem, where  for all .

Archer et al.~\cite{ABHK} use the path-variant Christofides' algorithm~\cite{H} as a black box to obtain a -approximation algorithm for the metric prize-collecting \st path problem. .

\begin{thm}[Archer et al.~\cite{ABHK}]\label{t:abhk}
Given a -approximation algorithm  for the  metric \st path TSP, one can obtain a -approximation algorithm for the metric prize-collecting \st path problem that uses  as a black box.
\end{thm}

This theorem, combined with Theorem~\ref{t:main}, readily produces an improvement. .

\begin{cor}
There exists a -approximation algorithm for the metric prize-collecting \st path problem.
\end{cor}

However, as the performance guarantee established by Theorem~\ref{t:main} is in terms of the Held-Karp optimum, the theorem enables a further improvement via an analysis analogous to Goemans~\cite{G:pc}. For the metric prize-collecting traveling salesman problem, Goemans~\cite{G:pc} combines the LP rounding algorithm due to Bienstock et al.~\cite{BGSW} and the primal-dual algorithm of Goemans \& Williamson~\cite{GW} (with the observation of \cite{CRW} and \cite{ABHK}) to achieve the best performance guarantee known for the problem.

One obstacle in applying this approach to the prize-collecting \st path problem is that, unlike the circuit-variant Held-Karp relaxation, the path-variant Held-Karp relaxation cannot be written as a set of edge-connectivity requirements between the pairs of vertices: the relaxation requires nonseparating cuts to have capacity of at least 2, whereas the edge connectivity between any two vertices can be as low as 1 in both a feasible Held-Karp solution and a (integral) Hamiltonian path. We will show that, despite this fact, the parsimonious property~\cite{GB} still can be used, and will analyze the performance guarantee given by the approach.

We start with the following LP relaxation of the problem:where  denotes the all-1 vector in . It can be easily verified that this is a relaxation of the prize-collecting \st path problem.

Given , consider a related problem of finding a minimum \st path on  that visits all the vertices in , and only those vertices. The following LP is a relaxation to this problem:

\begin{obs}\label{o:pchk}
Let  be the subgraph of  induced by . Projecting a feasible solution to \eqref{e:pcb} to  yields a feasible solution to the path-variant Held-Karp relaxation for .
\end{obs}

The following lemma shows that we can use the parsimonious property.

\begin{lemma}\label{l:pcpp}
The optimal solution value to \eqref{e:pcb} is equal to the optimal solution value to the following relaxation without the degree constraints:
\end{lemma}
\begin{proof}
Let . It suffices to show that, given a feasible solution  to \eqref{e:pcc}, how to construct a feasible solution to \eqref{e:pcb} whose cost is no greater than .

We will extend the graph (and ) so that the relaxation (almost) becomes a set of edge-connectivity requirements between pairs of vertices, and then use a similar approach as in Bienstock et al.~\cite{BGSW}, along with the following lemma:
\begin{lemma}[\cite{BGSW}]\label{l:pcsplit}
Let  be an Eulerian multigraph. Suppose that, for some  and , any two vertices in  other than  are -edge-connected. Let  be an arbitrary neighbor of ; then, there exists a neighbor  of  such that\begin{enumerate}
\item ; and
\item any two vertices in  other than  are still -edge-connected after splitting  and : i.e., replacing  and  (one copy each) with .
\end{enumerate}
\end{lemma}

Without loss of generality, we can assume  is rational.

Now we add three new vertices to the graph: ,  and . We set  and  for all ; :  and  will be the ``proxy'' of  and . We do not define the cost between  and other vertices: these costs do not affect the rest of the analysis. However, for notational convenience, we set these costs to be zero, potentially violating the triangle inequality. Let  be this extended graph.

We extend  into  as well: , and all other newly added edges are set to zero. Note that the (fractional) degree of ,  and  are 2.

Let ; we claim that any two vertices in  are 2-edge-connected.
\begin{claim}\label{c:pc2}
For any  such that  and , .
\end{claim}
\begin{proof}
Without loss of generality, assume . If , then at least one edge of the path  is in ; thus,

Suppose . If  then ; hence,Otherwise,  and thus,since .
\end{proof}

Now scale  by some large constant  so that  is integral and, in the multigraph on  whose edge multiplicities are given by , the degree of every vertex is even. Note that any two vertices in  are -edge-connected in this multigraph.

Let ;  is an even integer. We will modify  until  reaches 0: in particular, we split two edges in the multigraph so that\begin{enumerate}[(i)]
\item  decreases by 2; \label{i:pc1}
\item  do not increase; \label{i:pc2}
\item any two vertices in  are -edge-connected; \label{i:pc3}
\item the degrees of ,  and  all remain ; \label{i:pc4}
\item the only edges incident to  are  and ; and \label{i:pc5}
\item every vertex has even degree and hence the connected component containing  is Eulerian. \label{i:pc6}
\end{enumerate}It is clear that the invariants \eqref{i:pc3} through \eqref{i:pc6} initially hold.

If there exists an edge that is not reachable from any vertex in , we can remove all such edges without violating any of the conditions ( may decrease by more than 2).

If there exists  such that , then we apply Lemma~\ref{l:pcsplit} to pick two incident edges to split. Note that  since . \eqref{i:pc3} is maintained from the lemma. Splitting does not change the degree of any vertex other than ; hence \eqref{i:pc1}, \eqref{i:pc4} and \eqref{i:pc6} are satisfied. Neither of the chosen edges is incident to , as can be seen from \eqref{i:pc5}; thus, \eqref{i:pc5} is maintained and \eqref{i:pc2} follows from the triangle inequality.

Otherwise, we choose  such that .  from \eqref{i:pc6}. Again  from \eqref{i:pc4}; we can similarly verify all properties in this case as well.

Once  reaches , we remove  and its incident edges. None of these edges got split during the process: this is the reason why the cost of these edges can be left undefined.

Note that the degree of  and  now are , whereas  and  are . Concatenate  and , and  and , respectively; we scale this multigraph back by  to obtain a feasible solution to \eqref{e:pcb} whose cost is no greater than .
\end{proof}

We are now ready to apply the analyses of Goemans~\cite{G:pc} and Bienstock et al.~\cite{BGSW}. Let  and  be an optimal solution to \eqref{e:pca}.
\begin{lemma}\label{l:prounding}
Let  be an approximation algorithm for the \st path TSP that produces a path of cost at most  times the Held-Karp optimum. Let  for some . Running  on the subgraph  induced by  yields a path  with .
\end{lemma}
\begin{proof}The proof is basically the same as \cite{BGSW}. Observe that  is a feasible solution to \eqref{e:pcc}, as can be seen from \eqref{e:pca} and \eqref{e:pcc}. From Lemma~\ref{l:pcpp} and Observation~\ref{o:pchk}, the Held-Karp optimum for  is of cost no greater than .
\end{proof}

The primal-dual algorithm of Chaudhuri et al.~\cite{CGRT} can be used to obtain the following performance guarantee for the metric prize-collecting \st path problem.
\begin{lemma}[\cite{CGRT, ABHK}]\label{l:cgrtpg}
There exists a polynomial-time algorithm  that produces an \st path  satisfying
\end{lemma}

Now, the combined algorithm is as follows: let  and . The algorithm runs  with probability ; otherwise, it computes an optimal solution  and  to \eqref{e:pca}, samples  uniformly at random from , and run  on the subgraph induced by .

This algorithm can be derandomized since there are only  different 's possible.

\begin{thm}\label{t:pcfinal}
Let  be an approximation algorithm for the \st path TSP that produces a path of cost at most  times the Held-Karp optimum, for some ; then, there exists a -approximation algorithm for the metric prize-collecting \st path problem.
\end{thm}
\begin{proof}
The given algorithm is a polynomial-time algorithm. Let  denote the output path.

It can be easily verified that  and . From Lemma~\ref{l:prounding},
We have

From \eqref{e:pcfinal1}, \eqref{e:pcfinal2}, and Lemma~\ref{l:cgrtpg},
\end{proof}

Theorem~\ref{t:pcfinal} along with Theorem~\ref{t:main} yields the following:
\begin{cor}
There exists a deterministic -approximation algorithm for the metric prize-collecting \st path problem.
\end{cor}

\begin{cor}
The integrality gap of \eqref{e:pca} is smaller than .
\end{cor}

\subsection{Unit-weight graphical metrics}\label{ss:unit}

In this subsection, we study the \st path TSP for a special case where the cost function is a shortest-path metric defined by an underlying undirected, unit-weight graph.

Let  be an optimal solution to the path-variant Held-Karp relaxation;  be the underlying unit-weight graph defining the cost function.  is connected.

Mucha~\cite{M} gives an improved analysis of the -approximation algorithm of M\"omke and Svensson~\cite{MS}; following is from \cite{M}.

\begin{lemma}[\cite{M}]\label{l:m}
There exists an algorithm  for the \st path TSP under unit-weight graphical metrics, which returns a solution of cost at most
\end{lemma}

This immediately gives a -approximation algorithm for any . .
\begin{thm}[\cite{M}]\label{t:m}
There exists a -approximation algorithm for the \st path TSP under unit-weight graphical metrics, for any .
\end{thm}
\begin{proof}
Let  be the output of . From Lemma~\ref{l:m},where the last line holds since  for all .

Thus, there exists  such that  for all input that has  or more vertices. Smaller instances can be separately solved.
\end{proof}

It can be observed from Lemma~\ref{l:m} and Theorem~\ref{t:m} that the ``critical case'' determining the proven performance guarantee is when . We will show that three different constructions of Hamiltonian paths carry performance analyses with complementary critical cases.

Even though -narrow cuts function as a mere analytic tool in Section~\ref{s:improv}, we propose an algorithm that actually computes the -narrow cuts and utilize them: once the -narrow cuts are computed, the algorithm constructs an \st path that traverses from the first layer to the last, without ``skipping'' any layer in-between. If the path is inexpensive, the number of -narrow cuts is also small so the algorithm presented in Section~\ref{s:improv} produces a good solution. If the path is expensive but the Held-Karp optimum is close to , then we prove that the path already contains a large number of vertices and therefore can be augmented into a spanning Eulerian path with small additional cost. Lastly, if the Held-Karp optimum is bounded away from , then M\"omke \& Svensson's algorithm performs well provided that the graph has large number of vertices.

Algorithm~\ref{a:unitg} shows the entire algorithm (except the separate handling of small instances);  is a parameter to be chosen later. Let  be a function such that . For ,  denotes the subgraph of  induced by . Suppose ; this implies .

\begin{algorithm}[ht]
\caption{The algorithm for the \st path TSP under unit-weight graphical metrics}
\label{a:unitg}
\begin{algorithmic}[1]
	\REQUIRE Complete graph  with cost function ; endpoints .
	\ENSURE Hamiltonian path between  and .
	\STATE Run ; let  be the output Hamiltonian path.
	\STATE an optimal solution to the path-variant Held-Karp relaxation
	\STATE Run the algorithm from Section~\ref{s:improv}; let  be the output Hamiltonian path.
	\STATE Compute the partition  defining all the -narrow cuts .\label{as:unitg:0}
	\FOR {}
	\STATE Let  be the shortest edge in , where  and .
	\ENDFOR
	\FOR {}
	\STATE Let  be the shortest path from  to  within , under edge cost given by .
	\ENDFOR
	\STATE Let  be an \st path obtained by concatenating .\hspace{-2em}
	\STATE 
	\WHILE {the multigraph  is not spanning}\label{as:unitg:1}
	\STATE Choose  such that: ,  is isolated in , and  is not.
	\STATE Add two copies of  to .
	\ENDWHILE \label{as:unitg:2}
	\STATE Shortcut an Eulerian path of  to obtain a Hamiltonian path .
	\STATE Let  be the best among ,  and ; output .
\end{algorithmic}
\end{algorithm}

\begin{lemma}\label{l:wd}
Algorithm~\ref{a:unitg} is a well-defined, polynomial-time algorithm.
\end{lemma}
\begin{proof}
Steps~\ref{as:unitg:1}-\ref{as:unitg:2} start with an \st path, and augment it into a spanning multigraph that has an Eulerian path between  and . This follows from the preservation of the parity of degree. Choice of  satisfying  is always possible since  is connected.

 is an \st path since  and . Note that some of 's may be a length-0 path.

Step~\ref{as:unitg:0}, unlike the algorithm from Section~\ref{s:improv}, actually computes the layered structure of -narrow cuts, whereas this structure was only for the sake of analysis in Section~\ref{s:improv}. Yet, the layers can in fact be identified via a polynomial number of min-cut calculations; hence, the algorithm is a polynomial-time algorithm.
\end{proof}

\begin{lemma}\label{l:c1}

\end{lemma}
\begin{proof}
We haveandFrom \eqref{e:c1:1} and \eqref{e:c1:2},
\end{proof}
By symmetry, .

\begin{lemma}\label{l:c2}
For any , ,  and  such that\begin{enumerate}
\item ,
\item , and
\item ,
\end{enumerate}then .
\end{lemma}
\begin{proof}
We haveby symmetry,again by symmetry,

From \eqref{e:c2:1} through \eqref{e:c2:4},
\end{proof}

\begin{cor}\label{c:ccut}
For all , .
\end{cor}
\begin{proof}
From Lemma~\ref{l:c1} and Lemma~\ref{l:c2} applied for .
\end{proof}

\begin{cor}\label{c:clayer}
For all ,  weighted by (the projection of)  is -edge-connected.
\end{cor}
\begin{proof}
 and  are singleton; every cut in any other nonsingleton layer subgraphs are of capacity at least  from Lemma~\ref{l:c2}, applied for .
\end{proof}

Let  be some parameters to be chosen later.

\begin{lemma}\label{l:unitgmain}

\end{lemma}
\begin{proof}
Suppose ; from the proof of Theorem~\ref{t:m},thus, we can assume from now that .

\paragraph{Case 1.}

From Corollary~\ref{c:ccut} and the choice of ,For each layer  with , consider a bidirected flow network on  whose capacities are given by . From Corollary~\ref{c:clayer}, we can route flow of  from  to . This flow can be decomposed into cycles and paths from  to ; thus, by the choice of ,

From \eqref{e:ugm11} and \eqref{e:ugm12},

Let  be  after finishing the execution of Steps~\ref{as:unitg:1}-\ref{as:unitg:2} of Algorithm~\ref{a:unitg};  denotes the number of edges on . We havewhere the second last line follows from \eqref{e:case1} and \eqref{e:unitg1f}; the last from .

\paragraph{Case 2.}Note that, from the construction of , ; hence we have

From each -narrow cut , we can pick an edge  with  due to the connectedness of . Let , , , and . Note that this choice of  and  satisfies \eqref{e:later1}. Since the second condition on  of Lemma~\ref{l:fd} is not used to derive \eqref{e:later2} (it is used in the later part of the proof), we haveAs  for all ,
\end{proof}

\begin{cor}\label{c:unitgar}
Let . There exists a -ap\-proxi\-ma\-tion algorithm for the \st path TSP under unit-weight graphical metrics, for any .
\end{cor}

\begin{cor}
There exists a -approximation algorithm for the \st path TSP under unit-weight graphical metrics.
\end{cor}
\begin{proof}
Directly follows from Corollary~\ref{c:unitgar}: if we choose, for example, , , and , we have .
\end{proof}

\begin{cor}
The integrality gap of the path-variant Held-Karp relaxation under the unit-weight graphical metric is smaller than .
\end{cor}
\begin{proof}
Trivial for . Let  denote the optimal (integral) solution value.

Suppose . From a similar argument as in the proof of Lemma~\ref{l:wd}, if there exists a simple \st path with  edges in , . Thus, if there exists a simple \st path with at least two edges,Suppose there does not exist a simple \st path with more than one edge; then  and  is a bridge of . Let  be the \st cut defined by the removal of  from .  since ; therefore,and

Suppose . Choose , , and ; from Lemma~\ref{l:unitgmain},for some .
\end{proof}

\section{Open questions}\label{s:oq}

An immediate open question is in improving the performance guarantee. The fractional -join dominators constructed in the analyses are not directly derived from the algorithm; a different construction may lead to an improved performance guarantee. One related question is whether  and  can be chosen differently. In the proof of -approximation, Lemma~\ref{l:fd} can be considered as distributing  over the cuts of different capacities. An adaptive choice of  and  after seeing one such distribution does not appear to improve the analysis; from Yao's Lemma, oblivious but stochastic choice of  and  does not either.

A bigger open question is whether the techniques presented in this paper can be extended to the circuit case as well. Given the successful adaptation of the techniques devised in one variant to the other in the unit-weight graphical metric case, whether the present techniques can be extended to beat the longstanding  barrier of the general-metric circuit problem becomes an interesting question. It appears that the layered structure of -narrow cuts or the parity argument on them are less likely to directly extend to the circuit case, as the arguments rely on the characteristics of the path case; what could be more promising is the approach of repairing deficient cuts using a set of vectors obtained from an auxiliary flow network, since this approach might extend to work with some different type of ``fragile cut structure''.

\bibliography{path2}

\appendix

\section{An LP-based new analysis of the path-variant Christofides' algorithm}\label{ap:c53}

In this appendix, we present a new analysis of the path-variant Christofides' algorithm~\cite{C, H} for the metric \st path TSP, and show how the critical case characterized by this analysis can lead to an improvement. The analysis compares the output solution value to the LP optimum of the path-variant Held-Karp relaxation, thereby proving the upper bound of  on the integrality gap of the path-variant Held-Karp relaxation. We note that the LP optimum is never computed by the algorithm.

First we recall the following definition of the circuit-variant Held-Karp relaxation:
\begin{defn}[\cite{HK}]
\label{d:hkcircuit}
The \emph{circuit-variant Held-Karp relaxation} is the following:
\end{defn}

Let  be the input complete graph with cost function  and the endpoints . The path-variant Christofides' algorithm first finds a minimum spanning tree  of ; it then computes a minimum -join , where  is the set of the vertices with the ``wrong'' parity of degree in : i.e.,  is the set of odd-degree internal points and even-degree endpoints in . Lastly, the algorithm shortcuts an Eulerian path of the multigraph  to obtain the output Hamiltonian path .

We give two different bounds on the cost of , which together will establish the performance guarantee. Let  be the LP optimum of the path-variant Held-Karp relaxation.

\begin{lemma}
\label{l:c53j1}
.
\end{lemma}
\begin{proof}
As can be seen from Observation~\ref{o:hkequiv}, the path-variant Held-Karp polytope is contained in the spanning tree polytope. The lemma follows from this observation, since  is a minimum spanning tree.
\end{proof}

Lemmas~\ref{l:c53j2} and \ref{l:c53j3} give the two bounds.

\begin{lemma}
\label{l:c53j2}
.
\end{lemma}
\begin{proof}
Let : i.e.,  is obtained by ``adding'' the edge  to . Then  is a feasible solution to the circuit-variant Held-Karp relaxation (see \eqref{e:hkpath1} and \eqref{e:hkcircuit}). Let  be the optimal value of the circuit-variant Held-Karp relaxation and we havewhere the first inequality follows from \cite{W, SW}.
\end{proof}

\begin{lemma}
\label{l:c53j3}
.
\end{lemma}
\begin{proof}
Let  be the path between  and  on . Consider an edge set . Note that  is a -join:  has even degree in  if and only if  is internal; thus,  has even degree in the multigraph  if and only if  is an internal point, and this shows that  has odd degree in  if and only if .

We haveThe last inequality follows from Lemma~\ref{l:c53j1} and the triangle inequality.
\end{proof}

\begin{thm}
; therefore, the path-variant Christofides' algorithm is a -approximation algorithm, and the integrality gap of the path-variant Held-Karp relaxation is at most .
\end{thm}
\begin{proof}
We have
where the second inequality follows from Lemmas~\ref{l:c53j1}, \ref{l:c53j2} and \ref{l:c53j3}.
\end{proof}

We observe that the equality of \eqref{e:crit} is achieved when , and this is the critical case of this analysis that determines the performance guarantee proven. Hence, if we can improve the performance guarantee only near this critical case, such an improvement would lead to a better approximation ratio. We demonstrate this approach, by presenting how this analysis combines with the results of Oveis Gharan et al.~\cite{OSS} on the unit-weight graphical metric TSP to yield a comparable result in the \st path TSP.

We consider the \st path TSP under the unit-weight graphical metric; we show how to modify the algorithm of Oveis Gharan et al. for the path case and that, when  is close to , this modified algorithm carries a performance guarantee that is slightly better than .

First we review the results in Oveis Gharan et al.~\cite{OSS}. In the following, the parameters  and  can be chosen as follows: , , , , , and  denotes .

\begin{defn}[Nearly integral edges]\label{d:niedges}
An edge  is \emph{nearly integral} with respect to  if .
\end{defn}

\begin{defn}\label{d:approxm}
For some constant  and , a \emph{maximum entropy distribution over spanning trees with approximate marginal}  is a probability distribution  defined by  such that  for every spanning tree  and the marginal probability of every edge  is no greater than .
\end{defn}

\begin{defn}[Good edges]\label{d:goodedges}
A cut is -near-minimum if its weight is at most  times the minimum cut weight. An edge  is \emph{even} with respect to  if every -near-minimum cut containing  has even number of edges intersecting with .

For a circuit-variant Held-Karp feasible solution , consider  as the edge weight and let  be a spanning tree sampled from a maximum entropy distribution with approximate marginal . We say an edge  is \emph{good} with respect to  if the probability that  is even with respect to  is at least .
\end{defn}

\begin{thm}[Structure Theorem]
\label{t:structure}
Let  be a feasible solution to the circuit-variant Held-Karp relaxation, and let  be a maximum entropy distribution over spanning trees with approximate marginal . There exist small constants  such that at least one of the following is true:\begin{itemize}
\item[1.] there exists a set  such that  and every edge in  is good with respect to ;
\item[2.] there exist at least  edges that are nearly integral with respect to .
\end{itemize}
\end{thm}

\begin{lemma}
\label{l:case1}
Suppose that Case~1 of Theorem~\ref{t:structure} holds and  is sampled from . Let  be the set of odd-degree vertices in , then a minimum -join  satisfies
\end{lemma}

We are now ready to present the algorithm. Algorithm~\ref{a:au} describes the entire algorithm for the \st path TSP under the unit-weight graphical metric. It first computes the LP optimum . If  is close to , we run a modified version of Oveis Gharan, Saberi, and Singh's algorithm (Cases~A1 and A2); otherwise, we invoke Christofides' algorithm (Case~B). Parameters  and  are to be chosen later.

\begin{algorithm}[ht]
\caption{Algorithm for the \st path TSP under the unit-weight graphical metric}
\label{a:au}
\begin{algorithmic}[1]
	\REQUIRE Complete graph  with cost function ; endpoints .
	\ENSURE Hamiltonian path between  and .

	\STATE optimal solution to the path-variant Held-Karp relaxation
	\IF { for }
		\IF[Case A1]{at least  edges are nearly integral w.r.t. }
			\STATE Find a minimum spanning subgraph  containing all the nearly integral edges.
			\STATE Find a minimum spanning tree  of .
			\STATE Let  be the set of odd-degree internal points and even-degree endpoints in .
			\STATE Compute a minimum -join ; .
		\ELSE[Case A2]
			\STATE 
			\STATE Sample spanning tree  from max-entropy distribution with approx. marginal .
			\STATE Let  be the set of odd-degree vertices in .
			\STATE Compute a minimum -join ; .
			\STATE \textbf{if}  \textbf{then}  \textbf{else}  \textbf{end if}
		\ENDIF
	\ELSE[Case B]
		\STATE Find a minimum spanning tree  of .
		\STATE Let  be the set of odd-degree internal points and even-degree endpoints in .
		\STATE Compute a minimum -join ; .
	\ENDIF
	\STATE Shortcut an Eulerian path of the multigraph  to obtain a Hamiltonian path ; output it.
\end{algorithmic}
\end{algorithm}

First we show that we can have a Structure Theorem analogous to Theorem~\ref{t:structure} by adjusting  and replacing  with  in Case~2. The following corollary states that either there are good edges of significant weight with respect to  or there are many nearly integral edges with respect to .
\begin{cor}
\label{c:pathstructure}
Let  be a feasible solution to the path-variant Held-Karp relaxation and  . Let  be a maximum entropy distribution over spanning trees with approximate marginal . There exist small constants  such that at least one of the following is true:\begin{itemize}
\item[1.] there exists a set  such that  and every edge in  is good with respect to ;
\item[2.] there exist at least  edges that are nearly integral with respect to .
\end{itemize}
\end{cor}
\begin{proof}
By Theorem~\ref{t:structure}, at least one of the two cases of Theorem~\ref{t:structure} holds. Case~1 of Theorem~\ref{t:structure} and Case~1 of this corollary are identical, so consider when Case~2 of Theorem~\ref{t:structure} holds.

Recall that  was chosen as ; we choose .

Suppose .  has at least  nearly integral edges; thus,  has at least  nearly integral edges.

Suppose .  has at leastnearly integral edges.
\end{proof}

\begin{lemma}
\label{l:fca1}
In Case~A1,  for some .
\end{lemma}
\begin{proof}
The following proof is adapted from \cite{OSS} and modified for the path case.

Let  be the set of nearly integral edges. Since the metric is defined by an unweighted connected graph, . From , we know that  is a union of disjoint cycles and paths and the lengths of cycles are at least . Thus,  and . Let .

We construct a fractional -join dominator  as follows.We claim that  is a fractional -join dominator. Let  be any cut that has an odd number of vertices in  on one side. If there exists an edge , then . So suppose from now on that . Then .

If  is nonseparating,  contains odd number of odd-degree vertices, and thus  is odd. We have  from the Held-Karp formulation and thus

If  is an \st cut, then  contains even number of odd-degree vertices, and thus  is even. We have  since  is connected and

Thus  is a fractional -join dominator. Now,for some . For example, we can choose .
\end{proof}

\begin{lemma}
\label{l:fca2}
In Case~A2,  for some .
\end{lemma}
\begin{proof}
First we haveFrom Lemma~\ref{l:case1},

We haveand hencefor some  by choosing sufficiently small . For example, we can choose ,  and .
\end{proof}

\begin{lemma}
\label{l:fcb}
In Case~B,  for some .
\end{lemma}
\begin{proof}
Suppose that . From Lemmas~\ref{l:c53j1} and \ref{l:c53j2}, it follows that

Suppose . From Lemmas~\ref{l:c53j1} and \ref{l:c53j3},

Now choose .
\end{proof}

Lemmas~\ref{l:fca1}, \ref{l:fca2} and \ref{l:fcb} yield the following theorem.

\begin{thm}
For some , Algorithm~\ref{a:au} is a -approximation algorithm for the \st path TSP under the unit-weight graphical metric.
\end{thm}
\begin{proof}
In Cases~A1 and B, the multigraph  is the union of a spanning tree and a -join where  is the set of the vertices with the wrong parity of degree. Thus,  has an Eulerian path between the two endpoints.

In Case~A2,  is Eulerian and hence 2-edge-connected;  is therefore connected and  has an Eulerian path between the two endpoints.

By choosing ,  for example, we have  from Lemmas~\ref{l:fca1}, \ref{l:fca2} and \ref{l:fcb}. Thus, Algorithm~\ref{a:au} is a -approximation algorithm.
\end{proof}

\phantom{\cite{Full, CRW}}

\end{document}
