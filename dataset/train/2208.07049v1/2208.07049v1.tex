
\subsection{Abbreviations and Acronyms}
Define abbreviations and acronyms the first time they are used in the text, 
even after they have already been defined in the abstract. Abbreviations 
such as IEEE, SI, ac, and dc do not have to be defined. Abbreviations that 
incorporate periods should not have spaces: write ``C.N.R.S.,'' not ``C. N. 
R. S.'' Do not use abbreviations in the title unless they are unavoidable 
(for example, ``IEEE'' in the title of this article).

\subsection{Other Recommendations}
Use one space after periods and colons. Hyphenate complex modifiers: 
``zero-field-cooled magnetization.'' Avoid dangling participles, such as, 
``Using \eqref{eq}, the potential was calculated.'' [It is not clear who or what 
used \eqref{eq}.] Write instead, ``The potential was calculated by using \eqref{eq},'' or 
``Using \eqref{eq}, we calculated the potential.''

Use a zero before decimal points: ``0.25,'' not ``.25.'' Use 
``cm,'' not ``cc.'' Indicate sample dimensions as ``0.1 cm 
 0.2 cm,'' not ``0.1  0.2 cm.'' The 
abbreviation for ``seconds'' is ``s,'' not ``sec.'' Use 
``Wb/m'' or ``webers per square meter,'' not 
``webers/m.'' When expressing a range of values, write ``7 to 
9'' or ``7--9,'' not ``79.''

A parenthetical statement at the end of a sentence is punctuated outside of 
the closing parenthesis (like this). (A parenthetical sentence is punctuated 
within the parentheses.) In American English, periods and commas are within 
quotation marks, like ``this period.'' Other punctuation is ``outside''! 
Avoid contractions; for example, write ``do not'' instead of ``don't.'' The 
serial comma is preferred: ``A, B, and C'' instead of ``A, B and C.''

If you wish, you may write in the first person singular or plural and use 
the active voice (``I observed that '' or ``We observed that '' 
instead of ``It was observed that ''). Remember to check spelling. If 
your native language is not English, please get a native English-speaking 
colleague to carefully proofread your paper.

Try not to use too many typefaces in the same article. You're writing
scholarly papers, not ransom notes. Also please remember that MathJax
can't handle really weird typefaces.

\subsection{Equations}
Number equations consecutively with equation numbers in parentheses flush 
with the right margin, as in \eqref{eq}. To make your equations more 
compact, you may use the solidus (~/~), the exp function, or appropriate 
exponents. Use parentheses to avoid ambiguities in denominators. Punctuate 
equations when they are part of a sentence, as in


Be sure that the symbols in your equation have been defined before the 
equation appears or immediately following. Italicize symbols ( might refer 
to temperature, but T is the unit tesla). Refer to ``\eqref{eq},'' not ``Eq. \eqref{eq}'' 
or ``equation \eqref{eq},'' except at the beginning of a sentence: ``Equation \eqref{eq} 
is  .''

\subsection{\LaTeX-Specific Advice}

Please use ``soft'' (e.g., \verb|\eqref{Eq}|) cross references instead
of ``hard'' references (e.g., \verb|(1)|). That will make it possible
to combine sections, add equations, or change the order of figures or
citations without having to go through the file line by line.

Please don't use the \verb|{eqnarray}| equation environment. Use
\verb|{align}| or \verb|{IEEEeqnarray}| instead. The \verb|{eqnarray}|
environment leaves unsightly spaces around relation symbols.

Please note that the \verb|{subequations}| environment in {\LaTeX}
will increment the main equation counter even when there are no
equation numbers displayed. If you forget that, you might write an
article in which the equation numbers skip from (17) to (20), causing
the copy editors to wonder if you've discovered a new method of
counting.

{\BibTeX} does not work by magic. It doesn't get the bibliographic
data from thin air but from .bib files. If you use {\BibTeX} to produce a
bibliography you must send the .bib files. 

{\LaTeX} can't read your mind. If you assign the same label to a
subsubsection and a table, you might find that Table I has been cross
referenced as Table IV-B3. 

{\LaTeX} does not have precognitive abilities. If you put a
\verb|\label| command before the command that updates the counter it's
supposed to be using, the label will pick up the last counter to be
cross referenced instead. In particular, a \verb|\label| command
should not go before the caption of a figure or a table.

Do not use \verb|\nonumber| inside the \verb|{array}| environment. It
will not stop equation numbers inside \verb|{array}| (there won't be
any anyway) and it might stop a wanted equation number in the
surrounding equation.

\section{Units}
Use either SI (MKS) or CGS as primary units. (SI units are strongly 
encouraged.) English units may be used as secondary units (in parentheses). 
This applies to papers in data storage. For example, write ``15 
Gb/cm (100 Gb/in.'' An exception is when 
English units are used as identifiers in trade, such as ``3\textonehalf-in 
disk drive.'' Avoid combining SI and CGS units, such as current in amperes 
and magnetic field in oersteds. This often leads to confusion because 
equations do not balance dimensionally. If you must use mixed units, clearly 
state the units for each quantity in an equation.

The SI unit for magnetic field strength  is A/m. However, if you wish to use 
units of T, either refer to magnetic flux density  or magnetic field 
strength symbolized as . Use the center dot to separate 
compound units, e.g., ``Am.''

\section{Some Common Mistakes}
The word ``data'' is plural, not singular. The subscript for the 
permeability of vacuum  is zero, not a lowercase letter 
``o.'' The term for residual magnetization is ``remanence''; the adjective 
is ``remanent''; do not write ``remnance'' or ``remnant.'' Use the word 
``micrometer'' instead of ``micron.'' A graph within a graph is an 
``inset,'' not an ``insert.'' The word ``alternatively'' is preferred to the 
word ``alternately'' (unless you really mean something that alternates). Use 
the word ``whereas'' instead of ``while'' (unless you are referring to 
simultaneous events). Do not use the word ``essentially'' to mean 
``approximately'' or ``effectively.'' Do not use the word ``issue'' as a 
euphemism for ``problem.'' When compositions are not specified, separate 
chemical symbols by en-dashes; for example, ``NiMn'' indicates the 
intermetallic compound NiMn whereas 
``Ni--Mn'' indicates an alloy of some composition 
NiMn.

\Figure[t!](topskip=0pt, botskip=0pt, midskip=0pt){fig1.png}
{Magnetization as a function of applied field.
It is good practice to explain the significance of the figure in the caption.\label{fig1}}

Be aware of the different meanings of the homophones ``affect'' (usually a 
verb) and ``effect'' (usually a noun), ``complement'' and ``compliment,'' 
``discreet'' and ``discrete,'' ``principal'' (e.g., ``principal 
investigator'') and ``principle'' (e.g., ``principle of measurement''). Do 
not confuse ``imply'' and ``infer.'' 

Prefixes such as ``non,'' ``sub,'' ``micro,'' ``multi,'' and ``ultra'' are 
not independent words; they should be joined to the words they modify, 
usually without a hyphen. There is no period after the ``et'' in the Latin 
abbreviation ``\emph{et al.}'' (it is also italicized). The abbreviation ``i.e.,'' means 
``that is,'' and the abbreviation ``e.g.,'' means ``for example'' (these 
abbreviations are not italicized).

A general IEEE styleguide is available at \underline{http://www.ieee.org/}\break\underline{authortools}.

\section{Guidelines for Graphics Preparation and Submission}
\label{sec:guidelines}

\subsection{Types of Graphics}
The following list outlines the different types of graphics published in 
IEEE journals. They are categorized based on their construction, and use of 
color/shades of gray:

\subsubsection{Color/Grayscale figures}
{Figures that are meant to appear in color, or shades of black/gray. Such 
figures may include photographs, illustrations, multicolor graphs, and 
flowcharts.}

\subsubsection{Line Art figures}
{Figures that are composed of only black lines and shapes. These figures 
should have no shades or half-tones of gray, only black and white.}

\subsubsection{Author photos}
{Head and shoulders shots of authors that appear at the end of our papers. }

\subsubsection{Tables}
{Data charts which are typically black and white, but sometimes include 
color.}

\begin{table}
\caption{Units for Magnetic Properties}
\label{table}
\setlength{\tabcolsep}{3pt}
\begin{tabular}{|p{25pt}|p{75pt}|p{115pt}|}
\hline
Symbol& 
Quantity& 
Conversion from Gaussian and \par CGS EMU to SI  \\
\hline
& 
magnetic flux& 
1 Mx  Wb  Vs \\
& 
magnetic flux density, \par magnetic induction& 
1 G  T  Wb/m \\
& 
magnetic field strength& 
1 Oe  A/m \\
& 
magnetic moment& 
1 erg/G  1 emu \par  Am J/T \\
& 
magnetization& 
1 erg/(Gcm 1 emu/cm \par  A/m \\
4& 
magnetization& 
1 G  A/m \\
& 
specific magnetization& 
1 erg/(Gg)  1 emu/g  1 Am/kg \\
& 
magnetic dipole \par moment& 
1 erg/G  1 emu \par  Wbm \\
& 
magnetic polarization& 
1 erg/(Gcm 1 emu/cm \par  T \\
& 
susceptibility& 
1  \\
& 
mass susceptibility& 
1 cm/g  m/kg \\
& 
permeability& 
1  H/m \par  Wb/(Am) \\
& 
relative permeability& 
 \\
& 
energy density& 
1 erg/cm J/m \\
& 
demagnetizing factor& 
1  \\
\hline
\multicolumn{3}{p{251pt}}{Vertical lines are optional in tables. Statements that serve as captions for 
the entire table do not need footnote letters. }\\
\multicolumn{3}{p{251pt}}{Gaussian units are the same as cg emu for magnetostatics; Mx 
 maxwell, G  gauss, Oe  oersted; Wb  weber, V  volt, s  
second, T  tesla, m  meter, A  ampere, J  joule, kg  
kilogram, H  henry.}
\end{tabular}
\label{tab1}
\end{table}

\subsection{Multipart figures}
Figures compiled of more than one sub-figure presented side-by-side, or 
stacked. If a multipart figure is made up of multiple figure
types (one part is lineart, and another is grayscale or color) the figure 
should meet the stricter guidelines.

\subsection{File Formats For Graphics}\label{formats}
Format and save your graphics using a suitable graphics processing program 
that will allow you to create the images as PostScript (PS), Encapsulated 
PostScript (.EPS), Tagged Image File Format (.TIFF), Portable Document 
Format (.PDF), Portable Network Graphics (.PNG), or Metapost (.MPS), sizes them, and adjusts 
the resolution settings. When 
submitting your final paper, your graphics should all be submitted 
individually in one of these formats along with the manuscript.

\subsection{Sizing of Graphics}
Most charts, graphs, and tables are one column wide (3.5 inches/88 
millimeters/21 picas) or page wide (7.16 inches/181 millimeters/43 
picas). The maximum depth a graphic can be is 8.5 inches (216 millimeters/54
picas). When choosing the depth of a graphic, please allow space for a 
caption. Figures can be sized between column and page widths if the author 
chooses, however it is recommended that figures are not sized less than 
column width unless when necessary. 

There is currently one publication with column measurements that do not 
coincide with those listed above. Proceedings of the IEEE has a column 
measurement of 3.25 inches (82.5 millimeters/19.5 picas). 

The final printed size of author photographs is exactly
1 inch wide by 1.25 inches tall (25.4 millimeters31.75 millimeters/6 
picas7.5 picas). Author photos printed in editorials measure 1.59 inches 
wide by 2 inches tall (40 millimeters50 millimeters/9.5 picas12 
picas).

\subsection{Resolution }
The proper resolution of your figures will depend on the type of figure it 
is as defined in the ``Types of Figures'' section. Author photographs, 
color, and grayscale figures should be at least 300dpi. Line art, including 
tables should be a minimum of 600dpi.

\subsection{Vector Art}
In order to preserve the figures' integrity across multiple computer 
platforms, we accept files in the following formats: .EPS/.PDF/.PS. All 
fonts must be embedded or text converted to outlines in order to achieve the 
best-quality results.

\subsection{Color Space}
The term color space refers to the entire sum of colors that can be 
represented within the said medium. For our purposes, the three main color 
spaces are Grayscale, RGB (red/green/blue) and CMYK 
(cyan/magenta/yellow/black). RGB is generally used with on-screen graphics, 
whereas CMYK is used for printing purposes.

All color figures should be generated in RGB or CMYK color space. Grayscale 
images should be submitted in Grayscale color space. Line art may be 
provided in grayscale OR bitmap colorspace. Note that ``bitmap colorspace'' 
and ``bitmap file format'' are not the same thing. When bitmap color space 
is selected, .TIF/.TIFF/.PNG are the recommended file formats.

\subsection{Accepted Fonts Within Figures}
When preparing your graphics IEEE suggests that you use of one of the 
following Open Type fonts: Times New Roman, Helvetica, Arial, Cambria, and 
Symbol. If you are supplying EPS, PS, or PDF files all fonts must be 
embedded. Some fonts may only be native to your operating system; without 
the fonts embedded, parts of the graphic may be distorted or missing.

A safe option when finalizing your figures is to strip out the fonts before 
you save the files, creating ``outline'' type. This converts fonts to 
artwork what will appear uniformly on any screen.

\subsection{Using Labels Within Figures}

\subsubsection{Figure Axis labels }
Figure axis labels are often a source of confusion. Use words rather than 
symbols. As an example, write the quantity ``Magnetization,'' or 
``Magnetization M,'' not just ``M.'' Put units in parentheses. Do not label 
axes only with units. As in Fig. 1, for example, write ``Magnetization 
(A/m)'' or ``Magnetization (Am),'' not just ``A/m.'' Do not label axes with a ratio of quantities and 
units. For example, write ``Temperature (K),'' not ``Temperature/K.'' 

Multipliers can be especially confusing. Write ``Magnetization (kA/m)'' or 
``Magnetization (10 A/m).'' Do not write ``Magnetization 
(A/m)1000'' because the reader would not know whether the top 
axis label in Fig. 1 meant 16000 A/m or 0.016 A/m. Figure labels should be 
legible, approximately 8 to 10 point type.

\subsubsection{Subfigure Labels in Multipart Figures and Tables}
Multipart figures should be combined and labeled before final submission. 
Labels should appear centered below each subfigure in 8 point Times New 
Roman font in the format of (a) (b) (c). 

\subsection{File Naming}
Figures (line artwork or photographs) should be named starting with the 
first 5 letters of the author's last name. The next characters in the 
filename should be the number that represents the sequential 
location of this image in your article. For example, in author 
``Anderson's'' paper, the first three figures would be named ander1.tif, 
ander2.tif, and ander3.ps.

Tables should contain only the body of the table (not the caption) and 
should be named similarly to figures, except that `.t' is inserted 
in-between the author's name and the table number. For example, author 
Anderson's first three tables would be named ander.t1.tif, ander.t2.ps, 
ander.t3.eps.

Author photographs should be named using the first five characters of the 
pictured author's last name. For example, four author photographs for a 
paper may be named: oppen.ps, moshc.tif, chen.eps, and duran.pdf.

If two authors or more have the same last name, their first initial(s) can 
be substituted for the fifth, fourth, third letters of their surname 
until the degree where there is differentiation. For example, two authors 
Michael and Monica Oppenheimer's photos would be named oppmi.tif, and 
oppmo.eps.

\subsection{Referencing a Figure or Table Within Your Paper}
When referencing your figures and tables within your paper, use the 
abbreviation ``Fig.'' even at the beginning of a sentence. Do not abbreviate 
``Table.'' Tables should be numbered with Roman Numerals.

\subsection{Checking Your Figures: The IEEE Graphics Analyzer}
The IEEE Graphics Analyzer enables authors to pre-screen their graphics for 
compliance with IEEE Access standards before submission. 
The online tool, located at
\underline{http://graphicsqc.ieee.org/}, allows authors to 
upload their graphics in order to check that each file is the correct file 
format, resolution, size and colorspace; that no fonts are missing or 
corrupt; that figures are not compiled in layers or have transparency, and 
that they are named according to the IEEE Access naming 
convention. At the end of this automated process, authors are provided with 
a detailed report on each graphic within the web applet, as well as by 
email.

For more information on using the Graphics Analyzer or any other graphics 
related topic, contact the IEEE Graphics Help Desk by e-mail at 
graphics@ieee.org.

\subsection{Submitting Your Graphics}
Because IEEE will do the final formatting of your paper,
you do not need to position figures and tables at the top and bottom of each 
column. In fact, all figures, figure captions, and tables can be placed at 
the end of your paper. In addition to, or even in lieu of submitting figures 
within your final manuscript, figures should be submitted individually, 
separate from the manuscript in one of the file formats listed above in 
Section \ref{formats}. Place figure captions below the figures; place table titles 
above the tables. Please do not include captions as part of the figures, or 
put them in ``text boxes'' linked to the figures. Also, do not place borders 
around the outside of your figures.

\subsection{Color Processing/Printing in IEEE Journals}
All IEEE Transactions, Journals, and Letters allow an author to publish 
color figures on IEEE Xplore\textregistered\ at no charge, and automatically 
convert them to grayscale for print versions. In most journals, figures and 
tables may alternatively be printed in color if an author chooses to do so. 
Please note that this service comes at an extra expense to the author. If 
you intend to have print color graphics, include a note with your final 
paper indicating which figures or tables you would like to be handled that 
way, and stating that you are willing to pay the additional fee.

\section{Conclusion}
A conclusion section is not required. Although a conclusion may review the 
main points of the paper, do not replicate the abstract as the conclusion. A 
conclusion might elaborate on the importance of the work or suggest 
applications and extensions. 

\appendices

Appendixes, if needed, appear before the acknowledgment.

\section*{Acknowledgment}

The preferred spelling of the word ``acknowledgment'' in American English is 
without an ``e'' after the ``g.'' Use the singular heading even if you have 
many acknowledgments. Avoid expressions such as ``One of us (S.B.A.) would 
like to thank  .'' Instead, write ``F. A. Author thanks  .'' In most 
cases, sponsor and financial support acknowledgments are placed in the 
unnumbered footnote on the first page, not here.

\section*{References and Footnotes}

\subsection{References}
References need not be cited in text. When they are, they appear on the 
line, in square brackets, inside the punctuation. Multiple references are 
each numbered with separate brackets. When citing a section in a book, 
please give the relevant page numbers. In text, refer simply to the 
reference number. Do not use ``Ref.'' or ``reference'' except at the 
beginning of a sentence: ``Reference \cite{b3} shows  .'' Please do not use 
automatic endnotes in \emph{Word}, rather, type the reference list at the end of the 
paper using the ``References'' style.

Reference numbers are set flush left and form a column of their own, hanging 
out beyond the body of the reference. The reference numbers are on the line, 
enclosed in square brackets. In all references, the given name of the author 
or editor is abbreviated to the initial only and precedes the last name. Use 
them all; use \emph{et al.} only if names are not given. Use commas around Jr., 
Sr., and III in names. Abbreviate conference titles. When citing IEEE 
transactions, provide the issue number, page range, volume number, year, 
and/or month if available. When referencing a patent, provide the day and 
the month of issue, or application. References may not include all 
information; please obtain and include relevant information. Do not combine 
references. There must be only one reference with each number. If there is a 
URL included with the print reference, it can be included at the end of the 
reference. 

Other than books, capitalize only the first word in a paper title, except 
for proper nouns and element symbols. For papers published in translation 
journals, please give the English citation first, followed by the original 
foreign-language citation See the end of this document for formats and 
examples of common references. For a complete discussion of references and 
their formats, see the IEEE style manual at
\underline{http://www.ieee.org/authortools}.

\subsection{Footnotes}
Number footnotes separately in superscript numbers.\footnote{It is recommended that footnotes be avoided (except for 
the unnumbered footnote with the receipt date on the first page). Instead, 
try to integrate the footnote information into the text.} Place the actual 
footnote at the bottom of the column in which it is cited; do not put 
footnotes in the reference list (endnotes). Use letters for table footnotes 
(see Table \ref{table}).

\section{Submitting Your Paper for Review}

\subsection{Final Stage}
When you submit your final version (after your paper has been accepted), 
print it in two-column format, including figures and tables. You must also 
send your final manuscript on a disk, via e-mail, or through a Web 
manuscript submission system as directed by the society contact. You may use 
\emph{Zip} for large files, or compress files using \emph{Compress, Pkzip, Stuffit,} or \emph{Gzip.} 

Also, send a sheet of paper or PDF with complete contact information for all 
authors. Include full mailing addresses, telephone numbers, fax numbers, and 
e-mail addresses. This information will be used to send each author a 
complimentary copy of the journal in which the paper appears. In addition, 
designate one author as the ``corresponding author.'' This is the author to 
whom proofs of the paper will be sent. Proofs are sent to the corresponding 
author only.

\subsection{Review Stage Using ScholarOne\textregistered\ Manuscripts}
Contributions to the Transactions, Journals, and Letters may be submitted 
electronically on IEEE's on-line manuscript submission and peer-review 
system, ScholarOne\textregistered\ Manuscripts. You can get a listing of the 
publications that participate in ScholarOne at 
\underline{http://www.ieee.org/publications\_standards/publications/}\break\underline{authors/authors\_submission.html}.
First check if you have an existing account. If there is none, please create 
a new account. After logging in, go to your Author Center and click ``Submit 
First Draft of a New Manuscript.'' 

Along with other information, you will be asked to select the subject from a 
pull-down list. Depending on the journal, there are various steps to the 
submission process; you must complete all steps for a complete submission. 
At the end of each step you must click ``Save and Continue''; just uploading 
the paper is not sufficient. After the last step, you should see a 
confirmation that the submission is complete. You should also receive an 
e-mail confirmation. For inquiries regarding the submission of your paper on 
ScholarOne Manuscripts, please contact oprs-support@ieee.org or call +1 732 
465 5861.

ScholarOne Manuscripts will accept files for review in various formats. 
Please check the guidelines of the specific journal for which you plan to 
submit.

You will be asked to file an electronic copyright form immediately upon 
completing the submission process (authors are responsible for obtaining any 
security clearances). Failure to submit the electronic copyright could 
result in publishing delays later. You will also have the opportunity to 
designate your article as ``open access'' if you agree to pay the IEEE open 
access fee. 

\subsection{Final Stage Using ScholarOne Manuscripts}
Upon acceptance, you will receive an email with specific instructions 
regarding the submission of your final files. To avoid any delays in 
publication, please be sure to follow these instructions. Most journals 
require that final submissions be uploaded through ScholarOne Manuscripts, 
although some may still accept final submissions via email. Final 
submissions should include source files of your accepted manuscript, high 
quality graphic files, and a formatted pdf file. If you have any questions 
regarding the final submission process, please contact the administrative 
contact for the journal. 

In addition to this, upload a file with complete contact information for all 
authors. Include full mailing addresses, telephone numbers, fax numbers, and 
e-mail addresses. Designate the author who submitted the manuscript on 
ScholarOne Manuscripts as the ``corresponding author.'' This is the only 
author to whom proofs of the paper will be sent. 

\subsection{Copyright Form}
Authors must submit an electronic IEEE Copyright Form (eCF) upon submitting 
their final manuscript files. You can access the eCF system through your 
manuscript submission system or through the Author Gateway. You are 
responsible for obtaining any necessary approvals and/or security 
clearances. For additional information on intellectual property rights, 
visit the IEEE Intellectual Property Rights department web page at 
\underline{http://www.ieee.org/publications\_standards/publications/}\break\underline{rights/index.html}. 

\section{IEEE Publishing Policy}
The general IEEE policy requires that authors should only submit original 
work that has neither appeared elsewhere for publication, nor is under 
review for another refereed publication. The submitting author must disclose 
all prior publication(s) and current submissions when submitting a 
manuscript. Do not publish ``preliminary'' data or results. The submitting 
author is responsible for obtaining agreement of all coauthors and any 
consent required from employers or sponsors before submitting an article. 
The IEEE Access Department strongly discourages courtesy 
authorship; it is the obligation of the authors to cite only relevant prior 
work.

The IEEE Access Department does not publish conference 
records or proceedings, but can publish articles related to conferences that 
have undergone rigorous peer review. Minimally, two reviews are required for 
every article submitted for peer review.


\section{Reference Examples}

\begin{itemize}

\item \emph{Basic format for books:}\\
J. K. Author, ``Title of chapter in the book,'' in \emph{Title of His Published Book, x}th ed. City of Publisher, (only U.S. State), Country: Abbrev. of Publisher, year, ch. , sec. , pp. \emph{xxx--xxx.}\\
See \cite{b1,b2}.

\item \emph{Basic format for periodicals:}\\
J. K. Author, ``Name of paper,'' \emph{Abbrev. Title of Periodical}, vol. \emph{x, no}. pp\emph{. xxx--xxx, }Abbrev. Month, year, DOI. 10.1109.\emph{XXX}.123456.\\
See \cite{b3}--\cite{b5}.

\item \emph{Basic format for reports:}\\
J. K. Author, ``Title of report,'' Abbrev. Name of Co., City of Co., Abbrev. State, Country, Rep. \emph{xxx}, year.\\
See \cite{b6,b7}.

\item \emph{Basic format for handbooks:}\\
\emph{Name of Manual/Handbook, x} ed., Abbrev. Name of Co., City of Co., Abbrev. State, Country, year, pp. \emph{xxx--xxx.}\\
See \cite{b8,b9}.

\item \emph{Basic format for books (when available online):}\\
J. K. Author, ``Title of chapter in the book,'' in \emph{Title of
Published Book}, th ed. City of Publisher, State, Country: Abbrev.
of Publisher, year, ch. , sec. , pp. \emph{xxx--xxx}. [Online].
Available: \underline{http://www.web.com}\\
See \cite{b10}--\cite{b13}.

\item \emph{Basic format for journals (when available online):}\\
J. K. Author, ``Name of paper,'' \emph{Abbrev. Title of Periodical}, vol. , no. , pp. \emph{xxx--xxx}, Abbrev. Month, year. Accessed on: Month, Day, year, DOI: 10.1109.\emph{XXX}.123456, [Online].\\
See \cite{b14}--\cite{b16}.

\item \emph{Basic format for papers presented at conferences (when available online): }\\
J.K. Author. (year, month). Title. presented at abbrev. conference title. [Type of Medium]. Available: site/path/file\\
See \cite{b17}.

\item \emph{Basic format for reports and handbooks (when available online):}\\
J. K. Author. ``Title of report,'' Company. City, State, Country. Rep. no., (optional: vol./issue), Date. [Online] Available: site/path/file\\
See \cite{b18,b19}.

\item \emph{Basic format for computer programs and electronic documents (when available online): }\\
Legislative body. Number of Congress, Session. (year, month day). \emph{Number of bill or resolution}, \emph{Title}. [Type of medium]. Available: site/path/file\\
\textbf{\emph{NOTE: }ISO recommends that capitalization follow the accepted practice for the language or script in which the information is given.}\\
See \cite{b20}.

\item \emph{Basic format for patents (when available online):}\\
Name of the invention, by inventor's name. (year, month day). Patent Number [Type of medium]. Available: site/path/file\\
See \cite{b21}.

\item \emph{Basic format}\emph{for conference proceedings (published):}\\
J. K. Author, ``Title of paper,'' in \emph{Abbreviated Name of Conf.}, City of Conf., Abbrev. State (if given), Country, year, pp. \emph{xxxxxx.}\\
See \cite{b22}.

\item \emph{Example for papers presented at conferences (unpublished):}\\
See \cite{b23}.

\item \emph{Basic format for patents}\\
J. K. Author, ``Title of patent,'' U.S. Patent \emph{x xxx xxx}, Abbrev. Month, day, year.\\
See \cite{b24}.

\item \emph{Basic format for theses (M.S.) and dissertations (Ph.D.):}
\begin{enumerate}
\item J. K. Author, ``Title of thesis,'' M.S. thesis, Abbrev. Dept., Abbrev. Univ., City of Univ., Abbrev. State, year.
\item J. K. Author, ``Title of dissertation,'' Ph.D. dissertation, Abbrev. Dept., Abbrev. Univ., City of Univ., Abbrev. State, year.
\end{enumerate}
See \cite{b25,b26}.

\item \emph{Basic format for the most common types of unpublished references:}
\begin{enumerate}
\item J. K. Author, private communication, Abbrev. Month, year.
\item J. K. Author, ``Title of paper,'' unpublished.
\item J. K. Author, ``Title of paper,'' to be published.
\end{enumerate}
See \cite{b27}--\cite{b29}.

\item \emph{Basic formats for standards:}
\begin{enumerate}
\item \emph{Title of Standard}, Standard number, date.
\item \emph{Title of Standard}, Standard number, Corporate author, location, date.
\end{enumerate}
See \cite{b30,b31}.

\item \emph{Article number in~reference examples:}\\
See \cite{b32,b33}.

\item \emph{Example when using et al.:}\\
See \cite{b34}.

\end{itemize}