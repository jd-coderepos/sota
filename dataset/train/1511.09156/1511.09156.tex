\documentclass[11pt]{article}
\usepackage[margin=1in]{geometry}
\usepackage{amsmath,amssymb,amsbsy,amsgen,amsopn,amsthm}
\usepackage{graphicx}
\usepackage{mathrsfs}
\newcommand{\Qset}{\mathbb{Q}}
\newcommand{\Rset}{\mathbb{R}}
\newcommand{\Zset}{\mathbb{Z}}
\newcommand{\Nset}{\mathbb{N}}
\newcommand{\Ffam}{\mathcal{F}}
\newcommand{\Mfam}{\mathcal{M}}
\newcommand{\Cfam}{\mathcal{C}}
\newcommand{\Afam}{\mathcal{A}}
\newcommand{\Dfam}{\mathcal{D}}
\newcommand{\Lfam}{\mathcal{L}}
\newcommand{\Xfam}{\mathcal{X}}
\newcommand{\Vfam}{\mathcal{V}}
\newcommand{\opt}{\mathrm{OPT}}

\usepackage{algorithm,algorithmic}
\renewcommand{\algorithmicrequire}{\textbf{Input:}}
\renewcommand{\algorithmicensure}{\textbf{Output:}}

\newtheorem{theorem}{Theorem}
\newtheorem{lemma}{Lemma}
\newtheorem{corollary}{Corollary}
\newtheorem{definition}{Definition}

\begin{document}

\title{Approximation algorithms for highly
connected \\ multi-dominating sets in unit disk graphs}

\author{Takuro Fukunaga\thanks{This work was done when the authors
was at National Institute of Informatics, Japan. Email: {\tt takuro.fukunaga@riken.jp}}}

\date{}


\maketitle

\begin{abstract}
  Given an undirected graph on a node set 
  and positive integers  and ,
  a -connected -dominating set (-CDS)
  is defined as a subset  of  
  such that each node in 
  has at least  neighbors in , and
 a -connected subgraph is induced by .
  The weighted -CDS problem is to
  find a minimum weight -CDS
  in a given node-weighted graph.
  The problem is called
  the unweighted -CDS problem
  if the objective
  is to minimize the cardinality of a -CDS.
  These problems
  have been actively studied for unit disk graphs,
  motivated by the application of constructing a virtual backbone 
  in a wireless ad hoc network.
In this paper,
  we consider the case in which ,
  and we present a simple
  -approximation algorithm for the unweighted -CDS
 problem,
 and a primal-dual -approximation algorithm for the
 weighted -CDS problem. 
 \end{abstract}


 
\section{Introduction}
\label{sec.intro}


A wireless ad hoc network is a decentralized wireless network.
Compared with a traditional communication
network, it has the advantage of not requiring
any infrastructure, such as base stations and WiFi routers; this is a
great benefit when operating sensor networks, vehicle networks, or
networks in disaster areas.  However, for efficient operation of a
wireless ad hoc network, we have to overcome many technical challenges.
One of these is to reduce the redundant communication caused by flooding
messages.  
A typical solution 
for this task
is to construct a
\emph{virtual backbone network}, as follows:
we choose several backbone nodes from the network,
and then we construct a subnetwork that comprises only these backbone nodes.
When a message arrives, first, it is delivered to a backbone node,
next, it is flooded to all the backbone nodes via the virtual backbone
network,
and finally, the destination node receives the message from an adjacent backbone node.
This reduces redundancy more as the virtual backbone network is
smaller.
However, it is also important that the virtual backbone network
is fault tolerant.

Developing algorithms for constructing a virtual backbone network
is an active area of research.
A promising approach
is to formulate
a virtual backbone network 
as a \emph{connected dominating set} (CDS), and
to consider an algorithm for finding a minimum cardinality or a minimum
weight CDS.
For an undirected graph with a node set , a CDS
is defined as a subset  of  such that
each node in  is adjacent to at least one node in ,
and  induces a connected subgraph.
This approach has gained in popularity,
and many papers have been published~~\cite{AmbuhlEMN06,ChengHLWD03,GuhaK98,GuhaK99,MaratheBHRR95,ZouLGW09}.
It is typically assumed that the input graph is a unit disk graph,
which is a natural choice for modeling a wireless network.
Since the problem of finding the minimum
cardinality CDS
is NP-hard even for unit disk graphs~\cite{ClarkCJ90},
some studies have considered approximation algorithms.


A CDS does not give a fault-tolerant virtual backbone network.
This is because a CDS is only required to be connected,
and each node outside a CDS is required to have only one neighbor in the CDS.
Hence, if a backbone node fails,
the virtual backbone network may be disconnected,
or a non-backbone node may lose access to the virtual backbone network.
To overcome this disadvantage, 
Dai and
Wu~\cite{Dai:2006}
proposed replacing a CDS by a \emph{-connected -dominating set},
and they addressed the problem of finding a minimum -connected
-dominating set in a unit disk graph.
For a graph with the node set ,
a subset  of 
is called \emph{-connected} if the subgraph induced by 
is -connected (i.e., it is connected even if any  nodes are
removed),
and is called \emph{-dominating} if each node  has 
neighbors in .
Triggered by their study, 
much attention has been paid to this problem,
extending the notion of 
a -connected -dominating set to a more-general -connected -dominating
set (-CDS).

The problem of finding a minimum cardinality -CDS in a unit disk
graph is called the \emph{unweighted -CDS problem}.
If each node is given
a nonnegative weight, and
the objective is to minimize the weight of a -CDS,
then this is called the \emph{weighted -CDS problem}.
As for the unweighted -CDS problem,
several 
constant-approximation algorithms were given
for ~\cite{Shang:2007jg,ShiZZW16,WangKAGLZW13,Wang:2015,Zhang16}.
As for the weighted -CDS problem,
there are several constant-approximation algorithms for ~\cite{AmbuhlEMN06,ZouLGW09},
but no approximation algorithm was known for
the case of  before our study
(see Section~\ref{sec.relatedworks} for more literature reviews).

After these previous studies,
a natural question arises as to whether there
is a constant-approximation algorithm
for the unweighted -CDS problem with ,
and for the weighted problem with .
For the unweighted problem, this question has been already addressed in
both \cite{WangKAGLZW13} and \cite{Wang:2015}.


\subsection{Our results}
\label{sec.ourresults}
We answer the above question affirmatively.
The main contribution of this paper is to present
constant-approximation algorithms
for both the unweighted and the weighted -CDS problems
when  is a constant and .

Specifically, we present two algorithms;
one is an -approximation algorithm for the unweighted -CDS problem,
and the other is an -approximation algorithm for the
weighted -CDS problem.
Notice that both algorithms achieve a constant factor if  is a constant.
The approximation factor of the latter algorithm is better than that of
the former,
and it can be applied to the weighted problem, while the former algorithm is
restricted to the unweighted problem.
However, the former algorithm is simple, easy to analyze,
and can also be applied to other graph classes.
In fact,
 for ,
 the former algorithm is obtained by introducing more
 specification into the algorithms
 given in~\cite{Shang:2007jg,Wang:2015}.
 Hence our analysis on the former algorithm gives a simple proof for the
 fact that the algorithms in~\cite{Shang:2007jg,Wang:2015} achieve
 constant-approximation for .


When the author published a preliminary version \cite{Fukunaga15arxiv} of this work,
he noticed that
Shi, Zhang, and Du~\cite{ShiZD15} also presented an -approximation
algorithm for the weighted -CDS problem in unit disk graphs with
.
The journal version of their paper was published as Shi~et~al.\ \cite{ShiZMD17}.
Their study has been done independently from ours,
but the taken approach is similar to ours
in the fact that
both of their and our algorithms require
computing an -dominating set,
and then convert it into a -connected -dominating set.
In the first step,
both of the algorithms require
an approximation algorithm for the minimum weight -dominating
set problem.
Shi, Zhang, and Du~\cite{ShiZD15}  suggested using the algorithm
given in~\cite{WillsonZWD15} for this purpose, but
the algorithm of~\cite{WillsonZWD15} 
deals with a problem slightly different from the minimum weight -dominating set problem.
In the present paper, we give the first nontrivial approximation
algorithm for this problem (in Section~\ref{subs.dominating}),
and hence
the algorithm of Shi, Zhang, and Du~\cite{ShiZD15}
achieves the claimed approximation guarantee only when combined with our result.
In their journal version~\cite{ShiZMD17},
this fact is described.
In the second step of converting the -dominating set into a
-connected -dominating set,
the approach of  \cite{ShiZD15,ShiZMD17}
and ours are different.
We give an -approximation algorithm for
 the augmentation
 problem,
 which is a special case of the optimization problem arising in the
 second step.
 From this algorithm, we derive an -approximation for the general
 case.
 On the other hand,  \cite{ShiZD15,ShiZMD17} gave a reduction from the general case
 to the edge-weighted subset -connected subgraph problem.
 Their approach is weaker than our result
 in the fact that their reduction does not give a better algorithm for the
 augmentation problem.
 To be fair, their reduction approach has a merit that
  enables to use any algorithm for the edge-weighted subset -connected subgraph problem
 while our analysis applies to a specific primal-dual algorithm.



Although this is not our main focus, 
we also present an improved analysis of the algorithm for the unweighted
-CDS problem 
in Shang~et~al.~\cite{Shang:2007jg};
see Appendix~\ref{app.rect}.
Shi~et~al.~\cite{ShiZZW16} pointed out that the analysis of \cite{Shang:2007jg}
contains a flaw and the approximation factor given therein is not correct.
Shi~et~al.\ also gave a modified analysis, but our new analysis given in
Appendix~\ref{app.rect} attains a 
better approximation factor.

 \subsection{Organization}

 The remainder of this paper is organized as follows. 
 Section~\ref{sec.relatedworks} surveys related works.
 Section~\ref{sec.prelim} introduces preliminary facts used in this
 paper.
 Section~\ref{sec.simple} presents our -approximation
 algorithm for the unweighted -CDS problem, and
 Section~\ref{sec.primal-dual} provides our -approximation algorithm for the weighted -CDS
 problem. Section~\ref{sec.conclusion} concludes the paper.
 Appendix~\ref{app.rect} rectifies the analysis on the algorithm of Shang~et~al.~\cite{Shang:2007jg}
 for the unweighted -CDS problem.
 
\section{Related works}
\label{sec.relatedworks}

The study of the -CDS problem for general graphs was initiated 
by Guha and Khuller~\cite{GuhaK98}. They presented an
-approximation algorithm for the unweighted -CDS
problem in graphs
with maximum degree ,
where  denotes the -th harmonic number.
They also
gave a reduction from the set cover to the unweighted -CDS problem,
which shows that no polynomial-time algorithm achieves
an approximation factor 
for any fixed  unless
.
For the weighted -CDS problem,
they gave an -approximation
algorithm, where  is the number of nodes in
the given graph.
This algorithm was improved by the same authors to  in~\cite{GuhaK99}.

The unweighted -CDS problem is NP-hard, even in unit disk graphs~\cite{ClarkCJ90}.
 Marathe~et~al.~\cite{MaratheBHRR95}
showed that the unweighted -CDS problem in unit disk graphs admits a 10-approximation algorithm.
This has been improved by subsequent studies, and the current
best result is due to 
Cheng~et~al.~\cite{ChengHLWD03}, who gave a
polynomial-time approximation scheme
(PTAS) for the unweighted
-CDS problem in unit disk graphs; note that the existence of a PTAS means
that for
any fixed constant , there exists a 
()-approximation algorithm that runs in polynomial time.

The first constant-approximation algorithm for
the weighted -CDS problem in unit disk graphs was due to
Amb{\"{u}}hl~et~al.~\cite{AmbuhlEMN06}.
The current best approximation factor
for the same problem
is
, where  is the approximation factor for the
edge-weighted Steiner tree in general graphs.
 This is achieved by
combining the -approximation algorithm
due to Willson~et~al.\ \cite{WillsonZWD15}
for the minimum
weight 1-cover problem, and the -approximation
algorithm due to Zou~et~al.\ \cite{ZouLGW09}
for the node-weighted Steiner tree problem.
Currently it is known that ~\cite{ByrkaGRS13}.
Zou~et~al.\ \cite{ZouWXLDWW11} claimed a -approximation
algorithm for the weighted -CDS problem, but their algorithm does
not seem to run in polynomial time for general node weights.


The first study of the -CDS problem
was conducted by Dai and Wu~\cite{Dai:2006}.
They found several heuristic algorithms for the unweighted problem with
 in unit disk graphs.
Thus far, several constant-approximation algorithms have been given for the
unweighted -CDS problem in unit disk graphs~\cite{Shang:2007jg,WangTD09,WangKAGLZW13,Wang:2015}
For , Shang~et~al.\ \cite{Shang:2007jg} gave an 
approximation algorithm,
and it achieves a current best approximation factor for .
As mentioned in Section~\ref{sec.ourresults},
Shi~et~al.~\cite{ShiZZW16}
pointed out a flow in the analysis of
Shang~et~al.\ \cite{Shang:2007jg} for ,
and gave its rectification.
Simultaneously Shi~et~al.\ presented an algorithm for the unweighted -CDS problem on general
graphs. This algorithm achieves the current best approximation factor for
the unweighted -CDS problem even on
unit disk graphs.
In Appendix~\ref{app.rect},
we present an improved analysis of Shang~et~al.~\cite{Shang:2007jg} for
, but the obtained approximation factors are not better than 
Shi~et~al.
For ,
the current best approximation algorithm is 
due to
Zhang~et~al.\ \cite{Zhang16} (their algorithm is defined for general graphs).
Several previous papers
claimed constant-approximation algorithms for , but
Kim~et~al.~\cite{KimWLZW10} showed that
all of them had technical errors.
As far as we know, there was no known constant-approximation algorithm for the
unweighted -CDS problem with  or for
the weighted -CDS problem with  before our study.
For the weighted problem, it was not even known whether a
constant-approximation algorithm exists for the problem of finding a
minimum weight -dominating set in a unit disk graph when .



We note that 
distributed algorithms for
the unweighted -CDS problem are also actively being studied.
Since this paper will focus on centralized algorithms,
we will only refer to a few of these
studies~\cite{Alzoubi02,FunkeKMS06,YuJYLC2015}.


 \section{Preliminaries}
 \label{sec.prelim}

 
 Let  be an undirected graph with a node set  and an edge
 set .
 We denote  by .
 For , let  denote the subgraph of 
 induced by ; i.e., its node set is  and the edge set consists of
 the edges that join nodes in .
 Throughout the paper, on the power set of , we define maximality and minimality
with respect to inclusion.
 In other words,  is minimal in a family 
 if there is no  with ,
 and  is maximal in  if there is no 
 with .

 In a \emph{unit disk graph}, each node is placed on the two-dimensional
 Euclidean space,
 and two nodes are joined by an edge if and only if the distance between
 them is not larger than a unit length.
 The following property of unit disk graphs is well known,
 and it is used in \cite{Shang:2007jg}.

 \begin{lemma}\label{lem.kissingnum}
  Let  be a unit disk graph. Let ,
  and let  be distinct neighbors of .
  Then  includes an edge that joins two nodes in .
 \end{lemma}

 For the most part, our algorithms will require only the property stated
 in Lemma~\ref{lem.kissingnum}; the exception is the subroutine
 of computing an -dominating set
 in our -approximation algorithm
 (presented in Section~\ref{subs.dominating}).
 
For , we denote the set of neighbors of  in  by . In
other words, .
We also let  denote  for any .
For , we simply denote  by
.
The following property of the function 
has been frequently used in previous works on the survivable networks
design (see e.g., \cite{Nutov12}).


\begin{lemma}\label{lem.submodularity}
 For any ,
  the following hold:
 
 
\end{lemma}

 
Let .
For ,
 is called a
 \emph{-cut} if  and  , and it is called a
\emph{Steiner -cut} if
.
 Notice that a Steiner- cut is not necessarily a subset of 
 while a -cut is so.
For  and ,
 a -cut 
 (resp., a Steiner -cut )
 is called a -cut
 (resp., Steiner -cut)
 if .
A graph  is -connected
when it is connected even when any  or fewer nodes are removed from the graph.
We note that a graph on at most  nodes is -connected by definition if it is a complete graph.
By Menger's theorem, a graph  is -connected if and only if  for any nonempty  with .
A subset  of  is -connected if and only if 
for any -cut 
(recall that the -connectivity of  is defined by the
-connectivity of ).

Our algorithms take the same approach
proposed in
previous studies;
they compute an -dominating set in the first step, and
increase its connectivity to .
In our algorithms, this is achieved by solving
the following problems.

 \begin{definition}[Augmentation problem]
  \label{def.augmentation}
 Given an undirected graph , a nonnegative weight  for
 each node , and a -CDS
  of , find 
 that minimizes , 
 subject to the condition that  is -connected in  (i.e.,
   is a -connected graph).
 \end{definition}

 For , a path  is called a \emph{-path}
 if both end nodes of  are included in , and no inner nodes of 
 are included in . 

A family  of subsets of  is said to be 
\emph{laminar}
if any  satisfy , ,
or .
 is said to be
\emph{strongly laminar}
if any  satisfies , ,
or .
Let  be a laminar family.
If  is minimal in ,
we call  a \emph{leaf} of .
For some ,
if  is a maximal member of  
subject to ,
then we say that  is a \emph{child} of .


\section{Simple algorithm for the unweighted problem}
\label{sec.simple}

In this section, we present an -approximation algorithm for the unweighted
-CDS problem
with .
We may assume  is -connected,
since otherwise  has no -CDS.

Our algorithm computes an -dominating set by 
applying the constant-approximation algorithm given in
\cite{Shang:2007jg},
and then increases its connectivity by adding nodes.
Let us explain this in more details.
Let  be an -dominating set.
Note that .
First,
we choose  roots .
For each ,
we compute a node set 
such that the connectivity between  and each node  is at least  in the graph 
(i.e.,  includes  internally disjoint paths between 
and each node ).
Our solution is .

Let us see that  is a -CDS.
We can see that each pair of nodes in  is -connected in

(we say in this case that \emph{ is -connected in }),
as follows.
If  is not -connected in ,
there exists 
with 
such that  and  are not connected after removing .
Since , there exists  such that .
This  is not connected to  or  in the graph after removing
, which contradicts the definition of .
This means that  is -connected in
.
Moreover, from the facts that  is -dominating and ,
it follows that  is also -connected.
Indeed, we can observe that any node in  is not disconnected from other
nodes even after  nodes are removed from .
Therefore,   is a -CDS.

We show that  holds for each .
Since ,
this implies that .
Hence   achieves the approximation factor .

To compute , we repeat solving the augmentation problem.
Here, the definition of the augmentation problem is slightly different
from Definition~\ref{def.augmentation};
 is not necessarily -connected here.
Instead, the connectivity between  and each node  is  in  for some integer 
with ,
and we are required to compute a minimum size 
such that
 connectivity between  and each node in  is  in .
We call this version of the augmentation problem by the \emph{rooted
augmentation problem}.
For the rooted augmentation problem,
 we prove that it is possible to find a solution 
 such that .
 When we compute , we solve the problem
 with setting  to .
 This gives 
 holds for any .
  is defined as .
 Then ,
 indicating that .
 
 In the rest of this section, we discuss the rooted augmentation
 problem.
 For simplicity,
 we denote  by , , and , respectively.
 Namely, the connectivity between  and each  is 
 in , and we are asked to find a minimum size 
 such that the connectivity between  and each  is at least  in
 .


 By the definition of ,
all -cuts  satisfy .
We say that a -cut  is a \emph{demand cut}
if .
We also say that a -path  \emph{covers} a demand cut 
if one end node of  is in , and the other end node is in
.

   \begin{lemma}\label{lem.Ucut}
    Let ,
    and let  be an -dominating set of a graph 
    such that the connectivity between  and each node in  is
    at least  in .
  Let .
  If a -cut  in  satisfies ,
  then  is a demand cut covered by no -path in .
 \end{lemma}
   \begin{proof}
    Since  is an -dominating set, if , 
   then  holds,
   which contradicts the assumption.
   Hence  is a -cut.
  Since ,
    we have .
    This means that  is a demand cut.
    In particular, since ,
    no -path in  covers .
   \end{proof}

   The following lemma presents a characterization of feasible solutions
   for the rooted augmentation problem.
    \begin{lemma}
     \label{lem.feasibility}
         Let ,
    and let  be an -dominating set of a graph 
    such that the connectivity between   and each node in  is at least
      in .
     is feasible for the rooted augmentation
    problem with , , and 
    if and only if every demand cut  is covered by a
    -path in .
    \end{lemma}
     \begin{proof}
      If there exists a demand cut  covered by no -path in ,
      then, by Lemma~\ref{lem.Ucut},
      any -cut  in  satisfies .
      The ``if'' part follows from this fact.

      Let us prove the ``only if'' part.
     Let  be a demand cut,
      and .
     If  is covered by 
     no -path in ,
     then each path connecting  and   in  
      includes some node in . Since ,
     this implies that the connectivity between  and  is at most
      in .  Therefore,  is not feasible.
     \end{proof}

  For ,
  we let  denote
  the family of all demand cuts
   covered by no -path in .
  Lemma~\ref{lem.feasibility} implies that  is feasible if and only if .



The following lemma was used in previous studies.

\begin{lemma}[\cite{Shang:2007jg,WangKAGLZW13,Wang:2015}]\label{lem.path}
 Let , and 
 let  be an -dominating set  of a -connected graph 
 such that the connectivity between  and each node in 
 is at least  in .
 For every demand cut  of  with respect to 
 (i.e.,  is a -cut and  in ),  includes a -path that covers 
 and contains at most two inner nodes.
\end{lemma}

  
From these observations, we can consider
the following simple algorithm for the rooted augmentation
problem.
Initialize a solution  to an empty set.
If  is not feasible for the rooted augmentation problem,
there exists a demand cut .
The algorithm 
chooses such a demand cut , and adds to 
the inner nodes of a
-path covering  that is guaranteed by
Lemma~\ref{lem.path}.
The algorithm repeats this until  becomes feasible.
In fact, this is almost same as the algorithms proposed in
\cite{Shang:2007jg,Wang:2015}
for  while \cite{Shang:2007jg,Wang:2015} consider
the augmentation problem instead of the rooted augmentation problem.
Every iteration of this algorithm adds at most two nodes to .
Hence,
in order to obtain an approximation guarantee for this algorithm,
it is critical to analyze how many iterations are
required to ensure that  is feasible.

We analyze the number of iterations for a general connectivity requirement .
To do this, we make a slight modification to the algorithm.
We restrict the demand cut
 that is chosen in each iteration, as follows.
Instead of an arbitrary demand cut in ,
our algorithm
always chooses a minimal of such cuts.
This procedure is described in detail as Algorithm~\ref{alg.simple}.

 \begin{algorithm}
  \caption{Algorithm for the rooted augmentation problem}
  \label{alg.simple}
  \begin{algorithmic}
   \REQUIRE
   integers   with ,
   an undirected graph ,
   an -dominating set ,
   and a root  such that
    and each node in  is
   -connected in 
   \ENSURE  such that
    and each node in  is -connected in 
   \STATE 
   \WHILE{ is not feasible}
   \STATE  a minimal cut in 
   \STATE  a minimum-length -path that covers 
   \STATE add the inner nodes in  to 
   \ENDWHILE
   \STATE output 
  \end{algorithmic}
 \end{algorithm}

 Let us explain that each iteration of Algorithm~\ref{alg.simple} can be done in polynomial time.
 A minimum-length -path  that covers 
 can be computed by a shortest path algorithm.
 For computing a minimal cut in ,
 we first compute a minimal -cut  that minimizes
  in . This can be found by applying 
 a max-flow algorithm repeatedly to the graph  with unit
 node-capacities except on the sink and the source
 while setting the source to each node in  and the sink to .
 Since  is not feasible,  holds.
 Lemma~\ref{lem.Ucut} indicates .
 Indeed,  is a minimal cut in . To see this, 
 suppose that there exists  with .
 Let  be the subset of  that consists of nodes on
 -paths  in the graph  having end nodes in . Then, since  is 
 covered by no -path, , which
 contradicts the minimality of .
 
 
 
In the following theorem, we show that  iterations are sufficient to ensure that
Algorithm~\ref{alg.simple} computes a feasible solution.

 \begin{theorem}
  \label{thm.simple-alg}
 Algorithm~\ref{alg.simple} 
  outputs a solution after  iterations.
 \end{theorem}

Before presenting a proof of Theorem~\ref{thm.simple-alg},
let us give a consequence of Theorem~\ref{thm.simple-alg}.
Each iteration of
 Algorithm~\ref{alg.simple} adds at most two nodes to the solution.
 Hence, Theorem~\ref{thm.simple-alg} immediately implies
 that
 Algorithm~\ref{alg.simple} outputs a solution  such that .
 By the discussion above, this presents the following corollary.

   \begin{corollary}\label{cor.unweighted}
    There exists an -approximation algorithm for 
    the unweighted -CDS problem in unit disk graphs
    if .
   \end{corollary}

   We note that our algorithm for the rooted augmentation problem
   does not rely on any property specific to unit disk graphs, and so
  the result in Corollary~\ref{cor.unweighted}
   can be extended to any graph class that admits a 
   constant-approximation algorithm for
   finding a minimum -dominating set.
   One such example
   is the class of 
   bounded clique- and tree-width graphs~\cite{CicaleseCGMV14}.
   

 
 Now, we prove Theorem~\ref{thm.simple-alg}.
 \begin{lemma}
  \label{lem.crossing}
  Let  and .
  Let . If  is minimal in , then
   or  holds.
  \end{lemma}
\begin{proof}
 Suppose that
 some pair of  and  violates this claim.
  Then,  holds.
  The minimality of  implies that  does not
 hold.
  Hence, 
  holds.
 Also,  follows from
  and , and
  follows, because
 .
 Therefore, both  and  are -cuts.

 For each -path  in ,
 we add an edge joining two end nodes of  to .
 Let  denote the graph with the node set  obtained by this operation. 
 Consider inequality \eqref{eq.submodular}, where  is
 defined with respect to the graph .
 The left-hand side of  \eqref{eq.submodular} is exactly ,
 because ,
 and the right-hand side is at least ,
 because  and  are -cuts.
 Hence, the inequality holds with equality, and
 .
 This means that both  and  belong to .
 This contradicts the minimality  of .
\end{proof}

Define  as the family of demand cuts
chosen in the while loop of Algorithm~\ref{alg.simple}.
Theorem~\ref{thm.simple-alg} is implied by the following lemma.

 \begin{lemma}
  \label{lem.laminarity}
 .
 \end{lemma}
 \begin{proof}
  We prove that  is a laminar family
  on .
  The lemma follows from this, because the size of a laminar family
  on the set of cardinality  is at most .
  
Suppose that there exist  such that
 , , and .
  We may assume without loss of generality that
   is chosen in an earlier iteration than that in which  is chosen.
  Let  denote the subset at the beginning of the iteration during which  is chosen.
  Then
  both  and  belong to ,
  and  is minimal in .
  However, Lemma~\ref{lem.crossing} shows that
  .
    This contradicts the definitions of  and .
 \end{proof}

 

\section{Primal-dual algorithm for the weighted problem}
\label{sec.primal-dual}

In this section, we present an -approximation algorithm
for the weighted -CDS problem.
Our algorithm for this problem is also based on a subroutine that
solves the augmentation problem. We show that there is a
constant-approximation algorithm for the augmentation problem with general
node weights.
This algorithm is based on the primal-dual method, which is a technique
for computing an approximate solution from a linear programming (LP) relaxation of the problem.
Before introducing the primal-dual algorithm, we consider the
weighted -dominating set problem, which 
demands a minimum weight -dominating set;
we prove that the problem admits a constant-approximation algorithm.

\subsection{Approximation algorithm for the weighted  -dominating set problem}
\label{subs.dominating}

Our algorithm reduces the weighted -dominating set problem to the following
geometric problem.

  
 \begin{definition}[Disk multicover problem]
 We are given a set  of points and a set  of disks on the
 Euclidean plane,
  a demand  for each point ,
 and a nonnegative weight  for each disk .
  A subset  of 
is called a disk multicover if
 each point  is contained in at least  disks in .
  The problem requires finding a disk multicover  that minimizes the weight .
  \end{definition}

When  for all , this is called the
\emph{disk cover problem}.
We write  if a point  is included in a disk .

Bansal and Pruhs~\cite{Bansal:2012ik} presented a constant-approximation algorithm for the disk
multicover problem.
Their algorithm is an LP-rounding algorithm. That is to say, their
algorithm first solves the following LP relaxation of the problem:

then it computes a disk multicover  
that satisfies  for an optimal solution  to \eqref{eq.lp-diskcover}.

When , the problem of finding a minimum weight -dominating set
in a unit disk graph
can be reduced to the disk cover problem, as follows.
Define  as the set of unit disks corresponding to the nodes in the
unit disk graph, and define  as the set of the centers of the disks.
The weight  of a disk  is defined as the weight of the
corresponding node in the graph.
For each point ,
a disk cover in this instance includes at least one disk
that contains . 
This means that the set of nodes corresponding to the disks in the 
disk cover
is a -dominating set of the graph.

From the weighted -dominating set problem with ,
we can similarly define an instance of the disk multicover problem;
, , and  are defined in the same way,
and the demand  of each disk  is defined as .
By solving this instance, we can obtain
an -dominating set in the unit disk graph.
However, the minimum weight of disk multicovers in the obtained instance
is possibly too large, compared with the minimum weight of the -dominating sets.
To see this, let  be a disk multicover in the obtained instance of the
disk multicover problem.
The constraint in the disk multicover
problem demands that each point  is included in at least

disks in . On the other hand, in the weighted -dominating set problem,
if a solution includes
a node ,
it is feasible even if
it does not contain  neighbors of .
In other words, the constraint of the disk multicover problem
in the constructed instance is stronger than the one demanded in the original
instance of the weighted -dominating set problem.
Accordingly, there does not seem to exist a straightforward reduction
from the weighted -dominating set problem to the disk multicover problem.

Nevertheless, we show that
the weighted -dominating set problem in a unit disk graph
can be approximated via an algorithm for 
the disk multicover problem.
Our algorithm first solves the following LP relaxation of the
weighted -dominating set problem:


Although \eqref{eq.lp-dominatingset}  has an exponential number
of constraints, the separation can be done in polynomial time.
Namely, given , we can judge whether  is a feasible solution for
\eqref{eq.lp-dominatingset}, as follows.
For each , sort the neighbors  so that
.
If   for each
,
then the constraint defined from  and all  is
satisfied by .
If  does not hold for some
,
then the constraint defined from  and

( when )
is
violated by .
Hence, the separation can be done by checking whether the 
inequalities hold for each node .
Therefore, the ellipsoid method can be used to solve \eqref{eq.lp-dominatingset} in polynomial time.

Let  denote an optimal solution for \eqref{eq.lp-dominatingset}.
We define  as .
Our algorithm invokes
an algorithm for the disk multicover problem after the input is set as follows.
 is defined as
the set of disks corresponding to the nodes in ,
and  is defined as the set of the centers of the disks in .
The demand  of the point 
corresponding to a node 
is defined as .
We solve the obtained instance of the disk multicover problem
by using an LP rounding algorithm based on \eqref{eq.lp-diskcover}, such as
the algorithm of Bansal and Pruhs~\cite{Bansal:2012ik}.
Let  be the set of nodes corresponding to the disks in the obtained approximate solution.
Our algorithm outputs  as an approximate
solution for the weighted -dominating set problem.

\begin{theorem}
 There exists a -approximation algorithm for the weighted -dominating set problem in a
 unit disk graph if an algorithm for the disk multicover problem
 computes a solution of weight at most
  for an optimal solution  to
 \eqref{eq.lp-diskcover}.
\end{theorem}
\begin{proof}
 We prove that our algorithm given above is a -approximation algorithm.
 Let  denote the minimum weight of the -dominating sets,
 and let  denote an optimal solution for \eqref{eq.lp-dominatingset}. Then,
  holds
 because
 \eqref{eq.lp-dominatingset} relaxes 
  the weighted -dominating set problem 
 (i.e., the incidence vector of each -dominating set is a feasible solution to
 \eqref{eq.lp-dominatingset}).

For each , let  and  respectively denote the point in  and the
 disk in  corresponding to .
 Define  by  for each .
 Then, for each ,
 
 holds, where the first inequality follows from the constraint
 of \eqref{eq.lp-dominatingset}, and the second inequality follows from
 .
Hence,  is a feasible solution to \eqref{eq.lp-diskcover}.
 The algorithm for the disk multicover problem computes a solution  such that  for some constant .
 On the other hand, , because  for
 each .
 Therefore, .

 Let  be a node that is not contained in .
 Then,  has  neighbors in 
 and  neighbors in .
 Therefore,  is an -dominating set.
 \end{proof}

 A drawback to our algorithm is that it requires the ellipsoid method,
 which tends to be slow in practice.
 When ,
 this can be avoided by using the above-mentioned straightforward reduction to the disk multicover problem.
 Recall that the reduction does not work in general
 because
 a node in an -dominating set  may not have  neighbors in .
 However, when  is -connected for some ,
 each node  has  neighbors in . Note that  is not
 counted as a neighbor of .
 Hence, the minimum weight of the disk multicovers can be bounded by the
 minimum weight of the -CDSs,
 and the straightforward reduction gives an -dominating set
 that has a weight within a constant factor of the minimum weight of
 the -CDSs.


 \subsection{Algorithm for the augmentation problem}
 \label{ssec.augmentation}
 
First, let us give an overview of our algorithm for the augmentation
problem.
When the connectivity requirement  is equal to one, 
the augmentation problem is known as the \emph{node-weighted Steiner
tree problem}.
For general graphs, it is hard to approximate this problem within a factor of ,
because it extends the set cover problem~\cite{KleinR95}.
However, in unit disk graphs, there is a constant-approximation
algorithm for the node-weighted Steiner tree problem.
Zou~et~al.~\cite{ZouLGW09} proved the existence of such an algorithm from the 
fact that
any unit disk graph has a spanning tree of maximum degree five.
This property of unit disk graphs is well known, and it can be shown by
using the following observation: if there is a node  of degree more than
five in a spanning tree, then by Lemma~\ref{lem.kissingnum},
there is an edge  that joins two neighbors  and  of .
Replacing the edge  by another edge  transforms the spanning tree into
another spanning tree in which the degree of  is decreased by one
(to ensure the existence of a spanning tree of maximum degree five, we must consider the degree of , because it is increased by the operation).
This approach cannot be directly extended to the general connectivity
requirement , because this operation does not preserve the connectivity
of a graph.
To see this, consider the graph on seven nodes  such that
 form a cycle of length six, and  is adjacent to each of
.
This graph is 3-connected, and the degree of  is six. To decrease the
degree of , replace one edge  by another edge , and 
then  will have only two neighbors; hence, the
connectivity of the graph has been decreased to two.

Nevertheless, we will show that Lemma~\ref{lem.kissingnum} can be used to show 
that the augmentation problem admits a better approximation algorithm for unit disk graphs
than it does for general graphs. We will use the lemma in the framework
of the primal-dual method,
which has been applied successfully to many
network design problems~\cite{goemans1997primal}.
Our algorithm repeats growing several dual
variables simultaneously in an LP relaxation.
This approach has been used in the augmentation
problem with node weights~\cite{ChekuriEV12,Fukunaga15},
but 
its approximation factor depends on the number of nodes.
This is because the approximation factor is decided by the number of
dual variables that
are grown simultaneously in a single constraint.
In our LP relaxation of the augmentation problem, each dual variable
corresponds to
a demand cut, and each constraint corresponds to
a node in the given graph.
Since this number cannot be bounded, the primal-dual
method does not achieve a good approximation factor for
general graphs, but we will show that this number can be bounded in unit disk graphs, due to Lemma~\ref{lem.kissingnum}.

Let us explain the detail of our algorithm.
Consider an instance of the augmentation problem with a graph 
and an -CDS  of .
Recall that  is defined to be feasible for the augmentation
problem
if  is -connected.
Indeed,
as did in Section~\ref{sec.simple},
from the assumptions that  is -dominating and ,
we can see that  is feasible if and only if
 is -connected in
.
Since  is -connected,
each Steiner -cut  satisfies  in .
By the Menger's theorem,  is feasible for the augmentation problem
if and only if  for 
each Steiner -cut  with .

We choose a root node , and consider
the problem of finding a minimum weight node set
 such that
 holds for 
every Steiner -cut  with .
By solving this problem for different  roots and
outputting the union of the obtained solutions,
we can solve the augmentation problem.

 For the remainder of this subsection, we fix a root node .
We say that a Steiner -cut  is a \emph{demand cut}
if .
 (Note that this is slightly different from the definition in
 Section~\ref{sec.simple}).
We denote by  the family of all demand cuts.
 Observe that  is a feasible solution for the current problem defined
 from 
 if and only if  for each demand cut .
An LP relaxation of the problem can be formulated as follows:

The dual of this LP is 

We say that
a node  
\emph{covers} a demand cut 
if ,
and a node set 
covers  if there exists a node  that covers .



A subfamily  of  is called \emph{uncrossable} when 
any 
satisfy  or
.

is called \emph{-intersecting}
 if
any 
with  satisfy .
We will show that the augmentation problem admits  a
constant-approximation algorithm when  is -intersecting uncrossable.

In general, the family  is not -intersecting uncrossable.
If  is not -intersecting uncrossable, 
we use a decomposition result given by Nutov~\cite{Nutov12}.
Namely, our algorithm finds a -intersecting uncrossable subfamily

of ,
and it uses the algorithm for a -intersecting uncrossable family to find a node set that covers all demand cuts in .
After adding to the solution all the nodes in the obtained node set,
the algorithm updates , setting it equal to the residual family, which consists of all the demand
cuts that are not
covered by the current solution.
This is repeated until  becomes -intersecting uncrossable.
Nutov proved that the algorithm for a -intersecting uncrossable family is invoked
 times.
Below,
we present an algorithm for
covering a -intersecting uncrossable family in Section~\ref{sssec.covering}, and
combines the Nutov's decomposition result and the covering algorithm in Section~\ref{ssec.combine}.


 \subsubsection{Covering a -intersecting uncrossable family of demand cuts}
\label{sssec.covering}
 
 Here, we explain how to cover a -intersecting uncrossable family  of demand cuts.
 First, we introduce several properties of a -intersecting uncrossable family.
 Let a \emph{min-core} signify a minimal demand cut in .


 
 \begin{lemma}
  \label{lem.disjoint}
 Let  be an uncrossable family of subsets of . Let .
 If  is a min-core of , then 
 either  or  holds.
 In particular, if both  and  are min-cores of ,
  the latter condition holds.
 \end{lemma}
  \begin{proof}
   Note that
    holds if and only if .
 Suppose that  and
 .
 Since  is uncrossable,  or
  holds.
  implies ,
  and  implies .
  Hence, if either 
  or  holds, we have a contradiction with the minimality of .
  \end{proof}

For ,
let  denote .

 \begin{lemma}
  \label{lem.residual}
 If  is -intersecting uncrossable, then
   is also -intersecting uncrossable
 for any .
 \end{lemma}
\begin{proof}
 Let . Since  is uncrossable,  or  holds.

 Suppose that the former holds.
 If , then  includes a node
 . Since ,
  covers  or . However, this contradicts the fact that . Hence, .
 Similarly, we can prove , because
 .

 Next, suppose that the latter holds.
 We note that
  and
 .
 Hence, as above, we can prove that
 .
\end{proof}

Now we present  our algorithm for covering a -intersecting uncrossable family 
  of demand cuts.
  The algorithm consists of the increase phase and the reverse deletion
 phase.
 First, we present the increase phase.
In this phase, the algorithm maintains the following variables.

  \begin{itemize}
   \item  The dual solution .
	  This is initialized as , .
	  During the increase phase,  is always a
	  feasible solution to \eqref{eq.dual-lp}.
   \item  A solution .
	  This is initialized to :=. The increase phase terminates when  covers all the demand cuts in .
  \end{itemize}

   In each iteration, the algorithm simultaneously increases  for each min-core
    of .
   For ease of presentation, we will consider this over time.
   During  units of time,  is increased by
    for each min-core  of .
   When  becomes equal to  for some ,
 the algorithm stops increasing 
 and adds  to .
After this update, if  covers all demand cuts in , then
the algorithm terminates the increase phase and proceeds to the
 reverse deletion phase.
 Otherwise, the algorithm proceeds to the next iteration of the increase
   phase.
 By definition, in these steps,  is always a feasible solution to
 \eqref{eq.dual-lp}.
 At the end of the increase phase,  covers all the demand cuts in .
 
 Suppose that the increase phase
 starts at time  and ends at time .
 Let  be a moment during the increase phase.
 Let  denote , ,
 and the family of min-cores of , respectively,
 at time .
 For each , let  denote .
 Let .
 For the analysis given below, we observe the following properties.


 
  \begin{lemma}
   \label{lem.stronglaminar}
   is strongly laminar (i.e., any  satisfy , , or  holds).
  \end{lemma}
\begin{proof}
 Let . Suppose that  and 
 for some .
 Then, , because  means that  is
 not covered by .
 Hence, by Lemma~\ref{lem.disjoint},  or  holds.
\end{proof}
 
  \begin{lemma}
   \label{lem.degree-udg}
   If  is a unit disk graph, 
    holds for any  and .
  \end{lemma}
 \begin{proof}
  Let
   be the
  members of  whose neighbor sets include
  .
  Let  denote
  a neighbor of  in  for each .
  Notice that by Lemma~\ref{lem.residual},  is uncrossable.
  Thus, if , then by Lemma~\ref{lem.disjoint}, .
  If ,
   and  are adjacent 
  for some  with .
  However,
  this indicates that , which contradicts .
  \end{proof}

  In the reverse deletion phase, the algorithm 
  modifies  into an inclusionwise minimal node set that covers  as follows.
  Let , where  is the -th node added to  
  in the increase phase
  for each .
  The reverse deletion phase investigates nodes  in decreasing order
  of their subscripts. If
   covers all demand cuts in ,
   is removed from .
  Let  denote  after all nodes have been investigated.
  The algorithm outputs  as a solution.
  We will show that  is a 15-approximate solution.
  
  \begin{lemma}\label{lem.witness}
   Let ,
   and let  with .
   There exist
    and  such that ,
    , and
    or .
  \end{lemma}
   \begin{proof}
    When  is investigated in the reverse deletion phase,
    the solution set  is .
    Hence, 
     implies that
    there exists 
    such that 
    .
      Note that  holds because
    no node in  covers .
    Since ,
     there exists  with .
    By Lemma~\ref{lem.disjoint},  or  holds.
   \end{proof}

  For , we will call  in Lemma~\ref{lem.witness} a
  \emph{witness pair} of .

 \begin{figure}[t]
  \centering
 \includegraphics[bb=13 12 95 69]{counting2.pdf}
 \caption{When  for first-type nodes  and }
 \label{fig:counting2}
 \end{figure}

 \begin{figure}[t]
  \centering
\includegraphics[bb=9 10 161 82]{counting.pdf}
 \caption{Relationship of , , , and  for a second-type node }
 \label{fig:counting}
 \end{figure}
  
\begin{lemma}\label{lem.degree}
  for
 any .
\end{lemma}
  \begin{proof}
   Let  be a node in  such that 
   , from which
    follows
   because  covers no demand cut in  by
   their definitions.
   We categorize such a node  into two types.
   If there exists a witness pair
    of 
   such that ,  is said to be of the first type;
   otherwise,  is said to be of the second type.

   Let us count the number of nodes  of the first type.
   Let . Suppose that there are two nodes of the first type, ,
   such that .
   Then  holds.
   Recall that , 
   ,
   and 
   (see Figure~\ref{fig:counting2}).
   These imply that .
   Since 
   and  is -intersecting, we have
   . However, this contradicts the
   definition of .
   Therefore, for each ,
   there exists at most one node of the first type with . That is to say,
  there are at most  nodes of the first type.

  Next, we count the number of nodes of the second type.
   Let  be a node of the second type.
   There exists 
   such that  , because  is not covered by
   any node that was added to  earlier than  in the increase phase.
   Since  is not the node of the first type,
    is not included in , and thus .
   Let .
   Since
   , and by Lemma~\ref{lem.stronglaminar},
    is a strongly laminar family,
    is a strongly laminar family.
   We assume that  has a unique maximal member;
   if
   there is more than one maximal member of 
   , we add the node set  to , which has
   no effect on the following discussion.
   Let  be a member of  such that  became a
   min-core of the residual family of  when  was added to 
   in the increase phase.
   Then , and hence  and ,
   because  and .
   Let  be a minimal member of  such that  and ;
   Figure~\ref{fig:counting} illustrates these definitions.
   Note that  has at least two children in .
   Suppose that there exists a node of the second type, ,
   such that  and   
   .
   Then,
    holds, because ,
   and  because of the strong laminarity of
   .
   Since this contradicts the
   definition of ,
   no such  exists.
   In summary, this means that
   the number of nodes of the second type
   is at most ,
   where  denotes the number of children of  in
   .
   Since the leaf set of  is ,
   we have .

    Thus, ,
   and  for each , and the lemma has been proven.
  \end{proof} 

  \begin{theorem}
   \label{thm.coveringuncross}
  If  is a -intersecting uncrossable family of demand cuts on a unit disk graph,
  there exists a -approximation algorithm for finding a minimum
  weight node set that covers all demand cuts in .
  \end{theorem}
  \begin{proof}
  We will prove that the algorithm presented above is a -approximation
   algorithm.
   By definition, the algorithm computes a feasible solution
    to the problem, and a solution 
  is a feasible solution to \eqref{eq.dual-lp}.
   Since  is a lower bound on the optimal value, it suffices to prove that .
   
   When the increase phase terminates,
    satisfies
 .
   Moreover,
   for each node  and ,
    holds.
   For each ,
    holds,
   because  holds
 when the algorithm terminates.
 By Lemma~\ref{lem.degree},
 .
  \end{proof}

    Although we illustrated the algorithm by using a continuous measure of time,
  it can be easily discretized;
  Algorithm~\ref{alg.covering} 
  describes the details of our algorithm
  for covering a -intersecting uncrossable family of demand cuts.

     \begin{algorithm}
  \caption{Covering algorithm for a -intersecting uncrossable family of demand cuts}
  \label{alg.covering}
  \begin{algorithmic}
   \REQUIRE
   a unit disk graph , ,
   a nonnegative weight  of each node ,
   and a -intersecting uncrossable family  
   \ENSURE  that covers all demand cuts in 
   \STATE  , 
   \STATE  for each 
   \WHILE{}
   \STATE 
   \STATE 
   \STATE  for each 
   \STATE
   
   \STATE  
   \STATE 
   \STATE 
   for each 
   \ENDWHILE
   \FOR{}
   \STATE if  covers , then 
   \ENDFOR
   \STATE output 
  \end{algorithmic}
   \end{algorithm}



   

Let us discuss the running time of Algorithm~\ref{alg.covering}.
Lemma~\ref{lem.disjoint} indicates that the number of
min-cores in  is  for any .
Hence the algorithm runs in polynomial time if all min-cores of
 can be computed in polynomial time.
If  arises from the demand cuts of the augmentation problem in
the decomposition of Nutov~\cite{Nutov12},
a min-core of 
 corresponds to a minimal node cut in the graph .
Thus all min-cores can be computed by applying a max-flow algorithm
repeatedly in this case.
  
  \subsubsection{Combined decomposition and covering algorithm}
  \label{ssec.combine}
  
  We now summarize our algorithm for the augmentation problem.
  As mentioned earlier, 
  due to Nutov~\cite{Nutov12},
  we can find a node set that covers all demand cuts
  by applying an
  algorithm for covering a -uncrossable family of demand cuts 
  times.
Since Theorem~\ref{thm.coveringuncross} gives a constant-approximation algorithm
  for covering a -intersecting uncrossable family,
  we have an -approximation algorithm for covering all demand
  cuts.
  Recall that a demand cut is defined as a Steiner -cut for a
  fixed .
  For covering all Steiner -cuts  with ,
  we choose arbitrary  nodes ,
  and apply the algorithm fixing  to each of .
  The union of the obtained solutions is
  a feasible solution for the augmentation problem
  because, for any Steiner -cut ,
  there exists a root  such that  or  is a
  Steiner -cut.
  This presents 
  an -approximate
  solution for the augmentation problem.
  Therefore, we arrive at the following conclusion.

   \begin{corollary}
    \label{cor.augmentation}
    The augmentation problem admits 
    an -approximation algorithm.
    It outputs a solution  such that 
   is at most  times the optimal value of \eqref{eq.primal-lp}.
   \end{corollary}
  

  
 For the weighted -CDS problem,
 we first compute an -dominating set , using the algorithm given in
 Section~\ref{subs.dominating}.
 We incrementally increase the connectivity of  
 by solving the augmentation problem.
 This obviously gives an -approximation algorithm.
 This approximation factor can be slightly improved, as follows.

 \begin{corollary}\label{cor.primal-dual}
  There exists an -approximation algorithm for the
  weighted -CDS problem.
 \end{corollary}
 \begin{proof}
  Let  denote an optimal solution for the weighted -CDS
  problem, and let  denote its characteristic vector
  (i.e.,  if , and  if ).
  For each , we abbreviate  to .

 We will show that the algorithm described above achieves the approximation
  factor .
  Recall that we use the algorithm given in
  Corollary~\ref{cor.augmentation} for the augmentation
  problem, but with the connectivity requirement  changed
  from  to .
  Let  denote the solution output by the algorithm for the
  augmentation problem when the connectivity requirement is .
  Note that in this case, the node set  in the input is , where  is the
  -dominating set computed by the algorithm given in Section~\ref{subs.dominating}.
 Note that  is -connected.
  Hence, when , for each demand cut ,
   includes at least  nodes in .
  This implies that  is a feasible solution to \eqref{eq.primal-lp}
  when the connectivity requirement is ,
  and thus .
  The weight of the -CDS output by our algorithm
is at most

\end{proof}

 


 \section{Conclusion}
 \label{sec.conclusion}

 We presented two constant-approximation algorithms for the
 unweighted -CDS problem
 and for the weighted -CDS problem in unit
 disk graphs.
 The first of these is a simple algorithm that can be applied to a fairly general class of graphs,
 although it is restricted to the unweighted
 -CDS problem.
 The second is a primal-dual algorithm that has a better approximation factor and
 can be applied to the weighted -CDS problem.
 Both algorithms need an assumption that .
 
 
 
 Besides the -dominating sets,
  there are many other variations of dominating sets in graphs. For example,
  a subset  of a node set  is called an \emph{-tuple dominating set}
  if each node in the graph (including those in )
  has  neighbors in ,
  and it is called a \emph{vector dominating set}
  if each node  outside of  has  nodes in 
  for a given -dimensional vector .
  Refer to~\cite{CicaleseMV13} for other variations.
  Our algorithms for the augmentation problem can be used for
  increasing the connectivity of 
  these  variations if each node outside the solution has 
  neighbors
  in the solution, where  is the required connectivity.
  
 
 Our primal-dual algorithm
 for the weighted problem
 requires the ellipsoid method
 for computing an -dominating set when  (when ,
 using the ellipsoid method can be avoided; see the last paragraph of Section~\ref{subs.dominating}).
 However, the ellipsoid method is not practical, so 
another interesting future work will be to invent a
 constant-approximation algorithm for the minimum weight -dominating
 set problem, that does not rely on the ellipsoid method.

\section*{Acknowledgements}
The author thanks anonymous referees for their careful reading and many useful comments.
This work was supported by JST ERATO Grant Number JPMJER1201, and
JSPS KAKENHI Grant Number 17K00040.


\begin{thebibliography}{10}

\bibitem{Alzoubi02}
Khaled~M. Alzoubi, Peng-Jun Wan, and Ophir Frieder.
\newblock Message-optimal connected dominating sets in mobile ad hoc networks.
\newblock In {\em Proceedings of the 3rd ACM International Symposium on Mobile
  Ad Hoc Networking \& Computing}, MobiHoc '02, pages 157--164, 2002.

\bibitem{AmbuhlEMN06}
Christoph Amb{\"{u}}hl, Thomas Erlebach, Mat{\'{u}}s Mihal{\'{a}}k, and Marc
  Nunkesser.
\newblock Constant-factor approximation for minimum-weight (connected)
  dominating sets in unit disk graphs.
\newblock In {\em Approximation, Randomization, and Combinatorial Optimization.
  Algorithms and Techniques, 9th International Workshop on Approximation
  Algorithms for Combinatorial Optimization Problems, {APPROX} 2006 and 10th
  International Workshop on Randomization and Computation, {RANDOM} 2006.
  Proceedings}, pages 3--14, 2006.

\bibitem{Bansal:2012ik}
Nikhil Bansal and Kirk Pruhs.
\newblock Weighted geometric set multi-cover via quasi-uniform sampling.
\newblock {\em Journal of Computational Geometry}, 7(1):221--236, 2016.

\bibitem{ByrkaGRS13}
Jaroslaw Byrka, Fabrizio Grandoni, Thomas Rothvo{\ss}, and Laura Sanit{\`{a}}.
\newblock Steiner tree approximation via iterative randomized rounding.
\newblock {\em J. {ACM}}, 60(1):6:1--6:33, 2013.

\bibitem{ChekuriEV12}
Chandra Chekuri, Alina Ene, and Ali Vakilian.
\newblock Prize-collecting survivable network design in node-weighted graphs.
\newblock In {\em Approximation, Randomization, and Combinatorial Optimization.
  Algorithms and Techniques - 15th International Workshop, {APPROX} 2012, and
  16th International Workshop, {RANDOM} 2012. Proceedings}, pages 98--109,
  2012.

\bibitem{ChengHLWD03}
Xiuzhen Cheng, Xiao Huang, Deying Li, Weili Wu, and Ding{-}Zhu Du.
\newblock A polynomial-time approximation scheme for the minimum-connected
  dominating set in ad hoc wireless networks.
\newblock {\em Networks}, 42(4):202--208, 2003.

\bibitem{CicaleseCGMV14}
Ferdinando Cicalese, Gennaro Cordasco, Luisa Gargano, Martin Milanic, and Ugo
  Vaccaro.
\newblock Latency-bounded target set selection in social networks.
\newblock {\em Theoretical Computer Science}, 535:1--15, 2014.

\bibitem{CicaleseMV13}
Ferdinando Cicalese, Martin Milanic, and Ugo Vaccaro.
\newblock On the approximability and exact algorithms for vector domination and
  related problems in graphs.
\newblock {\em Discrete Applied Mathematics}, 161(6):750--767, 2013.

\bibitem{ClarkCJ90}
Brent~N. Clark, Charles~J. Colbourn, and David~S. Johnson.
\newblock Unit disk graphs.
\newblock {\em Discrete Mathematics}, 86(1-3):165--177, 1990.

\bibitem{Dai:2006}
Fei Dai and Jie Wu.
\newblock On constructing k-connected k-dominating set in wireless ad hoc and
  sensor networks.
\newblock {\em Journal of Parallel and Distributed Computing}, 66(7):947--958,
  July 2006.

\bibitem{Fukunaga15arxiv}
Takuro Fukunaga.
\newblock Constant-approximation algorithms for highly connected
  multi-dominating sets in unit disk graphs.
\newblock {\em CoRR}, abs/1511.09156, 2015.

\bibitem{Fukunaga15}
Takuro Fukunaga.
\newblock Spider covers for prize-collecting network activation problem.
\newblock In {\em Proceedings of the Twenty-Sixth Annual {ACM-SIAM} Symposium
  on Discrete Algorithms, {SODA} 2015}, pages 9--24, 2015.

\bibitem{FunkeKMS06}
Stefan Funke, Alexander Kesselman, Ulrich Meyer, and Michael Segal.
\newblock A simple improved distributed algorithm for minimum {CDS} in unit
  disk graphs.
\newblock {\em {ACM} Transactions on Sensor Networks}, 2(3):444--453, 2006.

\bibitem{goemans1997primal}
Michel~X Goemans and David~P Williamson.
\newblock The primal-dual method for approximation algorithms and its
  application to network design problems.
\newblock {\em Approximation algorithms for {NP}-hard problems}, pages
  144--191, 1997.

\bibitem{GuhaK98}
Sudipto Guha and Samir Khuller.
\newblock Approximation algorithms for connected dominating sets.
\newblock {\em Algorithmica}, 20(4):374--387, 1998.

\bibitem{GuhaK99}
Sudipto Guha and Samir Khuller.
\newblock Improved methods for approximating node weighted steiner trees and
  connected dominating sets.
\newblock {\em Information and Computation}, 150(1):57--74, 1999.

\bibitem{KimWLZW10}
Donghyun Kim, Wei Wang, Xianyue Li, Zhao Zhang, and Weili Wu.
\newblock A new constant factor approximation for computing 3-connected
  m-dominating sets in homogeneous wireless networks.
\newblock In {\em Proceedings of the 2010 {IEEE} Conference on Computer
  Communications, {INFOCOM} 2010}, pages 2739--2747, 2010.

\bibitem{KleinR95}
Philip~N. Klein and R.~Ravi.
\newblock A nearly best-possible approximation algorithm for node-weighted
  {S}teiner trees.
\newblock {\em Journal of Algorithms}, 19(1):104--115, 1995.

\bibitem{MaratheBHRR95}
Madhav~V. Marathe, H.~Breu, Harry B.~Hunt {III}, S.~S. Ravi, and Daniel~J.
  Rosenkrantz.
\newblock Simple heuristics for unit disk graphs.
\newblock {\em Networks}, 25(2):59--68, 1995.

\bibitem{Nutov12}
Zeev Nutov.
\newblock Approximating minimum-cost connectivity problems via uncrossable
  bifamilies.
\newblock {\em {ACM} Transactions on Algorithms}, 9(1):1, 2012.

\bibitem{Shang:2007jg}
Weiping Shang, Frances Yao, Pengjun Wan, and Xiaodong Hu.
\newblock On minimum m-connected k-dominating set problem in unit disc graphs.
\newblock {\em Journal of Combinatorial Optimization}, 16(2):99--106, December
  2008.

\bibitem{ShiZZW16}
Yishuo Shi, Yaping Zhang, Zhao Zhang, and Weili Wu.
\newblock A greedy algorithm for the minimum 2-connected m-fold dominating set
  problem.
\newblock {\em J. Comb. Optim.}, 31(1):136--151, 2016.

\bibitem{ShiZD15}
Yishuo Shi, Zhao Zhang, and Ding{-}Zhu Du.
\newblock Approximation algorithm for minimum weight (k, m)-cds problem in unit
  disk graph.
\newblock {\em CoRR}, abs/1508.05515, 2015.

\bibitem{ShiZMD17}
Yishuo Shi, Zhao Zhang, Yuchang Mo, and Ding{-}Zhu Du.
\newblock Approximation algorithm for minimum weight fault-tolerant virtual
  backbone in unit disk graphs.
\newblock {\em {IEEE/ACM} Trans. Netw.}, 25(2):925--933, 2017.

\bibitem{WangTD09}
Feng Wang, My~T. Thai, and Ding{-}Zhu Du.
\newblock On the construction of 2-connected virtual backbone in wireless
  networks.
\newblock {\em {IEEE} Transactions on Wireless Communications},
  8(3):1230--1237, 2009.

\bibitem{WangKAGLZW13}
Wei Wang, Donghyun Kim, Min~Kyung An, Wei Gao, Xianyue Li, Zhao Zhang, and
  Weili Wu.
\newblock On construction of quality fault-tolerant virtual backbone in
  wireless networks.
\newblock {\em {IEEE/ACM} Transactions on Networking}, 21(5):1499--1510, 2013.

\bibitem{Wang:2015}
Wei Wang, Bei Liu, Donghyun Kim, Deying Li, Jingyi Wang, and Yaolin Jiang.
\newblock A better constant approximation for minimum 3-connected m-dominating
  set problem in unit disk graph using {T}utte decomposition.
\newblock In {\em Proceedings of the 2015 {IEEE} Conference on Computer
  Communications, {INFOCOM} 2015}, pages 1796--1804, 2015.

\bibitem{WillsonZWD15}
James Willson, Zhao Zhang, Weili Wu, and Ding{-}Zhu Du.
\newblock Fault-tolerant coverage with maximum lifetime in wireless sensor
  networks.
\newblock In {\em Proceedings of the 2015 {IEEE} Conference on Computer
  Communications, {INFOCOM} 2015}, pages 1364--1372, 2015.

\bibitem{YuJYLC2015}
Jiguo Yu, Lili Jia, Dongxiao Yu, Guangshun Li, and Xiuzhen Cheng.
\newblock Minimum connected dominating set construction in wireless networks
  under the beeping model.
\newblock In {\em Proceedings of the 2015 {IEEE} Conference on Computer
  Communications, {INFOCOM} 2015}, pages 972--980, 2015.

\bibitem{Zhang16}
Zhao Zhang, Jiao Zhou, Yuchang Mo, and Ding{-}Zhu Du.
\newblock Performance-guaranteed approximation algorithm for fault-tolerant
  connected dominating set in wireless networks.
\newblock In {\em 35th Annual {IEEE} International Conference on Computer
  Communications, {INFOCOM} 2016, San Francisco, CA, USA, April 10-14, 2016},
  pages 1--8, 2016.

\bibitem{ZouLGW09}
Feng Zou, Xianyue Li, Suogang Gao, and Weili Wu.
\newblock Node-weighted {S}teiner tree approximation in unit disk graphs.
\newblock {\em Journal of Combinatorial Optimization}, 18(4):342--349, 2009.

\bibitem{ZouWXLDWW11}
Feng Zou, Yuexuan Wang, XiaoHua Xu, Xianyue Li, Hongwei Du, Peng{-}Jun Wan, and
  Weili Wu.
\newblock New approximations for minimum-weighted dominating sets and
  minimum-weighted connected dominating sets on unit disk graphs.
\newblock {\em Theoretical Computer Science}, 412(3):198--208, 2011.

\end{thebibliography}
 
\appendix
\section{Improved analysis of Shang et al. for the unweighted -CDS}
\label{app.rect}

Shang~et~al.\ \cite{Shang:2007jg} gave an
approximation
algorithm for the unweighted -CDS problem.
They claimed that their algorithm achieves an approximation factor of
 for , and  for .
However, Shi~et~al.~\cite{ShiZZW16} pointed out that its analysis
contains a flaw. Shi~et~al.~\cite{ShiZZW16} also rectified the analysis, and showed
that
its approximation factor is  for 
and  for .
Simultaneously, 
Shi~et~al.~\cite{ShiZZW16} presented an algorithm for the unweighted -CDS
problem on general graphs, and proved that their algorithm achieves
approximation factors  for ,
and 11 for  if the graphs are restricted to unit disk graphs.
In this section, we present an improved analysis of  Shang~et~al.~\cite{Shang:2007jg}.
It gives approximation factors 
   for , and 
   for .
Although these are not better than those of 
Shi~et~al.~\cite{ShiZZW16}, we believe that it is worth noting.
  

Let us begin with illustrating the algorithm of Shang~et~al.~\cite{Shang:2007jg}.
Let  denote the minimum size of -CDSs.
Let  be a maximal independent set of  for each , where .
The algorithm first computes ,
and  such that  and
 is connected.
The following properties are proven.
 \begin{itemize}
  \item[(i)]  for each .
  \item[(ii)]  is an -dominating set for
	      each ,
	      and hence  is a -CDS.
  \item[(iii)] Each cut-node of  is included in  or in .
  \item[(iv)]  for ,
	      and  for 
	      (this is slightly better than the conclusion in
	      \cite{Shang:2007jg}, but they proved this).
 \end{itemize}

 After this step,
the algorithm computes a
node set  such that  is 2-connected as follows.
First,  is initialized to be an empty set.
Then, the algorithm computes
a -cut  such that 
and no -path in 
covers .
For this -cut ,
the algorithm finds a -path that covers  with at most two inner nodes,
and it adds these inner nodes to the solution .
This procedure is repeated until  becomes 2-connected,
and the algorithm outputs 
as a -CDS.

Shang~et~al.\ \cite{Shang:2007jg} claimed that the number of iterations is at most
the number of cut-nodes in , which is 
at most , due to
properties (i) and (iii).
Since each iteration adds at most two nodes, when the algorithm terminates, 
 .
 This is their analysis of the algorithm.

 However, the number of iterations is not bounded by the number of cut-nodes in .
 This can be observed by considering a star with  leaves.
 The star has only one cut-node. To make it 2-connected,
 we need to add  paths to connect the leaves.

 We claim that a correct upper-bound on the number of iterations
 is the number of blocks of ,
 and this number is bounded by .
 A block of a connected graph is a maximal set of nodes that induces a connected subgraph without cut-nodes.
 Each block consists of at least two nodes.
 If a block consists of more than nodes, then it is a 2-connected component of the graph.
 If a block consists of only two nodes, then it induces an edge.

 For proving our claim, 
  let us consider the tree  that represents the block
  decomposition of .
  Namely, the node set of  is the disjoint union of two node sets 
  and . Each node in  corresponds to a block of
  ,
  and each node in  corresponds to a cut-node of .
  In what follows, we identify a node in  with the corresponding
  block of , and we identify a node in  with the
  corresponding
  cut-node of .
  A node  and a node  are joined by an edge in 
  when the block  includes the cut-node .

  In each iteration of the algorithm, a -path is selected to
  join two different blocks of ,
  and the inner nodes of the path are
  added to the solution .
  Let  and  denote the end nodes of a -path , and 
  let  be a block that includes .
  If more than one block includes , we let 
  denote the one nearest to the blocks including  on ;
   is defined in the same way.
  Let  be a cut-node of ,
  and let  and  be blocks that include .
  When the algorithm chooses a -path ,
  we add a virtual edge 
  that joins
   and  
  if , , and  are on the path between
   and  on .
  
  \begin{lemma}
    is -connected if
   virtual edges induce a connected graph
   on the set of neighbors of
    each cut-node  in .
  \end{lemma}
  \begin{proof}
   Suppose that  is not 2-connected even if the condition holds. Then there 
   exists a cut-node  of .
   In other words, after removing  from , some neighbors  and  of 
   are included in the different connected component.
   Since  is -dominating and , we can assume that .
   Hence  is also a cut-node of .
      
   Let  and  be the blocks of  including  and , respectively. 
   Since  and  are neighbors of , both  and  include  (i.e., 
    and  are neighbors of  in ).
   By the condition of the lemma,
    and  are connected by a path of the virtual edges. 
   This implies that there exists a path on  that connects  and ,
   and that does not pass through .
   Hence,  is not a cut-node in , which is a contradiction.
  \end{proof}

  The following lemma presents a bound on the number of iterations in this algorithm.
 
 \begin{lemma}\label{lem.iterationfor2}
  The number of iterations is at most
  .
 \end{lemma}
   \begin{proof}
    Let  be a cut-node on ,
    and let  denote the 
    number of connected components 
    induced by the virtual edges
    on the neighbor set of .
    In each iteration of the algorithm,
     is decreased by at least one
    for some cut-node .
    When the first iteration begins,
     is equal to the degree of  in .
    Since all leaves in  are included in ,
     holds at the beginning of the first iteration.
    The iterations terminate when
    .
    Hence,
    the number of iterations is at most .
    We will determine  below.

    Let  be a spanning tree on .
    We show that each block contains a node in
    
    or an edge in .
    To see this, suppose that a block  of  contains 
    no edge in .
    If  contains more than one node in ,
    then it includes an edge in .
    Since, by the assumption, this does not happen,  includes
    at most one node in .
    Hence, there exists a node 
     in .
    If , we are done. Suppose that .
    By property (ii),
     has neighbors in  and in .
    By property (iii),  is not a cut-node in , so
    these neighbors must be inside .
    If the neighbor in  is included in ,
     contains two nodes in .
    Since this contradicts the assumption, 
    contains a node in .

    Nodes in  and edges in  are
    not contained in more than one block of .
   The number of edges in  is ,
    and  by property (i).
    Hence .
   \end{proof} 

  From Lemma~\ref{lem.iterationfor2} and property (iv), we obtain the following
  approximation guarantee for the algorithm.
  \begin{theorem}
   The algorithm of Shang~et~al.\ \cite{Shang:2007jg}
   attains an approximation factor
    for , and 
    for .
  \end{theorem}



  

\end{document}
