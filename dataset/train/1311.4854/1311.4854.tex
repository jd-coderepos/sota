\documentclass{cccg12}

\usepackage{graphicx,amssymb,amsmath,dsfont}

\usepackage[small,bf,hang]{caption}
\setlength{\captionmargin}{12pt}

\usepackage{url}

\usepackage{verbatim}


\title{Computing the Coverage of an Opaque Forest \thanks{This work was supported by NSERC}}
\author{Alexis Beingessner \and Michiel Smid \thanks{School of Computer Science, Carleton University}}

\begin{document}
\thispagestyle{empty}
\maketitle



\begin{abstract}
We consider the problem of taking an opaque forest and determining the regions that are covered by it. We provide a tight upper bound on the complexity of this problem, and an algorithm for computing this area, which is worst-case optimal.
\end{abstract}




\section{Introduction}
Let a \emph{region} be any bounded, closed, and connected set of points in . Then a \emph{barrier}, or \emph{opaque forest}, of a finite set  of regions, is any finite set  of closed and bounded line segments, such that for any line : if  intersects  then  also intersects . A previously studied problem is as follows: given some set  or regions, compute a barrier  such that the length of all the segments in , , is minimal. The exact solution to this problem is not known even for specific cases, such as when  is a unit square. The best known bounds for this instance of the problem are . \cite{Dumitrescu2010}

The general problem of computing a minimal barrier for a given set of regions is a very difficult one. Currently there are no proven algorithms for computing this precisely, nor even known solutions for specific cases. For the internally optimal barrier, there is also no known algorithm. However, by further restricting the problem, it is reducible to well studied problems. If the internally optimal barrier is restricted to a single connected component, then this is easily reducible to the \emph{Minimal Steiner Tree Problem}. If the barrier is further restricted to a single polygonal chain, then the problem is reducible to the \emph{Travelling Salesman Problem}. Both of these problems are known to be NP-Hard in general, but can be much more easily computed or approximated when the input points are in convex position, which is the case for this problem \cite{Dumitrescu2010}.

In this paper we consider the following problem: given some barrier , compute a maximal set  of regions such that  is a barrier for . More precisely, given a set  of  line segments, compute  every line through  intersects . We say that  is the \emph{coverage} of .

We give an algorithm that computes the coverage of an opaque forest in
 time. We also provide an example of an opaque forest whose
coverage has size . Thus, our algorithm is worst-case
optimal.

\section{Maximal Regions}

 Let a \emph{maximal region} of a set  of points be a region
    such that for every point  in , there exists an open
   ball  centered at  such that .

\begin{lemma}
\label{lemma:hasNoLines}
If a maximal region of  is a line segment, then that line segment is part of .
\end{lemma}

\begin{proof}
Assume this is not the case. Then there is some line segment  that is a maximal region, but is not in . Therefore all lines that pass through a point  in  intersect , and there exists an open ball  of points around  such that every point  in  that is not in  has a line  through it which does not intersect .

Consider such a point . The line  through  that does not intersect  cannot intersect , or else the points it intersects in  are not actually in . We can select a point  such that it is arbitrarily close to , and the line  must therefore become ever more parallel to the line  lays on to avoid intersection. Therefore it must be the case that the line collinear with  intersects , but the line  that is parallel to  and arbitrarily close to it does not. Therefore, there must exist some line segment  that is parallel to . Further, there must be some opaque forests to the left and right of  that do not meet each other or , or else  can pass through . Therefore, there is a space for parallel lines to the left and right of . However, this implies that there is a line  that enters through one space and exits through the other which does not intersect  but passes through , which means there are points in  which are not in . If this were not the case, then  would intersect . So we have a contradiction, therefore if  is in ,  is in .\end{proof}

\begin{figure}[ht]
  \centering
  \includegraphics[scale=0.4]{noSegments}
  \caption{There is a line  which does not intersect  but passes through }
  \label{fig:noSegments}
\end{figure}

\begin{lemma}
\label{lemma:hasPoints}
 may contain maximal regions that are single points, but are not part of .
\end{lemma}

\begin{proof}
Consider the construction of three line segments found in Figure 2. 

\begin{figure}[ht]
  \centering
  \includegraphics[scale=0.4]{pointGenerator}
  \caption{A construction that creates a maximal region that is exactly one point}
  \label{fig:pointGenerator}
\end{figure}

 is not part of . Every line that passes through  intersects , so . Yet there exists an open ball of points centred at  such that every point in this ball except for  has a line through it that does not intersect . Therefore  is a maximal region of .\end{proof}



\section{Clear and Blocked Points}

Let a \emph{blocked point} be a point  with respect to some barrier  such that for every line  which passes through ,  intersects . Then a \emph{clear point} is a point which is not blocked. Every point of  is a blocked point. Moreover,  is the set of all blocked points with respect to , and the complement  or  is the set of all clear points. 

\begin{theorem}
\label{thm:tangency-boundary}
For every barrier , each maximal region  is the intersection of halfplanes defined by lines that pass through two vertices of .
\end{theorem}

\begin{proof}

Assume there exists some line  which is tangent to the boundary of a maximal region , but  does not touch . Then, because the complement of  is an open set,  can be translated to intersect  without intersecting . However that would mean  contains clear points, which is a contradiction. Therefore,  must be tangent to  at at least one point. Now assume  is tangent to  at exactly one point. Then  can still be rotated around the point of tangency, once more intersecting . This once more contradicts the fact that  is a subset of . Therefore  must be tangent to at least two points of . Further, since  is a set of line segments, only the end points of these segments need be considered, as tangency to a line segment is simply tangency to its two end points.\end{proof}

\begin{figure}[ht]
  \centering
  \includegraphics[scale=0.4]{halfPlaneBounding}
  \caption{The line  must be tangent to  at two vertices, if it defines the boundary of part of }
  \label{fig:halfPlaneBounding}
\end{figure}

Remark that this also implies that we need only finitely many halfplanes to define a maximal region of , and that every maximal region of  is convex.



\section{Connected Components}

 is a set of  line segments consisting of  connected components . Further,  is the convex hull of the connected component . Then for some point , we define  as follows:

\begin{enumerate}
\item If  is a single line segment, and  is collinear to , then  
\item Otherwise, if  lies on a vertex of , then  is the double-wedge defined by the lines of the two edges of  that meet at . 
\item Otherwise, if  lies inside , or on its boundary, , then 
\item Otherwise,  is the double-wedge defined by the tangents of  that pass through . 
\end{enumerate}

\begin{figure}[ht]
  \centering
  \includegraphics[scale=0.60]{pointCoverage}
  \caption{Various possible cases for }
  \label{fig:pointCoverage}
\end{figure}

Intuitively,  can be thought of as the set of all lines that pass through  and intersect . However this is not strictly true. For parts (3) and (4) of the definition, this does in fact hold. However, (2) describes the limiting behaviour of a point as it tends towards a vertex of  from outside. (1) ignores the behaviour of points collinear to a single disjoint line segment. This definition may seem counter-intuitive, but it is useful for us. Further, we will consider  to be a subset of , and not the actual lines that pass through . 

\begin{lemma}
Every point in  is a clear point with respect to .
\end{lemma}

\begin{proof}
In case (1), where  is a single line segment and  is a point collinear with it, this follows trivially, as . Therefore, even though , the only points that aren't clear are those of  itself. 
In case 2, where  is a vertex of ,  is completely contained within . 
	Since ,  can't 
	contain a blocked point. 
In case (3), where  lies inside of or on  this also follows trivially, as  is empty. 
In case (4), where  lies outside of ,  is once again completely contained within  and therefore,  can't contain a blocked point. 
Therefore  is a set of clear points.
\end{proof}

We now define  to be the set of all lines which intersect  and pass through , ignoring previously established special cases.

\begin{figure}[ht]
  \centering
  \includegraphics[scale=0.6]{LofQ}
  \caption{ is the set of all lines that intersect  and pass through some point }
  \label{fig:LofQ}
\end{figure} 

Since  is a set of clear points with respect to , we can further conclude that  has this property with respect to the whole of . Further, for some points  and , since  and  have this property,  also has this property. By DeMorgan's law for set compliments, we can also conclude that  has this property as well. Therefore given  we know  is a set that also has this property. 

\begin{theorem}
\label{thm:coverageComputation}
Let  be the closure of the interior of a set of points, then . Further,  is a finite set of disjoint points.
\end{theorem}

\begin{proof}
Since  is the set of all clear points with respect to , and  is a set of some clear points with respect to , . Therefore, . 

From Lemmas \ref{lemma:hasNoLines} and \ref{lemma:hasPoints}, we know that the only zero area maximal regions of  that aren't in  are individual points. Remark that  differs from  in that only the zero area maximal regions of  have been removed. Therefore, if , , and all that  and  may differ by are disjoint points. Since , and  has zero area, , so all that remains to be proven is . Equivalently, 

Assume some postive-area region  of points is in 
	. Consider a point . There 
	is some line  through  that does not intersect . 
	Then  can be rotated around  without intersecting  
	until it is tangent with some connected component  at some 
	point . We will call this rotated line . Now if 
	, then there exists some 
	, , which  is in. This would mean 
	there is some line  through  and  such that 
	 intersects . 

\begin{figure}[ht]
  \centering
  \includegraphics[scale=0.4]{horribleMess}
  \caption{There exists some , , which  is in}
  \label{fig:horribleMess}
\end{figure}

However  is , and if  intersects  then there are three possibilities. Either  intersects , we should have stopped at  before we got to , or  is tangent to  as well. For the first two cases we have a contradiction, so  must be tangent to . However, since  is part of some region with positive area, we may take a point  adjacent to  such that it lies on no such tangent, and for which this case is therefore not possible. Therefore  or else there is a contradiction. Remark that this argument holds for any choice of  that does not lie on a tangent between two connected components. If the points on these tangents were not in  this would imply a region of zero area exists in , but this is impossible. Therefore all points around  must be in , and therefore  must be as well. Therefore, if , , and therefore .

Since ,  is a set of disjoint points. To prove that there are finitely many points, recall that by Theorem \ref{thm:tangency-boundary} each maximal region of  is an intersection of halfplanes defined by the vertices of . The only way to get a point from this process is where three or more halfplane boundaries intersect at a point. Since there are finitely many vertices and therefore finitely many halfplanes, it follows that there are finitely many points. 
\end{proof}

\begin{comment}It may seem that excluding the degenerate case of collinearity with a line segment served only to complicate things, but it was necessary to get a correct result. For instance, consider a barrier that is a series of collinear line segments. Using a more naive approach we would get the coverage to be the line all of these segments lie on. This is clearly not correct. Further, excluding these lines does not remove any information because the coverage and barrier are closed sets. It is impossible to have a gap in the coverage that can be plugged by the ``width" of exactly one line. Either there is no gap, or infinitely many lines can fit in the gap.
\end{comment}


\section{Computing the Coverage of a Barrier}

Theorem \ref{thm:coverageComputation} provides a procedure for computing .

We will assume our input is given as a list  of  connected components , totalling  line segments. The first step of our algorithm will be to compute the convex hulls of all  components. Next, for each vertex  of each , we will compute  for each , and union together these  into  by sorting them by angle. Then we will construct an arrangement using all the lines of the . We can then determine our final result by determining how many  one cell is part of, and then traversing the dual while changing our count according to whether a given edge exits or enters an . Then we simply output those regions which were in every , as well as  itself. However this process returns , so we may still be missing a finite number of points.

To compute these points, recall that they must lay at the intersection of 3 or more halfplanes. While this is necessary, it is not sufficient. The only way we know of to be certain a point is in  is to perform a radial plane sweep on  from that point. Since there are  lines in the arrangement, there are  candidate points. We will consider a line  that makes up the arrangement. There are  points of intersection on this line. First we will perform a radial plane sweep on one of these points  to construct a set  of points on the interval  to , where each point  represents the angle of a tangent to some  from , and each point is labelled with the number of connected components the line through  at the angle   intersects. If every  is labelled with a non-zero value, then  and we return it. Now consider the intersection point  on  that is adjacent to . While most of the exact values of  will change, the ordering and labelling of the points will only change for those related to the tangents that bound this segment of . We can store this data in the vertices of the arrangement during construction, so we can just query  and  for this information. By updating just these values and checking if any are now labelled with , we now know if  is in . Repeating this process for all the points on , and then for all choices of , we will have determined all the points in .

Computing the convex hulls will take  time. Computing  requires computing two lines. In all the special cases this takes constant time, however in the case where we must actually compute the lines as tangents, we take  time to binary search 's vertices for the most extreme points. Since there are  , it takes  time to compute them all for one . Further, to union them together into , we need to sort their lines by angle, which will take  time. Since there are  , we take  time to compute them all. Since we now have  lines from all our , our arrangement will take  time to compute, whose dual we can navigate in  time.\cite{Berg08}

For each line of the arrangement we take at most  time to perform the plane sweep of the first point. Then for each other point, there are an amortized   other lines intersecting at this point, and we do  work per intersection, so we do  work per line. Therefore this step takes  time. 

Therefore our algorithm runs in  time. Now we must determine whether this is good or not.

Since  is at most , our algorithm will run in  time in the worst case. Consider the following barrier: Take a regular -gon, and shrink all the edges by a small amount, so that there are gaps where the vertices were. Now there are small regions of space where lines can travel between  each pair of vertices. These regions are equivalent to the planar embedding of . This partitions the space into  convex regions \cite{Freeman76}. So to even \emph{write} the output it would take  time and space. Therefore, our algorithm is indeed worst-case optimal.


\begin{figure}[ht]
  \centering
  \includegraphics[scale=0.4]{worstCase}
  \caption{The worst known case barrier and its coverage}
  \label{fig:worstCase}
\end{figure}

\section{Deciding Whether a Point is Part of a Barrier's Coverage}

Given a barrier  one can fairly simply determine whether a point  is in  in  time and  space using a plane sweep. However if  is already constructed, point queries can be done in  time using a structure that takes  extra space and  time to construct \cite{Kirkpatrick83}, where  is the number of edges in .

\addcontentsline{toc}{chapter}{References}
\small 
\bibliographystyle{abbrv}
\bibliography{bibliography}

\end{document}
