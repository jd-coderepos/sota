\documentclass{llncs}
\usepackage{amsmath}\usepackage{amsfonts}\usepackage{amssymb}\usepackage{graphicx}
\usepackage[english]{babel}
\usepackage{cite}
\usepackage[all]{xy}\usepackage{verbatim}
\usepackage{stmaryrd}
\numberwithin{equation}{section}

\newcommand{\Sm}{\mathcal{S}}   \newcommand{\Lm}{\mathcal{L}}  \newcommand{\Fm}{\mathbf{Fm}}  \newcommand{\Hom}{\mathrm{Hom}}   \newcommand{\Al}{\mathbf{A}}  \newcommand{\BB}{\mathbf{B}}   \newcommand{\Alg}{\mathbb{A}\mathrm{lg}}
\newcommand{\LL}{\mathcal{L}}
\newcommand{\SSS}{\mathcal{S}}
\newcommand{\la}{\langle}
\newcommand{\ra}{\rangle}
\newcommand{\LS}{\bf LS}
\newcommand{\s}{\mathcal{S}}
\newcommand{\Ls}{\mathcal{L}_{\s}}
\newcommand{\CRS}{\vdash_{\mathcal{S}}}
\newcommand{\aAA}{\boldsymbol{A}}
\newcommand{\Fi}{\mathrm{Fi}}
\newcommand{\leqas}{\leq^{\aAA}_{\s}}
\newcommand{\csa}{C_{\s}^{\boldsymbol{A}}}
\newcommand{\Idsa}{\mathrm{Id}_{\s}\boldsymbol{A}}
\newcommand{\p}{\mathcal{P}}
\newcommand{\aAAlg}{\mathsf{Alg}}
\newcommand{\bteor}{\begin{theorem}}
\newcommand{\eteor}{\end{theorem}}
\newcommand{\Fisa}{\mathrm{Fi}_{\s}\boldsymbol{A}}
\newcommand{\marginnote}[1]{\marginpar{\tiny{#1}}}

\setlength{\parindent}{0pt}

\begin{document}
\mainmatter              		\pagestyle{headings}
\title{An Abstract Algebraic Logic View on Judgment Aggregation\thanks{The research of the second and third author has been made possible by the NWO Vidi grant 016.138.314, by the NWO Aspasia grant 015.008.054, and by a Delft Technology Fellowship awarded in 2013.}}
\author{Mar\'ia Esteban\inst{1}
\and
Alessandra Palmigiano\inst{2}
\and
 Zhiguang Zhao\inst{2}}
\institute{Departament de L\`ogica, Hist\`oria i Filosofia de la Ci\`encia, Facultat de Filosofia, Universitat de Barcelona\\
\email{mariaesteban.edu@gmail.com}
\and Faculty of Technology, Policy and Management, Delft University of Technology\\
\email{\{a.palmigiano, z.zhao-3\}@tudelft.nl}}

\maketitle
\thispagestyle{empty}

\begin{abstract}
In the present paper, we propose  Abstract Algebraic Logic (AAL) as a general logical framework for Judgment Aggregation. Our main contribution is a generalization of Herzberg's algebraic approach to characterization results in  on judgment aggregation and propositional-attitude aggregation, characterizing certain Arrovian classes of aggregators as Boolean algebra and MV-algebra homomorphisms, respectively.
 The characterization result of the present paper applies to  agendas of formulas of an arbitrary {\em selfextensional} logic.
This notion comes from AAL, and encompasses a vast class of logics, of which classical, intuitionistic, modal, many-valued and relevance logics are special cases. To each selfextensional logic , a unique class of algebras  is canonically associated by the general theory of AAL. We show that for any selfextensional logic  such that  is closed under direct products, any algebra in  can be taken as the set of truth values on which an aggregation problem can be formulated.
In this way, judgment aggregation on agendas formalized in  classical, intuitionistic, modal, many-valued and relevance logic can be uniformly captured as special cases.
This paves the way to the systematic study of a wide array of ``realistic agendas'' made up of complex formulas, the propositional connectives of which are interpreted in ways which depart from their classical interpretation.
This is particularly interesting given that, as observed by Dietrich \cite{Di10},  nonclassical (subjunctive) interpretation of logical connectives can provide a strategy for escaping impossibility results.\\
{\em Keywords:} Judgment aggregation; Systematicity; Impossibility theorems; Abstract Algebraic Logic; Logical filter; Algebra homomorphism.\\
{\em Math. Subject Class.} 91B14; 03G27.
\end{abstract}
\section{Introduction}

\paragraph{\bf\em Social choice and judgment aggregation.} The theory of {\em social choice} is the formal study of mechanisms for collective decision making, and investigates issues of philosophical, economic, and political significance, stemming from the classical Arrovian problem of how the preferences of the members of a group can be ``fairly'' aggregated into one outcome.

In the last decades, many results appeared generalizing the original Arrovian problem, which gave rise to a research area called {\em judgment aggregation} (JA) \cite{LP09}. While the original work of Arrow \cite{Ar63} focuses on preference aggregation, this can be recognized as a special instance of the aggregation of consistent judgments, expressed by each member of a group of individuals over a given set of logically interconnected propositions (the {\em agenda}): each proposition in the agenda is either accepted or rejected by each group member, so as to satisfy certain requirements of logical consistency. Within the JA framework, the Arrovian-type {\em impossibility results} (axiomatically providing  sufficient conditions for aggregator functions to turn into degenerate rules, such as dictatorship) are obtained as consequences of {\em characterization theorems} \cite{NP02}, which provide necessary and sufficient conditions for agendas to have aggregator functions on them satisfying given axiomatic conditions.

In the same logical vein, in \cite{DiLi10}, {\em attitude aggregation theory} was introduced; this direction has been further pursued in \cite{He13}, where a characterization theorem has been given for certain many-valued propositional-attitude aggregators as MV-algebra homomorphisms. 

\paragraph{\bf\em The ultrafilter argument and its generalizations.} Methodologically, the {\em ultrafilter argument} is the tool underlying the generalizations and unifications mentioned above. It can be sketched as follows: to prove impossibility theorems for finite electorates, one shows that the axiomatic conditions on the aggregation function force the set of all decisive coalitions to be an (ultra)filter on the powerset of the electorate. If the electorate is finite, this implies that all the decisive coalitions must contain one and the same (singleton) coalition: the oligarchs (the dictator). First employed in \cite{KiSo72} for a proof of Arrow's theorem alternative to the original one, this argument was applied to obtain elegant and concise proofs of impossibility theorems also in judgment aggregation  \cite{DM10}. More recently, it gave rise to characterization theorems, e.g.\ establishing a bijective correspondence between Arrovian aggregation rules and ultrafilters on the set of individuals \cite{HE12}. Moreover,  the ultrafilter argument has been generalized by Herzberg \cite{He08} to obtain a bijective correspondence between certain judgment aggregation functions and ultraproducts of profiles, and---using the well-known correspondence between ultrafilters and Boolean homomorphisms---similar correspondences have been established between Arrovian judgment aggregators and Boolean algebra homomorphisms \cite{He10}.

\paragraph{\bf\em  Escaping impossibility via nonclassical logics.} While much research in this area explored the limits of the applicability of  Arrow-type results, at the same time the question of how to `escape impossibility' started attracting increasing interest. In \cite{Di10}, Dietrich shows that impossibility results do not apply to a wide class of realistic agendas
once propositions of the form `if  then ' are  modelled as {\em subjunctive} implications rather than material implications. Besides its theoretical value, this result is of practical interest, given that subjunctive implication models the meaning of if-then statements in natural language more accurately than  material implication.

\paragraph{\bf\em Aim.} A natural question arising in the light of Dietrich's result is how to highlight the role played by the logic (understood both as formal language and deductive machinery) underlying the given agenda in characterization theorems for JA.

The present paper focuses on {\em Abstract Algebraic Logic}  as a natural theoretical setting for Herzberg's results \cite{He08,He13}, and the theory of {\em (fully) selfextensional logics} as the  appropriate logical framework for a nonclassical interpretation of logical connectives, in line with the approach of \cite{Di10}.

\paragraph{\bf\em Abstract Algebraic Logic and selfextensional logics.} Abstract Algebraic Logic (AAL) \cite{FoJaPi03} is a forty-year old research field in mathematical logic. It was conceived as the framework for an algebraic approach to the investigation of classes of logics. Its main goal was establishing a notion of {\em canonical algebraic semantics} uniformly holding for classes of logics, and using it to systematically investigate (metalogical) properties of logics in connection with properties of their algebraic counterparts.

{\em Selfextensionality} is the metalogical property holding of those logical systems whose associated relation of logical equivalence on formulas is a congruence of  the term algebra.  W\'ojcicki \cite{Wo79} characterized  selfextensional logics as the logics which admit a so-called {\em referential semantics} (which is a general version of the well known possible-world semantics of modal and intuitionistic logics), and in \cite{JaPa06}, a characterization was given of the particularly well behaved subclass of the fully selfextensional logics in general duality-theoretic terms. This subclass includes many well-known logics, such as classical, intuitionistic, modal, many-valued and relevance logic. These and other results in this line of research (cf.\ e.g.\ \cite{Ja05,GJP10,Es13,EsJa14}) establish a systematic connection between possible world semantics and the logical account of intensionality.

\paragraph{\bf\em Contributions.}
In the present paper, we generalize and refine Herzberg's characterization result in \cite{He13} from the MV-algebra setting to any class of algebras canonically associated with some  selfextensional logic.

In  particular, the properties of agendas are formulated independently of a specific logical signature and are slightly different than those of Herzberg's setting.  In contrast with Herzberg's characterization result, which consisted of two slightly asymmetric parts, the two propositions which yield the characterization result in the present paper (cf.\ Propositions \ref{prop1} and \ref{prop2}) are symmetric. Aggregation of propositional attitudes modeled in classical, intuitionistic, modal, \L ukasiewicz and relevance logic can be uniformly captured as special cases of the present result.
 This makes it possible to fine-tune the expressive and deductive power of the formal language of the agenda, so as to capture e.g.\ intensional or vague statements.



\paragraph{\bf\em Structure of the paper} In Section \ref{sec:preliminaries} we give preliminaries on Abstract Algebraic Logic. In Section \ref{Sec:framework} we provide the formal framework for judgment aggregation. In Section \ref{Sec:Results} we prove the correspondence result. In Section \ref{Sec:Arrow} we state the impossibility theorem for judgment aggregation as a corollary, and discuss the setting of subjunctive implication.

\section{Preliminaries on Abstract Algebraic Logic}
\label{sec:preliminaries}
The present section collects the basic concepts of Abstract Algebraic Logic that we will use in the paper.
For a general view of AAL the reader is addressed to \cite{FoJa96} and the references therein.

\subsection{General approach.} As mentioned in the introduction, in AAL,  logics are not studied in isolation, and in particular,  investigation focuses on classes of logics and their identifying {\em metalogical properties}.  Moreover, the notion of {\em consequence} rather than the notion of {\em theoremhood} is taken as basic: consequently, {\em sentential logics}, the primitive objects studied in AAL, are defined as tuples  where  is the algebra of formulas of type  over a denumerable set of variables , and  is a {\em consequence relation } on (the carrier of)  (cf.\ Subsection \ref{subsec:logics}).

\noindent This notion encompasses logics that are defined by any sort of proof-theoretic calculus (Gentzen-style, Hilbert-style, tableaux, etc.), as well as logics arising from some classes of (set-theoretic, order-theoretic, topological, algebraic, etc.) semantic structures, and in fact it allows to treat logics independently of the way in which they have been originally introduced. Another perhaps more common approach in logic takes the notion of theoremhood as basic and consequently sees logics as sets of formulas (possibly closed under some rules of inference). This approach is easily recaptured by the notion of sentential logic adopted in AAL: Every sentential logic  is uniquely associated with the set   of its {\em theorems}.

\subsection{Consequence operations}
For any set , a \textit{consequence operation} (or closure operator) on   is a map  such that for every : (1) , (2) if , then  and (3) . The closure operator  is \textit{finitary} if in addition satisfies  (4) . For any consequence operation  on , a set  is -\textit{closed} if . Let  be the collection of -closed subsets of .

For any set , a \textit{closure system} on  is a collection  such that , and  is closed under intersections of arbitrary non-empty families. A closure system is \textit{algebraic} if it is closed under unions of up-directed\footnote{
For  a poset,  is  \emph{up-directed} when for any  there exists  such that . } families.

For any closure operator  on , the collection  of the -closed subsets of  is a closure system on . If  is finitary, then  is algebraic. Any closure system  on  defines a consequence operation  on  by setting

for every . The -closed sets are exactly the elements of . Moreover,  is algebraic if and only if  is finitary.

\subsection{Logics}\label{subsec:logics}
Let  be a propositional language (i.e.\ a set of connectives, which we will also regard as a set  of function symbols) and let  denote the algebra of formulas (or term algebra) of  over a denumerable  set  of variables. Let  be the carrier of the algebra . A \textit{logic} (or deductive system) of type  is a pair  such that  such that the operator  defined by  is a consequence operation with the property of  \textit{invariance under substitutions}; this means that for every substitution  (i.e.\ for every -homomorphism ) and for every ,



For every , the relation  is the {\em consequence} or {\em entailment} relation of . A logic is \textit{finitary} if the consequence operation  is finitary. Sometimes we will use the symbol  to refer to the propositional language of a logic .

The  \textit{interderivability relation} of a logic   is the relation  defined by   satisfies the \textit{congruence property} if  is a congruence of .

\subsection{Logical filters}
Let  be a logic of type  and let  be an -algebra (from now on, we will drop reference to the type , and when we refer to an algebra or class of algebras in relation with , we will always assume that the algebra and the algebras in the class are of type ).

A subset  is  an -\textit{filter} of  if for every  and every ,  The collection  of the -filters of  is a closure system. Moreover,  is an algebraic closure system if
 is finitary. The consequence operation associated with  is denoted by . For every , the closed set
 is the -filter of  generated by . If  is finitary, then  is finitary for every algebra .

On the algebra of formulas  , the closure operator  coincides with  and the -closed sets are exactly the -\textit{theories}; that is, the sets of formulas which are closed under the relation .

\subsection{-algebras and selfextensional logics}\label{subsec:Salgebras}

One of the basic topics of AAL is how to associate in a uniform way a class of algebras with an arbitrary logic . According to contemporary AAL \cite{FJa09}, the canonical algebraic counterpart of   is the class , whose elements are called \emph{-algebras}. This class can be defined via the notion of Tarski congruence.

For any algebra  (of the same type as ) and any closure system  on ,
the \emph{Tarski congruence of  relative to },  denoted by ,
is the greatest congruence which is compatible with all , that is, which does not relate elements of  with elements which do not belong to .
The Tarski congruence of the closure system consisting of all -theories relative to  is denoted by . The quotient algebra  is called the \emph{Lindenbaum-Tarski algebra} of .

For any algebra , we say that  is an \emph{-algebra}  (cf.\ \cite[Definition 2.16]{FJa09}) if the Tarski congruence of  relative to  is the identity. It is well-known (cf.\ \cite[Theorem 2.23]{FJa09} and ensuing discussion) that  is closed under direct products. Moreover, for any logic , the Lindenbaum-Tarski algebra is an -algebra (see page 36 in \cite{FJa09}).

A logic  is \emph{selfextensional} when the relation of \emph{logical equivalence} between formulas

is a congruence relation of the formula algebra .
An equivalent definition of selfextensionality (see page 48 in \cite{FJa09}) is given as follows:  is selfextensional iff  the Tarski congruence  and the relation of logical equivalence  coincide. In such case the Lindenbaum-Tarski algebra reduces to . Examples of selfextensional logics besides classical propositional logic are intuitionistic logic, positive modal logic \cite{CeJa99}, the -fragment of classical propositional logic,  Belnap's four-valued logic \cite{Be77} the local consequence relation associated with Kripke frames, and the (order-induced) consequence relation associated with MV-algebras and defined by ``preserving degrees of truth'' (cf.\ \cite{Fo03}). Examples of non-selfextensional logics include linear logic, the (1-induced) consequence relation associated with MV-algebras and defined by ``preserving absolute truth'' (cf.\ \cite{Fo03}), and the global consequence relation associated with Kripke frames.

From now on we assume that  is a selfextensional logic and . For any formula , we say that  is \emph{provably equivalent} to a variable iff there exist a variable  such that .

\section{Formal framework}
\label{Sec:framework}
In the present section, we generalize Herzberg's algebraic framework for aggregation theory from MV-propositional attitudes to -propositional attitudes, where  is an arbitrary selfextensional logic. Our conventional notation is similar to \cite{He13}. Let  be a logical language which contains countably many connectives, each of which has arity at most , and let  be the collection of -formulas.

\subsection{The agenda}\label{subsec:agenda}

The \emph{agenda} will be given by a set of formulas . Let  denote the closure of  under the connectives of the language. Notice that for any constant , we have .

We want the agenda to contain a sufficiently rich collection of formulas.
In the classical case, it is customary to assume that the agenda contains at least two propositional variables.
In our general framework, this translates in the requirement that the agenda contains at least  formulas that `behave' like propositional variables, in the sense that their interpretation is not constrained by the interpretation of any other formula in the agenda.

We could just assume that the agenda contains at least  different propositional variables, but we will deal with a slightly more general situation, namely, we assume that the agenda is -pseudo-rich:

\begin{definition}\label{def:pseudo:rich}
An agenda is -\emph{pseudo-rich}, if it contains at least  formulas  such that each  is provably equivalent to  for some set  of pairwise different variables.
\end{definition}

\subsection{Attitude functions, profiles and attitude aggregators}

An \emph{attitude function} is a function  which assigns each formula in the agenda to an element of the algebra .

The \emph{electorate} will be given by some (finite or infinite) set . Each  is called an \emph{individual}.

An \emph{attitude profile} is an -sequence of attitude functions, i.e.\ .
For each , we denote the -sequence  by .

An attitude aggregator is a function which maps each profile of individual attitude functions in some domain to a collective attitude function, interpreted as the set of preferences of the electorate as a whole.
Formally, an \emph{attitude aggregator} is a partial map .

\subsection{Rationality}
Let the agenda contain formulas , where  is an -ary connective of the language and .
Among all attitude functions , those for which it holds that  are of special interest. In general, we will focus on attitude functions which are `consistent' with the logic  in the following sense.

We say that an attitude function  is \emph{rational} if it can be extended to a homomorphism  of -algebras. In particular, if  is rational, then it can be uniquely extended to , and we will implicitly use this fact in what follows.

We say that a profile  is \emph{rational} if  is a rational attitude function for each .

We say that an attitude aggregator  is \emph{rational} if for all rational profiles  in its domain,  is a rational attitude function. Moreover, we say that  is \emph{universal} if  for any rational profile . In other words, an aggregator is universal whenever its domain contains all rational profiles, and it is rational whenever it gives a rational output provided a rational input.

\subsection{Decision criteria and systematicity}

A \emph{decision criterion} for  is a partial map  such that for all  and all ,


As observed by Herzberg \cite{He13}, an aggregator is independent if the aggregate attitude towards any proposition  does not depend on the individuals attitudes towards propositions other than :

An aggregator  is \emph{independent} if there exists some map  such that for all , the following diagram commutes (whenever the partial maps are defined):



An aggregator  is \emph{systematic} if there exists some \emph{decision criterion}  for , i.e. there exists some map  such that for all , the following diagram commutes (whenever the partial maps are defined):



Systematic aggregation is a special case of independent aggregation, in which the output of  does not depend on the input in the second coordinate.
Thus,  is reduced to a decision criterion .

An aggregator  is \emph{strongly systematic} if there exists some decision criterion  for , such that for all , the following diagram commutes (whenever the partial maps are defined):
	


Notice that the diagram above differs from the previous one in that the agenda  is now replaced by its closure  under the connectives of the language. If  is closed under the operations in , then systematicity and strong systematicity coincide.

A formula  is \emph{strictly contingent} if for all  there exists some homomorphism  such that .
Notice that for any , any -pseudo rich agenda (cf.\ Definition \ref{def:pseudo:rich}) always contains a strictly contingent formula.
Moreover, if the agenda contains some strictly contingent formula , then any universal systematic attitude aggregator  has a unique decision criterion (cf.\ \cite[Remark 3.5]{He13}).\label{page:unique:decision:criterion}

Before moving on to the main section, we mention four definitions which appear in Herzberg's paper, namely that of \emph{Paretian} attitude aggregator (cf.\ \cite[Definition 3.7]{He13}), \emph{complex} and \emph{rich} agendas (cf.\ \cite[Definition 3.8]{He13}), and \emph{strongly systematizable} aggregators (cf.\ \cite[Definition 3.9]{He13}). Unlike the previous ones, these definitions rely on the specific MV-signature, and thus do not have a natural counterpart in the present, vastly more general setting. However, as we will see, our main result can be formulated independently of these definitions.
Moreover, a generalization of the Pareto condition follows from the assumptions of   being universal, rational and strongly systematic, as then it holds that for any constant , and any , if  for all , then .

\section{Results}
\label{Sec:Results}
\begin{lemma}
Let  be an -pseudo-rich agenda, ,  be an -ary connective and .
Then there exist formulas  in the agenda and a rational attitude function  such that  for each .
\end{lemma}
\begin{proof}
As the agenda is -pseudo-rich, there are formulas  each of which is provably equivalent to a different variable . Notice that this implies that the formulas  are not pairwise interderivable. So the -equivalence cells  are pairwise different, and moreover there exists a valuation  such that  for all .
Let , where  is the restriction of the canonical projection  to . Then clearly  is the required rational attitude function.
\end{proof}

\begin{lemma}\label{lemma2}
Let  be an -pseudo-rich agenda, ,  be an -ary connective and .
Then there exist formulas  in the agenda and a rational attitude profile  such that  for each .
\end{lemma}
\begin{proof}
As the agenda is -pseudo-rich, there are formulas  each of which is provably equivalent to a different variable . By previous lemma, for each , there exists a rational attitude function  such that  for each .
Thus it is easy to check that the sequence of attitudes is  a rational profile such that  for each .
\end{proof}

Recall that given that  is -pseudo rich, there exists a unique decision criterion for any strongly systematic attitude aggregator  (cf.\ page \pageref{page:unique:decision:criterion}).

\begin{proposition}\label{prop1}
Let  be a rational, universal and strongly systematic attitude aggregator. Then the decision criterion of  is a homomorphism of -algebras.
\end{proposition}
\begin{proof}

By the strong systematicity of , there exists a decision criterion  of . Let us show that under the assumptions of the proposition,  is a homomorphism of -algebras, by showing that  for each constant , and that, for each -ary connective , and any ,



Let  be a constant and let  be a rational profile. By definition .
Since  is rational, it can be extended to  so that .
Moreover, as  is also rational, it can also be extended to  so that .
Therefore, by  being strongly systematic, we get:


Let  be an -ary connective, where recall that  and  let .
By Lemma \ref{lemma2} and the -pseudo-richness of the agenda, there are formulas  and a rational profile  such that


Notice that for each ,  being rational implies that it can be extended to  so that .
Therefore, by the definition of operation the  in the product algebra  we have


By the assumption of  being universal, it follows that . Moreover, since  is rational,  is a rational attitude function, hence it can be extended to  so that


Finally, since  is strongly systematic,



Hence,

\begin{center}
\begin{tabular}{r c l l}
&=& & \eqref{equ:2}\\
&=&& \eqref{equ:5}\\
&=&& \eqref{equ:3}\\
&=&& \eqref{equ:4}\\
&=&,& \eqref{equ:1}
\end{tabular}
\end{center}

as required.
\end{proof}
\begin{proposition}\label{prop2}
Let  be a homomorphism of -algebras. Then the function , defined for any rational profile  and any  by the following assignment:  is a rational, universal and strongly systematic attitude aggregator.
\end{proposition}
\begin{proof}
By definition  is a universal aggregator, and moreover its domain coincides with the set of all rational profiles.

Let  be a rational profile. Then it can be extended to a homomorphism . Hence,  is also a homomorphism. Let us show that the restriction of  to the agenda  coincides with . Indeed, for any  we have  as required. This shows that  is a rational attitude function, and hence  is a rational attitude aggregator.

It immediately follows from the definition that  is systematic. Let . By definition  is rational, so it can be uniquely extended to . Therefore, by a similar argument as in previous paragraph, we can uniquely extend  to  so that for any , . This shows that  is also strongly systematic, which finishes the proof.
\end{proof}
Finally, the conclusion of the following corollary expresses a property which is a generalization of the Pareto condition (cf.\ \cite[Definition 3.7]{He13}).

\begin{corollary}
If  is universal, rational and strongly systematic, then for any constant  and any , if  for all , then .
\end{corollary}
\begin{proof}
Let  and .
Notice that by definition of the product algebra, the sequence  is precisely .
If  for all , i.e.\ ,
then by Proposition \ref{prop1},
, as required.\end{proof}

\section{Applications}
\label{Sec:Arrow}
In the present section, we show how the setting in the present paper relates to existing settings in the literature. \subsection{Arrow-type impossibility theorem for judgment aggregation}
Let  be the classical propositional logic. Its algebraic counterpart  is the variety of Boolean algebras. Let  be its language (the connectives  are definable from the primitive ones). Let  be the two-element Boolean algebra. Let  be a 2-pseudo-rich agenda.

By Propositions \ref{prop1} and \ref{prop2}, for every electorate , there exists a bijection between rational, universal and strongly systematic attitude aggregators 
\footnote{Note that in this case an alternative presentation of  is , which is the standard one.}
and Boolean homomorphisms .

Recall that there is a bijective correspondence between Boolean homomorphisms  and ultrafilters of . Moreover, if  is finite, every ultrafilter of  is principal. In this case, a decision criterion corresponds to an ultrafilter exactly when it is dictatorial.

\begin{comment}
\subsection{Examples of Selfextensional Logics}

\subsubsection{Modal logic.}

Modal logic (see e.g. \cite{BRV01}) is used to formalize propositions involving statements like ``it is possible that'' (modal statements), ``it will be that'' (temporal statements), ``it is obligatory that'' (deontic statements), ``I know that'' (epistemic statements), ``I believe that'' (doxastic statements), and so on. Typically, we use  to denote that  is necessary,  is going to be true, I know that , etc., and  to denote that  is possible,  will be true sometime in the future,  is consistent with my knowledge, etc. The formulas of modal logic can be defined as follows:



A normal modal logic  is defined as a set of formulas (called the \emph{theorems} of ) in the language defined above which contains all the propositional logic tautologies, formulas of the form , and is closed under modus ponens (from  and  get ), substitution and necessitation (from  get ).

Given a normal modal logic , the weak consequence relation (see \cite{Ja95}) is defined as follows:\\

 is weakly deducible in  from a set of formulas  if  belongs to the smallest set including , all the theorems of  and closed under modus ponens. It can be shown that  is weakly deducible in  from a set of formulas  iff there is a finite  such that  is a theorem of . 

It can be checked that for the interderivability relation  of  is compatible with the connectives in the language: if  and , then ,  and . Therefore,  is a congruence relation of the formula algebra , and thus  with the weak deducibility relation is a selfextensional logic.

\subsubsection{Fuzzy logic}

In natural language, there are propositions involving vague properties like ``X is tall'', ``X is old''. Typically, this kind of propositions do not have an absolute truth value, but rather, a ``degree'' of truth. Propositions can also be compared for their truth degrees (e.g., X is taller than Y is old), and composed propositions can have their truth value based on the truth functions of the corresponding connectives. 

Fuzzy logic (see \cite{Ma14}) is the systematic study of the logics of a degree-based notion of truth. In the remaining of this section we will talk about T-norm based fuzzy logics preserving degrees of truth (see \cite{BEFGGTV08}).

  The formulas of fuzzy logic can be defined as follows:



Given a variety  of residuated lattices, the consequence relation (preserving degrees of truth)  is defined as follows:
\begin{itemize}
\item For finite set  of formulas,  iff , if  for all , then .
\item For infinite set  of formulas,  iff there exists a finite subset  such that .
\end{itemize}

By Proposition 3.1 in \cite{BEFGGTV08},  is selfextensional.


\subsubsection{Relevance logic}

In classical logic, an implicative formula  is true iff  is false or  is true, even if  and  share no common information, which is not intuitive in natural language. Relevance logic (see \cite{Du86}) concerns the relevance between the antecedent and the consequent of implications in order to avoid the paradox of material implication. 

In the following we use the formalisations of relevance logic in \cite{FoRo94}. The formulas of relevance logic can be defined as follows:



 is the abbreviation for . The set  of theorems is defined as the smallest set containing the following axioms and closed under the adjunction rule (from  and  get ) and the modus ponens rule(from  and  get ) (see Section 2, \cite{FoRo94}):

\begin{itemize}
\item ;
\item ;
\item ;
\item ;
\item ;
\item ;
\item ;
\item ;
\item ;
\item ;
\item ;
\item ;
\item .
\end{itemize}

The Weaker Deductive System () is defined as follows (see Definition 3.1, \cite{FoRo94}):  iff  and there are  s.t.\ .

By Lemma 4.11 in \cite{FoRo94},  is selfextensional.

\subsubsection{Conditional Logic}

\texttt{A paragraph about the phenomenon}

\texttt{A paragraph about the language}

\texttt{A paragraph about the system and consequence relation}

\texttt{A paragraph about its selfextensionality}












\subsubsection{Logic for resources and capability}

\texttt{A paragraph about the phenomenon}

\texttt{Reference, \cite{BiGrPaTz15}}

\texttt{A paragraph about the language}

\texttt{A paragraph about the system and consequence relation}

\texttt{A paragraph about its selfextensionality}

\end{comment}


























\subsection{An example of a natural interpretation for subjunctive implication}

In \cite{Di10}, Dietrich argues that, in order to reflect the meaning of connection rules (i.e.\  formulas of the form  or  such that  and  are conjunctions of atomic propositions or negated atomic propositions) as they are understood and used in natural language, the connective  should be interpreted subjunctively.That is, the formula  ”should not be understood as a statement about the actual
world, but about whether  holds in hypothetical world(s) where  holds,  depends on 's truth value in possibly non-actual worlds.
Dietrich proposes that, in the context of connection rules, any such implication should satisfy the following conditions:
\begin{itemize}
\item[(a)] for any atomic propositions  and ,   is inconsistent
with  but consistent with each of      ;

\item[(b)] for any  atomic propositions  and ,   is consistent with each of ,  ,   and  .
\end{itemize}
Clearly, the classical interpretation of  as  satisfies only condition (a) but not (b).  The subjunctive interpretation of  has been formalised in various settings based on possible-worlds semantics. One such setting, which is different from the one adopted by Dietrich's, is given by Boolean algebras with operators (BAOs). These are Boolean algebras endowed with an additional unary operation  satisfying the identities  and . Let us further restrict ourselves to the class of BAOs such that the inequality  is valid. This class coincides with , where  is the normal modal logic {\bf T} with the so-called local consequence relation. It is well known that {\bf T} is selfextensional and is complete w.r.t.\ the class of reflexive Kripke frames.  In this setting, let us stipulate that  is interpreted as . 

It is easy to see that this interpretation satisfies both conditions (a) and (b).
To show that  is inconsistent with , observe that . 

To show that  is consistent with ,consider the two-element BAO s.t.\  and . The assignment mapping  and  to  witnesses the required consistency statement. The remaining part of the proof is similar and hence is omitted.

Clearly, the characterization theorem given by Propositions \ref{prop1} and \ref{prop2} applies also to this setting. However, the main interest of this setting is given by the possibility theorems. It would be a worthwile future research direction to explore the interplay and the scope of these results.

\begin{comment}

\subsection{Comparison with other approaches}

Dietrich \cite{Di07} proposes a general model for judgment aggregation, which can be applied to general logic, including classical propositional logic, predicate logic, modal or conditional logics, and some multi-valued or fuzzy logics, as well as giving generalised impossibility theorems. In \cite{Di10}, Dietrich shows that impossibility results do not apply to a wide class of realistic agendas once propositions of the form `if  then ' are  modelled as {\em subjunctive} implications rather than material implications. Since both Dietrich's methodology and our methodology provide a uniform way to deal with judgment aggregation problems, it is worthwhile to compare his methodology with ours.

\subsubsection{Dietrich's methodology}

\texttt{To be added later, since there are too much ingredient, and I do not want the paper to contain 10 pages of definitions......}








\subsubsection{A comparison between the two methodologies}

As we can see, the underlying logical backgrounds of Dietrich's approach and our approach are similar in the sense that both approaches treat different logics satisfying certain properties uniformly in an abstract way. However, there are significant differences between the two approaches, which can lead to different results.

In the setting of the present paper, the aggregator  takes an -sequence  of attitude functions as input. Intuitively, our attitude functions give the algebraic interpretation of \emph{all} formulas in the agenda . In particular, the rational attitude functions are those which can be consistently extended to an algebraic assignment . Therefore, the inputs provide \emph{all} the relevant information of the formulas in the agenda.

In the setting of Dietrich, the inputs are subsets of agendas, i.e. the formulas that the individuals consider true according to certain criteria. Therefore, the underlying interpretation of the formulas are not directly seen if it is not based on classical propositional logic, but rather a ``black box'' which is implicit.

Take modal logic as an example: for judgment aggregation problem in modal logic, a rational judgment set can be the true formulas in the agenda  at , where  is a Kripke model and  is a state in it. The input for one individual is only the judgment set, instead of the underlying pointed Kripke model . Therefore, for a given judgment set, there might be different underlying pointed Kripke models  generating the same judgment set, which are dually equivalent to different algebraic assignments, i.e. different attitude functions.

Therefore, Dietrich's judgment set is ``local'' in the sense that the underlying interpretation is not fully revealed in the judgment set, but rather only the information about a certain state is given. In contrast, the attitude functions are ``global'' in the sense that the interpretations of the formulas in the agenda are fully given.

In the special case where  is the set of all formulas, the comparison is more clear: a rational attitude function will be the same as an algebraic assignment , which encodes all the information of the semantics. On the other hand, a judgment set could still have different underlying pointed Kripke models, and therefore different algebraic interpretations on the dual side.

In addition, in the present setting, the algebra  is fixed, which dually means that only a specific Kripke frame could be used, while in the setting of Dietrich, all Kripke models for the underlying logic could be the underlying model of the individuals.

\end{comment}


\begin{comment}

Other relevant papers:

Boolean case \cite{He10}, Judgment aggregators and Boolean algebra homomorphisms

\cite{He14}, Aggregating infinitely many probability measures

Predicate Logic case \cite{EcHe12}, The model-theoretic approach to aggregation: Impossibility results for finite and infinite electorates

Predicate logic, infinitary case \cite{EcHe12-b}, Impossibility Results for Infinite-Electorate Abstract Aggregation Rules

\cite{EcHe14}, The Problem of Judgment Aggregation in the Framework of Boolean-Valued Models
\end{comment}

\bibliographystyle{abbrv}
\bibliography{SocialChoice}{}

\end{document} 