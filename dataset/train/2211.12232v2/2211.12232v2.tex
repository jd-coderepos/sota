

\begin{figure}[t!]
    \centering
    \includegraphics[width=\linewidth]{graphics/encoder-block-abbriviated-rotated.png}
    \caption{Encoder layer.}
    \label{fig:encoder-block}
\end{figure}

\vspace{-0.2cm}
\section{Method}
\label{sec:model}





Given a signal of low resolution  that was downsampled from its high-resolution counterpart , where  is a fixed upsampling scaling factor, our goal is to reconstruct  and generate its missing high-frequency content.

Operating in the frequency domain, the waveform signal  is converted to  using \ac{STFT}. As demonstrated in \cite{choi}, the input to the model is represented as a concatenation of the real and imaginary parts of the complex-valued spectrogram. The model is then trained to directly predict the spectrogram of the high-frequency signal. The resulting spectrogram  is then transformed back to the reconstructed high-resolution signal  using the \ac{iSTFT}.
\vspace{-0.2cm}
\subsection{Architecture}

Inspired by Demucs \cite{hdemucs}, our model is a convolutional U-Net architecture that operates in the frequency domain, with four layers in the encoder and decoder each. 
The encoder accepts the signal in spectrogram form, and uses 1D convolutions that operate only on the frequency axis. A \ac{FTB} \cite{phasen} is added before each encoder layer. Within each layer are two compressed residual branches; as opposed to the original Demucs architecture, we use the Snake activation function \cite{snake}. For the inner layers of the encoder we utilize both LSTM and temporal-based attention modules. With a downsampling scheme of [4,4,2,2], the resulting latent vector is a projection of the input spectrogram, compressed -fold in the frequency axis. The encoder layer is visually described in Figure~\ref{fig:encoder-block}. Following the encoder, a decoder transforms the latent vector to a spectrogram of size equal to that of the input to the encoder.

Unlike prior studies that first perform upsampling and then optimize the network to fill the missing frequency range\cite{seanet, nugan}, in this work we directly multiply the hop-length and the window size of the \ac{iSTFT} by the scaling factor  (more details can be found in Section~\ref{subsec:spec_upsampling}). With the proposed upsampling technique, information in the encoding process is held across the whole range of frequencies instead of being limited by the Nyquist rate. To accommodate for this, we use concatenated skip connections instead of summation.





\begin{table*}[t!]
\centering
\caption{L, V and M denote LSD, ViSQOL and MUSHRA respectively. MUSHRA score is specified with a  Confidence Interval of 0.95. \label{tab:results}}
\resizebox{\textwidth}{!}{\begin{tabular}{@{\hskip6pt}l|ccc|ccc|ccc|ccc} 
    \toprule
     & 
     \multicolumn{3}{|c|}{8-16} & 
     \multicolumn{3}{|c|}{8-24} & 
     \multicolumn{3}{|c|}{4-16} & 
     \multicolumn{3}{|c}{11-44} \\
    \midrule
     & L & V & M \
     & L & V & M \
     & L & V & M \
     & L & V & M \\
    \midrule
        Reference &- &- &96.25\pmr{1.5} &- &- &97.16\pmr{1.4} &- &- &96.18\pmr{1.5} &- &- &95.30\pmr{2.5} \\
        Anchor &- &- &54.65\pmr{4.3} &- &- &56.21\pmr{4.4} &- &- &41.14\pmr{3.8} &- &- &46.55\pmr{7.4} \\
        \midrule
        Sinc &2.32 &3.41 &60.13\pmr{4.7} &2.96 &3.41 &59.49\pmr{4.8} &3.59 &2.27 &43.03\pmr{3.9} &3.91 &1.97 &47.61\pmr{8.0} \\
        TFiLM \cite{kuleshov} &1.27 &3.18 &58.53\pmr{4.0} &- &- &- &1.77 &2.25 &41.91\pmr{4.0} &- &- &- \\
        SEANet \cite{seanet} &0.79 &4.08 &91.23\pmr{2.9} &0.91 &4.06 &94.16\pmr{2.2} &0.99 &3.16 &89.40\pmr{3.2} &1.13 &2.88 &80.52\pmr{7.0} \\
        BEHMGAN \cite{behmgan} &- &- &- &- &- &- &- &- &- &1.80 &2.01 &46.27\pmr{8.3} \\
        \midrule
        Ours () &0.84 &4.02 &90.58\pmr{2.3} &0.99 &4.03 &\textbf{96.40\pmr{1.9}} &1.04 &3.04 &86.14\pmr{3.4} &1.16 &2.88 &81.21\pmr{6.4} \\
        Ours () &0.80 &4.11 &92.63\pmr{2.4} &0.91 &4.12 &95.41\pmr{2.0} &0.99 &3.15 &\textbf{92.05\pmr{2.7}} &1.16 &\textbf{2.89} &81.67\pmr{6.8} \\
        Ours () &\textbf{0.77} &\textbf{4.16} &\textbf{94.64\pmr{1.6}} &\textbf{0.90} &\textbf{4.17} &94.45\pmr{2.1} &\textbf{0.94} &\textbf{3.28} &90.61\pmr{3.1} &\textbf{1.12} &2.88 &\textbf{84.18\pmr{5.6}} \\
    \bottomrule
\end{tabular}
}
\end{table*}

\begin{table}[t!]
\small
\centering
\caption{12-48 results. We use the  configuration.\label{tab:12-48-results}}

\begin{tabular}{@{\hskip6pt}l|ccc} 
    \toprule
     & L & V & M \\     
    \midrule
        Reference &- &- &98.47\pmr{0.9} \\
        Anchor &- &- &67.76\pmr{4.1} \\
        \midrule
        Sinc &3.36 &4.33 &69.77\pmr{4.3} \\
        SEANet \cite{seanet} &\textbf{0.86} &\textbf{4.71} &96.17\pmr{1.6} \\
        NuWave2 \cite{nuwave2} &1.34 &4.42 &84.87\pmr{4.5} \\
        \midrule
        Ours &0.92 &4.67 &\textbf{96.71\pmr{1.8}} \\
    \bottomrule
\end{tabular}

\end{table}

\begin{figure}[t!]
    \centering
    \begin{subfigure}{0.49\columnwidth}
        \centering
        \includegraphics[width=\linewidth]{graphics/no_spec_upsampling.png}
        \label{fig:verge-artifact}
    \end{subfigure}~
    \begin{subfigure}{0.49\columnwidth}
        \centering
        \includegraphics[width=\linewidth]{graphics/spec_upsampling.png}
        \label{fig:verge-no-artifact}
    \end{subfigure}
    \vspace{-0.4cm}
    \caption{Spectral upsampling comparison. Both spectrograms are outputs of the same signal resulting from the model before the iSTFT stage, upsampled from 4kHz to 16kHz. The left spectrogram is generated with no spectral upsampling, whereas the one on the right is generated with spectral upsampling.\label{fig:spectrograms}}
    \vspace{-0.2cm}
\end{figure}

\vspace{-0.2cm}
\subsection{Training Objective}
\sloppy Similar to ~\cite{defossez2020real}, we use a multi-resolution \ac{STFT} loss \cite{yamamoto} using FFT bins , hop lengths  and window sizes . 
Additionally we apply the multi-scale adversarial and feature losses~\cite{seanet} that operate in the time domain.







Using only a spectral reconstruction loss we substantially outperform state-of-the-art methods in objective metrics. We found that this produces audible artifacts and significantly impacts subjective metrics. Adding adversarial and feature losses acts as a type of perceptual loss and removes artifacts while slightly reducing the objective metrics (see Table~\ref{tab:adv_ablation}).

\vspace{-0.2cm}
\subsection{Upsampling in the Frequency Domain}
\label{subsec:spec_upsampling}

As mentioned above, a typical first step in audio upsampling methods is to upsample the input signal to the target sample rate in the time domain using Sinc interpolation \cite{tfilm, nugan, nuwave2, seanet, bwe-all-you-need, mugan}. The resulting waveform potentially holds the Nyquist rate of a high-resolution signal, but, as no additional frequencies are actually added, the top segment of the corresponding spectrogram holds no significant information. We find that this technique produces significant artifacts at the verge between the observed low-frequency range and the generated high-frequency range, as seen in Figure~\ref{fig:spectrograms}.



To mitigate this, we propose a method for upsampling in the frequency domain. Using different window sizes and hop lengths at the \ac{STFT} and \ac{iSTFT} stages, we can start from a given low-resolution signal  and end with a high-resolution  -- upsampled by a factor of  -- while using a single \ac{STFT} representation of fixed size at the intermediate generation stage. With this technique, the input to the model holds information across the whole range of frequencies.

At the \ac{STFT} stage, the waveform signal  is converted to  with  frequency bins, a hop length of , and a window length of , where  defines the overlapping ratio between consecutive frames.
At the \ac{iSTFT} stage, the model's output  is transformed to  using parameters customized to the upsampling setting; this uses the same number of frequency bins , but with a hop length of  and a window length of . This process is illustrated in Figure~\ref{fig:model}.