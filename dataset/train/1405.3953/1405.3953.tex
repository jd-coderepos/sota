



\NeedsTeXFormat{LaTeX2e}
\pdfoutput=1

\documentclass{new_tlp}
\usepackage{url}
\usepackage{html}
\usepackage{ifthen}
\usepackage{calc}
\usepackage{mathptmx}
\usepackage{color}
\usepackage{listings}
\usepackage{multirow}
\usepackage[draft]{fixme}
\usepackage{graphicx}
\sloppy

\newcommand{\reffont}{\tt}
\newcommand{\functor}[2]{\mbox{\reffont #1/#2}}
\newcommand{\predref}[2]{\mbox{\reffont #1/#2}}
\newcommand{\module}[1]{\mbox{\reffont #1}}
\newcommand{\appref}[1]{appendix~\ref{sec:#1}}
\newcommand{\secref}[1]{section~\ref{sec:#1}}
\newcommand{\Secref}[1]{Section~\ref{sec:#1}}
\newcommand{\figref}[1]{figure~\ref{fig:#1}}
\newcommand{\Figref}[1]{Figure~\ref{fig:#1}}
\newcommand{\tabref}[1]{table~\ref{tab:#1}}
\newcommand{\Tabref}[1]{Table~\ref{tab:#1}}

\title[Theory and Practice of Logic Programming]
        {Pengines: Web Logic Programming Made Easy}

  \author[T. Lager and J. Wielemaker]
         {TORBJ\"ORN LAGER\\
	  University of Gothenburg, Sweden \\
	  \email{Torbjorn.Lager@ling.gu.se}
	  \and JAN WIELEMAKER\\
	  VU University Amsterdam, The Netherlands\\
	  \email{J.Wielemaker@vu.nl}}

\jdate{May 2003}
\pubyear{2014}
\submitted{14 May 2014}
\revised{14 May 2014}
\accepted{27 March 2014}

\pagerange{\pageref{firstpage}--\pageref{lastpage}}



\definecolor{lightgrey}{rgb}{0.95,0.95,0.95}

\newcommand{\yap}{\textsc{yap}}
\newcommand{\yapswi}{\textsc{yap/swi}}
\newcommand{\mfc}{\textsc{mfc}}
\newcommand{\ciao}{\texttt{Ciao}}
\newcommand{\ciaopp}{\texttt{CiaoPP}}
\newcommand{\swi}{\texttt{SWI-Prolog}}
\newcommand{\nt}[1]{\mbox{\it #1}}
\renewcommand{\arg}[1]{\textit{#1}}
\newcommand{\kbd}[1]{\mbox{\tt #1}}
\makeatletter
\def\@nodescription{false}
\newcommand{\onlinebreak}{}

\newcommand{\defentry}[1]{\definition{#1}}
\newcommand{\definition}[1]{\onlinebreak \ifthenelse{\equal{\@nodescription}{true}}{\def\@nodescription{false}\makebox[-\leftmargin]{\mbox{}}\makebox[\linewidth+\leftmargin-1ex][l]{\bf #1}\\}{\item[{\makebox[\linewidth+\leftmargin-1ex][l]{#1}}]}}
\newcommand{\nodescription}{\def\@nodescription{true}}
\def\predatt#1{\hfill{\it\footnotesize[#1]}}
\def\predicate{\@ifnextchar[{\@attpredicate}{\@predicate}}
\def\qpredicate{\@ifnextchar[{\@attqpredicate}{\@qpredicate}}
\def\@predicate#1#2#3{\ifthenelse{\equal{#2}{0}}{\defentry{#1}}{\defentry{#1({\it #3})}}\index{#1/#2}\ignorespaces}
\def\@attpredicate[#1]#2#3#4{\ifthenelse{\equal{#3}{0}}{\defentry{#2\predatt{#1}}}{\defentry{#2({\it #4})\predatt{#1}}}\index{#2/#3}\ignorespaces}
\def\@qpredicate#1#2#3#4{\ifthenelse{\equal{#3}{0}}{\defentry{#1:#2}}{\defentry{#1:#2({\it #4})}}\index{#1/#2}\ignorespaces}
\def\@attqpredicate[#1]#2#3#4#5{\ifthenelse{\equal{#4}{0}}{\defentry{#2:#3\predatt{#1}}}{\defentry{#2:#3({\it #5})\predatt{#1}}}\index{#2/#3}\ignorespaces}
\def\directive{\@ifnextchar[{\@attdirective}{\@directive}}
\def\@directive#1#2#3{\ifthenelse{\equal{#2}{0}}{\defentry{:- #1}}{\defentry{:- #1({\it #3})}}\index{#1/#2}\ignorespaces}
\def\@attdirective[#1]#2#3#4{\ifthenelse{\equal{#3}{0}}{\defentry{:- #2\predatt{#1}}}{\defentry{:- #2({\it #4})\predatt{#1}}}\index{#2/#3}\ignorespaces}
\newcommand{\termitem}[2]{\ifthenelse{\equal{}{#2}}{\definition{#1}}{\definition{#1({\it #2})}}\ignorespaces}
\makeatother

\begin{document}

\label{firstpage}

\maketitle

  \begin{abstract}
When developing a (web) interface for a deductive database,
functionality required by the client is provided by means of HTTP
handlers that wrap the logical data access predicates. These handlers
are responsible for converting between client and server data
representations and typically include options for paginating results.
Designing the web accessible API is difficult because it is hard to
predict the exact requirements of clients. Pengines changes this
picture. The client provides a Prolog program that selects the required
data by accessing the logical API of the server. The pengine
infrastructure provides general mechanisms for converting Prolog data
and handling Prolog non-determinism. The Pengines library is small (2000
lines Prolog, 150 lines JavaScript). It greatly simplifies defining an
AJAX based client for a Prolog program and provides non-deterministic
RPC between Prolog processes as well as interaction with Prolog engines
similar to Paul Tarau's engines. Pengines are available as a standard
package for SWI-Prolog~7.\footnote{A web-based demonstration of pengines
is available at \url{http://pengines.swi-prolog.org} and can be
downloaded from \url{https://github.com/SWI-Prolog/pengines}.}
  \end{abstract}

  \begin{keywords}
web programming, query languages, agent programming, distributed
programming
  \end{keywords}



\section{Introduction}

Distributed systems play a central role in modern IT systems. We
distinguish three different models for point-to-point communication
between systems: (1) based on a query language (e.g., SQL or SPARQL),
(2) based on a generic attribute-value exchange (e.g., HTTP) and (3) based
on methods and datatypes (e.g., SunRPC, SOAP, CORBA, JSON-RPC). If
Prolog is used as a component in such systems, the typical solution is
to embed Prolog in another language through the foreign language
interface. Alternatively, Prolog may be used to implement the wire
protocol directly (often used for HTTP) or wrap a foreign library that
implements the wire protocol (e.g., ODBC).

If the task of the Prolog-based component is simple and can easily be
expressed as a deterministic function call, the above solutions are
satisfactory. If, however, Prolog is used as a (deductive) database, the
above is not ideal. In this scenario, an SQL or SPARQL like query
language over the core relational predicates provided by the database is
much more comfortable because it allows the client to specify a desired
set of results in a uniform and flexible manner. Both SQL and SPARQL
provide (1) a language to express the desired result set, (2) a way to
paginate result sets (\textit{offset} and \textit{limit}) and (3) a
uniform access to data of different types (per column in SQL and the
various RDF object types for SPARQL).

Without something similar to SQL or SPARQL, the Prolog server developer
has to imagine all sensible ways to access the data based on the
deterministic procedures and wrap these access functions in HTTP or some
form of RPC. If the application is a-priori known and has a fairly fixed
functionality, a top-down design can be satisfactory, but once this is
not the case we will typically see a large and growing set of API
functions with many options to select the proper data and represent it
in a way that is suitable for the client application. This is where
Pengines come in. The idea behind pengines is simple:

\begin{itemize}
    \item A pengine is a thread on a (often remote) Prolog pengine
	  server.
    \item The query language is Prolog, i.e., the client uploads a
	  short Prolog program to the pengine that provides the data
	  exchange needed by the client based on the clean relational
	  interface of the deductive Prolog database.
    \item Subsequently, the client sends one or more Prolog queries
          with result templates to the pengine.
    \item The pengine answers with a set of answer tuples based on
	  answer bindings of a template.  Data representation is
	  standardized.  At the moment we have two formats:
	  Prolog syntax for Prolog clients and a standard representation
	  of Prolog terms in JSON for e.g., JavaScript clients.
    \item Pagination is based on Prolog backtracking.  As an option,
          results can be batched in chunks of a certain size, e.g.,
	  return (max) 20 results per communication.
\end{itemize}

A pengine is closely related to Paul Tarau's logical engines (see
\secref{related}). If a Prolog client is used, pengines can implement
natural \textit{non-deterministic} RPC (NDRPC) as well as coroutining. The
JavaScript client allows for creating a pengine from Prolog embedded on
the HTML page, sending the pengine a query and reacting on the `answer
events'.

This article is organized as follows. First we discuss related work, in
particular the relation with logical engines as realised by Paul Tarau.
Next, we informally introduce Pengines using a number of examples. In
\secref{PLTP} we describe the state machine used to realise Pengines and
communicate with them as well as the core primitives. \Secref{derived}
describes a derived high level primitive (Prolog-RPC). Before the final
conclusions section, we address security and future plans.


\section{Related work}
\label{sec:related}

The notion of explicit Prolog engines has been explored extensively by
Paul Tarau et al \cite{tarau2009interoperating} in the context of the
Jinni agent programming language and the BinProlog system. Tarau's
engines are in-process and primarily designed to provide a clean
alternative implementation for Prolog language constructs such as the
all-solutions predicates (e.g., findall/3), exception handling as well
as language constructs that are less common in the Prolog world, such as
multiple coroutining blocks.

In contrast, \textit{pengines} are designed primarily for creating and
accessing Prolog engines on a remote server and communicating with them
using multiple languages (currently Prolog and JavaScript).  Tarau's
engine API and the Prolog implementation of our Pengine client are
closely related.  We explain the differences below.

\begin{description}
    \item[Creation]
Tarau's engines are created using a \arg{Goal} and \arg{AnswerTemplate}.
As our pengines typically run remote, they are more heavy weight and we
decided that a created pengine can be used to execute multiple queries
using \predref{pengine\_ask}{3}, i.e., a pengine is an engine that
runs a goal that asks for goals to execute. Typically, a pengine is
created with a \textit{Prolog source}, providing the code to execute.
This is irrelevant for Tarau's engines as they run in the same process.
(Remote) pengine creation requires some additional information, such as
the address of the server and the data-format for exchanging events
(Prolog or JSON).

    \item[Yield/return]
Tarau's yield and return primitives are represented using
\predref{pengine\_output}{1} (return) and \predref{pengine\_input}{2}
(yield).

    \item[One-sided communication]
Designed to run over HTTP, only the `client' can take initiative in the
pengines world. The \predref{pengine\_pull\_response}{2} primitive
realises bi-directional initiative, based on the `long polling'
technique that is commonly used in the context of HTTP to achieve server
initiative. This is not needed if engines
are embedded in the same process or can use a bi-directional
communication channel. (See also \secref{plans}.)
\end{description}

Pengines can also be regarded as a high-level interface to Prolog,
similar to InterProlog \cite{DBLP:conf/jelia/Calejo04} and the
multi-language interface supported by ECLIPSE Prolog
\cite{DBLP:conf/padl/ShenSNS02}. Whereas InterProlog provides a
Java-Prolog interface, ECLIPSE supports Tcl/Tk and Visual Basic as well
as Java. At this time, Pengines only supports Prolog-Prolog and
JavaScript-Prolog communication, the latter of utmost importance for a
seamless integration between Prolog and the Web. Other languages can
easily be added. Pengines also handles backtracking over
non-deterministic queries as well as I/O, something which is not
supported by the other interfaces.




 
\section{Pengines by example}
\label{sec:examples}

We proceed to give a number of examples showing how pengines can be
created and controlled from any Prolog program, or from JavaScript
running in a web browser. We also show how to make non-deterministic
remote procedure calls (NDRPC) using a predicate implemented on top of
the Pengines core predicates.

\subsection{Prolog interacting with a pengine}
\label{sec:ex1}

In this example we load the Pengines library, use
\texttt{pengine\_create/1} to create a pengine in a remote
pengine server, and inject a number of clauses in it. We then  use
\texttt{pengine\_event\_loop/2}  to start an event loop that listens for
three kinds of  event terms. Running \texttt{main/0} will write  the
terms  \texttt{q(a)},  \texttt{q(b)}  and   \texttt{q(c)}  to  the
standard output of the local process.  Using \texttt{pengine\_ask/3}
with the option \texttt{template(X)}
would produce the output \texttt{a}, \texttt{b} and \texttt{c}. Removing
the \texttt{server('http://pengines.org')} option would solve the query
\texttt{q(X)} locally instead, although still concurrently.

\begin{verbatim}
:- use_module(library(pengines)).

main :-
    pengine_create(
        [ server('http://pengines.org'),
          src_text("q(X) :- p(X).
                    p(a). p(b). p(c).")
        ]),
    pengine_event_loop(handle, []).

handle(create(ID, _))           :- pengine_ask(ID, q(X), []).
handle(success(ID, [X], false)) :- writeln(X).
handle(success(ID, [X], true))  :- writeln(X), pengine_next(ID, []).
\end{verbatim}

\subsection{JavaScript interacting with a pengine}

In this example, we show  how  to   create  and  interact with a
pengine from JavaScript.   Loading the page brings up the
browser's prompt dialog, waits   for the user's input, and writes that
input in the browser  window.   If  the input was 'stop', it stops
there, else it repeats.\footnote{We could have implemented
\textit{main/0} as a recursive loop instead, but using a repeat-fail
loop nicely serves to demonstrate that Pengines gives the programmer the
same options as when programming against a simple command-line shell.}
Note that   I/O  works as expected. All we need to  do is to use
\texttt{pengine\_input/2}   instead  of \texttt{read/1}  and
\texttt{pengine\_output/1} instead of \texttt{write/1}.

\begin{verbatim}
<html lang="en">
  <head>
    <script src="/vendor/jquery/jquery-2.0.3.min.js"></script>
    <script src="/assets/js/pengines.js"></script>
    <script type="text/x-prolog">
      main :-
          repeat,
          pengine_input('myprompt>', X),
          pengine_output(X),
          X == stop.
    </script>
    <script>
      var pengine = new Pengine({
          oncreate: function() { pengine.ask('main'); },
          onprompt: function() { pengine.respond(prompt(this.data)); },
          onoutput: function() { _{HTTP}_{HTTP}_{HTTP}p(X,Y),q(Y,Z)$ where \arg{Y} is a large intermediate result that
is not needed by the client. By default, Pengines will transport all
variable bindings and thus also \arg{Y}. This can be avoided in one of
two ways. Either we can define a predicate \texttt{pq(X,Z)}, or we
can use the option \texttt{template(Z)} to only transport the binding
for \arg{Z}.

The impact of network latency is limited by reducing the number of round
trips. This is achieved with three features. First of all, the first
query may be passed with the \textit{create} command. Second, the
pengine by default self-destroys itself on determinstic completion,
failure or error of the initial query. Third, the \texttt{chunk(N)}
option, sends answers in chunks of the specified size. Calling
\texttt{pengine\_rpc(URL, Query, [chunk(N)])} will result in a call to
\texttt{find\_n(N, Query, Query,
Solutions)}\footnote{\texttt{find\_n(+N, ?Template, +Goal, ?List)} acts
like \texttt{findall/3} but returns only the first \texttt{N}
bindings of \texttt{Template} to \texttt{List}, on backtracking another
batch of \texttt{N} bindings, and so on.} on the pengine server located
at \texttt{URL}.


\section{Security}
\label{sec:security}

There are three layers of safety that relate to pengines. First of all, we
can rely on the safety of the underlying HTTP protocol. Considering that
Prolog has full access to the OS, this is like using an outdated
unencrypted telnet session to the client. Alternatively, we can use
HTTPS with authentication, which makes it similar to SSH access to a
shell.

To facilitate anonymous users, SWI-Prolog provides
library(sandbox),\footnote{\url{http://www.swi-prolog.org/pldoc/doc/swi/library/sandbox.pl}}
which exports \predref{safe\_goal}{1}. This predicate performs abstract
interpretation of the argument goal and either succeeds, throws an
\textit{instantiation\_error} if it cannot prove sufficient
instantiation to a meta-goal or throws a \textit{permission\_error} if
it encounters a possibility that a predicate may be called that is not
white-listed. The library provides a predefined whitelist consisting of
pure Prolog built-in predicates, which enables it to prove the safety of
many of the pure Prolog libraries.

The above is typically sufficient if the server uses plain Prolog. If it
uses, e.g., the SWI-Prolog RDF database, the C-defined query facilities
of this library can be added to the whitelist.

Finally, by means of various application settings the Pengines library also offers some protection against denial of service (DoS) attacks. One setting determines the number of pengines that can be run simultaneosly by an application, another setting the maximum number of slave pengines that a master process may create, and a third setting is a timeout that aborts the computation after a set time, thus protecting against runaway computations.


\section{Portability and interoperability}
\label{sec:portability}

The current implementation of pengines is heavily based on support for
multiple threads as well as the SWI-Prolog HTTP server and client
libraries \cite{wielemaker:tplp2008}. It can probably be ported fairly
easily to YAP. Porting to other Prolog implementations that implement
the ISO working draft for
threads\footnote{\url{http://logtalk.org/plstd/threads.pdf}} is probably
feasible if the systems provide HTTP server and client access. It is also
possible to realise the protocol using Prolog processes managed from a
conventional HTTP server. Such a coarse grain implementation is more
robust against malicious pengines, but individual pengines start much
slower and cannot easily share a Prolog database.

To achieve maximal interoperability between pengine platforms, that
will allow a client on one platform to send and execute code on another
platform, an effort must be made to standardise a subset of Prolog that
runs in the same way on all platforms. Fortunately, such an effort only
needs to deal with a safe subset of Prolog (see Section
\ref{sec:security}).


\section{Evaluation}
\label{sec:evaluation}

In this section we present insight into the overhead involved in using
pengines. \Tabref{basictiming} shows the basic HTTP overhead, a minimal
pengine RPC call and the execution time of the example from \secref{rdf}
in three scenarios. We minimised network overhead by using connections
to \textit{localhost}. Timings were performed on a machine with an Intel
Core i7-3770 CPU running Ubuntu 13.10 (kernel 3.11) and SWI-Prolog
7.1.14. Times are in milliseconds, taking the average of 10 runs, each
with 1,000 iterations.

\begin{table}[h]
    \caption{Basic timing}
    \label{tab:basictiming}
\begin{tabular}{p{0.6\linewidth}rr}
\hline\hline
Test	& CPU time (ms) & Wall time (ms) \\
\hline
Most simple HTTP request                      & 0.4 & 0.8 \\
RPC for \texttt{true}                         & 0.9 & 1.9 \\
\predref{event\_in\_area}{3} on server        & 7     &	7 \\
RPC \predref{event\_in\_area}{3}, chunk = 1   & 159   & 386 \\
RPC \predref{event\_in\_area}{3}, chunk = 128 & 11    & 39 \\
\hline\hline
\end{tabular}

\end{table}

The effect of chunking the result set is significant because fetching
the next result batch involves an HTTP request.  Nevertheless, using a
high value may not be the optimal choice if not all answers are needed.
\Figref{chunking} illustrates the elapsed and CPU time used for
fetching all 1981 solutions for \predref{event\_point}{2} with different
chunk sizes.

\begin{figure}[h]
	\includegraphics[width=0.8\linewidth]{chunk}
    \caption{Performance with varying chunk sizes}
    \label{fig:chunking}
\end{figure}


\section{Future plans}
\label{sec:plans}

The concept of pengines is simple and is not likely to change much. One
addition that we do however contemplate is a way to allow two
pengines that are created and ``owned" by different masters to
communicate. Currently, this is not possible.

Two areas need further attention: security and performance. The
current security model aims at retrieving data from the server without
compromising it. Future versions are likely to include support to deny
access to certain data and provide authorized access to certain data,
which may include update operations. Regarding performance, it is easy
to get into scenarios where the HTTP protocol and network connection
overhead become severely limiting factors, in particular for
\predref{pengine\_rpc}{3}. Protocol and connection overhead can be
reduced by using WebSockets.\footnote{\url{http://www.websocket.org/}}
We also plan to implement \emph{dynamic} chunking of the results. This
could either be the \predref{pengine\_rpc}{3} client dynamically
switching to larger chunks or the server computing additional answers
while waiting for a \textit{next} command and returning all available
answers when the \textit{next} arrives.


\section{Conclusions}

Pengines extend Tarau's Prolog
engines by allowing for Prolog engines to live in a remote (Prolog)
process. The communication with pengines is based on a state machine and
uses either Prolog or JSON for serialization of messages. The current
transport protocol is HTTP, which facilitates web applications and
allows pengines to communicate smoothly with common firewall
configurations. Prolog interaction with pengines can be used to
implement agent interaction protocols, control structures such as
coroutining blocks as well as convenient non-deterministic Prolog RPC.
JSON based interaction constitutes an ideal way to interface Prolog with
JavaScript for implementing web applications.

Creating a remote pengine may involve uploading a Prolog program to the
server. `Bringing the code to the data' greatly reduces data exchange
between client and server. It allows the server to implement only a
clean relational interface, while the client can ask questions that
involve multiple relations efficiently.

Pengines are versatile. Applications include providing a web interface
for a classical question/response Prolog application, complex network
transparent control structures, Prolog-to-Prolog RPC, a modular
alternative to the SPARQL query language for the semantic web or a web
application based on a Prolog server.

We would like to stress that Pengines is not just yet another library
for SWI-Prolog. We believe our work can be viewed more abstractly, as a
description of a general approach to web logic programming, that can be
given a concrete manisfestation not only for Prolog but also for other
one-tuple-at-a-time logic programming languages.


\bibliographystyle{acmtrans}
\bibliography{pengines}

\label{lastpage}





\end{document}
