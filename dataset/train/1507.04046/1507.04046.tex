

\subsection{Lower bound}

In the case of general trees, Gavoille et al~\cite{Gavoille200485} establish a lower bound  of  using an ingenious technique where they apply a distance labeling scheme to a special class of trees called -trees\footnote{Note that their exposition has some minor errors as pointed out (and corrected) in~\cite{esbenthesis}}. The following uses a generalization of -trees to improve their ideas and leads to a lower bound of .

\paragraph{-trees.}
We begin with some definitions. For integers  and a number  such that  is integral for all , an \emph{-tree} is a rooted binary tree  with edge weights in  that is constructed recursively as follows. For ,  is just a single node. For ,  is a claw (i.e.\ a star with three edges) with edge weights  for some  rooted at the leaf node of the edge with weight . For ,  consists of an -tree whose two leaves are the roots of two -trees .  We shall denote an -tree constructed in this way by 
An example for  can be seen in \Cref{fig:hMtree}. Note that the case  simply corresponds to the -trees defined in~\cite{Gavoille200485}.

\begin{figure}
	\centering
	\begin{tikzpicture}[sibling distance=3em, inner sep=0pt, minimum size=1ex]
	\tikzstyle{every node}=[draw,circle,fill=black]
	\Tree [.{} \edge node[draw=none,fill=none,auto=right] {}; [.{} 
		\edge node[draw=none,fill=none,auto=right] {}; [.{} \edge node[draw=none,fill=none,auto=right] {}; [.{} 
			\edge node[draw=none,fill=none,auto=right] {}; [.{} \edge node[draw=none,fill=none,draw=none,fill=none,auto=right] {}; [.{} \edge node[draw=none,fill=none,auto=right,pos=.6] {}; [ .{} ] \edge node[draw=none,fill=none,auto=left,pos=.6] {}; [ .{} ] ] ]
			\edge node[draw=none,fill=none,auto=left] {}; [.{} \edge node[draw=none,fill=none,auto=left] {}; [.{} \edge node[draw=none,fill=none,auto=right,pos=.6] {}; [ .{} ] \edge node[draw=none,fill=none,auto=left,pos=.6] {}; [ .{} ] ] ]
			]
		]
		\edge node[draw=none,fill=none,auto=left] {}; [.{}	\edge node[draw=none,fill=none,auto=left] {}; [.{}
			\edge node[draw=none,fill=none,auto=right] {}; [.{} \edge node[draw=none,fill=none,auto=right] {}; [.{} \edge node[draw=none,fill=none,auto=right,pos=.6] {}; [ .{} ] \edge node[draw=none,fill=none,auto=left,pos=.6] {}; [ .{} ] ] ]
			\edge node[draw=none,fill=none,auto=left] {}; [.{} \edge node[draw=none,fill=none,auto=left] {}; [.{} \edge node[draw=none,fill=none,auto=right,pos=.6] {}; [ .{} ] \edge node[draw=none,fill=none,auto=left,pos=.6] {}; [ .{} ] ] ] 
			] 
		] 
	] ]
	\end{tikzpicture}
	\caption{An -tree, where . We require that ,  and .
}
	\label{fig:hMtree}
\end{figure}


It is easy to see that an -tree
has  leaves and  nodes.  Further, it is straightforward to see that, if  are leaves in an -tree , then


\paragraph{Leaf distance labeling schemes.}
In the following we shall consider \emph{leaf distance labeling schemes} for the family of -trees: that is, distance labeling schemes where only the leaves in a tree need to be labeled, and where only leaf labels can be given as input to the decoder. Since an ordinary distance labeling scheme obviously can be used only for leaves, any lower bound on worst-case label sizes for a leaf distance labeling scheme is also a lower bound for an ordinary distance labeling scheme. We denote by  the smallest number of labels needed by an optimal leaf distance labeling scheme to label all -trees.
\begin{lemma} \label{lemm:distancehM}
For all  and , .
\end{lemma}
\begin{proof}
Fix an optimal leaf distance labeling scheme  which produces exactly  distinct labels for the family of -trees. For leaves  and  in an -tree, denote by  and , respectively, the labels assigned by . For , let  be the set consisting of pairs of labels  for all leaves  and  in all -trees .

The sets  and  are disjoint for , since every pair of labels in  uniquely determines  due to~\eqref{eq:disthM}. Letting , we therefore have . 
Since  contains pairs of labels produced by  from leaves in -trees , we clearly also have , and hence it only remains to prove that , which we shall do by showing that  for all .

The goal for the rest of the proof is therefore to create a leaf distance labeling scheme for -trees using only labels from the set  for some fixed . So let  be given and consider an -tree . Let  . From  we shall construct an -tree  for  such that every leaf node  in  corresponds to nodes  in  for .
The trees  are defined as follows.
If , so that  consists of a single node, then  for . 
If , then  is in the form  for some . We can write  in the form  for uniquely determined  with . For , we recursively define . Thus,  is an -tree that is similar to  but where we replace the top edge weight  by edge weights  and, recursively, do the same for all -subtrees. Note also that the corresponding edge weight  in  automatically is replaced by the edge weight  in  in order for  to be an -tree.

Denote by  the leaf in  corresponding to the leaf  in .

Consider now the -tree . Every leaf  in  corresponds to the leaves  in  where  for . 
Using  formula~\eqref{eq:disthM} for the distances in , it is straightforward to see that


We can now apply the leaf distance labeling scheme  to  and obtain a label for each leaf node in . In particular, the pair of leaves  corresponding to a node  in  will receive a pair of labels. We use this pair to label  in , whereby we have obtained a labeling of the leaves in  with labels from . Using the formula in~\eqref{eq:disthMdistances} we can construct a decoder that can compute the distance between two nodes in  using these labels alone, and hence we have obtained a leaf distance labeling scheme for -trees using only labels from  as desired.
\end{proof}

\begin{lemma} \label{lemm:distancehM2}
For all  and , .
\end{lemma}
\begin{proof}
The proof is by induction on . For  we note that an -tree has only one node, so that . \Cref{lemm:distancehM} therefore yields  from which it follows that . The lemma therefore holds for . Now let  and assume that the lemma holds for . \Cref{lemm:distancehM} and the induction hypothesis now yield

from which it follows that  as desired.
\end{proof}

The previous lemma implies that any (leaf and hence also ordinary) distance labeling scheme for -trees must have labels with worst-case length at least . Since the number of nodes in such a tree is , it follows that , and hence that  for sufficiently large . From this we see that the worst case label length is at least 

In the case where , we retrieve the bound of  obtained in~\cite{gavoillepelegperennesraz}. It seems that larger values of  only makes the above result weaker, but the the real strength of the above becomes apparent when we switch to the unweighted version of -trees, in which we replace weighted edges by paths of similar lenghts. Note that a distance labeling scheme for the family of unweighted -trees can be used as a distance labeling scheme for the weighted -trees, and hence any lower bound in the weighted version automatcially becomes a lower bound in the unweighted version.

The number of nodes  in an \emph{unweighted} -tree is upper bounded by

In the case , we get .

\begin{theorem} \label{theo:distancelowerbintrees}
Any distance labeling scheme for unweighted -trees, and hence also for general trees, has a worst-case label size of at least .
\end{theorem}
\begin{proof}
Choose the largest integer  with , and note that we must have .  Set  and consider the family of  -trees, which is a subfamily of the family of trees with  nodes. From \Cref{lemm:distancehM2} it therefore follows that the worst-case label length is

\end{proof}
