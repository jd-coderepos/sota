

\begin{proof}[of \cref{thm:languages-finitely-supported}]
By properties of nominal sets, for  finitely supported and  equivariant,  is finitely supported with . Let  be the function mapping each  to . We need to show that  is equivariant, that is, . Without loss of generality, we shall prove the right-to-left inclusion. Then, since  and  are arbitrary, one can prove the left-to-right inclusion starting from the state  and the permutation . Let . We shall prove that . Consider the unique (accepting) run  of  from , and  the unique run  of  from . By equivariance of the transition function, and definition of run, for all , we have , thus , therefore . 
\qed 
\end{proof}
\begin{proof}[of \cref{lem:deterministic-configuration-graph}]
 For each , if , recalling that  is injective, there is  with . By definition of \hdma, there is exactly one transition labelled with , let it be . Then by definition of configuration graph, we have . Since  is injective, there can not be other transitions labelled with  in the configuration graph. If , consider the only transition with label  from , namely .  Then we have  in the configuration graph; this transition is unique by definition.
\end{proof}

\begin{proof}[of \cref{pro:nset-to-nom}]
 A run  in the configuration graph clearly is also a run in the obtained automaton. As , also acceptance is the same on both sides. By \cref{lem:deterministic-configuration-graph} we get determinism. The proof is completed by noting that the obtained transition function is equivariant. For this, chose an edge  in the configuration graph, and look at \cref{def:configuration-graph}, thus consider a corresponding \hdma\ transition . The case when  is straightforward. When , thus  consider the permuted configuration ,  for any permutation . Since , also  , thus we have a transition , which is precisely the required permuted transition. \qed
\end{proof}



\begin{proof}[of \cref{prop:ndma-to-hdma}]
 The proof is similar to the equivalence results between categories of coalgebras given in \cite{CianciaM10}. First, we need to show that, for each transition  in the original nDMA, there is an edge  in the configuration graph of the derived \hdma. We look at the case ; the case with allocation is similar, even though technically more involved. By equivariance, from , we have . Then we have an \hdma\ transition  where . 
 By looking at the used permutations, we have . 
 Then, in the configuration graph, we have , thus by equivariance, we have 
 , 
 thus . Accordance of the accepting conditions is straightforward.
 \qed
\end{proof}


\begin{proof}[of \cref{prop:edges-correspondence}]
Let  and , .

\paragraph{Part \eqref{sync-to-each}.}
 
Let  and let 
 
be the transition inducing . We proceed by cases on the rule used to infer this transition:
\begin{itemize}
	\item (\textsc{Reg}): then the transition is inferred from , , such that either  or  is in . Suppose, w.l.o.g., . Then  and , so there is an edge  in the configuration graph of . The following chain of equations shows that :
	To prove the existence of an edge  in the configuration graph of , we have to consider the following two cases:
	\begin{itemize}
		\item If , then , by the rule premise ;
\item If , then  should be fresh, so we have to check . Suppose, by contradiction, that there is  such that , then , by definition of , which implies , by injectivity of , i.e. , but the premise of the rule states , so we have a contradiction. 
	\end{itemize}
	Now we have to check . Since we have , for  the equations \eqref{eq:rho} hold. For  we have:
		



	\item \allrule: then we have  and the transition is inferred from , . Since , we also have , so there are  with , for . Finally, we have to check that each  is as required: if  equations \eqref{eq:rho} hold; for  we have
	
	
\end{itemize} 


\paragraph{Part \eqref{each-to-sync}.} 
Since  is deterministic, there certainly is , for any . This edge, by the previous part of the proof, has a corresponding edge , for each . But then , by determinism of .


\qed
\end{proof}

\begin{proof}[of \cref{thm:bool-closure}]
We just consider , the other cases are analogous. Let  be ; this is a proper \hdma{}, thanks to \cref{rem:syncp-fin-det}. Given , let , and  be the runs for  in the configuration graphs of  and , respectively. Then, by \cref{thm:inf-correspondence}, we have , for each . From this, and the definition of , we have that  if and only if  and , i.e.\  if and only if  and .
\qed
\end{proof}
\begin{proof}[of \cref{thm:decidable}]
Let  be a \hdma{} for . Consider the set . This is finite, so we can use it as the alphabet of an ordinary deterministic Muller automaton , where  is a dummy state, and the transition function is defined as follows:  if and only if , and  for all other pairs . Clearly  if and only if , as words in  are sequence of transitions of  that go through accepting states infinitely often, and thus produce a word in , and viceversa. The claim follows by decidability of emptiness for ordinary deterministic Muller automata. Finally, to check equality of languages, observe that the language  is -regular nominal, thanks to \cref{thm:bool-closure}. Then we just have to check its emptiness, which is decidable.
\qed
\end{proof}
We give one straightforward lemma about configuration graphs.
\begin{lemma}
\label{lem:tr-names}
For all edges  we have .
\end{lemma}
We give one additional lemma about  defined in \cref{sec:up-words}.
\begin{lemma}
\label{lem:xI}
Given , suppose there is a positive integer  such that . Then .
\end{lemma}
\begin{proof}
Suppose .  implies , so , but this is against the assumption that  is the largest set satisfying .
\qed
\end{proof}




\begin{proof}[of \cref{lem:rho-forget}]
Observe that this sequence is such that , for all  such that . In fact, suppose there are , with . Then we would have , because  is injective. In general, , for , therefore . This means that  which, by \cref{lem:xI}, implies , against the hypothesis .

Now, suppose that . Then we would have an infinite subsequence  of pairwise distinct names that belong to , but  is finite, a contradiction.
\qed
\end{proof}


\begin{proof}[of \cref{lem:idI}]\hfill

\item Let  be the function  with its codomain restricted to . Then  is an element of the symmetric group on , so it has an order , that is a positive integer such that . Hence .
\qed
\end{proof}

\begin{proof}[of \cref{lem:forgetT}]
Let  be

This gives the number of transitions it takes to forget all the names assigned to . Let  be . For any , we can choose  as any -tuple of words that are recognized by the loop and such that, whenever , then  is different from  and all the previous symbols in , for all  and . Let us verify  separately on  and  (recall : we have , because all the names assigned to  have been replaced by fresh ones; and we have , so .
\qed
\end{proof}


\begin{proof}[of \cref{lem:initT}]
For each name , define a tuple  where  is the index of the transition that consumes the fresh name that will be assigned to , and  is how many traversals of  it takes for this assignment to happen (including the one where the transition  is performed). Formally,  is the smallest integer such that there are  defined as follows

Let  be the set of such tuples and let . Then we can construct  as follows

where by  fresh we mean different from elements of  and previous symbols in .

The second case in the definition of  is justified as follows. Suppose  is the register assignment for , then we have to show . Suppose, by contradiction, that , then by \cref{lem:tr-names} and by how we defined the symbols consumed we have , for some , and some set of fresh (in the mentioned sense) names .
But , by construction, and  cannot already be in , because there cannot be two distinct tuples in  that coincide on the first component. Therefore we must have
, which implies , because , but this contradicts our hypothesis.


It is easy to check that this constructions reaches a configuration where all  have been assigned the desired value. 
\qed
\end{proof}