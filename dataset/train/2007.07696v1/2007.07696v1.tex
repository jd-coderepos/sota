\pdfoutput=1
\documentclass[runningheads]{llncs}
\usepackage{graphicx}
\usepackage{comment}
\usepackage{amsmath,amssymb} \usepackage{color}
\usepackage{url}
\usepackage{subfiles}
\usepackage{times}
\usepackage{epsfig}
\usepackage{graphicx}
\usepackage{amsmath}
\usepackage{amssymb}
\usepackage{gensymb}
\usepackage{caption}
\newcommand{\TODO}[1]{ {\color{red}\textbf{#1}} }
\usepackage{booktabs}
\usepackage{multirow}
\usepackage{amsfonts}
\usepackage{threeparttable}
\usepackage{multirow}
\usepackage[toc,page]{appendix}

\usepackage{xspace}



\makeatletter
\DeclareRobustCommand\onedot{\futurelet\@let@token\@onedot}
\def\@onedot{\ifx\@let@token.\else.\null\fi\xspace}

\def\eg{\emph{e.g}\onedot} \def\Eg{\emph{E.g}\onedot}
\def\ie{\emph{i.e}\onedot} \def\Ie{\emph{I.e}\onedot}
\def\cf{\emph{c.f}\onedot} \def\Cf{\emph{C.f}\onedot}
\def\etc{\emph{etc}\onedot} \def\vs{\emph{vs}\onedot}
\def\wrt{w.r.t\onedot} \def\dof{d.o.f\onedot}
\def\etal{\emph{et al}\onedot}
\makeatother


\begin{document}
\pagestyle{headings}
\mainmatter
\def\ECCVSubNumber{4549}  

\title{P$^{2}$Net: Patch-match and Plane-regularization \\ for Unsupervised Indoor Depth Estimation}
\begin{comment}
\titlerunning{ECCV-20 submission ID \ECCVSubNumber} 
\authorrunning{ECCV-20 submission ID \ECCVSubNumber} 
\author{Anonymous ECCV submission}
\institute{Paper ID \ECCVSubNumber}
\end{comment}
\titlerunning{P$^{2}$Net for Unsupervised Indoor Depth}
\authorrunning{Yu et al.}
\author{Zehao Yu$^*$\inst{1,2} \and
    Lei Jin$^*$\inst{1,2} \and
    Shenghua Gao$^{\dag}$\inst{1} \\
}
\institute{ShanghaiTech Univsertiy \and DGene Inc \\
    {\email \{yuzh,jinlei,gaoshh\}@shanghaitech.edu.cn} \\
    \small \url{https://github.com/svip-lab/Indoor-SfMLearner}
}

\begin{comment}
\titlerunning{Abbreviated paper title}
\author{First Author\inst{1}\orcidID{0000-1111-2222-3333} \and
Second Author\inst{2,3}\orcidID{1111-2222-3333-4444} \and
Third Author\inst{3}\orcidID{2222--3333-4444-5555}}
\authorrunning{F. Author et al.}
\institute{Princeton University, Princeton NJ 08544, USA \and
Springer Heidelberg, Tiergartenstr. 17, 69121 Heidelberg, Germany
\email{lncs@springer.com}\\
\url{http://www.springer.com/gp/computer-science/lncs} \and
ABC Institute, Rupert-Karls-University Heidelberg, Heidelberg, Germany\\
\email{\{abc,lncs\}@uni-heidelberg.de}}
\end{comment}
\maketitle

\begin{abstract}
This paper tackles the unsupervised depth estimation task in indoor environments. The task is extremely challenging because of the vast areas of non-texture regions in these scenes.
These areas could overwhelm the optimization process in the commonly used unsupervised depth estimation framework proposed for outdoor environments. 
However, even when those regions are masked out, the performance is still unsatisfactory.  
In this paper, we argue that the poor performance suffers from the non-discriminative point-based matching. To this end, we propose P$^2$Net. We first extract points with large local gradients and adopt patches centered at each point as its representation. Multiview consistency loss is then defined over patches. This operation significantly improves the robustness of the network training. 
Furthermore, because those textureless regions in indoor scenes (\eg, wall, floor, roof, \etc) usually correspond to planar regions, 
we propose to leverage superpixels as a plane prior. We enforce the predicted depth to be well fitted by a plane within each superpixel.
Extensive experiments on NYUv2 and ScanNet show that our P$^2$Net outperforms existing approaches by a large margin.

\keywords{Unsupervised Depth estimation, Patch-based Representation, Multiview Photometric Consistency, Piece-wise Planar Loss}
\newcommand\blfootnote[1]{\begingroup
    \renewcommand\thefootnote{}\footnote{#1}\addtocounter{footnote}{-1}\endgroup
}
\blfootnote{$*$~Equal Contribution}
\blfootnote{$\dag$~Corresponding author}
\end{abstract}


\section{Introduction}
\subfile{sections/intro}

\section{Related Work}
\subfile{sections/relatedwork}

\section{Method}
\subfile{sections/methods}

\section{Experiments}
\subfile{sections/exp}


\section{Conclusion}
\subfile{sections/conclusion}

\section*{Acknowledgements}
The work was supported by National Key R\&D Program of China (2018AAA0100704), NSFC \#61932020,
and ShanghaiTech-Megavii Joint Lab. We would also like to thank Junsheng Zhou from Tsinghua University for detailed commonts of reproducing his work and some helpful discussions.

\clearpage
\appendix
\begin{appendix}
\section{Surface normal visualization}
We provide more visualizations of surface normal prediction on the ScanNet~\cite{dai2017scannet} dataset. In our implementation, we directly fit the surface normal from ground truth depth annotation. Black pixels indicate invalid regions where no ground truth depths are provided. Compared to MovingIndoor~\cite{zhou2019moving}, our surface normal estimation better preserves the boundary of the planar regions, thanks to our superpixel constraint. 


\begin{figure}[!t]
    \centering
    \includegraphics[height=20cm]{supp_eps/supp_1.png}
    \caption{Visualization of surface normal results on the Scannet~\cite{dai2017scannet} dataset. From left to right: input RGB, MovingIndoor~\cite{zhou2019moving}, our results and surface normal fitted from ground truth depth. Black pixels in ground truth indicate invalid regions where no depth ground truth are provided.}
    \label{fig:scannet_norm}
\end{figure}



\section{Point cloud visualization}
We further provide some point cloud visualization on NYUv2~\cite{nyu} and ScanNet~\cite{dai2017scannet} dataset in Figure~\ref{fig:pointcloud}.

\begin{figure}[!htbp]
    \centering
    \includegraphics[height=13.5cm]{supp_eps/supp_3.png}
    \caption{Point cloud visualization. From left to right: input RGB from NYUv2, point cloud in 3D, RGB from ScanNet, point cloud in 3D.}
    \label{fig:pointcloud}
\end{figure}


\section{The effect of different patterns.} 
Here, we compare the effect of different patterns in our Patch-match module. We experiment with different $N$s and report the result in Table~\ref{tab:diffpattern}. Setting $N$ to 3 gives best results. On the contrast, keeping a larger pattern ($N=4$) might introduce additional noise, this would lead to a decay in performance.
\begin{table}[!h] \begin{center}
        \begin{tabular}{c|cc|ccc}
            \toprule[1pt]
            $N$  & rms $\downarrow$ &  rel $\downarrow$ & $ \delta < 1.25$ $\uparrow$  & $ \delta < 1.25^2$ $\uparrow$  & $ \delta < 1.25^3$ $\uparrow$ \\ \hline
            1 & 0.629 & 0.173 & 0.746 & 0.939 & 0.984  \\
            2 & 0.618 & 0.170 & 0.748 & 0.937 & 0.984 \\
            3 & \textbf{0.612} & \textbf{0.166} & \textbf{0.758} & \textbf{0.945} & \textbf{0.985} \\
            4 & 0.634 & 0.173 & 0.741 & 0.938 & 0.984 \\
            \toprule[1pt]
        \end{tabular}
    \end{center}
    \caption{Comparison between different patterns in our Patch-match module.}
    \label{tab:diffpattern}
\end{table}
\end{appendix}
\clearpage

\bibliographystyle{splncs04}
\bibliography{egbib}


\end{document}
