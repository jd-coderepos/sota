











\documentclass[11pt]{amsart}
\usepackage{fullpage} 
\usepackage{algorithm}
\usepackage{algorithmic}
\usepackage{graphicx}
\usepackage{amsfonts}
\usepackage{amsmath}
\usepackage{url}

\newcommand{\amenable}{amenable}
\newcounter{foo}
\newtheorem{theorem}[foo]{Theorem}
\newtheorem{lemma}[foo]{Lemma}
\newtheorem{corollary}[foo]{Corollary}
\newtheorem{observation}[foo]{Observation}
\newtheorem{defn}[foo]{Definition}

\newtheorem{proposition}{Proposition}[section]
\newtheorem{claim}[foo]{Claim}





\begin{document}




\title{\Large Wireless Connectivity and Capacity}
\author[M. Halld\'orsson]{Magn\'us M. Halld\'orsson}
\address[M. Halld\'orsson]{School of Computer Science\\
Reykjavik University\\
Reykjavik 101, Iceland}
\email{mmh@ru.is}

\author[P. Mitra]{Pradipta Mitra}
\address[P. Mitra]{School of Computer Science\\
Reykjavik University\\
Reykjavik 101, Iceland}
\email{ppmitra@gmail.com}
\date{}

\maketitle

 


\begin{abstract} \small\baselineskip=9pt
Given  wireless transceivers located in a plane, a fundamental
problem in wireless communications is to construct a strongly
connected digraph on them such that the constituent links can be
scheduled in fewest possible time slots, assuming the SINR model of
interference.

In this paper, we provide an algorithm that connects an arbitrary
point set in  slots, improving on the previous best bound
of  due to Moscibroda.  This is complemented with a
super-constant lower bound on our approach to connectivity.
An important feature is that the algorithms allow for bi-directional
(half-duplex) communication.


One implication of this result is an improved bound of  on the worst-case capacity of wireless networks, matching the best
bound known for the extensively studied average-case.

\iffalse
, measured as the sustained
rate of data aggregation at an information sink, when using a compressible
aggregation function. This matches the best bound known for
\emph{average-case} capacity, which has been extensively studied from
an information theory viewpoint.
\fi

We explore the utility of oblivious power assignments, and show that 
essentially all such assignments result in a worst case bound of
 slots for connectivity.
This rules out a recent claim of a  bound using oblivious power. 
On the other hand, using our result we show that 
 slots suffice, where  is
the ratio between the largest and the smallest links in a minimum spanning tree of the points. 

Our results extend to the related problem of minimum latency
aggregation scheduling, where we show that aggregation scheduling with
 latency is possible, improving upon the previous best
known latency of .
We also initiate the study of network design problems in the SINR
model beyond strong connectivity, obtaining similar bounds 
for biconnected and -edge connected structures.
\end{abstract}

 


\section{Introduction}
A key architectural goal in wireless adhoc networks is to ensure that
each node in the network can communicate with every other node
(perhaps by routing through other nodes). This requires that the nodes
be connected through a communication overlay.  The problem can be
abstracted as such: Given  points on the plane (each representing a
wireless node), how \emph{efficiently} can one ensure connectivity
among the points?

The notion of efficiency in a wireless setting is crucially dependent on that distinguishing feature
of wireless networks: interference. Two or more simultaneous communications in the
same wireless channel interfere with each other, potentially destroying all or some of the communications.
Thus, easy as it might be to come up with a set of links (a link is an directed edge between two nodes) that connect the  nodes, it is highly 
unclear whether or not one can schedule these links in a small amount of time. 
This fundamental problem has been
the focus of substantial amount of research \cite{MoWa06,moscibroda06b,Moscibroda07,kumar2005,Dousse03impactof}.

The model of interference is of course a crucial
aspect. Traditionally, all theoretical results have been in graph-based
models, with either fixed radii (unit-disc graphs and quasi-unit
disc graphs) and variable radii (geometric radio networks and
protocol model), 
while engineering research has focused on largely non-algorithmic 
studies in more complex models.
We adopt the SINR (signal to noise and interference ratio) model,
a.k.a.\ the physical model, of interference. The main differences are
two-fold: the received signal is a decaying function of distance (rather than
being on/off), and interferences from multiple transmitters sum up.
While more involved analytically, the SINR models is known to
be more realistic than graph-based ones, as shown theoretically as well as
experimentally~\cite{GronkMibiHoc01,MaheshwariJD08,Moscibroda2006Protocol}.

The first worst-case guarantee for wireless connectivity 
in the SINR model was provided by Moscibroda and
Wattenhofer \cite{MoWa06}, who showed how to construct a strongly
connected set of links that can 
be scheduled in  slots. This was improved to  in \cite{moscibroda06b} and finally 
to  by Moscibroda \cite{Moscibroda07}, which is the 
best bound currently known.




Our main result is the following: Any minimum spanning tree
(arbitrarily oriented) on  nodes on the plane can be scheduled in
 slots. This immediately leads to a  worst-case
bound for strong connectivity, by orienting the tree towards an arbitrary
root and then using the same tree with the orientation reversed.
Thus we improve the connectivity bound by a  factor, while 
giving at the same time a simple characterization of the resultant network in
terms of the natural MST structure.

The connectivity problem is closely related to the capacity of a
wireless network, a subject of a vast literature.
The computational throughput capacity of a network is the sustained
rate at which data can be aggregated to an information sink, which is
really the \emph{raison d'\^etre} of wireless sensor networks.  
At each time step, data is introduced at each source node.
If the aggregation function is compressible, like sum or max, 
only one item of data needs to be forwarded on each link.
A short schedule that is repeated as needed yields high throughput using buffering.  
Bounds for the connectivity problem lead therefore immediately to
equivalent bounds for worst-case capacity of wireless network (for
compressible functions) \cite{Moscibroda07}.

\iffalse
Apart from being the most basic and important network structure,
wireless connectivity immediately leads to interesting
applications. To highlight just one, connectivity structures provide
bounds for worst-case capacity in wireless sensor data gathering
applications. 
The computational throughput capacity of a network is the sustained
rate at which data can be aggregated to an information sink, which is
really the \emph{raison d'\^etre} of wireless sensor networks.  Bounds
for the the connectivity problem immediately lead to equivalent bounds
for the data aggregation problem (for compressible functions like sum
and max) \cite{Moscibroda07}.
\fi

Indeed, this particular application also 
highlights the specific benefits of adopting the SINR model.  The best
known bound on the average-case capacity in the SINR model is
, given in the influential work of Gupta and Kumar
\cite{Kumar00}.  On the other
hand, whereas the average case throughput capacity in the protocol model is
 \cite{Kumar00}, the worst-case capacity
is only  \cite{Moscibroda07}.

We also study a variation of the connectivity problem inspired by the
sensor networking application mentioned above is known as
\emph{minimum-latency aggregation scheduling}. In this variation, one
seeks a tree aggregating to a information sink (as before), but with
the additional requirement that links must be scheduled after all
links below them in the tree are scheduled.  
A straightforward modification of our algorithm achieves this in
optimal  slots, improving on the  result previously known \cite{Li:2010:MAS:1868521.1868581}.

We conjecture that a logarithmic bound is necessary for connectivity.
One reason is that it matches the average-case bound, which has been a
highly researched topic \cite{Kumar00}.
We also give a construction that shows that our approach cannot yield a constant upper bound.
It is distinguished from all previous lower bound constructions in the SINR model in that it is (necessarily) not based on showing that pairs of links are incompatible. Without being able to show the existence of large ``cliques'', hardness results in the SINR with power control are hard to come by.


An important -- and perhaps surprising -- feature of our method is that it allows for bidirectional
communication. Namely, the links scheduled in each slot can
communicate in either direction without affecting or being affected by 
the other scheduled links.
This is important in a communication
setting because of the need to supply acknowledgements and flow
control, and is sometimes viewed as indispensable.
The previously studied algorithms 
\cite{MoWa06,moscibroda06b,Moscibroda07} all assumed unidirectional
communication. In fact, it was taken for granted that
unidirectionality could not be avoided, sometimes with references to 
lower bounds from graph-based models \cite{moscibroda06b}. Our algorithm uses
different power for the two directions of each link; we show that
to be unavoidable by constructing instances for which 
the use of symmetric power on bidirectional links 
forces the use of  slots.

\iffalse
Another issue with structure design is simplicity and naturalness.
Previous algorithms mix all three tasks in the same loops: deciding on structure (i.e.,
choosing the links), scheduling the links, and assigning
powers to senders. We start with the perhaps the most natural and best
studied of structures, the minimum spanning tree (or, a local
approximation thereof). We have a simple rule for selecting links for
each slot, and finish by assigning powers with a single sequential
pass through the links in each slot. 
\fi

Power assignments are yet another important issue in wireless protocols.
It is preferable if power settings are locally computable.
A power assignment is \emph{oblivious} if it depends only on the
length of the respective link.
Recently, a -slot connectivity
algorithm was claimed that used a particular oblivious power
assignment \cite{DBLP:conf/wdag/KowalskiR10}.
Unfortunately, there are problems with the proof (specifically, in Lemma 5 whose proof is not in the conference version) as acknowledged by one of the authors \cite{Kowalski11}. 
Actually, that general approach is bound to fail;
namely, we show that essentially all oblivious power
assignments (including the one used in \cite{DBLP:conf/wdag/KowalskiR10})
require  slots in the worst case.

On the other hand, when the edge lengths in the MST differ by a factor
of at most , then combining the results here with recent work
\cite{SODA11} gives a  slot
connectivity algorithm that uses a certain oblivious power assignment
called \emph{mean power}.



We use our approach as a starting point for the first excursion into
network design problems beyond strong connectivity.
By applying the connectivity routine a constant number of times, we
find that we can solve other connectivity problems with asymptotically
the same number of slots, including biconnectivity and -edge
connectivity.


\paragraph{Outline of the paper.}
We introduce the SINR model and related notation in
Sec.~\ref{sec:model}, followed by quick overview of related work in Sec.~\ref{sec:related}.
The connectivity algorithm is given in Sec.~\ref{sec:conn},
with a subsection on a limitation result.
We extend the method to a bi-directional model
of communication in Sec.~\ref{sec:bidi}, and examine the power of oblivious
power in Sec.~\ref{sec:oblivious}.
Extensions to other connectivity problems are treated in Sec.~\ref{sec:design}.

\section{Model and Preliminaries.}
\label{sec:model}

Given is a set  of points on the Euclidean plane.
A link  is a directed edge from point  (the ``sender") to point  (the ``receiver").
The goal is compute a set of links that strongly connect  and to schedule them
in  slots.

The distance between two points  and  is denoted . The asymmetric distance from link  to link  is the distance from
's sender to 's receiver, denoted . The length of link  is denoted  simply .
For a link set , let  denote the ratio between the maximum and minimum length of a link in .

When a point  transmits as a sender of link , it uses some transmission power .
We adopt the \emph{physical model} (or \emph{SINR model})
of interference: a communication over a link  succeeds if and only if the
following condition holds:

where  is the path loss constant,  is a universal constant denoting the ambient noise,  denotes the minimum
SINR (signal-to-interference-noise-ratio) required for a message to be successfully received,
and  is the set of concurrently scheduled links in the same \emph{slot} with .
We say that  is \emph{SINR-feasible} (or simply \emph{feasible}) if (\ref{eq:sinr}) is
satisfied for each link in . 

We will use the notion of affectance of \cite{HW09}, as refined in \cite{KV10}, 
which is a scaled interference measure from one link on another, defined as 
   
where  is a constant
depending only on the length and power of the link . 
As in previous work \cite{MoWa06,moscibroda06b,Moscibroda07,KesselheimSoda11}, we assume that
powers can be scaled up as needed, which implies that the effect of the noise  (and the coefficient ) can be ignored.
It holds that  is feasible in  iff

where  is the set of simultaneously transmitting links. We will
sometimes use this version of the SINR constraint instead of Eqn. \ref{eq:sinr}.


For a set of points , we will use  to denote a minimum spanning tree over the points in .
We will simply use  when  is clear from the context. Naturally,  contains undirected edges,
but when scheduling directed links, we need to orient  in some way. When no ambiguity
arises, we will simply use  to describe a particular oriented version of .


\section{Related Work}
\label{sec:related}
Abstract problems capturing aspects of wireless networks have a long history, but the adoption of the SINR
model in theoretical analysis has been a comparatively recent phenomenon. The first rigorous worst case
results were achieved in the seminal work of Moscibroda and
Wattenhofer \cite{MoWa06} (which involved the problem studied in this paper). Ever since, numerous paper
have appeared on the SINR model. For a recent overview, see 
\cite{GoussevskaiaPW10}. Apart
from the connectivity, another fundamental problem is the capacity problem, where one wants to find
the maximum feasible subset of a given set of links. First rigorous results for the capacity problem 
were established in \cite{GHWW09}, followed
by a number of other results.
Kesselheim achieved a breakthrough recently by proving the first -approximation algorithm for capacity
with power control \cite{KesselheimSoda11}, whose techniques we adopt into our analysis. In this regard,
this work can be considered to bring the approaches to connectivity and capacity together. Other recent progresses
made include a -approximation capacity algorithm for oblivious powers \cite{SODA11}, and the study of topological properties of wireless communication maps \cite{stoc_topology11}.


\section{ connectivity in the SINR model}
\label{sec:conn}

The starting point of our analysis is a criteria for wireless capacity
recently developed by Kesselheim \cite{KesselheimSoda11}. Kesselheim
showed that any set of links for which this criteria holds (defined in
Eqn.~\ref{feasibilitycond} below) can be scheduled in a single slot,
and provided an efficient algorithm to do so. We shall 
call this algorithm \textbf{Schedule}, which is described in Section 3 of
\cite{KesselheimSoda11}. 
For reference, 
we also include the algorithm in Appendix \ref{app:a}.

Our approach is as follows. We show, via a related criteria, that
given any , Eqn.~\ref{feasibilitycond} holds for
a constant fraction of the links in . Thus, a constant fraction of the
tree can be scheduled in a single step (and this holds
recursively). Naturally this process will end in 
steps. Our analysis applies to any orientation of . Thus to achieve
a strongly connected network, we simple schedule two trees. One is a
copy of  oriented towards some arbitrary root, another one oriented
away from the same root. Thus any two nodes in the network can
communicate by first routing from the source to the root, and then
routing from the root to the destination.

Our goal is then to prove the following.
\begin{theorem}
Let  be any set of points on the Euclidean plane. Let  be a minimum spanning tree on the points of ,
arbitrarily oriented. Then algorithm {\bf Connect} schedules  in  slots.
\label{mainth1}
\end{theorem}

\begin{algorithm}                      \caption{Connect(An arbitrarily oriented MST  on point set )}          \label{alg1}                           \begin{algorithmic}[1]                    \STATE 
     \WHILE{}
     	\STATE Use Algorithm \textbf{Schedule} to find a feasible subset  \label{alg:findfeasibleset}
	\STATE 
     \ENDWHILE
\end{algorithmic}
\label{alg1fig}
\end{algorithm}

For two links , define . For links  define
the function . Let  and for 
let . The function  can be thought of as a measure of how badly the link  might affect link  if
they were to transmit simultaneously.

We call a set of links  \textbf{\amenable} if the following holds:
for any link  ( not necessarily a member of ), 

for some constant  to be chosen later. The concept of an amenable set is closely related to the following
theorem due to Kesselheim (the connection is made explicit in Lemma \ref{lem:constantsizeapprox}).

\begin{theorem}[\cite{KesselheimSoda11}]
Assume  is a set of links such that for all ,

for a constant .
Then,  is feasible and there exists a polynomial time algorithm to find a power assignment to schedule  in a single slot.

Moreover, for any given set  assume  is the largest feasible subset of . Then \textbf{Schedule} finds a  of size  for which Eqn.~\ref{feasibilitycond} holds.
\label{kessel}
\end{theorem}

\begin{lemma}
If a set  of size  is amenable, then there are  links in  that can be scheduled in a single slot.
\label{lem:constantsizeapprox}
\end{lemma}
\begin{proof}
Since  is amenable, then by definition . Rearranging,
we get . By an averaging argument, there
must be a set  of at least  links for which 

This is almost exactly Eqn.~\ref{feasibilitycond} except for the use
of a different constant. To achieve the correct 
constant, a simple sparsification suffices. Start an empty set.
Go through links in  in increasing order of length, putting the link in the first set in which Eqn.~\ref{feasibilitycond} holds.
Start a new set if necessary. Clearly, no more that  sets will be necessary.

Thus, a set of size  can be found for which Eqn.~\ref{feasibilitycond} holds.
\end{proof}

The most important step is to prove that  is amenable.
\begin{lemma}
Let  where  is a minimum spanning tree on the points in . Then  is amenable.
\label{lem:approx}
\end{lemma}
\begin{proof}
Consider any link  (not necessarily in ) and assume without of loss of generality that its length is . 
We can do this because scaling all links to make  of length  does not change the values of the function 
.
To prove
amenability, we thus have to only consider links in  of length at least . Let  be this set and let  be the points that are incident to at least one edge in .

First we claim,
\begin{lemma}
Any disc of radius  contains at most 9 points from . 
\label{pointsincircle}
\end{lemma}
\begin{proof}
Let  be a disc of radius  and 
let  be the set of points from  in .  


We first observe that no two points  have a common
neighbor in .
If  and  were neighbors in  then a common neighbor would imply a cycle, while if they were non-neighbors,
replacing either of the edges to the common neighbor by the edge  
results in a cheaper spanning tree (since ).
Since each point in  has a neighbor in , it holds that  (where  
denotes the neighborhood of a point set  in ).

Let  be the center of , and consider any pair of points . 
We aim to show that the angle , which implies the lemma.
Let  () be the unique neighbor of  () in .
We observe first that the unique path in  between 
and  goes through neither  nor , since if it did, say
through , then replacing  by  results in a 
smaller tree.  

Consider now the triangle .  Let , 
, , and denote
, for points  and .
Note that , as
we could otherwise delete the edge  and add  to get a
better tree. Similarly, .
From the triangular inequality, our observations above, and the
fact that , we have that

and similarly .
By the sine law, ,
and .
For , this implies that since , computation shows that  as claimed.
\end{proof}

Now for , 
Thus it suffices to upper bound  for any arbitrary point  by a constant
to get the required bound.

Now take concentric circles  around  such that the  
circle has radius .
The proof of the following fact
can be found in Appendix B. 
\begin{lemma}
 can be covered by  circles of radius 
 (where  is the constant from Lemma \ref{pointsincircle}). 
The annulus  can be covered by  circles of radius , for .
\label{ddim}
\end{lemma}
Thus by Lemma \ref{pointsincircle}, there are at most  points from  in . Similarly,
 can be covered by  circles of radius  (see Lemma \ref{ddim}) and thus contains  points from .

The distance to  from any point in  is at least . Then,

\iffalse

\fi
for . This completes the proof of Lemma \ref{lem:approx},
assuming that  is at least twice the implicit constant in
the bound above.
\end{proof}


Theorem \ref{mainth1} now follows easily. By Lemma \ref{lem:approx}, the remaining set of links at each step
of the algorithm is amenable. Thus, by Lemma
\ref{lem:constantsizeapprox}, a constant factor of these links are feasible, and by Theorem \ref{kessel}, a constant
factor of those will be scheduled by {\bf Connect}. Clearly, this process terminates in  steps.

\smallskip
\noindent
\textbf{Remark.} We note that the assumption  is necessary.
Indeed, suppose points are placed at all integer coordinates within a
large circle, so all links will be of at least unit length.
Then, when , it can be shown with standard methods that 
there is no feasible subset of links of size larger than .


\subsection{A lower bound on our approach}
It is easy to construct an example where a link in  violates
Eqn.~\ref{feasibilitycond} (below, the set  provides an example of that). However,
this still leaves open the possibility that the spanning tree can be partitioned into a small number
 of subsets (for example, a constant number of subsets) such that Eqn.~\ref{feasibilitycond} holds
for each of them, thus improving upon the  result. In the following theorem, we show that one
cannot partition all the points into a constant number of subsets. This, naturally, is not a lower bound
on the connectivity problem, just on our particular approach.

\begin{theorem}
For any number , there exists a set of points on the line such that the minimum spanning tree  cannot be 
partitioned into  sets  such that Eqn.~\ref{feasibilitycond} holds for each .
\end{theorem}
\begin{proof}
For , we will recursively construct gadgets  such that a spanning tree on  cannot be partitioned into 
sets for which Eqn.~\ref{feasibilitycond} holds. 

Since we are considering points on a line, the minimum spanning tree is simply
the edges connecting each point to its immediate neighbors to the right and left. Our theorem holds
for any orientation of the links.

A gadget  is simply a set of points located on a line, with an implicit ordering from the left to the right. We will often use  to mean a translated copy
of  as well, which will be clear from the context.
For two gadgets  and , we will use 
to denote the joining of the two gadgets, which is a new gadget with  points. The first (starting from the left)  points
are a copy of , and the last  points are a translated copy of . In other words, the  point is both
the ending point of the gadget  and the starting point for the copy of gadget . For any collection
of points (or gadget) , let  be the diameter of .

For a gadget , we use  to mean a copy of the gadget scaled by a factor of . For example,
if , then .


\begin{figure*}
\begin{center}
\includegraphics[height=2.5cm]{lbcons2}
\caption{Construction of  from copies of . We join a number of copies of , each succeeding copy
scaled must larger than all the copies before it combined. All this is preceded by a huge new link between points  and  ( is very large)}
\label{gtconsfig}
\end{center}
\end{figure*}


We are ready to describe our construction.  contains the points .
For a gadget , define  where
 is the maximum distance from either end point of  to the left most point in . 
It is easy to verify that .
Now  is constructed by joining copies of  (each copy scaled much larger than preceding copies) and preceding all of it with a new point, such that
the distance between the new point and the beginning of the copies of  is humongous. This
is informally depicted in Fig.~\ref{gtconsfig}. 

To formally define , we first define . The value of  is set to 
 where  is the constant from Eqn. \ref{feasibilitycond}. The scaling factors are defined as such: ,  (for ) is chosen such that
. Let  be  translated so that its left most point is at location . Define the number . Then , i.e., a single point at location , followed by the gadget . 

Define the partition number  as the minimum number of partitions of a link set required so that Eqn.~\ref{feasibilitycond}
holds for each set. We claim:
\begin{claim}
 and  for all .
\end{claim}
Combined, these two claims clearly prove the theorem. The claim about  is easy to verify by direct computation.
Let us prove the inductive step.
For  consider the left-most link  in . This is of course a link of length  which is by construction
the unique largest link in  (since ). 
Thus . 
We claim that
 for all . To see this, consider any . 
Now  is of course part of the  copy of  for some . 
Now let .
We can  observe that 
. By construction, 
 and thus 
 (by definition of )) proving the claim. 

Given this and noticing the value of  chosen in the construction of
, it is clear that there must be some  such that no link in 
is in the same set as . This completes the proof of the claim and the theorem.
\end{proof}


\section{Bi-directionality}
\label{sec:bidi}
We have worked with the uni-directional model of wireless communication so far, where the links are directed from
a sender to a receiver. The bi-directional model, in contrast, has
two-way half-duplex communication
between the nodes of a link in the same slot. The advantage of this model is that it simplifies one-hop communication protocols. Two-way communication in a single slot without worrying about mutual interference can be achieved in practice in more than one way,
and we simply take that as given. The difficulty arises in that interferences from other links are now potentially
much larger, since we have to take into account both directions of each link.

We can model the bi-directional case as follows:  contains
 pairs . These two points implicitly define two unidirectional links  and . Each pair can be associated with two power levels  and , to be used by 
 and  respectively. We consider a set of pairs  feasible if for all ,

or equivalently,

where affectances are as defined for the unidirectional case.

We differentiate two versions of the bi-directional model. In the \emph{symmetric} model, we insist that 
for each . With this restriction, the model is essentially equivalent to the one
introduced in \cite{FKRV09}.
Without such a restriction,
we call it the \emph{asymmetric} model, which was briefly mentioned in \cite{KesselheimSoda11}.

First we show,
\begin{theorem}
There is an instance that 
requires  slots for connectivity in the symmetric bi-directional model.
\label{bidirlb}
\end{theorem}
\begin{proof}
Consider the pointset  given by
 and for , .
Observe that .
Consider any two pairs  and ,
and assume without loss of generality that .
Thus, there must be indices 
such that  and .

On the other hand  and .

Then,
 
Thus, any pair of links must be scheduled in different slots.
\end{proof}


In surprising contrast to the above strong lower bound in the symmetric model,
\begin{theorem}
In the asymmetric bi-directional model, any set of  points
can be strongly connected in  slots.
\end{theorem}

The argument in Section \ref{sec:conn} is as follows. First, we show that  is amenable. Then
we find a large subset for which Eqn \ref{feasibilitycond} holds, and finally, we schedule it in one slot.
The main difference in the bi-directional case is that we have to choose pairs in a feasible set, i.e., for any
pair  that we want to connect, we have to include  and  in the \emph{same} slot.
Note that since,  and  have no effect on each other, we can define  for
. The new definition of amenability is thus,


for a constant .

We first need to verify that Lemma \ref{lem:approx} still holds with the new definition. This happens to be easy. Indeed, the proof of Lemma \ref{lem:approx} does not use the dichotomy between sender and receiver, and thus automatically holds (up to a factor of 4). It is easy to see that the argument in Lemma \ref{lem:constantsizeapprox} continues to hold with minor differences.

Finally, we need to show that Thm. \ref{kessel} still holds with the new definition, i.e., the algorithm \textbf{Schedule} can still successfully find
and schedule the link set thus selected. The algorithm \textbf{Schedule} is robust in relation this, as \cite{KesselheimSoda11} points out. 

More specifically, to show that \textbf{Schedule} works for the bi-directional variant, we need the following version of 
Thm. \ref{kessel}:

\begin{proposition}
Assume  is a set of pairs such that for all ,

for a constant .
Then,  is feasible and there exists a polynomial time algorithm to find a power assignment to schedule  in a single slot.

Moreover, for any given set  assume  is the largest feasible subset of . Then \textbf{Schedule} finds a  of size  for which Eqn.~\ref{feasibilitycond2} holds.
\label{kessel2}
\end{proposition}


The last part about finding a large feasible subset is an implication of arguments like Lemmas \ref{lem:approx} and \ref{lem:constantsizeapprox}, which we already verified to be sound in the new regime.

For the first part of the algorithm, Eqn. \ref{feasibilitycond2} is
identical to Eqn. \ref{feasibilitycond} if we assume the link set to
be  except one caveat. For a
given , Eqn. \ref{feasibilitycond} includes the term
 for the two links of the same pair, and
Eqn. \ref{feasibilitycond2} doesn't (or rather  is
set to zero). However, this is not a problem, since we assume that
 and  do not interfere with each other. In relation to
all other pairs, Eqn. \ref{feasibilitycond2} is identical to
Eqn. \ref{feasibilitycond} and thus the argument is identical.  



\section{Oblivious Power Assignments}
\label{sec:oblivious}


We examine here the complexity of connectivity when using simple power
assignments. A power assignment is said
to be \emph{oblivious} if it depends only on the length of the link.
We show that any reasonable oblivious power assignment is ineffective
in that it requires  slots to connect some instance of  points.
On the other hand, we also find that if the diversity of the edge lengths in the MST is
small, then they can be quite effective. 



Moscibroda and Wattenhofer \cite{MoWa06} showed that for both uniform power (all links use the same power)
and linear power (), there are pointsets for which connectivity requires
 slots.  It is easy to verify that their construction
applies also to functions that grow slower than uniform (i.e., are decreasing)
or faster than linear.
We address here essentially all other reasonable
oblivious assignments, namely those that are monotone increasing but
grow slower than linear.  

We call a power function  
\emph{smooth} if  for all ,  when , and ,
and   defined by  is monotone
increasing and . This is true for mean power () and many similar power assignments (such as the
one used in \cite{DBLP:conf/wdag/KowalskiR10}, ).

\begin{lemma}
Let  be a set of points on the line such that , 
the minimum distance between any pair of points is , and 
, for each . 
Then, no two links between points in  can be scheduled simultaneously using power assignment .
\label{lem:incr1}
\end{lemma} 


\begin{proof}
Consider two links  and , where 
without loss of generality  and 
 is the sender of .  
We may assume that  and , since a point
cannot be involved in two transmissions simultaneously, if the signal
requirement .  The power is  on link  and  on link .

First, consider the case where  is the receiver of , . 
The affectance of  on  is

Explanations:
\begin{enumerate}
\item By sublinearity, , 
  and by monotonicity, .
\item Because .
\item Since .
\end{enumerate}
Thus these two links cannot be scheduled together.

Second, consider the case where  is the sender of , .  Let  denote .
The affectance of  on  is now

Explanations:
\begin{enumerate}
\item Because , since .
\item Since .
\item Since , by assumption.   \end{enumerate}
\vspace{-2em}
\end{proof}

We shall argue the lower bound for a more general class of structures
(similar to \cite{MoWa06}).
We say that a structure (set of links) on a pointset has property
 if each point is either a sender or receiver on at least
one link.

Since  is monotone increasing and eventually infinite, it has an
inverse .
We construct  points  on the line defined by
,  and .
The following result is now immediate from Lemma \ref{lem:incr1}.

\begin{theorem}
For any structure with property 
and any smooth oblivious power assignment,
there is an instance that requires  slots.
\end{theorem}

In spite of this highly negative statement, we do find that oblivious
power assignments are quite effective given some natural assumptions
about edge length distributions. 


\subsubsection*{Upper bounds}

Let  be a MST of the given pointset.  Let  denote the ratio
between longest to shortest edge length in .  Assume, by scaling,
that  for all . Let  denote the
\emph{length diversity} of the link set , or the number of length
groups. Note that .

\begin{theorem}
Any pointset can be strongly connected in 
slots using uniform (or linear) power assignment.
This is achieved on an orientation of the minimum spanning tree.
\label{logdelta}
\end{theorem}

\begin{proof}
Divide the links of the spanning tree into at most  length
classes, where links in the same class differ in length by a factor at
most 2. Consider one such color class .
Let  be the endpoints of links in  and let  be length of
the shortest link.
Consider an endpoint  of a link in .
By Lemma \ref{pointsincircle}, at most 9 points from  are within a
distance  from any point. Note that any radius- circle can be
covered with at most  radius- circles.
Thus, for any , there are at most  points
from  within a distance  from .
The links in  can then colored with  colors so that senders of any pair of
links are of distance at least . 
If , where  is the
Riemann function and , then it follows from Lemma 3.1 of
\cite{us:esa09full} that each colorset forms a feasible set using
uniform power.
The total number of slots used is then .
\end{proof}


This bound improves on a bound of  given by Moscibroda
and Wattenhofer \cite{MoWa06}. The construction in \cite{MoWa06} shows
also that the bound is best possible for uniform and linear power.

Exponentially weaker dependence on  can be achieved by using
mean power (the power is set proportional to the length to the power of ).

\begin{theorem}
Any pointset can be strongly connected in  slots using mean power assignment.
\label{loglogdelta}
\end{theorem}


\begin{proof}
The capacity of a linkset is the maximum number of links that can be
scheduled simultaneously.  Our main result is that any orientation of
the MST  yields a directed linkset with linear capacity:  
links can be scheduled in a single slot. 
A recent result \cite{SODA11} shows that for any linkset, the
optimal capacity with power control differs from optimal capacity with
mean power by a factor of .
Further, a constant approximation algorithm for mean power capacity is
given in \cite{SODA11}. That algorithm then schedules  links from  in a single slot. 
In  slots it will then have
scheduled all of .
\end{proof}


Note that the construction of Lemma \ref{lem:incr1} yields a lower
bound of  for mean power.

\section{Extensions to other connectivity problems}
\label{sec:design}

\subsection{Minimum-latency aggregation scheduling}

Recall the problem definition.  An \emph{in-arborescence}  is a
directed rooted tree that has a path from every node to the root. An
edge  in  is said to be a \emph{descendant} of edge  if
there is a directed path starting with  that includes .
Given a set of  points  on the plane, 
the MLAS problem is to
find  ordered disjoint linksets   such that
each  is feasible, the links in  form a spanning in-arborescence , and whenever  is a descendant of  then . Let us call this last condition the 
\emph{ordering requirement}.



Consider the following iterative algorithm. 
Let .
In step  the algorithm finds a feasible linkset  on  and derives a new pointset , repeating the process until  contains only a single node.
Given , we form the nearest-neighbor forest , where each node  provides the link  to its nearest point ; whenever links
whenever  contains a pair  and , we remove one of the two links. This forest  is a subset of some minimum spanning tree of , and therefore it is amenable by Lemma \ref{lem:approx}. 
Thus, we can find a feasible set  with 
using \textbf{Schedule}.
This set  is necessarily a (partial) matching on .
We form  by removing from  the tails of all the links in .

We first show that this algorithm uses  steps, which follows immediately from the following Lemma.

\begin{lemma}
, for some .
\end{lemma}
\begin{proof}
The forest  contains at least  edges.
By Theorem \ref{kessel}, \textbf{Schedule} finds a feasible matching  of size at least , for some .
Then, .
\end{proof}

We also need to show that the resulting link set forms an in-arborescence and that it satisfies the ordering requirement.
Both of these are easily verified.


Also, it can be easily verified that any aggregation tree satisfying the ordering requirement requires a schedule of length at least . 


Thus we get the following result.
\begin{theorem}
Given any set of  points on the plane a aggregation tree can be formed with  latency, and this is optimal.
\end{theorem}



\subsection{Biconnectivity and -edge connectivity}

We can use our basic connectivity method to achieve additional network
design criteria. 
As a warmup, we first show how to achieve biconnectivity at minimal extra cost.
A graph is biconnected if there are at least two vertex-disjoint paths between
any pair of vertices.
\begin{theorem}
Let  be any set of points on the Euclidean plane. Then  can be
strongly biconnected in  slots.
\label{mainth3}
\end{theorem}

To see this, take the minimum spanning tree  used for
Thm.~\ref{mainth1}. Let  be the set of degree-1 nodes in  and
form a minimum spanning tree  of . Apply the algorithm \textbf{Connect}
to the union of 
and , directed in both ways. Between any pair of nodes is a path in
, all of whose internal nodes are in , and a path in ,
with all its internal nodes in .

\iffalse To prove this, take the minimum spanning tree  used for
Thm.~\ref{mainth1}. Let  be its root, and let 
be the depth of  in  (when rooted at ). Let  and . Now we can schedule MSTs 
and  on each of them in  slots, each rooted at
. It is clear that if we remove any point  the union
of  still remains connected. However, 
itself is still an articulation point. To avoid this problem we can
orient  towards some other root, and perform the same process, thus
assuring that the removal of  is not problematic.
\fi

\smallskip

A directed graph is -edge strongly connected if the graph stays strongly connected after the removal of less than -edges.
Here we prove:
\begin{theorem}
Let  be any set of points on the Euclidean plane. Then  can be -edge strongly connected in 
slots.
\label{mainth2}
\end{theorem}
\begin{proof}{[Outline]}
The algorithm is as follows. We repeatedly compute  spanning trees
. Here,  is a minimum spanning tree, and
for ,  is a minimum spanning tree that does not use any
edge from . Once we schedule these trees in two
orientations, the resultant structure is clearly -edge strongly
connected.

We then claim that each  can be scheduled in 
slots from which the theorem follows. Proving that  can be
scheduled in  boils down to proving a version of Lemma
\ref{pointsincircle} for , 
given below.
The rest follows in a routine fashion.
\end{proof}



\begin{lemma}
Any disc of radius  contains at most  points from , where  is the set of points incident
to a link of length at least  in . 
\label{pointsincircle-kmst}
\end{lemma}
\begin{proof}
Consider  for  (we already have the bound for ). As before, let 
be the set of neighbors of  in .

Define .

\begin{lemma}
Let  with the following property:
There exist  such that ,
and .
Then .
\label{largeangle2}
\end{lemma}
The proof of this claim  is essentially identical to the same argument
in Lemma \ref{pointsincircle}.  The fact that  and
 simply mean that the links  and
 can be used in the argument as they are not ruled out by
being included in an earlier tree.

Assume from now on that  for .
We shall show that there exists then a set  of 10 points all of
whose pairs satisfy the conditions of Lemma \ref{largeangle2}, which
leads to a contradiction. 
Let  denote the subgraph of  induced by pointset .

We first argue that . More strongly,
we claim that no point in  has more than  neighbors in
. Suppose point  has a set  of 
neighbors in .  Since  is a union of  trees,  contains
at most  edges, which is strictly smaller than
, as . Thus there is a pair  that is non-adjacent each of the previous trees, in which case
we can argue as in Lemma \ref{pointsincircle} and claim that we can
delete  and add  to get a better tree.



The following is a general claim about points in relation to spanning trees.
\begin{claim}
For any set  of points,  contains
an independent set of size  in .
\label{largeis}
\end{claim}
\begin{proof}
Since  is a union of  trees, the average degree of any induced
subgraph is less than . The claim then follows from Tur\'an bound.
\end{proof}

Now, by Observation \ref{largeis},
there is an independent set  in  of size at least
.

If some ten points in  share a common neighbor in , then we
are done. Otherwise, there is a subset  of  of size at least 
 such that no two share the same neighbor in . 
Let,  be the neighbors of  in . By Observation \ref{largeis}, 
we can find a subset  of
size at least  which is independent in .
Since no two points in  share neighbors in , . Setting , we find that  contains at least 10 points all of whose
pairs satisfy the conditions of Lemma \ref{largeangle2}, which is a contradiction.
Hence, .
\end{proof}



\section{Conclusion}

We have shown that there the links of a minimum spanning tree of any pointset can be be scheduled in  slots in the SINR model. An open question is whether this is optimal; we conjecture that it is.
Another direction would be to derive effective distributed algorithms.

\bibliographystyle{plain}
\bibliography{references}		

\appendix


\section{The algorithm Schedule}
\label{app:a}
We include, as a reference, the algorithm \textbf{Schedule} due to Kesselheim \cite{KesselheimSoda11}.

\begin{algorithm}                      \caption{Schedule (Set  of  links)}          \label{alg2}                           \begin{algorithmic}[1]                    \STATE Sort links in increasing order of length , breaking ties arbitrarily
     \STATE 
     \FOR{ to } 
     \IF{} 
     \STATE 
      \ENDIF
     \ENDFOR

     \STATE Now schedule  by finding power assignment for all links in :
     \STATE 
     \FOR{ to } 
     \STATE  where  is the sender of  and  is the receiver of 
     \ENDFOR
     \STATE Scale powers to take care of noise.
\end{algorithmic}
\label{alg2fig}
\end{algorithm}

\section{Proof of Lemma \ref{ddim}: Covering by circles}
\label{app:b}
\noindent \textbf{Lemma \ref{ddim}:} \emph{
 can be covered by  circles of radius 
 (where  is the constant from Lemma \ref{pointsincircle}). 
The area of the annulus  can be covered by  circles of radius , for .
}

\begin{proof}
The first claim follows directly from the fact that the -dimensional space has a finite doubling dimension.
Namely, each unit circle can be covered by  radius- circles.
Thus it suffices to prove that  can be covered
by  unit circles.

Consider now the circle  concentric with  with
radius , i.e., in the middle of  and . 
The circumference of this circle is clearly contained in
. Now, place  equidistance
points  on this circle. Since the circumference of  is 
, the distance between consecutive points is . 
Now we
claim that all points in  are within a distance
 of a point in , thus proving that the unit circles
around points in  cover the whole annulus.

Let  be any point in . Consider the line
connecting this point to the center of .  Assume this line
intersects  at point . Now clearly . On the
other hand, there exists a  such that . By the triangle inequality , completing the
proof.
\end{proof}



\end{document}
