\documentclass{article}


\usepackage{arxiv}

\usepackage[utf8]{inputenc} \usepackage[T1]{fontenc}    \usepackage{hyperref}       \usepackage{url}            \usepackage{booktabs}       \usepackage{amsfonts}       \usepackage{nicefrac}       \usepackage{microtype}      \usepackage{lipsum}
\usepackage{soul}
\usepackage{tcolorbox}
\usepackage{times}
\usepackage{latexsym}
\usepackage{times}  \usepackage{graphicx}  \usepackage{epsfig}
\usepackage{multirow}
\usepackage{booktabs}
\usepackage{amssymb,amsmath}
\usepackage{epsfig}
\usepackage{algorithm,algorithmic}
\usepackage{xcolor,colortbl}
\usepackage{booktabs}
\usepackage{xcolor,colortbl}
\usepackage{makecell}
\usepackage{subcaption}
\usepackage{lipsum}
\usepackage{balance}
\frenchspacing  \setlength{\pdfpagewidth}{8.5in}  \setlength{\pdfpageheight}{11in}  \newcommand{\eqdef}{\overset{def}{=}}
\newcommand{\eqrank}{\overset{rank}{=}}


\newcommand\BibTeX{B{\sc ib}\TeX}

\title{An Editorial Network for Enhanced Document Summarization}

\author{
  Edward Moroshko\thanks{Work was done during a summer internship in IBM Research.} \\
  Electrical Engineering Dept.\\
  Technion -- Israel Institute of Technology\\
  Haifa, Israel \\
  \texttt{edward.moroshko@gmail.com} \\
   \And
  Guy Feigenblat, Haggai Roitman, David Konopnicki \\
  IBM Research\\
  Haifa University Campus\\
  Haifa, Israel \\
  \texttt{\{guyf,haggai,davidko\}@il.ibm.com} \\
  }

\date{}

\begin{document}
\maketitle
\begin{abstract}
We suggest a new idea of \textit{Editorial Network} -- a mixed extractive-abstractive summarization approach, which is applied as a post-processing step over a given sequence of extracted sentences. Our network tries to imitate the decision process of a human editor during summarization. Within such a process, each extracted sentence may be either kept untouched, rephrased or completely rejected. We further suggest an effective way for training the ``editor" based on a novel soft-labeling approach. Using the CNN/DailyMail dataset we demonstrate the effectiveness of our approach compared to state-of-the-art extractive-only or abstractive-only baseline methods.
\end{abstract}

\section{Introduction}\label{sec:intro}
Automatic text summarizers condense a given piece of text into a shorter version (the summary). This is done while trying to preserve the main essence of the original text and keeping the generated summary as readable as possible.

Existing summarization methods can be classified into two main types, either \textit{extractive} or \textit{abstractive}~\cite{Gambhir:2017}. Extractive methods select and order text fragments (e.g., sentences) from the original text source~\cite{Cheng2016ACL,Dlikman2016UsingML,DBLP:conf/emnlp/DongSCHC18,summit2017,Nallapati2017a,ZhangLatent2018}. Such methods are relatively simpler to develop and keep the extracted fragments untouched, allowing to preserve important parts, e.g., keyphrases, facts, opinions, etc. Yet, extractive summaries tend to be less fluent, coherent and readable and may include superfluous text.

Abstractive methods apply natural language paraphrasing and/or compression on a given text. A common approach is based on the encoder-decoder (seq-to-seq) paradigm~\cite{Sutskever2014Seq}, with the original text sequence being encoded while the summary is the decoded sequence. While such methods usually generate summaries with better readability, their quality declines over longer textual inputs, which may lead to higher redundancy~\cite{Paulus2017ADR}. Moreover, such methods are sensitive to vocabulary size, making them more difficult to train and generalize~\cite{See2017GetTT}.

A common approach for handling long text sequences in abstractive settings is through \textit{attention} mechanisms, which aim to imitate the attentive reading behaviour of humans~\cite{Chopra2016Abs}.
Two main types of attention methods may be utilized, either \textit{soft} or \textit{hard}. Soft attention methods first locate salient text regions within the input text and then bias the abstraction process to prefer such regions during decoding~\cite{Cohan2018ADA,Gehrmann2018BottomUpAS,HsuAUM2018,DBLP:conf/conll/NallapatiZSGX16,Li2018KIGN,Pasunuru2018a,Tan2017a}. On the other hand, hard attention methods perform abstraction only on text regions that were initially selected by some extraction process~\cite{ChenFastAS2018,Nallapati2017a,liu2018a}.

Compared to previous works, whose final summary is either entirely extracted or generated using an abstractive process, in this work, we suggest a new idea of  ``\textit{Editorial Network}" (EditNet) -- a \textit{mixed extractive-abstractive} summarization approach.  A summary generated by \textit{EditNet} may include sentences that were either extracted, abstracted or of both types. Moreover, per considered sentence, \textit{EditNet} may decide not to take either of these decisions and completely reject the sentence. 

Using the CNN/DailyMail dataset we demonstrate that, \textit{EditNet}'s summarization quality transcends that of state-of-the-art abstractive-only baselines. \textit{EditNet}'s summarization quality is also demonstrated to be highly competitive with that of \textit{NeuSum}~\cite{Qingyu2018}, which is, to the best of our knowledge, the best performing extractive-only baseline. Yet, while \textit{EditNet} obtains (more or less) a similar summarization quality to that of \textit{NeuSum}, compared to the latter which applies only extraction, the former (on average) applies abstraction to the majority of each summary's extracted sentences.



\section{Editorial Network}
\begin{figure*}[t!]
  \centering
  \includegraphics[width=4.00in]{architecture-short.eps}
  \caption{Editorial Network}\label{fig:architecture}
\end{figure*}

Figure~\ref{fig:architecture} now depicts the architecture of our proposed \textit{Editorial Network}-based approach.
We apply this approach as a post-processing step over a given summary whose sentences were selected by some extractor. The key idea is to try to imitate the decision process of a human editor who needs to edit the summary so as to enhance its quality.

Let  denote a summary which was extracted from a given text (document) .
The editorial process is implemented by iterating over sentences in  according to the selection order of the extractor.
For each sentence in , the ``editor" may make three possible decisions. The first decision is to keep the extracted sentence untouched (represented by label \textsf{E} in Figure~\ref{fig:architecture}). 
The second alternative is to rephrase the sentence (represented by label \textsf{A} in Figure~\ref{fig:architecture}). Such a decision, for example, may represent the editor's wish to simplify or compress the original source sentence. The last possible decision is to completely reject the sentence (represented by label \textsf{R} in Figure~\ref{fig:architecture}). For example, the editor may wish to ignore a superfluous or duplicate information expressed in the current sentence.
An example mixed summary generated by our approach is depicted in Figure~\ref{example summary}, further emphasizing the various editor's decisions.

\begin{figure}[tbh]
\begin{tcolorbox}[colback = white,boxrule=0.5pt]
  \textbf{Editor's automatic summary}:\\
  \textsf{E}: what was supposed to be a fantasy sports car ride at walt disney world speedway turned deadly when a lamborghini crashed into a guardrail.
  \textsf{A}: \textit{the crash took place sunday at the exotic driving experience}\footnote{Original extracted sentence: ``the crash took place sunday at the exotic driving experience , which bills itself as a chance to drive your dream car on a racetrack".}.
  \textsf{A}: \textit{the lamborghini 's passenger , gary terry , died at the scene}\footnote{Original extracted sentence: ``the lamborghini 's passenger , 36-year-old gary terry of davenport , florida , died at the scene , florida highway patrol said"}.
  \textsf{R}: \st{petty holdings , which operates the exotic driving experience at walt disney world speedway , released a statement sunday night about the crash.}
\end{tcolorbox}
\begin{tcolorbox}[colback = white,boxrule=0.5pt]
  \textbf{Ground truth summary}:\\
  the crash occurred at the exotic driving experience at walt disney world speedway. officials say the driver , 24-year-old tavon watson , lost control of a lamborghini. passenger gary terry , 36 , died at the scene.
\end{tcolorbox}\caption{An example mixed summary (annotated with the editor's decisions) taken from the CNN/DM dataset}\label{example summary}
\end{figure}



\subsection{Implementing the editor's decisions}
For a given sentence , we now denote by  and  its original (extracted) and paraphrased (abstracted) versions.
To obtain  we use an abstractor, whose details will be shortly explained (see Section~\ref{sec:EA sum}). Let  and  further denote the corresponding sentence representations of  and , respectively. Such representations allow to compare both sentence versions on the same grounds.

Recall that, for each sentence  (in order) the editor makes one of the three possible decisions: extract, abstract or reject .
Therefore, the editor may modify summary  by paraphrasing or rejecting some of its sentences, resulting in a mixed extractive-abstractive summary .

Let  be the number of sentences in . In each step , in order to make an educated decision, the editor considers both sentence representations  and  as its input, together with two additional auxiliary representations. The first auxiliary representation is that of the whole document  itself, hereinafter denoted . Such a representation provides a \textit{global context} for decision making. Assuming document  has  sentences, let . Following~\cite{ChenFastAS2018,Wu2018LearningTE},  is then calculated as follows:


where  and  are learnable parameters.

The second auxiliary representation is that of the summary that was generated by the editor so far, denoted at step  as , with . Such a representation provides a \textit{local context} for decision making. Given the four representations as an input, the editor's decision for sentence  is implemented using two fully-connected layers, as follows:


where  denotes the vectors concatenation, , ,  and  are learnable parameters.

In each step , therefore, the editor chooses the action  with the highest likelihood (according to Eq.~\ref{eq:action likelihood}), further denoted . Upon decision, in case it is either \textsf{E} or \textsf{A}, the editor appends the corresponding sentence version (i.e., either  or ) to ; otherwise, the decision is \textsf{R} and sentence  is discarded. Depending on its decision, the current summary representation is further updated as follows:


where  are learnable parameters,  is the summary representation from the previous decision step; and , depending on which decision is made.


Such a network architecture allows to capture various complex interactions between the different inputs. 
For example, the network may learn that given the global context, one of the sentence versions may allow to produce a summary with a better coverage. As another example, based on the interaction between both sentence versions with either of the local or global contexts (and possibly among the last two), the network may learn that both sentence versions may only add superfluous or redundant information to the summary, and therefore, decide to reject both.

\subsection{Extractor and Abstractor}\label{sec:EA sum}
As a proof of concept, in this work, we utilize the extractor and abstractor that were previously used in~\cite{ChenFastAS2018}, with a slight modification to the latter, motivated by its specific usage within our approach. 
We now only highlight important aspects of these two sub-components and kindly refer the reader to~\cite{ChenFastAS2018} for the full implementation details.

The extractor of~\cite{ChenFastAS2018} consists of two main sub-components. The first is an \textit{encoder} which encodes each sentence  into  using an hierarchical representation\footnote{\small Such a representation is basically a combination of a temporal convolutional model followed by a biLSTM encoder.}. The second is a \textit{sentence selector} using a \textit{Pointer-Network}~\cite{vinyals2015pointer}. For the latter, let  be the selection likelihood of sentence .

The abstractor of~\cite{ChenFastAS2018} is basically a standard encoder-aligner-decoder with a copy mechanism~\cite{See2017GetTT}. Yet, instead of applying it directly only on a single given extracted sentence , we apply it on a ``chunk" of three consecutive sentences\footnote{\small The first and last chunks would only have two consecutive sentences.}  , where  and  denote the sentence that precedes and succeeds  in , respectively. This in turn, allows to generate an abstractive version of  (i.e., ) that benefits from a wider local context. Inspired by previous soft-attention methods, we further utilize the extractor's sentence selection likelihoods  for enhancing the abstractor's attention mechanism, as follows. Let  denote the abstractor's original attention value of a given word  occurring in ; we then recalculate this value to be , with  and ;  denotes the normalization term.




\subsection{Sentence representation}
Recall that, in order to compare  with , we need to represent both sentence versions on as similar grounds as possible.
To achieve that, we first replace  with  within the original document . By doing so, we basically treat sentence  as if it was an ordinary sentence within , where the rest of the document remains untouched. We then obtain 's representation by encoding it using the extractor's encoder in a similar way in which sentence  was originally supposed to be encoded.
This results in a representation  that provides a comparable alternative to , whose encoding is expected to be effected by similar contextual grounds.


\subsection{Network training}\label{sec:network_train}
We conclude this section with the description of how we train the editor using a novel soft labeling approach.
Given text  (with  extracted sentences), let  denote its editing decisions sequence. We define the following ``soft" cross-entropy loss:


where, for a given sentence ,  denotes its soft-label for decision.  

We next explain how each soft-label  is estimated. 
To this end, we utilize a given summary quality metric  which can be used to evaluate the quality of any given summary  
(e.g., ROUGE~\cite{lin2004rouge}). Overall, for a given text input  with  sentences, there are  possible summaries  to consider. Let  denote the best decision sequence which results in the summary which maximizes . For , let  denote the average  value obtained by decision sequences that start with the prefix . Based on , the soft label  is then calculated\footnote{\small For  we have: .} as follows:




\section{Evaluation}
\subsection{Dataset and Setup}
We trained, validated and tested our approach using the non-annonymized version of the CNN/DailyMail dataset~\cite{Hermann:2015:TMR:2969239.2969428}. Following~\cite{DBLP:conf/conll/NallapatiZSGX16}, we used the story highlights associated with each article as its ground truth summary. We further used the F-measure versions of ROUGE-1 (R-1), ROUGE-2 (R-2) and ROUGE-L (R-L) as our evaluation metrics~\cite{lin2004rouge}.

The extractor and abstractor were trained similarly to~\cite{ChenFastAS2018} (including the same hyperparameters).
The Editorial Network (hereinafter denoted \textit{EditNet}) was trained according to Section~\ref{sec:network_train}, using the ADAM optimizer with a learning rate of  and a batch size of . To speedup the training time, we precalculated the soft labels (see Eq.~\ref{eq:soft label}).
Following~\cite{DBLP:conf/emnlp/DongSCHC18,Wu2018LearningTE}, we set the reward metric to be ; with ,  and , which were further suggested by~\cite{Wu2018LearningTE}.

\begin{table}[tb!]
\centering
\caption{Quality evaluation using ROUGE F-measure (ROUGE-1, ROUGE-2, ROUGE-L) on CNN/DailyMail non-annonymized dataset}
\begin{tabular}{|l|c|c|c|}
\hline
& R-1 & R-2 & R-L \\ \hline
\multicolumn{4}{|c|}{\textbf{Extractive}}                                                                   \\ \hline
\begin{tabular}[c]{@{}l@{}}Lead-3  \end{tabular}  & 40.00   & 17.50   & 36.20   \\
\begin{tabular}[c]{@{}l@{}}SummaRuNNer \scriptsize {\cite{Nallapati2017a}} \end{tabular}  & 39.60   & 16.20   & 35.30   \\
\begin{tabular}[c]{@{}l@{}}Refresh \scriptsize {\cite{DBLP:conf/naacl/NarayanCL18}} \end{tabular} & 40.00    & 18.20    & 36.60    \\
\begin{tabular}[c]{@{}l@{}}Rnes w/o coherence \scriptsize {\cite{DBLP:conf/aaai/WuH18}} \end{tabular}             & 41.25   & 18.87   & 37.75   \\
\begin{tabular}[c]{@{}l@{}}BanditSum \scriptsize {\cite{DBLP:conf/emnlp/DongSCHC18}} \end{tabular}             & 41.50   & 18.70   & 37.60   \\
\begin{tabular}[c]{@{}l@{}}Latent \scriptsize {\cite{ZhangLatent2018}}\end{tabular} & 41.05   & 18.77   & 37.54   \\
\begin{tabular}[c]{@{}l@{}}rnn-ext+RL \scriptsize {\cite{ChenFastAS2018}} \end{tabular}  & 41.47   & 18.72   & 37.76   \\
\begin{tabular}[c]{@{}l@{}}NeuSum \scriptsize {\cite{Qingyu2018}} \end{tabular}  & 41.59   & 19.01   & 37.98   \\\hline
\multicolumn{4}{|c|}{\textbf{Abstractive}}                                                                  \\ \hline
\begin{tabular}[c]{@{}l@{}}Pointer-Generator \scriptsize {\cite{See2017GetTT}} \end{tabular}   & 39.53   & 17.28   & 36.38   \\
\begin{tabular}[c]{@{}l@{}}KIGN+Prediction-guide \scriptsize {\cite{Li2018KIGN}} \end{tabular}   & 38.95   & 17.12   & 35.68   \\
\begin{tabular}[c]{@{}l@{}}Multi-Task(EG+QG) \scriptsize {\cite{GuoSoftLayer2018}} \end{tabular}   & 39.81   & 17.64   & 36.54   \\
\begin{tabular}[c]{@{}l@{}}RL+pg+cbdec \scriptsize {\cite{JiangClosedBook2018}} \end{tabular}   & 40.66   & 17.87   & 37.06   \\
\begin{tabular}[c]{@{}l@{}}Saliency+Entail. \scriptsize {\cite{Pasunuru2018a}}\end{tabular}       & 40.43   & 18.00   & 37.10   \\
\begin{tabular}[c]{@{}l@{}}Inconsistency loss \scriptsize {\cite{HsuAUM2018}}\end{tabular}  & 40.68   & 17.97   & 37.13   \\
\begin{tabular}[c]{@{}l@{}}Bottom-up \scriptsize {\cite{Gehrmann2018BottomUpAS}} \end{tabular}            & 41.22   & 18.68   & 38.34   \\
\begin{tabular}[c]{@{}l@{}}rnn-ext+abs+RL \scriptsize {\cite{ChenFastAS2018}} \end{tabular}  & 40.04   & 17.61   & 37.59   \\\hline
\multicolumn{4}{|c|}{\textbf{Mixed Extractive-Abstractive}}                                                                               \\ \hline
\textbf{EditNet}                                                            & 41.42   & 19.03   & 38.36   \\ \hline
\end{tabular}
\label{tab:ResultROUGE-F}
\end{table}


We further applied the \textit{Teacher-Forcing} approach~\cite{lamb2016professor} during training, where we considered the true-label instead of the editor's decision (including when updating  at each step  according to Eq.~\ref{eq:doc rep}).
Following~\cite{ChenFastAS2018}, we set  and .
We trained for  epochs, which has taken about  hours on a single GPU. We chose the best model over the validation set for testing. Finally, all components were implemented in Python  using the pytorch  package.

\subsection{Results}
Table~\ref{tab:ResultROUGE-F} compares the quality of \textit{EditNet} with that of several state-of-the-art extractive-only or abstractive-only baselines. This includes the extractor (\textit{rnn-ext-RL}) and abstractor (\textit{rnn-ext-abs-RL}) components of~\cite{ChenFastAS2018} that we further utilized for implementing\footnote{\small The \textit{rnn-ext-RL} extractor results reported in Table~\ref{tab:ResultROUGE-F} are the ones that were reported by~\cite{ChenFastAS2018}. Training the public extractor released by these authors, we obtained the following significantly lower results: R-1:38.43, R-2:18.07 and R-L:35.37.} \textit{EditNet} (see again Section~\ref{sec:EA sum}).

Overall, \textit{EditNet} provides a highly competitive summary quality, where it outperforms all baselines in the R-2 and R-L metrics. On R-1, \textit{EditNet} outperforms all abstractive baselines and almost all extractive ones. 
Interestingly, \textit{EditNet}'s summarization quality is quite similar to that of \textit{NeuSum}~\cite{Qingyu2018}. Yet, while \textit{NeuSum} applies an extraction-only approach, summaries generated by \textit{EditNet} include a mixture of sentences that have been either extracted or abstracted.

On average,  and  of \textit{EditNet}'s decisions were to abstract (\textsf{A}) or reject (\textsf{R}), respectively. Moreover, on average, per summary, \textit{EditNet} keeps only 33\% of the original (extracted) sentences, while the rest (67\%) are abstracted ones. 
This demonstrates that, \textit{EditNet} has a high capability of utilizing abstraction, while being also able to maintain or reject the original extracted text whenever it is estimated to provide the best benefit for the summary's quality.


\section{Conclusion and Future Work}
We have shown that instead of solely applying extraction or abstraction, a better choice would be a mixed one. As future work, we plan to evaluate other alternative extractor+abstractor configurations and try to train the network end-to-end. We further plan to explore reinforcement learning (RL) as an alternative decision making approach.


\balance
\bibliographystyle{plain}
\begin{thebibliography}{10}

\bibitem{ChenFastAS2018}
Yen-Chun Chen and Mohit Bansal.
\newblock Fast abstractive summarization with reinforce-selected sentence
  rewriting.
\newblock In {\em Proceedings of the 56th Annual Meeting of the Association for
  Computational Linguistics (Volume 1: Long Papers)}, pages 675--686.
  Association for Computational Linguistics, 2018.

\bibitem{Cheng2016ACL}
Jianpeng Cheng and Mirella Lapata.
\newblock Neural summarization by extracting sentences and words.
\newblock In {\em Proceedings of the 54th Annual Meeting of the Association for
  Computational Linguistics, {ACL} 2016, August 7-12, 2016, Berlin, Germany,
  Volume 1: Long Papers}, 2016.

\bibitem{Chopra2016Abs}
Sumit Chopra, Michael Auli, and Alexander~M. Rush.
\newblock Abstractive sentence summarization with attentive recurrent neural
  networks.
\newblock In {\em Proceedings of the 2016 Conference of the North American
  Chapter of the Association for Computational Linguistics: Human Language
  Technologies}, pages 93--98. Association for Computational Linguistics, 2016.

\bibitem{Cohan2018ADA}
Arman Cohan, Franck Dernoncourt, Doo~Soon Kim, Trung Bui, Seokhwan Kim, Walter
  Chang, and Nazli Goharian.
\newblock A discourse-aware attention model for abstractive summarization of
  long documents.
\newblock In {\em Proceedings of the 2018 Conference of the North American
  Chapter of the Association for Computational Linguistics: Human Language
  Technologies, Volume 2 (Short Papers)}, pages 615--621. Association for
  Computational Linguistics, 2018.

\bibitem{Dlikman2016UsingML}
Alexander Dlikman and Mark Last.
\newblock Using machine learning methods and linguistic features in
  single-document extractive summarization.
\newblock In {\em DMNLP@PKDD/ECML}, 2016.

\bibitem{DBLP:conf/emnlp/DongSCHC18}
Yue Dong, Yikang Shen, Eric Crawford, Herke van Hoof, and Jackie Chi~Kit
  Cheung.
\newblock Banditsum: Extractive summarization as a contextual bandit.
\newblock In {\em Proceedings of the 2018 Conference on Empirical Methods in
  Natural Language Processing, Brussels, Belgium, October 31 - November 4,
  2018}, pages 3739--3748, 2018.

\bibitem{summit2017}
Guy Feigenblat, Haggai Roitman, Odellia Boni, and David Konopnicki.
\newblock Unsupervised query-focused multi-document summarization using the
  cross entropy method.
\newblock In {\em Proceedings of the 37th International ACM SIGIR Conference on
  Research \& Development in Information Retrieval}, SIGIR '17. ACM, 2017.

\bibitem{Gambhir:2017}
Mahak Gambhir and Vishal Gupta.
\newblock Recent automatic text summarization techniques: A survey.
\newblock {\em Artif. Intell. Rev.}, 47(1):1--66, January 2017.

\bibitem{Gehrmann2018BottomUpAS}
Sebastian Gehrmann, Yuntian Deng, and Alexander Rush.
\newblock Bottom-up abstractive summarization.
\newblock In {\em Proceedings of the 2018 Conference on Empirical Methods in
  Natural Language Processing}, pages 4098--4109. Association for Computational
  Linguistics, 2018.

\bibitem{GuoSoftLayer2018}
Han Guo, Ramakanth Pasunuru, and Mohit Bansal.
\newblock Soft layer-specific multi-task summarization with entailment and
  question generation.
\newblock In {\em Proceedings of the 56th Annual Meeting of the Association for
  Computational Linguistics (Volume 1: Long Papers)}, pages 687--697.
  Association for Computational Linguistics, 2018.

\bibitem{Hermann:2015:TMR:2969239.2969428}
Karl~Moritz Hermann, Tom\'{a}\v{s} Ko\v{c}isk\'{y}, Edward Grefenstette, Lasse
  Espeholt, Will Kay, Mustafa Suleyman, and Phil Blunsom.
\newblock Teaching machines to read and comprehend.
\newblock In {\em Proceedings of the 28th International Conference on Neural
  Information Processing Systems - Volume 1}, NIPS'15, pages 1693--1701,
  Cambridge, MA, USA, 2015. MIT Press.

\bibitem{HsuAUM2018}
Wan-Ting Hsu, Chieh-Kai Lin, Ming-Ying Lee, Kerui Min, Jing Tang, and Min Sun.
\newblock A unified model for extractive and abstractive summarization using
  inconsistency loss.
\newblock In {\em Proceedings of the 56th Annual Meeting of the Association for
  Computational Linguistics (Volume 1: Long Papers)}, pages 132--141.
  Association for Computational Linguistics, 2018.

\bibitem{JiangClosedBook2018}
Yichen Jiang and Mohit Bansal.
\newblock Closed-book training to improve summarization encoder memory.
\newblock In {\em Proceedings of the 2018 Conference on Empirical Methods in
  Natural Language Processing}, pages 4067--4077. Association for Computational
  Linguistics, 2018.

\bibitem{lamb2016professor}
Alex~M Lamb, Anirudh Goyal ALIAS~PARTH GOYAL, Ying Zhang, Saizheng Zhang,
  Aaron~C Courville, and Yoshua Bengio.
\newblock Professor forcing: A new algorithm for training recurrent networks.
\newblock In {\em Advances In Neural Information Processing Systems}, pages
  4601--4609, 2016.

\bibitem{Li2018KIGN}
Chenliang Li, Weiran Xu, Si~Li, and Sheng Gao.
\newblock Guiding generation for abstractive text summarization based on key
  information guide network.
\newblock In {\em Proceedings of the 2018 Conference of the North American
  Chapter of the Association for Computational Linguistics: Human Language
  Technologies, Volume 2 (Short Papers)}, pages 55--60. Association for
  Computational Linguistics, 2018.

\bibitem{lin2004rouge}
Chin-Yew Lin.
\newblock Rouge: A package for automatic evaluation of summaries.
\newblock In {\em Text summarization branches out: Proceedings of the ACL-04
  workshop}, volume~8. Barcelona, Spain, 2004.

\bibitem{liu2018a}
Peter~J Liu, Mohammad Saleh, Etienne Pot, Ben Goodrich, Ryan Sepassi, Lukasz
  Kaiser, and Noam Shazeer.
\newblock Generating wikipedia by summarizing long sequences.
\newblock {\em arXiv preprint arXiv:1801.10198}, 2018.

\bibitem{Nallapati2017a}
Ramesh Nallapati, Feifei Zhai, and Bowen Zhou.
\newblock Summarunner: {A} recurrent neural network based sequence model for
  extractive summarization of documents.
\newblock In {\em Proceedings of the Thirty-First {AAAI} Conference on
  Artificial Intelligence, February 4-9, 2017, San Francisco, California,
  {USA.}}, pages 3075--3081, 2017.

\bibitem{DBLP:conf/conll/NallapatiZSGX16}
Ramesh Nallapati, Bowen Zhou, C{\'{\i}}cero~Nogueira dos Santos, {\c{C}}aglar
  G{\"{u}}l{\c{c}}ehre, and Bing Xiang.
\newblock Abstractive text summarization using sequence-to-sequence rnns and
  beyond.
\newblock In {\em Proceedings of the 20th {SIGNLL} Conference on Computational
  Natural Language Learning, CoNLL 2016, Berlin, Germany, August 11-12, 2016},
  pages 280--290, 2016.

\bibitem{DBLP:conf/naacl/NarayanCL18}
Shashi Narayan, Shay~B. Cohen, and Mirella Lapata.
\newblock Ranking sentences for extractive summarization with reinforcement
  learning.
\newblock In {\em Proceedings of the 2018 Conference of the North American
  Chapter of the Association for Computational Linguistics: Human Language
  Technologies, {NAACL-HLT} 2018, New Orleans, Louisiana, USA, June 1-6, 2018,
  Volume 1 (Long Papers)}, pages 1747--1759, 2018.

\bibitem{Pasunuru2018a}
Ramakanth Pasunuru and Mohit Bansal.
\newblock Multi-reward reinforced summarization with saliency and entailment.
\newblock In {\em Proceedings of the 2018 Conference of the North American
  Chapter of the Association for Computational Linguistics: Human Language
  Technologies, Volume 2 (Short Papers)}, pages 646--653. Association for
  Computational Linguistics, 2018.

\bibitem{Paulus2017ADR}
Romain Paulus, Caiming Xiong, and Richard Socher.
\newblock A deep reinforced model for abstractive summarization.
\newblock {\em CoRR}, abs/1705.04304, 2017.

\bibitem{See2017GetTT}
Abigail See, Peter~J. Liu, and Christopher~D. Manning.
\newblock Get to the point: Summarization with pointer-generator networks.
\newblock In {\em Proceedings of the 55th Annual Meeting of the Association for
  Computational Linguistics, {ACL} 2017, Vancouver, Canada, July 30 - August 4,
  Volume 1: Long Papers}, pages 1073--1083, 2017.

\bibitem{Sutskever2014Seq}
Ilya Sutskever, Oriol Vinyals, and Quoc~V. Le.
\newblock Sequence to sequence learning with neural networks.
\newblock In {\em Proceedings of the 27th International Conference on Neural
  Information Processing Systems - Volume 2}, NIPS'14, pages 3104--3112,
  Cambridge, MA, USA, 2014. MIT Press.

\bibitem{Tan2017a}
Jiwei Tan, Xiaojun Wan, and Jianguo Xiao.
\newblock Abstractive document summarization with a graph-based attentional
  neural model.
\newblock In {\em Proceedings of the 55th Annual Meeting of the Association for
  Computational Linguistics (Volume 1: Long Papers)}, pages 1171--1181.
  Association for Computational Linguistics, 2017.

\bibitem{vinyals2015pointer}
Oriol Vinyals, Meire Fortunato, and Navdeep Jaitly.
\newblock Pointer networks.
\newblock In {\em Advances in Neural Information Processing Systems}, pages
  2692--2700, 2015.

\bibitem{Wu2018LearningTE}
Yuxiang Wu and Baotian Hu.
\newblock Learning to extract coherent summary via deep reinforcement learning.
\newblock In {\em Proceedings of the Thirty-Second {AAAI} Conference on
  Artificial Intelligence, (AAAI-18), the 30th innovative Applications of
  Artificial Intelligence (IAAI-18), and the 8th {AAAI} Symposium on
  Educational Advances in Artificial Intelligence (EAAI-18), New Orleans,
  Louisiana, USA, February 2-7, 2018}, pages 5602--5609, 2018.

\bibitem{DBLP:conf/aaai/WuH18}
Yuxiang Wu and Baotian Hu.
\newblock Learning to extract coherent summary via deep reinforcement learning.
\newblock In {\em Proceedings of the Thirty-Second {AAAI} Conference on
  Artificial Intelligence, (AAAI-18), the 30th innovative Applications of
  Artificial Intelligence (IAAI-18), and the 8th {AAAI} Symposium on
  Educational Advances in Artificial Intelligence (EAAI-18), New Orleans,
  Louisiana, USA, February 2-7, 2018}, pages 5602--5609, 2018.

\bibitem{ZhangLatent2018}
Xingxing Zhang, Mirella Lapata, Furu Wei, and Ming Zhou.
\newblock Neural latent extractive document summarization.
\newblock In {\em Proceedings of the 2018 Conference on Empirical Methods in
  Natural Language Processing}, pages 779--784. Association for Computational
  Linguistics, 2018.

\bibitem{Qingyu2018}
Qingyu Zhou, Nan Yang, Furu Wei, Shaohan Huang, Ming Zhou, and Tiejun Zhao.
\newblock Neural document summarization by jointly learning to score and select
  sentences.
\newblock In {\em Proceedings of the 56th Annual Meeting of the Association for
  Computational Linguistics (Volume 1: Long Papers)}, pages 654--663.
  Association for Computational Linguistics, 2018.

\end{thebibliography}



\end{document}
