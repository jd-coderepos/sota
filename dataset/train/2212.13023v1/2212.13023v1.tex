\documentclass[acmsmall,screen]{acmart}


\AtBeginDocument{\providecommand\BibTeX{{Bib\TeX}}}

\setcopyright{acmcopyright}
\copyrightyear{2018}
\acmYear{2018}
\acmDOI{XXXXXXX.XXXXXXX}


\acmJournal{JACM}
\acmVolume{37}
\acmNumber{4}
\acmArticle{111}
\acmMonth{8}








\usepackage{multirow}
\usepackage{subcaption}
\def\eg{\emph{e.g.}} \def\Eg{\emph{E.g.}}
\def\ie{\emph{i.e.}} \def\Ie{\emph{I.e.}}
\def\cf{\emph{cf.}} \def\Cf{\emph{Cf.}}
\def\etc{\emph{etc.}} \def\vs{\emph{vs.}}
\def\wrt{w.r.t.} \def\dof{d.o.f.}
\def\iid{i.i.d.} \def\wolog{w.l.o.g.}
\def\etal{\emph{et al.}}
\def \mbf{\mathbf}
\def \mbb{\mathbb}
\def \tbf{\textbf}
\newcommand{\tabincell}[2]{\begin{tabular}{@{}#1@{}}#2\end{tabular}}

\begin{document}

\title{Improving Continuous Sign Language Recognition with Consistency Constraints and Signer Removal}

\author{Ronglai Zuo}
\email{rzuo@cse.ust.hk}
\orcid{0000-0002-7184-5137}
\author{Brian Mak}
\email{mak@cse.ust.hk}
\orcid{0000-0001-6787-5555}
\affiliation{\institution{The Hong Kong University of Science and Technology}
\country{Hong Kong}
}















\renewcommand{\shortauthors}{Ronglai Zuo and Brian Mak}

\begin{abstract}
Most deep-learning-based continuous sign language recognition (CSLR) models share a similar backbone consisting of a visual module, a sequential module, and an alignment module. 
However, due to limited training samples, a connectionist temporal classification loss may not train such CSLR backbones sufficiently. 
In this work, we propose three auxiliary tasks to enhance the CSLR backbones.
The first task enhances the visual module, which is sensitive to the insufficient training problem, from the perspective of consistency. 
Specifically, since the information of sign languages is mainly included in signers' facial expressions and hand movements, a keypoint-guided spatial attention module is developed to enforce the visual module to focus on informative regions, \ie, spatial attention consistency.
Second, noticing that both the output features of the visual and sequential modules represent the same sentence, to better exploit the backbone's power, a sentence embedding consistency constraint is imposed between the visual and sequential modules to enhance the representation power of both features.
We name the CSLR model trained with the above auxiliary tasks as consistency-enhanced CSLR, which performs well on signer-dependent datasets in which all signers appear during both training and testing.
To make it more robust for the signer-independent setting, a signer removal module based on feature disentanglement is further proposed to remove signer information from the backbone.
Extensive ablation studies are conducted to validate the effectiveness of these auxiliary tasks.
More remarkably, with a transformer-based backbone, our model achieves state-of-the-art or competitive performance on five benchmarks, PHOENIX-2014, PHOENIX-2014-T, PHOENIX-2014-SI, CSL, and CSL-Daily.
\end{abstract} 


\begin{CCSXML}
<ccs2012>
<concept>
<concept_id>10010147.10010178.10010224.10010225.10010228</concept_id>
<concept_desc>Computing methodologies~Activity recognition and understanding</concept_desc>
<concept_significance>500</concept_significance>
</concept>
</ccs2012>
\end{CCSXML}

\ccsdesc[500]{Computing methodologies~Activity recognition and understanding}
\keywords{continuous sign language recognition, auxiliary learning, signer-independent, feature disentanglement.}

\received{20 February 2007}
\received[revised]{12 March 2009}
\received[accepted]{5 June 2009}

\maketitle

\section{Introduction}
\label{sec:intro}
Sign language is usually the principal communication method among hearing-impaired people.
Sign language recognition (SLR) aims to transcribe sign languages into glosses (basic lexical units in a sign language), which is an important technology to bridge the communication gap between the normal-hearing and hearing-impaired people.
According to the number of glosses in a sign sentence, SLR can be categorized into (a) isolated SLR (ISLR), in which each sign sentence consists of only a single gloss, and (b) continuous SLR (CSLR), in which each sign sentence may consist of multiple glosses. 
ISLR can be seen as a simple classification task, which becomes less popular in recent years.
In this paper, we focus on CSLR which is more practical than its isolated counterpart.
In recent years, more and more CSLR models are built using deep learning techniques because of their superior performance over traditional methods \cite{stmc, vac, sfl}.
According to \cite{sfl}, the backbone of most deep-learning-based CSLR models is composed of three parts: a visual module, a sequential (contextual) module, and an alignment module.
Within this framework, visual features are first extracted from sign videos by the visual module.
After that, sequential and contextual information are modeled by the sequential module.
Finally, due to the difference between the length of a sign video and its gloss label sequence, an alignment module is needed to align the sequential features with the gloss label sequence and yields its probability.

\begin{figure}[t]
  \centering
   \includegraphics[width=0.7\linewidth]{figures/intro.pdf}
   \caption{An overview of the CSLR backbone and the three proposed auxiliary tasks. First, our SAC enforces the visual module to focus on informative regions by leveraging pose keypoints heatmaps. Second, our SEC aligns the visual and sequential features at the sentence level, which can enhance the representation power of both the features simultaneously. SAC and SEC constitute our preliminary work \cite{zuo2022c2slr}, consistency-enhanced CSLR (SLR). In this work, we extend SLR by developing a novel signer removal module based on feature disentanglement for signer-independent CSLR.}
   \label{fig:intro}
\end{figure}

Usually, such CSLR backbones are trained with the connectionist temporal classification (CTC) \cite{ctc} loss.
However, since CSLR datasets are usually small, only using the CTC loss may not train the backbones sufficiently \cite{iopt, dnf, cma, stmc, self-mutual, fcn, vac}.
That is, the extracted features are not representative enough to be used to produce accurate recognition results.
To relieve this issue, existing works can be roughly divided into two categories.
First, \cite{dnf} proposes a stage optimization strategy to iteratively refine the extracted features with the help of pseudo labels, which is widely adopted in \cite{dilated, iopt, cma, stmc, self-mutual}.
However, it introduces more hyper-parameters and is time-consuming since the model needs to adapt to a different objective in each new stage \cite{fcn}.
As an alternative strategy, auxiliary learning can keep the whole model end-to-end trainable by just adding several auxiliary tasks \cite{fcn, vac}.
In this work, three novel auxiliary tasks are proposed to help train CSLR backbones.

Our first auxiliary task aims to enhance the visual module, which is important to feature extraction but sensitive to the insufficient training problem \cite{vac, dnf, stmc}.
Since the information of sign languages is mainly included in signers' facial expressions and hand movements \cite{stmc, koller2020quantitative, hu2021global}, signers' face and hands are treated as informative regions.
Thus, to enrich the visual features, some CSLR models \cite{stmc, stmc_jour, papadimitriou20_interspeech} leverage an off-the-shelf pose detector \cite{cao2019openpose, sun2019deep} to locate the informative regions and then crop the feature maps to form a multi-stream architecture.
However, this architecture will introduce extensive parameters since each stream processes its inputs independently and the cropping operation may overlook the rich information in the pose keypoints heatmaps.
As shown in Figure \ref{fig:intro}, by visualizing the heatmaps, we find that they can reflect the importance of different spatial positions, which is similar to the idea of spatial attention.
Thus, as shown in Figure \ref{fig:framework}, we insert a lightweight spatial attention module into the visual module and enforce the spatial attention consistency (SAC) between the learned attention masks and pose keypoints heatmaps.
In this way, the visual module can pay more attention to the informative regions.

Only enhancing the visual module may not fully exploit the power of the backbone.
According to \cite{vac, self-mutual}, better performance can be obtained by explicitly enforcing the consistency between the visual and sequential modules.
VAC \cite{vac} adopts a knowledge distillation loss between the two modules by treating the visual and sequential modules as a student-teacher pair.
With a similar idea, SMKD \cite{self-mutual} transfers knowledge by shared classifiers.
Knowledge distillation can be treated as a kind of consistency since it is usually instantiated as the KL-divergence loss, a measurement of the distance between two probability distributions.
Nevertheless, the above two methods have a common deficiency that they measure consistency at the frame level, \ie, each frame has its own probability distribution.
We think that it is inappropriate to enforce frame-level consistency since the sequential module is supposed to gather contextual information; otherwise, the sequential module may be dropped.
Motivated by that both the visual and sequential features represent the same sentence, we propose the second auxiliary task: enforcing the sentence embedding consistency (SEC) between them.
As shown in Figure \ref{fig:framework}, we build a lightweight sentence embedding extractor that can be jointly trained with the backbone, and then minimize the distance between positive sentence embedding pairs while maximizing the distance between negative pairs.

We name the CSLR model trained with SAC and SEC as consistency-enhanced CSLR (SLR).
According to our experimental results (Table \ref{tab:SD}), with a transformer-based backbone, SLR can achieve satisfactory performance on signer-dependent datasets, in which all signers in the test set appear in the training set.
However, as shown in Table \ref{tab:2014SI}, SLR cannot outperform the state-of-the-art (SOTA) work on the more challenging but realistic signer-independent CSLR (SI-CSLR).
Under the SI setting, since the signers in the test set are unseen during training, removing signer-specific information can make the model more robust to signer discrepancy.
In this work, we further develop a signer removal module (SRM) based on the idea of feature disentanglement.
More specifically, we first extract robust sentence-level signer embeddings with statistics pooling \cite{snyder2018x} to ``distill" signer information, which is then dispelled from the backbone implicitly by a gradient reversal layer \cite{ganin2016domain}.
Finally, the SRM is trained with a signer classification loss.
To the best of our knowledge, we are the first to develop a specific module for SI-CSLR\footnote{Some works \cite{dnf, cma} evaluate their methods on SI-CSLR datasets, but none of them propose any dedicated modules for the SI setting. \cite{yin2016iterative} proposes a metric learning method to deal with the SI situation, but it focuses on ISLR.}.

In summary, our main contributions are:
\begin{itemize}
\item We propose to enforce the consistency between the learned attention masks and pose keypoints heatmaps to enable the visual module to focus on informative regions.
    \item We propose to align the visual and sequential features at the sentence level to enhance the representation power of both features simultaneously.
    \item We propose a signer removal module from the idea of feature disentanglement to implicitly remove signer information from the backbone for SI-CSLR. To the best of our knowledge, we are the first to focus on this challenging setting.
    \item Extensive experiments are conducted to validate the effectiveness of the three auxiliary tasks. More remarkably, with a transformer-based backbone, our model can achieve SOTA or competitive performance on five benchmarks, while the whole model is trained in an end-to-end manner.
\end{itemize}

This work is an extension to our 2022 CVPR paper, SLR \cite{zuo2022c2slr}. More specifically, we make the following new contributions:
\begin{itemize}
    \item Besides the investigation on signer-dependent continuous sign language recognition (SD-CSLR) in the CVPR paper, we propose in this paper an additional signer removal module (SRM) to tackle the more challenging signer-independent continuous sign language recognition (SI-CSLR) problem. More specifically, the SRM is designed to remove signer information from the backbone for SI-CSLR based on feature disentanglement. To the best of our knowledge, we are the first to propose a dedicated module to deal with SI-CSLR.
    \item We successfully adapt statistics pooling to SI-CSLR to extract robust sentence-level signer embeddings for the SRM.
    \item We conduct sufficient ablation studies to validate the effectiveness of the SRM, and the combination of SLR and SRM can achieve SOTA performance on an SI-CSLR benchmark.
    \item We also report additional experimental results of SLR on the latest large-scale Chinese sign language dataset, CSL-Daily \cite{zhou2021improving} with a vocabulary size of 2K and about 20K videos.
\end{itemize}



%
 \begin{figure*}[t]
  \centering
  \includegraphics[width=1.0\linewidth]{figures/framework.pdf}
  \caption{An overview of our SLR. For spatial attention consistency (SAC), we first insert a spatial attention module after the -th convolution layer, , of the visual module, and then guide it by pre-extracted pose keypoints heatmaps. For sentence embedding consistency (SEC), we extract the sentence embeddings of visual features, sequential features, and negative sequential features, respectively, and adopt a triplet loss to train the sentence embedding extractor along with the CSLR backbone. (GAP: global average pooling.)}
  \label{fig:framework}
\end{figure*}


\section{Related Works}
\subsection{Deep-learning-based CSLR}
According to \cite{sfl}, most deep-learning-based CSLR backbones consist of a visual module (3D-CNNs \cite{iopt, csl-3} or 2D-CNNs \cite{stmc, vac, self-mutual}), a sequential module (1D-CNNs \cite{dense, fcn}, RNNs \cite{stmc, vac, self-mutual, iopt, cma}, or Transformer \cite{sfl, slt}), and an alignment module (CTC \cite{stmc, vac, self-mutual} or hidden Markov models \cite{cnn-lstm-hmm, deep-sign}).
To relieve the insufficient training issue, \cite{dnf} proposes a stage optimization strategy to iteratively refine the extracted features by using pseudo labels, which is widely adopted in \cite{iopt, cma, stmc, self-mutual}.
Based on it, \cite{iopt} leverages LSTM to build an auxiliary decoder.
SMKD \cite{self-mutual} proposes a three-stage optimization strategy, which takes 100 epochs to train its model.
In a more time-efficient way, VAC \cite{vac} proposes visual enhancement and visual alignment constraints over the frame-level probability distributions to enhance the visual module and to enforce the consistency between the visual and sequential modules, respectively, and the whole model is still end-to-end trainable.
In this work, we enhance the visual module from a novel view of spatial attention consistency, and align the two modules at the sentence level to enforce their sentence embedding consistency.

Most of the existing works on CSLR only focus on the signer-dependent setting in which all signers in the test set appear during training.
Few works pay attention to the signer-independent (SI) setting, which is more realistic but challenging as all test signers are unseen in the training set.
In this work, we further propose a signer removal module based on feature disentanglement to make the model robust to signer discrepancy.


\subsection{Spatial Attention}
Spatial attention mechanism enables models to focus on specific positions, which is widely-adopted on many computer vision tasks, such as semantics segmentation \cite{fu2019dual}, object detection \cite{woo2018cbam, cao2019gcnet}, and image classification \cite{woo2018cbam, cao2019gcnet, hu2018gather, wang2017residual, linsley2018learning}.
However, the spatial attention module may not be well-trained with a single task-specific loss function.
Leveraging external information to guide the spatial attention module can be a solution to this issue.
In \cite{chen2019motion}, the spatial attention module is guided by motion information for video captioning.
\cite{pang2019mask} and \cite{li2020relation} propose mask and relation guidance for occluded pedestrian detection and person re-identification, respectively.
Interestingly, GALA \cite{linsley2018learning} leverages click maps collected in a game to supervise the spatial attention module.
In this work, we leverage pose keypoints heatmaps to guide the learning of the spatial attention module so that the visual module will focus on informative regions.

\subsection{Sentence Embedding}
Traditional methods \cite{palangi2016deep, liu2019cross} simply feed the word embedding sequence into RNNs, and take the last hidden state (two for bidirectional RNNs) as the sentence embedding. 
Recently, many powerful sentence embedding extractors \cite{reimers2019sentence, gao2021simcse, carlsson2020semantic} are built on BERT \cite{kenton2019bert}.
However, it is difficult to use these methods in our work because (1) they are too large to be co-trained along with the backbone; (2) they are pretrained on spoken languages, which are totally different to sign languages represented by videos.
In this work, we build a lightweight sentence embedding extractor that can be jointly trained with the CSLR backbone.

\subsection{Feature Disentanglement}
\label{sec:disent}
For SI-CSLR, each signer can be seen as a domain, and the key is to enable the model to generalize well to unseen domains, \ie, the test signers.
As an effective approach for domain generalization, feature disentanglement aims to decompose features into domain-invariant and domain-specific parts \cite{wang2021generalizing}.
Adversarial learning has been widely adopted for feature disentanglement by treating the feature extractor as the generator and the domain classifier as the discriminator \cite{xu2020investigating, cheng2021puregaze, liu2018exploring}.
For example, \cite{xu2020investigating} removes bias, \eg, gender and race, for facial expression recognition by training a series of domain classifiers in an adversarial manner.
Recently, \cite{cheng2021puregaze} proposes a self-adversarial framework to remove gaze-irrelevant factors to boost gaze estimation performance.
Another frequently-used feature disentanglement method is leveraging attention mechanism to highlight task-relevant features, and the residual features are treated as task-irrelevant ones.
For example, \cite{jin2020style} uses a channel attention module to remove style information for person re-identification, and \cite{huang2021age} uses both spatial and channel attention to remove age-related features for face recognition.
However, adversarial learning is usually complicated as the generator and discriminator are trained iteratively, and the attention modules would introduce extra parameters.
In this work, we adopt the gradient reversal (GR) layer \cite{ganin2016domain} that reverses the gradient coming from the domain (signer) classification loss when the back-propagation process arrives at the feature extractor (CSLR backbone) while keeping the gradient of the domain classifier unchanged.
It shares a similar idea with adversarial learning, but it is totally end-to-end and introduces no extra parameters compared to attention-based methods.
Thus, we believe it can serve as a simple baseline for future research on SI-CSLR.

 \section{Our Proposed Method}
\subsection{Framework Overview}
\label{sec:overview}
The blue arrows in Figure \ref{fig:framework} show the three components of the CSLR backbone: visual module, sequential module, and alignment module.
Taking a sign video with  RGB frames  as input, the visual module, which simply consists of several 2D-CNN\footnote{We only consider visual modules based on 2D-CNNs since a recent survey \cite{survey} shows that 3D-CNNs cannot provide as precise gloss boundaries as 2D-CNNs, and lead to worse performance.} layers () followed by a global average pooling (GAP) layer, first extracts visual features .
The sequential features  will be further extracted by the sequential module.
Finally, the alignment module computes the probability of the gloss label sequence  based on the widely-adopted CTC \cite{ctc}, where  and  denotes the length of the gloss sequence.


\subsection{Spatial Attention Consistency (SAC)}
Signers' facial expressions and hand movements are two major clues of sign languages \cite{koller2020quantitative, stmc}.
Thus, it is reasonable to expect the visual module can focus on signers' face and hands, \ie, informative regions (IRs).
From this view, we insert a spatial attention module into the visual module and enforce the consistency between the learned attention masks and keypoints heatmaps.
Since SAC is applied to all frames in the same way, we will omit the time steps in the formulation below.



\begin{figure}[t]
\begin{subfigure}[t]{.5\textwidth}
 \centering
 \includegraphics[width=\textwidth]{figures/sac.pdf}
 \caption{}
 \label{fig:sac}
\end{subfigure}
\hfill
\begin{subfigure}[t]{.45\textwidth}
  \centering
  \includegraphics[width=\textwidth]{figures/heatmap.pdf}
  \caption{}
  \label{fig:heatmap}
\end{subfigure}
\caption{(a) The architecture of our spatial attention module. (: the size of the input feature maps, GAP: global average pooling, CMP: channel-wise max pooling. (b) Two examples of the original and refined heatmaps.}
\end{figure}

\subsubsection{Spatial Attention Module}
We build our spatial attention module based on CBAM \cite{woo2018cbam} due to its simplicity and effectiveness.
As shown in Figure \ref{fig:sac}, we first pick the most informative channel via a channel-wise max pooling (CMP) operation:

where  is the squeezed feature map by CMP, and  is the input feature maps.

Besides CMP, CBAM also squeezes the feature maps with an average pooling operation along the channel dimension. 
However, we propose to dynamically weight the importance of each channel.
As shown in Figure \ref{fig:sac}, we first conduct global average pooling (GAP) over  to gather global spatial information. 
Then the channel weights  are simply generated by a channel-wise softmax layer.
By a weighted sum along the channel dimension, we can generate another squeezed feature map :

Finally, the spatial attention mask  is generated as:

where  is the sigmoid function,  is a 2D-CNN layer with a kernel size of 7\texttimes7, and  is a channel-wise concatenation operation. 
The output feature maps will be a product between  and .
In this way, important positions can be highlighted while trivial ones can be suppressed.

It should be noted that our channel weights are similar to the channel attention module in CBAM, but ours introduces no extra parameters and can even outperform the vanilla CBAM according to our ablation studies in Table \ref{tab:sac}.

\subsubsection{Keypoints Heatmap Extractor}
Simply training the spatial attention module with the backbone may lead to sub-optimal solutions.
Given the prior knowledge that signers' faces and hands are informative regions (IRs), we guide the spatial attention module with keypoints heatmaps extracted by the pretrained HRNet \cite{sun2019deep, andriluka20142d}.
Specifically, we first normalize the raw outputs of HRNet linearly to obtain the original heatmaps:

where  is the raw RGB frame,  is the pretrained HRNet, and  denotes the face, left hand, and right hand, respectively.


\subsubsection{Post-processing}
\label{sec:post-process}
There are some defects in the original heatmaps although they can roughly highlight the positions of IRs.
As shown in Figure \ref{fig:heatmap}, some trivial regions, \eg, the top of the face heatmap in the first row and the middle part of the left hand heatmap in the second row, may receive high activation values.
Besides, some highlighted regions, \eg, both of the face heatmaps in Figure \ref{fig:heatmap}, may not cover the IRs entirely.
In addition, there is usually a mismatch between the fixed heatmap resolution of the pretrained HRNet and that of the spatial attention masks.
Below we will elaborate our heatmap post-processing module that corrects the mismatch.

We first locate the center of each IR from the original heatmaps via a simple argmax operation: .
To fit different resolutions of spatial attention masks, we normalize the center as .
Suppose the spatial attention masks have a common resolution of , then a Gaussian-like refined keypoints heatmap is generated for each IR to reduce unwanted noise:

where , . 
, which denotes the transformed center for each IR under the resolution .
 and  are two hyper-parameters to control the scale of the highlighted regions.
In real practice, we set .
Finally, we merge the three processed IR heatmaps into a single one: .

\subsubsection{SAC Loss}
The spatial attention module is guided by the refined keypoints heatmaps via the SAC loss\footnote{For implementation, we further compute the average of  over all time steps.}:



\subsection{Sentence Embedding Consistency (SEC)}
\begin{figure}[t]
  \centering
  \includegraphics[width=0.5\linewidth]{figures/sec.pdf}
  \caption{The workflow of sentence embedding extraction. We omit LayerNorm \cite{layernorm} for simplicity.}
  \label{fig:sec}
\end{figure}
Some works \cite{vac, self-mutual} find that enforcing the consistency between the visual and sequential features can enhance their representation power, and lead to better performance.
Different from \cite{vac, self-mutual} that measure their consistency at the frame level, we impose a sentence embedding consistency between them.

\subsubsection{Sentence Embedding Extractor (SEE)}
Within a sign video, each gloss consists of only a few frames. 
We believe a good SEE for sign languages should take local contexts into consideration.
As shown in Figure \ref{fig:sec}, our SEE is built on QANet \cite{qanet}, which consists of a depth-wise temporal convolution network (TCN) layer and a transformer encoder layer.
The depth-wise TCN first extracts local contextual information from the frame-level feature sequence, then the transformer encoder models global contexts by its inner self-attention module.

Similar to the class token in BERT \cite{kenton2019bert}, we first prepend a learnable sentence embedding token, [SEN], to the sequential features  defined in Section \ref{sec:overview}:

The input of the SEE is the summation of the feature sequence and the positional embeddings \cite{transformer}; \ie, , where .

Within the SEE, the depth-wise TCN \cite{wu2018pay} layer first models local contexts with a residual shortcut: . 
Then the transformer encoder layer gathers information from all time steps to get the sentence embedding:

where  are the learned weights by the self-attention module in the transformer encoder.
We can also get the sentence embedding of visual features, , in the same way.

\subsubsection{Negative Sampling}
Directly minimizing the distance between  and  will result in trivial solutions.
For example, if the parameters of SEE are all zeros, then the outputs of SEE will always be the same.
A simple way to address this issue is introducing negative samples.
In this work, we follow the common practice \cite{schroff2015facenet, ye2019unsupervised, oord2018representation, hjelm2018learning} and sample another video from the mini-batch and take its sequential features as the negative sample.
Note that most CSLR models \cite{vac, self-mutual, stmc, stmc_jour} are trained with a batch size of 2, and our negative sampling strategy will degenerate to swapping under this setting:

where  is a mini-batch of the sequential features, and  denotes the corresponding negative sample.

\subsubsection{SEC Loss}
We implement SEC loss as a triplet loss \cite{schroff2015facenet} and minimize the distances between the sentence embeddings computed from the visual and sequential features of the same sentence, while maximizeing the distances between those from different sentences:

where ; 
 are sentence embeddings of visual and sequential features from the same sentence;
 are sentence embeddings of visual and sequential features from different sentences, and we treat the sentence embedding of the sequential features from a different sentence as the negative sample ;  is the margin.


\subsection{Signer Removal Module (SRM)}
\begin{figure}[t]
  \centering
  \includegraphics[width=0.7\linewidth]{figures/srm.pdf}
  \caption{The workflow of our signer removal module (SRM). We insert the SRM after the -th CNN layer, . The loss of SLR, , which is a sum of the CTC, SAC, and SEC losses, is used to train the backbone parameters . The signer classification loss  is used to train the SRM parameters  as usual, while the gradient coming from  will be reversed for .  is the loss weight for .}
  \label{fig:srm}
\end{figure}
To remove signer information from CSLR backbones, we further develop a signer removal module (SRM) based on statistics pooling and gradient reversal as shown in Figure \ref{fig:srm}. 

\subsubsection{Signer Embeddings}
We first extract signer embeddings to ``distill" signer information before dispelling it.
A na\"ive method is simply feeding the frame-level features into an MLP, and treat the outputs of MLP as signer embeddings.
In this work, motivated by the superior performance of x-vectors \cite{snyder2018x} in speaker recognition, we leverage statistics pooling to obtain more robust sentence-level signer embeddings.

Specifically, we first feed the intermediate visual features  into a global average pooling layer to squeeze the spatial dimension and obtain frame-level features\footnote{Here we misuse the notation  in Eq.\ref{eq:1}.} . 
Then a statistics pooling (SP) layer is used to aggregate frame-level information:

where  and  are the temporal mean and standard deviation of , respectively. 
In this way,  is capable to capture signer characteristics over the entire video instead of at the frame-level.

After that, a simple two-layer MLP with rectified linear unit (ReLU) function is used to project the statistics into the signer embedding space:

where  represent the two-layer MLP.

Finally, the signer embeddings  are fed into a classifier to yield signer probabilities , where  denotes the number of signers. 
The SRM is trained with the signer classification loss, which is simply a cross-entropy loss:

where  is the label of the signer.

\subsubsection{Gradient Reversal}
If the CSLR backbone is jointly trianed with , it will become the multi-task learning, which, however, cannot promise removing the signer information from the backbone.
In this work, we treat each signer as a domain and formulate SI-CSLR as a domain generalization problem in which no test signers are seen during training. 
The gradient reversal layer was proposed in \cite{ganin2016domain} to address the domain generalization problem by learning features that are discriminative to the main classification task while indiscriminate to the domain gap.
More specifically, according to \cite{ganin2016domain}, denoting the parameters of the feature extractor, label predictor, and domain classifier as , , and , respectively, the optimization of these parameters can be formulated as:

where  and  are the main classification and domain classification losses, respectively,  is the loss weight for , and  is the learning rate.



We adapt Eq. \ref{equ:grad_rev} by instantiating  and  as the backbone training loss  and signer classification loss , which are illustrated in Figure 
\ref{fig:srm}, respectively. We also merge  and  as  to denote the parameters of the backbone, and use  to represent the parameters of the SRM. The new optimization process can be formulated as:


As a result, the SRM itself is trained with  as usual, but the backbone is trained ``reversely" so that the extracted features cannot discriminate signers, and the signer information is implicitly removed.
We empirically validate the effectiveness of the SRM on two challenging SI-CSLR benchmarks, establishing a strong baseline for future works on SI-CSLR.


\subsection{Alignment Module and Loss Function}
We follow recent works \cite{self-mutual, vac, stmc, cma} to adopt CTC-based alignment module.
It yields a label for each frame which may be a repeating label or a special blank symbol.
CTC assumes that the model output at each time step is conditionally independent of each other.
Given an input sequence , the conditional probability of a label sequence , where  and  is the vocabulary of glosses, can be estimated by:

where  is the frame-level gloss probabilities generated by a classifier.
The final probability of the gloss label sequence is the summation of all feasible alignments:

where  is a mapping function to remove repeats and blank symbols in , and  is its inverse mapping.
Then the CTC loss is defined as:

Finally, the overall loss function is a combination of the CTC, SAC, SEC, and signer classification losses:

where  for signer-dependent datasets, and  for signer-independent ones.


\subsection{A Strong Sequential Module: Local Transformer}
\label{sec:lt}

\begin{figure}[t]
\begin{subfigure}[t]{.48\textwidth}
 \centering
 \includegraphics[width=\textwidth]{figures/lctr.pdf}
 \caption{Local Transformer (LT). We omit LayerNorm \cite{layernorm} for simplicity.}
 \label{fig:lctr}
\end{subfigure}
\hfill
\begin{subfigure}[t]{.48\textwidth}
  \centering
  \includegraphics[width=\textwidth]{figures/lcsa.pdf}
  \caption{Local Self-attention (LSA).}
  \label{fig:lcsa}
\end{subfigure}
\caption{We propose a strong sequential module, local transformer. It is based on QANet \cite{qanet}, which validates the effectiveness of combining TCNs with self-attention. The difference is that we further leverage Gaussian bias \cite{gau-1, gau-2} to introduce local contexts into the self-attention module, \ie, local self-attention. (: number of LT layers, which is set to 2 as default; : relative positional encoding \cite{rpe}; : window size of the Gaussian bias.)}
\end{figure}


The sequential module is an important component of the CSLR backbone.
Most existing CSLR works adopt globally-guided architectures, \eg, BiLSTM \cite{iopt, cma} and vanilla Transformer \cite{sfl, slt}, for sequence modeling due to their strong capability of capturing long-term temporal dependencies. 
However, within a sign video, each gloss is short, consisting of only a few frames.
This can explain why a locally-guided architecture, such as TCNs, can also achieve excellent performance \cite{fcn}.
In this subsection, we will elaborate a mixed architecture, Local Transformer (LT), which can leverage both global and local contexts for sequence modeling for CSLR.

Figure \ref{fig:lctr} shows the architecture of LT. 
Each LT layer consists of a depth-wise TCN layer, a local self-attention (LSA) layer, and a feed-forward network. 
Since the depth-wise TCN layer and the feed-forward network are the same as those used in \cite{qanet, transformer}, below we will only give the formulation of the LSA.

As shown in Figure \ref{fig:lcsa}, three linear layers first project the input feature sequence  into queries , keys , and values , respectively.
We then split  into , respectively, for multi-head self-attention as \cite{transformer}, where  and  is the number of heads.
The attention scores for each head can be obtained by the scaled dot-product attention as follows:


The vanilla self-attention treats each position equally. 
To emphasize local contexts, we adopt Gaussian bias \cite{gau-1, gau-2} to weaken the interactions between distant query-key (QK) pairs.
Given a QK pair (), the Gaussian bias (GB) is defined as:

where , and  is the window size of the Gaussian bias \cite{gau-1}.
Note that although we can assign Gaussian bias with a different value of  for each head, we find that a common Gaussian bias among all heads suffices to boost the performance of transformer significantly.
The final attention weights for each value vector are obtained from a softmax layer, and the output of the LSA is:

where  denotes the output linear layer.

We intuitively set  as the average ratio of frame length to gloss length: ,
where  is the number of training samples, based on the idea that a good window size should reflect the average frame length of a gloss.
More specifically,  for the PHOENIX datasets, CSL, and CSL-Daily, respectively.
 \section{Experiments}


\subsection{Datasets and Evaluation Metric}
\subsubsection{Datasets}
We evaluate our method on three signer-dependent datasets (PHOENIX-2014, PHOENIX-2014-T, and CSL-Daily) and two signer-independent datasets (PHOENIX-2014-SI and CSL).

\tbf{PHOENIX-2014} \cite{2014} is a German CSLR dataset with a vocabulary size of 1081. There are 5672, 540, and 629 videos performed by 9 signers in the training, development (dev), and test set, respectively. 

\tbf{PHOENIX-2014-T} \cite{2014T} is an extension of PHOENIX-2014 with a vocabulary size of 1085. 
There are 7096, 519, and 642 videos performed by 9 signers in the training, dev, and test set, respectively. 

\tbf{CSL-Daily} \cite{zhou2021improving} is the latest large-scale Chinese sign language dataset consisting of 18401, 1077, 1176 videos performed by 10 signers in the training, dev, and test set, respectively. Its vocabulary size is 2000.

\tbf{PHOENIX-2014-SI} \cite{2014} is the signer-independent version of PHOENIX-2014 consisting of 4376, 111, and 180 videos in the training, dev, and test set, respectively. It has 8 signers for training, and leaves the remaining one for validation and test. 

\tbf{CSL} \cite{iopt, csl-2, csl-3} is a Chinese CSLR dataset consisting of 4000 and 1000 videos in the training and test set, respectively, with a vocabulary size of 178. 
We follow \cite{fcn, vac} to conduct experiments on its signer-independent split in which 40 signers only appear in training while the remaining 10 signers only appear in testing.

Compared to some widely-adopted datasets in action recognition, \eg, Kinetics-600 \cite{k600} with ~500K videos and Something-Something v2 \cite{sthsthv2} with ~169K videos, the size of these sign language datasets are quite small. This can also explain why some specific training strategies, \eg, stage optimization and auxiliary training, are suggested necessary for CSLR before.


\subsubsection{Evaluation Metric}
We use word error rate (WER) to measure the dissimilarity between two sequences.

The official evaluation scripts provided by each dataset are used for measuring the WER.


\subsection{Implementation Details}
\subsubsection{Data Augmentation}
We first resize RGB frames to  and then crop them to .
Stochastic frame dropping (SFD) \cite{sfl} with a dropping ratio of 50\% is adopted for the PHOENIX datasets.
Since videos in CSL and CSL-Daily are much longer, we adopt a \textit{seg-and-drop} strategy that first segments the videos into short clips consisting of only two frames, and then one frame is randomly dropped from each clip.
In this way, the processed videos only consist of half of the original frames while most information can be kept.
After that, we further randomly drop 40\% frames using SFD from these processed videos.

\subsubsection{Backbones and Hyper-parameters}
We choose the following three representative backbones to validate the effectiveness of our method.
\begin{itemize}
    \item VGG11+TCN+BiLSTM (VTB). It is widely adopted in some recent works \cite{stmc, vac}. VGG11 \cite{vggnet} is used as the visual module, and the sequential module is composed of the TCN and BiLSTM to capture both local and global contexts.
    
    \item CNN+TCN (CT). This lightweight backbone only consists of a 9-layer 2D-CNN and a 3-layer TCN, which is proposed in \cite{fcn}.
    
    \item VGG11+Local Transformer (VLT). The sequential module is a 2-layer local transformer encoder described in Section \ref{sec:lt}. 
\end{itemize}
We set the number of output channels of the TCN layers in CT and VTB to 512 and the number of hidden units of BiLSTM in VTB to 2\texttimes256 to match the channel dimensions of the visual and sequential features. 
These modifications lead to comparable WERs with those reported in the original papers \cite{fcn, stmc}.
We insert the spatial attention module after the 5th CNN layer as a trade-off between heatmap resolution and GPU memory limitation.
In terms of post-processing, we set  according to the experimental results in Section \ref{sec:gamma}.
The kernel size of the depth-wise TCN layer in both our SEE and VLT backbone is set to 5, which is the same as in \cite{qanet}.
The margin  in Eq. \ref{equ:sec} is set to 2, which is the maximum difference between the negative and positive distance with a cosine distance function.
Regarding the signer removal module, we also put it after the 5th CNN layer, and the weight for , , is set to 0.75 by default.


\subsubsection{Training}
Following recent works \cite{self-mutual, vac, stmc}, all models are trained with a batch size of 2.
We adopt an Adam optimizer \cite{adam} with an initial learning rate of  and a weight decay factor of .
We empirically find that  decreases much faster than , thus we multiply the learning rate of the SEE with a factor of 0.1/0.01/0.1 for the three backbones, respectively, to match the training pace of the backbone and SEE.
As in \cite{slt}, we adopt plateau to schedule the learning rate: if the development WER does not decrease for 6 evaluation steps, the learning rate will be decreased by a factor of 0.7.
But since CSL does not have an official dev split, we decrease the learning rate after the 15th and 25th epoch and per 5 epochs after the 30th epoch.
The maximum number of training epochs is set to 60.

\subsubsection{Inference and Decoding.}
Following \cite{sfl}, to match the training condition, we evenly select every -th frame to drop during inference, where  is the dropping ratio.
We adopt the beam search algorithm with a beam size of 10 for decoding.
















\begin{figure*}[t]
  \centering
   \includegraphics[width=1.0\linewidth]{figures/vis_sac.pdf}
   \caption{Visualization results for learned spatial attention masks with or without the guidance of . We randomly select five samples () from the \tbf{test} set, and for each sample, we select one clear frame and one blurry frame. It is clear that the guidance of  can help the spatial attention module capture the informative regions (face and hands) more accurately.}
   \label{fig:vis_sac}
\end{figure*}

\subsection{Ablation Studies for SLR}
We first conduct ablation studies for SLR on PHOENIX-2014 following previous works \cite{vac, stmc, cma, self-mutual}.

\begin{table}[t]
  \centering
  \caption{Ablation study for SAC and SEC. During inference, since our SEC can be removed, only the spatial attention module in SAC will introduce negligible parameters and affect inference speed. ( denotes only inserting the spatial attention module but not guided by ; Par.: number of parameters; Sp.: inference speed measured on the same TITAN RTX GPU in seconds per video.)}
  \begin{tabular}{l|ccc|ccc}
    \toprule
    Backbone &  & SAC & SEC & WER\% & Par.(M) & Sp.(s)\\
    \midrule
    \multirow{5}{*}{VTB} & & & & 25.0 & 15.6359 & 0.169\\
    & \checkmark & & & 24.6 & +0.0001 & +0.002\\
    & & \checkmark & & 23.7 & +0.0001 & +0.002\\
    & & & \checkmark & 24.3 & +0.0000 & +0.000\\
    & & \checkmark & \checkmark & \tbf{22.6} & +0.0001 & +0.002\\
    
    \midrule
    \multirow{5}{*}{CT} & & & & 26.1 & 8.7504 & 0.095\\
    & \checkmark & & & 26.0 & +0.0001 & +0.001\\
    & & \checkmark & & 25.1 & +0.0001 & +0.001\\
    & & & \checkmark & 25.2 & +0.0000 & +0.000\\
    & & \checkmark & \checkmark & \tbf{24.5} & +0.0001 & +0.001\\
    
    \midrule
    \multirow{5}{*}{VLT} & & & & 21.5 & 16.1850 & 0.163 \\
    & \checkmark & & & 21.4 & +0.0001 & +0.002\\
    & & \checkmark & & 20.8 & +0.0001 & +0.002\\
    & & & \checkmark & 20.9 & +0.0000 & +0.000\\
    & & \checkmark & \checkmark & \tbf{20.4} & +0.0001 & +0.002\\
    \bottomrule
  \end{tabular}
  \label{tab:sacsec}
\end{table}


%
 \subsubsection{Effectiveness of SAC and SEC}
As shown in Table \ref{tab:sacsec}, both SAC and SEC generalize well on different backbones: the performance of all the three backbones can be clearly improved.
However, if the spatial attention module is inserted into the backbones without any guidance, \ie, , the model performance can only be improved slightly, which verifies the effectiveness of .
The effectiveness of SEC suggests that explicitly enforcing the consistency between the visual and sequential modules at the sentence level can strengthen the cross-module cooperation, which leads to the performance gain.
The improvements due to SAC and SEC are complementary so that using both of them can obtain better results than using only one of them.
Besides, since VLT performs the best among the three backbones, we will use it as the default backbone for the following experiments.

\subsubsection{Visualization Results for SAC}
Figure \ref{fig:vis_sac} shows the visualization results of the learned spatial attention masks of SAC (with ) and  (without ) for five test samples.
It should be noted that since SAC is deactivated during testing, our comparison is fair.
First, it is quite clear that the learned attention masks with the guidance of  look much better.
Without the guidance of , the attention masks are quite messy with horizontal lines at the top and many highlights at trivial regions, \eg, the left shoulder of , the hair of  and , and the waist of .
This explains why  can only slightly improve the performance of the backbones as shown in Table \ref{tab:sacsec}.
Second, our SAC is so robust that the IRs (face and hands) in blurry frames (right columns of  to ) can still be captured precisely.
Third, it is capable of dealing with different hand positions: \eg, both two hands are lower than the face (); one hand is near the face while the other one is not (), and hands are overlapped ().

\subsubsection{Channel Weights}
Within our spatial attention module, each channel can receive a weight to better measure its importance before squeezing the feature maps.
Removing the channel weights degenerates to the channel-wise average pooling in CBAM \cite{woo2018cbam} and achieves a WER of 21.3\%, which leads to a performance drop by 0.5\% as shown in Table \ref{tab:sac}.
Although our channel weights share a similar idea with the channel attention module of CBAM, which builds extra linear layers to generate the attention weights, no extra parameters are introduced in our spatial attention module.
To further validate their effectiveness, after removing the channel weights, we conduct one more experiment by adding the channel attention module back as CBAM; however, it can only lead to a slight performance gain and cannot beat ours even with extra parameters.

\begin{table}[t]
  \centering
  \caption{Ablation study for SAC.}
  \begin{tabular}{l|cc}
    \toprule
    Method & WER\% & \#Param(M)\\
    \midrule
    VLT + SAC & \tbf{20.8} & 16.1851 \\
    \ \ - channel weights & 21.3 & -0.0000 \\
    \ \ \ \ + channel attention  \cite{woo2018cbam} & 21.2 & +0.0335\\
    \ \ - post-processing & 21.7 & -0.0000 \\
    \ \ - face & 21.1 & -0.0000 \\
    \ \ - hands & 21.2 & -0.0000 \\
    \bottomrule
  \end{tabular}
  \label{tab:sac}
\end{table} \subsubsection{Heatmap Refinement}
We discuss in the Section \ref{sec:post-process} that the raw heatmaps of HRNet \cite{sun2019deep} consist of too many defects which may hinder the learning of the spatial attention module.
As shown in Table \ref{tab:sac}, the quality of the keypoints heatmaps can make a difference on model performance: directly using the original heatmaps without post-processing achieves a WER of 21.7\%, which reduces the performance of SAC by almost 1\%.


\subsubsection{Effect of Each Informative Region}
As shown in the last two rows in Table \ref{tab:sac}, removing either face or hands region can harm the performance of SAC. The results validate that both signers' face and hands play a key role in conveying information, which is also mentioned in \cite{stmc, koller2020quantitative}.


\begin{figure}[!t]
\centering
\includegraphics[width=0.7\linewidth]{figures/gamma.pdf}
\caption{Visualization results and performance comparison for different  in Eq. \ref{equ:post}. Since for real practice, the height and the width of the spatial attention masks are usually the same, we set  and  to the same value.}
\label{fig:gamma}
\end{figure}

\subsubsection{Effect of the Hyper-parameters  of Eq. \ref{equ:post}}
\label{sec:gamma}
We think  and  are two important hyper-parameters since they control the scale of highlighted regions in keypoints heatmaps.
Thus, we conduct experiments to compare the performance of different  as shown in Figure \ref{fig:gamma}. 
The model performance is worse when they are either too large (cannot cover the informative regions entirely) or too small (cover too many trivial regions).
When , the model achieves the best performance.

\begin{table*}[t]
\caption{Ablation study for SEC.}
\begin{subtable}[t]{0.48\textwidth}
\caption{Ablation study for the architecture of the sentence embedding extractor and negative sampling. (TF: Transformer; DTCN: depth-wise TCN; Neg. Sam.: negative sampling.)}
\centering
\resizebox{\linewidth}{!}{
\begin{tabular}{l|cc|c}
    \toprule
    Method & Extractor & Neg. Sam. & WER\% \\
    \midrule
    \multirow{4}{*}{VLT + SEC} & TF+DTCN & \checkmark & \tbf{20.9} \\
    & TF+DTCN & \texttimes & 21.5 \\
& TF & \checkmark & 21.1 \\
    & BiLSTM & \checkmark & 21.3 \\
    \bottomrule
\end{tabular}
}
\label{tab:sec_ext}
\end{subtable}
\quad
\begin{subtable}[t]{0.48\textwidth}
\caption{Ablation study for the constraint level. We fine-tune the loss factor of VA as \cite{vac} on the VLT for fair comparisons.}
\centering
\resizebox{\linewidth}{!}{
\begin{tabular}{l|c|c}
    \toprule
    Level & Constraint & WER\% \\
    \midrule
    Sentence & consistency & \tbf{20.9} \\
    \midrule
    \multirow{4}{*}{Frame} & consistency & 21.6 \\
    & visual enhancement (VE) \cite{vac} & 22.3 \\
    & visual alignment (VA) \cite{vac} & 21.9 \\
    & VE+VA \cite{vac} & 22.8 \\
\bottomrule
\end{tabular}
}
\label{tab:sec_lev}
\end{subtable}
\label{tab:sec}
\end{table*} \subsubsection{Sentence Embedding Extractor and Negative Sampling}
Our sentence embedding extractor consists of a depth-wise TCN layer and a transformer encoder aiming to model local and global contexts, respectively.
As shown in Table \ref{tab:sec_ext}, local contexts are important to sentence embedding extraction as dropping the TCN layer would lead to worse performance.
We also compare our method with the common practice, which concatenates the last two hidden states of BiLSTM and treats it as the sentence embedding.
Nevertheless, that it underperforms the transformer-based extractors implies the strength of the self-attention mechanism for sentence embedding extraction.
Table \ref{tab:sec_ext} also shows that negative sampling plays a key role in our SEC: without negative sampling, that is, directly minimizing the sentence embedding distance between the visual and sequential features, is not effective.

\subsubsection{Constraint Level}
\label{sec:cons_lev}
As shown in Table \ref{tab:sec_lev}, we implement some frame-level constraints to validate the effectiveness of our SEC.
First, we replace the sentence embeddings,  and  in Eq. \ref{equ:sec}, by their corresponding frame-level features to minimize the positive distances while maximizing the negative distances at the frame level.
However, it leads to a performance degradation of 0.7\% compared to our SEC.
We further compare our SEC with VAC \cite{vac}, which is composed of two frame-level constraints: visual enhancement (VE) and visual alignment (VA).
First, an extra classifier is appended to the visual module to yield frame-level probability distributions (visual distribution).
VE is implemented as a CTC loss computed between the visual distribution and the gloss label, which is the same as the one used for training the backbone.
Second, VA is simply a KL-divergence loss, which aims to minimize the distance between the visual distribution and the original probability distribution ( in Eq. \ref{equ:ctc}).
Table \ref{tab:sec_lev} shows that both VE and VA perform much worse than our SEC.
The results suggest that our SEC is a more proper way to measure the consistency between the visual and sequential modules.

\begin{figure}[t]
	\centering
	\includegraphics[width=0.7\linewidth]{figures/vgpair.pdf}
	\caption{Two examples of video-gloss pairs.}
	\label{fig:vgpair}
\end{figure}



\begin{table}[t]
\begin{minipage}{.5\linewidth}
    \caption{Examples of sentence embedding distances of the visual and sequential features.  and  are the videos in Figure \ref{fig:vgpair}.}
    \centering
    \begin{tabular}{c|cc}
    	\toprule
         &  &  \\
        \midrule
         & 0.01 & 1.99 \\
         & 1.76 & 0.37 \\
        \bottomrule
    \end{tabular}
    \label{tab:vgpair}
    \end{minipage}\begin{minipage}{.5\linewidth}
    \centering
    \vspace{20pt}
\caption{Effect of the value of  in Eq. \ref{equ:overall_loss}.}
    \resizebox{\linewidth}{!}{
    \begin{tabular}{l|ccccccc}
        \toprule
         & 0 & 0.25 & 0.5 & 0.75 & 1.0 & 1.25 & 1.5 \\
        \midrule
        Dev & 34.3 & 35.1 & 35.3 & \textbf{33.1} & 33.5 & 35.0 & 34.4 \\
        Test & 34.4 & 33.8 & 33.1 & \textbf{32.7} & 32.8 & 34.2 & 33.6 \\
        \bottomrule
    \end{tabular}
    }
    \label{tab:lambda}
    \end{minipage}
\end{table} 
\subsubsection{Examples of Video-gloss Pairs}
To verify whether  can really separate positive and negative samples, we provide two examples of video-gloss pairs denoted as  and  as shown in Figure \ref{fig:vgpair}.
The sentence embedding distances of the visual and sequential features of  and  are shown in Table \ref{tab:vgpair}. 
It is clear that the distance between the two features of the same video (diagonal entries, positive pairs) can be very small. 
Otherwise (off-diagonal entries, negative pairs), the distance can be very large (the maximum value is 2.00).


\subsection{Ablation Studies for the Signer Removal Module}
We further conduct ablation studies for our signer removal module (SRM) on the challenging signer-independent dataset, PHOENIX-2014-SI.

\subsubsection{Effect of the Hyper-parameter  of Eq. \ref{equ:overall_loss}}
\label{sec:lambda}
According to \cite{liu2018exploring}, the weight for the domain classification loss, \ie, our signer classification loss , is an important hyper-parameter.
We fine-tune it from 0 to 1.5 with an interval of 0.25 as shown in Table \ref{tab:lambda}.
When , the model degenerates to SLR and performs worse than other models with .
The results suggest the importance of removing signer information for SI-CSLR.
When , the model can achieve the best performance with a WER of 33.1\%/32.7\% on the dev and test set, respectively.

\begin{table}[t]
  \centering
  \caption{Ablation study for the signer removal module. (SP: statistics pooling; GR: gradient reversal)}
  \begin{tabular}{l|ccc|c|c}
    \toprule
    Method &  & SP & GR & WER\% & Type \\
    \midrule
    \multirow{6}{*}{SLR+} & & & & 34.4 & N/A \\
    \cmidrule(){2-6}
    & \checkmark & & & 34.9 & \multirow{2}{*}{Multi-task Learning} \\
    & \checkmark & \checkmark & & 33.5 & \\
    \cmidrule(){2-6}
    & \checkmark & & \checkmark & 33.6 & \multirow{2}{*}{Feature Disentanglement} \\
    & \checkmark & \checkmark & \checkmark & \textbf{32.7} & \\
    \bottomrule
  \end{tabular}
  \label{tab:srm}
\end{table} \subsubsection{Statistics Pooling and Gradient Reversal}
We further conduct ablation studies for the two major components of our SRM, the statistics pooling (SP) and gradient reversal (GR) layer.
First, the use of the GR layer decides the type of learning method: feature disentanglement or multi-task learning.
As shown in Table \ref{tab:srm}, it is clear that with the use of GR, models under the feature disentanglement setting can significantly outperform those under the multi-task learning setting.
The result implies that removing signer information is effective to SI-CSLR.
However, we find that the model, SLR+SP, can also outperform the baseline under the multi-task setting.
We think this is because the multi-task learning can be seen as a kind of regularization \cite{zhang2021survey}, which endows the shared networks between the CSLR and signer classification branches with better generalization capability.
Similar ideas also appear in some works that jointly train a speech recognition model and a speaker recognition model \cite{liu2018speaker, pironkov2016speaker}.
Finally, the effectiveness of SP also validates that sentence-level signer embeddings are more robust than frame-level ones to achieve signer classification, leading to better performance.

\begin{table}[t]
\centering
\caption{Performance comparison between seen and unseen signers.}
\begin{tabular}{l|c|c|c}
\toprule
\multirow{2}{*}{Method} & Seen Signers & Unseen Signers & Relative Gap \\
& (WER\%) & (WER\%) & (\%) \\
\midrule
SLR & 22.7 & 34.4 & 51.5 \\
SLR + SRM & 23.0 & 32.7 & \textbf{42.2} \\
\bottomrule
\end{tabular}
\label{tab:srm_gap}
\end{table} \subsubsection{Effect of the SRM over Seen and Unseen Signers}
We finally study the effect of the SRM over seen and unseen signers. We first build an extra test set consisting of only seen signers during training by removing videos performed by unseen signers from the official test set of PHOENIX-2014, and then retest ``SLR" and ``SLR+SRM" on this extra test set. As shown in Table \ref{tab:srm_gap}, with a comparable performance over the seen signers, adding the SRM can significantly narrow the performance gap between unseen and seen signers. The results suggest that our SRM can be more helpful for the real-world situation that most testing signers are unseen.


\subsection{Comparison with State-of-the-art Results}
\begin{table*}[!t]
\centering
\caption{Comparison on signer-dependent datasets. (R: RGB; F: optical flow; P: pose.)}
\resizebox{\linewidth}{!}{
\begin{tabular}{l|c|cc|cc|cc|cc}
\toprule
\multirow{2}{*}{Method}& \multirow{2}{*}{End-to-end} & \multicolumn{2}{c|}{Modalities} & \multicolumn{2}{c|}{PHOENIX-2014} & \multicolumn{2}{c|}{PHOENIX-2014-T} & \multicolumn{2}{c}{CSL-Daily} \\
& & Training & Inference & Dev & Test & Dev & Test & Dev & Test \\
\midrule
CNN-LSTM-HMMs \cite{cnn-lstm-hmm} & \texttimes & R & R & 26.0 & 26.0  & 22.1 & 24.1 & -- & -- \\
DNF (RGB) \cite{dnf} + SBD-RL \cite{wei2020semantic} & \texttimes & R & R & 23.4 & 23.5 & -- & -- & -- & -- \\
DNF \cite{dnf} & \texttimes & R+F & R+F & 23.1 & 22.9 & -- & -- & 32.8 & 32.4 \\
CMA \cite{cma} & \texttimes & R & R & 21.3 & 21.9 & -- & -- & -- & -- \\
SMKD \cite{self-mutual} & \texttimes & R & R & 20.8 & 21.0 & 20.8 & 22.4 & -- & -- \\
STMC \cite{stmc} & \texttimes & R+P & R & 21.1 & 20.7 & 19.6 & 21.0 & -- & -- \\
\midrule
SubUNets \cite{subunets} & \checkmark & R & R & 40.8 & 40.7 & -- & -- & 41.4 & 41.0 \\
LS-HAN \cite{csl-2} & \checkmark & R & R & -- & 38.3 & -- & -- & 39.0 & 39.4 \\
TIN + Transformer \cite{zhou2021improving} & \checkmark & R & R & -- & -- & -- & -- & 33.6 & 33.1 \\
SFL \cite{sfl} & \checkmark & R & R & 24.9 & 25.3 & 25.1 & 26.1 & -- & -- \\
FCN \cite{fcn} & \checkmark & R & R & 23.7 & 23.9 & 23.3 & 25.1 & 33.2 & 32.5 \\
SLT \cite{slt} & \checkmark & R & R & -- & -- & 24.6 & 24.5 & 33.1 & 32.0 \\
VAC \cite{vac} & \checkmark  & R & R & 21.2 & 22.3 & -- & -- & -- & -- \\
MMTLB \cite{chen2022simple} & \checkmark & R & R & -- & -- & 21.9 & 22.5 & -- & -- \\
SLR (ours) & \checkmark & R+P & R & \tbf{20.5} & \tbf{20.4} & 20.2 & \tbf{20.4} & \tbf{31.9} & \tbf{31.0} \\
\bottomrule
\end{tabular}
}
\label{tab:SD}
\end{table*} \subsubsection{Signer-dependent}
As shown in Table \ref{tab:SD}, we first evaluate our SLR on three signer-dependent benchmarks: PHOENIX-2014, PHOENIX-2014-T, and CSL-Daily.

Our SLR follows the idea of auxiliary learning, which also appears in some existing works, \eg, FCN \cite{fcn} and VAC \cite{vac}.
FCN proposes a gloss feature enhancement (GFE) module to introduce auxiliary supervision signals into the model training process.
However, the GFE module highly relies on pseudo labels (CTC decoded results), which may contain too many errors.
Our method only relies on pre-extracted heatmaps, which are quite accurate with the help of our post-processing algorithm, and the model's inherent consistency: the visual and sequential features represent the same sentence.
These two properties enable our method to outperform FCN by more than 3\% on both PHOENIX-2014 and PHOENIX-2014-T.
Recently, VAC proposes two auxiliary losses at the frame-level, which are not quite appropriate and perform worse than ours according to the comparison in Section \ref{sec:cons_lev}.
The SOTA work, STMC \cite{stmc}, adopts the complicated stage optimization strategy, which introduces extra hyper-parameters, and needs to manually decide when to switch to a new stage.
Our method is totally end-to-end trainable, and it can outperform STMC on both PHOENIX-2014 and PHOENIX-2014-T.
To the best of our knowledge, this is the first time that an end-to-end method can outperform those using the stage optimization strategy.

In terms of modality usage, our method just uses extra pose modality during training, while only RGB videos are needed for inference.
Thus, it is simpler for real application compared to DNF \cite{dnf} which is built on a two-stream architecture taking both RGB videos and optical flow as inputs.

Finally, the results reported on CSL-Daily may be more important due to its large vocabulary size.
Our method can still achieves SOTA performance on this large-scale dataset, which also validates the generalization capability of our method over different sign languages.







\begin{table*}[t]
\centering
\caption{Comparison on signer-independent datasets. (R: RGB; F: optical flow; P: pose; D: depth.)}
\begin{subtable}[t]{1.0\textwidth}
\centering
\caption{PHOENIX-2014-SI.}
\begin{tabular}{l|c|cc|cc}
    \toprule
    \multirow{2}{*}{Method} & \multirow{2}{*}{End-to-end} & \multicolumn{2}{c|}{Modalities} & \multirow{2}{*}{Dev} & \multirow{2}{*}{Test} \\
    & & Training & Inference & & \\
    \midrule
    Re-sign \cite{re-sign} & \texttimes & R & R & 45.1 & 44.1 \\
    DNF \cite{dnf} & \texttimes & R+F & R+F & 36.0 & 35.7 \\
    CMA \cite{cma} & \texttimes & R & R & 34.8 & 34.3 \\
    \midrule
    SLR (ours) & \checkmark & R+P & R & 34.3 & 34.4 \\
    SLR + SRM (ours) & \checkmark & R+P & R & \textbf{33.1} & \textbf{32.7} \\
    \bottomrule
\end{tabular}
\label{tab:2014SI}
\end{subtable}

\begin{subtable}[t]{1.0\textwidth}
\centering
\caption{CSL.}
\begin{tabular}{l|c|cc|c}
    \toprule
    \multirow{2}{*}{Method} & \multirow{2}{*}{End-to-end} & \multicolumn{2}{c|}{Modalities} & \multirow{2}{*}{Test} \\
    & & Training & Inference & \\
    \midrule
    LS-HAN \cite{csl-2} & \texttimes & R & R & 17.3 \\
    DPD + TEM \cite{csl-3} & \texttimes & R & R & 4.7 \\
    STMC \cite{stmc} & \texttimes & R+P & R & 2.1 \\
    \midrule
CTF \cite{ctf} & \checkmark & R & R & 11.2 \\
    HLSTM-attn \cite{HLSTM-attn} & \checkmark & R & R & 10.2 \\
    FCN \cite{fcn} & \checkmark & R & R & 3.0 \\
    VAC \cite{vac} & \checkmark & R & R & 1.6 \\
    MSeqGraph \cite{tang2021graph} & \checkmark & R+P+D & R+P+D & 0.6 \\
    SLR (ours) & \checkmark & R+P & R & 0.90 \\
    SLR + SRM (ours) & \checkmark & R+P & R & 0.68 \\
    \bottomrule
\end{tabular}
\label{tab:csl}
\end{subtable}
\label{tab:SI}
\end{table*}







 \subsubsection{Signer-independent}
As shown in Table \ref{tab:SI}, we further evaluate our SRM on the following two signer-independent benchmarks: PHOENIX-2014-SI and CSL.

Although some works, \eg, DNF \cite{dnf} and CMA \cite{cma}, evaluate their method on PHOENIX-2014-SI, none of them propose any dedicated module to deal with the challenging SI setting.
In this work, we develop a simple yet effective signer removal module (SRM) for SI-CSLR to make the model more robust to signer discrepancy. 
As shown in Table \ref{tab:2014SI}, our SLR can already achieve competitive performance on PHOENIX-2014-SI, and the SRM can further improve the performance significantly.
The result validates that feature disentanglement is an effective method to remove signer-relevant information, and we believe our SRM can serve as a strong baseline for future works on SI-CSLR.

As shown in Table \ref{tab:csl}, our SRM can lead to a relative performance gain of 24.4\% over the baseline SLR on CSL\footnote{Although the SI setting itself is challenging, since the sentences in the CSL test set all appear in the training stage, the WER can be very low (\%).}.
It is worth noting that the SOTA work, MSeqGraph \cite{tang2021graph}, uses three modalities including RGB, pose, and depth.
However, our method only uses RGB and pose information for training, and only RGB frames are needed for inference.
Thus, with a comparable performance to the SOTA work, we believe our method is more applicable in real practice.





















 \section{Conclusion}
In this work, we enhance CSLR backbones by developing three auxiliary tasks.
First, we insert a keypoint-guided spatial attention module into the visual module to enforce the visual module to focus on informative regions.
Second, we impose a sentence embedding consistency constraint between the visual and sequential features to enhance the representation power of both features.
We conduct proper ablation studies to validate the effectiveness of the two consistency constraints both quantitatively and qualitatively.
Finally, on top of the consistency-enhanced CSLR backbone, a signer removal module based on feature disentanglement is proposed for signer-independent CSLR.
More remarkably, our model can achieve SOTA or competitive performance on five benchmarks, while the whole model is trained in an end-to-end manner.

%
 
\begin{acks}
The work described in this paper was supported by a grant from the Research Grants Council of the Hong Kong Special Administrative Region, China (Project No. HKUST16200118)
\end{acks}

\bibliographystyle{ACM-Reference-Format}
\bibliography{main}

\end{document}
