\documentclass{LMCS}

\usepackage{amsmath}
\usepackage{amssymb}
\usepackage{stmaryrd}
\usepackage{subfigure}
\usepackage{gastex,amsfonts,amssymb,latexsym}
\usepackage[all]{xy}
\usepackage{arrows}
\usepackage{enumerate,hyperref}





\newcommand{\rbox}[2]{\raisebox{#1}[-#1]{#2}}


\renewcommand{\emptyset}{\varnothing}

\newcommand{\mc}[1]{\mathcal{#1}}
\newcommand{\inv}[1]{\{#1\}}
\newcommand{\nil}{{\bf 0}}
\newcommand{\mydef}{\stackrel{{\rm def}}{=}}
\newcommand{\mb}{\mathbb}
\newcommand{\lb}{\llbracket}
\newcommand{\rb}{\rrbracket}
\newcommand{\llb}{\llbracket}
\newcommand{\rrb}{\rrbracket}

\newcommand{\mbf}{\mathbf}

\newcommand{\eoe}{\hfill}

\newcommand{\<}{\langle}
\renewcommand{\>}{\rangle}
\newcommand{\les}{\leqslant}
\newcommand{\ges}{\geqslant}

\newcommand{\nv}[1]{\mathrel{\hspace{.4em}\not\hspace{-.4em}\trans{#1\;}}}


\renewcommand{\iff}{\mathrm{iff}}

\newcommand{\bigO}{{\cal O}}
\newcommand{\mX}{\mathcal{X}}
\newcommand{\mP}{\mathcal{P}}
\newcommand{\mG}{\mathcal{G}}
\newcommand{\mD}{\mathcal{D}}
\newcommand{\mC}{\mathcal{C}}
\newcommand{\mM}{\mathcal{M}}
\newcommand{\mB}{\mathcal{B}}
\newcommand{\mA}{\mathcal{A}}
\newcommand{\mL}{\mathcal{L}}
\newcommand{\mU}{\mathcal{U}}
\newcommand{\mR}{\mathcal{R}}
\newcommand{\mE}{\mathcal{E}}
\newcommand{\mF}{\mathcal{F}}
\newcommand{\mK}{\mathcal{K}}
\newcommand{\mV}{\mathcal{V}}
\newcommand{\mZ}{\mathcal{Z}}
\newcommand{\mQ}{\mathcal{Q}}

\newcommand{\sI}{Z}
\newcommand{\sO}{\mathsf{O}}

\newcommand{\trule}{\rule[-3mm]{0mm}{8mm}}

\newcommand{\DTSHS}{\textsc{DTSHS}}
\newcommand{\GBnA}{\textsc{GBA}}
\newcommand{\NFnA}{\textsc{NFA}}
\newcommand{\GDBA}{\textsc{GDBA}}


\newcommand{\BDD}{\textsc{BDD}}
\newcommand{\MTBDD}{\textsc{NTBDD}}
\newcommand{\BSCC}{\textsc{BSCC}}
\newcommand{\DAG}{\textsc{DAG}}
\newcommand{\DFT}{\textsc{DFT}}
\newcommand{\IOIMC}{\textsc{i/o-IMC}}
\newcommand{\WCC}{\textsc{WCC}}
\newcommand{\SCC}{\textsc{SCC}}
\newcommand{\BF}{\textsc{BF}}
\newcommand{\CTL}{\textsc{CTL}}
\newcommand{\CSP}{\textsc{CSP}}
\newcommand{\ACTL}{\textsc{ACTL}}
\newcommand{\ECTL}{\textsc{ECTL}}
\newcommand{\CSLTA}{\CSL}
\newcommand{\cpCTL}{\begin{small}{cp}\end{small}\textsc{CTL}}
\newcommand{\pCTMC}{\begin{small}\end{small}\textsc{CTMC}}
\newcommand{\pctmcL}{{\cal C}^{(\mX)}=(S,\R^{(\mX)},s_0,L)}
\newcommand{\CTMC}{\textsc{{CTMC}}}
\newcommand{\CTMDP}{\textsc{{CTMDP}}}
\newcommand{\ICTMC}{\textsc{ICTMC}}
\newcommand{\NHPP}{\textsc{NHPP}}
\newcommand{\CTMP}{\textsc{CTMP}}
\newcommand{\DTMP}{\textsc{DTMP}}
\newcommand{\DTA}{\textsc{DTA}}
\newcommand{\DTAr}{\DTA}
\newcommand{\DTAo}{\DTA}
\newcommand{\DMTAr}{\DMTA}
\newcommand{\DMTAo}{\DMTA}
\newcommand{\TA}{\textsc{TA}}
\newcommand{\DBTA}{\textsc{DBTA}}
\newcommand{\RA}{\textsc{RA}} \newcommand{\DFnA}{\textsc{DFA}}
\newcommand{\DRaA}{\textsc{DRA}}
\newcommand{\DFS}{\textsc{DFS}}
\newcommand{\DBA}{\textsc{DBA}}
\newcommand{\DTMA}{\textsc{DTA}}
\newcommand{\NBA}{\textsc{NBA}}
\newcommand{\ECL}{\textsc{ECL}}
\newcommand{\MITL}{\textsc{MITL}}
\newcommand{\MTL}{\textsc{MTL}}
\newcommand{\uMTL}{{\MITL_{0,\infty}}}
\newcommand{\uMITL}{{\MITL_{0,\infty}}}
\newcommand{\TBA}{\textsc{TBA}}
\newcommand{\BFS}{\textsc{BFS}}
\newcommand{\DTMC}{\textsc{DTMC}}
\newcommand{\FPS}{\textsc{FPS}}
\newcommand{\MRM}{\textsc{MRM}}
\newcommand{\MUS}{\textsc{MUS}}
\newcommand{\False}{\mathrm{f\!f}}
\newcommand{\HKSP}{\textsc{HKSP}}
\newcommand{\FKSP}{\textsc{FKSP}}
\newcommand{\HSP}{\textsc{HSP}}
\newcommand{\KSP}{\textsc{KSP}}
\newcommand{\FSP}{\textsc{FSP}}
\newcommand{\SP}{\textsc{SP}}
\newcommand{\LTL}{\textsc{LTL}}
\newcommand{\LTS}{\textsc{LTS}}
\newcommand{\MDP}{\textsc{MDP}}
\newcommand{\MRMC}{\textsc{MRMC}}
\newcommand{\PCTL}{\textsc{PCTL}}
\newcommand{\PRCTL}{\textsc{PRCTL}}
\newcommand{\CSL}{\textsc{CSL}}
\newcommand{\PRISM}{\textsc{PRISM}}
\newcommand{\REA}{\textsc{REA}}
\newcommand{\PTaB}{\textsc{PTaB}}
\newcommand{\RE}{\textsc{RE}}
\newcommand{\PTA}{\textsc{PTA}}
\newcommand{\PTS}{\textsc{PTA}}
\newcommand{\MTA}{\textsc{MTA}}
\newcommand{\DMTA}{\textsc{DMTA}}
\newcommand{\MPTA}{\textsc{MPTA}}
\newcommand{\MECA}{\textsc{MECA}}
\newcommand{\PDP}{\textsc{PDP}}
\newcommand{\PDMP}{\textsc{PDP}}
\newcommand{\IMC}{\textsc{IMC}}
\newcommand{\PDMDP}{\textsc{PDDP}}
\newcommand{\SCA}{\textsc{CSA}}
\newcommand{\ECA}{\textsc{ECA}}
\newcommand{\True}{\mathrm{t\!t}}
\newcommand{\XOR}{\mathop{\,\textsc{xor}\,}}
\newcommand{\ODE}{\textsc{ODE}}
\newcommand{\PDE}{\textsc{PDE}}
\newcommand{\NP}{\textsc{NP}}
\newcommand{\PTIME}{\textsc{PTIME}}
\newcommand{\PNF}{\textsc{PNF}}
\newcommand{\GNBA}{\textsc{GNBA}}


\newcommand{\sgn}{{\it sgn}}
\newcommand{\Exp}{{\it Exp}}


\newcommand{\Expand}{\textit{Expand}}
\newcommand{\Refine}{\textit{Refine}}
\newcommand{\Access}{\textit{Access}}
\newcommand{\Stable}{\textit{Stable}}

\newcommand{\ulpath}[4]{\pi^{#1}_{\lbrack #2,\,#3 \rbrack}(s,#4)}
\newcommand{\ulQ}[4]{Q^{#1}_{\lbrack #2,\,#3 \rbrack}(s, #4)}

\newcommand{\PI}{\mathbf{\Pi}}



\newcommand{\mv}[1]{\singlearrow{#1}}
\newcommand{\wmv}[1]{\mathrel{\stackrel{#1}{\Rightarrow}}}

\newcommand{\mdpm}{{\cal M}}
\newcommand{\mdp}{{\cal M}=(S,\steps,L)}

\newcommand{\dtmc}{\mathcal{D}=(S,\P,L)}
\newcommand{\dtmca}{\mathcal{D}=(S,\P,\hat{s})}

\newcommand{\DRA}{{\cal A}=(Q,\Sigma,\delta,q_0,Acc)}
\newcommand{\DFA}{{\cal A}=(S,\Sigma,s_0,\delta,\hat{t})}
\newcommand{\DFAD}{{\cal A}_{\cal D}=(S,\Sigma,s_0,\delta,Acc)}

\newcommand{\embm}{\mathit{emb}({\cal M})}
\newcommand{\emb}{\textsl{emb}}
\newcommand{\unif}{\mathit{unif}}

\newcommand{\bdU}{{\bf U}}
\newcommand{\bdD}{{\bf D}}
\newcommand{\bdE}{{\bf E}}
\newcommand{\bdF}{{\bf F}}
\newcommand{\bdB}{{\bf B}}
\newcommand{\bdW}{{\bf W}}
\newcommand{\bdM}{{\bf M}}
\newcommand{\bdI}{{\bf I}}
\newcommand{\bdPhi}{{\bf \Phi}}
\newcommand{\bdPi}{\mathbf{\Pi}}

\renewcommand{\P}{{\bf P}}
\newcommand{\bfU}{{\bf U}}
\newcommand{\R}{{\bf R}}
\newcommand{\pR}[1]{\R(#1)}
\newcommand{\Q}{{\bf Q}}
\newcommand{\pQ}[1]{\Q(#1)}
\newcommand{\E}{\vec{E}}
\newcommand{\mr}{\mathbb{R}}
\newcommand{\mrp}{\mathbb{R}_{\geqslant 0}}

\newcommand{\AP}{\textsc{AP}}
\newcommand{\Act}{\textit{Act}}
\newcommand{\Sched}{\mathit{Sched}}


\newcommand{\pathABS}{\mathit{Paths}_{\it abs}}
\newcommand{\pgphi}{\pathABS(s,\phi)}
\newcommand{\pathM}{\mathit{Paths}}

\newcommand{\DA}{{\cal D}\otimes{\cal A}}
\newcommand{\MA}{{\cal M}\otimes{\cal A}}


\newcommand{\reward}{w}

\newcommand{\Loc}{\textit{Loc}}
\newcommand{\distr}{{\it Distr}}
\newcommand{\Distr}{{\it Distr}}
\newcommand{\steps}{{\it Steps}}
\newcommand{\Steps}{{\it Steps}}

\newcommand{\paths}{{\it Paths}}
\newcommand{\Paths}{{\it Paths}}
\newcommand{\Pathsn}{{\it Paths^n}}
\newcommand{\Pathss}{{\it Paths^*}}
\newcommand{\Path}{{\it Paths}}
\newcommand{\pathsfin}{{\it Paths}^{\star}}
\newcommand{\Pathsfin}{{\it Paths}^{\star}}
\newcommand{\infPaths}{{\it Paths}^{\omega}}
\newcommand{\Pathsinf}{{\it Paths}^{\omega}}
\newcommand{\pathinf}{{\it Paths}^{\omega}}

\newcommand{\AccPaths}{{\it AccPaths}}

\newcommand{\last}{{\it last}}
\newcommand{\diag}{{\it diag}}

\newcommand{\goal}{{\it goal}}

\newcommand{\simp}{{\it simple}}

\newcommand{\simple}{{\it sim}}
\newcommand{\all}{{\it all}}
\newcommand{\Prob}{{\it Prob}}
\newcommand{\fair}{{\it fair}}
\newcommand{\pr}{{\it pr}}
\newcommand{\Unif}{{\it Unif}}
\newcommand{\PP}{{\it PP}}
\newcommand{\qt}{{\it qt}}
\newcommand{\prj}[2]{#1{\downharpoonright_{#2}}}

\newcommand{\Pro}{{\mathbb{P}}}

\newcommand{\Pref}{{\it Pref}}
\newcommand{\Pre}{{\it Pred}}
\newcommand{\Pred}{{\it Pred}}

\newcommand{\size}{{\it size}}

\newcommand{\U}{\mathbin{\mathsf{U}}}
\newcommand{\Release}{\mathbin{\mathsf{R}}}
\renewcommand{\S}{\mathbin{\mathsf{S}}}
\newcommand{\Since}{\mathbin{\mathsf{S}}}
\newcommand{\bU}[1]{\mathbin{\mathsf{U}^{\leqslant #1}}}  \newcommand{\oU}[1]{\mathbin{\mathsf{U}^{#1}}}  \newcommand{\W}{\mathbin{\mathsf{W}}}   \newcommand{\bW}[1]{\mathbin{\mathsf{W}^{\leqslant #1}}}

\newcommand{\bP}[1]{\mathcal{P}_{#1}}
\newcommand{\csls}{\bP{p}(\Phi\bU{t}\Psi)}
\newcommand{\cslp}{\Phi \bU{t} \Psi}

\newcommand{\Sat}[1]{\mbox{\sl Sat}(#1)}

\newcommand{\F}{\mathop{\diamondsuit}}
\newcommand{\bF}[1]{\mathop{\diamondsuit^{\leqslant #1}}}
\newcommand{\G}{\mathop{\square}}
\newcommand{\bG}[1]{\mathop{\square^{\leqslant #1}}}
\newcommand{\inva}{\textit{inv}}
\newcommand{\Inv}{\mathit{Inv}}
\newcommand{\gp}{\mathit{gp}}
\newcommand{\gc}{\mathit{gc}}
\newcommand{\lp}{\mathit{lp}}
\newcommand{\ip}{\mathit{ip}}









\newcommand{\set}[1]{\{ \, #1 \, \}}
\newcommand{\Union}{\, \cup \,}
\newcommand{\To}{\longrightarrow}
\newcommand{\sem}[1]{[\![\, #1 \, ]\!]}


\newcommand{\HS}{\hat{S}}
\newcommand{\SQ}{S^?}
\newcommand{\SYES}{S^{\it{Yes}}}
\newcommand{\SNO}{S^{\it{No}}}
\newcommand{\HAP}{\hat{\AP}}
\newcommand{\ePr}{\Pr_{\textit{emb}}}

\renewcommand{\tt}{\textrm{tt}}
\newcommand{\ff}{\textrm{ff}}



\newcommand{\Nats}{\mathbb{N}}
\newcommand{\Reals}{\mathbb{R}}
\newcommand{\pReals}{\mathbb{R}_{\geqslant0}}
\newcommand{\Rationals}{\mathbb{Q}}
\newcommand{\Int}{\mathbb{Z}}
\newcommand{\real}{\Reals}
\newcommand{\Always}{\mathop{\square}}
\newcommand{\Ever}{\F}
\newcommand{\Next}{\mbox{}}
\newcommand{\X}{\Next}


\newcommand{\updownrightarrow}[2]{\mathop{\longrightarrow}\limits^{#1}_{#2}}

\newcommand{\updownstackrel}[3]{\overset{#1}{\underset{#2}{#3}}}

\newcommand{\updownmapsto}[4]{\xymatrix{#1\;\ar @{|->}[r]^{#2}_{#3}&#4}}

\newcommand{\updownsquigarrow}[4]{\xymatrix{#1\;\ar @{~>}[r]^{#2}_{#3}&#4}}

\newcommand{\updownhookarrow}[4]{\xymatrix{#1\;\ar @{^{(}->}[r]^{#2}_{#3}&#4}}

\newcommand{\trans}[1]{\stackrel{#1}{\rightarrow}}

\newcommand{\smv}[1]{\stackrel{#1}{\rightarrow}}

\newenvironment{packed_itemize}{
\begin{itemize}
  \setlength{\itemsep}{1pt}\setlength{\parskip}{-3pt}
  \setlength{\parsep}{-1pt}
}{\end{itemize}}


\newenvironment{packed_enumerate}{
\begin{enumerate}
  \setlength{\itemsep}{-2ex}
  \setlength{\parskip}{0pt}
  \setlength{\parsep}{0pt}
}{\end{enumerate}}


\newcommand{\fref}[1]{Fig.\,\ref{#1}}



\newcommand{\tpred}{\mathsf{timepred}}
\newcommand{\tsplit}{\mathsf{timesplit}}
\newcommand{\dpred}{\mathsf{discpred}}
\newcommand{\ppred}{\mathsf{discpred}}
\newcommand{\tsucc}{\mathsf{timesucc}}
\newcommand{\psucc}{\mathsf{discsucc}}
\newcommand{\until}{\textsf{until}}
\newcommand{\veck}{\vec{k}}

\newenvironment{ft}{
\begin{footnotesize}
}{\end{footnotesize}}


\newcommand{\mpar}[1]{\marginpar{\framebox{\begin{minipage}{2cm}#1\end{minipage}}}}

\def\topbotatom#1{\hbox{\hbox to 0pt{\hss}}}
\newcommand*{\topbot}{\mathrel{\mathchoice{\topbotatom\displaystyle}
{\topbotatom\textstyle}
{\topbotatom\scriptstyle}
{\topbotatom\scriptscriptstyle}}}


\makeatletter
\newcommand{\be}{\begin{group}
\eqnarray \@ifstar{\nonumber}{}}
\newcommand{\ee}{\endeqnarray\endgroup}
\makeatother
 
\def\doi{7 (1:12) 2011}
\lmcsheading {\doi}
{1--34}
{}
{}
{Dec.~25, 2009}
{Mar.~29, 2011}
{}

\begin{document}

\title[Model Checking of CTMCs Against Timed Automata]{Model Checking of Continuous-Time Markov Chains \\
Against Timed Automata Specifications}

\author[T.~Chen]{Taolue Chen\rsuper a} \address{{\lsuper a}Formal Methods and Tools,
University of Twente, The Netherlands}  \email{chent@ewi.utwente.nl}  \thanks{{\lsuper a}This research is funded by the DFG
research training group 1295 AlgoSyn, the SRO DSN project of CTIT, University of Twente, the EU FP7 project QUASIMODO
and the DFG-NWO ROCKS project}  

\author[T.~Han]{Tingting Han\rsuper b} \address{{\lsuper{b,d}}Software Modelling and
Verification, RWTH Aachen University, Germany
}  \email{\{tingting.han,mereacre\}@cs.rwth-aachen.de}  \thanks{}   

\author[J.-P.~Katoen]{Joost-Pieter Katoen\rsuper c}  \address{{\lsuper c}Software Modelling and
Verification, RWTH Aachen University, Germany;\newline Formal Methods and Tools, University of Twente, The Netherlands}    \email{katoen@cs.rwth-aachen.de}  \thanks{}   

\author[A.~Mereacre]{Alexandru Mereacre\rsuper d}   \address{\vskip-6 pt}  \thanks{}

\keywords{continuous-time Markov chains, deterministic timed automata,
linear-time specification, model checking, piecewise-deterministic Markov processes}
\subjclass{D.2.4}

\begin{abstract}
\noindent
We study the verification of a finite continuous-time Markov chain \CTMC
 against a linear real-time specification given as a deterministic
timed automaton \DTA  with finite or Muller acceptance
conditions.
The central question that we address is: what is the probability of the set of
paths of  that are accepted by , i.e., the likelihood that
 satisfies ?
It is shown that under finite acceptance criteria this equals the reachability
probability in a finite piecewise deterministic Markov process \PDP, whereas
for Muller acceptance criteria it coincides with the reachability probability
of terminal strongly connected components in such a PDP.
Qualitative verification is shown to amount to a graph analysis of the PDP.
Reachability probabilities in our PDPs are then characterized as the least solution
of a system of Volterra integral equations of the second type and are shown
to be approximated by the solution of a system of partial differential equations.
For single-clock \DTA, this integral equation system can be transformed into
a system of linear equations where the coefficients are solutions of ordinary
differential equations.
As the coefficients are in fact transient probabilities in CTMCs, this result implies that
standard algorithms for CTMC analysis suffice to verify single-clock DTA specifications.
\end{abstract}

\maketitle
\vfill


\section{Introduction}

Continuous-time Markov chains (\CTMC s) are one of the most prominent models
in performance and dependability analysis.
They are exploited in a broad range of applications, and constitute the underlying
semantical model of a plethora of modeling formalisms for real-time probabilistic
systems such as Markovian queueing networks, stochastic Petri nets, stochastic
variants of process algebras, and calculi for systems biology.
\CTMC\ model checking has been mainly focused on the branching-time temporal
logic CSL (Continuous Stochastic Logic \cite{ASSB00,BHHK03}), a variant of timed
\CTL\ where the \CTL\ universal and existential path quantifiers are replaced by a
probabilistic operator.
Like \CTL\ model checking, CSL model checking of finite CTMCs proceeds by a
recursive descent over the parse tree of the CSL formula.
One of the key ingredients is that time-bounded reachability probabilities can be
approximated arbitrarily closely by a reduction to transient analysis in \CTMC s
\cite{BHHK03}.
This results in an efficient polynomial-time algorithm that has been realized in
model-checking tools such as PRISM \cite{HKNP06} and MRMC \cite{KHHJZ09} and
has been successfully applied to various case studies from diverse application areas.

Verifying a finite CTMC  against linear-time (but untimed) specifications
in the form of a regular or -regular language is rather straightforward
and boils down to computing reachability probabilities in discrete-time Markov
chains (DTMCs).
This can be seen as follows.
Assume that the specification is provided as a deterministic automaton 
on finite words, or alternatively as a deterministic Muller automaton .
The underlying idea is that the evolution of a CTMC is ``synchronized'' with an
accepting run of  by considering the state labels in a CTMC, i.e., atomic
propositions, as letters read by .
As  does not constrain the timing of events in the CTMC , it
suffices to take a synchronous product of  and 's embedded
DTMC, denoted , which is obtained by just ignoring the random
state residence times in  while keeping all other ingredients, in particular
the transition probabilities and state labels.
For finite acceptance criteria, the probability that , i.e., the
probability of the set of paths in  that are accepted by ,
 for short, is obtained as the reachability probability in
the product  of the final states in .
Since  is deterministic,  is a DTMC.
In case of Muller acceptance criteria,  corresponds
to the reachability probability of accepting terminal strongly connected components
in .  This follows directly from results in~\cite{CY95}.
The reachability probabilities in a DTMC can be obtained by solving a system of linear
equations whose size is linear in the size of the DTMC, see, e.g., \cite{HJ94}.

In this paper, we consider the verification of CTMCs against linear \emph{real-time}
specifications that are given as deterministic \emph{timed} automata (DTA)~\cite{AD94}.
That is to say, we explore the following problem: given a \CTMC\ , and a
linear real-time specification provided as a \emph{deterministic timed automaton}
, what is the probability of the set of paths of  that are accepted
by , i.e., what is ?

\begin{exa}
Let us illustrate the usage of \DTA\ specifications by means of a small example.
Consider a robot randomly moving in some area. It starts in some zone (, say)
and has to reach zone  within 10 time units, cf.\ Figure\,\ref{fig:Robot_CTMC}.
(For simplicity, all zones on the map are equally-sized, but this is not a restriction.)
The robot randomly moves through the zones, and resides in a zone for an
exponentially distributed amount of time.
The robot may pass through all zones to reach , but should not stay longer
than 2 time units in any gray zone.
The specification ``reach  from  within 10 time units while residing in any
gray zone for at most 2 time units'' is modeled by a simple \DTA\ which accepts
once location  is reached, cf.\ Figure\,\ref{fig:Robot_DTA}.
Clock  controls the timing constraint on the residence times of the gray zones
(assumed to be labeled with ), while clock  controls the global time constraint
to reach zone .
In state , the robot traverses non-gray zones, in  gray zones, and in
 it has reached the goal zone .
\begin{figure}[h]
\begin{center}
\hspace{-0.7cm}
\subfigure[Robot map]{\scalebox{0.95}{ \begin{picture}(59,62)(0,-62)
\put(0,-62){}
\node[Nmr=0.0](n1)(16.0,-16.0){}

\node[Nmr=0.0](n2)(24.0,-16.0){}

\node[Nmr=0.0](n3)(32.0,-16.0){}

\node[Nmr=0.0](n4)(40.0,-16.0){}

\node[NLangle=0.0,Nmarks=r,Nmr=0.0](n5)(48.0,-16.0){}

\node[Nmr=0.0](n6)(16.0,-24.0){}

\node[Nfill=y,fillgray=0.7,Nmr=0.0](n7)(24.0,-24.0){}

\node[Nmr=0.0](n8)(32.0,-24.0){}

\node[Nmr=0.0](n9)(40.0,-24.0){}

\node[Nfill=y,fillgray=0.7,Nmr=0.0](n10)(48.0,-24.0){}

\node[Nmr=0.0](n11)(16.0,-32.0){}

\node[Nfill=y,fillgray=0.7,Nmr=0.0](n12)(24.0,-32.0){}

\node[Nmr=0.0](n13)(32.0,-32.0){}

\node[Nmr=0.0](n14)(40.0,-32.0){}

\node[Nmr=0.0](n15)(48.0,-32.0){}

\node[Nmr=0.0](n16)(16.0,-40.0){}

\node[Nmr=0.0](n17)(24.0,-40.0){}

\node[Nmr=0.0](n18)(32.0,-40.0){}

\node[Nfill=y,fillgray=0.7,Nmr=0.0](n19)(40.0,-40.0){}

\node[Nmr=0.0](n20)(48.0,-40.0){}

\node[NLangle=0.0,Nmarks=i,Nmr=0.0](n21)(16.0,-48.0){}

\node[Nmr=0.0](n22)(24.0,-48.0){}

\node[Nmr=0.0](n23)(32.0,-48.0){}

\node[Nmr=0.0](n24)(40.0,-48.0){}

\node[Nmr=0.0](n25)(48.0,-48.0){}



\end{picture} \label{fig:Robot_CTMC}}}\hspace{0.6cm}
\subfigure[Two-clock DTA]{\scalebox{0.75}{\begin{picture}(88,75)(0,-75)
\put(0,-75){}
\node[NLangle=0.0,Nmarks=i](n26)(16.0,-16.0){}

\node[NLangle=0.0](n27)(16.0,-48.0){}

\node[NLangle=0.0,Nmarks=r](n28)(68.0,-32.0){}

\drawedge(n26,n28){}

\drawedge[ELside=r,ELdist=1.83](n27,n28){}

\drawloop(n26){}

\drawloop[loopangle=-90.0](n27){}

\drawedge[curvedepth=6.0](n26,n27){}

\drawedge[curvedepth=5.64](n27,n26){}



\end{picture} \label{fig:Robot_DTA}}}
\end{center}
\caption{A robot example\label{fig:robot}}
\end{figure}
\end{exa}

Like in the untimed setting discussed before, we consider two variants: DTA that accept
finite timed words, and DTA that accept infinite timed words according to a Muller
acceptance condition.
(Note that DTA with Muller acceptance condition are strictly more expressive than
DTA with B\"uchi acceptance conditions~\cite{AD94}.)
The considered verification problem is substantially harder than the case for untimed
linear specifications, e.g., as the DTA may constrain the timing of events in , it
does not suffice to take the embedded DTMC  as starting-point.
In addition, the product of a CTMC and a DTA is neither a CTMC nor a DTA, and has an
infinite state space.
It is unclear which (and whether a) stochastic process is obtained from such infinite
product, and if so, how to analyze it.

We tackle the verification of a finite CTMC against a DTA specification as follows:
\begin{enumerate}[(1)]
\item
We first show that the problem  is well-defined in the
sense that the set of paths of  that are accepted by  is measurable.
\item
We define the product  for \CTMC\  and \DTA\
 as a variant of \DTA\ in which, besides the usual ingredients of timed
automata like guards and clock resets, the location residence time is exponentially
distributed, and define a probability space over sets of timed paths in this model.
In particular, we show that the probability of  coincides with
the reachability probability of accepting paths in .
\item
We adapt the standard region construction for timed automata \cite{AD94} to this
variant of \DTA, and show that the thus obtained region automata are in fact \emph{piecewise deterministic Markov processes} (\PDP s) \cite{Dav93}, a model that is frequently
used in, e.g., stochastic control theory and financial mathematics.
The characterization of region automata as PDPs sets the ground for obtaining the
following results concerning qualitative and quantitative verification of CTMCs against
DTA.
\item
For finite acceptance criteria, we show that  equals the
reachability probability in the embedded PDP of .
Under Muller acceptance criteria,  equals the reachability
probability of accepting terminal strongly connected components in this embedded
PDP.
In case of qualitative verification ---does CTMC  satisfy  with probability
larger than zero, or equal to one?--- a graph traversal of the (embedded) PDP suffices.
\item
We then show that reachability probabilities in our PDPs can be characterized as the
least solution of a system of \emph{Volterra integral equations} of the second type
\cite{AWW95}.
This probability is shown to be approximated by the solution of a system of partial
differential equations (PDEs).
\item For the case of single-clock DTA, we show that the system of integral equations can be
transformed into a system of \emph{linear equations}, whose coefficients are solutions
of some ordinary differential equations (ODEs).
For these coefficients either an analytical solution (for small state space) can be obtained
or an arbitrarily closely approximated solution can be determined efficiently.
\end{enumerate}

\subsection*{Related work}
Model checking CTMCs against linear real-time specifications has received scant attention
so far.
To our knowledge, this issue has only been (partly) addressed in \cite{DHS09,BCHKS07}.
Baier et al.~\cite{BCHKS07} define the logic asCSL where path properties are characterized
by (time-bounded) regular expressions over actions and state formulas.
The truth value of path formulas depends not only on the available actions in a given time
interval, but also on the validity of certain state formulas in intermediate states.
asCSL is strictly more expressive than CSL~\cite{BCHKS07}.
Model checking asCSL is performed by representing the regular expressions as finite-state
automata, followed by computing time-bounded reachability probabilities in the product
of CTMC  and this automaton.
In \CSLTA~\cite{DHS09}, time constraints of until modalities are specified by single-clock
\DTA; the resulting logic is at least as expressive as asCSL~\cite{DHS09}.
The combined behavior of  and DTA  is interpreted as a Markov renewal process
and model checking \CSLTA\ is reduced to computing reachability probabilities in a \DTMC\
whose transition probabilities are given by subordinate \CTMC s.
This paper takes a completely different approach.
The technique of \cite{DHS09} cannot be generalized to multiple clocks, whereas our approach
does not restrict the number of clocks and thus supports more specifications than \CSLTA.
The DTA specification of our robot example, for instance, can neither be expressed in \CSLTA\
nor in asCSL.
For the single-clock case, our approach produces the same result as~\cite{DHS09}, but
yields a (in our opinion) conceptually simpler formulation whose correctness can be
derived by simplifying the system of integral equations obtained for the general case.
Moreover, measurability has not been addressed in \cite{DHS09}.
Other related work~\cite{BBBBG07,BBBBG08,BBBM08} provides a quantitative interpretation
to timed automata where delays and discrete choices are interpreted probabilistically.
In this approach, delays of unbounded clocks are governed by exponential distributions
like in \CTMC s.
Decidability results have been obtained for almost-sure properties~\cite{BBBBG08} and
quantitative verification~\cite{BBBM08} for (a subclass of) single-clock timed automata.

\subsection*{Organization of the paper.}
Section~\ref{sec:prelim} defines the three models that are central to this paper: CTMCs,
DTA, and PDPs.
Section~\ref{sec:DTA} shows that the set of paths in CTMC  accepted by DTA
 is measurable and coincides with reachability probabilities in the product
.
It also shows that the underlying region graph of  is a (simple
instance of a) PDP.
Section~\ref{sec:finite} constitutes the main part of the paper and deals with the verification
of DTA with finite acceptance conditions, and analyzes the quantitative reachability problem
in our PDPs, for both the general case and single-clock DTA.
Section~\ref{sec:infinite} considers DTA with Muller acceptance criteria, as well as qualitative
verification.
Finally, section~\ref{sec:concl} concludes.

This paper extends the conference paper~\cite{CHKM09_LICS} with complete proofs,
illustrative examples, and by considering Muller acceptance criteria.

\section{Preliminaries}\label{sec:prelim}

Given a set , let  be a probabi\-li\-ty
measure on the measurable space , where 
is a -algebra over . Let  denote the set of
probability measures on this measurable space.

\subsection{Continuous-time Markov chains}
\begin{defi}[\CTMC] A (labeled) \emph{continuous-time Markov chain} (\CTMC) is a
tuple  where  is a \emph{finite}
set of \emph{states}; \ \  is a finite set of \emph{atomic
propositions}; \ \  is the \emph{labeling
function};\ \  is the \emph{initial
distribution};\ \  is a stochastic
\emph{transition probability matrix}; \ and  is the \emph{exit rate function}.
\end{defi}
The probability to exit state  in  time units is given by ; the probability to take the transition
 in  time units equals .
A state  is \emph{absorbing} if .
The \emph{embedded} discrete-time Markov chain (\DTMC) of \CTMC\
 is obtained by deleting the exit rate function , i.e.,
.

\begin{defi}[Timed paths]
Let  be a \CTMC.

is the set of paths of length  in ; the set of finite
paths in  is defined by

and  is the set of infinite paths in .

denotes the set of all paths in .
\end{defi}

We denote a path 
( for short) as the sequence
 starting in state 
such that for  ( is the number of
transitions in  if  is finite);  is the
-th state of  and  is the time spent in
state . Let  be the state occupied in  at
time , i.e.  where  is
the smallest index such that . We assume w.l.o.g.\  for any .

The definition of a Borel space on paths through \CTMC s follows
\cite{Var85,BHHK03}. A \CTMC\  yields a probability measure  on paths as
follows.  Let  with 
for  and  nonempty
intervals in .
Let  denote the \emph{cylinder
set} consisting of all paths  such that
 (), and  ().
 is the smallest -algebra on
 which contains all sets  for all state sequences  with
 () and 
range over all sequences of nonempty intervals in .
The probability measure  on 
is the unique measure defined by induction on  by
 and for :


\begin{figure}
 \scalebox{0.85}{\begin{picture}(73,36)(0,-36)
\put(0,-36){}
\node[Nmarks=i](n0)(8.32,-16.3){}

\node(n1)(32.51,-16.3){}

\drawedge[curvedepth=3.0](n0,n1){1}

\drawedge[curvedepth=3.0](n1,n0){}

\node(n2)(52.48,-10.19){}

\node(n3)(52.48,-22.19){}

\drawedge(n1,n2){0.2}

\drawedge[ELside=r,ELdist=1.31](n1,n3){0.3}

\drawloop[loopdiam=6.0,loopangle=0.0](n2){1}

\drawloop[loopdiam=6.0,loopangle=0.0](n3){1}

\node[Nframe=n](n4)(8.48,-10.16){}

\node[Nframe=n](n5)(32.48,-10.16){}

\node[Nframe=n](n6)(56.16,-4.16){}

\node[Nframe=n](n7)(56.16,-28.16){}

\node[Nframe=n](n8)(49.16,-28.16){}

\node[Nframe=n](n9)(49.16,-4.16){}

\node[Nframe=n](n10)(32.16,-22.16){}

\node[Nframe=n](n11)(8.16,-22.16){}



\end{picture} }\vspace{-0.6cm}\caption{An example CTMC\label{fig:CTMC_1c_1}}
\end{figure}

\begin{exa}
An example \CTMC\ is illustrated in Figure~\ref{fig:CTMC_1c_1}, where  and  is
the initial state, i.e.,  and  for any
. The exit rates are indicated at the states, whereas the
transition probabilities are attached to the transitions. An example
timed path is  with  and .
\end{exa}

\subsection{Deterministic timed automata}



\newcommand{\mCC}{\mathcal{C}\mathcal{C}}
\newcommand{\mRe}{\mathcal{R}e}

Let  be a set of \emph{nonnegative} real-valued variables,
called \emph{clocks}.
An -valuation is a function  assigning
to each variable  a nonnegative real value .
Let  denote the set of all valuations over .
A \emph{clock constraint} on , denoted by , is a conjunction of expressions
of the form  for clock , comparison operator  and .
Let  denote the set of clock constraints over .
An -valuation  \emph{satisfies} constraint , denoted
, if and only if ; it satisfies a conjunction
of such expressions if and only if  satisfies all of them.
Let  denote the valuation that assigns 0 to all clocks.
For a subset , the reset of , denoted , is the
valuation  such that  and
.
For  and -valuation , 
is the -valuation  such that . , which implies that all clocks proceed at the same speed.

\begin{defi}[\DTA]\ A \emph{deterministic timed automaton} (or
  \DTA\ for short)
is a tuple   where
 is a finite \emph{alphabet};  is a finite set of \emph{clocks};
 is a nonempty, finite set of \emph{locations} with \emph{initial location}
;  is the \emph{acceptance condition}, which is either:
\begin{enumerate}[]
\item
, a set of \emph{accepting locations} (reachability or finite acceptance), or
\item
, an \emph{acceptance family} (Muller acceptance).
\end{enumerate}
The relation  is the \emph{edge relation} satisfying:

\end{defi}

We refer to  as an \emph{edge}, where  is an input symbol,
the \emph{guard}  is a clock constraint on the clocks of ,  is the set of
clocks that are to be reset and  is the successor location.
Intuitively, the edge  asserts that the \DTA\  can move from
location  to  when the input symbol is  and the guard  holds, while the
clocks in  should be reset when entering .
DTA are deterministic as they have a single initial location, and outgoing edges of a
location labeled with the same input symbol are required to have disjoint guards.
In this way, the next location is uniquely determined for a given location and a given
clock valuation.
In case no guard is satisfied in a location for a given clock valuation, time can progress.
If the advance of time will never reach a situation in which a guard holds, the DTA will
stay in that location ad infinitum.
Note that DTA do not have location invariants, as in safety timed automata.
For the sake of simplicity, diagonal constraints like  are not considered.
This restriction does, however, not harm the expressiveness~\cite{BPDG98}.

An (infinite) \emph{timed path} of \DTA\  is of the form  such that , and for all , it holds
, , , where 
is the clock evaluation when \emph{entering} .
The definitions on timed paths (such as , , and so forth) for CTMCs
can readily be adapted for DTA.
We consider DTA with two types of acceptance criteria.
Let \DTAr\ and \DTAo\ denote the set of \DTA\ with reachability and Muller acceptance
conditions, respectively.
\DTA\ denotes the general case covering both \DTAr\ and \DTAo.

\begin{defi}[DTA accepting paths]
An infinite timed path  is \emph{accepted} by a \DTAr\ if  for
some ;  is accepted by a \DTAo\ if , where  is the set of states  such that  for infinitely many .
\end{defi}

The timed path  is accepted according to a reachability criterion if it reaches
some final location, whereas it is accepted according to a Muller acceptance condition
if the set of infinitely visited locations equals some  set in .
As a convention, we assume each location  in \DTAr\ to be a sink.

\begin{exa}
Figure~\ref{fig:DTA_1c_ss} depicts an example \DTAr\ over the alphabet 
with initial location .
The timed automaton is deterministic as  is the only initial location and both
-labeled edges have disjoint guards.
Any timed path ending in  is accepting.

Figure~\ref{fig:DTMA_Muller_s} depicts an example \DTAo\ over the alphabet .
Its initial location is ; its Muller acceptance family equals .
Any accepting path should cycle between the locations  and  \emph{finitely}
often, and between  and  \emph{infinitely} often.
\end{exa}

\begin{figure}\begin{center}
\subfigure[\DTAr\
 ]{\scalebox{0.8}{\begin{picture}(63,32)(0,-32)
\put(0,-32){}
\node[Nmarks=i](n24)(12.16,-12.32){}

\node[Nmarks=r](n25)(48.16,-12.32){}

\drawloop[loopdiam=6.0](n24){}

\drawloop[loopdiam=6.0,loopangle=270](n24){}

\drawedge(n24,n25){}



\end{picture}
 \label{fig:DTA_1c_ss}}
 }\subfigure[\DTAo\
 ]{\hspace{-0.8cm}\scalebox{0.7}{\begin{picture}(118,31)(0,-31)
\put(0,-31){}
\node[iangle=90.0,Nmarks=ir](n91)(64.0,-20.0){}

\node[Nmarks=r](n92)(108.0,-20.0){}

\node(n93)(20.0,-20.0){}

\drawedge[curvedepth=6](n91,n92){}

\drawedge[curvedepth=6](n92,n91){}

\drawedge[ELside=r,curvedepth=-6](n91,n93){}

\drawedge[ELside=r,curvedepth=-6](n93,n91){}



\end{picture} \label{fig:DTMA_Muller_s}}
 }
\end{center}\caption{DTA with (a) reachability and (b) Muller acceptance conditions \label{fig:DTMA_Muller_sss}}
\end{figure}

\begin{rem}Expressive power of \DTAo
\DTAo \ is the set of (deterministic) timed Muller automata, \emph{(D)MTA}, for short.
A (deterministic) timed B\"uchi automaton, \emph{(D)TBA} for short, has a set  of
accepting locations, and accepts an infinite timed path  if  visits
some location in  infinitely often, i.e., .
The expressive power of \emph{(D)TMA} and \emph{(D)TBA} is related as follows~\cite{AD94}:

Note that in nondeterministic \emph{TMA} and \emph{TBA}, guards on edges emanating from a
location may overlap.
\emph{DTMA} are closed under all Boolean operators union, intersection,
and complement, while \emph{DTBA} are \emph{not} closed under complement.
\end{rem}

\begin{rem}Successor location
Since DTA are deterministic, the edge relation  can be replaced
by a (partial) function .
If only the successor location is of interest, we simpy use the function
, i.e., .
\end{rem}

\subsection{Piecewise-deterministic Markov processes}\label{sec:PDP}

PDPs~\cite{Dav84} constitute a general model for stochastic systems without
diffusions~\cite{Dav93} and has been applied to a variety of problems in
engineering, operations research, management science, and economics.
Powerful analysis and control techniques for PDPs have been developed~\cite{Len85,Len91,Cos88}.
A \PDP\ is a hybrid stochastic process involving discrete control (i.e.,
locations) and continuous variables.

Let us introduce some auxiliary notions.
Let  be a set of variables in .
Note that clock variables are a special case of these variables.
A \emph{constraint} over , denoted by , is a subset of .
Let  denote the set of constraints over .
An -valuation  satisfies constraint , denoted , if and only if .
For , a constraint over , let  be the closure of ,  the interior
of , and  the boundary
of .
For instance, for , we have
, , and  equals .

To each control location  of a PDP, an \emph{invariant}  is associated,
a constraint over  which constrains the variable values in .
The state of a PDP is a pair  with control location  and 
a variable valuation.
Let ,
where  is the set of locations.
The notions of closure, interior and boundary can be lifted to  in
a straightforward manner, e.g.,  is the boundary of ; 
and  are defined in a similar way.
\begin{defi}[\PDP\,\cite{Dav93}]
A \emph{piecewise-deterministic Markov process} (PDP) is a tuple  where  is a finite set of
\emph{locations},  is a finite set of \emph{variables},  is an \emph{invariant function}, and
\begin{enumerate}[]
\item
 is a \emph{flow
function}, which is the solution of a system of ODEs with a Lipschitz continuous vector
field,
\item
 is an \emph{exit rate function}
satisfying for any :

where ,  and
\item 
is the \emph{transition probability function} satisfying:

where  denotes ,  is a -algebra
generated by  with , and
 is a -algebra generated by .
\end{enumerate}
\end{defi}





\noindent Let us explain the behavior of a PDP.
A PDP can reside in a state  as long as
 holds.
In state , the \PDP\ can either \emph{delay} or take a
\emph{Markovian jump}.
Delaying by  time units yields the next state , i.e., the
PDP remains in location  while all its continuous variables are updated
according to .
The flow function  defines the time-dependent behavior in a single
location, in particular, it specifies how the variable valuations change when
time elapses.
In case of a Markovian jump in state , the next state  is reached with probability .
The residence time of a state is exponentially distributed; this is defined
by the function .
A third possibility for a PDP to evolve is by taking \emph{forced transitions}.
When the variable valuation  satisfies the boundary of the invariant, i.e.,
, the \PDP\ is forced to take a boundary jump,
i.e., it has to leave state .
With probability  it then moves to state .
For any , the function  is integrable as the
interval  can be divided into finitely many small intervals, on which
by equation , the function  is integrable.

A \PDP\ is named piecewise-deterministic because in each location (one piece)
the behavior is deterministically determined by the flow function .
The PDP is Markovian as the current state contains all the information to
determine the future progress of the PDP.

\subsection{Embedded PDP}
The embedded  \emph{discrete-time Markov process} (\DTMP) 
of the \PDP\  has the same state space  as  and is
equipped with a transition probability function .
The one-jump \emph{transition probability} from a state  to a set  of states (with different location as ), denoted
, is given by~\cite{Dav93}:

where  is the
minimal time to hit the boundary if such time exists;  otherwise.

is the accumulative (one-jump) transition probabi\-li\-ty from  to  and
 is the characteristic function such that  when  and  otherwise.
Term~\eqref{eq:embedded} specifies the probability to delay to state 
(on the same location) and take a Markovian jump from  to .
Note the delay  can take a value from .
Term~\eqref{eq:embedded2} is the probability to stay in the same location for
 time units and then it is forced to take a boundary jump from  to  since  will be by any delay invalid.

\begin{figure} \begin{center}\scalebox{1}{
\begin{picture}(85,20)(0,-20)
\node[Nw=15.0,Nh=15.0,Nmr=9.0](n0)(43.61,-8.31){}

\node[Nframe=n](n1)(43.61,-4.31){}

\node[Nframe=n](n2)(43.61,-8.31){}

\node[Nframe=n](n3)(43.61,-12.31){}

\node[Nw=15.0,Nh=15.0,Nmr=9.0](n4)(73.61,-8.31){}

\drawedge(n0,n4){}

\node[Nframe=n](n5)(73.61,-4.31){}

\node[Nframe=n](n6)(73.61,-8.31){}

\node[Nframe=n](n7)(73.61,-12.31){}

\node[Nw=15.0,Nh=15.0,Nmr=9.0](n9)(10.25,-8.31){}

\node[Nframe=n](n10)(10.25,-4.31){}

\node[Nframe=n](n11)(10.25,-8.31){}

\node[Nframe=n](n12)(10.25,-12.31){}

\drawedge[ELside=r,ELdist=0.89](n0,n9){}
\end{picture} }\end{center}\vspace{-0.3cm}\caption{An example \PDP\ with constant exit rate
 and boundary measure \label{fig:PDP}}
\end{figure}

\begin{exa}
Figure~\ref{fig:PDP} depicts a -location \PDP\  with ,
where  and .
Solving  yields the flow function 
for .
The state space of  is .
Let exit rate  for any .
For , let
,
 and the
boundary measure be given as .
The time for  to hit the boundary is .
For set of states  and state ,
 if , and
 if .
This yields for the transition probability from state  to  in
 is:

\end{exa}

\section{The Product of a CTMC and a DTA}\label{sec:DTA}

In this section, we will make the first steps towards the quantitative and qualitative
verification of CTMCs against linear real-time properties specified by \DTA.
The aim is to computing the probability of the set of paths in \CTMC\ 
accepted by a \DTA\ , i.e., .
We first prove that this question is well-defined, i.e., that this set of paths is
measurable.
The next step is to define the product of a CTMC  and a DTA .
As we will see, this is neither a CTMC nor a DTA, but a mixture of the two.
We define the semantics of such products and define a probability space on their
paths.
The central result of this section is that  equals
the reachability probability in the product of  and , cf. Theorem
\ref{th:CTMC=MTA}.
In order to facilitate the effective computation of these reachability probabilities,
we adapt the region construction of timed automata to the product , and show that this yields a PDP.
The analysis of these PDPs will be the subject of the next two sections.


To simplify the notations, we assume w.l.o.g.\ that a \CTMC\ has a single initial
state , i.e., , and  for .
The state labels of the CTMC will act as input symbols of the DTA.
Thus, the alphabet of DTA equals the powerset of the atomic propositions, i.e.,
.
A timed path in a CTMC is accepted by a DTA  if there exists a corresponding
accepting path in .

\begin{defi}[CTMC paths accepted by a DTA]
Let \CTMC\  and \DTA\ .
The \CTMC\ path  is \emph{accepted
by } if there exists a corresponding \DTA\ path

which is accepted by , where ,  is the (unique) guard
in  such that  and ,
and  is the clock evaluation when entering , for all .
\end{defi}

\subsection{Measurability}
The quantitative verification of \CTMC\  against \DTA\  amounts to
compute the probability of the set of paths in  that is accepted by .
Formally, let

We first prove its measurability:

\begin{thm}\label{lem:measurability}
For any \CTMC\  and \DTA\ ,  is measurable.
\end{thm}

\proof
It suffices to show that  can be written as a finite union or
intersection of measurable sets.
The proof is split in two parts: DTA with (1) reachability acceptance, and (2) Muller acceptance.
The proof of the first case is carried out by (1a) considering DTA that only contain strict
inequalities as guards, (1b) equalities, and (1c) non-strict inequalities.
(Note that constraint  can be obtained by ).
\begin{enumerate}[\hbox to8 pt{\hfill}]
\item\noindent{\hskip-12 pt\bf (1a):}\
Let \DTAr\  only contain strict inequalities as clock constraints.
As all accepting paths are finite, , where  is the set of paths
of length  accepted by .
Let .
Then there exists a corresponding path  of  which is induced by the sequence:

with  such that there exist 
with 1) ; 2) ; and 3) , where  is the clock valuation when entering .

We prove the measurability of  by showing that for any path
 there exists
a cylinder set  ( for short) such that:

This is proven in two steps:
\begin{enumerate}[a.]
\item
(.)
Let .
We define  by considering intervals  with rational bounds that are based on
.
Let  such that  if , and
 otherwise, such that:

where
,
with  denoting the fractional part.
Since DTA  only contains strict inequalities, for any  with , it follows .
\item
(.)
Let .
Let  and .
It remains to show that .
Observe that , and for any  and clock variable ,

Given that guard  only contains strict inequalities, it follows .
This can be seen as follows.
Let  for some natural .
As  and ,
it follows .
Note that , and thus
.
Hence,  since, by definition, .
It follows that .
A similar argument applies to the case  and extends to conjunctions of strict
inequalities.
Thus, , and .
\end{enumerate}
By (\ref{prfobli}) and the fact that , we have:

As each interval in  has rational bounds,  is measurable.
It follows that  is a union of \emph{countably many}
cylinder sets, and hence is measurable.

\item\noindent{\hskip-12 pt\bf (1b):}\
Consider \DTAr\  with equalities of the form  for natural .
Measurability is shown by induction on the number of equalities in .
The base case (only strict inequalities) has been shown above.
Now suppose there exists an edge  in  where
 contains the constraint .
Let \DTAr\  be obtained from  by deleting all the outgoing
edges from  except .
We then consider the \DTA\ , , and
 where  is obtained from
 by replacing  by ;  is obtained
from  by replacing  by  and  is
obtained from  by replacing  by .
Since  is deterministic, it follows that

By the induction hypothesis, the sets
,
 and
 are measurable.
Hence,  is measurable.
Furthermore, as

where all guards  of edge  are equalities, it follows that 
is measurable.

\item\noindent{\hskip-12 pt\bf (1c):}\
Let \DTAr\  have clock constraints of the form  where .
We consider the \DTA\  and , where 
is obtained from  by changing all constraints of the form  by
, and  is obtained from  by changing any
constraint  by , with 
and  otherwise.
Clearly, .
As it was shown before that  and  are measurable, it follows that 
is measurable.
\item\noindent{\hskip-12 pt\bf (2):}\
Let \DTAo  with .
 where  is the set of paths
in CTMC  whose corresponding DTA paths are accepted by , i.e., .
We have:

where  with  the set of CTMC states whose
corresponding DTA states are , and  is the cylinder set such that each timed path of the cylinder set of the form  is a prefix of an accepting
path of .
It follows that  is measurable.
Thus,  is measurable.\qed
\end{enumerate}



\subsection{The product of a CTMC and a DTA}\label{sec:big_product}
A central step in the verification of a CTMC  against a DTA  is
to construct its synchronous product .  The resulting
object is neither a CTMC nor a DTA, but a mixture of the two.  We first define this
model, called deterministic Markovian timed automata, and define a measurable
space over its paths.  In Section~\ref{sec:finite}, we consider
the computation of 
which is based on this product.

\begin{defi}[\DMTA]\label{def:MTA} A \emph{deterministic Markovian
timed automaton} (\DMTA) is a tuple , where  is a nonempty finite set of \emph{locations}; 
is a finite set of \emph{clocks};  is the \emph{initial location};
 is the \emph{acceptance condition} with  the reachability condition and  the Muller condition;  is the
\emph{exit rate function}; and  is an \emph{edge relation} such that:

\end{defi}

DMTA closely resemble DTA, but have in addition to DTA an exit rate function
that determines the random residence time in a location, and an edge relation
where the target of an edge is a probability distribution over the locations.
Concepts such as clock valuation, clock constraints and so forth are defined as
for \DTA.  We refer to  for distribution
 as an \emph{edge} and to  with  as a \emph{transition} of this edge.  The intuition
is that when entering location , the \DMTA\ chooses a residence time which
is governed by an exponential distribution with rate .  Thus, the
probability to leave  within  time units is .  Due to the
determinism of the edge relation, at most one edge, say , is enabled.  The probability to jump to  via this edge
equals .   Similar as for \DTA s, \DMTAr\ and \DMTAo\ are defined
and \DMTA\ refers to both classes.

\begin{defi}[DMTA paths]
An infinite \emph{symbolic path} of \DMTA\  is of the form:



A symbolic path induces \emph{infinite paths} of the form  such that , , and
 where  and
 is the clock valuation of  in  when
\emph{entering} location .
The path  is \emph{accepted} by a \DMTAr\ if there exists ,
such that . It is \emph{accepted} by
\DMTAo\ if and only if .
\end{defi}

\subsection*{\DMTA\ semantics. }
Consider clock valuation  in location .
As the DMTA is deterministic, at most one guard is enabled in state .
The \emph{one-jump} probability of taking the transition 
within time interval  starting at clock valuation  in location , denoted
, is defined as follows:

Note the resemblance with \eqref{eqn:ctmc}.
Actually, part (i) characterizes the delay  at location  which is exponentially
distributed with rate ; (ii) is the \emph{characteristic function}, where
 if and only if .
It compares the current valuation  with guard 
and rules out those violating .
Part (iii) indicates the probability of the transition under consideration.
Note that (i) and (iii) are features from \CTMC s while (ii) stems from \DTA.
The characteristic function  is Riemann integrable as it is bounded
and its support is an interval; therefore,  is well-defined.
The one-jump probability can be uniquely defined in this way because it relates to
a fixed clock evaluation .

\bigskip

The above characterisation of the one-jump probability provides the basis for
defining the probability of a set of DMTA paths.
Let  be the cylinder set with  and .
It denotes a set of paths in DMTA  such that for any such path ,  and .
Let  denote the
probability of  such that  is the initial clock
valuation in location .
Let , where  is inductively defined as follows:
1ex]
   \displaystyle \int_{I_{i}}\
\underbrace{E(\ell_i){\cdot}e^{-E(\ell_{i})\tau} \cdot \mathbf{1}_{g_i}
(\eta+\tau) \cdot p_i}_{(\star)}\,\cdot\,\underbrace{\Pro^{\mc{M}}_{i+1}(\eta')}_{(\star\star)}\
d\tau & \mbox{ if } 0 \les i < n,
  \end{array} \right.

\dfrac{\P(s,s')>0\ \wedge\ q\mv{L(s), g, X}q'}
    {\updownsquigarrow{\<s,q\>}{g,X}{}{\zeta}}\mbox{ such that } \zeta(\<s',q'\>)=\P(s,s').

\AccPaths^{\mc{C}\otimes\mc{A}} \ := \
\{\,{\tau\in\Paths^{\mc{C}\otimes\mc{A}}}\mid{\tau \mbox{
is accepted by }\mc{C}{\otimes}\mc{A}}\;\}.

\tau = \<s_0,q_0\> \mv{t_0} \<s_1,q_1\> \cdots \<s_{n-1},q_{n-1}\> \mv{t_{n-1}}
\<s_n,q_n\>,

\tau = \<s_0,q_0\> \mv{t_0} \cdots \mv{t_{n-1}} \<s_n,q_n\> \in
\AccPaths^{\mc{C}\otimes\mc{A}},
{\Pr}^{\mc{C}}\left(\Paths^{\mc{C}}(\mc{A})\right) \ = \
{\Pr}^{\mc{C}\otimes\mc{A}}_{\vec{0}}\left({\AccPaths}^{\mc{C}\otimes\mc{A}} \right).{\Pr}^{\mc{C}}\big(C(s_0,I_0,\ldots,I_{n-1},s_n)\big)
= \prod_{0\leqslant  i< n}\int_{I_i}\P(s_i,s_{i+1})\cdot
E(s_i)\cdot e^{-E(s_i)\tau} d\tau. 
\Pro^{\mc{C}\otimes\mc{A}}_{i}(\eta_{i})=\int_{I_{i}}\mathbf{1}_{g_i}(\eta_i+\tau_i)
{\cdot}p_i{\cdot}E(\ell_i){\cdot}e^{-E(\ell_i)\tau_i}\cdot\Pro^{\mc{C}\otimes\mc{A}}_{i+1}(\eta_{i+1})\
d\tau_i,

\updownmapsto{\ell_{0}}{g_{0},X_{0}}{p_{0}}{\ \ \ell_{1}} \cdots
\updownmapsto{\ell_{n-1}}{g_{n-1},X_{n-1}}{p_{n-1}}{\ \ \ell_{n}}
p_i=\P(s_i, s_{i+1})\qquad\mbox{ and }\qquad E(\ell_i)=E(s_i).\label{eq:rate}
\Pro^{\mc{C}\otimes\mc{A}}_{i}(\eta_{i})&=&\int_{I_{i}}\mathbf{1}_{g_i}(\eta_i+\tau_i){\cdot}
p_i{\cdot}E(\ell_i){\cdot}e^{-E(\ell_i)\tau_i}\cdot\Pro^{\mc{C}\otimes\mc{A}}_{i+1}(\eta_{i+1})\
d\tau_i\\
&\stackrel{\textrm{I.H.}}{=}& \int_{I_{i}}p_i{\cdot}E(\ell_i){\cdot}e^{- E(\ell_i)\tau_i}d\tau_i\cdot\Pro^{\mc{C}\otimes\mc{A}}_{i+1}(\eta_{i+1})\\
&\stackrel{\textrm{Eq.}\eqref{eq:rate}}{=}&\int_{I_{i}}\P(s_i,s_{i+1}){\cdot}E(s_i){\cdot}e^{- E(s_i)\tau_i}d\tau_i\cdot\Pro^{\mc{C}\otimes\mc{A}}_{i+1}(\eta_{i+1}).\\
{\Pr}_{\vec{0}}^{\mc{C}\otimes
\mc{A}}\big(C(\ell_0,I_0,\ldots,I_{n-1},\ell_n)\big)\ :=\
\Pro^{\mc{C}\otimes \mc{A}}_{0}(\vec{0})= \prod_{0\leqslant i<
n}\int_{I_i}\P(s_i,s_{i+1}){\cdot}E(s_i){\cdot}e^{-E(s_i)\tau}
d\tau ,\lfloor \eta(x_i) \rfloor = \lfloor \eta'(x_i) \rfloor \quad \mbox{and} \quad
		             \{ \eta(x_i) \} \les \{ \eta'(x_i) \} \ \mbox{iff} \ { \eta(x_j) } \les { \eta'(x_j) },
    
\Lambda(v) \ = \ \left\{  \begin{array}{ll}
E(v{\downharpoonright_1}) & \mbox{ if } v \stackrel{p,X}{\hookrightarrow}
v' \mbox{ for some } v' \in V \

\item  is the
\emph{transition edge relation}, such that:\begin{enumerate}[]
\item
 if , and
 is the successor region of ;

\item
 if  with , and .
\end{enumerate}
\end{enumerate}
\end{defi}\medskip

\noindent Any vertex in the region graph is a pair consisting of a location and a region.
Edges of the form  are called delay edges,
whereas those of the form  are called Markovian
edges.   Note that Markovian edges emanating from a boundary region do \emph{not}
contribute to the reachability probability as the time to hit the boundary is always zero
(i.e.,  in Eq.~\eqref{eq:Markovian}, page\,\pageref{eq:Markovian}).
Therefore, we can safely remove all the Markovian edges emanating from boundary
regions and combine each such boundary region with its unique non-boundary (direct)
successor.  In the sequel, by slight abuse of notation, we refer to this \emph{simplified region
graph} as . Note that then
 in the last item of
Definition~\ref{def:region}.

\begin{rem}Exit rates
The exit rate  equals  if only delay transitions emanate from .
The probability to take the delay edge within time  is ,
while the probability to take Markovian edges is .
\end{rem}

\begin{exa}
For the \DMTAr\  in Figure~\ref{fig:MTA_1c},
the reachable part (forward reachable from the initial vertex and
backward reachable from the accepting vertices) of the simplified
region graph  is shown in
Figure~\ref{fig:region_1c}. Note that the exit rates on  and  are ,
as only a delay edge is enabled in these vertices.
\end{exa}

The following result asserts that the region graph obtained from a DMTA is
in fact a PDP.  This is an important observation, as verification
now reduces to analyzing this PDP.

\begin{lem}
The region graph of any \DMTA\ induces a PDP.
\end{lem}

\begin{proof}
Let \DMTAr\ 
with region graph  .
Define 
where for any :
\begin{enumerate}[]
\item
 and the state space
;
\item
;\item
;\item
if  in , then
, provided ;
\item
if  in , then
, provided .
\end{enumerate}
It follows directly that  is a
PDP.
\end{proof}
\noindent Note that the acceptance conditions play no role in the definition of a PDP,
thus this lemma applies to both \DMTAr\ and \DMTAo.

\section{Verifying CTMCs Against Finite DTA Specifications} \label{sec:finite}

The characterization of the region graph of  as a PDP paves
the way to the verification of CTMC  against \DTAr\ specification .
This section concentrates on the quantitative verification problem and deals with
single-clock DTA separately.

\subsection{Quantitative verification with arbitrarily many clocks}\label{sec:general_DTA}

The central issue in quantitative verification is to compute the probability of the set
of paths in  accepted by .   By Theorem~\ref{th:CTMC=MTA}, this is
equal to computing reachability probabilities in DTMA .   The
remaining question is how to determine these probabilities.  To that end, we show
that this amounts to determine reachability probabilities of untimed events in the
embedded PDP of  (cf.\ Theorem~\ref{th:MTA=DTMP} below).  These probabilities are characterized by a Volterra integral equation
system of second type.  As solving this integral equation system is typically hard, we
present an effective approximation algorithm.

\subsection*{Characterizing reachability probabilities.}
\label{sec:general_DTA_characterization}
We first consider determining unbounded reachability probabilities in the \PDP\
.
This is done by considering its embedded PDP, the DTMP , as
for unbounded reachability probabilities, the timing aspects are not important.
Note that the set of locations of PDP  and  are equal.
Besides, the discrete probabilistic evolution of  and 
coincide.  The main difference is that  is time-abstract whereas
 is not.

Let  initial state  and  be the set of goal locations.
For state , let ,  
for short, denote the probability to reach some state in  from state  in .
These probabilities are recursively defined as follows.
For vertex , we have: 1ex]
\Prob_{v,\delta}(\eta,T) +
{\sum}_{v\stackrel{p,X}{\hookrightarrow}v'} \Prob_{v,v'}(\eta,T)
&\mbox{otherwise}
\end{array}\right.
\label{eq:delay}
{\Prob}_{v,\delta}(\eta, T) \ = \
   e^{-\Lambda(v){\cdot}\flat(v,\eta)} \cdot {\Prob}_{v'}\big(\eta{+}\flat(v,\eta), T \big),
\label{eq:Markovian}
{\Prob}_{v, v'} (\eta, T) \ = \
\int_0^{\flat(v,\eta)}p{\cdot}\Lambda(v) {\cdot} e^{{-}\Lambda(v){\cdot}\tau} \cdot
{\Prob}_{v'}\big((\eta+\tau)[X:=0], T\big)\ d\tau.
\Prob_{v_1}(x_1,x_2)=\Prob_{v_1,\delta}(x_1,x_2)+\Prob_{v_1,v_3}(x_1,x_2),\Prob_{v_1,\delta}(x_1,x_2)=e^{-(2-x_1)r_0}{\cdot}\Prob_{v_2}(2,2)\Prob_{v_1,v_3}(x_1,x_2)=\int_0^{2-x_1}r_0{\cdot}e^{-r_0\tau}{\cdot}\Prob_{v_3}(0,x_2+\tau)\
d\tau

{\Pr}^{\mc{C}\otimes \mc{A}}_{\vec{0}}\big(\AccPaths^{\mc{C}\otimes \mc{A}}\big)
\mbox{ is the least solution of } {\Prob}^\mc{D}_{v_0}(\vec{0},V_F),

\Pr(\ell,\eta) = \left\{ \begin{array}{ll}
1 & \mbox{ if } \ell\in Loc_F \
Informally,  is the probability to reach the set of locations  from
location  and clock valuation .
The above integral can be simplifed as follows.
W.l.o.g. assume clock constraints to be of the form , where  and .
Then we have:

where  and 
for any .

If , the theorem follows directly.  In the remainder of the proof, assume
.  Our proof is based on showing that for any  and
clock valuation ,

where  is the initial vertex in the region graph  with , and .
This is done as follows.
For natural , let  be the probability to reach  in 
steps in .
For , we have  if  and 0, otherwise.
For , we define inductively:

Similarly, let  be the probability to reach the set of goal states
 in  steps:
1ex]
1,& \quad \mbox{otherwise}
\end{array}\right.\\
{\Prob}_{v}^{s,n}(\eta,V_F)&=&
\int_0^{\flat(v,\eta)}\hspace{-0.6cm}\Lambda(v) {\cdot}
e^{-\Lambda(v)\tau}{\cdot}\sum_{v\stackrel{p,
X}{\hookrightarrow}v'}p{\cdot}{\Prob}_{v'}^{n-1}\big((\eta{+}\tau)[X{:=}0],V_F
\big)\ d\tau,\\
{\Prob}_{v,\delta}^n(\eta,V_F)&=&e^{-\Lambda(v)\flat(v,
\eta)}\cdot{\Prob}_{v'}^n\big(\eta+\flat(v,\eta),V_F\big).\label{eq:prob_n_v_delta}
\label{eq:unbpresind}
{\Pr}^{n}(\ell,\eta) = \Prob_{v_0}^{n}(\eta,V_F).

t_1=\sum_{i=0}^{m-1}\flat(v_i,{\hat\eta}_i) \qquad\mbox{and} \qquad
t_2=\sum_{i=0}^{k}\flat(v_i,{\hat\eta}_i)

p_{v_0}^{n+1}(\eta) & = &
e^{-t_1\Lambda(v_0)}\cdot p_{v_m}^{n+1}(\hat\eta_m), \mbox{ where} \nonumber \


\noindent
We now derive:
1ex]
& = &e^{-t_1\Lambda(v_0)}{\cdot}\int_0^{\flat(v_m,\hat\eta_m)}
\!\!\!\Lambda(v_m){\cdot}e^{-\Lambda(v_m)\tau}{\cdot}\!\!\!\sum_{v_m\stackrel{
p,X}{\hookrightarrow}v_m'}p{\cdot}{\Prob}^{n}_{v_{m}'}
\big((\hat\eta_m{+}\tau)[X:=0],V_F \big)d\tau\nonumber\\
&=& \int_{t_1}^{t_1+\flat(v_m,\hat\eta_m)}
\!\!\!\Lambda(v_m){\cdot}e^{-\Lambda(v_m)\tau}{\cdot}\!\!\!\sum_{v_m\stackrel{
p,X}{\hookrightarrow}v_m'}p{\cdot}{\Prob}^{n}_{v_{m}'}
\big((\hat\eta_m{+}\tau{-}t_1)[X:=0],V_F \big)d\tau.

p_{v_0}^{n+1}(\eta) =
e^{-t_1\Lambda(v_0)}{\cdot}\Prob^{n+1}_{v_m,\delta}(\hat\eta_m,V_F) +
e^{-t_1\Lambda(v_0)}{\cdot}\Prob^{s,n+1}_{v_m}(\hat\eta_m,V_F).

\quad \quad p_{v_0}^{n+1}(\eta) & \!\! = \!\!
&\sum_{i=0}^{k-m}\int_{t_1+\sum_{j=0}^{i-1}
\flat(v_{m+j},\hat\eta_{m+j})}^{t_1+\sum_{j=0}^{i}\flat(v_{m+j},\hat\eta_{
m+j})}\!\!\!\Lambda(v_{m+i}){\cdot}e^{-\Lambda(v_{m+i})\tau}
\!\!\!\\\nonumber\\
&\cdot&
\underbrace{\sum_{v_{m+i}\stackrel{
p,X}{\hookrightarrow}v_{m+i}'}p{\cdot}{\Prob}^{n}_{v_{m+i}'}
\big((\hat\eta_{m+i}{+}\tau{-}t_1{-}\sum_{j=0}^{i-1}
\flat(v_{m+j},\hat\eta_{m+j}))[X:=0],V_F \big)}_{= F^n(\tau)} \, d\tau.
\label{eq:genxxx}
p_{v_0}^{n+1}(\eta)=\int_{t_1}^{t_2} \Lambda(v_0){\cdot}e^{
-\Lambda(v_0)\tau}{\cdot}F^{n}(\tau) \, d\tau.

\hat\eta_{m+i}=\eta+ \underbrace{\sum_{j=0}^{m-1}\flat(v_j,\hat\eta_j)}_{= \, t_1} +\sum_{j=0}^{i-1} \flat(v_{m+j},\hat\eta_{m+j}).

\hat\eta_{m+i}+t-t_1-\sum_{j=0}^{i-1}\flat(v_{m+j},\hat\eta_{m+j})=\eta+t.

\quad \quad \quad
F^{n}(t)
& = &
\sum_{v_{m+i}\stackrel{p,X}{\hookrightarrow}v_{m+i}'}p{\cdot}{\Prob}^{n}_{v_{m+i}'}
\big((\hat\eta_{m+i}{+}t{-}t_1{-}\sum_{j=0}^{i-1} \flat(v_{m+j},\hat\eta_{m+j}))[X:=0],V_F \big)\\
&=&
\sum_{v_{m+i}\stackrel{p,X}{\hookrightarrow}v_{m+i}'}p{\cdot}{\Prob}^{n}_{v_{m+i}'}
\big((\eta{+}t))[X:=0],V_F \big)\\
&=&
\sum_{v_{m+i}\stackrel{p,X}{\hookrightarrow}v_{m+i}'}p{\cdot}{\Pr}^{n}(\ell',(\eta{+}t))[X:=0])\\
&=&\sum_{\updownmapsto{\ell}
{g,X}{p}{\ell'}}p{\cdot}{\Pr}^{n}(\ell',(\eta+t))[X:=0]).

p_{v_0}^{n+1}(\eta)=\int_{t_1}^{t_2} \Lambda(\ell){\cdot}e^{
-\Lambda(\ell)\tau}{\cdot}\sum_{\updownmapsto{\ell}
{g,X}{p}{\ell'}}p{\cdot}{\Pr}^{n}(\ell',(\eta{+}\tau))[X:=0])d\tau.

{\Pr}^{\mc{C}}\left(\Paths^{\mc{C}}(\mc{A}[t_{\!f}])\right) \ = \
\sum_{\bar v\in V_F}\int_{Inv(\bar v)}\hbar_{v_0}^{\bar v}(t_f,\vec{0},d\eta),
\label{eq:davispde}
\frac{\partial\hbar_{v}^{\bar v}(y,\eta,\bar\eta)}{\partial
y}+\sum_{i=1}^{|\mc{X}|}
\frac{\partial\hbar_{v}^{\bar v}(y,\eta,\bar\eta)}{\partial\eta^{(i)}}+
\Lambda(v){\cdot}\!\!\sum_{v\stackrel{p,X}{\hookrightarrow}v'}p{\cdot}
\big(\hbar_{v'}^{\bar v}(y,\eta[X:=0],\bar\eta)-\hbar_{v}^{\bar
v}(y,\eta,\bar\eta)\big) \ = \ 0,

\bdM_0(x)=\left(
\begin{array}{ccc}
0& 1{\cdot} r_0{\cdot}e^{-r_0x} & 0\\
0.5{\cdot}r_1{\cdot} e^{-r_1x}& 0 &0.2{\cdot}r_1{\cdot} e^{-r_1x} \\
0 &0& 0
\end{array} \right)
\quad
\bdF_0 =\left(
\begin{array}{cccc}
1& 0 & 0 & 0\\
0& 1 & 0 & 0\\
0& 0 & 1 & 0
\end{array} \right)

\bdM_1(x)=\left(
\begin{array}{cccc}
0& 0 & 0 & 0\\
0 & 0& 0&0\\
0 & 0 & 0 & r_2{\cdot} e^{-r_2x} \\
0 &0& 0 &0
\end{array} \right)
\quad \bdB_1(x) =\left(
\begin{array}{ccc}
0 & r_0{\cdot}e^{-r_0x} &0\\
0.5{\cdot}r_1{\cdot}e^{-r_1x} & 0 &0.2{\cdot}r_1{\cdot}e^{-r_1x}\\
0 & 0 &0\\
0 & 0 &0
\end{array} \right)

\bdF_1 =\left(
\begin{array}{cc}
0 & 0\\
0 & 0 \\
1 & 0 \\
0 & 1
\end{array} \right)
\quad \bdM_1^a(x)=\left(
\begin{array}{ccccccc}
0& 0 & 0 & 0 & 0&  r_0{\cdot}e^{-r_0x}&0\\
0 & 0& 0&0 & 0.5 {\cdot} r_1{\cdot}e^{-r_1x} & 0 & 0.2 {\cdot} r_1{\cdot}e^{-r_1x}\\
0 & 0 & 0 & r_2{\cdot} e^{-r_2x} & 0 & 0 & 0\\
0 &0& 0 &0 & 0 & 0 & 0 \\
0 &0& 0 &0 & 0 & 0 & 0 \\
0 &0& 0 &0 & 0 & 0 & 0 \\
0 &0& 0 &0 & 0 & 0 & 0
\end{array} \right)

\bdPi(x) =\int_{0}^{x}\bdM(\tau)\bdPi(x-\tau)d\tau+\bdD(x).

\vec{\wp}(t)&=&\alpha\cdot\bdPi(t),\\
\frac{d\vec{\wp}(t)}{dt}&=&\vec{\wp}(t)\cdot\Q,\label{eq:trandistr}
\label{eq:delay_m}
\vec{U}_i(x)= \int_{0}^{\Delta
c_i-x}\bdM_i(\tau)\vec{U}_i(x+\tau)d\tau+\int_{0}^{\Delta
c_i-x}\bdB_i(\tau)d\tau\cdot\vec{U}_0(0) +\bdD_i(\Delta
c_i-x)\cdot \bdF_i\vec{U}_{i+1}(0),

\vec{U}_m(x)= \int_{0}^{\infty}\hat{\bdM}_m(\tau)\vec{U}_m(x{+}\tau)d\tau+ \vec{1}_F+\int_{0}^{\infty}\bdB_m(\tau)d\tau\cdot\vec{U}_0(0)\label{eq:last_m} \hat{\bdM}_2(x)=\left( \begin{array}{cc}  0&  r_2{\cdot}e^{-r_2x}\\ 0&0 \end{array}
\right) \quad \hat{\P}_2=\left( \begin{array}{cc} 0& 1\\ 0&0
\end{array} \right)\bdPi_i^a(x)=\left(\begin{array}{c|c}  \bdPi_i(x) & \bar\bdPi_i^a(x)\\
\hline \mathbf{0}&\mathbf{I} \end{array}\right),

\vec{U}_i(0) \ = \
\left\{
\begin{array}{ll}
\bdPi_i(\Delta c_i)\cdot \bdF_i \cdot \vec{U}_{i+1}(0)+\bar \bdPi_i^{a}(\Delta c_i)\cdot \vec{U}_{0}(0)
& \mbox{ if } i < m
\
where  if ;  otherwise and
.
\end{thm}

\proof
Distinguish two cases:  and .
\begin{enumerate}[(1)]
\item (.)
Consider the augmented \CTMC\  with  states.
From equation~\eqref{eq:delay_m}, and the fact that  contains
reset edges of , we have:

where ,  is
the vector representing the reachability probabilities for the augmented states in ,
 such that

is the incidence matrix for delay edges and , and finally
.
\noindent
The proof of the theorem for  proceeds in two steps.
\begin{enumerate}[(a)]
\item[(a)]
We first show that:
1ex]
\bdPi_i^a(x) & = & \displaystyle \int_{0}^{x}\bdM_i^a(\tau)\cdot \bdPi_i^a(x{-}\tau) \, d\tau + \bdD_i^a(x).
\end{array}

\vec{U}_i^{a,(0)}(x)&=& \vec{0}\\
\vec{U}_i^{a,(j+1)}(x)&=&\int_{0}^{c_{i,x}}\bdM_i^a(\tau)\cdot\vec{U}_i^{a,(j)}(x{+}
\tau) \, d\tau+\bdD_i^a(c_{i,x})\cdot\bdF_i^a\cdot\vec{\hat U}_i(0).

\bdPi_i^{a,(0)}(c_{i,x})&=& \mathbf{0}\\
\bdPi_i^{a,(j+1)}(c_{i,x})&=&\int_{0}^{c_{i,x}}\bdM_i^a(\tau)\cdot\bdPi_i^{a,(j)}(c_{
i,x}{-}\tau) \, d\tau+\bdD_i^a(c_{i,x}).
\nonumber
\vec{U}_i^{a,(j)}(x)=\bdPi_i^{a,(j)}(c_{i,x})\cdot\bdF_i^a\cdot\vec{\hat
U}_i(0).

\quad\qquad \qquad\qquad\vec{U}_i^{a,(j+1)}(x) & =&
\int_{0}^{c_{i,x}}\bdM_i^a(\tau)\vec{U}_i^{a,(j)}(x+\tau)d\tau+\bdD_i^a(c_{i,x}
)\cdot\bdF_i^a\vec{\hat U}_i(0)\\
&=&\int_{0}^{c_{i,x}}\bdM_i^a(\tau)\bdPi_i^{a,(j)}(c_{i,x}{-}
\tau)\cdot\bdF_i^a\vec{\hat U}_i(0)d\tau+\bdD_i^a(c_{i,x})\cdot\bdF_i^a\vec{\hat
U}_i(0)\\
& =& \Bigg(\int_{0}^{c_{i,x}}\bdM_i^a(\tau)\bdPi_i^{a,(j)}(c_{i,x}-\tau)d\tau
+\bdD_i^a(c_{i,x})\Bigg)\cdot \bdF_i^a\vec{\hat
U}_i(0)\\
&  =&\bdPi_i^{a,(j+1)}(c_{i,x})\cdot\bdF_i\vec{\hat U}_i(0).
\nonumber
\vec{U}_i^a(0)=\bdPi_i^a(c_{i,0})\cdot\bdF_i^a\vec{\hat U}_i(0).
\nonumber
\left(\begin{array}{c}
\vec{U}_i(0)\\ \hline \vec{U}_i'(0)\\\end{array}\right)&=&\bdPi_i^a(\Delta c_i)\left(\begin{array}{c|c}{\bdF'_i}&{\bdB'_i}\end{array}\right)\left(\begin{array}{c}
\vec{U}_{i+1}(0)\\ \hline \vec{U}_{0}(0)\\\end{array}\right)\\
&=&\left(\begin{array}{c|c}\bdPi_i(\Delta c_i)& \bar\bdPi_i^a(\Delta c_i)\\ \hline \mathbf{0}&\mathbf{I} \end{array}\right)\left(\begin{array}{c|c}\bdF_i& \mathbf{0}\\ \hline \mathbf{0}&\mathbf{I} \end{array}\right)\left(\begin{array}{c}
\vec{U}_{i+1}(0)\\ \hline \vec{U}_{0}(0)\\\end{array}\right)\\
&=&\left(\begin{array}{c|c}\bdPi_i(\Delta c_i)\bdF_i& \bar\bdPi_i^a(\Delta c_i)\\ \hline \mathbf{0}&\mathbf{I} \end{array}\right)\left(\begin{array}{c}
\vec{U}_{i+1}(0)\\ \hline \vec{U}_{0}(0)\\\end{array}\right)\\
&=&\left(\begin{array}{c}
\bdPi_i(\Delta c_i)\bdF_i\vec{U}_{i+1}(0)+\bar \bdPi_i^a(\Delta c_i)\vec{U}_{0}(0)\\ \hline \vec{U}_{0}(0)\\\end{array}\right).
\nonumber
\vec{U}_i(0)=\bdPi_i(\Delta c_i)\bdF_i\vec{U}_{i+1}(0)+\bar \bdPi_i^a(\Delta c_i)\vec{U}_{0}(0)

\Prob_{v_0,\delta}(0) \ = \ e^{-r_0{\cdot}1}{\cdot}\Prob_{v_9}(1) \ = \
  e^{-r_0{\cdot}1}{\cdot}0 \ = \ 0.

\Prob_{v_0,v_1}(0)
\ = \
\int_0^10.4{\cdot} r_0{\cdot}e^{-r_0{\cdot}\tau}{\cdot}\Prob_{v_1}(\tau) \, d\tau
\ = \
\int_0^10.4{\cdot} r_0{\cdot}e^{-r_0{\cdot}\tau} \, d\tau.

{\Pr}^\mC \big( \Paths^\mC(\mA) \big)
\ = \
\int_0^1(0.4+0.6){\cdot} r_0{\cdot}e^{-r_0{\cdot}\tau} \, d\tau
\ = \
\int_0^1 r_0{\cdot}e^{-r_0{\cdot}\tau}d\tau=1-e^{-r_0}.

\end{exa}


\iffalse
\begin{rem}\label{remark:whyBSCC}
We deliberately consider BSCCs in the DTMP 
rather than BSCCs in .
For instance, our example contains two \BSCC s in :
 and , cf.\
Figure~\ref{fig:infinite_region}.
As turned out in the example, only the latter forms a \BSCC\ in the DTMP
 while the former does not.
\marginpar{explain why.}
This is because the guards on the transitions play a role in determining the
acceptance of a path.
The impact of guards, however, is not immediate in ,
but is implicit in the region graph.
\end{rem}
\fi

\subsection*{Verifying qualitative specifications}\label{sec:qualitative}
Until now we have investigated the quantitative verification problem, which is to determine
the value of .
The qualitative verification problem, on the other hand, is to determine whether the
probability that  satisfies  exceeds zero, or, dually, equals one.
For stochastic processes such as finite CTMCs and finite DTMCs, qualitative verification
problems are known to be decidable by means of a simple graph analysis.

\begin{prop}
 For any \CTMC\  and \DTA\ ,
\begin{enumerate}[(1)]
\item
,
\item
,
\end{enumerate}
where  for \DTAr,
 for \DTAo, and  denotes the
weak until operator.
\end{prop}

\begin{proof}
Similar to the case for discrete-time Markov chains~\cite[Chapter 10]{BaKa08}.
\end{proof}

From the above theorem, it follows that the qualitative properties can be verified using a
standard graph-based CTL model checking algorithm, i.e., by just considering the
underlying finite digraph of the PDP  ---basically the
region graph of --- while ignoring the transition probabilities.



\section{Conclusion}\label{sec:concl}
This paper addressed the quantitative (and qualitative) verification of a finite \CTMC\
 against a linear real-time specification given as a deterministic timed automaton
(DTA).
We studied DTA with finite and Muller acceptance criteria.
The key result (for finite acceptance) is that the probability of  equals
the reachability probability in the embedded discrete-time Markov process of a \PDP.
This PDP is obtained via a standard region construction.
Reachability probabilities in the thus obtained PDPs are characterized by a system of
Volterra integral equations of the second type and are shown to be approximated by a
system of  PDEs.
For Muller acceptance criteria, the probability of  equals the reachability
probability of the accepting terminal SCCs in the embedded \PDP.
These results apply to DTA with arbitrarily (but finitely) many clocks.
For single-clock DTA with finite acceptance,  is obtained by
solving a system of linear equations whose coefficients are solutions of a system of ODEs.
As the coefficients are in fact transient probabilities in CTMCs, this result implies that
standard algorithms for CTMC analysis suffice to verify single-clock DTA specifications.

An interesting future research direction is the verification against non-deterministic timed
automata (NTA).
NTA are strictly more expressive than DTA, and thus would allow more linear real-time
specification.
Following the approach in this paper requires a nondeterministic variant of PDP.
Another challenging open problem is to consider real-time linear temporal logics as
specifications such as metric temporal logic (MTL)~\cite{Koy90} or variants thereof.

\section*{Acknowledgement}
\begin{small}
We thank Jeremy Sproston (University of Turin) for fruitful and insightful discussions and
Beno\^it Barbot (ENS Cachan) for pointing out some flaws in an earlier version of this
paper.
We are grateful to the reviewers for providing many useful suggestions on improving the
presentation of the paper.
\end{small}

\bibliographystyle{abbrv}
\bibliography{../bib/myBib,../bib/mybibProceedings,../bib/mybibArticles,../bib/mybibInroceedings,../bib/mybibInbooks}



\end{document}
