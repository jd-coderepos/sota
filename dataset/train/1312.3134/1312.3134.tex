\documentclass{article}
\usepackage{spconf,amsmath,graphicx}
\usepackage{MnSymbol}
\usepackage{algorithmicx}
\usepackage{algpseudocode}
\usepackage{pgfplots}

\def\x{{\mathbf x}}
\def\L{{\cal L}}

\title{Approximate Least Squares\footnote{Preprint of the paper submitted to IEEE International Conference on Acoustics, Speech, and Signal Processing (ICASSP) 2014}}
\name{Michael Lunglmayr , Christoph Unterrieder, Mario Huemer}

\address{ Klagenfurt University, Embedded Systems and Signal Processing, 9020 Klagenfurt, Austria \\
     Johannes Kepler University Linz, Institute of Signal Processing, 4040 Linz, Austria\\
    michael.lunglmayr@aau.at}



\begin{document}
\onecolumn
\maketitle
\sloppy
\begin{abstract}
We present a novel iterative algorithm for approximating the linear least squares 
solution with low complexity. 
After a motivation of the algorithm we discuss the algorithm's
properties including its complexity, and we present theoretical results as well as 
simulation based performance results.
We describe the analysis of its convergence behavior and show that in the noise free 
case the algorithm converges to the least squares solution.
\end{abstract}

\begin{keywords}
least squares, approximation, iterative algorithm, complexity.
\end{keywords}
\section{Introduction}
\label{sec:intro}
The linear least squares (LS) approach is an important and extensively
studied problem in many areas of signal processing with many 
practical applications from localization \cite{localization} 
to battery state estimation \cite{batterystate}. 
In applying the linear LS approach for a vector parameter ,
we assume a signal model  disturbed by noise  such 
that

where  is a known  observation matrix () with 
full rank ,  is a known  vector 
(typically from measurements),  is an unknown  parameter 
vector that is to be estimated and  is an  noise vector.
For the LS approach, the statistical properties of  need not to be known.
For simplicity we only consider real vectors and matrices in this work, 
however, the presented concepts can  
easily be extended for complex vectors and matrices. 
The vector  that minimizes the cost function

is the solution to the LS problem. Here 
 is the  row of  and 
 is the  element of , respectively. The LS 
solution is given by

with  as the pseudoinverse 
of . Numerically more stable algorithms avoiding explicitly calculating , e.g. based on 
the QR decomposition, can for example be found in \cite{numeric}. 
A solution as in (\ref{eqn:batch}) is often called batch solution in literature \cite{kay}.

For real time applications one usually wants to avoid the calculation of the batch solution due to 
its computational complexity and its large memory requirements. Alternatives are
sequential algorithms such as the Sequential Least Squares (SLS) algorithm 
-- described in the next section -- or
gradient based approaches such as the iterative LS (ILS) \cite{numeric} algorithm. 
The latter algorithm is based on the steepest descent approach and iteratively calculates

for iteration . Here  
is the gradient of  at .
For ,  converges to  given that
the iteration step width  fulfills  \cite{numeric}, 
with  as the largest singular value of .
Alternatively, (\ref{eqn:ls_iteration}) can be written as

Analyzing the complexity of this approach one can see that  multiplications are required per iteration.
In addition, every iteration of ILS requires the availability of all elements 
of the measurement vector .

Based on the principle of ILS we propose a novel iterative way of 
approximating 
the least squares solution that we call approximate least squares (ALS). 
As we will 
show, the complexity of this approach is significantly 
lower than for ILS and it requires only
one measurement value  per iteration.

When analyzing (\ref{eqn:ILS_grad}), the gradient can be interpreted as a 
sum of the partial gradients 

as schematically depicted in Fig.~\ref{fig:partgrad}.
\begin{figure}[h]
\centering{\includegraphics[width=.5\columnwidth]{Conv}}
\caption{Gradient and partial gradients of ILS.\label{fig:partgrad}}
\end{figure}
The idea of ALS is to use only \emph{one} of these partial gradients per iteration. 
Instead of moving a small step (due to ) in a steepest descent way in the negative direction of the gradient as done by ILS, 
ALS moves a small step in the negative direction of only a partial
gradient.
This has the advantage
of a lower complexity, but -- as we will discuss below -- also has the disadvantage of a higher noise sensitivity. 
Following this general idea,
two issues have to be addressed. First, the number of iterations 
of the algorithm to achieve satisfying performance results may be higher than the number of rows of . Second, the noise sensitivity has
to be reduced.
To cope with the first issue we suggest to re-use the rows  of {\bf H} in a cyclic manner.
Let the operator ``'' be defined such that
for a positive natural number : m. From this it follows 
that . For better readability we do not
write the dependence of this operator 
on  in the operator's symbol. For ALS,  is always the number of rows of the matrix . 
An ALS iteration is now defined as 

That means for ALS that if  reaches , for the following iterations the first rows of  and the first 
elements of  are used again in a cyclic manner. 
We will now address the second issue, namely the noise sensitivity. 
As we will discuss below, if  then  
converges to  as . For the usual case
, a noise dependent error remains. This
error can be greatly reduced by introducing a simple averaging 
process in the last  iterations. A formal 
justification for this averaging will be given within the error analysis 
in Sect.~\ref{sec:conv_behav}. 
Summarizing, this leads to an overall formulation of the algorithm: \\ 
\clearpage
\vskip-6.5\smallskipamount
\noindent \rule{\columnwidth}{.4pt}
\vskip-1\smallskipamount
\noindent Algorithm: ALS\\
\vskip-7\smallskipamount
\noindent \rule{\columnwidth}{.4pt}
\begin{algorithmic}                    
\State 
\State 
\For {}
\State 
\If {}
\State 
\EndIf
\EndFor
\State 
\end{algorithmic}
\vskip-2\smallskipamount
\noindent \rule{\columnwidth}{.4pt}
Here  denotes the number of iterations of the algorithm and 
 is the approximation of  
that is output by the algorithm.
When analyzing the above algorithm, and only counting the multiplications, one can 
see that  overall multiplications are required to perform the algorithm. 
Compared to ILS a factor of around  fewer multiplications per 
iteration are required. 
Although more iterations are usually needed for ALS  
its overall complexity is significantly lower
as will be demonstrated in Sect.~\ref{sec:sim_res}.
This decrease in complexity is bought with only a 
small degradation in performance. An additional advantage of ALS is 
that per iteration only one value  and only one row 
 of  are required. 
This significantly 
reduces the required number of other operations (additions,
memory accesses,...) and also simplifies the memory management as well 
as the architecture when thinking of a hardware implementation.
\section{Relation to prior work}  
\label{sec:prior} 
ALS not only has similarities to ILS but also to the SLS approach \cite{kay}.
For SLS the update equation

is sequentially calculated  times, requiring an update of the gain vector 
 at every iteration.
Although, the algorithm can deliver  after  iterations, 
the update 
of  requires significant effort, including the multiplication of full 
matrices (although symmetry can be exploited
to reduce the complexity). ALS uses the same update equation 
(usually more than  times),
with the simplified choice .

Update equation (8) is arithmetically similar to the Least Mean Squares 
(LMS) filter update step \cite{LMS}. However, the LMS update step uses a random 
(filter input) vector and one sample of a desired signal as input, whereas the ALS 
update step only uses one sample  of the measurement vector  as input. 
Also the original formulation of the LMS algorithm for the so-called 
ADALINE \cite{LMS_first, ADALINE}
approach was based on a random input vector, providing an adaptive 
approach with a potentially unlimited set of input patterns.
Instead 
of the random input vector in the LMS case the deterministic and fixed rows of the 
observation matrix H are used in the update equation of the ALS. The row vectors 
 and the measurement values  are cyclically re-used. 
Another difference is 
the averaging at the last  iterations which is unique for the ALS algorithm. And 
finally, the convergence behavior of ALS can be described in a completely 
deterministic manner, whereas the convergence of the LMS is usually only described 
in the mean. Anyhow, the authors are confident that some ideas improving LMS -- e.g. adjusting the 
step size \cite{LMS_variable, LMS_variable2} 
might be also
used to further improve the performance of ALS. 
\section{Convergence Behavior}
\label{sec:conv_behav} 
By rewriting (\ref{eqn:als_main_equation}) as

and defining the error vector of ALS 
 together with 
 one gets

Subtracting  left and right from the equation leads to 

When defining 

one can write the above equation as 

with  as the initial error.
Here 
the product of the matrices is defined as .
When analyzing the above equation one can see that the error at iteration  depends on the initial error 
represented in  by the part 
as well as on an error term introduced by noise represented by . With this one can write

If no noise is present then

When choosing  as an integer multiple of  and defining 
 one obtains

In \cite{journal} we show that for the choice

the matrix  has a -norm smaller than one (although the proof is 
not complicated it is omitted here due to length constraints).  This implies that all  
eigenvalues of  have absolute values
smaller than one. 
From this it directly follows that  converges to zero as 
, i.e. .
This means that if no noise is present  
converges to .
However if  a persistent error  remains.
In \cite{journal} we will give a more detailed analysis of
, showing that  features almost a 
periodic behavior 
from an index  on, wherefrom  can be considered negligible.
This particular index , which can also be specified analytically, can be 
used to define  e.g. as .

By analyzing the ALS algorithm one can see the importance of the averaging 
in the final  iterations. 
As we already noted  is highly dependent on the noise for large  (
vanishes with increasing ). 
The averaging over the last  vectors yields


That means by averaging over the last  vectors  an averaging 
over the 
corresponding error vectors occurs.
Since for a practical application it is highly unlikely that all these error vectors 
have equal length and 
point in the same direction
(in this case averaging would have no effect) this averaging step typically 
significantly reduces the error norm.
The averaging only has to be done \emph{once}, it therefore presents only a minor 
complexity 
increase (overall only  additions and  multiplications with the constant 
).

\section{Simulation Results} \label{sec:sim_res} 
We first show simulation results of a typical example of least squares estimation: the estimation
of amplitudes of sine signals in noise. For this demonstration example we 
chose  as a  matrix 
with elements .
The frequencies  are not necessarily integer multiples of a base frequency.
The elements of the noise vector have been 
sampled indepentently from a normal distribution with zero mean and a 
standard deviation . 
The amplitudes  
have been estimated using  values, forming the vector . 
The step size for ILS was chosen as  and
for ALS as .
The estimation performance
has been measured by calculating the norm of the difference vector 
between the true vector  and the estimated vectors, respectively. Fig.~\ref{fig:error_sin}
shows a typical simulation result for ILS and ALS. 
In this figure,  is the estimated parameter vector resulting 
after averaging the 
final  out of  vectors , represented as a horizontal line for 
illustration purposes.
As one can see, ILS requires significantly less
iterations than ALS, but with about  times more multiplications per iteration. 
The
performance of ALS is only slightly worse than the performance of ILS
but ALS features a significantly lower overall complexity. 
In this figure one can observe an interesting behavior of ALS. After a certain number of iterations 
the influence of  becomes negligible. 
This reflects in an oscillatory behavior 
of the error norm as can be seen in Fig.~\ref{fig:error_sin}.
This oscillatory behavior comes from the fact that the values  are cyclically 
re-used in the  ALS iterations. As a consequence also the noise values appear
in a cyclic manner.
The averaging at the end of ALS is most effective if  is chosen large enough
so that the effects of  are negligible. Such a value for  can be found with 
simulations or based on analytical results as will be presented in \cite{journal}.
To provide a fair comparison, in Fig.~\ref{fig:error_sin_mul} we compared ALS, ILS and SLS in 
terms of its error norms over the number of
calculated multiplications. 
As one can see, if the error performance of 
ALS is sufficient for a given application, its complexity is significantly lower. 
In this example ILS needs about  times more multiplications than ALS to obtain
the same error norm. 
But as stated above,
this complexity analysis is only based on the number of multiplications per iteration. 
Including other operations (additions, memory accesses) would furthermore favor 
ALS. Due to page constraints we omitted a more detailed complexity analysis in this paper. 
If the error performance of ALS is not sufficient for 
a given application one could choose a different approach, but extended variants of the ALS, e.g. with 
adjusting  during the iterations
show promising first results towards 
further reducing the error norm. One can immediately see the benefits of such an approach in
(\ref{eqn:first_error_splitting}) because 
the noise dependent part of the error vector scales with , as will be
described in detail in \cite{journal}. 
\begin{figure}[tb]
\begin{center}
\includegraphics[width=.6\columnwidth]{ALS_ICASSP_plot1}
\caption{ Example of errors of ILS and ALS \label{fig:error_sin} }
\end{center}
\end{figure}
\begin{figure}[tb]
\begin{center}
\includegraphics[width=.6\columnwidth]{ALS_ICASSP_plot2}
\caption{ Error norms over to the number of multiplications \label{fig:error_sin_mul} }
\end{center}
\end{figure}
But as extensive performance simulations showed, ALS' performance is on average very close to the LS solution. 
Table.~\ref{tab:perf_res} shows performance results for random  matrices. 
The entries of these matrices have been sampled from a uniform distribution out of . Every 
simulation has been done for white Gaussian noise with 
, 
respectively,
with  random matrices  per  value and  
random vectors 
(with random entries also sampled from a uniform
distribution out of ) per  matrix. For every  
value the averages  and 

over the simulated results have been calculated.
Table.~\ref{tab:perf_res} shows the maximum relative increase of ALS'
averaged error norm over the averaged error norms of LS, whereas the maximization
has been done over the elements of : 
. 
We furthermore want to note that 
the relative increase of the averaged
error norms remained nearly constant over all simulated  values.
As one can see in this table, the performance of ALS shows on average only a minor degradation 
compared to the LS solution.
\begin{table}[htb]
\centering
\begin{tabular}{|l|c||l|c|}
\hline
dim &  & dim &  \\
\hline
   &   &    &  \\
   &   &    &  \\
   &  &    & \\
   &  &    & \\
  &    &   & \\    
\hline
\end{tabular}
\caption{Performance results for random matrices. \label{tab:perf_res} }
\end{table}
\section{Conclusion} \label{sec:sim_res}
We presented a novel algorithm for 
approximating the solution of the linear least squares problem. We discussed its 
convergence behavior and
demonstrated that the algorithm provides a close solution to the least squares solution
with low complexity. The presented algorithm 
shows promising potential for further extension in theory and implementation as well 
for use in a variety of applications.


\vfill\pagebreak

\begin{thebibliography}{1}

\bibitem{localization}
Choi, K.H.; Ra, W.-S., Park, S.-Y.; Park, J.B., ``Robust Least Squares Approach to Passive Target Localization Using Ultrasonic Receiver Array,''
\emph{IEEE Transactions on Industrial Electronics}, vol. 61, no. 4, pp. 1993-2002, Apr. 2014.

\bibitem{batterystate}
Unterrieder, C., Lunglmayr, M., Marsili, S., Huemer, M., ``Battery state-of-charge estimation using polynomial enhanced prediction,''
\emph{IET Electronics Letters}, vol. 48, no. 21, pp. 1363-1365, Oct. 2012.

\bibitem{numeric}
A. Bj\"orck (1996), \emph{Numerical Methods for Least Squares Problems}, SIAM, Philadelphia, 1996.

\bibitem{kay}
S. M. Kay, \emph{Fundamentals of Statistical Signal Processing: Estimation Theory}, Prentice Hall, 2005.

\bibitem{LMS}
B.  Widrow, J. R. Glover, J. M. McCool, J. Kaunitz, C.  S.  Williams,
R. H. Heam,  J.  R. Zeidler, E. Dong, and R. C. Goodlin, ``Adaptive
noise cancelling: Principles and applications,''  \emph{Proc.  IEEE},  vol. 63, pp. 1692-1716, Dec. 1975.

\bibitem{LMS_first}

Widrow, B., Hoff M. E., ``Adaptive Switching Circuits,'' \emph{IRE WESCON Convention Record}, 
Part 4, pp. 96-104, 1960.

\bibitem{ADALINE}
Widrow, B., ``Thinking about thinking: the discovery of the LMS algorithm,'' 
\emph{IEEE Signal Processing Magazine}, vol. 22, no. 1, pp. 100-106, Jan. 2005.

\bibitem{LMS_variable}
R. Harris, D. Chabries, P. Bishop, ``A variable step
(VS) adaptive filter algorithm,'' \emph{IEEE Trans. Acoust. Speech Signal Processing}, 
vol. ASSP-34, pp. 309-316, Apr. 1986.

\bibitem{LMS_variable2}
Bhotto, M.Z.A.; Antoniou, A., ``A Family of Shrinkage Adaptive-Filtering Algorithms,''
\emph{IEEE Transactions on Signal Processing}, vol. 61, no. 7, pp. 1689-1697, Apr. 2013.

\bibitem{journal}
M. Lunglmayr,  C. Unterrieder, M. Huemer, ``Approximate Least Squares: Convergence 
and Performance Analysis,'' in preparation. 

\end{thebibliography}
\end{document}
