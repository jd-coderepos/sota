\documentclass{sig-alternate-05-2015}
\usepackage[numbers,sort&compress]{natbib}
\usepackage{balance}
\usepackage{color, colortbl, fancyvrb,alltt}
\definecolor{LightBlue}{rgb}{0.749,0.75,1.0}\definecolor{LightOrange}{rgb}{0.99,0.799,0.6} \newcommand{\myarr}{} \newcommand{\symbolic}{} \newcommand{\monad}{} \newcommand{\mysens}{}  \newcommand{\mydelta}{}  \newcommand{\mylambda}{}  \let\proof\relax
\let\endproof\relax
\clearpage{}\newcommand{\THESYSTEM}{\textsf{PrivInfer}\xspace}
\usepackage{xspace}
\usepackage{mathpartir}
\usepackage{stmaryrd}
\usepackage{color}
\usepackage{float}
\usepackage{listings}
\usepackage{xfrac}
\usepackage{bbm}
\usepackage{bbold}
\usepackage{url}
\usepackage{paralist}
\usepackage{bussproofs}
\usepackage{xifthen}\usepackage{graphicx}
\usepackage{subcaption}

\pdfoutput=1

\DeclareCaptionFormat{myformat}{#1#2#3\hrulefill}
\captionsetup[figure]{format=myformat}
\newenvironment{myfigure}{\begin{figure}}{\end{figure}}



\usepackage{amsthm}
\usepackage{mathabx}

\theoremstyle{plain}
\newtheorem{lemma}{Lemma}[section]
\newtheorem{theorem}{Theorem}[section]


\theoremstyle{definition}
\newtheorem{definition}{Definition}[section]

\theoremstyle{corollary}
\newtheorem{corollary}{Corollary}[section]


\def\p#1{\mathrel{\ooalign{\hfil\hfil\cr}}}
\renewcommand{\implies}{\Rightarrow}

\usepackage{listings}

\usepackage[T1]{fontenc}
\usepackage{microtype}



\usepackage[scaled]{beramono}
\newcommand\Small{\fontsize{8.2pt}{8.4pt}\selectfont}
\newcommand*\LSTfont{\Small\ttfamily\SetTracking{encoding=*}{-60}\lsstyle}




\lstset{
         language=ML,
         basicstyle=\LSTfont,
extendedchars=true,
         breaklines=true,
         keywordstyle=\bfseries,
         morekeywords={mod,lap,expmech,expect,match,mlet,munit,cunit,clet},
mathescape=true,
         literate={->}{{}}1{=>}{{}}3{-->}{{}}2,
         stringstyle=\ttfamily,
         showspaces=false,
         showtabs=false,
         xleftmargin=8pt,
         showstringspaces=false
 }

\newcommand{\lstt}[1]{\mbox{\LSTfont #1}}
\newcommand\lstiny{\fontsize{6.6pt}{6.8pt}\selectfont}
\newcommand{\lstts}[1]{\mbox{\lstiny\ttfamily\SetTracking{encoding=*}{-60}\lsstyle #1}}
 


\definecolor{DarkGreen}{rgb}{0.1,0.5,0.1}
\definecolor{DarkRed}{rgb}{0.5,0.1,0.1}
\definecolor{DarkBlue}{rgb}{0.1,0.1,0.5}
\usepackage{hyperref}
\hypersetup{
    unicode=false,          pdftoolbar=true,        pdfmenubar=true,        pdffitwindow=false,      pdftitle={},    pdfauthor={}
    pdfsubject={},   pdfnewwindow=true,      pdfkeywords={keywords}, colorlinks=true,       linkcolor=DarkRed,          citecolor=DarkGreen,        filecolor=DarkRed,      urlcolor=DarkBlue,          }
\usepackage[capitalise]{cleveref}

\newcommand{\pys}[1]{\ifdraft\textcolor[rgb]{.30,0.40,0.70}{[Pierre-Yves: #1]}\fi}
\newcommand{\gb}[1]{\ifdraft\textcolor[rgb]{0,0.80,0.80}{[Gilles: #1]}\fi}
\newcommand{\mg}[1]{\ifdraft\textcolor[rgb]{.90,0.00,0.00}{[Marco: #1]}\fi}
\newcommand{\gp}[1]{\ifdraft\textcolor[rgb]{.50,0.00,0.50}{[Gian Pietro: #1]}\fi}
\newcommand{\eg}[1]{\ifdraft\textcolor[rgb]{0.07,0.55,0.07}{[Emilio: #1]}\fi}
\newcommand{\jh}[1]{\ifdraft\textcolor[rgb]{0.99,0.55,0.07}{[Justin: #1]}\fi}
\newcommand{\ar}[1]{\ifdraft\textcolor[rgb]{1,0,0}{[Aaron: #1]}\fi}

\crefname{section}{\S}{\S}
\Crefname{section}{\S}{\S}

\newcommand{\ra}{\rightarrow}
\newcommand{\la}{\leftarrow}

\def\reals{\ensuremath{\mathbb{R}}\xspace}
\def\nats{\ensuremath{\mathbb{N}}\xspace}

\newcommand{\mapx}[3]{{#1}_{#2}^{#3}}

\DeclareMathOperator{\FV}{FV}

\def\vars{\ensuremath{\mathcal{X}}\xspace}
\def\tyvars{\ensuremath{\mathcal{X}_\tau}\xspace}
\def\pcf{\ensuremath{\mathbf{PCF}}\xspace}
\def\pcfp{\ensuremath{\mathbf{PCF}_{p}}\xspace}
\def\vpcf{\ensuremath{\mathbf{vPCF}}\xspace}
\def\pcfty{\ensuremath{\mathbf{Ty}}\xspace}
\def\pcfcoty{\ensuremath{\mathbf{CoreTy}}\xspace}
\def\indct{\ensuremath{\mathcal{I}}}

\def\tyctts{\ensuremath{\mathbb{I}}\xspace}
\def\atytag{\ensuremath{\mathcal{A}}}

\def\ctts{\ensuremath{\mathcal{C}}\xspace}
\def\ctors{\ensuremath{\ctts_{\indtag}}\xspace}
\def\actts{\ensuremath{\ctts_{\atytag}}\xspace}

\def\indtag{\ensuremath{\mathcal{I}}}
\def\indctts{\ensuremath{\tyctts_{\indtag}\xspace}}
\def\atyctts{\ensuremath{\tyctts_{\atytag}\xspace}}

\def\ddomain{\ensuremath{\mathcal{D}}}
\def\adomain{\ensuremath{\ddomain_\atytag}}

\newcommand{\indctn}[1]{\ensuremath{\indct/{#1}}}
\newcommand{\ctorn}[2]{\ensuremath{{#1}/{#2}}}
\newcommand{\tysch}[1]{\tau_{#1}}
\newcommand{\tyinst}[2]{{#1} \mathrel{{\lightning}} {#2}}
\newcommand{\coty}[1]{\widetilde{#1}}

\def\kwcase{\mathtt{case}}
\def\kwwith{\mathtt{with}}
\def\kwis{\mathtt{is}}
\def\kwif{\mathtt{if}}
\def\kwthen{\mathtt{then}}
\def\kwelse{\mathtt{else}}
\def\kwlet{\mathtt{let}}
\def\kwletrec{\mathtt{letrec}}
\def\kwin{\mathtt{in}}
\def\kwunit{\mathtt{return}}
\def\kwbind{\mathtt{mlet}}
\def\kwdatatype{\mathtt{datatype}}
\def\kwinl{\mathtt{inl}}
\def\kwinr{\mathtt{inr}}
\def\kwlist{\mathtt{list}}
\def\kwsk{\mathtt{sk}}
\def\kwbool{\mathtt{bool}}
\def\kwptxt{\mathtt{ptxt}}
\def\kwctxt{\mathtt{ctxt}}
\def\kwint{\mathtt{int}}
\def\kwposreal{\mathtt{posreal}}

\newcommand{\snmark}{\downarrow}
\newcommand{\markvar}{\mathfrak{m}}
\newcommand{\sncond}{\mbox{-guard}}

\newcommand{\slam}[2]{\lambda {#1} .\, {#2}}
\newcommand{\slet}[3]{\kwlet\ {#1} = {#2}\ \kwin\ {#3}}
\newcommand{\sletrec}[4]{\kwletrec^{#1}\ {#2}\ {#3} = {#4}}
\newcommand{\ssnletrec}[3]{\sletrec{\snmark}{#1}{#2}{#3}}
\newcommand{\sif}[3]{\kwif\ {#1}\ \kwthen\ {#2}\ \kwelse\ {#3}}
\newcommand{\sifseq}[5]{\kwcase\ {#1}\ \kwwith\ [\snil \Rightarrow {#5} \vbar \scons{#2}{#3} \Rightarrow {#4}]}
\newcommand{\scase}[2]{\kwcase\ {#1}\ \kwwith\ {#2}}
\newcommand{\sbranch}[2]{{#1} \Rightarrow {#2}}
\newcommand{\sunitM}[1]{\kwunit\ {#1}}
\newcommand{\sbindM}[3]{\kwbind\ {#1} = {#2}\ \kwin\ {#3}}
\newcommand{\sunitC}[1]{{#1}_\uparrow}
\newcommand{\sbindC}[3]{\kwlet_\uparrow\ {#1} = {#2}\ \kwin\ {#3}}
\newcommand{\sone}[0]{()}
\newcommand{\strue}[0]{\mathtt{true}}
\newcommand{\sfalse}[0]{\mathtt{false}}
\newcommand{\spair}[2]{({#1}, {#2})}
\newcommand{\sinl}[1]{\kwinl\ {#1}}
\newcommand{\sinr}[1]{\kwinl\ {#1}}
\newcommand{\snil}[0]{\ensuremath{\epsilon}}
\newcommand{\scons}[2]{{#1} \mathrel{::} {#2}}

\newcommand{\stfun}[2]{{#1} \rightarrow {#2}}
\newcommand{\stapply}[2]{(#2)\,#1}
\newcommand{\stmod}[1]{\mathfrak{M}[{#1}]}
\newcommand{\stmodc}[1]{\mathfrak{C}[{#1}]}
\newcommand{\stunit}[0]{\bullet}
\newcommand{\stbool}[0]{\mathbb{B}}
\newcommand{\stnat}[0]{\mathbb{N}}
\newcommand{\stint}[0]{\mathbb{Z}}
\newcommand{\streal}[0]{\mathbb{R}}
\newcommand{\stuint}[0]{\mathbb{I}}
\newcommand{\stxreal}[0]{\overline{\mathbb{R}}}
\newcommand{\stprod}[2]{{#1} \times {#2}}
\newcommand{\stsum}[2]{{#1} + {#2}}
\newcommand{\stlist}[1]{{#1}\ \kwlist}
\newcommand{\stref}[3]{\{ {#1} : {#2} \vbar {#3} \}}

\newcommand{\sdatatype}[3]{\kwdatatype\ ({#1})\ {#2} = {#3}}

\def\skty{\ensuremath{\mathbb{S}}}

\newcommand{\sk}[2]{\kwsk^{#1}_{#2}}
\newcommand{\ska}[1]{\kwsk_{#1}}

\def\lvmark{\triangleleft}
\def\rvmark{\triangleright}

\renewcommand{\l}[1]{#1_\lvmark}
\renewcommand{\r}[1]{#1_\rvmark}
\def\svar{\mathfrak{s}}

\newcommand{\rembed}[1]{|{#1}|}
\newcommand{\rrembed}[1]{\|{#1}\|}
\newcommand{\rmark}[1]{{#1}^{\Join}}

\newcommand{\tflip}[1]{{#1}^{\leftrightarrow}}

\def\pvars{\ensuremath{\mathcal{X}_{\mathcal{P}}}\xspace}
\def\rvars{\ensuremath{\mathcal{X}_{\mathcal{R}}}\xspace}

\newcommand{\rvar}[1]{\mathbb{#1}}

\newcommand{\expr}[0]{\ensuremath{\mathcal{E}}}
\newcommand{\rexpr}[0]{\ensuremath{\expr^{\Join}}}

\newcommand{\rtypes}[0]{\ensuremath{\mathcal{T}}\xspace}
\newcommand{\rassert}[0]{\ensuremath{\mathcal{A}}\xspace}

\newcommand{\rtmod}[3]{\mathfrak{M}_{#1,#2}[{#3}]}
\newcommand{\rtmodc}[1]{\mathfrak{C}[{#1}]}
\newcommand{\rtprod}[3]{\Pi ({#1} :: {#2}) .\, {#3}}
\newcommand{\rtref}[3]{\{ {#1} :: {#2} \vbar {#3} \}}
\newcommand{\tprod}[3]{\Pi ({#1} : {#2}) .\, {#3}}
\newcommand{\tref}[3]{\{ {#1} : {#2} \vbar {#3} \}}


\def\rfalse{\bot}
\def\rtrue{\top}

\newcommand{\fquant}[4]{{#1}\ ({#2} : {#3}) .\, {#4}}
\newcommand{\rquant}[4]{{#1}\ ({#2} :: {#3}) .\, {#4}}

\def\quantvar{\mathcal{Q}}
\def\rformc{\mathcal{C}}
\def\rpred{\mathcal{P}}

\newcommand{\rmapx}[4]{{#1}{\left\{\substack{\l{#2} \,\mapsto\, {#3}\\ \r{#2} \,\mapsto\, {#4}}\right\}}}

\newcommand{\renv}[1]{\mathcal{#1}}
\newcommand{\rsubst}[2]{{#1} \hookrightarrow {#2}}

\def\red{\mathrel{\rightarrow}}
\def\redctx{\mathtt{C}[\bullet]}
\def\redval{\mathcal{V}}

\newcommand{\tycttinterp}[1]{\vv{#1}}
\newcommand{\ctorinterp}[1]{\vv{#1}}

\newcommand{\tyinterp}[1]{\llbracket {#1} \rrbracket}
\newcommand{\interp}[2]{\llbracket {#2} \rrbracket_{#1}}
\newcommand{\rinterp}[2]{\llparenthesis {#2} \rrparenthesis_{#1}}

\def\tyle{\preceq}

\newcommand{\dotted}[1]{\ensuremath{{#1}_\bot}}
\newcommand{\cfun}{\multimap}



\newcommand{\Pt}[1]{\mathscr{P}({#1})}

\DeclareMathOperator{\dom}{dom}

\newcommand{\vbar}[0]{\mathrel{|}}
\newcommand{\vv}[1]{\overline{#1}}
\newcommand{\vvi}[2]{[{#2}]_{#1}}



\newcommand{\fseven}{\textsc{F}7\xspace}
\newcommand{\fstar}{\textsc{F}\xspace}
\newcommand{\liquidH}{\textsc{LiquidHaskell}\xspace}
\newcommand{\rfstar}{\textsc{RF}\xspace}




\newcommand{\tmod}[1]{\mathfrak{M}[{#1}]}
\newcommand{\tmodsub}[1]{\mathfrak{M}_{\leq}[{#1}]}




\newcommand{\Expect}{\mathbf{E}}

\newcommand{\lift}[2]{\mathcal{L}_{#1}(#2)}
\newcommand{\Lap}{\mathsf{Laplace}}


\def\fprov{\vdash_{F}}
\def\dprov{\vdash_{D}}
\def\distone{\mathfrak{D}}
\def\disttwo{\mathcal{D}}
\def\realone{\mathrm{r}}
\def\natone{\mathrm{n}}
\def\R{\mathbb{R}}
\def\N{\mathbb{N}}

\def\typeone{\sigma}
\def\typetwo{\tau}
\def\typethree{\mu}

\def\tyo{\typeone}
\def\tyw{\typetwo}
\def\tyt{\typethree}
\def\rrb{\rrparenthesis}
\def\llb{\llparenthesis}

\newcommand{\cass}[3]{{#1} :_{[#2]} {#3}}
\newcommand{\cempty}[2]{{#1} :_{\Box} {#2}}


\def\contextone{\Gamma}
\def\contexttwo{\Delta}
\def\contextthree{\Sigma}

\def\ctxo{\contextone}
\def\ctxw{\contexttwo}
\def\ctxt{\contextthree}
\newcommand{\iconstraintone}{\Phi}
\newcommand{\iconstrainttwo}{\Psi}

\def\cso{\iconstraintone}
\def\csw{\iconstrainttwo}

\newcommand{\icontextone}{\phi}
\newcommand{\icontexttwo}{\psi}

\def\tctxo{\icontextone}
\def\tctxw{\icontexttwo}

\def\bcase{\mathop{\bf case}}
\def\binj{\mathrel{\bf inj}}
\def\bfix{\mathop{\bf fix}}
\def\bFix{\mathop{\bf Fix}}
\def\bof{\mathop{\bf of}}
\def\bfail{\mathop{\bf fail}}

\def\Rextp{\ensuremath{\mathbb{S}}\xspace}

\def\tmo{\termone}
\def\termone{e}

\newcommand{\DFuzz}{{{\em DFuzz}}\xspace}
\newcommand{\Fuzz}{{{\em Fuzz}}\xspace}

\newcommand{\rname}[1]{\quad #1}

\def\primone{\mathrm{f}}
\def\lin{\multimap}

\def\breturn{\mathop{\mathbf{return}}}

\newcommand{\exvar}[1]{\lceil #1\rceil}

\newcommand{\ruleitapp}{\ensuremath{\forall I}}
\newcommand{\ruleetapp}{\ensuremath{\forall E}}
\newcommand{\rulenate}{\ensuremath{\N~E}}
\newcommand{\ruleliste}{\ensuremath{\List{\tau}~E}}

\def\kindone{\kappa}
\def\ikind{\iota}
\def\rkind{\upsilon}
\newcommand{\List}[1]{\mathsf{list}\,{#1}}

\def\bnil{\mathop{\bf nil}}
\def\sizeivarone{i}
\def\sizeivartwo{j}
\def\sensivarone{k}
\def\ivarone{a}
\def\ivartwo{b}
\def\ivarthree{c}
\def\itermone{I}
\def\sensitermone{R}
\def\sensitermtwo{T}
\def\sensitermthree{W}
\def\sizeitermone{S}
\def\sizeitermtwo{V}
\def\sizeitermthree{U}
\def\sizemetaone{I}
\def\sizemetatwo{J}
\newcommand{\constant}[1]{{#1}}

\def\rprob{\mathcal{P}}
\def\mtypeone{A}
\def\bncase{\mathop{\bf case_{\N}}\nolimits}


\newcommand{\eraseDP}[1]{| #1 |}

\newcommand{\equref}[1]{{=_{#1}}}
\newcommand{\distance}[1]{{d({#1})}}

\newcommand{\rplusinfty}{\ensuremath{\overline{\R}^+}}
\newcommand{\rplusbottom}{\ensuremath{\R^+_\bot}}
\newcommand{\rplus}{\R^+}

\newcommand{\M}[2][]{\ifthenelse{\isempty{#1}}{\mathsf{M}\, {#2}}
              {\mathsf{M}_{#1}\, {#2}}}

\newcommand{\aty}{\mathcal{A}} \newcommand{\ity}{\mathcal{I}} \newcommand{\ty}{\mathcal{T}}
\newcommand{\etyr}[1]{\overline{\mathcal{#1}}}
\newcommand{\ety}{\etyr{T}}
\newcommand{\uty}{\star} \newcommand{\bty}{\mathcal{B}}
\newcommand{\assn}{\mathcal{F}}
\newcommand{\rty}[3]{ \{ #1 :: #2 \ |\ #3 \}}
\newcommand{\exty}[3]{\exists #1 : #2 \cdot #3}
\newcommand{\real}[0]{\ensuremath{\mathbb{R}}\xspace}
\newcommand{\intg}[0]{\ensuremath{\mathbb{Z}}\xspace}
\newcommand{\natty}[0]{\ensuremath{\mathbb{N}}\xspace}
\newcommand{\V}{\mathcal{V}}

\newcommand{\E}{\mathcal{E}}
\newcommand{\C}{\mathcal{C}}
\newcommand{\D}{\mathcal{D}}
\newcommand{\Q}{\mathcal{Q}}
\renewcommand{\P}{\mathcal{P}}
\newcommand{\bra}[1]{\ensuremath{\{#1\}}}

\newcommand{\lam}[2]{\lambda #1 . #2}
\newcommand{\fixp}[2]{\mu #1. #2}
\newcommand{\vlet}[3]{\mathsf{let}~ #1\ = #2 \mathop{\mathsf{in}} #3}
\newcommand{\mlet}[3]{\mathsf{bind}~ #1 \leftarrow #2 ~\mathsf{in}~ #3}
\newcommand{\mletrec}[3]{\mathsf{let rec}~ #1\ #2 = #3}
\newcommand{\mcase}[2]{\mathsf{match}~ #1 ~\mathsf{with~} #2}
\newcommand{\mmatchfix}[3]{\mathsf{let rec}~ #1 = \mcase{#2}{#3}}
\newcommand{\unit}[1]{\mathsf{unit}~ #1}
\def\lpRHL{\textsf{RFuzz}\xspace}
\def\PCF{\textsf{PCF}}

\newcommand{\side}[1]{\langle #1 \rangle}
\newcommand{\sidel}{{\side{1}}}
\newcommand{\sider}{{\side{2}}}
\newcommand{\sidei}{{\side{i}}}
\newcommand{\erase}[1]{| #1 |}

\newcommand{\observe}[3]{\mathsf{observe}\, #1\Rightarrow #2\,
  \mathsf{in}\, #3}
\newcommand{\binfer}[1]{\mathsf{infer}(#1) }
\newcommand{\bran}[1]{\mathsf{ran}(#1) }
\newcommand{\bernoulli}[1]{\mathsf{bernoulli}(#1) }
\newcommand{\normal}[2]{\mathsf{normal}(#1,#2) }
\newcommand{\bivnorm}[2]{\mathsf{bivariate-normal}(#1,#2) }
\newcommand{\laplace}[2]{\mathsf{lapMech}(#1,#2) }
\newcommand{\gauss}[2]{\mathsf{gaussMech}(#1,#2) }
\newcommand{\betad}[2]{\mathsf{beta}(#1,#2) }
\newcommand{\uniform}{\mathsf{uniform}() }
\newcommand{\stdist}[1]{\mathfrak{D}[{#1}]}
\newcommand{\exponential}[3]{\mathsf{expMech}(#1,#2,#3) }
\newcommand{\fdiv}{\ensuremath{f}}
\newcommand{\distr}{\mathcal{D}}
\newcommand{\Distr}[1]{\mathsf{distr}\,{#1}}
\newcommand{\charfun}{\ensuremath{\mathbb{1}}}
\providecommand{\eqdef}{\raisebox{-.2ex}[.2ex]{}}
\clearpage{}
\begin{document}
\CopyrightYear{2016}
\setcopyright{acmlicensed}
\conferenceinfo{CCS'16,}{October 24 - 28, 2016, Vienna, Austria}
\isbn{978-1-4503-4139-4/16/10}\acmPrice{\\scriptstyle *\#1065060\#1513694\#360368\mathsf{HOARe}^2\mathsf{observe}\mathsf{observe}\mathsf{infer}\mathsf{HOARe}^2\mathsf{HOARe}^2f\mathsf{HOARe}^2\ell_1\mathsf{observe}\ell_p\Pr(\xi)\xix\Pr(\xi \mid x)\Pr(x \mid \xi)\xix\mathcal{L}_{x}(\xi)\xixx\xix\Pr(x)\Pr(\xi \mid x)\xi\epsilon\sf ExpMech_{\epsilon}{\sf Q}{\sf Q}\epsilonnd,d'01\xi[0,1]a,b\in \rplusB\xi\sf infer{\sf observe}{\sf beta}(a,b)(\lambda r. {\sf bernoulli}(r)={\sf obs}){\sf Q}{\sf
  obs}1bb={\sf obs}01(\epsilon,0){\sf
  beta}(a',b')a',b'{\sf  beta}(a',b')\sf Q(a',b')\ell_1d((a,b),(a',b'))=|a-a'|+|b-b'|{\sf  beta}(a',b')\Delta_{\mathcal{H}}({\sf  beta}(a,b),{\sf
  beta}(a',b'))\distr(A)A\mu:A\to [0,1]\mathsf{support}(\mu)=\{x \mid \mu\,x\neq 0 \}\sum_{x\in
  A}\mu\,x=1\mathsf{uniform},
\mathsf{bernoulli}\mathsf{normal},\mathsf{beta}a\in
A\charfun_aa\mathsf{bind}\, \mu\, M\muAMABddd, d'dd'd\, \Phi\, d'\epsilon, \delta > 0DRM : D \to
  \distr(R)(\epsilon, \delta)d, d' \in Dd\ \Phi\ d'S \subseteq R\epsilon\epsilon\textnormal{-}\mathsf{D}\epsilon, \delta \in \rplus DRM : D \to \distr(R)(\epsilon, \delta)\epsilon\mathsf{D}(M(d), M(d'))\leq \deltad,d'\epsilon\mathsf{D}(\mu_1,\mu_2)\equiv\displaystyle \max_{E\subseteq R} \big( \Pr_{x\leftarrow \mu_1} [x\in E] - e^{\epsilon}\cdot\Pr_{x\leftarrow \mu_2}[x\in E] \big)\mu_1, \mu_2\in \distr(R)M:D\rightarrow\distr(R)(\epsilon, \delta)N:R\rightarrow \distr(R')\lambda d. \mathsf{bind}\, (M\, d)\, N:D\rightarrow\distr(R')(\epsilon,\delta)M_1:D\rightarrow\distr(R_1)M_2:D\rightarrow\distr(R_2)(\epsilon_1,\delta_1)(\epsilon_2,\delta_2)M:D\rightarrow \distr(R_1\times
R_2)M(x)\equiv (M_1(x),M_2(x))M(\epsilon_1+\epsilon_2,\delta_1+\delta_2)k \in \rplusf : A \to BABd_Ad_Bfka, a' \in Ak\epsilon \in \rplusf : D \to \mathbb{R}d \in Df(d) +
  \nu\nu{1/\epsilon}fk(k\epsilon,
  0)(\epsilon, 0)\epsilon(\epsilon,
\delta)\delta \in \rplus\epsilon, \delta \in \mathbb{R}f : D \to
  \mathbb{R}d \in Df(d) + \nu\nufkk < 1/\epsilon(k\epsilon, \delta)Rq : D \times
R \to \mathbb{R}r \in Rq\epsilon \in \rplusRq :
  D \times R \to \mathbb{R}d \in Dr \in Rfkdr \in R(k\epsilon, 0)\fdiv(x)x>0\fdiv(1)=0\mu_1,\mu_2A\mu_1\mu_2\Delta_\fdiv(\mu_1\ \mid \mu_2)0 \cdot \fdiv(\frac{0}{0})=0\Delta_\fdiv(\mu_1\ \mid \mu_2)\leq \delta\mu_1\mu_2(\fdiv,\delta)\epsilon{\fdiv}(x)\mathsf{SD}(x)\frac{1}{2}\ |x-1|\displaystyle{\sum_{a\in
      A}}\frac{1}{2}|\mu_1(a)-\mu_2(a)|\mathsf{HD}(x)\frac{1}{2}\ (\sqrt{x}-1)^2\displaystyle{\sum_{a\in
      A}}\frac{1}{2}\Big(\sqrt{\mu_1(a)}-\sqrt{\mu_2(a)}\Big)^2\mathsf{KL}(x)x\ln(x)-{x}+1\displaystyle{\sum_{a\in
      A}}\mu_1(a)\ln\Big(\frac{\mu_1(a)}{\mu_2(a)}\Big)\epsilon\mathsf{D}(x)\max(x-e^\epsilon,0)\displaystyle{\sum_{a\in
      A}}\max\Big({\mu_1(a)}- e^\epsilon {\mu_2(a)},0\Big)f\mathsf{SD}\mathsf{HD}\mathsf{KL}\epsilon\epsilon\mathsf{D}\fdiv\epsilon(\epsilon,\delta)\mathcal{F}\fdiv\fdiv\in\mathcal{F}\mu_1,\mu_2AMAB\fdiv_1,\fdiv_2,\fdiv_3\in\mathcal{F}(\fdiv_1,\fdiv_2)\fdiv_3A,B\mu_1,\mu_2AM_1,M_2AB (\epsilon_1\text{-}\mathsf{D}, \epsilon_2\text{-}\mathsf{D})(\epsilon_1+\epsilon_2)\text{-}\mathsf{DP}(\mathsf{SD}, \mathsf{SD})\mathsf{SD}(\mathsf{HD}, \mathsf{HD})\mathsf{HD}(\mathsf{KL}, \mathsf{KL})\mathsf{KL}\begin{array}{rcl}
    e
      & ::=   & x \vbar c \vbar e\ e \vbar \slam{x}{e} \\ & \vbar & \sletrec{}{f}{x}{e} \vbar
                \scase{e}{\vvi{i}{\sbranch{d_i\ \vv{x_i}}{e_i}}}\\
      & \vbar &  \sunitM{e}    \vbar \sbindM{x}{e}{e}\\
&\vbar & \observe{x}{e}{e} \vbar \binfer{e} \vbar \bran{e}\\
&\vbar & \bernoulli{e} \vbar \normal{e}{e} \vbar  \betad{e}{e} \vbar
\uniform\\
 &\vbar &
 \laplace{e}{e}\vbar \gauss{e}{e}
\vbar \exponential{e}{e}{e}
  \end{array}c\cttsx\mathcal{X}\mathcal{X}\mathsf{uniform}: \stdist{[0,1]}\mathsf{bernoulli}: [0,1] \to \stdist{\stbool}\mathsf{beta}: \rplus\times \rplus \to
    \stdist{[0,1]}\mathsf{normal}: \reals\times \rplus \to \stdist{\R}\mathsf{lapMech}: \rplus\times \R \to
\stmod{\reals}\mathsf{gaussMech}: \rplus\times \R \to \stmod{\R}\mathsf{expMech}: \R\times ((D,R)\to \R) \times D \to
\stmod{R}\begin{array}{rcl}
    \tau,\sigma
      & ::= & \coty{\tau} \vbar
              \stmod{\coty{\tau}} \vbar               
              \stmod{\stdist{\coty{\tau}}} \vbar               
              \stdist{\coty{\tau}} \vbar
              \stfun{\tau}{\sigma}\\
    \coty{\tau}
      & ::= & \stunit \vbar \stbool \vbar \stnat \vbar \R \vbar \rplus\vbar
              \rplusinfty \vbar [0,1]\vbar
              \stlist{\coty{\tau}} .
  \end{array}\coty{\tau}\Gamma\vdash e:\tau\Gamma\stunit\stbool\stnat\R\rplus\rplusinfty[0,1]\stmod{\coty{\tau}}\coty{\tau}\stdist{\coty{\tau}}\coty{\tau}\stmod{\coty{\tau}}\sbindM{x}{e_1}{e_2}\sunitM{e}\bernoulli{e}\normal{e_1}{e_2}{\sf getParams}\laplace{e_1}{e_2}\gauss{e_1}{e_2}\exponential{e_1}{e_2}{e_3}\observe{x}{e_1}{e_2}e_2e_1xe_1\binfer{e}\bran{e}\binfer{e}\tyinterp{\stunit}=\{\stunit\}\tyinterp{\stbool}=\{\mathsf{true},\mathsf{false}\}\tyinterp{\stnat}=\{0,1,2,\ldots\}\tyinterp{\tau\to \sigma}\tyinterp{\tau}\to\tyinterp{\sigma}\stmod{\tau}\tau\in \{\coty{\tau},\stdist{\coty{\tau}}\}\tau\stdist{\coty{\tau}}\stdist{\stbool}\theta\theta\Gamma\forall x:\tau\in \Gamma\theta(x)\in \tyinterp{\tau}\interp{\theta}{\sunitM{e}}\interp{\theta}{e}\sbindM{x}{e_1}{e_2}\mathsf{observe}\, x\Rightarrow t\,  \mathsf{in}\, u\interp{\theta}{u}x\Rightarrow t\mathsf{infer}maybe\mathsf{AlgInf}\interp{\theta}{\bernoulli{e}}\mathsf{ran}\bernoulli{e}\Gamma\vdash e:\tau\theta\Gamma\interp{\theta}{e}\in\tyinterp{\tau}\rvars\pvarsx\in \rvars\l{x}\r{x}\rmark{\rvars}\bigcup_{x \in \rvars} \{ \l{x}, \r{x} \}\rmark{\vars}\rmark{\rvars} \cup \pvars\expr\pcfp(\pvars)\rexpr\pcfp(\rmark{\vars})\rtypes = \{ T, U, \ldots \}\rassert = \{ \phi, \psi, \ldots \}\begin{array}{rrl}
    T, U \in \rtypes
      & ::= & \coty{\tau} \vbar
              \rtmod{\fdiv}{\delta}{\rtref{x}{\coty{\tau}}{\phi}}
\vbar
              \rtmod{\fdiv}{\delta}{\rtref{x}{\stdist{\coty{\tau}}}{\phi}}\\
&&
              \vbar
              \stdist{\coty{\tau}} \vbar
              \rtprod{x}{T}{T} \vbar
              \rtref{x}{T}{\phi}\.3em]
                \rquant{\quantvar}{x}{T}{\phi} & (x \in \rvars)\.3em]
    \rformc &= & \{ \sfrac{\rtrue}{0}, \sfrac{\rfalse}{0},
                     \sfrac{\neg}{1}, \sfrac{\vee}{2}, \sfrac{\wedge}{2},
                     \sfrac{\implies}{2} \} , 

  \end{array}\fdiv, \delta, \rmark{e} \in \rexpr\quantvar\in\{\forall, \exists\}\rtref{x}{T}{\phi}\phi\Delta^{\mathfrak{D}}_\fdiv(\rmark{e},\rmark{e})\leq \delta\fdiv\in\mathcal{F}\ \rmark{e} = \rmark{e}\rmark{e} \le \rmark{e}\rtmod{\fdiv}{\delta}{\rtref{x}{T}{\phi}}T\in\{\coty{\tau},\stdist{\coty{\tau}}\}\fdiv\delta\rtprod{x}{T}{S}\renv{G}(x :: T)x\rvarsx\renv{G}x\renv{G}\rembed{\cdot}\rrembed{\cdot}x_\svar\rrembed{\renv{G}} =
x\rembed{\renv{G}}x\in \dom(\renv{G})\svar \in \{
\lvmark, \rvmark \}(x::T)\rrembed{(x::T)}= \l{x}:\rembed{T},\r{x}:\rembed{T}\renv{G} \vdash e_1 \sim e_2 :: T\Gamma \vdash e :: T\Gamma \vdash e
\sim e :: T(\fdiv_1,\fdiv_2)\fdiv_3\delta''\delta\delta'\delta\delta'(\fdiv,\delta)(\fdiv,\delta)\Psi\Psi\subseteq T_1\times T_2\ \delta\in\rplus\mu_1\in\tmod{T_1}\mu_2\in\tmod{T_2}(\fdiv, \delta)\Psi\lift{(\fdiv, \delta)}{\Psi}\mu_L,\mu_R\in\tmod{T_1\times T_2}\mu_i\, (a,b) > 0(a,b)\in \Psii\in\{L,R\}\pi_1\, \mu_L =\mu_1\land \pi_2\, \mu_R=\mu_2\Delta_{\fdiv}(\mu_L, \mu_R)\leq \delta\pi_1\, \mu = \lambda x. \sum_{y} \mu\, (x,y)\pi_2\, \mu
= \lambda y. \sum_{x} \mu\, (x,y)\mu_L\mu_R\theta\renv{G}\theta \VDash \renv{G}\theta \vDash
 \rrembed{\renv{G}}\forall x \in \dom(\renv{G})(\l{x}\theta, \r{x}\theta) \in \rinterp{\theta}{x\renv{G}}\interp{\theta}{\phi} \in \{ \top, \bot \}\phi\theta \vDash \Gamma\Delta_\fdiv^\mathfrak{D}(\rmark{e}_1, \rmark{e}_2)\leq \delta\fdiv\rinterp{\theta}{T}T\theta \vDash \rrembed{\renv{G}}\stdist{\coty{\tau}}\stdist{\coty{\tau}}\renv{G} \vdash e_1 \sim e_2 :: T\theta\models\renv{G}(\interp{\theta}{e_1},\interp{\theta}{e_2})\in \rinterp{\theta}{T}\vdash e ::  \rtmod{\fdiv}{\delta}{\rtref{y}{\tau}{\l{y}=\r{y}}}(\mu_1,\mu_2)\in\interp{}{e}\Delta_{\fdiv}(\mu_1,\mu_2)\leq \delta\vdash e :: \rtref{x}{\sigma}{\Phi} \rightarrow
  \mathfrak{M}_{\epsilon\textnormal{-}\mathsf{D},\delta}{\rtref{y}{\tau}{\l{y}=\r{y}}}\interp{}{e}(\epsilon,\delta)\interp{}{\Phi}\begin{array}{r@{\;}c@{\;}l}
    
      \interp{\theta}{\fdiv\in\mathcal{F}} & = &
        \interp{\theta}{\fdiv}\in\mathcal{F}\.5em]
    
    \end{array}\ell_1\Phi\rightarrow\rightarrow\{x::T \mid = \}\{x::T \mid  \l{x}=\r{x} \}{\sf learnBias}\fdiv\in\mathcal{F}\epsilon\mathfrak{D} [[0,1]]\ell_1(\epsilon,\delta)\ell_1\ell_{1}\ell_12\epsilonnuk=\big(\frac{1}{hV^2}+\frac{n}{kv^2}\big)^{-1}s=\frac{hV}{ kv + hV}kvhV\rhod_1, d_2: \stboold_1 \Phi d_2a,b \in \rplus\Pr(\xi)={\sf Beta}(a,b)\Delta_{\mathsf{HD}}(\Pr(\xi \mid d_1), \Pr(\xi \mid d_2))\leq
\sqrt{1-\frac{\pi}{4}}=\rho\rhol_1l_2l_1\ \Phi\ l_2\rho(\rho\epsilon,0)d_1, d_2: \stboold_1 \Phi d_2a,b \in \rplus\Pr(\xi)={\sf Beta}(a,b)\Delta_{\mathsf{SD}}(\Pr(\xi \mid d_1), \Pr(\xi \mid d_2))\leq
\sqrt{2(1-\frac{\pi}{4})}=\zetak\in \mathbb{N}^{\geq2}d_1, d_2: \stlist{ [k]}d_1 \Phi d_2a_1,a_2,\dots,a_k\in \rplus\Pr(\xi)={\sf Dirichlet}(a_1,a_2,\dots,a_k)\Delta_{\mathsf{HD}}(\Pr(\xi \mid d_1), \Pr(\xi \mid d_2))\leq
\sqrt{1-\frac{\pi}{4}}=\rho(\rho\epsilon,0)\mathsf{HOARe}^2$, a relational
refinement type system that was recently proposed by \citet{BartheGAHRS15}. This
system has been used for verifying differential privacy of algorithms, and more
general relational properties like incentive compatibility from the field of
mechanism design. However, it cannot model probabilistic inference.

\paragraph*{Probabilistic programming}
Research in probabilistic programming has emerged early in the 60s and
70s, and is nowadays a very active research area. Relevant to our work is in particular the research in probabilistic programming
for machine learning and statistics which has been very active in
recent years. Many probabilistic programming languages have been
designed for these applications, including WinBUGS~\citep{LunnTBS00},
IBAL~\citep{Pfeffer01}, Church~\citep{DBLP:conf/uai/GoodmanMRBT08},
Infer.net~\citep{InferNET12}, Tabular~\citep{GordonGRRBG14},
Anglican~\citep{TolpinMW15}, Dr. Bayes~\citep{TorontoMH15}. Our goal is not
to provide a new language but instead is to propose a framework where
one can reason about differential privacy for such languages.
For instance, we compiled programs written in Tabular ~\citep{GordonGRRBG14}  into \THESYSTEM
so that differential privacy could be verified. More information on this translation can be found in the supplementary material section.
Another related work is the one by \citet{AdamsJ15} proposing a type theory for Bayesian inference. While
technically their work is very different from ours it shares the 
same goal of providing reasoning principles for Bayesian inference.
Our work considers a probabilistic \pcf for discrete distributions.
It would be interesting to extend our techniques to higher-order languages with continuous distributions and conditioning,
by building on the rigorous foundations developed in recent work \cite{lics16,icfp16}.

\section{Conclusion}
We have presented \THESYSTEM, a type-based framework for differentially private
Bayesian inference. Our framework allows to write data analysis as
functional programs for Bayesian inference and add to noise to them in different
ways using different metrics. Besides, our framework allows to reason about general
\fdiv-divergences for Bayesian inference. 

Future directions include exploring the use of this approach to
guarantee robustness for Bayesian inference and other machine learning
techniques~\citep{Dey1994287}, to ensure differential privacy using conditions
over the prior and the likelihood similar to the ones studied by
\citet{Zheng16,ZhangRD16}, and investigating further uses of \fdiv-divergences for
improving the utility of differentially private Bayesian
learning. On the programming language side it would also be
interesting to extend our framework to continuous distributions
following the approach by~\citet{sato2016}. We
believe that the intersection of programming languages, machine
learning, and differential privacy will reserve us many exciting results.

\bibliographystyle{abbrvnat}
\bibliography{newbib}
\end{document}
