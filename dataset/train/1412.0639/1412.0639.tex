In this section, we show our reduction from $\alpha$-composition pair isomorphism to low-degree graph isomorphism.  Our reduction also extends to reducing $\alpha$-composition pair canonization to computing canonical forms of low-degree graphs.  Our proofs follow an outline similar to the analogous reduction from composition series isomorphism to low-degree graph isomorphism in the case of $p$-groups~\cite{rosenbaum2013c}, but are more complex due to the more general structure of solvable groups.

\subsection{Isomorphism testing}
At a high level, our algorithm consists of the following steps.  First, we augment our $\alpha$-composition pair $(P_1, P_2)$ by choosing an ordered generating set $\bmg$ for the subgroup $P_1$ (which corresponds to the large primes) to obtain the \emph{augmented $\alpha$-composition pair} $(P_1, S_2, \bmg)$.  We say that a mapping $\phi : G \ra H$ is an isomorphism between the augmented $\alpha$-decompositions $(P_1, S_2, \bmg)$ and $(Q_1, S_2', \bmh)$ for $G$ and $H$ if $\phi$ is an $\alpha$-composition pair isomorphism for $(P_1, S_2)$ and $(Q_1, S_2')$ and $\phi(\bmg) = \bmh$.  The reason for choosing an augmented $\alpha$-composition pair is so that we can reduce the degree of the part of the graph we construct that corresponds to $P_1$ using the trick due to Wagner~\cite{wagner2011a} mentioned in the introduction.

Since one can fix an ordered generating set $\bmg$ for $P_1$ and consider all possible ordered generating sets for $Q_1$, it is easy to see that $\alpha$-composition pair isomorphism is $n^{\log_{\alpha} n + O(1)}$ Turing-reducible to augmented $\alpha$-composition pair isomorphism.  (Recall that we will later set $\alpha = \log n / \log \log n$ so this is $n^{O(\log n / \log \log n)}$ time and is less than the complexity we are aiming for.)  We state this in the following lemma.

\begin{lemma}
  \label{lem:alpha-comp-red}
  Testing isomorphism of the $\alpha$-composition pairs $(P_1, S_2)$ and $(Q_1, S_2')$ for the solvable groups $G$ and $H$ is $n^{\log_{\alpha} n + O(1)}$ deterministic time Turing reducible to testing isomorphism of augmented $\alpha$-composition pairs for $(P_1, S_2)$ and $(Q_1, S_2')$ where $p$ is the smallest prime dividing the order of the group.
\end{lemma}

We then construct a tree whose leaves represent the elements of $G$; by using the ordered generating set $\bmg$ chosen above, we are able to ensure that the degree of this tree is at most $\alpha + O(1)$.  By augmenting this tree with gadgets that represent the multiplication table of the group, we obtain an object that represents the isomorphism class of the augmented $\alpha$-composition pair $(P_1, P_2, \bmg)$.  The final step of the algorithm is to apply the following result\intoct{ due to Babai and Luks~\cite{babai1983a,babai1983b}} for computing canonical forms of low-degree graphs.

\intoctornot{
\begin{theorem}
  \label{thm:const-deg-can}
  Canonization of colored graphs of degree at most $d$ is in $n^{O(d)}$ time.
\end{theorem}}{
\begin{theorem}[Babai and Luks~\cite{babai1983a,babai1983b}]
  \label{thm:const-deg-can}
  Canonization of colored graphs of degree at most $d$ is in $n^{O(d)}$ time.
\end{theorem}}

The main challenge compared to $p$-group isomorphism~\cite{rosenbaum2013c} is dealing with the fact that some of the prime divisors of a solvable group can be small while others may be large.  This is the main reason why the correctness proof is significantly more complex than for $p$-groups.  Since a $p$-group has exactly one prime divisor, it was possible to handle the cases of small and large primes separately using a graph-isomorphism based $p$-group algorithm~\cite{rosenbaum2013c} (which is fast when the prime is small) and the generator-enumeration algorithm (which is fast when the prime is large).  On the other hand, for solvable groups, it is necessary to design a hybrid algorithm that is fast for both cases simultaneously.

As mentioned above, the first step in the graph construction is to define a tree for an augmented $\alpha$-composition pair $(P_1, P_2, \bmg)$.  We do this by constructing trees $T_1$ and $T_2$ whose leaves correspond to the elements of $P_1$ and $P_2$.  In order to define the part of the tree corresponding to $P_1$, we need a way to canonically order the elements of a group given an ordered generating set.  For completeness, we state and prove the required properties from~\cite{rosenbaum2013c}.

\intoctornot{
\begin{definition}
  \label{defn:gen-ord}
  Let $G$ be a group with an ordered generating set $\bmg = (g_1, \ldots, g_k)$.  Define a total order $\prec_{\bmg}$ on $G$ by $x \prec_{\bmg} y$ if $w_{\bmg}(x) \prec w_{\bmg}(y)$ where each $w_{\bmg}(x) = (x_1, \ldots, x_j)$ is the first word in $\{g_1, \ldots, g_k\}^*$ under the standard ordering such that $x = x_1 \cdots x_j$.
\end{definition}}{
\begin{definition}[\cite{rosenbaum2013c}]
  \label{defn:gen-ord}
  Let $G$ be a group with an ordered generating set $\bmg = (g_1, \ldots, g_k)$.  Define a total order $\prec_{\bmg}$ on $G$ by $x \prec_{\bmg} y$ if $w_{\bmg}(x) \prec w_{\bmg}(y)$ where each $w_{\bmg}(x) = (x_1, \ldots, x_j)$ is the first word in $\{g_1, \ldots, g_k\}^*$ under the standard ordering such that $x = x_1 \cdots x_j$.
\end{definition}
}

\intoctornot{
We will also need the following lemma from~\cite{rosenbaum2013c}.

\begin{lemma}
  \label{lem:gen-ord}
  Let $G$ and $H$ be groups with ordered generating sets $\bmg = (g_1, \ldots, g_k)$ and $\bmh = (h_1, \ldots, h_k)$, and let $x, y \in G$.  Then

  \begin{enumerate}
  \item $\prec_{\bmg}$ is a total ordering on $G$.
  \item if $\phi : G \ra H$ is an isomorphism such that each $\phi(g_i) = h_i$, then $x \prec_{\bmg} y$ \ifft $\phi(x) \prec_{\bmh} \phi(y)$.
  \item we can decide if $x \prec_{\bmg} y$ in $O(n \abs{\bmg})$ time.
  \end{enumerate}
\end{lemma}}{
\begin{lemma}[\cite{rosenbaum2013c}]
  \label{lem:gen-ord}
  Let $G$ and $H$ be groups with ordered generating sets $\bmg = (g_1, \ldots, g_k)$ and $\bmh = (h_1, \ldots, h_k)$, and let $x, y \in G$.  Then

  \begin{enumerate}
  \item $\prec_{\bmg}$ is a total ordering on $G$.
  \item if $\phi : G \ra H$ is an isomorphism such that each $\phi(g_i) = h_i$, then $x \prec_{\bmg} y$ \ifft $\phi(x) \prec_{\bmh} \phi(y)$.
  \item we can decide if $x \prec_{\bmg} y$ in $O(n \abs{\bmg})$ time.
  \end{enumerate}
\end{lemma}
}

\begin{proof}
  Let $S = \{g_1, \ldots, g_k\}$.  For part (a), it is clear that $\prec_{\bmg}$ is a total order since $w_{\bmg} : G \ra S^*$ is clearly injective and the standard ordering on $S^*$ is a total order.

  For part (b), consider an isomorphism $\phi : G \ra H$ such that each $\phi(g_i) = h_i$.  Then if $w_{\bmg}(x) = (x_1, \ldots, x_j)$, $w_{\bmh}(\phi(x)) = (\phi(x_1), \ldots, \phi(x_j))$.  Thus, $x \prec_{\bmg} y$ \ifft $w_{\bmg}(x) \prec w_{\bmg}(y)$ \ifft $w_{\bmh}(\phi(x)) \prec w_{\bmh}(\phi(y))$ \ifft $x \prec_{\bmh} y$.

  For part (c), it suffices to show how to compute $w_{\bmg}(x)$ in polynomial time.  Consider the Cayley graph $\cay(G, S)$ for the group $G$ with generating set $S$.  Then the word $w_{\bmg}(x)$ corresponds to the edges in the minimum length path from $1$ to $x$ in $\cay(G, S)$ that comes first lexicographically.  We can find this path in $O(n \abs{\bmg})$ time by visiting the nodes in breadth-first order starting with $1$.  At the \nth{j} stage, we know $w_{\bmg}(y)$ for all $y \in G$ at a distance of at most $j$ from the root.  We then compute $w_{\bmg}(x)$ for each $x$ at a distance of $j + 1$ from the root by selecting the minimal word $w_{\bmg}(x) \concat g_{x, y}$ over all edges $(x, y)$ associated with an element $g_{x, y}$ of $S$.
\end{proof}

Now we can define the tree that corresponds to $P_1$.  We do this by choosing a balanced binary tree whose leaves are elements of $P_1$.  The choice of this tree is arbitrary so long as it depends only on $\prec_{\bmg}$.  The reason for constructing the trees for $P_1$ and $P_2$ separately is that this allows us to ensure that the tree for $P_1$ has only constant degree.  Otherwise, it would have degree $\Omega(n)$ for groups divisible by large primes which would result in a very slow algorithm.  Later on, we will combine the trees for $P_1$ and $P_2$ to obtain a tree whose leaves correspond to elements of $G$.

\begin{definition}
  \label{defn:TP}
  Let $P_1$ be a group with ordered generating set $\bmg = (g_1, \ldots, g_k)$.  To construct the rooted tree $T(P_1, \bmg)$, we create a leaf node for each element of $P_1$ and color each node by the number that corresponds to its position in the ordering $\prec_{\bmg}$; we then arrange the nodes on a line from smallest to largest according to their colors.  We attach a parent node to each pair of adjacent leaves starting with the smallest pair; if $\abs{P_1}$ is odd, we attach a single parent node to the last leaf.  We then arrange the parent nodes just generated on a line according to the ordering on their children and add new parent nodes for them in the same way.  We continue in this manner until we obtain a single root node from which all the leaves are descended; this yields the tree $T(P_1, \bmg)$.
\end{definition}

Next, we define the tree for the $S_2$ using a definition from~\cite{rosenbaum2013c}.  We start by letting $P_2$ be the root of the tree.  We then partition $P_2$ into the cosets obtained by taking $P_2$ mod the subgroup before $P_2$ in $S_2$.  These are the children of the node $P_2$.  We continue this partitioning process until we obtain cosets in $P_2 / 1$; these correspond to the leaves.  We state the definition for general groups, but in our case the groups will always be solvable.

\intoctornot{
\begin{definition}
  \label{defn:TS}
  Let $P_2$ be a group and consider the composition series $S_2$ given by the subgroups $P_{2, 0} = 1 \tril \cdots \tril P_{2, m} = P_2$.  Then $T(S_2)$ is defined to be the rooted tree whose nodes are $\bigcup_i \left(P_2 / P_{2, i}\right)$.  The root node is $P_2$.  The leaf nodes are $\{x\} \in P_2 / 1$ which we identify with the elements $x \in P_2$.  For each node $x P_{2, i + 1} \in P_2 / P_{2, i + 1}$, there is an edge to each $y P_{2, i}$ such that $y P_{2, i} \subseteq x P_{2, i + 1}$.
\end{definition}}{
\begin{definition}[\cite{rosenbaum2013c}]
  \label{defn:TS}
  Let $P_2$ be a group and consider the composition series $S_2$ given by the subgroups $P_{2, 0} = 1 \tril \cdots \tril P_{2, m} = P_2$.  Then $T(S_2)$ is defined to be the rooted tree whose nodes are $\bigcup_i \left(P_2 / P_{2, i}\right)$.  The root node is $P_2$.  The leaf nodes are $\{x\} \in P_2 / 1$ which we identify with the elements $x \in P_2$.  For each node $x P_{2, i + 1} \in P_2 / P_{2, i + 1}$, there is an edge to each $y P_{2, i}$ such that $y P_{2, i} \subseteq x P_{2, i + 1}$.
\end{definition}
}

In order to obtain a tree whose leaves correspond to elements of $G$, we need to combine the trees for $P_1$ and $S_2$.  For this, we require a variant of the rooted product~\cite{godsil1978a} called a leaf product~\cite{rosenbaum2013c}.  Given two rooted trees, their leaf product is obtained by identifying the root node of a copy of the second tree with each leaf node.

\intoctornot{
\begin{definition}
  \label{defn:leaf-prod}
  Let $T_1$ and $T_2$ be trees rooted at $r_1$ and $r_2$.  Then the leaf product $T_1 \leafprod T_2$ is the tree rooted at $r_1$ with vertex set
  \begin{equation*}
    V(T_1) \cup \setb{(x, y)}{x \in L(T_1) \text{ and } y \in V(T_2) \setminus \{r_2\}}
  \end{equation*}
  The set of edges is
  \begin{align*}
    E(T_1) & \cup \setb{(x, (x, y))}{x \in L(T_1) \text{ and } (r_2, y) \in E(T_2)} \\
    {} & \cup \setb{((x, y), (x, z))}{x \in L(T_1) \text{ and } (y, z) \in E(T_2) \text{ where } y, z \not= r_2}
  \end{align*}
\end{definition}}{
\begin{definition}[\cite{rosenbaum2013c}]
  \label{defn:leaf-prod}
  Let $T_1$ and $T_2$ be trees rooted at $r_1$ and $r_2$.  Then the leaf product $T_1 \leafprod T_2$ is the tree rooted at $r_1$ with vertex set
  \begin{equation*}
    V(T_1) \cup \setb{(x, y)}{x \in L(T_1) \text{ and } y \in V(T_2) \setminus \{r_2\}}
  \end{equation*}
  The set of edges is
  \begin{align*}
    E(T_1) & \cup \setb{(x, (x, y))}{x \in L(T_1) \text{ and } (r_2, y) \in E(T_2)} \\
    {} & \cup \setb{((x, y), (x, z))}{x \in L(T_1) \text{ and } (y, z) \in E(T_2) \text{ where } y, z \not= r_2}
  \end{align*}
\end{definition}
}

For convenience, we identify the tuples $(x, (y, z))$, $((x, y), z)$ with $(x, y, z)$ in the vertex set so that leaf products are associative.  It is also useful to define leaf products of tree isomorphisms and bijections between the leaves of two trees\intoct{ as in~\cite{rosenbaum2013c}}.

\intoctornot{
\begin{definition}
  \label{defn:leaf-prod-iso}
  For each $1 \leq i \leq k$, let $T_i$ and $T_i'$ be trees rooted at $r_i$ and $r_i'$ and let $\phi_i : L(T_i) \ra L(T_i')$ be a bijection that extends to a unique isomorphism which we denote by $\hat \phi : T_i \ra T_i'$.  Then the leaf product $\bigleafprod_{i = 1}^k \phi_i : \bigleafprod_{i = 1}^k T_i \ra \bigleafprod_{i = 1}^k T_i'$ sends each $(x_1, \ldots, x_j)$ to $(\hat \phi_1(x_1), \ldots, \hat \phi_j(x_j))$ where each $x_i \in L(T_i)$ for $i < j$, $x_j \in V(T_j) \setminus \{r_j\}$ and $j \leq k$.
\end{definition}}{
\begin{definition}[\cite{rosenbaum2013c}]
  \label{defn:leaf-prod-iso}
  For each $1 \leq i \leq k$, let $T_i$ and $T_i'$ be trees rooted at $r_i$ and $r_i'$ and let $\phi_i : L(T_i) \ra L(T_i')$ be a bijection that extends to a unique isomorphism which we denote by $\hat \phi : T_i \ra T_i'$.  Then the leaf product $\bigleafprod_{i = 1}^k \phi_i : \bigleafprod_{i = 1}^k T_i \ra \bigleafprod_{i = 1}^k T_i'$ sends each $(x_1, \ldots, x_j)$ to $(\hat \phi_1(x_1), \ldots, \hat \phi_j(x_j))$ where each $x_i \in L(T_i)$ for $i < j$, $x_j \in V(T_j) \setminus \{r_j\}$ and $j \leq k$.
\end{definition}
}

It is easy to see that $\bigleafprod_{i = 1}^k \phi_i$ is a well-defined isomorphism from $\bigleafprod_{i = 1}^k T_i$ to $\bigleafprod_{i = 1}^k T_i'$.  We are now finally in a position to define the tree for a augmented $\alpha$-composition pair.

\begin{definition}
  \label{defn:aug-comp}
  Let $(P_1, S_2, \bmg)$ be an augmented $\alpha$-composition pair for a solvable group $G$.  We define $T(P_1, S_2, \bmg) = T(P_1, \bmg) \leafprod T(S_2)$.
\end{definition}

As in the case of $p$-groups, we cannot attach the aforementioned multiplication gadgets directly to the tree $T(P_1, S_2, \bmg)$ because each leaf be attached to $n$ gadgets and would thus have degree $\Omega(n)$; this would cause our algorithm to be extremely slow.  We resolve this by utilizing the leaf product of $T(P_1, S_2, \bmg)$ with itself so that each multiplication gadget is only attached to a constant number of leaves.

The following notation is convenient as it allows us to easily associate elements of $G$ with nodes in the tree $T(P_1, S_2, \bmg)$.  Let $* : \setb{(x_1, x_2)}{x_i \in P_i} \ra G$ by $*(x_1, x_2) = x_1 x_2$ and note that this is a bijection.  Similarly, we define $\bullet : \setb{(x_1, x_2)}{x_i \in Q_i} \ra H$ by $\bullet(x_1, x_2) = x_1 x_2$.  We can then represent each $x \in G$ by the node $*^{-1}(x)$ in $T(P_1, S_2, \bmg)$ and attach the gadget for each multiplication rule $x y = z$ to the nodes $*^{-1}(x)$, $*^{-1}(y)$ and $*^{-1}(z)$.  We formalize this in the following definition.


\begin{figure}[H]
  \centering
  \includegraphics[scale=1.2]{X-PS.pdf}
  \caption{The graph $X(P_1, S_2, \bmg)$ with the multiplication gadget for $x y = z$ where $z = x y$, $*^{-1}(x) = (x_1, x_2)$, $*^{-1}(y) = (y_1, y_2)$ and $*^{-1}(z) = (z_1, z_2)$}
  \label{fig:X-PS}
\end{figure}


\begin{definition}
  \label{defn:X-PS}
  Let $(P_1, S_2, \bmg)$ be an augmented $\alpha$-composition pair for a solvable group $G$ and define $M$ to be the tree with a root connected to three nodes $\la$, $\ra$ and $=$ with colors ``left'', ``right'' and ``equals'' respectively.  We construct $X(P_1, S_2, \bmg)$ by starting with the tree $T(P_1, S_2, \bmg) \leafprod T(P_1, S_2, \bmg) \leafprod M$ and connecting multiplication gadgets to the leaf nodes.  For each $x, y \in G$, we create the path $((*^{-1}(x), *^{-1}(y), \la), (*^{-1}(y), *^{-1}(x), \ra), (*^{-1}(x y), *^{-1}(y), =))$.  We color each node $(x_1, 1)$ where $x_1 \in P_1$ ``second identity.''  Finally, we color the remaining nodes ``internal.''

\end{definition}

The graph $X(P_1, S_2, \bmg)$ can be thought of a rooted tree with edges added between some nodes at the same levels.  The edges from the original tree are called \emph{tree edges} and the edges between nodes at the same level are called \emph{cross edges}.  We show $X(P_1, S_2, \bmg)$ in \figref{X-PS}.

The correctness of our reduction is based on the fact that two augmented composition pairs $(P_1, S_2, \bmg)$ and $(Q_1, S_2', \bmh)$ are isomorphic \ifft $X(P_1, S_2, \bmg)$ and $X(Q_1, S_2', \bmh)$ are isomorphic.  We prove this in the remainder of this subsection.

Some additional terminology is required for the proof.  We define $\augcomp$ to be the \classorcat\spc of augmented composition pairs for finite solvable groups \inpuborpriv{and}{and isomorphisms between them;} let $\augcomptree$ be the \classorcat\spc of graphs that are isomorphic to the graph $X(P_1, S_2, \bmg)$ for some augmented composition pair $(P_1, S_2, \bmg)$\inpriv{ and isomorphisms between such graphs}.  We overload the symbol $X$ from \defref{X-PS} by defining $X(\phi) : X(P_1, S_2, \bmg) \ra X(Q_1, S_2', \bmh)$ to be $\restr{\phi}{P_1} \leafprod \restr{\phi}{P_2} \leafprod \restr{\phi}{P_1} \leafprod \restr{\phi}{P_2} \leafprod \id_M$ for each $\alpha$-composition pair isomorphism $\phi : (P_1, S_2, \bmg) \ra (Q_1, S_2', \bmh)$\inpriv{ (thus obtaining a functor)}.

In order to prove the correctness of our reduction, we need to show that the augmented $\alpha$-composition pairs $(P_1, S_2, \bmg)$ and $(Q_1, S_2', \bmh)$ are isomorphic \ifft the graphs $X(P_1, S_2, \bmg)$ and $X(Q_1, S_2', \bmh)$ are isomorphic.  The forward direction of the implication \inpuborpriv{is equivalent to}{follows from} the assertion that \inpuborpriv{$X_{(P_1, S_2, \bmg), (Q_1, S_2', \bmh)} : \iso((P_1, S_2, \bmg), (Q_1, S_2', \bmh)) \ra \iso(X(P_1, S_2, \bmg), X(Q_1, S_2', \bmh))$ is well-defined}{$X$ is a functor}.  Proving the converse is more difficult and is one of the main lemmas of this subsection.



\begin{lemma}
  \label{lem:X-functor}
  Let $(P_1, S_2, \bmg)$ and $(Q_1, S_2', \bmh)$ be augmented $\alpha$-composition pairs for the solvable groups $G$ and $H$.  Then \inpuborpriv{the map
    \begin{equation*}
      X_{(P_1, S_2, \bmg), (Q_1, S_2', \bmh)} : \iso((P_1, S_2, \bmg), (Q_1, S_2', \bmh)) \ra \iso(X(P_1, S_2, \bmg), X(Q_1, S_2', \bmh))
    \end{equation*}
is well-defined.}{$X : \augcomp \ra \augcomptree$ is a functor.}
\end{lemma}

Before proceeding with the proof, it is convenient to introduce additional notation.  Let $x, y \in G$.  Consider the sequence of nodes that starts at $*^{-1}(x)$, follows tree edges (away from the root) to a node colored ``left'', follows a cross edge to a node colored ``right'', then follows tree edges (towards the root) to $*^{-1}(y)$, follows tree edges (away from the root) back to the same node colored ``right'' and finally follows a cross edge to a node colored ``equal''; we call this a $W$-\emph{sequence} from $x$ to $y$ to $x y$ since its shape resembles a $W$ (see \figref{X-PS}).  Since $W$-sequences correspond to multiplication gadgets, there is exactly one $W$-sequence from $*^{-1}(x)$ to $*^{-1}(y)$: namely, the one that results from the multiplication gadget
\begin{equation*}
  ((*^{-1}(x), *^{-1}(y), \la), (*^{-1}(y), *^{-1}(x), \ra), (*^{-1}(x y), *^{-1}(y), =)).
\end{equation*}
Therefore, we denote \emph{the} $W$-sequence from $x$ to $y$ to $x y$ by $W(x, y)$.  We now proceed with our proof.

\begin{proof}
  Consider the augmented $\alpha$-composition pairs $(P_1, S_2, \bmg)$ and $(Q_1, S_2', \bmh)$ for the solvable groups $G$ and $H$.  Let $\phi : (P_1, S_2, \bmg) \ra (Q_1, S_2', \bmh)$ be an isomorphism and let $P_{2, 0} = 1 \tril \cdots \tril P_{2, m} = P_2$ and $Q_{2, 0} = 1 \tril \cdots \tril Q_{2, m} = Q_2$ be the subgroup chains for $S_2$ and $S_2'$.  Because $\phi(\bmg) = \bmh$, it follows from~\lemref{gen-ord} that $\restr{\phi}{P_1}$ extends to a unique isomorphism between the rooted colored trees $T(P_1, \bmg)$ and $T(Q_1, \bmh)$.  Moreover, since each $\phi[P_{2, i}] = Q_{2, i}$, we see that $\restr{\phi}{P_2}$ extends to a unique isomorphism from $T(S_2)$ to $T(S_2')$.  Thus, $\restr{\phi}{P_1} \leafprod \restr{\phi}{P_2}$ is an isomorphism from $T(P_1, \bmg) \leafprod T(S_2)$ to $T(Q_1, \bmh) \leafprod T(S_2')$; therefore, $X(\phi) = \restr{\phi}{P_1} \leafprod \restr{\phi}{P_2} \leafprod \restr{\phi}{P_1} \leafprod \restr{\phi}{P_2} \leafprod \id_M$ is a tree isomorphism.

  Let $x, y \in G$ and let $*^{-1}(x) = (x_1, x_2)$.  Then $X(\phi)$ maps $*^{-1}(x)$ to $(\phi(x_1), \phi(x_2)) = \bullet^{-1}(\phi(x))$ as $\phi(x) = \phi(x_1) \phi(x_2)$.  Similarly, recalling that we identified expressions of the forms $((x_1, x_2), (y_1, y_2))$ and $(x_1, x_2, y_1, y_2)$, we see that $X(\phi)$ maps $(*^{-1}(x), *^{-1}(y))$ to $(\bullet^{-1}(\phi(x)), \bullet^{-1}(\phi(y)))$

  Consider the path
  \begin{equation*}
    ((*^{-1}(x), *^{-1}(y), \la), (*^{-1}(y), *^{-1}(x), \ra), (*^{-1}(x y), *^{-1}(y), =))
  \end{equation*}
  in $X(P_1, S_2, \bmg)$.  The image of this path under $X(\phi)$ is
  \begin{equation*}
    ((\bullet^{-1}(\phi(x)), \bullet^{-1}(\phi(y)), \la), (\bullet^{-1}(\phi(y)), \bullet^{-1}(\phi(x)), \ra), (\bullet^{-1}(\phi(x y)), \bullet^{-1}(\phi(y)), =)).
  \end{equation*}
  By \defref{X-PS}, this path is one of the multiplication gadgets in $X(Q_1, S_2', \bmh)$.  Thus, $X(\phi)$ maps each $W$-sequence in $X(P_1, S_2, \bmg)$ to a $W$-sequence in $X(Q_1, S_2', \bmh)$.  Moreover,
 $X(\phi)$ maps each node $(x_1, 1)$ to $(\phi(x_1), 1)$, so it respects the ``second identity'' color.  This implies that $X(P_1, S_2, \bmg) \cong X(Q_1, S_2', \bmh)$ since both graphs have the same number of multiplication gadgets (and hence the same number of $W$-sequences).\inpriv{

    Finally, if $(R_1, S_2'', \bmk)$ is an $\alpha$-composition pair and $\psi$ is an isomorphism from $(Q_1, S_2', \bmh)$ to $(R_1, S_2'', \bmk)$, then $X(\psi \phi) = X(\psi) X(\phi)$ and $X(\id_{(P_1, S_2, \bmg)}) = \id_{X(P_1, S_2, \bmg)}$.  Thus, $X$ is a functor.}
\end{proof}

In order to show if that if the graphs $X(P_1, S_2, \bmg)$ and $X(Q_1, S_2', \bmh)$ are isomorphic then so are the augmented $\alpha$-composition pairs $(P_1, S_2, \bmg)$ and $(Q_1, S_2', \bmh)$, it suffices to show that \inpuborpriv{the map $X_{(P_1, S_2, \bmg), (Q_1, S_2', \bmh)} : \iso((P_1, S_2, \bmg), (Q_1, S_2', \bmh)) \ra \iso(X(P_1, S_2, \bmg), X(Q_1, S_2', \bmh))$ is surjective}{$X$ is a full functor}.  This is the key to our correctness proof and implies that augmented $\alpha$-composition pair isomorphism reduces to testing isomorphism of the resulting graphs. To do this, we need to show that every isomorphism from $X(P_1, S_2, \bmg)$ to $X(Q_1, S_2', \bmh)$ can be written as a leaf product of group isomorphisms.  We accomplish this by restricting the isomorphism between the graphs to certain subsets of nodes and showing that the isomorphism is the leaf product of these restrictions (which turn out to be group isomorphisms).  An isomorphism $\theta : X(P_1, S_2, \bmg) \ra X(Q_1, S_2', \bmh)$ induces the bijection $\phi = \bullet \circ \theta \circ *^{-1} : G \ra H$.  We call this $\phi$ the \emph{induced bijection} for $\theta$.

\begin{lemma}
  \label{lem:iso-decomp}
  Let $X(P_1, S_2, \bmg)$ and $X(Q_1, S_2', \bmh)$ be augmented $\alpha$-composition pairs for the solvable groups $G$ and $H$, let $\theta : X(P_1, S_2, \bmg) \ra X(Q_1, S_2', \bmh)$ be an isomorphism and let $\phi$ be its induced bijection.  Then



  \begin{enumerate}
  \item $\phi : G \ra H$ is a group isomorphism,
  \item $\phi_1 = \restr{\phi}{P_1} : P_1 \ra Q_1$ and $\phi_2 = \restr{\phi}{P_2} : P_2 \ra Q_2$ are group isomorphisms,
  \item $\theta = \phi_1 \leafprod \phi_2 \leafprod \phi_1 \leafprod \phi_2 \leafprod \id_M$ and
  \item $\phi : (P_1, S_2, \bmg) \ra (Q_1, S_2', \bmh)$ is an augmented $\alpha$-composition pair isomorphism.
  \end{enumerate}
\end{lemma}

\begin{proof}
  Let us start with part (a).  It follows from the assumption that $\theta$ is an isomorphism (and hence bijective) that $\phi$ is a bijection.

  Let $x, y \in G$.  Now, $\theta$ maps the nodes $*^{-1}(x)$ and $*^{-1}(y)$ in $X(P_1, S_2, \bmg)$ to $\bullet^{-1}(\phi(x))$ and $\bullet^{-1}(\phi(y))$ by definition of $\phi$.  It follows that $\theta$ maps the $W$-sequence $W(x, y)$ from $x$ to $y$ to $xy$ in $X(P_1, S_2, \bmg)$ to the $W$-sequence $W(\phi(x), \phi(y))$ in $X(Q_1, S_2', \bmh)$.  Now, since $\theta$ maps $*^{-1}(xy)$ to $\bullet^{-1}(\phi(xy))$, it follows that the $W$-sequence $W(\phi(x), \phi(y))$ in $X(Q_1, S_2', \bmh)$ is from $\phi(x)$ to $\phi(y)$ to $\phi(xy)$.  Therefore, by \defref{X-PS}, $\phi(x y) = \phi(x) \phi(y)$ so $\phi$ is a group isomorphism.



  Now we prove (b).  Let $x_1 \in P_1$.  Because $\theta$ respects the ``second identity'' color, it follows that it maps $(x_1, 1)$ to $(x_1', 1)$ for some $x_1' \in Q_1$.  Then $x_1' = \phi(x_1)$ which implies that $\phi[P_1] = Q_1$.

  Now let $x_2 \in P_2$.  Because $\phi$ is an isomorphism, $\phi(1) = 1$; thus, $\theta$ sends the node $(1, 1)$ to $(1, 1)$ which implies that it maps $1$ to $1$.  Thus, for some $x_2' \in Q_2$,
  \begin{align*}
    \theta(1, x_2)      & = (1, x_2') \\
    \theta(*^{-1}(x_2)) & = \bullet^{-1}(x_2') \\
    \phi(x_2)           & = x_2'.
  \end{align*}
  Thus, $\theta(1, x_2) = (1, \phi(x_2))$ so $\phi[P_2] = Q_2$ and $\phi_2$ is a group isomorphism.

  For part (c), let $x, y \in G$ and $*^{-1}(x) = (x_1, x_2)$.  By part (b), $\theta$ sends the node $x_1$ to $\phi_1(x_1)$.  Therefore, for some $x_2' \in Q_2$,
  \begin{align*}
    \theta(x_1, x_2)          & = (\phi(x_1), x_2') \\
    \bullet(\theta(x_1, x_2)) & = \phi(x_1) x_2' \\
    \phi(x)                   & = \phi(x_1) x_2'.
  \end{align*}
  Since $\phi(x) = \phi(x_1) \phi(x_2)$, this implies that $x_2' = \phi(x_2)$ so $\theta$ maps $*^{-1}(x) = (x_1, x_2)$ to $\bullet^{-1}(\phi(x)) = (\phi(x_1), \phi(x_2))$.
  
  Now consider a node $(*^{-1}(x), *^{-1}(y), \ell)$ where $x, y \in G$ and $\ell \in \{\la, \ra, =\}$.  As $(*^{-1}(x), *^{-1}(y))$ is in the subtree rooted at $*^{-1}(x)$, $\theta$ sends it to a node of the form $(\bullet^{-1}(\phi(x)), \bullet^{-1}(b))$ for some $b \in H$.  Similarly, $\theta$ maps the node $(*^{-1}(y), *^{-1}(x))$ to a node of the form $(\bullet^{-1}(\phi(y)), \bullet^{-1}(a))$ for some $a \in H$.  Now, because $(*^{-1}(x), *^{-1}(y))$ and $(*^{-1}(y), *^{-1}(x))$ are in the $W$-sequence from $x$ to $y$ to $x y$, $(\bullet^{-1}(\phi(x)), \bullet^{-1}(b))$ and $(\bullet^{-1}(\phi(y)), \bullet^{-1}(a))$ are in the $W$-sequence from $\phi(x)$ to $\phi(y)$ to $\phi(x y)$.  Then by \defref{X-PS}, $a = \phi(x)$ and $b = \phi(y)$.  Therefore, $\theta$ maps $(*^{-1}(x), *^{-1}(y))$ to $(*^{-1}(\phi(x)), *^{-1}(\phi(y)))$.  Because of the coloring of the leaves in \defref{X-PS}, it follows that $\theta = \phi_1 \leafprod \phi_2 \leafprod \phi_1 \leafprod \phi_2 \leafprod \id_M$.

  Finally, let us prove part (d).  We already know that $\phi$ is a group isomorphism by part (a).  By part (b), we know that each $\phi[P_i] = Q_i$.

  Let $P_{2, 0} = 1 \tril \cdots \tril P_{2, m} = P_2$ and $Q_{2, 0} = 1 \tril \cdots \tril Q_{2, m} = Q_2$ be the subgroup chains for $S_2$ and $S_2'$.  We need to show that each $\phi[P_{2, i}] = Q_{2, i}$.  By part (c), $\theta$ maps $(1, 1)$ in $X(P_1, S_2, \bmg)$ to $(1, 1)$ in $X(Q_1, S_2', \bmh)$.  Now the path from the root of $X(P_1, S_2, \bmg)$ to $(1, 1)$ contains the nodes $(1, P_{2, m}), \ldots, (1, P_{2, 0})$ (in that order).  Moreover, the descendants of the node $(1, P_{2, i})$ that are in $P_1 \times P_2$ are $\setb{(1, x_2)}{x_2 \in P_{2, i}}$.  Similarly, the path from the root of $X(Q_1, S_2', \bmh)$ to $(1, 1)$ contains the nodes $(1, Q_{2, m}), \ldots, (1, Q_{2, 0})$ (in that order) and the descendants of the node $(1, Q_{2, i})$ that are also in $Q_1 \times Q_2$ are $\setb{(1, x_2')}{x_2' \in Q_{2, i}}$.  Therefore, $\theta$ maps each set $\setb{(1, x_2)}{x_2 \in P_{2, i}}$ to $\setb{(1, x_2')}{x_2' \in Q_{2, i}}$.  Then, by definition of $\phi$, $\phi[P_{2, i}] = Q_{2, i}$ and part (d) is proved.
\end{proof}

We now prove that \inpuborpriv{$X_{(P_1, S_2, \bmg), (Q_1, S_2', \bmh)}$ is bijective}{$X$ is a fully faithful functor}.  For isomorphism testing, we only need to show that it is \inpuborpriv{surjective}{full}.  However, we will need it to be \inpuborpriv{injective}{faithful} later when we discuss canonical forms.

\begin{theorem}
\label{thm:X-fff}
\inpuborpriv{Let $(P_1, S_2, \bmg)$ and $(Q_1, S_2', \bmh)$ be augmented $\alpha$-composition pairs for the solvable groups $G$ and $H$.  Then $X_{(P_1, S_2, \bmg), (Q_1, S_2', \bmh)}$ is a bijection}{$X : \augcomp \ra \augcomptree$ is a fully faithful functor and can be evaluated in polynomial time}.\inpub{  Moreover, both $X(P_1, S_2, \bmg)$ and $X(\phi)$ where $\phi \in \iso((P_1, S_2, \bmg), (Q_1, S_2', \bmh))$ can be computed in polynomial time.}
\end{theorem}

\begin{proof}
We know that \inpuborpriv{$X_{(P_1, S_2, \bmg), (Q_1, S_2', \bmh)}$ is well-defined}{$X$ is a functor} by \lemref{X-functor}.  Let $\theta : X(P_1, S_2, \bmg) \ra X(Q_1, S_2', \bmh)$ be an isomorphism.  By \lemref{iso-decomp}, the induced bijection $\phi : (P_1, S_2, \bmg) \ra (Q_1, S_2', \bmh)$ is an isomorphism and $\theta = \phi_1 \leafprod \phi_2 \leafprod \phi_1 \leafprod \phi_2 \leafprod \id_M$ where each $\phi_i = \restr{\phi}{P_i}$.  Then $X(\phi) = \theta$ so $X_{(P_1, S_2, \bmg), (Q_1, S_2', \bmh)}$ is surjective.

Let $\phi, \psi : (P_1, S_2, \bmg) \ra (Q_1, S_2', \bmh)$ be isomorphisms and suppose that $X(\phi) = X(\psi)$.  Then $\phi_1 \leafprod \phi_2 \leafprod \phi_1 \leafprod \phi_2 \leafprod \id_M = \psi_1 \leafprod \psi_2 \leafprod \psi_1 \leafprod \psi_2 \leafprod \id_M$ where each $\phi_i = \restr{\phi}{P_i}$ and each $\psi_i = \restr{\psi}{P_i}$.  Therefore, each $\phi_i = \psi_i$ so $X_{(P_1, S_2, \bmg), (Q_1, S_2', \bmh)}$ is injective.
\end{proof}

Correctness of our reduction now follows.

\begin{corollary}
  \label{cor:aug-alpha-red-cor}
  Let $(P_1, S_2, \bmg)$ and $(Q_1, S_2', \bmh)$ be augmented $\alpha$-composition pairs for the solvable groups $G$ and $H$.  Then $(P_1, S_2, \bmg) \cong (Q_1, S_2', \bmh)$ \ifft $X(P_1, S_2, \bmg) \cong X(Q_1, S_2', \bmh)$.
\end{corollary}

Because $X$ is defined in terms of leaf products of structures that can be computed in polynomial time, it is immediate that $X$ can also be evaluated in polynomial time.

\begin{lemma}
  \label{lem:X-poly}
  Let $(P_1, S_2, \bmg)$ and $(Q_1, S_2', \bmh)$ be augmented $\alpha$-composition pairs for the solvable groups $G$ and $H$ and let $\phi : (P_1, S_2, \bmg) \ra (Q_1, S_2', \bmh)$ be an isomorphism.  Then both $X(P_1, S_2, \bmg)$ and $X(\phi)$ can be computed in polynomial time.
\end{lemma}

The last ingredient that we require for our algorithm for augmented $\alpha$-composition pair isomorphism is a bound on the degree of the graph.

\begin{lemma}
  \label{lem:aug-alpha-graph}
  Let $(P_1, S_2, \bmg)$ be an augmented $\alpha$-composition pair for the solvable group $G$.  Then the graph $X(P_1, S_2, \bmg)$ has degree at most $\max\{\alpha + 1, 4\}$ and size $O(n^2)$.
\end{lemma}

\begin{proof}
  The trees $T(P_1, \bmg)$, $T(S_2)$ and $M$ have degrees $3$, at most $\alpha + 1$ and $3$ respectively.  Since $\abs{P_1} \abs{P_2} = n$, the size of $T(P_1, \bmg) \leafprod T(S_2)$ is $O(n)$.  Thus, $T(P_1, \bmg) \leafprod T(S_2) \leafprod T(P_1, \bmg) \leafprod T(S_2) \leafprod M$ has size $O(n^2)$ and degree at most $\max\{\alpha + 1, 4\}$.
\end{proof}

Finally, we obtain our result for augmented $\alpha$-composition pair isomorphism.

\begin{theorem}
  \label{thm:aug-alpha-iso}
  Let $(P_1, S_2, \bmg)$ and $(Q_1, S_2', \bmh)$ be augmented $\alpha$-composition pairs for the solvable groups $G$ and $H$.  Then we can test if $(P_1, S_2, \bmg) \cong (Q_1, S_2', \bmh)$ in $n^{O(\alpha)}$ time.
\end{theorem}

\begin{proof}
  By \lemref{X-poly}, we can compute the graphs $X(P_1, S_2, \bmg)$ and $X(Q_1, S_2', \bmh)$ in polynomial time.  By \lemref{aug-alpha-graph} and \thmref{const-deg-can}, we can decide if $X(P_1, S_2, \bmg) \cong X(Q_1, S_2', \bmh)$ in $n^{O(\alpha)}$ time.  Finally, \corref{aug-alpha-red-cor} tells us that $(P_1, S_2, \bmg) \cong (Q_1, S_2', \bmh)$ \ifft $X(P_1, S_2, \bmg) \cong X(Q_1, S_2', \bmh)$.\end{proof}

Using \lemref{alpha-comp-red}, we obtain the following corollary.


\begin{corollary}
  \label{cor:alpha-iso}
  Let $(P_1, S_2)$ and $(Q_1, S_2')$ be $\alpha$-composition pairs for the solvable groups $G$ and $H$.  Then we can test if $(P_1, S_2) \cong (Q_1, S_2')$ in $n^{O(\alpha) + \log_{\alpha} n}$ time.
\end{corollary}

\subsection{Canonization}
In this subsection, we extend our results for testing isomorphism of $\alpha$-composition pairs to canonization.  This result can be leveraged to obtain faster algorithms for solvable-group isomorphism via collision arguments~\cite{rosenbaum2013b}.  Our canonization algorithm requires another \inpuborpriv{map}{functor} $Y$ that reverses the action of $X$ by sending back to the augmented $\alpha$-composition pairs from which they arise.  We start with the definition for $Y$.  As with $X$, we overload notation so that $Y$ can also be applied to isomorphisms between graphs.

\begin{definition}
  \label{defn:Y-A}
  For each augmented $\alpha$-composition pair $(P_1, S_2, \bmg)$ for a solvable group $G$ and each graph $A \cong X(P_1, S_2, \bmg)$, we fix an arbitrary isomorphism $\pi : X(P_1, S_2, \bmg) \ra A$.  Let $P_{2, 0} = 1 \tril \cdots \tril P_{2, m} = P_2$ be the subgroup chain for $S_2$.  Then we define $Y(A) = (\pi[P_1 \times \{1\}], \pi[\{1\} \times P_{2, 0}] \tril \cdots \tril \pi[\{1\} \times P_{2, m}], \pi(\bmg))$.

  Here, $\pi[\setb{(x_1, x_2)}{x_i \in P_i}]$ is interpreted as a group containing each $\pi[\{1\} \times P_{2, i}]$ as a subgroup.  For each $x_i, y_i, z_i \in P_i$, we define $\pi(x_1, x_2) \pi(y_1, y_2) = \pi(z_1, z_2)$ \ifft there exists a path $(a_{\pi(x)} a_{\pi(y)}, a_{\pi(z)})$ colored $(\text{``left''}, \text{``right''}, \text{``equals''})$, such that $a_{\pi(x)}$, $a_{\pi(y)}$ and $a_{\pi(z)}$ are descendants of the nodes $\pi(x_1, x_2)$, $\pi(y_1, y_2)$ and $\pi(z_1, z_2)$ in the image of the tree $T(P_1, \bmg) \leafprod T(S_2) \leafprod T(P_1, \bmg) \leafprod T(S_2) \leafprod M$ under $\pi$.

  Let $(P_1, S_2, \bmg)$ and $(Q_1, S_2', \bmh)$ be augmented $\alpha$-composition pairs for the groups $G$ and $H$ and consider the graphs $A \cong X(P_1, S_2, \bmg)$ and  $A' \cong X(Q_1, S_2', \bmh)$.  Let
  $\pi : X(P_1, S_2, \bmg) \ra A$ and $\pi' : X(Q_1, S_2', \bmh) \ra A'$ be the fixed isomorphisms chosen above.  Then for each isomorphism $\theta : A \ra A'$, we define $Y(\theta) : \pi[\setb{(x_1, x_2)}{x_i \in P_i}] \ra \pi'[\setb{(x_1, x_2)}{x_i \in Q_i}]$ to be $\restr{\theta}{\pi[\setb{(x_1, x_2)}{x_i \in P_i}]}$.
\end{definition}

As for $X$, we define $Y_{A, A'} : \iso(A, A') \ra \iso(Y(A), Y(A'))$ by $\theta \mapsto Y(\theta)$ for each pair of graphs $A, A' \in \augcomptree$.

Our first step is to show that $Y$ is well-defined.  Once this is proved, we can leverage \thmref{X-fff} to show that each $Y_{A, A'}$ is bijective.  This allows us to define a canonical form for augmented $\alpha$-composition pairs in terms of $\can_{\graph}$, $X$ and $Y$.

\begin{lemma}
  \label{lem:Y-well-def}
  Let $(P_1, S_2, \bmg)$ be an augmented $\alpha$-composition pair for the solvable group $G$, let $A$ be a graph and let $\pi : X(P_1, S_2, \bmg) \ra A$ be an isomorphism.  Then $Y(A)$ is a well-defined augmented composition pair and can be computed in polynomial time.  Moreover, $Y(\pi) : (P_1, S_2, \bmg) \ra Y(A)$ is an isomorphism.
\end{lemma}

\begin{proof}
  We claim that $\pi[\setb{(x_1, x_2)}{x_i \in P_i}]$ is indeed a group if interpreted according to~\defref{Y-A}.  Let $x_i, y_i, z_i \in P_i$.  Then $\pi(x_1, x_2) \pi(y_1, y_2) = \pi(z_1, z_2)$ \ifft there exists a path $(a_{\pi(x)} a_{\pi(y)}, a_{\pi(z)})$ colored $(\text{``left''}, \text{``right''}, \text{``equals''})$, such that $a_{\pi(x)}$, $a_{\pi(y)}$ and $a_{\pi(z)}$ are descendants of the nodes $\pi(x_1, x_2)$, $\pi(y_1, y_2)$ and $\pi(z_1, z_2)$ in $A$.  Since $\pi$ is an isomorphism, this is equivalent to the existence of a path $(a_{x} a_{y}, a_{z})$ colored $(\text{``left''}, \text{``right''}, \text{``equals''})$, such that $a_x$, $a_y$ and $a_z$ are descendants of the nodes $(x_1, x_2)$, $(y_1, y_2)$ and $(z_1, z_2)$ in $X(P_1, S_2, \bmg)$.

  This is in turn equivalent to the existence of a $W$-sequence from $x$ to $y$ to $z$ where $x = x_1 x_2$, $y = y_1 y_2$ and $z = z_1 z_2$.  By definition, this $W$-sequence exists \ifft $x y = z$.  Therefore, $\pi[\setb{(x_1, x_2)}{x_i \in P_i}]$ is a group and $Y(\pi)$ is a group isomorphism from $G$ to $\pi[\setb{(x_1, x_2)}{x_i \in P_i}]$.  It is immediate that $Y(A)$ is an augmented $\alpha$-composition pair and $Y(\pi)$ is an augmented $\alpha$-composition pair isomorphism.

  Now we show how to compute $Y(A)$ in polynomial time.  Let $\ell = \lceil \log \abs{P_1} \rceil$ and let the subgroup chain for $S_2$ be $P_{2, 0} = 1 \tril \cdots \tril P_{2, m}$.  Then $\ell$ is the height of $T(P_1, \bmg)$ and $m$ is the height of $T(S_2)$.  Thus, by~\defref{X-PS}, $\pi[P_1 \times \{1\}]$ consists of the nodes in $A$ colored ``second identity'' at a depth of $\ell + m$ from the root.

  To compute each $\pi[\{1\} \times P_{2, k}]$, we first find the node $\pi(1, 1)$; this is the identity element of the group $\pi[\setb{(x_1, x_2)}{x_i \in P_i}]$.  The node $\pi(1, P_{2, k})$ is the node on the path from the root to $\pi(1, 1)$ in $A$ that is at a distance of $\ell + k$ from the root.  Then, by~\defref{X-PS}, each $\pi[\{1\} \times P_{2, k}]$ consists of the nodes in $A$ descended from $\pi(1, P_{2, k})$ that are at a distance of $m - k$ from $\pi(1, P_2)$.
\end{proof}

Now we can show that \inpuborpriv{each $Y_{A, A'}$ is surjective}{$Y$ is a full functor}.

\begin{theorem}
  \label{thm:Y-fff}
  \inpuborpriv{Consider the graphs $A, A' \in \augcomptree$.  Then $Y_{A, A'}$ is a bijection and both $Y(A)$ and $Y(\theta)$ where $\theta \in \iso(Y(A), Y(A'))$ can be computed in polynomial time}{$Y : \augcomptree \ra \augcomp$ is a fully faithful functor and can be evaluated in polynomial time}.
\end{theorem}

\begin{proof}
  Let $(P_1, S_2, \bmg)$ and $(Q_1, S_2', \bmh)$ be augmented $\alpha$-composition pairs for the solvable groups $G$ and $H$ such that $\pi : X(P_1, S_2, \bmg) \ra A$, $\pi' : X(Q_1, S_2', \bmh) \ra A'$ and $\theta : A \ra A'$ are isomorphisms.
  
  First, we observe that $Y$ respects composition\inpriv{ and the identity} and let $\psi = \theta \pi : X(P_1, S_2, \bmg) \ra A'$.  Since $\theta$ and $\pi$ are isomorphisms so is $\psi$; \lemref{Y-well-def} then implies that $Y(\psi) = Y(\theta) Y(\pi)$ is also an isomorphism.  Therefore, $Y(\theta) = Y(\psi) (Y(\pi))^{-1}$ is an isomorphism and so \inpuborpriv{$Y_{A, A'}$ is a well-defined function}{$Y$ is a well-defined functor}.

  Now we prove that \inpuborpriv{$Y_{A, A'}$ is a bijection}{$Y$ is fully faithful}.  It follows from Definitions~\ref{defn:X-PS} and~\ref{defn:Y-A} that \inpuborpriv{$Y X = I_{\augcomp}$}{$Y_{X_{(P_1, S_2, \bmg)}, X_{(Q_1, S_2', \bmh)}} X_{(P_1, S_2, \bmg), (Q_1, S_2', \bmh)} = I_{\augcomp}$}.  By \thmref{X-fff}, $X_{(P_1, S_2, \bmg), (Q_1, S_2', \bmh)}$ is bijective; this implies that $Y_{X(P_1, S_2, \bmg), X(Q_1, S_2', \bmh)}$ is also bijective since the identity is bijective.  Now we just need to show that $Y_{A, A'}$ is bijective.  For each isomorphism $\theta : A \ra A'$, there exists an isomorphism $\rho : X(P_1, S_2, \bmg) \ra X(Q_1, S_2', \bmh)$ such that $\theta = \pi' \rho \pi^{-1}$.  It follows that $Y(\theta) = Y(\pi') Y(\rho) Y(\pi^{-1})$ from which we see that $Y_{A, A'}$ is indeed bijective.\inpriv{  Thus, $Y$ is fully faithful.}

  We already showed that $Y(A)$ can be computed in polynomial time in \lemref{Y-well-def} and it follows easily from \defref{Y-A} that $Y(\theta)$ can be computed in polynomial time.
\end{proof}

While \thmref{Y-fff} is enough to obtain our canonization results, we point out that $X$ and $Y$ form a category equivalence \inpuborpriv{when viewed as functors}{(see \appref{X-Y-equiv})}.\inpub{  Moreover, the results of this section can be derived from this more general fact.}

To construct our canonical form for augmented $\alpha$-composition pairs, we convert our augmented $\alpha$-composition pairs to graphs of degree at most $\alpha + O(1)$ by applying $X$.  Then we compute the canonical form of the resulting graph using \thmref{const-deg-can} and convert it back into an augmented $\alpha$-composition pair by applying $Y$.  We use $\can_{\graph}$ to denote the map from graphs to their canonical forms from \thmref{const-deg-can}.

\begin{theorem}
  \label{thm:aug-alpha-can}
  $Y \circ \can_{\graph} \circ X$ is a canonical form for augmented $\alpha$-composition pairs.  Moreover, for any $\alpha$-composition pair $(P_1, S_2, \bmg)$, we can compute $(Y \circ \can_{\graph} \circ X)(P_1, S_2, \bmg)$ in $n^{O(\alpha)}$ time.
\end{theorem}

\begin{proof}
  Consider two $\alpha$-composition pairs $(P_1, S_2, \bmg)$ and $(Q_1, S_2', \bmh)$ for the solvable groups $G$ and $H$.  By \corref{aug-alpha-red-cor}, $(P_1, S_2, \bmg) \cong (Q_1, S_2', \bmh)$ \ifft 
  \begin{equation*}
    X(P_1, S_2, \bmg) \cong X(Q_1, S_2', \bmh).
  \end{equation*}
Thus, $(P_1, S_2, \bmg) \cong (Q_1, S_2', \bmh)$ \ifft
\begin{equation*}
  \can_{\graph}(X(P_1, S_2, \bmg)) = \can_{\graph}(X(Q_1, S_2', \bmh))
\end{equation*}
Now, clearly, if $(P_1, S_2, \bmg) \cong (Q_1, S_2', \bmh)$,
\begin{equation*}
  Y(\can_{\graph}(X(P_1, S_2, \bmg))) = Y(\can_{\graph}(X(Q_1, S_2', \bmh)))
\end{equation*}
On the other hand, if $(P_1, S_2, \bmg) \not\cong (Q_1, S_2', \bmh)$, then
\begin{align*}
  \can_{\graph}(X(P_1, S_2, \bmg))    & \not\cong \can_{\graph}(X(Q_1, S_2', \bmh)) \\
  Y(\can_{\graph}(X(P_1, S_2, \bmg))) & \not\cong Y(\can_{\graph}(X(Q_1, S_2', \bmh))) \\
  Y(\can_{\graph}(X(P_1, S_2, \bmg))) & \not= Y(\can_{\graph}(X(Q_1, S_2', \bmh))).
\end{align*}
 Thus, $Y \circ \can_{\graph} \circ X$ is a complete invariant.  Also, $X(P_1, S_2, \bmg) \cong \can_{\graph}(X(P_1, S_2, \bmg))$ so since $Y X = I_{\augcomp}$, we have $(P_1, S_2, \bmg) \cong Y(\can_{\graph}(X(P_1, S_2, \bmg)))$ by \thmref{Y-fff}.  Thus, $Y \circ \can_{\graph} \circ X$ is a canonical form.

  Lastly, we show that $Y(\can_{\graph}(X(P_1, S_2, \bmg)))$ can be computed in $n^{O(\alpha)}$ time.  By \thmref{X-fff}, we can compute $X(P_1, S_2, \bmg)$ in polynomial time.  By \lemref{aug-alpha-graph} and \thmref{const-deg-can}, it takes $n^{O(\alpha)}$ time to compute $\can_{\graph}(X(P_1, S_2, \bmg))$.  Finally, by \thmref{Y-fff}, we can compute $Y(\can_{\graph}(X(P_1, S_2, \bmg)))$ in polynomial time from $\can_{\graph}(X(P_1, S_2, \bmg))$.
\end{proof}
