In this section, we show our reduction from -composition pair isomorphism to low-degree graph isomorphism.  Our reduction also extends to reducing -composition pair canonization to computing canonical forms of low-degree graphs.  Our proofs follow an outline similar to the analogous reduction from composition series isomorphism to low-degree graph isomorphism in the case of -groups~\cite{rosenbaum2013c}, but are more complex due to the more general structure of solvable groups.

\subsection{Isomorphism testing}
At a high level, our algorithm consists of the following steps.  First, we augment our -composition pair  by choosing an ordered generating set  for the subgroup  (which corresponds to the large primes) to obtain the \emph{augmented -composition pair} .  We say that a mapping  is an isomorphism between the augmented -decompositions  and  for  and  if  is an -composition pair isomorphism for  and  and .  The reason for choosing an augmented -composition pair is so that we can reduce the degree of the part of the graph we construct that corresponds to  using the trick due to Wagner~\cite{wagner2011a} mentioned in the introduction.

Since one can fix an ordered generating set  for  and consider all possible ordered generating sets for , it is easy to see that -composition pair isomorphism is  Turing-reducible to augmented -composition pair isomorphism.  (Recall that we will later set  so this is  time and is less than the complexity we are aiming for.)  We state this in the following lemma.

\begin{lemma}
  \label{lem:alpha-comp-red}
  Testing isomorphism of the -composition pairs  and  for the solvable groups  and  is  deterministic time Turing reducible to testing isomorphism of augmented -composition pairs for  and  where  is the smallest prime dividing the order of the group.
\end{lemma}

We then construct a tree whose leaves represent the elements of ; by using the ordered generating set  chosen above, we are able to ensure that the degree of this tree is at most .  By augmenting this tree with gadgets that represent the multiplication table of the group, we obtain an object that represents the isomorphism class of the augmented -composition pair .  The final step of the algorithm is to apply the following result\intoct{ due to Babai and Luks~\cite{babai1983a,babai1983b}} for computing canonical forms of low-degree graphs.

\intoctornot{
\begin{theorem}
  \label{thm:const-deg-can}
  Canonization of colored graphs of degree at most  is in  time.
\end{theorem}}{
\begin{theorem}[Babai and Luks~\cite{babai1983a,babai1983b}]
  \label{thm:const-deg-can}
  Canonization of colored graphs of degree at most  is in  time.
\end{theorem}}

The main challenge compared to -group isomorphism~\cite{rosenbaum2013c} is dealing with the fact that some of the prime divisors of a solvable group can be small while others may be large.  This is the main reason why the correctness proof is significantly more complex than for -groups.  Since a -group has exactly one prime divisor, it was possible to handle the cases of small and large primes separately using a graph-isomorphism based -group algorithm~\cite{rosenbaum2013c} (which is fast when the prime is small) and the generator-enumeration algorithm (which is fast when the prime is large).  On the other hand, for solvable groups, it is necessary to design a hybrid algorithm that is fast for both cases simultaneously.

As mentioned above, the first step in the graph construction is to define a tree for an augmented -composition pair .  We do this by constructing trees  and  whose leaves correspond to the elements of  and .  In order to define the part of the tree corresponding to , we need a way to canonically order the elements of a group given an ordered generating set.  For completeness, we state and prove the required properties from~\cite{rosenbaum2013c}.

\intoctornot{
\begin{definition}
  \label{defn:gen-ord}
  Let  be a group with an ordered generating set .  Define a total order  on  by  if  where each  is the first word in  under the standard ordering such that .
\end{definition}}{
\begin{definition}[\cite{rosenbaum2013c}]
  \label{defn:gen-ord}
  Let  be a group with an ordered generating set .  Define a total order  on  by  if  where each  is the first word in  under the standard ordering such that .
\end{definition}
}

\intoctornot{
We will also need the following lemma from~\cite{rosenbaum2013c}.

\begin{lemma}
  \label{lem:gen-ord}
  Let  and  be groups with ordered generating sets  and , and let .  Then

  \begin{enumerate}
  \item  is a total ordering on .
  \item if  is an isomorphism such that each , then  \ifft .
  \item we can decide if  in  time.
  \end{enumerate}
\end{lemma}}{
\begin{lemma}[\cite{rosenbaum2013c}]
  \label{lem:gen-ord}
  Let  and  be groups with ordered generating sets  and , and let .  Then

  \begin{enumerate}
  \item  is a total ordering on .
  \item if  is an isomorphism such that each , then  \ifft .
  \item we can decide if  in  time.
  \end{enumerate}
\end{lemma}
}

\begin{proof}
  Let .  For part (a), it is clear that  is a total order since  is clearly injective and the standard ordering on  is a total order.

  For part (b), consider an isomorphism  such that each .  Then if , .  Thus,  \ifft  \ifft  \ifft .

  For part (c), it suffices to show how to compute  in polynomial time.  Consider the Cayley graph  for the group  with generating set .  Then the word  corresponds to the edges in the minimum length path from  to  in  that comes first lexicographically.  We can find this path in  time by visiting the nodes in breadth-first order starting with .  At the \nth{j} stage, we know  for all  at a distance of at most  from the root.  We then compute  for each  at a distance of  from the root by selecting the minimal word  over all edges  associated with an element  of .
\end{proof}

Now we can define the tree that corresponds to .  We do this by choosing a balanced binary tree whose leaves are elements of .  The choice of this tree is arbitrary so long as it depends only on .  The reason for constructing the trees for  and  separately is that this allows us to ensure that the tree for  has only constant degree.  Otherwise, it would have degree  for groups divisible by large primes which would result in a very slow algorithm.  Later on, we will combine the trees for  and  to obtain a tree whose leaves correspond to elements of .

\begin{definition}
  \label{defn:TP}
  Let  be a group with ordered generating set .  To construct the rooted tree , we create a leaf node for each element of  and color each node by the number that corresponds to its position in the ordering ; we then arrange the nodes on a line from smallest to largest according to their colors.  We attach a parent node to each pair of adjacent leaves starting with the smallest pair; if  is odd, we attach a single parent node to the last leaf.  We then arrange the parent nodes just generated on a line according to the ordering on their children and add new parent nodes for them in the same way.  We continue in this manner until we obtain a single root node from which all the leaves are descended; this yields the tree .
\end{definition}

Next, we define the tree for the  using a definition from~\cite{rosenbaum2013c}.  We start by letting  be the root of the tree.  We then partition  into the cosets obtained by taking  mod the subgroup before  in .  These are the children of the node .  We continue this partitioning process until we obtain cosets in ; these correspond to the leaves.  We state the definition for general groups, but in our case the groups will always be solvable.

\intoctornot{
\begin{definition}
  \label{defn:TS}
  Let  be a group and consider the composition series  given by the subgroups .  Then  is defined to be the rooted tree whose nodes are .  The root node is .  The leaf nodes are  which we identify with the elements .  For each node , there is an edge to each  such that .
\end{definition}}{
\begin{definition}[\cite{rosenbaum2013c}]
  \label{defn:TS}
  Let  be a group and consider the composition series  given by the subgroups .  Then  is defined to be the rooted tree whose nodes are .  The root node is .  The leaf nodes are  which we identify with the elements .  For each node , there is an edge to each  such that .
\end{definition}
}

In order to obtain a tree whose leaves correspond to elements of , we need to combine the trees for  and .  For this, we require a variant of the rooted product~\cite{godsil1978a} called a leaf product~\cite{rosenbaum2013c}.  Given two rooted trees, their leaf product is obtained by identifying the root node of a copy of the second tree with each leaf node.

\intoctornot{
\begin{definition}
  \label{defn:leaf-prod}
  Let  and  be trees rooted at  and .  Then the leaf product  is the tree rooted at  with vertex set
  
  The set of edges is
  
\end{definition}}{
\begin{definition}[\cite{rosenbaum2013c}]
  \label{defn:leaf-prod}
  Let  and  be trees rooted at  and .  Then the leaf product  is the tree rooted at  with vertex set
  
  The set of edges is
  
\end{definition}
}

For convenience, we identify the tuples ,  with  in the vertex set so that leaf products are associative.  It is also useful to define leaf products of tree isomorphisms and bijections between the leaves of two trees\intoct{ as in~\cite{rosenbaum2013c}}.

\intoctornot{
\begin{definition}
  \label{defn:leaf-prod-iso}
  For each , let  and  be trees rooted at  and  and let  be a bijection that extends to a unique isomorphism which we denote by .  Then the leaf product  sends each  to  where each  for ,  and .
\end{definition}}{
\begin{definition}[\cite{rosenbaum2013c}]
  \label{defn:leaf-prod-iso}
  For each , let  and  be trees rooted at  and  and let  be a bijection that extends to a unique isomorphism which we denote by .  Then the leaf product  sends each  to  where each  for ,  and .
\end{definition}
}

It is easy to see that  is a well-defined isomorphism from  to .  We are now finally in a position to define the tree for a augmented -composition pair.

\begin{definition}
  \label{defn:aug-comp}
  Let  be an augmented -composition pair for a solvable group .  We define .
\end{definition}

As in the case of -groups, we cannot attach the aforementioned multiplication gadgets directly to the tree  because each leaf be attached to  gadgets and would thus have degree ; this would cause our algorithm to be extremely slow.  We resolve this by utilizing the leaf product of  with itself so that each multiplication gadget is only attached to a constant number of leaves.

The following notation is convenient as it allows us to easily associate elements of  with nodes in the tree .  Let  by  and note that this is a bijection.  Similarly, we define  by .  We can then represent each  by the node  in  and attach the gadget for each multiplication rule  to the nodes ,  and .  We formalize this in the following definition.


\begin{figure}[H]
  \centering
  \includegraphics[scale=1.2]{X-PS.pdf}
  \caption{The graph  with the multiplication gadget for  where , ,  and }
  \label{fig:X-PS}
\end{figure}


\begin{definition}
  \label{defn:X-PS}
  Let  be an augmented -composition pair for a solvable group  and define  to be the tree with a root connected to three nodes ,  and  with colors ``left'', ``right'' and ``equals'' respectively.  We construct  by starting with the tree  and connecting multiplication gadgets to the leaf nodes.  For each , we create the path .  We color each node  where  ``second identity.''  Finally, we color the remaining nodes ``internal.''

\end{definition}

The graph  can be thought of a rooted tree with edges added between some nodes at the same levels.  The edges from the original tree are called \emph{tree edges} and the edges between nodes at the same level are called \emph{cross edges}.  We show  in \figref{X-PS}.

The correctness of our reduction is based on the fact that two augmented composition pairs  and  are isomorphic \ifft  and  are isomorphic.  We prove this in the remainder of this subsection.

Some additional terminology is required for the proof.  We define  to be the \classorcat\spc of augmented composition pairs for finite solvable groups \inpuborpriv{and}{and isomorphisms between them;} let  be the \classorcat\spc of graphs that are isomorphic to the graph  for some augmented composition pair \inpriv{ and isomorphisms between such graphs}.  We overload the symbol  from \defref{X-PS} by defining  to be  for each -composition pair isomorphism \inpriv{ (thus obtaining a functor)}.

In order to prove the correctness of our reduction, we need to show that the augmented -composition pairs  and  are isomorphic \ifft the graphs  and  are isomorphic.  The forward direction of the implication \inpuborpriv{is equivalent to}{follows from} the assertion that \inpuborpriv{ is well-defined}{ is a functor}.  Proving the converse is more difficult and is one of the main lemmas of this subsection.



\begin{lemma}
  \label{lem:X-functor}
  Let  and  be augmented -composition pairs for the solvable groups  and .  Then \inpuborpriv{the map
    
is well-defined.}{ is a functor.}
\end{lemma}

Before proceeding with the proof, it is convenient to introduce additional notation.  Let .  Consider the sequence of nodes that starts at , follows tree edges (away from the root) to a node colored ``left'', follows a cross edge to a node colored ``right'', then follows tree edges (towards the root) to , follows tree edges (away from the root) back to the same node colored ``right'' and finally follows a cross edge to a node colored ``equal''; we call this a -\emph{sequence} from  to  to  since its shape resembles a  (see \figref{X-PS}).  Since -sequences correspond to multiplication gadgets, there is exactly one -sequence from  to : namely, the one that results from the multiplication gadget

Therefore, we denote \emph{the} -sequence from  to  to  by .  We now proceed with our proof.

\begin{proof}
  Consider the augmented -composition pairs  and  for the solvable groups  and .  Let  be an isomorphism and let  and  be the subgroup chains for  and .  Because , it follows from~\lemref{gen-ord} that  extends to a unique isomorphism between the rooted colored trees  and .  Moreover, since each , we see that  extends to a unique isomorphism from  to .  Thus,  is an isomorphism from  to ; therefore,  is a tree isomorphism.

  Let  and let .  Then  maps  to  as .  Similarly, recalling that we identified expressions of the forms  and , we see that  maps  to 

  Consider the path
  
  in .  The image of this path under  is
  
  By \defref{X-PS}, this path is one of the multiplication gadgets in .  Thus,  maps each -sequence in  to a -sequence in .  Moreover,
  maps each node  to , so it respects the ``second identity'' color.  This implies that  since both graphs have the same number of multiplication gadgets (and hence the same number of -sequences).\inpriv{

    Finally, if  is an -composition pair and  is an isomorphism from  to , then  and .  Thus,  is a functor.}
\end{proof}

In order to show if that if the graphs  and  are isomorphic then so are the augmented -composition pairs  and , it suffices to show that \inpuborpriv{the map  is surjective}{ is a full functor}.  This is the key to our correctness proof and implies that augmented -composition pair isomorphism reduces to testing isomorphism of the resulting graphs. To do this, we need to show that every isomorphism from  to  can be written as a leaf product of group isomorphisms.  We accomplish this by restricting the isomorphism between the graphs to certain subsets of nodes and showing that the isomorphism is the leaf product of these restrictions (which turn out to be group isomorphisms).  An isomorphism  induces the bijection .  We call this  the \emph{induced bijection} for .

\begin{lemma}
  \label{lem:iso-decomp}
  Let  and  be augmented -composition pairs for the solvable groups  and , let  be an isomorphism and let  be its induced bijection.  Then



  \begin{enumerate}
  \item  is a group isomorphism,
  \item  and  are group isomorphisms,
  \item  and
  \item  is an augmented -composition pair isomorphism.
  \end{enumerate}
\end{lemma}

\begin{proof}
  Let us start with part (a).  It follows from the assumption that  is an isomorphism (and hence bijective) that  is a bijection.

  Let .  Now,  maps the nodes  and  in  to  and  by definition of .  It follows that  maps the -sequence  from  to  to  in  to the -sequence  in .  Now, since  maps  to , it follows that the -sequence  in  is from  to  to .  Therefore, by \defref{X-PS},  so  is a group isomorphism.



  Now we prove (b).  Let .  Because  respects the ``second identity'' color, it follows that it maps  to  for some .  Then  which implies that .

  Now let .  Because  is an isomorphism, ; thus,  sends the node  to  which implies that it maps  to .  Thus, for some ,
  
  Thus,  so  and  is a group isomorphism.

  For part (c), let  and .  By part (b),  sends the node  to .  Therefore, for some ,
  
  Since , this implies that  so  maps  to .
  
  Now consider a node  where  and .  As  is in the subtree rooted at ,  sends it to a node of the form  for some .  Similarly,  maps the node  to a node of the form  for some .  Now, because  and  are in the -sequence from  to  to ,  and  are in the -sequence from  to  to .  Then by \defref{X-PS},  and .  Therefore,  maps  to .  Because of the coloring of the leaves in \defref{X-PS}, it follows that .

  Finally, let us prove part (d).  We already know that  is a group isomorphism by part (a).  By part (b), we know that each .

  Let  and  be the subgroup chains for  and .  We need to show that each .  By part (c),  maps  in  to  in .  Now the path from the root of  to  contains the nodes  (in that order).  Moreover, the descendants of the node  that are in  are .  Similarly, the path from the root of  to  contains the nodes  (in that order) and the descendants of the node  that are also in  are .  Therefore,  maps each set  to .  Then, by definition of ,  and part (d) is proved.
\end{proof}

We now prove that \inpuborpriv{ is bijective}{ is a fully faithful functor}.  For isomorphism testing, we only need to show that it is \inpuborpriv{surjective}{full}.  However, we will need it to be \inpuborpriv{injective}{faithful} later when we discuss canonical forms.

\begin{theorem}
\label{thm:X-fff}
\inpuborpriv{Let  and  be augmented -composition pairs for the solvable groups  and .  Then  is a bijection}{ is a fully faithful functor and can be evaluated in polynomial time}.\inpub{  Moreover, both  and  where  can be computed in polynomial time.}
\end{theorem}

\begin{proof}
We know that \inpuborpriv{ is well-defined}{ is a functor} by \lemref{X-functor}.  Let  be an isomorphism.  By \lemref{iso-decomp}, the induced bijection  is an isomorphism and  where each .  Then  so  is surjective.

Let  be isomorphisms and suppose that .  Then  where each  and each .  Therefore, each  so  is injective.
\end{proof}

Correctness of our reduction now follows.

\begin{corollary}
  \label{cor:aug-alpha-red-cor}
  Let  and  be augmented -composition pairs for the solvable groups  and .  Then  \ifft .
\end{corollary}

Because  is defined in terms of leaf products of structures that can be computed in polynomial time, it is immediate that  can also be evaluated in polynomial time.

\begin{lemma}
  \label{lem:X-poly}
  Let  and  be augmented -composition pairs for the solvable groups  and  and let  be an isomorphism.  Then both  and  can be computed in polynomial time.
\end{lemma}

The last ingredient that we require for our algorithm for augmented -composition pair isomorphism is a bound on the degree of the graph.

\begin{lemma}
  \label{lem:aug-alpha-graph}
  Let  be an augmented -composition pair for the solvable group .  Then the graph  has degree at most  and size .
\end{lemma}

\begin{proof}
  The trees ,  and  have degrees , at most  and  respectively.  Since , the size of  is .  Thus,  has size  and degree at most .
\end{proof}

Finally, we obtain our result for augmented -composition pair isomorphism.

\begin{theorem}
  \label{thm:aug-alpha-iso}
  Let  and  be augmented -composition pairs for the solvable groups  and .  Then we can test if  in  time.
\end{theorem}

\begin{proof}
  By \lemref{X-poly}, we can compute the graphs  and  in polynomial time.  By \lemref{aug-alpha-graph} and \thmref{const-deg-can}, we can decide if  in  time.  Finally, \corref{aug-alpha-red-cor} tells us that  \ifft .\end{proof}

Using \lemref{alpha-comp-red}, we obtain the following corollary.


\begin{corollary}
  \label{cor:alpha-iso}
  Let  and  be -composition pairs for the solvable groups  and .  Then we can test if  in  time.
\end{corollary}

\subsection{Canonization}
In this subsection, we extend our results for testing isomorphism of -composition pairs to canonization.  This result can be leveraged to obtain faster algorithms for solvable-group isomorphism via collision arguments~\cite{rosenbaum2013b}.  Our canonization algorithm requires another \inpuborpriv{map}{functor}  that reverses the action of  by sending back to the augmented -composition pairs from which they arise.  We start with the definition for .  As with , we overload notation so that  can also be applied to isomorphisms between graphs.

\begin{definition}
  \label{defn:Y-A}
  For each augmented -composition pair  for a solvable group  and each graph , we fix an arbitrary isomorphism .  Let  be the subgroup chain for .  Then we define .

  Here,  is interpreted as a group containing each  as a subgroup.  For each , we define  \ifft there exists a path  colored , such that ,  and  are descendants of the nodes ,  and  in the image of the tree  under .

  Let  and  be augmented -composition pairs for the groups  and  and consider the graphs  and  .  Let
   and  be the fixed isomorphisms chosen above.  Then for each isomorphism , we define  to be .
\end{definition}

As for , we define  by  for each pair of graphs .

Our first step is to show that  is well-defined.  Once this is proved, we can leverage \thmref{X-fff} to show that each  is bijective.  This allows us to define a canonical form for augmented -composition pairs in terms of ,  and .

\begin{lemma}
  \label{lem:Y-well-def}
  Let  be an augmented -composition pair for the solvable group , let  be a graph and let  be an isomorphism.  Then  is a well-defined augmented composition pair and can be computed in polynomial time.  Moreover,  is an isomorphism.
\end{lemma}

\begin{proof}
  We claim that  is indeed a group if interpreted according to~\defref{Y-A}.  Let .  Then  \ifft there exists a path  colored , such that ,  and  are descendants of the nodes ,  and  in .  Since  is an isomorphism, this is equivalent to the existence of a path  colored , such that ,  and  are descendants of the nodes ,  and  in .

  This is in turn equivalent to the existence of a -sequence from  to  to  where ,  and .  By definition, this -sequence exists \ifft .  Therefore,  is a group and  is a group isomorphism from  to .  It is immediate that  is an augmented -composition pair and  is an augmented -composition pair isomorphism.

  Now we show how to compute  in polynomial time.  Let  and let the subgroup chain for  be .  Then  is the height of  and  is the height of .  Thus, by~\defref{X-PS},  consists of the nodes in  colored ``second identity'' at a depth of  from the root.

  To compute each , we first find the node ; this is the identity element of the group .  The node  is the node on the path from the root to  in  that is at a distance of  from the root.  Then, by~\defref{X-PS}, each  consists of the nodes in  descended from  that are at a distance of  from .
\end{proof}

Now we can show that \inpuborpriv{each  is surjective}{ is a full functor}.

\begin{theorem}
  \label{thm:Y-fff}
  \inpuborpriv{Consider the graphs .  Then  is a bijection and both  and  where  can be computed in polynomial time}{ is a fully faithful functor and can be evaluated in polynomial time}.
\end{theorem}

\begin{proof}
  Let  and  be augmented -composition pairs for the solvable groups  and  such that ,  and  are isomorphisms.
  
  First, we observe that  respects composition\inpriv{ and the identity} and let .  Since  and  are isomorphisms so is ; \lemref{Y-well-def} then implies that  is also an isomorphism.  Therefore,  is an isomorphism and so \inpuborpriv{ is a well-defined function}{ is a well-defined functor}.

  Now we prove that \inpuborpriv{ is a bijection}{ is fully faithful}.  It follows from Definitions~\ref{defn:X-PS} and~\ref{defn:Y-A} that \inpuborpriv{}{}.  By \thmref{X-fff},  is bijective; this implies that  is also bijective since the identity is bijective.  Now we just need to show that  is bijective.  For each isomorphism , there exists an isomorphism  such that .  It follows that  from which we see that  is indeed bijective.\inpriv{  Thus,  is fully faithful.}

  We already showed that  can be computed in polynomial time in \lemref{Y-well-def} and it follows easily from \defref{Y-A} that  can be computed in polynomial time.
\end{proof}

While \thmref{Y-fff} is enough to obtain our canonization results, we point out that  and  form a category equivalence \inpuborpriv{when viewed as functors}{(see \appref{X-Y-equiv})}.\inpub{  Moreover, the results of this section can be derived from this more general fact.}

To construct our canonical form for augmented -composition pairs, we convert our augmented -composition pairs to graphs of degree at most  by applying .  Then we compute the canonical form of the resulting graph using \thmref{const-deg-can} and convert it back into an augmented -composition pair by applying .  We use  to denote the map from graphs to their canonical forms from \thmref{const-deg-can}.

\begin{theorem}
  \label{thm:aug-alpha-can}
   is a canonical form for augmented -composition pairs.  Moreover, for any -composition pair , we can compute  in  time.
\end{theorem}

\begin{proof}
  Consider two -composition pairs  and  for the solvable groups  and .  By \corref{aug-alpha-red-cor},  \ifft 
  
Thus,  \ifft

Now, clearly, if ,

On the other hand, if , then

 Thus,  is a complete invariant.  Also,  so since , we have  by \thmref{Y-fff}.  Thus,  is a canonical form.

  Lastly, we show that  can be computed in  time.  By \thmref{X-fff}, we can compute  in polynomial time.  By \lemref{aug-alpha-graph} and \thmref{const-deg-can}, it takes  time to compute .  Finally, by \thmref{Y-fff}, we can compute  in polynomial time from .
\end{proof}
