









\documentclass[journal]{IEEEtran}























\ifCLASSINFOpdf
\else
\fi














\usepackage{amsmath}\usepackage{amssymb}
\usepackage{cite}
\usepackage{color}   
\usepackage{graphicx}   \usepackage{epsfig} 
\usepackage{mathtools}
\usepackage{marvosym}
\usepackage{soul}
\usepackage{xcolor}
\usepackage{enumitem}

\usepackage{amsthm}
\usepackage{ams fonts} \usepackage{xcolor}
\usepackage{epstopdf}
\usepackage[normalem]{ulem}
\usepackage{cancel}
\usepackage{url}
\usepackage{algorithm}
\usepackage{algpseudocode}
\usepackage{pifont}

\newcommand{\norm}[1]{\left\lVert#1\right\rVert} 
\newcommand{\abs}[1]{\left\lvert#1\right\rvert}
\newtheorem{ass}{\bf{Assumption}} 
\newtheorem{myeg}{\bf{Example}}
\newtheorem{lem}{\bf{Lemma}}
\newtheorem{Def}{\bf{Definition}}
\newtheorem{prop}{\bf{Proposition}}
\newtheorem{thm}{\bf{Theorem}}
\newtheorem{remark}{Remark}

\newcommand{\qie}[1]{{\normalsize{{{\color{cyan}(#1)}}}}}
\newcommand{\yh}[1]{{\normalsize{{{\color{magenta}#1}}}}}


































\newcommand{\revision}[1]{{\normalsize{{{\color{blue}#1}}}}}


\begin{document}
\title{Secure Estimation based Kalman Filter for Cyber-Physical Systems against Adversarial Attacks}





\author{Young~Hwan~Chang, Qie~Hu, ~Claire~J.~Tomlin
\thanks{Y. H. Chang is with the Department of Biomedical Engineering, Oregon Health and Science University, Portland, OR 97201 USA (e-mail:).}
\thanks{Q. Hu, C.J Tomlin are with the Department of Electrical Engineering and Computer Sciences, University of California, Berkeley, CA 94720 USA (e-mail:).}
\thanks{*These authors contributed equally}}










\maketitle

\begin{abstract}
Cyber-physical systems are found in many applications such as power networks, manufacturing processes, and air and ground transportation systems. Maintaining security of these systems under cyber attacks is an important and challenging task, since these attacks can be erratic and thus difficult to model. Secure estimation problems study how to estimate the true system states when  measurements are corrupted and/or control inputs are compromised by attackers. The authors in \cite{Fawzi2014} proposed a secure estimation method when the set of attacked nodes (sensors, controllers) is fixed. In this paper, we extend these results to scenarios in which the set of attacked nodes can change over time. We formulate this secure estimation problem into the classical error correction problem \cite{Candes_Tao} and we show that accurate decoding can be guaranteed under a certain condition.
Furthermore, we propose a combined secure estimation method with our proposed secure estimator and the Kalman Filter for improved practical performance. Finally,  we demonstrate the performance of our method through simulations of two scenarios where an unmanned aerial vehicle is under adversarial attack.

\end{abstract}



\begin{IEEEkeywords}
Cyber-physical systems, Error correction, Secure estimation
\end{IEEEkeywords}


\IEEEpeerreviewmaketitle




\section{Introduction}

Cyber-physical systems (CPS) consist of physical components such as actuators, sensors and controllers that communicate with each other over a network. 
For example, unmanned aerial vehicles (UAV) may obtain position measurements from a Global Positioning System (GPS) or communicate with remote control centers.
Although communication networks are often protected by security measures, cyber attacks can still take place when a malicious attacker obtains unauthorized access, launching jamming attacks \cite{Gligor} or spoofing sensor readings and sending erroneous control signals to actuators \cite{Mo}. For CPS, cyber attacks not only compromise information but can also cause damage in the physical process. This presents new challenges and thus demands new strategies and algorithms \cite{Sastry}. 

There has been extensive work on the security of CPS. 
Each of them relies on specific assumptions about attackers' strategies and it is rarely the case, if not impossible, that one estimator/detector can protect against all possible attacks. 
\cite{Tong, KwonACC, Reiter, Sastry2} studied optimal attack strategies for different control systems and applications. 
From the controller's point of view, researchers have studied how to detect attacks \cite{Blanke, Willsky} and how to accurately estimate the states and control the system when it is under attack. One approach for the latter, which is adopted in robust control and filtering methods, is to model the attack signal as process or measurement noise, and assume that they are bounded \cite{Zhou_Doyle} or follow a certain probabilistic distribution \cite{Bullo, Liu}.
An alternative approach uses game theory, where the controller and attacker are players with competing goals in a game \cite{Wu, Basar, Basar2, Walrand, Pappas}. Finally, the authors of \cite{KwonCDC} proposed a hybrid controller, where each constituent controller protects against a specific type of attack.


Recently, \cite{Fawzi2014} studied secure estimation of a linear time invariant system where attack signals can be arbitrary and unbounded, thus protecting the system against more general cyber-attacks. Later, \cite{Pajic2014} and \cite{shoukry2016smt} extended this work by relaxing the assumption of having an exact system model and proposing an SMT-based observer that handles large systems with thousands of sensors, respectively. 
One limiting assumption of \cite{Fawzi2014,Pajic2014,shoukry2016smt} is that the set of attacked nodes/sensors is fixed and can not change over time. 
If a malicious attacker is aware of this, then he or she can exploit this weakness and attack different sensors at different time steps so that such a decoder would fail.

In this paper, we are interested in the case in which the set of attacked nodes can change over time. 
By doing so, our proposed decoder can protect the system against more general attack scenarios than that presented in \cite{Fawzi2014}. 
We believe this is a significant contribution, because it is difficult, if not impossible, to anticipate cyber attackers' strategies and behavior, thus a decoder that is able to handle more general attacks is an improvement. 
Furthermore, security studies on the current traffic infrastructure \cite{ghena2014traffic} demonstrated that once a cyber attacker gains access to the traffic network at a single point, the attacker can send commands to any traffic intersection in the network. In other words, the attacker can freely attack a different set of traffic signals (sensors) at any time. Indeed, an attacker who desires to travel through a set of roads as fast as possible would attack different traffic lights to always give him/herself green lights as he/she moves through the road network. 




\subsection{Contributions}

There are four main contributions in this paper:
\begin{enumerate}[listparindent=1.5em]
\item
We propose a secure decoder for a linear time invariant system under sensor attack, where the attacked sensors can change with time, and attack signals can be unbounded and arbitrary.
The proposed decoder is based on  optimization and is computationally efficient.
\item
We prove the maximum number of sensor attacks that can be corrected with our decoder, which turns out to be the same as that of the decoder proposed in \cite{Fawzi2014}. This is a very nice result as it shows that, compared to \cite{Fawzi2014}, our decoder can protect the system against more general attacks, and at the same time, it does not compromise the number of attacks that can be corrected.
\item
We propose a practical method for decoder design that guarantees accurate decoding. 
First, we formulate the secure estimation problem into the classical error correction (EC) problem \cite{Candes_Tao}. In EC, accurate decoding can be guaranteed if the coding matrix satisfies the Restricted Isometric Properties (RIP), which unfortunately are very difficult to check, in addition, we cannot choose a random coding matrix \textit{a priori} in our problem setting. 
Instead of using RIP, in Theorem 1, we provide a sufficient condition for perfect recovery of the system states against sensor attacks.



\item
Finally, we propose to combine our decoder with a Kalman Filter (KF) to improve its practical performance. 
The KF filters out both occasional estimation errors by the secure decoder and noisy measurements.
We demonstrate the effectiveness of our combined estimator using two examples of UAVs under adversarial attack.
\end{enumerate}




\subsection{Organization of  the Paper}
This paper is organized as follows. Section \ref{sec:overview} gives an overview of secure estimation for CPS when attacked nodes are fixed, as well as compressive sensing and error correction. 
Section \ref{sec:main} formulates the problem of secure estimation when attacked nodes can change over time and compares it with the case when attacked nodes are fixed. 
In Section \ref{sec:design}, we describe our decoder design method and assess the decoder's practical performance through extensive simulations. In addition, we describe how to combine it with a KF to improve its performance in practice. Finally we present two more realistic numerical examples of UAVs subject to adversarial attack in Section \ref{sec:examples}. 
In this paper, we focus on estimation of states under sensor attack, hence the terms `estimator' and `decoder', `sensor' and `node', `attack vector' and `error vector' are used interchangeably.
 
\subsection{Notation}
\begin{itemize}
\item 
  denotes the support of vector , i.e., the number of nonzero components in \footnote{If  is any real-valued or vector-valued function on a topological space , the support of , denoted by , is the closure of the set points where  is nonzero: .}. 
\item 
 where . Note that this is not the same as  defined in \cite{Fawzi2014}. 
\item  
For a matrix ,  represents the null space of .  denotes the range space of , and is defined as the set of all possible linear combinations of its column vectors.
\end{itemize}









\section{Overview}\label{sec:overview}
\subsection{Secure Estimation for Fixed Attacked Nodes \cite{Fawzi2014} }Consider a linear dynamical system in the presence of attacks:

where  represents the state of the system at time ,  is the output of the sensors at time  and  represents attack signals injected by malicious agents at the sensors. 

In \cite{Fawzi2014}, the authors proposed an elegant state estimation algorithm against adversarial attacks, where the attack signals can be unbounded and arbitrary, but they assumed that the set of attacked nodes  does not change over time. More precisely, if  is the set of nodes that were attacked, then we have for all , . The problem of reconstructing the initial state  of the plant from the corrupted observations  was formulated as follows:

\begin{Def} \emph{(\hspace{1sp}\cite{Fawzi2014})}  errors are correctable after  steps by the decoder  if for any , any  with , and any sequence of vectors  in  such that , we have  where  for .
\end{Def}

\begin{prop}  \emph{(\hspace{1sp}\cite{Fawzi2014})}  \label{prop:Fawzi} Let . The following are equivalent:\\
(i) There is a decoder that can correct  errors after  steps;\\
(ii) For all , .
\end{prop}


\noindent The authors then proposed the optimal decoder:

where the  ``norm" of matrix  is the number of nonzero rows in  \cite{Fawzi2014}:

where  represents the -th row of .  and  is a linear map such that . Therefore

which we refer to as the error matrix in the sequel, and consists of horizontally stacked attack/error vectors from time  to .
In other words, the decoder finds the  that minimizes the number of nonzero rows in the resulting error matrix. The foundation for this decoder to work is the assumption of fixed attacked nodes.
As an illustration, consider a system with a single attacked node and construct the error matrix

where  denotes a nonzero component (i.e., attack or corruption). Observe that the set of nonzero rows corresponds to the set of attacked nodes, and furthermore, if a small number of nodes are attacked, then necessarily only a small number of rows are not identically zero and the error matrix has a small row support. The decoder in \cite{Fawzi2014} works by leveraging this property of fixed attacked nodes. 


What happens if the attacked nodes can change over time? To answer this, let's look at another example where the system has a single attacked node again, however the attacked node is cycled through all nodes such that the error matrix is as follows

Observe that although a single node is attacked at any time , since the attacked node is changing over time, the error matrix has full row support. Therefore a decoder that attempts to find the optimal  by minimizing the row support of the error matrix does not work here. One may argue that this decoder can be used with , to solve a secure estimation problem where the attacked nodes change over time. 
However, when , the number of correctable errors cannot be large.
From Proposition 1 (i.e. Proposition 2 in \cite{Fawzi2014}), ``the decoder can correct  errors after  step if and only if for all , ''. 
As an example, for any  where ,  has a nontrivial null space, hence there exists  such that ; in other words, zero errors are correctable. Therefore, the decoder for fixed attacked nodes cannot be easily extended to when the attacked nodes can change over time, and a new decoder is required for the latter.

In this paper, we propose such a new decoder, one that is suitable for when the set of attacked nodes can change over time. Our decoder is inspired by the error correction problem \cite{Candes_Tao} and by observing that if we construct the error matrix by stacking the error vectors  vertically, instead of horizontally, then even when the attacked nodes can change over time, as long as the number of attacked nodes at each time  is small, the error matrix (which is a large error vector in this case) is sparse. Before we dive into details of this method and its properties, we first give a brief overview about compressive sensing and error correction in the following section.




\subsection{Overview: Compressive Sensing and the Error Correction Problem \cite{Candes_Tao}} 

\subsubsection{Compressive Sensing}
Sparse solutions , are sought to the following problem:

where  are the measurements,  is a sensing matrix and  denotes the number of nonzero elements of . The following lemma provides a sufficient condition for a unique solution to (\ref{eq:CS}) \cite{Candes_Tao}:

\begin{lem} \emph{(\hspace{1sp}\cite{David_Chang})} \label{lem:CS}
If the sparsest solution to (\ref{eq:CS}) has ,  and all subsets of  columns of  are full rank, then the solution is unique. 
\end{lem}
\begin{proof}
Suppose the solution is not unique, hence there exists  such that  and  where . Then  and . Since  and all  columns of  are full rank (i.e. linearly independent), it is impossible to have  that satisfies  (contradiction).
\end{proof}



\subsubsection{The Error Correction Problem  \cite{Candes_Tao}} \label{sec:error_correction}
Consider the classical error correction problem:  where  is a coding matrix  and assumed to be full rank. We wish to recover the input vector  from corrupted measurements . Here,  is an arbitrary and unknown sparse error vector. To reconstruct , note that it is obviously sufficient to reconstruct the vector  since knowledge of  together with  gives , and consequently  since  has full rank \cite{Candes_Tao}. In \cite{Candes_Tao}, the authors construct a matrix  which annihilates  on the left, i.e.   for all . Then, they apply  to the output  and obtain

Thus, the decoding problem can be reduced to that of reconstructing a sparse vector  from the observations . Therefore, by Lemma \ref{lem:CS}, if all subsets of  columns of  are full rank, then we can reconstruct any  whose  . We refer to a decoder that can correct  errors as a -error-correcting decoder.

\subsubsection{ Equivalence between the two programs  and }\label{sec:equiv}
In \cite{Candes_Tao}, the authors mentioned that the computational intractability of the -program led researchers to develop alternatives, and a frequently discussed approach considers a similar program in the  norm which goes by the name of Basis Pursuit. Motivated by the problem of finding sparse decompositions of special signals in the field of mathematical signal processing, a series of beautiful and ground breaking works \cite{Donoho2003, Elad2002, Gribonval2003, Tropp2004} showed exact equivalence between the two programs  and  when the RIP conditions are satisfied. Therefore, the  norm in (\ref{eq:CS}) can be approximated by an  norm to give a convex decoder and is therefore computationally feasible. 
We will discuss in more detail the conditions required to ensure accurate decoding using an -optimization based decoder in Section \ref{sec:design}. 




\section{Secure Estimation when Attacked Nodes can Change with Time}  \label{sec:main}


\subsection{Problem Formulation}
Consider the linear time invariant system as follows:
 
where ,  and  are the states, measurements and control inputs at time .  is the attack signal, and we assume that the attacked nodes can change over time. We define the number of correctable errors as follows:

\begin{Def}\label{def:num_err_change}
When the set of attacked nodes can change over time,  errors are correctable after  steps by the decoder  if for any  and any sequence of vectors  in  such that , 
we have  where  for .
\end{Def}

In addition, assume that a local control loop implements secure state feedback: , i.e. the local control loop is not subject to attack. The resulting closed loop system matrix is  . 
In practice, this represents the following scenario: a physical system possesses a local control loop that has direct access to the state of the plant and can control the evolution of the physical system. This is reasonable if the sensors are connected to the local controller through a wired link that is not subject to external attacks. Also, as part of the overall plant, a higher-level supervisory and monitoring system receives measurements from the sensors through wireless and vulnerable communication links that are subject to attacks \cite{Fawzi2014}. 
A concrete example is a UAV that uses measurements from onboard, hardwired sensors such an Inertial Measurement Unit (IMU) for its autopilot and trajectory following (i.e. secure local control loop), and communicates wirelessly with a remote control center (i.e. vulnerable link subject to attack).




\subsection{Reformulation into an Error Correction Problem} \label{sec:err_corr}
The problem that we want to solve is how to reconstruct the initial state  of the system from the corrupted observations () where .
Let  denote the large vector with error vectors from time  to  stacked vertically: , where each  satisfies  as in Definition \ref{def:num_err_change}. 

where  is the set of vertically stacked corrupted measurements and  represents the observability matrix of the closed loop system. Assume , which is reasonable because otherwise, the system is unobservable and thus  cannot be determined even if there is no attack ().
\begin{itemize}
\item Open-loop case :  A full column rank condition represents the pair  being observable. In other words, if not, one cannot reconstruct  even if there are no errors in the measurements.
\item State-feedback case: Since state-feedback may affect the observability of a system (even though the pair  is observable), we have to satisfy  for the closed-loop system with state-feedback. \end{itemize}
Note that the closed-loop system with state-feedback is controllable if and only if the open-loop system is controllable. However, state-feedback may affect the observability of a system. 

Proposition \ref{prop:equivalent} below, gives two equivalent sufficient conditions for the existence of a unique solution  to (\ref{eq:decoder_Phi}).

\begin{prop} \label{prop:equivalent}
Given ,  is full rank and , then the following are equivalent: \\ all subsets of  columns of  are linearly independent, where   is the QR decomposition of  and ;\\
  for all .
\end{prop}
\begin{proof}
 Suppose there exist  columns of  that are linearly dependent. Then, there exists  such that  where . Observe that , therefore there exists  such that  (i.e., ). Then,  (contradiction).

: We again resort to contradiction. Suppose that there exists  such that . Let  and  be two disjoint subsets of  with  and  such that  (such  and  exist since ). Let  be the vector obtained from  by setting all the components outside of  to 0, and similarly let  (i.e., ). Then we have  with  and  with  and . Now, consider 

The last equality is due to  (i.e. ). Since all subsets of  columns of  are linearly independent,  (contradiction).
\end{proof}


Next, we present two methods, inspired by error correction techniques \cite{Candes_Tao}\cite{David_Chang}, for estimating . We show their equivalence and provide sufficient conditions for the existence of a unique solution.

\noindent
{\bf Decoder 1:~} The first method determines the error vector  first, and then solves for . Consider the  decomposition of ,

where  is orthogonal,  and  is a rank- upper triangular matrix. Pre-multiply (\ref{eq:decoder_Phi}) by :

We can now solve for  using the second block row:

where . From Lemma \ref{lem:CS}, Equation (\ref{eq:E_est}) has a unique, -sparse solution if all subsets of  columns of  are full rank (this is a reasonable assumption if ). 
Therefore, the secure decoder has two steps, we first solve the following -minimization problem

Note that the  norm in (\ref{eq:CS}) was replaced with with the  norm in (\ref{eq:solve_E}) to give a convex program, which was proposed by Candes and Tao in \cite{Candes_Tao}. In addition, in Section \ref{sec:equiv} we discussed the conditions for equivalence of these two programs.

With  in hand, we then solve for  in the second step using the first block row of Equation (\ref{eq:QR}):


\noindent
{\bf Decoder 2:~} 
The second method recovers  from the corrupted data  directly by solving the following -minimization problem \cite{Candes_Tao}:


\begin{lem} \label{lem:equivalent}
  is the unique solution of (\ref{eq:direct_l1}) if and only if  is the unique solution of (\ref{eq:solve_E}).
\end{lem}
\begin{proof} (By \cite{Candes_Tao}) Observe that on one hand, since  and we may decompose , hence 

On the other hand, the constraint  means that  for some  and, therefore,

Thus, (\ref{eq:solve_E}) and (\ref{eq:direct_l1}) are equivalent programs \cite{Candes_Tao}.
\end{proof}
\noindent Even though we are interested in the state  and not necessarily the error vectors , Lemma \ref{lem:equivalent} states that if the attack vectors cannot be uniquely determined from (\ref{eq:solve_E}), then we cannot estimate  uniquely from (\ref{eq:direct_l1}). \cite{Fawzi2014} also mentioned this notion: the existence of a decoder that can correct  errors is equivalent to saying that the map,  has an inverse for the first  components of its domain where  since the attack vectors are uniquely determined by  and the 's, i.e., . 






\subsection{Comparison with Secure Estimation for Fixed Attacked Nodes}
We refer to a decoder that can correct  errors as a -error-correcting decoder.
It is interesting to compare the conditions for the existence of a -error-correcting decoder for when the attacked nodes are fixed (Proposition \ref{prop:Fawzi}) and when the attacked nodes can change over time (Proposition \ref{prop:equivalent}) are :
,

It is easy to see that when , conditions  and  are equivalent as both of them reduce to  for all .
When  (i.e., with dynamics), the comparison is not so straightforward. In \cite{Fawzi2014}, the authors proved an equivalent condition for :

\begin{lem} \label{lem:distinct}
Assume  has  distinct positive eigenvalues () and . Then, the following are equivalent:

\end{lem}
\begin{proof} Refer to the proof in \cite{Fawzi2014}.
\end{proof}
\noindent
The significance of this lemma is that in order to check whether a decoder can guarantee accurate decoding of  errors when the attacked nodes are fixed, one no longer needs to check satisfiability of condition  which is stated for all  and hard to check, instead, one can simply check condition  for the eigenvectors of  which is much simpler. Next, we derive a similar result for our decoder for when the attacked nodes can change with time.

\begin{thm} Let . Assume that  is full rank,  is observable and  has  distinct positive eigenvalues such that . Define:
\begin{itemize}
\item
, where  is an eigenvector of , \item
,
\item
For every , let  be any subset of  with  elements, define .
Then  is such that  for all subsets , i.e. all subsets of  elements from the set .
\end{itemize}
Choose  such that  .
Then, the following are equivalent:

\end{thm}
\noindent
In order to prove Theorem 1, we make use of Lemmas \ref{lem:two_vec}, \ref{lem:three_vec} and Proposition \ref{prop:m_vec} (see Appendix): 

\begin{proof} (Proof of Theorem 1)

First, it is simple to prove that  and  are equivalent: . 

Second, we want to show that  and  are equivalent. The direction  is trivial, since  is a specific case of (iii) with . The other direction is more complex. Note that  is diagonalizable, therefore its eigenvectors form a basis for . Now consider the decomposition of  in the eigenbasis of , i.e.  with  for at least one . 
\begin{enumerate}
\item : Suppose there exists  such that . Without loss of generality, let  and  for all , then,  for all  (contradiction, ). 
\item : By Lemma \ref{lem:two_vec}, if we choose .
\item : By Lemma \ref{lem:three_vec} and Proposition \ref{prop:m_vec}, if we choose  for each value of , respectively.
\end{enumerate}
We need to choose  to satisfy the worst case for any  such that . Thus, if , then  and  are also equivalent.
\end{proof}



\begin{prop}\label{prop:equivalent2}
Given  is full rank, the closed-loop matrix  has  distinct positive eigenvalues, the open-loop pair  is controllable, the closed-loop pair  is observable and  is chosen to satisfy Theorem 1. Then, the condition for secure estimation of -errors when the set of attacked nodes is fixed ((i) in (\ref{eq:connection})) is the same as the condition for when the set of attacked nodes can change over time ((ii) in (\ref{eq:connection})), except the condition on .
\end{prop}
\begin{proof}
Since the pair  is controllable, there exists a feedback matrix  such that the eigenvalues of the closed-loop matrix , i.e.,  can be arbitrarily located on the complex plane. Then Proposition \ref{prop:equivalent2} directly follows from Proposition \ref{prop:equivalent}, Lemmas \ref{lem:distinct} and Theorem 1 
\end{proof}

Theorem 1 and Proposition \ref{prop:equivalent2} state that if the feedback system and the secure decoder are designed such that the conditions in Proposition \ref{prop:equivalent2} are satisfied, then we can guarantee accurate correction of  errors using our proposed secure decoder for attacked nodes that can change over time, by checking the following very simple condition:

And interestingly, this is the exact same condition that one should check if one is designing the decoder from \cite{Fawzi2014} for fixed attack nodes. In other words, it is equally easy to check satisfiability of the sufficient condition for -error-correction for both types of decoders.




\subsection{Discussion on Sufficient Condition of }

Unsurprisingly, the condition on  from Theorem 1, is different from that for a decoder designed for fixed attack nodes (). In \cite{Fawzi2014}, since the set of attack nodes is fixed, one can leverage the property of fixed attacked nodes (i.e., the number of nonzero rows in (\ref{eq:opt_decoder})). However, in our setting, since the set of attack nodes can change over time, we cannot leverage this property and thus, we need more time steps .
In general, it is difficult to see what the value of  is, from the formula in Theorem 1. However it is reasonable to assume that one would design the feedback matrix  and matrix  to maximize the number of errors that can be corrected, i.e.  (proved in Section \ref{sec:max_q}). To achieve this, we must design  and  such that  for all eigenvectors of . In other words,  for all  and for all subsets . 
In this case, it is easy to see that according Theorem 1, for any  that is an even number, we must choose: .
In most cases, this  is larger than  (the value of  required for the decoder for fixed attacked nodes).

At the first look, this may seem like an unsatisfying result, however there are two important points to note. First, a larger value of  merely translates to a longer initial delay from time  to  when the decoder collects enough () measurements. From time  onwards, the decoding is in real time and takes place at every time step  as the decoder uses a ``sliding window'' of observations. Second, extensive simulation results in this paper show that a value of  is often sufficient for our proposed decoder to achieve good estimation, i.e. to perfectly recover the state under attacks where the attacked nodes change over time. 

This is because we consider the worst case (conservative) scenario in Theorem 1. 
In the proof, we consider the case where  has  distinct positive eigenvalues. What if  has complex eigenvalues? For instance, assume that   and it has one pair of complex conjugate eigenvalues and one real eigenvalue, i.e.,  where ,  and  represents the complex conjugate. We denote ,  and  where , ,  are linearly independent. Any  can be represented by a linear combination of  independent vectors in , i.e.,  where ,  and  and . Therefore, the same results for real eigenvalues applies. In other words, if  and , then  where  represents the number of cancelled support of linear combinations of  and  at time step .



\subsection{Number of Correctable Errors}\label{sec:max_q}

Given that the set of attacked nodes can change over time and  satisfies  for all , we prove in Proposition \ref{prop:maximum} (see below) that the maximum number of correctable errors (as defined in Definition \ref{def:num_err_change}) by our decoder is , where  is the number of measurements. This is in fact the same as the maximum number of correctable errors for the decoder proposed in \cite{Fawzi2014} which is for fixed attacked nodes.
This is a pleasing result, because it demonstrates that with our proposed decoder, we can relax the assumption of fixed attacked nodes and protect the system against more general attacks, without compromising the maximum number of correctable errors. 



\begin{prop}\label{prop:maximum} 
Let ,  and  and assume that the pair (, ) is controllable,  is full rank and each row of  is not identically zero. Then there exists a finite set  such that for any choice of  numbers  such that , there exists  such that:
\begin{itemize}
\item
The eigenvalues of the closed-loop matrix  are .
\item
If the pair () is observable, then the number of correctable errors for the pair () is maximal after  time steps and is equal to , where  is the value of  from Theorem 1. 
\end{itemize}
\end{prop}

\begin{proof}
The proof for Proposition 4 in \cite{Fawzi2014} shows that if the chosen poles  are distinct, positive and do not fall in some finite set , then there is a choice of  such that the eigenvalues of  are exactly the , and the corresponding eigenvectors  are such that . Thus, by Proposition \ref{prop:equivalent2}, the number of correctable errors for  is .
\end{proof}





In addition, recall that  consists of vertically stacked error vectors from , and observe that our proofs for accurate decoding are independent of how the individual error (nonzero) terms are distributed in the vector . Thus, if we remove the assumption:  for all , and allow  to appear in an arbitrary fashion, e.g.  and , as long as , then our -error-correcting decoder can still recover the true states. In other words, our proposed decoder can protect the system against more general attacks, where the number of attacked nodes is not necessarily less than or equal to  at every time.











\begin{figure*}[!t]
\center
\includegraphics[width=0.85\textwidth]{performance_n8p10.pdf}
\caption{Success rate and mean error of  decoder on different systems (ideal coding matrix, designed state feedback and poorly designed system with ,  and , where black dot lines show the fundamental limit for dynamical systems and ideal coding matrix case respectively. We see that as the number of attacked nodes increase, success rate decreases. Also, by designing state feedback gain properly, we improve success rate and decrease mean error. }
\label{fig:ex_n8p10}
\end{figure*}



\section{Optimal Decoder Design}\label{sec:design}
 
In the classical error correction problem, to ensure accurate decoding, the coding matrix must satisfy the RIP conditions  \cite{Candes_Tao}, which are extremely difficult to check in general. In practice, Theorem 1.4 from\cite{Candes_Tao} is almost always used to design a coding matrix {\it a priori}. This theorem states that a coding matrix whose entries are sampled from independent and identical distributions satisfies the RIP condition with overwhelming probability. 
In secure estimation, however, it is impossible to choose such a coding matrix {\it a priori} because it is the observability matrix , which is structurally constrained: as shown in (\ref{eq:decoder_Phi}),  consists of 's where . In this Section, we use Lemma \ref{lem:distinct}, the results from Proposition 4 in \cite{Fawzi2014} and state feedback to design a matrix  for accurate decoding. 




\subsection{System and Decoder Design}\label{sec:decoder_design}

Since conditions  and  in Theorem 1 are equivalent, condition  can be used to design a state feedback controller such that the closed system can achieve accurate decoding. Therefore, given a controllable open-loop pair , design  and choose an adequate feedback control law  and construct a secure decoder such that:
\begin{enumerate}
	\item Each row of  is not identically zero, and  is full rank;
	\item The closed-loop matrix  has  distinct positive eigenvalues: ;
	\item  is observable;
	\item The length of the sliding window of measurements  of the decoder satisfies Theorem 1\footnote{We found that much smaller 's are often sufficient for good secure estimation performance, i.e. to perfectly recover the attack signals. In all simulations in this paper,  is used, where  is the number of states.};
	\item Maximize  subject to:  where , .
\end{enumerate}
Without loss of generality, the first condition holds. For example, if there exists a zero row in , we can simply remove that row from  without changing the system's behavior. Conditions 2, 3 and 4 are required for equivalence in Theorem 1. The last condition is needed for accurate decoding and for maximizing the number of correctable attacks. From Proposition \ref{prop:maximum} the maximum number of correctable errors can be achieved when  (i.e., the number of measurements) for all eigenvectors of . 

Conditions 2, 3 and 5 depend on the feedback controller. So how do we choose a controller that achieves good performance in both control and secure estimation? Below, we describe an approach that we have taken in all simulations in this paper, and has proved to work well. This is by no means the only method. First, we design a controller that achieves good control, for example, Linear Quadratic Regulators (LQR), which are optimal with respect to a certain quadratic cost function. However, these controllers may not have good secure estimation properties, meaning the value of  that satisfies  for all eigenvectors of  may be small, i.e., the resulting decoder can only correct few errors. It is often easy to increase the value of  and make the decoder more resilient to attacks by slightly perturbing the closed-loop poles from those resulting from the LQR controller, such as placing the poles closer to the origin, and making the poles more spread out amongst themselves. We chose to keep the perturbations small as to not lose too much control performance.  
Although this is a heuristic method, it is relatively easy to carry out in order to satisfy the above conditions; whereas in the classical error correction method \cite{Candes_Tao}, checking whether a coding matrix satisfies RIP is extremely difficult.

To summarize, we start from some optimal controller which may not result in a good decoder, then we perturb the closed-loop poles slightly to improve the resulting decoder's secure estimation capability. Therefore there is a trade-off between a system's control and secure estimation performances, and the feedback controller can be designed to achieve a desired trade-off between them.





\begin{figure}
\center
\includegraphics[width=0.45\textwidth]{performance_overall.pdf}
\caption{Success rate and mean error of  decoder on three different systems (ideal coding matrix, designed state feedback and poorly designed system with  and ) with different . Black solid lines show the fundamental limit for dynamical systems and black dashed lines show the fundamental limit for the ideal coding matrix case. We see that as the number of attacked nodes increases, the success rate decreases. Also, by designing the state feedback gain properly, we improve success rate and decrease mean error. }
\label{fig:ex_n8_overall}
\end{figure}





\subsection{Practical Performance}\label{sec:prac_perf}
In this Section, we show the performance of our proposed decoder using an arbitrary system with ,  and , where  is chosen arbitrarily (i.e., the entries of  are chosen as independent and identically distributed random variables) and  is chosen such that every row of  has only one nonzero component (i.e., each row of  is not identically zero) and is full rank. 
For each value of  (i.e. number of total attacked nodes during  steps), we test the decoder on 500 independent trials. For each trial, both the system matrices and initial conditions are re-generated. The initial conditions  were randomly generated from the standard Gaussian distribution and the set of attacked nodes are chosen at random and can change over time. Since we increase  in steps of  as shown in Figure \ref{fig:ex_n8p10}, we first distribute  attacks arbitrary for each time step  and randomly distribute the remaining  attacks amongst . For example, in Figure \ref{fig:ex_n8p10}, for , we have at least 2 attacks for each  where  and 
distribute the remaining  attacks arbitrary. 

Figures \ref{fig:ex_n8p10} and \ref{fig:ex_n8_overall} show the performance of our proposed decoder on three different systems: (1) an ideal random coding matrix (i.e., with i.i.d. Gaussian entries), (2) a LTI system with a state feedback controller designed using the method in Section \ref{sec:decoder_design} and (3) a LTI system with a poorly designed feedback controller. 
The performance metric used is the \textit{success rate}, which is defined as the fraction of the total 500 trials that our decoder is able to perfectly recover the attack signals.
System 1 acts as the baseline for the comparison because a random coding matrix has good RIP and represents the best achievable decoding performance. However it is not a realistic dynamical system. For a realistic LTI system, the coding matrix  is constructed by stacking 's as in (\ref{eq:decoder_Phi}) where . This structural constraint can reduce the restricted isometry constant (i.e., a measure of how good the RIP is, larger restricted isometry constants correspond to better RIP), and indeed, the success rates for system 3 are lower than those for system 1. Figure 1 also shows that by designing the feedback controller, we can recover some of the lost RIP due to the structure in the system dynamics, and improve the success rate (system 2).
Fundamental limits of the -optimization for both secure decoding and ideal coding are also shown in Figure \ref{fig:ex_n8p10}. 


Figure \ref{fig:ex_n8_overall} shows that as  increases, more measurements () are needed to correctly recover the state of the system. In practice, this can be done by fusing different types of measurements and sensors together. For example, consider a simplified UAV dynamics where  and 's and 's represent positions and velocities, respectively. The observation equations of the IMU and GPS are as follows:

By combining measurements from both IMU and GPS, we can increase the types of measurements (), .

Finally, we want to point out that  in all these simulations, which is much smaller than that dictated by Theorem 1 ().
Nevertheless, our decoder's success rates are quite high: in Figure \ref{fig:ex_n8p10}, it achieves success rates of more than 0.8 for system~2 for all attacks where the number of attacked nodes is below the maximum number of correctable errors, in Figure \ref{fig:ex_n8_overall}, these success rates increase to above 0.9 when 12 measurements are available. These results suggest that the sufficient condition on  given in Theorem 1 is conservative, and can be relaxed in practice.

\begin{figure}
\center
\includegraphics[width=0.45\textwidth]{est_illu.pdf}
\caption{Illustrative comparison of three schemes: KF only (KF), secure estimator only (SE), secure estimator with a KF (SE+KF). KF fails to estimate the true state as attack signal is non-Gaussian. SE correctly estimates the system state most of the time but has occasional large estimation errors. SE+KF tracks the true state trajectory perfectly.}
\label{fig:estimation}
\end{figure}




\subsection{Combination of Secure Estimation and Kalman Filter}\label{sec:estimation}
Consider the state estimation problem for the following LTI system under attack:

where , ,  and  are as defined in (\ref{eq:system_model_se});
 and  are zero mean i.i.d. Gaussian process noise and measurement noise, respectively. 

We can model the attack signal as noise and use a KF to estimate the states. More specifically, one would define a new measurement noise vector , so that the measurement equation becomes . A KF can then estimate the states from the inputs  and the corrupted measurements  \cite{KwonACC}. However, KFs are derived using the assumption that both process and measurement noises are zero mean i.i.d. white Gaussian processes, but attack signals are usually erratic and may be poorly modeled by Gaussian processes \cite{KwonACC}. For example, in GPS spoofing attacks, attack signals are often structured to resemble normal GPS signals or can be genuine GPS signals captured elsewhere. When the system is subjected to attacks that are poorly modeled by Gaussian processes, a KF is expected to fail to recover the true states. Figure \ref{fig:estimation} gives an illustrative example where an attack signal that increases linearly with time is injected into the measurements of state . The red dashed line shows a plausible estimated state trajectory from a KF.

On the other hand, our proposed secure decoder works for arbitrary and unbounded attacks.
Nevertheless, Figures \ref{fig:ex_n8p10} and \ref{fig:ex_n8_overall} show that when the number of attacked nodes are close to the theoretical maximum number of correctable errors, our decoder occasionally fails to perfectly recover the states.
The green dashed line in Figure \ref{fig:estimation} depicts a possible result from this decoder: the estimated state trajectory follows the true trajectory most of the time with occasional errors.

Therefore, to improve the performance, we propose to combine our secure decoder with a KF as follows:
\begin{algorithm}
\caption{Combined secure estimator with KF}
\label{al:se_kf}
\begin{algorithmic}[1]
\State Initialize the KF
\For{each }
	\If{}
		\State Estimate the attack signal at time , , using secure estimator
	\Else
		\State Set 
	\EndIf
	\State Form a new measurement equation: , where  and 
	\State Apply standard KF using  and  
\EndFor
\end{algorithmic}
\end{algorithm}

\noindent
The intuition is that the secure decoder acts as a pre-filter for the KF, so that  is close to a zero mean i.i.d. Gaussian process even when the true attack signal  is not. More specifically, the secure decoder usually perfectly recovers , thus  and . What happens when the secure decoder fails? Equation (\ref{eq:sys_err_corr}) shows that the estimated state at time , , does not directly depend on the estimated state at another time point  (). As a result, when the secure decoder fails, its estimation error, , appears to be quite random. Putting these together:  is closer to a zero mean white Gaussian process than the original attack signal , and this improves the KF's performance. 

Finally, the \textit{if} statement in Algorithm 1 ensures that the secure estimator always has access to  past measurements, as required by Theorem 1.
A more realistic example illustrating these behaviors is shown in Figure \ref{fig:ex_uav_remote}.




\section{Numerical Examples}\label{sec:examples}

On February 15, 2015, the Federal Aviation Administration proposed to allow routine use of certain small, non-recreational UAVs in today's aviation system \cite{faa}. Thus in the near future, we may see thousands of UAVs such as Amazon Prime Air \cite{Amazon} and Google Project Wing vehicles \cite{Google} sharing the airspace simultaneously. To ensure safety of this immense UAV traffic, UAVs may periodically update their position and velocity measurements wirelessly to a Remote Control Center (RCC) for traffic management (Channel 1 in Figure \ref{fig:ex_uav_pic}). At the same time, UAVs may broadcast this information to other UAVs in its vicinity for collaborative collision avoidance (Channel 2 in Figure \ref{fig:ex_uav_pic}). Finally, autonomous UAVs may use GPS for their position measurements (Channel 3 in Figure \ref{fig:ex_uav_pic}). 
All these communication channels are subject to cyber attacks. 
If corrupted information are used in collision avoidance or path planning algorithm, they can lead to possible collisions or loss of UAVs, causing physical and financial damage and even injury to civilians.
To help protect against these attacks and consequences, participating entities such as the UAVs and the RCC can use secure estimation to estimate a target UAV's true position and velocity before using any received information for collision avoidance, for instance.
In this section, we focus on 2 types of adversarial cyber attacks on UAVs and demonstrate the effectiveness of our secure estimator through simulations.

\begin{figure}
\center
\includegraphics[width=0.35\textwidth]{uav_pic.pdf}
\caption{Different communication channels that are subject to adversarial attacks.}
\label{fig:ex_uav_pic}
\end{figure}



\begin{figure}[b]
\center
\includegraphics[width=0.40\textwidth]{supp_ev_2.pdf}
\caption{ for all eigenvectors  of closed-loop matrix  for 2 feedback controllers: a LQR and a controller designed by pole-placement. Black dashed line is at , i.e., the number of measurements.}
\label{fig:ex_pole}
\end{figure}


\subsection{UAV Model}
We consider a quadrotor with the following dynamics:

\begin{figure*}

\end{figure*}
\noindent 
where  is the state vector. ,  and  represent the quadrotor's position along the ,  and  axis, respectively. ,  and  represent its velocities.  and  are the pitch and roll angles respectively,  and  are their corresponding angular velocities.  is the input vector:  is the reference pitch or roll angle, and  is the commanded thrust in the vertical direction.  represents compromised position measurements from the GPS under attack signal .  and  represent process and measurement noise respectively.  is a constant vector which represents gravitational effects, and can be dropped without loss of generality because we can always subtract it out in .  refers to the -th entry of the subsystem matrix of the discretized rotational dynamics , and  refers to the -th entry of the input-to-state map  for the discretized rotational dynamics.  is the discrete time step,  is the gravitational acceleration,  is the mass of the quadrotor and  is a thrust coefficient. Further details about this model and its derivation can be found in \cite{Bouffard}. Finally, the matrix  depends on the particular measurements taken in each example.



\subsection{Decoder Design via Pole-Placement}

Assume that the UAV uses the state feedback control law , where  is the feedback matrix which can be designed\footnote{In the GPS spoofing example, direct uncorrupted state measurements are not available. Therefore a KF is used to give estimated states which are then used for state feedback control.}. If the pair  is controllable, then we can choose  to place the closed loop poles anywhere in the complex plane. We first design a Linear Quadratic Regulator (LQR) and evaluate its secure estimation performance by checking whether the sufficient condition for -error correction  (i.e.,  for all ) holds. 
Figure \ref{fig:ex_pole} shows the results for a matrix  (i.e., 5 measurements) and observe that  for  and . Furthermore,  for  and , therefore the resulting secure decoder can correct zero errors!
To improve the secure estimation performance, we perturb the closed-loop poles slightly until  for all , as shown in Figure \ref{fig:ex_pole}. Therefore the resulting secure decoder can achieve the maximum number of correctable errors within the limits of  (i.e., the number of measurements). By keeping the perturbations on the poles small, our final controller achieves both good control and estimation performances (see Figure \ref{fig:ex_pp_est}).




\subsection{UAV under Adversarial Attack}

\subsubsection{MITM Attack in Communication with a RCC or with other UAVs} \label{sec:uav_utm}
In this section, we consider MITM attacks targeted at Channels 1 and 2 in Figure \ref{fig:ex_uav_pic}, where a malicious agent spoofs the information being sent and/or received over these channels. The goal of the RCC or other UAVs is to accurately estimate the true flight path of a target UAV from corrupted measurements. 
Note that the true path of the target UAV is unaffected by the attack.
Assume that the attacker spoofs the position measurements in order to deceive the receiver that the target UAV is deviating in the -direction, i.e., he/she injects a continuous and increasing signal in the -position measurement.
To make the estimation task even harder for the receiver, the attacker also injects a random Gaussian noise to an additional measurement, and the choice of this measurement can change at each time step. 

In this example, we first demonstrate the effectiveness of our proposed decoder design via pole-placement method by comparing the estimation performance of the decoder resulting from (1) a LQR controller and (2) a controller designed using pole-placement as described in the previous section.
We then implement the latter feedback controller, and compare the performance of three different state estimation schemes: (1) KF only (KF), (2) secure estimator only (SE), and (3) secure estimator combined with KF (KF+SE). 

Throughout this example, , measurements include the ,  and  positions and 2 additional randomly selected states. The left plots in Figure \ref{fig:ex_pp_est} show the true attack signal on all 5 sensors (solid lines) and the estimated attack signals (dashed lines) by the secure decoder if the feedback controller is a LQ regulator (top) or one designed via pole-placement (bottom). It is obvious that the latter estimates the attack signal much more accurately. The right plots of this figure highlights this observation by explicitly showing the estimation error of the attack signal for each measurement.

The same information is shown in Figure \ref{fig:ex_pp_err}, where each row corresponds to one sensor, and the first 3 rows are the ,  and  position measurements, respectively. This figure highlights three points: first, the attacked sensors change with time; second, the number of attacked sensors at each time  is less or equal to 2; third, only position measurements are corrupted.

Figure \ref{fig:ex_uav_remote} shows the estimated flight paths by all three methods.
The true path of the UAV (solid blue line) starts from the position marked by the blue triangle and ends at the position marked by the blue square. KF fails to filter out the attack signal in the -position measurements as the attack is highly non-Gaussian, and the estimated trajectory (dashed red line) significantly differs from the true one. On the other hand, SE correctly estimates some portions of the trajectory and the final position of the vehicle, nevertheless it produces spontaneous errors in the  direction. 
Finally the combined method KF+SE perfectly recovers the true path of the target UAV.


\begin{figure}
\center
\includegraphics[width=0.45\textwidth]{uav_pp_est}
\caption{True attack signal, estimated attack signal and estimation error in the attack signal of the estimator (SE) with 2 different feedback controllers: LQR, controller designed via pole-placement (PP); with 5 measurements. In the left plots, solid lines are true attack signals, dashed lines are estimated signals. The right plots show the estimation error in the attack signal.}
\label{fig:ex_pp_est}
\end{figure}


\begin{figure}
\center
\includegraphics[width=0.48\textwidth]{uav_pp_error.pdf}
\caption{Estimated attack signal, true attack signal and estimation error in the attack signal of the estimator (SE) with 2 different feedback controllers: LQR, controller designed via pole-placement (PP); with 5 measurements. Left column shows estimated attack signals. Middle column shows true attack signal. Right column shows estimation error. Each row corresponds to one type of measurement. Red pixels indicate positive values, green pixels are negative values and black indicates zero. }
\label{fig:ex_pp_err}
\end{figure}


\begin{figure}
\center
\includegraphics[width=0.5\textwidth]{uav_lq_traj}
\caption{Estimated UAV trajectory by three methods under MITM attack: KF only (KF), secure estimator only (SE), secure estimator with KF (KF+SE). Solid blue lines are the true UAV trajectories. They start from the blue triangle and end at the blue square. Red dotted lines represent estimated trajectories by each method, with 5 measurements.}
\label{fig:ex_uav_remote}
\end{figure}




\subsubsection{GPS Spoofing}

In this section, we focus on adversarial attacks in the GPS navigation system (Channel 3 in Figure \ref{fig:ex_uav_pic}). Consider the scenario where a UAV uses a Linear Quadratic Gaussian (LQG) controller to follow a desired path, , designed by LQ control. In other words, a KF takes compromised and noisy measurements  and outputs a state estimate , which is then used for state feedback control: , where  is the feedback matrix. Note that in the previous example (Section \ref{sec:uav_utm}), the feedback controller had access to uncorrupted state measurements , therefore the true path of the UAV is unaffected by attacks. On the other hand, in this example, the UAV uses estimated states  for feedback control and path following. Hence, if measurements are corrupted and the state estimates are poor, then the UAV may not be able to follow its desired path and may deviate away from it. The goal is to correctly estimate the true states of the UAV and therefore, follow the desired path. Assume an attacker spoofs the GPS position measurements in order to deviate the UAV from its planned path. He/she injects a sinusoidal signal to -position measurement, as well as a Gaussian noise to a randomly chosen position measurement at each time step. 

In this example, we explore the effect of the number of sensor measurements on the secure estimation performance of two schemes: (a) KF only, (b) KF+SE.
We first assume that the UAV only uses GPS for navigation, i.e., 3 positional measurements. 
Figure \ref{fig:ex_uav_error} shows that KF completely fails to estimate the attack signal (KF, , plots in Row 1), consequently the actual UAV trajectory (red dashed line)  deviates significantly from its desired path (solid blue line) as shown in Figure \ref{fig:ex_uav_traj}, and deviations are largest along the - and -axis (Figure \ref{fig:ex_uav_est}).
On the other hand, Figures \ref{fig:ex_uav_error} (KF + SE, ) shows that KF+SE's estimated attack signals are significantly more accurate with only a small estimation error in the -position (plots in Row 2). 
Therefore the UAV can follow its planned path much more closely (Figures \ref{fig:ex_uav_traj}  and \ref{fig:ex_uav_est}).
Recall from Proposition \ref{prop:maximum} that the maximum number of correctable errors for a system with  measurements is , which equals 1 in this case. There are at most 2 attacked nodes at any time  in this example, which exceeds the above limit. This explains the estimation error in the -position. Despite this small estimation error, the combined scheme KF+SE still outperforms the KF on its own.

We now show the effect of increasing the number of measurements (, or equivalently ) through sensor fusion, on the estimation performance and consequently, the UAV's path following performance. Autonomous UAVs often use IMUs in addition to GPS for navigation, the former provides additional measurements such as the UAV's velocities, pitch and roll angles. Figure \ref{fig:ex_uav_error} shows that increasing the number of measurements has no affect on the KF's estimation accuracy (compare plots in Rows 1, 3 and 5). 
Even when 8 measurements are used the UAV equipped with a KF still fails to follow the desired path (Figures \ref{fig:ex_uav_traj} and \ref{fig:ex_uav_est}). On the other hand, increasing the number of measurements improves the estimation performance of the secure decoder SE and consequently the performance of the combined scheme KF+SE (compare plots in Rows 2, 4 and 6 in Figure  \ref{fig:ex_uav_error}). Observe that for both 3 and 5 measurements, the combined scheme KF+SE perfectly estimate the attack signals and therefore, can completely subtract them out from the corrupted measurements. As a result, the UAV can follow its original planned path perfectly (KF + SE , KF + SE  in Figures \ref{fig:ex_uav_traj} and \ref{fig:ex_uav_est}).


\begin{figure}
\center
\includegraphics[width=0.45\textwidth]{uav_lqg_error}
\caption{Estimated attack signal, true attack signal and estimation error in different cases: KF and KF+SE, each using 3, 5 and 8 different measurements. Left column shows estimated attack signals. Middle column shows true attack signal. Right column shows estimation error. Each row corresponds to one sensor measurement and the first three rows in each plot are the ,  and  position measurements, respectively. Red pixels indicate positive values, green pixels are negative values and black indicates zero.}
\label{fig:ex_uav_error}
\end{figure}



\begin{figure}
\center
\includegraphics[width=0.5\textwidth]{uav_lqg_traj}
\caption{Desired and actual UAV trajectory in different cases: KF and KF+SE, each using 3, 5 and 8 different measurements. Blue solid lines are the desired trajectory. Red dash lines are the actual UAV trajectory under adversarial attack.}
\label{fig:ex_uav_traj}
\end{figure}


\begin{figure}
\center
\includegraphics[width=0.5\textwidth]{uav_lqg_est}
\caption{Desired and actual UAV trajectory in different cases: KF and KF+SE, each using 3, 5 and 8 different measurements. Blue solid lines are the desired ,  and  trajectories. Red dashed lines are the actual UAV trajectories under adversarial attack.}
\label{fig:ex_uav_est}
\end{figure}





\section{Conclusion}
In this paper, we consider the problem of secure estimation for CPS under adversarial attacks. Unlike \cite{Fawzi2014} where the attacked nodes are assumed to be fixed, we allow the set of attacked nodes to change over time, and propose a computationally efficient secure decoder for the latter scenario that works for arbitrary and unbounded attacks. In addition, we propose to combine the secure decoder with a KF for improved practical performance. We demonstrate through numerical examples, that our proposed secure estimator based KF outperforms standard KF. Furthermore, we illustrate practical applications of secure estimation in UAVs under adversarial cyber attacks. This is important not only for today's aviation system but also UAV delivery systems in the near future. 



\section*{APPENDIX}

\subsection*{(Proof of Theorem 1)}
In the following lemmas and proposition, we assume the following:
\begin{enumerate}
\item  has  distinct positive eigenvalues such that .\item  is full rank. 
\item The pair  is observable.
\item  where   (i.e.,  is an eigenvector of ), .
\end{enumerate}

Recall that we want to show that given  for all eigenvectors  of , then  for all . Now,  has  distinct eigenvalues, hence the eigenvectors of  form a basis for , and any  can be expressed in the eigenbasis of , i.e., . Therefore, , and thus, the only way that the number of nonzero terms in  may be less than  is if too many nonzero terms in  are cancelled during this summation. 
In other words, there are rows that are nonzero in the vectors , however after the scaling by 's and summation, these rows become zero in . Therefore our goal is to prove upper bounds on the number of such cancellations, and use these upper bounds to derive a value of  such that even in the worst case (i.e., with most number of cancellations), the number of nonzero terms in  is greater than  for all .

Before we present the proof, we define the following notations:  is the -th row of , , . For all ,  where .
In addition, with these notations, .



We will first consider two simple cases where  is spanned by two and three eigenvectors ( and ) respectively, and then generalize the results to . The proofs use the following result.

\begin{lem} \label{lem:gv}
For  real numbers , and  positive integers , the Generalized Vandermonde matrix  defined as
\nonumber
is nonsingular.
\end{lem}
 
\begin{proof}
 is a submatrix of a Vandermonde matrix  defined as
\nonumber
where  is a positive integer, ,  and .

 is a Totally Positive (TP) matrix \cite{fallat2011tnm}, and by definition of TP matrices, all minors of , i.e., the determinant of all submatrices of , are positive. Therefore  is nonsingular.
\end{proof}




\begin{lem}\label{lem:two_vec}
()  Consider  (i.e.,  , , ):
\begin{enumerate}
\item 
If there exists  such that  and  but the -th row of , i.e., there is a cancellation of the -th row at time , then the -th row of  for all  where . In other words, there cannot be another cancellation of the -th row at any other time step, or equivalently, there is a maximum of one cancellation of the -th row over  time steps.
\item 
If we choose , then . 
\end{enumerate}
\end{lem}


\begin{proof}
(1) (Suppose not) There exist  and  such that :

Without loss of generality, assume . We can reformulate above equation as follows:

Now, the first matrix on the left hand side (LHS) is a Generalized Vandermonde matrix  with  and , thus by Lemma \ref{lem:gv} it is nonsingular. The second matrix on the LHS is also nonsingular as  and . Therefore we must have  (contradiction, since  and  by assumption).

(2) Let  be three disjoint subsets of  such that , , and  where the superscript  represents the set complement. Then, , , ,   and . 
Also, possible cancellations only occur in the subset  by definition. 

where  is the number of cancelled support in  at time step . 
More specifically,  if  and , but .
From (1), we have the followings:

Thus, , and

\end{proof}



\begin{lem} \label{lem:three_vec} Consider  where , ,  and  for .   
\begin{enumerate}
\item 
 where .
\item  
If we choose ,  then .
\end{enumerate}
\end{lem}
\begin{proof} (1) Claim: the -th row of  is cancelled at most 2 times over  time steps, i.e. for at most 2 distinct values of . 
	
	(Suppose not) There exist 3 distinct time steps  such that . Without loss of generality, assume :
	
	
Again, the first matrix on the LHS is a Generalized Vandermonde matrix satisfying the conditions of Lemma \ref{lem:gv}, hence it is nonsingular. The second matrix on the LHS is also nonsingular as ,  and . Therefore we must have  (contradiction). A similar derivation as in Lemma \ref{lem:two_vec} then shows . 

(2) Consider  and :\\

\begin{tabular}[!b]{ccc} 
  \hline
   &    &   \\
  \hline
	 & {\bf 0 } & {\bf 0} \\
  {\bf 0}  &   & {\bf 0} \\
  {\bf 0} & {\bf 0} &   \\
  &   & {\bf 0 } \\
  {\bf 0} &   &     \\
  & {\bf 0} &     \\
 &    &  \\
  \hline
\end{tabular}\bigskip\\
\noindent Without loss of generality, assume  (recall ):

where
 ,  and  and note that .
\end{proof}

\begin{prop} \label{prop:m_vec}
Consider  eigenvector combinations . Then, the total number of cancellations over  time steps satisfies  where .
\end{prop}
\begin{proof}
Claim: the -th row of  is cancelled at most () times over  time steps, i.e. for at most () distinct values of 
(Suppose not)

Again, the first matrix on the LHS is a Generalized Vandermonde matrix satisfying the conditions of Lemma \ref{lem:gv}, hence it is nonsingular. The second matrix on the LHS is also nonsingular as  for all . Therefore we must have  (contradiction). A similar derivation as in Lemma \ref{lem:two_vec} then shows .
\end{proof}



























\section*{Acknowledgment}
This work was supported by the NSF CPS project ActionWebs under grant number 0931843, NSF CPS project FORCES under grant number
1239166.





\bibliographystyle{IEEEtran}


\bibliography{reference}


\end{document}
