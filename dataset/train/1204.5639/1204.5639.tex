The -induction method \cite{SheeranEtAl:FMCAD2000,EenSorensson:2003}
inductively proves a reachability property for a system or
discovers a counter-example while trying to prove the property.
In the following, we will extend -induction,
which was originally proposed as a verification method for finite-state systems,
to a complete verification method for STTS.

As the base case of an inductive proof, -induction shows that no bad
state can be reached within  steps starting from an initial state
for some . As the inductive step, -induction shows that it is impossible under the transition relation of the system to have a path consisting for  good (property-satisfying) states followed by a bad (property-violating) state. Together, base case and inductive step prove that the property holds in any reachable state.

For an untimed system ,
both the base case and inductive step can be proven using a SAT solver.
The base case holds iff the formula

is unsatisfiable. Likewise, the inductive step holds iff the formula

is unsatisfiable.
Initially, -induction attempts an inductive proof with . If unsuccessful,  is increased until the inductive proof succeeds or a counter-example is found while checking the base case.
Note that the large overlap both between the formulas for checking base case and inductive step and between the checks before and after increasing  can be exploited by incremental SAT solvers \cite{EenSorensson:2003}.


While correct, the described approach is not complete due to the fact that the induction step is not guaranteed to hold even if the property checked is satisfied by the system. -induction can, however, be made complete for finite-state systems by only considering simple (non-looping) paths when checking the inductive step.
The most straightforward way to enforce paths to be simple, is to add a quadratic number disequality constraints to the SAT formula, requiring any pair of states to be distinct. Experimental evidence, however, suggests that it is beneficial to only add disequality constraints for pairs of states for which it is observed that disequality constraints are needed \cite{EenSorensson:2003}. 

\paragraph{-induction for STTS.}
Both base case and inductive step formulas can be applied to an STTS  simply by replacing ,  and  in these formulas by ,  and  and using an SMT solver instead of a SAT solver.
However, unlike for untimed systems,
termination is not even guaranteed when adding disequality constraints.
For untimed systems,
disequality constraints guarantee termination due to the fact that
in a finite state system, there are no simple paths of infinite length and, thus, the simple path inductive step check is guaranteed to be unsatisfiable with sufficiently large~.
Timed systems, in contrast, typically have no upper bound for the length of a simple path and, thus, disequality constraints are not sufficient for completeness.
However, the infinite state space of an STTS can be split into a finite number of regions.
Thus, any reasoning made for finite state systems can be applied to regions of states.
In particular, -induction is complete and correct
when only paths that do not visit two states belonging to the same region are considered in the inductive step \cite{deMouraEtAl:CAV2003}.
By enforcing this property on inductive step paths using region-disequality
constraints,
complete -induction can be performed using ,  and  (almost) without modification.



In order to specify that two states of an STTS belong to different regions,
region-disequality constraints need to individually constrain the integer and
fractional parts of clock values.
As only some SMT-solvers, such as Yices \cite{DBLP:conf/cav/DutertreM06},
allow referring to integer and fractional parts of real-valued variables,
we provide a region-disequality constraint encoding that does not rely on
such a feature.\footnote{In \cite{KindermannJunttilaNiemela:submitted} we give an alternative encoding for region-disequality constraints in a BMC setting.}
Instead, we split each clock variable  into two variables:
 represents the integer
and
 the fractional part of 's value.
This ``splitting of clocks'' requires rewriting of
,  and  by
replacing each atom involving a clock with
a formula as follows:


\begin{center}
 \footnotesize
   \begin{tabularx}{\textwidth}{| l | X |c| l | X |}
\cline{1-2} \cline{4-5}
    Atom & Replacement,  &&
    Atom & Replacement, 
    \\
    \cline{1-2} \cline{4-5}
    \cline{1-2} \cline{4-5}
     & 
    &&
     & 
    \\
    \cline{1-2} \cline{4-5}
     & 
    &&
     & 
    \\
    \cline{1-2} \cline{4-5}
     & 
    &&
     & 
    \\
    \cline{1-2} \cline{4-5}
     &  
    &
    \multicolumn{3}{ c }{}
    \\ 
    \hline
     & \multicolumn{4}{ p{9.5cm} |}{} \\
    \hline
  \end{tabularx}
\end{center}


\iffalse
\begin{center}
 \footnotesize
   \begin{tabularx}{\textwidth}{| l | X || l | X |}
\hline
    Atom & Replacement,  &
    Atom & Replacement
    \\
    \hline
    \hline
     & 
    &
     & 
    \\
    \hline
     & 
    &
     & 
    \\
    \hline
     & 
    &
     & 
    \\
    \hline
     &  
    &
    \multicolumn{2}{ c |}{}
    \\ 
    \hline
     & \multicolumn{3}{ p{9.5cm} |}{} \\
    \hline
  \end{tabularx}
\end{center}
\fi

\iffalse
\begin{center}
 \footnotesize
  \begin{tabularx}{\textwidth}{| l | X | r|}
    \hline
    Atom & Replacement, 
    \\
    \hline
    \hline
     &  \\
    \hline
     &  \\
    \hline
     &  \\
    \hline
     &  \\
    \hline
     &  \\
    \hline
     &  \\
    \hline
     &  \\
    \hline
     & \\
    \hline
  \end{tabularx}
\end{center}
\fi
\newcommand{\DiffRegions}[2]{\mathit{DiffRegion}^{[#1,#2]}}
\newcommand{\DiffRegionsPrec}[2]{\mathit{DiffRegion}_{\precsim}^{[#1,#2]}}
Then,
two states with indices  and  can be forced to
be in different regions by the following
region-disequality constraint :
\begin{small}

\end{small}\iffalse

\fi where the shorthand

detects whether the clock  exceeds its maximum relevant value .


















\iffalse K-induction \tocite\ inductively proves a reachability property for a system or discovers a counter-example wile trying to prove the property. In the following, we will extend K-induction, which was originally proposed as a verification method for finite-state systems, to a complete verification method for STTS.

As base case of an inductive prove, K-induction shows that no bad state can be reached within  steps starting from an initial state for some . As inductive step, K-induction shows that it is impossible under the transition relation of the system to have a path consisting for  good (property-satisfying) states followed by a bad (property-violating) state. Together, base case and inductive step prove that the property holds in any reachable state.

For an untimed system , both base case and inductive step can be proven using a SAT solver. The base case holds iff the formula

is unsatisfiable. Likewise, the inductive step holds iff the formula

is unsatisfiable.
Initially, K-induction attempts an inductive proof with . If unsuccessful,  is increased until the inductive proof succeeds or a counter-example is found while checking the base case.
Note that the large overlap both between the formulas for checking base case and inductive step and between the checks before and after increasing  can be exploited by incremental SAT solvers \tocite.
\todo{Delete this:?} Furthermore, the overlap can be increased by adding the conjunct  to the base case, which does not affect the correctness of the method as at any given time reaching a bad state in less than  steps has been ruled out by previous base case checks.


While correct, the described approach is not complete due to the fact that the induction step is not guaranteed to hold even if the property checked is satisfied by the system \tocite. K-induction can, however, be made complete for finite-state systems by only considering simple (non-looping) paths when checking the inductive step \tocite.
The most straightforward way to enforce paths to be simple, is to add a quadratic number disequality constraints to the SAT formula, requiring any pair of states to be distinct. Experimental evidence, however, suggests that it is beneficial to only add disequality constraints for pairs of states for which they are really needed \tocite. That is, a constraint requiring two states two be distinct is only added once these two states are actually observed to be equal in a model returned by the SAT-solver as a counter-example to the inductive step \tocite.
Again, disequality constraints can be kept for the base case check to increase the overlap between successive checks \tocite .

\paragraph{K-induction for STTS.} Both base case and inductive step formulas can be applied to \aExpsttsTerm\  simply by replacing ,  and  with ,  and  and using an SMT solver instead of a SAT solver. Like for untimed systems, checking base case and inductive step formula without using disequality constraints does not guarantee termination. Unlike for untimed systems, however, adding disequality constraints does not yield guaranteed termination either.

For untimed systems, disequality constraints guarantee termination due to the fact that in a finite state system, there are no simple paths of infinite length. As a result that the simple path inductive step check is guaranteed to be unsatisfiable with sufficiently large  \tocite. Timed systems in contrast typically have an infinite number of reachable states and thus no upper bound for the length of a simple path. Therefore, even adding disequality constraints does not yield completeness.
While the size of the state space of \aExpsttsTerm\ is typically infinite, the state space of the corresponding region automaton is always finite. Thus, any reasoning made for finite state systems can be applied to the region automaton. In particular, K-induction is complete and correct for region automata when only simple paths are considered both in base case and inductive step.
K-induction can be performed on the region automaton by simply considering any valuation of state variables and clocks a representative for the corresponding state in the region automaton. This way, ,  and  can be used (almost) without modification in base case and inductive step checks. Disequality constraints in this approach require any pair of states to correspond to different states of the region automaton.

\begin{table}[tp]
  \caption{Subformula replacements applied to any matching subformula of any relation of \aExpsttsTerm\ where  and  are variables,  is the clock difference variable and  an arbitrary integer constant TODO: add indices to all variables (?)}
  \label{t:int_frac_replacements}
  \begin{center}
  \begin{tabularx}{\textwidth}{| l | X | r|}
  	\hline
  	Formula & Replacement \\
  	\hline
  	\hline
  	 &  \\
  	\hline
  	 &  \\
  	\hline
  	 &  \\
  	\hline
  	 &  \\
  	\hline
  	 &  \\
  	\hline
  	 &  \\
  	\hline
  	 &  \\
  	\hline
  	 & \\
  	\hline
  	any other formula & Recursively replace subformulas \\
  	\hline
  \end{tabularx}
  \end{center}
\end{table}

In order to specify that two states of an STTS belong to different region automata states, disequality constraints need to individually constrain the integer and fractional parts of clock values. While some SMT-solvers like Yices \tocite\ allow to individually refer to integer and fractional part of any real-valued variable, we will provide a region-disequality constraint encoding that does not rely on the availability of such non-standard features. Instead, we explicitly use two variables per clock variable. For any clock , we represent by  the integer part of 's value in the th state and by  its fractional part. Splitting clock values into two parts requires minor modifications made to ,  and . The modifications are listed in Table~\ref{t:int_frac_replacements}, which covers any type of formula in which clock variables are used in \aExpsttsTerm. \todo{This potentially would be a place to refer to the BMC paper as we use the same technique there and also explain the non-standard way to split real values into integer and fractional parts that is supported by Yices}
Then, states with indices  and  can be forced to correspond two different states in the region automaton using the following region-disequality constraint:

 \todo{bring into same format used in IC3 section}
 For simplicity, we use the shorthand  do denote situations in which the clock  exceeds its maximum relevant value, i.e.
.
The first big disjunction is satisfied the state variable valuation of the two states is satisfied while the following big disjunctions each correspond to one condition in the region definition (cf. Section~\ref{s:regions}) being violated.
\fi
