\documentclass{article}


\usepackage{amsmath, amsthm, amssymb}
\usepackage[dvipdfm]{graphicx}
\usepackage{comment}

\theoremstyle{plain}
\newtheorem{lemma}{Lemma}
\newtheorem{proposition}{Proposition}
\newtheorem{theorem}{Theorem}
\newtheorem{corollary}{Corollary}

\theoremstyle{remark}
\newtheorem{example}{Example}


\newcommand{\LHC}{\ensuremath{\mathcal{L\hspace*{-0.5mm}H}}}
\newcommand{\RHC}{\ensuremath{\mathcal{R\hspace*{-0.5mm}H}}}
\newcommand{\HC}{\ensuremath{\mathcal{H}}}

\newcommand{\NL}{{\tt NL}}
\newcommand{\NSPACE}{{\tt NSPACE}}

\newcommand{\CSL}{\ensuremath{\mathrm{CSL}}}
\newcommand{\REG}{\ensuremath{\mathrm{REG}}}

\newcommand{\HCS}{\ensuremath{\mathrm{HCS}}}

\newcommand{\aPref}{\ensuremath{\mathrm{Pref}_\alpha}}
\newcommand{\Pref}{\ensuremath{\mathrm{Pref}}}
\newcommand{\Suff}{\ensuremath{\mathrm{Suff}}}

\newcommand{\ind}{\ensuremath{\mathrm{ind}}}
\newcommand{\maxind}{\ensuremath{\mathrm{maxind}}}

\newcommand{\calpha}{\ensuremath{\overline{\alpha}}}



\title{On the regularity of iterated hairpin completion of a single word}
\author{Lila Kari, Steffen Kopecki, and Shinnosuke Seki}

\begin{document}

\maketitle

\begin{abstract}
	Hairpin completion is an abstract operation modeling a DNA bio-operation which receives as input a DNA strand , and outputs , where  denotes the Watson-Crick complement of . 
	In this paper, we focus on the problem of finding conditions under which the iterated hairpin completion of a given word is regular. 
	According to the numbers of words  and  that initiate hairpin completion and how they are scattered, we classify the set of all words . 
	For some basic classes of words  containing small numbers of occurrences of  and , we prove that the iterated hairpin completion of  is regular. 
	For other classes with higher numbers of occurrences of  and , we prove a necessary and sufficient condition for the iterated hairpin completion of a word in these classes to be regular. 
\end{abstract}

\section{Introduction}

\begin{figure}
\begin{center}
\includegraphics{selfPCR.eps}
\caption{
	Hairpin completion by polymerase chain reaction \cite{HAKSY00, SKKGYISH99}. 
	The operation input is , the output is , and the primer is . 
}
\label{fig:selfPCR}
\end{center}
\end{figure}

A DNA strand can be abstractly viewed as a word over the alphabet , where in {\tt A} is Watson-Crick complementary to {\tt T} and {\tt C} to {\tt G}, and two complementary DNA single strands of opposite orientation bind together to form a double DNA strand (intermolecular structure). 
Also, if subwords of a DNA strand are complementary, the strand may bind to itself forming intramolecular structures such as stem-loops, also known more commonly as {\it hairpins} (Figure~\ref{fig:selfPCR} (2)). 
Hairpins can be a building block of a larger-scale structure of RNA strands, and play a role in determining various chemical and thermodynamical properties (stability, structures, functions) of the structure, and make significant contributions to the genetic information processing as illustrated in their function as a stopper for messenger RNA (mRNA) transcription. 
A {\tt CG}-rich sequence of an mRNA folds into its Watson-Crick complement on the RNA and forms a stable hairpin. 
Transcription of the mRNA is terminated when RNA polymerase reaches the hairpin. 
At that time, {\it nusA} protein bound to the polymerase interacts with the hairpin and takes the polymerase off the mRNA. 
This hairpin-driven mechanism is called {\it intrinsic termination} \cite{WilsonHippel95}. 
As such, hairpins tend to interfere with reactions, and therefore were given the cold shoulder by DNA computing experimentalists. 
See \cite{Adleman94, AritaKobayashi02, JoKeMa02, JonoskaMahalingam04, KaKoLoSoTh06, PaRoYo01} about this problem and about some of the ``good'' designs of DNA strands that are free of hairpins. 

Hairpin is not a foe to all DNA computing experiments; many molecular computing machineries have been proposed which make good use of hairpins. 
Such hairpin-driven systems include DNA RAM \cite{KaYaOhYaHa08, TakinoueSuyama04, TakinoueSuyama06} and Whiplash PCR \cite{HAKSY00, SKKGYISH99}. 
In particular, Whiplash PCR features a self-directed polymerase chain reaction (PCR) of DNA strand, which practically motivates the investigation of a formal language operation called {\it hairpin completion}. 
Hairpin completion proceeds as follows (Figure~\ref{fig:selfPCR}): 
Starting from a DNA strand , a segment  at the 3'-end of  binds to its Watson-Crick complementary strand  on the strand (annealing). 
A polymerase chain reaction then extends  at its 3'-end in the  direction so as to generate the strand  (let us call  and  that bind with each other to initiate this PCR reaction {\it primers}). 
Despite the intrinsic  polarity of polymerases, a mechanism exists to make polymerase reaction work in the  direction (Okazaki fragment \cite{OkOkSaSuSu68}). 

As an abstract model of the above-mentioned self-directed PCR, Cheptea, Mart\'{i}n-Vide, and Mitrana proposed the hairpin completion in \cite{ChMaMi06}, and since then this abstract operation has been studied on its algorithmic and formal linguistic aspects \cite{DiekertKopecki11, MaViMi09, MaMiYo09, MaMiYo10} together with its variant called {\it bounded hairpin completion} \cite{ItLeMaMi11, Kopecki10}, where the length of extension in one operation is bounded by a constant. 
Ito et al.~\cite{ItLeMaMi11} and Kopecki~\cite{Kopecki10} proved that all classes in the Chomsky Hierarchy are closed under iterated bounded hairpin completion. 
In contrast, the class of regular languages was proved not to be closed under iterated (unbounded) hairpin completion \cite{ChMaMi06}, and a surprising fact is that iterated hairpin completion of {\it a} word can be non-regular \cite{Kopecki10}. 
In this paper, we focus on a problem proposed by Kopecki in \cite{Kopecki10}; is it decidable whether the iterated hairpin completion of a given word is regular? 
The iterated hairpin completion of a singleton language (a word) is known to be in {\tt NL} \cite{ChMaMi06}, but can be non-regular as shown in the following example. 

\begin{example}\label{ex:nonx_cs}
	Let  and , where  are all distinct letters. 
	Then the intersection of the iterated hairpin completion of  with  is . 
	This intersection is not context-free, and neither is the iterated hairpin completion. 
\end{example}

In this paper, we give a partial answer to the regularity-test decidability problem. 
We focus our attention on the number of primers a given word contains as its factors and on how these primers are scattered over the given word. 
All the words are classified in accordance with these two criteria, and for some basic classes, we give a necessary and sufficient condition for the iterated hairpin completion of a word in the class to be regular. 


\section{Preliminaries}

Let  be an alphabet,  be the set of all words over , and for an integer ,  be the set of all words of length  over . 
The word of length 0 is called the empty word, denoted by , and let . 
A subset of  is called a language over . 
For a word , we employ the notation  when we mean the word as well as the singleton language  unless confusion arises. 
For a language , we denote by  the set . 

We equip  with a function  satisfying ; such a function is called an {\it involution}. 
This involution  is naturally extended to words as: for , . 
For example, over the 4-letter alphabet , if we define an involution  as  and , then , being thus extended, maps the Watson strand of a complete DNA double strand into its Crick strand. 
The involution  is called the Watson-Crick involution \cite{KariMahalingam08}. 
For a word , we call  the {\it complement of }, being inspired by this application. 
A word  is called a {\it pseudo-palindrome} if . 
For a language , . 

For words , if  holds for some words , then  is called a {\it factor} of ; a factor that is distinct from  is said to be {\it proper}. 
If the equation holds with  (), then the factor  is especially called a {\it prefix} (resp.~a {\it suffix}) of . 
The prefix relation can be regarded as a partial order  over ;  means that  is a prefix of . 
Analogously, by  we mean that  is a suffix of . 
For a word  and a language , a factor  of  is {\it minimal with respect to } if  and none of the proper factors of  is in . 

A nonempty word  is {\it primitive} if  implies  for any nonempty word . 
It is well-known that for any nonempty word , there exists a unique primitive word  with . 
Such  is called the {\it primitive root of } and denoted by . 
Two words  {\it commute} if , and this is known to be equivalent to . 
See \cite{ChKa97} for details of primitivity and commutativity of words and related results. 

Now we introduce the operation investigated in this paper, that is, hairpin completion, and define it formally. 
Imagine that we have a DNA sequence . 
The suffix  can find its -image as a factor  on this sequence. 
Hence, this DNA sequence may bend over into a hairpin form by  binding with . 
This formation of hairpin structure leaves  as a free sticky-end, and DNA polymerase converts it into the complete double strand by extending its 3'-end by . 
This exemplifies the mechanism of hairpin completion. 
We call two words whose thus binding initiate hairpin completion {\it primers}. 
In the above example,  and  are primers. 

Let  be a constant that is assumed to be the length of a primer. 
Throughout this paper, we will not use the notation `' for any other purpose. 
Let  be a primer. 
If a given word  has a factorization  for some  and , then its {\it right hairpin completion with respect to } results in the word . 
As long as  is clear from context, this operation is simply called {\it (single-primer) right hairpin completion}. 
By , or by , we mean that  can be obtained from  by right hairpin completion (with respect to ). 
The {\it left hairpin completion} is defined analogously as an operation to derive  from , and the relation  is naturally introduced. 
By  and , we denote the reflexive transitive closure of  and that of , respectively. 
The relation  is defined as the union of  and . 

For a given language , we define the set of words obtained by left hairpin completion from , and the set of words obtained by iterated left hairpin completion from , respectively, as follows: 

Analogously,  and  are defined based on  and , and  and  are defined based on  and 

\begin{proposition}\label{prop:presentation_RH}
	For a word , . 
\end{proposition}


\section{Word structures relevant to the power of iterated hairpin completion}
	\label{sec:apref_casuf}

In this section, we describe several structural properties of a word  that will be relevant for the characterization of its iterated hairpin completion , where  is a fixed parameter. 

A word  is called an {\it -prefix} of a word  if  for some word . 
In a similar manner, a word  is an {\it -suffix} of  if  for some . 
If  begins with , then this prefix can bind with the occurrence of  (unless they overlap with each other), and left hairpin completion results in . 
By  and , we denote the set of all -prefixes and that of all -suffixes of , respectively. 
One can easily observe that . 
Throughout this paper, we let  and  for some . 
It will be convenient to assume that these -prefixes are sorted in the ascending order of their length. 
Likewise, we assume that . 

Our investigation on the properties of -prefix and -suffix of word begins with a basic observation. 

\begin{proposition}\label{prop:aprefix_basic}
	For a word , the following statements hold: 
	\begin{enumerate}
	\item	for any , ; 
	\item	for any , ; 
\end{enumerate}
\end{proposition}
\begin{proof}
	The first statement derives directly from the definition of -prefix. 
	For the second one, induction on  works. 
	Due to the first statement,  so that proving  is reduced to proving . 
\end{proof}

From this proposition, we can easily deduce that for a word  and , , which means . 
This deepens the above observation further as follows. 

\begin{corollary}\label{cor:apref_casuff_transitive}
	For a word , any word in  has  as its prefix. 
\end{corollary}

Due to the second statement of Proposition~\ref{prop:aprefix_basic},  holds for -prefixes . 
Hence, from a word , one-step right hairpin completion can produce at least the words .\footnote{.} 
Now, if we know that one-step hairpin completion extends the word to the right by , what can we say about the word ? 
Firstly, as long as , we can say that  by definition of hairpin completion. 
Moreover, Corollary~\ref{cor:apref_casuff_transitive} enables us to find  such that . 
Then, one can let  for some prefix  of . 
Since ,  is an -prefix of  that is properly shorter than . 
By defining  to be the index satisfying , we have ; recall that elements of  is sorted with respect to their length. 
The above argument is summarized by the next lemma. 

\begin{lemma}\label{lem:aprefix_prefix}
	Let . 
	If a word  satisfies , then there exists an integer  such that  for some . 
\end{lemma}

A more natural setting is to assume that each of  is either an element of  {\it or} an element of  because, by left hairpin completion, the complement of a -suffix of  can be produced to the left of . 
We need to generalize the function  by extending its domain as follows: for ,  if . 
Note that this generalized  is not a function any more in cases when , but this will not cause any problem in this paper. 

\begin{lemma}\label{lem:apref_casuf_apref}
	Let . 
	If a word  satisfies , then there exists an integer  such that , where 
	 
\end{lemma}
\begin{proof}
	As done previously, we can find  and a nonempty word  satisfying  and . 
	Since this prefix relation can be rewritten as , if  is an -suffix of , so is . 
	The case when  is clear from the previous argument. 
\end{proof}

Having considered prefix relations among -prefixes and -suffixes of a word, now we proceed our study to more general factor relationships among them. 

\begin{lemma}\label{lem:suf_relation_aprefs}
	If  for some integers , then . 
\end{lemma}
\begin{proof}
	We can let  for some . 
	Combining this with Proposition~\ref{prop:aprefix_basic}, we have  so that . 
	Since ,  is in . 
\end{proof}

\begin{lemma}\label{lem:factor_relation_shortests}
	If  is a factor of , then . 
\end{lemma}
\begin{proof}
	Let  for some . 
	Unless ,  would be a nonempty -prefix of  that is properly shorter than , and causes a contradiction. 
	Thus,  must be empty so that . 
	Now, Lemma~\ref{lem:suf_relation_aprefs} leads us to . 
\end{proof}

Finally, let us introduce interesting results that illustrate the close relationship between -prefixes, commutativity, and primitivity, essential notions in combinatorics on words. 

\begin{lemma}\label{lem:primitive_apref}
	Let  and . 
	Then . 
\end{lemma}
\begin{proof}
	Due to the first statement of Proposition~\ref{prop:aprefix_basic},  enables us to let  for some . 
	Its solution is well-known to be  and  for some  and  such that . 
	Hence, . 
\end{proof}

An immediate implication of this lemma is that the shortest nonempty -prefix of a word that begins with  must be primitive. 
We should make one more step forward. 
Imagine that a word  has an -prefix . 
If  is possible, then  is also possible. 
Thus, repeating the extension of  to the right by   times amounts to extending  by  once. 
In other words, the process to extend a word by  is not essential unless  is primitive because it can be always simulated by multiple processes to extend a word by . 

The next lemma proves that all nonempty -prefixes of length at most  commute with each other, and hence, only the shortest one is essential in the above sense. 

\begin{lemma}
	For nonempty words , if  and  hold, then . 
\end{lemma}
\begin{proof}
	If , then the prefix relation immediately gives , and the conclusion of this lemma is trivial. 
	Hence, we assume . 
	Combining  with , we can deduce that the word  has a period . 
	Likewise,  has a period , and hence,  also has this period. 
	As a result,  has two periods , and moreover it is of length at least the sum of these periods. 
	Thus, Fine and Wilf's theorem \cite{ChKa97, FiWi65} leads us to the conclusion of this lemma. 
\end{proof}

\subsection{Non-crossing words and their properties}



A word  is an {\it --word}, or simply an {\it -word} when  is clear from the context, if  and . 
Informally speaking, an -word is a word on which  occurs  times and  does  times. 
For a pseudo-palindromic  (), we regard an occurrence of  also as that of , and as such, any word is an -word for some . 

We say that  is {\it non--crossing} if the rightmost occurrence of  precedes the leftmost one of  on . 
When  is understood from the context, we simply say that  is non-crossing. 
Otherwise, the word is {\it -crossing} or {\it crossing}. 
Note that if , then for a word  which is either a -word or -word, , and otherwise ( is an -word for some ),  can be considered crossing. 
Thus, whenever the non--crossing word is concerned, we assume that . 
The definition of a word being non--crossing does not force the word to begin with  or end with . 
However, it is not until  is a primer that this notion becomes useful in our work. 
Thus, the word should be in either  or . 
Actually, in the rest of this paper, we assume both of these conditions and consider only {\it single-primer iterated hairpin completion}; thus, we can assume that . 
As let previously, elements of  are denoted by , those of  by , and they are sorted so that this assumption imposes . 

Our main focus lies on the characterization of non-crossing words whose iterated hairpin completion is regular in terms of combinatorics on words. 
Thus, in this subsection, we prove some combinatorial properties of non-crossing words. 
Let us begin with an easy observation about the longest -prefix and -suffix of . 

\begin{proposition}\label{prop:mirror}
	 if and only if  and for all , . 
\end{proposition}

Next, we will see that one-step hairpin completion can extend  to the left by any of  or to the right by any of  due to the following lemma. 

\begin{lemma}\label{lem:length_nonoverlap_apref_casuf}
	Let  be a non-crossing word with  and . 
	Then . 
\end{lemma}
\begin{proof}
	Suppose that this inequality did not hold. 
	Being non-crossing,  can be written as  for some  with . 
	Hence, . 
	Let  be a nonempty word satisfying . 
	Since  is non-crossing,  must hold, from which we have . 
	Combining this with  enables us to find an -suffix  of , but this would be longer than the longest -suffix of , a contradiction. 
\end{proof}

This lemma does not rule out the possibility that  cannot be extended to the right by  by hairpin completion because the rightmost occurrence of  might overlap with the suffix . 
The analogous argument is valid for  and left hairpin completion. 
However, Lemma~\ref{lem:length_nonoverlap_apref_casuf} leads us to one important corollary on non-crossing -words for  that hairpin completion can extend  to the right by the complement of any of its -prefix and to the left by the complement of any of its -suffix. 

\begin{corollary}\label{cor:1st_step_mn_ge2}
	Let  be a non-crossing -word with . 
	Then . 
\end{corollary}

Any word obtained from a non-crossing word by hairpin completion is non-crossing. 
Though being easily confirmed, this closure property forms the foundation of our discussions in this paper. 

\begin{proposition}\label{prop:nonx_HC}
	Let  with , and  be a non-crossing word. 
	Then any word in  is non-crossing. 
\end{proposition}

We conclude this section with a characterization of a non--crossing word in terms of minimal factors with respect to the language . 
With Proposition~\ref{prop:nonx_HC}, this characterization will bring a unique factorization theorem (Theorem~\ref{thm:nonx_initial_once}) of any word  in  as  for some words . 

\begin{lemma}
	Let  with . 
	A word  is non-crossing if and only if it contains exactly one minimal factor  from .
\end{lemma}
\begin{proof}
	Let us consider the contrapositive of the converse implication. 
	So, if  is crossing, then we can find an occurrence of  (let us denote it by ) which precedes an occurrence of  (). 
	 is guaranteed to be preceded by another occurrence of  () because  begins with . 
	Thus, the factor of  that spans from  to  is a minimal factor from . 
	By the same token, the factor of  that spans from  to its right adjacent occurrence of  becomes another minimal factor. 

	In order to prove the direct implication, suppose that  contains two minimal factors from . 
	These two factors must overlap with each other because otherwise the suffix  of the first factor precedes the prefix  of the second one and  would be crossing. 
	However, if they overlap, then the overlapped part would be in , and this contradicts the minimality of the two factors. 
\end{proof}

\begin{theorem}\label{thm:nonx_initial_once}
	Let  with , and  be a non-crossing word. 
	On any word in ,  occurs exactly once as a factor. 
\end{theorem}
\begin{proof}
	From the two facts that any word in  is non-crossing (Proposition~\ref{prop:nonx_HC}) and that these words contain at least one occurrence of  as a factor by definition of hairpin completion, we can reach this conclusion. 
\end{proof}


\section{Iterated hairpin completion of non-crossing words}

This section contains the main contribution of this paper: characterizations of the regularity of iterated hairpin completion of a non-crossing -word  (recall that  is assumed). 
Throughout this section,  is thus assumed with  and . 

Let us begin with a proof that one-sided hairpin completion of a non-crossing word is regular (Theorem~\ref{thm:nonx_oneside_regular}). 
Then we will show that the iterated hairpin completion of a non-crossing -word for any  or -word is always regular (Theorems~\ref{thm:m1nonx_regular} and \ref{thm:22nonx_regular}). 
Using these results and combinatorial results shown in Section~\ref{sec:apref_casuf}, we characterize the set of all non-crossing -words whose iterated hairpin completion is regular, in terms of commutativity (Theorem~\ref{thm:32nonx_iff_regular}). 

\begin{theorem}\label{thm:nonx_oneside_regular}
	For a non-crossing word , both  and  are regular. 
\end{theorem}
\begin{proof}
	First, we prove the regularity of . 
	Let  be an -prefix of . 
	A right hairpin completion of  can produce . 
	Note that the suffix  of this resulting word does not contain  due to the non-crossing assumption on , and this means that the longest -prefix of  is the same as that of . 
	Thus, the language  can be obtained by iterated bounded hairpin completion from , and hence, is regular \cite{Kopecki10}. 

	For the regularity of , it suffices to observe that  is also non-crossing. 
	Using the result just proved,  is regular, and according to Proposition~\ref{prop:presentation_RH}, . 
	Note that the class of regular languages is closed under . 
\end{proof}

\subsection{Iterated hairpin completion of  non-crossing words}

In this subsection, we consider the case  ( is an -word), and prove that  is regular. 
For , it is easy to see that hairpin completion cannot generate any word but , that is, . 
Hence, we assume . 

Lemma~\ref{lem:length_nonoverlap_apref_casuf} means that right hairpin completion can extend  to the right by any of , 
In contrast, the operation can extend  to the right by  if and only if , i.e., the  to the right of  does not overlap with the suffix  of . 
As a result, if  but this inequality does not hold, then . 
Therefore, we can advance our discussion on the assumption that  is valid. 

Note that  is a non-crossing -word. 
Applying Lemma~\ref{lem:length_nonoverlap_apref_casuf} to this word, we can see that . 
Hence, hairpin completion can extend  further to the right by not only by any of  but also by . 

Let us define the following regular language: 

We claim that this language is the language obtained from  by iterated hairpin completion. 

First, we prove that . 
Let . 
By definition, any word in  can be factorized as . 
Compare the leftmost factor  and the complement of the rightmost factor  with respect to their index. 
Assume that . 
Then . 
Hence, one-step left hairpin completion can derive  from the word . 
In the case when , the same argument implies that . 
Due to , the repetition of this process eventually reduces  into a word  for some . 
Because of the condition on  and our discussion above,  is valid. 
Thus, . 

Secondly, we prove the opposite inclusion by induction on the length of derivation by hairpin completion. 
Clearly . 
Let us assume that a word in  can be written as  with . 
Let . 
If left hairpin completion extends this word to the left by , then  and this means  (see Lemma~\ref{lem:aprefix_prefix}). 
Thus, there exist  such that  and . 
It it trivial that this inequality remains valid in the right hairpin completion.  

\begin{comment}
On , if the -th occurrence of  overlaps with the suffix  (primer), then the primer cannot combine with the occurrence of , and hence, hairpin completion cannot extend  to the right by . 
Such  is said to be {\it initially-inert}. 
Let  be the largest index such that  is initially-active. 
Then . 
For , we define the value ; this specifies how long  has to be extended to the right so that its suffix  can combine with the -th occurrence of  on  to add  by right hairpin completion. 
Then we can describe the set of all ``ways'' to extend  to the right by hairpin completion as follows: 


Then

This is a regular language, and we will prove that this is actually the hairpin completion of . 


\begin{theorem}\label{thm:m1_nonx_representation}
	For any  and a non-crossing  word , . 
\end{theorem}
\begin{proof}
	Let us prove  first. 
	By definition, any word  in  can be factorized as . 
	Compare the leftmost factor  and the complement of the rightmost factor  with respect to their index. 
	Assume that . 
	Then . 
	Hence, one-step left hairpin completion can derive  from the word . 
	In the case when , the same argument implies that . 
	Due to , the repetition of this process eventually reduces  into a word  for some . 
	Since , iterated right hairpin completion can produce  from . 
	Thus, . 

	Next we prove the converse inclusion on the length of derivation by hairpin completion. 
	Clearly . 
	Let us assume that a word in  can be written as  with . 
	If left hairpin completion extends this word to the left by , then  and this means  (see Lemma~\ref{lem:aprefix_prefix}). 
	Thus, there exist  such that  and . 
	It it trivial that this inequality remains valid in the right hairpin completion. 
\end{proof}
\end{comment}

\begin{theorem}\label{thm:m1nonx_regular}
	For any  and a non-crossing  word , the language  is regular. 
\end{theorem}

The key idea in the above discussion is that if a word in  begins with the longest -prefix  of , then hairpin completion can extend it to the right by any of -prefix of . 
This idea has a broader range of applications. 
Let  be a non-crossing -word for some  with  and . 
Proposition~\ref{prop:mirror} says that if , then . 
For , the rightmost occurrence of  on  does not overlap with the suffix  of  (Lemma~\ref{lem:length_nonoverlap_apref_casuf}). 
Thus, . 

\begin{corollary}\label{cor:mn_palindrome-like_regular}
	Let  be a non-crossing -word. 
	If , then  is regular. 
\end{corollary}

\subsection{Iterated hairpin completion of  non-crossing words}

In contrast to the result obtained in the previous subsection, Example~\ref{ex:nonx_cs} shows that there exists an  non-crossing word whose iterated hairpin completion is non-regular with . 
This result motivates the study of  non-crossing words reported here. 
Let  be a non-crossing -word. 
We can employ Corollary~\ref{cor:1st_step_mn_ge2} to see that . 
This further implies that the suffix  of any word in  can bind with the second  on the (unique) factor  on the word for right hairpin completion. 

Let us define the following regular language: 

We will show that this language is exactly the set of words obtained by iterated hairpin completion from . 

In order to prove that , it suffices to present the following process: 


Next, we prove the opposite inclusion by induction on the length of derivation by hairpin completion from . 
Obviously, . 
Assume that all words obtained from  by at most -times hairpin completion are in . 
Let  be such a word and consider a word  such that . 
Consider the case when this hairpin completion is right one. 
The rightmost occurrence of  on  is the second  on its (unique) factor . 
Therefore, if we let  and then . 
Since  and  are the respective shortest nonempty -prefix and -suffix of , Lemma~\ref{lem:apref_casuf_apref} implies that . 
Note that  is closed under catenating a word in  to the right. 
Thus, . 
The case when  can be proved in a symmetric manner. 

Due to the symmetry of  and , we can easily construct a regular language  which is equivalent to . 
Now the regularity of  has been proved. 


\begin{theorem}\label{thm:22nonx_regular}
	For a  non-crossing word , the language  is regular. 
\end{theorem}


\subsection{Iterated hairpin completion of  non-crossing words}

Theorem~\ref{thm:22nonx_regular} and Example~\ref{ex:nonx_cs} motivate our investigation of non-crossing  words. 
Actually, Theorem~\ref{thm:32nonx_iff_regular}, a main contribution of this paper, provides a characterization of the regularity of iterated hairpin completion of a non-crossing -word in terms of the commutativity of the -prefixes and -suffixes of the word. 

Let  be a non-crossing -word (so ) with  and . 
Note that  () must be primitive; otherwise, its primitive root is also an -prefix (resp.~-suffix) of  and  would not be a -word any more. 
As a result,  commute with  () if and only if  (resp.~). 
Recall also that  must hold for  to be -word (Proposition~\ref{prop:mirror}).
Thus, if  and  commute, then  and . 
In other words, the commutativity between  and  is reduced to the commutativity between  and  and the commutativity between  and , and hence, not essential. 

Corollary~\ref{cor:1st_step_mn_ge2} states that . 
Let us ask the question of whether iterated hairpin completion can generate a same word from  and . 
We partially answer this question in a broader setting for arbitrary  and . 

\begin{lemma}\label{lem:mn_nonx_include_or_emptyset}
	Let  be a non-crossing -word for some  and  with . 
	For integers  with , if , then ; otherwise, . 
\end{lemma}
\begin{proof}
	Let  for some .
	Lemma~\ref{lem:length_nonoverlap_apref_casuf} implies that  is possible. 
	Thus, the inclusion holds. 
	Conversely, if the intersection is not empty, then Theorem~\ref{thm:nonx_initial_once} implies that  for some . 
	Then, due to Lemma~\ref{lem:suf_relation_aprefs}, this equation gives ; thus, . 
\end{proof}



We can employ Lemma~\ref{lem:mn_nonx_include_or_emptyset} in our current setting of non-crossing -words to observe that if , then ; otherwise, . 
Thus, for example, if , then . 

In this subsection, we first prove that the commutativity of  with  or with  is a sufficient condition for  to be regular. 

\begin{lemma}\label{lem:22nonx_u2isv2_regular}
	If , then the language  is regular. 
\end{lemma}
\begin{proof}
	Let  for some . 
	Observe that  is a non-crossing -word with  being its nonempty -prefix. 
	Lemma~\ref{lem:length_nonoverlap_apref_casuf} implies that , which means that hairpin completion can extend  to the right by  and result in . 
	If , then hairpin completion can also generate , but it is not essential in the following discussion whether this is possible or not. 
	Let us consider only the case when it is possible. 
	Then , which is regular due to Theorem~\ref{thm:m1nonx_regular}, is . 
	As we have seen above, if , then either  or . 
	In any case, , and hence, is regular. 
\end{proof}

Now it is easy to see that  is regular when  commutes with . 
Since  is -word,  must be primitive and  is equal to either  or . 
In the former case,  is a proper prefix of  so that  has  and would not be a -word. 
Thus, the latter must be the case. 
In this case, the prefix  of , which is the primitive root of , is an -prefix of  (Lemma~\ref{lem:primitive_apref}), and hence, in order for  to be a -word,  must hold, and this brings the conclusion according to Lemma~\ref{lem:22nonx_u2isv2_regular}. 

\begin{lemma}
	If , then the language  is regular. 
\end{lemma}
\begin{proof}
	Lemma~\ref{lem:22nonx_u2isv2_regular} makes it sufficient to consider the case when  does not commute with . 
	Since  (when the reader check this, recall Lemma~\ref{lem:length_nonoverlap_apref_casuf}) and , we will show the regularity of the second and third terms of this equation and that is enough for our purpose. 

	First, we prove that  is regular. 
	Let , where  is a -word with  and . 
	We can easily check that 
	
	As done in the proof of Lemma~\ref{lem:22nonx_u2isv2_regular}, the non-commutativity between  and  implies that . 
	Thus, . 
	Since  is a non-crossing -word so that  is regular (Theorem~\ref{thm:22nonx_regular}), and hence, so is . 

	Next, we prove the regularity of . 
	We can let  for some -word . 
	This means that  is a -word with  and the empty -suffix. 
	Thus, 
	
	Using the essentially same argument as above, we obtain . 
	Since the iterated hairpin completion of non-crossing -word is regular (Theorem~\ref{thm:m1nonx_regular}),  is regular and so is . 

	Combining what have been proved in the previous two paragraphs together, we conclude the regularity of . 
\end{proof}

To summarize the results obtained so far, any of two of the -prefixes and the complements of -suffixes of , i.e., , must not commute in order for  not to be regular. 

\begin{lemma}
	If , then the language  is regular. 
\end{lemma}
\begin{proof}
	Due to Lemma~\ref{lem:22nonx_u2isv2_regular}, it suffices to consider this problem under the assumption , which is equivalent to that  does not commute with  under our problem setting.  

	We have . 
	As done before, we will check that the second, third, and fourth terms of the union above are regular. 
	The regularity of the third one is from  and  and Corollary~\ref{cor:mn_palindrome-like_regular}. 

	In order to check that the second term is regular, let , where  is a -word. 
	Then  is a -word, and 
	
	Since  and , we have . 
	The regularity of  is due to Theorem~\ref{thm:m1nonx_regular} so that  is regular. 

	What remains to be considered is the fourth term. 
	One can let  for some non-crossing -word . 
	Then  holds, and we can easily see that . 
	The regularity of  was proved. 
\end{proof}

\begin{theorem}\label{thm:32nonx_iff_regular}
	Let  be a non-crossing -word with  and . 
	Then  is regular if and only if one of the following three conditions holds: 
	\begin{enumerate}
	\item	 commutes with ; 
	\item	 commutes with ; 
	\item	. 
	\end{enumerate}
\end{theorem}
\begin{proof}
	Let , which is a regular language. 
	Under the assumption that none of the conditions 1-3 holds,  holds.
	As mentioned previously, if the second condition does not hold, which is equivalent to , then  cannot contain any word in the above intersection. 
	Thus, . 
	Using Lemmas~\ref{lem:suf_relation_aprefs} and \ref{lem:factor_relation_shortests}, we can easily prove the emptiness of the second intersection of the above sum. 
	This check is left to the reader, and the authors recommend them to check at least  because this check involves the important fact that  implies  and causes a contradiction. 
	As a result, we have . 
	Informally speaking, in order to produce a word in  from , we first have to extend  to the right by . 

	Now we can extend  to the right by  -times to obtain . 
	If this obtained word is extended to the left, then the word will be in . 
	Let us check that . 
	If the intersection is not empty, then  for some . 
	Due to Lemma~\ref{lem:apref_casuf_apref}, , but actually we can say  for  is the second shortest nonempty -prefix of . 
	However, this means that either the condition 1 or 2 holds, and contradicts our assumption. 
	Thus, we have only one choice; extending  to the right by . 

	As mentioned above,  cannot hold so that we cannot extend  further to the right to obtain a word in . 
	Thus, we should extend this word to the left either by  for some  or by . 
	Lemmas~\ref{lem:suf_relation_aprefs} and \ref{lem:factor_relation_shortests} prove that the former choice will not lead us to any word in . 
	Now it suffices to mention that extending  further to the left because such an extension force the contradictory relation  to hold. 
\end{proof}


\section{Conclusion}

In this paper, we focused on finding conditions that a word  must satisfy so that its iterated hairpin completion  is a regular language. 
We classified the set of all non-crossing words according to the number  of occurrences of  and the number  of occurrences of  on a given word. 
For the cases when  and when , we proved that the iterated hairpin completion of a non-crossing -word is regular. 
We also found a necessary and sufficient condition under which the iterated hairpin completion of a non-crossing -word is regular. 
This approach can be generalized to arbitrary non-crossing -words, with the cases  and  being the induction base of an inductive proof. 
Future works include considering the same problem for crossing-words. 
In this case, Lemma~\ref{lem:length_nonoverlap_apref_casuf} or Theorem~\ref{thm:nonx_initial_once} does not hold any more, and hence, it may get harder to analyze the derivation processes of how a word is obtained from a given word  by iterated hairpin completion. 
In addition, we investigated only the case when the suffix of length  of an initial word  is the complement of its prefix of the same length, but we eventually have to consider  in , where  might not be equal to  (double-primer hairpin completion). 
We can easily observe that one-step hairpin completion with respect to  () derives a word in  (resp.~) from . 
Thus, results obtained in this study of single-primer hairpin completion are important step towards this most general setting of the regularity test problem of iterated hairpin completion of a single word. 
Another direction of research is to consider stopper sequences as in Whiplash PCR \cite{HAKSY00, SKKGYISH99}. 


\bibliographystyle{plain}
	\bibliography{sekibib}


\end{document}
