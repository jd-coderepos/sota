\subsection{A first analysis}

Our first step is to relax~\eqref{program-alg1-frlp} into a linear
program. For that, we adjust the objective function as
in~\cite{JainMMSV03}, and we approximate the inequalities with square
roots using inequalities given by Lemma~\ref{corollary-alg1-inequality}. 
For simplicity, here we will use only the inequalities corresponding
to . With this, we will prove that  is not greater than \MAGIC{}.  Later, we
will improve the obtained result by using a whole set of inequalities
from Lemma~\ref{corollary-alg1-inequality}, and using a more standard
factor-revealing analysis for the \smflp. The relaxed lower
 factor-revealing linear program is:


As \eqref{program-alg1-reveal-tres} is a relaxation of
\eqref{program-alg1-frlp}, we have that  and thus
an upper bound on  is also an upper bound on
. Solving linear
program~\eqref{program-alg1-reveal-tres} using CPLEX for \MAGIC{}, we
obtain the next lemma.

\begin{lemma}\label{lemma-alg1-lower-tres}
.
\end{lemma}

To obtain an upper bound on their factor-revealing linear program,
Jain~\etal~\cite{JainMMSV03} presented a general dual solution of a
relaxed version of the lower bound factor-revealing linear
program. This solution is deduced from computational experiments and
empirical results for small values of~. In their analysis, they
guessed 2- and 3-step functions for a set of dual variables, and used
a long verification to show that the value of such solution was not
greater than . For the squared metric case, if we use step
functions for the dual variables, the bound on the factor would be as
bad as . One can improve the obtained factor to \MAGIC{}
by guessing a piecewise function whose pieces are either constants or
hyperboles.

Instead of looking for a good general dual solution, we use an alternative 
analysis and derive a linear minimization program from~\eqref{program-alg1-reveal-tres}
whose feasible solutions are upper bounds on . 
Afterwards, we give an upper bound on the approximation factor by 
presenting a feasible solution for this program of value less than 
\MAGIC{}.

The idea is to determine a conical combination of the inequalities
of~\eqref{program-alg1-reveal-tres} that imply
inequality~\eqref{inequality-gamma} for a  as small as possible.  The
linear minimization program will help us to choose the coefficients of such
conical combination.

First, rewrite the third inequality of program~\eqref{program-alg1-reveal-tres},
so that the right-hand side is zero. For each  and , we multiply
the corresponding inequality by . Denote by  the sum of all
these inequalities, that is,





The fourth and fifth inequalities of program~\eqref{program-alg1-reveal-tres}
can be relaxed to the set of inequalities , one for each  such that . For each~ and~,
we multiply the corresponding inequality by  and denote by 
the inequality resulting of summing them up, that is,


The coefficients of  in  and  are, respectively,

and the coefficients of  in  and  are, respectively,


Now, we sum inequalities  and  and obtain a new inequality :


We want to find values for , , and  so that
the corresponding coefficients of~ are such that inequality~\eqref{eq-alg1-323-c}
implies, for sufficiently large , that

Moreover, we want  as small as possible.
To obtain inequality~\eqref{eq-alg1-323-objective} from
inequality~\eqref{eq-alg1-323-c}, it is enough that, for each , coefficient
, , and .
Hence, this can be expressed by the following linear program.


The interested reader may observe that program~\eqref{program-alg1-dual-tres}
is the dual of a relaxed version of the lower bound factor-revealing
linear program~\eqref{program-alg1-reveal-tres}.  Therefore, its optimal value
is an upper bound on the optimal value of~\eqref{program-alg1-reveal-tres}, 
that is,  for every .





\newcommand{\PasteLemmaAlgOneThreeTwoThree}{
.
}
\begin{lemma}\label{lemma-alg1-323}
\PasteLemmaAlgOneThreeTwoThree
\end{lemma}
\begin{proof}
  We start by observing that  does not decrease if
  we restrict attention to values of  that are multiples of a fixed
  positive integer~. Indeed, for an arbitrary positive integer~, by
  making  replicas of a solution of~\eqref{program-alg1-reveal-tres}
  for~, and scaling the variables by~, we obtain a solution
  of~\eqref{program-alg1-reveal-tres} for~, that is, we deduce that
  . So we may assume that  has the form  with  and  positive integers, and our goal is to prove that 
  .

  We will use program~\eqref{program-alg1-dual-tres} to obtain a tight
  upper bound on . The size of this program however
  depends on , which can be arbitrarily large. So we will use a
  scaling argument to create another linear minimization program with
  a fixed number (depending only on ) of variables, and obtain a
  feasible solution for program~\eqref{program-alg1-dual-tres} from a
  solution for this smaller program. Then, we will show that the value
  of the generated solution for~\eqref{program-alg1-dual-tres} is
  bounded by the value of the small solution.

  Consider variables ,  for , and  for . For
  an arbitrary , let . We obtain a
  candidate solution for program~\eqref{program-alg1-dual-tres} by taking


Let us calculate each coefficient of  for this solution.







Now, we want to find the minimum value of  and values for  and
 such that the candidate solution for program~\eqref{program-alg1-dual-tres}
is feasible. We may define the following linear program,
named the \emph{upper bound factor-revealing program}.


Consider an optimal solution for program~\eqref{program-alg1-dual-tres}.
Replacing it in~\eqref{eq-alg1-323-c}, that is, in inequality , we obtain
. Thus,
. Now, consider an optimal solution for
program~\eqref{program-alg1-upfrlp-tres} and the corresponding generated
solution for program~\eqref{program-alg1-dual-tres}. We obtain  and conclude that , 
and that holds for every positive integer .

Using CPLEX to solve program \eqref{program-alg1-upfrlp-tres}, we obtained
\MAGIC{}, and this concludes the proof
of Lemma~\ref{lemma-alg1-323}.
\end{proof}
 






