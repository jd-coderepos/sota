\section{Results} \label{sec:Results}
In this section we present the results from our study. In Section \ref{subsec:studydata} we overview the raw data from our study (e.g., how many participants returned for each rehearsal round?) and some simple metrics (e.g., how many of these participants remembered their action-object pairs?)
In Section \ref{subsubsec:SurveyResults} we discuss the results of a survey we sent to participants that did not return for a rehearsal phase.
In Section \ref{sec:CoxRegression} we briefly overview Cox regression --- a tool for performing survival analysis that we used to compare several of our study conditions. In Section \ref{subsec:findings} we use the data from our study to evaluate and compare different study conditions.


\subsection{Study Data} \label{subsec:studydata}
\begin{table*}[htb]
\centering
 \begin{tabular}{|p{2cm}||p{0.58cm}|c|c|c|c|c|c|c|c|c|} 
\hline
                & {\bf Initial} &\multicolumn{9}{c|}{\begin{tabular}[x]{@{}c@{}},\\,\\   confidence interval\end{tabular}   } \\
\hline \begin{tabular}[x]{@{}c@{}}{\bf Rehearsal~}\\{\bf Condition}\end{tabular}
 &     &  &  &  &  &  &  &  &  &  \\
\hline
m\_\HeavyStart  & 80 & \begin{tabular}[x]{@{}c@{}}51\\ \\ 0.838,0.988 \end{tabular}  & \begin{tabular}[x]{@{}c@{}}41\\ \\ 0.914,1 \end{tabular} & \begin{tabular}[x]{@{}c@{}}38\\ \\ 0.907,1 \end{tabular} & \begin{tabular}[x]{@{}c@{}}37\\ \\ 0.858,0.999 \end{tabular} & \begin{tabular}[x]{@{}c@{}}36\\ \\ 0.903,1 \end{tabular} & \begin{tabular}[x]{@{}c@{}}36\\ \\ 0.855,0.999 \end{tabular} & \begin{tabular}[x]{@{}c@{}}34\\ \\ 0.847,0.999 \end{tabular} & \begin{tabular}[x]{@{}c@{}}30\\ \\ 0.735,0.978 \end{tabular} & \begin{tabular}[x]{@{}c@{}}25\\ \\ 0.465,0.85 \end{tabular} \\
\hline
t\_\HeavyStart
  & 100  &\begin{tabular}[x]{@{}c@{}}71\\ \\ 0.790,0.950 \end{tabular}   &  \begin{tabular}[x]{@{}c@{}}54\\ \\ 0.873,0.995 \end{tabular}  & \begin{tabular}[x]{@{}c@{}}51\\ \\ 0.930,1 \end{tabular} & \begin{tabular}[x]{@{}c@{}}51\\ \\ 0.930,1 \end{tabular} & \begin{tabular}[x]{@{}c@{}}51\\ \\ 0.836,0.988 \end{tabular} & \begin{tabular}[x]{@{}c@{}}48\\ \\ 0.857,0.995 \end{tabular} & \begin{tabular}[x]{@{}c@{}}44\\ \\ 0.754,0.962 \end{tabular} & \begin{tabular}[x]{@{}c@{}}39\\ \\ 0.635,0.907 \end{tabular} & 
\begin{tabular}[x]{@{}c@{}}39\\ \\ 0.683,0.961 \end{tabular}   \\
\hline
m\_\Aggressive
 & 75  & \begin{tabular}[x]{@{}c@{}}65\\ \\ 0.648,0.845 \end{tabular}  & \begin{tabular}[x]{@{}c@{}}45\\ \\ 0.817,0.986 \end{tabular} & \begin{tabular}[x]{@{}c@{}}40\\ \\ 0.912,1 \end{tabular}  & \begin{tabular}[x]{@{}c@{}}39\\ \\ 0.865,0.999 \end{tabular}  & \begin{tabular}[x]{@{}c@{}}37\\ \\ 0.858,0.999 \end{tabular}   & \begin{tabular}[x]{@{}c@{}}32\\ \\ 0.838,0.999 \end{tabular}  &
\begin{tabular}[x]{@{}c@{}}28\\ \\ 0.718,0.977 \end{tabular}  & N/A & N/A  \\
\hline
m\_\HeavyStart\_2
  & 81 &  \begin{tabular}[x]{@{}c@{}}50\\ \\ 0.929,1 \end{tabular}  & \begin{tabular}[x]{@{}c@{}}42\\ \\ 0.916,1 \end{tabular}    & \begin{tabular}[x]{@{}c@{}}42\\ \\ 0.916,1 \end{tabular} & \begin{tabular}[x]{@{}c@{}}41\\ \\ 0.914,1 \end{tabular}  &\begin{tabular}[x]{@{}c@{}}38\\ \\ 0.907,1 \end{tabular} & \begin{tabular}[x]{@{}c@{}}37\\ \\ 0.905,1 \end{tabular}   & \begin{tabular}[x]{@{}c@{}}36\\ \\ 0.903,1 \end{tabular}  & \begin{tabular}[x]{@{}c@{}}33\\ \\ 0.894,1 \end{tabular} & \begin{tabular}[x]{@{}c@{}}30\\ \\ 0.735,0.979 \end{tabular} \\
\hline
m\_\HeavyStart\_1
  & 86  & \begin{tabular}[x]{@{}c@{}}64\\ \\ 0.943,1 \end{tabular}     & \begin{tabular}[x]{@{}c@{}}52\\ \\ 0.932,1 \end{tabular}   & \begin{tabular}[x]{@{}c@{}}49\\ \\ 0.927,1 \end{tabular}    & \begin{tabular}[x]{@{}c@{}}49\\ \\ 0.927,1 \end{tabular}  & \begin{tabular}[x]{@{}c@{}}47\\ \\ 0.925,1 \end{tabular}   &  \begin{tabular}[x]{@{}c@{}}46\\ \\ 0.923,1 \end{tabular}    & \begin{tabular}[x]{@{}c@{}}45\\ \\ 0.922,1 \end{tabular}   & \begin{tabular}[x]{@{}c@{}}44\\ \\ 0.920,1 \end{tabular}  & 
\begin{tabular}[x]{@{}c@{}}43\\ \\ 0.918,1 \end{tabular}  \\
\hline
m\_\VeryConservative  & 83  & \begin{tabular}[x]{@{}c@{}}72\\ \\ 0.759,0.931 \end{tabular}   & \begin{tabular}[x]{@{}c@{}}53\\ \\ 0.899,1.000 \end{tabular}  &  \begin{tabular}[x]{@{}c@{}}51\\ \\ 0.930,1 \end{tabular}  & \begin{tabular}[x]{@{}c@{}}51\\ \\ 0.930,1 \end{tabular}  & \begin{tabular}[x]{@{}c@{}}49\\ \\ 0.927,1 \end{tabular} & \begin{tabular}[x]{@{}c@{}}46\\ \\ 0.885,0.999 \end{tabular}  & \begin{tabular}[x]{@{}c@{}}43\\ \\ 0.918,1 \end{tabular}   & \begin{tabular}[x]{@{}c@{}}42\\ \\ 0.874,0.999 \end{tabular} &  \begin{tabular}[x]{@{}c@{}}42\\ \\ 0.827,0.994 \end{tabular}  \\
\hline
m\_\HeavierStart  & 73 & \begin{tabular}[x]{@{}c@{}}40\\ \\ 0.831,0.994 \end{tabular}  &  \begin{tabular}[x]{@{}c@{}}27\\ \\ 0.872,1 \end{tabular}  & \begin{tabular}[x]{@{}c@{}}26\\ \\ 0.868,1 \end{tabular}  & \begin{tabular}[x]{@{}c@{}}24\\ \\ 0.858,1 \end{tabular}  & \begin{tabular}[x]{@{}c@{}}22\\ \\ 0.846,1 \end{tabular}  & \begin{tabular}[x]{@{}c@{}}22\\ \\ 0.846,1 \end{tabular} & \begin{tabular}[x]{@{}c@{}}22\\ \\ 0.846,1 \end{tabular} & \begin{tabular}[x]{@{}c@{}}22\\ \\ 0.846,1 \end{tabular} & \begin{tabular}[x]{@{}c@{}}22\\ \\ 0.846,1 \end{tabular} \\
\hline
\hline \begin{tabular}[x]{@{}c@{}}{\bf Rehearsal~}\\{\bf Condition}\end{tabular}
 &  &     &     &  &  &  & &  &  &  \\
\hline
m\_\HeavierStart & & \begin{tabular}[x]{@{}c@{}}21\\ \\ 0.762,0.999 \end{tabular} & \begin{tabular}[x]{@{}c@{}}20\\ \\ 0.683,0.988 \end{tabular} & \begin{tabular}[x]{@{}c@{}}17\\ \\ 0.681,0.998 \end{tabular} & & & & & & \\
 \hline
m\_\VeryConservative &  & \begin{tabular}[x]{@{}c@{}}36\\ \\ 0.903,1 \end{tabular} & N/A  & N/A &  &  & &  &  &  \\
\hline 
\end{tabular}
\caption{/ with  binomial confidence intervals.  ``mnemonic," ``text"}
\label{tab:remembered}
\end{table*}

One of the primary challenges in analyzing the results from our study is that some participants were dropped from the study because they were unable to return for one of their rehearsals in a timely manner. We do not know how many of  these participants would have been able to remember their stories under ideal circumstances. We consider several different ways to estimate the true survival rate in each condition. Before we present these metrics we must introduce some notation.

{\noindent \bf Notation:} We use  to denote the number of participants from study condition  who remembered their secret action-object pair(s) during rehearsal   with  incorrect guesses per action-object pair, and we use  to denote the number of participants who also remembered their action-object pair(s) during every prior rehearsal. We use  to denote the total number of participants who returned for rehearsal  and we use  to denote the total number of participants who survived through rehearsal  and returned for rehearsal . Finally, we use  to denote the time of rehearsal , as measured from the initial memorization phase. Because the study condition  is often clear from the context we will typically omit it in our presentation.

Observe that  denotes the conditional probability that a participant remembers his PAO stories during rehearsal  given that s/he has survived through rehearsal  and returned for rehearsal . Figures \ref{fig:FacetedConditionalSurvivalProbability} and  \ref{fig:ConditionalSurvivalProbability} plot the conditional probability of survival for participants in different study conditions. Table \ref{tab:remembered} shows how many participants who had never failed before returned in each rehearsal round as well as their conditional probability of success with   confidence intervals.  

\cut{
Our first estimate for the survival rate uses the conditional probability that a participant remembers all of their secret action-object pairs during rehearsal  given that s/he remembered during all previous rehearsal rounds and s/he returned for rehearsal . Our second estimate simply ignores any data from a participant who did not return for rehearsal . Our third estimate is pessimistic. This estimate ignores any data from a successful participant who did not return for rehearsal , but included data from any participant who failed previously.    

 We compare three different metrics to estimate the survival rate of participants in our study under ideal circumstances. 

We first overview the metrics we use to evaluate the performance of participants in different study conditions.

{\noindent \bf Notation:} Given a participant  we use the indicator function  (resp. ) if and only if   returned for rehearsal  (resp. if and only if  remembered his words during rehearsal  with  incorrect guesses per action-object pair. ). We use the function  to indicate whether  remembered his words with  incorrect guesses per action-object pair during rehearsal  and during every earlier rehearsal . We use the  function  to indicate whether  survived rehearsals  to  with no failures and returned for rehearsal . Given an indicator function  and a study condition  (e.g., a set of participants) we use 

to denote the number of participants selected by the indicator function . For example,  denotes the number of participants in condition  who survive through rehearsal . We will usually omit the  and write  when discussing results within a particular condition. Finally, we use  to denote the time of rehearsal , as measured from the initial memorization phase.\\

 Figure \ref{fig:ConditionalSuccessMethod2} plots the probability of success for all participants who returned for rehearsal  (e.g., ) with corresponding values in Table \ref{tab:survived}. }


\begin{table}[t]
\centering
\resizebox{\columnwidth}{!}{\begin{tabular}{|p{1.9cm}||p{0.5cm}|p{1.05cm}|p{1.05cm}|p{1.05cm}|p{1.05cm}|} 
\hline
                & {\bf Initial} &\multicolumn{4}{c|}{\begin{tabular}[x]{@{}c@{}},\\,\\   confidence interval\end{tabular}   } \\
\hline \begin{tabular}[x]{@{}c@{}}{\bf Rehearsal~}\\{\bf Condition}\end{tabular}
 &  i0 &\centering  \cut{ &  &  &  &  &  &\centering}  &\centering   &\centering  & \centering \arraybackslash  \\
\hline
m\_\HeavyStart  & 80 & \cut{\begin{tabular}[x]{@{}c@{}}51\\ \\ 0.838,0.988 \end{tabular}  & \begin{tabular}[x]{@{}c@{}}44\\ \\ 0.813,0.986 \end{tabular} & \begin{tabular}[x]{@{}c@{}}41\\ \\ 0.801,0.985 \end{tabular} & \begin{tabular}[x]{@{}c@{}}40\\ \\ 0.763,0.97.2 \end{tabular} & \begin{tabular}[x]{@{}c@{}}40\\ \\  0.763,0.97.2  \end{tabular} & \begin{tabular}[x]{@{}c@{}}40\\ \\ 0.732,0.958 \end{tabular} &} \begin{tabular}[x]{@{}c@{}}38\\ \\ 0.719,0.956 \end{tabular} & \begin{tabular}[x]{@{}c@{}}34\\ \\ 0.621,0.913 \end{tabular} & \begin{tabular}[x]{@{}c@{}}31\\ \\ 0.360,0.727 \end{tabular} & N/A \\
\hline
t\_\HeavyStart
  & 100  & \cut{\begin{tabular}[x]{@{}c@{}}71\\ \\ 0.790,0.950 \end{tabular}   &  \begin{tabular}[x]{@{}c@{}}61\\ \\ 0.738,0.930 \end{tabular}  & \begin{tabular}[x]{@{}c@{}}60\\ \\ 0.734,0.929 \end{tabular} & \begin{tabular}[x]{@{}c@{}}59\\ \\ 0.750,0.939 \end{tabular} & 
\begin{tabular}[x]{@{}c@{}}59\\ \\ 0.690,0.903 \end{tabular} & \begin{tabular}[x]{@{}c@{}}58\\ \\ 0.666,0.888 \end{tabular} &
} \begin{tabular}[x]{@{}c@{}}56\\ \\ 0.559,0.812 \end{tabular} & \begin{tabular}[x]{@{}c@{}}55\\ \\ 0.423,0.697 \end{tabular} & \begin{tabular}[x]{@{}c@{}}50\\ \\ 0.355,0.645 \end{tabular} & N/A \\
\hline
m\_\Aggressive
 & 75  & \cut{\begin{tabular}[x]{@{}c@{}}65\\ \\ 0.648,0.845 \end{tabular}  & \begin{tabular}[x]{@{}c@{}}57\\ \\ 0.603,0.844 \end{tabular} & \begin{tabular}[x]{@{}c@{}}52\\ \\ 0.631,0.875 \end{tabular}  & \begin{tabular}[x]{@{}c@{}}50\\ \\ 0.618,0.869 \end{tabular}  & \begin{tabular}[x]{@{}c@{}}49\\ \\ 0.589,0.851 \end{tabular}   & 
 \begin{tabular}[x]{@{}c@{}}42\\ \\ 0.580,0.861 \end{tabular}  &}
 \begin{tabular}[x]{@{}c@{}}39\\ \\ 0.472,0.788 \end{tabular}  & N/A & N/A & N/A  \\
\hline
m\_\HeavyStart\_2
  & 81 & \cut{ \begin{tabular}[x]{@{}c@{}}50\\ \\ 0.929,1 \end{tabular}  & \begin{tabular}[x]{@{}c@{}}42\\ \\ 0.916,1 \end{tabular}    & \begin{tabular}[x]{@{}c@{}}42\\ \\ 0.916,1 \end{tabular} & \begin{tabular}[x]{@{}c@{}}41\\ \\ 0.914,1 \end{tabular}  &\begin{tabular}[x]{@{}c@{}}38\\ \\ 0.907,1 \end{tabular} & \begin{tabular}[x]{@{}c@{}}37\\ \\ 0.905,1 \end{tabular}   & }\begin{tabular}[x]{@{}c@{}}36\\ \\ 0.903,1 \end{tabular}  & \begin{tabular}[x]{@{}c@{}}33\\ \\ 0.894,1 \end{tabular} &
\begin{tabular}[x]{@{}c@{}}30\\ \\ 0.735,0.979 \end{tabular} & N/A \\
\hline
m\_\HeavyStart\_1
  & 86  & \cut{ \begin{tabular}[x]{@{}c@{}}64\\ \\ 0.943,1 \end{tabular}     & \begin{tabular}[x]{@{}c@{}}52\\ \\ 0.932,1 \end{tabular}   & \begin{tabular}[x]{@{}c@{}}49\\ \\ 0.927,1 \end{tabular}    & \begin{tabular}[x]{@{}c@{}}49\\ \\ 0.927,1 \end{tabular}  & \begin{tabular}[x]{@{}c@{}}47\\ \\ 0.925,1 \end{tabular}   &  \begin{tabular}[x]{@{}c@{}}46\\ \\ 0.923,1 \end{tabular}    & }\begin{tabular}[x]{@{}c@{}}45\\ \\ 0.922,1 \end{tabular}   & \begin{tabular}[x]{@{}c@{}}44\\ \\ 0.920,1 \end{tabular}  & \begin{tabular}[x]{@{}c@{}}43\\ \\ 0.918,1 \end{tabular} & N/A  \\
\hline
m\_\VeryConservative  & 83  & \cut{\begin{tabular}[x]{@{}c@{}}72\\ \\ 0.759,0.931 \end{tabular}   & \begin{tabular}[x]{@{}c@{}}61\\ \\ 0.738,0.930 \end{tabular}  &  \begin{tabular}[x]{@{}c@{}}60\\ \\ 0.734,0.929 \end{tabular}  & \begin{tabular}[x]{@{}c@{}}60\\ \\ 0.734,0.929\end{tabular}  & \begin{tabular}[x]{@{}c@{}}58\\ \\ 0.72.6,0.927 \end{tabular} & \begin{tabular}[x]{@{}c@{}}55\\ \\ 0.691,0.909 \end{tabular}  &} \begin{tabular}[x]{@{}c@{}}53\\ \\ 0.680,0.906 \end{tabular}   & \begin{tabular}[x]{@{}c@{}}51\\ \\ 0.669,0.902 \end{tabular} & \begin{tabular}[x]{@{}c@{}}49\\ \\ 0.634,0.882 \end{tabular} &\begin{tabular}[x]{@{}c@{}}44\\ \\ 0.673,0.918 \end{tabular} \\
\hline
m\_\HeavierStart  & 73 & \cut{ \begin{tabular}[x]{@{}c@{}}40\\ \\ 0.831,0.994 \end{tabular}  &  \begin{tabular}[x]{@{}c@{}}27\\ \\ 0.872,1 \end{tabular}  & \begin{tabular}[x]{@{}c@{}}26\\ \\ 0.868,1 \end{tabular}  & \begin{tabular}[x]{@{}c@{}}24\\ \\ 0.858,1 \end{tabular}  & \begin{tabular}[x]{@{}c@{}}22\\ \\ 0.846,1 \end{tabular}  & \begin{tabular}[x]{@{}c@{}}22\\ \\ 0.846,1 \end{tabular} &} \begin{tabular}[x]{@{}c@{}}22\\ \\ 0.846,1 \end{tabular} & \begin{tabular}[x]{@{}c@{}}22\\ \\ 0.846,1 \end{tabular} & \begin{tabular}[x]{@{}c@{}}22\\ \\ 0.846,1 \end{tabular} &
\begin{tabular}[x]{@{}c@{}}21\\ \\ 0.762,0.999 \end{tabular} \\
\hline
\hline \begin{tabular}[x]{@{}c@{}}{\bf Rehearsal~}\\{\bf Condition}\end{tabular}
 &  &\centering     &\centering     &   &  \\
\hline
m\_\HeavierStart &   & \begin{tabular}[x]{@{}c@{}}21\\ \\ 0.637,0.970 \end{tabular} & \begin{tabular}[x]{@{}c@{}}17\\ \\ 0.566,0.962 \end{tabular}  & & \\
 \hline
\end{tabular}}
\caption{/ with  binomial confidence intervals.  ``mnemonic," ``text"}
\label{tab:survived}
\end{table}

\begin{figure}[tb]
\centering
\begin{subfigure}[b]{0.45 \textwidth}
\includegraphics[scale=0.4]{Images/faceted_survival_of_those_who_never_failed_previously_1.pdf}
\caption{Faceted. Mean Time Since Memorization.}
\label{fig:FacetedConditionalSurvivalProbability}
\end{subfigure}
\begin{subfigure}[b]{0.45 \textwidth}
\centering
\includegraphics[scale=0.4]{Images/survival_of_those_who_never_failed_previously_1.pdf}

\caption{Together. Mean Time Since Memorization. }
\label{fig:ConditionalSurvivalProbability}
\end{subfigure}
\caption{ vs.  }
\end{figure}

\begin{figure}[tb]
\centering
\includegraphics[scale=0.4]{Images/faceted_conditional_survival_with_failures_not_carried_forward.pdf}
\caption{ vs }
\label{fig:ConditionalSuccessMethod2}
\end{figure}

We compare three different metrics to estimate the survival rate of participants in our study under ideal circumstances (e.g., if all of our participants were always able to return to rehearse in a timely manner). Our first estimate is    We plot this value in Figures \ref{fig:EstimatedFacetedConditionalSurvivalProbability2}\cut{ and \ref{fig:EstimatedSurvivalProbability2}}. 

Our second estimate, shown in Figure \ref{fig:EstimatedSurvivalProbabilityMethod2}, is quite simple:  

Our first estimate could be biased if participants are less likely to return for future rehearsals whenever they think they might forget their words. Our second estimate could be biased if participants were less likely to return for future rehearsal rounds after previous failures. However, we did not observe any obvious correlation between prior failure and the return rate. Sometimes the return rate was higher for participants who had failed earlier than for participants who had never failed and sometimes the return rate was lower. Furthermore, in our survey of participants who did not return in time for a rehearsal round no one self-reported that they did not return because they were not confident that they would be able to remember (see Section \ref{subsubsec:SurveyResults}). Both methods consistently yielded close estimates. 

We also consider a pessimistic estimate of the survival rate (see Figure \ref{fig:FacetedTotalSurvival})  
  
Here,  counts the number of participants who failed to remember at least one of his action-object pairs with  guesses during {\em any} rehearsal  --- even if that participant did not return for later rehearsal rounds. This estimate is most likely overly pessimistic because it includes every participant who failed early on during the study before dropping out, but it excludes every participant who dropped out without failing. For example, a participant who succeeded through rehearsal  but could not return for rehearsal  would not be included in the estimate for . However, if our participant had failed during round  before not returning for rehearsal  then s/he would still be included in the estimate.\cut{\footnote{As an example, consider a participant who correctly remembered his stories during the first three rehearsals, but was not able to return for the fourth rehearsal (e.g., because he went on vacation). The results of this participant would be dropped. However, if the same participant had failed during round three then his results would be included because we would not have asked him to return for the fourth rehearsal while he was on vacation. Suppose that participants who return for rehearsal one succeed with probability  and that any participant who is able to return for rehearsal  will succeed with probability . The true survival rate would be  for all , but if each participant is not able to return to complete rehearsal  independently with probability  then the total survival rate among participants who always returned will always tend to  as  grows. }.}
\begin{figure}[tb]
\centering
\includegraphics[scale=0.4]{Images/estimated_faceted_survival_of_those_who_never_failed_previously_1.pdf}
\caption{ vs }
\label{fig:EstimatedFacetedConditionalSurvivalProbability2}
\end{figure}


\begin{figure}
\centering
\includegraphics[scale=0.4]{Images/estimated_faceted_survival_of_those_who_never_failed_previously_2.pdf}
\caption{ vs }
\label{fig:EstimatedSurvivalProbabilityMethod2}
\end{figure}


\begin{figure}
\centering
\includegraphics[scale=0.40]{Images/faceted_survival_with_failures_carried_forward.pdf}
\caption{ vs .}
\label{fig:FacetedTotalSurvival}
\end{figure}



\subsection{Survey Results} \label{subsubsec:SurveyResults} We surveyed  participants who did not return to complete their first rehearsal to ask them why they were not able to return. The results from our survey are presented in Figure \ref{fig:SurveyReturned}. The results from our survey strongly support our hypothesis that the primary reason that participants do not return is because they were too busy, because they did not get our follow up message in time, or because they were not interested in interacting with us outside of the initial Mechanical Turk task, and not because they were convinced that they would not remember the story --- no participant indicated that they did not return because they thought that they would not be able to remember the action-object pairs that they memorized.

\begin{figure}[htb]
\begin{center}
\hspace*{-0.1\columnwidth}
\includegraphics[width=1.1\columnwidth]{Images/SurveyResultsBar2.pdf}
\end{center}
\caption[Survey: Which of the following reasons best describes why you were unable to return to take the follow up test?]{Survey Results}
\label{fig:SurveyReturned}
\end{figure}

\paragraph*{Fun} We had several participants e-mail us to tell us how much fun they were having memorizing person-action-object stories. The results from our survey are also consistent with the hypothesis that memorizing person-action-object stories is fun (e.g., no participants said that they no longer wished to participate in the study). 

