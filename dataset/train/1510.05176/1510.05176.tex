\documentclass[a4paper, 11pt]{article}
\usepackage[pdftex, a4paper]{geometry}
\usepackage{graphicx}				

\usepackage{cite}
\bibliographystyle{unsrtnat}


\newcommand*\mcap{\mathbin{\mathpalette\mcapinn\relax}}
\newcommand*\mcapinn[2]{\vcenter{\hbox{}}}



\newcommand*\mcup{\mathbin{\mathpalette\mcupinn\relax}}
\newcommand*\mcupinn[2]{\vcenter{\hbox{}}}


\DeclareFontFamily{OT1}{pzc}{}
\DeclareFontShape{OT1}{pzc}{m}{it}{<-> s * [1.200] pzcmi7t}{}
\DeclareMathAlphabet{\mathpzc}{OT1}{pzc}{m}{it}
			

		

\usepackage{amsmath}
\usepackage{amsfonts}
\usepackage{dsfont}
\usepackage{mathrsfs}
\usepackage{amssymb}
\usepackage{bbm}
\usepackage{epsfig}
\usepackage[english]{babel}
\usepackage[all]{xy}
\usepackage{subfigure}
\usepackage{color}
\def\T{^{\rm\tiny T}}

\interfootnotelinepenalty=10000




\topmargin -0.2in
\voffset -0.2in
\textwidth 6.85in
\oddsidemargin -0.30in
\evensidemargin -0.0in
\textheight 9.30in

\def\baselinestretch{1.4}
\parindent 1.2em
\frenchspacing

\newtheorem{theorem}{Theorem}
\newtheorem{definition}{Definition}
\newtheorem{coro}{Corollary}
\newtheorem{lemma}{Lemma}
\newtheorem{remark}{Remark}
\newtheorem{proposition}{Proposition}
\newtheorem{assumption}{Assumption}
\newtheorem{example}{Example}


\title{\bf Network  Flows that Solve Linear   Equations}


\author{Guodong Shi, Brian D. O. Anderson and Uwe Helmke\thanks{A brief version of the current manuscript  was presented at  the American Control Conference in Boston, July 2016 \cite{acc2016}.
}\thanks{G. Shi is with the Research School of Engineering, College of Engineering and Computer Science, The Australian National University, Canberra 0200, Australia. E-mail: guodong.shi@anu.edu.au.}
\thanks{B. D. O. Anderson is with the National ICT Australia (NICTA) and the Research School of Engineering, College of Engineering and Computer Science, The Australian National University, Canberra 0200, Australia. E-mail: brian.anderson@anu.edu.au.}
\thanks{U. Helmke is deceased but was  with  Institute of Mathematics, University of W\"{u}rzburg, 97074 W\"{u}rzburg, Germany. }}
\date{}


\begin{document}

\maketitle


\begin{abstract}
We study  distributed network flows as  solvers  in continuous time for the linear algebraic equation . Each node  has access to a row  of the matrix  and  the corresponding entry  in the vector . The first ``consensus + projection" flow under investigation  consists of two terms, one from standard consensus dynamics and the other contributing  to projection onto each affine subspace specified by the  and . The second ``projection consensus" flow on the other hand simply replaces the relative state feedback in consensus dynamics with projected relative state feedback.
Without  dwell-time assumption on switching graphs, we prove that all node states converge to a common solution of the linear algebraic equation, if there is any. The convergence is global for the ``consensus + projection" flow while local for the ``projection consensus" flow in the sense that the initial values must lie on the affine subspaces.  If the linear equation has no exact solutions, we show that the node states can converge to a ball around the least squares solution whose radius can be made arbitrarily small through selecting a sufficiently large gain for the ``consensus + projection" flow for a fixed bidirectional graph. Semi-global convergence to approximate least squares solutions is also demonstrated  for switching balanced directed graphs under suitable conditions. It is also shown that the ``projection consensus" flow drives the average of the node states to the least squares solution with a complete graph. Numerical examples are provided as illustrations of the established results.
\end{abstract}
\section{Introduction}

In the past decade, distributed consensus algorithms have attracted a significant amount of research attention \cite{jad03, xiao04,  saber04,  ren05}, due to their wide applications in distributed control and estimation \cite{martinez07,kar12}, distributed signal processing \cite{Rabbat2010}, and distributed optimization methods \cite{rabbat2004,nedic09}. The basic idea is that a network of interconnected nodes with different initial values can reach a common value, namely an  agreement or a consensus, via local information exchange as long as the communication graph is well connected. The consensus value is informative in the sense that it can depend on all nodes' initial states  even if the value is not the exact average \cite{jackson}; the consensus processes are robust against random or deterministic switching of network interactions as well as against noises \cite{moreau05, lin07, jad08, shisiam}.

As a generalized  notion, constrained consensus seeks to reach state agreement  in the intersection of a number of convex sets, where each set serves  as the   supporting state space for a particular node \cite{nedic10}. It was shown that with doubly stochastic arc weights, a projected consensus algorithm,  where each node iteratively projects a weighted average of its neighbor's values onto its supporting set,  can guarantee convergence to constrained consensus when the intersection set is bounded and contains an interior point \cite{nedic10}. This idea was then extended to the continuous-time case under which the graph connectivity and arc weight conditions can be relaxed with convergence still being guaranteed \cite{shitac}. The projected consensus algorithm was shown to be convergent  even under randomized projections \cite{shiauto}.  In fact, those developments are closely related to a class of so-called alternating projection algorithms,  which was first studied by von Neumann in the 1940s \cite{jvn49} and then extensively investigated across the past many decades \cite{aron50, gubin1967,deut83,B-B-SIAM}.


A related but different   intriguing problem  lies in how to develop distributed algorithms that solve a linear algebraic equation  with respect to variable , to which tremendous efforts have been devoted with many algorithms developed under different assumptions of information exchanges among the nodes \cite{anderson97, mehmood05, lu09-1,lu09-2,liu13,mou13,Morse-TAC15, asuman14, jadbabaie15, brian15}. Naturally, we can assume that a node  has access to a row vector,   of  as well as the corresponding entry,   in . Each node will then be able to determine with straightforward calculations an affine subspace whose elements satisfy the equation . This means that, as long as the original equation  has a nonempty solution set, solving the equation would be equivalent to finding a point in the intersection of all the affine subspaces and therefore  can be put into the constrained consensus framework. If however the linear equation has no exact solution and  its least squares solutions are of interest, we ended up with a convex optimization problem with quadratic cost and linear constraints. Many distributed optimization algorithms, such as \cite{nedic09,nedic10,elia,jmf,nedic11,Rabbat2012,lu12,cotes14,jakovetic14}, developed for much more complex models, can therefore be directly applied. It is worth emphasizing that the above ideas of distributed consensus and optimization were traced back to the seminal work of Bertsekas and J. Tsitsiklis \cite{tsi,tsibook}.


In this paper, we study two distributed network flows as distributed solvers for such linear algebraic equations in continuous time. The first so-called ``consensus + projection" flow consists of two additive  terms, one from standard consensus dynamics and the other contributing to projection onto each affine subspace. The second ``projection consensus" flow on the other hand simply replaces the relative state feedback in consensus dynamics with projected relative state feedback. Essentially only relative state information is exchanged among the nodes for both of the two flows, which justifies their full {\it distributedness}.  To study the asymptotic behaviours of the two distributed flows with respect to the solutions of the linear equation, new challenges arise in the loss of intersection boundedness and interior points for the exact solution case as well as in the complete loss of intersection sets for the least squares solution case. As a result, the analysis cannot be simply mapped back to the studies in \cite{nedic10,shitac}.

The contributions of the current paper are summarized as follows.

\begin{itemize}
\item Under mild conditions on the communication graphs (without requiring a  dwell-time for switching graphs and  only  requiring a  positively lower bound on the integral of arc weights over certain time intervals), we prove that all node states asymptotically converge to a common solution of the linear algebraic equation for the two flows, if there is any. The convergence is global for the ``consensus + projection" flow, and local for the ``projection consensus" flow in the sense that the initial values must be put into the affine subspaces  (which is a very minor restriction indeed). We manage to characterize the node limits for balanced or fixed graphs.

\item If the linear equation has no exact solutions, we show that the node states can be forced to converge to a ball of fixed but arbitrarily small radius surround the  least squares solution by taking  the gain of the consensus dynamics to be sufficiently large for ``consensus + projection" flow under fixed and undirected graphs. Semi-global convergence to approximate least squares solutions is established  for switching balanced directed graphs under suitable conditions.  A minor, but more explicit result arises where it is also shown that the ``projection consensus" flow drives the average of the node states to the least squares solution with complete communication graphs.
\end{itemize}

These results rely on our development of new technique of independent interest for treating the interplay between consensus dynamics and the projections onto affine subspaces in the absence of  intersection boundedness and interior point assumptions. All the convergence results  can be sharpened to provide exponential convergence with suitable conditions on the switching graph signals.

The remainder of this paper is organized as follows. Section \ref{Sec:Problem} introduces the network model, presents the distributed flows under consideration, and defines the problem of interest. Section \ref{Sec:exact} discusses the scenario where the linear equation has at least one solution. Section \ref{Sec:least-squares} then turns to the least squares case where the linear equation has no solution at all. Finally, Section \ref{Sec:numerical} presents a few numerical examples illustrating the established  results and Section \ref{Sec:conclusions} concludes the paper with a few remarks.


\subsection*{Notation and Terminology}

A directed graph (digraph) is an ordered pair of two sets denoted by  \cite{god}. Here   is a finite set of  vertices (nodes). Each element in  is an ordered pair of two distinct  nodes in , called an arc.  A  directed path in  with length  from  to  is a  sequence of distinct nodes, , such that  , for all . A digraph  is termed {\it strongly connected} if for any two distinct nodes , there is a  path from  to . A digraph is called {\it bidirectional} when  if and only if  for all  and . A strongly connected bidirectional digraph  is  simply called {\it connected}. All vectors are column vectors and denoted by bold, lower case letters, i.e., ,  etc.; matrices are denoted with bold, upper case letters, i.e.,  ,  etc.;  sets are denoted with , etc. Depending on the argument,  stands for the absolute value of a real number or the cardinality of a set.  The Euclidean inner product between two vectors  and  in  is denoted as , and sometimes  simply  .  The Euclidean norm of a vector is denoted as .







\section{Problem Definition}\label{Sec:Problem}

\subsection{Linear Equations}
Consider the following  linear algebraic equation:

with respect to variable , where  and . We know from the basics of linear algebra that overall there are three cases.

\begin{itemize}
\item[(I)] There exists a unique solution satisfying  Eq. (\ref{LinearEquation}):  and .

\item[(II)] There is an infinite set of  solutions  satisfying  Eq. (\ref{LinearEquation}):  and .


\item[(III)] There exists no solution   satisfying  Eq. (\ref{LinearEquation}): .
\end{itemize}

We denote

with  being the -th row vector of . For the ease of presentation and with inessential loss of generality we assume throughout the rest of the paper that
 Introduce

for each , which is an affine subspace.  It is then clear that Case (I) is equivalent to the condition that  is a singleton, and that Case (II) is equivalent to the condition  that  is an affine space with a nontrivial dimension. For Case (III), a least squaress solution of (\ref{LinearEquation}) can be defined via the following  optimization problem:

which yields a unique solution  if .


Consider a network with nodes  indexed in the set .  Each node  has access to the value of  and  without the knowledge of  or  from other nodes. Each node   holds a state  and exchanges this state information with other neighbor nodes, these being determined by the edges of the graph of the network.
We are interested in distributed flows for the  that asymptotically solve the equation (\ref{LinearEquation}), i.e.,  approaches some solution of  (\ref{LinearEquation}) as  grows.
\subsection{Network Communication Structures}
Let  denote the set of all directed graphs associated with node set . Node interactions are described by a  signal .
The digraph that  defines at time  is denoted as , where  is the set of arcs. The neighbor set of node  at time , denoted , is given by

This is to say, at any given time , node  can only receive information from the nodes in the set .

Let  and  be the sets of nonnegative and positive real numbers, respectively. Associated with each ordered pair  there is a function  representing the weight of the possible connection . We impose the following assumption, which will be adopted throughout the paper without specific  further mention.

\medskip

\noindent {\bf Weights Assumption.} The function  is continues except for at most a set with measure zero over  for all ; there exists  such that  for all  and all .

\medskip




Denote  as an indicator function for all , where   if  and  otherwise.  We impose the following definition on the connectivity of the network communication structures.

\begin{definition}
\begin{itemize}
\item[(i)] An arc  is said to be a -arc of  for the
time interval   if 

\item[(ii)]  is -uniformly jointly strongly connected (-) if there exists  such that  the  -arcs of   on time interval  form a  strongly connected digraph for all ;

\item[(iii)]  is -bidirectionally infinitely  jointly  connected (-) if    is bidirectional for all  and the  -arcs of   on time interval  form a connected graph for all .
    \end{itemize}
\end{definition}
\subsection{Distributed Flows}
Define the mapping  as the projection onto the affine subspace . Let  be a given constant. We consider the following continuous-time network flows:

\medskip

\noindent[{\bf ``Consensus + Projection" Flow}]:

\noindent[{\bf ``Projection Consensus" Flow}]:




\begin{figure*}
\begin{minipage}[t]{0.5\linewidth}
\centering
\includegraphics[width=2.4in]{projectionplusconsensus.png}
\end{minipage}\begin{minipage}[t]{0.5\linewidth}
\centering
\includegraphics[width=2.4in]{projectionconsensus.png}
\end{minipage}
\caption{An illustration of the ``consensus + projection" flow (left) and the ``projection consensus" flow (right). The blue arrows mark the vector of  for the two flows, respectively.}
\label{fig:twoflows}
\end{figure*}


The first term of (\ref{1}) is a standard  consensus flow \cite{jad03}, while along

 will be asymptotically projected onto  since (\ref{r1}) is a gradient flow with , where  is a   convex function. The flow (\ref{2}) simply replaces the relative state in standard consensus flow with relative projective state.


Notice that a particular equilibrium point of these equations is given by  , where  is a solution of Eq. (\ref{LinearEquation}). The aim of the two flows is to ensure that the  asymptotically tend to a solution of Eq. (\ref{LinearEquation}). Note that, the projection consensus flow (\ref{2}) can be rewritten as\footnote{A rigorous treatment will be given later in Lemma \ref{lem:subspace-affine-projection}.}

Therefore, in both of the ``consensus + projection" and the  ``projection consensus" flows, a node  essentially only receives the information

from its  neighbors, and the flows can then be utilized in addition with the  and  it holds. In this way,   the  flows (\ref{1}) and (\ref{2}) are {\it distributed}.

Without loss of generality we assume the initial time is . We denote by   a trajectory along any of the two flows.

\subsection{Discussions}

\subsubsection{Geometries}
 The two network flows are intrinsically different in their geometries. In fact, the ``consensus + projection" flow  is a special case of the optimal consensus flow proposed in \cite{shitac} consisting of two parts, a consensus part and another projection part. The ``projection consensus" flow, first  proposed and studied in \cite{brian15} for fixed bidirectional graphs,  is the continuous-time analogue of the projected consensus algorithm   proposed in \cite{nedic10}. Because it is a gradient descent on a Riemannian manifold if the communication graph is undirected and fixed, there is guaranteed convergence to an equilibrium point.  The two flows are closely related to the alternating projection algorithms,  first studied by von Neumann in the 1940s \cite{jvn49}. We refer to  \cite{B-B-SIAM} for a thorough survey on the developments of alternating projection algorithms. We illustrate the intuitive  difference between the two flows in Figure \ref{fig:twoflows}.

\subsubsection{Relation with Previous Work}
The dynamics described in systems (\ref{1}) and (\ref{2}) are linear, in contrast to the nonlinear dynamics associated with the general convex sets  studied in \cite{nedic10,shitac}. However, we would like to emphasize that new challenges arise with the   systems (\ref{1}) and (\ref{2}) compared to the work of \cite{nedic10,shitac}. First of all, for both of the Cases (I) and (II),  contains no interior point. This interior point condition  however is essential to the  results   in   \cite{nedic10}.
Next, for Case (II) where   is an affine space with a nontrivial dimension, the boundedness condition for the intersection of the convex sets no longer holds, which plays a key role in the analysis of  \cite{nedic10,shitac}. Finally, for Case (III),   becomes an empty set. The least squares solution case then completely falls out from the discussions of constrained consensus in \cite{shitac,nedic10}.

\subsubsection{Grouping of Rows}

We have assumed hitherto that each node  only has access to the value of  and  from the equation (\ref{LinearEquation}) and therefore there are a total of  nodes. Alternatively and as a generalization, there can be  nodes with node  holding  rows, denoted  of  with . In this case, we can revise the definition of  to

which is nonempty if (\ref{LinearEquation}) has at least one solution. Then the two flows (\ref{1}) and (\ref{2}) can be defined in the same manner for the  nodes. Of course, if  is large, finding an initial condition consistent with  becomes computationally more burdensome.

Let\footnote{Note that, it is not necessary to require the  to be disjoint.} 
Note  that,  the  are still affine subspaces,  while solving (\ref{LinearEquation}) exactly  continues to be equivalent to finding a point in . Consequently, all our results for Cases (I) and (II) apply also to this new setting with row grouping.


\section{Exact Solutions}\label{Sec:exact}
In this section, we show how the two distributed flows  asymptotically  solve the equation (\ref{LinearEquation}) under quite general conditions for Cases (I) and (II).
\subsection{Singleton  Solution Set}
We first focus on the case when Eq. (\ref{LinearEquation}) admits a unique solution , or equivalently,   is a singleton. For the ``consensus + projection" flow (\ref{1}), the following theorem holds.



\begin{theorem}\label{thm1}
Let (I) hold. Then  along  the ``consensus + projection" flow (\ref{1}), there holds  for all initial values if  is either  - or -.
\end{theorem}


The ``projection consensus" flow (\ref{2}), however, can only guarantee local convergence for a particular set of initial values. The following theorem holds.
\begin{theorem}\label{thm2}
Let (I) hold. Suppose  for all . Then  along  the  ``projection consensus" flow (\ref{2}), there holds
 if  is either  - or -.
\end{theorem}


\begin{remark}
Convergence along the ``projection consensus" flow relies on specific  initial values  due to the existence of equilibriums other than the desired consensus states within the set : if  are all equal, then obviously they will stay there for ever along the flow (\ref{2}). It was suggested in \cite{brian15} that one can add another term in the ``projection consensus" flow and arrive at

for , then convergence will be global under (\ref{100}). We would like to point out that (\ref{100}) has a similar structure  as the ``consensus + projection" flow with the consensus dynamics being replaced by projection consensus.  In fact, our analysis of the ``consensus + projection" flow developed  in  this paper can be refined and generalized to  the flow (\ref{100}) and then leads to  results under the same graph conditions for both exact and least squares solutions.
\end{remark}

\subsection{Infinite  Solution Set}
We now turn to the scenario when Eq. (\ref{LinearEquation}) has an infinite set of  solutions, i.e.,  is an affine space with a nontrivial dimension. We note that in this case  is no longer a bounded set; nor does it contain interior points. This is in contrast to the situation studied in \cite{nedic10,shitac}.

For the ``consensus + projection" flow (\ref{1}), we present the following result.


\begin{theorem}\label{thm3}
Let (II) hold. Then   along  the ``consensus + projection" flow (\ref{1}) and for any initial value , there exists  , which is a solution of (\ref{LinearEquation}), such that    if  is either  - or -.
\end{theorem}


For the ``projection consensus" flow, convergence relies on restricted initial nodes states.
\begin{theorem}\label{thm4}
Let (II) hold. Then   along  the `projection consensus" flow  (\ref{2}) and for any initial value  with  for all , there exists  , which is a solution of (\ref{LinearEquation}), such that    if  is either  - or -.
\end{theorem}

\subsection{Discussion: Convergence Speed/The Limits}


For any given graph signal , the value of  in Theorems \ref{thm3} and \ref{thm4}  depends only on the initial value . We manage to provide a characterization to   for balanced switching graphs or fixed graphs. Denote  as the projection operator over .

\begin{theorem}\label{prop1} The following statements hold for both the ``consensus + projection" and the ``projection consensus" flows.
\begin{itemize}
\item[(i)] Suppose  is balanced, i.e.,  for all  and  for all . Suppose in addition  that  is either  - or -. Then

\item[(ii)] Suppose  for some fixed, strongly connected, digraph  and for any ,  for some constant .  Let  with  be the left eigenvector  corresponding to the simple eigenvalue zero of the Laplacian\footnote{The Laplacian matrix  associated with the graph  under the given arc weights is defined as  where  and  In fact, we have  for all  if  is strongly connected \cite{magnusbook}.}  of the digraph .
Then we have


\end{itemize}
\end{theorem}



 Due to the linear nature of the systems (\ref{1}) and (\ref{2}), it is straightforward to see that  the convergence stated in Theorems \ref{thm1}, \ref{thm2}, \ref{thm3} and \ref{thm4} is   exponential if  is periodic and  -. It is however difficult to provide tight approximations of the exact convergence rates because of the general  structure with switching interactions adopted by our network model. For time-invariant networks with constant edge weights, the two flows become linear time-invariant,  and then  the convergence rate is determined by the spectrum of the state transition matrix: the rate of convergence for the  ``consensus + projection" flow is determined together by both the graph Laplacian and the structure of the linear manifolds  and one in fact cannot dominate another;  the rate of convergence for the ``projection consensus" flow is upper bounded by the smallest nonzero eigenvalue of the graph Laplacian \cite{brian15}.




\subsection{Preliminaries and Auxiliary Lemmas}
Before presenting the detailed proofs for the stated results, we recall  some preliminary theories from  affine subspaces, convex analysis,  invariant sets, and  Dini derivatives, as well as a few auxiliary lemmas which will be useful  for the analysis.



\subsubsection{Affine Spaces and  Projection Operators}



An affine space is a set  that admits a free transitive action of a vector space . A set  is said to be {\it convex} if  whenever  and  \cite{rock}.
For any set , the intersection of all convex sets
containing  is called the {\it convex hull} of , denoted by
. Let  be a closed convex subset in  and denote
 as the distance between  and .  There is a unique element  satisfying
 associated to any
 \cite{aubin}.  The map  is called the {\it projector} onto \footnote{The projections  and  introduced earlier are consistent with this definition.}. The following lemma holds  \cite{aubin}.
\begin{lemma}\label{lemconvex}
(i) .

(ii) .

(iii)  The function  is continuously differentiable at every  with .
\end{lemma}


\subsubsection{Invariant Sets and Dini Derivatives}
Consider the following
autonomous system

where  is a  continuous function. Let  be a solution of
(\ref{i1}) with initial condition . Then  is called a {\it positively invariant
set} of (\ref{i1}) if, for any  and any ,
we have , , along  every solution  of (\ref{i1}).

The  {\it upper Dini
derivative} of a continuous function  () at  is defined as

When  is continuous on ,  is
non-increasing on  if and only if  for any
. Hence if  for all  and  is lower bounded then the limit of  exists as a finite number when  tends to infinity.

The next
result is convenient for the calculation of the Dini derivative \cite{dan}.

\begin{lemma}
\label{lemdini}  Let  be
 a continuously differentiable  function with respect to both  and   and . Let  be an absolutely continuous function. If 
is the set of indices where the maximum is reached at , then

\end{lemma}


\subsubsection{Key Lemmas}

The following lemmas, Lemmas \ref{lem:affineprojection}, \ref{lem:subspace-affine-projection}, \ref{lem:doubleprojection} can be easily proved using the properties of affine spaces, which turn out to be useful throughout our analysis. We therefore collect them below and the details of their proofs are omitted.
\begin{lemma}\label{lem:affineprojection}
Let  be an affine subspace, where  with , and . Denote  as the projection onto . Then  for all . Consequently there holds  for all .
\end{lemma}
\begin{lemma}\label{lem:subspace-affine-projection}
Let  be an affine subspace in . Take an arbitrary point  and define , which is a subspace in .  Denote   and  as the projectors onto  and , respectively. Then  for all .
\end{lemma}
\begin{lemma}\label{lem:doubleprojection}
Let  and  be two affine subspaces in  with . Denote  and   as the projections onto  and , respectively. Then 
\end{lemma}
\begin{lemma}\label{lem:invariant}
Suppose either (I) or (II) holds. Take  as an arbitrary solution of (\ref{LinearEquation}) and let  be arbitrary.  Define

Then \begin{itemize}
\item[(i)]  is a positively invariant set for the ``consensus + projection" flow (\ref{1});
    \item[(ii)]   is a positively invariant set for the ``projection consensus" flow (\ref{2}).
\end{itemize}
\end{lemma}
{\it Proof.} Let  be a trajectory along the flows (\ref{1}) or (\ref{2}).  Define 
Denote .

\noindent (i) From Lemma \ref{lemdini} we obtain that along the flow (\ref{1})

where  follows from the fact that

in view of Lemma \ref{lem:affineprojection} and the fact that  is a solution of (\ref{LinearEquation});  makes use of the elementary inequality

 follows from the definition of  and .

We therefore know from (\ref{3}) that  is always a non-increasing functions. This implies that,  if , i.e.,   for all , then   for all  and all . Equivalently,   is a positively invariant set for the flow (\ref{1}).

\noindent(ii) Let . The structure of the flow (\ref{2}) immediately tells that  for all .  Furthermore, again by Lemma \ref{lemdini} we obtain

where  follows from Lemma \ref{lem:affineprojection},  holds due to the fact that  is a projection  matrix,  is again based on the inequality (\ref{4}), and  holds because  (so that ) and   that   is a projection  matrix (so that ).

We can therefore readily conclude that  is a positively invariant set for the ``projection consensus" flow (\ref{2}). \hfill

\subsection{Proofs of Statements}
We  now have the tools in place to present the proofs of the stated results.
\subsubsection{Proof of Theorems \ref{thm1} and \ref{thm2}}
When (I) holds,  is a singleton, which is obviously a bounded set. The ``consensus + projection" flow (\ref{1}) is a special case of the optimal consensus flow proposed in \cite{shitac}. Theorem \ref{thm1} readily follows  from Theorems 3.1 and Theorems 3.2 in \cite{shitac} by adapting the treatments to the Weights Assumption adopted  in the current paper using the techniques established in \cite{shisiam}. We therefore omit the details.

Being a special case of  Theorem \ref{thm4},  Theorem \ref{thm2} holds true as a direct consequence of Theorem \ref{thm4}, whose proof will be presented below.

\subsubsection{Proof of Theorem \ref{thm3}}
Recall that   is  an arbitrary solution of (\ref{LinearEquation}). With Lemma \ref{lem:invariant}, for any initial value ,   the set

which  is  obviously bounded, is a positively invariant set for the ``consensus + projection" flow (\ref{1}). Define 
and .  As a result,  recalling  Theorem 3.1 and Theorem 3.2 from \cite{shitac}\footnote{Again, the arguments in \cite{shitac} were based on switching graph signals with dwell time and absolutely bounded weight functions. We can however borrow the treatments in \cite{shisiam} to the generalized graph and arc weight model considered here under which we can rebuild the results in \cite{shitac}. More details will be provided in the proof of Theorem \ref{thm4}.   }, if  is either  - or -, there hold

(i)  for all ;

(ii)   for all  and .

 We still need to show that the limits of the  indeed exist. Introduce

and .

 With Lemma \ref{lem:doubleprojection},  we see that
 
 As a result, taking  from both the left and right sides of (\ref{1})\footnote{Note that,  is a projector onto a subspace.}, we obtain

This is to say, if  is either  - or -, we can invoke Theorem 4.1 (for - graphs) and Theorem 5.2 (for - graphs) in \cite{shisiam} to conclude:  for any initial value , there exists  such that

While on the other hand  leads to

where the last equality follows from Lemma \ref{lem:subspace-affine-projection}. We conclude from (\ref{5}) and (\ref{6}) that

We have now completed the proof of Theorem \ref{thm3}.
\subsubsection{Proof of Theorem \ref{thm4}}
We continue to use the notation   
and  introduced in the proof of Theorem \ref{thm3}. In view of Lemma \ref{lem:invariant}, for any initial value ,
  is a positively invariant set for the ``projection consensus" flow (\ref{2}). Define


Let .  We obtain from Lemma \ref{lemconvex}.(iii) that

In (\ref{7}), the argument used to establish (\ref{8}) can be carried through when   is replaced by  . Accordingly, we obtain\footnote{Note that the inequalities in (\ref{8}) hold without relying   on the fact that  is a constant.}, . This immediately implies that   there is a  constant  such that . As a result, for any , there exists  such that



Now that  is a nonnegative constant satisfying  , we can in fact show that  with suitable graph conditions, as summarized in the following  two technical lemmas. Details of their proofs can be found  in the Appendix.


\medskip




\begin{lemma}\label{lem:directed}
Let   be -. Then . In fact,  due to the following two contradictive conclusions:

  (i) If , then  there holds    for all   along the ``projection consensus" flow (\ref{2}).

  (ii) If      for all   along the ``projection consensus" flow (\ref{2}), then .
\end{lemma}




\medskip

\begin{lemma}\label{lemma-bidirectional}
Suppose  is -. Then  along the ``projection consensus" flow (\ref{2}).
\end{lemma}

\medskip






Recall  . Applying  to both the left and right sides of (\ref{2}), we have

where we have used Lemma \ref{lem:subspace-affine-projection} to derive
 We can again make use of the argument  in the proof of Theorem \ref{thm3} and conclude that the limits of the  exist and they are obviously the same.

We have now completed the proof of Theorem \ref{thm4}.

\subsubsection{Proof of Theorem \ref{prop1}}

We provide detailed proof for the ``projection+consensus" flow.  The analysis for the projection consensus flow can be similarly established in view of  (\ref{25}).


\noindent (i) With (\ref{20}), we have 
along the ``projection+consensus" flow. Note that (\ref{22}) is a standard consensus flow with arguments being , . As a result, there holds

if  is balanced for all   \cite{saber04}. As a result, similar to  (\ref{21}), we have


\noindent (ii) If  for some fixed digraph  and for any ,  for some constant , then along (\ref{22}) we have   \cite{saber04}

where  is the left eigenvector corresponding to eigenvalue zero for the Laplacian matrix . This immediately gives us

which completes the proof.
\section{Least Squares Solutions}\label{Sec:least-squares}

In this section, we turn to  Case  (III) and consider that (\ref{LinearEquation}) admits a unique least squares solution . Evidently, neither of the two continuous-time distributed flows (\ref{1}) and (\ref{2}) in general can yield exact convergence to the least squares solution of (\ref{LinearEquation}) since, even for a fixed interaction graph,  is not an equilibrium of the two network flows.


It is indeed possible to find the least squares solution using  double-integrator node dynamics \cite{elia,cotes14}.  However, the use of double integrator dynamics was  restricted to   networks with fixed and undirected (or balanced) communication graphs \cite{elia,cotes14}. In another direction,  one can also borrow the idea of  the use of    square-summable step-size sequences with infinite sums in discrete-time algorithms \cite{nedic10} and  build the following flow

for .
The least squares case can then be solved under graph conditions of connectedness and balance \cite{nedic10}, but the convergence rate is at best . This means (\ref{9}) will be fragile against noises.



For the ``projection+consensus" flow, we can show that under fixed and connected bidirectional interaction graphs,  with a sufficiently large , the node states will converge to  a ball around the least squares solution whose radius can be made arbitrarily small. This approximation is global in the sense that the required  only depends on the accuracy between the node state limits and the .
\begin{theorem}\label{thm5}
Let (III) hold with  and  denote the unique least squares solution of (\ref{LinearEquation}) as .  Suppose  for some bidirectional, connected graph  and for any ,  for some constant . Then   along  the flow (\ref{1}), for any , there exists  such that  exists and
 
for any initial value  if .
\end{theorem}



The intuition behind Theorem \ref{thm5} can be described briefly as follows (see its proof presented later for a full exposure of this intuition). With fixed and undirected network topology,  the ``consensus + projection" flow (\ref{1}) is a gradient flow  in the form of

where

There  holds

if each  is normalized with a unit length. Therefore, for large , the trajectory of (\ref{1}) will asymptotically  tend to  be close to the solution of the following optimization problem:

Any solution of the above problem  consists of   copies of the solution to

which is  the unique least squares  solution  if . Therefore, a large  can drive the node states to somewhere near  along the flow (\ref{1}) as stated in Theorem \ref{thm5}.

\begin{remark}[Least Squares Solutions without Normalizing ]
From the above argument, for the case when  is not normalized, we can replace the ``consensus + projection" flow (\ref{1}) with

 for all . The statement  of Theorem \ref{thm5} will continue to hold for the flow (\ref{r3}).
\end{remark}


\begin{remark}[Tradeoff between Convergence Rate and Accuracy]
Under the assumptions of Theorem \ref{thm5}, the flow defines a  linear time-invariant system

where  is the Laplacian of the graph ,   is a block-diagonal matrix, and . Therefore, the rate of convergence is given by
 
   which is influenced by both  (the graph) and  (the linear equation). We also know that   in general cannot grow to infinity by increasing  due to the presence of the zero eigenvalue in . One can however speed up the convergence by adding a weight  to the projection term in the flow (\ref{1}) and obtain
   
  Certainly the convergence rate (to a ball centered at  with radius ) of (\ref{r9}) can be arbitrarily large  by selecting sufficiently large  and . However, it is clear from the proof of Theorem \ref{thm5} that the required  for a given accuracy   will in turn depend on  and require a large  for a large .







 \end{remark}




  We also manage to establish the following semi-global result for  switching but balanced graphs under two further assumptions.

\medskip

\noindent {\bf [A1]} The set  is a bounded set.

\noindent {\bf [A2]} .



\begin{theorem}\label{thm6}
Let (III) hold with  and  denote the unique least squares solution of (\ref{LinearEquation}) as . Assume [A1] and [A2].  Suppose  is balanced for all  and   - with respect to . Then   for any  and  any , there exist two constants  and  such that
when  and , there holds 
along  the flow (\ref{1}) with the initial value .
\end{theorem}



\begin{remark}
The two assumptions, [A1] and [A2] are indeed rather strong assumptions. In general,  holds if  is a nonempty bounded set \cite{B-B-SIAM}, which is exactly opposite to the least squares solution case under consideration. We conjecture that at least for  case,  should hold  when  are distinct vectors.  The assumption  requires a strong  symmetry in the affine spaces , which turns out to be essential for the result to stand.
\end{remark}



For the ``projection consensus" flow, we present the following result.

\begin{theorem}\label{thm7}
Let (III) hold with  and  denote the unique least squares solution of (\ref{LinearEquation}) as . Suppose  is fixed, complete, and  for all .
Then    for any  initial value  , there holds

along  the flow (\ref{2}).
\end{theorem}

 Based on simulation experiments,  the requirement that   does not have to be complete for the Theorem \ref{thm7} to be valid for certain choices of the . Moreover, because the flow (\ref{2}) is a gradient flow over the manifold , the  will converge but perhaps to different limits at different nodes.
\subsection{Proof of Theorem \ref{thm5}}
 Suppose  for some bidirectional, connected graph  and for any ,  for some constant .  Fix . Note that, we have


 Therefore, the flow (\ref{1}) is a gradient flow written as

where   is a  convex function.  Denote . There must hold





The following lemma holds.

\begin{lemma}
Suppose . Then  \begin{itemize}
\item[(i)]  is a singleton, which implies that  converges to a fixed point asymptotically;
\item[(ii)]  is a bounded set for all .
\end{itemize}
\end{lemma}
{\it Proof.} (i) Recall that  is the Laplacian of the graph ,    is a block-diagonal matrix, and . With Lemma \ref{lem:affineprojection},  the equation  can be written as

or, in a compact form,



Consider

We immediately know from the second term of  that   only if .  On the other hand,  obviously

for any  if . Therefore,  is positive-definite, which yields


\noindent(ii) By the Courant-Fischer Theorem (see Theorem 4.2.11 in \cite{horn}), we have

This immediately implies

for all  and consequently,   is obviously a bounded set. This proves the desired lemma. \hfill



Now we introduce  . Let  with  and define

 and
 
 with  being the unique least squares solution of (\ref{LinearEquation}). We see that  and  are finite numbers due to the boundedness of  . The remainder of the proof contains two steps.

 \medskip

 \noindent{\bf Step 1}. Let . Then    satisfies  (\ref{200}), or in equivalent form,


 Denote\footnote{In the rest of the proof we sometimes omit  in , , and  in order to simplify the presentation. One should however keep in mind that they always depend on .} . Let
 
be the consensus manifold. Denote  as the second smallest eigenvalue of . We can now conclude that

where   holds from the fact that the zero eigenvalue of  corresponds to a unique unit eigenvector whose entries  are all the same,  is from (\ref{32}), and  holds from the definition of  and the fact that . This allows us to further derive

and then 
Therefore, for any  we can find  that

for all .

\noindent {\bf Step 2}. Applying Lemma \ref{lem:affineprojection}, we have 
Then with (\ref{31}) and noticing  the continuity of  , for any , there exists  such that

if

 Consequently, we can conclude without loss of generality\footnote{If  then we can replace  with  in (\ref{201}) with (\ref{51}) continuing to hold; if  then both (\ref{31}) and (\ref{51}) hold directly. } that for any  we can find  so that both  (\ref{31}) and (\ref{51})  hold when  .

Now noticing , from (\ref{32}) we have

Therefore, with (\ref{31}), we can find another  such that

for all .

Take . We can finally conclude that

for all , where  is from (\ref{51}),  is from the convexity of ,  is based on direct computation of  in (\ref{52}),  is due to (\ref{53}), and  holds because  from the definition of  and .


 This completes the proof of Theorem \ref{thm5}.






\subsection{Proof of Theorem \ref{thm6}}

Consider the following dynamics for the considered network model:

where ,  is a given constant,  are weight functions satisfying our standing assumption, and   is a piecewise continuous  function. The proof is based on the following lemma on the robustness of  consensus subject to noises, which is a special case of Theorem 4.1 and Proposition 4.10  in \cite{shisiam}.

\begin{lemma}\label{lemrobust}
Let  be - with respect to . Then along (\ref{60}), there holds that for any ,  there exist a sufficiently small   and sufficiently large  such that  
for all initial value  when  and , where .
\end{lemma}


With Assumption [A1], the set

is a compact set, and is obviously positively invariant along the flow (\ref{1}). Therefore, we can define

as a finite number. Now along the trajectory  of (\ref{1}), we have

for all  and all . Invoking Lemma \ref{lemrobust} we have  for any  and  any initial value , there exist  and  such that
 
if  and .

Furthermore,  with Lemma \ref{lem:subspace-affine-projection}, we have

which implies

if Assumption [A2] holds and  is balanced. While  if , then 
This certainly guarantees  for all  in view of (\ref{66}). The proof for the fact that
 
 can then be built using exactly the same analysis as the final part of the proof of Theorem \ref{thm5}.

We have now completed the proof of Theorem \ref{thm6}.






\subsection{Proof of Theorem \ref{thm7}}
The proof is based on the following lemma.

\begin{lemma}
There holds  for all .
\end{lemma}
{\it Proof.} From Lemma \ref{lem:subspace-affine-projection} we can easily know  for all . As a result, we have

On the other hand,

The desired lemma thus holds. \hfill

Suppose  is fixed, complete, and  for all . Now along the flow (\ref{2}), we have

for any  initial value  .  The desired result follows straightforwardly and this concludes the proof of Theorem \ref{thm7}.

\begin{figure*}
\begin{minipage}[t]{0.5\linewidth}
\centering
\includegraphics[width=3in]{E1-plus.png}
\end{minipage}\begin{minipage}[t]{0.5\linewidth}
\centering
\includegraphics[width=3in]{E1-projection.png}
\end{minipage}
\caption{The trajectories of the node states along the ``consensus + projection" flow (left) and the ``projection consensus" flow (right).}
\label{fig:fig1}
\end{figure*}

\begin{figure*}
\begin{minipage}[t]{0.5\linewidth}
\centering
\includegraphics[width=3in]{E2-plus.png}
\end{minipage}\begin{minipage}[t]{0.5\linewidth}
\centering
\includegraphics[width=3in]{E2-projection.png}
\end{minipage}
\caption{The trajectories of    along the ``consensus + projection" flow (left) and the ``projection consensus" flow (right).}
\label{fig:fig2}
\end{figure*}



\section{Numerical Examples}\label{Sec:numerical}
In this section, we provide a few examples illustrating the established results.

\medskip

\noindent {\bf Example 1}. Consider three nodes in the set  whose interactions form a fixed, three-node directed cycle. Let , , and  for all .  Take , ,  and , ,   so the linear equation (\ref{LinearEquation}) has a unique solution at  corresponding to Case (I). With the same initial value , , and , we plot the trajectories of , respectively, for the ``consensus + projection" flow (\ref{1}) and the ``projection consensus" flow (\ref{2}) in Figure \ref{fig:fig1}. The plot is consistent with the result of Theorems \ref{thm1} and \ref{thm2}.


\medskip

\noindent {\bf Example 2}. Consider three nodes in the set  whose interactions form a fixed, three-node directed cycle.  Let , , and  for all .   Take , ,  and , ,   so corresponding to Case (II), the linear equation (\ref{LinearEquation}) admits an infinite solution set  
 For the initial value  , , and , we plot the trajectories of 
 respectively, for the ``consensus + projection" flow (\ref{1}) and the ``projection consensus" flow (\ref{2})  in Figure \ref{fig:fig2}. The plot is consistent with the result of Theorems \ref{thm3}, \ref{thm4}, and \ref{prop1}.


\medskip


\noindent {\bf Example 3}. Consider four nodes in the set  whose interactions form a fixed, four-node undirected cycle.  Let ,  and  for all .   Take , , ,  and ,   , ,  so corresponding to Case (III) the linear equation (\ref{LinearEquation}) has  a unique least squares solution .
 For the initial value , , , and  we plot the trajectories of 
along  the ``consensus + projection" flow (\ref{1}) for , respectively in Figure \ref{fig:fig3}.   The plot is consistent with the result of Theorem \ref{thm5}.

\begin{figure}
\centering
\includegraphics[width=3.2in]{E3-LS-plus.png}
\caption{The evolution of   for , respectively along the ``consensus + projection" flow.}
\label{fig:fig3}
\end{figure}








\section{Conclusions}\label{Sec:conclusions}

Two distributed network flows were studied  as distributed solvers for a linear algebraic equation , where a  node  holds a row  of the matrix  and the entry  in the vector . A ``consensus + projection" flow consists of two terms, one from standard consensus dynamics and the other as projection onto each affine subspace specified by the  and .  Another  ``projection consensus" flow  simply replaces the relative state feedback in consensus dynamics with projected relative state feedback. Under mild graph conditions, it was shown that  that all node states converge to a common solution of the linear algebraic equation, if there is any. The convergence is global for the ``consensus + projection" flow while local for the ``projection consensus" flow.  When the linear equation has no exact solutions but has a well defined least squares approximate solution, it was proved  that the node states can converge to somewhere  arbitrarily near the least squares solution as long as the gain of the consensus dynamics is sufficient large for ``consensus + projection" flow under fixed and undirected graphs.  It was also shown that the ``projection consensus" flow drives the average of the node states to the least squares solution if the communication graph is complete. Numerical examples were provided verifying  the established results.  Interesting future direction includes more precise   comparisons of the existing continuous-time or discrete-time linear equation solvers in terms of convergence speed, computational complexity, and communication complexity.




\section*{Appendix}

\subsection*{A. Proof of Lemma \ref{lem:directed}. (i)}
Suppose . We show    for all  by a contradiction argument. Suppose (to obtain a contradiction) that there exists a node    with . Therefore, we can find a time instant  with

In other words, there is an absolute positive distance between  and the limit  of .




\noindent  Let   be -. Consider the  time intervals . In view of the arguments in (\ref{8}), we similarly obtain

which leads to

for all  noticing (\ref{10}). Denoting  and applying  Gr\"{o}nwall's inequality we can further conclude that

for all , where .

Now that    is -, there must be a node  for which  is a -arc over the time  interval . Thus, there holds

for . Noticing the definition of -arcs and that , we invoke the Gr\"{o}nwall's inequality again and conclude from (\ref{11}) that

 where  and . This further allows us to apply the estimation of  over the interval  to node  for the interval  and obtain
 
for all .
Since   is -, the above analysis can be recursively applied to the intervals , for which nodes  can be found, respectively, with  such that
 
for 
This implies
 
when  is sufficiently small. However, we have known that non-increasingly there holds , and therefore such  does not exist, i.e.,   for all . The statement of Lemma \ref{lem:directed}. (i) is proved.

\subsection*{B. Proof of Lemma \ref{lem:directed}. (ii)}

Suppose  for all . Then for any , there exists  such that

Moreover, for the ease of presentation  we assume that  are distinct vectors since otherwise we can always combine the nodes with the same  as a cluster to be treated together in the following arguments.
Denote the angle between the two unit vectors  as . Then\footnote{Again, note that  for all .}

This leads to (cf., (\ref{8}))

with 
for all .
This will in turn give us

for all . Now we have

if   is -. Combining (\ref{16}) and (\ref{17}) we arrive at

which leaves  the only possibility since  can be arbitrary number. This proves Lemma \ref{lem:directed}. (ii).

\subsection*{C. Proof of Lemma \ref{lemma-bidirectional}}
 Again without loss of generality we can assume that   are distinct vectors. With (\ref{19}), we have

where . It is easy  to see from (\ref{19}) using  a contradiction argument that if   is -, there must hold
 
 Therefore, we conclude that  immediately and this proves the desired lemma.


\section*{Acknowledgement}
The authors  thank Zhiyong Sun for his generous help in the preparation of the numerical examples.


\begin{thebibliography}{99}
\bibitem{acc2016} G. Shi and B. D. O. Anderson, ``Distributed network flows solving linear algebraic equations," {\em American Control Conference}, Boston, MA, pp. 2864-2869, 2016.
\bibitem{god}
C. Godsil and G. Royle,
\newblock {\em Algebraic Graph Theory.}
\newblock New York: Springer-Verlag, 2001.

\bibitem{aubin} J. Aubin and A. Cellina. {\em Differential Inclusions}.  Berlin: Speringer-Verlag, 1984.

\bibitem{rock}
R. T. Rockafellar. {\em Convex Analysis.} New Jersey: Princeton University Press, 1972.

\bibitem{horn}  R. A. Horn  and C. R. Johnson. {\em  Matrix Analysis}. Cambridge
University Press. 1985.

\bibitem{magnusbook} M. Mesbahi and M. Egerstedt. {\em Graph Theoretic Methods in Multiagent Networks}. Princeton University Press, 2010.

\bibitem{dan} J. Danskin, ``The theory of max-min, with applications,"
{\em SIAM J. Appl. Math.}, vol. 14, 641-664, 1966.

\bibitem{jad03}
A. Jadbabaie, J. Lin, and A. S. Morse,
``Coordination of groups of mobile autonomous agents using nearest neighbor rules,"
{\em IEEE Trans. Autom. Control}, vol. 48, no. 6, pp. 988-1001, 2003.

\bibitem{xiao04}  L. Xiao and S. Boyd, ``Fast linear iterations for distributed averaging,"
{\em Syst. Contr. Lett.}, vol. 53, pp. 65-78, 2004.

\bibitem{saber04} R. Olfati-Saber and R. Murray, ``Consensus problems in the networks
of agents with switching topology and time delays," {\em  IEEE Trans.
Autom. Control}, vol. 49, no. 9, pp. 1520-1533, 2004.

\bibitem{ren05} W. Ren and R. Beard, ``Consensus seeking in multi-agent systems under dynamically changing interaction topologies," {\em IEEE Trans. Autom. Control}, vol. 50, no. 5, pp. 655-661, 2005.

\bibitem{martinez07} S. Martinez, J. Cort\'{e}s, and F. Bullo, ``Motion coordination with distributed
information," {\em IEEE Control Syst. Mag.}, vol. 27, no. 4, pp.
75-88, 2007.

\bibitem{kar12} S. Kar, J. M. F. Moura and K. Ramanan, ``Distributed parameter estimation in sensor networks:
nonlinear observation models and imperfect communication,"  {\em IEEE Trans.
Information Theory}, 58(6), pp. 3575-52, 2012.

\bibitem{Rabbat2010} A. G. Dimakis,  S. Kar,
J. M. F. Moura,  M. G. Rabbat,  and
A Scaglione,  ``Gossip algorithms for
distributed signal processing,"  {\em Proceedings of IEEE}, vol. 98, no. 11, pp. 1847-1864, 2010.

\bibitem{rabbat2004} M. Rabbat and R. Nowak, ``Distributed optimization in sensor networks,"
in {\em Proc. IPSN’04},  pp. 20-27, 2004.

\bibitem{nedic09} A. Nedi\'{c} and A. Ozdaglar, ``Distributed subgradient methods for multi-agent optimization,"
{\em IEEE Trans. Autom. Control}, vol. 54, no. 1, pp. 48-61, 2009.

\bibitem{jackson}B. Golub,  and M. O. Jackson,  ``Naive learning in social networks
and the wisdom of crowds,"  {\em American Economic Journal: Microeconomics}, 2(1): 112-149, 2010.

\bibitem{moreau05} L. Moreau, ``Stability of multiagent systems with time-dependent communication links," {\em IEEE Trans. Automat. Control}, 50, pp. 169–182, 2005.

\bibitem{lin07} Z. Lin, B. Francis, and M. Maggiore,
``State agreement for continuous-time coupled nonlinear systems," {\em SIAM J. Control Optim.}, vol. 46, no. 1, pp. 288-307, 2007.

\bibitem{jad08} A. Tahbaz-Salehi and A. Jadbabaie, ``A necessary and sufficient condition
for consensus over random networks," {\em IEEE Trans. Autom. Control},
vol. 53, no. 3, pp. 791-795, 2008.

\bibitem{shisiam}     G. Shi and K. H. Johansson, ``Robust consusus for continuous-time multi-agent dynamics," {\em SIAM J. Control Optim.}, vol. 51, no. 5, pp. 3673-3691, 2013.

\bibitem{nedic10} A. Nedi\'{c}, A. Ozdaglar and P. A. Parrilo, ``Constrained consensus and optimization in multi-agent networks," {\em IEEE Trans. Autom. Control}, vol. 55, no. 4, pp. 922-938, 2010.

\bibitem{shitac} G. Shi, K. H. Johansson and Y. Hong,
``Reaching an optimal consensus: dynamical systems that compute intersections of convex sets," {\em IEEE Trans.
Autom. Control},  vol. 58, no.3, pp. 610-622,  2013.

\bibitem{shiauto} G. Shi and  K. H. Johansson, ``Randomized optimal consensus of multi-agent systems,"  {\em Automatica}, 48(12): 3018-3030, 2012.

\bibitem{jvn49}  J. von Neumann, ``On rings of operators - reduction theory," {\em  Ann. of Math.}, 50(2): 401-485, 1949.

\bibitem{aron50} N. Aronszajn, ``Theory of reproducing kernels," {\em Trans. Amer. Math. Soc.},
vol. 68, no. 3, pp. 337-404, 1950.


\bibitem{gubin1967} L. G. Gubin, B. T. Polyak, and E. V. Raik, ``The method of projections for finding the common point of convex sets," {\em U.S.S.R. Computational Mathematics and Mathematical Physics}, 7:1-24, 1967.

\bibitem{deut83}  F. Deutsch, ``Rate of convergence of the method of alternating projections,"
in {\em Parametric Optimization and Approximation}, B. Brosowski
and F. Deutsch, Eds.,ol. 76, pp. 96-107, Basel, Switzerland:
Birkh\"{a}user, 1983.

\bibitem{B-B-SIAM}   H. H. Bauschke and J. M. Borwein,  ``On projection algorithms for solving convex
feasibility problems," {\em SIAM Review}, Vol. 38, No. 3, pp. 367-426,  1996.

\bibitem{anderson97}  C. Anderson, ``Solving linear eqauations on parallel distributed memory architectures by ex-trapolation," Technical Report, Royal Institute of Technology, 1997.

\bibitem{mehmood05} R. Mehmood and J. Crowcroft, ``Parallel iterative solution method of large sparse linear
equation systems," Technical Report, University of Cambridge, 2005.

\bibitem{lu09-1} J. Lu and C. Y. Tang, ``Distributed asynchronous algorithms for solving
positive definite linear equations over networks - Part II: wireless
networks. In {\em Proc first IFAC Workshop on Estimation and Control of
Networked Systems}, pages 258-264, 2009.

\bibitem{lu09-2} J. Lu and C. Y. Tang, ``Distributed asynchronous algorithms for solving
positive definite linear equations over networks - Part I: agent networks,"
In {\em Proc first IFAC Workshop on Estimation and Control of Networked
Systems}, pages 22-26, 2009.

\bibitem{liu13} J. Liu, S. Mou, and A. S. Morse, ``An asynchronous distributed algorithm
for solving a linear algebraic equation,"  In {\em Proceedings of the 2013 IEEE
Conference on Decision and Control}, pp. 5409-5414, 2013.

\bibitem{mou13} S. Mou and A. S. Morse, ``A fixed-neighbor, distributed algorithm for
solving a linear algebraic equation,"  In {\em Proceedings of the 2013 European
Control Confrence}, pages 2269-2273, 2013.


\bibitem{asuman14} C. E. Lee, A. Ozdaglar, and D. Shah, ``Solving systems of linear equations: locally and asynchronously," preprint  arXiv 1411.2647, 2014.

\bibitem{jadbabaie15}R. Tutunov, H. B. Ammar, and A. Jadbabaie,  ``A fast distributed solver for symmetric diagonally dominant linear equations,"  preprint  arXiv 1502.03158, 2015.




\bibitem{Morse-TAC15} S.  Mou,  J, Liu, and  A. S.  Morse, ``A distributed algorithm for solving a linear algebraic equation," {\em IEEE Trans. Autom. Control},  [Online] in press, 2015.

\bibitem{brian15} B. D. O. Anderson, S. Mou, A. S. Morse, and U. Helmke, ``Decentralized gradient algorithm for solution
of a linear equation," {\em Numerical Algebra, Control and Optimization}, to appear (preprint arXiv:1509.04538).



\bibitem{elia} J. Wang and N.  Elia,  ``Control approach to distributed optimization," {\em The 48th  Annual Allerton Conference}, pp. 557-561, 2010.


\bibitem{jmf} D.  Jakoveti\'{c}, J. Xavier and J. M. F. Moura, ``Cooperative convex optimization in networked
systems: augmented Lagrangian algorithms
with directed gossip communication," {\em IEEE Trans. Signal Processing}, vol. 59, no. 8, pp. 3889-3902, 2011.

\bibitem{nedic11}K. Srivastava and A. Nedi\'{c}, ``Distributed asynchronous constrained
stochastic optimization," {\em IEEE J. Select. Topics Signal Process.}, vol. 5,
no. 4, pp. 772-790, 2011.

\bibitem{Rabbat2012} K. I. Tsianos and M. G. Rabbat, ``Distributed strongly convex optimization," in the {\em 50th Annual Allerton Conference on Communication, Control,
and Computing}, pp. 593-600, 2012.

\bibitem{lu12} J. Lu and C. Y. Tang, ``Zero-gradient-sum algorithms for distributed convex optimization: the continuous-time case," {\em IEEE Trans. Automat. Contr.}, 57(9): 2348-2354, 2012.


\bibitem{cotes14} B. Gharesifard and  J. Cort\'{e}s, ``Distributed continuous-time convex optimization on weight-balanced digraphs," {\em IEEE Trans.  Automatic Control}, vol. 59, no. 3, pp. 781-786, 2014.

\bibitem{jakovetic14} D. Jakoveti\'{c}, J. M. F. Moura and J. Xavier, ``Fast distributed gradient methods," {\em IEEE
Trans. Automatic Control}, 59(5):  1131-1146, 2014.

\bibitem{tsi}
J. Tsitsiklis, D. Bertsekas, and M. Athans, ``Distributed asynchronous
deterministic and stochastic gradient optimization algorithms," {\em IEEE Trans. Autom. Control}, vol. 31, no. 9, pp. 803-812, 1986.

\bibitem{tsibook} D. Bertsekas and J. Tsitsiklis. {\em Parallel and Distributed Computation:
Numerical Methods}. Princeton, NJ: Princeton, 1989.

\end{thebibliography}

\end{document}
