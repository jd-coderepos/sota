\documentclass[a4paper,11pt]{article}






\bibliographystyle{plain}

\usepackage{xspace}
\usepackage{comment}
\usepackage{amsmath}
\usepackage{amsfonts}
\usepackage{amssymb}
\usepackage{amsthm}
\usepackage{fullpage}
\usepackage{graphicx}

\newtheorem{lemma}{Lemma}[section]
\newtheorem{proposition}[lemma]{Proposition}
\newtheorem{observation}[lemma]{Observation}
\newtheorem{theorem}[lemma]{Theorem}
\newtheorem{corollary}[lemma]{Corollary}

\theoremstyle{definition}
\newtheorem{definition}[lemma]{Definition}

\newcommand{\dist}{\delta}
\newcommand{\cost}{\mathrm{cost}}
\newcommand{\val}{\mathrm{value}}
\newcommand{\cP}{\mathcal{P}}
\newcommand{\cC}{\mathcal{C}}
\newcommand{\cS}{\mathcal{S}}
\newcommand{\cB}{\mathcal{B}}
\newcommand{\eps}{\varepsilon}






\title{On Pairwise Spanners\footnote{Partially supported by the ERC Starting Grant NEWNET 279352. This work was partially done while the third author was visiting IDSIA.}}



\author{
 Marek Cygan \\
  IDSIA, University of Lugano, Switzerland \\
\small  \texttt{marek@idsia.ch}  \\
\and
Fabrizio Grandoni \\
  IDSIA, University of Lugano, Switzerland \\
\small   \texttt{fabrizio@idsia.ch}  \\
\and
Telikepalli Kavitha \\
Tata Institute of Fundamental Research, India \\
\small \texttt{kavitha@tcs.tifr.res.in}
}


\begin{document}

\maketitle

\begin{abstract}
Given an undirected -node unweighted graph , a spanner with \emph{stretch function}  is a subgraph  such that, if two nodes are at distance  in , then they are at distance at most  in . Spanners are very well studied in the literature. The typical goal is to construct the sparsest possible spanner for a given stretch function. 

In this paper we study \emph{pairwise spanners}, where we require to approximate the - distance 
only for pairs  in a given set .
Such -spanners were studied before [Coppersmith,Elkin'05] only in the special case that  is the identity function, i.e. distances between relevant pairs must be preserved exactly (a.k.a. {\em pairwise preservers}). 

Here we present pairwise spanners which are at the same time sparser than the best known preservers (on the same ) and of the best known spanners (with the same ).
In more detail, for arbitrary , we show that there exists a -spanner of size  with . Alternatively, for any , there exists a -spanner of size  with . We also consider the relevant special case that there is a critical set of nodes , and we wish to approximate either the distances within nodes in  or from nodes in  to any other node. We show that there exists an -spanner of size  with , and an -spanner of size  with . All the mentioned pairwise spanners can be constructed in polynomial time.
\end{abstract}





\section{Introduction}
\label{intro}

Let  be an undirected unweighted graph. A subgraph  of  is a \emph{spanner} with stretch function  if, given any two nodes  at distance  in , the distance  between the same two nodes in  is at most . An  spanner is a spanner with stretch functions . ( and  are the \emph{multiplicative stretch} and \emph{additive stretch} of the spanner, respectively). If  the spanner is called \emph{multiplicative}, and if  the spanner is called \emph{purely-additive}. 
Spanners are very well studied in the literature (see Section \ref{sec:related}). The typical goal is to achieve the sparsest possible spanner for a given stretch function  \cite{BKMP05,BS03,DHZ97,E05,EP04,HZ96,PS89,RZ04,RMZ05,TZ06}.

In this paper we address the natural problem of finding (even sparser) spanners in the case that we want to approximately preserve distances only among a given subset  of pairs. More formally a \emph{pairwise spanner} on pairs , or -spanner for short, with stretch function  is a subgraph  such that, for any , . In particular, a classical (all-pairs) spanner is a -spanner. Pairwise spanners capture scenarios where we only (or mostly) care about some distances in the graph. 

To the best of our knowledge, pairwise spanners were studied before only in the special case that  is the identity function, i.e.  distances between relevant pairs have to be preserved exactly.  Coppersmith and Elkin \cite{CE05} call such spanners \emph{pairwise (distance) preservers}, and show that one can compute pairwise preservers of size (i.e., number of edges) . 

The authors left it as an open problem to study the {\em approximate} variants of these 
preservers, i.e. what we call pairwise spanners here. This paper takes the first step in answering this question. We show that (for suitable ) it is possible to achieve -spanners which are at the same time sparser than the preservers in \cite{CE05} (on the same set ) and than the sparsest known classical spanners (with the same stretch function). 








\subsection{Our Results and Techniques}

In this paper we present some polynomial-time algorithms to construct  -spanners for unweighted graphs. Our spanners are either purely-additive (i.e. ) or \emph{near-additive} (i.e.  for an arbitrarily small ). 
For arbitrary , we achieve the following main results (see Section \ref{section-pairwise}).
\begin{theorem}
\label{thm:pairwise} {\bf (near-additive pairwise)}
For any  and any , there is a polynomial time algorithm to compute a  -spanner of size . 
\end{theorem}
\begin{theorem}
\label{thm:pairwise2} {\bf (purely-additive pairwise)}
For any integer  and any , there is a polynomial time algorithm to compute a  -spanner of size\newline .
\end{theorem}

We also consider the relevant special case that all the pairs involve at least one node from a critical set . More precisely, we distinguish two types of such pairwise spanners: in \emph{subsetwise spanners} (see Section \ref{section-s-spanners}) we wish to approximate distances \emph{between} nodes in , i.e. ; in \emph{sourcewise spanners} (see Section \ref{section-sourcewise}) we wish to approximate distances \emph{from} nodes in , i.e. . We obtain the following improved results for the mentioned cases. 

\begin{theorem}
\label{thm:s-spanners} {\bf (subsetwise)}
For any , there is a polynomial time algorithm to compute a  -spanner of size . 
\end{theorem}
\begin{theorem}
\label{thm:sourcewise} {\bf (sourcewise)}
For any  and any integer , there is a polynomial time algorithm to compute a  -spanner of size  .
\end{theorem}


In particular, by choosing , we obtain a  sourcewise spanner of size  , and a 
 pairwise spanner of size
.

All our spanners rely on a path-buying strategy which was first exploited in the  spanner by Baswana et al. \cite{BKMP05}.
The high-level idea is as follows. There is an initial clustering phase, where we compute a suitable clustering of the nodes, and an associated subset of edges which are added to the spanner. Then there is a path-buying phase, where we consider an appropriate sequence of paths, and decide whether to add or not each path in the spanner under construction\footnote{In the spanner from Theorem \ref{thm:pairwise} there is also a final step where we add a  multiplicative -spanner.}. In particular, each path has a \emph{cost}  which is given by the number of edges of the path not already contained in the spanner, and a \emph{value} which measures \emph{how much} the path helps to satisfy the considered set of constraints on pairwise distances. If the value is sufficiently larger than the cost, we add the considered path to the spanner, otherwise we do not.

In more detail, all our pairwise spanners exploit the same clustering phase. We compute a partition  of a subset of the nodes, and call \emph{unclustered} the remaining nodes . The initial value of the spanner is , where  contains all the edges of  but possibly a subset of the \emph{inter-cluster} edges (with endpoints in two different clusters).
The common clustering phase is described in Section \ref{section-clustering}.



During the path-buying phase we add to the spanner some extra \emph{inter-cluster} edges. Here we need to finely tune the sequence of paths that we consider, and also the definition of value of a path. In our subsetwise and sourcewise spanners the value of a path  reflects the number of pairs , where  is the endpoint of some pair and  is a cluster, such that adding  to the current spanner decreases the distance between  and (the closest node in) . In the remaining pairwise spanners, we use a similar notion of value, but considering the distance between pairs of clusters . 

The sequence of paths used in our subsetwise spanner and near-additive pairwise spanner is simply given by the shortest paths among the relevant pairs. This naturally generalizes the set of paths considered in \cite{BKMP05}. However, for the sourcewise spanner and the purely-additive pairwise spanner we need to consider a carefully constructed sequence of paths, which includes slightly suboptimal paths. In more detail, we start with the set of shortest paths between the relevant pairs. Then, for each such path , if the cost of  is sufficiently smaller than its value, we include  in the spanner. Otherwise, we replace  with a \emph{slightly longer} path  between the same endpoints which is \emph{much cheaper}, and iterate the process on . After a small number of iterations, the considered path becomes cheap enough and hence we include it in the spanner. 
This (non-trivial) iterative construction of candidate paths during the path-buying phase is probably the main algorithmic contribution of this paper. 


\subsection{Related Work}
\label{sec:related}


Graph spanners were introduced by Peleg and Schaffer~\cite{PS89} in 1989. Spanners have been extensively 
studied since then, and there are numerous applications involving spanners, such as algorithms for 
approximate shortest paths~\cite{ABCP98,C93,E05}, labeling schemes~\cite{P00,GPPR01}, 
approximate distance oracles~\cite{TZ01,BS04,BK06}, 
routing~\cite{AP92,C01,CW04}, and network design~\cite{PU89}. 

There are several algorithms for computing multiplicative and additive spanners in weighted and 
unweighted graphs. In unweighted graphs, for any integer , Halperin and Zwick~\cite{HZ96} 
gave a linear time algorithm to compute a multiplicative -spanner of size , 
where  is the number of vertices. 
Note that for  one obtains a spanner with multiplicative stretch  and with size : we will use this type of spanner in Theorem \ref{thm:pairwise}. 
Analogous results are also known for weighted graphs~\cite{BS03,RMZ05,RZ04}.


The first purely-additive spanner (for unweighted graphs) is due to Dor et al.~\cite{DHZ97}. They describe a  spanner of 
size . This was subsequently improved to ~\cite{EP04}. Note that our subsetwise spanner from Theorem \ref{thm:s-spanners} generalizes this result: in particular, it has the same stretch function and is sparser whenever .
Baswana et al. \cite{BKMP05} describe a -spanner of size . Whenever  for some constant , we achieve an asymptotically sparser pairwise spanner with constant additive stretch (depending on ). The same holds for our sourcewise spanner if .






The result in \cite{HZ96} shows an elegant trade-off between the size of the spanner and its multiplicative stretch. No such trade-off is known for purely-additive spanners. In particular, the spanner in \cite{BKMP05} is the sparsest known purely-additive spanner. Theorems \ref{thm:pairwise2} and \ref{thm:sourcewise} show a non-trivial trade-off between the size and additive stretch of pairwise spanners. 

There have also been several results on near-additive spanners~\cite{EP04,E05,TZ06}. For example, there is a -spanner of size  for any 
~\cite{EP04}. Our pairwise spanner from Theorem \ref{thm:pairwise} has the same stretch function, and is sparser for . 

Compared to the preservers in \cite{CE05}, we achieve sparser  pairwise spanners with additive stretch  for , and a sparser subsetwise spanners for . 
Interestingly, our sourcewise spanners are always sparser than the pairwise preservers from \cite{CE05}. 








\section{Clustering}
\label{section-clustering}

A \emph{clustering} of a graph  is a collection  of pairwise disjoint subsets of nodes . Note that we do not require  to span all the nodes : we call \emph{unclustered} the nodes .

We will crucially exploit the following construction of a clustering  and of an associated \emph{cluster subgraph} .

\begin{lemma}
\label{lem:clustering}
There is a polynomial time algorithm which,
given  and a graph , computes a clustering  with at most  clusters and a subgraph  of size  such that:
\begin{enumerate}
\item \label{prop:clustering:prop2} {\bf (missing-edge property)} If an edge  is absent in , then  and  belong to two different clusters.
  \item \label{prop:clustering:prop3} {\bf (cluster-diameter property)} The distance in  between any two vertices of the same cluster is at most .
\end{enumerate}
\end{lemma}
\begin{proof}
Let  be the set of nodes which are not yet clustered (initially we set .
As long as there exists a vertex  with at least  neighbors in , let  contain exactly  arbitrary
neighbors of  in .
Add  to , set  and add to  
all the edges of  with both endpoints in . 
When no node  satisfies the mentioned property, we stop creating new clusters and add to  all the edges incident to the final set of unclustered nodes .

By construction, clusters are pairwise disjoint. Each time we create a new cluster, the size of  decreases by at least , hence there cannot be more than  clusters. Any two nodes in the same cluster  have some common neighbor  in , hence Property \ref{prop:clustering:prop3} is satisfied. By construction, all the edges incident to unclustered nodes plus the intra-cluster edges (with both endpoints in the same cluster) belong to , which implies Property \ref{prop:clustering:prop2}.

It remains to bound the number of edges of . Each time we create a new cluster, the number of edges of  grows by at most : this gives  edges altogether. When we stop creating clusters, each (clustered or unclustered) node  has at most  neighbors in : consequently the number of edges incident to unclustered nodes that we add at the end of the procedure is at most .  
\end{proof}





The following technical lemma turns out to be useful in the remaining sections.
\begin{lemma}
\label{lem:num-clusters}
Let  and  be constructed with the procedure from Lemma \ref{lem:clustering} w.r.t. a given graph . If the shortest path  in  between any two nodes  contains  edges that are absent in , 
then there are at least  clusters of 
having at least one vertex on .
\end{lemma}

\begin{proof}
We prove the lemma by counting pairs , where  is an edge of 
absent in  and  is one of the endpoints of : let  be the set of such pairs.
Since  contains  edges that are absent in  there are exactly 
pairs in  (each edge  belongs to two pairs: 
 and ).
We say that a cluster  {\em owns} a pair  if .
By the missing-edge property, each edge  of  absent in  has both endpoints clustered,
hence each pair of  is owned by some cluster.

Let us assume that there are  clusters of  having
at least one vertex on .
By the cluster-diameter property any cluster  contains at most  vertices on ,
since otherwise  would not be a shortest path between  and .
However, if a cluster  contains exactly  vertices on ,
those have to be consecutive vertices  of , since  is a shortest path and we know by the cluster-diameter property that there is a path of length at most 2 between every 
pair in . By the missing-edge property
both edges  and  are present in , and consequently 
owns at most two pairs of .
Clearly if a cluster  contains at most  vertices on ,
then it owns at most  pairs of .
Therefore each cluster owns at most  pairs of :  
since  has  pairs we have .
\end{proof}

\section{Subsetwise Spanners}
\label{section-s-spanners}

In this section we present our algorithm to compute a subsetwise spanner, and prove Theorem~\ref{thm:s-spanners}. 

Our algorithm 
consists of two main phases: a clustering phase and a path-buying phase. In the clustering phase we invoke Lemma~\ref{lem:clustering} 
and obtain a cluster subgraph  of  of size ,
together with a set  of at most  clusters. The value of  will be defined later.

In the path-buying phase we proceed as follows. 
Initially set  and let ,  denote
the set of  shortest paths between all pairs of vertices in . We let  denote the endpoints of .
Next, we iterate over the paths  for . 
To determine which paths are affordable, we define the functions 
and :
\begin{itemize}
\item let  be the number of edges of  that are absent in 
\item let  be the number of pairs , where  and 
 is a cluster, such that  contains at least one vertex of 
and the distance between  and  in the graph 
is strictly greater than the distance between  and  in , 
i.e., .
\end{itemize}





Our path-buying strategy is as follows. If 
 
then we buy the path , that is we set  
(in words,  is given by  plus the edges of  not in ).
Otherwise (i.e., ), we do not buy 
and set .
The subsetwise spanner is given by .












The next two lemmas bound the stretch and the size of the constructed spanner , respectively

\begin{lemma}
\label{lem:s-1}
For any ,  .
\end{lemma}

\begin{proof}
Clearly the claim holds if our algorithm bought the path ,
hence we assume .
Let , that is there are exactly 
edges of  which are not present in the graph .
By Lemma~\ref{lem:num-clusters} there are at least  clusters having
at least one vertex on .
If there is no cluster  among them such that 
 and
,
then all these clusters would contribute to  (either with  or with  or both)
which leads to a contradiction, because .


Thus there is a cluster  having a vertex of  such that 
 and
. This implies:

where the first inequality is because  is a subgraph 
of ,
the second inequality holds since any two vertices of  are at distance at most two in  (by the cluster-diameter property)
and the last inequality follows from the assumption that  contains a vertex of .
\end{proof}

\begin{lemma}
\label{lem:s-2}
For  such that  
the graph  contains  edges.
\end{lemma}

\begin{proof}
The clustering phase produces a graph with  edges.
Let  be the set of paths bought in the path-buying phase.
The total number of edges that appear in  and do not appear in 
is equal to , which is 
upper bounded by .
Observe that after the first contribution of a pair 
to the above sum, the distance between  and  is at most ,
hence each pair  can contribute to the sum at most  times.
Therefore the total number of edges added in the second phase of our algorithm
is upper bounded by .
\end{proof}

The proof of Theorem~\ref{thm:s-spanners} follows from Lemmas~\ref{lem:s-1} and \ref{lem:s-2}.



\section{Sourcewise spanners}
\label{section-sourcewise}

In this section we present our algorithm to compute a sourcewise spanner from sources , and prove Theorem \ref{thm:sourcewise}. 

Our algorithm again consists of two phases, where the first 
is a clustering phase and the second is a path-buying phase.
The clustering phase is as in the algorithm from previous section, for a proper value of  to be defined later. Let  and  be the resulting clustering and cluster subgraph. 


At the start of the second phase we set 
and define  as the 
set of shortest paths between any two vertices of 
such that at least one of them belongs to .
Let us assume that the path  is a shortest path 
between  and .
Next, we iterate over paths  for .
For a given  we are going to define paths , where ,
maintaining the following invariants:\smallskip

\begin{itemize}
  \item[(i)]  is a path between  and  in  of length at most ,
  \item[(ii)] any cluster  contains at most three vertices of ,
  \item[(iii)] , where  is the number of edges of  absent in ,
and .
\end{itemize}

\smallskip\noindent Our algorithm will buy exactly one path  for , which will ensure 
(by Invariant (i)) that in , the distance between  and  is at most .

We set . 
Observe that for , Invariant (i) is trivially satisfied,
Invariant (ii) is satisfied by the cluster-diameter property  
(otherwise  would not be a shortest path),
and Invariant (iii) is satisfied because there are at most  clusters in 
and consequently by Lemma~\ref{lem:num-clusters} . 

Say we have constructed , where .
Let us define the function  as the number of clusters  
such that  contains a vertex of  and the distance
between  and  in  is strictly greater than
the distance between  and  in , i.e., .
Now we check the condition 
 

If that is the case, then we buy the path . That is,  is set to .
We ignore the remaining values of  and proceed with the next value of .
Else we construct  as follows:


Let  be the longest suffix of  containing exactly  edges that are absent in .
Observe that the first node of  is clustered:
by the maximality of , the edge  of  preceding 
is absent in , and hence both the endpoints of  (one of which is the first node of ) are clustered by the missing-edge property of . 
Consequently at least  
vertices of  are clustered, as  contains  edges
absent in  and the endpoints of these edges are clustered. 

By Invariant (ii) there are at least  clusters in 
having at least one vertex of . Since we did not buy , there
exists a cluster  containing a vertex  of  such that the distance between  
and  in  is at most the distance between  and  in . We construct the path  by taking
a shortest path in  from  to the closest node , then we add a path of length at most two between  and  (which exists in  hence in  by the cluster-diameter property), and finally add the suffix of  starting at  (see Fig.~\ref{fig1}).




Let us show that  maintains the invariants. Note that by construction, Invariant (i) is satisfied, since the length of  is at most the length of  plus .
Then, as long as there is a cluster  containing at least four vertices on ,
we let ,  be the vertices of  closest to  and  respectively.
Note that there are at least three edges on  between  and , hence we
can replace the subpath of  by adding the at most two edges of 
guaranteed by the cluster-diameter property.
Consequently, Invariant (ii) is satisfied. Moreover, by the choice of , Invariant (iii) is also satisfied.
This finishes the construction of . 

\begin{figure}[t]
\begin{center}
\includegraphics{fig1.pdf}
\end{center}
\caption{The solid edges represent a path , while the dashed edges
  denote the new prefix of the path .}
\label{fig1}
\end{figure}

Observe that by Invariant (iii) we have : since 
 has only integral values, it has to be that ,
which ensures that we buy a path  for some .

Finally, as our spanner  we take .






\begin{lemma}\label{lem:stretchSourcewise}
For any pair ,  .
\end{lemma}
\begin{proof}
From the above discussion, we buy at least one path  for some . By Invariant (i), the length of the latter path is  at most the length of the shortest path  between  and  plus .
\end{proof}


\begin{lemma}
\label{lem:sizeSourcewise}
For  such that , the subgraph  contains \linebreak  edges. 
\end{lemma}
\begin{proof}
To bound the size of 
we recall that in the first phase we have inserted  edges.
Let  be the index of a path  bought for a given .
We claim, that any cluster  contributes to  of at most
 bought paths.
This holds because when for  a supported path is bought 
the distance between  and  is at most  greater than 
the distance between  and  in : otherwise one could shorten 
by more than , obtaining a contradiction with Invariant (i).
Therefore the total number of edges added during the second phase is upper bounded by
,
since each cluster  supports at most  bought paths. The claim follows.
\end{proof}
The proof of Theorem \ref{thm:sourcewise} follows from Lemmas \ref{lem:stretchSourcewise} and \ref{lem:sizeSourcewise}. 




\section{Pairwise spanners}
\label{section-pairwise}

In this section we present our pairwise spanners for arbitrary . We start with a near-additive spanner (see Section \ref{section-pairwise-near}) and then present a purely-additive spanner (see Section \ref{section-pairwise-pure}). In both cases we let  denote the set of pairs, .



\subsection{A Near-Additive Pairwise Spanner}
\label{section-pairwise-near}

Our algorithm to construct the near-additive -spanner from Theorem \ref{thm:pairwise} consists of three phases. First, 
we use Lemma~\ref{lem:clustering} with the value of 
to be determined later, obtaining a cluster subgraph  of 
of size  together with a set  of at most  clusters.

At the start of the second phase we set  and consider the set of paths  , where  
is a shortest path between  and  in .
Next we iterate over the paths  for  .
By  we denote the number of edges of 
absent in , and by  we denote the number of 
pairs of clusters , such that both  and  contain at
least one vertex of  and .
For a given  if

then we buy , that is we set . Otherwise we set .

In the third phase we add to  the multiplicative  spanner of size  given in \cite{HZ96}: this way we obtain the desired spanner .






In the following two lemmas we bound the stretch and size of , respectively.

\begin{lemma}
\label{lem:stretchPairwise1}
For each , .
\end{lemma}

\begin{proof}
Clearly we can assume that the path  was not bought in the second phase,
since otherwise the claim trivially holds.
Therefore 


Let 
and let  be the set of all indices
 such that  is clustered.
Observe that if , then by the missing-edge property
the whole path  is present in , and hence the claim holds.
Therefore denote , where  and .
Let  be two indices, such that ,  (for some ),
 and the value of  is maximized.
Note that such a pair of indices  always exists, since we can take .

Let . Observe that any cluster  contains at most  vertices of ,
since otherwise by the cluster-diameter property  
would not be a shortest - path.
Therefore there are at least  clusters  having at least one vertex in the set ,
and at least  clusters  having at least one vertex in the set .
However, each of the at least  pairs of clusters in  contributes to  
since the difference between indices in the corresponding set is at least 
.
Therefore  and hence
.

The latter bound on  is sufficient to prove the claim. In fact, consider the path between  and  in  obtained by concatenating the following paths
(as illustrated in Fig.~\ref{fig2}):
\begin{itemize}
  \item A shortest path in  from  to . Note that in the prefix of  between  and  there
  are  clustered nodes and hence at most  edges absent in  (by the missing-edge property). Since  contains the -spanner added in the third phase, each missing edge can be replaced by at path of length . Consequently, there is a path from  to  of length at most  in .
  \item A shortest path in  from  to . Let  be the clusters containing  and  respectively.
  We know that in  there is a path from  to  of length at most ,
  which can be extended to a path between  and  in  by adding at most  edges
  (by the cluster-diameter property).
  \item A shortest path in  from  to . Observe that in the suffix of  between  and 
  there are at most  edges absent in  by the same argument as above. Hence, thanks to the -spanner added in the third phase, there is a path from  to  of length at most  in .
\end{itemize}
The resulting path is of length at most 
 
where the last inequality
follows from  together with .
\begin{figure}[t]
\begin{center}
\includegraphics{fig2.pdf}
\end{center}
\caption{Illustration of the three paths concatenation in the proof of Lemma~\ref{lem:stretchPairwise1}.}
\label{fig2}
\end{figure}
\end{proof}






\begin{lemma}
\label{lem:sizePairwise1}
For  such that  the 
size of  is .
\end{lemma}

\begin{proof}
The clustering phase gives  edges, which matches the desired bound on .
Let  be the set of paths  bought in the path-buying phase.
Observe, that if a pair of clusters  contributes to  of a bought path ,
then when  is bought we have , since otherwise
the subpath of  between  and  might be shortened (by the cluster-diameter property). It follows that each pair of clusters contributes at most  times to , and hence 

The total number of edges added in the second phase is therefore upper bounded by 



Finally, in the last phase we insert only  edges when adding the -spanner.
\end{proof}

Having Lemmas~\ref{lem:stretchPairwise1} and \ref{lem:sizePairwise1}, the proof of Theorem~\ref{thm:pairwise} follows.



\subsection{A Purely-Additive Pairwise Spanner}
\label{section-pairwise-pure}

In this section we describe an algorithm to compute the purely-additive -spanner from Theorem \ref{thm:pairwise2}. To that aim we will combine ideas from the proofs of Theorems \ref{thm:pairwise} and \ref{thm:sourcewise}.

Our algorithm consists of the usual clustering phase (for an appropriate parameter ) followed by a path-buying phase that we next describe. 

Let  and  be the clustering and the associated cluster graph. At the beginning of the path-buying phase, we set  and consider the set ,
where  is a shortest path between  and  in .
Next we iterate over the paths  for .
For a given  we are going to define paths , where ,
maintaining the following invariants:\smallskip

\begin{itemize}
  \item[(i)]  is a path between  and  in  of 
  length at most ,
  \item[(ii)] any cluster  contains at most three vertices of ,
  \item[(iii)] , where  is the number of edges of  absent in ,
and .
\end{itemize}

\smallskip\noindent Our algorithm will buy exactly one path , which will ensure by Invariant (i) that 
in  the distance between  and  is at most .
By  let us denote the number 
of pairs of clusters , such that both  and  contain at
least one vertex of  and .

We set .
Observe that for  Invariant (i) is trivially satisfied,
Invariant (ii) is satisfied by the cluster-diameter property 
(otherwise  would not be a shortest path),
and Invariant (iii) is satisfied because there are at most  clusters in 
and consequently by Lemma~\ref{lem:num-clusters}, . 

Say we have constructed , where .
If

then we buy the path ,
i.e. as  we take the union of  and ,
ignore remaining values of  and proceed with the next value of .
Otherwise (i.e., ), 
we construct a path  as follows:


Let 
and let  be the set of all indices
 such that  is clustered.
Observe that if , then by the missing-edge property
the whole path  is present in , and hence it 
is of zero cost, which contradicts the assumption
.
Therefore denote , where  and . Let  be two indices,
such that ,  (for some ),
 and the value of  is maximized.
Note that such a pair of indices  always exists, since we can take .

Let . By Invariant (ii) there are at least  clusters  having at least 
one vertex in the set ,
and at least  clusters  having at least one vertex 
in the set .
However, each of the at least  pairs of clusters in  contributes to  since the difference between indices in the corresponding set is at least 
.
Therefore 



We construct the path  by appending the following three
paths , , and :
\begin{itemize}
  \item As  we take the prefix of  from  to .
  Note that this prefix contains  clustered nodes and hence at most  edges absent 
  in  (by the missing-edge property of ).
  \item Let  be the clusters containing  and  respectively.
  We know that in  there is a path from  to  of length at most ,
  which can be extended to a path  between  and  in  by adding at most  edges
  (by the cluster-diameter property).
  \item As  we take the suffix of  from  to ,
  which contains at most  edges absent in  by the same argument as above.
\end{itemize}
Observe that  contains at most  edges absent in , hence by \eqref{eqn:pairwise2} we ensure Invariant (iii).
Moreover the length of  is at most the length of  plus , which ensures Invariant (i).
In order to ensure Invariant (ii), as long as there exists a cluster 
containing at least  vertices of  we let
 and  be two such vertices closest to  and  on  respectively
and replace the subpath of  between  and  (which is of length at least three)
by a path of length at most two in  (which exists by  the cluster-diameter property).

Observe that by Invariant (iii) we have , hence  which ensures that we buy a path  for some . Finally, as our spanner  we take .




\begin{lemma}\label{lem:stretchPairwise2}
For each , .
\end{lemma}
\begin{proof}
From the above discussion,  contains at least one path  between  and  for some . The claim follows by Invariant (i). 
\end{proof}



\begin{lemma}\label{lem:sizePairwise2}
For  such that  the
size of  is\newline .
\end{lemma}

\begin{proof}
The clustering phase gives  edges, 
which matches the desired bound on .
Let  be the index of a path  bought for a given .
We claim, that any pair of clusters contributes 
to  of at most  bought paths.
Observe, that if a pair of clusters  contributes to ,
then when  is bought we have
, since otherwise
the subpath of  between  and  might be shortened 
by more than , contradicting Invariant (i).
The total number of edges added in the second phase is upper bounded by 

By substituting  and  the claim follows.
\end{proof}

Theorem~\ref{thm:pairwise2} follows from Lemmas \ref{lem:stretchPairwise2} and \ref{lem:sizePairwise2}


\section{Conclusions}

We considered a natural extension to the problem of computing a sparse spanner in an undirected unweighted graph. Along with the input
graph ,  a subset  of relevant pairs
of vertices is also given here and we seek a sparse subgraph  of  
such that for every 
pair  in , the - distance  in the subgraph is close to the 
- distance  in . We showed sparse subgraphs  where  is a small additive or near-additive stretch away from . 

The pairwise preservers in \cite{CE05} are at the same time more accurate and sparser than our spanners for small enough values of . In particular, in that range of values of  the authors exploit a construction which does not seem to benefit from allowing a larger stretch. The authors also present lower bounds on the size of any preserver, however it is unclear whether those lower bounds extend to the case of pairwise spanners (where distances have to be approximated rather than preserved). Obtaining sparser pairwise spanners for very small , if possible, is an interesting open problem.











\begin{thebibliography}{20}


\bibitem{ABCP98}
B. Awerbuch, B. Berger, L. Cowen, and D. Peleg.
Near-linear time construction of sparse neighborhood covers.
{\em SIAM Journal on Computing}, 28(1):263-277, 1998.

\bibitem{AP92}
B. Awerbuch and D. Peleg.
Routing with polynomial communication-space trade-off.
{\em SIAM Journal on Discrete Math.}, 5(2):151-162, 1992.

\bibitem{BK06}
S.~Baswana and T.~Kavitha.
Faster algorithms for all-pairs approximate shortest paths in undirected graphs.
{\em SIAM Journal on Computing}, 39(7):2865-2896, 2010.

\bibitem{BKMP05}
S.~Baswana, T.~Kavitha, K.~Mehlhorn, S.~Pettie.
Additive Spanners and -Spanners.
{\em ACM Transactions on Algorithms}, 7(1): 5, 2010.

\bibitem{BS03}
S. Baswana and S. Sen.
A simple linear time algorithm for computing a -spanner of  size in weighted graphs.
In {\em Proc. 30th Int. Colloq. on Automata, Languages, and Programming (ICALP)}, pages 384-396, 2003.

\bibitem{BS04}
S. Baswana and S. Sen.
Approximate distance oracles for unweighted graphs in  time.
In {\em Proc. 15th ACM-SIAM Symposium on Discrete Algorithms (SODA)}, pages 271-280, 2004.


\bibitem{C93}
E. Cohen.
Fast algorithms for constructing -spanners and paths of stretch .
In {\em Proc. 34th IEEE Symp. on Foundations of Computer Science (FOCS)},  pages 648-658, 1993.

\bibitem{CE05}
D. Coppersmith and M. Elkin.
Sparse source-wise and pair-wise distance preservers.
In {\em Proc. 16th ACM-SIAM Symposium on Discrete Algorithms (SODA)}, pages 660-669, 2005.

\bibitem{C01}
L. J. Cowen.
Compact routing with minimum stretch.
{\em Journal of Algorithms}, 28:170-183, 2001.

\bibitem{CW04}
L. J. Cowen and C. G. Wagner.
Compact roundtrip routing in directed networks.
{\em Journal of Algorithms}, 50(1):79-95, 2004.

\bibitem{DHZ97}
D. Dor, S. Halperin, and U. Zwick.
All-pairs almost shortest paths.
{\em SIAM Journal on Computing}, 29(5):1740-1759, 2004.

\bibitem{E05}
M. Elkin.
Computing almost shortest paths.
In {\em ACM Transactions on Algorithms}, 1(2):283-323, 2005.

\bibitem{EP04}
M. Elkin and D. Peleg.
-spanner construction for general  graphs.
{\em SIAM Journal on Computing}, 33(3):608-631, 2004.

\bibitem{GPPR01}
C. Gavoille, D. Peleg, S. Perennes, and R. Raz.
Distance labeling in graphs.
In {\em Proc. 12th ACM-SIAM Symposium on Discrete Algorithms (SODA)}, pages 210-219, 2001.


\bibitem{HZ96}
S. Halperin and U. Zwick.
Unpublished result, 1996.

\bibitem{P00}
D. Peleg.
Proximity-preserving labeling schemes.
{\em Journal of Graph Theory}, 33(3):167-176, 2000.


\bibitem{PS89}
D. Peleg and A. A. Schaffer.
Graph Spanners.
{\em Journal of Graph Theory}, 13:99-116, 1989.

\bibitem{PU89}
D. Peleg and J. D. Ullman.
An optimal synchronizer for the hypercube.
{\em SIAM Journal on Computing}, 18:740-747, 1989.

\bibitem{RZ04}
L. Roditty and U. Zwick.
On dynamic shortest paths problems.
In {\em Proc. 12th Annual European Symposium on Algorithms (ESA)}, pages 580-591, 2004.


\bibitem{RMZ05}
L. Roditty, M. Thorup, and U. Zwick.
Deterministic constructions of approximate distance oracles and spanners.
In {\em Proc. 32nd Int. Colloq. on Automata, Languages, and Programming (ICALP)}, pages 261-272, 2005.

\bibitem{TZ01}
M. Thorup and U. Zwick.
Approximate Distance Oracles.
{\em Journal of the ACM}, 52(1):1-24, 2005.


\bibitem{TZ06}
M. Thorup and U. Zwick.
Spanners and emulators with sublinear distance errors.
In {\em Proc. 17th ACM-SIAM Symposium on Discrete Algorithms (SODA)}, pages 802-809, 2006.


\end{thebibliography}
\end{document}
