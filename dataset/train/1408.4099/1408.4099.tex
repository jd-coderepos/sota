\documentclass{cccg14}
\usepackage{graphicx,amssymb,amsmath}
\usepackage{xspace}
\usepackage{refcount}
\usepackage{amssymb}
\usepackage{amsmath}
\usepackage{xcolor}







\graphicspath{{figures/}}



\newcommand{\etal}{\emph{et al.}\xspace}
\newcommand{\cyao}{\ensuremath{cY(\theta)}\xspace}
\newcommand{\cyaoOneTwenty}{\ensuremath{cY(2\pi/3)}\xspace}
\newcommand{\cyaopi}{\ensuremath{cY(\pi)}\xspace}
\newcommand{\cyaoangle}[1]{\ensuremath{cY(#1)}\xspace}
\newcommand{\creflect}{\ensuremath{c^*}\xspace}
\newcommand{\vreflect}{\ensuremath{v^*}\xspace}
\newcommand{\sector}{disk sector\xspace}
\newcommand{\ray}[1]{\ensuremath{\vec{#1}}\xspace}
\newcommand{\spanningRationCyao}{\ensuremath{6.0411}\xspace}

\newcommand{\red}[1]{\textcolor{red}{#1}}



\title{Continuous Yao Graphs}

\author{
Luis Barba\thanks{School of Computer Science, Carleton University. Email: \texttt{jit@scs.carleton.ca, jdecaruf@cg.scs.carleton.ca, andre@cg.scs.carleton.ca, sander@cg.scs.carleton.ca}. Research supported in part by NSERC and Carleton University's President's 2010 Doctoral Fellowship.}\ \thanks{D\'epartement d'Informatique, Universit\'e Libre de Bruxelles. Email:  \texttt{lbarbafl@ulb.ac.be}}
\and
Prosenjit Bose\footnotemark[1]
\and
Jean-Lou De Carufel\footnotemark[1]
\and
Mirela Damian\thanks{Department of Computing Sciences, Villanova University. Email: \texttt{mirela.damian@villanova.edu}. Research supported by NSF grant CCF-1218814.}
\and
Rolf Fagerberg\thanks{Department of Computer Science, University of Southern Denmark. Email: \texttt{rolf@imada.sdu.dk}.}
\and
Andr\'e van Renssen\footnotemark[1]
\and
Perouz Taslakian\thanks{School of Science and Engineering, American University of Armenia. Email: \texttt{perouz.taslakian@ulb.ac.be}.}
\and
Sander Verdonschot\footnotemark[1]
}

\index{Barba, Luis}
\index{Bose, Prosenjit}
\index{De Carufel, Jean-Lou}
\index{Damian, Mirela}
\index{Fagerberg, Rolf}
\index{van Renssen, Andr\'e}
\index{Taslakian, Perouz}
\index{Verdonschot, Sander}



\begin{document}
\thispagestyle{empty}
\maketitle

\begin{abstract}
In this paper, we introduce a variation of the well-studied Yao graphs.
Given a set of points  and an angle ,
we define the \emph{continuous Yao graph}  with vertex set  and angle  as follows. For each , we add an edge from  to  in  
if there exists a cone with apex  and aperture  such that  is the closest point to  inside this cone.

We study the spanning ratio of \cyao for different values of .
Using a new algebraic technique, we show that  is a spanner when . We believe that this technique may be of independent interest. We also show that  is not a spanner, and that  may be disconnected for .

\end{abstract}

\section{Introduction}
Let  be a set of points in the plane.
The complete geometric graph with vertex set  has a straight-line edge connecting each pair of points in . 
Because the complete graph has quadratic size in terms of number of edges, several methods for ``approximating'' this graph with a graph of linear size have been proposed. 

A geometric -spanner  of  is a spanning subgraph of the complete geometric graph of  with the property that for all pairs of points  and  of , the length of the shortest path between  and  in  is at most  times the Euclidean distance between  and .

The \emph{spanning ratio} of a spanning subgraph is the smallest  for which this subgraph is a -spanner. For a comprehensive overview of geometric spanners and their applications, we refer the reader to the book by Narasimhan and Smid \cite{NS06}.

A simple way to construct a -spanner is to first partition the plane around each point  into a fixed number of cones\footnote{The orientation of the cones is the same for all vertices.} and then add an edge connecting  to a closest vertex in each of its cones. These graphs have been independently introduced by Flinchbaugh and Jones~\cite{flinchbaugh1981strong} and Yao~\cite{yao1982constructing}, and are referred to as \emph{Yao graphs} in the literature.
It has been shown that Yao graphs are good approximations of the complete geometric graph
\cite{clarkson1987approximation,althofer1993sparse,bose2004approximating,bose2012piArxiv,damian2012yao,bose2012pi,el2009yao,barba2014new}.

We denote the Yao graph defined on  by , where  is the number of cones, each having aperture .
Clarkson~\cite{clarkson1987approximation} was the first to remark that  is a -spanner in 1987.
Alth{\"o}fer~\etal~\cite{althofer1993sparse} showed that for every , there is a  such that  is a -spanner. 
For , Bose~\etal~\cite{bose2004approximating} showed that  is a geometric spanner with spanning ratio at most . 
This was later strengthened to show that for ,  is a -spanner~\cite{bose2012piArxiv}. 
Damian and Raudonis~\cite{damian2012yao} proved a spanning ratio of  for , which was later improved by Barba~\etal to ~\cite{barba2014new}. In~\cite{barba2014new} the authors also improve the spanning ratio of  for all odd values of  to . In particular, they show an upper bound on the spanning ratio for  of .
Bose~\etal~\cite{bose2012pi} showed that  is a -spanner. For , El Molla~\cite{el2009yao} showed that there is no constant  such that  is a -spanner. 

Yao graphs are based on the implicit assumption that all points use identical cone orientations with respect to an extrinsic fixed direction. From a practical point of view, if these points represent wireless devices and edges represent communication links for instance, the points would need to share a global coordinate system to be able to orient their cones identically. Potential absence of global coordinate information adds a new level of difficulty by allowing each point to spin its cone wheel independently of the others. In this paper we take a first step towards reexamining Yao graphs in light of intrinsic cone orientations, by introducing a new class of graphs called \emph{continuous Yao graphs}.

Given an angle , the continuous Yao graph with angle , denoted by , is the graph with vertex set , and an edge connecting two points  and  of  if there exists a cone with angle  and apex  such that  is the closest point to  inside this cone. In contrast with the classical construction of Yao graphs, for the continuous version the orientation of the cone is arbitrary. We can imagine rotating a cone with angle  around each point  and connecting it to each point that becomes the closest to  inside the cone during this rotation. To avoid degenerate cases, we assume \emph{general position}, i.e., we assume that for each , there are no two points at the same distance from .

In contrast with the Yao graph, the continuous Yao graph has the property that  for any . This property provides consistency as the angle of the cone changes and could be useful in potential applications requiring scalability. Another advantage of continuous Yao graphs over regular Yao graphs is that they are invariant under rotations of the input point set. However, unlike Yao graphs that guarantee a linear number of edges, continuous Yao graphs may have a quadratic number of edges in the worst case. (Imagine, for instance, the input points evenly distributed on two line segments that meet at an angle . For any , \cyao includes edges connecting each point on one line segment to each point on the other line segment.)

In this paper, we focus on the spanning ratio of the continuos Yao graph.
In Section~\ref{section:Small cones}, we show that  has spanning ratio at most  when . 
However, the argument used in this section breaks when . To deal with this case, we introduce a new algebraic technique based on the description of the regions where induction can be applied.
To the best of our knowledge, this is the first time that algebraic techniques are used to bound the spanning ratio of a graph. 
As such, our technique may be of independent interest. In Section~\ref{section:The 2pi/3 cyao}, we use this technique to show that  is a -spanner.
In Section~\ref{Section:Other angles}, we study the case when . Using elliptical constructions, we are able to show that  is not a constant spanner. While the algebraic techniques presented in Section~\ref{section:The 2pi/3 cyao} appear to extend beyond , it remains open whether or not  with angle  is a constant spanner.
Finally, we study the connectivity of  and show that  is connected provided that . Moreover, for , there exist point sets for which  is not connected.

\vspace{-0.5em}
\section{Continuous Yao for narrow cones}\label{section:Small cones}

In this section, we study the spanning ratio of \cyao for . 
In this case, we make use of an inductive proof similar to those used to bound the spanning ratio of Yao graphs~\cite{barba2014new}.

\vspace{-0.2em}
\begin{lemma}
 \label{lem:basicyao}\emph{[Lemma 1 of~\cite{barba2014new}]}
Let ,  and  be three points such that  and . Then\vspace{-.1in}

\end{lemma}
Given two points  and  of \cyao,
let  be the cone with apex  and  on its angle bisector.
The cone  is defined analogously.

\begin{theorem}
 \label{thm:contyaosmall}
 The graph \cyao has spanning ratio at most  for .
\end{theorem}
\begin{proof}
 We need to show that there exists a path of length at most  between any two vertices  and . We prove this by induction on the distance . In the base case  and  form the closest pair. Hence, the edge  is added by any cone of~ that contains , as no other vertex can be closer to~.
 
 For the inductive step, we assume that the theorem holds for any two vertices whose distance is less than . If the edge  is in the graph, the proof is finished, so assume that this is not the case. That means that there is a vertex closer to  in every cone with apex  that contains . In particular, this also holds for the cone . Let  be the vertex that is closest to  in . Since   has aperture , the angle  is at most , and Lemma~\ref{lem:basicyao} gives us that 
 . Note that since , we have that , which means that  and hence . Therefore our inductive hypothesis applies to  and , which tells us that there exists a path between them of length at most . Adding the edge  to this path yields a path between  and  of length at most

\vspace{-1em}

This completes the proof.
\end{proof}


\section{The graph  is a Spanner}\label{section:The 2pi/3 cyao}

Let  be the largest root of the polynomial

In this section, we prove that \cyaoOneTwenty is a -spanner.
That is, we show that for any two points  and  in \cyaoOneTwenty, there exists a path from  to  of length at most . The way we derive this polynomial will become clear by the end of this section.

The proof proceeds by induction on the rank of the distance  among all distances between vertices of \cyaoOneTwenty.
In the base case,
 and  define the closest pair among the points of \cyaoOneTwenty.
Hence, the edge  is added by any cone of  that contains ,
as no other vertex can be closer to .

We spend the remainder of this section proving the inductive step.
Assume that the result holds for any two points whose distance is smaller than .
Without loss of generality,
assume that  and ,
so that .
We start with a simple observation that follows from the general position assumption.
Define  be the \emph{inductive set of  with respect to }
(see Fig.~\ref{fig:Inductive Region}).
\begin{figure}[t]
\centering
\includegraphics[width=0.45\linewidth]{InductiveRegion.pdf}
\caption{\small The inductive set  for different values of .}
\label{fig:Inductive Region}
\end{figure}
Symmetrically,
let  be the inductive set of  with respect to .

\begin{lemma}\label{lemma:Contained in circle}
The inductive set  is contained in the disk  with center  and radius .
Moreover, any point  on the boundary of  lies outside of .
\end{lemma}
\vspace{-0.8em}
\begin{proof}
Let  be a point in .
Because  and , we have that 
.
Consequently,
 lies strictly inside the circle with center  and radius .
\end{proof}

\vspace{-0.5em}
Recall that  denotes the cone with apex  and  on its angle bisector.
Let  and  be the neighbors of  and  in cones  and , respectively.
The inductive set  satisfies the \emph{inductive property}:
if ,
then there is a path from  to  with length at most .
Indeed,
because , Lemma~\ref{lemma:Contained in circle} implies that .
Therefore,
we can apply the induction hypothesis and obtain a path from  to  of length at most . 
Because , adding the edge  to this path yields a path from  to~ 
of length at most  as desired.
The inductive set  has an analogous inductive property.

Note that if 
or ,
then we are done by the inductive property.
Thus, we assume that  and .
Since  and ,
the set of points on the boundary of  satisfy

which defines a quartic curve in  and . 
Let  and  be the intersection points of the boundaries of  and  and assume that  lies above ;
see Fig.~\ref{fig:Configuration}.
\begin{figure}[t]
\centering
\includegraphics[width=0.6\linewidth]{ConfigurationYao}
\caption{\small The inductive sets  and  are shown. The circular sectors where  and  can lie are depicted in light blue and light red, respectively.}
\label{fig:Configuration}
\end{figure}
Because the triangles  and  are equilateral,
we have  and .
Let

be the intersection point of the boundary of  with the segment . 
Symmetrically,
let

be the intersection of the boundary of  with the segment .
There are two cases to deal with. Either   and  lie on the same side of the -axis
or  they lie on opposite sides.

Given three points ,  and  such that ,
we denote by  the circular sector with apex  that is contained between  and ,
counter-clockwise.\vspace{.05in}

\textbf{Case } Assume first that  and  both lie above the -axis.
Because  and  lie in the circular sectors  and , respectively, we have that .
Therefore, we can apply induction on  to obtain a path  from  to  of length at most .
Consider the path  from  to . 
We show that the length of  is at most .
To this end, we provide a bound on the length of the segment .

\vspace{-0.5em}
\begin{lemma}\label{lemma:Case one maximized length}
In the configuration of Case  depicted in Fig.~\ref{fig:Configuration},
.
\end{lemma}
\vspace{-1em}
\begin{proof}
Recall that  must lie in the circular sector . Moreover,
because we assumed that  lies outside of ,  lies in the region . 
Let  be the convex hull of  and let  be the intersection point between  and the circular arc of ; see Fig.~\ref{fig:Neighbor regions Case 1}. Analogously, let  be the intersection between  and the circular arc of .
Then,  is bounded by the segments ,  and the circular arc joining  and  with center  and radius 1. We define  analogously as the convex hull of .

\begin{figure}[htb]
\centering
\includegraphics[width=0.8\linewidth]{ZoomCase1}
\caption{\small The neighbor regions of  and  in Case .}
\label{fig:Neighbor regions Case 1}
\vspace{-.1in}
\end{figure}

Because  and , we get an upper bound on the distance between  and  by computing the maximum distance between a point in  and a point in . We refer to two points realizing this distance as a  \emph{maximum --pair}.
Since the Euclidean distance function is convex and since both  and  are convex sets, a maximum --pair must have one point on the boundary of  and another on the boundary of .

In fact, we claim that we need only to consider the boundaries of the triangles  and  to find a maximum --pair.
To prove this claim, consider the lune defined by . For any point  in this lune, consider its farthest point  in  and notice that the circle with center on  that passes through  leaves either  or  outside (or both). This is because the radius of this circle is smaller than the radius of the circular arc on the boundary of ; see Fig.~\ref{fig:Neighbor regions Case 1}. Therefore, either  or  is farther than  from  and hence, the maximum --pair cannot have an endpoint in this lune. That is, the maximum --pair includes a point on the boundary of the triangle . The same argument holds for  and  proving our claim.

As we know the coordinates of the boundary vertices of  and , we can verify that 
,  and  are all 
maximum --pairs (notice that this is true for any ).
\end{proof}

Because the length of  is at most , and since  and  are both at most 1,
the length of the path  is at most 
by Lemma~\ref{lemma:Case one maximized length}. 
We now prove that .
Since , , 
and ,
 the inequality  is equivalent to 
  \vspace{-.1in}
 
which is true,
provided that
 and .
Since  is bigger than the largest real root of ,
we are done.
Therefore,
whenever we are in the configuration of Case ,
we can apply induction and obtain a path  from  to  of length at most .\vspace{.05in}

\textbf{Case } The proof of Case  is a bit more involved but follows the same line of reasoning as the proof of Case .
If  and  lie on different sides of , we can assume without loss of generality that~ lies below the -axis while  lies above it.
Recall that  is the intersection of the boundaries of  and  that lies below the -axis.

\begin{figure*}[htb]
\centering
\includegraphics[width=0.82\textwidth]{RotatedConfiguration}
\vspace{-0.5em}
\caption{\small a) Point  and angle  b) Cone  is obtained by rotating  counter-clockwise  degrees.}
\vspace{-0.5em}
\label{fig:Rotated Configuration}
\end{figure*}

Since  is not an edge of \cyaoOneTwenty,
 must lie inside .
Let  be the intersection of the boundary of  with the circular arc of ; see Fig.~\ref{fig:Rotated Configuration}.
This intersection point always exists because
 lies inside 
and  lies outside of  by Lemma~\ref{lemma:Contained in circle}.
The circular arc of  is part of the circle defined by .
Therefore,
from~\eqref{eq:quartic},
 \vspace{-.07in}



Let ; see Fig.~\ref{fig:Rotated Configuration}a.
Since ,
from~\eqref{expression vreflect}
we have \vspace{-.09in}

from which  and hence, .
Consider the cone 
(respectively the point )
obtained by rotating 
(respectively )
counter-clockwise around  by an angle .
Note that ; see Fig.~\ref{fig:Rotated Configuration}b.
Let  be the neighbor of  inside . 
If  lies inside , we are done by the inductive property.
Therefore, assume that .
Because ,  cannot lie inside  and hence,
 must lie above the -axis. 
Let  be the convex hull of . Then  
must lie inside of ; see Fig.~\ref{fig:Zoom Case 2} for an illustration.
As in Case ,  must lie inside of the region  being the convex hull of .

Let  be the intersection of the boundaries of  and 
(see Fig.~\ref{fig:Zoom Case 2}).
From~\eqref{expression tan psi},
the equation of the line supported by  and  is

Thus,
the -coordinate of  is given by the expression

and the -coordinate of  is given by the expression

Thus,

and .

A proof similar to that of Lemma~\ref{lemma:Case one maximized length} (moved to the appendix due to space constraints) yields the following result. \begin{lemma}\label{lemma:Maximum length Case 2}
In the configuration of Case , the distance between  and  is at most .
\end{lemma}

By Lemma~\ref{lemma:Maximum length Case 2}, the distance between  and  is at most . Therefore, we can apply the induction hypothesis to obtain a path  from  to  of length at most .


Let  be a path from  to . 
Similarly to what we observed in Case ,
the length of  is at most  by Lemma~\ref{lemma:Maximum length Case 2}.

We now prove that .
Since ,  and , using the exact expressions for 
we find that ,
provided that .
Because we chose   to be equal to the largest real root of , we infer that .
Therefore,
whenever we are in the configuration of Case ,
we can apply induction and obtain a path  from  to  of length at most .

\begin{figure}[b]\centering
\includegraphics[width=0.8\linewidth]{ZoomCase2}
\caption{\small ,  and maximum --pair .}
\label{fig:Zoom Case 2}
\end{figure}


In summary,
given any two points  and  of \cyaoOneTwenty and a constant ,
we can construct a path from  to  which uses edges of \cyaoOneTwenty and has length at most . We obtain the following result.

\vspace{-0.3em}
\begin{theorem}
 \label{thm:contyao}
 The graph  \cyao has spanning ratio at most  if , or  if .
\end{theorem}

\vspace{-1.5em}
\section{Larger angles}\label{Section:Other angles}

\vspace{-0.5em}
Theorem~\ref{thm:contyao} provides upper bounds for the spanning ratio of  for values of .
But what happens when  is larger than ?
The next result shows that if  is very large, the graph can be disconnected.

\begin{figure}[ht]
 \centering
 \includegraphics[width=0.7\linewidth]{Disconnected}
 \caption{\small  \cyao\ can be disconnected when .}
 \label{fig:Disconnected}
\end{figure}

\vspace{-0.5em}
\begin{theorem}
 For , there are point sets for which \cyao is disconnected.
\end{theorem}
\begin{proof}
 Let , for any . Take a regular polygon  with interior angles of at least  radians, and let  be a copy of . Now place  and  such that the distance between them is larger than the distance between two consecutive vertices on  (see Fig.~\ref{fig:Disconnected}). Consider a vertex  on . The exterior angle at  is at most  radians. As this is less than , any cone with apex  will include one of 's neighbors on . And since the distance between  and  is larger than the distance between  and its neighbors,  will never connect to a vertex on . As the choice of  was completely arbitrary, and  is a duplicate of , this implies that no edge of \cyao will connect  to .
\end{proof}


Indeed,  is the true breaking point here: the continuous Yao graph with  is always connected (for a proof, see Appendix~\ref{Appendix}).
Next we show that, despite being connected,  is not a constant spanner.

\begin{theorem}
 \label{thm:CyaoPi No Spanner}
 The continuous Yao graph \cyaopi is not a constant spanner.
\end{theorem}
\begin{proof}
 Consider two points  and  at unit distance. We will add points such that the shortest path between  and  in \cyaopi is arbitrarily long. The construction is illustrated in Fig.~\ref{fig:LowerBound}. We place these additional points on an ellipsis that is obtained from the circle with diameter  by stretching it vertically by a factor of , for a fixed real .  (Fig.~\ref{fig:LowerBound}a). We start by placing four points, each at distance  from  or  (Fig.~\ref{fig:LowerBound}b). Then we place points at distance  from these points, and so on, until the two chains meet (when the distance between the last point on the upwards chain from  and the symmetric point from  is less than : Fig.~\ref{fig:LowerBound}c).
 
 With these points, any half-plane through a vertex  that contains vertices on the other side of the ellipsis also contains a neighbor of .
As these neighbors are always closer (before the end of the chain), no diagonals are created. Thus \cyaopi forms a convex polygon, following the contour of the ellipsis (Fig.~\ref{fig:LowerBound}d).

\begin{figure}[t]
 \centering
 \includegraphics[width = 0.85\linewidth]{LowerBound}
 \caption{\small  Establishing a lower bound for the spanning ratio of \cyao\ for large values of .}
 \label{fig:LowerBound}
\end{figure}

 As we increase , the number of vertices on each chain grows. When the chains each have  vertices, the shortest path between  and  has length at least . Since the distance between  and  remains fixed, and we can make  arbitrarily large, there is no constant  such that \cyaopi is a -spanner.
\end{proof}

\section{Conclusions}
We introduced a new class of graphs, called continuous Yao graphs, and studied their spanning properties. We showed that, for any angle , the continuous Yao graph \cyao is a spanner, whereas for , it is not. Furthermore, we showed that \cyao is connected for , and possibly disconnected for . The question whether \cyao is a spanner for  remains open. While the construction in the proof of Theorem~\ref{thm:CyaoPi No Spanner} does give a lower bound on the spanning ratio of the continuous Yao graphs in this range, this bound seems hard to express in terms of . For the upper bound, the proof from Section~\ref{section:The 2pi/3 cyao} appears to extend beyond , but we have not yet determined where the breaking point lies.




An alternative problem variant that maintains a linear number of edges in the output graph is one that permits each point to randomly select an  initial orientation of the entire cone wheel (as opposed to sweeping one cone continuously around the apex point). 
From Theorem~\ref{thm:CyaoPi No Spanner} we obtain as a corollary that there are point sets for which the Yao graph  is not a spanner, regardless of the orientation of the cones.
However, Theorem~\ref{thm:contyao} leaves open the possibility that  and above \emph{are} spanners under these conditions.






\section*{Acknowledgement}

The research for this paper was initiated at the first \emph{Workshop on Geometry and Graphs}, organized at the Bellairs Research Institute, March 10-15, 2013.



\small
\bibliographystyle{unsrt}
\bibliography{contyao}

\newpage
\appendix

\section{Omitted proofs}\label{Appendix}

\textbf{Lemma~\ref{lemma:Maximum length Case 2}}\emph{
In the configuration of Case , the distance between  and  is at most .
}
\begin{proof}
Because  and , we obtain an upper bound on the distance between  and  by computing the maximum distance between a point in  and a point in .
Using the same arguments as in the proof of Lemma~\ref{lemma:Case one maximized length}, we can show that the maximum distance is achieved by a point on the boundary of  and a point on the boundary of .
We refer to a pair of points that realizes this maximum distance as a \emph{maximum --pair}.

One can verify that every point in  is farther from  than from any other point in . Therefore, it suffices to find the point farthest from  in . Note also that the circle centered at  that passes through any point in the circular arc of  does not contain . Therefore, it suffices to find the point farther from  in the boundary of the triangle .

As we have exact expressions for  and for the vertices on the boundary of , we can verify that the maximum --pair is found when when  and , proving our result.\end{proof}

\begin{theorem}\label{theorem: cyan pi is connected}
 For , the continuous Yao graph \cyao is connected.
\end{theorem}
\begin{proof}
 Consider a set  of cones whose union is exactly the right half-plane. Such a set can be constructed by starting with the cone whose left boundary aligns with the positive -axis, and rotating by  degrees until the right boundary aligns with the negative -axis. Since , this set is non-empty. 
 Now, if a vertex  is not a rightmost vertex, there is a cone  in  that is not empty. Since  is completely contained in the right half-plane, the closest vertex in  must lie further to the right than . Thus, there is an edge connecting  to a vertex to its right. Since we only have finitely many points, by repeating this, we obtain a path from any vertex to a rightmost vertex. Finally, by slightly rotating the right half plane at each rightmost point (so that it includes only rightmost vertices), we obtain a path connecting all rightmost vertices (if several rightmost vertices exist). 
 Thus, by concatenating the paths from two arbitrary points  and  to rightmost vertices to the path connecting these rightmost vertices, we obtain a path between  and .
\end{proof}


\end{document}
