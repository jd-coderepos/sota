\documentclass{article}


\usepackage{breakurl}             






 

\usepackage{color}
\usepackage{pstricks,pst-node,pst-tree,pstricks-add}
\usepackage{multido}


\usepackage{enumerate}
\usepackage[english]{babel}
\usepackage{amsthm}
\usepackage{amssymb}
\usepackage{amsmath}
\usepackage{amsfonts}
\usepackage[footnote,marginclue]{fixme}

\theoremstyle{plain}
\newtheorem{theorem}{Theorem}
\newtheorem{lemma}[theorem]{Lemma}
\newtheorem{corollary}[theorem]{Corollary}
\newtheorem{proposition}[theorem]{Proposition}
\newtheorem{fact}[theorem]{Fact}
\newtheorem{q}{Question}
\newtheorem*{con}{Conjecture}
\newtheorem*{theoremC}{Open Question}
\newtheorem*{theoremB}{Lyndon Interpolation Theorem}
\theoremstyle{definition}
\newtheorem{rem}[theorem]{Remark}
\newtheorem{definition}[theorem]{Definition}
\newtheorem{example}[theorem]{Example}
\newtheorem{remark}{Remark}
\newtheorem{claim}{Claim}

\newcommand{\sg}{\sigma}
\newcommand{\rel}{rel}
\newcommand{\mA}{{\mathfrak A}}
\newcommand{\ma}{{\mathfrak A}}
\newcommand{\Dom}{{\rm Dom}}
\newcommand{\dom}{{\rm dom}}
\DeclareMathOperator{\str}{Str}
\DeclareMathOperator{\Mod}{Mod}

\newcommand{\restrict}[2]{{#1}({#2})}

\newcommand{\calC}{{\mathcal C}}
\newcommand{\calS}{{\mathcal S}}
\newcommand{\mX}{{\mathfrak X}}
\newcommand{\mY}{{\mathfrak Y}}
\newcommand{\mZ}{{\mathfrak Z}}




\newcommand{\FO}[1]{{\rm FO(#1)}}
\newcommand{\conjFOqf}[1]{{\rm \{\wedge\}\!-\!FO({#1})}}
\newcommand{\AndExistForallFO}[1]{{\rm \{\wedge, \exists, \forall\}\!-\!FO({#1})}}
\newcommand{\AndExistFO}[1]{{\rm \{\wedge, \exists\}\!-\!FO({#1})}}
\newcommand{\AndExistForallDep}{{\rm \{\wedge, \exists, \forall\}\!-\!\df}}
\newcommand{\AndExistDep}{{\rm \{\wedge, \exists\}\!-\!\df}}
\newcommand{\AndExistInd}{{\rm \{\wedge, \exists\}\!-\! \Ind}}


\newcommand{\free}[1]{\textsf{free}(#1)}

\newcommand{\MSO}{{\rm MSO}}
\newcommand{\Fr}{{\rm Fr}}
\newcommand{\Var}{{\rm Var}}
\newcommand{\SO}{{\rm SO}}
\newcommand{\ESO}{{\rm ESO}}

\newcommand*\dep{{=\mkern-1.2mu}} 
\newcommand{\Ind}{\mathcal{I}}
\newcommand{\df}{\rm{D}}
\newcommand{\FOinclusion}{{\rm FO(\subseteq)}}

\newcommand{\ESOarity}[1]{{\rm ESO}({#1}\mbox{-ary})}
\newcommand{\ESOfarity}[1]{{\rm ESO}_f({#1}\mbox{-ary})}
\newcommand{\ESOfarityOcc}[1]{{\rm ESO}_f^1({#1}\mbox{-ary})}
\newcommand{\ESOfvar}[1]{{\rm ESO}_f({#1}\forall)}
\newcommand{\ESOfvarOcc}[1]{{\rm ESO}_f^1({#1}\forall)}
\newcommand{\ESOarityvar}[2]{{\rm ESO}({#1}\mbox{-ary},{#2}\forall)}
\newcommand{\ESOfarityvar}[2]{{\rm ESO}_f({#1}\mbox{-ary},{#2}\forall)}

\newcommand{\fo}{{\rm FO}}
\newcommand{\dforall}[1]{{\rm D}({#1}\forall)}
\newcommand{\ddep}[1]{\fo(\dep(\ldots))({#1}\mbox{\rm-dep})}





\newcommand{\RAM}{{\rm RAM}}
\newcommand{\NTIME}{{\rm NTIME}}
\newcommand{\np}{{\rm NP}}
\newcommand{\Ptime}{{\rm PTIME}}
\newcommand{\ph}{{\rm PH}}
\newcommand{\nl}{{\rm NL}}
\newcommand{\logspace}{{\rm L}}
\newcommand{\GFP}{{\rm GFP}}
\newcommand{\LFP}{{\rm LFP}}
\newcommand{\PGFP}{{\rm GFP}^+}


\newcommand{\pr}{\mathrm{Pr}}
\newcommand{\tu}[1]{\overline{#1}}
\newcommand{\calP}{\mathcal{P}}

\newcommand{\indep}[3]{{#1}\ \bot_{#2}\ {#3}}
\newcommand{\var}[1]{\mathsf{#1}}
\newcommand{\col}[1]{\mathbf{#1}}

\newcommand{\mAsat}[2]{\mA \models_{#1}  #2}
\newcommand{\mAnsat}[2]{\mA \models_{#1}  \neg #2}


\newcommand{\commentmargin}[2]{\marginpar{\tiny{\textbf{#1 : }\textit{#2}}}}
\newcommand{\commenttext}[1]{ \begin{center} {\fbox{\begin{minipage}[h]{0.9
            \linewidth}   {\sf #1} \end{minipage} }} \end{center}} 

\newcommand{\arnaud}[1]{ {\leavevmode\color{green}\commenttext{Arnaud: #1}}}
\newcommand{\juha}[1]{{\color{red}\commenttext{Juha: #1}}}
\newcommand{\jouko}[1]{{\color{blue}\commenttext{Jouko: #1}}}
\newcommand{\nicolas}[1]{{\color{orange}\commenttext{Nicolas: #1}}}

\newcommand{\arnaudmargin}[1]{\commentmargin{\color{green}Ar}{\color{orange}#1}}
\newcommand{\juhamargin}[1]{\commentmargin{\color{red}Ju}{\color{red}#1}}
\newcommand{\joukomargin}[1]{\commentmargin{\color{blue}Jo }{\color{blue}#1}}
\newcommand{\nicolasmargin}[1]{\commentmargin{\color{orange}Ni }{\color{orange}#1}}



\newcommand{\makeprob}[3]{
\noindent 
\begin{tabular}{l}
\textsc{#1} \\ 
\hspace{1ex}
\begin{tabular}{lp{11cm}}
{\it Input: }  & {#2} \\
{\it Output: } & {#3} 
\end{tabular}
\end{tabular}\vspace{.7ex}
}
 \newcommand{\pb}[1]{\textsc{#1}}
 
 \newcommand{\type}[1]{\textsf{typ}(#1)}
\newcommand{\Disjunctiondepth}[1]{\textsf{d}_{\vee}(#1)}
\newcommand{\Existsdepth}[1]{\textsf{d}_{\exists}(#1)}
\newcommand{\algoone}{\text{algo}_1}
\newcommand{\algotwo}{\text{algo}_2}
\newcommand{\Falgoone}[2]{\text{Fal}_1\left(#1,#2\right)}
\newcommand{\Falgotwo}[3]{\text{Fal}_2\left(#1,#2,#3\right)}

\newcommand{\ESOHORN}{\SO\exists\textrm{-Horn}}
\newcommand{\SDHORN}{\df^*\textrm{-Horn}}
\newcommand{\indlogic}{\rm{FO} (\bot_{\rm c})}
\def\one{w_1}
\def\two{w_2}
\def\three{w_3}

 




\begin{document}



\title{Tractability Frontier of Data Complexity in Team Semantics}

\author{Arnaud Durand\\
IMJ-PRG, CNRS UMR 7586, Universit\'e de Paris, France

\and

Juha Kontinen\\
Department of Mathematics and Statistics,\\
 University of Helsinki, Finland

\and Nicolas de Rugy-Altherre\\
Universit\'e de Lorraine, France

\and
Jouko V\"{a}\"{a}n\"{a}nen\\
Department of Mathematics and Statistics,\\
 University of Helsinki, Finland\\
and
Institute for Logic, Language and Computation,\\
 University of Amsterdam, The Netherlands}
\date{}



\maketitle





\begin{abstract}We study the data complexity of model-checking for logics with team semantics. We focus on dependence, inclusion, and independence logic formulas under both strict and lax team semantics. Our results delineate a clear tractability/intractability frontiers in data complexity of both quantifier-free and quantified formulas for each of the logics. 
For inclusion logic under the lax semantics, we reduce the model-checking problem to the satisfiability problem of so-called \emph{dual-Horn} Boolean formulas. Via this reduction, we give an alternative proof for the known result that the data complexity of inclusion logic is in  PTIME.
\end{abstract}

\section{Introduction}

In this article we study the data complexity of model-checking of dependence, independence, and inclusion logic formulas. Independence  and inclusion logic  \cite{gradel10,galliani12} are variants of  dependence logic \cite{vaananen07}  that  extends first-order logic by dependence atoms of the form 

expressing that the value of  is functionally determined by the values of the variables . 
In independence and inclusion logic dependence atoms are replaced by independence and inclusion atoms 
 and  
respectively. The meaning of the independence atom is that, with respect to any fixed value of  , the variables   are independent of the variables , whereas the  inclusion atom expresses that all the values of  appear also as values for . 

  
Team semantics is a  framework for formalizing and studying  various notions of dependence and independence  pervasive in many areas of science.  Team semantics differs from Tarski's semantics by interpreting formulas using sets of assignments instead of single assignments as in  first-order logic.  
 Reflecting this, dependence logic  has higher expressive power than classical logics used for these purposes previously.  Dependence, inclusion, and independence  atoms are intimately connected to the  corresponding functional, inclusion, and multivalued  dependencies studied in  database theory, see, e.g., \cite{DBLP:conf/foiks/HannulaK14}.  Interestingly, independence atoms correspond naturally to a qualitative analogue of the notion of conditional independence in statistics  \cite{DBLP:journals/networks/GeigerVP90}. On the other hand,  team semantics can be naturally generalized to a probabilistic variant in which probabilistic independence can be taken  as an atomic formula (see \cite{HannulaHKKV19,HKMV18} for further details). 

 Dependence logic and its variants can be  used to formalize and study dependence and independence notions in various areas. For example, in the foundations of quantum mechanics, there  are a range of notions of independence playing a central role in celebrated No-Go results such as Bell's theorem. 
Similarly, in the foundations of social choice theory, there are results such as Arrow's Theorem which  can also be formalized in the team semantics setting   \cite{2014arXiv1409.5537H,PacuitY16}.


For the applications it is important to understand the complexity theoretic aspects  of dependence logic and its variants. During the past few years,  these aspects have been addressed in several studies. We will next briefly discuss some previous work. 
The data complexity of inclusion logic  is sensitive to the choice between the two main variants of team semantics: under the so-called  lax semantics it is equivalent to positive greatest fixed point logic () and captures  over finite (ordered) structures \cite{gallhella13}. Recently a  fragment of inclusion logic that captures  has been identified in \cite{HannulaH19}. The same article also exhibits surprisingly simple formulas of inclusion logic whose data complexity is  and -complete (see equations  \eqref{HH1} and \eqref{HH2}).
 On the other hand, under the strict semantics, inclusion logic is equivalent to  (existential second order logic) and hence captures  \cite{galhankon13}.  

 In \cite{durand11} the fragment of dependence logic allowing only sentences in which dependence atoms of arity at most  may appear (atoms   satisfying ) was shown to correspond to the -ary fragment   of ESO in which second-order quantification is restricted to at most -ary functions and relations.  
Several  similar results have been obtained  also for independence and inclusion logic, e.g.,  in \cite{galhankon13,DBLP:journals/corr/HannulaK14,Ronnholm18,Ronnholm19}.

 The combined complexity of the model-checking problem of dependence logic, and many of its variants, was shown to be NEXPTIME-complete \cite{gradel12}. On the other hand, the satisfiability problem for the two variable fragment of dependence logic (and many of its variants) was shown to be NEXPTIME-complete in \cite{KontinenKLV11, KontinenKV14}. Furthermore,  during the past few years, the complexity aspects of propositional and modal logics in team semantics have been also systematically studied (see \cite{Luck19,HannulaKVV18} and the references therein). 


The starting point for the present work are the following results of \cite{kontinenj13} showing that the non-classical interpretation of disjunction in team semantics makes the model-checking of certain quantifier-free
formulas very complicated.  Define  and  as follows:
\begin{enumerate}
\item  is the formula , and
\item\label{phi2}  is  the formula .
\end{enumerate}
Surprisingly, the data complexity of the model-checking problem of  and    are already  NL-complete and NP-complete, respectively. In  \cite{kontinenj13} it was also shown that  model-checking for  where  and  are -\emph{coherent} quantifier-free formulas of dependence logic is always in . A formula  is  called -coherent if, for all  and ,  , if and only if,   
 for all  such that . Note that the left-to-right implication is always true due to the downwards closure property of dependence logic formulas. The  downwards closure property also implies that, for dependence logic formulas,  the strict and the lax semantics are equivalent. For independence and inclusion logic formulas this is not the case.


In this article our goal is to shed light on  the tractability frontier of data complexity of dependence, independence, and inclusion logic  formulas under both strict and lax team semantics.  In order to state our results, we define  a new syntactic measure called the disjunction-width   of a formula . Our results show that,  for quantifier-free formulas  of dependence logic, the data complexity of model-checking is in NL if . Surprisingly, for independence logic the case of quantifier-free formulas turns out to be  more fine-grained. In particular, we exhibit  a quantifier-free formula  with  whose data-complexity is NP-complete and also identify a more restricted fragment with data complexity in  .
For quantified formulas,  the complexity 
is shown to be   NP-complete already with simple formulas constructed in terms of existential quantification and conjunction   in the empty non-logical vocabulary. 

For inclusion logic, we  show that model-checking can be reduced to the satisfiability problem of dual-Horn propositional formulas. While interesting in its own right, this also provides an alternative proof for the fact (see \cite{gallhella13}) that the  data complexity of  inclusion logic is in PTIME, and is also analogous  to the classical result of Gr\"adel on the Horn fragment of second-order logic \cite{DBLP:journals/tcs/Gradel92}.  We  also show that, under the strict semantics, the tractability frontier of model-checking of  (both quantifier-free and quantified) inclusion logic formulas becomes similar to that of  dependence and independence logic.



\section{Preliminaries}
In this section we briefly discuss the basic definitions and results needed in this article.
 
 
 \begin{definition}
Let  be a structure with domain , and  be a finite (possibly empty) set  of
variables.
\begin{itemize}
\item A \emph{team}  of  with domain   is a finite set of assignments .
\item For a tuple , where  ,   is the -ary relation of , where .
\item For ,  denotes  the team obtained by restricting all assignments of  to  .
\item  The set of free variables of a formula  is  defined as in first-order logic, taking into account that free variables may arise also from dependence, independence and inclusion atoms, and is denoted by .
\end{itemize}
\end{definition}
We will consider two variants of the semantics called the strict and
the original semantics given in \cite{vaananen07} is a combination of these variants (with the lax disjunction and the strict existential quantifier). For dependence logic formulas, the two variants of the semantics are easily seen to be equivalent, but for independence and inclusion logic this is not the case. We first define the lax  team semantics for first-order formulas in negation normal form. Below   refers to the satisfaction in first-order logic, and  is the assignment such that , and    for . The power set of a set  is denoted by .

\begin{definition}
Let  be a structure,  be a team of , and  be a first-order formula such that . \begin{description}
\item[lit:] For a first-order literal ,  if and only if, for all , .
\item[:]   if and only if,  there are  and  such that ,   and .
\item[:]  if and only if,  and .
\item[:]   if and only if, there exists a function  such that , where .
\item[:]  if and only if, , where .
\end{description}
A sentence  is  \emph{true} in  (abbreviated ) if . Sentences  and  are  \emph{equivalent}, , if for all models , .
\end{definition}
In the \emph{Strict Semantics}, the semantic rule for disjunction is modified by adding the requirement , and  the clause for the existential quantifier  is replaced by 
\begin{description} 
\item  if and only if, there exists a function  such that , where . \end{description}
The meaning of first-order formulas is invariant under the choice between the strict and the lax semantics. 
First-order formulas satisfy what is known as the \emph{flatness} property:
, if and only if,    for all .
Next we will give the semantic clauses for the new dependency atoms:
\begin{definition}
\begin{itemize}
\item Let  be a tuple of variables and let  be another variable. Then  is a \emph{dependence atom}, with the semantic rule
\begin{description}
\item  if and only if for all , if , then ;
\end{description}
\item Let , , and  be tuples of variables (not necessarily of the same length). Then  is a \emph{conditional independence atom}, with the semantic rule
\begin{description}
\item  if and only if for all  such that , there exists an assignment  such that .
\end{description}
Furthermore, when  is empty, we write  as a shorthand for , and  call it a \emph{pure independence atom};
\item Let  and  be two tuples of variables of the same length. Then  is an \emph{inclusion atom}, with the semantic rule \begin{description}
\item  if and only if for all  there exists a  such that . 
\end{description}
\end{itemize}
\end{definition}
The formulas of dependence logic, , are obtained by extending the syntax of  by dependence atoms. The semantics of -formulas is obtained by extending Definition 2 by the semantic rule defined above for dependence atoms.  Independence logic, , and inclusion logic, , are defined analogously using independence and inclusion atoms, respectively.


It is easy to see that the flatness property is lost immediately when  is extended by any of the atoms defined above. On the other hand,  it is straightforward to check that all -formulas satisfy the following  strong \emph{downwards closure} property:  if  and   , then .
Another basic property of logics in team semantics is called \emph{locality}:
\begin{proposition}[Locality]
Let   be a formula of any of the logics  ,   or . Then, under the lax semantics, for all structures    and teams  :
 
\end{proposition}
Under the strict semantics locality  holds only for dependence logic formulas (see \cite{galliani12} for details).

In this article we study the  data complexity  of model-checking  of dependence, independence, and inclusion logic formulas. In other words, for a fixed formula  of one of the aforementioned logics, we study the complexity of the following model-checking  problem:  given a finite structure   and a team ,  decide whether . 
Note that when we are working with the lax semantics, we may assume without loss of generality that the  domain of  is exactly . \emph{If not explicitly mentioned otherwise, all  results are valid under both strict and lax semantics}.
We assume that the reader is familiar with the basics of computational 
complexity theory such as NP-completeness and the log-space bounded classes  and .

\subsection{Complexity classes and satisfiability problems}

In the paper, we will make use of some well-known computational problems whose complexity is recalled here (unless specified all proofs can be found in~\cite{GareyJ1979}). We suppose the reader familiar with basic complexity classes such as  (logarithmic space),  (non deterministic logarithmic space),  (polynomial time),  (non deterministic polynomial time).

Let , a propositional formula is in -\pb{cnf} if it is in conjunctive normal form with clauses of length at most . It is positive, if it equivalent to a formula without negation. 

It is well known that, - the satisfiability problem for -\pb{cnf} formulas is -complete for  and -complete for . \pb{horn-sat} (resp. \pb{dual-horn-sat}) is the satisfiability problem of \pb{cnf} formulas with at most one positive (resp. negative) literal per clause. This problem is known to be -problem.

Given a positive -\pb{cnf} formula, deciding whether  is satisfiable such that exactly one variable is set to true in each clause is also known to be -complete. This problem is called \pb{1-in-3-sat}.  


\section{Dependence and independence logics}In this section we consider the complexity of model-checking for quantifier-free and quantified  formulas of dependence and independence logic. 

\subsection{The case of quantifier-free formulas}
In this section we consider the complexity of model-checking for quantifier-free formulas of dependence and independence logic. For dependence logic the problem has  already been essentially settled in \cite{kontinenj13}.  The following theorems delineate a clear barrier between tractability and intractability for quantifier-free dependence logic formulas.
\begin{theorem}[\cite{kontinenj13}]\label{JarmoKontinenNL} The model checking problem for the formula 
 
is -complete. More generally, the model-checking for  where  and  are -coherent quantifier-free formulas of  is always in .
\end{theorem}

When  two disjunctions can be used, the model checking problem becomes intractable as shown by the following results.

\begin{theorem}[\cite{kontinenj13}]\label{JarmoKontinenNP}
The model checking problem for the formula 

 is -complete.
\end{theorem}




In order to give a syntactic generalization of  Theorem \ref{JarmoKontinenNL}, we define next the disjunction-width of a formula. 


\begin{definition} Let  be a relational signature. The disjunction-width of a -formula , denoted , is defined as follows:

\end{definition}

The next theorem  is a syntactically defined analogue of Theorem  \ref{JarmoKontinenNL}.
\begin{proposition}\label{disjunction depth 2}
The data complexity of  model-checking of quantifier-free -formulas   with  is in  .
\end{proposition}

\begin{proof} We will first show that a formula   with  is -coherent.
 This follows by induction using the following facts \cite{kontinenj13}: 
 \begin{itemize}
\item dependence atoms are -coherent, and first-order formulas are -coherent,
\item if  is -coherent, then  is also -coherent assuming   is first-order,
\item  if    is -coherent and   is -coherent, then   is -coherent.
\end{itemize}
It is also straightforward to check that the data complexity of a formula   with  is in L (the formula  can be  expressed in FO assuming the team  with domain  is represented by the -ary relation ). We will complete the  proof using induction on  with  
 .  Suppose that  , where  . Then the claim follows by Theorem  \ref{JarmoKontinenNL}. The case  is also clear. Suppose finally  that , where  is first-order. Note that by downward closure and flatness
 
where . Now since  can be computed in L,  the model-checking problem of  can be decided in NL by the induction assumption for . \end{proof}


In the rest of this section we examine potential  analogues of Theorems \ref{JarmoKontinenNL} and \ref{JarmoKontinenNP} for  independence logic. It is well-known that the dependence atom  is logically equivalent to the independence atom . Hence, the following is immediate from Theorem~\ref{JarmoKontinenNP}.


\begin{corollary}\label{3 disjunction relativized independance}
The model checking problem for the formula 

 is -complete.
\end{corollary}


For independence logic, the situation is not as clear as for dependence logic concerning tractability.
 In the following we will exhibit a fragment of independence logic whose data complexity is in  and which is in some sense the maximal such fragment. 
 















\newcommand{\bcindepfo}{\mathsf{BC}(\bot,\fo)}
\newcommand{\CCm}{\mathfrak{C}^{-}}
\newcommand{\CCp}{\mathfrak{C}^{+}}

\begin{definition} The Boolean closure of an independence atom by first-order formulas, 
denoted , is defined as follows:

\begin{itemize} 
\item Any independence atom  is in .

\item If , then for any formula ,  and  are  in . 
\end{itemize}
\end{definition}

Let . Up to permutation of disjunction and conjunction,  can be put into the following normalized form:



For a structure ,  and , where 
 . 


The next lemma gives a combinatorial criterion for the satisfaction of  formulas. 

\begin{lemma}\label{bcindepfo}
For all  of the form~(\ref{normalized}), structures ,  and teams   the following are equivalent:
\begin{description}
\item [(A)].
\item [(B)] (1) and (2) below hold: 
\begin{description}

\item[(1)] For all  either  or . 

\item[(2)]\label{c2}  For all  such that , and ,   there exists ,
such that: . 




\end{description}

\end{description}Assignments   as in (2) will be called \textit{compatible} for formula  and team . Furthermore,  as in (2) will be called a \textit{witness} of  for formula .
\end{lemma}


\begin{proof} Note that (A) is equivalent to the existence of subteams  and , , of  such that 

\begin{enumerate}
\item[(i)] 
\item[(ii)]  for ,
\item[(iii)]  for 
\item [(iv)] .
\end{enumerate}
This can be proved by induction on  as follows: The claim is clearly true for . Suppose it holds for , where . We know  is equivalent to  Hence  if and only if  such that  and . In particular,  and 
By Induction Hypothesis there are subteams  and , , of  such that (i)-(iv) hold with  replaced by .
Now the sequence  satisfies (i)-(iv) and is therefore as required for the induction claim. The converse is similar.

(A) implies (B): To prove (1), suppose . Thus , where  are as defined earlier. Suppose  

 Then , whence . Hence , whence , and therefore   Eventually, step by step, we verify  and hence
 (1) is proved. 
 
 To prove (2), suppose  such that . By (i)-(iii), . By (iv) there is a witness as required.
 
 (B) implies (A): Let us denote by  a minimal superset of    satisfying   . Such a set  exists by the assumption (2). Then, for , define
 
 Suppose  but . By (1),  If , then  
  Hence .
 If , then 
  Hence 
 . Continuing this way we see that if , then , contrary to the assumption  . We have proved (i). Clearly (ii) and  (iii) hold. Finally, (iv) holds by the definition of .
\end{proof}









\begin{theorem}\label{BC(top,FO)} The data complexity of any  is in . This is true both in lax and in strict semantics.\end{theorem}
\begin{proof}
It is well known that checking whether a tuple  belongs to the query result  of a first-order formula can be done in logarithmic space~\cite{Immerman99}.  Therefore, by the characterization of Lemma \ref{bcindepfo}, deciding whether  is also in .
For the strict semantics, it suffices to note that the proof of Lemma \ref{bcindepfo}  goes through  also under the strict semantics. We can use the fact that the formulas  have the downwards closure property to force the  subteams  in the decomposition of  to be pairwise disjoint.
\end{proof}

Interestingly the analogue of Lemma \ref{bcindepfo} does not hold for inclusion atoms; It was recently shown in \cite{HannulaH19} that the data complexity of the formula

is already -complete and for 

the problem becomes -complete. These results hold under both strict and lax semantics as the other disjunct in both formulas is first-order and satisfies the downwards closure property.
Furthermore, in the last section of this article we construct a quantifier free inclusion logic formula with -complete data complexity under the strict semantics.
\begin{theorem}\label{thm13}
The data complexity of  formulae of the form   with  is in  under the lax semantics.
\end{theorem}


\begin{proof} 
The proof is given by a log-space reduction to the -complete problem -\pb{sat}.
Given a structure  and a  team  we  construct a -\pb{cnf} propositional formula  such that:


 Recall that if a team  is such that   then, there exists  such that  and .
For each assignment , we introduce two Boolean variables  and . Our Boolean formula  will be defined below with these   variables the set of which is denoted by . It will express that the set of assignments must split into  and  but also make sure that incompatible (see Lemma~\ref{bcindepfo}) assignments do not appear in the same subteam. 
 
For each pair  that are incompatible for  on team , one adds the -clause: . The conjunction of these clauses is denoted by .
Similarly, for each pair  that are incompatible for  on team , one adds the clause:  and call   the conjunction of these clauses.

Finally, the construction of  is completed by adding the following conjunctions:





It is not hard to see  that the  formula
 
  can be constructed deterministically in log-space.
 It remains to show that the equivalence \eqref{indep-translation} holds.
 
Assume that the left-hand side of the equivalence holds. Then, there exists  such that ,  and  .
We construct a propositional assignment  as follows. For all , we set  and for all , we set similarly . It is now immediate that the all of the clauses in   are satisfied by .
 
 Let us consider a clause   for an incompatible pair . Then,  or  must hold. For a contradiction, suppose that . Then since  holds, by construction  and  must be compatible for . Hence we get a contradiction and may conclude that  satisfies . The situation is similar for each clause . 

Let us then assume that   is satisfiable, and let  be a satisfying assignment for . Since , we get that  or  for all  . Let 
   
 Now  and clauses   () ensure that each  () satisfies condition (B,1) of Lemma   \ref{bcindepfo}.
 
We will next show how the  sets  and  can be extended to sets  and  satisfying also condition (B,2) of Lemma   \ref{bcindepfo} and consequently  and  .
 Note first that, since  satisfies , for all ,  cannot have a clause of the form , and hence  are compatible for . Analogously we see that all   are compatible for  .
We will define the sets   and  incrementally by first initializing them to   and , respectively. Note that even if , no decision has been made regarding the membership of   assignments  in  (resp. ) such that  (resp. ).  Let us first consider . Until no changes occur,  we consider all pairs  (where the sets  and  are taken with respect to ) such that    and add into  (if they are not already in) all tuples  such that  is a witness for the pair  regarding formula . Since by construction  are compatible then at least one such   exists (but may be initially outside of ). We prove below that this strategy is safe.
First of all, it is  easily seen that any pair among  is compatible for . Therefore, it remains to show that the new assignments   are compatible with every other element  added to  so far. Suppose this is not the case and that there exists  such that  and  are incompatible for . Note that  for incompatibility we  must have . Since , and  are in  they are all pairwise compatible. Hence, there exists  such that  is a witness for the pair .
Then,  , and  . Consequently,  is also a witness for  hence,  and  are compatible which is a contradiction. Therefore,  the assignment   can be safely added to . The set   is defined analogously. By the construction,  the sets  and  satisfy conditions (B,1) and (B,2) of Lemma   \ref{bcindepfo} for  and , respectively,  and hence it holds that    and  .
  
 




\end{proof}


It is worth noting that the last step of the proof of the above theorem does not seem to  work  under strict semantics. It is an open question whether Theorem \ref{thm13} holds under the strict semantics.

We will next show that a slight relaxation on the form of the formula immediately yields intractability 
of model-checking.
\begin{theorem}
There exists a formula  the model-checking problem of which is -complete and such that:

\begin{itemize}
    \item  and
    \item  is the conjunction of \textit{two} independence atoms.
\end{itemize} 
\end{theorem}


\begin{proof}
Define , . We will reduce  -SAT to the  model-checking problem  of . Let  be  a -SAT instance. 
 Each  is of the form  with  . To this instance we associate a structure  of the empty vocabulary and a team  on the variables . The the universe of the structure  is composed of ,  new elements  and of . For each clause  we add  in  the  assignments displayed on the left below,  and for each variable ,  we add to  the  assignments on the right:

We will next show that  is satisfiable if and only if .



Suppose there is an assignment   that evaluates   to true, i.e., at least one literal in each clause is evaluated to . We have to split  into two sub-teams  such that  and . We must put every assignment  such that  in . There are exactly three such assignments per clause. We put in  every assignment  such that  if , and  if . The other assignments are put into .
	
	For each clause , one literal  is assigned to  by . Then there is at least one assignment  in . In , the  assignments mapping  to  map  to  or . Thus . 
	
	If   are such that , then  (analogously ) is  if ,  otherwise.  Therefore, , and hence   holds.
	
	As for , it is immediate that . The only pair of assignments  in  such that  are  and , for some . Only one of them is in  ( if ,  otherwise). Thus .
	


Suppose then that  such that  and . Define an assignment   of the variables of  by:  if  is in ,  if  is in . Since ,  and because there is no  such that  and , for each  at most one of ,  can be in . Similarly, because , only one of them can be in . Thus  is indeed a function.

Since  and there is no assignment in  such that , every pair  such that  must have the same value of  and . Every assignment representing a clause in  respects the choice of . Furthermore, since  and  are in ,  must be in , i.e., at least one assignment per clause is in . By the above we may conclude that  satisfies : at least one literal per clause is evaluate to  by .


Finally it is not difficult to check that  the splitting of  into  and , if possible, can be  realized with disjoint  and . Hence the hardness holds  under both strict and lax semantics 

\end{proof}

It is worth noting that the formula in the previous result has disjunction-width two whereas the formula in Corollary~\ref{3 disjunction relativized independance} simulating dependence atoms by conditional independence atoms has width three. We will next show  an analog of these results for  pure (i.e., non-conditional) independence atoms. 

\begin{theorem}\label{pure3-disjunction}
	The model checking problem 	is -complete for  
	
\end{theorem}


\begin{proof} Membership in  is obvious. Hardness is proved by reduction from the following -complete problem -clique cover (see~\cite{GareyJ1979}): Given  a finite graph, decide if there exists a collection of disjoint triangle subgraphs of , that cover the vertex set  of .

	So, let  be a finite graph,  be a first-order structure of the empty signature, and  the team
	 
	  where   denotes the assignment  with  and  .
		 We are going to show that  has a -clique cover if and only if .
	 
	 
	


Suppose that  has a -clique cover, i.e., there exists  three cliques such that . We have to prove  . For , let  and .
	 
	 Because it is a vertex cover, every assignment of the form  is contained in  and not in , i.e. .
	 
	 Let  and  be two assignments. If  and , then  and there exists two assignments   in  such that  and  by construction. Similarly if  and , there exists in  the assignments  and  (even if  or ). Finally, if  and , the assignments  are in  and so are . The above implies that .
	 
For the converse implication,   suppose that , then  such that  for  and .
	Let ,  for ,  be the graph whose vertices are  and  edges are 
	 
	 Note that some  can be empty but they form a vertex cover of  as no assignment  is in . If  then  and  are in . By independence,  and  are also in . Therefore the edge  is in :  is a clique. Therefore,  is covered by the three disjoint cliques ,  and . 
	
		If we are in the strict semantics, the sets  are disjoint and  so are the cliques. If we are in lax semantic,  is covered by the disjoint cliques ,  and . In both cases,  has a -clique cover.


\end{proof}






\subsection{The case of quantified formulas}

In this section we show that  existential quantification even without disjunction and even in the empty vocabulary makes the model checking problem hard  for both dependence and independence logic. 


\begin{theorem}\label{Exists+wedge} 
There exists a formula  with -complete model-checking problem where  and  is a conjunction of two dependence atoms.



\end{theorem}

\begin{proof}
Before giving the proof, let us consider the following problem: Given  a graph  with   vertices, are the vertices of  colourable with  colors (such that no adjacent vertices carries the same color). This problem is easily seen to be a generalization of the well-known -complete -coloring problem (see~\cite{GareyJ1979}). Indeed, let  be an undirected graph with  vertices, let  be the complete graph with  vertices (hence  edges) and let  be the graph with vertex set  with  of size  (hence ) and edge set . Then, it is easy to see that  is -colorable iff  is  
	-colorable.   


Let us now define  the formula  as follows:
	 
	We will reduce the problem of determining whether a graph  with   vertices is -colorable to the model-checking problem of . 

	Let  be a graph with  vertices ,  a first order structure of the empty signature and  be a team such that :

	\begin{itemize}
		\item  and . In other words,  is the decomposition of  in base .
		\item  and . In other words,  is the decomposition of   in base .
		
		\item  if  and .\item , if , or if there is an edge between  and  with . Otherwise .
	\end{itemize}
	
	For example, for  and , we  obtain the following team on the universe :
	
	
	
	We are going to demonstrate that  if and only if  is -colourable.
	
	First the left to right implication. Since  there exists  a mapping   such that .  By downwards closure, we may assume without loss of generality that  is a singleton for all .
Since   holds,   induces a mapping  , by 
, for the unique  such that . If there is an edge between  and , , then . Furthermore,  but  and . Therefore, because the atom   holds,  we must have . Thus   if there is an edge between  and .  This shows that  is a colouring of  with  colours.
	
	Let us then consider the right to left implication. Let  be an  colouring. We extend  to variable  with a new team  such that . The value of  depends only on , which is encoded in , i.e., .
	
	 Let  be two assignments of . Suppose that  but . In this case we must check that  is different from  (because ). Now it holds that  because . Furthermore, since , either  or . Let us suppose . Because , there is an edge between  and  in . Therefore  and .
	\end{proof}

By encoding dependence atoms in terms of conditional independence atoms we get the analogous results for free for independence logic.
\begin{corollary}
There is a formula  of independence logic of empty non-logical vocabulary built with  and  whose model-checking problem is  -complete.
 
\end{corollary}
We will next show that this corollary can be strengthened in the case of independence logic under the strict semantics. In other words, we will now show a  version of Theorem \ref{Exists+wedge}  for pure independence atoms under the strict semantics. Let  be the following formula over signature , where  is a ternary relation symbol and   and  are free variables:



\begin{proposition}\label{StrictInd}
For all propositional formulas  in 3-\pb{cnf}, one can compute in polynomial time a  team , with  and a structure  such that:


under the strict semantics.
\end{proposition}

\begin{proof}
Without loss of generality, let  be a -\pb{cnf} formula over a set  of variables of size . Let , with . We first describe the  relation  built on the domain :


Let .
Finally, team  is the union of the two assignment sets  and  below:


 

Variable  encodes the type of the object in consideration:  for a clause,  for a Boolean variable. The first  assignments deal with variables (hence the value of  is set to , by convention), the last  assignments deal with clauses (hence,  is set to ). It now remains to show that 


where   is the formula .
Suppose first that  is satisfiable and let  be such that . Let  be such that: 
\begin{enumerate}
\item\label{item1-exists-indep} if  and ,  then  if  and  if . Similarly, if , , then   if  and  if .

\item\label{item2-exists-indep} if , , then , for , such that . Such a  always exists since .
\end{enumerate}

Let . It is clear that for all ,  holds by the construction. In , variable  takes only two values  and . Let now



Now to  show the claim  , it remains to  show . For this it suffices to show that . Note that, for all , either  or  belongs to . Also, by the construction, . Indeed, by item~(\ref{item1-exists-indep}), for any  with  and   it holds that , where  and .
Suppose now that there exists  such that  and . Clearly, for such an ,  , for some . Then, by the construction of the function , , for , such that . But then again by the definition of ,  which is a contradiction. Therefore, , , and hence  .

We now prove the other implication.
Suppose   and let  be such that

 
 
 Because  and  for all , it holds that , where  and  are as defined in the first part of the proof above. Together with the definition of the relation , this implies that if  such that  and , then . 
 Define now the Boolean assignment  by
 
 

Now consider  with . By definition of ,  (for some ). Since  there must exists  s.t. ,  and, also immediately by the construction of ,  and . The definition of  implies that  and that clause  is satisfied.
\end{proof}






We end this section by noting that existential quantifiers cannot be replaced by universal quantifiers in the above theorems.

\begin{proposition}\label{forall+wedge} The model-checking problem for formulas of  dependence or independence  logic using only  universal quantification and conjunction is in .
\end{proposition}


\begin{proof} Given , we first transform it into prenex normal-form exactly as in first-order logic \cite{vaananen07}. We may hence assume that  has the form

where  is either a first-order, dependence, or independence atom. Let   be a model,  and   be a team of  with domain . As in   \cite{vaananen07}, the formula  can be expressed by a first-order sentence  when the team  is represented by the -ary relation , that is,

Since  is a first-order definable extension of   it is clear that we can construct a FO-sentence  such that 

holds for all structures  and teams  with domain . The claim follows from the fact that the data complexity of FO is in .
\end{proof}




\section{Inclusion logic under the lax semantics}

Recall that a \pb{cnf} formula  is called dual-Horn if each of its clauses contains at most one negative literal. The satisfiability problem of dual-Horn \pb{cnf} formulas, \pb{dual-horn-sat}, is known to be -complete (see~\cite{GareyJ1979}).  

 In this section we show that the model-checking problem of inclusion logic formulas under the lax semantics  can be reduced to \pb{dual-horn-sat}.
  
For a team , , and , we denote by  the restriction of  to the  variables . In this section,  denotes a relational signature.

  \begin{proposition}\label{FO inclusion lax}
  There exists an algorithm which, given 
   ,  a structure  over , and  a team  such that , outputs a propositional formula  in dual-Horn form such that:  is satisfiable.
Furthermore, when  is fixed, the algorithm runs in logarithmic space in the size of  and .
  \end{proposition}
  
  \begin{proof}
Let  be as above. For any team , we  will consider the set  of propositional variables  for .
Starting from , , and  we decompose step by step the formula  into subformulas (until reaching its atomic subformulas) and different teams , , ... and control the relationships between the different teams by propositional dual-Horn formulas built over the propositional variables issued from ..
Let , where , and . The propositional formula  is now constructed inductively as follows.
  
  As long as , we apply the following rule:  Pick  in  and apply the following rules.
  
  \begin{itemize}
  	\item If  is   with  a literal of  then:   and 
  	.
Clearly, it holds that  iff  is satisfiable.
  	
  	
  	\item If  is   then:   and  
  	
It holds that  iff  is satisfiable.
  	  	
  	
  	\item If  is  , then:  and 
  	
\noindent where the ,  are new  propositional variables (not used in ).
If  then, there exists a function , such that . In other words,   for some team  defined by the solutions  of the constraint  (which define a suitable function ). Conversely, if  for a team  as above defined from , then clearly .

  	
  	\item If  is   , then:  and 
  	  		


  	  	 \noindent where the ,  are new  propositional variables (not used in ). The conclusion is similar as for the preceding case.
  	  	 
  	\item  If  is    then: 
  	and  is unchanged. By definition,  iff .
  	 \item 	If  is    then:  and

\noindent where again the  and ,  are  new propositional variables  (not used in ). Here again,  if and only if  and  for some suitable  and  such that  which is exactly what is stated by the Boolean constraints.
  \end{itemize}
  
  Observe that each new clause added to  during the process is of dual-Horn form, i.e., contains at most one negative literal. Observe also, that applied to some , the algorithm above only adds triples   whose first component is a proper subformula of  and eliminates  . When the formula  is atomic, no new triple is added afterwards.  Hence the algorithm will eventually terminate with . 
 Setting , it can easily be proved by induction that:   iff   is satisfiable.  
 
  



Observe also that each clause in  can be constructed from  and  by simply running through their elements (using their index) hence  in logarithmic space. 
  \end{proof}
  
 \begin{remark} 
The construction of Proposition~\ref{FO inclusion lax} can  be done in principle for any kind of atom: dependence, independence, exclusion, constancy etc. 
To illustrate this remark, one could translate in the above proof a dependence atom of the form  by (using the notations of the proof):
 

 
 The additional clauses are of length two. A similar treatment can be done for independence atoms .  In the two cases however, the resulting formula is not in Dual-Horn form anymore and there is no way to do so (unless ).
 \end{remark}
  
  
  Since deciding the satisfiability of a propositional formula in dual-Horn form can be done in polynomial time we obtain the following already known (\cite{gallhella13}) corollary.
  
\begin{corollary}
  The data complexity of  under the lax  semantics  is in .
  \end{corollary}
  



\section{Inclusion logic under the strict semantics}
In this  section we consider model-checking of inclusion logic formulas  under the strict semantics. By the result of \cite{galhankon13}, inclusion logic with the strict semantics is equi-expressive with dependence logic. The following theorem shows that NP-completeness can be attained with quite simple formulas as in Theorem \ref{Exists+wedge}  combining strict existential quantification and inclusions atoms.
\begin{proposition}\label{StrictInc1}
	Let  be the following formula over signature :
	
	For all propositional formulas  in -\pb{cnf}, one can compute in polynomial time a team  with domain  and a structure  such that:   is satisfiable   under the strict semantics.
\end{proposition}




\begin{proof}
	Let  be a -\pb{cnf} formula over a set  of variables. Let , with .
	
	The domain of the structure  is . The relation  in this structure is:
	
	
	
	Finally, the team  isgiven by the following table: 
		
	
	Now we claim that  is satisfiable, if and only if .
	
Let us suppose that . Then there is an extension  of  to the variables  and  such that 
		. Note that  
	 has, e.g.,  the following shape:
		

Note that for each row of  the variables  and  get exactly one value in .

Let  be the following assignment  of the variables of : let   such that ; we set  if ,  otherwise (i.e., if ). Because  such  exists and . Furthermore,  is the only assignment in  such that  so there is no ambiguity in the definition of .
		
Now we have to check that for every  clause of  there is a literal which is evaluated to  by . Let  be a clause of  and  the element of  such that .  Since  is a literal of  and  , there exists  such that . But  by definition, thus a literal of  is evaluated to  by  and hence  satisfies .
	
Suppose then that  is satisfiable and let  an assignment of the variables of  which satisfies . We extend  to a team  over variables  as follows: for , if  is such that  we set  if , and   otherwise. Furthermore, .
		
		For , if  is such that , we set  and  where  is a literal of  which is evaluated to  by . It is now easy to check that 
		, and hence  . 
\end{proof}

The next proposition shows that -completeness can be also attained by combining strict disjunction with inclusions atoms.
\begin{proposition}\label{StrictInc2}
	There exists formulas  built with  such that the model checking problem for  under the strict semantics is NP-complete.
	
\end{proposition}

\begin{proof} Membership in  is obvious. For hardness, we exhibit a polynomial time reduction to . 

Let  be the following formula over variables :



	We will show that for all   positive -\pb{cnf} formulas , one can compute in polynomial time a team  and a structure  such that:
	
Let  be a positive -\pb{cnf} formula over a set  of variables. Let .  Recall that  is an instance of  the problem -in--SAT if and only if there is a truth assignment such that each  clause of  has exactly one true variable.
	
The domain of the structure  is . The team  is , see Table~\ref{Y}.
\begin{table}
	
\caption{\label{Y}}
\end{table}

	Now we claim that  is an instance of \pb{1-in-3-sat}, if and only if 
	.

Let us suppose that , i.e., there exists a partition of  into three subsets ,  and  such that 
			
			
			We will define  an assignment  over  witnessing that  is an instance of \pb{1-in-3-sat}: for  there exists a unique  such that . If , we set  and  otherwise. As  is unique and the sets   are disjoint (because we are in the strict semantics),  is well defined.
			
			We have to check that for any ,  exactly one of the  variables of clause  is evaluated to  by the assignment .
			
		Now 	,  and  imply that every assignment  such that  must be in . Therefore .	Similarly .

			The variable    stores the index of a clause. Because , every clause  has an assignment  such that , and the same holds for the sets    and in . Thus the claim follows.
			
Let  be an assignment witnessing that  is an instance of \pb{1-in-3-sat}. Define a partition of  into  as follows. Let .
		
		\begin{enumerate}
		    \item\label{c1}  If  and , we assign . If  and  we assign ,
		    \item\label{c22} 	For every , . Let  be such that . We send the unique  such that  and  to  and assign exactly one of the remaining two such assignments  to  and ,\item\label{c3} If ,  we send  to . 
		\end{enumerate}
		 By \eqref{c1} and \eqref{c22} it now  clearly holds that  . Furthermore, by \eqref{c22} and \eqref{c3} it holds that  and    , for  , respectively.


		
		
		
\end{proof}



\section{Conclusion}
On this paper we have studied the  tractability/intractability frontier  of data complexity of both quantifier-free and quantified  dependence, independence, and inclusion logic formulas. Furthermore, we defined a novel translation of  inclusion logic formulas into dual-Horn propositional formulas, and used it to show that the data-complexity of inclusion logic is in PTIME. In a paper under preparation we shall consider similar questions for quantifier-free formulas containing, in addition to dependence and independence atoms, also so-called anonymity atoms.
Although our results shed light on the tractability/intractability frontiers studied in this article, many open  questions related to the data-complexity of quantifier-free independence logic formulas remain. The general goal is to find fragments where we can prove PTIME/NP-complete dichotomy results. Our results on disjunctions of independence atoms show that the data-complexity question is more complex than merely the length of the disjunction, as is the case with dependence atoms. In a different direction,  it is an open question whether Theorem 13 holds under the strict semantics. 


\section*{Acknowledgements}
The second author was supported by the Academy of Finland grant 308712. The fourth author was supported by the Faculty of Science of the University of Helsinki and the Academy of Finland grant 322795. This project has received funding from the European Research Council (ERC) under the European Union’s Horizon 2020 research and innovation programme (grant agreement No 101020762).
\bibliographystyle{abbrv}
\bibliography{biblio}








\end{document}
