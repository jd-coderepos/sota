









\documentclass[journal]{IEEEtran}









\usepackage{ifpdf,color}







\usepackage{cite}







\ifCLASSINFOpdf
   \usepackage[pdftex]{graphicx}
\else
\usepackage[dvips]{graphicx}
\fi






\usepackage[cmex10]{amsmath}






\usepackage{algorithmic}





\usepackage{array}


























\usepackage{url,subfigure}








\hyphenation{op-tical net-works semi-conduc-tor}

\newcommand{\RR}[1]{\textcolor{black}{#1}}
\newcommand{\ca}[1]{\textcolor{black}{#1}}
\newcommand{\DA}[1]{\textcolor{black}{#1}}

\begin{document}
\title{Gait Velocity Estimation using ``time-interleaved'' between Consecutive Passive IR Sensor Activations}


\author{Rajib~Rana,~\IEEEmembership{Member,~IEEE,}
        Daniel~Austin,~\IEEEmembership{Member,~IEEE,}
	Peter~Jacobs,~\IEEEmembership{Member,~IEEE,}
	Mohanraj~Karunanithi,~\IEEEmembership{Member,~IEEE,}
        and~Jeffrey~Kaye,~\IEEEmembership{Member,~IEEE}\thanks{R. Rana is with the Centre for Health Sciences Research at the University of Southern Queensland, Australia (For contact information visit sites.google.com/site/rajibranadr/).}\thanks{D. Austin and J. Kaye are with the Department of Neurology, and P. Jacobs is with the Departments of Biomedical Engineering and Otolaryngology, Oregon Health \& Science University, Portland, OR 97239 USA.}
\thanks{M. Karunanithi is with the Australian E-Health Research Centre, CSIRO, Australia.}
}





\markboth{Journal of \LaTeX\ Class Files,~Vol.~11, No.~4, December~2012}{Shell \MakeLowercase{\textit{et al.}}: Gait velocity estimation from Passive IR sensor activations}














\maketitle

\begin{abstract}
Gait velocity has been consistently shown to be an important indicator and predictor of health status, especially in older adults. It is often assessed clinically, but the assessments occur infrequently and do not allow optimal detection of key health changes when they occur. In this paper, we show that the “time gap” between activations of a pair of Passive Infrared (PIR) motion sensors installed in the consecutively visited room-pair carry rich latent information about a person's gait velocity. We name this time gap ``transition time'' and show that despite a six second refractory period of the PIR sensors, transition time can be used to obtain an accurate representation of gait velocity.  

Using a Support Vector Regression (SVR) approach to model the relationship between transition time and gait velocity, we show that gait velocity can be estimated with an average error  less than  cm/sec.  This is demonstrated with data collected over a 5 year period from  older adults monitored in their own homes. 

This method is simple and cost-effective, and has advantages over competing approaches such as: obtaining 20-100x more gait velocity measurements per day and offering the fusion of location-specific information with time stamped gait estimates.  These advantages allow stable estimates of gait parameters (maximum or average speed, variability) at shorter time scales than current approaches.  This also provides a pervasive in-home method for context-aware gait velocity sensing that allows for monitoring of gait trajectories in space and time.



\end{abstract}

\begin{IEEEkeywords}
Unobtrusive monitoring, gait velocity, walking speed, passive infrared (PIR) motion sensors, transition time, support vector regression.
\end{IEEEkeywords}






\IEEEpeerreviewmaketitle



\section{Introduction}
Gait velocity, also referred to as the speed of walking, is both an important \ca{indicator of individuals' current health} state and predictor of future adverse cognitive and physical health outcomes.  Gait velocity can distinguish between patients with dementia and healthy controls~\cite{bruce2012relationship} and has been shown to decrease prior to cognitive impairment~\cite{buracchio2010trajectory} and in Alzheimer's disease~\cite{Goldman1999}.  Decreased gait velocity is prevalent in dementia~\cite{Beauchet2008} and is predictive of future hospitalization~\cite{Studenski2003}.  Gait is known to require substantial cognitive resources~\cite{Faulkner2006} and gait velocity may be directly related to several cognitive processes such as attention and executive function~\cite{Holtzer2006,Holtzer2012}.  Gait has also been linked to risk of falls~\cite{Kelsey2012,Cuaya2013} and risk of future disability~\cite{Guralnik2000a,studenski2011gait}.

Gait velocity is most commonly assessed clinically with a stopwatch timed walk - such as the 25-ft timed walk~\cite{Larson2013} - although more comprehensive assessments are also used.  A large shortcoming of clinic-based assessments is infrequent test administration.  Often, 6 months to a year or more passes between assessments~\cite{hagler2010unobtrusive}, making it difficult to detect acute changes when they occur or to distinguish between abrupt changes in function and slower changes occurring over time.  Many pervasive computing approaches have been successfully proposed and validated to estimate gait velocity that overcome the infrequency problem of the current clinical assessment methodologies. The existing approaches can be grouped into two categories.  

The first category is based on instrumenting the body with a worn device, such as an accelerometer~\cite{Culhane2005,Gietzelt2013,Dalton2013}.  Accelerometry is accurate and effective but is not well suited for studies lasting more than a few days without substantial requirements of the patients or research/clinical staff participation.  In particular, patients must remember to wear the device, 
\ca{position it correctly on the body, regularly charge the device, and follow any procedures needed for ensuring download and transmission of the data.}
This is especially problematic in older and cognitively impaired populations who may benefit the most from long-term gait monitoring, as they may forget or be unable to perform the tasks required for obtaining reliable and continuous data.  Research staff can mitigate some of these shortcomings at the expense of increased cost, however, this drastically reduces scalability for big studies requiring large-scale device deployment.  

Accelerometry is also not location-aware (unless additional sensors are added to the system or initial coordinates are known apriori) and thus gait velocity estimates obtained from accelerometry cannot immediately be associated with the activity performed during the walking event (e.g., whether someone is headed toward the bathroom). 

\ca{Approaches to device-free velocity tracking are typically} based on instrumenting the environment (most often the home) with unobtrusive sensors.  Strategies in this category include the use of ``restricted" infrared sensors arranged in a walking line~\cite{hagler2010unobtrusive,kaye2012one}, referred to as a ``sensor-line'' or \ca{a camera-based sensors such as a Microsoft Kinect system} to estimate gait~\cite{Clark2013,Stone2013,sivapalan2011compressive}.  Both the camera and sensor-line methods overcome some of the issues posed by the body-worn devices and are readily deployable for long-term monitoring and in large scale studies.  However, these methods have limitations in capturing unobtrusive monitoring of gait.  For example, both systems only detect walks when a resident passes within the field-of-view of the sensors.  This can result in data sparsity when a subject passes through the instrumented area infrequently. The higher installation and maintenance cost (as compared to the method presented below) and dedicated equipment required for both methods tends to restrict the deployment of sensor-lines or cameras to a single place in a person's home. \RR{As a result, these methods can be used to measure gait velocity typically at one \emph{fixed} location and thus may miss important features of gait that occur in other locations in the home.}
In addition, camera-based methods can also suffer issues with occlusion.  This can further limit the ability to measure gait in several common circumstances.  For example, if a resident moves their furniture or changes behavior (e.g., they regularly choose a path through the room for which the camera focal point has not been optimized). 
Lastly, depending on the use case, camera-based systems may be limited by a loss of privacy~\cite{demiris2009older,boise2013willingness}.


We demonstrate in this paper that the time-interleaved between two consecutive passive infrared motion sensor (placed in the consecutively visited room pair) activations can accurately estimate gait velocity.  The proposed approach offers several improvements over the current state of the art. 

First, since velocity is measured every time a person moves between rooms in their home, the proposed method can gather 20-100x more estimates of gait velocity per day than other unobtrusive systems. \RR{In Fig.~\ref{fig:statGAITvsTT} we present the number of measurements for transition time and gait velocity for one representative subject. For this subject, the average number of gait velocities measured using the walking line was approximately six per day, whereas the average number of velocities measured using transition time was approximately 121 per day.”} 



\begin{figure}
\centering
\includegraphics[width=0.95\linewidth]{statGAITvsTT}
\caption{Comparison of the number of measurements per day: Gait Velocity versus Transition Time.}
\label{fig:statGAITvsTT}
\end{figure}


Our approach is also less sensitive to sensor placement. We demonstrated that room transition times can be used to accurately estimate gait velocity than other approaches and it uses sensors that are usually already deployed in a smart home.  

Our approach also provides location-specific gait velocity (e.g., the speed to or from the bathroom or phone) with less than \RR{2.5 cm/s} of error, on average. The estimation model is developed using sensor-line estimated gait velocity as ground truth for proof of concept. Once developed, the model can be used without having another gait measurement system simultaneously deployed. This location-specific system will allow further investigation into the interplay between gait velocity and context, which may account for some observed variability in speed throughout the day. 




 Finally, studies have indicated that measuring fast vs. slow vs. average walking speeds~\cite{Fitzpatrick2007}  and measuring variability in these walking speeds~\cite{Ijmker2012} may be critical in passively assessing patient health using in-home monitoring.  The approach that we present using transition times enables the stable measurement of fast, slow, and average walking speeds throughout the home.   The fact that this method can acquire far more velocity estimates than a walking line located in a restricted location in the home could potentially enable earlier detection of movement-related health changes.



\section{Study Description }
In this study, we used a data set collected from 74 non-demented older (mean age  85.9 years) men and women living independently who were part of the Intelligent Systems for Assessing Aging changes (ISAAC) project conducted by Oregon Center for Aging and Technology (ORCATECH) Living Laboratory. The ISAAC study is a longitudinal community cohort study using an unobtrusive home-based assessment platform installed in the homes of many seniors in the Portland, OR (USA) metropolitan area and is described in detail elsewhere~\cite{Kaye2011}.  \DA{Subjects living alone in the ISAAC cohort lived in a variety of different home sizes with 5.7 rooms on average (SD=2.2 rooms) and a range from studio style (3 rooms) to large houses (16 rooms).  For these homes, the average size is 900 ft (SD=448 ft) with a range from 324 ft to 3560 ft.  All subjects provided written informed consent and the study was approved by the Oregon Health \& Science University Institutional Review Board (OHSU IRB 2353).}


\begin{figure}
\centering
\subfigure[]{
\includegraphics[type=pdf,ext=.pdf,read=.pdf,width = .8\linewidth]{home.13}
\label{fig:home.13}
}
\subfigure[]{
\includegraphics[width = 0.3\linewidth]{ceilingImage}
\label{fig:ceilingImage.eps}
}
\subfigure[]{
\includegraphics[width = 0.3\linewidth]{walkingImage}
\label{fig:walkingImage.eps}
}
\caption{(a) Sensor distribution in participant's home. Squares in red and cyan are representing passive infrared sensor. Blue rectangles are reed switches (b) sensor array in the ceiling (c) Schematics of a person walking through the sensor line.}
\label{fig:miscelleneous}
\end{figure}

\subsection{Data Collection and Preprocessing}
\subsubsection{Data}
\DA{\RR{Two} types of data were used from each participant.  First are the \emph{transition times}, estimated as the time between sensor firings from X10 motion detectors (MS16A; X10.com) positioned in adjacent rooms. Each room in each home has one (unrestricted) motion detector (Fig.~\ref{fig:home.13} illustrates the sensor placement in one participant's house) placed such that the 30 degree by 90 degree field-of-view (FOV) of the sensor spans as closely as possible the room in which the sensor is installed.  Different homes have different room sizes and floor plans, therefore the distance between the FOV of sensors in adjacent rooms depends on characteristics of the specific home in which it is installed.  This is why the support vector regression (SVR) must be trained separately for each pair of sensors as discussed below; the actual distance between the FOV of sensors placed in adjacent rooms of a home is not known a priori. All motion sensors used in this study have a six second refractory period and are time stamped to the nearest millisecond.  
Participants self-reported via an online survey such events as when overnight visitors were present in the house, days in which technical staff visited the home, or times when sensors did not function properly (e.g. due to a dead battery for  example). Data from days with overnight guests, when staff visited the home, or with sensor outage were excluded in our analysis. \RR{In total, we had  days of data from our  participants. \ca{On average, we had 630.32325.8 days of data from each participant.}}} 


The second type of data is in-home gait velocity estimated from a \emph{sensor-line}~\cite{hagler2010unobtrusive}.  This data is used as the ground-truth for the purpose of this study. The in-home gait velocities are estimated using four infrared sensors with restricted fields of view positioned in a linear array on the ceiling \DA{with 2 ft between each sensor (See Fig.~\ref{fig:ceilingImage.eps})}.  As a person moves underneath each sensor in the array, they fire in order (See Fig.~\ref{fig:walkingImage.eps}).  The time between the firing and the position of each sensor are used to estimate the gait velocity. There are various events that can cause variability in this method of estimating walking velocity.  First, a participant may not pass through the sensor at a near-constant velocity (e.g., they may pass part way through, stop, then continue walking). Second, an undetected issue with one or more sensors may cause the data to be corrupted in a way that influences velocity estimation.  As a result, the restricted-view sensors can yield poor estimates of gait velocity that manifest as outliers in the data set.  





\begin{figure}[ht]
\centering
\subfigure[]{
\includegraphics[width = .45\columnwidth]{histogram1}
\label{fig:histogram}
}
\subfigure[]{
\includegraphics[width = .45\columnwidth]{QQplot_1}
\label{fig:QQplot}
}
\caption{a) Distribution  of the gait velocity. b) Quantile-Quantile (QQ) Test of ``normal distribution'' of gait velocity estimated from restricted sensor line. c) Outlier Detection.}
\label{fig:walkSpedAnalysis}
\end{figure}


\subsubsection{Outlier Removal and Feature Selection}
\label{sec:DataAnalysisAndFeatureSelection}

In order to estimate gait velocity from transition time, we remove outliers from these two quantities, followed by extracting features from them. Below we describe the outlier removal and the feature selection methods in detail.


\noindent{\bf Gait-Velocity:}
In order to exclude the outliers we only included data within two standard deviations of the mean velocity. After outlier removal, the majority of the values were approximately normally distributed with all measurements less than the physiologically reasonable value of  cm/sec (Fig.~\ref{fig:histogram}). 



There is not a one-to-one mapping between in-home gait velocities and transition times since they are measured in different locations and with different sensors, thus do not co-occur. In other words, we do not get a transition time and an in-home velocity that both correspond to the same movement at the same time.  Because of this, we aggregated the in-home velocity estimates across an entire day and used the mean gait velocity as the target to be estimated from the transition time. We use the mean of gait velocity as the feature since  we conjecture that gait velocity is well represented by a normal distribution. 


We use Q-Q plot~\cite{wilk1968probability} ("Q" stands for quantile) to verify gait velocity is normally distributed. Q-Q plot is a graphical method for comparing two probability distributions by plotting their quantiles against each other. We plotted the quantiles of the empirical in-home gait velocity distribution with that of a standard normal distribution. 
If the distributions are ``linearly related'', the points in the Q-Q plot will approximately lie on a line, but not necessarily on the line . 

In Fig.~\ref{fig:QQplot}, we plot the Q-Q test result for a representative participant. We found that the points in the Q-Q plot mostly lie on a line with a regression coefficient , and we observed this trend across all the participants. Therefore, the in-home estimated gait velocity is well approximated by a normal distribution. For normally distributed data, the sample mean is identical to the mean of the population distribution. Therefore, we use the \emph{sample mean} as a representative feature for the in-home estimated gait velocity. 


\noindent{\bf Transition Time:}
Some transitions between home locations do not occur frequently enough to permit a characterization of the distribution of the transition times.  This makes them unsuitable for use in velocity estimation.  This can occur, for example, when a room is infrequently visited such as a guest room. Note that the room transitions were mostly between adjacent rooms. However, because a sensor will occasionally not fire when a resident passes within the sensor’s field of view (due to the refractory period, for example), we will occasionally see instances where sensors in non-adjacent rooms are considered as transitions. To identify and remove these infrequent transitions, we removed all room pairs with  or fewer observed transitions (\ca{over the entire period of data collection with the exception of when sensors were not functional or when visitors were reported to be in the home}) where  was an empirically chosen threshold which allows reliable estimation of distributional parameters (e.g., the mean or different percentiles). We report the percentage of transitions for various room pairs in Fig.~\ref{fig:roomVisitFrequency} for one participant. Data taken from  days were used to generate this plot. 
Some transitions, for example, kitchen to the bathroom are very rare and should not be modeled due to the large statistical variability resulting from small sample sizes.


The way transitions are measured can confound the speed at which a person travels between rooms and the time they spend in a room (dwell time).  This is because there is a refractory period of 6 seconds in the X10 motion sensors.  A person could, for example, trigger a motion sensor in the kitchen, wait 5 seconds before leaving the kitchen to go to the living room and then trigger the living room sensor.  Since the kitchen sensor could not fire again before the person left, the measured transition time would appear to be long when, in fact, it really represents a combination of dwell time and transition time.  For this reason, the mean value of the transition time may not be the best feature describing movement speed (as was used for the in-home gait velocity). This is demonstrated in Fig.~\ref{fig:skewedDistribution}, where we observe that
transition times are skewed.  Intuitively, smaller percentiles of the transition time distribution should be less likely to include dwell time. We, therefore, consider the \emph{, , and  percentile, along with the  first quartile, mean, and median }as potential features best summarizing the movement part of the transition time distribution.

\begin{figure}
\centering
\includegraphics[width = 0.8\linewidth]{roomVisitFrequency_7}
\caption{Percentage transitions in various room pairs.}
\label{fig:roomVisitFrequency}
\end{figure}


\begin{figure}[ht]
\centering
\subfigure[Kitchen to sensor line.]{
\includegraphics[width = 0.4\columnwidth]{skew2}
\label{fig:svrFig1}
}
\subfigure[Living room to walk-in-closet.]{
\includegraphics[width = 0.4\columnwidth]{skew3}
\label{fig:svrFig3}
}
\caption{Distribution (skewed) of transition time for two randomly chosen room pairs.}
\label{fig:skewedDistribution}
\end{figure}

\section{Support Vector Regression for Gait Velocity estimation}
\label{subsec:svr}



Our approach to estimating gait velocity is based on learning the functional relationship between the transition times and gait velocity.  To learn this relationship, we used a support vector regression model, which is widely used for prediction~\cite{rana2013feasibility,rana2011adaptive}.

The complete description of SVR is outside the scope of this paper. However, we will provide intuition sufficient to understand the working principles of SVR. Consider a training set , where s are the mean transition time and s are mean gait velocity. Support vector regression computes the function  that has the largest  deviation from the actual observed  for the complete training set.
Errors below  are not penalized, but for errors above  certain amount of loss would be associated with the estimate. The loss rises linearly with the absolute difference between  and  above .



For simplicity, let us assume a linear relationship between the variables of the form , where  (weight vector) and  are parameters to be estimated. 
When , the functional relationship between  and  is least complex or in other words, there is no relationship between  and . Therefore, the overall error is very high. On the other hand when  is very large, the error is minimized but the function  looses generalization; consequently, the arrival of new points may be highly penalized. 

Support vector regression seeks to minimize the value of  without incurring much estimation error. This is done by solving the following optimization problem\eqref{eqn:lossFunction}:



Note that in many cases the relationship between the variables is non-linear. 
In those cases the SVR method needs to be extended, which is done by transforming  into a feature space . The feature space linearizes  
the  relationship between  and , therefore, the linear approach can be used to find the regression solution. A mapping function or namely Kernel function is used to transform into feature space. 

There are four different functions which are frequently used as kernels within support vector regression: linear, RBF (Radial Basis Function), polynomial, and sigmoid. When the feature set is small, the RBF kernel is preferable over others. We use only one feature: transition time, therefore we use the RBF Kernel. However, we empirically verify that the RBF kernel performs better than the linear kernel. 







\section{Results and Discussion}
\subsection{Simulation Setup}
We used the Matlab library LIBSVM~\cite{chang2011libsvm} to implement SVR. There are two functions: \texttt{svmtrain} and \texttt{svmpredict} for training and testing, respectively. We construct input feature  using the transition time features discussed in Section~\ref{sec:DataAnalysisAndFeatureSelection}. The function \texttt{svmtrain} uses the input features to estimate  and . The prediction method \texttt{svmpredict} then uses these values and some other parameters (for details please review~\cite{chang2011libsvm}) to estimate the gait velocity. We used five-fold cross-validation to assess the model fits via RMS estimation error.  We input various features such as ,  and  percentile, first quartile, mean and median of transition time to estimate the gait velocity.

\subsection{Results}
We report the transitions producing minimum estimation error while using different features in Table~\ref{tab:filter_coeff_2}. 
We observe that the transition producing the minimum estimation error is variable. For example, the transition from Bathroom to Living is the best predictor for the  percentile, whereas Kitchen to Refrigerator is the best predictor for the  percentile. We even observe that this also varies person to person. For example, for participant  the transition from Bathroom to Living is the best predictor for the -percentile, but this may not be true for participant .  This is, however, realistic since transition times can be a person or home-specific.
\begin{table}[t]
\small
\centering
\caption{Transitions producing the minimum estimation error.}
\setlength{\tabcolsep}{1pt}
\resizebox{6cm}{!} {
\begin{tabular}{|c|c|} \hline
\parbox{3.5cm}{\bf  Features (Giving minimum estimation error)}&{\bf Room pair}\\ \hline
 Percentile&Bathroom to Living  \\ \hline
 Percentile & Kitchen to Refrigerator \\ \hline
 Percentile &Bathroom to Living  \\ \hline
First Quartile&Kitchen to Refrigerator\\ \hline
Mean&Refrigerator to Kitchen\\ \hline
Median&Bed to Living\\ \hline
\end{tabular}
}
\label{tab:filter_coeff_2}
\end{table}

\begin{figure}
\centering
\includegraphics[width = 1\linewidth]{nonClinicalNew1}
\caption{Gait velocity predicted from transition time across 74 participants. Various transition time features are used along the x-axis. }
\label{fig:WalkSpeedVersusTransitionTime}
\end{figure}

In order to aggregate the estimation accuracy across all the participants, for each participant we calculate the estimation error for various features and average the error over all participants. The mean and standard deviation of these estimation errors for various features are presented in Fig.~\ref{fig:WalkSpeedVersusTransitionTime}. 

We observe a minimum at the first quartile or  percentile. \DA{Quantitatively, the 25th percentile produces an average estimation error less than \RR{} cm/s}. \RR{Intuitively, the transitions below the  percentile may not be typical; we speculate that these transitions may be observed when a person rushes from one room to the other room. Furthermore, the transitions above  percentile may be more likely to incorporate dwelling time possibly caused by the six-second refractory period. 
}

\RR{Finally, we plot the mean ground truth gait velocity (measured using sensor-line) and the mean predicted (using  percentile of transition time) gait velocity for all the 74 participants in Fig.~\ref{fig:compareTwoMean}. Applying linear regression on the points we find that the points fit () a straight line with slope  and intercept  with narrow  confidence intervals. 
For perfect estimation, all the points should be aligned to the line  (slope  and intercept 0). Therefore, our estimation performance is very close to ideal and approximately unbiased.}

\DA{Fig.~\ref{fig:compareTwoMean} also demonstrates the variability of the proposed estimator as a function of velocity.  The dotted gray lines in Fig.~\ref{fig:compareTwoMean} represents 95\% confidence intervals of the estimates, suggesting a reasonable spread of individual estimates around the average.  Additionally, the slight widening of the confidence intervals at the lowest and highest speeds indicate that these regions of the velocity curve are estimated less precisely, largely because there are fewer instances of the slowest and fastest walks with which to train the estimator.}





\begin{figure}
\centering
\includegraphics[width=.8\linewidth]{2STDFinal}
\caption{The relationship between predictive and true gait velocity showing a highly linear
and strong correlation.}
\label{fig:compareTwoMean}
\end{figure}

\subsection{Discussion}
\begin{table*}
\centering
\caption{Comparison of various gait velocity estimation methods.}
\setlength{\tabcolsep}{1pt}
\resizebox{12cm}{!} {
\begin{tabular}{|l|l|c|c|c|}\hline
{\bf Method}&{\bf Accuracy}&{\bf Device-free}&{\bf Location/context} &{\bf Privacy} \\ 
&&&{\bf aware}& {\bf Conscious}\\ \hline
Room transitions &2.5 cm/s&Yes&Yes&Yes \\
(Current manuscript)& & & & \\ \hline
Accelerometer~\cite{Dalton2013} &4 cm/s compared to &No&No&Yes\\  
&GAITRite&&&\\ \hline
Sensor line~\cite{hagler2010unobtrusive}]&1.1 cm/s compared to &Yes&\parbox{3.5cm}{Yes(Limited to location where installed)}&Yes\\ 
&GAITRite&&&\\ \hline
Video~\cite{wang2013}& \% difference from&Yes&\parbox{3.5cm}{Yes (Limited to location where installed)}&No \\ 
&GAITRite (0.18\%)&&& \\ \hline
In-home gait mat~\cite{low2009initial}&Close to &Yes&\parbox{3.5cm}{Yes (Limited to location where installed)}&Yes\\ 
& GAITRite&&&\\ \hline
\end{tabular}
}
\label{tab:comparisonTable}
\end{table*}

The results demonstrate that transition time can be used to accurately estimate gait velocity.  \ca{A comparison with other related technologies is presented in Table~\ref{tab:comparisonTable}. The other studies mentioned in Table~\ref{tab:comparisonTable} appear to have validated gait velocities against clinical gold standards but have not all reported results for the same accuracy measures, making a direct comparison of accuracy difficult. However, Table~\ref{tab:comparisonTable} clearly shows that our proposed method offers high accuracy while being location-aware and non-invasive.}



Our methodology has three main advantages over other in-home sensing based technologies. First, we can estimate a gait velocity every time a person switches rooms in their home.  This can produce substantially more estimates of gait velocity than are available from competing methodologies.  Using the sensor-line as an example, we typically measure between 0 - 20 gait velocities a day.  On the other hand, a resident can move between rooms from 200-500 times a day.  As a result, measuring gait speed through transitions can provide a very dense set of velocity measurements over the course of each day, which can improve estimates of aggregated parameters (e.g., mean or maximum daily walking speed) and provide a rich set of time-specific movement information.   


Second, our system does not require expensive dedicated sensors, such as the camera-based sensors. The cost reduction from not having dedicated sensors for gait velocity estimation can be considerable, especially when scaling to many homes.  


Finally, or perhaps most importantly, because the proposed system can estimate a gait velocity every time someone switches rooms, the proposed method produces multiple estimates of different {\it location-specific} gait velocities throughout the course of the day.  This context-rich data will allow future studies to explore and account for variability in gait velocity associated with the location, which is not currently possible on a large scale.  Combining this with the time information (as mentioned above) will further allow the study of time-space velocity trajectories - the study of how fast and when people move through their environment.

There are also some shortcomings with the proposed methodology.  \ca{First, we focused on single resident homes.  Living alone is a relatively common scenario for our target population of older adults~\cite{greenberg2011profile}, but may limit generalizability to other populations.  However, our approach may be extended to dual or multiple occupancy homes in the future as recent studies have shown that passive motion sensors can be used for resident identification and disambiguation of passively collected data~\cite{Banerjee2012,Austin2011}. Besides multiple residents, pets can also potentially interfere with our proposed system. We have not conducted any experiments with pets, but it is reported that the range of the motion detector can be altered to eliminate motions close to the floor such as from pets~\cite{caudle2004illumination}.}

Another shortcoming is that the proposed method requires that the model is trained using ground truth gait velocity collected within each residents' home.  In this study, a sensor-line was used (although a camera-based or other method could also have been used). In practice, deploying a sensor-line increases the cost of the system (4 extra sensors) and maintenance expenses (battery changes, repair or replace etc.). However, a sensor-line (or other ground truth system) is required for a reasonably short period of time to train the system, so it could conceivably be removed and deployed in a different residence after the training period is complete.  

In order to determine a sufficient training period, we generated Fig.~\ref{fig:LearningTimeNew}. In this figure we report the  estimates (as in Fig.~\ref{fig:compareTwoMean}) for the various training period. We observe that when the mean training period is around 100 days, the  estimate is , which resembles a good estimation of the gait velocity. On average we have 630 days of data from Each participant. Therefore, we can construct an accurate model using approximately 15\% of the time period. 


An alternative solution of the sensor-line produced gait-velocity training would be to use the transition times themselves as a proxy for true velocities.  This has been done previously to estimate mobility~\cite{Austin2014}.  In health monitoring applications, clinicians are often more interested in identifying the change in gait velocity due to a health event, suggesting that absolute gait velocity may be less important than change.  For example, if a person has fallen, their gait velocity may change relative to the prior week.

Our approach works best (as with all in-home based systems) in populations who spend much of their time at home.  Because of this, a truly ubiquitous monitoring system would use our approach in conjunction with other in-home technology and ambulatory approaches such as GPS or wearable devices for truly pervasive and ubiquitous health monitoring. 





\begin{figure}
\centering
\includegraphics[width=0.65\linewidth]{LearningTimeNew}
\caption{Accuracy Versus Learning Period}
\label{fig:LearningTimeNew}
\end{figure}

\section{Conclusion}
In this paper, we demonstrated that the time-interleaved between consecutive PIR sensor activations located in consecutively visited room pairs, can be used to accurately estimate gait velocity.   Using support vector regression for estimating gait velocity from the transition times, we show that the estimation accuracy of the approach is very high; quantitatively we can estimate the gait velocity with \RR{less than 2.5 cm/sec error}.  This is demonstrated using data from  participants collected over a five-year period.  Using transition times to estimate gait velocity has several advantages over competing approaches such as increased frequency of measurements, less sensitivity to sensor placement, and the ability to monitor time and location specific velocities.  In summary, the gait estimation approach described in this paper is simple, cost-effective, and highly accurate. It can be readily implemented in smart homes facilitating an accurate assessment of gait velocity, which has been shown to be an important predictor and indicator of healthy aging.













\ifCLASSOPTIONcaptionsoff
  \newpage
\fi







\bibliographystyle{IEEEtran}
\bibliography{referenceGaitTransitionTime,sigProc}










\vspace{-1.2 cm}
\begin{IEEEbiography}
[{\includegraphics[height=.9in,keepaspectratio]{RajibRana}}]{Rajib Rana, PhD}
received the PhD degree in computer science and engineering from the University of New South Wales, Sydney, Australia, in 2011. Currently, he is working as a Research Scientist in the Centre for Health Sciences Research, at USQ. He has expertise in the field of  signal
processing (compressive sensing, sparse coding),  optimization and machine learning. He is passionate about data-driven health care for maintaining wellness. His research interests span developing reliable data extraction algorithms from low-power mobile devices, and harness this burgeoning data stream to model behavioral, physiological, physical and emotional states of human being.
\end{IEEEbiography}
\vspace{-1.4 cm}
\begin{IEEEbiography}
[{\includegraphics[width=1in,height=1.25in,keepaspectratio]{austin}}]{Daniel Austin, PhD}
received the B.S. degree in electronic engineering technology from the Oregon Institute of Technology, Portland, OR, USA in 2006, the M.S. degree in electrical engineering from the University of Southern California, Los Angeles, CA, USA in 2008, and the Ph.D. degree in biomedical engineering at the Oregon Health \& Science University, Portland, OR, USA.
	
He is an Adjunct Assistant Professor in the Department of Neurology at Oregon Health \& Science University and a Data Scientist at AppNexus.  His research interests include computational modeling of human behavior with applications to the early detection of physical and cognitive decline.
\end{IEEEbiography}
\vspace{-1.4 cm}
\begin{IEEEbiography}
[{\includegraphics[width=1in,height=1.2in,keepaspectratio]{PeterJacobs-small.jpg}}]{Peter G. Jacobs, PhD}
is an Assistant Professor in the Department of Biomedical Engineering at Oregon Health \& Science University (OHSU).  He received his in electrical engineering from OHSU, his masters in electrical and computer engineering from the University of Wisconsin in Madison, and his bachelors in engineering from Swarthmore College. His interests are in the area of researching, designing, and translating novel medical devices and systems for use within natural living environments.  Numerous projects are ongoing in the lab which fit broadly within the areas of (1) medical device and algorithm development primarily in the area of diabetes technologies and (2) ubiquitous computing for delivering home-based health care solutions including wearable and ambient systems.                 

His current appointment is as an Assistant Professor in the Departments of Biomedical Engineering and Otolaryngology at Oregon Health \& Science University and as a Research Investigator at the Portland VA Medical Center, in Portland Oregon at the National Center for Rehabilitative Auditory Research.  His current interests are within the fields of telemedicine, hearing sciences, biomedical device design, and signal processing.

Dr. Jacobs is a member of the IEEE Engineering in Medicine and Biology Society, the IEEE Signal Processing Society, the Acoustical Society of America, and the Association for Research in Otolaryngology.  
\end{IEEEbiography}
\vspace{-1.4 cm}
\begin{IEEEbiography}
[{\includegraphics[width=1in,height=1.2in,keepaspectratio]{Karunanithi_Mohan_AEHRCgray}}]{Mohanraj Karunanithi, PhD}
obtained his doctorate in Biomedical Engineering at the University of New South Wales. He has worked in cardiac research for 10 years on ventricular function. He has worked and led cardiac function research in a leading cardiac research institute in Australia. He is currently a Research Group Leader at the Australian Research eHealth Research Centre, CSIRO coordinating and managing research projects, scientists, and students in the area of Integrated mobile health application in the management of chronic diseases, aged care and independently living. The research work includes a recent publication on the world’s first evidence based mobile health delivery of cardiac rehabilitation.  He is a senior member of the IEEE and founding chair of the Engineering Medicine Society in Queensland, Australia.\end{IEEEbiography}
\vspace{-1 cm}
\begin{IEEEbiography}
[{\includegraphics[height=.7in,keepaspectratio]{JKAYE.jpg}}]{Jeffrey Kaye}
Jeffrey Kaye received his M.D. degree from New York Medical College, New York.
He trained in Neurology at Boston University, Boston, MA, subsequently completing a fellowship in Movement Disorders and Neuropharmacology. He then was a Medical Staff Fellow in brain aging at the National Institute on Aging, National Institutes of Health, Bethesda, MD. He is currently the Layton Professor of Neurology and Biomedical Engineering, Director of the Oregon Center for Aging and Technology and the Director of the Layton Aging and Alzheimer’s Disease Center, all at Oregon Health and Science University. His research interests include the design, study and application of ubiquitous unobtrusive technologies and systems for community-based health research, and the identification of behavioral and biological biomarkers of healthy aging. 
\end{IEEEbiography}


\end{document}
