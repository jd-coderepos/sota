\documentclass[a4paper]{article}
\pdfoutput=1 

\usepackage{amsmath}
\usepackage{a4wide}
\usepackage{color}
\usepackage{amssymb}
\usepackage{amsthm}
\usepackage{graphicx}
\usepackage{wrapfig}
\usepackage{algorithmic}
\usepackage{algorithm}
\usepackage{hyperref}
\usepackage{multirow}
\usepackage{verbatim}
\usepackage{cite}
\usepackage{lineno}


\newtheorem{lemma}{Lemma}
\newtheorem{theorem}{Theorem}
\newtheorem{proposition}{Proposition}
\newtheorem{corollary}{Corollary}
\newtheorem{claim}{Claim}
\newtheorem{observation}{Observation}


\newcommand{\Poly}{\ensuremath{\mathcal{P}}}               \newcommand{\bd}{\ensuremath{\partial \Poly} }                 \newcommand{\Vis}{\ensuremath{\mathrm{Vis}_\Poly}} \newcommand{\VisC}{\ensuremath{\mathrm{Vis}_{\mathcal{C}}}} \newcommand{\VisCprime}{\ensuremath{\mathrm{Vis}_{\mathcal{C'}}}} \newcommand{\E}{\ensuremath{\mathrm{E}}}                    \newcommand{\Rin}{\ensuremath{r}}      \newcommand{\Rout}{\ensuremath{\bar{r}}}      \newcommand{\Hout}{\ensuremath{\bar{h}}}      \newcommand{\chain}{\ensuremath{{\mathrm{Chain}}}}     \newcommand{\region}{\ensuremath{{{\mathcal R}}}}     
\newcommand{\regionC}{\ensuremath{{{\mathcal R_{\mathcal C}}}}}     
\newcommand{\regone}{\ensuremath{{{\mathcal R_{\mathcal C_{1}}}}}}     
\newcommand{\regtwo}{\ensuremath{{{\mathcal R_{\mathcal C_{2}}}}}}
\newcommand{\coneC}{\ensuremath{{{\Delta_{\mathcal C}}}}} 
\newcommand{\C}{\ensuremath{{\mathcal C}}}     



\newcommand{\marrow}{\marginpar[\hfill]{}}
\newcommand{\remark}[2]{\textcolor{red}{\textsc{#1 says:} \marrow\textsf{#2}}}
\newcommand{\rodrigo}[1]{\remark{Rodrigo}{#1}}
\newcommand{\mati}[1]{\remark{Mati}{#1}}
\newcommand{\luis}[1]{\remark{Luis}{#1}}


\graphicspath{{figures/}}



\title{Computing a visibility polygon using few variables\thanks{A preliminary version of this paper appeared in the proceedings of the 22nd International Symposium on Algorithms and Computation (ISAAC 2011)~\cite{bkls-cvpufv-11}.}}



\author{
Luis Barba
\thanks{Universit\'e Libre de Bruxelles (ULB), Brussels, Belgium. {\tt \{lbarbafl,stefan.langerman\}@ulb.ac.be}}
\and Matias Korman
\thanks{Universitat Polit\`{e}cnica de Catalunya (UPC), Barcelona, Spain. {\tt  \{matias.korman, rodrigo.silveira\}@upc.edu}. With the support of the Secretary for Universities and Research of the Ministry of Economy and Knowledge of the Government of Catalonia, the European Union, the FP7 Marie Curie Actions Individual Fellowship PIEF-GA-2009-251235, and ESF EUROCORES programme EuroGIGA - ComPoSe IP04 - MICINN Project EUI-EURC-2011-4306.}
\and Stefan Langerman\footnotemark[2]
\thanks{Directeur de Recherches du FRS-FNRS.}
\and Rodrigo I. Silveira\footnotemark[3]
}






\date{}

\begin{document}
\maketitle

\begin{abstract}
We present several algorithms for computing the visibility polygon of a simple polygon  of  vertices (out of which  are reflex) from a viewpoint inside , when  resides in read-only memory and only few working variables can be used.
The first algorithm uses a constant number of variables, and outputs the vertices of the visibility polygon in  time, where  denotes the number of reflex vertices of  that are part of the output. Whenever we are allowed to use  variables, the running time decreases to  (or  randomized expected time), where . This is the first algorithm in which an exponential space-time trade-off for a geometric problem is obtained.
\end{abstract}   


\section{Introduction}


The \emph{visibility polygon} of a simple polygon  from a viewpoint  is the set of all points of  that can be seen from , where two points  and  can see each other whenever the segment  is contained in .
The visibility polygon is a fundamental concept in computational geometry and one of the first problems studied in planar visibility. 
The first correct and optimal algorithm for computing the visibility polygon from a point was found by  Joe and Simpson\cite{js-clvpa-87}. It computes the visibility polygon from a point in linear time and space. 
We refer the reader to the survey of O'Rourke \cite{r-v-04} and the book of Gosh~\cite{g-vap-07} for an extensive discussion of such problems.





In this paper we look for an algorithm that computes the visibility polygon of a given point and uses few variables. This kind of algorithm not only provides an interesting trade-off between running time and memory needed, but is also useful in portable devices where important hardware constraints are present (such as the ones found in digital cameras or mobile phones). 
In addition, this model has direct applications in concurrent environments where several devices with limited memory resources perform some computation on a large centralized input. 
Since several devices may access the input at the same time, allowing writing to the input memory can result in compromising its integrity.


A significant amount of research has focused on the design of algorithms that use few variables, some of them even dating from the 80s \cite{mp-ssls-80}. Although many models exist, most of the research considers that the input is in some kind of read-only data structure. In addition to the input values, we are allowed to use few additional variables (typically a variable holds a logarithmic number of bits). 

One of the most studied problems in this setting is that of selection. For any constant , Munro and Raman \cite{mr-sromswmdm-96} gave an algorithm that runs in  time and uses  variables. Frederickson~\cite{Frederickson87} extended this result to the case in which  working variables are available (and ). Raman and Ramnath~\cite{rr-iubtstsls-98} gave several exact and approximation algorithms for the case in which fewer variables are available. Among other results, they provide a -approximation of the median that runs in  time using  variables (for ), or  time, using  variables. More recently Chan \cite{Chan} provided several lower bounds for performing selection with few variables. 

In recent years there has been a growing interest in finding algorithms for geometric problems that use a constant number of variables.
An early example is the well-known gift-wrapping algorithm (also known as Jarvis march~\cite{j-ich-73}), which can be used to  report the points on the convex hull of a set of  points in  time using a constant number of variables, where \Hout\ is the number of vertices on the convex hull. 
Recently, Asano and Rote~\cite{ar-cwagp-09} and afterwards Asano {\em et al.} \cite{amw-cwaspsp-10,abbkmrs-mcasp-11} gave efficient methods for computing well-known geometric structures, such as the Delaunay triangulation, the Voronoi diagram, a polygon triangulation, and a minimum spanning tree (MST) using a constant number of variables. These algorithms run in  time (except computing the MST, which needs  time). Observe that, since these structures have linear size, they are not stored but reported. Prior to this work, there was no algorithm for computing the visibility polygon in memory-constrained models. Indeed, this problem was explicitly posed as an open problem by Asano {\em et al.}~\cite{amrw-cwagp-10} for the case in which only a constant number of variables are allowed. 



\subsubsection*{Results}


In this paper we present a novel approach for computing the visibility polygon of a given point inside a simple polygon. 
It is easy to see that reflex vertices have a much larger influence on the visibility polygon than convex vertices. Therefore, whenever possible we express the running time of our algorithms not only in terms of , the complexity of , but also in terms of   and  (the number of reflex vertices of  that are present in the input and in the output, respectively). This approach continues a line of research relating the combinatorial and computational properties of polygons to the number of their reflex vertices. We refer the reader to \cite{bdhilm-ghsc-07,akpv-got-12,bchm-gwp-10} and references found therein for a deep review of existing similar results.

In Section~\ref{sec:Preliminaries} we begin the paper with some preliminaries, followed by some observations and basic algorithms in Section~\ref{sec:SimpleAlgorithm}. In Section \ref{section:SequentialAlgorithms} we give an output-sensitive algorithm that reports the vertices of the visibility polygon in  time using  variables. Using this algorithm as a stepping stone, in Section \ref{Section:Divide and conquer} we present a divide-and-conquer algorithm. This algorithm runs in  time (or  randomized expected time)  using  variables (for any ), giving an exponential trade-off between running time and space. 



{\bf Remark:} prior to this research there was no known method for computing visibility polygons using few variables. Following the conference version of this paper~\cite{bkls-cvpufv-11}, De {\em et al.}~\cite{dmn-seavpsp-12} provided a linear-time algorithm that uses -variables. Parallel to this research, Barba {\em et al.}~\cite{bklss-sttosba-13} gave a general method for transforming stack-based algorithms into memory constrained workspaces. Since Joe and Simpson's algorithm for computing the visibility polygon~\cite{js-clvpa-87} is stack-based, their approach can be used for this problem as well. 

\iffalse
\begin{table}
\centering
\begin{tabular}{|c|c||l|} \hline
Space & Time & Notes \\ \hline
 &  & Thm.~\ref{theoImprov}. Output Sensitive. \\ \hline
 &  & Thm.~\ref{theo_determ}.  \\ \hline
 &  & Thm.~\ref{theo_determ}. \\ \hline
 &  & Thm.~\ref{theo_randomi}. Randomized expected time. \\ \hline
 &  & Given in~\cite{js-clvpa-87} . \\ \hline
\end{tabular}
\caption{Running time of the known memory-constrained algorithms for computing the visibility polygon as a function of the working space . Unless otherwise stated, running times are deterministic.}
\label{tabresults}
\end{table}
\fi

\section{Preliminaries}
\label{sec:Preliminaries}
\subsubsection*{Model definition and considerations on input/output precision}
We use a slight variation of the constant workspace model, introduced by Asano and Rote~\cite{ar-cwagp-09}. In this model the input of the problem resides in a read-only data structure and we are allowed to perform random access to any of the values of the input in constant time. 
An algorithm can use a constant number of variables and we assume that each variable or pointer contains a data word of  bits. 
Implicit storage consumption required by recursive calls is also considered part of the workspace. This model is also referred as {\em log space}~\cite{AB09} in the complexity literature.

Many other similar models have been studied. 
We note that in some of them (like the {\em streaming}~\cite{gk-seocqs-01} or the {\em multi-pass} model~\cite{cc-mpga-07}) the values of the input can only be read once or a fixed number of times. 
As in the constant workspace model of  Asano and Rote~\cite{ar-cwagp-09}, our model allows scanning the input as many times as necessary. 
However, our model differs from theirs in two aspects: we are allowed to use a workspace of  variables (instead of ), and we do not require random access to the vertices of the input. 

The input to our problem is a simple polygon  in a read-only data structure and a point  in the plane, from where the visibility polygon needs to be computed.
We do not make any assumptions on whether the input coordinates are rational or real numbers (in some implicit form). The only operations that we perform on the input are determining whether a given point is above, below or on a given line and determining the intersection point between two lines. In both cases, the line is defined as passing through two points of the input, hence both operations can be expressed as finding the root of linear equations whose coefficients are values of the input. We assume that these two operations can be done in constant time. Moreover, if the coordinates of the input are algebraic values, we can express the coordinates of the output as ``the intersection point of the line passing through points  and  and the line passing through  and " (where  and  are vertices of the input). 


\subsubsection*{Definitions and basic properties}
We say that a point  is \emph{visible} from  (with respect to ) if and only if the segment  is contained in  (note that we regard  as a closed subset of the plane). The set of points visible from  is called the \emph{visibility polygon} of \Poly, and is denoted \Vis. Note that if  is outside of \Poly\ then \Vis\ is by definition empty. Thus, when considering visibility with respect to polygons, we always assume that  is inside the polygon. From now on, we assume that  is fixed, hence we omit the ``with respect to " term.

We assume we are given  as a list of its vertices in counterclockwise order along its boundary (denoted by ). Let  be a point on  closest to  on the horizontal line passing through . It is easy to see that  is visible and can be computed in linear time. In the following, we will treat  as a vertex of \Poly\ (even though it does not need to be one). By implicitly reordering the vertices of the input, we can assume that we are given the vertices of  in counterclockwise direction starting from  (i.e., ). 

For simplicity of exposition, we assume that the vertices of  are in a weak general position; that is, we assume that there do not exist two vertices  such that , and  are aligned (but we note that the algorithms can be extended easily for the general case). 


Along this paper, we will often work with polygonal chains (instead of polygons). However, we will restrict our scope to polygonal chains contained in . For any two points  on , there is a unique path from  to  that travels counterclockwise along ; let  be the set of points traversed in this path, including both  and  (this set is called the {\em chain} between  and ). We say that  and  are the \emph{endpoints} of , and we refer to the rest of the points on  as its \emph{interior points}. 

We now extend the concept of visibility to chains. Due to technical reasons, we define this concept for chains contained in  whose endpoints are visible from .  We say that a chain  is \emph{independent} if and only if its endpoints are visible points of . Given a chain  with endpoints  and , let \regionC\ denote the polygon enclosed by the union of \C\ and the segments  and  (equivalently, we use the notation ).

Given a point  and an independent chain \C\ with endpoints  and , we say that a point  is \emph{visible} from  with respect to \C\ if and only if  is visible from  with respect to \regionC.
The set of points that are visible with respect to  is called the \emph{visibility polygon} of , and is denoted \VisC. We start by observing that both concepts of visibility are equivalent (hence we need not distinguish between them).

\begin{observation}
If  is an independent subchain, then \regionC\ is a simple polygon. Moreover, a point  is visible with respect to  if and only if it is visible with respect to . 
\end{observation}

The above observation certifies that visibility within independent chains is well-defined. Since we will only consider independent chains, from now on we omit the ``with respect to " (or to ), and simply say that a point  is visible. 

True to its name, the visibility polygon of an independent chain  can be computed without having knowledge of the remainder of . 

\begin{observation}\label{DivideObs}
Let  be an independent chain such that  for some . Let  be a visible interior point of  lying on the edge  for some . If we let  and ,
then  

\end{observation}





Given a point  on the plane, let  be the ray emanating from  and passing through .
We define  as the angle that  makes with the positive -axis, . 
We call  the \emph{CCW-angle} of .








\begin{figure}[tb!]
\centering
\includegraphics{polygon}
\caption{Left:	general setting, vertices that are reflex with respect to  are shown with a white point (black otherwise). Right: the visibility polygon .
}
\label{fig:polygon}
\end{figure}

We also need to define what a \emph{reflex} vertex is in our context. Given any vertex , the line  passing through  and  splits  into disjoint components. A vertex  is \emph{reflex with respect to } if the angle at the vertex interior to  is more than  and the vertices  and  lie on the same connected component of  (see Fig. \ref{fig:polygon}). 
Observe that any vertex that is reflex with respect to  is a reflex vertex (in the usual sense), but the converse is not true. Since the point  is fixed, from now on we omit the ``with respect to " term and simply refer to these points as reflex. 
Note that being reflex is a local property that can be verified in  time.
Intuitively speaking, reflex vertices with respect to  are the vertices where important changes occur in the visibility polygon. That is, where the polygon boundary can change between visible or not-visible. Let  be the number of reflex of  vertices of . We also define  as the number of reflex vertices of  that are present in . Naturally, we always have .

Given two points  and  on a chain , we say that  lies \emph{before}  (resp.  lies \emph{after} ) if, when we walk from  towards  along , we first pass through  and then through . We say that a chain is {\em visible} if all the points of the chain are visible. A visible chain  is \emph{CCW-maximal} if no other visible chain starting at  strictly contains . In this case, we say that  is the maximal chain {\em starting} at  and {\em ending} at .




Given a visible reflex vertex  on a chain , we say that a point  on  is the \emph{shadow} of  if  is collinear with  and , and  is visible from  (this point is denoted by ()). Due to the general position assumption,  is uniquely defined and  must be an interior point of an edge. That is, each visible reflex vertex is uniquely associated to a shadow point (and vice versa). We say that a visible reflex vertex  is of \emph{type R} (resp. \emph{type L}) if its shadow lies after (resp. before) ; see Fig.~\ref{fig:RightLeftReflex and ConeDefs} (left). 
Equivalently, a vertex  is of type R if  and  make a \emph{right} turn (where  and  are the predecessor and the successor of  on , respectively). 
Analogously, vertices of type L make a \emph{left} turns instead. 

A \emph{ray shooting query} is a basic operation that, given an independent chain \C\ and a point , considers the ray  and reports the last visible point in  with respect to \C\ (i.e., the one furthest away from ). Observe that when  is a visible reflex vertex we obtain its shadow. We denote the output of this operation by \textsc{RayShooting}. It is easy to see that a ray shooting query can be performed in linear time, using only  extra variables, by scanning the edges of \C\ one by one and computing their intersections with . 

Finally, for   we define  as the cone with apex  that contains every point in the plane with CCW-angle between  and ; see Fig.~\ref{fig:RightLeftReflex and ConeDefs} (right).




\section{Understanding the visibility polygon}
\label{sec:SimpleAlgorithm}
The basic scheme of our algorithms is to partition the input polygon into independent subchains so that the visibility within one subchain is unaffected by the others. We will use  as the starting chain, thus the first chain will be closed, but the following chains will be open. Notice that since \Poly\ is simple, any chain of it will be simple too. 

In this section we present some observations about the independence between chains.


\begin{figure}[tb]
\centering
\includegraphics{RightLeftReflex}
\caption{\small Left:  is a reflex vertex of type R, while  is of type L. The chain between  and , together with the segment joining them, bound a simple polygon containing no visible points other than  and .  Observe that the chain  is CCW-maximal. 
Right: A polygonal chain  and its associated polygon . Point  is visible while points  and  are not. 
Note that every visible point of  lies inside .}
\label{fig:RightLeftReflex and ConeDefs}
\end{figure}

\begin{observation}\label{obs:VisibleChains}
Let  be a visible reflex vertex of  of type R (resp. L) whose shadow is . The  (resp. ) contains no visible point other than its endpoints. In particular, a visible chain cannot contain an interior reflex vertex.
\end{observation}








The following important lemma characterizes the endpoints of CCW-maximal chains.



\begin{lemma}\label{lem_key}
Let  be an independent chain, let  be a visible point and let  the first visible reflex vertex encountered when walking from  towards  (or  if none exists). Let  be the CCW-maximal chain starting at . The point  is either equal to  (if  or  is of type R), or equal to the shadow of  (if  is of type L).
\end{lemma}
\begin{proof}
Clearly, all the points lying after  are visible if and only if .
If , then, when walking on ,  is the last visible point of the chain before a change in visibility occurs (i.e. points at distance  lying after  are not visible for sufficiently small values of ).
This can happen for only two reasons: either  is a reflex vertex of type R, or there is some reflex vertex  of type L such that  is the shadow of .
In the former case,  is equal to  since no reflex vertex lies in the interior of  by Observation~\ref{obs:VisibleChains}.
In the latter case,  must be the first visible reflex vertex lying after  since no point between  and  can be visible by Observation~\ref{obs:VisibleChains}.
\end{proof}

The following result is a direct consequence of Lemma~\ref{lem_key} that allows us to report  of an independent chain  with no interior reflex vertices. 

\begin{corollary}\label{corollary:CharacterizationVisibleReflex}
Let  be a visible point such that  is not a reflex vertex of type R. 
Let  be the first visible reflex vertex lying after  and let  be its shadow. The following statements hold:
\begin{itemize}
\vspace{-1mm}\item[-] If  is of type R, then  is CCW-maximal.
\vspace{-1mm}\item[-] If  is of type L, then  is CCW-maximal.
\end{itemize}
\end{corollary}







\iffalse
\begin{lemma}\label{DivideLemma}
Let  be an independent subchain of  (i.e.  are visible).
Let  be a visible interior point of  lying on the edge  for some .
Let  and ,
then  

\end{lemma}
\begin{proof}
For every , the segment  is contained in  (since  is -visible).
That is, no point of the subchain  crosses , 
which implies that  is -visible or -visible (depending on whether  belongs to  or to ), and it is both if and only if . Thus, .

The other inclusion is proven by contradiction. Assume that there exists a point  such that , and the segment  intersects  at a point  (since otherwise  would be -visible). By definition of -visible,  is contained in the polygon , hence  lies in the interior of . Since  is a visible point,  and  have disjoint interiors, therefore  does not belong to .
Thus, by the Jordan curve Theorem, 
the curve  contained on  joining  with  either crosses , in which case  would not be a simple polygon; or  crosses either the edge  or , in which case we have a contradiction since both  and  are visible points of .
Therefore,  does not intersect  and consequently  is -visible. 
Note that this also implies that  and hence .
The proof when  is analogous.
\end{proof}
\rodrigo{What about making the previous lemma an observation, and removing the proof? The result looks rather obvious to me, but the proof is rather convoluted---given the simplicity of the result}

Lemma~\ref{DivideLemma} allows us to divide the problem of finding  into a series of subproblems (computing the visibility polygon of a polygonal subchain) that can be solved independently.

\fi
 



\section{Output Sensitive Algorithm}\label{section:SequentialAlgorithms}
In this section we present an algorithm that will be used as stepping stone for our divide and conquer algorithm presented in  Section~\ref{Section:Divide and conquer}.
This base algorithm computes the visibility polygon of an independent chain using  extra variables.

For this purpose, we introduce an operation that we call . 
This operation receives an independent chain  and a visible point  on . 
Its objective is to compute the next \emph{visible} reflex vertex lying after  on \C, along with its shadow. That is, let  be the first reflex vertex lying after  on . If  is visible, then  should return  and its shadow. Otherwise, we know that at some point when walking along the path from  to  we change from a visible to a non-visible region. By Lemma~\ref{lem_key}, this change occurs at the shadow of some visible reflex vertex. In this case,   should return this reflex vertex (as well as its shadow). For well-definedness purposes, we say that  should return  if  contains no reflex vertex.

The following observation (illustrated in Fig.~\ref{fig:ImprovedAlgorithm}) allows us to compute  efficiently. 

\begin{figure}[tb]
  \centering
\includegraphics{ImprovedAlgorithm}
\caption {\small When the first reflex vertex  lying after  is not visible, there must exist a visible reflex vertex in  angularly located between  and . The one with smallest CCW-angle inside  ( in the figure) will determine the change between visible and non-visible regions between  and .}
\label{fig:ImprovedAlgorithm}
\end{figure}








\begin{observation}\label{obs:SmallestAngle}
For any independent chain , and visible point  that is not an R-type reflex vertex, let  be the first reflex vertex encountered when walking from  towards  on  (or  if none exists). If  is not visible, then the CCW-maximal chain starting at  ends at the shadow of the L-type reflex vertex in  with smallest CCW-angle.
\end{observation}

\begin{lemma}\label{lem_next}
For any independent chain  of  vertices and a visible point , \linebreak can be computed in  time using  additional variables.
\end{lemma}
\begin{proof}
Let  and let  be the first reflex vertex lying after  on .
In  time we can perform a ray shooting query: if  is visible then we output it (and its shadow) as the result of . Otherwise, we use Observation~\ref{obs:SmallestAngle} and compute the reflex vertex on  with smallest CCW-angle (among those that are in ). This vertex is found by walking along  and keeping track of every time we enter or leave . Note that since  is visible and  is simple, we can only enter or leave  when we cross line segment , hence this can be checked in constant time per edge of \C. Since a constant number of operations is needed per vertex, at most  time will be needed for computing . 

Finally note that Observation~\ref{obs:SmallestAngle}  holds whenever  is not an R-type reflex vertex. Thus, if  is a reflex vertex of type R, it suffices to first compute its shadow, and  return the same as   would.
\end{proof}


Given an  independent chain , our base algorithm works as follows: start from , use  to obtain the next visible reflex vertex, and report the CCW-maximal chain starting from . We then repeat the procedure starting from the last reported vertex until we reach  (see the details in Algorithm \ref{alg:subroutine}). 


\begin{theorem}\label{theo:Sequential algorithm}
Algorithm~\ref{alg:subroutine} reports the visibility polygon of an independent chain of  vertices and  visible reflex vertices in counterclockwise order in  time, using  additional variables.
\end{theorem}
\begin{proof}
Correctness of the algorithm is given by Lemma \ref{lem_next} and Corollary~\ref{corollary:CharacterizationVisibleReflex}.  It is easy to verify that Algorithm \ref{alg:subroutine} uses a constant number of variables, hence it remains to show a bound on the running time. Notice that at each iteration of the algorithm we report a visible reflex vertex. Hence, we can charge the cost of  operation to the reported vertex. Since no vertex is reported twice and operation  needs linear time, the total running time is bounded by . 
\end{proof}

\begin{algorithm}
  \begin{algorithmic}[1]  
	\STATE   (or  if  is an R-type reflex vertex)
	\REPEAT
		\STATE 
		\IF{}
			\STATE  (*  The remainder of the chain is visible *)
			\STATE 
			\STATE 
		\ELSE 
		        \STATE (* We found next visible reflex . The reported chain will depend on the type of  *)
			\IF{ is of type R}
				\STATE 
				\STATE 
			\ELSE
				\STATE 
				\STATE 
			\ENDIF
		\ENDIF
    \STATE Report every vertex between  and 
    \STATE 
    \UNTIL{}
  \end{algorithmic}
\caption{Computing the visibility polygon of an independent chain }
\label{alg:subroutine}
\end{algorithm}

\section{A divide-and-conquer approach}\label{Section:Divide and conquer}


We now consider the case in which we are allowed a slightly larger amount of variables. We parametrize the running time of our algorithms by the number of working variables allowed, which we denote by . Our aim is to obtain an algorithm whose running time decreases as  grows. 

Using the result of the previous section as base algorithm, we now present a divide-and-conquer approach to solve the problem. The general scheme of our algorithm is the natural one: choose a reflex vertex  inside the cone , perform a ray shooting query to find the visible point in the direction of , and split the polygonal chain into two smaller independent subchains  (see Fig.~\ref{fig:OutlineAlgorithm}, left). We repeat the process recursively, until either 
 a chain  has a constant number of reflex vertices (see Fig.~\ref{fig:OutlineAlgorithm}, right) or 
 the depth of the recursion is such that we would exceed the number of allowed working variables. 
Whenever either of these two conditions is met, we compute the visibility polygon of the chain using Algorithm \ref{alg:subroutine}. See a scheme of this approach in Algorithm~\ref{alg:DivideAndConquer}.




\begin{figure}[tb]
\centering
\includegraphics{OutlineAlgorithm}
\includegraphics{Casek=0}
\caption{Left: Split of  into two subchains  using a visible point  in the direction of a reflex vertex .
Right: A polygonal chain  with no reflex vertices inside the cone , only one subchain of  is visible. }
\label{fig:OutlineAlgorithm}
\end{figure}

\begin{algorithm}
\caption{Given a polygonal chain  such that  are both visible points of  and a positive integer depth  (initially ), compute }
\label{alg:DivideAndConquer}
\begin{algorithmic}[1]
\STATE  number of reflex vertices of  inside the cone 
\IF{ or }
	\STATE\label{algoconst} Run Algorithm~\ref{alg:subroutine} on 
\ELSE
	\STATE 
\STATE 
	\STATE Algorithm~\ref{alg:DivideAndConquer}
	\STATE Algorithm~\ref{alg:DivideAndConquer}
\ENDIF
\end{algorithmic}
\end{algorithm}


In order to control the depth of the recursion we use a depth counter, hence the algorithm stops dividing once  (for some value  that will be determined later). Note that the split direction is decided by subroutine . This procedure should give a direction so that the resulting subchains  and  have roughly the same complexity. Naturally, it must also run reasonably fast and use  variables.

In this section we propose two different methods to choose the partition vertex. The first one uses the approximate median finding algorithm of  Raman and Ramnath~\cite{rr-iubtstsls-98}. The second one is randomized, and simply chooses a random reflex vertex among those lying in the cone . We first show that, regardless of the partition method used, the visibility polygon will be correctly computed. 

\begin{lemma}\label{lem_correctdiv}
Algorithm~\ref{alg:DivideAndConquer} correctly reports the visibility polygon of an independent chain in counterclockwise order, using  variables.
\end{lemma}
\begin{proof}
The divide-and-conquer approach repeatedly partitions the input into independent chains. Each subchain will eventually be reported. By Observation~\ref{DivideObs}, the union of reported vertices is equal to the visibility polygon, hence correctness is derived from the correctness of  Algorithm~\ref{alg:subroutine}.





Regarding space, the subroutines called in this algorithm (\textsc{findPartitionVertex} and \linebreak\textsc{RayShooting}) use  and  variables, respectively. Once the procedure finishes, their working space can be reused for further calls, hence we never use more than  working space at the same time. It remains to consider the memory used implicitly for handling the recursion. Since each step of the algorithm needs a constant number of variables, the total memory needed will be proportional to the recursion depth. Since , the claim is shown.
\end{proof}

In what follows we present two implementations of . 

\subsection{Deterministic variant}\label{secdeter}

We start by giving a deterministic algorithm for  using  extra variables. For this purpose, we will use the algorithms by Raman and Ramnath presented in~\cite{rr-iubtstsls-98}, which compute an approximation of the median value of a given set using reduced workspace.

Given a chain , let \C. An element  is called a \emph{-median of } if there are at most  elements in  smaller that  and at most  greater that .
Given two elements  of  such that , we say that  is an \emph{approximate median pair} if at most  elements of  are smaller than , at most  lie between  and , and at most  elements are greater that .
Notice that if  is an approximate median pair, then either  or  is a -median of .
Moreover, we can determine which of the two is a -median with one scan of . Thus, we say that every approximate median pair induces a -median of .

Raman and Ramnath~\cite{rr-iubtstsls-98} presented two algorithms to find an approximate median pair. The first one is used whenever , and allows us to find a -median of  in  time using  variables (Lemma 5 of~\cite{rr-iubtstsls-98}, assigning their parameter  to ). For the case , they propose another algorithm (stated in Lemma 3 of~\cite{rr-iubtstsls-98}) that computes an approximate median pair in  time, using  variables. We note that Raman and Ramnath used these algorithms to afterwards obtain the exact median, but in here we only need an approximation. 

 will execute the approximate median pair algorithm, and return the reflex vertex  that induces a -median of . By construction, each of the two cones obtained from , by shooting a ray through , will contain at most  of the reflex vertices in . 

Let  be the running time of the approximate median finding algorithm on a chain  of length  with  reflex vertices inside  (that is,  if ,  otherwise). We set  (if ) or  (otherwise). Observe that in both cases we have . We now prove an upper bound on the running time of Algorithm~\ref{alg:DivideAndConquer} with this implementation of .


\begin{theorem}\label{theo_determ}
For any ,  can be computed in   time using  variables.
\end{theorem}
\begin{proof}
Consider any node  of the recursion tree of the algorithm and let  be the chain processed at this node. Let  and  be the size of  and the number of reflex vertices lying inside , respectively.

If  is a non-terminal node of the recursion, then the running time at  is bounded by . 
Note that a vertex of the input can only appear in at most two chains of the same depth. Hence, the total cost of all non-terminal nodes of a fixed level  in the recursion tree is bounded by . By definition, there are at most  levels of recursion, hence the time spent in all the non-terminal executions of the algorithm is bounded by .

It remains to consider the time spent in the terminal nodes. Each terminal node  will need  time, where  denotes the number of visible reflex vertices on . Recall that terminal nodes are only executed whenever either  has a constant number of reflex vertices or we have reached  levels of recursion. Further note that, at each level of recursion at least a third of the reflex vertices are discarded. In particular, we have that . 

Similar to non-terminal nodes, vertices cannot be present in more than two terminal nodes. Thus, the total time spent in the terminal nodes is bounded by  Therefore, the total time spent by the algorithm becomes  . 

This expression can be simplified by distinguishing between different workspace sizes. 

\begin{description}
\item[Case .] In this case we have , and in particular  (since  is a parameter that can be chosen to be at least 3). Further recall that , and in particular  . Thus, the second term simplifies to . 
Since  and the running time decreases as  grows, we have   . 

That is, the running time of our algorithm is expressed as the sum of two terms that, when , the first one is at most  whereas the second one is at least . Asymptotically speaking, the first one can be ignored, and the running time is dominated by  (i.e., applying Algorithm~\ref{alg:subroutine} to each terminal node). 
\item[Case .] In this case we can use the faster method of Raman and Ramnath (i.e. ). Recall that in this case we set , hence  . In particular, the running time of the second term simplifies to . Since  , the running time is dominated by the first term (i.e., finding the split direction), which is . 
\end{description}
Observe that in both cases the running time is bounded by , hence the claim holds. 
\end{proof}

{\bf Remark} Note that, although we only consider algorithms that use up to  variables, one could study what happens whenever more space is allowed. However, we note that increasing the size of our workspace will not reduce the running time, since the time bottleneck of this approach is determined by the approximate median method of Raman and Ramnath. 





\subsection{Randomized approach}\label{second}
Whenever , the running time of the previous algorithm is dominated by the \textsc{FindPartitionVertex} procedure. Motivated by this, in this section we consider a faster (albeit randomized) partition method. 

The randomized method proceeds as follows: let  be the  number of reflex vertices of  lying inside  (note that  can be computed in linear time by scanning ). Select a random number  between  and  (uniformly at random). The idea is to output the -th reflex vertex inside  (computed by walking counterclockwise along ). However, we must first check that this vertex will make a balanced partition. In order to check so, we make another scan of , and we count the angular rank of the -th reflex vertex (among the reflex vertices in ). If its rank is between  and  , we use it as partition vertex. Otherwise, we pick another random number and repeat the process until such a vertex is found.

\begin{lemma}\label{lem_runningtime}
The randomized version of \textsc{FindPartitionVertex} has expected running time .
\end{lemma}
\begin{proof}
The probability that, when choosing a reflex vertex uniformly at random, we pick one whose rank is between  and  is exactly . If each time we make the choices independently, we are performing Bernoulli trials whose probability of success is . Hence, the expected number of times we have to choose a random index is a constant (three in this case). For each try we only need to check its rank (which can be done in  time by performing two scans of the input).
Therefore we conclude that the expected running time of \textsc{FindPartitionVertex} is .
\end{proof}
 

\begin{theorem}\label{theo_randomi}
For any ,   can be computed in  expected time using  variables.
\end{theorem}
\begin{proof}
The proof is identical to the proof of Theorem \ref{theo_determ}, just taking into account that now , hence the running time of the second term is decreased by a  factor. 
\end{proof}


Observe that this algorithm is faster than the previous one when .
We conclude by summarizing the different algorithms presented in this paper. 

\begin{theorem}
Given a polygon  of  vertices, out of which  are reflex, the visibility polygon of a point  can be computed in  time using  variables (where  is the number of reflex vertices of ) or  time using  variables (for any ). If  variables are available, the running time decreases to  time or  randomized expected time.
\end{theorem}








 






\bibliographystyle{abbrv}
\bibliography{visi}

\end{document}
