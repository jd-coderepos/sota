

\documentclass[US letter, 9 pt, conference]{ieeeconf}  \usepackage{setspace}
\usepackage{graphics} \usepackage{graphicx}


\usepackage{epsfig} \usepackage{float}
\usepackage{times} \usepackage{stdmath}
\usepackage{amsmath}  \usepackage{amsfonts}
\newtheorem{theorem}{Theorem}
\newtheorem{definition}{Definition}
\newtheorem{lemma}{Lemma}
\newtheorem{remark}{Remark}
\newtheorem{corollary}{Corollary}
\usepackage{graphicx}
\usepackage{caption}
\usepackage{subcaption}
\makeatother
\newcommand{\p}{\partial}
\newcommand{\pfs}{\frac{\partial}{\partial s}}
\newcommand{\pft}{\frac{\partial}{\partial t}}
\newcommand{\pfx}{\frac{\partial}{\partial x}}
\newcommand{\pfxi}{\frac{\partial}{\partial \xi}}
\newcommand{\om}{\bar{\Omega}}
\newcommand{\pos}{\bar{\mathbb{R}}_+}
\newcommand{\igzo}{\int_0^1}
\newcommand{\igzx}{\int_0^x}
\newcommand{\igxo}{\int_x^1}
\newcommand{\igzxi}{\int_0^\xi}
\newcommand{\igxix}{\int_\xi^x}
\newcommand{\igxxi}{\int_x^\xi}
\newcommand{\igxio}{\int_\xi^1}
\newcommand{\igzs}{\int_0^s}
\newcommand{\igso}{\int_s^1}
\newcommand{\mc}{M_C}
\newcommand{\mo}{M_O}
\newcommand{\mcx}{M_{C,x}}
\newcommand{\mox}{M_{O,x}}
\newcommand{\mcxx}{M_{C,xx}}
\newcommand{\moxx}{M_{O,xx}}
\newcommand{\konex}{K_{1,x}}
\newcommand{\ktwox}{K_{2,x}}
\newcommand{\konexx}{K_{1,xx}}
\newcommand{\ktwoxx}{K_{2,xx}}
\newcommand{\Pm}{u_{11}}
\newcommand{\Pmo}{P_{M_O}}
\newcommand{\Pmc}{P_{M_C}}
\newcommand{\zdm}{Z_{d}}
\newcommand{\wh}{\hat{w}}
\newcommand{\wt}{\tilde{w}}
\newcommand{\zt}{\tilde{z}}
\newcommand{\lt}{L_2(0,1)}
\newcommand{\ct}{C([0,1])}
\newcommand{\hlf}{\frac{1}{2}}
\newcommand{\mcl}[1]{\mathcal{#1}}
\newcommand{\pop}{\mathcal{P}}
\newcommand{\pinv}{\mathcal{P}^{-1}}
\newcommand{\sop}{\mathcal{S}}
\newcommand{\sinv}{\mathcal{S}^{-1}}
\newcommand{\zh}{\hat{z}}
\allowdisplaybreaks



\IEEEoverridecommandlockouts                              

\overrideIEEEmargins                                      





\title{\LARGE \bf
Output Feedback Control of Inhomogeneous Parabolic PDEs with Point Actuation and Point Measurement using SOS and Semi-Separable Kernels
}


\author{Aditya Gahlawat and Matthew M. Peet\thanks{Aditya Gahlawat is with the Department
of Mechanical, Materials and Aerospace Engineering at the Illinois Institute of Technology, Chicago,
IL, 60616 USA and with the Grenoble Image Parole Signal Automatique Lab., Universit\'{e} Joseph Fourier/Centre National de la Recherche Scientifique, St. Martin d'Heres, France
        {\tt\small agahlawa@hawk.iit.edu}}\thanks{Matthew. M. Peet is with the School of Engineering of Matter, Transport and Energy at Arizona State University, Tempe, AZ, 85287-6106 USA
        {\tt\small mpeet@asu.edu}}}


\begin{document}



\maketitle
\thispagestyle{empty}
\pagestyle{empty}


\begin{abstract}
In this paper we use SOS and SDP to design output feedback controllers for a class of one-dimensional parabolic partial differential equations with point measurements and point actuation. Our approach is based on the use of SOS to search for positive quadratic Lyapunov functions, controllers and observers. These Lyapunov functions, controllers and observers are parameterized by linear operators which are defined by SOS polynomials. The main result of the paper is the development of an improved class of observer-based controllers and evidence which indicates that when the system is controllable and observable, these methods will find a observer-based controller for sufficiently high polynomial degree (similar to well-known results from backstepping).
\end{abstract}


\section{INTRODUCTION}

Parabolic Partial Differential Equations (PDEs) are a class of system used to model processes such as diffusion, transport and reaction.  Some examples of systems which have been modelled using parabolic PDEs include plasma in a tokamak~\cite{witrant2007control}, heat propagation, and spatial dynamics of population in an ecosystem~\cite{murray2002mathematical}. In this paper we consider a class of inhomogeneous linear scalar valued Parabolic PDEs. We assume that only boundary control and sensing is available for the PDEs. The control is exercised through a Dirichlet boundary condition and a Neumann boundary measurement of the state is available. The goal of this article is to use this boundary measurement to construct a boundary controller which ensures that the state of the system remains bounded in the presence of an exogenous input (has finite -gain). We refer to this as output feedback based boundary control.

In order to design output feedback based boundary controllers, we design a Luenberger observer where the error dynamics have finite  gain from disturbance to error. We then design a controller for the system which utilizes the state of the observer and show that for the resulting closed-loop system, there is a bound on the  gain from disturbance to output. Our approach is based on parameterizing the set of quadratic Lyapunov functions by the set positive operators, which in turn is parameterized by polynomials and ultimately by Sum-of-Squares (SOS) polynomials and positive matrices - leading to a Linear Matrix Inequality (LMI). The approach we take in this paper is akin to LMI methods for control of linear Ordinary Differential Equations (ODEs) using Lyapunov inequalities and a variable substitution trick. However, because our inequalities are expressed as operators in Hilbert space, we refer to our approach as a Linear Operator Inequality (LOI).

This article extends our work in~\cite{gahlawat2011designing} wherein we designed output feedback boundary controllers for a one-dimensional homogenous heat equation by considering a simpler class of positive operators (also parameterized by SOS polynomials). This paper improves on the work in~\cite{gahlawat2011designing} by a) Considering the larger class of inhomogeneous, possibly unstable parabolic PDEs b) By considering a larger class of Lyapunov functions defined by positive multiplier and integral operators with semi-separable kernels and c) providing evidence (but not a proof) that this new class of operators can be used to design output-feedback based controllers whenever the system is observable and controllable. Specifically, the class of Lyapunov functions we use has the form
 where

and were ,  and  are polynomials and  represents the spatially distributed state of the PDE. A kernel  of this form is referred to as \textit{semi-separable}.

One popular and relatively straightforward method for output feedback boundary control of PDEs is backstepping~\cite{krstic2008boundary}. This method relies on constructing an invertible operator which, in closed loop, maps the state of the system to the state of a chosen stable system for which a quadratic Lyapunov function exists. Our approach varies in the fact that we search for both the controller and the quadratic Lyapunov function. Although our approach is different, similar to backstepping, the numerical results indicate that we can construct output feedback controllers for any controllable and observable system. Some other examples of work which use Lyapunov functions for analysis and control of PDEs are~\cite{coron2008dissipative}, \cite{coron2007strict}. An example of application of LMIs for the control of PDEs is~\cite{fridman2009lmi} where the authors synthesize stabilizing boundary controllers for uncertain semi-linear PDEs using quadratic Lyapunov functions parameterized by positive scalars. Early results on the use of SOS for analysis and control of infinite-dimensional systems can be found in~\cite{peet2006positive}, \cite{papachristodoulou2006analysis}. Additional recent work on the application of polynomials to infinite-dimensional systems can be found in the research done by our colleagues in~\cite{Valmo_1} and~\cite{Valmo_2}.

The paper is organized as follows: Section~\ref{sec:pro_state} outlines the problem statement and presents background on SOS polynomials. In Section~\ref{sec:posop} we define the class of positive operators which we utilize. Section~\ref{sec:prelim} provides a controller synthesis condition and related inequalities which are later used to prove the main result. In Sections~\ref{sec:control} we provide the main results wherein we construct output feedback controllers. Section~\ref{sec:num_results} provides the numerical results for an example PDE.
\section{NOTATION}\label{sec:notation}


 denotes the set of real -by- matrices.  is the subspace of symmetric matrices.  is the identity matrix of dimension  and we denote  when  is clear from context. For any ,  is the space of -times continuously differentiable functions defined on . Similarly, for any ,  is the space of functions which are -times and -times continuously differentiable on  and  respectively.  The shorthand  denotes the partial derivative of  with respect to independent variable . We use  to denote the Hilbert space of square integrable functions from  to .
Unless otherwise indicated,  denotes the inner product on  and  denotes the norm induced by the inner product. Similarly,  denotes the Hilbert space of square integrable functions from  to  equipped with the norm

 is the Sobolev subspace equipped with inner product . For Hilbert spaces  and , the set  is the Banach space of bounded linear operators from  to  endowed with the induced norm .  denotes the identity operator.
We define  to be the column vector of all monomials in variables  of degree  or less. For brevity, we sometimes use .

\section{PROBLEM STATEMENT}\label{sec:pro_state}
In this paper, we consider the following scalar parabolic PDE

where , , with mixed boundary conditions of the form
 Here ,  and  are polynomials with , for . Additionally,  is the \textit{exogenous input} and  is the \textit{control input}. The output of the system is . Note that we have also considered several other types of observer-controller boundary conditions, which will be listed in the section on numerical results. The first goal of the paper is to find a control operator  such that if , then the closed-loop PDE system is stable.

Next, using the Luenberger framework, we assume our observer has the form
 with boundary conditions

where the function  and scalar  must be chosen such that the dynamics of the error  are stable. The second goal of the paper, then, is to find such  and  and show that if , then the coupled system of parabolic PDEs is stable and satisfies

for some . Note that for the system and the observer, we assume the existence of classical solutions belonging to . This assumption can be validated using the analysis presented in~\cite{balogh2004stability} and~\cite{fridman2009lmi}.

\subsection{SOS and Operators}\label{subsec:sos}
SOS is an approach to  the optimization of positive polynomial variables. Given a polynomial , , the feasibility problem of determining if the polynomial is globally positive ( for all ) is NP-hard~\cite{blum1998complexity}. To overcome this difficulty, there are a number of sufficient conditions for polynomial positivity. A particularly important such condition is that the polynomial, , be a Sum-of-Squares (SOS), so that , for polynomials   and which is denoted . The importance of the SOS condition lies in the fact that it can be readily enforced using LMIs. This is due to the easily proven fact that for a polynomial  of degree ,  if and only if  for some , where  is the vector of monomials of degree  or less~\cite{parrilo2000structured}.  A recent survey for alternatives to SOS based methods may be found in~\cite{kamyar2014polynomial}.

We can use SOS to construct positive operators on . For example, define the operator , , where  is a polynomial. If, for , , then the operator  is positive on . Therefore, we may conclude that  is positive on  if there exists a  such that
 . By equating the coefficients on the left and right-hand sides, we obtain an LMI test for positivity of the operator. Of course, the operators considered in this paper are significantly more complicated than .
\section{POSITIVE OPERATORS ON }\label{sec:posop}
In this paper, our results are expressed as optimization over a set of positive operators. To solve these optimization problems, we use positive matrices to parameterize a subset of positive operators on  as described in~\cite{peetlmi}. Specifically, we consider operators of the form
 with semi-separable kernel

where  and  are polynomials.  In~\cite{peet2008using}, we gave necessary and sufficient conditions for positivity of multiplier and integral operators of similar form using pointwise constraints on the functions ,  and . Recently, in~\cite{peetlmi}, these conditions were sharpened - See Theorem~. The following theorem is an extension of this result.
\begin{theorem}\label{thm:jointpos}
Let

where ,  and  and

Then the operator  defined in Eqn.~\eqref{eqn:Poperator} is  self-adjoint and satisfies

where ,  is the maximum eigenvalue of , and

\end{theorem}
\begin{proof}
The proof is based on the result in~\cite{peetlmi} and is omitted for brevity.
\end{proof}
For convenience, we define the set of multipliers and kernels which satisfy Theorem~\ref{thm:jointpos}.

Of course, since such operators are positive definite and bounded on , the inverse of these operators exist and are bounded~\cite{kreyszig1989introductory}. However, as will become apparent in subsequent sections, we need a method of constructing the inverse of operators defined by elements of . Fortunately, such methods do exist in literature and we use one such method. Using the terminology presented in~\cite{gohberg1984time} it can be shown that the operators defined by  are the input-output maps of well-posed Linear Time Varying (LTV) systems. For this class of operators, the inverse can be constructed as explained in~\cite{gohberg1984time}.
\section{PRELIMNARY INEQUALITIES}\label{sec:prelim}
In this section we provide a couple of inequalities which we will use for the controller and observer synthesis. We begin by defining the operator  (infinitesimal genearator) which defines the class of PDEs under consideration.
 where recall ,  and  are polynomial functions and , for . Before presenting the inequalities, we define a pair of mappings which relate the functions  to the derivative of the Lyapunov function . The first mapping considers .
\begin{definition}\label{def:dual}
For scalar  and polynomials ,  and  which define the PDE under consideration, we say  if

\end{definition}
The second mapping relates the functions  to the derivative of the \textit{dual} functional .
\begin{definition}\label{def:primal}
Given scalar  and polynomials ,  and  which define the PDE under consideration, we say  if the following hold

\end{definition}

The proofs of the following lemmas are provided in the appendix.

 The first allows us to represent 
\begin{lemma}\label{lem:dual_LOI}
For any , , let . Then, for any , if the operator  is given by
 with
 and operator  is given by Equation~\eqref{eqn:A_prelim}, we have that

where  for any  with . Here we define the operator  as
 with

\end{lemma}
The second lemma allows us to represent the derivative of .
\begin{lemma}\label{lem:primal_LOI}
For any , , let . Then, for any , if operator  is given by

 with

and operator  is given by Equation~\eqref{eqn:A_prelim},
we have that
 for any  with . Here we define the operator , for any , as
 with

\end{lemma}
\section{OUTPUT FEEDBACK CONTROLLER SYNTHESIS}\label{sec:control}

Our approach to design of output-feedback controllers is based on three steps. First, we design the control operator  which maps that state to the control input as . However, because we cannot measure the state, we find function  and scalar  which define the observer which outputs an estimate of the state . Finally, we prove that the controller coupled to the observer as  produces a closed-loop system with bounded  gain from exogenous input to controlled output.

\subsection{Control Design}
We begin by designing the control operator . Consider the following observer dynamics


with boundary conditions
 The following lemma defines the operator .
\begin{lemma}\label{lem:control}
Suppose there exist  and  such that
 and

Let  where ,  is as in Eqn.~\eqref{eqn:pop} and the operator  is defined as

where  is any scalar such that .
Now, if  where  is as in Equation~\eqref{eqn:pop}. Then for any solution  of Eqns.~\eqref{eqn:obserror_PDE} and~\eqref{eqn:obserror_PDE_BC} with input  , we have that

for some  where  and  is defined in Lemma~\ref{lem:dual_LOI}.
\end{lemma}


\begin{proof}
We begin by taking the time derivative of the Lyapunov function  along the trajectories of~\eqref{eqn:obserror_PDE}-\eqref{eqn:obserror_PDE_BC}
 where we use that  is self-adjoint have simplified the derivative using the definition of operator  provided in Equation~\eqref{eqn:A_prelim}. We rewrite this as
 Now define , then
 Now, applying Lemma~\ref{lem:dual_LOI} and using the facts that  and  produces
 Since , . Thus
 From the boundary condition in~\eqref{eqn:obserror_PDE_BC} we get
 Using the definition of ,
 Substituting into Equation~\eqref{eqn:cont:1} and using the definition  we get
 Substituting this expression into~\eqref{eqn:cont:Vdot1}
 Now, since  is a scalar such that , there exists a scalar  such that . Hence

\end{proof}
\subsection{Observer Design}

We now design the function  and scalar  which define the observer.
We begin by subtracting Equations~\eqref{eqn:prob:PDE_form}-\eqref{eqn:prob:PDE_form_BC} from~\eqref{eqn:prob:observer_form}-\eqref{eqn:prob:observer_form_BC} to obtain the dynamics of the error variable  given by
 with boundary conditions
 where we have used the definition of the measurement  and . We present the following lemma.
\begin{lemma}\label{lem:observer}
Suppose there exist , such that

where  (See Defn.~\ref{def:primal}).
Let  be defined as in Eqn.~\eqref{eqn:sop}. Then choose a scalar  such that
 and let  where .
Define . Then for any  which satisfies Eqns.~\eqref{eqn:error_PDE}-\eqref{eqn:error_PDE_BC} with  and  as defined here, we have

for some scalar  where  is as defined in Lemma~\ref{lem:primal_LOI}.
\end{lemma}

\begin{proof}
We begin by taking the time derivative of the Lyapunov function  along the trajectories of~~\eqref{eqn:error_PDE}-\eqref{eqn:error_PDE_BC}, yielding
 where we have again used the definition of  from Eqn.~\eqref{eqn:A_prelim} and we have also usd the fact that  is self-adjoint. Now, since from the theorem statement we have that  and , applying Lemma~\ref{lem:primal_LOI} produces
 where we have used the fact that since , . We have the boundary condition  and since , we have that . Substituting in~\eqref{eqn:obs:Vdot_2},
 Since  is a scalar such that , let
 Then . Now, using the definition of  we get that
 Substituting Eqns.~\eqref{eqn:obs:gain1}-\eqref{eqn:obs:gain2} into Eqn.~\eqref{eqn:obs:Vdot_3}, we find

\end{proof}


\subsection{Output Feedback Based Control}
We now have the following set of coupled parabolic PDEs.

with boundary conditions


We now prove that the previously designed controller and the observer can be coupled such that norm of the system state remains bounded in the presence of an exogenous input.
\begin{theorem}\label{thm:coupled}
Suppose there exist scalars ,   and polynomials  and , such that
 where  and   as provided in Definitions~\ref{def:dual} and~\ref{def:primal} respectively. Let  be defined as in Equation~\eqref{eqn:pop} and  as in Equation~\eqref{eqn:sop}.

Then for any solution  of the coupled dynamics~\eqref{eqn:couple1}-\eqref{eqn:couple4}, if  is given by Lemma~\ref{lem:control} and   and  are given by Lemma~\ref{lem:observer}, there exists a scalar  such that
 for any .
\end{theorem}
\begin{proof}
For the Lyapunov function , we have from Lemma~\ref{lem:control} that there exists  scalar  such that
 We have from Lemma~\ref{lem:observer} that . Therefore


For the Lyapunov function , we have from Lemma~\ref{lem:observer} that there exists a scalar  such that
 From Equations~\eqref{eqn:coupled:Vobs}-\eqref{eqn:coupled:Vcont} we conclude that for any 
 where , for any , and the inner product is defined on . Now, since , we have that . Therefore, for any , it can be established using Schur complement that for a large enough ,
 Therefore
 Substituting into Equation~\eqref{eqn:coupled:Vtotal_1}, we get
 Let , thus
 Adding  to both sides,
 Since , we have that . Hence, using Schur complement we conclude that

Therefore, from Equation~\eqref{eqn:coupled:Vtotal_2} we conclude that
 Since the operator  is defined using , we have from Theorem~\ref{thm:jointpos} that, for all ,
 Similarly, since  is defined using , using Theorem~\ref{thm:jointpos} it can established that
 Since , from the previous expression we have that
 Substituting Equation~\eqref{eqn:coupled:bound2} in Equation~\eqref{eqn:couple:Vtotal_3} and using~\eqref{eqn:coupled:bound1}, we get
 Integrating in time from  to some , we get
 Now,
. Additionally, if we assume zero initial conditions, then
. Therefore we conclude from Equation~\eqref{eqn:couple:Vtotal_4} that
 where . Since , taking the limit , we get
 Hence, we conclude that
 Since , . Therefore
 Setting  completes the proof.
\end{proof}
\section{NUMERICAL RESULTS}\label{sec:num_results}
In this section we consider a couple of examples on which we test the conditions of Theorem~\ref{thm:coupled} using SOS and SDP. These numerical results are obtained using the Matlab toolbox SOSTOOLS~\cite{prajna2001introducing}.

We consider the following two PDEs. First consider the classical heat equation with an unsteady source term.

Without feedback, this system is unstable for . Next, we consider a randomly generated PDE.
 with boundary conditions
 By using stability analysis and numerical simulation, we estimate that PDE is unstable for .

In these examples, we find the maximum , using a bisection search, for which we can construct stabilizing output-based boundary feedback controllers. We test the conditions of Theorem~\ref{thm:coupled} with , ,  and increasing values of  and . Table~\ref{table:exmp1:cont} presents the maximum   for which we can construct output feedback controllers for PDE~\eqref{eqn:exmp1} as a function of the degree  of the polynomials which define the controller, observer, and Lyapunov function. Table~\ref{table:exmp2:cont} presents the maximum  for PDE~\eqref{eqn:exmp2}.

\begin{table}{}
\caption{Maximum  as a function of polynomial degree  for which we can construct output feedback boundary controllers for PDE~\eqref{eqn:exmp1}.}
\vspace{-10pt}
\begin{center}
    \begin{tabular}{l *{7}{c}}\hline
   &  &  &  &   \\ \hline
   &  &  &  & 
\end{tabular}
\end{center}
\label{table:exmp1:cont}
\end{table}

\begin{table}{}
\caption{Maximum  as a function of polynomial degree  for which we can construct output feedback boundary controllers for PDE~\eqref{eqn:exmp2}.}
\vspace{-10pt}
\begin{center}
    \begin{tabular}{l *{7}{c}}\hline
   &  &  &  &   \\ \hline
   &  &  &  & 
\end{tabular}
\end{center}
\label{table:exmp2:cont}
\end{table}

The numerical results suggest that increasing the degree  of the polynomial representation leads to higher values of . Moreover, the value of  does not appear to be upper bounded, which implies that the method is asymptotically accurate. That is given any , we conjecture that we can construct output feedback controllers for a large enough degree .

Figures~\ref{fig:1}-\ref{fig:2} represent the simulation of PDE~\eqref{eqn:exmp2} with  subject to the output feedback based control in the presence of exogenous input .
\begin{figure}[h!]
\vspace{-10pt}
    \centering
    \includegraphics[width=0.4\textwidth]{state}
    \vspace{-5pt}
\caption{State of PDE~\eqref{eqn:exmp2} with point observation and point actuation.}
\label{fig:1}
\end{figure}
\begin{figure}[h!]
\vspace{-10pt}
    \centering
    \includegraphics[width=0.4\textwidth]{control}
    \vspace{-5pt}
\caption{Boundary control effort  for PDE~\eqref{eqn:exmp2}.}
\label{fig:2}
\end{figure}

One of the key technical advances of this paper is the use of semi-separable kernels , ,  and  and this advance leads to significantly more complex stability conditions. Therefore we wish to establish if the inclusion of the variables , ,  and  does, in fact, provide any significant performance gain. In order to do this, we check the conditions of Theorem~\ref{thm:coupled} while setting  (similar to our previous approach~\cite{gahlawat2011designing}) and applied these conditions to the example PDEs.  Table~\ref{table:simple_lyap1:cont1} presents these results for PDE~\eqref{eqn:exmp1} and Table~\ref{table:simple_lyap1:cont2} presents results for PDE~\eqref{eqn:exmp2}.

\begin{table}{}
\caption{Maximum  as a function of polynomial degree,  for PDE~\eqref{eqn:exmp1} for which we can construct controllers using  with .}
\vspace{-10pt}
\begin{center}
    \begin{tabular}{l *{7}{c}}\hline
     &  &  &   \\ \hline
     &  &   & 
\end{tabular}
\end{center}
\label{table:simple_lyap1:cont1}
\end{table}

\begin{table}{}
\caption{Maximum  as a function of polynomial degree,  for PDE~\eqref{eqn:exmp2} for which we can construct controllers using  with .}
\vspace{-10pt}
\begin{center}
    \begin{tabular}{l *{7}{c}}\hline
     &  &  &   \\ \hline
     &  &   & 
\end{tabular}
\end{center}
\label{table:simple_lyap1:cont2}
\end{table}

Comparing Tables~\ref{table:simple_lyap1:cont1}-\ref{table:simple_lyap1:cont2} with Tables~\ref{table:exmp1:cont}-\ref{table:exmp2:cont} we observe that the inclusion of kernels , ,  and  allows the construction of output feedback based controllers for significantly higher values of . Moreover, by setting , the numerical results appear to show an upper bound to the  for which we can design controllers without the use of these kernel functions. We conjecture, therefore, that kernel functions are a necessary part of any Lyapunov-based method for analysis and control of PDEs.

Finally, as we previously stated, the SOS conditions for the design of output feedback controllers can be easily modified for systems with other types of boundary conditions. To this end, we provide the numerical results for controller synthesis for PDEs~\eqref{eqn:exmp1} and \eqref{eqn:exmp2} with boundary conditions defined in Table~\ref{table:alt_BC}.
\begin{table}{}
\caption{Alternative boundary conditions and outputs for PDE~\eqref{eqn:exmp1}-\eqref{eqn:exmp2}.}
\begin{center}
    \begin{tabular}{l *{3}{c}}\hline
  & Boundary Condition & Output   \\ \hline
   Dirichlet & {} &   \\ \hline
   Neumann & {} &  \\ \hline
   Robin & {} & 
\end{tabular}
\end{center}
\label{table:alt_BC}
\end{table}
Table~\ref{table:alt:cont1} illustrates the maximum  for which we can construct output feedback controllers as a function of  for PDE~\eqref{eqn:exmp1} with boundary conditions and outputs given in Table~\ref{table:alt_BC}. Similarly, Table~\ref{table:alt:cont2} illustrates these results for PDE~\eqref{eqn:exmp2}.

\begin{table}{}
\caption{Maximum  as a function of polynomial degree,  for PDE~\eqref{eqn:exmp1} with boundary conditions and outputs given in Table~\ref{table:alt_BC}.}
\begin{center}
    \begin{tabular}{l *{7}{c}}\hline
  & &  &  &  &   \\ \hline
  Dirichlet &  &  &  &  &    \\
  Neumann &  &  &  &  &  \\
  Robin &  &  &  &  &  \\
\end{tabular}
\end{center}
\label{table:alt:cont1}
\end{table}

\begin{table}{}
\caption{Maximum  as a function of polynomial degree,  for PDE~\eqref{eqn:exmp2} with boundary conditions and outputs given in Table~\ref{table:alt_BC}.}
\begin{center}
    \begin{tabular}{l *{7}{c}}\hline
  & &  &  &  &   \\ \hline
  Dirichlet &  &  &  &  &    \\
  Neumann &  &  &  &  &  \\
  Robin &  &  &  &  &  \\
\end{tabular}
\end{center}
\label{table:alt:cont2}
\end{table}

We note that the backstepping method has been applied to Example~\eqref{eqn:exmp1} and is also able to construct exponentially stabilizing output feedback boundary controllers for arbitrary  (see~\cite{krstic2008adaptive}). Therefore, while we cannot necessarily claim any improvement in performance over this established methods, our approach is at least competitive and may have certain advantages such as relative ease of implementation (changing the system is a one-line edit) and the fact that our approach does not require numerical integration of a PDE.
\section{CONCLUSIONS}
In this paper we developed an algorithmic approach for designing output feedback boundary controllers for a class of linear scalar valued inhomogeneous parabolic PDEs. Our approach is based on a parameterization of positive multiplier and integral operators with semi-separable kernels. We tested the approach on homogeneous and inhomogeneous PDEs using several different types of boundary feedback and several different types of point measurements. Furthermore, we tested our approach with and without kernel functions to determine if kernel functions are a necessary part of Lyapunov theory for PDEs. Our numerical results indicate that kernel functions are a necessary part of Lyapunov functions for PDEs. Further, our numerical results indicate there is little or no conservativity in the method and that our approach is competitive with well-established approaches such as backstepping.
Note that as yet, the observer-based controllers in this paper are not optimal in any norm. Therefore, an obvious future direction of this work is to extend our approach to -optimal control.
\section*{APPENDIX}

To prove Lemmas~\ref{lem:dual_LOI} and~\ref{lem:primal_LOI}, we use the following identity.
\begin{lemma}[\cite{hardy1952inequalities},\cite{krstic2008boundary}]
\label{lem:wirtinger}
 let  be a scalar function. Then
  
\end{lemma}

\begin{proof}[Lemma~\ref{lem:dual_LOI}]
We begin by considering the following decomposition
 where
 where we have used the fact that


Before we proceed, we calculate the boundary condition at . Since , for any  with , we have that . Using the definition of ,
 where we have used the fact that . Since , we conclude that
 Applying integration by parts twice and using the boundary condition at , we get

 
From Theorem~\ref{thm:jointpos} it is readily established that . Additionally, we have that . Therefore,  and we may apply Lemma~\ref{lem:wirtinger} to produce

Therefore, 
Similarly, applying integration by parts once,

Now, note that for , we have  and thus . Utilizing this property and applying integration by parts twice
 Dividing the double integrals,
 Changing the order of integration, switching between  and  and using the fact that  in the last two double integral produces
  Applying integration by parts once and employing the previously performed change of order of integration
  Finally, applying a change of order of integration as applied to  and ,
 Substituting \eqref{eqn:dual_gamma_1}-\eqref{eqn:dual_gamma_5} in \eqref{eqn:dual_gamma} and using Definition~\ref{def:dual},

Finally, using the definition of operator ,

\end{proof}

\begin{proof}[Lemma~\ref{lem:primal_LOI}]
Using the self-adjointedness of operator  we begin with the following decomposition

 where
  Here we have used the fact that



Applying integration by parts twice and using the boundary condition  yields
 
 Since  and , we have . Thus, by application of Lemma~\ref{lem:wirtinger} we get
 
Therefore, we conclude that
 
Similarly, applying integration by parts once
 
Now, note that for , we have  and thus . Exploiting this property and using the boundary condition, we may apply integration by parts twice and use  to obtain 
 We can divide the two double integrals as

  Changing the order of integration in the last two double integrals, switching the variables  and  and using ,
 
  Applying integration by parts once and following the same procedure as for ,
  
Finally, employing a change of order of integration as done for  and  produces
 
 Substituting \eqref{Gamma_1}-\eqref{Gamma_5} into \eqref{Gamma_eqn} and using Definition~\ref{def:primal} gives us
  
 Finally, using the definition of operator ,
 
\end{proof}
\section*{ACKNOWLEDGMENT}

This research was carried out with the financial support of NSF CAREER Grant CMMI-1151018.





\bibliographystyle{plain}
\bibliography{CDC_015}









\end{document}
