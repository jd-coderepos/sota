\subsection{Experiments}
\label{ch7:experiments}

 For our experiment, we have collected a set of 68 programs from different sources.
\begin{enumerate}
\item A set of 30 programs from SV-COMP'15 repository\footnote{http://sv-comp.sosy-lab.org/2015/benchmarks.php} \cite{DBLP:conf/tacas/000115} (recursive category)  and translated them to Horn clauses using inter-procedural encoding of SeaHorn \cite{DBLP:conf/tacas/GurfinkelKN15,DBLP:conf/cav/GurfinkelKKN15} producing  (mostly) non-linear Horn clauses. 

\item A set of 38 problems taken from the source repository\footnote{https://github.com/sosy-lab/sv-benchmarks/tree/master/clauses/LIA/Eldarica}, compiled by the authors of the tool Eldarica \cite{DBLP:conf/fm/HojjatKGIKR12}. This set consists of problems, among others, from  the NECLA static analysis suite, from the paper \cite{DBLP:conf/tacas/JhalaM06}. These tasks  are also considered in \cite{WangJ2015} and  are interpreted over  \emph{integer linear arithmetic}.
\end{enumerate}

We made the following comparison between the tools.

\begin{enumerate}
\item We compare $\rahit$  with  $\rahft$, which compares the effect of removing a set of traces rather than a single trace.  

\item We compare $\rahit$ with the \emph{trace-abstraction} tool \cite{WangJ2015} (\emph{TAR} from now on).   $\rahit$  uses polyhedral approximation combined with trace abstraction refinement  whereas  \emph{TAR} uses only trace abstraction refinement.
\end{enumerate}





The results are summarized in Table~\ref{ch7:tbl:experiments}. 

\paragraph{Implementation:}
Most of the tools in our tool-chain depicted in Figure \ref{ch7:fig:toolchain} are implemented in Ciao Prolog \cite{DBLP:journals/tplp/HermenegildoBCLMMP12} except the one for  determinisation of FTA, which is implemented in Java following the algorithm described in  \cite{DBLP:journals/corr/GallagherAK15}. Our tool-chain obtained by combining  various tools using a \emph{shell script} serves as a proof of concept which is not optimised at all. For handling constraints, we  use the Parma polyhedra library \cite{DBLP:journals/scp/BagnaraHZ08} and the Yices SMT solver \cite{Dutertre:cav2014} over \emph{linear real arithmetic}. The construction of tree interpolation uses constrained based algorithm presented in \cite{DBLP:journals/jsc/RybalchenkoS10} for computing interpolant of two formulas. 

\paragraph{Description:} In Table~\ref{ch7:tbl:experiments},  {\it Program} represents a verification task, {\it Time (secs) RAHFT} and {\it Time (secs) RAHIT} - respectively represent the time in seconds taken by the the tool  $\rahft$ and  $\rahit$ respectively for solving a given task. Similarly,  the number of \emph{abstraction-refinement} iteration needed in these cases to solve a task are represented by   {\it \#Itr. RAHFT} and {\it \#Itr. RAHIT}. Similarly, {\it Time (secs) TAR} and {\it \#Itr. TAR} represent the time taken and the number of iterations  needed by the  tool \emph{TAR}. The experiments were run on a MAC computer running  OS X on  2.3 GHz Intel core i7 processor  and 8 GB memory.

\paragraph{Discussion:}
The comparison between $\rahft$ and $\rahit$ would reflect purely the role of \emph{interpolant tree automata} in Horn clause verification (Table~\ref{ch7:tbl:experiments}) since the only difference between them is the refinement part using \emph{(interpolant) tree automata}. 
The results show that $\rahit$ is more effective in practice than its counterpart $\rahft$. This is justified by the  number of tasks 61/68 solved by $\rahit$ using fewer iterations  compared to $\rahft$, which only solves 56/68 tasks.   This is due to the generalisation of a spurious counterexample during refinement, which also captures other infeasible traces. Since these traces  can be removed in the same iteration,  it (possibly) reduces the number of refinements, however the solving time goes up because of the cost of computing an interpolant automaton.  It is not always the case that $\rahit$ takes less iterations for a task (for example  \emph{Addition03 false-unreach}) than $\rahft$. This is because  the  restructuring of the program obtained  as a result of removing a set of traces may  or may not favour   polyhedral approximation. It is still not clear to us how to produce a right  restructuring which favours  polyhedral approximation.  $\rahit$ times out on \texttt{cggmp2005\_true-unreach}  whereas  $\rahft$ solves it in 5 iterations. We suspect that this is due to the cost of generating  interpolant automata. We are not sure about the complexity of interpolant generation algorithm we used (the size of the formula generated was quite large with respect to the original program, magnitude not known) and there are several calls to the  theorem prover to label each tree node with interpolants.  So the bigger is the trace-tree, the longer it takes to compute the interpolant tree.  In average,  $\rahit$  needs 2.08 iterations and 11.40 seconds time to solve a task whereas $\rahft$ needs 2.32 iterations and 10.55 seconds.



The use of \emph{interpolant tree automata} for trace generalisation and the  tree automata based operations for trace-refinement are same  in both $\rahit$ and  \emph{TAR}. Since \emph{TAR} is not publicly available, we chose the same set of benchmarks  used by \emph{TAR} for the purpose of comparison and presented the results (the results corresponding to \emph{TAR} are taken from \cite{WangJ2015}). The computer used in our experiments and in \emph{TAR} \cite{WangJ2015} have similar characteristics. $\rahit$ solves more than half of the problems  only with \emph{abstract interpretation} over the domain of convex polyhedra without needing any refinement, which indicates its power. $\rahit$ solves 33/38 problems where as \emph{TAR} solves 28/38 problems. In average, $\rahit$ takes less time than \emph{TAR}. In many cases \emph{TAR} solves a task faster than $\rahit$,   however  it spends much longer time in some tasks.  
Our current constraint solver is over   \emph{linear real arithmetic}. If we use it over \emph{linear integer arithmetic} then the results may differ. 
We made some observation with   the problems  \emph{boustrophedon.c},  \emph{boustrophedon\_expansed.c} and \emph{cousot.correct} (which are supposed to be interpreted over integers). In them, if we replace strict inequalities ($>, <$)  with non-strict inequalities ($\geq,\leq$) over integers  (for example replace $X>Y$ with $X \geq Y+1$), then we can solve them only with \emph{abstract interpretation} without refinement which were not solved before the transformation using our solver.  On the other hand, $\rahit$ times out for \emph{ mergesort.error } whereas \emph{TAR} solves it in a single iteration. This indicates that the choice of a spurious counterexample and the quality of interpolant generated from it for generalisation have some effects on verification.



\begin{table}
\resizebox{1.0\textwidth}{106mm}{
    \begin{tabular}{|l|l|l|l|l|l|l|}
    \hline
    \textbf{Program}                              & \textbf{Time (secs) RAHFT} & \textbf{\#Itr. RAHFT} & \textbf{Time (secs) RAHIT} & \textbf{\#Itr. RAHIT} & \textbf{Time (secs) TAR \cite{WangJ2015}} & \textbf{\#Itr. TAR} \\ \hline
    addition                                                 & 1                                     & 0                                          & 1                                     & 0                                          & 0.26                                                            & 3                                        \\ \hline
    anubhav.correct                                          & 2                                     & 0                                          & 2                                     & 0                                          & 1.72                                                            & 9                                        \\ \hline
    bfprt                                                    & 1                                     & 0                                          & 1                                     & 0                                          & 0.43                                                            & 6                                        \\ \hline
    binarysearch                                             & 2                                     & 0                                          & 2                                     & 0                                          & 0.36                                                            & 5                                        \\ \hline
    blast.correct                                            & 5                                     & 1                                          & 11                                    & 1                                          & 8.93                                                            & 65                                       \\ \hline
    boustrophedon.c                                          & TO                                    & -                                          & TO                                    & -                                          & 53.65                                                           & 193                                      \\ \hline
    boustrophedon\_expansed.c                                & TO                                    & -                                          & TO                                    & -                                          & 69.06                                                           & 340                                      \\ \hline
    buildheap                                                & 44                                    & 9                                          & 44                                    & 9                                          & TO                                                              & -                                        \\ \hline
    copy1.error                                              & 11                                    & 0                                          & 11                                    & 0                                          & 12.79                                                           & 19                                       \\ \hline
    countZero                                                & 1                                     & 0                                          & 1                                     & 0                                          & TO                                                              & -                                        \\ \hline
    cousot.correct                                           & TO                                    & -                                          & TO                                    & -                                          & TO                                                              & -                                        \\ \hline
    gopan.c                                                  & 3                                     & 0                                          & 3                                     & 0                                          & TO                                                              & -                                        \\ \hline
    halbwachs.c                                              & TO                                    & -                                          & TO                                    & -                                          & TO                                                              & -                                        \\ \hline
    identity                                                 & 1                                     & 0                                          & 1                                     & 0                                          & 7.67                                                            & 34                                       \\ \hline
    inf1.error                                               & 4                                     & 1                                          & 9                                     & 1                                          & 0.51                                                            & 6                                        \\ \hline
    inf6.correct                                             & 5                                     & 1                                          & 5                                     & 1                                          & 1.96                                                            & 33                                       \\ \hline
    insdel.error                                             & 2                                     & 0                                          & 2                                     & 0                                          & 0.17                                                            & 1                                        \\ \hline
    listcounter.correct                                      & 1                                     & 0                                          & 1                                     & 0                                          & TO                                                              & ~                                        \\ \hline
    listcounter.error                                        & 9                                     & 1                                          & 9                                     & 1                                          & 0.21                                                            & 1                                        \\ \hline
    listreversal.correct                                     & 4                                     & 0                                          & 4                                     & 0                                          & 35.79                                                           & 149                                      \\ \hline
    listreversal.error                                       & 9                                     & 0                                          & 9                                     & 0                                          & 0.3                                                             & 1                                        \\ \hline
    loop.error                                               & 3                                     & 0                                          & 3                                     & 0                                          & 3                                                               & 3                                        \\ \hline
    loop1.error                                              & 8                                     & 0                                          & 8                                     & 0                                          & 10.87                                                           & 19                                       \\ \hline
    mc91.pl                                                  & 139                                   & 24                                         & 7                                     & 3                                          & 0.57                                                            & 7                                        \\ \hline
    merge                                                    & 2                                     & 0                                          & 2                                     & 0                                          & 0.86                                                            & 10                                       \\ \hline
    mergesort.error                                          & TO                                    & -                                          & TO                                    & -                                          & 0.32                                                            & 1                                        \\ \hline
    palindrome                                               & 2                                     & 0                                          & 2                                     & 0                                          & 0.61                                                            & 6                                        \\ \hline
    parity                                                   & 3                                     & 1                                          & 4                                     & 1                                          & 0.62                                                            & 7                                        \\ \hline
    rate\_limiter.c                                          & 3                                     & 0                                          & 3                                     & 0                                          & 49.96                                                           & 130                                      \\ \hline
    remainder                                                & 1                                     & 0                                          & 1                                     & 0                                          & 1.5                                                             & 17                                       \\ \hline
    running                                                  & 3                                     & 1                                          & 8                                     & 2                                          & 0.4                                                             & 5                                        \\ \hline
    scan.error                                               & 3                                     & 0                                          & 3                                     & 0                                          & TO                                                              & -                                        \\ \hline
    string\_concat.error                                     & 6                                     & 0                                          & 6                                     & 0                                          & TO                                                              & -                                        \\ \hline
    string\_concat1.error                                    & TO                                    & -                                          & TO                                    & -                                          & TO                                                              & -                                        \\ \hline
    string\_copy.error                                       & 3                                     & 0                                          & 3                                     & 0                                          & TO                                                              & -                                        \\ \hline
    substring.error                                          & 5                                     & 0                                          & 5                                     & 0                                          & 0.55                                                            & 1                                        \\ \hline
    substring1.error                                         & 15                                    & 0                                          & 15                                    & 0                                          & 2.84                                                            & 5                                        \\ \hline
    triple                                                   & 27                                    & 10                                         & 13                                    & 1                                          & 0.86                                                            & 6                                        \\ \hline
    \hline    \textbf{average (over 38)} & ~                                     & ~                                          & \textbf{8.78}              & \textbf{0.93}                   & \textbf{9.52}                                        & \textbf{38.64}                \\ \hline
    \textbf{solved/total}                         & ~                                     & ~                                          & \textbf{33/38}             & -                                          & \textbf{28/38}                                       & ~                                        \\ \hline
    \hline    Primes\_true-unreach                  & 16                                    & 4                                          & 4                                     & 1                                          & ~                                                               & ~                                        \\ \hline
    sum\_10x0\_false-unreach                                 & 5                                     & 2                                          & 12                                    & 2                                          & ~                                                               & ~                                        \\ \hline
    afterrec\_false-unreach                                  & 2                                     & 1                                          & 3                                     & 1                                          & ~                                                               & ~                                        \\ \hline
    id\_o3\_false-unreach                                    & 6                                     & 3                                          & 7                                     & 3                                          & ~                                                               & ~                                        \\ \hline
    cggmp2005\_variant\_true-unreach                         & 2                                     & 1                                          & 3                                     & 1                                          & ~                                                               & ~                                        \\ \hline
    recHanoi01\_true-unreach                                 & 8                                     & 3                                          & 10                                    & 3                                          & ~                                                               & ~                                        \\ \hline
    cggmp2005b\_true-unreach                                 & 3                                     & 1                                          & 3                                     & 1                                          & ~                                                               & ~                                        \\ \hline
    gcd02\_true-unreach                                      & 11                                    & 4                                          & 11                                    & 4                                          & ~                                                               & ~                                        \\ \hline
    diamond\_false-unreach                                   & 3                                     & 1                                          & 3                                     & 1                                          & ~                                                               & ~                                        \\ \hline
    Addition03\_false-unreach                                & 6                                     & 2                                          & 13                                    & 5                                          & ~                                                               & ~                                        \\ \hline
    diamond\_true-unreach-call1                              & 2                                     & 1                                          & 3                                     & 1                                          & ~                                                               & ~                                        \\ \hline
    id\_i5\_o5\_false-unreach                                & 19                                    & 8                                          & 12                                    & 5                                          & ~                                                               & ~                                        \\ \hline
    diamond\_true-unreach-call2                              & 6                                     & 1                                          & 5                                     & 1                                          & ~                                                               & ~                                        \\ \hline
    cggmp2005\_true-unreach                                  & 10                                    & 5                                          & TO                                    & -                                          & ~                                                               & ~                                        \\ \hline
    gsv2008\_true-unreach                                    & 3                                     & 1                                          & 3                                     & 1                                          & ~                                                               & ~                                        \\ \hline
    Fibocci01\_true-unreach                                  & 52                                    & 10                                         & 29                                    & 6                                          & ~                                                               & ~                                        \\ \hline
    id\_b3\_o2\_false-unreach                                & 5                                     & 2                                          & 3                                     & 1                                          & ~                                                               & ~                                        \\ \hline
    Ackermann02\_false-unreach                               & 68                                    & 17                                         & 25                                    & 7                                          & ~                                                               & ~                                        \\ \hline
    mcmillan2006\_true-unreach                               & 2                                     & 1                                          & 3                                     & 1                                          & ~                                                               & ~                                        \\ \hline
    ddlm2013\_true-unreach                                   & TO                                    & -                                          & 17                                    & 7                                          & ~                                                               & ~                                        \\ \hline
    sum\_2x3\_false-unreach                                  & 2                                     & 1                                          & 3                                     & 1                                          & ~                                                               & ~                                        \\ \hline
    fibo\_5\_true-unreach                                    & TO                                    & -                                          & 77                                    & 7                                          & ~                                                               & ~                                        \\ \hline
    Addition01\_true-unreach                                 & 6                                     & 2                                          & 5                                     & 2                                          & ~                                                               & ~                                        \\ \hline
    Ackermann04\_true-unreach                                & TO                                    & -                                          & 59                                    & 8                                          & ~                                                               & ~                                        \\ \hline
    Addition02\_false-unreach                                & 4                                     & 2                                          & 5                                     & 2                                          & ~                                                               & ~                                        \\ \hline
    id\_i10\_o10\_false-unreach                              & TO                                    & -                                          & 39                                    & 10                                         & ~                                                               & ~                                        \\ \hline
    gcd01\_true-unreach                                      & 9                                     & 4                                          & 5                                     & 2                                          & ~                                                               & ~                                        \\ \hline
    id\_o10\_false-unreach                                   & TO                                    & -                                          & 38                                    & 10                                         & ~                                                               & ~                                        \\ \hline
    gcnr2008\_false-unreach                                  & 13                                    & 4                                          & 6                                     & 2                                          & ~                                                               & ~                                        \\ \hline
    Fibocci04\_false-unreach                                 & TO                                    & -                                          & 91                                    & 11                                         & ~                                                               & ~                                        \\ \hline
    \hline    \textbf{average (over 68)} & \textbf{10.55}             & \textbf{2.32}                   & \textbf{11.40}             & \textbf{2.08}                   & ~                                                               & ~                                        \\ \hline
    \textbf{solved/total}                         & \textbf{56/68}             & ~                                          & \textbf{61/68}             & ~                                          & ~                                                               & ~                                        \\ \hline
    \end{tabular}
   }
   
        \caption{Experiments on  software verification problems. In the table  ``TO'' means  time out which is set for 300 seconds,  ``-'' indicates the insignificance of the result.}
   \label{ch7:tbl:experiments}
\end{table}




