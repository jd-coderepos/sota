\documentclass[preprint,11pt]{elsarticle}
\usepackage{amsfonts,amssymb,amsmath,latexsym,ae,aecompl}
\usepackage{epsfig}
\usepackage{graphicx}
\usepackage{algorithmic}
\usepackage{psfrag}
\usepackage{pstricks, pst-all, pstricks-add}
\usepackage{epsfig}
\usepackage{float,epsfig, floatflt,here}
\usepackage{subfig}
\usepackage{a4wide}
\newtheorem{fact}{Fact}
\newtheorem{result}{Result}
\newcommand{\F}{\vspace*{2pt}}
\newcommand{\FF}{\vspace*{4pt}}
\newcommand{\FFF}{\vspace*{6pt}}
\newcommand{\B}{\vspace*{-2pt}}
\newcommand{\BB}{\vspace*{-4pt}}
\newcommand{\BBB}{\vspace*{-6pt}}

\newcommand{\cA}{{\mathcal A}}
\newcommand{\cO}{{\mathcal O}}
\newcommand{\cW}{{\mathcal W}}
\newcommand{\cM}{{\mathcal M}}
\newcommand{\cE}{{\mathcal E}}
\newcommand{\cF}{{\mathcal F}}
\newcommand{\cH}{{\mathcal H}}
\newcommand{\cI}{{\mathcal I}}
\newcommand{\cD}{{\mathcal D}}
\newcommand{\cR}{{\mathcal R}}
\newcommand{\cS}{{\mathcal S}}
\newcommand{\cB}{{\mathcal B}}
\newcommand{\cC}{{\mathcal C}}
\newcommand{\cP}{{\mathcal P}}
\newtheorem{theorem}{Theorem}
\newtheorem{corollary}{Corollary}


\newtheorem{definition}[theorem]{Definition}


\newenvironment{proof}{\noindent{\bf Proof: }}{\qed \smallbreak}
\newcommand{\Paragraph}[1]{\BBB\paragraph{#1}}
\newcommand{\myparagraph}[1]{\BBB\paragraph{#1}}
\usepackage[ruled,vlined]{algorithm2e}

\begin{document}
\begin{frontmatter}

\title{Designing Path Planning Algorithms for\\ Mobile Anchor towards Range-Free Localization}
\author[label1]{Kaushik Mondal}
\author[label2]{Arindam Karmakar}
\author[label1]{Partha Sarathi Mandal}
\address[label1]{Indian Institute of Technology Guwahati, India}
\address[label2]{Tezpur University, India}







\begin{abstract}
Localization is one of the most important factor in wireless sensor networks as
many applications demand position information of sensors.
Recently there is an increasing interest on the use of mobile anchors for localizing sensors.
Most of the works available in the literature either looks into the aspect of reducing path
length of mobile anchor or tries to increase localization accuracy.
The challenge is to design a movement strategy for a mobile anchor that reduces path length
while meeting the requirements of a good range-free localization technique.
In this paper we propose two cost-effective movement strategies i.e., path planning for a mobile anchor
so that localization can be done using the localization scheme \cite{Lee2009}.
In one strategy we use a hexagonal movement pattern for the mobile anchor to localize all sensors inside a bounded
rectangular region with lesser movement compared to the existing works in literature.
In other strategy we consider a connected network in an unbounded region where the mobile anchor
moves in the hexagonal pattern to localize the sensors. In this approach, we guarantee localization of
all sensors within  error-bound where  is the communication range of the mobile anchor and sensors.
Our simulation results support theoretical results along with localization accuracy.
\end{abstract}

\begin{keyword}
Localization\sep Range-free\sep Beacon point\sep Path planning\sep Mobile anchor\sep Wireless Sensor Networks
\end{keyword}

\end{frontmatter}

\section{Introduction}
\label{sec:intro}
Localization of wireless sensors with high degree of accuracy is required for many
wireless sensor networks (WSNs) applications, such as security and surveillance,
object tracking, detecting accurate location of a target etc.
To meet this purpose, many sensor localization schemes
\cite{Ammar2010,Chen2012,Delaet11,Lee2009,Seow:2008,Ssu2005,Xiao2008,zhang2006} have been proposed for WSNs.
These schemes can be viewed as range-based or range-free. In range-based schemes, the sensor locations
are calculated using distance and/or angle information among sensors by ranging hardware.
On the other hand, connectivity constraints such as hop-count, anchor beacons etc are used in range-free
schemes. Usually range-based schemes are more accurate than range-free schemes. But range estimation
techniques in range-based schemes are erroneous as well as costly due to requirement of special hardware,
which encouraged researcher to design range-free schemes for sensor localization.
A number of static anchors are needed in the localization schemes like \cite{Ammar2010,Chen2012,Delaet11,Seow:2008,zhang2006}
which uses static anchors.
To minimize the number of static anchors, localization schemes \cite{Lee2009,Ssu2005,Xiao2008} using mobile anchor are proposed.
One mobile anchor with a suitable path planning is equivalent to many static anchors,
which localizes whole network. By using mobile anchor, we can save large number of anchors with deployment cost
in the expense of the mobility of the mobile anchor. So, path planning of the mobile anchor has
become an important issue in the area of localization. There are few proposed movement strategies which localizes sensors using some basic
techniques like trilateration \cite{Shih2010}, which causes large localization error. Our aim is to
propose a movement strategy such that we can use existing range-free localization schemes which
yields better accuracy.

In this paper we have proposed path planning schemes for the mobile anchor where localization
is done using localization scheme proposed by Lee et al.\cite{Lee2009}.We have proposed two different movement strategies for two different assumptions.
One movement strategy is proposed on the assumption that the network is connected.
The anchor localizes every sensor of the network with connectivity guided movement.
The other movement strategy is proposed to localize sensors over a bounded rectangular
region where the anchor has to cover the whole rectangle to ensure localization of all sensors.
In this strategy, only boundary information is used to localize all the sensors irrespective of
deployment and underlying network topology.

\subsection*{Our contribution:}
In this paper we have proposed a hexagonal movement strategy for a mobile anchor
to localize static sensors with improved (lesser) movement of the mobile anchor
compared to the existing results in literature. We have divided our work in two parts.
First part assumes connectivity in the network whereas later we have used known boundary
of a rectangular region where the static sensors reside. Our achievements are the following.
\begin{itemize}
\item We have proposed a distributed range-free movement strategy to localize all sensors within  error-bound
in a connected network, where  is the transmission range of the sensors and the mobile anchor.
\item We have given another path planning scheme for a bounded rectangular region
using same hexagonal movement pattern.
\item Theoretically we have shown that
the length of the path traversed by the anchor is lesser in the
proposed strategy compared to other existing path planning methods for covering
a rectangular region. \item Our simulation results support all theoretical results for path planning with localization accuracy.
Simulation results show  to  on an average improvement of our scheme over different schemes in
terms of path length while covering a bounded rectangular region.
\end{itemize}

The organization of the remaining part of the paper is as follows. In section \ref{sec:reltdwrk} we discuss
about related works. The theoretical results of our
proposed path planning on a connected network are explained in section \ref{sec:tech1}.
The algorithm along with system model are given in section \ref{sec:algomodel}.
In section \ref{sec:tech2}, we propose a movement strategy of mobile anchor for a bounded rectangular region.
The simulation results are presented in section \ref{sec:sim}, along with performance
comparison with existing approaches. Finally we conclude in section \ref{sec:conclusion}.

\section{Related Works}
\label{sec:reltdwrk}
Path planning algorithms set path for mobile anchor along which it moves in
the network while localization process goes on. First we look at a
brief overview of the existing range free localization schemes which provides
good accuracy and can be used for localization. Ssu et al. proposed
a localization scheme in \cite{Ssu2005} where the sensor's position is estimated as the
intersection of perpendicular bisector of two calculated chords.
However this scheme suffers from short chord length problem. Xiao et al.\cite{Xiao2008} improved
over that scheme using pre-arrival and post-departure points along with the beacon points
to localize a sensor. Later Lee et al. used beacon distance more effectively as another
geometric constraint and proposed a more accurate localization scheme in \cite{Lee2009}.


We can view the path planning problem in two different ways depending on the knowledge of the area of sensor deployment
and the underlying topology formed by the sensors. Topology-based path planning, can be viewed as a
graph traversal problem. Sensors have information about their neighbors which they send to the mobile anchor
for determining the path. Li et al. proposed two algorithms named breadth first
and backtracking greedy algorithms in \cite{Li2008}.
Mitton et al. in \cite{Mitton2012} proposed a depth first traversal scheme by the mobile anchor
to localize the sensors. Both these works need range estimations.
Kim et al. proposed a path planning in \cite{Kim2011} for  randomly deployed sensors
using trilateration method for localization. An already localized sensor
becomes a reference point to help other sensors to find their position
which reduces path length but localization error may propagate.
Chang  et al. proposed another path planning algorithm of the mobile anchor
in \cite{Chang2012} where localization have been done using the scheme proposed by Galstyan et al.
in \cite{Galstyan2004} and mobile sensor calculates its trajectory by moving around already localized sensors.
Our aim is to propose a path planning algorithm which can decide its trajectory without using any range
estimation in a connected network. Using connectivity of the network, we discover neighbors of a sensor as
well as localize it by the scheme \cite{Lee2009} using our proposed path planning algorithm.

The other way of viewing the path planning problem is to cover a rectangular area
by the mobile anchor where all the sensors are deployed.
,  and  schemes are proposed by Koutsonikolas et al. in \cite{Koutsonikolas2007}.
They used the localization scheme proposed in \cite{Sichitiu2004}. Scan covers the whole area
uniformly where the mobile anchor travels in line segments along -axis (or -axis) keeping a
fixed distance between two line segments. In , anchor moves along both -axis
and -axis, which improves localization accuracy in the expense of traveled distance.
 reduces both error and path length with compare to the other two.
Huang et al. proposed two path planning schemes namely  and {\it S-curves} in \cite{Huang2007}.
Simulation results show that these two schemes produce better results than those discussed above.
Based on trilateration, Han et al.  proposed a path planning scheme in \cite{Han2013} for a
mobile anchor. Using Received Signal Strength Indicator (RSSI) technique, sensor measures distances from three
different non collinear points and finds its position. Chia-Ho-Ou et al. proposed
a movement strategy in \cite{Chia-Ho-Ou2013} of the anchor which helps sensors to localize
with good accuracy by reducing the short chord length problem of Ssu's scheme \cite{Ssu2005}.
Our aim is to propose a path planning which minimizes the path length compared to the
existing ones and guarantee positioning of each sensor using scheme
proposed by Lee et al. in \cite{Lee2009}.

\section{Path Planning for Connected Network}
\label{sec:tech1}
In this section we discuss path planning to localize an arbitrary connected network of any number of sensors.
The mobile anchor broadcasts beacon with its position information after every  time interval.
We may use the term `anchor' instead `mobile anchor' in the rest part of this paper. Here required definitions are given below.
\begin{definition}
{\rm(}Beacon distance{\rm)} Distance traveled by the mobile anchor between two consecutive broadcasts
of beacon is called {\it beacon distance} and is denoted by .
\end{definition}
\begin{definition}
{\rm(}Communication circle{\rm)} The circle with radius  centering at the sensor, where  is the communication range of the sensor.
\end{definition}
\begin{definition}
{\rm(}LRH{\rm)} Largest regular hexagon inscribed within the communication circle of any sensor.
\end{definition}
\begin{definition}
{\rm(}Beacon point{\rm)}
The position of the anchor that is extracted from the beacon received by a sensor at time  is denoted as a beacon point for the sensor if and only if the sensor does not receive any beacon either in time interval  or in time interval , where  is the waiting time such that  and  is time interval of periodical broadcasts of beacon by the anchor.
\end{definition}
In this paper we use  as a sensor as well as a point in the plane that defines the location of that sensor.
The algorithm begins with localizing a sensor  by random movement of the anchor.
The localized sensor broadcasts its position which is received by the anchor and according to our movement strategy,
the anchor reaches at any point  on the communication circle of the localized sensor .
The anchor computes the LRH inscribed within the communication circle with  as a vertex as shown in Figure \ref{f:fig7}. At the same time all the other vertices of LRH are also computed by the anchor. Then the anchor starts moving along the LRH and broadcasting beacons with its current position along with all vertices of the LRH at regular interval so that any sensor which receives a beacon, knows the LRH.
If the time interval between two beacons received by a sensor is more than , the sensor can easily identify the LRH on which the anchor is moving. At the same time all the neighbors of the localized sensor  marks at least two beacon points which help them to
compute two probable positions of themselves according to the scheme \cite{Lee2009}. We briefly discuss below how the localization scheme \cite{Lee2009} works.
\begin{figure}[h]
\psfrag{A}{}
\psfrag{C}{}
\psfrag{C'}{\hspace{-1mm}}
\psfrag{C''}{}
\psfrag{T}{}
\psfrag{T'}{}
\psfrag{Q}{\hspace{-1mm}}
\psfrag{Q'}{\hspace{-1mm}}
\psfrag{N}{}
\psfrag{N'}{\hspace{-1mm}}
    \centering
    \subfloat[ and  are the two possible positions]{\label{f:fig0}\includegraphics[width=0.35\textwidth]{pmfig0.eps}}
    ~~~~~~~
    \subfloat[Sensor chooses its position at ]{\label{f:fig7}\includegraphics[width=0.35\textwidth]{pmfig7.eps}}
\caption{Detection of correct position using hexagonal movement pattern of mobile anchor}
\end{figure}


Let  and  be two beacon points marked by a sensor as shown in  Figure \ref{f:fig0}.
Sensor lies on the one of the two intersections of the circular laminae with radius
 and  centering at the beacon points  and ,  where u is the beacon distance.
As the intersection is an area, the position of the sensor is considered as the intersection point  of  and .
So maximum localization error is equal to .
Similarly,  is another possible position. A third beacon point helps to choose the correct one among  and .
In our work we do not need the third beacon point to localize a sensor. Hexagonal movement strategy helps to
choose the correct one among those two. According to  Figure \ref{f:fig7}, the sensor chooses  as its
position instead of  depending on the received beacons other than those beacon points  and . Since sensor
knows the LRH, it finds that if  would have been its position than it should have received
beacons at those positions indicated by filled circles on the LRH as shown in  Figure \ref{f:fig7}.
Instead of that, those unfilled circles shown in  Figure \ref{f:fig7} on the LRH are the beacons received by .
So, sensor selects  as its position. This selection method will fail only if there exist a common set of beacons
for two different sensor positions, which is not possible.
Following theorems guarantee that all neighbors of any sensor can localize by one complete movement of the
mobile anchor along the LRH around that sensor.


\begin{theorem}
\label{the1:errbd}
Using the scheme \cite{Lee2009} of Lee et al., localization error remains less than   for a suitable beacon
distance  if , where  is the distance between two beacon points and  is the communication range.
\end{theorem}
\begin{proof}
There are two cases depending on the length of .
\begin{description}
\item[Case 1 ():]
Figure \ref{f:figCase1} illustrates the case.
According to  Figure \ref{f:fig1},  and  be the beacon points received by a sensor such that ,
where .
\begin{figure}[h]
\psfrag{C'}{}
\psfrag{C''}{}
\psfrag{T}{}
\psfrag{T'}{}
\psfrag{N}{}
\psfrag{N'}{}
\psfrag{M}{}
    \centering
    \subfloat[Showing the line segment ]{\label{f:fig1}\includegraphics[width=0.31\textwidth]{pmfig1.eps}}
    ~~
    \subfloat[]{\label{f:fig2}\includegraphics[width=0.27\textwidth]{pmfig3.eps}}
    ~~
    \subfloat[ when ]{\label{f:fig3}\includegraphics[width=0.31\textwidth]{pmfig2.eps}}
\caption{Illustration of case 1}\label{f:figCase1}
\end{figure}

From the figure, .

Now consider the angle  in Figure \ref{f:fig2}. Since  is parallel
with , so  implies . Then from triangle
, . Since the minimum value of  is ,
so,  implies , which is same as .
Therefore,  implies .
Hence,  implies .
\begin{figure}[h]
\psfrag{C'}{}
\psfrag{C''}{}
\psfrag{T}{}
\psfrag{T'}{}
\psfrag{N}{}
\psfrag{N'}{}
\centering
\includegraphics[width=0.5\textwidth]{pmfig4.eps}
\caption{Illustration of case 2}\label{f:fig4}
\end{figure}

From triangle  in Figure \ref{f:fig3}, we can say,  when , i.e.,
 when . Hence,

\item[Case 2 ():] From Figure \ref{f:fig4}, . This is maximum when
 is minimum, i.e., . Now,  if , i.e.,
 if . Again from Figure \ref{f:fig4}, . So,  if .
Hence,

\end{description}

From the above two inequalities \ref{eq:1} and \ref{eq:2}, for , the error bound of the scheme \cite{Lee2009}
is  if , i.e., error is less than  if .
\end{proof}

\begin{theorem}
\label{the2:localz}
If an anchor completes its movement along the LRH around a sensor , then all other sensors lying inside the circle of radius  centering at  can be localized with error less than  for suitable beacon distance ,
if  has been localized within  error.
\end{theorem}
\begin{proof}
In Figure \ref{f:fig5},  is the calculated position of a sensor and the anchor moves along LRH around .
Here the LRH is .
\begin{figure}[h]
\psfrag{A}{}
\psfrag{B}{}
\psfrag{C}{}
\psfrag{D}{}
\psfrag{C'}{}
\psfrag{C''}{}
\psfrag{E}{}
\psfrag{F}{}
\psfrag{G}{}
\psfrag{H}{}
\psfrag{J}{}
\psfrag{P'}{}
\psfrag{P''}{}
\psfrag{Q'}{}
\psfrag{Q}{}
    \centering
    \subfloat[Position of the sensor is within the ]{\label{f:fig5a}\includegraphics[width=0.4\textwidth]{pmfig5.eps}}
     ~~~~~~~~~
    \subfloat[Position of the sensor is within the region bounded by , ,  and the arc ]{\label{f:fig5b}\includegraphics[width=0.35\textwidth]{pmfig6.eps}}
\caption{Position of a sensor within the sector }\label{f:fig5}
\end{figure}
Due to localization error,  may lie anywhere within the smallest dotted circle in Figure \ref{f:fig5}
with radius  centering at .
We prove that if a sensor lies anywhere within the largest dotted circle with radius  centering at 
then it marks at least  two beacon points such that distance between them is at least , where  is the beacon
distance. Length of each side of the regular hexagon  is  and . The anchor starts
its movement from a vertex of the hexagon and broadcasts beacon maintaining beacon distance  for some integer ,
which ensures that the anchor broadcasts beacon at every vertex of the LRH along with some other points on the LRH.

We divide the largest dotted circle shown in Figure \ref{f:fig5} in six symmetric sectors by
joining each vertex of the hexagon with  and then extending the line up to the boundary of the largest circle.
So any sensor lies within the largest circle must lie in any one of the six symmetric sectors. If a sensor lies
on common boundary on any two sectors then it can be considered within any of the sector.
Without loss of generality, let a sensor lie in the sector bounded by the line segments ,  and
by the arc , where  is the mid-point of  on the arc. We further divide the sector in two parts i.e.,  and the region  bounded by , ,  and the arc , where  defines the triangle with vertices  and . There are two cases based on the position of a sensor belonging to  or region .
\begin{description}
  \item[Case 1.]  is the location of sensor inside  as shown in Figure \ref{f:fig5a}, where  is an equilateral triangle with side length . Then irrespective of the position of , , . Hence the communication circle of  intersects  and / or intersects  and /. So distance  between two beacon points is at least  i.e., . So  can be localized within  error using Theorem \ref{the1:errbd}.
  \item[Case 2.]  is the location of sensor inside the region . The distance  between the beacon points decreases, if position of  moves away from the center  of the LRH. As any point on the arc  is furthest from , hence, it is sufficient to show  when  lies on the arc . Due to the symmetric nature of the arcs  and , it is sufficient to show for positions of  on arc . Let position of  is at . Since , circle with radius  centering at  intersects  and  as shown in Figure \ref{f:fig5b}. Hence  marks two beacon points such that .
      One can easily verify that as position of  changes along the arc  towards , distance  between two beacon points decreases and  is least when  is at . So, it is sufficient to show that  when  is at .
      As shown in Figure \ref{f:fig5b}, communication circle of , i.e., the circle centering at , intersects  and  at  and  respectively i.e., . Let ,  be the beacon points considering the worst case such that .
      We can write  and . Let . Then from triangle  of Figure \ref{f:fig5b},
      . Solving this equation, we get . Now from  ,
      , i.e., . So,  if , i.e., , i.e., .
\end{description}
So, we can conclude that for , all the sensors which lie within the circle of radius  and centering at , can be localized with error less than .
\end{proof}


\begin{corollary}
\label{cor:cor}
If , all the sensors which lie within the circle of radius  centered at ,
can be localized with error less than  after mobile anchor completes its movement along the LRH around the sensor .
\end{corollary}
\begin{proof}
Follows from Theorem \ref{the1:errbd} and Theorem \ref{the2:localz}.
\end{proof}

\begin{theorem}
\label{thm3:nbrslclztn}
If a mobile anchor completes its movement along the LRH around a sensor which is localized within  error, then all its neighbors
are localized.
\end{theorem}
\begin{proof}
If  a sensor is localized within  error, its neighbors lie within the circle of radius  centering the sensor.
Hence the statement follows by Theorem \ref{the2:localz}.
\end{proof}

\section{Distributed Algorithm for Path Planning}
\label{sec:algomodel}
We assume sensors form a connected network. The number of sensors in the network is not an input of our algorithm.
Each sensor has unique id and knows the id of its one hop neighbors. Set of neighbors of a sensor  is denoted by .
We define {\it NLN-degree} as the number of non-localized neighbors of a sensor . Initially NLN-degree, where
 is the cardinality of the set . We assume at the beginning of localization, a sensor  is localized itself by
random movement of the mobile anchor such that the localization error is within , where  is the communication range. This can be done by choosing beacon points such that they are at least  distance apart, where  is the beacon distance. After localization, 
broadcasts its position.
Whenever the mobile anchor hears the position of , it starts moving along the sides of the LRH of . Computation of LRH is explained in section \ref{sec:tech1}. Hence each  localizes within the error bound  by Theorem \ref{thm3:nbrslclztn} and broadcasts the position to . After receiving position of a neighbor , sensor  updates NLN-degree by NLN-degree. During the process of completion of LRH, each sensor which receives a neighbor's position, keeps updating its NLN-degree. So, when all neighbors of sensor  are localized at the end of a LRH movement, each  has computed their NLN-degree. After completing LRH, the anchor moves  distance towards . Moving  distance towards  is required to ensure communication between the anchor and  because the positional error is bounded by  according to Theorem \ref{the2:localz}. Then anchor sends a message to  for its next destination of movement.
To do so,  sends a request to all  for NLN-degree along with their positions. Then  selects
a neighbor sensor  that achieves the value NLN-degree and sends position of  with NLN-degree to the anchor for next destination of movement. The anchor moves to the closest point of the communication circle of  and starts the LRH movement around  to continue localization. The mobile anchor maintains a STACK along with its operations PUSH, POP and variable TOP with usual dynamic stack data structure during its travel through the connected network.
Initially the STACK is empty. Before making LRH movement around a sensor , the anchor PUSH id  on the STACK. If NLN-degree then anchor POP  from the STACK. When the STACK becomes empty, the algorithm terminates, otherwise anchor revisits the communication circle of the sensor  (say), which is at the TOP of the STACK and then sends a message to  for its next destination of movement.
Mobile anchor decides its path according to the Algorithm \ref{alg:ALG}: \textsc{HexagonalLocalization}.
\begin{algorithm}[]
\caption{\textsc{HexagonalLocalization}}
\begin{algorithmic}[1]
                    \STATE Mobile anchor localizes a sensor by its random movement then PUSH id of the sensor into the STACK.
                    \STATE \label{lin2} Computes LRH centering at the sensor whose id  is at the TOP of the STACK and broadcasts beacons periodically with period  until the LRH movement completes.
                    \STATE \label{lin3} The anchor moves  distance towards  and sends a message to  for next destination of its movement.
                    \STATE Sensor  sends a message to all  for NLN-degree along with their positions.
                    \STATE On receiving the replies, sensor  selects
                        a neighbor sensor  that achieves the value NLN-degree and sends position of  with NLN-degree to the anchor for next destination of movement.
                    \STATE If NLN-degree then the anchor PUSH  into the STACK and moves to the closest point of the communication circle of  and executes step \ref{lin2}, otherwise POP from the STACK.
                    \STATE The algorithm terminates if STACK is empty, otherwise the anchor revisits the sensor whose id is at the TOP of the STACK and executes step \ref{lin3}.
\end{algorithmic}
\label{alg:ALG}
\end{algorithm}

\subsection{Correctness and complexity analysis}
\begin{theorem}
Algorithm \ref{alg:ALG} ensures localization of all sensors in a connected network.\label{th:correctness}
\end{theorem}
\begin{proof}
We prove correctness of the algorithm by  method of contradiction. Let us assume that one sensor , (say), is not localized but the algorithm terminates, i.e., the STACK becomes empty. Since all sensors are in a connected network, then there exist at least one sensor , (say), which is a neighbor of  and is localized. According to the algorithm,  localizes itself by marking beacon points on a LRH movement of the anchor around one of its neighbors, say . At this moment  is the TOP of the STACK. As  is not localized,  would send non zero NLN-degree and its position to . Since other sensors are localized, NLN-degree is the maximum among those received by . Hence  should send the position of  to the anchor for the next destination of movement according to algorithm \ref{alg:ALG}. Whenever anchor visits  and makes the LRH movement,  becomes localized. So  cannot be popped without pushing  which implies localization of . Hence  is in the STACK until  is localized. It contradicts our assumption that the STACK is empty but  is not localized. Hence proved.
\end{proof}
\begin{theorem}
The time complexity of Algorithm \ref{alg:ALG} is .\label{th:timecomplexity}
\end{theorem}
\begin{proof}
To analyze complexity of Algorithm \ref{alg:ALG}, we calculate maximum travel distance of the mobile anchor to localize all sensors in any connected graph  topology. Here  and  are the set of vertices corresponding to the sensors and the set of edges of  respectively. If degree of the graph decreases then the anchor needs to make more LRH movements. The line graph achieves lowest degree among connected graphs and hence the anchor attends maximum LRH movements to localize all sensors. In the worst case, for a line graph our algorithm matches with DFS visit of . In this case anchor has to make LRH movement around  sensors if it initiates its movement from one end of the graph. Total LRH movement is  since each LRH movement equals to perimeter  of LRH, where  is the communication range of the sensors. In addition to this, anchor has to move maximum  distance to reach other sensors and then to return to the initiator.  Hence both time complexity and complexity in terms of distance traveled by the anchor are same and equal to .
\end{proof}
\section{Path Planning for Rectangular Region}
\label{sec:tech2}
In this section we describe path planning of a mobile anchor, which covers a given rectangular region to ensure localization of all sensors deployed
on the region. Let  be a regular hexagon with side length  and center  as shown in  Figure \ref{f:fig8}.
If communication circle of a sensor touches the hexagon then we say that the sensor is within the coverage area of the hexagon.
Obviously the area inside the hexagon is a part of the coverage area.  As shown in Figure \ref{f:fig8},  is a regular hexagon with side  and center . The communication circles of the sensors located at the vertices of the larger hexagon touches the smaller hexagon at one vertex. Communication circle of any sensor which lies within or on the larger hexagon except at the vertices, intersects the smaller hexagon at at least two points.
\begin{figure}[h]
\psfrag{O}{}
\psfrag{A}{}
\psfrag{M}{\hspace{-1mm}}
\psfrag{M'}{\hspace{-1mm}}
\psfrag{B}{}
\psfrag{C}{}
\psfrag{D}{}
\psfrag{E}{}
\psfrag{F}{}
\psfrag{A'}{}
\psfrag{B'}{}
\psfrag{C'}{}
\psfrag{D'}{}
\psfrag{E'}{}
\psfrag{F'}{}
\psfrag{A''}{}
\psfrag{B''}{\hspace{-1mm}}
\psfrag{C''}{}
\psfrag{D''}{}
\psfrag{E''}{}
\psfrag{F''}{}
  \centering
    \subfloat[The hexagon  is the coverage area when mobile anchor moves along the hexagon  ]{\label{f:fig8}\includegraphics[width=0.35\textwidth]{pmfig8.eps}}
     ~~~~~~~~~
    \subfloat[The coverage area  ensures two beacon points on ]{\label{f:fig9}\includegraphics[width=0.4\textwidth]{pmfig9.eps}}
\caption{Showing coverage area when a mobile anchor moves along the regular hexagon }
\end{figure}
So  forms a coverage area of .
Let the anchor moves along  and broadcasts beacons periodically with time period . For localization with the scheme \cite{Lee2009}, each sensor should mark at least two beacon points on . To ensure marking of two beacon points, we reduce the coverage area by reducing the hexagon  to  with side  as shown in  Figure \ref{f:fig9}. Now, we have to find a suitable value of  such that all sensors located inside  can be localized.
As the vertices of  are the furthest points from , so if we find  in such a way that a sensor, which lies at any vertex of , marks at least two beacon points, then so does all other sensors, which lies on or inside  for that same .

Let  as shown in Figure \ref{f:fig9}. Then the sensor at  marks two beacon points at , , if .
Let . From the triangle  , , which implies .
Approximately  as . Hence, if mobile anchor moves along a regular hexagon of side , then all the sensors which lie within a regular
hexagon with side , mark at least two beacon points, where .

As shown in Figure \ref{f:fig10}, the mobile anchor moves along the blue solid lines starting from  to cover the rectangle .
The movement terminates at .  The directions of movement are shown by red dashed arrows. We now compute the path length of the movement. According to Figure \ref{f:fig10}, each hexagon with blue solid lines and side  covers one larger hexagon with black dotted lines whose width is equal to \\ .
\begin{figure}[h]
\begin{center}
\psfrag{Z}{}
\psfrag{R}{}
\psfrag{A}{}
\psfrag{G}{}
\psfrag{H}{}
\psfrag{1}{\hspace{1mm}}
\psfrag{2}{\hspace{1mm}}
\psfrag{3}{\hspace{1mm}}
\psfrag{4}{\hspace{1mm}}
\includegraphics[width=0.80\textwidth]{fig3p4.eps}
\caption{\label{f:fig10} Showing path planning to cover a rectangular region }
\end{center}
\end{figure}
If  is the width of the rectangle, then the number of hexagon in a row would be .
Two rows of hexagon cover an area with height equal to . Hence, if  is the hight of
the rectangle, then the number of rows of hexagon would be . Then total number of
hexagon with side  required to cover a  rectangle is .
Total path length required to reach all the hexagons in a row by moving from one hexagon to another hexagon is at most .
So, total distance traversed to connect the hexagons for all rows is equal to  no. of rows, i.e., .
Total path length required to reach all the rows by moving from one row to another row is at most  again.
In each pair of consecutive rows, an extra hexagon is needed i.e., an extra  movement is required in each row on an average.
So, total path traversed to cover  rectangle is equal to

.

\subsection{Comparison with existing schemes}

Theoretical comparison with existing schemes \cite{Huang2007,Koutsonikolas2007,Chia-Ho-Ou2013} in terms of path length of mobile anchor
is presented in this section. Let  be the length of the sides of a square region and  be the communication range of sensors and anchor.
To normalize all the expressions of path length given in \cite{Chia-Ho-Ou2013}, we have taken  for some integer .
We have denoted path length of mobile anchor in \cite{Chia-Ho-Ou2013} by . Path lengths of mobile anchor in different schemes in \cite{Koutsonikolas2007} are denoted by ,  and . Path lengths of mobile anchor in different schemes in \cite{Huang2007} are denoted by  and . We have denoted the path length of mobile anchor of our proposed scheme by . So, path lengths of different schemes according to \cite{Chia-Ho-Ou2013} are given by,

. This is same as .

.

.

+L, where .

Expression of  is modified in such a way that it can also cover the corner points of the rectangle, instead of the largest circle inscribed within the rectangle.

.\\
The path length for our proposed scheme is

.

In all the above expressions, the highest order term is . So, we have compared the coefficients of  as shown in Table \ref{table:1} to decide minimality of path length theoretically.

\begin{table}[h]
\center
\caption{Coefficients of higher order term  in the expressions of different schemes}
\label{table:1}
\begin{tabular}{c|c|c|c|c|c}
\hline
\hline
 &         &  &       &      &   \\
&&&&&\\
\hline
 &  &   &     &  & \\
\hline
\hline
\end{tabular}
\end{table}

In , , and , the coefficient of higher order term  is equal to .
For , the coefficient of  is . The coefficient of  in  is .
The coefficient of  in  is .
We show that coefficient of  in  is lesser than all the above mentioned coefficients for all .
Among the existing path planning schemes  does best in terms of path length to cover a circular region. But for a square region, larger circles are needed to cover the corners of the square, which increases path length and makes  a bad strategy for covering a square.
Since >> for all , so if we can show
> for all , then we can claim the minimality of path length of our scheme as the coefficient of the highest order term  of our scheme is least among all the existing ones.
One can easily verify that > for . Again, for all ,  but . Hence for any , there exists  such that path length of  is minimum.


\section{Simulation Results}
\label{sec:sim}
We have used MATLAB platform to study the performances of our proposed schemes. We have randomly generated a connected graph of
sensors in a 50 meter  50 meter square region. According to the section \ref{sec:tech1}, we have taken values of beacon
distance . Following Figure \ref{f:fig11} and Figure \ref{f:fig12} show the hexagonal movement path (red solid lines)
of a mobile anchor through the network during localization, where black bold lines are transition path between LRHs.
The blue crosses and red circles are the actual and calculated positions of the sensors respectively and pairs are joined by dotted lines.
The number of sensors , communication ranges , path length  and average positioning error are given in the caption of each figure.
All the values showing in the tables and figures are in meter. As the area remains fixed and connectivity is maintained by increasing
communication range of sensors and the anchor, path length does not decrease much with the number of sensors.
\begin{figure}[h]
\centering
\subfloat[, , , Average error=1.42]{\label{f:fig11a} \includegraphics[width=0.5\textwidth]{plot25.eps}}
\subfloat[, , , Average error=1.09]{\label{f:fig11b} \includegraphics[width=0.55\textwidth]{plot50.eps}}
\caption{Hexagonal movement pattern in a connected network}
\label{f:fig11}
\end{figure}
\begin{figure}[h]
\centering
\subfloat[, , , Average error=0.92]{\label{f:fig12a} \includegraphics[width=0.5\textwidth]{plot75.eps}}
\subfloat[, , , Average error=0.76]{\label{f:fig12b} \includegraphics[width=0.53\textwidth]{plot100.eps}}
\caption{Hexagonal movement pattern in a connected network}
\label{f:fig12}
\end{figure}
\begin{table}[]
\centering
\caption{Showing average error (in meter) for different communication range and beacon distance \label{table:5}}
\begin{tabular}
{p{4.5cm}|p{1.2cm}|p{1.2cm}|p{1.2cm}|p{1.2cm}|p{1.2cm} }
\hline
\hline
Beacon distance   &  &  &  &  &  \\
Communication range()  &&&&&\\
\hline
10 &1.47  & 1.10 & 0.61 & 0.42 & 0.33  \\
15 &2.35  & 1.36 & 1.01 & 0.73 & 0.54 \\
20 &2.74  & 1.90 & 1.28 & 0.93 & 0.62 \\
25 &4.35  & 2.38 & 1.53 & 1.22 & 0.94 \\
30 &5.89  & 3.14 & 2.12 & 1.64 & 1.13 \\
\hline
\hline
\end{tabular}
\end{table}

In Table \ref{table:5}, we have shown localization error for different communication ranges  and beacon distances .
According to Table \ref{table:5}, localization error decreases as beacon distance decreases for any fixed communication range.
Here, , where  according to Theorem \ref{the1:errbd} and Theorem  \ref{the2:localz}.
We have also shown path lengths of the mobile anchor for different number of sensors forming a connected network in a square region of fixed side length 50 meter in  Table \ref{table:2}. Here we keep the communication range fixed at 10 meter. Results show that path length does not increase much with number of sensors. It happens since degree of connectivity increases with the number of sensors in a fixed region.

Next we show the simulation results for the path planning scheme for a bounded square region.
We have simulated the proposed scheme for a bounded region by varying  from  to  and compared with
\cite{Chia-Ho-Ou2013}. We have fixed the value of communication range  meter. We have compared path lengths with the existing schemes
by varying  for a fixed square region with side length  meter. Results are shown in Table \ref{table:3}.
\begin{table}[]
\center
\caption{Average path length (in meter) of our scheme varying number of sensors}
\label{table:2}
\begin{tabular}{p{4cm}|p{1.5cm}|p{1.5cm}|p{1.5cm}|p{1.5cm}|p{1.5cm}}
\hline
\hline
   No. of sensors  & 100   & 150     & 200  & 250     & 300  \\
  &      &     &  & &      \\
\hline
 Average path length (meter) & 1490 & 1550 & 1640 & 1675 & 1754   \\
\hline
\hline
\end{tabular}
\end{table}
\begin{table}[h]
\center
\caption{Comparison of path length (in meter) of our scheme with existing schemes varying }
\label{table:3}
\begin{tabular}
{p{1cm}|p{1.4cm}|p{2.2cm}|p{1.6cm} |p{1.4cm}|p{1.4cm}|p{1.4cm}}
\hline
\hline
X() &   &   &   &   &  &  \\
\hline
 & 4987 & 5945 & 6019 & 5377 & 6228 & 5431 \\
 & 4292 & 5724 & 5827 & 5185 & 6013 & 5124 \\
 & 4271 & 5728 & 5735 & 5094 & 5911 & 5420 \\
\hline
\hline
\end{tabular}
\end{table}
\begin{table}[h]
\center
\caption{Percentage (\%) of improvement of our scheme in terms of path length compared to existing schemes for communication range }
\label{table:6}
\begin{tabular}{p{5.4cm}|p{2.2cm}|p{1.7cm}|p{1.2cm}|p{1.2cm}|p{1.2cm}}
\hline
\hline
Existing schemes &   &   &   &  &  \\
\hline
\% of improvement of our scheme &      &     &  & &      \\
compared to the existing schemes in terms of path length & 22.12\% & 22.93\% & 13.45\% & 25.35\% & 15.19\% \\
\hline
\hline
\end{tabular}
\end{table}
We have shown the percentages of improvement of our scheme over various schemes compared to path length for  in Table \ref{table:6}. We have taken averages for different  values to find average improvement on path length. Results show  to  improvement over different schemes. We have also shown values of average localization error of our scheme in Table \ref{table:4} varying communication ranges  and beacon distance . We have taken  in Table \ref{table:4}. Here also, localization error decreases as beacon distance decreases for any fixed communication range.

\begin{table}[]
\centering
\caption{Showing average error (in meter) for different communication range and beacon distance \label{table:4}}
\begin{tabular}
{p{4.5cm}|p{1.2cm}|p{1.2cm}|p{1.2cm}|p{1.2cm}|p{1.2cm} }
\hline
\hline
Beacon distance   &  &  &  &  &  \\
Communication range()  &&&&&\\
\hline
10 &1.59  &1.17  &0.78  &0.57  &0.46  \\
15 &2.75  &1.58  &1.22  &0.98  &0.85  \\
20 &2.93  &1.95  &1.61  &1.17  &0.89  \\
25 &4.28  &2.47  &1.61  &1.19  &0.99  \\
30 &4.94  &2.65  &1.72  &1.36  &1.11  \\
\hline
\hline
\end{tabular}
\end{table}

\section{Conclusion}
In this paper we have proposed path planning for a mobile anchor in a connected network and also in a bounded rectangular region.
Our movement strategy reduces the requirement of three beacon points for localization to two beacon points, which helps improving
path length for a rectangular region.
In a connected network, once a sensor is localized, our path planning is able to localize all its neighbors
with one hexagonal movement around the sensor. After completing one hexagonal movement, anchor decides its next
destination depending upon received information from the neighboring sensors and localize all sensors along
the way it moves. The novelty is that without knowing the boundary of the network,
our distributed algorithm localizes all sensors using connectivity without any range estimation.
We have also computed the length of the path traversed by the anchor for different number of sensors with good
localization accuracy based on simulation.
We have compared path length of our proposed movement strategy for rectangular region theoretically with existing
literature to show better performance. Simulation results show  to  improvement of our path
planning strategy over existing strategies in terms of path length of mobile anchor for rectangular region.
In future we will try to investigate path planning in presence of obstacles.
\label{sec:conclusion}

\section*{Acknowledgement}
The first author is thankful to the Council of Scientific and Industrial Research (CSIR), Govt. of India,
for financial support during this work.

\bibliographystyle{plain}
\bibliography{ref}
\end{document}
