\documentclass{llncs}

\usepackage{dsfont}
\usepackage{amsmath}
\usepackage{authblk}
\usepackage[utf8]{inputenc}
\usepackage[OT1]{fontenc}
\newcommand{\G}{\mathds{G}}
\newcommand{\Z}{\mathds{Z}}
\newcommand{\N}{\mathds{N}}
\newcommand{\set}[1]{\{#1\}}
\newcommand{\abs}[1]{\left|#1\right|}

\usepackage[printonlyused]{acronym}



\begin{document}

\title{Oblivious Lookup Tables}

\author{Stefan Rass\and Peter Schartner\and Markus Wamser}
\institute{ Institute of Applied Informatics, Universit\"{a}t Klagenfurt,
Universit\"{a}tsstrasse 65-67, 9020 Klagenfurt, Austria, \{stefan.rass,
peter.schartner\}@aau.at,\\
 Institute for Security in Information Technology, Technical University
of Munich, Arcisstraße 21, 80333 Munich, Germany, wamser@tum.de}


\maketitle              \abstract{We consider the following question: given a group-homomorphic
public-key encryption , a ciphertext  hiding a value  using
a key , and a "suitable" description of a function , can we evaluate
 without decrypting ? We call this an \emph{oblivious lookup
table} and show the existence of such a primitive. To this end, we describe a
concrete construction, discuss its security and relations to other
cryptographic primitives, and point out directions of future investigations
towards generalizations.}

\noindent\textbf{Keywords.} homomorphic encryption; private function
evaluation; secure function evaluation



\section{Introduction and Concept}
This short note is about the following setting: let  be a mapping
between finite sets. Assume that the sizes of  and  are sufficiently
small to permit a specification of  via a lookup-table. Let 
denote a group-homomorphic encryption of a message  under a key ,
where  can be symmetric or asymmetric. Let  be group-homomorphic in the
sense that , where
 denotes the respective group operations within the plain- and
ciphertext space.

In this setting, we consider the following question: given  and an
encrypted value , can we compute  without
decrypting ? We call any such implementation of  an \emph{oblivious
lookup table}, as it shall effectively hide the evaluation of , or
equivalently, evaluate  only on ciphertexts by virtue of conventional
homomorphic encryption.

Becoming more specifically, let  be a safe prime (i.e.,  is a
prime too), and let  denote the -order subgroup generated
by some element . We first describe the lookup technique in plain
form, and subsequently wrap the encryption around the necessary operations.

Let  be an enumeration of (distinct)
values to be looked up. To each such element , we associate a vector
. Notice that
 whenever  implies that the vectors  are all linearly independent, as they essentially form
the rows of a Vandermonde-matrix . Without loss of generality, let us
assume , say, by allowing multiple occurrences of the same
element in  in case that  is not injective. Under this convention, let
the (not necessarily pairwise distinct) elements of  be enumerated as
.

We will construct the value of  by a scalar product of 
with a vector-representation of the lookup table. That is, the lookup table
itself is a vector  with the property that  for all . To this end, let us choose an
arbitrary but invertible matrix  with columns . Define the lookup table as  for some (yet to be determined) vector . Now, let us look at the scalar product of
 with  to yield . This
results in a linear equation . Establishing this condition for all , we end up
observing that, to find , we need to solve the linear system
 for . The coefficient matrix  is invertible by
construction, and hence we can easily lookup values  by computing
, taking  multiplications and
additions.

Now, let us see if we can equivalently do all the necessary steps when the
pre-image is encrypted. For that matter, we take an element-wise commitment
to the  from above to represent . That is, the value  now
comes committed and encrypted as ,
, ,\ldots,
, so that the matrix of exponents
remains  with  and as such
invertible. Since the order of  is a prime, we can -- in a setup phase
where the exponents are known -- straightforwardly work out the values
 and the lookup table , which
is supplied in plain (not encrypted) form to the instance that seeks to
evaluate .

To evaluate , let the encrypted input value  be given as . Then, we can compute the lookup value  as


The last equality is instantly obtained by writing out the exponents for
 and rearranging terms properly when summing up.

A final remark is judicious here: the formula yields only a single value
based on an input vector. To properly implement the lookup to be
\emph{repeatable}, i.e., to model iterations like 
or generally functions , we need to look up all the elements of the
output vector via separate tables. So, the overall lookup table is no longer
a -dimensional vector, but an -matrix . The -th such lookup table  must
then be designed to return , whenever the input value  is
represented by a sequence  in the exponents. That is,
the mapping , acting on  being represented by encrypted values
, requires  lookups that successively
yield , each of which by \eqref{eqn:lookup}
requires  exponentiations and multiplications. So, the total cost of an
oblivious lookup comes to  exponentiations (subsuming multiplications
as the cheaper operation here).

Considering security, each lookup table is indeed available in plaintext, but
since it is independent of a particular input and the input and output values
remain encrypted at all times, knowledge of  cannot release any
information about the secret  being transferred into the secret result
. Probabilistic encryptions like ElGamal an offer the additional appeal
of enforced re-randomization of the resulting ciphertexts. That is, if a
distrusted third party does several lookups, it nevertheless cannot recognize
any results as being identical to previous ones.

\section{Related Work}

This work closely relates to \ac{PFE}, which provides a system where the
function-to-be-evaluated  \emph{and} the inputs are private and the
evaluator learns nothing about either aside from the (encrypted) results of
the evaluation of the function on the inputs. This can be realized using
\ac{SFE} over a universal circuit
(\cite{KolesnikovSchneider2008,Valiant1976}), to which  has to be
converted first. Another approach is to use a (non-universal) circuit
representation of  and employ a \ac{FHE} scheme
\cite{Gentry2009,Silverberg2013}. However, all mentioned approaches carry
complexities that are too high for practical applications. Conceptually
closest to our ideas seem to be \cite{KatzMalka2011} and
\cite{MohasselSadeghian2013}, both based on singly homomorphic encryption.
The former realises \ac{PFE} in a strict two-party setting with one party
providing the function and the other providing the inputs. Evaluation is done
through a common virtual machine. The latter is based on a framework that
splits the task into \ac{CTH} and \ac{PGE} which together enable \ac{PFE}
with linear complexity in all standard settings. However, both \ac{PFE}
protocols require an interactive setting while we are aiming for the
non-interactive setting. The security implications tied to our simple scheme
when being lifted to two-operand functions (if that it possible at all) are,
however, far from clear and probably intricate (cf. \cite{Boneh2011}) and
will be discussed along the research sketched in this abstract.

\section{Open Issues}
A yet open issue is a proper formalization of security for oblivious lookup
tables. Intuitively, the attacker should be unable to learn anything
meaningful from the lookup table as such, since this is nothing but a bunch
of ciphertexts and hence indistinguishable from self-made cryptograms,
provided that the encryption is semantically secure. However, a full-fledged
formal argument and definition of security is subject of future
considerations. Also, the idea does not obviously generalize to functions of
multiple inputs, which poses another interesting question for future
research.

\bibliographystyle{plain}\bibliography{cecc15}

\section*{Acronyms}
\begin{acronym}
\acro{CTH}{Circuit Topology Hiding}
\acro{FHE}{Fully Homomorphic Encryption}
\acro{PFE}{Private Function Evaluation}
\acro{PGE}{Private Gate Evaluation}
\acro{SFE}{Secure Function Evaluation}
\end{acronym}

\end{document}
