\documentclass[a4paper,11pt]{article}
\usepackage[margin=26mm]{geometry}
\usepackage{amsthm}
\usepackage{amsmath}
\usepackage{mathbbol}
\usepackage{amsfonts}
\usepackage{amssymb}
\usepackage{graphicx}
\usepackage{url,hyperref}
\usepackage{epsfig}
\usepackage{wrapfig}
\usepackage{color}
\usepackage{fancybox}
\usepackage{xcolor}

\usepackage{subfig}

\usepackage{algo} 
\usepackage{comment}

\usepackage[mathlines]{lineno}

\renewcommand\linenumberfont{\normalfont\tiny\sffamily\color{gray}}

\newcommand*\patchAmsMathEnvironmentForLineno[1]{\expandafter\let\csname old#1\expandafter\endcsname\csname #1\endcsname
  \expandafter\let\csname oldend#1\expandafter\endcsname\csname end#1\endcsname
  \renewenvironment{#1}{\linenomath\csname old#1\endcsname}{\csname oldend#1\endcsname\endlinenomath}}\newcommand*\patchBothAmsMathEnvironmentsForLineno[1]{\patchAmsMathEnvironmentForLineno{#1}\patchAmsMathEnvironmentForLineno{#1*}}\AtBeginDocument{\patchBothAmsMathEnvironmentsForLineno{equation}\patchBothAmsMathEnvironmentsForLineno{align}\patchBothAmsMathEnvironmentsForLineno{flalign}\patchBothAmsMathEnvironmentsForLineno{alignat}\patchBothAmsMathEnvironmentsForLineno{gather}\patchBothAmsMathEnvironmentsForLineno{multline}}


\newcommand{\KK}{\mathbb{K}}
\newcommand{\RR}{\mathbb{R}}
\renewcommand{\SS}{\mathbb{S}}

\newcommand{\KKnew}{\mathbb{K}_{\rm new}}
\newcommand{\calKnew}{\mathcal{K}_{\rm new}}
\newcommand{\Gnew}{G_{\rm new}}
\newcommand{\Tnew}{T_{\rm new}}

\newcommand{\N}{\mathbb{N}}
\newcommand{\Q}{\mathbb{Q}}
\newcommand{\Z}{\mathbb{Z}}
\renewcommand{\S}{\mathbb{S}}
\newcommand{\G}{\mathbb{G}}
\newcommand{\T}{\mathbb{T}}
\newcommand{\eps}{\epsilon}

\newcommand{\D}{\mathcal{D}}
\newcommand{\calK}{\mathcal{K}}

\newcommand{\ignore}[1]{}

\newcommand{\freespace}{\mathbb{F}}
\newcommand{\configspace}{\mathbb{X}}
\newcommand{\redsquares}{\mathbb{Y}}

\newcommand{\cycle}{\mathit{cycle}}

\def\find{\mbox{\sc Find}}
\def\findext{\mbox{\sc FindExt}}
\def\union{\mbox{\sc Union}}
\def\unionext{\mbox{\sc UnionExt}}
\def\makeset{\mbox{\sc MakeSet}}
\def\rank{\mathit{rank}}
\def\parent{\mathit{parent}}
\def\parity{\mathit{parity}}
\newcommand\CR{\mbox{\tt cr}_2}		  
\newcommand{\myparagraph}[1]{\textbf{#1.}}

\def\DEF#1{\textbf{\emph{#1}}}

\newtheorem{theorem}{Theorem}
\newtheorem{lemma}[theorem]{Lemma}
\newtheorem{proposition}[theorem]{Proposition}
\newtheorem{corollary}[theorem]{Corollary}
\newtheorem{definition}[theorem]{Definition}
\newtheorem{remark}[theorem]{Remark}
\newtheorem{note}[theorem]{Note}


\def\marrow{\marginpar[\hfill]{}}
\def\michael#1{\textcolor{red}{\textsc{Michael says: }{\marrow\sf #1}}}
\def\sergio#1{\textcolor{red}{\textsc{Sergio says: }{\marrow\sf #1}}}

\title{Semi-dynamic connectivity in the plane}
\author{
	Sergio Cabello\thanks{Department of Mathematics, IMFM, and
			Department of Mathematics, FMF, University of Ljubljana, Slovenia.
			\texttt{sergio.cabello@fmf.uni-lj.si}.
        	Supported by the Slovenian Research Agency, program P1-0297.}
\and
	Michael Kerber\thanks{Max-Planck-Institut f\"ur Informatik, Saarbr\"ucken, Germany.
                              \texttt{mkerber@mpi-inf.mpg.de}.
                              Supported by the Max Planck Center for Visual Computing and Communication.}
}
\date{\today}


\begin{document}
\maketitle

\begin{abstract}
	Motivated by a path planning problem we consider the following procedure.
	Assume that we have two points  and  in the plane and take .
	At each step we add to  a compact convex set that does not
	contain  nor .
	The procedure terminates when the sets in  separate  and .
	We show how to add one set to 
	in  amortized time plus the time needed 
	to find all sets of  intersecting the newly added set,
	where  is the cardinality of ,
	 is the number of sets in  intersecting the newly added set, and 
	 is the inverse of the Ackermann function.
\end{abstract}


\section{Introduction}

Consider the \emph{path planning problem} from robotics, 
also known as the \emph{piano mover's problem}~\cite{lavalle}~\cite[Ch.13]{dutchbook}:
Given an initial and a target configuration of a robot, the task
is to decide whether the robot can move from the initial to the target
configuration without colliding with itself or a surrounding object
(and to find such a transformation if it exists).
The problem is typically tackled by setting up a \DEF{configuration space} 
where every robot position is encoded as a single point.
Then  is partitioned into a \DEF{free space}  
of allowed configurations and its complement 
 denoting configurations
that collide with obstacles. The initial and final state are denoted by two points 
and  in , and the task is to decide whether  and  are in the same
path-connected component of .

The following approach to solve the path planning problem
is discussed by Wang, Chiang and Yap~\cite{wcy-soft}. 
Assume for simplicity
that the configuration space  is a unit cube in .
For any given subcube, which we call \DEF{box} from now, we can decide whether the box 
is entirely contained in , entirely contained in ,
or both contains points of  and . We color a box
green, red, or yellow, respectively, depending on the predicates outcome.
Now, starting with the entire , we build a quadtree structure
and keep subdividing yellow boxes
into  boxes of equal size until one of the following events occur:

\begin{itemize}
\item[(1)] Points  and  lie in green boxes and are connected by a path that lies entirely in green boxes. Such a path is a solution to the path planning problem.
See Figure~\ref{fig:problem_illustration}, left, for an illustration.
\item[(2)] Each path from  to  intersects some red square.
In this case, no collision-free path from  to  can exist, and
we say that the red boxes separate  and .
See Figure~\ref{fig:problem_illustration}, right, for an illustration.
\end{itemize}

The described subdivision strategy is also used for the task of segmentation of digital images; see~\cite{aizawa} and references therein.
In that situation, the approach would decide whether the pixels  and  belong to the same connected component of the image.

How quickly can we decide whether one of the two conditions is satisfied?
Condition (1) can be easily checked by union-find~\cite{tarjan}: just create a new element
for each new green box and make unions to keep together adjacent green boxes, 
always checking whether the boxes containing  and  fall into the same set. 
That means that the amortized complexity
of checking condition (1) is almost linear in the number of green boxes produced.
For condition (2), the case seems less clear~-- an alternative way of phrasing
the condition is to check whether the union of green and yellow boxes contains
 and  in the same connected component. The union-find approach cannot directly be applied
because yellow regions might turn into red and, therefore, the area covered by the boxes
may shrink.
In this paper, we discuss how to test the second condition in the planar case ().

\begin{figure}[thb]
\centering
	\includegraphics[width=6cm,page=1]{subdivision}
	\hspace{0.5cm}
	\includegraphics[width=6cm,page=2]{subdivision}
	\caption{Left: Configuration space with two (convex) holes. 
	When subdividing the marked yellow box according to the dashed lines,  and  become connected.
	Right: Configuration space with an annulus-shaped obstacle. When subdividing the marked yellow box, the union
	of red boxes separates  and , so no path can exist.}
	\label{fig:problem_illustration}
\end{figure}

We consider the following generalization of the problem.
We have two points  and  in the plane.
We get a set  of compact, convex sets in the plane
iteratively, adding the sets one by one. 
Each of the sets added to  is disjoint from  and .
In the motivating problem,
the red boxes would be the elements of .
At the end of the insertion of a new compact convex set into , 
we want to know whether  separates  and . That is,
we want to know whether each path from  to  has to intersect
some element of .
Thus, we want a semi-dynamic data structure to store  that
allows the insertion of new elements to  and decides whether
 separates  and .

We show that we can maintain  under insertions 
using a slightly more sophisticated union-find approach. 
The time to insert a new set  into  is the time
we need to find all the  elements of  intersecting ,
plus  union-find operations.
The idea is based on a classical parity argument saying that 
 and  are separated if and only if we
can find a closed curve contained in the union of the elements of  
that is crossed an odd number of times by the line segment  from  to . 
We maintain a union-find data structure for the sets of  and augment
it by storing additional information about the parity of crossings with the line segment .
Using this additional knowledge, we can quickly decide whether adding a new set
to  forms a cycle that separates  and , 
and the information can be maintained under unions
and path compressions without asymptotic overhead.

If in the motivating subdivision procedure we always subdivide a largest
yellow box, we obtain  time per yellow box and 
 amortized time per red box, where  is the number of red boxes and
 is the inverse of the Ackermann function.
Thus, we obtain the same asymptotic behavior for testing conditions (1) and (2).

\paragraph{Roadmap.}
In Section~\ref{sec:static} we discuss a criterion to decide when 
separates  and  in the static case.
In Section~\ref{sec:dynamic} we extend this to the semi-dynamic case,
where sets get added to .
In Section~\ref{sec:subdivision} we discuss the application to the motivating
subdivision procedure.

Our aim is to provide a self-contained exposition.
Some of the arguments are an adaptation of Cabello and Giannopoulos~\cite{cg-15} 
to this simpler setting,
others can be shorten substantially using machinery from Algebraic Topology.

 
\section{Static connectivity}
\label{sec:static}

Let  denote a finite family of compact convex sets in the plane,
and let  denote their union. 
We use the notation .
Let  and  be points in .

The set  \DEF{separates}
 and  if they are in different path-connected components of .
Equivalently,  separates  and  if 
each path in the plane from  to  intersects .
We also say that  separates  and .

In the next subsection we discuss a criterion to decide when  separates  and .
The criterion is based on considering all polygonal paths contained in , and thus is
computationally unfeasible.
In Subsection~\ref{sec:intersection} we discuss how this criterion can be checked 
in the intersection graph of , and thus obtain a discrete version suitable
for computations.

We will consistently use Greek letters  only for (polygonal) curves.

\subsection{Topological criterion for separation} 

A polygonal curve  is \DEF{generic} (with respect to  and ) if  does not contain
 nor  and the line segment from  to  does neither contain an endpoint
of  nor a self-intersection of . 
We will assume in our discussion that all the polygonal curves are generic.
We can enforce this assumption making a rotation, so that  is horizontal,
and replacing the point  by , for an infinitesimal .
We always use the same perturbed point .
Since  is finite,
separation of  and  with  is equivalent to separation of  and  with .
The computations can then be made using simulation of simplicity~\cite{sos}.

We fix  as the line segment joining  and .
The \DEF{crossing number} of  with a polygonal curve 
is the number of intersections of  and .
We denote by  the modulo  value of the crossing number of  and .
Thus,  if and only if the crossing number is odd.
For the whole paper, \emph{any arithmetic involving  is done modulo 2.}

It is important to use always the same perturbed point . Then, if a polygonal curve  is 
the concatenation of  and , we have
. If we would use different perturbed points
and the common endpoint of  and  lies in the line segment ,
then the inequality does not necessarily hold.

A polygonal curve  is \DEF{closed} if its endpoints coincide.
It is \DEF{simple} if it does not have any self-intersections, except for the common
endpoint in the case of closed polygonal paths. 

Note that in the following lemma we do not require simple curves.


\begin{lemma}
\label{lem:parity_folklore}
	The set  separates  and  if and only if
	there exists a closed polygonal curve  contained in  
	such that .
\end{lemma}
\begin{proof}
	We use the following classical argument, 
	which sometimes is an intermediary step towards a proof of the Jordan's curve theorem:
	A simple closed polygonal curve  separates  and  if and only if 
	 and  have an odd crossing number. 
	See, for example, Mohar and Thomassen~\cite[Section 2.1]{mt-01} for a formal proof. 

	Assume that  contains a closed polygonal curve  
	such that  and  have an odd crossing number. 
	The curve  may have self-intersections.
	If  is not simple, we can split it at self-intersections arbitrarily to obtain
	simple, closed polygonal paths  that have, all together,
	the same image as . 
	Since we have ,
	at least one of the curves  has .
	Such a curve  separates the endpoints of , and thus separates  and .
	It follows that there is no path in  from  to .
	Since , there is no path in 
	from  to .

	Assume that there is no path in  from  to . 
	Consider the path-connected component  of  that contains . 
	Since  is in a different cell of 
	and  is a finite collection of compact, convex bodies, 
	there exists a simple closed curve  contained in the boundary of  that separates  and .
	We can make shortcuts in  to obtain a simple closed polygonal curve  contained in 
	that separates  and . (This can be shown formally using the convexity of
	the elements of  and the compactness of .)
	The resulting simple polygonal path  separates  and , and thus
	the crossing number of  and  is odd.
\end{proof}

\begin{lemma}
\label{lem:same}
	Let  and  be two compact convex sets of .
	For any two generic polygonal curves  and  contained in  
	with the same endpoints, we have .
\end{lemma}
\begin{proof}
	First note that  does not separate  and . This can be 
	seen as follows. Let  be the set of directions.
	Consider the set of directions of the vectors ,
	for all . Since  is convex and , this directions
	cover less than half of . A similar statement holds for .
	It follows that there exists some ray from  to infinity
	in . Similarly, there exists a ray from  to infinity
	in . Those two rays and an extra path far enough
	can be combined to obtain a path from  to  in . Thus,
	 does not separate  and . 

	Since  does not separate  and , 
	Lemma~\ref{lem:parity_folklore} implies that any closed path  contained
	in  has .
	The concatenation of  and the reverse of  is a closed path contained
	in  and therefore .
\end{proof}

\subsection{Criterion on the Intersection Graph} 
\label{sec:intersection}

Consider the \DEF{intersection graph} of  and denote it by .
Each element  is a node of ; we will denote the node by  to 
match standard graph theory notation.
There is an edge  in  if and only if  and  intersect. 
The graph  is an abstract graph. 
Next we provide a geometric representation.

For each node  of  choose a point  in . 
For each edge  of , let  be a polygonal path from  to 
contained in the union . Since  and  are convex and intersect, 
we can always choose  with at most  edges.
The pair 
 
is a drawing of .
(It is not necessarily an embedding because drawings of edges may cross, for example when four 
axis-parallel squares have disjoint interiors but share a vertex.)
For each walk  in , let  be the polygonal path obtained
by concatenating . 
If  is a closed walk, then  is a closed polygonal curve.

\begin{lemma}
\label{lem:nerve}
	The set  separates  and  if and only if
	there exists a closed walk  in  such that .
\end{lemma}
\begin{proof}
	Assume that  separates  and .
	Because of Lemma~\ref{lem:parity_folklore}, 
	there is some polygonal curve  contained in  such that . 
	We break the path  into pieces such that each piece is contained
	in the union of  sets from .
	Let  be the resulting pieces, each of them a polygonal curve. 
	For each piece , let  and  be the endpoints
	of , and let  and  be the elements of  that contain  and ,
	respectively, so that  is contained in .
	Note that  is an edge of . 
	Let  be the closed walk with edges .
	
	We claim that . 
	To see this, consider for each piece  the polygonal curve 
	 from  to  obtained by concatenating 
	the line segment from  to , 
	followed by , 
	and followed by the line segment from  to . 
	See Figure~\ref{fig:notation} for an example.
	For each piece , the polygonal curves  and  have the same
	endpoints and are contained in the union .
	Because of Lemma~\ref{lem:same}, we have
	.
	It follows that, if we define  as the concatenation of ,
	we have . 
	Moreover, 
	because  is essentially  with some spokes connecting  to ,
	where the number of crossings evens out.
	We conclude that .
	This finishes one direction of the proof.
	
	\begin{figure}[thb]
	\centering
		\includegraphics[page=1,scale=.9]{notation}
		\caption{Notation in the proof of Lemma~\ref{lem:nerve}.}
		\label{fig:notation}
	\end{figure}

	For the other direction, assume that  has a closed walk  such that the crossing
	number of  and  is odd. Since the closed polygonal path
	 is contained in  by
	construction, Lemma~\ref{lem:parity_folklore} implies that  separates  and .
	This proves the other direction.
\end{proof}

We extend Lemma~\ref{lem:nerve} 
to a necessary and sufficient condition for  and  being disconnected
that involves only a few cycles of . 
Let  be any maximal spanning forest of , that is,  contains a spanning
tree of each connected component of . 
For each edge  of , let  be the unique cycle
in , and let  be the curve .
That is,  is the polygonal curve describing  in the drawing.

\begin{lemma}
\label{lem:homology_condition}
	Let  be a maximal spanning forest of .
	The set  separates  and  if and only if
	there exists some edge  such that .
\end{lemma}
\begin{proof}
	The essential idea is to use the so-called cycle space of a graph and the fact that
	 is a basis. We next provide the details
	using no background.
	
	Since we can treat each connected component of  (and thus ) independently,
	we will just assume that  has one connected component. This means that  is a spanning
	tree of .
	
	Fix any node  and take the point  as a basepoint. 
	For each node , let  be the simple walk in  from  to .
	For each edge  of  we define a closed polygonal curve  
	as the concatenation of , , and the reverse of .
	Note that  is a closed polygonal path through .
	
	When ,  is  with a spoke following ,
	where  is the last common node of  and .
	This implies that 
	
	When , then  walks twice the same polygonal curve,
	and therefore 
	
	
	Assume that the points  and  lie in different path-components of .
	Because of Lemma~\ref{lem:nerve}, there exists some closed walk  in  
	with .
	Let  be the sequence of edges in , where . 
	Using arithmetic modulo  we have  
	
	where in the third equality we have used that each node of  
	is the endpoint of two consecutive edges of , which implies that the new terms
	cancel out.
	This means that, for some edge  of , we have
	. This edge  cannot be in 
	because of \eqref{eq:T}.
	Therefore we have some edge  in , where ,
	with . Because of \eqref{eq:noT} we have
	.
	This finishes the proof of one direction of the statement.
	
	For the other direction, assume that there exists some edge 
	such that . Taking  and using
	that  by definition,
	this means that  is a closed walk in  with .
	It follows from Lemma~\ref{lem:nerve} that  separates  and .	
\end{proof}



\section{Semi-dynamic connectivity}
\label{sec:dynamic}

In this section we discuss the separation of  and  
under the addition of new sets to .
We first describe a standard union-find data structure because we will build on it.
Then we describe the setting and the notation we will use.
It follows a description of the extension of the union-find data structure
for our setting. Finally, we describe the data structure, its maintenance,
and its correctness.


\subsection{Preliminaries: Union-find}

Here we review a standard union-find data structure and some of its properties.
See~\cite[Chapter 21]{cormen}, \cite[Chapter 5]{dpv} or~\cite{e-14} for a comprehensive exposition.

A \DEF{union-find data structure} represents a disjoint set system supporting
the operations \makeset\ (create a new disjoint set with a single element), \union\
(merge two sets), and \find\ (return a representative of a given set). 
We can test whether two elements belong two the same set by testing whether the output
of \find\ for those two elements is the same.
A common realization
is to represent each disjoint set by a rooted tree in which each node holds one element of the set.
The root of the tree holds the representative of the set. Each node has a pointer to its parent,
while the root points to itself.
Then \find\ simply follows the parent pointer until it finds the root of the tree.
The union operation merges two trees by making the root of one subtree a child of the root of the other.
Thus, given two elements, we first locate the roots of their corresponding trees calling \find,
and then we proceed with the union.

Two optimizations are commonly used to obtain an efficient realization. 
\emph{Union-by-rank} determines which root gets merged in a union operation: 
each root has a rank associated to it, in an union we simply make the root of lower rank a child
of the root with larger rank, and we increase the rank of the root if both roots had the same rank. 
\emph{Path compression} makes all nodes found on a search path 
from a node to its root direct children of the root. 
For later reference and modification, 
we include pseudocode in Figure~\ref{fig:code1}.
Combining these two optimizations, each operation has an amortized time
complexity of , where  is the number of elements in the set system and
 is the extremely slow growing inverse Ackermann function.
See references~\cite[Chapter 21]{cormen}, \cite{e-14} or~\cite{ss-05} 
for an analysis of the time complexity.

\begin{figure}[htb]
	\hfill
	\ovalbox{
	\parbox{6.6cm}
	{\begin{algorithm}{\find}{}
		\qif  \qthen\\
			
		\qfi\\
		\qreturn 
	\end{algorithm}}}
	\hfill
	\ovalbox{
	\parbox{6.6cm}
	{\begin{algorithm}{\union}{}
		\\
		\\
		\qif  \qthen\\
			
		\qelse (*  *)\\
			\\
			\qif  \qthen\\
				
			\qfi
		\qfi
	\end{algorithm}}}
	\hfill
	\caption{The main two operations in the union-find data structure. 
			 and  are nodes of the tree.}
	\label{fig:code1}
\end{figure}


\subsection{Setting}
\label{sec:setting}
Let  and  be two points in the plane.
We have a finite family of convex sets , all of them disjoint from  and .
Following the previous notation, we denote by  the union of the sets in , 
and by  the intersection graph of .

Consider the addition of a new compact convex set  to . 
We use  for the resulting set,  for the union of its sets,
and  for the intersection graph of .
 
The analysis of our data structure is based on a maximal spanning forest 
of the intersection graph of the convex sets.
The definition of such spanning forest is iterative, as follows.
Let  be an enumeration of the edges incident to  in .
That is,  are the sets of  intersecting the new set .
We consider adding the edges  to  one by one.
We thus define  as the union of  and a new vertex  for .
For each index , we define the graph .
Note that . The intermediate graphs  are not
intersection graphs of  or , but something in between.

If at the time of adding  the vertices  and  are already connected 
in the graph , then we call  a \DEF{cycle edge}.
Otherwise,  merges two connected components of 
and we call it a \DEF{merge edge}.
Note that whether an edge is a cycle edge or a merge edge depends on the order used in the addition of edges.

Let  be the maximal spanning forest of .
We define  as the union of  and a new vertex  for . 
For each index  we define

It is easy to see by induction that, for each index ,
 is a maximal spanning forest of .
We define  as . Thus  is a maximal spanning
forest of .

As it was done in Section~\ref{sec:intersection},
for each  we choose a point  in  and for
each edge  we choose a polygonal curve .
These choices are made in the first appearance of the node or edge,
and remain invariant from there onwards.
 

\subsection{Augmented union-find}
\label{sec:extended}

We maintain a union-find data structure for the connected components 
of the graphs .
Recall that  is a maximal spanning forest of .
For each node  of , we store a \DEF{parity bit}, denoted as , 
with the following property:
\begin{itemize}
	\item If  is the root of a union-find tree, then .
	\item If  has parent  in a union-find tree, 
		then . That is,
		we look at the parity of the crossing number of  with
		the polygonal curve from  to  defined by the drawing of .
\end{itemize}
For the rest of the paper, \emph{any arithmetic involving parity bits is done modulo 2.}

We next argue that the correct parity bits can be maintained in the same complexity as the union-find operations, assuming that only certain unions are made.
That is clear for \makeset\ by giving the new node parity . 

Consider the \find\ operation, which changes parent pointers due to path compression. 
Note that the graphs  and  do not change, but the union-find data structure does.
Let  be nodes such that, in the union-find data structure,
 is parent of  and  is parent of .
Note that 

Therefore, when we update ,
we just have to set 
to restore  to its correct value.

We can now easily realize the augmented path compression.
We define an extended function  that, for 
all nodes  from  to the root  of the tree containing ,
sets  and updates the value  accordingly. 
Pseudocode is given in Figure~\ref{fig:code2}.
It easily follows by induction that \findext\ correctly maintains 
the parity bit of all elements.


\begin{figure}[htb]
	\centering
	\ovalbox{
	\parbox{9cm}
	{\vspace{.4cm}\begin{algorithm}{\findext}{}
		\qif  \qthen\\
			\\
			\\
			
		\qfi\\
		\qreturn 
	\end{algorithm}}}
	\caption{Extended find operation for an element .}
	\label{fig:code2}
\end{figure}

Finally, we discuss the extension \union\ to \unionext. 
Its arguments are two nodes  and  such that
 is a merge edge and the union-find data structure stores the
connectivity of . Since  is a merge edge,
we have .
This means that the sets  and  intersect but 
 and  were in different connected components of .
Like before, we first find the roots  and  of their trees 
using . After this it holds that
, and similarly
.

The walk 
can be split into , , and .
Thus we have

The last values are either available through  
or computable in constant time. 
If, for example,  gets  as its parent, then
we have . The other case
is similar. We provide the resulting pseudocode in Figure~\ref{fig:code3}.

\begin{figure}[htb]
	\centering
	\ovalbox{
	\parbox{8cm}
	{\vspace{.4cm}\begin{algorithm}{\unionext}{}
		\\
		\\
		\\
		\qif  \qthen\\
			\\
			
		\qelse (*  *)\\
			\\
			\\
			\qif  \qthen\\
				
			\qfi
		\qfi
	\end{algorithm}}}
	\caption{Extended union for two nodes  and .}
	\label{fig:code3}
\end{figure}

The properties of union-find imply that each of the extended operations,
\unionext\ and \findext, has an amortized
complexity of , where  is the cardinality of .

\subsection{Semi-dynamic data structure}

We now describe the data structure to maintain .
The data structure supports one operation: add a new
compact convex set  to  and then 
report whether  separates  and .
We use the notation from Sections~\ref{sec:setting} and~\ref{sec:extended}.

The data structure has the following elements:
\begin{itemize}
	\item an augmented union-find data structure as described in Section~\ref{sec:extended};
	\item for each element  of , we store the point ;
	\item a semi-dynamic data structure  that can find, for the new , all the 
		objects of  that intersect .
\end{itemize}
The intersection graph  and the maximal spanning forest  are not kept.
They are used only for the analysis. 

We next describe how to insert .
We use the data structure  to find the sets  of 
that intersect . We then insert  in the data structure  to obtain 
. We choose a point  in 
and create a new node  in the extended union-find data structure.

We then iterate over the edges . 
We first decide whether the considered edge  is a merge edge or a cycle edge 
by checking whether 
 and 
 return the same representative.
If  is a merge edge, we just call  and continue with the next step
of the filtration.

Otherwise,  is a cycle edge, and we proceed as follows. 
We want to check whether  is  or .
For this, we use that  and  have already the same parent because of the calls 
 and .
If we denote such a common parent by , then

If , then we conclude 
that  separates  and  and we finish the algorithm.
If , we proceed to the next edge .
Pseudocode for the insertion of  is given in Figure~\ref{fig:code4}.
This finishes the description of the algorithm.

\begin{figure}[htb]
	\centering
	\ovalbox{
	\parbox{10.5cm}
	{\vspace{.4cm}\begin{algorithm}{Adding  to }{}
		\\
		\\
		\qfor  intersecting  \qdo\\
			\\
			\\
			\qif  \qthen\\
				
			\qelse (*  a cycle edge *)\\
				\\
				\qif  \qthen\\
					\qreturn `` separates  and !!"
				\qfi
			\qfi
		\qrof
	\end{algorithm}}}
	\caption{Procedure for the addition of .}
	\label{fig:code4}
\end{figure}

It follows from the invariants of the extended union-find discussed in
Section~\ref{sec:extended}, that we are correctly computing the value
. 
If , then Lemma~\ref{lem:homology_condition}
implies that  separates  and . From that point on,
we only need to remember that  and  are separated.

If , then  will
remain  for all future maximal spanning forests .
This is so because the maximal spanning forest we maintain is monotone increasing:
we only add vertices and edges, but never remove anything.
Thus, we never need to check  again later.
In particular, if  did not separate  and  and 
we have  for all ,
then

Since  is a maximal spanning forest of ,
Lemma~\ref{lem:homology_condition} implies that  
does not separate  and . 

For each edge  we make 2 calls to \findext, at most one call to \unionext,
and additional  work. This means that for each edge we spend
 amortized time, where  is the cardinality of .
We also need the time needed to find the elements of  intersecting the new
element . We conclude.

\begin{theorem}
\label{thm:main}
	Let  and  be two points in the plane.
	There is a semi-dynamic data structure to maintain a family  of  compact convex
	sets in the plane under insertions to decide whether  separates  can .
	The insertion of a new set  in  that intersects  sets of  
	takes  amortized time, plus the time needed to find
	the  elements of  intersecting .
\end{theorem}

Of course, once  and  are separated by , the insertion of each new
set can be carried out in constant time, since we only need to remember that
 separates  and .


\section{Application to dynamic connectivity under subdivision}
\label{sec:subdivision}

We consider now the motivating application discussed in the Introduction for .

We have two points  and  inside the unit square .
Initially, the box  is colored yellow.
In each iteration, we take a \emph{largest} yellow box, subdivide it into
4 subboxes, and color each of them as red, yellow, or green 
depending on the outcome of some oracle.
The boxes containing  or  are always colored yellow or green.
We want to know at which point the red boxes separate  and , meaning
that each path from  to  contained in the unit square 
intersects some red box. 

Boxes are assumed to contain their boundary, 
so that any two boxes intersect if their boundaries intersect, 
possibly only at a common vertex.

For our arguments it is convenient to surround  with 8 red boxes of the same size
as . This reduces the problem to finding certain
curves within the red region. Without those additional squares, we should also consider 
boundary-to-boundary curves.

We maintain through the algorithm the intersection graph  of the yellow and red boxes.
This intersection graph  has one node for each box that is yellow or red,
and an edge between two nodes whenever the corresponding boxes intersect.
The graph  is stored using an adjacency list representation~\cite[Chapter 22]{cormen}.
The adjacency list of each vertex is stored as a doubly linked list.
Moreover, for the appearance of a node  in the adjacency list of ,
we keep a pointer to the appearance of  in the adjacency list of .
With this, we can perform the deletion of a node  in time proportional to its
degree.

When we want to subdivide a yellow box  represented by a node , 
we can locate its set of neighbors  in the graph , delete  from
the graph, subdivide  into four boxes, create the at most four new nodes
representing the yellow and red boxes arising from the subdivision of ,
check for intersection each of them against each of the nodes in , 
and update the graph  accordingly. 
All this takes time  time.

If we always subdivide a largest yellow box, there are at most  
other boxes intersecting it. This means that we can update the 
intersection graph  of yellow and red boxes in  time.
For choosing always a largest yellow box, we can use for example a queue
for the yellow boxes.
Thus, we spend  time per subdivided yellow box and,
for each red box, we get its neighboring red boxes in  time.
Using Theorem~\ref{thm:main} for the red boxes, and a normal union-find
for the green boxes, as discussed in the Introduction,
we obtain the following result.

\begin{theorem}
	Consider the subdivision procedure described in the Introduction where
	we always subdivide a largest yellow box.
	We can perform the subdivision until condition (1) or (2) occurs in  time,
	where  is the number of subdivisions performed.	
\end{theorem}

Of course we can also perform the first  steps of the subdivision procedure
in  time, and correctly report that neither condition (1) nor (2) hold.


\paragraph{Acknowledgments} We thank Chee Yap for posing to us the problem about connectivity under subdivisions.

\begin{thebibliography}{10}

\bibitem{aizawa}
K.~Aizawa, S.~Tanaka, K.~Motomura, and R.~Kadowaki.
\newblock Algorithms for connected component labeling based on quadtrees.
\newblock {\em International Journal of Imaging Systems and Technology}
  19(2):158--166, June 2009, \href{http://dx.doi.org/10.1002/ima.v19:2}{doi:10.1002/ima.v19:2}, \url{http://dx.doi.org/10.1002/ima.v19:2}.

\bibitem{dutchbook}
M.~de~Berg, M.~van Kreveld, M.~Overmars, and O.~Schwarzkopf.
\newblock {\em Computational Geometry: Algorithms and Applications}.
\newblock Springer, 2nd edition, 2000.

\bibitem{cg-15}
S.~Cabello and P.~Giannopoulos.
\newblock The complexity of separating points in the plane.
\newblock {\em Algorithmica}, to appear,
  \href{http://dx.doi.org/10.1007/s00453-014-9965-6}{doi:10.1007/s00453-014-9965-6}.

\bibitem{cormen}
T.~H. Cormen, C.~E. Leiverson, R.~L. Rivest, and C.~Stein.
\newblock {\em Introduction to Algorithms}.
\newblock MIT Press, 3rd edition, 2009.

\bibitem{dpv}
S.~Dasgupta, C.~H. Papadimitriou, and U.~V. Vazirani.
\newblock {\em Algorithms}.
\newblock McGraw-Hill, 2008.

\bibitem{sos}
H.~Edelsbrunner and E.~P. M{\"{u}}cke.
\newblock Simulation of simplicity: a technique to cope with degenerate cases
  in geometric algorithms.
\newblock {\em {ACM} Transactions on Graphics} 9(1):66--104, 1990,
  \href{http://dx.doi.org/10.1145/77635.77639}{doi:10.1145/77635.77639}.

\bibitem{e-14}
J.~Erickson.
\newblock Algorithms notes: Maintaining disjoint sets (``union-find"), 2015.
\newblock Lecture nodes available at
  \url{http://web.engr.illinois.edu/~jeffe/teaching/algorithms/}.

\bibitem{lavalle}
S.~M. Lavalle.
\newblock {\em Planning Algorithms}.
\newblock Cambridge University Press, 2006.

\bibitem{mt-01}
B.~Mohar and C.~Thomassen.
\newblock {\em Graphs on Surfaces}.
\newblock Johns Hopkins University Press, 2001.

\bibitem{ss-05}
R.~Seidel and M.~Sharir.
\newblock Top-down analysis of path compression.
\newblock {\em {SIAM} Journal of Computing} 34(3):515--525, 2005,
  \href{http://dx.doi.org/10.1137/S0097539703439088}{doi:10.1137/S0097539703439088}.

\bibitem{tarjan}
R.~E. Tarjan.
\newblock Efficiency of a good but not linear set union algorithm.
\newblock {\em Journal of the ACM} 22(2):215--225, 1975,
  \href{http://dx.doi.org/10.1145/321879.321884}{doi:10.1145/321879.321884}.

\bibitem{wcy-soft}
C.~Wang, Y.-J. Chiang, and C.~Yap.
\newblock On soft predicates in subdivision motion planning.
\newblock {\em Proceedings of the Twenty-ninth Annual Symposium on
  Computational Geometry}, pp.~349--358. ACM, SoCG '13, 2013,
  \href{http://dx.doi.org/10.1145/2462356.2462386}{doi:10.1145/2462356.2462386}.

\end{thebibliography}

\end{document}
