
\subsection{Background}
The continuous-time replica exchange Langevin diffusion (reLD) $\{ \bbeta_t\}_{t\ge 0 }:=\left\{\begin{pmatrix}{}
\bbeta_t^{(1)}\\
\bbeta_t^{(2)}
\end{pmatrix}\right\}_{t\ge 0 }$ is a Markov process compounded with a Poisson jump process. In particular, the Markov process follows the stochastic differential equations
\begin{equation}
\label{sde_2}
\begin{split}
    d \bbeta^{(1)}_t &= - \nabla U(\bbeta_t^{(1)}) dt+\sqrt{2\tau_1} d\bW_t^{(1)}\\
    d \bbeta^{(2)}_t &= - \nabla U(\bbeta_t^{(2)}) dt+\sqrt{2\tau_2} d\bW_t^{(2)},\\
\end{split}
\end{equation}
where $\bbeta_t^{(1)}, \bbeta_t^{(2)}$ are the particles (parameters) at time $t$ in $\mathbb{R}^d$, $\bW^{(1)}, \bW^{(2)}\in\mathbb{R}^d$ are two independent Brownian motions, $U:\mathbb{R}^d\rightarrow \mathbb{R}$ is the energy function, $\tau_1<\tau_2$ are the temperatures. The jumps originate from the swaps of particles $\bbeta_t^{(1)}$ and $\bbeta_t^{(2)}$ and follow a Poisson process where the jump rate is specified as the Metropolis form $r (1\wedge S(\bbeta_t^{(1)}, \bbeta_t^{(2)}))dt$. Here $r\geq 0$ is a constant, and $S$ follows
\begin{equation*}
\label{swap_1}
\begin{split}
    S(\bbeta_t^{(1)}, \bbeta_t^{(2)})=e^{ \left(\frac{1}{\tau_1}-\frac{1}{\tau_2}\right)\left(U(\bbeta_t^{(1)})-U(\bbeta_t^{(2)})\right)}.\\
\end{split}
\end{equation*}
Under such a swapping rate, the probability $\nu_t$ associated with reLD at time $t$ is known to converge to the invariant measure (Gibbs distribution) with density
\begin{equation*}
\label{pt_density}
\begin{split}
    \pi(\bbeta^{(1)}, \bbeta^{(2)})\propto e^{-\frac{U(\bbeta^{(1)})}{\tau_1}-\frac{U(\bbeta^{(2)})}{\tau_2}}.
\end{split}
\end{equation*}

In practice, obtaining the exact energy and gradient for reLD (\ref{sde_2}) in a large dataset is quite expensive. We consider the replica exchange stochastic gradient Langevin dynamics (reSGLD), which generates iterates $\{\widetilde \bbeta^{\eta}(k)\}_{k\ge 1}$ as follows
\begin{equation}
\label{reSGLD}
       \begin{split}
           \widetilde \bbeta^{\eta(1)}(k+1) &= \widetilde \bbeta^{\eta(1)}(k)- \eta \nabla \widetilde U (\widetilde \bbeta^{\eta(1)}(k))+\sqrt{2\eta\tau_1} \bxi_k^{(1)}\\
    \widetilde \bbeta^{\eta(2)}(k+1) &= \widetilde \bbeta^{\eta(2)}(k) - \eta\nabla \widetilde U(\widetilde \bbeta^{\eta(2)}(k))+\sqrt{2\eta\tau_2} \bxi_k^{(2)},\\
       \end{split}
   \end{equation}
where $\eta$ is considered to be a fixed learning rate for ease of analysis, and $\bxi_k^{(1)}$ and $\bxi_k^{(2)}$ are independent Gaussian random vectors in $\mathbb{R}^d$. Moreover, the positions of the particles swap based on the stochastic swapping rate. In particular, $\widetilde S(\bbeta^{(1)}, \bbeta^{(2)}):=S(\bbeta^{(1)}, \bbeta^{(2)})+\psi$, and the stochastic gradient $\nabla\widetilde U(\cdot)$ can be written as $\nabla U(\cdot)+\bphi$, where both $\psi\in \mathbb{R}^1$ and $\bphi\in\mathbb{R}^d$ are random variables with mean not necessarily zero. We also denote $\m_k$ as the probability measure associated with $\{\widetilde \bbeta^{\eta}(k)\}_{k\ge 1}$ in reSGLD (\ref{reSGLD}) at step $k$, which is close to $\nu_{k\eta}$ in a suitable sense.


\subsection{Overview of the analysis}


We aim to study the convergence analysis of the probability measure $\m_{k}$ to the invariant measure $\pi$ in terms of 2-Wasserstein distance,
\begin{equation}
\label{w2}
    \mathcal{W}_2(\m, \nu):=\inf_{\Gamma\in \text{Couplings}(\m, \n)}{\sqrt{\int\|\bbeta_{\m}-\bbeta_{\n}\|^2 d \Gamma(\bbeta_{\m},\bbeta_{\n})}},
\end{equation}
where $\|\cdot\|$ is the Euclidean norm, and the infimum is taken over all joint distributions $\Gamma(\bbeta_{\mu}, \bbeta_{\nu})$ with $\mu$ and $\nu$ being the marginals distributions.

By the triangle inequality, we easily obtain that for any $k\in \mathbb{N}$ and $t=k\eta$, we have 
\begin{equation*}
    \mathcal{W}_2(\m_k, \pi)\le \underbrace{\mathcal{W}_2(\m_k, \nu_{t})}_{\text{Discretization error}} +\underbrace{\mathcal{W}_2(\nu_{t}, \pi)}_{\text{Exponential decay}}.
\end{equation*}
We start with the discretization error first by analyzing how reSGLD (\ref{reSGLD}) tracks the reLD (\ref{sde_2}) in 2-Wasserstein distance. The critical part is to study the discretization of the Poisson jump process in mini-batch settings. To handle this issue, we follow \citet{Paul12} and view the swaps of positions as swaps of temperatures. Then we apply standard techniques in stochastic calculus \citep{chen2018accelerating, yin_zhu_10, Issei14, Maxim17} to discretize the Langevin diffusion and derive the corresponding discretization error.


Next, we quantify the evolution of the 2-Wasserstein distance between $\n_t$ and $\pi$. The key tool is the exponential decay of entropy (Kullback-Leibler divergence) when $\pi$ satisfies the log-Sobolev inequality (LSI) \citep{Bakry2014}. To justify LSI, we first verify LSI for reSGLD without swaps, which is a direct result given a proper Lyapunov function criterion \citep{Cattiaux2010} and the Poincar\'{e} inequality \citep{chen2018accelerating}. Then we follow \citet{chen2018accelerating} and verify LSI for reLD with swaps by analyzing the Dirichlet form. Finally, the exponential decay of the 2-Wasserstein distance follows from the Otto-Villani theorem by connecting the 2-Wasserstein distance with the entropy \citep{Bakry2014}. 


Before we move forward, we first lay out the following assumptions:
\begin{assumption}[Smoothness]\label{assump: lip and alpha beta}
The energy function $U(\cdot)$ is $C$-smoothness, which implies that there exists a Lipschitz constant $C>0$, such that for every $x,y\in\hR^d$, we have $\|\nabla U(x)-\nabla U(y)\|\le C\|x-y\|.$ \footnote{$\|\cdot\|$ denotes the Euclidean $L^2$ norm.}
\end{assumption}{}


\begin{assumption}[Dissipativity]\label{assump: dissipitive}
The energy function $U(\cdot)$ is $(a,b)$-dissipative, i.e. there exist constants $a>0$ and $b\ge 0$ such that $\forall x\in\mathbb R^d$,  $\la x,\nabla U(x)\ra \ge a\|x\|^2-b.$
\end{assumption}{}

Here the smoothness assumption is quite standard in studying the convergence of SGLD, and the dissipativity condition is widely used in proving the geometric ergodicity of dynamic systems \citep{Maxim17, Xu18}. Moreover, the convexity assumption is not required in our theory.



\subsection{Analysis of discretization error}

The key to deriving the discretization error is to view the swaps of positions as swaps of the temperatures, which has been proven equivalent in distribution \citep{Paul12}. Therefore, we model reLD using the following SDE, 
\begin{equation}\label{replica exchange}
    d\bbeta_t=-\nabla G(\bbeta_t)dt+\Si_td\bW_t,
\end{equation}{}
where  $G(\bbeta_t)=\begin{pmatrix}{}
U(\bbeta_t^{(1)})\\
U(\bbeta_t^{(1)})
\end{pmatrix}$, $\bW\in\mathbb{R}^{2d}$ is a Brownian motion, $\Si_t$ is a random matrix in continuous-time that swaps between the diagonal matrices $\mathbb{M}_1=\begin{pmatrix}{}
\sqrt{2\tau_1}\mathbf I_d&0\\
0&\sqrt{2\tau_2}\mathbf I_d
\end{pmatrix}$ and $\mathbb{M}_2=\begin{pmatrix}{}
\sqrt{2\tau_2}\mathbf I_d&0\\
0&\sqrt{2\tau_1}\mathbf I_d
\end{pmatrix}$ with probability $r S(\bbeta_t^{(1)}, \bbeta_t^{(2)})dt$, and $\mathbf I_d\in \mathbb R^{d\times d}$ is denoted as the identity matrix.

Moreover, the corresponding discretization of replica exchange SGLD (reSGLD) follows:
\begin{equation}
\begin{split}
\label{resgld_2}
    \widetilde \bbeta^{\eta}(k+1)=\widetilde \bbeta^{\eta}(k)-
    \eta\nabla \widetilde G(\widetilde \bbeta^{\eta}(k)) + \sqrt{\eta}\widetilde \Si^{\eta}(k)\bxi_k,
\end{split}
\end{equation}
where $\bxi_k$ is a standard Gaussian distribution in $\mathbb{R}^{2d}$, and $\widetilde \Si^{\eta}(k)$ is a random matrix in discrete-time that swaps between $\mathbb{M}_1$ and $\mathbb{M}_2$ with probability $r\widetilde S(\widetilde \bbeta^{\eta(1)}(k), \widetilde \bbeta^{\eta(2)}(k))\eta$. We denote $\{\widetilde \bbeta_t^{\eta}\}_{t\ge 0 }$ as the continuous-time interpolation of $\{\widetilde \bbeta^{\eta}(k)\}_{k\ge 1}$, which satisfies the following SDE, 
\begin{equation}\label{SGD continuous time interpolation}
	\widetilde \bbeta_t^{\eta}=\widetilde \bbeta_0-\int_0^t\nabla \widetilde G(\widetilde \bbeta^{\eta}_{\lfloor s/\eta \rfloor \eta})ds+\int_0^t\widetilde\Si^{\eta}_{\lfloor s/\eta \rfloor\eta}d\bW_s.
\end{equation}
Here the random matrix $\widetilde \Si_{\lfloor s/\eta\rfloor \eta}^{\eta}$ follows a similar trajectory as $\widetilde \Si^{\eta}(\lfloor s/\eta \rfloor)$. For $k\in \mathbb N^{+}$ with $t=k\eta$, the relation $\widetilde \bbeta_t^{\eta}=\widetilde \bbeta_{k\eta}^{\eta}=\widetilde \bbeta^{\eta}(k)$ follows.


\begin{lemma}[Discretization error]\label{discretization}
Given the smoothness and dissipativity assumptions \eqref{assump: lip and alpha beta} and \eqref{assump: dissipitive}, and the learning rate $\eta$ satisfying $0<\eta<1 \land a/C^2$, there exists constants $D_1, D_2$ and $D_3$ such that
\begin{equation}
	\begin{split}
&	\hE[\sup_{0\le t\le T}\|\bbeta_t-\widetilde \bbeta^{\eta}_t||^2] \le D_1 \eta + D_2 \max_{k}\hE[\|\bphi_k\|^2]+D_3\max_{k}\sqrt{\hE\left[|\psi_{k}|^2\right]},\\
	\end{split}
\end{equation}
where $ D_1$ depends on $\tau_1,\tau_2,d, T, C,a,b$; $D_2$ depends on $T$ and $C$; $D_3$ depends on $r, d, T$ and $C$.
\end{lemma}{}

\begin{proof}
Based on the replica exchange Langevin diffusion $\{\bbeta_t\}_{t\ge 0}$ and the continuous-time interpolation of the stochastic gradient Langevin diffusion $\{\widetilde \bbeta_t^{\eta}\}_{t\ge 0}$, we have the following SDE for the difference $\bbeta_t-\widetilde \bbeta^{\eta}_t$. For any $t\in [0,T]$, we have
\begin{equation*}
	\begin{split}
		\bbeta_t-\widetilde \bbeta^{\eta}_t&=-\int_0^t(\nabla G(\bbeta_s)-\nabla\widetilde G(\widetilde \bbeta^{\eta}_{\lfloor s/\eta \rfloor\eta})ds+\int_0^t (\Si_s-\widetilde \Si^{\eta}_{\lfloor s/\eta \rfloor\eta})d\bW_s
	\end{split}
\end{equation*}
Indeed, note that
\begin{equation*}
	\begin{split}
		\sup_{0\le t\le T}\|\bbeta_t-\widetilde \bbeta^{\eta}_t\|&\le \int_0^T\|\nabla G(\bbeta_s)-\nabla\widetilde G(\widetilde \bbeta^{\eta}_{\lfloor s/\eta \rfloor\eta})\|)ds+\sup_{0\le t\le T}\left\|\int_0^t (\Si_s-\widetilde \Si^{\eta}_{\lfloor s/\eta \rfloor\eta})d\bW_s\right\|
	\end{split}
\end{equation*}
We first square both sides and take expectation, then apply the Burkholder-Davis-Gundy inequality and  Cauchy-Schwarz inequality, we have 
\begin{equation}\label{sup norm estimate 1}
	\begin{split}
	\hE[	\sup_{0\le t\le T}\|\bbeta_t-\widetilde \bbeta^{\eta}_t\|^2]&\le 2\hE\left[\left(\int_0^T\|\nabla G(\bbeta_s)-\nabla\widetilde G(\widetilde \bbeta^{\eta}_{\lfloor s/\eta \rfloor\eta})\|ds\right)^2+\sup_{0\le t\le T}\left\|\int_0^t (\Si_s-\widetilde \Si^{\eta}_{\lfloor s/\eta \rfloor\eta})d\bW_s\right\|^2\right]\\
	&\le \underbrace{2T\hE\left[\int_0^T\|\nabla G(\bbeta_s)-\nabla\widetilde G(\widetilde \bbeta^{\eta}_{\lfloor s/\eta \rfloor\eta})\|^2ds \right]}_{\cI}+\underbrace{8\hE\left[\int_0^T\|\Si_s-\widetilde \Si^{\eta}_{\lfloor s/\eta \rfloor\eta}\|^2 ds\right]}_{\cJ}
	\end{split}
\end{equation}

\noindent
$\textbf{Estimate of stochastic gradient:}$
For the first term $\cI$, by using the inequality
$$\|a+b+c\|^2\le 3(\|a\|^2+\|b\|^2+\|c\|^2),$$ 
we get  
\begin{equation}
	\begin{split}
		\cI=&2T\hE\left[\int_0^T\left\|\left(\nabla G(\bbeta_s)-\nabla G(\widetilde \bbeta^{\eta}_s)\right)+\left(\nabla G(\widetilde \bbeta^{\eta}_s)-\nabla  G(\widetilde \bbeta^{\eta}_{\lfloor s/\eta \rfloor\eta})\right)+\left(\nabla  G(\widetilde \bbeta^{\eta}_{\lfloor s/\eta \rfloor\eta})-\nabla \widetilde G(\widetilde \bbeta^{\eta}_{\lfloor s/\eta \rfloor\eta})\right)\right\|^2ds \right]\\
		\le& \underbrace{6T\hE\left[\int_0^T\|\nabla G(\bbeta_s)-\nabla  G(\widetilde \bbeta^{\eta}_s)\|^2ds\right]}_{\cI_1}+\underbrace{6T\hE\left[\int_0^T\|\nabla  G(\widetilde \bbeta^{\eta}_s)-\nabla  G(\widetilde \bbeta^{\eta}_{\lfloor s/\eta \rfloor\eta})\|^2ds\right]}_{\cI_2}\\
		&+\underbrace{6T\hE\left[\int_0^T\|\nabla  G(\widetilde \bbeta^{\eta}_{\lfloor s/\eta \rfloor\eta})-\nabla \widetilde G(\widetilde \bbeta^{\eta}_{\lfloor s/\eta \rfloor\eta})\|^2ds\right]}_{\cI_3}\\
		\le& \cI_1+\cI_2+\cI_3.
	\end{split}
\end{equation}
By using the smoothness assumption \ref{assump: lip and alpha beta}, we first estimate 
\begin{equation*}
    \cI_1 \le 6TC^2\hE\left[\int_0^T\|\bbeta_s-\widetilde \bbeta^{\eta}_s\|^2ds\right].
\end{equation*}{}
By applying the smoothness assumption \ref{assump: lip and alpha beta} and discretization scheme, we can further estimate 
\begin{equation}\label{est cI2}
    \begin{split}
        \cI_2&\le 6TC^2\hE\left[\int_0^T\|\widetilde \bbeta^{\eta}_s-\widetilde \bbeta^{\eta}_{\lfloor s/\eta \rfloor\eta}\|^2ds\right]\\
        &\le 6TC^2\sum_{k=0}^{\lfloor T/\eta\rfloor} \hE\left[\int_{k\eta}^{(k+1)\eta}\|\widetilde \bbeta^{\eta}_s-\widetilde \bbeta^{\eta}_{\lfloor s/\eta \rfloor\eta} \|^2ds \right]\\
        &\le 6TC^2\sum_{k=0}^{\lfloor T/\eta\rfloor} \int_{k\eta}^{(k+1)\eta}\hE\left[\sup_{k\eta\le s<(k+1)\eta}\|\widetilde \bbeta^{\eta}_s-\widetilde \bbeta^{\eta}_{\lfloor s/\eta \rfloor\eta} \|^2\right]ds 
    \end{split}{}
\end{equation}{}
For $\forall~ k\in\mathbb N$ and $s\in [k\eta,(k+1)\eta)$, we have
\begin{equation*}
    \begin{split}
        \widetilde \bbeta^{\eta}_s-\widetilde \bbeta^{\eta}_{\lfloor s/\eta \rfloor\eta}=\widetilde \bbeta^{\eta}_s-\widetilde \bbeta^{\eta}_{k\eta}=-\nabla\widetilde G(\widetilde \bbeta^{\eta}_{k\eta})\cdot(s-k\eta)+\widetilde\Si^{\eta}_{k\eta}\int_{k\eta}^sd\bW_r
    \end{split}{}
\end{equation*}{}
which indeed implies
\begin{equation*}
    \begin{split}
     \sup_{k\eta\le s<(k+1)\eta}  \| \widetilde \bbeta^{\eta}_s-\widetilde \bbeta^{\eta}_{\lfloor s/\eta \rfloor\eta}\|\le \|\nabla\widetilde G(\widetilde \bbeta^{\eta}_{k\eta})\|(s-k\eta)+\sup_{k\eta\le s<(k+1)\eta} \|\widetilde\Si^{\eta}_{k\eta}\int_{k\eta}^sd\bW_r\|
    \end{split}{}
\end{equation*}{}
Similar to the estimate \eqref{sup norm estimate 1}, square both sides and take expectation, then apply the Burkholder-Davis-Gundy inequality, we have
\begin{equation*}
    \begin{split}
        \hE\left[\sup_{k\eta\le s<(k+1)\eta}  \| \widetilde \bbeta^{\eta}_s-\widetilde \bbeta^{\eta}_{\lfloor s/\eta \rfloor\eta}\|^2 \right]&\le 2\hE[ \|\nabla\widetilde G(\widetilde \bbeta^{\eta}_{k\eta})\|^2(s-k\eta)^2 ]+8\sum_{j=1}^{2d}\hE\left[\left(\widetilde \Si^{\eta}_{k\eta}(j)\la \int_{k\eta}^{\cdot}d\bW_r \ra_s^{1/2}\right)^2 \right]\\
         &\le  2(s-k\eta)^2\hE[ \|\nabla \widetilde G(\widetilde \bbeta^{\eta}_{k\eta})\|^2]+32d\tau_2(s-k\eta),
    \end{split}{}
\end{equation*}{}
where the last inequality follows from the fact that $\widetilde \Si^{\eta}_{k\eta}$ is a diagonal matrix with diagonal elements $\sqrt{2\tau_1}$ or $\sqrt{2\tau_2}.$
For the first term in the above inequality, we further have
\begin{equation*}
    \begin{split}
  2(s-k\eta)^2 \hE[ \|\nabla \widetilde G(\widetilde \bbeta^{\eta}_{k\eta})\|^2]&=        2(s-k\eta)^2\hE[ \|(\nabla G(\widetilde \bbeta^{\eta}_{k\eta})+\bphi_k)\|^2 ]\\
&\le 4\eta^2\hE[ \|\nabla  G(\widetilde \bbeta^{\eta}_{k\eta})-\nabla  G(\bbeta^*) \|^2+\|\bphi_k\|^2 ]\\
   &\le 8C^2\eta^2\hE[\|\widetilde \bbeta^{\eta}_{k\eta}\|^2+\|\bbeta^*\|^2]+4\eta^2\hE[\|\bphi_k\|^2 ],
    \end{split}{}
\end{equation*}
where the first inequality follows from the separation of the noise from the stochastic gradient and the choice of stationary point $\bbeta^*$ of $G(\cdot)$ with $\nabla G(\bbeta^*)=0$, and $\bphi_k$ is the stochastic noise in the gradient at step $k$. Thus, combining the above two parts and integrate $ \hE\left[\sup_{k\eta\le s<(k+1)\eta}  \| \widetilde \bbeta^{\eta}_s-\widetilde \bbeta^{\eta}_{k\eta}\|^2 \right]$ on the time interval $[k\eta,(k+1)\eta)$, we obtain the following bound 
\begin{equation}\label{est cI2 part two}
    \begin{split}
        \int_{k\eta}^{(k+1)\eta}  \hE\left[\sup_{k\eta\le s<(k+1)\eta}  \| \widetilde \bbeta^{\eta}_s-\widetilde \bbeta^{\eta}_{k\eta}\|^2 \right] ds&\le 8C^2\eta^3 \left(\sup_{k\ge 0}\hE[ \|\widetilde \bbeta_{k\eta}^{\eta}\|^2+ \|\bbeta^*\|^2] \right)+4\eta^3 \max_{k}\hE[\|\bphi_k\|^2]+32d\tau_2\eta^2
    \end{split}{}
\end{equation}{}
By plugging the estimate \eqref{est cI2 part two} into estimate \eqref{est cI2}, we obtain the following estimates when $\eta\le 1$,
\begin{equation}
    \begin{split}
        \cI_2&\le 6TC^2(1+T/\eta)\left[8C^2\eta^3 \left(\sup_{k\ge 0}\hE[ \|\widetilde \bbeta_{k\eta}^{\eta}\|^2+\|\bbeta^*\|^2] \right)+4\eta^3 \max_{k}\hE[\|\bphi_k\|^2]+32d\tau_2\eta^2\right]\\
        & \le \tilde \delta_1(d, \tau_2, T, C, a, b) \eta + 24TC^2(1+T)\max_{k}\hE[\|\bphi_k\|^2],
    \end{split}{}
\end{equation}{}
where $\tilde \delta_1(d, \tau_2, T, C, a, b)$ is a constant depending on $d, \tau_2, T, C, a$ and $b$. Note that the above inequality requires a result on the bounded second moment of   $\sup_{k\ge 0}\hE[ \|\widetilde \bbeta_{k\eta}^{\eta}\|^2]$, and this is majorly\footnote{The slight difference is that the constant in the RHS of (C.38) \citet{chen2018accelerating} is changed to account for the stochastic noise.} proved in Lemma $C.2$ in \citet{chen2018accelerating} when we choose the stepzise $\eta\in (0, a/C^2)$. We are now left to estimate the term $\cI_3$ and we have
\begin{equation}
    \begin{split}
   \cI_3  &\le  6T\sum_{k=0}^{\lfloor T/\eta \rfloor} \hE\left[\int_{k\eta}^{(k+1)\eta}\|\nabla  G(\widetilde \bbeta^{\eta}_{k\eta})-\nabla \widetilde G(\widetilde \bbeta^{\eta}_{k\eta})\|^2ds \right]\\
   &\le 6T (1+T/\eta)\max_{k}\hE[ \|\bphi_k\|^2]\eta\\
   &\le 6T (1+T)\max_{k}\hE[ \|\bphi_k\|^2].
    \end{split}{}
\end{equation}{}
Combing all the estimates of $\cI_1,\cI_2$ and $\cI_3$, we obtain
\begin{equation}
    \begin{split}
        \cI\le& \underbrace{6TC^2\int_0^T\hE\left[\sup_{0\le s\le T}\|\bbeta_s-\widetilde \bbeta^{\eta}_s\|^2\right]ds}_{\cI_1}+\underbrace{\tilde \delta_1(d, \tau_2, T, C, a, b) \eta+ 24TC^2(1+T)\max_{k}\hE[\|\bphi_k\|^2]}_{\cI_2}\\
        &+\underbrace{6T (1+T)\max_{k}\hE[ \|\bphi_k\|^2]}_{\cI_3}.\\
    \end{split}{}
\end{equation}{}
\noindent
$\textbf{Estimate of stochastic diffusion:}$
For the second term $\cJ$, we have 
\begin{equation}
	\begin{split}
		\cJ&=8\hE\left[\int_{0}^{T} \|\Sigma_s(j)-\widetilde \Sigma_{{\lfloor s/\eta \rfloor}\eta}(j)\|^2 ds\right] \\
		& \leq 8\sum_{j=1}^{2d}\sum_{k=0}^{{\lfloor T/\eta \rfloor}}\int_{k \eta}^{(k+1)\eta}\hE\left[\|\Sigma_s(j)-\widetilde \Sigma^{\eta}_{k \eta}(j)\|^2\right]ds \\
		& \leq 8\sum_{j=1}^{2d}\sum_{k=0}^{{\lfloor T/\eta \rfloor}}\int_{k \eta}^{(k+1)\eta}\hE\left[\|\Sigma_s(j)- \Sigma_{k \eta}^{\eta}(j)+\Sigma_{k \eta}^{\eta}(j)-\widetilde \Sigma_{k \eta}^{\eta}(j)\|^2\right]ds \\
		& \leq 16\sum_{j=1}^{2d}\sum_{k=0}^{{\lfloor T/\eta \rfloor}}\left[\underbrace{\int_{k \eta}^{(k+1)\eta}\hE\left[\|\Sigma_s(j)- \Sigma_{k \eta}^{\eta}(j)\|^2\right]ds}_{\cJ_1}+\underbrace{\int_{k \eta}^{(k+1)\eta}\hE\left[\|\Sigma_{k \eta}^{\eta}(j)-\widetilde \Sigma_{k \eta}^{\eta}(j)\|^2\right]ds}_{\cJ_2}\right] 
	\end{split}.
\end{equation}
where $\Si_{k \eta}^{\eta}$ is the temperature matrix for the continuous-time interpolation of $\{\bbeta^{\eta}(k)\}_{k\ge 1}$, which is similar to \eqref{SGD continuous time interpolation} without noise generated from mini-batch settings and is defined as below
\begin{equation}
	 \bbeta_t^{\eta}= \bbeta_0-\int_0^t\nabla  G( \bbeta^{\eta}_{k\eta})ds+\int_0^t\Si^{\eta}_{k \eta}d\bW_s.
\end{equation}

We estimate $\cJ_1$ first, considering that $\Si_s$ and $\Si^{\eta}_{\lfloor s/\eta\rfloor \eta}$ are both diagonal matrices, we have
\begin{equation*}
    \begin{split}
        \cJ_1&=4(\sqrt{\tau_2}-\sqrt{\tau_1})^2\int_{k \eta}^{(k+1)\eta}\hP(\Si_s(j)\neq \Si^{\eta}_{k \eta}(j))ds\\
        &=4(\sqrt{\tau_2}-\sqrt{\tau_1})^2\hE\left[\int_{k \eta}^{(k+1)\eta}\hP(\Si_s(j)\neq \Si^{\eta}_{k \eta}(j)|\bbeta^{\eta}_{k \eta})ds\right]\\
        &= 4(\sqrt{\tau_2}-\sqrt{\tau_1})^2 r\int_{k \eta}^{(k+1)\eta} [(s-k \eta)+\mathcal R(s-k \eta)]ds\\
        &\le \tilde \delta_2(r, \tau_1,\tau_2)\eta^2,
    \end{split}{}
\end{equation*}{}
where $\tilde \delta_2(r, \tau_1,\tau_2)=4(\sqrt{\tau_2}-\sqrt{\tau_1})^2 r$, and the equality follows from the fact that the conditional probability $\hP(\Si_s(j)\neq \Si^{\eta}_{k \eta}(j)|\bbeta^{\eta}_{k \eta})=r S(\bbeta^{\eta(1)}_{k \eta},\bbeta^{\eta(2)}_{k \eta})\cdot(s-\eta)+r\mathcal R(s-k \eta)$. Here $\mathcal R(s-k \eta)$ denotes the higher remainder with respect to $s-k \eta$. The estimate of $\cJ_1$ without stochastic gradient for the Langevin diffusion is first obtained in \citet{chen2018accelerating}, we however present here again for reader's convenience.  
As for the second term $\cJ_2$, it follows that
\begin{equation}
	\begin{split}
	\label{new_prob}
	    \cJ_2&=4(\sqrt{\tau_2}-\sqrt{\tau_1})^2\int_{k \eta}^{(k+1)\eta}\hP(\Sigma_{k \eta}(j)\neq \widetilde \Sigma_{k \eta}(j))ds \\
	    &= 4(\sqrt{\tau_2}-\sqrt{\tau_1})^2 r\eta \hE\left[\left|S(\bbeta_{k \eta}^{\eta(1)}, \bbeta_{k \eta}^{\eta(2)})-\tilde S(\widetilde \bbeta_{k \eta}^{\eta(1)}, \widetilde \bbeta_{k \eta}^{\eta(2)})\right|\right] \\
	    &\le \tilde \delta_2(r, \tau_1,\tau_2) \eta \sqrt{\hE\left[\left|S(\bbeta_{k \eta}^{\eta(1)}, \bbeta_{k \eta}^{\eta(2)})-\tilde S(\widetilde \bbeta_{k \eta}^{\eta(1)}, \widetilde \bbeta_{k \eta}^{\eta(2)})\right|^2\right]}\\
	    & \leq \tilde \delta_2(r, \tau_1,\tau_2) \eta \sqrt{\hE\left[|\psi_{k}|^2\right]},
	\end{split}
\end{equation}
where $\psi_{k}$ is the noise in the swapping rate. Thus, one concludes the following estimates combing $\cI$ and $\cJ$.
\begin{equation}
	\begin{split}
	&\hE[\sup_{0\le t\le T}\|\bbeta_t-\widetilde \bbeta^{\eta}_t||^2] \le \underbrace{6TC^2\int_0^T\hE\left[\sup_{0\le s\le T}\|\bbeta_s-\widetilde \bbeta^{\eta}_s\|^2\right]ds}_{\cI_1}+\underbrace{\tilde \delta_1(d, \tau_2, T, C, a, b) \eta+ 24TC^2(\eta+T)\max_{k}\hE[\|\bphi_k\|^2]}_{\cI_2}\\
	&\ \ \ \ \ +\underbrace{6T (1+T) \hE[\|\bphi_{k}\|^2]}_{\cI_3}+\underbrace{32d(1+T)\tilde \delta_2(r, \tau_1,\tau_2)\left(\eta+\max_{k}\sqrt{\hE\left[|\psi_{k}|^2\right]}\right)}_{\cJ}. 
	\end{split}
\end{equation}
Apply Gronwall's inequality to the function
\begin{equation*}
    t\mapsto \hE\left[\sup_{0\le u\le t} \|\bbeta_u-\widetilde \bbeta_u^{\eta}\|^2 \right],
\end{equation*}{}
and deduce that 
\begin{equation}
	\begin{split}
&	\hE[\sup_{0\le t\le T}\|\bbeta_t-\widetilde \bbeta^{\eta}_t||^2] \le D_1 \eta + D_2 \max_{k}\hE[\|\bphi_k\|^2]+D_3\max_{k}\sqrt{\hE\left[|\psi_{k}|^2\right]},\\
	\end{split}
\end{equation}
where $ D_1$ is a constant depending on $\tau_1,\tau_2,d, T, C,a,b$; $D_2$ depends on $T$ and $C$; $D_3$ depends on $r, d, T$ and $C$. \qed 


\end{proof}


\subsection{Exponential decay of Wasserstein distance in continuous-time}


We proceed to quantify the evolution of the 2-Wasserstein distance between $\n_{t}$ and $\pi$. We first consider the ordinary Langevin diffusion without swaps and derive the log-Sobolev inequality (LSI). Then we extend LSI to reLD and obtain the exponential decay of the relative entropy. Finally, we derive the exponential decay of the 2-Wasserstein distance.


In order to distinguish from the replica exchange Langevin diffusion $\bbeta_t$ defined in \eqref{replica exchange}, we call it $\hat \bbeta_t$ which follows, 
\begin{equation}
    d\hat \bbeta_t=-\nabla G(\hat \bbeta_t)dt+\Si_td\bW_t.
\end{equation}{}
where $\Si_t\in \hR^{2d\times 2d}$ is a diagonal matrix with the form $\begin{pmatrix}{}
\sqrt{2\tau_1}\mathbf I_d&0\\
0&\sqrt{2\tau_2}\mathbf I_d
\end{pmatrix}$. 
The process $\hat \bbeta_t$ is a Markov diffusion process with infinitesimal generator $\cL$ in the following form, for $x_1\in\hR^d$ and $x_2\in\hR^d$,
\beaa
\cL=&-\la\nabla_{x_1}f(x_1,x_2),\nabla U(x_1)\ra+\tau_1\Delta_{x_1}f(x_1,x_2)\\
&-\la \nabla_{x_2}f(x_1,x_2),\nabla U(x_2)\ra+\tau_2\Delta_{x_2}f(x_1,x_2)
\eeaa
Note that since matrix $\Si_t$ is a non-degenerate diagonal matrix, operator $\cL$ is an elliptic diffusion operator. According to the smoothness assumption \eqref{assump: lip and alpha beta}, we have that $\nabla^2 G\ge -C\mathbf I_{2d}$, where $C>0$, the unique invariant measure $\pi$ associate with the underlying diffusion process satisfies the Poincare inequality and LSI with the Dirichlet form given as follows,
\bea\label{dirichlet form}
\cE(f)=\int \Big(\tau_1\|\nabla_{x_1}f\|^2+\tau_2\|\nabla_{x_2}f\|^2 \Big)d\pi(x_1,x_2),\qq f\in\cC_0^2(\hR^{2d}).
\eea
In this elliptic case with $G$ being convex, the proof for LSI follows from standard Bakry-Emery calculus \cite{Bakry85}. Since, we are dealing with the non-convex function $G$, we are particularly interested in the case of $\nabla^2 G\ge -C\mathbf I_{2d}$.
To obtain a Poincar\'{e} inequality for invariant measure $\pi$, \citet{chen2018accelerating} adapted an argument from \citet{Bakry08} and \citet{Maxim17} by constructing an appropriate Lyapunov function for the replica exchange diffusion without swapping $\hat \bbeta_t$. Denote $\n_t$ as the distribution associated with the diffusion process $\{\hat \bbeta_t\}_{t\ge 0}$, which is absolutely continuous with respect to $\pi$. It is a direct consequence of the aforementioned results that the following log-Sobolev inequality holds.



\begin{lemma}[LSI for Langevin Diffusion]\label{LSI no swaping}
Under assumptions \eqref{assump: lip and alpha beta} and \eqref{assump: dissipitive}, we have the following log-Sobolev inequality for invariant measure $\pi$, for some constant $c_{\text{LS}}>0$, 
\beaa
D(\n_t||\pi)\le 2c_{\text{LS}}\cE(\sqrt{\frac{d\n_t}{d\pi}}).
\eeaa
where $D(\n_t||\pi)=\int d\nu_t \log\frac{d\nu_t}{d\pi}$ denotes the relative entropy and the Dirichlet form $\cE(\cd)$ is defined in \eqref{dirichlet form}.
\end{lemma}{}
\begin{proof}

According to \citet{Cattiaux2010}, the sufficient conditions to establish LSI are:
\begin{enumerate}
\item There exists some constant $C\ge 0$, such that $\nabla^2 G\succcurlyeq -C I_{2d}$.
\item $\pi$ satisfies a Poincar\'{e} inequality with constant $c_{p}$, namely, for all probability measures $\nu\ll\pi$, $\chi^2(\nu||\pi)\leq c_p \cE(\sqrt{\frac{d\n_t}{d\pi}})$, where $\chi^2(\nu||\pi):=\|\frac{d\nu}{d\pi}-1\|^2$ is the $\chi^2$ divergence between $\nu$ and $\pi$.
\item There exists a $\cC^2$ Lyapunov function $V: \mathbb{R}^{2d}\rightarrow [1, \infty)$ such that $\frac{\cL V(x_1, x_2)}{V(x_1, x_2)} \leq \kappa - \gamma (\|x_1\|^2 + \|x_2\|^2)$
for all $(x_1, x_2)\in \mathbb{R}^{2d}$ and some $\kappa, \gamma>0$.
\end{enumerate}
Note that the first condition on the Hessian is obtained from the smoothness assumption \eqref{assump: lip and alpha beta}. Moreover, the Poincar\'{e} inequality in the second condition is derived from Lemma C.1 in \citet{chen2018accelerating} given assumptions \eqref{assump: lip and alpha beta} and \eqref{assump: dissipitive}. Finally, to verify the third condition, we follow \citet{Maxim17} and construct the Lyapunov function 
$V(x_1,x_2):=\exp\left\{a/4 \cdot \left(\frac{\|x_1\|^2}{\tau_1}+\frac{\|x_2\|^2}{\tau_2}\right)\right\}$. From the dissipitive assumption \ref{assump: dissipitive}, $V(x_1, x_2)$ satisfies the third condition because
\begin{equation}
\begin{split}
    \cL(V(x_1, x_2))&=\left(\frac{a}{2\tau_1}+\frac{a}{2\tau_2}+\frac{a^2}{4\tau_1^2}\|x_1\|^2+\frac{a^2}{4\tau_2^2}\|x_2\|^2-\frac{a}{2\tau_1^2}\langle x_1, \nabla G(x_1)-\frac{a}{2\tau_2^2}\langle x_1, \nabla G(x_2)\rangle\right) V(x_1, x_2)\\
    &\leq \left(\frac{a}{2\tau_1}+\frac{a}{2\tau_2}+\frac{ab}{2\tau_1^2}+\frac{ab}{2\tau_2^2}-\frac{a^2}{4\tau_1^2}\|x_1\|^2-\frac{a^2}{4\tau_2^2}\|x_2\|^2\right) V(x_1, x_2)\\
    &\leq \left(\kappa-\gamma (\|x_1\|^2+\|x_2\|^2)\right) V(x_1, x_2),\\
\end{split}
\end{equation}
where $\kappa=\frac{a}{2\tau_1}+\frac{a}{2\tau_2}+\frac{ab}{2\tau_1^2}+\frac{ab}{2\tau_2^2}$, and $\gamma=\frac{a^2}{4\tau_1^2}\land \frac{a^2}{4\tau_2^2}$. Therefore, the invariant measure $\pi$ satisfies a LSI with the constant
\begin{equation}
    c_{\text{LS}}=c_1+(c_2+2)c_p,
\end{equation}
where $c_1=\frac{2C}{\gamma}+\frac{2}{C}$ and $c_2=\frac{2C}{\gamma}\left(\kappa+\gamma\int_{\mathbb{R}^{2d}}( \|x_1\|^2 + \|x_2\|^2)\pi(dx_1 dx_2)\right)$. \qed




\end{proof}{}



We are now ready to prove the log-Sobolev inequality for invariant measure associated with the replica exchange Langevin diffusion \eqref{replica exchange}. We use a similar idea from \citet{chen2018accelerating} where they prove the Poincar\'{e} inequality for the invariant measure associated with the replica exchange Langevin diffusion \eqref{replica exchange} by analyzing the corresponding Dirichlet form. In particular, a larger Dirichlet form ensures a smaller log-Sobolev constant and hence results in a faster convergence in the relative entropy and Wasserstein distance.

\begin{lemma}[Accelerated exponential decay of $\mathcal{W}_2$]\label{exponential decay}
Under assumptions \eqref{assump: lip and alpha beta} and \eqref{assump: dissipitive}, we have that the replica exchange Langevin diffusion converges exponentially fast to the invariant distribution $\pi$:
\begin{equation}
    \mathcal{W}_2(\nu_t,\pi) \leq  D_0 e^{-k\eta(1+\delta_S)/c_{\text{LS}}},
\end{equation}
where $D_0=\sqrt{2c_{\text{LS}}D(\nu_0||\pi)}$, $\delta_{S}:=\inf_{t>0}\frac{\cE_S(\sqrt{\frac{d\n_t}{d\pi}})}{\cE(\sqrt{\frac{d\n_t}{d\pi}})}-1$ is a non-negative constant depending on the swapping rate $S(\cd, \cd)$ and obtains $0$ only if $S(\cd, \cd)=0$.
\end{lemma}{}

\begin{proof}
Consider the infinitesimal generator associated with the diffusion process \eqref{replica exchange}, denoted as $\cL_{S}$, contains an extra term arising from the temperature swapping. The operator $\cL_{S}$ in this particular case, indeed, has the following form
\bea
\cL_{S}=\cL+ rS(x_1,x_2)\cd (f(x_2,x_1)-f(x_1,x_2)).
\eea
According to Theorem 3.3 \citep{chen2018accelerating}, the Dirichlet form associated with operator $\cL_{S}$ under the invariant measure $\pi$ has the form
\bea\label{dirichlet swap}
\cE_{S}(f)=\cE(f)+\underbrace{\frac{r}{2}\int S(x_1,x_2)\cd (f(x_2,x_1)-f(x_1,x_2))^2d\pi(x_1,x_2)}_{\text{acceleration}},~ f\in\cC_0^2(\hR^{2d}),
\eea
where $f$ corresponds to $\frac{d\nu_t}{d\pi(x_1, x_2)}$, and the asymmetry of $\frac{\nu_t}{\pi(x_1, x_2)}$ is critical in the acceleration effect \citep{chen2018accelerating}. Given two different temperatures $\tau_1$ and $\tau_2$, a non-trivial distribution $\pi$ and function $f$, the swapping rate $S(x_1,x_2)$ is positive for almost any $x_1, x_2\in\mathbb{R}^d$. As a result, the Dirichlet form associated with $\cL_{S}$ is strictly larger than $\cL$. Therefore, there exists a constant $\delta_{S}> 0$ depending on $S(x_1,x_2)$, such that $\delta_{S}=\inf_{t>0}\frac{\cE_S(\sqrt{\frac{d\n_t}{d\pi}})}{\cE(\sqrt{\frac{d\n_t}{d\pi}})}-1$. From Lemma \ref{LSI no swaping}, we have
\begin{equation}
    D(\n_t||\pi)\le 2c_{\text{LS}}\cE(\sqrt{\frac{d\n_t}{d\pi}})\le 2c_{\text{LS}} \sup_t\frac{ \cE(\sqrt{\frac{d\n_t}{d\pi}})}{\cE_{S}(\sqrt{\frac{d\n_t}{d\pi}})}\cE_{S}(\sqrt{\frac{d\n_t}{d\pi}})= 2 \frac{c_{\text{LS}}}{1+\delta_{S}}\cE_{S}(\sqrt{\frac{d\n_t}{d\pi}}).
\end{equation}
Thus, we obtain the following log-Sobolev inequality for the unique invariant measure $\pi$  associated with replica exchange Langevin diffusion $\{\bbeta_t\}_{t\ge 0}$ and its corresponding Dirichlet form $\cE_{S}(\cd)$. In particular, the LSI constant $ \frac{c_{\text{LS}}}{1+\delta_{S}}$ in replica exchange Langevin diffusion with swapping rate $S(\cd, \cd)>0$ is strictly smaller than the LSI constant $c_{\text{LS}}$ in the replica exchange Langevin diffusion with swapping rate $S(\cd, \cd)=0$. By the exponential decay in entropy \citep{Bakry2014}[Theorem 5.2.1] and the tight log-Sobolev inequality in Lemma \ref{LSI no swaping}, we get that, for any $t\in[k\eta,(k+1)\eta)$, 
\begin{equation}
    D(\nu_t||\pi)\leq D(\nu_0||\pi) e^{-2t(1+\delta_S)/c_{\text{LS}}}\leq D(\m_0||\pi) e^{-2k\eta(1+\delta_S)/c_{\text{LS}}}.
\end{equation}
Finally, we can estimate the term $\mathcal{W}_2(\nu_t,\pi)$ by the
Otto-Villani theorem \citep{Bakry2014}[Theorem 9.6.1],
\begin{equation}
    \mathcal{W}_2(\nu_t,\pi) \leq \sqrt{2 c_{\text{LS}} D(\nu_t||\pi)}\leq \sqrt{2c_{\text{LS}}D(\m_0||\pi)} e^{-k\eta(1+\delta_S)/c_{\text{LS}}}.
\end{equation}
\end{proof}




\subsection{Summary: Convergence of reSGLD}

Now that we have all the necessary ingredients in place, we are ready to derive the convergence of the distribution $\m_{k}$ to the invariant measure $\pi$ in terms of 2-Wasserstein distance, 




\begin{theorem}[Convergence of reSGLD]Let the assumptions \eqref{assump: lip and alpha beta} and \eqref{assump: dissipitive} hold. For the unique invariant measure $\pi$ associated with the Markov diffusion process \eqref{replica exchange} and the distribution $\{\m_{k}\}_{k\ge 0}$ associated with the discrete dynamics $\{\widetilde \bbeta^{\eta}(k)\}_{k\ge 1}$, we have the following estimates, for $0\le k\in \mathbb N^{+}$ and the learning rate $\eta$ satisfying $0<\eta<1 \land a/C^2$, 
\begin{equation}
    \mathcal{W}_2(\m_{k}, \pi) \le  D_0 e^{-k\eta(1+\delta_S)/c_{\text{LS}}}+\sqrt{\delta_1 \eta +  \delta_2 \max_{k}\hE[\|\bphi_k\|^2]+ \delta_3\max_{k}\sqrt{\hE\left[|\psi_{k}|^2\right]}}
\end{equation}
where $D_0=\sqrt{2c_{\text{LS}}D(\m_0||\pi)}$, $\delta_{S}:=\min_{k}\frac{\cE_S(\sqrt{\frac{d\m_k}{d\pi}})}{\cE(\sqrt{\frac{d\m_k}{d\pi}})}-1$ is a non-negative constant depending on the swapping rate $S(\cd, \cd)$ and obtains the minimum zero only if $S(\cd, \cd)=0$.
\end{theorem}{}

\begin{proof}
We reduce the estimates into the following two terms by using the triangle inequality,
\begin{equation}\label{w2 triangle}
    \mathcal{W}_2(\m_{k}, \pi) \leq \mathcal{W}_2(\m_{k}, \n_t) + \mathcal{W}_2(\nu_t,\pi),\qq t\in[k\eta,(k+1)\eta).
\end{equation}
The first term $\mathcal{W}_2(\m_{k}, \n_t)$ follows from the analysis of discretization error in Lemma.\ref{discretization}.
Recall the very definition of the $\mathcal{W}_2(\cd,\cd)$ distance defined in (\ref{w2}). Thus, in order to control the distance $\mathcal{W}_2(\m_{k},\nu_t)$, $t\in[k\eta,(k+1)\eta)$, we need to consider the diffusion process whose law give $\m_{k}$ and $\n_t$, respectively. Indeed, it is obvious that $\n_t=\cL(\bbeta_t)$ for $t\in[k\eta,(k+1)\eta)$. For the other measure $\m_k$, it follows that $\m_{k}=\tilde \n_{k\eta}$ for $t=k\eta$, where $\tilde\n_{k\eta}=\cL(\widetilde \bbeta_t^{\eta})$ is the probability measure associated with the continuous interpolation of reSGLD (\ref{resgld_2}). By Lemma.\ref{discretization}, we have that for $k\in\mathbb{N}$ and $ t\in [k\eta,(k+1)\eta)$, 
\begin{equation}
    \mathcal{W}_2(\m_{k}, \n_t)=\mathcal{W}_2(\tilde\n_{k\eta}, \n_t) \leq \sqrt{\hE[\sup_{0\le s\le t}\|\bbeta_s- \widetilde \bbeta_s^{\eta}\|^2]}\leq \sqrt{\delta_1 \eta + \delta_2 \max_{k}\hE[\|\bphi_k\|^2]+\delta_3\max_{k}\sqrt{\hE\left[|\psi_{k}|^2\right]}},
\end{equation}

Recall from the accelerated exponential decay of replica exchange Langevin diffusion in Lemma.\ref{exponential decay}, we have
\begin{equation}
    \mathcal{W}_2(\nu_t,\pi)\leq \sqrt{2c_{\text{LS}}D(\nu_0||\pi)} e^{-k\eta(1+\delta_S)/c_{\text{LS}}}= \sqrt{2c_{\text{LS}}D(\m_0||\pi)} e^{-k\eta(1+\delta_S)/c_{\text{LS}}}.
\end{equation}

Combing the above two estimates completes the proof.
\qed
\end{proof}{}
