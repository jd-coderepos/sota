
\documentclass{article}
\usepackage{amssymb}



\newtheorem{theorem}{Theorem}
\newtheorem{acknowledgement}[theorem]{Acknowledgement}
\newtheorem{algorithm}[theorem]{Algorithm}
\newtheorem{axiom}[theorem]{Axiom}
\newtheorem{case}[theorem]{Case}
\newtheorem{claim}[theorem]{Claim}
\newtheorem{conclusion}[theorem]{Conclusion}
\newtheorem{condition}[theorem]{Condition}
\newtheorem{conjecture}[theorem]{Conjecture}
\newtheorem{corollary}[theorem]{Corollary}
\newtheorem{criterion}[theorem]{Criterion}
\newtheorem{definition}[theorem]{Definition}
\newtheorem{example}[theorem]{Example}
\newtheorem{exercise}[theorem]{Exercise}
\newtheorem{lemma}[theorem]{Lemma}
\newtheorem{notation}[theorem]{Notation}
\newtheorem{problem}[theorem]{Problem}
\newtheorem{proposition}[theorem]{Proposition}
\newtheorem{remark}[theorem]{Remark}
\newtheorem{solution}[theorem]{Solution}
\newtheorem{summary}[theorem]{Summary}
\newenvironment{proof}[1][Proof]{\noindent\textbf{#1.} }{\ \rule{0.5em}{0.5em}}
\input{tcilatex}
\usepackage{graphicx}

\begin{document}


\begin{center}
{\Large On a constructive characterization of a class of trees related to
pairs of disjoint matchings}

R.R. Kamalian, V. V. Mkrtchyan*,

{\small *Department of Informatics and Applied Mathematics, Yerevan State
University, 0025, Armenia}

{\small Institute for Informatics and Automation Problems of
National Academy of Sciences of Armenia, 0014, Armenia}

{\small e-mail: rrkamalian@yahoo.com}

{\small vahanmkrtchyan2002@\{ysu.am, ipia.sci.am, yahoo.com,\}}

{\small The author is supported by a grant of the Armenian
National Science and Educational Fund}
\end{center}

\bigskip

\begin{center}
\textbf{Abstract}
\end{center}

{\small For a graph consider the pairs of disjoint matchings which union
contains as many edges as possible, and define a parameter }{\small \ which eqauls the cardinality of the largest matching in those pairs. Also,
define }{\small \ to be the cardinality of a maximum matching of the
graph.}

{\small We give a constructive characterization of trees which satisfy the }{\small \ equality. The proof of our main theorem is based on
a new decomposition algorithm obtained for trees.}

{\small Keywords: tree, pair of disjoint matchings, maximum matching}

\bigskip

\begin{center}
\textbf{Introduction}\bigskip
\end{center}

Let  denote the set of non-negative integers. We consider finite,
undirected graphs without loops or multiple edges. Let  and 
denote the sets of vertices and edges of a graph , respectively.

If  then let  denote the degree of a vertex  in a
graph . For a bridge  of a connected graph , let  be the connected components of . Define the graphs  as follows:

\begin{center}



\end{center}

where, without loss of generality, it is assumed, that .

For a graph , let  denote the cardinality of a maximum
matching of . Define:

\begin{center}
 is a maximum matching of 




\end{center}

Let us also define:

\begin{center}
are matchings of  with ,

  and ,

 ,


\end{center}

It is known that every graph  contains a maximum -matching that
includes a maximum matching of  (see [7]). In contrast with the theory of 
-matchings, in an arbitrary graph  we cannot always guarantee the
existence of a "maximum" pair of disjoint matchings (i.e. pair of disjoint
matchings the union of which contains  edges), which includes a
maximum matching. The following is the best we can do here: for every graph  the following inequality is true [10]:

\begin{center}
 .
\end{center}

Let us also note that in her master thesis [11] Tserunyan gave an elegant
and very deep characterization of graphs which achieve the bound . Her theorem particularly implies that these graphs contain a spanning
subgraph every component of which is isomorph to the minimal graph that
satisfies the  equality.

In the light of this fact, the characterization of graphs which satisfy the  equality becomes a problem of notable importance. Moreover,
the problem is interesting not only because on its own but also because of
the equivalence :

\begin{center}
a graph  satisfies the equality  if and only if .
\end{center}

Though, the calculation of  is -hard in general [4], the
Ford-Fulkerson algorithm for finding a maximum flow in a network implies
that it is indeed polynomial-time calculable for bipartite graphs. And, once
we are given a bipartite graph  satisfying the equality , we can calculate  easily.This is important, since  remains 
-hard calculable even for connected bipartite graphs  with maximum
degree three [5]. Let us also note that there is a polynomial algorithm
which constructs a maximum matching  of a tree  such that  (to be presented in [6]).

The aim of present paper is the characterization of trees that satisfy the  equality. An early result in this direction is given in [8]:
for every matching covered tree  the equality 
holds (a graph  is referred to be matching covered if its every edge
belongs to a maximum matching of the graph [7, 9], complete characterization
of those trees can be found in [2,3]). The characterization given in the
paper is constructive, more specifically, we define four operations, with
the help of which we prove that a tree  satisfies the equality  if and only if it can be built from  or  (the trees
containing one or two vertices, respectively) by using these operations. Our
proof is based on a new decomposition algorithm obtained for the class of
trees.

Non-defined terms and concepts can be found in [1, 7, 12].

\bigskip

\begin{center}
\textbf{Some auxiliary results about }\textbf{\ }
and 
\end{center}

\bigskip

\textbf{Lemma 1} Let  be a graph,  be a vertex with , and  be the edge incident to it. Then

\begin{tabular}{ll}
1. [8] & There is , such that . \\ 
2. [6] & There is , such that .\end{tabular}

\textbf{Lemma 2 }[8]. Let  be a graph,  satisfying the conditions:    for  (fig
1). Then the following is true:

\begin{center}
, .

\FRAME{ftbpF}{5.348in}{3.0018in}{0in}{}{}{figure 1.jpg}{\raisebox{-3.0018in}{\includegraphics[height=3.0018in]{Figure1.JPG}}}

Figure 1\bigskip
\end{center}

\textbf{Lemma 3}. Let  be a graph and let . Then

\begin{tabular}{ll}
(1) & ; \\ 
(2) & if  and then  and ; \\ 
(3) & if  and then .\end{tabular}

\textbf{Lemma 4}. Let  be a connected graph,  be a bridge of , and
let ,  be the connected components of . Then

\begin{tabular}{ll}
(1) & ; \\ 
(2) & if there is  with then  and \\ 
& ; \\ 
(3) & if there is  with  then 
.\end{tabular}

\textbf{Proof}. (1) Choose ,  with  ( (1) of lemma 1). Define:

\begin{center}
,

.
\end{center}

Clearly,  and  are disjoint, and

\begin{center}
.
\end{center}

(2) Note that  and \ are pairs of disjoint matchings in  and , respectively. Hence

\begin{center}


,
\end{center}

therefore

\begin{center}
.
\end{center}

Note that this and lemma 1 imply that

\begin{center}
.
\end{center}

(3) (2) implies that

\begin{center}
 and ,
\end{center}

hence

\begin{center}
, or

.
\end{center}

The proof of lemma 4 is completed.

\textbf{Lemma 5 }[6]. Let  be a connected graph,  be a bridge of ,
and let ,  be the connected components of . Then

\begin{center}
.
\end{center}

\bigskip

\begin{center}
\textbf{The main result}
\end{center}

\bigskip

\ In this section we introduce four elementary operations. They have the
property of preserving the equality , that is, if the graph
satisfies the equality then so does the graph obtained from original one by
the application of any of them. In the end of the section we prove that the
tree  satisfying  can be built from  or  by using only these operations.

\bigskip

\ \textbf{Operation A}. Let ,..., () be different
vertices of a graph . Consider the graphs  and  obtained from  in the following way (figure 2):

\begin{center}
\FRAME{ftbpF}{5.348in}{3.0018in}{0in}{}{}{figure 2.jpg}{\raisebox{-3.0018in}{\includegraphics[height=3.0018in]{Figure2.JPG}}}

Figure 2\bigskip
\end{center}

\ Since there are 
and  such
that   and , we imply that (lemma 3)

\begin{center}

\end{center}

Note that the following equalities are also true [6]:

\begin{center}

\end{center}

Hence

\textbf{Lemma 6}. Either the graphs , , 
satisfy the equality  or none of them does.\bigskip

Now, we proceed to the definitions of the three other operations. In
contrast with operation A, these ones are not always defined. This is the
main reason why the description of each operation is preceded by the
description of the cases when the operation is applicable.

\bigskip

\textbf{Operation B}.

\textbf{Definition 1}. A vertex  of a graph  is referred to be
applicable for the operation B if either  or there is  satisfying the conditions:

\begin{tabular}{ll}
(a) & ; \\ 
(b) &  for ; \\ 
(c) & ,  (figure 3)\end{tabular}

\begin{center}
\FRAME{ftbpF}{5.348in}{3.0018in}{0pt}{}{}{figure 3.jpg}{\raisebox{-3.0018in}{\includegraphics[height=3.0018in]{Figure3.JPG}}}

Figure 3\bigskip
\end{center}

If  is a graph, and  is an applicable vertex for operation B, then  (the result of operation B) is defined as follows (figure 4):

\begin{center}
\FRAME{ftbpF}{5.348in}{3.0018in}{0in}{}{}{figure 4.jpg}{\raisebox{-3.0018in}{\includegraphics[height=3.0018in]{Figure4.JPG}}}

\bigskip Figure 4
\end{center}

\textbf{Lemma 7}.  if and only if .

\textbf{Proof}. First of all note that .
The statement is true if . Assume that . Then

\begin{center}
,
\end{center}

and due to (1) of lemma 1 and (3) of lemma 4

\begin{center}
.
\end{center}

This shows that the statement of lemma 6 is true for the case of .

Therefore, we may assume that . Since  is applicable for
operation B, there is  satisfying the
conditions (a), (b), (c) of definition 1. Let  and , . Lemma 3 implies that to complete the proof it suffices
to show that there is , such that , or .

Choose any , and assume that . Without loss of generality,
we may assume that  and . We claim that .
Suppose that . Define:

\begin{center}
, .
\end{center}

Note that

\begin{center}
 and ,
\end{center}

which is impossible. Thus . Define:

\begin{center}
.
\end{center}

Note that  and . The proof of lemma
7 is completed.

\bigskip

\textbf{Operation C}.

\textbf{Definition 2}. A vertex  of a graph  is referred to be
applicable for the operation C if either

(1) there is  with

\qquad 
\begin{tabular}{ll}
(1a) & ; \\ 
(1b) & , , , \\ 
&  for ; ,  (figure 5a); \\ 
(1c) & , , where  and ;\end{tabular}

or

(2) there is  with

\qquad 
\begin{tabular}{ll}
(2a) & ; \\ 
(2b) & , , , \\ 
&  for ;  (figure
5b); \\ 
(2c) & , where , and .\end{tabular}

\begin{center}
\FRAME{ftbpF}{5.348in}{3.096in}{0in}{}{}{figure 5.jpg}{\raisebox{-3.096in}{\includegraphics[height=3.096in]{Figure5.JPG}}}

Figure 5a \ \ \ \ \ \ \ \ \ \ \ \ \ \ \ \ \ \ \ \ \ \ \ \ \ \ \ \ \ \ \ \ \
\ \ \ \ \ \ \ \ \ \ \ Figure 5b\bigskip
\end{center}

If  is a graph, and  is an applicable vertex for operation C, then  (the result of operation C) is defined as follows (figure 6):

\begin{center}
\FRAME{ftbpF}{5.348in}{3.0018in}{0in}{}{}{figure 6.jpg}{\raisebox{-3.0018in}{\includegraphics[height=3.0018in]{Figure6.JPG}}}

Figure 6\bigskip
\end{center}

\textbf{Lemma 8}. If  then .

\textbf{Proof}. Case 1: There is 
satisfying (1) of definition 2 (figure 5a).

Note that

\begin{center}
 and due to (*)

 , hence

, or .
\end{center}

Let  (figure 7).

\begin{center}
\FRAME{ftbpF}{5.348in}{3.0018in}{0in}{}{}{figure 7.jpg}{\raisebox{-3.0018in}{\includegraphics[height=3.0018in]{Figure7.JPG}}}

Figure 7\bigskip
\end{center}

We claim that there is no   containing the
edge . Assume the contrary, and let  contain the edge .

Due to lemma 5

\begin{center}
.
\end{center}

Choose a maximum matching   (lemma 1).
Note that . Define:

\begin{center}
.
\end{center}

Note that

\begin{center}
 and


\end{center}

which is a contradiction.

This implies that there is  
containing . Note that  (otherwise we would have an
augmenting path), therefore due to lemma 5

\begin{center}
.
\end{center}

On the other hand, lemma 2 implies that

\begin{center}
, hence

.
\end{center}

Case 2: There is  satisfying (2) of
definition 2 (figure 5b).

Note that

\begin{center}
 and due to (*)

 , hence

, and

.
\end{center}

Let  (figure 8).

\begin{center}
\FRAME{ftbpF}{5.348in}{3.0018in}{0in}{}{}{figure 8.jpg}{\raisebox{-3.0018in}{\includegraphics[height=3.0018in]{Figure8.JPG}}}

Figure 8\bigskip
\end{center}

Let us show that there is 
containing . Take any  , and
assume that . Note that

\begin{center}
 and  (otherwise we would
have an augmenting path).
\end{center}

Without loss of generality we may assume that . It is not hard
to see that

\begin{center}
 (lemma 5),

, and therefore

.
\end{center}

Let . Define  as follows:

\begin{center}
.
\end{center}

Clearly

\begin{center}
 hence , and

,
\end{center}

hence  and . Lemma 5 and (**) imply that

\begin{center}
.
\end{center}

Lemmata 2,4 imply that

\begin{center}
, hence

.
\end{center}

The proof of lemma 8 is completed.

\bigskip

\textbf{Operation D}.

\textbf{Definition 3}. A vertex  of a graph  is referred to be
applicable for the operation D if either

(1) there is  with

\qquad 
\begin{tabular}{ll}
(1a) & ; \\ 
(1b) & , ,  for  (figure 9a);\end{tabular}

or

(2) there is  with

\qquad 
\begin{tabular}{ll}
(2a) & ; \\ 
(2b) & , , , \\ 
&  for ;  (figure
9b);\end{tabular}

or

(3) there is  with

\qquad 
\begin{tabular}{ll}
(3a) & ; \\ 
(3b) & , ,  for  (figure 9c); \\ 
(3c) & , , where , and .\end{tabular}

\begin{center}
\FRAME{ftbpF}{5.348in}{3.0649in}{0in}{}{}{figure 9.jpg}{\raisebox{-3.0649in}{\includegraphics[height=3.0649in]{Figure9.JPG}}}

Figure 9a \qquad \qquad \qquad Figure 9b \qquad \qquad \qquad Figure
9c\bigskip
\end{center}

If  is a graph, and  is an applicable vertex for operation D, then  (the result of operation D) is defined as follows (figure 10):

\begin{center}
\FRAME{ftbpF}{5.348in}{3.0018in}{0in}{}{}{figure 10.jpg}{\raisebox{-3.0018in}{\includegraphics[height=3.0018in]{Figure10.JPG}}}

Figure 10\bigskip
\end{center}

\textbf{Lemma 9}. If  then .

\textbf{Proof}. Case 1: There is 
satisfying (1) of definition 3 (figure 9a).

Note that lemma 2 implies that

\begin{center}
,

, therefore

, or

.
\end{center}

Case 2: There is  satisfying (2) of
definition 3 (figure 9b).

From lemma 2 we have

\begin{center}
,

, therefore

, or

.
\end{center}

Case 3: There is  satisfying (3) of
definition 3 (figure 9c).

Note that

\begin{center}
,

,

 (lemma 4), hence

, or 
\end{center}

Let  (figure 11).

\begin{center}
\FRAME{ftbpF}{5.348in}{3.0649in}{0in}{}{}{figure 11.jpg}{\raisebox{-3.0649in}{\includegraphics[height=3.0649in]{Figure11.JPG}}}

Figure 11\bigskip
\end{center}

We claim that there is no 
containing the edge . On the opposite assumption, consider  with . Note that

\begin{center}
 and  (lemma 5).
\end{center}

Let . Note that . Define  as
follows:

\begin{center}
.
\end{center}

Clearly

\begin{center}
 and

,
\end{center}

which is a contradiction. This implies that there is   containing . Note that as  (otherwise we
would have an augmenting path), we imply that

\begin{center}
 (lemma 5).
\end{center}

On the other hand, lemma 2 implies that (see the definition of operation B)

\begin{center}
,
\end{center}

hence

\begin{center}
.
\end{center}

The proof of lemma 9 is completed.

\bigskip

\textbf{Theorem}. A tree  satisfies the equality 
if and only if it is either  or , or can be obtained from them
by the application of the operations A, B, C or D.

\textbf{Proof}. Note that\  and  satisfy the equality , and lemmata 6,7,8,9 imply that the operations A, B, C or D
preserve this property, that is, whatever tree  we build from  or  by these operations we will always have .

Let us show that the converse is also true, i.e. every tree  satisfying  can be built from  or  by A, B, C or D.

The proof is on induction. Clearly, the statement is true if . Assume that the statement is true for all trees  which satisfy the equality  and , and let us show that it also holds for the tree 
satisfying .

First of all note that we may always assume that there is no  with , ,  for . On the
opposite assumption, consider the set  comprised of vertices  satisfying these conditions. Set:

\begin{center}
.
\end{center}

The definition of operation B and lemma 4 imply that . The induction hypothesis implies that 
can be built from  or  by A, B, C or D, and since  can be
built from  by operation B, we are done.

Now let us show that we may also assume that there is no  with  for ; , , . If  is
such a set, then set:

\begin{center}
.
\end{center}

The definition of operation B and lemma 7 imply that  and therefore due to induction hypothesis,  can be built from  or  by A, B, C or D. As  is applicable for B and  is
built from  by applying B, we conclude that  can be built
from  or  by A, B, C or D.

Define:

\begin{center}
,
\end{center}

and for  let

\begin{center}
 where .
\end{center}

Consider a mapping  defined as:

\begin{center}
for  .
\end{center}

Note that for each vertex  there is at most one vertex  with 
\ and .

Since  is not a path, we imply that it contains a vertex of degree at
least three. Now, choose a vertex  satisfying the conditions:

\begin{center}
 and .
\end{center}

Note that the choice of  implies that there are paths  () of  satisfying the conditions:

\begin{center}
for every  , ,  and ;

 (figure 12).

\FRAME{ftbpF}{5.348in}{3.0018in}{0pt}{}{}{figure 12.jpg}{\raisebox{-3.0018in}{\includegraphics[height=3.0018in]{Figure12.JPG}}}

Figure 12\bigskip
\end{center}

We claim that without loss of generality we may assume that  and  are of length two for every vertex  satisfying the
conditions  and .

Note that every path from  is of length at most two. Now,
let us show that paths  may be assumed to have lengths
equal to two. Let  have a length equal to one, and let . Consider the trees  - the connected
components of . Note that (see
operation A)

\begin{center}
,
\end{center}

and since  we imply that , . Due to hypothesis of induction we conclude that , , can be built from  or  by A, B, C or D. Note
that since  is built from  by
operation A, we are done. This shows that the lengths of paths  may be assumed to be equal to two, and therefore we may
also assume that  for every vertex  satisfying the
conditions:  and .

As  is not a path and  we imply that for every
vertex  with  and 
there is a unique  such that 
and .

Now, choose a vertex  satisfying the conditions:

\begin{center}
,  and ,
\end{center}

where  is the abovementioned vertex corresponding to .

Note that the choice of  implies that  and . Let us show that we may also assume that . Suppose that , and let  be vertices shown in the figure below:

\begin{center}
\FRAME{ftbpF}{5.348in}{3.0018in}{0in}{}{}{figure 13.jpg}{\raisebox{-3.0018in}{\includegraphics[height=3.0018in]{Figure13.JPG}}}

Figure 13\bigskip
\end{center}

Let us show that there is  such that

\begin{center}
.
\end{center}

Choose  and, without loss of
generality, assume that , .
Define:  and  as follows:

\begin{center}
,  if ,

,  if .
\end{center}

Note that  and .

It is not hard to see that this implies that there is  such that . lemma 3 implies that

\begin{center}
,
\end{center}

hence the tree  also satisfies the  equality. Due to hypothesis of induction  can be built from  or  by A, B, C or D. Note
that  is obtained from  by operation B since
the vertex  is applicable for it. This shows that  can also be built
from  or  by A, B, C or D.

Thus, we may assume that . Let us show that we may
also assume that  is not adjacent to a vertex  with . On the opposite assumption, consider a vertex  satisfying
conditions:  and . Let 
be vertices shown in the figure below:

\begin{center}
\FRAME{ftbpF}{5.348in}{3.0018in}{0in}{}{}{figure 14.jpg}{\raisebox{-3.0018in}{\includegraphics[height=3.0018in]{Figure14.JPG}}}

Figure 14\bigskip
\end{center}

We claim that there is  with . Take any  with  (lemma 1), and suppose that . Note that one of the edges  and  does not belong to . Assume that . Since  we have . Define:

\begin{center}
.
\end{center}

Note that  and . This and lemma 3 imply that

\begin{center}
,
\end{center}

hence the tree  also satisfies the  equality. Due to hypothesis of induction  can be built from  or  by A, B, C or D.
Note that  is obtained from  by
operation D since the vertex  is applicable for it. This shows that 
can also be built from  or  by A, B, C or D.

Thus, we may assume that  is not adjacent to a vertex  with . Now, we claim that we may assume that there is no a vertex  such that

\begin{center}
,   and .
\end{center}

On the opposite assumption, consider a vertex  satisfying these
conditions, and let  be vertices shown in the figure below:

\begin{center}
\FRAME{ftbpF}{5.348in}{3.0018in}{0in}{}{}{figure 15.jpg}{\raisebox{-3.0018in}{\includegraphics[height=3.0018in]{Figure15.JPG}}}

Figure 15\bigskip
\end{center}

We claim that there is  with . Take any .

Case 1: . Note that one of the edges  and  does not belong to . Assume
that . Since  we have . Define:

\begin{center}
.
\end{center}

Note that  and .

Case 2: . Define ,  as
follows:

\begin{center}
,

.
\end{center}

Clearly,  and .

This and lemma 3 imply that

\begin{center}
,
\end{center}

hence the tree  also satisfies the  equality. Due to hypothesis of induction  can be built from  or  by A, B, C or D.
Note that  is obtained from  by
operation D since the vertex  is applicable for it. This shows that 
can also be built from  or  by A, B, C or D.

Thus, we may assume that  is not adjacent to another vertex  satisfying the conditions:

\begin{center}
,  .
\end{center}

It is not hard to see that there are paths  () starting from the vertex  and satisfying the conditions:

\begin{center}
for every  , ,  and .
\end{center}

Now, we will consider the remaining two cases:

Case 1: . Let  be vertices shown in
the figure below:

\begin{center}
\FRAME{ftbpF}{5.348in}{3.0018in}{0in}{}{}{figure 16.jpg}{\raisebox{-3.0018in}{\includegraphics[height=3.0018in]{Figure16.JPG}}}

Figure 16\bigskip
\end{center}

Assume:

\begin{center}
, .
\end{center}

We claim that there is no  containing the edge . Suppose there is. Note that

\begin{center}
,

 (lemma 5),
\end{center}

and since , we have

\begin{center}
,
\end{center}

contradicting lemma 2 which imples that

\begin{center}
.
\end{center}

This immediately implies that  and, consequently, . Let us show that . Suppose that . Choose . Since  we
have , therefore , hence

\begin{center}
 (lemma 5).
\end{center}

Choose , and define  as
follows:

\begin{center}
.
\end{center}

Note that  and

\begin{center}
, hence

 and ,
\end{center}

which is impossible.

Hence  and . Let us show that .
Note that

\begin{center}
,
\end{center}

hence

\begin{center}
 or .
\end{center}

(*) and (**) imply that

\begin{center}
,

,
\end{center}

we imply that , and therefore due to hypothesis of induction  can be built from  or  by A,
B, C or D. Note that  is obtained from 
by operation C since the vertex  is applicable for it. This shows that 
can also be built from  or  by A, B, C or D.

Case 2: . Let  be vertices shown in
the figure below:

\begin{center}
\FRAME{ftbpF}{5.348in}{3.0018in}{0in}{}{}{figure 17.jpg}{\raisebox{-3.0018in}{\includegraphics[height=3.0018in]{Figure17.JPG}}}

Figure 17\bigskip
\end{center}

Assume:

\begin{center}
, .
\end{center}

We need to consider two cases:

Case 2a: there is no  with .

First of all note that since there is a maximum matching of \  which does
not contain the edge , we have

\begin{center}
, therefore

 and .
\end{center}

Let us show that . Suppose that . Choose . Since  we have , therefore , hence

\begin{center}
 (lemma 5).
\end{center}

Choose , and define  as
follows:

\begin{center}
.
\end{center}

Note that  and

\begin{center}
, hence

 and ,
\end{center}

which is impossible.

Hence  and . Let us show that . Note that lemma 2 implies that

\begin{center}
,
\end{center}

hence

\begin{center}
 or .
\end{center}

As (see operation B, lemma 4)

\begin{center}
,

,
\end{center}

we imply that , and therefore due to hypothesis of induction  can be built from  or  by
A, B, C or D. Note that  is obtained from  by operation D since the vertex  is applicable for
it. This shows that  can also be built from  or  by A, B, C
or D.

Case 2b: there is  with .

Clearly,

\begin{center}
 and, due to lemma 5, .
\end{center}

Let us show that . Lemma 2 implies that

\begin{center}
,
\end{center}

therefore . (*) and (**) imply that

\begin{center}
,

,
\end{center}

therefore , and due to hypothesis of induction  can be built from  or  by A, B, C or D.
Note that  is obtained from  by
operation C since the vertex  is applicable for it. This shows that 
can also be built from  or  by A, B, C or D.

The proof of the Theorem is completed.

\bigskip

\textbf{Acknowledgement}. \ We would like to thank Hasmik Sargsyan for her
simplification of the operation B. We are also indebted to Vahe Musoyan and
Anush Tserunyan for their careful reading of the manuscript and for their
useful comments and suggestions that helped us to improve the paper.

\bigskip

\begin{center}
\textbf{References}\bigskip
\end{center}

[1] R. Diestel, Graph theory, Springer-Verlag Heidelberg, New York, 1997,
2000, 2005.

[2] F. Harary, Graph Theory, Addison-Wesley, Reading, MA, 1969.

[3] F. Harary, M.D. Plummer, On the core of a graph, Proc. London Math. Soc.
17 (1967), pp. 305--314.

[4] I. Holyer, The -completeness of edge coloring, SIAM J. Comput. 10, N
4, 718-720, 1981.

[5] R.R. Kamalian, V. V. Mkrtchyan, On complexity of special maximum
matchings constructing, Discrete Mathematics, to appear.

[6] R.R. Kamalian, V. V. Mkrtchyan, Two polynomial algorithms for special
maximum matching constructing in trees, manuscript.

[7] L. Lovasz, M.D. Plummer, Matching theory, Ann. Discrete Math. 29 (1986).

[8] V. V. Mkrtchyan, On trees with a maximum proper partial 0-1 coloring
containing a maximum matching, Discrete Mathematics 306, (2006), pp. 456-459.

[9] V. V. Mkrtchyan, A note on minimal matching covered graphs, Discrete
Mathematics 306, (2006), pp. 452-455.

[10] V. V. Mkrtchyan, V. L. Musoyan, A. V. Tserunyan, On edge-disjoint pairs
of matchings, Discrete Mathematics 2006, (submitted)

[11] A. V. Tserunyan, Characterization of a class of graphs related to pairs
of disjoint matchings, Discrete Mathematics 2006, (submitted)

[12] D. B. West, Introduction to Graph Theory, Prentice-Hall, Englewood
Cliffs, 1996.

\end{document}
