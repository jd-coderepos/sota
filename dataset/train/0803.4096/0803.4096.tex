\documentclass[]{iopart}
\usepackage{amstext,amsfonts,amssymb}
\usepackage{mathptmx}  \usepackage{graphics}

\newfont{\gl}{eufm10 scaled \magstep1} 

\newcommand{\dd}{\hbox{d}}
\newcommand{\RR}{\ensuremath{\mathbb R}}
\def\Ker{{\rm Ker}}
\def\Im{{\rm Im}}
\def\Rk{{\rm Rk}}
\def\ee{{\rm e}}
\begin{document}


\title{\bf{On the number of -cycles in the assignment problem 
for random matrices}}

\author{Jos\'e G. Esteve and Fernando Falceto  } 
\address{ Departamento de F\'{\i}sica Te\'orica, Facultad de Ciencias and \\
Instituto de Biocomputaci\'on y F\'{\i}sica de sistemas complejos.\\
Universidad de Zaragoza,
E-50009 Zaragoza (Spain)
}
\ead{esteve@unizar.es and falceto@unizar.es}
\date{\today}



\begin{abstract}


We continue the study of the assignment problem
for a random cost matrix. We analyse the number of -cycles
for the solution and their dependence on the 
symmetry of the random matrix. We observe that for a
symmetric matrix one and two-cycles are dominant
in the optimal solution. In the antisymmetric case 
the situation is the opposite
and the one and two-cycles are suppressed.
We solve the model for a pure random matrix
(without correlations between its entries) 
and  give analytic arguments to explain
the  numerical results in the symmetric and 
antisymmetric case. We show that the results can be
explained to great accuracy by a simple
ansatz that connects the expected number of
-cycles to that of one and two cycles. 


\end{abstract}
\pacs{02.60.Pn, 02.70.Rr, 64.60.Cn}


\section{Introduction}

The {\it assignment problem} (AP) for a given cost or {\it distance}\footnote{We use the term {\it distance} matrix although  are not 
necessarily true distances in a mathematical sense,
in particular they do not need to be positive or symmetric.}
matrix 
consists in finding the permutation  that minimises
the total distance . 


There are other problems related to this 
with additional constraints on the permutations allowed.
Probably, the most renowned one is the {\it traveling salesman problem}
(TSP) that can be formulated like the previous AP but
admitting only cyclic permutations 
(we insist that unlike in the standard TSP
our matrix does not need to be a true distance matrix).
The list includes also the {\it minimum weight simple matching 
problem} (SMP)
where only permutations composed of two-cycles are allowed
(obviously in this case  has to be even) and the, somehow opposite case
of  the {\it minimum weight directed  2-restricted 1-factor problem} (1FP), 
for which one-cycles and two-cycles are 
forbidden. If the matrix is symmetric the latter problem can be also 
seen as a {\it minimum weight non directed 2-factor problem} (2FP).
 
{}From the point of view of complexity theory,
it is well known (see \cite{Papa}-\cite{LP}) that the TSP is  NP-hard 
while the 2FP  the AP  and the SMP 
can be solved in a time the scales polynomially with .

In this paper we are interested in the study of the AP for random cost or
{\it distance} matrices.  
 This problem
 has been studied for many years,
 focusing mainly on the minimal 
 distance .
 For example, for random matrices whose entries have probability density 
  ( is the 
Heaviside step function), it was first conjectured by G. Parisi
 \cite{P1} and then proved rigorously (\cite{prueb1}-\cite{prueb3}) that
the expected length is 

 with   the number of points to be matched. Furthermore,
 for general random distances whose densities  behave like  
 
near , 
it is known (\cite{P21}-\cite{P4}) that

where  is the Riemann's zeta function. 

It is also 
known that  for the TSP on symmetric random matrices with ,
 the mean length of the minimal tour is (\cite{atp1},\cite{P5}) 
 
and the next  corrections are (\cite{atp2},\cite{atp3})


Different probabilistic relations 
among the problems considered in the previous paragraphs
are also well known in the literature. Namely, since the seminal work of Karp
\cite{Karp} we know that for purely asymmetric random matrices
with uniformly distributed entries we have

See also \cite{Frieze} and references therein for more precise
estimates of this convergence.

The case of symmetric random matrices is however different, and in this 
situation the expected length of the solution in the TSP and in the 
AP do not coincide in the large  limit. 
A different problem that has been shown to 
be closer to the TSP in probabilistic terms is 
the, above mentioned, 2FP where one-cycles 
and two-cycles are excluded. 
In ref. \cite{Frieze1} it is shown that the expected value 
of the minimal distance for TSP and 2FP with symmetric random matrix
coincides in the large  limit.
These results make clear that the structure of cycles in the optimal 
permutation for the AP depends strongly on the symmetry of 
the distance matrix and 
gives the clue to compare, at a probabilistic level, the different 
related problems.

 Actually, in a recent paper \cite{EstFal}, we found that 
 depending on the characteristics of the distance matrix the AP can interpolate
 between those situations which are near the 
 SM problem (in the sense that the optimal permutation is composed
 approximately of  cycles) and those whose 
 optimal permutation is composed of a few cycles 
(just one in some cases) and one and two cycles are absent. 
These can be considered {\it near} the TSP or 2FP solution.
 The transition between both limits is governed by 
 the correlation of the distances  and :
 for positive correlations the AP problem is in
 the ``SM regime'', whereas for anti correlated distances it is
 ``near'' the TSP regime. The transition point is located where there is no
 correlation between the entries 
 (that is all the distances are independent random variables), 
a situation that can be solved analytically as we shall see.


In this paper we shall study the expected number of -cycles in the optimal
permutation and its dependence on the symmetry of the distance matrix. 
We shall show  analytic and numerical results with special emphasis in the 
large   limit. In particular we put into relation the probability of a permutation
to be the solution of the AP with the number of one-cycles and two-cycles 
it contains. This ansatz can account for the numerical results
with high accuracy.

The paper is organised as follows. In the next section we 
describe the problem with full precision. The numerical 
results for the expected value of the number of -cycles
are presented in section 3. In the next three sections
we give analytic arguments to explain the numeric results
in the three regimes: the pure random case, the antisymmetric region 
and the symmetric one. We finally end the paper with some comments and 
conclusions.

\section{Description of the problem}

Given an  matrix  we are interested in the
permutation  that minimises the total {\it distance}

This problem is usually named as the {\bf assignment problem} or 
{\bf bipartite matching problem}.
The novelty of our approach is that rather than looking
at the minimum distance itself we focus on the permutation
 that gives this minimum. More concretely we are interested
in the number of -cycles, ,  in the 
permutation  (note that this numbers, determine the 
conjugacy class of  inside ).

{}From this point of view we shall consider equivalent those
matrices  whose minimum total distance corresponds to  
permutations in the same 
conjugacy class. This implies the following equivalence relation:


In this paper  is a random matrix that depends on a
constant , we sometimes denote it by ,
and it is constructed in the following way:
take a random   matrix  whose entries are equally 
distributed, independent, real  random variables with probability density
, then the entries of  are given by

Note that, unlike the others, the diagonal 
elements depend on a single random variable
and read .
Observe that  is symmetric for ,  
antisymmetric for  and purely random (without any correlation
among its entries) for .   

{}From the definition of  we have

and,  therefore  for 
 and  
for .

As it was mentioned before we are interested in
the number of -cycles  or rather
in its expected value in the distribution
generated by , we call it . 
We shall consider  that ranges from
the antisymmetric matrix for  
to the symmetric one for .
On the other hand, given the previous equivalence
( for ),
the results with  repeat
themselves for .
Then in an effective way we cover 
the whole positive real line.
For the negative part things are different
as we have  for ;
but, if the probability density for the entries of  is
such that  for some constant , then
the distribution of the optimal permutation with  
is again identical to the one for .

In the next sections we shall present the results for 
 and , where  is the 
total number of cycles in the optimal permutation.
It is interesting to observe how they change
with  from the antisymmetric point, ,
to the symmetric one, . 
Different values for the dimension  are considered
to study the large  limit.

We also vary the distribution  used to define the model.
We mainly focus on the uniform distribution between ,
with density , and on the exponential one, 
. 
Note that  and then, in this particular
case, the interval  for 
is enough to cover the whole
real line. On the other hand, as mentioned in the previous section, 
 has been extensively used in
studies of the assignment problem for random matrices 
\cite{P1},\cite{P5} which motivates our choice.

The two distributions considered in the previous paragraph
have the same limit for the density in the minimum
of its support . 
Many of the results obtained in the next sections hold independently 
of the distribution used to generate the random matrix provided 
its density function have a non 
zero limit in the minimum of its support. 
The same property is invoked in \cite{P1},\cite{P5}
to have a minimal distance with finite limit when 
goes to infinity.    

\section{Numerical results.}

We carried out a numerical simulation of the
statistical ensemble described in the previous section.
For that we generated between  and 
random instances for , using the corresponding
probability distributions for the elements . 
The number of instances depends
on the dimension of the matrix, which ranges from  to 
.

Once we generate the matrix  we solve the assignment
problem for it using the algorithm of R. Jonker and A. Volgenant
\cite{JV} and compute the number of -cycles  obtained in this way.
 In Fig. \ref{fig1} we plot the value of 
; 
there one can see the phase transition between the two regimes of 
 for  and . 
In the first case () the expected value of 
 behaves like  and is (almost) constant with . 
For  the values of  
grow linearly with  and 
\cite{EstFal}.


To understand the behaviour of  in both regimes
we analyse separately the average number of -cycles, , 
as a function of  and .
In the rest of the section we present the values
obtained in the numerical simulation.  In the following 
sections we shall give a theoretical explanation of these results.
 
\begin{figure}[h!]
\resizebox{130mm}{80mm}{
\includegraphics  {fig1.eps}}
\caption{\small Mean value of the number of cycles of the optimal solution for
  the assignment problem at different values of  and .
The dots and the joined plots are obtained with the distributions 
, 
 and
 respectively. }
\label{fig1}
\end{figure}


{\it i) One cycles:}
In the second plot (Fig. \ref{p1}) we show  as a function of  
for different values of the dimension . The dots
correspond to  for dimensions , , , 
 and . The joined plots represent the results for
 with ,  and . We show no error bars 
because these are negligible. 

We observe that  vanishes in all cases in the left part
of the diagram, it attains a common value  for 
 and finally it takes a value 
that grows like  for .
We finally note that the joined plots, corresponding 
to a different probability density ,
lay very close to their respective dots
(for ) and the fit 
gets better as  grows.

The scaling of  with  is shown 
 in the inset of Fig. \ref{p1} where  we 
plot   as a function of  
for different values of . 

\begin{figure}[h!]
{
\includegraphics{p1.eps}}
\caption{\small Average number of one-cycles in the optimal solution for
  the assignment problem at different values of  and . 
The dots are obtained with the uniform distribution with density 
, 
and the joined plots with
. Statistical errors corresponding to three standard 
deviation are not visible. The inset shows the behaviour of 
 as a function of  for
 different values of .}
\label{p1}
\end{figure}


{\it ii) Two cycles:}
In the next plot (Fig. \ref{p2}) we represent  versus .
As in the previous case we show it for different
values of the dimension and different distributions:
the dots correspond to  and the joined plots
 to . 

We again see that  vanishes near ,
takes the value  for 
and grows, in an approximately linear way,
in the symmetric region, , to a value
close to  for . We also observe that the
points corresponding to  fit very well with those
of the joined plot corresponding to . 
The inset shows the linear scaling of  with , 
for .

\begin{figure}[h!]
\includegraphics {p2.eps}
\caption{\small 
Average number of two-cycles (multiplied by ) in the optimal solution for
the assignment problem at different values of  and . 
The results obtained with the densities 
 () are displayed as 
points (joined plot) respectively. Statistical errors are negligible. 
The inset corresponds to 
 versus  for different values of .  }
\label{p2}
\end{figure}


{\it iii) Three cycles:}
The situation changes drastically when we plot 
as a function of  in
Fig. \ref{p3}.
The dots correspond to , 
and  for . The joined plot represents the case
of dimension  with . 

We see that  gets a constant value equal to 
for almost all values of 
and all values of  and . Only near 
things depend on  and as  grows
the value of  tends to .  
This limiting behaviour is common for all 
probability densities .
\begin{figure}[h!]
\resizebox{130mm}{80mm}{
\includegraphics  {p3.eps}}
\caption{\small Average number of 3-cycles (multiplied by 3) 
in the optimal solution for
the assignment problem.
Symbols correspond to 
 and different values of  and the joint plot
is for  and .
The error bars correspond to 
three standard deviations from the mean.
}
\label{p3}
\end{figure}


Similar results are obtained for other odd cycles
of small length compared to   i.e.  or 
are equal to  for all values of  except
near , but it tends to  everywhere when 
tends to infinity.

{\it iv) Four cycles.}
In the next plot (Fig. \ref{p41}) we represent the behaviour of four cycles
plotting  versus  for different values of  and 
. Dots represent the values obtained for different
dimensions ,  and , 
all with the uniform distribution, with density . 
The joined plot corresponds to  with .
\begin{figure}[h!]
\resizebox{130mm}{80mm}{
\includegraphics  {p4.eps}}
\caption{\small Mean value of  4-cycles (multiplied by 4) in the optimal 
solution for the assignment problem.
Symbols correspond to 
 and different values of  and the joint plot
is for  and .
Error bars represent three standard deviations.}
\label{p41}
\end{figure}


Comparing with the previous plot of , we see no change in the left part,
. However the right half is quite different.
We observe that  always vanishes at the symmetric point, 
and it follows a smooth curve (even in the large  limit)
from  at  to  for . 
A similar result is obtained for other {\it short} cycles of even length
like , , ...:
all of them vanish at , only the shape of the curve
changes, it is more horizontal near  
and steeper as we approach the symmetric point.

{\it v) Intermediate cycles:}
In the Fig. \ref{pk} we show the cycles of intermediate length for dimension 
and the density .
As an example we draw  for  and .
We see that, as in previous cases, the behaviour for  is 
always constant and equal to . For positive  we see
a fast transition from  to  at a value for  that diminishes
as  increases. Other intermediate values of  and different values of  or  
give similar results (see also Fig. \ref{kpkteo} for odd values of ).
\begin{figure}[h!]
\resizebox{130mm}{80mm}{
\includegraphics  {kpk.eps}}
\caption{\small Average number of -cycles (multiplied by )
in the optimal solution for the assignment problem at different 
values of  and  for . 
Error bars represent three standard deviations. }
\label{pk}
\end{figure}


{\it vi)  cycles:}
In the Fig. \ref{pnm1} we draw  for 
and . We see a peak, sharper as  increases while
its maximum moves toward . It always takes
the unit value at .
As before, different distributions give similar results.
This plot, as well as those of  and 
which are plotted in the Fig. \ref{teornm2} 
are qualitatively very different from the previous ones
and also different from each other.
In section 5 we shall introduce a simple ansatz that accounts for this,
with great accuracy.
\begin{figure}[h!]
\resizebox{130mm}{80mm}{
\includegraphics  {pnm1.eps}}
\caption{\small Average number of ()-cycles (times ) 
in the optimal solution for
  the assignment problem at different values of  and  .  Error bars represent three standard deviations. Note the common value  
 at  for all values of .}
\label{pnm1}
\end{figure}


{\it vii)  cycles:}
Finally in the Fig. 8 we present the results for  for different 
dimensions.
Note that it is again constant near 
but, contrary to the previous cases, the constant is not  but rather
. It takes the value 
 for  and vanishes for . The width 
of the transition is inverse proportional to . Different distributions 
give similar results. 
\begin{figure}[h!]
\resizebox{130mm}{80mm}{
\includegraphics  {pn.eps}}
\caption{\small Average number of -cycles (times ) in the
 optimal solution for
  the assignment problem at different values of  and  .
  Error bars represent three standard deviations.}
\label{pn}
\end{figure}


To summarise the results of this section
we have that for small cycles,
with odd ,
 for all  in the large  limit.
Small cycles with even  have a smooth decay to  at . 
For cycles of intermediate length 
 from  until it has an abrupt 
decay at a positive value of 
that depends on . 
Cycles of length close to  have a very different behaviour
one from each other. 
And finally, one and two
cycles are absent for  and grow
like  and  respectively for .


\section{Solution of the model for .}

We start with the theoretical study of the model
by analysing the point . In this case
 and the entries of our matrix
are identical, independent random variables.
Due to this fact we can show that all permutations  
have the same probability of giving rise to the minimal 
distance. 

The proof is very simple.
Given  call 
 
the probability distribution in the space of matrices
for .
It is then clear that

for any permutation .
But if  is the permutation that minimises
the distance  for  then
 gives the minimum distance for
. It implies then that  and 
have the same probability of being the optimal permutation,
which leads to the uniform distribution in 


Once we have established that at 
all permutations have the same probability
our problem is a purely combinatorial one,
and reduces to compute how many -cycles
there are in . This number, that we call , 
is well known to be 

as one can derive from simple counting arguments, i. e.
 where
the different factors count repectively the possible
choices of  indexes to form the cycle, their orderings
and the permutations of the rest of indexes. Note that in this 
way every permutation is counted as many times as the number of
-cycles it contains, hence the result follows. 
\footnote{We can also use the following iteration 
.
The first term in the iteration
counts the number of -cycles that
persist when one add a new index
while the second term stands for  the number of ways
one can add a new index to a -cycle to make it 
one unit larger. The  is there because
for one-cycles, when adding a new index linked to itself 
rather than to any of the preexisting ones, the number of one-cycles 
is increased by one}.



For latter purposes we shall present here
a different, more cumbersome, way to derive  that makes use of the 
generating function \cite{aSt},\cite{Riordan}. Let

be the generating function for the number of -cycles in 
in the sense that

But we rather want to compute the number of -cycles in .
To do this we observe that the generator for the
permutations in  are obtained by simply taking the exponential
 
The procedure to obtain the number
of -cycles in  is then simple. 
We introduce

so that when we take the exponential of 
the power of  in every term indicates the number
of -cycles that the corresponding permutation contains.
Therefeore,  is given by


The expected number of -cycles for
 is then

Note that this result is independent of  and of the probability 
density  we used to generate the ensemble. 
This explains why in all the results
showed in the previous section  for . Finally,
 the expected value of
 is:

where  is the Harmonic series


\section{The antisymmetric region, .}

In this section we study the behaviour
of   for . We start by the observation
that one-cycles and two-cycles are strongly suppressed
for . The absence of one and two-cycles in the 
solution of the AP makes it equivalent to the corresponding 
1FP as it was mentioned in the introduction. 

This fact can be heuristically understood if one considers that 
the optimal permutation for  comes from the choice 
of  elementary distances  out of  and, apart from the 
diagonal elements which are , the rest of elements are half of them negative
and half of them positive.
Then, for large , the shortest total distance will be typically
obtained when we chose only negative elements and this excludes
the possibility of having one-cycles ()
and two-cycles () that always include 
non negative entries.
The rest of cycles have no correlation
among their elements and therefore it is
reasonable to assume the equiprobability of all
permutations that do not contain one-cycles or 
two-cycles. With this assumption we reduce the problem to a combinatorial
one and we can proceed like in section 3.

Our goal, however, is to understand the expected number of -cycles
in the whole negative region  that interpolates
between the absence of one and two cycles for  
to the expected values  and  at .
This goal can be achieved with the following ansatz.
We assume that, at least in the large  limit, the probability for a 
permutation to be the 
shortest distance depends only on the number of one-cycles and 
two-cycles it contains.
This is consistent with the fact that only one and two cycles
are sensible to the symmetry of the matrix, bonds of longer
cycles are uncorrelated.
Namely for a permutation with  one-cycles and
 two-cycles the probability is proportional
to , where  and  vanish
for  and  for .
The new generating function is then:

That implements the idea outlined above, as in the exponential 
of  every term has a weight
.
{}From this we derive the normalising factor
(the total weight of the space of permutations)

while the expected value for the number of -cycles can be 
obtained as in previous section by introducing the factor 
multiplying  and taking the derivative of the exponential 
at .
The result for  is

To compute these quantities we use the singularity analysis
approximation \cite{Flajolet}. In the case at hand the  
coefficient in the power series is approximated by the residue at 
the pole in . It then gives

and 

For small values of  and  (compared with ) this approximation 
can be used and we obtain  for , which is compatible 
with the numerical results of section 3, (see figures 4, 5 and 6). 

 and  do not follow the general formula but
 
which, in the singularity analysis approximation, gives

And 
 
so that

Then for small values of  and  and the values of
 we are considering in the paper (from  to )
we can take  and  with a very good 
accuracy (that covers the  region since there  and 
 are less than ).

For long cycles  the singularity analysis approximation 
is not valid any
more. In this case, however, it is very easy to compute (\ref{kcycles}) 
explicitly.
Therefore with the precision given by that of 
 we get:


\begin{figure}[h!]
\resizebox{130mm}{80mm}{
\includegraphics {teor.eps}}

\caption{\small Average number of  (for the largest values of ) 
in the optimal solution for
  the assignment problem at different values of  and for .
  The points are the result of our simulation and the error bars represent
 three standard deviations from the mean. The joined plot is the theoretical
 prediction  using  (\ref{longcycles}).}
\label{teornm2}
\end{figure}


In Fig. \ref{teornm2} we plot  for  and  . The continuous
line is the theoretical value obtained from (\ref{longcycles})
where we take  and . One can see
that the agreement is excellent. A similar match holds for the 
other cases.

Thus, from the previous expressions
we see that the behaviour of  for for   
for 
is completely determined by  and . In the rest of the section 
we shall study the behaviour with  and  of this two factors.
Many of the results presented below
are independent on the distribution 
used to generate the random matrices, provided
the probability density fulfils the non vanishing
property in the minimum  of its support that was discussed
in section 2. In the rest of the paper we shall assume that this property
holds.

Our first observation is the relation between  and 
 for the same value of  and .
One can check that .
A plot showing the extremely good fit between the two values as a function of
 for  and different  is shown in Fig. \ref{q1q2}.
This relation can be expressed as the fact that the probability of
a permutation to produce the minimal total distance,
is unchanged if we change the permutation by substituting 
a two-cycle by two one-cycles. 
An argument for this comes from the fact
that given two indexes  and ,  while 
. Then both
sums are identical random variables.
\begin{figure}[h!]
\resizebox{130mm}{80mm}{
\includegraphics  {q1cuadq2.eps}}
\caption{\small Values of  versus  for  and
 , . The continuous plot is the line .}
\label{q1q2}
\end{figure}


\begin{figure}[h!]
\resizebox{130mm}{80mm}{
\includegraphics  {q1.eps}}
\caption{\small The points in the upper curve represent the  values of  
as a function of  for different values of 
 and  and for matrices generated with probability density . The 
 tangent line at  is the  theoretical prediction given by
 (\ref{slope}). The lower curve is the same
 but for matrices generated with the exponential density.}
\label{q1}
\end{figure}
The second important property we observe in the region  
is the invariance under scaling of  and  (see Fig. \ref{q1}). In fact 
one can check that for a given probability density , 
.
And as any  can be obtained
from  according to the formulae above, this 
scale invariance is true also for any .

The scaling relations presented in the previous paragraph are obtained  
by taking a fixed probability density 
to generate the ensemble, while we change 
 and . We want to examine now how
 depends on the distribution near
the random point . Given the result that we can 
rescale  and  without changing 
it is natural to think that  can be determined by 
looking at only a few elements of the matrix .
A confirmation of this conjecture is not available yet, but
some partial results can be verified. Concretely
we can reproduce the slope of  at ,
that depends on the distribution,
by the following formula:  

Where  depends solely on the distribution and is 
determined as follows:\hfill\break
for a given value of lambda fix  
and define , 
also define . Now 
compute 

where with  we denote the Heaviside step function. The
coefficient  is obtained by
 
As we mentioned before the value of  depends only on the 
probability density  and can be computed with the following formula

The meaning of  is the following: it measures the probability
for an extra diagonal element of a pair to be  smaller than its pair
and than two entries in the diagonal.  
It, somehow, reproduces at a small scale (only four random variables 
involved) the mechanism for the disappearance of one-cycles 
(diagonal entries) in the real problem as  starts to be 
negative. 
Recall that the argument for the disappearance of one and two-cycles
was based in the fact that for negative 
one of every pair of extra diagonal terms is smaller
(in average) than the diagonal terms
(or than half the sum of the extra diagonals).
It then implies that the appearence of one and two-cycles 
in the optimal permutation is disfavoured. This property is 
quantitatively studied by means of the function .

Our result has been checked with different distributions and
the agreement is very good. As an example we show in Fig. \ref{q1} 
the lines for  and  with slope
 and  respectively, as obtained from (\ref{slope}). 
We can see that these lines are, as predicted,
tangent to the curve of  at .

\section{The symmetric region }
As shown in Fig. \ref{p1} and \ref{p2}, the first relevant 
fact in this region
is that  and  grow from  and  respectively 
for , to values proportional to  
in the first case and to  in the second for .
A first attempt to account for this behaviour
is to adjust the corresponding parameters
 and  to fulfil equations (\ref{eqPuno}) and 
(\ref{eqPdos}), (note that now  and  can be  so the terms
of order  and  can be important).
 The values of  and  obtained
in this way are used to compute  for 
different values of . 

\begin{figure}[h!]
\resizebox{130mm}{80mm}{
\includegraphics  {p3teo.eps}}
\caption{\small Numerical value of  (dots) and the theoretical prediction using
 equations (\ref{eqPuno}) and 
(\ref{eqPdos}) with the corrected values of  and  (continuous line) and without the
 corrections (discontinuous line).}
\label{p3teo}
\end{figure}


This procedure, however, fails to predict the numerical
results in two different aspects.
First, if we try to fit  we obtain a large deviation
with respect to the numerical value near the symmetric point.
This is shown in Fig. \ref{p3teo} where the dots represent the 
numerical value and the dashed line represents 
the theoretical prediction obtained as outlined above. 
Also the disappearance of even cycles at ,
as shown in fig. \ref{p41},
is not taken into account within this approximation
i. e. the theoretical value for  does not vanish at .
These two facts happen to be connected and will be 
discussed in the next paragraph.

Is is first important to understand why even cycles
disappear when . 
The reason is very simple, for if we had a
cycle of even length i. e.  ,
with  except , 
then either the links in odd position
 
or those in even position 
 
have a smaller sum. 
Assume that

then the new permutation  which is equal to  except
for  gives a smaller
total distance. To see this, it is enough to realise that, 
given that  is a symmetric matrix, the sum of the odd links 
for  is replaced by that of the even links in 
which lowers the total distance. Hence it is impossible
to have cycles of even length larger than two, in the optimal permutation
of a symmetric {\it distance} matrix.

The mechanism for disappearance of even cycles we
outlined in previous paragraph can be stated by saying that
-cycles break into  two-cycles. This is the key point
behind the improvement of the approximation  
in order to account for small cycles.
The idea is that in equations (\ref{eqPuno}) 
and (\ref{eqPdos}) instead of using the value of 
obtained in the numerical simulations we subtract to it
the two-cycles that come from what would be cycles of even length.
The procedure is then clear: we start with a value for  and ,
say  and , we compute with this values the theoretical
number of cycles of even length and subtract from it the real one
obtained in the numerical simulations. These are the cycles that break
into a number of two cycles. We subtract this number from , 
introduce the new value of  into equation (\ref{eqPdos})
and compute again  and . The procedure is iterated until the
desired convergence is reached. In practise in 4 or 5 iterations we obtain
a very good precision.

\begin{figure}[h!]
\resizebox{130mm}{80mm}{
\includegraphics  {kcic-teo.eps}}
\caption{\small Values of  for intermediate values of ,
  and . The continuous line is the theoretical prediction. }
\label{kpkteo}
\end{figure}


In Fig. \ref{p3teo} we plot the numerical values for 
(dots) and  the theoretical curves using the uncorrected version 
for  (dashed line) 
and the corrected ones (solid line). 
We see that the fit is much better in the second instance. 
The theoretical prediction can be also applied to 
the intermediate cycles as shown in Fig. \ref{kpkteo}.
The theoretical 
and numerical values for  with 
using the corrected  and , show a very good 
agreement.

\begin{figure}[h!]
\resizebox{130mm}{80mm}{
\includegraphics  {q1q2+.eps}}
\caption{\small Values of  versus  for positive  
. The plot includes the points obtained for 
  and with the probability density   and . }
\label{q1q2+}
\end{figure}


Our last point is the relation between  and  that 
extends for positive values of  the fit shown in
fig. \ref{q1q2}.
We find that the dependence changes in this case. 
A very good fit is obtained by taking 

 As it is shown in fig. \ref{q1q2+} the agreement is rather good and 
it gets better in the large  limit.
 
\section{Conclusions and outlook.}

The expected number of -cycles in the optimal 
permutation of the assignment problem for random matrices,
can be understood to great accuracy in terms of only
two parameters,  and  associated to one and two-cycles.
More precisely, the ansatz is that in the large  limit
the probability for a permutation to be the solution of the  
is proportional to , with  the number
of one and two-cycles of the permutation respectively. 
The ansatz can be substantiated by considering that with the cost or 
{\it distance} matrices used in the paper only one and two-cycles
are sensible to the symmetry of the matrix, as bonds of longer cycles are 
uncorrelated. On the other hand in the large  limit we can consider
the occurrence of short cycles as independent events.  

With this ansatz we are able to explain, with great accuracy,
the expected number of -cycles in the solution of the AP for
cost matrices ranging from the symmetric to the antisymmetric one.  
The parameters suffer an abrupt 
transition (in the large  limit)
when moving from a matrix mostly symmetric ()
to another one mostly antisymmetric ().

We also find some universal scaling relations in the variables
which are valid in the antisymmetric region.
Based in this scaling behaviour we are able to give a 
theoretical prediction for the slope of  
at the critical point, .

An open problem is to understand the behaviour of the cycles of even 
length in the symmetric region. It is clear that, as it is argued
in the paper, all of them  (except the two cycles) should be absent 
at the symmetric point 
(), but for the moment we do not know
how to explain the curves that the average number 
of even cycles follow to reach the zero value.
Finally, it would be nice to have a full theoretical study of the model
(or a reliable approximation to it) that could 
explain the facts mentioned above.

\noindent{\bf Acknowledgements:} Research partially supported by
grants FIS2006-01225 and FPA2006-02315, MEC (Spain).

\section*{References}
\begin{thebibliography}{99}
\bibitem{Papa} {C. H. Papadimitriou and M. Yannakakis,
 The traveling salesman problem with distances one and two.
 {Mathematics of Operations Research}, 18:1-11,1993.}

\bibitem{Hart}{D. Hartvigsen, Extensions of Matching Theory
, Phd. thesis, Department of Mathematics, Carnegie Mellon University, Pittsburgh, PA, 1984}

\bibitem{JV} {R. Jonker  and  A. Volgenant,
  A Shortest Augmenting Path
      Algorithm for Dense and Sparse Linear  Assignment Problems,
 { Computing } {\bf 38}, 325 (1987).}

\bibitem{LP}{L. Lov\' asz and M. D. Plummer, Matching Theory, North-Holland Publishing Co., Amsterdan, 1986}

\bibitem{P1}{ G. Parisi, A conjecture on random bipartite matching, 
 cond-mat/9801176 (1998).}

\bibitem{prueb1} { V. J. Dotsenko,
Exact solution of the random  bipartite matching model. 
{ J. Phys. A} {\bf 33}, 2015 (2000).}

\bibitem{prueb0} {D. J. Aldous,
The  Limit in the Random Assignment Problem.
 Random structures Algorithms {\bf 18}, 381 (2001).}

\bibitem{prueb2}{S. Linusson, and J. W\"astlund,
  A Proof  of Parisi's Conjecture on the Random
 Assignment Problem. http://arXiv.org/abs/math/0303214  (2003).}

\bibitem{prueb3}
{ C. Nair, B. Prabhakar and Sharma M.,  A Proof of the
 Conjecture due to Parisi for the finite Random Assignment Problem,
available at http://www.stanford.edu/~balaji/rap.html.}

\bibitem{P21}
{ M. M\'ezard and G. Parisi,
Replicas and optimization. 
{ J. Physique Lett.} {\bf 46}, L-771 (1985).}

\bibitem{P22}
{ M. M\'ezard and G. Parisi,
Mean-Field Equations for the Matching
 and the Travelling Salesman Problems.
 { Euro. Phys. Lett.} {\bf 2}, 913 (1986).}

\bibitem{P23}
{ M. M\'ezard and G. Parisi,
 The Euclidean Matching Problem.
 { J. Physique} {\bf 49}, 2019 (1988).}

\bibitem{P3}{R. Brunetti, W. Krauth, M. M\'ezard and G. Parisi,
 Extensive Numerical Simulations
 of Weighted Matchings: Total Length and Distribution of Links in the
 Optimal Solution.
 { Euro. Phys. Lett.} {\bf 14}, 295 (1991).}

\bibitem{P4}
{G. Parisi, M. and Rati\'eville,
On the finite size corrections to some random matching problems.
 { Euro. Phys. J. } {B29}, 457 (2002).} 

\bibitem{atp1}{W. Krauth and M. M\'ezard,
The cavity method and the travelling salesman problem.
 Europhys. Lett. {\bf 8},213 (1989).}

\bibitem{P5}{G. Parisi,
  Constrain optimization and statistical physics.
Lectures given at the Varenna summer school. arXiv: cs.CC/0312011.}

\bibitem{atp2}{A. Percus, The Traveling Salesman 
and Related Stochastic Problems, PhD thesis 2007
 cond-mat/9803104, Apendix E. Note that in this article 
  and consequently they obtain a value for  which 
is half of that of the reference \cite{atp1}}

\bibitem{atp3}{N. J. Cerf, J. Boutet de Monvel, O. Bohigas, O. C. Martin and A. Percus,
The Random Link Approximation for the Euclidean Traveling Salesman Problem.
J. Phys. France, {\bf 7},117 (1997). }

\bibitem{Karp} {R. M. Karp, A pattching algorithm for the non-symmetric traveling salesman problem
SIAM Journal on computing {\bf 8}, 561 (1979).}

\bibitem{Frieze}{A. M. Frieze and G. Sorkin, The probabilistic relationship between the 
assignment and asymmetric  traveling salesman problems, Proceedings of the 15th Anual
ACM-SIAM Symposium on Discrete Algorithms, Baltimore MD (2001),652-660.}

\bibitem{Frieze1}{A. M. Frieze, On Random symmetric traveling salesman problems, FOCS2002, 789, (2002). and 
Mathematics of Operations Research, {\bf 29}, 878 (2004).}

\bibitem{EstFal}{J. G. Esteve and F. Falceto,
Phase transition in the assignment problem for random matrices.
 EuroPhys. Lett. 
{\bf 72},691 (2005). }



\bibitem{aSt}{L. Comtet,
  Advanced Combinatorics: The Art of Finite and
      Infinite Expansions, rev. enl. ed.  Dordrecht, Netherlands: Reidel,
    1974, pag 256.}

\bibitem{Flajolet}{P. Flajolet and R. Sedgewick,
   Analytic combinatorics, A preliminary version is available at:
 http://algo.inria.fr/flajolet/Publications/books.html}

\bibitem{Riordan}{J.  Riordan,
  An Introduction to 
Combinatorial Analysis. New York: Wiley, 1980,
pag 75.}



\end{thebibliography}
\end{document}
