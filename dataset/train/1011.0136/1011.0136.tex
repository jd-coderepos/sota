\documentclass{llncs}
\usepackage{amsmath}
\usepackage{amssymb}
\usepackage{graphicx,xspace,tikz}
\usetikzlibrary{arrows,automata}

\newcommand{\Michel}[1]{{\bf #1}}
\newcommand{\leaveout}[1]{}

\newcommand{\sosrule}[2]{\frac{\raisebox{.7ex}{\normalsize{}}}
                        {\raisebox{-1.0ex}{\normalsize{}}}}
\newcommand{\trans}[1]{\,{\stackrel{{#1}}{\rightarrow}}\,}
\newcommand{\ltsntrans}[1]{\,{\stackrel{{#1}}{\nrightarrow}}\,}
\newcommand{\ltstrans}[1]{\xrightarrow{#1}}
\newcommand{\kstrans}{\to}

\newcommand{\bisim}{\mbox{}}
\newcommand{\bbisim}{\mbox{}}
\newcommand{\dsbbisim}{\mbox{}}
\newcommand{\dbstuttering}{\,\approx_{\mathrm{dbs}}\,}
\newcommand{\stuttering}{\,\approx_{\mathrm{s}}\,}

\newcommand{\lts}{\mathsf{lts}}
\newcommand{\ltsrev}{\lts^{-1}}
\newcommand{\ks}{\mathsf{ks}}
\newcommand{\ksrev}{\ks^{-1}}

\newcommand{\ie}{\emph{i.e.}}
\newcommand{\eg}{\emph{e.g.}}
\newcommand{\viz}{\emph{viz.}}
\newcommand{\etal}{\emph{et al}\xspace}

\def\lastname{Reniers \and Willemse}

\title{Folk Theorems on the Correspondence between State-Based and Event-Based Systems}
\author{Michel A. Reniers\inst{1} \and Tim A.C. Willemse\inst{2}
}
\institute{
Department of Mechanical Engineering, Eindhoven University of Technology,
\\ {P.O.~Box~513}, NL-5600~MB~~Eindhoven, The Netherlands
\and
Department of Computer Science, Eindhoven University of Technology,
\\ {P.O.~Box~513}, NL-5600~MB~~Eindhoven, The Netherlands
}


\thispagestyle{plain}
\pagestyle{plain}


\begin{document}
\begin{frontmatter}

\maketitle

\begin{abstract}
Kripke Structures and Labelled Transition Systems are the two most
prominent semantic models used in concurrency theory. Both models
are commonly believed to be equi-expressive. One can find many ad-hoc
embeddings of one of these models into the other. We build upon the
seminal work of De Nicola and Vaandrager that firmly established the
correspondence between stuttering equivalence in Kripke Structures
and divergence-sensitive branching bisimulation in Labelled Transition
Systems. We show that their embeddings can also be used for a range of
other equivalences of interest, such as strong bisimilarity, simulation
equivalence, and trace equivalence. Furthermore, we extend the results by
De Nicola and Vaandrager by showing that there are additional translations
that allow one to use minimisation techniques in one semantic domain to
obtain minimal representatives in the other semantic domain for these
equivalences.

\end{abstract}

\end{frontmatter}

\pagenumbering{arabic}

\newcommand{\KS}{\textsf{KS}\xspace}
\newcommand{\LTS}{\textsf{LTS}\xspace}

\section{Introduction}\label{Sect:intro}

Concurrency theory, and process theory in general, deal with the analysis
and specification of behaviours of reactive systems, \ie, systems that
continuously interact with their environment. Over the course of the
past decades, a rich variety of formal languages have been proposed
for modelling such systems effectively. At the level of the semantics,
however, consensus seems to have been reached over the models used to
represent these behaviours. Two of the most pervasive models are the
state-based model generally referred to as \emph{Kripke Structures}
and the event-based model known as \emph{Labelled Transition Systems},
henceforth referred to as \KS and \LTS.

The common consensus is that both the \KS and \LTS models are on
equal footing. This is supported by several embeddings of one model
into the other that have been studied in the past, see below for a
brief overview of the relevant literature.  As far as we have been
able to trace, in all cases embeddings of both semantic models were
considered modulo a single behavioural equivalence. For instance, in
their seminal work~\cite{DBLP:journals/jacm/NicolaV95}, De Nicola and
Vaandrager showed that there are embeddings in both directions showing
that stuttering equivalence~\cite{DBLP:journals/tcs/BrowneCG88}
in \KS coincides with divergence-sensitive branching
bisimulation~\cite{vanGlabbeek96} in \LTS. The embeddings, however,
look a bit awkward from the viewpoint of concrete equivalence relations.

On the basis of these results, one cannot arrive at the conclusion that
the embeddings also work for a larger set of equivalences. For instance,
it is very easy to come up with a mapping that reflects and preserves
branching-time equivalences while breaking linear-time equivalences, by
exposing observations of branching through the encodings.  Note that it is
equally easy to construct encodings that break branching-time equivalences
while reflecting and preserving some linear-time equivalences, \eg,
by including some form of determinisation in the embeddings.

Our contributions are as follows. Using the
\KS-\LTS embeddings  and  of De Nicola and
Vaandrager in~\cite{DBLP:conf/litp/NicolaV90}, in Section \ref{Sect:preservations} we formally establish the
following relations under these embeddings:
\begin{enumerate}
\item bisimilarity in \KS reflects and preserves bisimilarity in \LTS;
\item similarity in \KS reflects and preserves similarity in \LTS;
\item trace equivalence in \KS reflects and preserves completed trace equivalence
in \LTS.
\end{enumerate}
These results add to the credibility that indeed both worlds are on
equal footing, and it may well be that the embeddings  and 
are in fact canonical.

As already noted in~\cite{DBLP:conf/litp/NicolaV90}, there is no
immediate correspondence between the embeddings  and .
For instance, one cannot move between \KS and \LTS and back again by
composing  and . We mend this situation by introducing two
additional translations, \viz,  and , that can be used
to this end. Moreover, we show that combining these with the original
embeddings enables one to minimise with respect to an equivalence in
\KS by minimising the embedded artefact in \LTS (and \emph{vice versa}).

From a practical point of view, our contributions allow one to smoothly
move between both semantic models using a single set of translations.
This reduces the need for implementing dedicated software in one setting
when one can take advantage of state-of-the-art machinery available in
the other setting.

\paragraph{Related Work}

In their seminal paper (see~\cite{DBLP:journals/jacm/NicolaV95})
on logics for branching bisimilarity, De Nicola and Vaandrager
established, among others, a firm correspondence between the
divergence-sensitive branching bisimilarity of~\cite{vanGlabbeek96},
and stuttering equivalence~\cite{DBLP:journals/tcs/BrowneCG88}.
Their results spawned an interest in the relation
between temporal logics in the \LTS and the \KS setting, see
\eg~\cite{DBLP:journals/cn/NicolaFGR93,DBLP:conf/litp/NicolaV90}. The
latter both contain the embeddings that we use in this paper, differing
slightly from the ones proposed in~\cite{DBLP:journals/jacm/NicolaV95},
which in turn were in part inspired by the (unpublished) embedding
by Emerson and Lei~\cite{EL:84}.  The tight correspondence between
stuttering equivalence and branching bisimilarity that was exposed,
led Groote and Vaandrager to define algorithms for deciding said
equivalences in~\cite{DBLP:conf/icalp/GrooteV90}. Their algorithms
(and their correctness proofs), however, are stated directly in terms of
the appropriate setting, and do not appear to use the embeddings 
and  (but they might have acted as a source of inspiration).

Apart from the few documented cases listed above, many ad-hoc
embeddings are known to work for equivalences that are not sensitive to
abstraction. For instance, one can model the state labelling in a Kripke
Structure by means of labelled self-loops, or directly on the edges to
the next states, thereby exposing the same information. Such embeddings,
however, fail for equivalences that are sensitive to abstraction, such
as stuttering equivalence, which basically compresses sequences of states
labelled with the same state information.

\paragraph{Outline} In Section~\ref{Sect:preliminaries}, we formally
introduce the computational models \KS and \LTS, along with the
embeddings  and . The latter are proved to preserve and
reflect the additional three pairs of equivalences relations stated
above. In Section~\ref{sec:minimisations}, we introduce the inverses
 and , and we show that these can be combined with
 and , respectively, to obtain our minimisation results.
We finish with a brief summary of our contributions and an outlook to
some interesting open issues.

\section{Preliminaries}\label{Sect:preliminaries}

Central in both models of computation that we consider, \ie, \KS and
\LTS, are the notions of \emph{states} and \emph{transitions}. While the
\KS model emphasises the information contained \emph{in} such states,
the \LTS model emphasises the \emph{state changes} through some action
modelling a real-life event. Let us first recall both models of computation.

\newcommand{\AP}{\ensuremath{AP\xspace}}
\newcommand{\tuple}[1]{\ensuremath{\langle\,{#1}\,\rangle}}
\begin{definition}
A \emph{Kripke Structure} is a structure , where
\begin{itemize}
\item  is a set of states;

\item  is a set of atomic propositions;

\item  is a total transition relation, \ie, for
all , there exists , such that ;

\item  is a state labelling.
\end{itemize}
\end{definition}
By convention, we write  whenever .

\begin{remark}
The transition relation in the \KS model is traditionally required to be
total. Our results do not depend on the requirement of totality, but
we choose to enforce totality in favour of a smoother presentation and
more concise definitions. Without totality,
slightly more complicated treatments of the notions of paths and traces
(see also Section~\ref{sec:traces}) are needed.
\end{remark}
With the above restriction in mind, we define the \LTS model with
a similar restriction imposed on it.

\newcommand{\act}{\ensuremath{A}ct}
\begin{definition}[Labelled Transition System]
A structure  is an \LTS, where:
\begin{itemize}
\item  is a set of states;
\item  is a set of actions;
\item  is a total
transition relation,
\ie, for all , there are , , such that
.
\end{itemize}
\end{definition}
In lieu of the convention for \KS, we write  whenever
.

Note that in the setting of the \LTS model, a special constant 
is assumed outside the alphabet of the set of actions  of any
concrete transition system. This constant is used to represent so-called
silent steps in the transition system, modelling events that are
unobservable to any witness of the system.



In~\cite{DBLP:conf/litp/NicolaV90}, De Nicola and Vaandrager considered
embeddings called  and , which allowed one to move from \KS
models to \LTS models, and, \emph{vice versa}, from \LTS models to \KS
models. We repeat these embeddings below, starting with the embedding
from \KS into \LTS.

\newcommand{\shadow}[1]{\ensuremath{\bar{#1}}}
\begin{definition}
\label{translationlts}
The embedding  is defined as  for arbitrary Kripke Structures , where:

\begin{itemize}
\item , where
it is assumed that  for all ;
\item ;
\item  is the smallest relation satisfying:

\end{itemize}
\end{definition}
The fresh symbol  is used to signal a forthcoming encoding of the state information of the Kripke Structure. Encoding the state information by means of a self-loop  introduces problems in preserving and reflecting equivalences that are sensitive to abstraction.

\begin{definition}
\label{translationks}
The embedding  is formally defined as
 for Labelled Transition
System , where:
\begin{itemize}
\item ;
\item , where ;
\item  is the least relation satisfying:

\item  for , and .
\end{itemize}
\end{definition}
In this embedding the fresh symbol  is used to label the states from the Labelled Transition System.
The reason to treat -transitions different from ordinary actions is that otherwise equivalences that abstract from sequences of -transitions are not reflected well.

Observe that, as already stated in~\cite{DBLP:conf/litp/NicolaV90},
due to the artefacts introduced by the embeddings, moving from \LTS
to \KS and back again yields transition systems incomparable to the
original ones. Consequently, in \LTS, one cannot take advantage of tools
for minimising in the setting of \KS, and \emph{vice versa}. We defer
further discussions on this matter to Section~\ref{sec:minimisations}.

\section{Preservations and Reflections of Equivalences Under  and }
\label{Sect:preservations}

The embeddings  and  have already been shown to preserve and
reflect stuttering equivalence~\cite{DBLP:journals/tcs/BrowneCG88} and divergence-sensitive
branching bisimulation~\cite{vanGlabbeek96} by De Nicola and Vaandrager. In this
section, we introduce three additional pairs of equivalences and show
that these are also preserved by the embeddings  and . Our
choice for these four equivalences is motivated largely by the limited
set of equivalence's available in the \KS model (contrary to the \LTS model,
which offers a very fine-grained lattice of equivalence relations).

\begin{remark}
For reasons of brevity, throughout this paper we define equivalence
relations on states within a single \LTS (resp.\ \KS) rather than
equivalence relations between different models in \LTS (resp.\ \KS). Note
that this does not incur a loss in generality, as it is easy to define
the latter in terms of the former.

\end{remark}

\subsection{Similarity}
\label{sec:similarity}

Both \KS and \LTS have well-developed theories revolving around
similarity. We first formally define both notions.

\newcommand{\simm}{\simeq}
\begin{definition}
Let  be a Kripke Structure. A relation
 is a \emph{simulation relation} iff for every
 satisfying :
\begin{itemize}
\item ;
\item for all , if , then  for
some  such that .

\end{itemize}
For states , we say  is simulated by  if there is
a simulation relation , such that . States  are said to be \emph{similar}, denoted  iff
there are simulation relations  and , such that 
and .

\end{definition}
\begin{remark}
It should be noted that when lifting our notion of similarity to
an equivalence relation between different models in \KS,
the first requirement is sometimes stated as ,
where  is the state labelling of the second \KS model, and 
is the set of atomic propositions of the first \KS model. In this
case, some form of abstraction is included already, and care should be
taken to deal with such abstractions properly when lifting all our results to
such a setting.
\end{remark}

\begin{definition} Let  be a Labelled
Transition System. A relation  is a
\emph{simulation relation} iff for every  satisfying
:
\begin{itemize}
\item for all  and ,
if , then 
for some  such that .

\end{itemize}
State  is said to be simulated by state  if there is
a simulation relation , such that . States 
are \emph{similar}, denoted
 iff there are simulation relations  and , such
that  and .

\end{definition}

The theorems below state that indeed, embedding  preserves
and reflects \KS-similarity through \LTS-similarity (see
Theorem~\ref{th:lts_similarity}), and \emph{vice versa}, embedding
 preserves and reflects \LTS-similarity through \KS-similarity
(Theorem~\ref{th:ks_similarity}).

\begin{theorem}
\label{th:lts_similarity}
Let  be an arbitrary Kripke Structure.
Then, for all , we have
 iff .
\end{theorem}

\begin{proof}
See Appendix \ref{pf:th:lts_similarity}.\qed
\end{proof}

\begin{theorem}
\label{th:ks_similarity}
Let  be a Labelled Transition System.
Then for all , we have  iff
.
\end{theorem}

\begin{proof}
See Appendix \ref{pf:th:ks_similarity}.\qed
\end{proof}

\subsection{Bisimilarity}
\label{sec:bisimilarity}

A slightly stronger notion of equivalence that is rooted in the same concepts
as similarity, is \emph{bisimilarity}. Again, bisimilarity has been defined
in both \KS and \LTS, and we here show that both definitions agree through
the embeddings  and .

\begin{definition}
Let  be a Kripke Structure. States
 are said to be \emph{bisimilar}, denoted
 iff there is a \emph{symmetric
simulation relation} , such that .

\end{definition}
Similarly, we define bisimilarity in the setting of \LTS as follows:

\begin{definition}
Let  be a Labelled Transition System.
States  are \emph{bisimilar}, written  iff
there is a \emph{symmetric simulation relation} , such that
.

\end{definition}

\begin{theorem}
\label{th:lts_bisimilarity}
Let  be a Kripke Structure. Then for all
, we have  iff .
\end{theorem}

\begin{proof}
The proof is along the lines of the proof for similarity. For details,
see Appendix \ref{pf:th:lts_bisimilarity}.\qed
\end{proof}

\begin{theorem}
Let  be a Labelled Transition System.
For all , we have  iff .
\end{theorem}

\begin{proof} Again, the proof is along the lines of the proof for
similarity. \qed
\end{proof}


\subsection{Stuttering Equivalence -- Divergence-Sensitive Branching
Bisimilarity}
\label{sec:stuttering}

In this section, we merely repeat the definitions for stuttering
equivalence and divergence-sensitive branching bisimilarity. In
Section~\ref{sec:minimisations}, we come back to these equivalence
relations and state several new results for these.\\

The following definition for stuttering equivalence is taken
from \cite{DBLP:journals/jacm/NicolaV95}, where it is shown to coincide
with the original definition by Brown, Clarke and Grumberg~\cite{DBLP:journals/tcs/BrowneCG88}.
We prefer the former phrasing because of its coinductive nature.
\begin{definition}
Let  be a Kripke Structure. A
\emph{symmetric} relation  is a
\emph{divergence-blind stuttering equivalence} iff for all :
\begin{itemize}
\item ;
\item for all , if , then there exist
, such that  and , and
for all ,  and .
\end{itemize}
\end{definition}
\begin{definition} Let  be a Kripke
Structure. Let the Kripke Structure  be defined as follows:
\begin{itemize}
\item  for some fresh state ;

\item  for some fresh proposition ;

\item sL(s)s=s_d;

\item for all , , and .

\end{itemize}
States  are said to be \emph{stuttering equivalent},
notation:  iff there is a divergence-blind
stuttering equivalence relation  on  of , such that
.
\end{definition}
The origins of divergence-sensitive branching bisimilarity can be
traced back to~\cite{vanGlabbeek96}.
In~\cite{DBLP:journals/fuin/GlabbeekLT09}, Van Glabbeek
\etal demonstrate that various incomparable phrasings of
the divergence property all coincide with the original definition.
For our purposes the following formulation is most suitable.
\newcommand{\nat}{\ensuremath{\mathbb{N}}}
\begin{definition}
Let  be a Labelled Transition System.
A symmetric relation  is a \emph{divergence-sensitive
branching simulation relation} iff for all :
\begin{itemize}

\item if there is an infinite sequence of states 
such that  and  for all , then
there exist a mapping , and an infinite sequence of
states  such that ,  and  for all ;

\item for all  and , if
, then  and , or
 for some 
such that  and  .

\end{itemize}
States  are divergence-sensitive branching bisimilar,
notation  iff there is a symmetric divergence-sensitive
branching simulation relation , such that .
\end{definition}



\subsection{Trace Equivalence -- Completed Trace Equivalence}
\label{sec:traces}

Trace equivalence and completed trace equivalence are the only
linear-time equivalence relations that we consider in this paper. In
defining these equivalence relations, we require some auxiliary notions,
basically defining what a \emph{computation} is in our respective
models of computation.

\newcommand{\paths}[1]{\ensuremath{\mathsf{Paths}(#1)}}
\begin{definition} Let  be a Kripke
Structure. A \emph{path} starting in state  is an infinite
sequence , such that  for
all , and . The set of all paths starting in  is denoted
.
\end{definition}
Basically, a path formalises how a single computation evolves in
time. Actually, it is the information contained in the states that are
visited along such a computation that is often of interest, as it shows
how the state information evolves in time. This is exactly captured
by the notion of a \emph{trace}.

\newcommand{\trace}[1]{\ensuremath{\mathsf{Trace}(#1)}}
\newcommand{\traces}[1]{\ensuremath{\mathsf{Traces}(#1)}}
\newcommand{\traceeq}{\ensuremath{\simeq_{\mathrm{t}}}}

\begin{definition} Let  be a Kripke
Structure. Let  be a path starting in . The
\emph{trace} of , denoted , is the infinite sequence
.  For a set of paths , we set

States  are \emph{trace equivalent}, denoted , if .
\end{definition}

\begin{remark} In the presence of non-totality of the transition relation
of a Kripke Structure, it no longer suffices to consider only the
infinite paths as the basis for defining trace equivalence. Instead,
\emph{maximal} paths are considered, which in addition to the infinite
paths, also contains paths made up of sequences of states that end in
a sink-state, \ie, a state without outgoing edges.
\end{remark}
For models in \LTS, we define similar-spirited concepts; for the
origins of the definition, we refer to Van Glabbeek's lattice of
equivalences~\cite{vanGlabbeek01}.


\newcommand{\runs}[1]{\ensuremath{\mathsf{Runs}(#1)}}
\newcommand{\bareruns}[1]{\ensuremath{\mathsf{Runs_b}(#1)}}
\begin{definition} Let  be a Labelled
Transition System.  A \emph{run} starting in a state  is an
infinite, alternating sequence of states and actions  satisfying  for all , and
.  The set of all runs starting in  is denoted .

\end{definition}


\begin{definition} Let  be a Labelled
Transition System.  The \emph{trace} of a run , denoted , is the infinite sequence
.  For a set of runs , we define

States  are \emph{completed trace equivalent}, denoted by
 iff .

\end{definition}



\begin{theorem}
\label{th:ks_traceeq}
Let  be a Kripke Structure. For all
, we have  iff
.
\end{theorem}

\begin{proof}
See Appendix~\ref{pf:th:lts_bisimilarity} for details.
\qed
\end{proof}
In a similar vein, we obtain that completed trace equivalence in
\LTS is preserved and reflected by trace equivalence in \KS.
\begin{theorem}
\label{th:lts_traceeq}
Let  be a Labelled Transition System.
Let  be arbitrary states. We have  iff
.

\end{theorem}
\begin{proof}
Along the lines of the proof for Theorem~\ref{th:ks_traceeq}.\qed
\end{proof}


\section{Minimisations in \LTS and \KS}
\label{sec:minimisations}

As we concluded in Section~\ref{Sect:preliminaries}, the mappings 
and  cannot be used to freely move to and fro the computational
models.  Instead, we introduce two additional mappings, \viz, 
and  that act as inverses to  and , respectively, and
we show that these can be used to come to our results for minimisation.
Here, we focus on the computationally most attractive
equivalences, \viz, \emph{bisimilarity} and \emph{stuttering equivalence}.

\renewcommand{\min}[1]{{#1}\textrm{-min}}
\newcommand{\minKS}[1]{{#1}\textrm{-min}_{\KS}}
\newcommand{\minLTS}[1]{{#1}\textrm{-min}_{\LTS}}
\newcommand{\quotient}[2]{\ensuremath{{#1}_{/#2}}}
Let  and  be arbitrary equivalence
relations on \KS and \LTS, respectively.  For a given model  in \KS, its \emph{quotient}
with respect to  is denoted . Similarly, for a given model  in \LTS, its \emph{quotient}
with respect to  is denoted . We assume
unique functions  for \KS, and 
for \LTS that uniquely determine transition systems that are isomorphic
to the quotient. If, from the equivalence relation , the setting
is clear, we drop the subscripts and write  instead.

\subsection{Minimisation in \KS via minimisation in \LTS}


We first characterise a subset of models of \LTS for which we can
define our inverse  of .

\begin{definition}
Let  be a Labelled Transition System.
Then  is reversible iff
\begin{enumerate}
\item , for some set ;

\item for all  and ,
if , then ;

\item for all  such that  and
, we require that
 and  implies  for all
actions .

\end{enumerate}
\end{definition}
Note that any embedding  of a Kripke Structure 
is a reversible Labelled Transition System. Reversibility
is preserved by the quotients for  and ,
as stated by the following proposition.
\begin{proposition}
\label{prop:reversibility}
Let  be an arbitrary reversible Labelled Transition
System. Then , for ,
is reversible. \qed

\end{proposition}
The embedding 
introduces a fresh, \emph{a priori} known action label .
We treat this constant differently from all other actions in
our reverse embedding.
\begin{definition}
\label{translationltsreverse}
Let  be a reversible Labelled Transition
System.
We define the Kripke Structure  as the structure
, where:
\begin{itemize}
\item ;

\item  is such that ;

\item  is the least relation satisfying the single rule:


\item  for the unique  such that .

\end{itemize}

\end{definition}

\noindent
The following proposition establishes that  is the
inverse of embedding .

\newcommand{\id}{\ensuremath{\mathsf{Id}}}
\begin{proposition}
\label{prop:lts_inverse}
We have .\qed
\end{proposition}

\begin{proof}
Establishing the isomorphism follows immediately from the definitions
and the observation that  is reversible. See Appendix \ref{pf:prop:lts_inverse}.\qed
\end{proof}

\newcommand{\simKS}{\sim_{\ks}}
\newcommand{\simLTS}{\sim_{\lts}}


Note that reversibility of a Labelled Transition System  is too
weak to obtain , as the following example
illustrates:
\begin{example}
Consider the Labelled Transition System left below.
\begin{center}
\begin{tikzpicture}[>=stealth',node distance=40pt]
\tikzstyle{every state}=[inner sep=1pt, minimum size=6pt];

\node[state] (t) {};

\node (t') [right of=t] {};
\node (t'') [right of=t'] {};

\node[state] (u) [right of=t''] {};

\node (u') [right of=u] {};
\node (u'') [right of=u'] {};

\node[state] (v) [right of=u''] {};
\node[state] (v') [right of=v,xshift=-20pt] {};

\draw[->] (t) edge [loop left] node [left] {} (t)
          (t) edge [loop right] node [right] {} (t)
          (u) edge [loop right] (u)
          (v) edge [loop left] node [left] {} (v)
          (v) edge [bend left] node [above] {} (v')
          (v') edge [bend left] node [below] {} (v)
;

\draw (u) node [left] {};
\draw[->,dashed] (t') edge node [above] {} (t'')
                 (u') edge node [above] {} (u'');

\end{tikzpicture}
\end{center}
Clearly, the
Labelled Transition System
is reversible, so the mapping  is applicable. Its result is
given by the Kripke Structure in the middle. Applying  to the middle Kripke Structure
yields the Labelled Transition System at the right. It is clear that the latter is not isomorphic
to the original Labelled Transition System.\qed
\end{example}

\begin{lemma}
\label{lem:lts_bisim} We have .

\end{lemma}


\begin{proof}
See Appendix \ref{pf:lem:lts_bisim}.\qed
\end{proof}


\begin{lemma}
\label{lem:lts_stut}
We have .
\end{lemma}

\begin{proof}
See Appendix \ref{pf:lem:lts_stut}.\qed
\end{proof}
Before we present the main theorems concerning the minimisations in \KS
through minimisations in \LTS, we first show that it suffices to prove
such results for Kripke Structures that are already minimal; see the
lemma below.
\begin{lemma}
\label{lem:minimal}
Let  and
 such that  preserves and reflects  through
.  Then

\end{lemma}

\begin{proof}
Assume that we have

By definition of , we find
.
Since, by assumption,  preserves and reflects  through
, we derive
.
By definition of , this means that we have:

As  is functional, and  preserves
reversibility, we immediately obtain:

The desired conclusion then follows by combining~\ref{eq:*} and~\ref{eq:**}.
\qed
\end{proof}
We finally state the two main theorems in this section.
\begin{theorem}
\label{th:ks2lts_bisim_minimal}
We have .
\end{theorem}

\begin{proof}
Lemma~\ref{lem:lts_bisim} guarantees

Functionality of , combined with Proposition~\ref{prop:reversibility},
we find:

By Lemma~\ref{lem:minimal}, we then have our desired conclusion:

\qed
\end{proof}


\begin{theorem}
\label{th:ks2lts_stut_minimal}
We have  .
\end{theorem}

\begin{proof}
Similar to
Theorem~\ref{th:ks2lts_bisim_minimal}, using Lemma~\ref{lem:lts_stut}
instead of Lemma~\ref{lem:lts_bisim}. \qed
\end{proof}


\subsection{Minimisation in \LTS via minimisation in \KS}

In the previous section, we showed that one can minimise in \KS
with respect to bisimilarity or stuttering equivalence,
using the embedding , a matching equivalence relation in \LTS and
converting to \KS again. In a similar vein, we propose a reverse
translation for , which allows one to return to \LTS from
\KS. We first characterise a set of Kripke Structures that are amenable
to translating to Labelled Transition Systems.

\begin{definition} Let  be a Kripke
Structure. Then  is reversible iff
\begin{enumerate}
\item  for some set ;
\item  for all ;
\item for all  for which , we require that
for all ,  and  implies both  and
.

\end{enumerate}

\end{definition}
\begin{proposition}
\label{prop:reversibility2}
Let  be an arbitrary reversible Kripke Structure.
Then , for ,
is reversible.

\end{proposition}

\begin{definition}
Let  be a reversible Kripke Structure.
The Labelled Transition System  is the structure
, where:
\begin{itemize}
\item ;

\item  is such that ;

\item  is the least relation satisfying:


\end{itemize}
\end{definition}
\begin{proposition}
\label{prop:ks_inverse}
We have .
\end{proposition}

\begin{proof}
Similar to the proof of Proposition \ref{prop:lts_inverse}.\qed
\end{proof}

\noindent
Without further elaboration, we state the final results.
\begin{theorem}
\label{th:lts2ks_bisim_minimal}
We have .\qed
\end{theorem}



\begin{theorem}
\label{th:lts2ks_stut_minimal}
We have .
\qed
\end{theorem}



\section{Conclusions}\label{Sect::conc}

Our results in Section~\ref{Sect:preservations} naturally extend
the fundamental results obtained by De Nicola and Vaandrager
in~\cite{DBLP:conf/litp/NicolaV90,DBLP:journals/jacm/NicolaV95}.
In a sense, we can now state that their embeddings
 and  are canonical for four commonly used equivalence
relations.

While the stated embeddings have traditionally been used to come to
results about the correspondence between logics, the question whether
they support minimisation modulo behavioural equivalences was never
answered. Thus, in addition to the above stated results, we proved that
indeed the embeddings  and  can serve as basic tools in the
problem of minimising modulo a behavioural equivalence relation. To this
end, we defined inverses of the embeddings to compensate for the fact that
composing  and  does not lead to transition systems that are
comparable (in whatever sense) to the one before applying the embeddings.
The latter results are clearly interesting from a practical perspective,
allowing one to take full advantage of state-of-the-art minimisation
tools available for one computational model, when minimising in the other.

Our minimisation results are for two of the most commonly used equivalence
relations that are, arguably, still efficiently computable. However,
we do intend to extend our results also in the direction of (completed)
trace equivalence and similarity. As a slightly more esoteric research
topic, one could look for improving on the embedding , as, compared
to the embedding , it introduces more ``noise''. For instance, it
yields Labelled Transition Systems that have runs that cannot sensibly
be related to paths in the original Kripke Structure.


\bibliographystyle{plain}
\bibliography{lit}

\cleardoublepage
\appendix
\section{Proofs for Section \ref{Sect:preservations}}

\subsection{Proof of Theorem \ref{th:lts_similarity}}
\label{pf:th:lts_similarity}

Consider states  and  in a Kripke structure . Assume that  and that this is witnessed by the simulation relations  and  with  and .
We show that, with respect to the Labelled Transition System associated with the Kripke structure, the relation  is a simulation relation with  . In a similar way a simulation relation  with  can be defined. This part is omitted.

First consider an arbitrary pair . This is due to the fact that . By construction the only transitions for  and  are  and . From the fact that  it follows that . This suffices to satisfy all transfer conditions for the pair .

Next, consider an arbitrary pair . This is due to the fact that . Let us consider all transitions from .
\begin{itemize}
\item . Since  we have . Since  it also follows that .

\item  for some  such that  and .
Since  and  is a simulation, it follows that  and  for some  such that . Since  we have , and therefore  as well. Thus, by construction . From  we obtain .

\item  for some  such that  and .
Since  and  is a simulation, it follows that  and  for some  such that . Since  we have , and therefore  as well. Thus, by construction . From  we obtain . \qed
\end{itemize}

\subsection{Proof of Theorem \ref{th:ks_similarity}}
\label{pf:th:ks_similarity}

Consider states  and  in a Labelled Transition System . Assume that  and that this
is witnessed by the simulations  and  with 
and . We show that, with respect to the Kripke structure
associated with the Labelled Transition System, the relation  is a simulation relation with . Here  is
the set of states of that Kripke structure as prescribed by Definition
\ref{translationks}. Similarly, a simulation relation  with  can be defined.

First consider a pair  that is present in  due to its presence in . Since  we have by definition that .
We consider all possible transitions from . By construction the only possible transitions for  are the following.

\begin{itemize}
\item  for some  such that . Since  and  is a simulation relation, we have  for some  such that . By construction then also  and .
\item  for some  and  such that  and . Since  and  is a simulation relation, we have  for some  such that . By construction then also . Note that  since  and .
\end{itemize}


Next, consider a pair  that is present in  due to presence of both  and  in . By construction . Let us consider all transitions from .
The only possible transition is . Since ,  and  is a simulation relation it follows that  for some  such that . By construction then also  and . \qed


\subsection{Proof of Theorem \ref{th:lts_bisimilarity}}
\label{pf:th:lts_bisimilarity}

Consider states  and  in a Kripke structure . Assume that  and that this is witnessed by the bisimulation relation  with . Thus  is a simulation relation with  and with .
We define the relation . It follows from the proof of Theorem \ref{th:lts_similarity} that  is a simulation relation for  and for . \qed

\subsection{Proof of Theorem \ref{th:ks_traceeq}}
\label{pf:th:ks_traceeq}

Before we prove Theorem \ref{th:ks_traceeq} in this section, we establish
an intermediate result concerning the relation between paths ---and
their prefixes--- of a Kripke Structure and the subset of bare runs,
defined below ---and their prefixes--- in the \LTS-embedding of the same
Kripke Structure.


\begin{definition}[Bare run] A run  is said
to be a \emph{bare run} iff the labels occurring on the run differ from
. The set of bare runs starting in a state , for 
is given by the set .
\end{definition}
\newcommand{\prefruns}[1]{\ensuremath{\mathsf{Runs}^p(#1)}}
\newcommand{\prefbareruns}[1]{\ensuremath{\mathsf{Runs}^p_b(#1)}}
\newcommand{\bare}[1]{\ensuremath{\beta(#1)}}

Let .  Let  be the set of prefixes of runs starting in states ;
likewise,  is the set of prefixes of bare runs
starting in  satisfying .

Given a (finite) trace , we write
 if there is some 
ending in state  such that .

\begin{definition} Let .
Denote the sequence , obtained from 
by deleting
\begin{itemize}
\item all subsequences of the form ;
\item  in case  ends
as such.
\end{itemize}
\end{definition}
It is not hard to see that 
implies , \ie,
any trace  is generated by a bare run.
\begin{lemma}
\label{lem:bare_vs_ordinary}
For all ,  such that
 and all , we have
 iff
\begin{enumerate}
\item
 and
, \emph{or}

\item  and
.

\end{enumerate}

\end{lemma}
\begin{proof} By induction on the length of .
\end{proof}
The above lemma firmly establishes a connection between a trace 
of a run starting in a state  and the trace . Intuitively,
as bare runs only pass through states that can perform a 
transition, any trace generated by a bare run can be ``pumped up'' to generate
an arbitrary trace that can lead to the same state as its corresponding
bare run, simply by following the loop , for some action label .

\begin{lemma}
\label{lem:bareruns1}
Let  be a Kripke Structure.
Then
 implies
 in 
for precisely one
infinite sequence . \emph{Vice versa,}
if for some infinite sequence , we have
 in ,
then .
\end{lemma}
\begin{proof}
Follows by definition of .
\end{proof}
Informally, the above lemma states that for each path in a Kripke
Structure, there is a unique matching bare run in its \LTS embedding,
and, \emph{vice versa}, for every bare run in its \LTS embedding, there
is a unique path in the Kripke Structure.

We next establish that the embedding  is such that for the trace
equivalence of two states in a Labelled Transition System resulting from
the embedding , it suffices to prove that the traces of all bare runs
coincide.  Formally, we have:

\begin{lemma}
\label{lem:bareruns2}
Let  be a Kripke Structure. Let , with  be such that  in . Then also  in .

\end{lemma}

\begin{proof} By induction on the length of the traces.
\end{proof}
Since all runs are infinite, in the limit, any trace  is also in the set .


\begin{proof}[Theorem~\ref{th:ks_traceeq}]
Let states  in a Kripke Structure be trace equivalent.
Suppose  and  are such that . Because of Lemma~\ref{lem:bareruns1}, we find that there
must be unique  and 
passing through the exact same states as the paths  and 
respectively. That is:

By construction of , we have  if
 and  otherwise (and similarly for
). But from
the fact that , we find that  for all . Hence, also  for all . But
this means that . Appealing to
Lemma~\ref{lem:bareruns1}, \emph{all} bare runs correspond to paths
in the Kripke Structure. Hence, we find that
. Since , Lemma~\ref{lem:bareruns2} yields the desired conclusion that
.

For the other direction, we assume that states  are
trace equivalent in . In short, this means that the set of
bare runs starting in  and  produce the same traces. Let
 be a bare run starting in ,
and  be a bare run starting in
, such that . Using Lemma~\ref{lem:bareruns1}, we find
that there are unique matching paths 
and .
By construction of , we find that
this implies that  for all  satisfying that 
for the least  such that . For all , we
observe that , which can only be if
 for . Likewise, .
Since \emph{all} traces starting in  and  are the same in
, also ,
which can only be the case when . But then also
 for all . Hence, .
Since all paths starting in
 and  correspond to unique bare runs starting in  and 
in , this means we have considered all possible paths and
therefore all possible traces.
\end{proof}



\section{Proofs for Section \ref{sec:minimisations}}

\subsection{Proof of Proposition \ref{prop:lts_inverse}}
\label{pf:prop:lts_inverse}

This theorem follows directly from the definitions.
Consider arbitrary Kripke structure . Let  and . We will show that , ,  and , thus establishing the isomorphism of  and .

From the definition of  (applied to ) it follows that
\begin{itemize}
\item ;
\item ;
\item 
\end{itemize}
and application of  (applied to ) gives
\begin{itemize}
\item . Since  iff  we obtain
    .
\item .
\item 
\item  where  is such that  for some . Therefore  and . From this it follows that . So . \qed
\end{itemize}


\subsection{Proof of Lemma \ref{lem:lts_bisim}}
\label{pf:lem:lts_bisim}

Consider a Kripke structure  that is minimal w.r.t.\ strong bisimilarity (on \KS). We have to show that  is minimal w.r.t.\ strong bisimilarity (on \LTS). We show that (1) the identity relation on the states of  is a bisimulation relation, and (2) that this bisimulation relation is maximal.

We know, since  is minimal, that the identity relation on  is a maximal bisimulation relation. From this it follows that the identity relation on  (the states of ) is a bisimulation relation as well.

Now assume that the identity relation on  is not the maximal bisimulation relation, i.e.,  there exists a bisimulation relation  that relates at least one pair of different states. First, we show that it has to be the case that at least one pair of different states from  is related by .

This can be seen as follows. Consider a pair of different states  and  related by . Suppose that  and . In this case, by definition of , , but . Hence  and  cannot be related by a bisimulation relation. The case that  and  is similar.
In case both  and , by definition  and  for some  with . Then, by definition of , the only transitions of  and  are  and . In order for  and  to be related by  necessarily  and  need to be related by . Thus we can safely conclude that  relates a pair of different states  and , both from .

Now we show that  is a bisimulation relation on \KS, thus contradicting the assumption that the identity relation on  is the maximal bisimulation relation.

Consider a pair of different states  and , both from , that are related by . We show that .
 This follows from the following observations. Both  and  each have a single -transition:  and . Then, also  and  are related by . These states each have only one transition:  and . From this it follows that .

 Assume that  for some . We distinguish two cases:
 \begin{itemize}
 \item . Then . Then  for some  such that . Since  cannot relate states from  (such as ) with states outside , also . Therefore, by definition of , .
 \item  . Then . Then  for some  such that . Since  cannot relate states from  (such as ) with states outside , also . Then, by definition of  it has to be the case that  and .
 \end{itemize}
 In each case it follows that  and  and  are related by ., which was to be shown.

The case that  for some  needs to be mimicked is similar.

From the contradiction obtained it can be concluded that the identity relation on  is the maximal bisimulation relation.


\subsection{Proof of Lemma \ref{lem:lts_stut}}
\label{pf:lem:lts_stut}

Consider a Kripke structure  that is minimal w.r.t.\ stuttering equivalence (on \KS). We have to show that  is minimal w.r.t.\ divergence-sensitive branching bisimilarity (on \LTS). We show that (1) the identity relation on the states of  is a divergence-sensitive branching bisimulation relation, and (2) that this bisimulation relation is maximal.

We know, since  is minimal, that  is minimal with respect to
divergence-blind stuttering equivalence. Denote the states of  by
. Hence, the identity relation on  is a
maximal divergence-blind stuttering bisimulation relation with respect
to the Kripke structure . From this it follows that the identity
relation on  (the states of ) is a divergence-sensitive
branching bisimulation relation as well.

Now assume that the identity relation on  is not the maximal bisimulation relation, i.e.,  there exists a divergence-sensitive branching bisimulation relation  such that there are different states  and  from  with . We distinguish four cases:
\begin{itemize}
\item  and . In this case, by definition of , , but  and . Therefor the transition from  cannot be mimicked from . So this case cannot occur.

\item  and . Similar to the previous case.

\item  and . We have to show that there exists a divergence-blind stuttering bisimulation relation  with .

First we consider the case that  for some . We can distinguish two cases
\begin{itemize}
\item Suppose that . By definition . Then, by definition of divergence-sensitive branching bisimulation, we have  and , or the existence of  and  such that
    
     with  (using the Stuttering Lemma) and . In the first case we have  and in the second case we have   with  and .
\item Suppose that . By definition there is an infinite sequence
    
    of states with the same label. Therefore, in  there is an infinite sequence
    
    where all states have the same label .
    Hence, there is an infinite sequence
    
    Therefore, in , there is an infinite sequence
    
    where  for all .
    Thus  as required.
\end{itemize}

\item  and . By definition the only transition of  is of the form  for some . Since  obviously the only way to mimic the transition is by means of  for some  with . Necessarily . We have established in the previous item that such  and  cannot be related. Therefore, also  and  cannot be related.
\end{itemize}

Second, we show that . Since  we have . Then,  and  with , ,  and .
It follows that  and from the fact that  it follows that  as well. Similarly, . Since  it also follows that . Thus we have obtained .

We have shown that  was not minimal. Therefore the assumption that  is not minimal is flawed, which completes the proof.

\end{document}
