\documentclass[11pt]{article}
\usepackage{amsthm,amsfonts,amsmath}
\usepackage{enumerate,paralist}
\usepackage{xspace}
\usepackage{boxedminipage}
\usepackage{color}
\usepackage{hyperref}
\usepackage{cite}
\usepackage{graphicx, wrapfig}

\fontencoding{T1}

\textheight 9.1in
\advance \topmargin by -1.0in
\textwidth 6.7in
\advance \oddsidemargin by -0.8in
\newcommand{\myparskip}{3pt}
\parskip \myparskip

\newcommand{\headers}[3]{
\newpage\setcounter{page}{1}
\def\@oddhead{}
\def\@oddfoot{\hfill\thepage\hfill}}

\newtheorem{fact}{Fact}[section]
\newtheorem{lemma}{Lemma}[section]
\newtheorem{theorem}[lemma]{Theorem}
\newtheorem{assumption}[lemma]{Assumption}
\newtheorem{definition}[lemma]{Definition}
\newtheorem{corollary}[lemma]{Corollary}
\newtheorem{prop}[lemma]{Proposition}

\newtheorem{remark}[lemma]{Remark}
\newtheorem{prob}{Problem}
\newtheorem{conjecture}{Conjecture}
\newtheorem{question}{Question}
\newtheorem{observation}{Observation}

\renewenvironment{proof}{\vspace{-0.1in}\noindent{\bf Proof:}}{\hspace*{\fill}\par}
\newenvironment{proofof}[1]{\smallskip\noindent{\bf Proof of #1:}}{\hspace*{\fill}\par}
\newenvironment{proofsketch}{\vspace{-0.1in}\noindent{\bf Proof Sketch:}}{\hspace*{\fill}\par}

\DeclareMathOperator{\argmax}{\arg\max}
\def\etal{\emph{et al.}\xspace}

\def\opt{\mathrm{OPT}}
\def\eps{\varepsilon}
\def\bar{\overline}
\def\floor#1{\lfloor {#1} \rfloor}
\def\ceil#1{\lceil {#1} \rceil}
\def\script#1{\mathcal{#1}}

\def\card#1{\left|#1\right|}
\def\set#1{\left\{#1\right\}}
\def\pair#1{\left<#1\right>}
\def\sep{\;|\;}
\def\polylog{\mathrm{polylog}}
\def\poly{\mathrm{poly}}

\def\whp{w.h.p.\xspace}

\def\sA{\script{A}}
\def\sC{\script{C}}
\def\sD{\script{D}}
\def\sG{\script{G}}
\def\sH{\script{H}}
\def\sP{\script{P}}
\def\sQ{\script{Q}}
\def\sS{\script{S}}
\def\sT{\script{T}}
\def\sW{\script{W}}

\def\bC{\script{\mathbf{C}}}
\def\vN{\mathbf{N}}

\def\mypar#1{{\medskip\noindent \textbf{#1}}}

\def\prob#1{\textsf{\textup{#1}}\xspace}

\def\Ex{\mathrm{Ex}}
\def\capacity{\mathrm{cap}}
\def\minDSlp{\prob{\minDS-LP}}
\def\minCDSlp{\prob{\minCDS-LP}}
\def\nwST{\prob{NW-Steiner-Tree}}
\def\nwSTlp{\prob{\nwST-LP}}

\def\NP{\mathrm{NP}}
\def\DTIME{\mathrm{DTIME}}
\def\PTAS{\mathrm{PTAS}}


\def\CDSpart{\prob{CDS-Partition}}
\def\CDSpack{\prob{CDS-Packing}}
\def\capCDSpart{\prob{Cap-CDS-Partition}}
\def\capCDSpack{\prob{Cap-CDS-Packing}}

\def\edgeCDSpart{\prob{Edge-CDS-Partition}}
\def\edgeCDSpack{\prob{Edge-CDS-Packing}}
\def\minCDS{\prob{Min-Cost-CDS}}
\def\minDS{\prob{Min-Cost-DS}}

\def\cost{\mathrm{cost}}
\def\vx{\mathrm{\mathbf{x}}}
\def\vy{\mathrm{\mathbf{y}}}

\sloppy

\begin{document}

\title{Connected Domatic Packings in Node-capacitated Graphs}
\author{
Alina Ene\thanks{Dept.\ of Computer Science, University of Illinois,
Urbana, IL, 61801. Supported in part by NSF grant CCF-1016684 and a
Chirag Foundation graduate fellowship. This work was done while the
author was visiting the IBM Almaden Research center.
{\tt ene1@illinois.edu}.}
\and
Nitish Korula\thanks{
Google Research, New York, NY 10011. Part of this work was done while
the author was a student at University of Illinois.
{\tt nitish@google.com}.}
\and
Ali Vakilian\thanks{
Dept.\ of Computer Science, University of Illinois, Urbana, IL,
61801. Supported in part by NSF grant CCF-1016684 and a Siebel Scholar award. 
{\tt vakilia2@illinois.edu}.}}

\date{\today}

\pagenumbering{gobble}
\maketitle

\begin{abstract}
	A set of vertices in a graph is a dominating set if every vertex
	outside the set has a neighbor in the set. A dominating set is
	connected if the subgraph induced by its vertices is connected.
	The connected domatic partition problem asks for a partition of
	the nodes into connected dominating sets. The connected domatic
	number of a graph is the size of a largest connected domatic
	partition and it is a well-studied graph parameter with
	applications in the design of wireless networks.
In this note, we consider the fractional counterpart of the
	connected domatic partition problem in \emph{node-capacitated}
	graphs. Let  be the number of nodes in the graph and let 
	be the minimum capacity of a node separator in . Fractionally
	we can pack at most  connected dominating sets subject to the
	capacities on the nodes, and our algorithms construct packings
	whose sizes are proportional to . Some of our main
	contributions are the following:
\begin{itemize}
		\item An algorithm for constructing a fractional connected
		domatic packing of size  for
		node-capacitated planar and minor-closed families of graphs.
\item An algorithm for constructing a fractional connected
		domatic packing of size  for
		node-capacitated general graphs.
\end{itemize}
\end{abstract}


\newpage
\setcounter{page}{1}
\pagenumbering{arabic}

\section{Introduction}
\label{sec:intro}

Let  be an undirected and connected graph with  nodes.
A set  of nodes is a dominating set if every node not in  has a
neighbor in . A connected domatic partition is a collection of
connected dominating sets that are node disjoint. The connected
domatic number is the size of a largest connected domatic partition.
In this note, we consider the problem of constructing large
fractional connected domatic packings; a fractional packing is a
weight function on connected dominating sets such that, for each
vertex , the total weight of the connected dominating sets that
contain  is at most one.

Connected domatic partitions and packings have several applications
in the design of wireless networks. In these applications, a
connected dominating set is used as a virtual backbone, and the rest
of the nodes use the connected dominating set to exchange messages
and route traffic \cite{DasB97,DasSB97,MahjoubM10}.  Motivated by the
goal of improving the energy efficiency and the lifetime of the
network, several papers \cite{MisraM09,MoscibrodaW05,PemmarajuP06}
have proposed using several connected dominating sets; these
approaches first compute a large connected domatic packing or
partition and they rotate between the connected dominating sets.
Additionally, the recent work of Censor-Hillel \etal \cite{CHGK}
establishes a close connection between the fractional connected
domatic number and the throughput of store-and-forward algorithms for
routing in wireless networks.

Integer and fractional packings of combinatorial structures are
connected to each other and to the corresponding optimization problem
that asks for the minimum cost combinatorial structure; we refer the
reader to Section~{5} in \cite{CalinescuCV09} for an overview of
these connections. In particular, an -approximation for the
minimum-cost \prob{Connected Dominating Set} (\minCDS) problem implies an
-approximation for the \prob{Connected Domatic Packing}
(\CDSpack) problem; this connection was shown by
Carr and Vempala \cite{CarrV02}.  This result and the 
approximation algorithm for \minCDS given by Guha and Khuller
\cite{GuhaK98} imply an  approximation for \CDSpack. In
very recent work, Censor-Hillel \etal \cite{CHGK} gave the first
poly-logarithmic approximation for the \prob{Connected Domatic
Partition} problem; their algorithm achieves
an  approximation. The results of \cite{CHGK} guarantee
partitions and packings whose sizes are a poly-logarithmic fraction
of the \emph{vertex connectivity}\footnote{A graph  is
-vertex-connected iff, for any subset  of size
less than , the removal of  does not disconnect the graph. The
vertex connectivity of  is the maximum  such that  is
-vertex-connected.}. These guarantees are independent of the size
of the largest partition or packing and thus they are not
approximation results \emph{per se}.  Since the connectivity of the
graph is an upper bound on the fractional connected domatic number
and thus the connected domatic number as well, these absolute results
give us approximation guarantees as a byproduct.

In several applications in wireless networks, each node has a certain
battery life that constrains how long the node can be used as part of
a virtual backbone for the network. We can model such networks using
node-capacitated graphs, where the capacity represents the battery
life of the node. Motivated in part by these applications, we
consider the more general problem of constructing large connected
packings in node-capacitated graphs. In this setting, each vertex 
has a capacity  and the goal is to find a fractional
packing of maximum total weight such that the fractional weight of
the connected dominating sets that contain each vertex is at most the
capacity of the vertex. We refer to the capacitated analogue of
\CDSpack as \capCDSpack.  We can reduce the capacitated problem to
the uncapacitated one by replacing each node by a clique whose size
is equal to the capacity of the node. However, this reduction does
not run in polynomial time if the capacities are large and it does
not preserve the special structure of certain graphs, such as planar
or minor-free graphs.  Real-world wireless networks are typically not
arbitrary graphs but rather they are nearly planar or have restricted
structure. We give an algorithm that constructs improved fractional
packings for such networks.

\begin{theorem} \label{thm:cds-packing-minor-free}
	Let  be a node-capacitated graph that belongs to a
	minor-closed family  of graphs. Let  be the minimum
	capacity of a node separator\footnote{A set  is a node
	separator in  if the graph  has at least two connected
	components, where  is the graph obtained from  by
	removing the nodes of } in . There is a polynomial time
	algorithm that constructs a fractional connected domatic packing
	in  of size , where the constant depends only on
	the family .
\end{theorem}

\noindent
Our approach can also be used to construct fractional packings for
general graphs with arbitrary node capacities. This result was
shown in \cite{CHGK} for uncapacitated graphs using very different
techniques.

\begin{theorem} \label{thm:cds-packing-general}
	Let  be a node-capacitated graph. Let  be the minimum
	capacity of a node separator in . There is a polynomial time
	algorithm that constructs a fractional connected domatic packing
	in  of size .
\end{theorem}

\noindent
Our algorithm for \capCDSpack is based on a connection between the
size of a fractional packing and the integrality gap of a standard LP
relaxation for \minCDS; we describe this LP relaxation in
Section~\ref{sec:cds-packing}. We show that, if the relaxation has an
integrality gap of , we can construct a packing of size  in
polynomial time using an -approximate rounding algorithm for the
\minCDS LP and the ellipsoid method. One of our contributions is a
constant upper bound on the integrality gap of \minCDS LP in
minor-closed families of graphs, where the constant depends only on
the family.  In the process, we also show that the integrality gap of
a standard LP relaxation for the minimum cost \prob{Dominating Set}
(\minDS) problem is constant in minor-free graphs. The \minDS problem
admits a  in planar graphs \cite{Baker94}, but this result
does not establish an upper bound on the integrality gap. Our
algorithms can be easily adapted to give analogous integrality gap
upper bounds for the Steiner variants of \minDS and \minCDS; in the
Steiner problems, we are given a subset of the vertices called the
terminals and the goal is to select a (connected) set that dominates
the terminals.

\begin{theorem} \label{thm:ds-gap-minor-free}
	The standard LP relaxation for the \minDS problem has an 
	integrality gap in planar and minor-closed families of graphs.
	Moreover, there is a polynomial time algorithm that rounds any
	fractional solution to an integral solution whose cost is at most
	 times larger than the cost of the fractional solution.
\end{theorem}

\begin{theorem} \label{thm:cds-gap-minor-free}
	The standard LP relaxation for the \minCDS problem has an 
	integrality gap in planar and minor-closed families of graphs.
	Moreover, there is a polynomial time algorithm that rounds any
	fractional solution to an integral solution whose cost is at most
	 times larger than the cost of the fractional solution.
\end{theorem}

\vspace{-0.1in}
\mypar{Other related work:}
Domatic partitions have received considerable attention; we refer the
reader to \cite{HedetniemiL91, HaynesHS98a, HaynesHS98b} for a
comprehensive treatment of graph domination. Feige \etal
\cite{FeigeHKS02} gave a polynomial time algorithm that constructs a
domatic partition of size , where  is
the minimum degree of the graph and they showed that this is best
possible unless .
C\u{a}linescu \etal \cite{CalinescuCV09} considered the more general
problem of packing disjoint bases in a polymatroid.


\section{Algorithm for fractional connected domatic packings}
\label{sec:cds-packing}

In this section, we give polynomial time algorithms for constructing
fractional connected domatic packings in node-capacitated graphs.

We start by introducing the following natural LP relaxation for the
\minCDS problem. Let  be a graph with costs 
associated with the nodes. For each vertex , we let 
denote the set of all neighbors of  in . For a set  of
nodes, we let  denote the set of all nodes  such that
 is not in  and  has a neighbor in .  Let . The relaxation has a variable  for
each vertex  with the interpretation that  iff  is in
the connected dominating set. Let  be the collection of all sets
 such that , , and  are all
non-empty; note that, for each set , the set 
is a node separator that separates  from .
The relaxation \minCDSlp is given below.

\begin{center}
\begin{boxedminipage}{0.45\textwidth}
\vspace{-0.15in}

\end{boxedminipage}
\end{center}

Note that the LP is a valid relaxation for the \minCDS problem. A
dominating set must contain a vertex from  for each
vertex . Additionally, a connected dominating set must contain a
vertex from each node separator.

The two main steps of our approach for constructing large fractional
packings are the following. The first step is to show that we can
construct in polynomial time a packing of size , where
 is the capacity of a minimum node separator in  and  is an
upper bound on the integrality gap of \minCDSlp; we refer the reader
to Corollary~\ref{cor:cds-pack-decomp} for a precise statement of the
result.  The second step is to upper bound the integrality gap of
\minCDSlp. For general graphs, it follows easily from previous work
that the integrality gap is  and thus we can find a
packing of size . For planar graphs and more
generally, minor-free graphs, we will show that the integrality gap
of \minCDSlp is a constant and thus we can find a packing of size
.

In the following, we say that a rounding algorithm  for an LP
relaxation is an \emph{-approximate} rounding algorithm for the
LP if, given any fractional solution to the LP, the algorithm
constructs an integral solution of value at most  times the value
of the fractional solution.

Consider an instance  of \capCDSpack, where
 is a graph from a family  of graphs and 
is a capacity function on the nodes of . Let  be the minimum
capacity of a node separator in . Our goal is to show that we can
construct a fractional packing of size  provided that we
have an -approximate rounding algorithm for \minCDSlp.  This will
follow from the theorem below, which is an immediate corollary of
Theorem~{2} in \cite{CarrV02}.

\begin{theorem}[Carr and Vempala \cite{CarrV02}] \label{thm:convex-decomp}
	Let  be a family of graphs. Let  be a fractional
	solution to \minCDSlp for an instance of \minCDS for which the
	graph  is in . Let  be a polynomial time rounding
	algorithm for \minCDSlp that is -approximate on instances for
	which the graph is in . Given  and , we can find
	in polynomial time a collection of polynomially many connected
	dominating sets  with associated weights
	 such that  and, for each vertex , we have .
\end{theorem}

\noindent
The theorem above gives us the following corollary.

\begin{corollary} \label{cor:cds-pack-decomp}
	Let  be a family of graphs. Let  be a polynomial time
	rounding algorithm for \minCDSlp that is -approximate on
	instances for which the graph is in .  Let  be an instance of \capCDSpack such that .
	Let  be the minimum capacity of any node separator in .
	Given  and , we can find in polynomial
	time a collection of polynomially many connected dominating sets
	 and associated weights  such that 
	and, for each vertex , we have . Differently said,  is a feasible
	fractional connected domatic packing of size .
\end{corollary}
\begin{proof}
	Consider the following fractional solution :  for each vertex . We can verify that
	 is a feasible solution to \minCDSlp as follows. Consider a
	vertex . We can assume that  is a node separator;
	otherwise,  and  trivially holds. Since  is a
	node separator, it follows that .
	Therefore we have
		
	and thus  satisfies the first set of constraints. Consider a
	set . Since  is a node separator in  it
	follows that . Therefore we have
		
	and thus  satisfies the second set of constraints.

	We apply Theorem~\ref{thm:convex-decomp} to  and  in
	order to get a collection of connected dominating sets  and associated weights . For each , let . We can verify that  is the desired packing as
	follows. We have
		
	Additionally, for each vertex , we have
		
\end{proof}

\medskip\noindent
In the second step, we upper bound the integrality gap of \minCDSlp.
To this end, we will first relate the integrality gap of \minCDSlp to
the integrality gaps of the standard LP relaxations for the
minimum-cost \prob{Dominating Set} (\minDS) problem and the minimum
node-weighted \prob{Steiner Tree} (\nwST) problem, and then we will
upper bound the integrality gaps of these two relaxations. The LP
relaxation for \minDS is the relaxation \minDSlp given below.  In the
\nwST problem, we are given a graph  with non-negative
weights  on the nodes and a set  of nodes called
\emph{terminals}. The goal is to select a minimum weight subgraph 
of  that spans all the terminals, where the weight of  is the
total weight of the nodes in . (Note that we may assume that 
is a node-induced connected subgraph of .) Let  be the
collection consisting of all sets  such that  separates the
terminals; more precisely,  and  are both
non-empty.  For each set , at least one vertex in
 must be in the solution. The LP relaxation for \nwST is
the relaxation \nwSTlp given below.

\begin{center}
\begin{boxedminipage}{0.45\textwidth}
\vspace{-0.15in}

\end{boxedminipage}
\begin{boxedminipage}{0.45\textwidth}
\vspace{-0.15in}

\end{boxedminipage}
\end{center}

\noindent
The following straightforward propositions allow us to relate the
integrality gap of \minCDSlp to the integrality gaps of \minDSlp and
\nwSTlp.

\begin{prop} \label{prop:feasible-dominating-set}
	Consider an instance of \minCDS; let  be the input graph and
	let  be a feasible solution to \minCDSlp for this instance.
	Then  is a feasible solution to \minDSlp for any instance of
	\minDS in which the input graph is .
\end{prop}

\begin{prop} \label{prop:feasible-steiner-tree}
	Consider an instance of \minCDS; let  be the input graph and
	let  be a feasible solution to \minCDSlp for this instance.
	Let  be any subset of the vertices and let  be the
	following fractional solution:  if  and
	 otherwise. Then  is a feasible solution to
	\nwSTlp for any instance of \nwST in which the input graph is 
	and the set of terminals is .
\end{prop}
\begin{proof}
	Consider a set . If  is
	non-empty,  and the fact that  is a feasible
	solution to \minCDSlp gives us that  is at least
	one. Therefore we may assume that . Since
	 separates the terminals,  contains a terminal and
	thus  is at least one.
\end{proof}

\begin{corollary}
	Let  be a family of graphs.  Let  be a polynomial
	time rounding algorithm for \minDSlp that is -approximate
	on instances in which the graph is in . Let
	 be a polynomial time rounding algorithm for \nwSTlp that
	is -approximate on instances in which the graph is in .
	Given  and , we can design a
	polynomial time a rounding algorithm for \minCDSlp that is
	-approximate on instances in which the graph is in
	.
\end{corollary}
\begin{proof}
	Let  be an instance of \minCDS, where . Let  be a feasible solution to \minCDSlp for this
	instance. Let . Our goal is to
	show that we can construct in polynomial time a connected
	dominating set  whose cost  is at most
	. By Proposition~\ref{prop:feasible-dominating-set},
	 is a feasible solution to \minDSlp for the instance
	.  Thus we can run  with  as input
	in order to get a dominating set  such that . Once we have the dominating set , we consider the
	following instance of \nwST. The nodes in  will be the
	terminals. We define a set of weights as follows: for each vertex
	, we have  if  and 
	otherwise. We define a fractional solution  as follows: for
	each vertex , we have  if  and
	 otherwise. Let ;
	note that . By
	Proposition~\ref{prop:feasible-steiner-tree},  is a
	feasible solution to \nwSTlp for the instance .
	Thus we can run  with  as input in order to get a
	node-induced connected subgraph  of  that spans  and it
	has weight . Let ; since 
	is a subset of ,  is a dominating set. Additionally,
	.
\end{proof}

\medskip\noindent
Consider the relaxation \minDSlp. For general graphs, we can show an
 upper bound on the integrality gap using the following
standard randomized rounding approach. Given a fractional
solution , we select a set  of nodes as follows: for each
vertex , we add  to  independently at random with
probability , where  is a large
enough constant. With high probability, the resulting set  is a
dominating set. For minor-closed families of graphs, we give a
primal-dual algorithm in Section~\ref{sec:min-cost-ds-planar} that
shows that the integrality gap is . We remark that the \minDS
problem admits a  in planar graphs \cite{Baker94}, but the
algorithm of \cite{Baker94} does not give an upper bound on the
integrality gap of the LP.

\begin{theorem} \label{thm:min-ds-planar-integrality-gap}
	Let  be a minor-closed family of graphs. There is a
	polynomial time rounding algorithm for \minDSlp that is
	-approximate on instances in which the graph is in ,
	where  is a constant that depends only on the family
	.
\end{theorem}

\noindent
Finally, consider the relaxation \nwST. Guha \etal \cite{GuhaMNS99}
showed that the integrality gap is  for general graphs,
and Demaine \etal \cite{DemaineHK09} showed that the integrality gap
is  for minor-closed families of graphs.  This completes the
proof of Theorem~\ref{thm:cds-packing-general} and
Theorem~\ref{thm:cds-packing-minor-free}.

\section{Algorithm for \minDS in minor-closed families of graphs}
\label{sec:min-cost-ds-planar}

In this section, we give a primal-dual algorithm for the minimum cost
\prob{Dominating Set} problem (\minDS) in minor-closed families of
graphs that achieves a constant factor approximation. The algorithm
will also establish a matching upper bound on the integrality gap of
the standard LP relaxation for the problem that was given in
Section~\ref{sec:cds-packing}.

Let  be a node-weighted graph, and let  denote
the cost of . As before, for each vertex , we let 
denote the set of all neighbors of  in . Let . The primal and dual LPs are described below;
we omit the constraint  from the primal LP, since it is
redundant.

\begin{center}
\begin{boxedminipage}{0.45\textwidth}
\vspace{-0.15in}

\end{boxedminipage}
\vspace{0.1in}
\begin{boxedminipage}{0.45\textwidth}
\vspace{-0.15in}

\end{boxedminipage}
\end{center}

\noindent
The algorithm is based on the primal-dual framework of Goemans and
Williamson \cite{GoemansW95}. The algorithm selects a dominating set
 for . Initially,  consists of all vertices with zero cost.
We also maintain a dual solution ; initially,  for all
. We proceed in iterations. Consider an iteration  and
let  be the set of nodes selected in the first 
iterations.  Let  be the set of all vertices  such that
 is empty. If  is empty, 
is a dominating set and we return .  Otherwise, we
increase the dual variables  uniformly
until a dual constraint for a node  becomes tight, i.e., we have
; we add all the tight
vertices to .

We note that, in each iteration , it is possible to increase the
dual variables corresponding to the nodes of . The set  contains all the vertices whose dual constraints are tight at the
beginning of iteration . Thus, at the beginning of iteration ,
for each vertex  and each vertex  such that , the dual constraint corresponding to  is slack,
i.e., we have . Therefore
the algorithm terminates in at most  iterations.

Finally, we perform a \emph{reverse-delete step}. Let  be the
dominating set selected by the primal-dual algorithm. We select a
subset  as follows. We start with . We order the vertices
of  in the reverse of the order in which they were selected by the
primal-dual algorithm. We consider the vertices of  in this order.
Let  be the current vertex. If  is a dominating set, we
remove  from .

The algorithm described above is well-defined on general graphs, but
its approximation is . In the following, we show that we
can take advantage of the fact that minor-free graphs are sparse in
order to show that the algorithm achieves a constant factor
approximation in minor-closed families of graphs; the constant
depends on the family.

We start by noting that the dual solution  satisfies the
complementary slackness conditions.

\begin{prop} \label{prop:complementary-slackness}
	For each vertex , we have .
\end{prop}

\noindent
The following lemma gives us a very convenient way to upper bound the
approximation ratio. The lemma follows from a standard primal-dual
analysis and the fact that the algorithm increases the dual variables
uniformly in each iteration. Recall that  is the final dominating
set after performing reverse-delete, and  is the set of
vertices selected in the first  iterations of the algorithm.

\begin{lemma} \label{lem:approx-condition}
	Let . Suppose that there exists a 
	such that, for each iteration  of the algorithm, we have
		
	Then the cost of  is at most , where
	 is the cost of the optimal solution to \minDSlp.
\end{lemma}
\begin{proof}
	By Proposition~\ref{prop:complementary-slackness}, we have
		
	By rearranging the second summation, we get that
		
	Since  is a feasible dual solution, by weak duality, we have
		
	Therefore it suffices to show that
		
	We can prove the inequality above by induction on the number of
	iterations. Initially,  for all vertices  and the
	inequality clearly holds. Now consider an iteration .
	Let  be the amount by which the dual variables
	 are increased in iteration . The
	right-hand side of the inequality increases by . Thus, if we can show that the left-hand side
	increases by at most , the inequality
	will follow. The left hand side of the inequality
	increases by . For each , we have  is empty, and thus
		
	where the last inequality follows from the assumption in the
	statement of the lemma. Therefore the left-hand side increases by
	at most , and the lemma follows.
\end{proof}

\medskip\noindent
Therefore, in order to upper bound the approximation ratio of the
algorithm, it suffices to prove the following key lemma. The lemma
follows from the minimality of  and the fact that minor-free
graphs are sparse, in the sense that the number of edges is
proportional to the number of vertices.

\begin{lemma} \label{lem:counting}
	Suppose that the input graph  belongs to a minor-closed
	familty  of graphs. There is a constant  depending
	only on  such that, for each iteration  of the algorithm,
	we have
		
	where .
\end{lemma}

\noindent
We devote the rest of this section to the proof of
Lemma~\ref{lem:counting}. We will prove the lemma in two steps. In
the first step, we use the sparsity of minor-free graphs to show that
the sum  is
at most a constant times larger than . In
the second step, we use the minimality of  to show that
.

\begin{lemma} \label{lem:counting-first}
	Let  be a constant such that, for each graph , we have .  For each
	iteration , we have
		
\end{lemma}
\begin{proof}
	Consider an iteration  of the algorithm.  We have
		
	We can upper bound the second sum in the equation above as
	follows.  Let  be the subgraph of  whose vertices are
	 and whose edges are all the edges of  with one
	endpoint in  and the other in . Note that
	 is equal to
	the number of edges of .  Let  be the subgraph of 
	whose vertices are  and whose edges are all the edges of 
	with one endpoint in  and the other in .
	Finally, let  be the subgraph of 
	induced by . Note that  is equal to the number of edges of
	 plus the number of edges of . Therefore we have
		
	Therefore we have
		
\end{proof}

\begin{lemma} \label{lem:counting-second}
	For each iteration , we have .
\end{lemma}
\begin{proof}
	Consider a vertex . We claim that, since we could not
	remove  in the reverse-delete step, there is a vertex  such that .
	We can show this as follows. Since we could not remove , there
	is a vertex  such that . Since  is not dominated by
	,  is in . Thus each vertex  has a
	\emph{witness} vertex  such that . Now we claim that each vertex  is a witness vertex for at most one vertex of .
	Suppose for contradiction that a vertex  is a witness
	vertex for two vertices  and  in . Without loss of
	generality,  was selected by the algorithm after .
	Consider the iteration of the reverse-delete step that considered
	. At this point  had not been considered yet and thus
	it is in . Thus ,
	which contradicts the fact that . Therefore , as
	desired.
\end{proof}

\medskip\noindent
Lemma~\ref{lem:counting} follows from Lemma~\ref{lem:counting-first}
and Lemma~\ref{lem:counting-second}. We have the following upper
bounds on the constant  (see
Lemma~\ref{lem:counting-first}). If  is a minor-closed family,
there is a constant-sized graph  such that  is the family of
all graphs that do not have  as a minor. As shown by Kostochka
\cite{Kostochka84}, we have  for
the family of -minor-free graphs. If  is a planar graph, we
have ; if  is also bipartite, the constant improves
to . Thus the algorithm achieves an -approximation for planar
graphs.

\begin{remark}
	The algorithm above can be easily adapted to give a constant
	factor approximation for the minimum cost \prob{Steiner
	Dominating Set} problem in minor-closed families of graphs. In
	the Steiner problem, we are given a subset of vertices called
	terminals and the goal is to select a set that dominates the
	terminals.
\end{remark}

\begin{remark}
	A constant factor approximation for the minimum cost \prob{Steiner
	Dominating Set} problem in minor-free graphs can also be obtained
	via iterated rounding.
\end{remark}

\mypar{Acknowledgements:}
The results in Section~\ref{sec:cds-packing} were developed in joint
work with Chandra Chekuri and we thank him for his help. We also
thank Chandra for several other fruitful discussions and suggestions.
\newpage
\bibliographystyle{plain}
\bibliography{cds}


\end{document}
