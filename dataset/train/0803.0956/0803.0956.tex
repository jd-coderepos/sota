
\documentclass[11pt]{article}
\usepackage[latin1]{inputenc}
\usepackage{amsmath}
\usepackage{amsfonts}
\usepackage{amssymb}
\usepackage{latexsym}
\usepackage{graphicx}
\usepackage[a4paper]{geometry}

\parskip=3pt

\newtheorem{theorem}{Theorem}
\newtheorem{lemma}{Lemma}
\newtheorem{claim}{Claim}
\newtheorem{corollary}{Corollary}

\newenvironment{proof}{\noindent \emph{Proof.}\ }{\hfill
    \vspace{1em}}

\newenvironment{proofcl}{\noindent \emph{Proof.}\ }{Thus the claim
holds.  \hfill \vspace{1em}}



\title{Characterizing path graphs by forbidden induced subgraphs}
\author{Benjamin L\'ev\^eque\thanks{ROSE, EPFL, Lausanne, Switzerland}
\and Fr\'ed\'eric Maffray\thanks{C.N.R.S., Laboratoire G-SCOP,
Grenoble, France} \and Myriam Preissmann}
\begin{document}
\maketitle

\begin{abstract}
A path graph is the intersection graph of subpaths of a tree.  In
1970, Renz asked for a characterization of path graphs by forbidden
induced subgraphs.  We answer this question by determining the
complete list of graphs that are not path graphs and are minimal with
this property.
\end{abstract}

\section{Introduction}

All graphs considered here are finite and have no parallel edges and
no loop.  A \emph{hole} is a chordless cycle of length at least four.
A graph is \emph{chordal} (or \emph{triangulated}) if it contains no
hole as an induced subgraph.  Gavril~\cite{Gav74} proved that a graph
is chordal if and only if it is the intersection graph of a family of
subtrees of a tree.  In this paper, whenever we talk about the
intersection of subgraphs of a graph we mean that the \emph{vertex
sets} of the subgraphs intersect.

An \emph{interval graph} is the intersection graph of a family of
intervals on the real line; equivalently, it is the intersection graph
of a family of subpaths of a path.  An \emph{asteroidal triple} in a
graph  is a set of three non adjacent vertices such that for any
two of them, there exists a path between them in  that does not
intersect the neighborhood of the third.  Lekkerkerker and
Boland~\cite{LB62} proved that a graph is an interval graph if and
only if it is chordal and contains no asteroidal triple.  They derived
from this result the list of minimal forbidden subgraphs for interval
graphs.

An intermediate class is the class of path graphs.  A graph is a
\emph{path graph} if it is the intersection graph of a family of
subpaths of a tree.  Clearly, the class of path graphs is included in
the class of chordal graphs and contains the class of interval graphs.
Several characterizations of path graphs have been
given~\cite{Gav78,MW86, Ren70} but no characterization by forbidden
subgraphs was known, whereas such results exist for intersection
graphs of subpaths of a path (interval graphs~\cite{LB62}), subtrees
of a tree (chordal graphs~\cite{Gav74}), and also for directed
subpaths of a directed tree~\cite{Pan99}.

In 1970, Renz~\cite{Ren70} asked for a complete list of graphs that
are chordal and not path graphs and are minimal with this property,
and he gave two examples of such graphs.  Reference~\cite{TonGutSzw05}
extends the list of minimal forbidden subgraphs for path graphs; but
that list is incomplete.  Here we answer Renz's question and obtain a
characterization of path graphs by forbidden induced subgraphs.  We
will prove that the graphs presented in
Figures~\ref{fig:nosimpl}--\ref{fig:4cospe} are all the minimal
non-path graphs.  In other words:
\begin{theorem}
    \label{th:main}
    A graph is a path graph if and only if it does not contain any of
     as an induced subgraph.
\end{theorem}



\section{Special simplicial vertices in chordal graphs}

In a graph , a \emph{clique} is set of pairwise adjacent vertices.
Let  be the set of all (inclusionwise) maximal cliques
of .  When there is no ambiguity we will write  instead
of .

Given two vertices  in a graph , a \emph{-separator}
is a set  of vertices of  such that  and  lie in two
different components of  and  is minimal with this
property.  A set is a \emph{separator} if it is a -separator
for some  in .  Let  be the set of separators
of .  When there is no ambiguity we will write  instead
of .

The neighborhood of a vertex  is the set  of vertices
adjacent to .  Let us say that a vertex  is \emph{complete} to a
set  of vertices if .  A vertex is
\emph{simplicial} if its neighborhood is a clique.  It is easy to see
that a vertex is simplicial if and only if it does not belong to any
separator.  Given a simplicial vertex , let  and
.  Since  is simplicial, we have
.  Remark that  is not necessarily in
; for example, in the graph  with vertices  and edges , we have  and
.

A classical result~\cite{HajSur58,Ber60} (see also~\cite{Gol04})
states that, in a chordal graph , every separator is a clique;
moreover, if  is a separator, then there are at least two
components of  that contain a vertex that is complete to
, and so  is the intersection of two maximal cliques.

A \emph{clique tree}  of a graph  is a tree whose vertices are
the members of  and such that, for each vertex  of ,
those members of  that contain  induce a subtree of
, which we will denote by .  A classical result~\cite{Gav74}
states that a graph is chordal if and only if it has a clique tree.

For a clique tree , the \emph{label} of an edge  of  is
defined as .  Note that every edge  satisfies
; indeed, there exist vertices  and , and the set  is a -separator.  The number of times an element  of 
appears as a label of an edge is equal to , where  is the
number of components of  that contain a vertex complete
to  \cite{Gav74,MacMac}.  Note that this number is at least one and
that it depends only on  and not on , so for a given  it is the same in every clique tree.

Given , let  denote the subgraph of 
induced by all the vertices that appear in members of .  If  is
a clique tree of , then  denotes the subtree of  of
minimum size whose vertices contains .  Note that if , then
 is a path.

Given a subtree  of a clique-tree  of , let 
be the set of vertices of  and  be the set of
separators of .

Dirac~\cite{Dir61} proved that a chordal graph that is not a clique
contains two non adjacent simplicial vertices.  We need to generalize
this theorem to the following.  Let us say that a simplicial vertex
 is \emph{special} if  is a member of  and is
(inclusionwise) maximal in .


\begin{theorem}
\label{special}
In a chordal graph that is not a clique, there exist two non adjacent
special simplicial vertices.
\end{theorem}
\begin{proof}
We prove the theorem by induction on .  By the
hypothesis,  is not a clique, so  and .

\emph{Case 1:  has only one maximal element .} Let
 be two maximal cliques such that .  Let  and .  The set  is the only
maximal separator and it does not contain  or .  So  and
 do not belong to any element of , and so they are
simplicial and , so they are special.

\emph{Case 2:  has two distinct maximal elements }.
So .  Let  be a clique tree of .  Let  be members of  such that ,
, and  appear in this order along
the path  (possibly ).  Let  be
the subtree of  that contains , and let  be
the tree that consists of  plus the vertex  and the edge
.  The subtree  does not contain , so  has strictly fewer maximal cliques than ; and  is not a
clique.  By the induction hypothesis, there exist two non adjacent
simplicial vertices  of  such that  are maximal elements of .  At most one of 
is in  since they are not adjacent, say  is not in .  We
claim that  is a simplicial vertex of  and that  is a
maximal element of .  Vertex  does not belong to any
element of .  If it belongs to an element of , then it must also belong to , a contradiction.  So  does not belong to any
element of  and so it is a simplicial vertex of .  The
set  is a maximal element of .  If it is not a
maximal element of , then it is included in , a contradiction.  So  is a special simplicial vertex of .
Likewise, let  be the subtree of  that contains
, and let  be the tree that consists of  plus the vertex
 and the edge .  Just like with , we can find a
simplicial vertex  of  not in  that is a
simplicial vertex of  with  being a maximal element of
.  Vertices  and  are not adjacent since  is a
-separator.  So  and  are the desired vertices.
\end{proof}

Algorithms LexBFS~\cite{RTL76} and MCS~\cite{TY84} are linear time
algorithms that were developed to find a simplicial vertex in a
chordal graph.  But a simplicial vertex found by these algorithms is
not necessarily special.  For example, on the graph with vertices
 and edges , every application
of LexBFS or MCS will end on one of simplicial vertices , which
are not special.  The proof of Theorem~\ref{special} can be turned
into a polynomial time algorithm to find a special simplicial vertex
in a chordal graph.  We do not know how to find such a vertex in
linear time.

\section{Forbidden induced subgraphs}

A \emph{clique path tree}  of  is a clique tree of  such
that, for each vertex  of , the subtree  induced by cliques
that contain  is a path.  Gavril \cite{Gav78} proved a graph is a
path graph if and only if it has a clique path tree.

Consider graphs  presented in
Figures~\ref{fig:nosimpl}--\ref{fig:4cospe}.  Let us make a few
remarks about them.  Each graph in Figure~\ref{fig:univ} is obtained
by adding a universal vertex to some minimal forbidden subgraph for
interval graphs.  Clearly, in a path graph the neighborhood of every
vertex is an interval graph; so  are not path
graphs.  Graphs  are also forbidden in interval
graphs.  Graphs  and  are from Renz~\cite[Figures~1
and~5]{Ren70}.  For ,
Panda~\cite{Pan99} proved that  is a minimal non directed path
graph, so  is a directed path graph for every vertex
 (obviously every directed path graph is a path graph).  In general
we have the following:
\begin{theorem}
    \label{th:minimal}
     are minimal non path graphs.
\end{theorem}
\begin{proof}
Clearly,  is a minimal non path graph.  For the other graphs, we
prove the theorem in one case and then show how the same arguments can
be applied to all cases.

Consider , ; see Figure~\ref{fig:4simpl}.  Name
its vertices such that  are the simplicial
vertices of degree , clockwise;  are the two
neighbors of  (), with subscripts modulo
; and  are the remaining vertices.  Let  be the
maximal clique that contains  (), and call
these  cliques ``peripheral''.  Let  and  be the maximal
cliques that contain respectively  and , and call these two
cliques ``central''.  Thus .  Since  is chordal, it admits a clique tree.  Let 
be any clique tree of .  Then  and  are adjacent in 
(for otherwise, there would be at least one interior vertex  on the
path , so we should have , but no
member  of  satisfies this
inclusion).  By the same argument, each  ()
must be adjacent to  or  in .  Suppose that we are trying
to build a clique path tree  for .  By symmetry, we may assume
that  is adjacent to .  Then, for 
successively,  must be adjacent to  (if  is even) and to
 (if  is odd) in , for otherwise, for some  the subtree  induced by the cliques that contain  would
not be a path.  Note that in this fashion we obtain a clique path tree
 of .  Now if  is adjacent to
, then the subtree  is not a path, and if if
 is adjacent to , then the same holds for
.  This shows that  is not a path graph.

Now consider any vertex  of .  If  is one of the 's,
then by symmetry we may assume that , and we have seen
above that  is a path graph with clique path tree .
Suppose that  is one of the 's, say .  Then by
adding vertex  and edge  to , it is easy to
see that we obtain a clique path tree of .  Finally,
suppose that  is one of , say .  Then the tree with
vertices  and edges  is a clique path tree of .  So  is a
minimal non path graph.

When  is any other  (), the same arguments
apply as follows.  For , call peripheral the three
cliques that contain a simplicial vertex.  For ,
call peripheral the cliques that contain a simplicial vertex of degree
, plus, in the case of , the clique that contain the bottom
simplicial vertex (which has degree ).  Call central all other
maximal cliques.  Then it is easy to prove, as above, that the central
cliques must form a subpath in any clique tree of , and all the
peripheral cliques except one can be appended to either end of that
subpath, but whichever way this is done, when the last clique is
appended, the subtree  is not a path for some vertex  of .
Moreover, when any vertex  is removed, it is possible to build a
clique path tree for .
\end{proof}



    \begin{figure}[e]
      \centering
\begin{tabular}{c}
\includegraphics[scale=0.5]{t.eps} \\
 \\
\end{tabular}
\caption{Forbidden subgraphs with no simplicial vertices}
\label{fig:nosimpl}
    \end{figure}


    \begin{figure}[e]
      \centering
\begin{tabular}{ccccc}
\includegraphics[scale=0.5]{a.eps} \ &\
\includegraphics[scale=0.5]{h.eps} \ &\
\includegraphics[scale=0.5]{i.eps} \ &\
\includegraphics[scale=0.5]{q.eps} \ &\
\includegraphics[scale=0.5]{f.eps} \\
 &  &  &  & 
\end{tabular}
      \caption{Forbidden subgraphs with a universal vertex}
\label{fig:univ}
    \end{figure}

    \begin{figure}[e]
      \centering
\begin{tabular}{ccccc}
\includegraphics[scale=0.5]{b.eps} \ &\
\includegraphics[scale=0.5]{c.eps} \ &\
\includegraphics[scale=0.5]{r.eps} \ &\
\includegraphics[scale=0.5]{p.eps}  \ &\
\includegraphics[scale=0.5]{g.eps} \\
 &  &  &  &  \\
\end{tabular}
      \caption{Forbidden subgraphs with no universal vertex and exactly
      three simplicial vertices}
\label{fig:3simpl}
    \end{figure}

    \begin{figure}[e]
      \centering
\begin{tabular}{ccccc}
\includegraphics[scale=0.5]{l.eps}\ &\
\includegraphics[scale=0.5]{o.eps} \ &\
\includegraphics[scale=0.5]{u.eps}\ &\
\includegraphics[scale=0.5]{n.eps}  \ &\
\includegraphics[scale=0.5]{m.eps} \\
 &  &
 &  &  \\
\end{tabular}
\caption{Forbidden subgraphs with at least one simplicial vertex that
    is not co-special. (bold edges form a clique)}
\label{fig:4simpl}
    \end{figure}


    \begin{figure}[e]
      \centering
\begin{tabular}{c}
\includegraphics[scale=0.5]{v.eps} \\
 \\
\end{tabular}
\caption{Forbidden subgraphs with  simplicial vertices that
    are all co-special. (bold edges form a clique)}
\label{fig:4cospe}
    \end{figure}



\section{Co-special simplicial vertices}

Let us say that a simplicial vertex  is \emph{co-special} if 
is a separator such that  has exactly two components.
Note that in that case  is a minimal element of  and
it appears exactly once as a label of any path tree of .


\begin{lemma}
\label{cospecial}
Let  be a minimal non path graph.  Then either  is one of
 or every simplicial vertex of  is
co-special.
\end{lemma}
\begin{proof}
Suppose on the contrary that  is a minimal non path graph,
different from , and there is a simplicial
vertex  of  that is not co-special.  All simplicial vertices of
 are co-special, so  is not any of these
graphs; moreover it does not contain any of them strictly (for
otherwise  would not be minimal).  Therefore  contains none of
.

Let  be a clique path tree of .  Let  be such that .  If
, then we can add  to  to obtain a clique path tree of
, a contradiction.  So , and  (as
there is a vertex  and  is a -separator).

Let  be the maximal subtree of  that contains  and such
that no label of the edges of  is included in .  Remark that
 plus vertex  and edge  is a clique tree of  (but not necessarily a clique path tree), and in
that tree only one label is included in .  Since  is not
co-special, there is an edge of  whose label is included in
, and so  has strictly fewer vertices than .  So
 is a path graph.  Let  be a clique
path tree of this graph.

We claim that  is a leaf of .  If not, then there are at least
two labels of  that are included in , which contradicts the
definition of  (the number of times a label appears in a clique
tree is constant).

Let  be the subtrees of 
().  For , let  be the edge
between  and  with  and .  Note that
 are pairwise disjoint (but  are not necessarily pairwise disjoint).  Let  and .  Let  be the intersection graph of , that is,  and
.

\begin{claim}
 contains no odd cycle.
\end{claim}
\begin{proofcl}
Suppose on the contrary, without loss of generality, that
--- is an odd cycle in , with
length  ().  Let  (), with .  Suppose that for some  we
have ; then there is a common vertex in
the cliques , and the number of
different cliques among these is at least three, which contradicts the
fact that  is a clique path tree as these three cliques do not
lie on a common path of .  For , let .
By the preceding remark, the 's are pairwise distinct.  By the
definition of , we have  for each , so the 's are all in  and .  Let .  Let us consider the subgraph induced by .  Each of the non-adjacent vertices  and  is
adjacent to all of the clique formed by the 's.  Each vertex
 is adjacent to  and  (with ) and not to
any other  or to .  Vertex  can have at most two neighbors
among the 's.  If  has zero or one neighbor among them, then
 induce respectively
 or .  If  has two
consecutive neighbors , then  induce .  If  has two
non-consecutive neighbors , then we can assume that
 and  is odd,  with , and
then  induce
.  In all cases we obtain a contradiction.
\end{proofcl}


By the preceding claim,  is a bipartite graph.

For , let .  Let .
\begin{claim}
 contains no odd path between two vertices in .
\end{claim}
\begin{proofcl}
Suppose on the contrary, without loss of generality, that
-- is an odd path in  between two
vertices  of  (with , ), and assume that
 is minimum with this property.  By the minimality, all interior
vertices  () are not in .  For , let 
be a vertex in .  As in the preceding claim, the
's are pairwise distinct and lie in  and .  Let  be the
path .  If , then  is not in , so
, for otherwise  would not be a path; then 
is not in , so , and so on.  Thus the two extremities
of  are  and .  Since  and  are in , the sets  are non empty.



Let  be the closest vertex to  in  such that there
exists an edge incident to  with label in , and let
 be such an edge and  be its label (such an edge exists
because ).  Similarly, let  be the
closest vertex to  in  such that there exists an edge
incident to  with label in , and let  be
such an edge and  be its label.  So ,
 and , .
Each of  may be in  or not.  Since  is a clique path
tree,  lies between  and  and between  and 
along .  So  lie in this order on ,
and  is included in all labels between  and  in ,
and  is included in all labels between  and  in .

Let  and .
Since  is a clique path tree,  and  are
distinct from  and not adjacent to .

Let  and .  Then  and
 are adjacent, and  and  are adjacent.  Since 
is a clique path tree, if  or  is not in , then  and
 are different from each other, from  and
from .  Furthermore, if  is not in ,
then  is not adjacent to any of ; and if 
is not in , then  is not adjacent to any of .

Let  and .  Then
 and  are not adjacent, and  and  are not
adjacent.  Since  is a clique path tree, if  or  is in
, then  and  are different from each other, from  and from .  Furthermore, if
 is in , then  is adjacent to  and to ; and if  is in , then  is adjacent to  and
to .

Note that the set 
induces a clique in .  Moreover,  is adjacent to , 
is adjacent to , for ,  is adjacent to
 and , and there is no other edge between  and that clique.

Suppose that .  Then  and  is not in .
By the definition of , there exists .
Vertex  is distinct from all 's as it is not in , and it
is adjacent to all of  and to none of .  Then  induce
, a contradiction.  So , and 
and  are distinct non adjacent vertices.  We can choose
vertices  () not in  and on the labels
of  such that ---- is a
chordless path in .  Vertices  are distinct from
and adjacent to , and they are distinct
from and not adjacent to any of .

Suppose that  and .  Then  and  are
not in .  If , then   induce .  If , then   induce
.  If , then  induce , a contradiction.

Suppose now that  and .  Then  is not
in  and we may assume that  is in .  If , then   induce
.  If , then  induce , a contradiction.

Suppose finally that  and .  Then we may
assume that  and  are in .  If , then  induce .  If , then  induce .  If , then
 induce
, a contradiction.
\end{proofcl}


By the preceding two claims,  is a bipartite graph  such that .  Now all the subtrees 
can be linked to  to get a clique path tree of  as follows.  For
each , we add an edge  between  and .  This
creates a clique path tree on the corresponding subset of cliques
because  is a stable set of  and  is a leaf of .
For each , let  be such that
 and the length of  is
maximal.  Since , we have , so
 and we can add an edge  between  and
.  This creates a clique path tree of  because  is a stable
set of  and by the definition of , a contradiction.
\end{proof}

\section{Characterization of path graphs}

In this section we prove the main theorem, that is, path graphs are
exactly the graphs that do not contain any of .
We could not find a characterization similar to the one found by
Lekkerkerker and Boland~\cite{LB62} for interval graphs (``an interval
graph is a chordal graph with no asteroidal triple'').  We know that
in a path graph, the neighborhood of every vertex contains no
asteroidal triple; but this condition is not sufficient.  So we prove
directly that a graph that does not contain any of the excluded
subgraphs is a path graph.

\begin{lemma}
\label{lem:PAT}
In a graph that does not contain any of ,
the neighborhood of every vertex does not contain an asteroidal
triple.
\end{lemma}

\begin{proof}
    It suffices to check that when a universal vertex is added to a
    minimal forbidden induced subgraph for interval graphs
    (\cite{LB62}), then one obtains a graph that contains one of .  The easy details are left to the reader.
\end{proof}

Given three non adjacent vertices , we say that  is the
\emph{middle} of  if every path between  and  contains a
vertex from .  If  is not an asteroidal triple, then at
least one of them is the middle of the others.


\begin{lemma}
    \label{lem:middle}
In a chordal graph  with clique tree , a vertex  is the
middle of two vertices  if and only if for all cliques  and
 such that  and , there is an edge of the
path  such that  is complete to its label.
\end{lemma}

\begin{proof}
Suppose that  is the middle of .  Let  and  be
cliques such that  and , and suppose there is no
edge of  such that  is complete to its label.  For
each edge on , one can select a vertex that is not
adjacent to .  Then the set of selected vertices forms a path from
 to  that uses no vertex from , a contradiction.

Suppose now that for all cliques  and  with  and
, there is an edge of the path  such that 
is complete to its label.  Suppose that there exists a path
--, with  and  and none of the 's
is in .  We may assume that this path is chordless.  For , let  be a maximal clique containing .
Then  appear in this order along a subpath of .
On each  (), vertex  is not
adjacent to , so  is not complete to any label of , but  contains  and  contains , a
contradiction.
\end{proof}

Now we are ready to prove the main theorem.  Part of the proof has be
done in the previous section.  Lemma~\ref{cospecial} deals with the
case where there exists a simplicial vertex that is the middle of two
other vertices; now we have to look at the case where all simplicial
vertices are not the middle of any pair of vertices.

\paragraph*{Proof of Theorem~\ref{th:main}}
\setcounter{claim}{0}

By Theorem~\ref{th:minimal}, a path graph does not contain any of
.  Suppose now that there exists a graph  that
does not contain any of  and is a minimal non
path graph.  Since  contains no , it is chordal.  By
Theorem~\ref{special}, there is a special simplicial vertex  of
.  By Lemma~\ref{cospecial},  is co-special.  Let  and
.  Let  be a clique path tree of
.  Let  be such
that .  We add the edge  to  to obtain a
clique tree  of .

\begin{claim}\label{clq}
     For all non-adjacent vertices , there exists a path
     between  and  that avoids the neighbourhood of .
\end{claim}
\begin{proofcl}
Suppose the contrary.  Let  be such that 
and .  We have  since  are not adjacent.  By
Lemma~\ref{lem:middle}, there is an edge of  whose label is
included in , contradicting that  is co-special.
\end{proofcl}

For each clique , let  be the
neighbor of  along .  Let .  Let
 be the set of labels of edges incident to  in .
Let  be the clique in  such
that  and no other edge of
 has a label included in .  (Possibly
.)

Let  be the set of cliques  of  such that no element of 
contains~.  For each clique , we
define a subtree  of , where  is the biggest subtree
of  that contains  and for which no label is included in
.  Note that  is in  and  is not
in .  Since  is special and co-special we have , so  contains .

\begin{claim}
   \label{claim:L'}
   For each clique  we have .
   \end{claim}
\begin{proofcl}
Suppose on the contrary that .  Then .  When we remove the edges  and 
from , there remain three subtrees , where 
is the subtree that contains ,  is the subtree that contains
 and , and  is the subtree that contains
.  Let  be the tree formed by 
plus the edge .  Then, since ,  is a clique tree of .  The set
 contains strictly fewer maximal cliques than
, so there exists a clique path tree  of .  Label  is on the edge  of
, so it is also a label of .  Consequently there is an edge
 of  with a label  such that .  (Possibly ).  Suppose that .  Then there is an edge of  or  with label
.  But no label of  can be  by the definition of ; and all the labels of  that are included in  are also
included in , so no label of  can be , a
contradiction.  So .  Now if we remove the edge
 from  and replace it by the subtree  and edges 
and , we obtain a clique path tree of , a
contradiction.
\end{proofcl}



Let  be the set of all  such that 
is a strict subtree of .  For every vertex  of
 let  be the subtree of  induced
by the cliques that contain .  Recall that  is a path
because  is a clique path tree.  Let  be the set of vertices
 of  such that  is a vertex of  that is not a leaf.
Then  is not empty, for otherwise  would be a clique path
tree of .  Moreover:
\begin{claim}\label{cla}
    For any , the two leaves of  are in  and
    at least one of them is in~.
    \end{claim}
\begin{proofcl}
Let  be the leaves of , and, for , let
.  We have , and  is
not in any member of .  Thus .  Similarly .  The three vertices  are adjacent to , so they do not form an asteroidal
triple by Lemma~\ref{lem:PAT}, and so one of them is the middle of the
other two.  Vertex  cannot be the middle of  by
Claim~\ref{clq}.  So we may assume up to symmetry that  is the
middle of .  So, by Lemma~\ref{lem:middle}, there is an
edge of  with a label included in .  So
 is a strict subtree of  and so
.
\end{proofcl}

The preceding claim implies that  is not empty.  We
choose  such that the subtree  is maximal.
Let  be the label of the edge of  that is incident
to .  Vertex  is special and co-special, so there exists 
in , and we have .  Therefore no
clique of  contains .  We
add the edge  to  to obtain a clique tree  of
.  Since  is a strict subtree of
, we can consider a clique path tree  of .  Note that  is a leaf of , for otherwise there are at
least two labels of  that are included in , which contradicts
the definition of .

\begin{claim}
\label{cl:orange}
Let  be such that both leaves of  are not in .
Let  be a leaf of  that belongs to .  Then
 is in , and every edge  of  with  satisfies .
\end{claim}
\begin{proofcl}
By Claim~\ref{cla},  exists.  Since the labels of the edges of
 are not included in , they are also not included in
.  So  is a subtree of .  By the maximality of
, we have .  By Claim~\ref{claim:L'},  is in
.  By the definition of , every edge  of  with
 satisfies .
\end{proofcl}

\begin{claim}
\label{claim:p}
There exist  such that
 is an edge of , , ,  and .
\end{claim}
\begin{proofcl}
We define sets  as follows:


We observe that the members of  are pairwise disjoint.
For if there is a vertex  in  for some , then  is on three labels (namely 
and ) of  that do not lie on a common path, contradicting
that  is a clique path tree.

We define sets  () and  () as follows:


Consider the smallest  such that there exists  with .  If no such  exists,
then let .  The claim states that , so let us suppose
on the contrary that .  For all  and all
, we have ; let  be such that  and the length of  is maximal.  Remark that  is included in  if and only
if all members of  that intersect  contain .
Let us prove that:


Suppose that there exists , ,
such that , and let  be minimum with this
property.  Let  be such that
, ,  and .  Pick  and .  Let  be such that , ,
\ldots,  with .  We claim that .  For otherwise there exists  such that .  Then one of 
contains elements of  but not all, and so , which contradicts the minimality of .

By the definition of the 's, none of 
is in .  Let  (maybe ).  So .  None of  can
contain  by the definition of .  Note that  is
in  and ; on the other hand we have .  So there exists a clique  of  such that
, ,  and .  Vertex  is on
 as .  Let .  We
can find vertices  on the labels of 
such that none of them is in  and ---- is
a chordless path in .  Let .  By
Claim~\ref{clq}, there exists a path  between  and  whose
vertices are not neighbors of .

If , then let .  As  is
special and co-special, we have , so let .  Then  form an asteroidal triple
(because of paths , ---- and
---), and they lie in the neighborhood of , a
contradiction.  So .  Let .
If , then  form an asteroidal triple
(because of paths , ---- and
--), and they lie in the neighborhood of , a
contradiction.  So .  The 's are all
included in  and so in  too.  They are pairwise disjoint, for
otherwise  is not a clique path tree.  Vertex  is not in
any of the 's, and  is adjacent to all of  and to none of .

Suppose that .  Then we may assume
that , so  is in  and the two leaves of 
are not in .  By Claim~\ref{cl:orange}, the leaf  of
 that is in  is such that  is in
, so .  But  is in , so it is
not in ; thus , which contradicts the
end of Claim~\ref{cl:orange}.  Therefore , so , , .  Now, if
, then  induce .  If ,
then  induce .  If ,
then  induce
, a contradiction.  Therefore (\ref{u2nde})
holds.

\medskip

Suppose that  is finite.  Let  be such that , ,
, and .
Let  and .  Pick vertices , ,
\ldots,  with .  By the definition of
, none of  is in .  Let .  Suppose that .  Then we
can assume that , so  is in  and the two leaves of
 are not in .  By Claim~\ref{cl:orange}, the leaf
 of  that is in  is such that
 is in , so .  But  is in 
and not in , so , which
contradicts the end of Claim~\ref{cl:orange}.  Therefore , and , , .  Let .  Vertex  is not
adjacent to any of  because
, and by the minimality of , vertex  is
not adjacent to .  Then  induce
, a contradiction.

Now  is infinite.  Then the members of  are included in  and pairwise disjoint, for otherwise 
is not a clique path tree.  For each member  of , let  be the component of 
that contains~.  Starting from the path tree  and the trees
 (), we build a new tree as
follows.  For each , we add the
edge  between  and .  For each , we add the edge  between  and .  For
each , we
add the edge  between  and .  For each , we define  such that  and the length of
 is maximal.  By the definition of , we have
, so , so  is a vertex of 
on  and it contains  as .  Then we
can add the edge  between  and .  Thus we obtain a
clique path tree of , a contradiction.  So , and there exist
 and  such that ,  and .
\end{proofcl}


Let  be as in the preceding claim.  Let .  Vertex  is not adjacent to .  Let  and .

\begin{claim}
\label{claim:egal}
    .
\end{claim}
\begin{proofcl}
Assume on the contrary that .  Then  is a proper
subset of .  Suppose that there exists .  Then  is in  and the two leaves of 
are not in .  By Claim~\ref{cl:orange}, the leaf  of
 that is in  is such that  is in ,
so .  But , so Claim~\ref{cl:orange} is
contradicted.  Therefore .  By the
definition of  and , there exists  and .  So , , .  Since  is in , we have .  The labels of the edges
of  are not included in , so they are also not in .
Thus we can choose vertices  on the labels of
 such that none of the 's is in , ,
, and ---- is a path from  to 
that avoids .  If , then  is different from  and
, and  induce .  If ,
then, if  is adjacent to , vertices  induce , and if  is not adjacent to , vertices
 induce .  Finally, if ,
then  induce .
In all cases we obtain a contradiction.
\end{proofcl}



\begin{claim}
\label{claim:W}
.
\end{claim}

\begin{proofcl}
If , then, by Claim~\ref{claim:egal}, we have
 and , as desired.  So suppose .  By the definition of , there is a vertex , and so .  Let  be the leaves of
 such that  lie in this order on that path.
Let  be the member of  that is closest to  on
.  Clearly .  The edges of  are not
included in , so they are also not in  and not in .
So  contains .  If , then 
by the maximality of , so , which contradicts
Claim~\ref{claim:L'}.  Thus .  This means that
, and so the labels of  are
not included in , in particular .  Let
 be the edge of  such that  contains  and 
does not (maybe , ).  The set  contains  but not
all of , and the members of  do not contain .  So no element of  contains , which means that , a contradiction to the definition of .
\end{proofcl}



By Claim~\ref{claim:W}, we have .  By
Claim~\ref{claim:egal}, we have , so  is also maximal
and what we have proved for  can be done for .  Thus, by
Claim~\ref{claim:p}, there exists  such that  is an
edge of  with  and .  Let  and .
Vertex  is not in , for otherwise it would also be in 
and in .  Vertex  is not in , for otherwise it would
also be in  and in .  Vertex  is not in  ().
So  are pairwise non adjacent.

Suppose that there exists a vertex .
So , but none of the two leaves of  can satisfy
Claim~\ref{cl:orange}, a contradiction.  Therefore .

Suppose that , and let .  So 
is not in .  Let  ().  So  is not in
.  If , then 
induce , a contradiction.  So  is in one of , say  (if  is in  the argument is similar).
Since  is in , there is a vertex .  Vertex  is adjacent to  and not to .
Then  induce ,  or ,
a contradiction.  Therefore .

Let , so .  Suppose .  If there
exists , then  is also in  and  induce , a contradiction.  So .
Let .  Then  and .  Let ; so , , .  If  is adjacent to
, then  induce , else  induce , a contradiction.  So .
Let ; so .  If , then  induce , a contradiction.  So .  Let
 ( exists because  is special and
co-special).  Since ,  is adjacent to at most
one of , and then  induce  or
, a contradiction.  This completes the proof of
Theorem~\ref{th:main}.  \hfill 



\section{Recognition algorithm}

The proof that we give above yields a new recognition algorithm for
path graphs, which takes any graph  as input and either builds a
clique path tree for  or finds one of .  We
have not analyzed the exact complexity of such a method but it is easy
to see that it is polynomial in the size of the input graph.  More
efficient algorithms were already given by Gavril~\cite{Gav78},
Sch\"affer~\cite{Sch93} and Chaplick~\cite{Cha08}, whose complexity is
respectively ,  and  for graphs with 
vertices and  edges.  Another algorithm was proposed in
\cite{DahBai96} and claimed to run in  time, but it has only
appeared as an extended abstract (see comments in~\cite[Section
2.1.4]{Cha08}).

There are classical linear time recognition algorithms for
triangulated graphs~\cite{RTL76}, and, following \cite{BL76}, there
have been several linear time recognition algorithms for interval
graphs, of which the most recent is~\cite{HMPV}.  We hope that the
work presented here will be helpful in the search for a linear time
recognition algorithm for path graphs.



\begin{thebibliography}{99}

\bibitem{Ber60}
C. Berge.  Les probl\`emes de coloration en th\'eorie des graphes.
{\it Publ.  Inst.  Stat.  Univ.  Paris} 9 (1960) 123--160.

\bibitem{BL76}
K.S.~Booth, G.S.~Lueker.  Testing for the consecutive ones property,
interval graphs and graph planarity using PQ-tree algorithm.
{\it J.~Comput.  Syst.  Sci.}  13 (1976) 335--379.

\bibitem{Cha08}
S. Chaplick.  {\it PQR-trees and undirected path graphs.}
M.Sc.~Thesis, Dept.  of Computer Science, University of Toronto, 2008.

\bibitem{DahBai96}
E. Dahlhaus, G. Bailey.  Recognition of path graphs in linear time.
5th Italian Conference on Theoretical Computer Science (Revello, 1995)
World Sci.  Publishing, River Edge, NJ, 1996, 201--210.

\bibitem{Dir61}
G.A.~Dirac.  On rigid circuit graphs.  \emph{Abh.  Math.  Sem.  Univ.
Hamburg} 38 (1961) 71--76.

\bibitem{Gav74}
F.~Gavril.  The intersection graphs of subtrees in trees are exactly
the chordal graphs.  \emph{J. Combin.  Theory B} 16 (1974) 47--56.

\bibitem{Gav78}
F.~Gavril.  A recognition algorithm for the intersection graphs of
paths in trees.  \emph{Discrete Math.} 23 (1978) 211--227.

\bibitem{Gol04}
M. C. Golumbic.  {\it Algorithmic graph theory and perfect graphs.}
Annals Disc.  Math.  57, Elsevier, 2004.

\bibitem{HMPV}
M.~Habib, R.~McConnell, C.~Paul, L.~Viennot.  Lex-BFS and partition
refinement, with applications to transitive orientation, interval
graph recognition and consecutive ones testing.  \emph{Theoretical
Computer Science} 234 (2000) 59--84.

\bibitem{HajSur58}
A. Hajnal and J. Sur\'anyi.  \"Uber die Aufl\"osung von Graphen in
vollst\"andige Teilgraphen.  {\it Ann.  Univ.  Sci.  Budapest
E\"otv\"os, Sect.  Math.} 1 (1958) 113--121.



\bibitem{LB62}
C.~Lekkerkerker, D.~Boland.  Representation of finite graphs by a set
of intervals on the real line.  \emph{Fund.  Math.} 51 (1962) 45--64.

\bibitem{MacMac}
T.A. McKee and F.R. McMorris.  {\it Topics in intersection graph
theory.} SIAM Monographs on Discrete Mathematics and Applications,
Philadelphia, 1999.

\bibitem{MW86}
C.L.~Monma, V.K.~Wei.  Intersection graphs of paths in a tree.
\emph{J. Combin.  Theory B} 41 (1986) 141--181.

\bibitem{Pan99}
B. S. Panda.  The forbidden subgraph characterization of directed
vertex graphs.  {\it Discrete Mathematics} 196 (1999) 239--256.

\bibitem{Ren70}
P.L.~Renz.  Intersection representations of graphs by arcs.
\emph{Pacific J. Math.} 34 (1970) 501--510.

\bibitem{RTL76}
D.J.~Rose, R.E.~Tarjan, G.S.~Lueker.  Algorithmic aspects of vertex
elimination of graphs.  \emph{SIAM J. Comput.} 5 (1976) 266--283.



\bibitem{Sch93}
A.A.~Sch\"affer.  A faster algorithm to recognize undirected path
graphs.  \emph{Discrete Appl.  Math.} 43 (1993) 261--295.

\bibitem{TY84}
R.E.~Tarjan, M.~Yannakakis.  Simple linear time algorithms to test
chordality of graphs, test acyclicity of hypergraphs, and selectively
reduce acyclic hypergraphs.  \emph{SIAM J. Comput.} 13 (1984)
566--579.

\bibitem{TonGutSzw05}
S. Tondato, M. Gutierrez, J. Szwarcfiter.  A forbidden subgraph
characterization of path graphs.  {\it Electronic Notes in Discrete
Mathematics} 19 (2005) 281--287.





\end{thebibliography}


\end{document}