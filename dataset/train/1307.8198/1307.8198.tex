

\documentclass[times]{ettauth}

\usepackage{moreverb}




 \usepackage{amssymb}
  \usepackage{latexsym}
  \usepackage{amsfonts}
  \usepackage{amsthm}
  \usepackage{amsmath}
  \usepackage{graphics}
  \usepackage{setspace}
  \usepackage{graphicx}
  \usepackage{epstopdf}
  \usepackage{color}
  \usepackage{float}
  \usepackage{wrapfig}
  \usepackage{color}
  \usepackage{rotating}
  \usepackage{array} 
  \usepackage{dblfloatfix}
\usepackage[hyphens]{url}
  \usepackage[table]{xcolor}
  \usepackage{multirow}

  \usepackage{graphics}
  \usepackage{setspace}
  \usepackage{algorithm}
  \usepackage{algorithmic}
  \usepackage{graphicx}
  \usepackage{epstopdf}  
  \usepackage{color}
\usepackage{float}
  \usepackage{wrapfig}
  \usepackage{color}








\usepackage[colorlinks,bookmarksopen,bookmarksnumbered,citecolor=red,urlcolor=red]{hyperref}


\newenvironment{noindlist}
 {\begin{list}{\labelitemi}{\leftmargin=1.2em \itemindent=-.5em}}
 {\end{list}}



\newcommand\BibTeX{{\rmfamily B\kern-.05em \textsc{i\kern-.025em b}\kern-.08em
T\kern-.1667em\lower.7ex\hbox{E}\kern-.125emX}}

\def\volumeyear{2014}

\begin{document}

\runningheads{Perera et al.}{Sensing as a Service Model for Smart Cities Supported by Internet of Things}

\articletype{RESEARCH ARTICLE}

\title{Sensing as a Service Model for Smart Cities Supported by Internet of Things}

\author{A.~N.~Other\corrauth}


\author{Charith Perera,
 Arkady Zaslavsky,
Peter Christen,
Dimitrios Georgakopoulos}













\address{Research School of Computer Science, The Australian National University, Canberra, ACT 0200, Australia\\
CSIRO ICT Center, Canberra, ACT 2601, Australia}




\corraddr{CSIRO ICT Center, Canberra, ACT 2601, Australia. E-mail: charith.perera@csiro.au}

\begin{abstract}
The world population is growing at a rapid pace. Towns and cities are accommodating half of the world's population thereby creating tremendous pressure on every aspect of urban living. Cities are known to have large concentration of resources and facilities. Such environments attract people from rural areas. However, unprecedented attraction has now become an overwhelming issue for city governance and politics. The enormous pressure towards efficient city management has triggered various \textit{Smart City} initiatives by both government and private sector businesses to invest in ICT to find sustainable solutions to the growing issues. The Internet of Things (IoT) has also gained significant attention over the past decade. IoT envisions to connect billions of sensors to the Internet and expects to use them for efficient and effective resource management in Smart Cities. Today infrastructure, platforms, and software applications are offered as services using cloud technologies. In this paper, we explore the concept of sensing as a service and how it fits with the Internet of Things.  Our objective is to investigate the concept of sensing as a service model in technological, economical, and social perspectives and identify the major open challenges and issues.
\end{abstract}








\maketitle



\section{Introduction}
\label{sec:Introduction}

The Internet of Things (IoT) \cite{P001} and Smart Cities (SC) \cite{P532} are recent phenomena that have attracted the attention from both academia and industry. While both ideas consolidate similar ideology, they have different origins. Both IoT and SC do not have clear and concise definitions due to their short history and broadness. Examining the origins of both ideas in brief allows us to understand their potentials. Even though the term `\textit{Internet of Things}' was coined in 1999 \cite{P065}, the technologies that enable IoT such as sensor networks existed since the 1990s. Due to the advances in sensor and cloud technology, processing and storage capability, and decreased sensor production cost, the growth of sensor deployments has increased over the last five years \cite{ZMP003}. The European Commission has predicted that by 2020, there will be 50 to 100 billion devices connected to the Internet \cite{P029}. According to Figure \ref{Figure:IoT_Statistics}, the number of things connected to the Internet exceeded the number of people on earth in 2008.  

By definition, \textit{IoT allows people and things to be connected anytime, anyplace, with anything and anyone, ideally using any path/network and any service} \cite{P019}. As we can observe, IoT is primarily driven by technological advances, not by the applications or user needs. In contrast, SC \cite{P534} originated to solve the problems in modern cities. As a result of rural migration and suburban concentration towards cities, the urban living has become a significant challenge to both citizens and to the city governance. Waste, traffic, energy, water, education, unemployment, health, and crime management are some of the critical issues \cite{P535}. SC are expected to address these challenges efficiently and effectively using information and communication technologies (ICT). By definition, \textit{Smart Cities have six characteristics: smart economy, smart people, smart governance, smart mobility, smart environment and smart living} \cite{P528}. As illustrated in Figure \ref{Figure:Smart_Cities_and_IoT_Models}, SC and IoT, which have different origins, are moving towards each other to achieve a common goal. We believe that the sensing as a service model resides in between these two with many other technological and business models.

\begin{figure}[t]
 \centering
 \includegraphics[scale=.38]{./Images/35-IoT_Statistics.pdf}
\caption{Growth of \textit{`things'} connected to the Internet \cite{P574}}
 \label{Figure:IoT_Statistics}	
\end{figure}


The remainder of this paper is organized as follows: we briefly review the trend of everything as service in Section \ref{sec:The_Trends}. In Section \ref{sec:Sensing_as_a_Service Model}, the sensing as a service model is presented. Subsequently, we explain the sensing as a service model using a futuristic scenario in Section \ref{sec:The Future}. In Section \ref{sec:Action}, we discuss several use case scenarios that highlight the different aspects of the sensing as a service model. Advantages in sensing as a service model are discussed in Section \ref{sec:Advantages}. Later, in Section \ref{sec:Challenges}, we highlight some of the major open challenges and issues related to sensing as a service model. Open challenges are identified under three main categories: technological, economical, and social. Finally, we present the concluding remarks in Section \ref{sec:Conclusions}.





\section{The Trends: Everything as a Service}
\label{sec:The_Trends}



Everything as a service (XaaS) \cite{P533} is a category of models introduced with cloud computing \cite{P498}. Similar to IoT, cloud computing also has a short history. It became popular with a number of industry initiatives such as Salesforce.com (1999) and Amazon Web Service (2002). The basic idea behind cloud computing is to concentrate resources such as hardware and software into few physical locations and offer those resources as services to a large number of consumers who are located in many different geographical locations around the globe over the Internet in an efficient manner. There are three major service models that are closely bound to cloud computing from its initial stage: infrastructure-as-a-service (IaaS), platform-as-a-service (PaaS), and software-as-a-service (SaaS). The commonality among these models is that they all provide resources as a service. With the popularity of these models, several similar type models are also proposed. The service models offered in cloud computing are discussed in \cite{P502} with popular industry based examples.


\begin{figure*}[t]
 \centering
 \includegraphics[scale=.80]{./Images/72-Smart_Cities_and_IoT_Models_white.pdf}
\caption{Relationship among sensing as a service model, SC and IoT}
 \label{Figure:Smart_Cities_and_IoT_Models}	
\vspace{-0.20cm}	
\end{figure*}


Let us briefly discuss the reasons behind the success of everything as a service model in the cloud paradigm. One major reason is the cost effectiveness. XaaS model promotes the \textit{`pay as you go'} method or in other terms \textit{`pay only for what you use'}.  This allows the consumers to consume a service from a service provider by paying only for the amount of resources they use. This is an efficient way compared to the traditional methods of consuming resources where consumers need to buy resources in predefined discreet quantities with higher expenses. For example, consider a retail online business which has peak and off-peak seasons. In traditional method, the business has to buy significant amount of compute servers (and other resources) to facilitate the customer needs during the peak season. However, these resources become idle during the off-peak season which makes the business process inefficient. In XaaS, online retail applications are hosted in servers facilitated by cloud service provider where the business is only required to pay for the resource it consumes. This model works similar to the utility services such as electricity. Further, cloud computing service models provide many other benefits such as business agility, scalability and elasticity, reliability, green initiatives, less maintenance work including backup and disaster recovery. Ultimately, XaaS allows businesses to focus more on core competency and innovation instead of ICT \cite{P539}. Further explanation on characteristics, features and benefits of cloud computing are presented in \cite{P498,P501}.





Smart City initiatives have become another trend during the past decade. Various city councils, business organizations, research and academic institutions, and the governments have invested significantly in projects to study, design, and build solutions to address the problems in urban cities using ICT. \textit{IBM Smart Planet and Smart Cities}, \textit{Oracle iGovernment},\textit{ Amsterdam Smart City}, \textit{Dubai SmartCity}, \textit{EuropeanSmartCities}, and\textit{ Smart Cities Future} are some of the leading Smart City projects \cite{P537, P536}. The following statistics show the magnitude of both trends. Global cloud computing and XaaS market is expected to grow from \121.1 billion by 2015, growing at a compound annual growth rate (CAGR) of 26.2\% from 2010 to 2015 \cite{P539}. Similarly, the global Smart City market is expected to exceed \_{2}2. As \textit{Mike} likes \textit{DairyIceCream} products, he agrees to the 3\% discount offer instead of the monthly fee as shown in  step (5). A week later, \textit{Mike} receives an email from a company called \textit{ProductiveAnalytics} which has been sent on behalf of the \textit{GoldenCheese} company, a cheese manufacturer, with an similar offer. This request  also comes through \textit{EasySensing}. However, the offer is either 4\% discount on every product purchase by \textit{GoldenCheese} or a monthly fee of \8,000 \cite{P632}. Recently, different third party companies started offering consumer surveys on behalf of businesses. One such solution is Google Consumer Surveys (www.google.com/insights/consumersurveys). Google Consumer Surveys allows businesses to target user groups with specific criteria and conduct the survey. Currently, one user response cost around \$0.10, 1/10th of the cost of similar quality research conduct using traditional methods. 

Even though such approaches have reduced the cost of surveys, they still have deficiencies such as latency, inaccuracies, and so on. In the sensing as a service model, all the data is directly coming from the sensor without user intervention. This also helps to reduce the cost of data acquisition. Due to privacy concerns it is important to anonymise the sensor data collected. We discuss privacy matters later. In the smart home scenario we discussed in Section \ref{sec:The Future}, we explained how a  single sensor attached to a refrigerator, and cheap passive RFID tags attached to consumer products, produce valuable information of consumer behaviour that can be used by thousand of companies. This drastically reduces the consumer survey cost as well as pay off the cost of attaching sensors to the products.



\item  \textit{Innovations}: Due to a reduction in sensor data acquisition cost, larger number of interest groups will be able to access to them. Further, the availability of sensor data which was not available previously can also significantly stimulate innovation . Sensing as a service model itself provides space for innovation in the ESP layer. The cloud-based value added services provided in the ESP layer allows the sensor data consumers to achieve their objective easily and faster in many different application domains. 



\item  \textit{Applications}: Easily accessible sensor data allows government authorities, academia, research institutions, and businesses to address different challenges in Smart Cities such as traffic, energy, water, education, and unemployment, health, and crime management. For example, accurate data on energy consumption in a city allows managing electric grids efficiently by analysing and predicting energy consumption behaviours, patterns, future trends, and needs.


\item  \textit{Real-time data for decision making and policy making}: This model enables collecting sensor data in real-time, from a variety of different domains, which facilitates the decision making processes. Such data is expensive to collect and usually unavailable for decision making in traditional sensor deploying environments. For example, data collected from sensors deployed in vehicles and roads allow the authorities to monitor and manage traffic in real-time. Further, sensor data collected over a period of time (archived) can be used to make policy decisions. For example, traffic data over a period on a specific city will help a city governance to make long term strategic decisions such as whether to invest on a tram service across the city or not. In addition to the points discussed above, there are many other direct and indirect benefits in the sensing as a service model.


\item  \textit{Direct and indirect benefits}: The sensing as a service model creates a win-win situation for all the parties involved. Based on the scenario we presented in Section \ref{sec:The Future}, \textit{Mike} (sensor owners' perspective) is getting a return (a valuable offer). In \textit{DairyIceCream}  perspective, now they have real-time data about product consumer behaviour (e.g. when \textit{Mike} eats ice cream, how frequent, whether \textit{Mike} use substitutions and so on). Therefore, \textit{DairyIceCream} is no longer required to conduct manual surveys and market analyses. 



\item  \textit{Privacy preservation}: Finally and more importantly, this model provide complete control of the privacy of sensor owners in  their own hands. The final decision of whether to publish their sensors or not is taken by the sensor owners. It allows the sensor owners to control and protect their privacy. Additionally, the sensing as a service model needs to be supported by anonymization techniques. For example, lets consider security and privacy challenges \cite{P636} related to the  smart home scenario we presented in Section \ref{sec:The Future}. During the configuration process, it is important to identify the information and preferences related to \textit{Mike}. In order to protect the  privacy of the users, SPs and ESPs should not provide personal information to the sensor data consumers. Such approach helps to preserve user privacy. Additionally, once the deal between the sensor owner, sensor consumer and the sensor provider is done, data retrieves from  \textit{Mike's} sensors should be explicitly anonymized. It is important to develop new algorithms and security devices that can anonymize sensitive information (such as exact location).






































\end{noindlist}




\begin{table*}[t]
\centering
\small


\renewcommand{\arraystretch}{-4}
\caption{Open challenges and issues in sensing as a service model}
\vspace{-0.3cm} 
\begin{tabular}{ m{0.05cm} m{2.0cm} m{12cm}  }

\hline  
   \multicolumn{1}{r}{}   &    
\begin{center} Open challenges and issues \end{center} & 
\begin{center} Description, significance, and research directions to address the challenges
 \end{center} 


\\ \hline \hline 
  



\begin{sideways}Technological\end{sideways}     
&   \begin{noindlist}
 		 \item Architectural Designs, \newline Sensor Configuration, \newline Data Fusing / Filtering,
 		 \newline Processing / Storage, \newline
 		 Infrastructure, \newline Energy Consumption,\newline
 		 \vspace{60pt}
 		 \item Standardization, \newline  Accuracy \newline Security and Privacy,  

    \end{noindlist}  
 &  \begin{noindlist}
  		 \item Technology plays the most important role in enabling the sensing as a service model. This model uses the same infrastructure that IoT envisions. Therefore, most of the technological solutions that are developed to facilitate sensing as a service can be used to realize the vision of IoT. The sensing as a service model is expected to facilitate billions of sensors and parallel sensor data streams. A major challenge is to develop middleware solutions that allow to handle such demand \cite{P377}. Similarly, this model needs significant improvements in data communication bandwidth \cite{P667} over the existing infrastructure (e.g. fiber). Another major challenge is the sensor configuration. The term \textit{`sensor configuration'} encapsulates different aspects of configuration that needs to be done: sensor embedded software, intermediate devices, and  cloud (middleware) software. In reference to the scenario we presented in Section \ref{sec:The Future}, sensors in \textit{Mike's} new refrigerator need to be configured autonomously so they can communicate with the SP. Such an approach needs to deal with challenges such as heterogeneity: sensor types (e.g. RFID, temperature), protocols and communications technologies (e.g. Wi-Fi, Zigbee). In addition, once a deal is done, sensor behaviour need to be configured according to the agreement between the sensor consumer and the sensor owner (e.g. sampling rate, data communication frequency and so on). Further, SPs and ESPs may need to configure their cloud software accordingly. The sensing as a service model is a distributed system. It is critical to utilise computational devices with different capabilities and capacities (e.g. sensors, mobile phones, Raspberry Pi, laptops, servers) \cite{ZMP005}. Another challenge is to ensure the interoperability among different sensor hardware and cloud solutions. Complying with common standards in key areas in the architecture (such as communication interfaces and data formats) is critical. Energy conservation is also a challenge that needs to be addressed across all the entities in the model due to the large scale and the resource restricted nature of the sensors. Other than the sensor data, it is important to capture context information (e.g. battery level of the sensors, redundant sensors, access to energy sources, accuracy, reliability) as well  \cite{ZMP007}. Context information allows to design optimized sensing schedules and strategies that ensure the sustainability of the IoT infrastructure.
  		 
  		 
  		 
  		 \item Standardization is the key to interoperability. We have experienced the value of interoperability in service computing and many other occasions throughout the history of computing. Standardization efforts need to be carried out as early as possible to avoid significant frustrations and costs that may occur at later stages. Technology needs ensure the accuracy of the data up to an acceptable level as it is one of the main motivations behind the Sensing as a service model. It is important to anonymize the sensor data collected. Sensitive information such as location need to be implicitly altered to protect the sensor owner privacy. This should be done in both the hardware and software levels. For the hardware level, we need to develop next generation security appliance that can be used to anonymize data at the ground level (i.e. physically close to the sensor owners). Techniques similar to privacy preserving data sharing ad matching \cite{P637} need to be developed in order to combine sensor data to anonymize entities / profiles (excluding sensitive data) later at the server level. 
     \end{noindlist}  

\\ \hline
\begin{sideways} Economical  \end{sideways}      
&   \begin{noindlist}
		 \item Innovation, \newline Entrepreneurship, \newline Entry Barriers\newline
		 \vspace{30pt}
 		 \item Sustainability,\newline Licensing,\newline Business Practices, \newline Credibility
\end{noindlist}  
 &
  \begin{noindlist}
 		 \item The sensing as a service model will collect enormous amount of data that need to be processed and understood. It will  open up opportunities for thousands of new businesses. The entry barriers need to be kept at a minimum  to stimulate new start-ups to be established to provide more value added services (e.g. search sensors based on context information \cite{ZMP004} and user requirements \cite{ZMP006}). The opportunities are ranging from the point where data is collected and to the point data is delivered. As we argued earlier, most of the users who may consume sensor data will not have technical expertise. Therefore, understanding data and extract valuable information from sensor data, by data fusing and reasoning, can also provide value added services.
 		 
  		 \item The sensing as a service model promotes a healthy competition among parties involved as it helps both the sensor data owners and sensor data consumers. Sustainability needs to be ensured by having a fair and transparent financial model which motivates all the parties to be retained in the business. Sensor data and knowledge produced using them need to be accurate and credible so consumers can make important and potentially costly strategic decisions based on them.
 \end{noindlist}  

\\ \hline

\begin{sideways} Social  \end{sideways}      
&   \begin{noindlist}
		 \item Trust, \newline Social Acceptance, 
		 \newline Change Management, \newline Awareness 
		 		 
 		 \item Security and Privacy, \newline Safety, \newline Accessibility,    \newline Usability, \newline Legal Terms 
\end{noindlist}  
 &
  \begin{noindlist}
 		 \item Trust and social acceptance in vital towards the adaptation of the sensing as a service model. If sensor owners do not trust the sensing as a service, the entire model will fail. In order to win the trust, a long term change management process is required. It needs to be supported by increasing the awareness about inner-workings and benefits of the model. New privacy protection and security protocols \cite{P669} need to be introduced in order to make the model sustainable by winning the trust of all parties involved.
 		 
  		 \item Security and privacy is a must \cite{P670}. It needs to be implemented in number of levels. First, at the technology level, secondly, in government and business policy level and finally, through strict legal terms and conditions. Policies need to be set in order to keep the accessibility fairly open to the sensor data consumers while validating and monitoring all the parties involved in the model. Maximum usability at both ends (the sensor owner and sensor data consumer end) helps the model to be adopted by the wider community. Automated sensor configuration plays a significant role in usability because most of the sensor owner will be non-technical. 
 \end{noindlist}  

\\ \hline


\end{tabular}

\label{Tbl:Comparison of Semantic Technologies}
\vspace{-0.6cm}
\end{table*}





\section{Open Challenges}
\label{sec:Challenges}

The sensing as a service model can contribute significantly to address the challenges in the IoT and SC. There are many open challenges and issues that need to be tackled. We identify some of the major challenges in the sensing as a service model under three categories in Table \ref{Tbl:Comparison of Semantic Technologies}: technological, economical, and social, where some can be discussed under multiple categories. Each of these challenges shows research directions for future work in the sensing as a service domain.











\section{Conclusions}
\label{sec:Conclusions}

This paper provides a comprehensive overview of the sensing as a service model and its applicability towards Smart Cities in the Internet of Things paradigm. Our vision is backed up by a number of projects initiated around the globe, including FP7 ICT project OpenIoT \cite{P377}. We discussed the model from three different perspectives including technological, economical, and social. We examined how the sensing as a service can be a sustainable, scalable, and powerful model. The sensing as a service model allows utilizing  resources efficiently so limited resource can be used to accommodate large numbers of consumers. Further, it also creates a win-win situation for all the parties involved. We identified a number of major open challenges and issues which need to be addressed in order to realise the vision of sensing as a service. Finally, this model will create an unprecedented amount of opportunities to build innovative value added solutions that makes the decision making process efficient and effective in IoT paradigm.
















\section*{Acknowledgement}

The authors acknowledge support from SSN TCP, CSIRO, Australia and ICT OpenIoT Project, which is co-funded by the European Commission under Seventh Framework program, contract number FP7-ICT-2011-7-287305-OpenIoT. The Author(s) acknowledge help and contributions from all partners of the OpenIoT project.




\bibliography{Bibliography}
\bibliographystyle{IEEEtran}




\end{document}
