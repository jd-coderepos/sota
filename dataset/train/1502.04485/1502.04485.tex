

\subsection{\emph{In-vivo}}\label{sec:results:invivo}





\pgfplotstableread{
Subject	P1A	P2A	P1B	P2B P3Speller P1M P2M
1	0.0	0.0	0.0	0.0	0.0	0.0	0.0
2	0.11	0.0	0.0	0.0	0.07	0.05	0.0
3	0.0	0.0	0.0	0.0	0.13	0.0	0.0
4	0.0	0.0	0.0	0.0	0.07	0.0	0.0
5	0.0	0.0	0.0	0.14	0.17	0.0	0.08
6	0.17	0.0	0.13	0.14	0.28	0.15	0.08
7	0.11	0.0	0.0	0.0	0.1	0.05	0.0
8	0.0	0.0	0.0	0.0	0.0	0.0	0.0
9	0.0	0.0	0.0	0.0	0.13	0.0	0.0
10	0.0	0.0	0.13	0.0	0.13	0.09	0.0

}\vivoerrorseltable







Concerning sentence A, we observed larger OCM in both turn 1 and 2 of PolyMorph session with respect to that obtained by using P3Speller 
(; ; -value ; Wilcoxon test: -value  in both cases).
It was also detected 
relevant enhancement of OCM from turn 1 to turn 2 (Wilcoxon test: -value ). 

The rmANOVA showed a significant interaction between writing system and spelled sentence (; -value ). 
With regard to PolyMorph, we found a substantially increased OCM in turn 1 of sentence A compared with the OCM of the same turn of sentence B  
(; -value ). 
Finally, the OCM obtained during turn 2 is larger than that of turn 1 for both sentence A and B 
(; -value  and ; -value , respectively). 



\pgfplotstableread{
Subject P1A     P2A     P1B     P2B     P3Speller       P1M     P2M
1       6.48    8.94    4.16    8.94    2.51    5.32    8.94
2       2.85    5.03    2.35    5.03    1.07    2.60    5.03
3       4.11    5.65    2.64    5.65    1.06    3.37    5.65
4       2.76    3.79    1.78    3.79    0.78    2.27    3.79
5       3.87    5.32    2.49    3.80    0.88    3.18    4.56
6       3.76    8.94    3.06    6.39    1.07    3.41    7.66
7       3.94    6.92    3.23    6.92    1.40    3.59    6.92
8       4.37    5.01    2.81    6.02    1.54    3.59    5.51
9       3.66    4.28    2.35    5.03    0.93    3.01    4.66
10      4.37    5.12    2.12    6.02    1.15    3.25    5.57

}\vivoOCMtable

\begin{figure}[!h]
\begin{center}
\resizebox{\columnwidth}{!}{\begin{tikzpicture}[scale=0.70]
\begin{axis}
[
width=22cm,
height=6cm,
ymajorgrids,
ybar,
xlabel = Subjects,
nodes near coords, 
every node near coord/.append style={rotate=90, anchor=west},
symbolic x coords={1,2,3,4,5,6,7,8,9,10},
xtick={1,2,3,4,5,6,7,8,9,10},
xticklabels={\subject{1},\subject{2},\subject{3},\subject{4},\subject{5},\subject{6},\subject{7},\subject{8},\subject{9},\subject{10}},
ytick={3,6,9},
ylabel = {Mean OCM},
ymin = 0,
ymax=11.5,
legend columns=-1,
legend style={/tikz/every even column/.append style={column sep=0.3cm},at={(0.5,-0.4)}, anchor=north}
]
\addplot[P3SpellerBorderColor,fill=P3SpellerColor] table[y =P3Speller] from \vivoOCMtable;
\addlegendentry{P3Speller};
\addplot[P1MBorderColor,fill=P1MColor] table[y =P1M] from \vivoOCMtable;
\addlegendentry{PolyMorph - Turn 1};
\addplot[P2MBorderColor,fill=P2MColor] table[y =P2M] from \vivoOCMtable;
\addlegendentry{PolyMorph - Turn 2};
\end{axis}
\end{tikzpicture}}
\caption{Mean output characters per minute \emph{in-vivo}: the values 
reported for PolyMorph turn 1 and 2 are the means of the spelling times 
of sentence A and B in turn 1 and turn 2 respectively.}\label{fig:OCM_vivo}
\end{center}
\end{figure}



When accuracy was assessed, there were significant differences among conditions 
(; , -value ). Post-hoc analysis show a significant enhancement of accuracy 
when spelling the  sentence A for the  time (turn 1) by PolyMorph (and, partially, also for turn 2) 
with respect to that obtained by P3Speller (Wilcoxon test: -value  and -value , respectively).  






\begin{table}[!h]
\begin{center}
\resizebox{\columnwidth}{!}{\subfloat[Accuracy (AC). Data are reported in the format ``(number of correct selections)/(total number of selections)''.\label{table:accuracy}]{
\begin{tabular}{lccccc}\toprule
\headcol & \multicolumn{3}{c}{Sentence A}& \multicolumn{2}{c}{Sentence B}\\
 \rulefiller\cmidrule(r){2-4} \cmidrule(r){5-6}
\headcol Subj &  PM & PM  & P3S & PM  &  PM  \\
\midrule
\subject{1} &  &  &  &  &  \\
\rowcol  \subject{2}&  &  &  &  & \\
\subject{3} &  &  &  &  & \\
\rowcol \subject{4} &  &  &  &  & \\
\subject{5} &  &  & &   & \\
\rowcol \subject{6} &  &  &  &   & \\
\subject{7} &  &  &  &   & \\
\rowcol \subject{8} &  &  &  &   & \\
\subject{9} &  &  &  &   & \\
\rowcol \subject{10} &  &  &  &   & \\
\midrule
\rescol Total &  &  &  & 
 & \\
\bottomrule
\end{tabular}}
\hskip3mm
\subfloat[Errors per character (EC). Data are reported in the format ``(wrong selections)/(number of characters in the sentence)''.\label{table:errors_per_character}]{
\begin{tabular}{lccccc}\toprule
\headcol & \multicolumn{3}{c}{Sentence A}& \multicolumn{2}{c}{Sentence B}\\
 \rulefiller\cmidrule(r){2-4} \cmidrule(r){5-6}
\headcol Subj &  PM & PM  & P3S & PM  &  PM  \\
\midrule
\subject{1} &  &  &  &  &  \\
\rowcol \subject{2} &  &  &  &  & \\
\subject{3} &  &  & &  & \\
\rowcol \subject{4} &  &  &  &  & \\
\subject{5} &  &  & &   & \\
\rowcol \subject{6} &  &  &  &   & \\
\subject{7} &  &  &  &   & \\
\rowcol \subject{8} &  &  &  &   & \\
\subject{9} &  &  &  &   & \\
\rowcol \subject{10} &  &  &  &   & \\
\midrule
\rescol Total &  &  &  & 
 & \\
\bottomrule
\end{tabular}}}
\caption{Accuracy (AC) and errors per character (EC) for PolyMorph turn 1 (PM ),  turn 2 (PM ),  
and P3Speller (P3S) for \emph{in-vivo} experiments.}
\end{center}
\end{table}



Errors per each character selected were also assessed. Test indicates that the distribution of results significantly differ 
among conditions (;  , -value ). Post-hoc analysis 
exhibits a significant reduction of errors per each character in both PolyMorph turn 1 and 2 of sentence A 
with respect to errors per each character 
obtained when using P3Speller (Wilcoxon test: -value is   in both cases). 

