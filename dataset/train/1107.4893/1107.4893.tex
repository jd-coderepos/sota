\documentclass{llncs}
\usepackage{epsfig}
\begin{document}

\title{Approximating minimum-power edge-multicovers}

\author{Nachshon Cohen \and Zeev Nutov}
\institute{The Open University of Israel,\\
\email{nachshonc@gmail.com, nutov@openu.ac.il}}

\maketitle

\begin{abstract}
Given a graph with edge costs, the {\em power} of a node is the
maximum cost of an edge incident to it, and the power of a graph is
the sum of the powers of its nodes. Motivated by applications in
wireless networks, we consider the following fundamental problem 
in wireless network design. Given a graph  
with edge costs and degree bounds , 
the {\sf Minimum-Power Edge-Multi-Cover} ({\sf MPEMC}) problem  is to find a
minimum-power subgraph  of  such that the degree of every node  in  
is at least . We give two approximation algorithms for {\sf MPEMC},
with ratios  and , where  is the maximum 
degree bound.
This improves the previous ratios  and , and implies ratios 
 for the {\sf Minimum-Power -Outconnected Subgraph} and 
 for the {\sf Minimum-Power -Connected Subgraph}
problems; the latter is the currently best known ratio for the min-cost version
of the problem. 
\end{abstract}

\section{Introduction} \label{s:intro}

\subsection{Motivation and problems considered}

Wireless networks are studied extensively due to their wide applications. 
The power consumption of a station determines its transmission range, and thus also 
the stations it can send messages to; the power typically increases at least
quadratically in the transmission range. Assigning power levels to
the stations (nodes) determines the resulting communication network.
Conversely, given a communication network, the power required at 
only depends on the farthest node reached directly by .
This is in contrast with wired networks, in which every pair of stations 
that communicate directly incurs a cost. 
An important network property is fault-tolerance, which is often measured
by minimum degree or node-connectivity of the network.
Node-connectivity is much more central here than edge-connectivity,
as it models stations failures. Such power minimization problems were vastly studied;
see for example \cite{ACMP,HKMN,LYN,N-dir,zeev-p} and the references therein for 
a small sample of papers in this area.
The first problem we consider is finding a low power network with 
specified lower degree bounds.
The second problem is the {\sf Min-Power -Connected Subgraph} problem. 
We give approximation algorithms for these problems,
improving the previously best known ratios.


\begin{definition}
Let  be a graph with edge-costs .
For a node  let  denote the set of edges incident to  in .
The {\em power}  of  is the maximum cost of an edge in  incident to , 
or  if  is an isolated node of ;
i.e.,  if ,
and  otherwise. For  the power of  w.r.t.  
is the sum  of the powers of the nodes in .
\end{definition}

Unless stated otherwise, all graphs are assumed to be undirected and simple.
Let . 
Given a graph  with edge-costs , 
we seek to find a low power subgraph  of  that satisfies some prescribed property.
One of the most fundamental problems in Combinatorial Optimization is finding 
a minimum-cost subgraph that obeys specified degree constraints 
(sometimes called also ``matching problems'') c.f. \cite{Sch}.
Another fundamental property is fault-tolerance (connectivity).
In fact, these problems are related, and we use our algorithm 
for the former as a tool for approximating the latter.

\begin{definition}
Given degree bounds , we say that an edge-set 
on  is an {\em -edge cover} if  for every ,
where  is the degree of  in the graph .
\end{definition}


\noindent
{\sf Minimum-Power Edge-Multi-Cover} ({\sf MPEMC}): \\
{\em Instance:} 
A graph  with edge-costs , 
degree bounds \hphantom{\em Instance: } 
. \\
{\em Objective:} 
Find a minimum power -edge cover .

\vspace{0.2cm}

Given an instance of {\sf MPEMC}, let  denote the maximum requirement.

We now define our connectivity problems. 
A  graph is {\em -outconnected from } if it contains 
internally-disjoint -paths for all .
A graph is {\em -connected} if it is -outconnected from every node, 
namely, if it contains  internally-disjoint -paths for all .

\vspace{0.2cm}

\noindent
{\sf Minimum-Power -Outonnected Subgraph} ({\sf MPOS}): \\
{\em Instance}: \ 
A graph  with edge-costs , a root , and 
\hphantom{\em Instance: } an integer . \\
{\em Objective}: 
Find a minimum-power -outconnected from  spanning subgraph  
\hphantom{\em Objective:} of .

\vspace{0.2cm}

\noindent
{\sf Minimum-Power -Connected Subgraph} ({\sf MPCS}): \\
{\em Instance}: \ 
A graph  with edge-costs  and an integer . \\
{\em Objective}: 
Find a minimum-power -connected spanning subgraph  of .


\subsection{Our Results}
The previous best approximation ratio for {\sf MPEMC} was  \cite{KMNT}. 
Our main result improves this ratio to .

\begin{theorem} \label{t:mpec}
{\sf MPEMC} admits an -approximation algorithm.
\end{theorem}

For small values of , the problem admits also the ratios 
for arbitrary  \cite{HKMN}, while for  the best known ratio is  \cite{KN-cov}. 
Our second result extends the latter ratio to arbitrary .

\begin{theorem} \label{t:mpec'}
{\sf MPEMC} admits a -approximation algorithm.
\end{theorem}

For small values of , say , the ratio  is better than
 because of the constant hidden in the  term.
And overall, our paper gives the currently best known ratios for all values .

\vspace*{0.1cm}

In \cite{LYN} it is proved that an -approximation for {\sf MPEMC} implies an 
-approximation for {\sf MPOS}.
The previous best ratio for {\sf MPOS}  was 
 \cite{LYN} for large values of , 
and  for small values of  \cite{zeev-p}.
From Theorem~\ref{t:mpec} we obtain the following.

\begin{theorem} \label{c:mpoc}
{\sf MPOS} admits an -approximation algorithm.
\end{theorem}

In \cite{HKMN} it is proved that an -approximation for {\sf MPEMC}
and a -approximation for {\sf Min-Cost -Connected Subgraph} implies a 
-approximation for {\sf MPCS}.
Thus the previous best ratio for {\sf MPCS} was
 \cite{KMNT}, where  is the best ratio for {\sf MCCS}
(for small values of  better ratios for {\sf MPCS} are given in \cite{zeev-p}).
The currently best known value of  is  \cite{zeevnew},
which is , unless .
From Theorem~\ref{t:mpec} we obtain the following.

\begin{theorem} \label{c:mpkc}
{\sf MPCS} admits an -approximation algorithm,
where  is the best ratio for {\sf MCCS}. 
In particular, {\sf MPCS} admits an -approximation algorithm.
\end{theorem}

\subsection{Overview of the techniques}

Let the {\em trivial solution} for {\sf MPEMC} be obtained by picking for every 
node  the cheapest  edges incident to .
It is known and easy to see that this produces an edge set of power at most 
, see \cite{HKMN}. 

Our -approximation algorithm uses the following idea.
Extending and generalizing an idea from \cite{KMNT},
we show how to find an edge set  of power 
such that for the residual instance, 
the trivial solution value is reduced by a constant fraction.
We repeatedly find and add such an edge set  to the constructed solution, 
while updating the degree bounds accordingly to .
After  steps, the trivial solution value is reduced to , 
and the total power of the edges we picked is .
At this point we add to the constructed solution the trivial solution of the residual problem,
which at this point has value , obtaining an -approximate solution.

Our -approximation algorithm uses a two-stage reduction.
The first reduction reduces {\sf MPEMC} to a constrained version of {\sf MPEMC} 
with , where we also have lower bounds  on the power 
of each node ; these lower bounds are determined by the trivial solution
to the problem. We will show that a -approximation 
algorithm to this constrained version implies a -approximation
algorithm for {\sf MPEMC}. The second reduction reduces the constrained version 
to the {\sf Minimum-Cost Edge Cover} problem with a loss of  in the 
approximation ratio. As {\sf Minimum-Cost Edge Cover} admits a polynomial time algorithm, 
we get a ratio  for the constrained problem,
which in turn gives the ratio  for {\sf MPEMC}.

\section{An -approximation (proof of Theorem~\ref{t:mpec})}

As in \cite{KMNT}, we reduce {\sf MPEMC} to {\sf Bipartite MPEMC},
where  is a bipartite graph with sides , and  for every  
(so, only the nodes in  may have positive degree bound).
This is done by taking two copies  and  of ,
for every edge  adding the two edges  and  of cost  each,
and for every  setting  and .
It is proved in \cite{KMNT} that this reduction invokes a factor of  in the
approximation ratio, namely, that a -approximation for bipartite {\sf MPEMC} 
implies a -approximation for general {\sf MPEMC}.

Let  denote the optimal solution value of a problem instance at hand.
For , let  be the cost of the -th least cost edge incident to  
in  if , and  otherwise.
Given a partial solution  to {\sf Bipartite MPEMC} let 
 be the {\em residual bound} of  w.r.t. .
Let 


The main step in our algorithm is given in the following lemma, 
which will be proved later.
 
\begin{lemma} \label{l:1}
There exists a polynomial time algorithm that given an edge set , 
an integer , and a parameter , either correctly establishes that 
, or  
returns an edge set  such that 
 and , where
. 
\end{lemma}

\begin{lemma} \label{l:2}
Let  and let  be an edge set obtained by picking 
 least cost edges in  for every . 
Then  is an -edge-cover and: ,
. 
\end{lemma}
\begin{proof}
Since  is an -edge-cover,  is an -edge-cover.
By the definition of , for any -edge-cover ,
 for all .
In particular, if  is an optimal -edge-cover, then 

Also, 

Finally,  since

This concludes the proof of the lemma.
\qed
\end{proof}

Theorem \ref{t:mpec} is deduced from Lemmas \ref{l:1} and \ref{l:2} as follows.
We set  to be constant strictly greater than , say .
Then .
Using binary search, we find the least integer  such that the following procedure 
computes an edge set  satisfying .  

\vspace{0.2cm}

\noindent
{\em Initialization:} . \\
{\em Loop:} \ Repeat  times: \\
\hphantom{Loop:} \ \ \ \ Apply the algorithm from Lemma~\ref{l:2}: \\ 
\hphantom{Loop:} \ \ \ \ - If it establishes that  then return ``ERROR'' and STOP. \\
\hphantom{Loop:} \ \ \ \ - Else do . 

\vspace{0.2cm}

After computing  as above, we compute an edge set  as in Lemma~\ref{l:2}.
The edge-set  is a feasible solution, by Lemma~\ref{l:2}.
We claim that for any  the above procedure returns an edge 
set  satisfying ; thus binary search indeed applies.
To see this, note that  and thus

Consequently, the least integer  for which the above procedure does not return ``ERROR''
satisfies . Thus
.
Also, by Lemma~\ref{l:2}, . 
Consequently, 


In the rest of this section we prove Lemma~\ref{l:1}. 
It is sufficient to prove the statement in the lemma for the residual 
instance  with edge-costs restricted to ; 
namely, we may assume that .
Let .

\begin{definition}
An edge  incident to a node  is {\em -cheap} if 
.
\end{definition}

\begin{lemma} \label{l:cheap}
Let  be an -edge-cover, let , and let 
 
be the set of -cheap edges in .
Then  and .
\end{lemma}
\begin{proof}
Let . 
Since for every  there is an edge  incident to
 with ,
we have  for every .
Thus

This implies . 
Note that for every ,
 and hence .
Thus we obtain:
 

To see that  note that

This concludes the proof of the lemma.
\qed
\end{proof}

In \cite{KMNT} it is proved that the following problem,
which is a particular case of submodular function  minimization subject to 
matroid and knapsack constraint (see \cite{LNMN}) admits a 
-approximation algorithm.

\vspace{0.2cm}

\noindent
{\sf Bipartite Power-Budgeted Maximum Edge-Multi-Coverage} ({\sf BPBMEM}): \\
{\em Instance:} \ A bipartite graph  with edge-costs ~and
\hphantom{\em Instance: } node-weights ,  
degree bounds , and a \hphantom{\em Instance: } budget .\\
{\em Objective:} Find  with  that maxmizes


\vspace{0.2cm}

The following algorithm computes an edge set as in Lemma~\ref{l:1}. 
\begin{enumerate}
\item
Among the -cheap edges, compute a -approximate
solution  to {\sf BPBMEM}.
\item
If  then return , where 
; \\
Else declare ``''.
\end{enumerate}

\vspace{0.2cm}

Clearly, . By Lemma~\ref{l:cheap}, .
Thus .

Now we show that if  then .
Let  be the set of cheap edges in some optimal solution. 
Then . 
By Lemma~\ref{l:cheap} , namely, 
 reduces  by at least .
Hence our -approximate solution  to {\sf BPBMEM}
reduces  by at least .
Consequently, we have 
, 
as claimed.

The proof of Theorem~\ref{t:mpec} is complete.

\section{A -approximation (proof of Theorem~\ref{t:mpec'})}

We say that an edge set  {\em covers} a node set , 
or that  is a {\em -cover}, if  for every . 
Consider the following auxiliary problem:

\vspace{0.2cm}

\noindent
{\sf Restricted Minimum-Power Edge-Cover}  \\
{\em Instance:} \ 
A graph  with edge-costs , , and~degree 
\hphantom{\em Instance: } 
bounds . \\
{\em Objective:} 
Find a power assignment  that minimizes ,
\hphantom{\em Objective:}
such that  for all , and such that the edge set 
\hphantom{\em Objective:}~ covers .

\vspace{0.2cm}

Later, we will prove the following lemma.

\begin{lemma} \label{l:restricted}
{\sf Restricted Minimum-Power Edge-Cover} admits a -approximation algorithm.
\end{lemma}

Theorem~\ref{t:mpec'} is deduced from Lemma~\ref{l:restricted} and the following statement. 

\begin{lemma} \label{l:alpha}
If {\sf Restricted Minimum-Power Edge-Cover} admits a -ap\-pro\-ximation algorithm,
then {\sf Minimum-Power Edge-Multi-Cover} admits a -approxima\-tion algorithm.
\end{lemma}
\begin{proof}
Consider the following algorithm. 
\begin{enumerate}
\item 
Let  be the power assignment computed by the 
-approximation algorithm for {\sf Restricted Minimum-Power Edge-Cover}  
with  and bounds  for all . 
Let .
\item 
For every  let  be the edge-set obtained by picking 
the least cost  edges in  and let 
. 
\end{enumerate}
Clearly,  is a feasible solution to {\sf Minimum-Power Edge-Multi-Cover}.
Let  denote the optimal solution value for {\sf Minimum-Power Edge-Multi-Cover}.
In what follows note that  and that 
.

We claim that 
 
As , this implies  
.

For  let  be the set of neighbors of  in the graph . 
The contribution of each edge set  to the total power is at most 
.
Note that  and  for every ,
hence .
This implies 

Now observe that 
 and that 
for every .
Thus 

Finally, using the fact that , we obtain

This finishes the proof of the lemma. \qed
\end{proof}

In the rest of this section we prove Lemma~\ref{l:restricted}.

We reduce {\sf Restricted Minimum-Power Edge-Cover} to the following problem that admits an exact polynomial time algorithm, c.f. \cite{Sch}.

\vspace{0.2cm}

\noindent
{\sf Minimum-Cost Edge-Cover}: \\
{\em Instance:} \ A multi-graph (possibly with loops)  with edge-costs 
\hphantom{\em Instance: }~. \\
{\em Objective:} Find a minimum cost edge-set  that covers . 

\vspace{0.2cm}

Our reduction is not approximation ratio preserving, but incurs a loss of  in the appro\-xi\-mation ratio.
That is, given an instance  of {\sf Restricted Minimum-Power Edge-Cover}, 
we construct in polynomial time an instance  of {\sf Minimum-Cost Edge-Cover} such that:
\begin{itemize}
\item[(i)] 
For any -cover  in  corresponds a feasible 
solution  to  with .
\item[(ii)] 
, where  is an optimal solution value 
to {\sf Restricted Minimum-Power Edge-Cover} and
 is the minimum cost of a -cover in .
\end{itemize}
Hence if  is an optimal (min-cost) solution to , then 
.

Clearly, we may 
set  for all .
For  let 

Here is the construction of the instance , where  and  
consists of the following three types of edges, where for every edge 
corresponds a set  of one edge or of two edges.  

\begin{enumerate}
\item 
For every ,  has a loop-edge  with 
 where  is is an arbitrary chosen minimum cost edge 
in . \\
Here .
\item 
For every  such that ,  has an edge  with 
. \\
Here .
\item 
For every pair of edges  such that ,  has an edge  
with . \\
Here .
\end{enumerate}

\begin{lemma} \label{l:D}
Let  be a -cover in , 
let , and let  
be a power assignment defined on  by .
Then ,  is a -cover in ,
and  is a feasible solution to .
\end{lemma}
\begin{proof}
We have that  is a -cover in , by the definition of 
and since  covers both endnodes of every .  
By the definition of , we have that . 
Hence  is a feasible solution to , as claimed.

We prove that .
For  let  if  is a type~1 edge,
and  otherwise.
Note that 
, hence
 
By the definition of  and since  is a -cover
.
Also, , by the definition of .
Thus we have

It is easy to see that 

Finally, note that  for every  
(if  is a type~1 edge, this follows from our assumption that 
).
Combining we get

\qed
\end{proof}

\begin{lemma} \label{l:pi}
Let  be a feasible solution to an instance 
of {\sf Restricted Minimum-Power Edge-Cover}.
Then there exists a -cover  in  such that 
.
\end{lemma}
\begin{proof}
Let  be an inclusion minimal -cover.
We may assume that  for every .
Since any inclusion minimal -cover is a collection of node disjoint stars,
it is sufficient to prove the statement for the case when  is a star.
Then  has at most one node not in , 
and if there is such a node, then it is the center of the star, if ;
in the case  consists of a single edge , then we define the center of  
to be the endnode of  in  if such exists, 
or an arbitrary endnode of  otherwise.

We define a -cover  in , and show that 

Let  be the center of  and let  be the leaves of  
ordered by descending order of costs . 
The -cover  is defined as follows.
We cover each pair , , by a type~3 edge.
This covers all the nodes except , and maybe  if  is odd.
We add an additional edge  of type 1 or 2, if there are nodes in 
( and/or ) that remain uncovered by the picked type~3 edges. 
Formally, we have the following 4 cases, see Figure~1.
\begin{figure} \label{f:cases}
\centering
\epsfbox{cases.eps}
\caption{Illustration to the definition of the -cover .}
\end{figure}


\begin{enumerate}
\item
 is even and , see Figure~1(a). 
Then  is covered by type~3 edges. 
\item
 is odd, and , see Figure~1(b).  
Then we add a type~1 edge  to cover . 
\item
 is odd and , see Figure~1(c). 
Then we add a type~2 edge  to cover .
\item
 is even and , see Figure~1(d). 
Then we add a type~1 edge  to cover .
\end{enumerate}



Consider a type~3 edge . 
Let .
Note that . The key point is that 

This is since  while .
Therefore, 


Now, we prove that (\ref{e:pi}) hold in each one of our four cases.

\begin{enumerate}
\item 
 and  is even. Note that .
Then: 
\item 
 and  is odd. In this case  is a loop type~1 edge, so 
. 
This implies 

Thus

\item 
 and  is odd. In this case , so 
. 
This implies 
.
Thus

\item 
 and  is even. In this case  is a loop type~1 edge, 
so . This implies .
Thus

\end{enumerate}
This concludes the proof of the lemma.
\qed
\end{proof}

As was mentioned, Lemmas \ref{l:D} and \ref{l:pi} imply Lemma~\ref{l:restricted}.
Lemmas \ref{l:restricted} and \ref{l:alpha} imply Theorem~\ref{t:mpec'},
hence the proof of Theorem~\ref{t:mpec'} is now complete.

\section{Conclusions and open problems}

The main result of this paper is a new approximation algorithm for {\sf MPEMC}
with ratio . This improves the ratio  of \cite{KMNT}.
We also gave a -approximation algorithm, which is better than our 
-approximation algorithm for small values of  (roughly ).

The main open problem is whether the ratio  shown in this paper
is tight, or the problem admits a constant ratio approximation algorithm. 

\bibliographystyle{abbrv}
\bibliography{L}

\end{document}
