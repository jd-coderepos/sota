We now extend the algorithm of Section~\ref{sec:unconstrained} to the
setting $(\profit, \Feas)$ where $\Feas$ is a matroid constraint. In
particular, $\Feas$ is the set of all independent sets of a matroid
over $U$. We skip a precise definition of matroids and will only use the following facts: $\Feas$ is a downward closed feasibility constraint and there exist an exact offline and an $O(\sqrt{\log r})$ online algorithm for
$(\Phi,\Feas)$, where $\Phi$ is a sum-of-values objective and $r$ is the rank of the matroid. 

In the unconstrained setting, we showed that there always exists either a density prefix or a single element with near-optimal profit. So in the online setting it sufficed to determine the density threshold for a good prefix. In constrained settings this is no longer true, and we need to develop new techniques. Our approach is to develop a general reduction from the $\profit$ objective to two different sum-of-values type objectives over the same feasibility constraint. This allows us to employ previous work on the $(\Phi,\Feas)$ setting; we lose only a constant factor in the competitive ratio. We will first describe the reduction in the offline setting and then extend it to the online algorithm using techniques from Section~\ref{sec:unconstrained}.

\paragraph{Decomposition of $\profit$.}

For a given density $\gamma$, we define the {\em shifted density
function} $\sdf_\gamma( )$ over sets as $ \sdf_\gamma(A)
:= \sum_{e \in A} \left(\rho(e) - \gamma \right) s(e) $ and the {\em
fixed density function} $\fdf_\gamma( )$ over sizes as
$ \fdf_\gamma(s) := \gamma s - \cost(s)$. For a set $A$ we use
$\fdf_\gamma(A)$ to denote $\fdf_\gamma(s(A))$. It is immediate that
for any density $\gamma$ we can split the profit function as
$\profit(A) = \sdf_\gamma(A) + \fdf_\gamma(A)$. In particular
$\profit(\OPT)= \sdf_{\gamma}(\OPT)+\fdf_{\gamma}(\OPT)$. Our goal
will be to optimize the two parts separately and then return the
better of them.

Note that the function $\sdf_{\gamma}$ is a sum of values function
where the value of an element is defined to be $(\rho(e) - \gamma)
s(e)$. Its maximizer is a subset of $P_\gamma$, the set of elements
with nonnegative shifted density $\rho(e)-\gamma$. In order to ensure
that the maximizer, say $A$, of $\sdf_{\gamma}$ also obtains good
profit, we must ensure that $\fdf_{\gamma}(A)$ is nonnegative, and
therefore $\profit(A)\ge \sdf_{\gamma}(A)$. This is guaranteed for a
set $A$ as long as $s(A)\le \dcostinv(\gamma)$.

Likewise, the function $\fdf_{\gamma}$ increases as a function of size
$s$ as long as $s$ is at most $\dcostinv(\gamma)$, and decreases
thereafter. Therefore, in order to maximize $\fdf_\gamma$, we merely
need to find the largest (in terms of size) feasible subset of size no
more than $\dcostinv(\gamma)$. As before, if we can ensure that for
such a subset $\sdf_\gamma$ is nonnegative (e.g. if the set is a
subset of $P_\gamma$), then the profit of the set is no smaller than
its $\fdf_{\gamma}$ value. This motivates the following definition of
``bounded'' subsets:
\begin{definition}
\label{def:bounded}
Given a density $\gamma$ a subset $A \subseteq U$ is said to be $\gamma$-bounded if $A\subseteq P_\gamma$ and $ s(A) \leq \dcostinv( \gamma)$.
\end{definition}
 
We begin by formally proving that the function $\fdf_\gamma$ increases
as a function of size $s$ as long as $s$ is at most
$\dcostinv(\gamma)$.
\begin{lemma}
  \label{lem:monotone}
  If density $\gamma$ and sizes $s$ and $t$ satisfy $s \leq t \leq \dcostinv(\gamma)$, then $ \fdf_\gamma(s)
  \leq \fdf_\gamma(t) $.
\end{lemma}

\begin{proof}
Since $\cost(\cdot)$ is convex, we have that its marginal, $\dcost()$,
is monotonically non-decreasing. Thus we get the following chain of
inequalities,
  \begin{align*}
    \cost(t) - \cost(s) &= \sum_{z=s+1}^t \dcost(z) \leq (t - s)
    \times \dcost(t) \leq (t - s) \gamma.
  \end{align*}
The last inequality follows from the assumption that $t$ is no more than $\dcostinv (\gamma) $ and hence $ \dcost(t) \leq \gamma$. By
 definition of $\fdf_\gamma()$ we get the desired claim.
\end{proof}

\begin{proposition}
\label{lem:profit-lb}
For any $\gamma$-bounded set $A$, $\profit(A)  \geq \sdf_\gamma(A)$ and  $\profit(A)  \geq \fdf_\gamma(A)$.
\end{proposition}

\begin{proof}
 Since $\profit(A) = \sdf_\gamma(A) + \fdf_\gamma(A)$, it is
 sufficient to prove that $ \sdf_\gamma(A)$ and $\fdf_\gamma(A)$ are
 both non-negative. The former is clearly non-negative since all
 elements of $A$ have density at least $\gamma$. Lemma
 \ref{lem:monotone} implies that the latter is non-negative by taking
 $t = s(A)$ and $s = 0$.
\end{proof}

\noindent
For a density $\gamma$ and set $T$ we define $H_\gamma^T$ and $G_\gamma^T$ as follows. (We write $H_\gamma$ for $H_\gamma^U$ and $G_\gamma$ for $G_\gamma^U$.)
 \begin{align*}
    H_\gamma^T &= \argmax_{H \in \Feas, H \subset P_\gamma^T} \sdf_{\gamma}(H)
&
    G_\gamma^T &= \argmax_{G \in \Feas, G \subset
      \bP_\gamma^T } s(G) 
\end{align*}

Following our observations above, both $H_\gamma$ and $G_\gamma$ can
be determined efficiently (in the offline setting) using standard
matroid maximization. However, we must ensure that the two sets are
$\gamma$-bounded. Further, in order to compare the performance of
$G_{\gamma}$ against $\OPT$, we must ensure that its size is at least
a constant fraction of the size of $\OPT$. We now show that there
exists a density $\gamma$ for which $H_\gamma$ and $G_\gamma$ satisfy
these properties.

Once again, we focus on the case where no single element has high enough profit by itself. Recall that $\FOPT$ denotes $\FOPT(\I)$, $s^*$ denotes the size of this set and $\bP_\gamma$ denotes $P_\gamma \setminus \{ e_\gamma \}$. Before we proceed we need the following fact about the fractional
optimal subset $\FOPT$. 

\begin{lemma}
\label{lem:opt_frac}
 If $\FOPT$ has an element of density $\gamma$ then $s^*$ is at most
 $\dcostinv(\gamma)$.
\end{lemma}
\begin{proof}
The proof is by contradiction. Say $s^*$ is more than
$\dcostinv(\gamma)$. Recall that $e_\gamma$ denotes the element 
with density $\gamma$. We show that in such conditions reducing the
fractional contribution of $e_\gamma$, say by $\epsilon$, increases profit
giving us a better fractional solution. This is turn contradicts the
optimality of $\FOPT$.

Write $s = s(e_\gamma)$ and note that
\begin{align*} 
\profit(\FOPT) & = v(\FOPT) - \cost(s^*) \\ & = \left[ \left( v(\FOPT)
  - \gamma \times \epsilon s \right) - \cost(s^* - \epsilon s )
  \right] - \left[ \cost(s^*) - \cost(s^* - \epsilon s ) - \gamma
  \times \epsilon s \right].
\end{align*}
If $\epsilon$ is such that $s^* -\epsilon > \dcostinv(\gamma) $, then we have $ \cost(s^*) - \cost(s^* -
\epsilon s ) > \gamma \times \epsilon s $; thus we get that the term 
$ \left[ \cost(s^*) - \cost(s^* - \epsilon s ) - \gamma \times \epsilon s
  \right]$ is positive which proves the claim.
\end{proof}


\begin{definition}
Let $\rhoc$ be the largest density such that $\preflarge$ has a feasible set of size at least $s^*$.
\end{definition}

\noindent
We now state a useful property of $\rhoc$. 
\begin{lemma}
  \label{lem:rhoc-indep}
Any feasible set in $\bP_\rhoc$ is $\rhoc$-bounded and has size less than $s^*$. Moreover for any density $\gamma > \rhoc$ all feasible subsets of $P_\gamma$ are $\gamma$-bounded. 
\end{lemma}

\begin{proof}
By definition, the size of any feasible set contained in $P_\rhoc
\setminus \{ e_\rhoc \}$ is no more than $s^*$.

We will show that $s^* \leq \dcostinv( \rhoc)$. Then the first part of the lemma
follows immediately. For the second part we have $\gamma > \rhoc$ and hence 
$\dcostinv(\rhoc) \leq \dcostinv(\gamma) $. Overall a size bound of $s^*$ also implies that 
feasible sets in $P_\gamma$ satisfy the size requirement for being
$\gamma$-bounded. Hence we get that all feasible sets in $P_\rhoc
\setminus \{ e_\rhoc \}$ are $\rhoc$-bounded and all feasible sets
in $P_\gamma$ are $\gamma$-bounded.

The size of $\FOPT$ is at most the size of its support. Thus the
support of $\FOPT$ is a feasible set of size at least $s^*$. By
definition, $\rhoc$ is the largest density such that $P_\rhoc$
contains a feasible set of size $s^*$. Hence we get that $\FOPT$
contains an element of density less than or equal to $\rhoc$. Applying
Lemma \ref{lem:opt_frac} we get $s^* \leq \dcostinv(\rhoc)$ and the
lemma follows.
\end{proof}

\noindent
The following is our main claim of this section. 
\begin{lemma}
  \label{lem:goodprefix} 
For any density $\gamma>\rhoc$,
$\profit(\OPT(\bP_\gamma )) \le \profit(H_\gamma)+\profit(G_\gamma)$. Furthermore, $\profit(\OPT) \leq \profit(H_\rhoc) + \profit(G_\rhoc) + 2\max_{e\in U} \profit(e)$.
\end{lemma}

\begin{proof}
Let $P$, $H$, and $G$ denote $\preflarge$, $H_{\rhoc}$, and
$G_{\rhoc}$ respectively. As in the unconstrained setting, there can
be at most one element in the intersection of $\OPT$ and $\Fr$ 
(see proof of Lemma \ref{lemma:offline-knapsack}). Note that $\profit()$ is subadditive, hence
$\profit( \OPT \cap \I) + \max_{e \in U } \profit(e) \geq
\profit(\OPT)$.  In the analysis below we do not consider elements
present in $\Fr$ and show that $\profit(H) + \profit(G) + \max_{e \in
  U } \profit(e) \geq \profit(\OPT \cap \I)$.  This in turn
establishes the second part of the Lemma.



For ease of notation we denote $e_\rhoc$ as $e'$. Without loss of
generality, we can assume that $H$ does not contain $e'$ since
$\rho(e') - \rhoc = 0$.  Therefore set $H$ is contained in $P
\setminus \{ e' \}$. By Lemma \ref{lem:rhoc-indep} we get that $H$ is
$\rhoc$-bounded.

Note that by definition $G$ does not contain $e'$, hence its size is
at most $s^*$.  Also, $G \cup \{e' \}$ is no smaller than the
maximum-size feasible subset contained in $P$. So, by definition of
$\rhoc$, we also have $s(G) + s(e') \geq s^*$. Thus there exists
$\alpha < 1$ such that the fractional set $F = G \cup \{ \alpha e' \}
$ has size exactly equal to $s^*$.





Next we split the profit of $\FOPT$ into two parts, and bound the first
by $\sdf_\rhoc(H)$ and the second by $\fdf_\rhoc (F)$:
\begin{align*}
  \profit(\FOPT) &= \sdf_\rhoc( \FOPT) + \fdf_\rhoc(s^*) 
  \leq \sdf_\rhoc(H) + \fdf_\rhoc(F) .
\end{align*}
\noindent Note that we can drop elements which have a negative contribution to
the sum to get
\begin{align*}
\sdf_\rhoc(\FOPT) & \leq \sdf_\rhoc(\FOPT \cap \preflarge ) \\
& \leq \sdf_\rhoc \left( \supp(\FOPT) \cap \preflarge \right) \\
& \leq \sdf_\rhoc(H) \\
& \leq \profit(H).
\end{align*}
The second inequality follows since we can only increase the value of
a subset by ``rounding up'' fractional elements. The third inequality
follows from the optimality of $H$, and the fourth from the fact that
it is $\rhoc$-bounded.

To bound the second part we note that $s(F) = s^*$, hence
$\fdf_\rhoc(F) = \rhoc s^* - \cost(s^*)$. Elements in $F$ have density
no less than $\rhoc$ and its size is bounded above by
$\dcostinv(\rhoc)$, hence it is a $\rhoc$-bounded set implying that
$\profit(F) \geq \fdf_\rhoc(F) $.  Note that $F = G \cup \{ \alpha e'
\}$, and by sub-additivity of $\profit()$ we have $\profit(G) +
\profit(\alpha e' ) \geq \profit(F) $. Moreover $e' \in \I$ implies
$\profit(\alpha e' ) \leq \profit(e')$ and hence we get
\begin{align*}
\profit(\FOPT) & \leq \profit(H) + \fdf_\gamma(F) \leq \profit(H) + \profit(G) + \profit(e'),
\end{align*}
which proves the second half of the lemma.

The first half of the lemma follows along similar lines. We 
have the standard decomposition, $\profit(\OPT( \bar{P}_\gamma) ) =
\sdf_\gamma(\OPT(\bar{P}_\gamma)) + \fdf_\gamma(\OPT(
\bar{P}_\gamma))$. By definition, $H_\gamma$ is the
constrained maximizer of $\sdf_\gamma$, hence we get  
$\sdf_\gamma(\OPT( \bar{P}_\gamma) ) \leq \sdf_\gamma(H_\gamma)$. We 
note that all feasible sets in $P_\gamma$ are $\gamma$-bounded, 
for density $\gamma > \rhoc$ (Lemma \ref{lem:rhoc-indep}). Hence, by Lemma \ref{lem:monotone}, $\fdf_\gamma$ strictly increases with size when restricted to feasible sets in $\bar{P}_\gamma$. $G_\gamma$ is the largest such set, hence we get $\fdf_\gamma(G_\gamma) \geq \fdf_\gamma(\OPT(\bar{P}_\gamma))$. $H_\gamma$ and $G_\gamma$ are $\gamma$-bounded and hence by Proposition \ref{lem:profit-lb} we have $\profit(H_\gamma) \geq \sdf_\gamma(H_\gamma)$ and
$\profit(G_\gamma) \geq \fdf_\gamma(G_\gamma)$. This establishes the lemma.
\end{proof}

This lemma immediately gives us an offline approximation algorithm for
$(\profit,\Feas)$: for every element density $\gamma$, we find the
sets $H_{\gamma}$ and $G_{\gamma}$; we then output the best (in terms of profit) of these sets or the best individual element. We obtain the following theorem:

\begin{theorem}
\label{theorem:matroid_offline}
Algorithm~\ref{Algorithm:offline-matroid} $4$-approximates
$(\profit,\Feas)$ in the offline setting.  
\end{theorem}


\begin{algorithm}
  \caption{Offline algorithm for single-dimensional $(\profit, \Feas)$}
\label{Algorithm:offline-matroid}
\begin{algorithmic}[1]
\STATE Set $A_{\max{}} \leftarrow \argmax_{H \in \{ H_\gamma\}_\gamma } \profit(H)$
\STATE Set $B_{\max{}} \leftarrow \argmax_{ G \in \{G_\gamma \}_\gamma } \profit(G)$ 
\STATE Set $ e_{\max{}} \leftarrow \argmax_{e \in
  U} \profit(e) $ 
\STATE Assign $\alg(U) \leftarrow \argmax_{ S \in \{ A_{\max{}}, B_{\max{}}, e_{\max{}} \}} \profit(S)$
\end{algorithmic}
\end{algorithm}

\paragraph{The online setting.}
Our online algorithm, as in the unconstrained case, uses a sample $X$
from $U$ to obtain an estimate $\tau$ for the density $\rhoc$. Then
with equal probability it applies the online algorithm for
$(\sdf_\tau,\Feas)$ on the remaining set $Y\cap P_\tau$ or the online
algorithm for $(s,\Feas)$ (in order to maximize $\fdf_\tau$) on $Y\cap
P_\tau$. Lemma~\ref{lem:goodprefix} indicates
that it should suffice for $\tau$ to be larger than $\rhoc$ while
ensuring that $\profit(\OPT(P_\tau))$ is large enough. As in
Section~\ref{sec:unconstrained} we define the density $\rhob$ as the
upper limit on $\tau$, and claim that $\tau$ satisfies the required
properties with high probability.

\begin{algorithm}
  \caption{Online algorithm for single-dimensional $(\profit,\Feas)$ }\label{Algorithm:online-matroid}
\begin{algorithmic}[1]
\STATE Draw $k$ from $\textrm{Binomial}(n,1/2)$.  
\STATE Select the first $k$ elements to
be in the sample $X$. Unconditionally reject these elements.  \STATE
Toss a fair coin.  \IF {Heads} \STATE Set output $O$ as the first
element, over the remaining stream, with profit higher than $\max_{e \in X} \profit(e)$.
\ELSE \STATE Determine $\alg(X)$ using the offline Algorithm
\ref{Algorithm:offline-matroid}.  \STATE Let $\eone$ be a specified
constant and let $\guess$ be the highest density such that
$\profit(\alg(\prefguess^X)) \geq \frac{\eone}{16}\profit(\alg(X))$.
\STATE \label{step:online-start} Toss a fair coin.  \IF {Heads}
\STATE \label{step:redn-1} Let $O_1$ be the result of executing an
online algorithm for $(\sdf_\guess,\Feas)$ on the subset $P^Y_\tau$ of the remaining
stream with the objective function
\[\sdf_{\guess}(A) = \sum_{e \in A} (\rho(e) - \guess) s(e)\]
\STATE Set $O \leftarrow \emptyset$.  \FOR {$e \in
  O_1$} \label{step:for-1} \IF {$\profit(O \cup \{e\}) - \profit(O)
  \geq 0$} \STATE Set $O \leftarrow O \cup \{e\}$.  \ENDIF \ENDFOR
\STATE Output $O$.  \ELSE \STATE\label{step:redn-2} Let $O_2$ be the result
of executing an online algorithm for $\Feas$ on the subset $P^Y_\tau$
of the remaining stream with objective function $s()$.  \STATE
Set $O \leftarrow \emptyset$.  \FOR {$e \in O_2$} \label{step:for-2}
\IF {$\profit(O \cup \{e\}) - \profit(O) \geq 0$} \STATE Set $O
\leftarrow O \cup \{e\}$.  \ENDIF \ENDFOR
\STATE\label{step:online-end} Output $O$.  \ENDIF \ENDIF
\end{algorithmic}
\end{algorithm}
 
Note that we use an algorithm for
$(\Phi,\Feas)$ where $\Phi$ is a sum-of-values objective in Algorithm
\ref{Algorithm:online-matroid} as a black box. For example if the
underlying feasibility constraint is a partition matroid we execute
the partition-matroid secretary algorithm in steps~\ref{step:redn-1}
and \ref{step:redn-2} as subroutines. Since these subroutines are online algorithms, 
we can execute steps~\ref{step:redn-1} and
\ref{step:redn-2} in parallel with the respective `for' loops in
steps~\ref{step:for-1} and \ref{step:for-2}. This ensures that all
accepted elements have positive profit increment.

\begin{definition}
For a fixed parameter $\eone \leq 1 $, let $\rhob$ be the highest density with $\profit(\OPT(\prefsmall)) \geq (\eone/16)^2 \profit(\OPT)$.
\end{definition}

\begin{lemma}
  \label{lem:main} 
For fixed parameters $\const[1] \geq 1 $, $\const[2] \leq 1$,
$\eone \leq 1$ and $\etwo \leq 1$ suppose that there is no element
with profit more than $\frac{1}{\const[1]}\profit(\OPT)$. Then with
probability at least $\const[2]$, we have that if $\guess$ is the highest density  
such that $\profit(\alg(\prefguess^X)) \geq \frac{\eone}{16}\profit(\alg(X))$,
then $\guess$ satisfies
$\rhob\geq \guess \geq \rhoc$ and
$\profit(\OPT(\prefguess^Y)) \geq \etwo(\eone/16)^2 \profit(\OPT)$.
\end{lemma}

\begin{proof}
We first define two events analogous to the events $E_1$ and $E_2$ in
Section~\ref{sec:unconstrained}.
\begin{align*}
E_1: \profit(\OPT(\preflarge^X)) & \geq \eone
\profit(\OPT(\preflarge)) \\ E_2: \profit(\OPT(\prefsmall^Y)) & \geq
\etwo \profit(\OPT(\prefsmall))
\end{align*}

We claim that the event $E_1$ immediately implies
$\rhob\ge\guess\ge\rhoc$. Furthermore, when $\rhob\ge\guess\ge\rhoc$,
we get the containment
$\prefsmall^Y \subset \prefguess^Y \subset \preflarge^Y$, which
implies
$\profit(\OPT(\prefguess^Y)) \geq \profit(\OPT(\prefsmall^Y))$. This
inequality, when combined with event $E_2$ and the definition of
$\rhob$, proves the second condition. We furthermore claim that $E_1$
and $E_2$ simultaneously hold with probability at least $\const[2]$,
which would give the desired result.

Thus, we are done if we demonstrate that event $E_1$ implies
$\rhob\ge\guess\ge\rhoc$, and that the probability of $E_1$ and $E_2$
occurring simultaneously is at least $\const[2]$; we now proceed to prove each of
these claims in turn.






\begin{claim}
\label{lem:upperbound_guess}
  Event $E_1$ implies that $\rhob \geq \guess$ and so $\prefsmall^Y
  \subset \prefguess^Y$.
\end{claim}

\begin{proof}
  First, by containment and optimality, we observe that
  \[\profit(\OPT(\prefguess))
  \geq \profit(\OPT(\prefguess^X)) \geq \profit(\alg(\prefguess^X)).\] 
  By definition of $\guess$, we have $\profit(\alg(\prefguess^X)) \geq
  \frac{\eone}{16}\profit(\alg(X))$. Furthermore, \[\profit(\alg(X))
  \geq \frac{1}{4}\profit(\OPT(X)) \geq
  \frac{1}{4}\profit(\OPT(\preflarge^X)). \]
  Lemma \ref{lem:goodprefix} gives us $\profit(\OPT(\preflarge)) \geq
  \frac{1}{4}\profit(\OPT)$. This together with event $E_1$ implies
  \[\profit(\OPT(\preflarge^X)) 
  \geq \eone \profit(\OPT(\preflarge)) \geq \frac{\eone}{4}
  \profit(\OPT).\]
  Thus, we have that $\profit(\OPT(\prefguess)) \geq
  \left(\frac{\eone}{16}\right)^2 \profit(\OPT)$.  We have defined
  $\rhob$ to be the largest density for which the previous profit
  inequality holds. Hence we conclude that $\rhob \geq \guess$.
\end{proof}
\begin{claim}
\label{lem:lowerbound_guess}
  Event $E_1$ implies that $\guess \geq \rhoc$.
\end{claim}

\begin{proof}

As stated before, Lemma \ref{lem:goodprefix} gives us
$\profit(\OPT(\preflarge)) \geq \frac{1}{4}\profit(\OPT)$. We have
that $\profit(\alg(\preflarge^X)) \geq \frac{1}{4}
\profit(\OPT(\preflarge^X))$. Now, $E_1$ implies that
  \[\profit(\OPT(\preflarge^X)) 
  \geq \eone \profit(\OPT(\preflarge)) \geq \frac{\eone}{4}
  \profit(\OPT).\] Combining these we get that
  \[\profit(\alg(\preflarge^X)) \geq \frac{\eone}{16} \profit(\OPT) \geq
  \frac{\eone}{16} \profit(\alg(X)).\]

  Since $\guess$ is the largest density for which the above inequality
  holds we have $\guess \geq \rhoc$.
\end{proof}



\begin{claim}
\label{lem:chernoff-matroid}
For a fixed constant $\const[1]$, if no element of $S$ has profit more
than $\frac{1}{\const[1]} \profit( \OPT) $ then $\textrm{Pr}[E_1
  \wedge E_2] \geq 0.52$.
\end{claim}
\begin{proof}
The proof of this claim is similar to that of Lemma
\ref{lemma:chernoff-unconstrained}. We show that $\Pr[E_1 ] \ge 0.76
$ and $\Pr[ E_2] \ge 0.76$; the result then follows by applying the
union bound.  We begin by observing that
$\profit(\OPT(\preflarge^X)) \geq \profit(\OPT(\preflarge) \cap X)$,
and so it suffices to bound the probability that
$\profit(\OPT(\preflarge) \cap X) \geq \eone
\profit(\OPT(\preflarge))$ and likewise the probability that $\profit(\OPT(\prefsmall)
\cap Y) \geq \etwo \profit(\OPT(\prefsmall))$.



We fix an ordering over elements of $\OPT(\preflarge)$, such that the
profit increments are non-increasing. That is, if $L_i$ is the set
containing elements $1$ through $i$ and hence the profit increment of
the $i$th element is $\incprofit(i) := \profit(L_i) -
\profit(L_{i-1})$, then we have $\incprofit(1) \geq \incprofit(2) \geq
\ldots$. Note that such an ordering can be determined by greedily
picking elements from $\OPT(\preflarge)$ such that the profit
increment at each step is maximized.

We set $\const[1] \geq \lOneMinK$, now the fact that no element in $S$ has
profit more than $\frac{1}{\lOneMinK} \profit( \OPT) $ implies no element by
itself has profit more than $1/\lOneSize \ \profit(\OPT(\preflarge))$, since
$ \profit(\OPT(\preflarge)) \geq 1/3 \profit(\OPT)$. Profit increments
of elements are upper bounded by their profits, therefore we can apply
Lemma \ref{lemma:concentration} with $\OPT(\preflarge)$ as the fixed
set and profit increments as the weights. By optimality of
$\OPT(\preflarge)$ we have that these profit increments are
non-negative hence the required conditions of Lemma
\ref{lemma:concentration} hold and we have $ \profit(\OPT(\preflarge)
\cap X) \geq \eone \profit( \OPT(\preflarge)) $ with probability at
least $0.76$.  Since $\profit(\OPT(\preflarge^X)) \geq
\profit(\OPT(\preflarge) \cap X)$ we get $\Pr[ E_1 ] \ge 0.76$ with
$\eone = \lOneRatio$.

By definition of $\rhob$ the profit of $\OPT(\prefsmall)$ is at least
$\left(\frac{\eone}{9}\right)^2 \profit(\OPT)$. With $\const[1] \geq
\lOneSize \left( \frac{9}{\eone}\right)^2 $ and $\etwo = \lOneRatio$ we can again
apply Lemma \ref{lemma:concentration} to show that $\Pr[ E_2]
\ge0.76$. Hence we can set $\const[2] = 0.52$ and this completes the proof
of the claim.
\end{proof}

With the demonstration that the above three claims hold, our proof is complete.
\end{proof}

To conclude the analysis, we show that if the online algorithms for
$(\sdf_\tau,\Feas)$ and $(s,\Feas)$ have a
competitive ratio of $\alpha$, then we obtain an $O(\alpha)$
approximation to $\profit(\OPT(P^Y_\tau))$. We therefore get the
following theorem.

\begin{theorem}
\label{thm:online-matroid}
If there exists an $\alpha$-competitive algorithm for the matroid
secretary problem $(\Phi, \Feas)$ where $\Phi$ is a sum-of-values objective,
then Algorithm~\ref{Algorithm:online-matroid} achieves a competitive ratio of
$O(\alpha)$ for the problem $(\profit, \Feas)$.
\end{theorem}

Before we proceed to prove Theorem~\ref{thm:online-matroid}, we show
that in steps \ref{step:online-start} to \ref{step:online-end} the
algorithm obtains a good approximation to $\OPT(\prefguess^Y)$.

\begin{lemma}
  \label{lem:secondstage}
Suppose that there is an $\alpha$-competitive algorithm for
$(\Phi,\Feas)$ where $\Phi$ is any sum-of-values objective. For a fixed set
$Y$ and threshold $\guess$, satisfying $\guess \geq \rhoc$, we have $
\E_\sigma[\profit(O_1) + \profit(O_2)] \geq \alpha
\ \profit(\OPT(\prefguess^Y))$, where the expectation is over all
permutations $\sigma$ of $Y$.
\end{lemma}

\begin{proof}
 The threshold $\tau$ is either equal to or strictly greater than
 $\rhoc$. In the former case $e_\rhoc$ must have been in the sample
 set $X$ and hence $O_1, O_2 \subseteq \bP_\rhoc$. By Lemma \ref{lem:rhoc-indep} we show that $O_1$ and $O_2$ are
 $\rhoc$-bounded and hence $\tau$-bounded. On the other hand if $\tau
 > \rhoc$ we can again apply Lemma \ref{lem:rhoc-indep} and get that
 $O_1$ and $O_2$ are $\tau$-bounded.
 
Hence by Proposition \ref{lem:profit-lb} we get the inequalities
$\E_\sigma[\profit(O_1)] \geq \E_\sigma\left[ \sdf_\guess(O_1)\right]$
and $\E_\sigma[\profit(O_2)] \geq \E_\sigma[ \fdf_\guess(O_2)]$.
  
  By applying the $\alpha$-competitive matroid secretary algorithm with
  objective $\sdf_\guess$ (Step 11 of Algorithm
  \ref{Algorithm:online-matroid} ) we get
  \begin{align*}
    \E_\sigma[ \sdf_\guess(O_1) ] &\geq \alpha \times \sdf_\guess(
    H_\guess^Y) \\ &\geq \alpha \times \sdf_\guess(\OPT(\prefguess^Y)),
  \end{align*}
\noindent where the second inequality follows from the optimality of $H_\guess^Y$. 

Next we bound $\E_\sigma[\fdf_\guess(O_2)]$. Let
$K$ be the largest feasible subset contained in $\prefguess^Y$. The
fact that the underlying algorithm is $\alpha$-competitive implies
$\E_\sigma[s(O_2)] \geq \alpha \times s(K)$.

Note that, as observed above, $\prefguess^Y \subset \bP_\rhoc$. Since $K \subset \prefguess^Y $, by definition of $\rhoc$ we get that $s(K) \leq
s^*$. So, for $O_2$ we have
  \begin{align*}
    \E_\sigma[\fdf_\guess(O_2)] &\geq \E_\sigma[\guess s(O_2) -
      \cost(s(O_2)]\\ &\geq \E_\sigma\left[\guess
      \left(\frac{s(O_2)}{s(K)}\right) s(K) -
      \left(\frac{s(O_2)}{s(K)}\right) \cost(s(K))\right] \\
      & = \E_\sigma\left[\frac{s(O_2)}{s(K)}\right] \left(\guess s(K) -
    \cost(s(K))\right)\\ 
    &\geq \alpha (\guess s(K) - \cost(s(K)))\\ & = \alpha
    \ \fdf_\guess(K) \\ & \geq \alpha \ \fdf_\guess(\OPT(\prefguess^Y)).
  \end{align*}

\noindent Since $s(\OPT(\prefguess^Y)) \leq s(K) \leq s^* \leq
\dcostinv(\guess)$ we get the last inequality by applying Lemma
\ref{lem:monotone}.

  The conclusion of the lemma now follows from the decomposition
  $\profit(\OPT(\prefguess^Y)) = \sdf_\guess(\OPT(\prefguess^Y)) +
  \fdf_\guess (\OPT(\prefguess^Y) ) $.
  \end{proof}

\begin{proof}[Proof of Theorem~\ref{thm:online-matroid}]
With probability $\frac{1}{2}$ we apply the standard secretary algorithm which is 
$e$-competitive. If an element has profit more than
$\frac{1}{\const[1]} \profit(\OPT)$, in expectation we get a profit of
$\frac{1}{2 \const[1] e}$ times the optimal.

We have $\Pr[E_1 \wedge E_2]\geq k_2 $, for a fixed constant $k_2$. Also, the events $E_1$ and $E_2$ 
depend only on what elements are in $X$ and $Y$, 
and not on their ordering in the stream. So conditioned on $E_1$
and $E_2$, the remaining stream is still a uniformly random
permutation of $Y$. Therefore, if no element has profit more than
$\frac{1}{\const[1]} \profit(\OPT)$ we can apply the second inequality
of Lemma \ref{lem:main} and Lemma \ref{lem:secondstage} along with the
fact that we output $O_1$ and $O_2$ with probability $\frac{1}{4}$
each,  to show that
\begin{align*}
\E[\profit(O)] &\geq \frac{1}{4} \E[\profit(O_1) +
  \profit(O_2)\ |\ E_1 \wedge E_2]\times \Pr[E_1 \wedge E_2]  \\ &\geq \frac{\alpha \const[2]
}{4}\E[\profit(\OPT(\prefguess^Y))\ |\ E_1 \wedge E_2]\\ &\geq
\frac{\alpha \const[2] \etwo}{4}\left(\frac{\eone}{16}\right)^2
\profit(\OPT).
\end{align*}

Overall, we have that \[\E[\profit(O)] \geq \min\left\{\frac{ \alpha
  \const[2]\etwo}{4}\left(\frac{\eone}{16}\right)^2, \frac{1}{2
  \const[1] e}\right\} \cdot \profit(\OPT). \] Since all the involved
parameters are fixed constants we get the desired result.
\end{proof}
















































