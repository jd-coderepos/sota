We now extend the algorithm of Section~\ref{sec:unconstrained} to the
setting  where  is a matroid constraint. In
particular,  is the set of all independent sets of a matroid
over . We skip a precise definition of matroids and will only use the following facts:  is a downward closed feasibility constraint and there exist an exact offline and an  online algorithm for
, where  is a sum-of-values objective and  is the rank of the matroid. 

In the unconstrained setting, we showed that there always exists either a density prefix or a single element with near-optimal profit. So in the online setting it sufficed to determine the density threshold for a good prefix. In constrained settings this is no longer true, and we need to develop new techniques. Our approach is to develop a general reduction from the  objective to two different sum-of-values type objectives over the same feasibility constraint. This allows us to employ previous work on the  setting; we lose only a constant factor in the competitive ratio. We will first describe the reduction in the offline setting and then extend it to the online algorithm using techniques from Section~\ref{sec:unconstrained}.

\paragraph{Decomposition of .}

For a given density , we define the {\em shifted density
function}  over sets as  and the {\em
fixed density function}  over sizes as
. For a set  we use
 to denote . It is immediate that
for any density  we can split the profit function as
. In particular
. Our goal
will be to optimize the two parts separately and then return the
better of them.

Note that the function  is a sum of values function
where the value of an element is defined to be . Its maximizer is a subset of , the set of elements
with nonnegative shifted density . In order to ensure
that the maximizer, say , of  also obtains good
profit, we must ensure that  is nonnegative, and
therefore . This is guaranteed for a
set  as long as .

Likewise, the function  increases as a function of size
 as long as  is at most , and decreases
thereafter. Therefore, in order to maximize , we merely
need to find the largest (in terms of size) feasible subset of size no
more than . As before, if we can ensure that for
such a subset  is nonnegative (e.g. if the set is a
subset of ), then the profit of the set is no smaller than
its  value. This motivates the following definition of
``bounded'' subsets:
\begin{definition}
\label{def:bounded}
Given a density  a subset  is said to be -bounded if  and .
\end{definition}
 
We begin by formally proving that the function  increases
as a function of size  as long as  is at most
.
\begin{lemma}
  \label{lem:monotone}
  If density  and sizes  and  satisfy , then .
\end{lemma}

\begin{proof}
Since  is convex, we have that its marginal, ,
is monotonically non-decreasing. Thus we get the following chain of
inequalities,
  
The last inequality follows from the assumption that  is no more than  and hence . By
 definition of  we get the desired claim.
\end{proof}

\begin{proposition}
\label{lem:profit-lb}
For any -bounded set ,  and  .
\end{proposition}

\begin{proof}
 Since , it is
 sufficient to prove that  and  are
 both non-negative. The former is clearly non-negative since all
 elements of  have density at least . Lemma
 \ref{lem:monotone} implies that the latter is non-negative by taking
  and .
\end{proof}

\noindent
For a density  and set  we define  and  as follows. (We write  for  and  for .)
 

Following our observations above, both  and  can
be determined efficiently (in the offline setting) using standard
matroid maximization. However, we must ensure that the two sets are
-bounded. Further, in order to compare the performance of
 against , we must ensure that its size is at least
a constant fraction of the size of . We now show that there
exists a density  for which  and  satisfy
these properties.

Once again, we focus on the case where no single element has high enough profit by itself. Recall that  denotes ,  denotes the size of this set and  denotes . Before we proceed we need the following fact about the fractional
optimal subset . 

\begin{lemma}
\label{lem:opt_frac}
 If  has an element of density  then  is at most
 .
\end{lemma}
\begin{proof}
The proof is by contradiction. Say  is more than
. Recall that  denotes the element 
with density . We show that in such conditions reducing the
fractional contribution of , say by , increases profit
giving us a better fractional solution. This is turn contradicts the
optimality of .

Write  and note that

If  is such that , then we have ; thus we get that the term 
 is positive which proves the claim.
\end{proof}


\begin{definition}
Let  be the largest density such that  has a feasible set of size at least .
\end{definition}

\noindent
We now state a useful property of . 
\begin{lemma}
  \label{lem:rhoc-indep}
Any feasible set in  is -bounded and has size less than . Moreover for any density  all feasible subsets of  are -bounded. 
\end{lemma}

\begin{proof}
By definition, the size of any feasible set contained in  is no more than .

We will show that . Then the first part of the lemma
follows immediately. For the second part we have  and hence 
. Overall a size bound of  also implies that 
feasible sets in  satisfy the size requirement for being
-bounded. Hence we get that all feasible sets in  are -bounded and all feasible sets
in  are -bounded.

The size of  is at most the size of its support. Thus the
support of  is a feasible set of size at least . By
definition,  is the largest density such that 
contains a feasible set of size . Hence we get that 
contains an element of density less than or equal to . Applying
Lemma \ref{lem:opt_frac} we get  and the
lemma follows.
\end{proof}

\noindent
The following is our main claim of this section. 
\begin{lemma}
  \label{lem:goodprefix} 
For any density ,
. Furthermore, .
\end{lemma}

\begin{proof}
Let , , and  denote , , and
 respectively. As in the unconstrained setting, there can
be at most one element in the intersection of  and  
(see proof of Lemma \ref{lemma:offline-knapsack}). Note that  is subadditive, hence
.  In the analysis below we do not consider elements
present in  and show that .  This in turn
establishes the second part of the Lemma.



For ease of notation we denote  as . Without loss of
generality, we can assume that  does not contain  since
.  Therefore set  is contained in . By Lemma \ref{lem:rhoc-indep} we get that  is
-bounded.

Note that by definition  does not contain , hence its size is
at most .  Also,  is no smaller than the
maximum-size feasible subset contained in . So, by definition of
, we also have . Thus there exists
 such that the fractional set  has size exactly equal to .





Next we split the profit of  into two parts, and bound the first
by  and the second by :

\noindent Note that we can drop elements which have a negative contribution to
the sum to get

The second inequality follows since we can only increase the value of
a subset by ``rounding up'' fractional elements. The third inequality
follows from the optimality of , and the fourth from the fact that
it is -bounded.

To bound the second part we note that , hence
. Elements in  have density
no less than  and its size is bounded above by
, hence it is a -bounded set implying that
.  Note that , and by sub-additivity of  we have . Moreover  implies
 and hence we get

which proves the second half of the lemma.

The first half of the lemma follows along similar lines. We 
have the standard decomposition, . By definition,  is the
constrained maximizer of , hence we get  
. We 
note that all feasible sets in  are -bounded, 
for density  (Lemma \ref{lem:rhoc-indep}). Hence, by Lemma \ref{lem:monotone},  strictly increases with size when restricted to feasible sets in .  is the largest such set, hence we get .  and  are -bounded and hence by Proposition \ref{lem:profit-lb} we have  and
. This establishes the lemma.
\end{proof}

This lemma immediately gives us an offline approximation algorithm for
: for every element density , we find the
sets  and ; we then output the best (in terms of profit) of these sets or the best individual element. We obtain the following theorem:

\begin{theorem}
\label{theorem:matroid_offline}
Algorithm~\ref{Algorithm:offline-matroid} -approximates
 in the offline setting.  
\end{theorem}


\begin{algorithm}
  \caption{Offline algorithm for single-dimensional }
\label{Algorithm:offline-matroid}
\begin{algorithmic}[1]
\STATE Set 
\STATE Set  
\STATE Set  
\STATE Assign 
\end{algorithmic}
\end{algorithm}

\paragraph{The online setting.}
Our online algorithm, as in the unconstrained case, uses a sample 
from  to obtain an estimate  for the density . Then
with equal probability it applies the online algorithm for
 on the remaining set  or the online
algorithm for  (in order to maximize ) on . Lemma~\ref{lem:goodprefix} indicates
that it should suffice for  to be larger than  while
ensuring that  is large enough. As in
Section~\ref{sec:unconstrained} we define the density  as the
upper limit on , and claim that  satisfies the required
properties with high probability.

\begin{algorithm}
  \caption{Online algorithm for single-dimensional  }\label{Algorithm:online-matroid}
\begin{algorithmic}[1]
\STATE Draw  from .  
\STATE Select the first  elements to
be in the sample . Unconditionally reject these elements.  \STATE
Toss a fair coin.  \IF {Heads} \STATE Set output  as the first
element, over the remaining stream, with profit higher than .
\ELSE \STATE Determine  using the offline Algorithm
\ref{Algorithm:offline-matroid}.  \STATE Let  be a specified
constant and let  be the highest density such that
.
\STATE \label{step:online-start} Toss a fair coin.  \IF {Heads}
\STATE \label{step:redn-1} Let  be the result of executing an
online algorithm for  on the subset  of the remaining
stream with the objective function

\STATE Set .  \FOR {} \label{step:for-1} \IF {} \STATE Set .  \ENDIF \ENDFOR
\STATE Output .  \ELSE \STATE\label{step:redn-2} Let  be the result
of executing an online algorithm for  on the subset 
of the remaining stream with objective function .  \STATE
Set .  \FOR {} \label{step:for-2}
\IF {} \STATE Set .  \ENDIF \ENDFOR
\STATE\label{step:online-end} Output .  \ENDIF \ENDIF
\end{algorithmic}
\end{algorithm}
 
Note that we use an algorithm for
 where  is a sum-of-values objective in Algorithm
\ref{Algorithm:online-matroid} as a black box. For example if the
underlying feasibility constraint is a partition matroid we execute
the partition-matroid secretary algorithm in steps~\ref{step:redn-1}
and \ref{step:redn-2} as subroutines. Since these subroutines are online algorithms, 
we can execute steps~\ref{step:redn-1} and
\ref{step:redn-2} in parallel with the respective `for' loops in
steps~\ref{step:for-1} and \ref{step:for-2}. This ensures that all
accepted elements have positive profit increment.

\begin{definition}
For a fixed parameter , let  be the highest density with .
\end{definition}

\begin{lemma}
  \label{lem:main} 
For fixed parameters , ,
 and  suppose that there is no element
with profit more than . Then with
probability at least , we have that if  is the highest density  
such that ,
then  satisfies
 and
.
\end{lemma}

\begin{proof}
We first define two events analogous to the events  and  in
Section~\ref{sec:unconstrained}.


We claim that the event  immediately implies
. Furthermore, when ,
we get the containment
, which
implies
. This
inequality, when combined with event  and the definition of
, proves the second condition. We furthermore claim that 
and  simultaneously hold with probability at least ,
which would give the desired result.

Thus, we are done if we demonstrate that event  implies
, and that the probability of  and 
occurring simultaneously is at least ; we now proceed to prove each of
these claims in turn.






\begin{claim}
\label{lem:upperbound_guess}
  Event  implies that  and so .
\end{claim}

\begin{proof}
  First, by containment and optimality, we observe that
   
  By definition of , we have . Furthermore, 
  Lemma \ref{lem:goodprefix} gives us . This together with event  implies
  
  Thus, we have that .  We have defined
   to be the largest density for which the previous profit
  inequality holds. Hence we conclude that .
\end{proof}
\begin{claim}
\label{lem:lowerbound_guess}
  Event  implies that .
\end{claim}

\begin{proof}

As stated before, Lemma \ref{lem:goodprefix} gives us
. We have
that . Now,  implies that
   Combining these we get that
  

  Since  is the largest density for which the above inequality
  holds we have .
\end{proof}



\begin{claim}
\label{lem:chernoff-matroid}
For a fixed constant , if no element of  has profit more
than  then .
\end{claim}
\begin{proof}
The proof of this claim is similar to that of Lemma
\ref{lemma:chernoff-unconstrained}. We show that  and ; the result then follows by applying the
union bound.  We begin by observing that
,
and so it suffices to bound the probability that
 and likewise the probability that .



We fix an ordering over elements of , such that the
profit increments are non-increasing. That is, if  is the set
containing elements  through  and hence the profit increment of
the th element is , then we have . Note that such an ordering can be determined by greedily
picking elements from  such that the profit
increment at each step is maximized.

We set , now the fact that no element in  has
profit more than  implies no element by
itself has profit more than , since
. Profit increments
of elements are upper bounded by their profits, therefore we can apply
Lemma \ref{lemma:concentration} with  as the fixed
set and profit increments as the weights. By optimality of
 we have that these profit increments are
non-negative hence the required conditions of Lemma
\ref{lemma:concentration} hold and we have  with probability at
least .  Since  we get  with
.

By definition of  the profit of  is at least
. With  and  we can again
apply Lemma \ref{lemma:concentration} to show that . Hence we can set  and this completes the proof
of the claim.
\end{proof}

With the demonstration that the above three claims hold, our proof is complete.
\end{proof}

To conclude the analysis, we show that if the online algorithms for
 and  have a
competitive ratio of , then we obtain an 
approximation to . We therefore get the
following theorem.

\begin{theorem}
\label{thm:online-matroid}
If there exists an -competitive algorithm for the matroid
secretary problem  where  is a sum-of-values objective,
then Algorithm~\ref{Algorithm:online-matroid} achieves a competitive ratio of
 for the problem .
\end{theorem}

Before we proceed to prove Theorem~\ref{thm:online-matroid}, we show
that in steps \ref{step:online-start} to \ref{step:online-end} the
algorithm obtains a good approximation to .

\begin{lemma}
  \label{lem:secondstage}
Suppose that there is an -competitive algorithm for
 where  is any sum-of-values objective. For a fixed set
 and threshold , satisfying , we have , where the expectation is over all
permutations  of .
\end{lemma}

\begin{proof}
 The threshold  is either equal to or strictly greater than
 . In the former case  must have been in the sample
 set  and hence . By Lemma \ref{lem:rhoc-indep} we show that  and  are
 -bounded and hence -bounded. On the other hand if  we can again apply Lemma \ref{lem:rhoc-indep} and get that
  and  are -bounded.
 
Hence by Proposition \ref{lem:profit-lb} we get the inequalities

and .
  
  By applying the -competitive matroid secretary algorithm with
  objective  (Step 11 of Algorithm
  \ref{Algorithm:online-matroid} ) we get
  
\noindent where the second inequality follows from the optimality of . 

Next we bound . Let
 be the largest feasible subset contained in . The
fact that the underlying algorithm is -competitive implies
.

Note that, as observed above, . Since , by definition of  we get that . So, for  we have
  

\noindent Since  we get the last inequality by applying Lemma
\ref{lem:monotone}.

  The conclusion of the lemma now follows from the decomposition
  .
  \end{proof}

\begin{proof}[Proof of Theorem~\ref{thm:online-matroid}]
With probability  we apply the standard secretary algorithm which is 
-competitive. If an element has profit more than
, in expectation we get a profit of
 times the optimal.

We have , for a fixed constant . Also, the events  and  
depend only on what elements are in  and , 
and not on their ordering in the stream. So conditioned on 
and , the remaining stream is still a uniformly random
permutation of . Therefore, if no element has profit more than
 we can apply the second inequality
of Lemma \ref{lem:main} and Lemma \ref{lem:secondstage} along with the
fact that we output  and  with probability 
each,  to show that


Overall, we have that  Since all the involved
parameters are fixed constants we get the desired result.
\end{proof}
















































