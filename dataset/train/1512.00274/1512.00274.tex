\documentclass[9pt,conference]{IEEEtran} \usepackage{times}
   \usepackage{mathptm}
   \usepackage{caption}
   \usepackage{epsfig}
    \usepackage{subfigure}
   \usepackage{bbold}
   \usepackage{color}
   \usepackage{latexsym}
   \usepackage{pslatex}
   \usepackage{algorithmic}
   \usepackage{algorithm}
   \usepackage{graphicx}
  \usepackage{multirow}

 


\newtheorem{definition}{Definition}
\newtheorem{lemma}{Lemma}
\newtheorem{theorem}{Theorem}
\newtheorem{prop}{Property}

\newcommand{\procname}[1]{\textsc{#1} }

\begin{document}

\title{On Constructing Secure and Hardware-Efficient Invertible Mappings}
\author{
Elena~Dubrova  \\
Royal Institute of Technology (KTH), Stockholm, Sweden \\ 
dubrova@kth.se
}

\maketitle
 
\begin{abstract}
Our society becomes increasingly dependent on wireless communications. The tremendous growth in the number and type of wirelessly connected devices in a combination with the dropping cost for performing cyberattacks create new challenges for assuring security of services and applications provided by the next generation of wireless communication networks. The situation is complicated even further by the fact that many end-point Internet of Things (IoT) devices have very limited resources for implementing security functionality. This paper addresses one of the aspects of this important, many-faceted problem - the design of hardware-efficient cryptographic primitives suitable for the protection of resource-constrained IoT devices. We focus on cryptographic primitives based on the invertible mappings of type .
In order to check if a given mapping is invertible or not, we generally need an exponential in  number of steps.
In this paper, we derive a sufficient condition for invertibility which can be checked in  time, where  is the size of representation of the largest function in the mapping. Our results can be used for constructing cryptographically secure invertible mappings which can be efficiently implemented in hardware.
\end{abstract}

\begin{keywords} 
Invertible mapping, permutation, Boolean function, NLFSR
\end{keywords}


\section{Introduction}

This paper addresses the problem of constructing hardware-efficient cryptographic primitives 
suitable for assuring trustworthiness of resource-constrained devices used in services and applications provided by the next generation of wireless communication networks.
 

The importance of improving security of wireless networks for our society is hard to overestimate. 
In 2014, the annual loss to the global economy from cybercrimes was more than \x \rightarrow f(x)\{0,1,\ldots,2^n-1\} \rightarrow \{0,1,\ldots,2^n-1\}2^nn\{0,1\}^n \rightarrow \{0,1\}^nx_i \in \{0,1\}ixf_i: \{0,1\}^n \rightarrow \{0,1\}if(x)i \in \{0,1,\ldots,n-1\}+2^nnO(n^2 N)N\oplus\cdot\overline{x}x\overline{x} = 1 \oplus xf: \{0,1\}^{n} \rightarrow \{0,1\}GF(2)c_{i} \in \{0,1\}\cdot\sum(i_{0} i_{1} \ldots i_{n-1})ix^{i_{j}}_{j}i_{j}x_{j}x^{i_{j}}_{j} = x_{j}i_{j} = 1x^{i_{j}}_{j} = 1j \in \{0,1, \ldots, n-1\}f: \{0,1\}^n \rightarrow \{0,1\}f|_{x_j=k} = f(x_0, \ldots, x_{j-1}, k, x_{j+1}, \ldots, x_{n-1})k \in \{0,1\}x \rightarrow f(x)f(x) = f(y)x = ylrfx \rightarrow f(x)\{0,1,\ldots,2^n-1\} \rightarrow \{0,1,\ldots,2^n-1\}f_if(x)f_ig_ih_iiNf_ig_ih_if_ij \in \{0,1,\ldots,n-1\}g_ix_jijj > if_2x_1,x_2x_3f(x)\{0,1,\ldots,2^n-1\} \rightarrow \{0,1,\ldots,2^n-1\}f_{i-1}(x)ix_0, x_1, \ldots, x_{i-1}f_ig_if_ij > ipp2^n2^nn > 2a_1(a_2 + a_4 + \ldots)(a_3 + a_5 + \ldots)\circ  \in \{+,-,\oplus\}ijj > innnn-1nfx_iii \in \{0,1,\ldots,n-1\}nf(x_0,x_1,x_2,x_3) = x_0 \oplus x_3 \oplus x_1 x_2 \oplus x_2 x_3nnn-1fnx = (x_0, x_1, \ldots, x_{n-1})y = (\overline{x}_0, x_1, \ldots, x_{n-1})f(x)f(y)g(x_1, \ldots, x_{n-1})x_0g(x_1, \ldots, x_{n-1})xyx_0xyf(x) \not= f(y)xyf\{0,1\}^n \rightarrow \{0,1\}^nn2^n2^n\{0,1\}^n \rightarrow \{0,1\}^nx, y \in \{0,1\}^nx \not= yf_i(x) \not= f_i(y)i \in \{0,1,\ldots,n-1\}nO(n^2 N)x_i \in dep(f)fx_i \not\in dep(g_i)f\Phi(f)fi \in \{0,1,\ldots,n-1\}f_ix_{j_i} \in \Phi(f_i)f_{0}, f_{1}, \ldots, f_{n-1}f_{i_0}, f_{i_1}, \ldots, f_{i_{n-1}}\cupx_{j_k} \in \Phi(f_{i_k})k \in \{0,1,\ldots,n-1\}a = (a_{i_0},a_{i_1},\ldots,a_{i_{n-1}}) \in \{0,1\}^na' = (a'_{i_0},a'_{i_1},\ldots,a'_{i_{n-1}}) \in \{0,1\}^na \not= a'{a}_{i_1} \rightarrow a_{j_1} \oplus g_{i_0}{a'}_{i_0} \rightarrow a'_{j_0} \oplus g_{i_0}a_{j_0} \oplus g_{i_0} = a'_{j_0} \oplus g_{i_0}a_{j_0} = a'_{j_0}a_{i_1} \rightarrow a_{j_1} \oplus g_{i_1}(a_{j_0})a'_{i_1} \rightarrow a'_{j_1} \oplus g_{i_1}(a'_{j_0})a_{j_1} \oplus g_{i_1}(a_{j_0}) = a'_{j_1} \oplus g_{i_1}(a'_{j_0})a_{j_0} = a'_{j_0}a_{j_1} = a'_{j_1}\ldotsa_{i_{n-1}} \rightarrow a_{j_{n-1}} \oplus g_{i_{n-1}}(a_{j_0},a_{j_1}\ldots,a_{j_{n-2}})a'_{i_{n-1}} \rightarrow a'_{j_{n-1}} \oplus g_{i_{n-1}}(a'_{j_0},a'_{j_1}\ldots,a'_{j_{n-2}})a_{j_{n-1}} \oplus g_{i_{n-1}}(a_{j_0},a_{j_1}\ldots,a_{j_{n-2}}) = a'_{j_{n-1}} \oplus g_{i_{n-1}}(a'_{j_0},a'_{j_1}\ldots,a'_{j_{n-2}})a_{j_0} = a'_{j_0}a_{j_1} = a'_{j_1}\ldotsa_{j_{n-2}} = a'_{j_{n-2}}a_{j_{n-1}} = a'_{j_{n-1}}a = a'\Boxn!n!(0,1,2,3) \rightarrow (1,2,3,0)g_i(x_0,x_2\ldots,x_{i-1})g_ig_ig_iNx_1 \oplus x_1 x_2\Phi = \emptysetin-1f_i = x_jf_i = x_j \oplus 1j \in \{0,1,\ldots,n-1\}x_j\Phif_iin-1f_if_ix_k \in \Phi~ \forall x_k \in dep(g_i)x_j\Phif_iin-1f_i\Phif_inf_if_i = x_jf_i = x_j \oplus 1j \in \{0,1,\ldots,n-1\}x_j\Phif_iO(1)O(n)f_if_ix_k \in dep(g_i)\Phix_j\Phif_iO(1)O(N)O(n N)nO(n^2 N)f_iO(n)O(n^2 N)n< 252^n-12^{20}-12^{20}-1f_if_{16} = x_{17} \oplus x_4 x_{11}f_{13} = x_{14} \oplus x_{13}f_4 = x_5 \oplus x_1f_{19} = x_{0} \oplus x_{16}f_{15} = x_{16} \oplus x_{3} \oplus x_1 x_{15}f_i = x_{(i+1)~mod~20}i \in \{0,1,\ldots,19\}\{0,1,\ldots,2^n-1\} \rightarrow \{0,1,\ldots,2^n-1\}O(n^2 N)N2^w2^n-1$ using cross-join pairs,'' {\em IEEE Transactions on Information
  Theory}, vol.~1, no.~59, pp.~703--709, 2013.

\bibitem{Du12l}
E.~Dubrova, ``A list of maximum-period {NLFSR}s.'' Cryptology ePrint Archive,
  Report 2012/166, 2012.
\newblock http://eprint.iacr.org/2012/166.

\end{thebibliography}



\end{document}
