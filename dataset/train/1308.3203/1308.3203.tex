\documentclass[a4paper,11pt]{llncs}

\usepackage{ucs}
\usepackage[utf8x]{inputenc}
\usepackage[T1]{fontenc}

\usepackage{amssymb}
\usepackage{amsmath}
\usepackage{amsfonts}
\usepackage{color,proof}
\usepackage{marginnote}
\usepackage{proof}
\usepackage{listings}

\usepackage{pdflscape}

\title{Detecting Data Races on OpenCL Kernels with Symbolic Execution\thanks{This work was supported by the EU FP7 project CARP (project number 287767).}}
\author{Dino Distefano \and Jeremy Dubreil}
\institute{Monoidics Ltd}
\date{}

\clearpage{}\usepackage{xspace}
\usepackage{tikz}

\definecolor{mygreen}{rgb}{0,0.6,0}

\usetikzlibrary{matrix,arrows,automata,positioning}

\tikzset{
    initial text={},
    every state/.style={minimum size=.4cm,draw=blue!50,fill=blue!20},
    secret/.style={circle, draw},
    join/.style={minimum size=.4cm,draw=red!50,very thick,fill=red!20,rectangle},
    whitebox/.style={rectangle, draw=black, minimum width=1cm, minimum height=1cm, rectangle, rounded corners, thick},
    mutebox/.style={rectangle, draw=white, minimum width=1cm, minimum height=1cm, rectangle, rounded corners, thick},
    badbox/.style={rectangle, minimum width=1.4cm, minimum height=1cm, draw=red!50,fill=red!20,very thick},
    goodbox/.style={rectangle, minimum width=1.4cm, minimum height=1cm, draw=mygreen!50,fill=mygreen!20,very thick},
    formula/.style={rectangle, minimum width=1.4cm, minimum height=0.8cm, draw=gray},
}


\newcommand{\tid}{\texttt{tid}}
\newcommand{\size}{\texttt{size}\xspace}
\newcommand{\start}{\texttt{start}}
\newcommand{\exit}{\texttt{exit}}
\newcommand{\new}{\texttt{new}}
\newcommand{\barrier}{\texttt{barrier}\xspace}
\newcommand{\assume}{\texttt{assume}\xspace}
\newcommand{\gsize}{\texttt{gsize}\xspace}
\newcommand{\error}{\top}
\newcommand{\wait}{\boxdot}


\newcommand{\lvars}{\mathit{LVar}}
\newcommand{\locs}{\mathit{Locs}}
\newcommand{\vals}{\mathit{Vals}}
\newcommand{\heaps}{\mathit{Heaps}}
\newcommand{\stacks}{\mathit{Stacks}}
\newcommand{\states}{\mathit{States}}
\newcommand{\sstates}{\mathit{SStates}}
\newcommand{\tstates}{\mathit{TStates}}
\newcommand{\gstates}{\mathit{GStates}}
\newcommand{\vars}{\mathit{PVar}}
\newcommand{\gvars}{\mathit{GVar}}
\newcommand{\defeq}{\stackrel{\mbox{\tiny \it def}}{=}}
\newcommand{\fin}{\mathsf{fin}}
\newcommand{\psto}{{\,{\mapsto}\,}}
\newcommand{\lseg}{\mathsf{ls}}
\newcommand{\SH}{\mathsf{SH}}
\newcommand{\sem}[1]{[ \! | #1 | \! ]}
\newcommand{\trans}[1]{\stackrel{#1}{\to}}
\newcommand{\triple}[3]{\{ #1 \} \; #2 \; \{ #3 \}}
\clearpage{}

\newcommand{\dinocomment}[1]{
\begin{center}
\fbox{
\begin{minipage}{4.5in}
{\bf Dino's comment:} {\it #1}
\end{minipage}}
\end{center}}

\newcommand{\jeremycomment}[1]{
\begin{center}
\fbox{
\begin{minipage}{4.5in}
{\bf Jeremy's comment:} {\it #1}
\end{minipage}}
\end{center}}

\newcounter{note_number}
\newcommand{\marginNote}[2]{
	\refstepcounter{note_number}\textcolor{#1}{(\arabic{note_number})}\marginnote{\textcolor{#1}{(\arabic{note_number})} {{\scriptsize #2}}}}

\newcommand{\Jeremy}[1]{\marginNote{red}{#1}}

\newcommand{\Jeremyadd}[1]{\textcolor{red}{
	\begin{quote}#1\end{quote}
	}}

\begin{document}
\maketitle


\begin{abstract}
We present an automatic analysis technique for checking data races on OpenCL kernels. 
Our method defines symbolic execution techniques based on separation logic with suitable abstractions to automatically detect {\em non-benign} racy behaviours on kernels.

\end{abstract}

\section{Introduction}
Graphics Processing Units (GPU) are becoming a popular way to accelerate general-purpose applications.
OpenCL (Open Computing Language) is a cross-platform parallel programming language, developed by the KRONOS group\footnote{\tt http://www.khronos.org}, that is becoming popular for programming GPUs.
OpenCL is a rather low-level language and developing correct parallel applications is challenging and error-prone.


OpenCL programs are composed by two parts: the first one 
called the \textit{host} is
executing on the CPU;  the second part  is a set of functions, called \textit{kernels}, executing on specialised devices like GPUs.

In this paper we develop techniques to automatically analyze OpenCL kernels and discover possible  data races.
In doing so, we focus on distinguishing between {\em benign} and {\em non-benign} races. 
Here, for benign race we intend the possibility for more than one thread to write in the same memory cell the same value. Technically this is a race, however 
it will not result in a non-determinist behaviour of the program as 
the threads write the same value. 
OpenCL programmers explicitly use benign data races as they are harmless. 
An analysis that would flag data races ragardless whether benign or not would give way too many false alarms and would hardly be used in practice.
  
Our method is based on symbolic execution~\cite{BCO05} for computing an overapproximation of the possible behaviours of the program.
More specifically, we develop a new symbolic execution based on separation logic~\cite{REYNOLDS02} tailored to the specifics of OpenCL.
This is challenging task as it requires a special way to deal with the SIMD (Single Instruction Multiple Data) model used by GPUs.





In summary this paper makes the following contribution:
\begin{itemize}
\item We define an operational semantics for SIMD model able to detect non-benign data race.
\item We define a extension of separation logic suitable for the analysis of the SIMD model.
\item We define a new symbolic execution based on separation logic for the SIMD model able to detect non-benign data race.
\end{itemize}
The paper is organised as follows.
In Section~\ref{sec:background} we give a summary of the problems related with kernel and data race.
In Section~\ref{sec:progr-language} we define a minimal OpenCL language.
In Section~\ref{sec:concrete-semantics} we describe a concrete semantics for our language.
In Section~\ref{sec:symbolic-heaps} we define a subset of separation logic for SIMD.
In Section~\ref{sec:symbolic-execution} we describe the symbolic execution.
In Section~\ref{sec:related-work} we discuss other work in the literature. 
Section~\ref{sec:conclusion} concludes the paper.

\section{OpenCL Kernels}
\label{sec:background}
OpenCL Kernels are not full program, but functions embedded in the main program (called the host), and that will be executed on the GPU.
Kernels are executing following the Single Instruction Multiple Data (SIMD) model.
In this model when a kernel is launched by the host, a number of threads (also called {\em work-items}) are executing the same code but on different data.
Threads are grouped in work-groups and both threads and work-groups have a unique identifier that can be referred from the kernel. 
Threads can also query the size of the work-group they belong to, i.e. the number of threads executing the same code in parallel.

\begin{example}
The program in Figure~\ref{fig:kernel} is an example of OpenCL kernel. The special variable {\tt tid} stands for {\em thread id}, the unique identifier of the thread.
Two distinguished threads will only differ from the runtime value of their {\em thread id} which will allow them to manipulate different data.
The {\tt barrier} command synchronises the memory for all the threads executing the kernel. When a thread executes {\tt barrier} it waits until all the other threads have done the same. At that point  the computation resumes.
\end{example}\begin{figure}[t]
   \centering
\begin{verbatim}
  kernel( int A[], int B[], int R[]) { 
   
      R[tid] = A[tid-1] + B[tid+1];
      barrier;
      R[tid] = 2 * R[tid + 1];
  }
\end{verbatim}
 \caption{Example kernel}
 \label{fig:kernel}
\end{figure}



\section{Kernel Programming Language}
\label{sec:progr-language}
We consider the following language for kernels.
It is presented as a control flow
graph consisting of instruction nodes with edges from the nodes to their successors.
Let  be a set of program variables.
The set  of commands is defined by the following grammar.

An expression can be a constants , a variable  (), the special variable  for the thread id, the size of an array, or an expression constructed with n-ary operations (e.g. , etc.).
Boolean expression  are standard.
The commands have assignment to a variable or to an element of an array.
The special command  is used in kernels to syncronise threads.
Its informal semantics is the following: when a thread reaches  waits until all threads have reached the same instruction . At that point, threads resume their computation.\\

\newcommand{\Node}{\mathsf{CFGNode}}
\newcommand{\locals}{\mathit{Locals}}
\newcommand{\args}{\mathit{Args}}

\noindent
A kernel is a tuple: 
where the control flow is given as a relation . The set of control flow graph nodes are given by  with . The kernel begins at the
unique node  and terminate at the unique  node.
A kernel has a unique identifier  and a set of argument   passed by the host program and a set of private variables .



\section{Concrete Semantics}
\label{sec:concrete-semantics}
Let  be a set of logical
variables, disjoint from program variables , to be used in the
assertion language. Let  be a set of global shared variables. 
Let  be a countable set of
locations, and let  be a set of values that includes
, i.e., . The storage model is given by:

where  denotes a finite partial map. 
  is the set of  {\em shared states} and they   
  are modelled by a pair  of a stack and a heap.
 is the set of 
{\em thread states}. They are modelled by a tuple  of a stack, a heap and a thread
identifier. We write  for all the thread states of thread , i.e., .
 is the set of global states (i.e., all thread local states and global states) of the system. A global state has the form:

and includes information for the all the local state of the threads and the shared state of the system.

The concrete semantics is now defined with two transition relations, one reflecting the local computation of a thread (thread-local semantics) and the other describing the global computation of the system (global semantics).

\paragraph{Thread-local semantics.}
The concrete operational semantics of the execution of a thread is defined by a transition relation: 

Hence, configurations of our semantics are of the kind: 
 describing the effect of a thread executing the command  in the private state  and shared global state .
The result of a step of a transition can be either a modified state  or 
one of the special states , or ,  or .
The special state  denotes a runtime error (e.g., a race, an access out of bounds) or that the execution is aborted.  denotes that a path on a computation does not have continuation.
The special state  indicated that the thread has suspended on the current state .

In the following we indicate  and  for the stack and the heap part of . Similarly  and  denote the stack and the heap of , respectively. For a function  we indicate with  the function update where  now maps to . More precisely:



The transition relation is defined by the following rules:











The barrier rule expresses  that a thread executing a barrier suspend its computation in the current local and shared state (indicated by ).

\paragraph{Global Semantics.}
The global semantics is defined with a transition relations

In the global semantics we have the following rules. The first two rules give the interleaving of kernels operations. If a thread makes a local step, this is observed in the global 
execution of the system. 

The second rule records the special case when a thread
waits at a barrier


The next rule describes the case where all the threads have reached the barrier and are in a waiting state. If the shared state  is the same for every threads, then no races have occurred. The rule removes the waiting state allowing threads to resume their computation.

Finally, the last rule detects if a race has occurred. This is identified when there exists a thread  that is waiting in a shared state  which is different from the global shared state . The global shared state is  obtained when the last thread in the particular schedule considered entered in the waiting state.  In the case the system has a race and  the global computation reaches the error state.

Notice that these rules describe any scheduling for threads. Moreover, notice that the race rule reaches the error state only for {\em non-benign} races. 
In case of benign races (i.e., more than one thread writing the same variable or memory location with the same value) the computation continues normally. 
\begin{definition}
A command  has a {\em race} if there exists a thread state  and a shared state  such that .
\end{definition}
According to our operational semantics, a program is racy if it presents non-deterministic behaviour on the shared state when the threads syncronise by means of a barrier. This includes also the last implicit barrier at the end of the kernel.

\section{Symbolic Heaps}
\label{sec:symbolic-heaps}
Symbolically, we use formulae in separation logic to represent set of states
during the computation of the kernel.
The subset of formulae of separation logic we use in this paper is an extension of symbolic heaps~\cite{DistefanoOHearnYang06}. We also make specific considerations about arrays as they are fundamental data structures in OpenCL.\\

\noindent
Arrays are encoded in separation logic as follows:
\begin{itemize}
 \item we denote by  the value at address 
 \item we denote by  an array segment, that is:
 	
\end{itemize}

\noindent
Symbolic heaps are defined by the following grammar:

Primed variables are implicitly existentially quantified.
The other expression in  are standard.
The points-to predicate  is  standard and denotes an allocated cell at address  with content .
The predicate  is used to encode arrays. Its  meaning is  that  points-to an array segment with index ranging from  to  and values defined by the function expression . 
The latter, in the simple case, is a function  that given an index  returns the value of the array at that index\footnote{Note that  may depend on }.
Alternatively, function expressions  can be defined by successions of updates using the constructor :

Therefore  describe the content of the array to be  for the subset of index which make  true otherwise .
This expression allows us to describe updates of the array done simultaneously by several threads\footnote{Note that we do not check here if such updates introduces data-races. This will be done in section~\ref{sec:symbolic-execution}}.\\

\noindent
In the sequel, we will use the following notation:


when an array contains  in all its elements.

\paragraph{Semantics in Thread state. }
The semantics of the fragment of separation logic we use here is defined on thread states. The definition is divided in three parts. We first give the semantics  for the pure part of the fragment. 

We then define the semantics  of functions defining the content of arrays. It is defined in terms of  and the semantics of pure expressions  as follows:
2ex]
\sem{\lambda j . \eta(\Pi, F_1, F_2)}_{s,h,i} & = & \left\{\begin{array}{ll}
	\sem{F_1(j)}_{s,h,i} & \mbox{ if } 
	\\
	\sem{F_2(j)}_{s,h,i} & \mbox{ otherwise}
	\\
	\end{array}
	\right.
\end{array}
\begin{array}{lclll}
s,h,i & \models_S & E_1 \psto E_2 & \mbox{ iff } &  dom(h)=\{ s(E_1)\} \mbox{ and } h(s(E_1))=s(E_2)\\ 
s,h,i & \models & E_1 \psto A[n, m | F]  & \mbox{ iff } & dom(h)=\{s(E_1)+n,\dots,s(E_1)+m \} 
\\ & & & & \mbox{and } n\leq j \leq m: h(s(E_1)+j)=\sem{F(j)}_{s,h,i} \\
s,h,i & \models_S & emp & \mbox{ iff } & dom(h)=\emptyset\\ 
s,h,i & \models_S & \Sigma_1 * \Sigma_2 & \mbox{ iff } & \exists h_1,h_2{:}  \ dom(h_1) \cap dom(h_2) = \emptyset, h_1 \cup h_2 = h \mbox{ and} \\
& &   & &  s,h_1,i \models_S \Sigma_1 \ \mbox{ and } \ s,h_2,i \models_S \Sigma_2 
\end{array}
s,h,i  \models_S  \exists \vec {x'} .(\Pi \wedge \Sigma) & \mbox{ iff } & \exists \vec v{:} \ 
s,h,i \models_P \Pi[\vec v/\vec {x'}]  \mbox{ and } s,h,i \models_S \Sigma[\vec v / \vec{x'}]
2ex] 

\2ex]
 
\2ex]
\infer{\assume(b),~ \Pi \wedge \Sigma \implies (\Pi \wedge b) \wedge \Sigma}{(\Pi \wedge b) \wedge \Sigma \nvdash false}	
\2ex]
\infer{\texttt{assert}(b),~ \Pi \wedge \Sigma \implies \Pi \wedge \Sigma}{ \Pi \wedge \Sigma \vdash b}	
\
	\Pi \wedge \Sigma \vdash \Pi' \wedge \Sigma'
 \begin{array}{ccc}
	\infer{C,~ \mathcal{H} \Longrightarrow_s  {\cal H'}}{ {\cal H'}=\{ H' \mid \exists H \in \mathcal{H},~ C, H \implies H' \}  \qquad \bot \notin {\cal H'}}
	& \qquad &
		\infer{C,~ \mathcal{H} \Longrightarrow_s  \bot}{   \exists H \in \mathcal{H}: ~ C, H \implies \bot }
\end{array}

\begin{array}{l} 
compare((e_1, H_1), (e_2, H_2))
\ 
 tries to prove that the expressions  and  denote the same value in both symbolic heaps.
 tries to prove that expressions  and  denotes different values. These functions return  when it cannot be stated from the formula that the two expressions  and  denote the same value () or different ().

\begin{definition}
\label{def:no_benign_race}
Let  be a symbolic state. We say that   is (data) {\em race-free}, 
 written , when:

\end{definition}
This definition states that taking any pair of symbolic heaps ,  in  instantiated to two arbitrary distinct instances of  modelling two different threads, there is no data race whenever the shared state has the same value.
More precisely, we need to check that for all the memory cells ,  in the shared state for which we cannot prove that they represent different addresses, we should be able to prove that the values they point to are the same. 
This definition reflects our intuitive notion of benign race. We flag a potential race only if we cannot prove that the share state of all threads at barrier time is deterministic.
Notice that when checking for disjointness we require that we should be able to prove  in both  and  that the left-hand side of the points-to  and  are different. Otherwise, we require that we can prove in both heaps that  and  are equal. Failing that our definition should declare that there might be a potential race.


Based on this, we define the abstract rule for \barrier as follows:

Notice that in the reading of the precondition of the second  rule should be intended to mean when is it not possible to prove .

\subsection{Discussion}
\label{discussion}
Although the symbolic execution described in this section will be able to detect interesting data races on most concrete examples, the analysis is not sound. Indeed, there can be cases where the analysis will fail to detect branches depending on the values of arrays. For example, the code sample of Figure~\ref{fig:counterexample}, the analysis will not be able to detect the race of the global variable .
\begin{figure}[t]
   \centering
\begin{verbatim}
  kernel( int A[]) { 
    g = 0
    A[tid] = 0
    barrier
    A[tid] = 1
    if A[tid + 1] = 0 then g = 1
    barrier
  }
\end{verbatim}
 \caption{Counter example}
 \label{fig:counterexample}
\end{figure}
Indeed, g can in practice have the value 0 or the value 1. But, according to the rules for the symbolic execution, the array  will get the value . Then, the symbolic execution will consider the test \texttt{A[tid + 1] == 0} as always false, and forget the branch where  gets updated to  and miss the data race. It seems that this issue can be fixed by just keeping the same rules, but considering the interleaving product of two threads with two distinct thread id  and  such that . The predicate {\sf NoRace} does not need to be updated as these two arbitrary threads  and  will be the same than the ones present in Definition~\ref{def:no_benign_race}. 
We intend to investigate the possible extensions to a sound approach in the future.
The soundness should be proved against the concrete semantics defined in Section~\ref{sec:concrete-semantics}.



\section{Related Works}
\label{sec:related-work}

Recently several approaches to formal analysis of GPU kernels have
been
proposed~\cite{PUG,BettsCDQT_OOPSLA2012,CollingbourneDKQ_ESOP2013,DBLP:conf/ppopp/LiLSGGR12,KLEECL,PLDI2012,BYTECODE}.

In~\cite{BettsCDQT_OOPSLA2012}, the authors present an approach to verify intra-group data races and barrier divergence. Their approach is based on modelling the OpenCL kernels semantics using predicated execution and a synchronous, delayed visibility semantics to deal with the modifications of shared memory. With this delayed visibility semantics, similar to transactional memory, every thread updates a local copy of shared variables, and the consistency on these modifications are verified at the barrier level. At these synchronisation points, it must be true that no shared variable modified by one thread has been read or modified by another thread in the delayed semantics. Practically, this theory is implemented by translating OpenCL programs into Boogie~\cite{Boogie} and use the Z3~\cite{Z3} theorem prover to prove consistency between delayed reads on writes. Instead, in our approach, we consider a more abstract definition of race freedom based on the absence of non-determinism at the barrier level. We also use a crafted logic based on separation logic to define the abstract states of our symbolic execution. Unlike approaches based on theorem prover, our approach aims at being fully automatic.  The technique of~\cite{PUG} shares similarities with~\cite{BettsCDQT_OOPSLA2012}: it is also based on generating constraints to be solved by a theorem prover.


Another line of work that is worth comparing with our approach is the tool GKLEE~\cite{DBLP:conf/ppopp/LiLSGGR12}.
In this paper, the author present a technique based on the existing theory behind the tool KLEE~\cite{klee} which uses static analysis techniques to generate test cases, and evaluate the test coverage provided by a set of test cases.
The approach of~\cite{DBLP:conf/ppopp/LiLSGGR12} is based on defining a symbolic virtual machine that will accurately model the behaviours of kernel threads. Based on this, the tool generates sets of tests to force kernels to execute and detect errors relevant for GPU programming such as data races, but also to outline possible performance issues related to memory access. Unlike~\cite{BettsCDQT_OOPSLA2012}, the approach of~\cite{DBLP:conf/ppopp/LiLSGGR12} does not aim at providing soundness on data races freedom, but provides instead a level of confidence with the concept of test coverage.  The KLEE-CL tool~\cite{KLEECL} is a similar method for analysing GPU kernels, also based on KLEE.


\section{Conclusions and Future Work}
\label{sec:conclusion}
We have introduced a novel automatic approach for the analysis of OpenCL Kernels based on separation logic.
Our method uses a tailored symbolic execution able to detect non-benign races.

In the future we plan to extend our technique from a race detector to a race-free verifier. One avenue we intend to try  discussed in Section~\ref{discussion}.
Moreover, in the future we plan  to  automatically synthesise loop invariants. This requires the design of  new sophisticated finite abstractions for arrays in OpenCL. We will use techniques from abstract interpretation~\cite{CousotCousotPOPL77} 

Another avenue we intend to investigate is the possibility to automatically generate  preconditions such that kernels can be run in a data-race free way. 

An implementation is left for future work.


\bibliographystyle{plain}
\bibliography{biblio}

\end{document}
