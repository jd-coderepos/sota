\documentclass[conference]{IEEEtran}
\usepackage{graphicx}
\usepackage{algorithm2e}
\usepackage{algorithmic}
\usepackage{amsmath}
\usepackage{amsfonts}
\usepackage{amssymb}
\usepackage{wasysym}
\usepackage{multirow}
\setlength{\paperwidth}{215.9mm}
\setlength{\hoffset}{-9.7mm}
\setlength{\oddsidemargin}{0mm}
\setlength{\textwidth}{184.3mm}
\setlength{\columnsep}{6.3mm}
\setlength{\marginparsep}{0mm}
\setlength{\marginparwidth}{0mm}

\setlength{\paperheight}{279.4mm}
\setlength{\voffset}{-7.4mm}
\setlength{\topmargin}{0mm}
\setlength{\headheight}{0mm}
\setlength{\headsep}{0mm}
\setlength{\topskip}{0mm}
\setlength{\textheight}{235.2mm}
\setlength{\footskip}{12.4mm}
\setlength{\parindent}{1pc}
\newtheorem{thrm}{Theorem}
\newtheorem{proof}{Proof}
\begin{document}
\title{Secret Sharing Homomorphism and Secure E-voting}
\author{\IEEEauthorblockN{Binu V.P}
\IEEEauthorblockA{Department of Computer Applications\\
Cochin University\\
Kochi, India\\
binuvp@gmail.com}
\and
\IEEEauthorblockN{Divya G Nair}
\IEEEauthorblockA{Department of Computer Science \\
	Cochin University\\
	Kochi, India\\
	divyagnr@gmail.com}
\and
\IEEEauthorblockN{Sreekumar A}
\IEEEauthorblockA{Department of Computer Applications \\
Cochin University\\
Kochi, India\\
sreekumar@cusat.ac.in}
}
\maketitle
\begin{abstract}
Secure E-voting is a challenging protocol.Several approaches based on homomorphic crypto systems, mix-nets  blind signatures are proposed in the literature.But most of them need complicated homomorphic encryption which involves complicated encryption decryption process and key management which is not efficient.In this paper we propose a secure and efficient E-voting scheme based on secret sharing homomorphism.Here E-voting is viewed as special case of multi party computation where several voters jointly compute the result without revealing his vote.Secret sharing schemes are good alternative for secure multi party computation and are computationally efficient and secure compared with the cryptographic techniques.It is the first proposal, which  make use of the additive homomorphic property of the Shamir's secret sharing scheme and the encoding-decoding of votes to obtain the individual votes obtained by each candidates apart from the election result.We have achieved  integrity and privacy while keeping the efficiency of the system.
\end{abstract}
Keywords:E-voting, homomorphism,multi-party computation,secret sharing,individual votes
\section{Introduction}
Voting is a distributed decision making process involving several people.Each participant called the voter casts a vote and the computations are performed  on the vote casts by different voters to select the preferred item.Voting can be modeled as a secure multi party computation system, since multiple parties submit input and obtain the result without knowing any details of other inputs.

The  process involved in traditional election is quite tedious, time and resource consuming. To overcome these difficulties E-voting system  is introduced.The evolving new technologies made e-voting practical.But the research in this direction has to go a long way.The reliability and security are the major challenge.E-voting provides a lot of benefits compared with traditional voting.It avoids the requirement of geographical proximity of users.The cost can be greatly reduced because the resources can be reused. The use of E-voting must satisfy the security requirements such as authentication, voter privacy, confidentiality, integrity, etc.The security flaws make E-voting vulnerable than traditional system.Gritzalis et al \cite{gritzalis2002principles},\cite{gritzalis2003secure} mentioned the requirements of a secure E-voting system.

Confidentiality,Authenticity,Integrity and Verifiability are the major security requirements in E-voting scenario.Confidentiality ensures that nobody knows whom the voter is voted.Authentication is an important process where each voter must be identified as a person he claims to be and he should not be allowed to vote again.Integrity of the votes are also important.The system should ensure that the votes are valid and any modification must be detected.Verifiability means any one can verify at later time that, the voting is properly performed or his vote was properly registered and has been taken into account in the final tally \cite{fujioka1993practical}.

The are several proposal for efficient secret ballot elections based on mix-nets \cite{park1994efficient} \cite{sako1995receipt} \cite{jakobsson1998practical}, homomorphic encryption \cite{cohen1985robust} \cite{sako1994secure} \cite{benaloh1987verifiable} \cite{benaloh1994receipt} \cite{cramery1997secure} and blind signatures \cite{fujioka1993practical} \cite{okamoto1998receipt}.
There are different methods addressing the security and reliability of the E-voting scheme.Most of the approaches are based on cryptography.The major objective is to protect the voters identity from the vote.Secure E-voting using Blind Signature is proposed in \cite{ibrahim2003secure}.RSA \cite{rivest1978method} and Blind signatures are the major cryptographic algorithms involved\cite{camenisch1995blind},\cite{chaum1983blind}.Homomorphic encryption techniques are used in several implementations \cite{peng2005multiplicative}.Several proposal for verifiable secret sharing schemes are also given in \cite{benaloh1987verifiable},
\cite{benaloh1994receipt}.
Several modifications and use of homomorphic encryption and verifiable secret shuffle are mentioned in \cite{lee2000receipt}
\cite{hirt2000efficient} \cite{neff2001verifiable}.
Malkhi et al \cite{malkhi2003voting} in 2003 gave constructions without cryptographic technique which uses secret sharing homomorphism.Iftene \cite{iftene2007general} in 2007 proposed a general secret sharing for E-voting using Chinese remainder theorem . Pailliar's crypto system and its application to voting is proposed by Damagaard et al \cite{damgaard2010generalization} in 2010.Discrete logarithm problem and secret sharing are used by Chen et al \cite{chen2014secure} in 2014.Scheme with enhanced confidentiality and privacy is suggested by Pan et al \cite{pan2014enhanced} in 2014.

Secret sharing and many variations of its form an important primitive in several security protocols and applications.In the proposed method we make of Shamir's \cite{shamir1979} secret sharing techniques and its additive homomorphism property for efficient E-voting and vote tallying.Hence it avoids the complicated encryption decryption process and key management.The secret sharing schemes are originally proposed by Shamir \cite{shamir1979} and Blackley \cite{blakley1979} in 1979.
The motivation was to safeguard cryptographic keys.	The secret keys are stored at several locations as shares and when authorized number of users collaborate together, they can retrieve the secret.This provides both security,reliability and trust.The shares are information theoretically secure and provides no information about the key.The schemes are  threshold schemes where any  number of users can collaborate to recover the secret out of  users.Less than  users cannot obtain any information about the secret.Secret sharing scheme is perfect if less than  shares gives no information about the secret.The secret sharing scheme is ideal if the share size is same as the secret size.Shamir's scheme is perfect and also ideal.It is easy to implement and is based on polynomial interpolation.Blackley's scheme is not perfect.It is based on hyperplane geometry and it is difficult to implement.There are other schemes based on boolean operations \cite{kuri2009xor} \cite{wang2007ssboln} and number theory \cite{mignotte1983} \cite{asmuth1983}.


Properties of polynomials give Shamir's scheme a  homomorphic property.The secret domain and the share domain is same ( integers modulo ).There are other schemes \cite{asmuth1983modular},\cite{kothari1985generalized} which also have  homomorphism, we consider Shamir's scheme for the ease of implementation and also it is perfect and information theoretically secure.
The homomorphism also provides verifiable secret sharing.It is very important in secure multi party computation.The first proposal of verifiable secret sharing was done by Chor et al \cite{chor1985verifiable}.In secret sharing not only the participant but also the dealer may be malicious.So the participant must be able to verify whether the shares are consistent.A set of  shares is  consistent if every subset of  of the  shares defines the same secret.Publicly verifiable secret sharing scheme's are introduced by Stadler in 1996 \cite{stadler1996publicly}.Schoenmakers \cite{schoenmakers1999simple} in 1999 proposed a publicly verifiable secret sharing scheme(PVSS) with applications to E-voting.The scheme is better than \cite{cramery1997secure},\cite{cohen1985robust}. The issue of homomorphic secret sharing for PVSS is also discussed.An efficient PVSS is suitable for the secure implementation of E-voting.

\section{Preliminaries}

The proposed scheme uses Shamir's threshold  secret sharing scheme \cite{shamir1979}, which is having all the required property for efficient implementation of E-voting.The cryptographic techniques used for E-voting makes use of several mathematical assumptions.The use of secret sharing homomorphism avoids these problems and provides perfect secrecy.This section explores Shamir's secret sharing technique and the use of homomorphism property for e-voting.

\subsection{Shamir's Secret Sharing Scheme}

Let , a  secret sharing protocol allows the shares of the secret to be distributed to  participants and any  of them can collaborate to retrieve the secret.Shamir \cite{shamir1979} uses a polynomial based construction for the implementation of   threshold scheme.It is based on the following theorem.

\begin{thrm}
	Let .Also let , where  is prime and  are pairwise distinct.Then there exist  polynomial  of degree 
	with 
\end{thrm}
\begin{proof}
	The polynomial can be obtained from the points  using Lagrange interpolation formula.
	
	
It satisfies .

We can determine number of such polynomials of degree .
 

We can obtain a system of linear equations from 

Lets consider the coefficient matrix 


The coefficient matrix is a Vandermonde matrix \cite{bjorck1970solution}. Since the  are distinct the determinant  is non zero and the coefficient matrix has rank .Thus the kernal of the coefficient matrix has rank .So the number of polynomials is  since each coefficient can take any of the  values.
\end{proof}
\  \\
The secret sharing protocol consist of following phases.\\
\subsubsection{Initialization}
In this phase, the dealer who wants to distribute the secret  must choose a prime number  larger than the secret and also  must be greater than , where  is the number of participants.The dealer then choose different  corresponds to each participants.The  are then published.\\
\subsubsection{Secret Sharing}
The secret  is distributed as shares to the participants securely in this phase.

The dealer chooses  and construct a polynomial  of degree  such that the constant term  represents the secret.


The shares corresponds to each participants are constructed by evaluating the polynomial at corresponding  values..

The shares,   are then distributed to 'th participants
securely .

\subsubsection{Secret Reconstruction}

When  or more participants collaborate together,they can retrieve the secret  by combining the shares.Let  ,be the  shares.Then by using Lagrange Interpolation, the polynomial of degree  can be reconstructed from these  points using the formula 



As per theorem 1 , there is exactly one such polynomial of degree . The participants can obtain the secret  as


It is noted that less than  share holders  get no information about the secret.All constant term are equally likely and is an element in the field.The scheme is information theoretically secure.

\subsection{Secret Sharing Homomorphism}

Secret sharing homomorphism introduced by Benaloh in 1987 \cite{benaloh1987secret}.It is noted that Shamir's scheme is additive homomorphic.He stated that any  of the  agents can determine the super secret and no conspiracy of fewer than  agents can gain any information at all about any of the sub secrets. That is the sum of the shares of different sub secret when added up and then interpolate according to the threshold mentioned to obtain the master secret which is the sum of the sub secrets.He also mentioned the importance of secret sharing homomorphism to E-voting.

Shamir's secret sharing scheme has the (+,+) homomorphism property. For example, assume there are two secrets: ,  and are shared using polynomials  and .If we add the shares  then each of these  can be treated as the shares corresponds to the secret .The polynomial  and .
But each voter choose 1 or 0 ( vote or no vote).The shares are send to  tellers.Any  of them can collaborate to retrieve the result back.

In case of PVSS, two operations are defined.One on the shares  , and the other operation  on the encrypted shares such that for all participants

If the underlying secret sharing scheme is homomorphic then by decrypting the combined encrypted shares, the recovered secret will be equal to .

\section{Proposed Scheme}

The proposed system is a modification of the existing electronic voting scheme's used in India.Currently electronic voting machines are used in polling booth.These machines are costly and also not reliable.We propose an alternative solution for this using Internet and secret sharing homomorphism.This add trust and reliability to the existing voting scheme by incorporating secret sharing based techniques.The secrecy of vote is an important issue.This needs to be addressed with ultimate care. In the current Electronic Voting System, when a vote is casted,the corresponding candidates data base entry is updated and it can be easily tracked.But in the proposed scheme, it is difficult to track the vote because the shares of the votes is added to all the servers. We also add trust to the existing scheme by maintaining more than one server to keep the voting details.We are not considering on-line verification of the authenticity of the voter as in general e-voting scheme.Here we assume that the polling officers in each polling booth has to do it manually using the electoral role.The major components of the proposed e-voting schemes are
\begin{enumerate}
	\item Voting Terminal
	\item Share Generation 
	\item Collection Centers
	\item Result Computation
\end{enumerate}

We have considered the user authentication process which is done manually.The voting takes place in a Polling station.A voter is allowed to vote after his identity is verified. A polling station may contain many voting terminals.The user interface shows a voting panel which contains the list of all contesting candidates and their party symbols. Voting panel is setup and managed by the Chief Election Officers.

The share generation module is responsible for receiving the vote casted by each voter and make shares of it according to the threshold secret sharing scheme.
The shares are generated according to the vote casted for each candidates.Each candidate vote is represented as an encoded binary code.So when a vote is casted, the shares of the decimal value corresponds to the encoded binary vote of each candidate is generated using the Shamir's secret sharing scheme.The number of bits in the encoded binary code corresponds to each candidate vote depends on the number of contesting candidates and also total number of voters.

Let us assume that there are  candidates  and  voters .Then the binary encoding of the vote corresponds to each candidate will consist of  number of bits.Here we consider the fact that all voters may vote to the same candidate.So the number of bits required for the representation of votes for each candidate is equal to the number of bits required to represent the total number of voters which is .

The encoding of the vote corresponds to each contesting candidate is explained below with an example.Let us consider that there are three candidates and seven voters.So the total number of bits of each encoded vote will be nine.The bit pattern corresponds to the vote of each candidate is obtained by setting the corresponding bit  to 1 in the code  and other bits  to 0.For example the code corresponds to the vote of candidate  is .So depending on the vote casted, it is encoded into a decimal code of 1,8 or 64 respectively.This bit wise encoding helps in computing  the total votes obtained by each candidate using the additive homomorphism.
 
The encoded vote is then shared using Shamir's threshold secret sharing scheme.The shares are then send to different Collection Centres(CC).The Collection Centres are responsible for receiving and summing up the shares corresponds to each vote casted.We can set up the threshold and also set number of collection centres required.If there are  collection centres  and a threshold  is set so that we can get back the result from any  collection centres.This provides trust and reliability.Based on the number of collection centres and threshold set up, Shamir's scheme can be used for a threshold  secret sharing.A  degree polynomial  is constructed with constant term representing the encoded vote value in decimal.The other coefficients are chosen randomly from the field , where  is larger than the encoded vote values and the number of participants.The shares are generated by evaluating the polynomial  at  different values .These  values represent different collection centres known only to the Chief Election Officer and are kept secret.These shares are then send securely to the  collection centres.Any  of them can be used for result evaluation and verification.The shares look totally random and the collection centres have no idea regarding which secret (vote) share it is, from the share value.The share size is also same as the secret size and hence it provides information theoretical security.
Once all the collection centres receives the vote share, the voting terminal is intimated to receive the next vote or it is the confirmation that the vote is registered properly.

The collection centres are responsible  for summing up the shares they receive for vote tallying.Here the shares are always valid. They are generated automatically from the terminal program embedded.So there is no need
to check the consistency of the shares received by the collection centres.But proper measures must be taken for the secure and error free communication between the voting terminal and collection centres.Collection centres behave as group of authorized parties.In a real time voting scenario, a single machine can act as a collection centre by maintaining database which contains collection of shares.How ever in this case the collection centre must be trusted.We can maintain a hierarchy of collection  centres for collecting vote shares according to the geographical location which compute the local sum of shares. The local sum is then send to the top level collection centres which further add the sums of shares received from local collection centres.A separate communication module can be incorporated for the efficient and secure communication of shares.The collection centres can also keep the shares received from each polling booth
or booths belongs to the same area as a separate entity for the computation of region wise voting details.The strategy for share maintenance, number of collection centres etc can be determined based on the requirement.The implementation issues also depend on the hierarchal structure used.


The Result Computation module is responsible for computing and declaring the final result.From the sum of shares stored on collection centres, the result can be obtained using Lagrange Interpolation.If there are  collection centres and a  threshold secret secret sharing scheme is used , then any  of the share sum from the collection centres can be used for computing the final result.These  shares can be used to get back a  degree polynomial  and the encoded result will be .The result is then decoded by converting  into binary and then separating the bits corresponds to each candidates.The decimal equivalent of the separated bits represent the total vote obtained by each candidate.Based on this the result can be announced.

It is noted that the result computation cannot be performed by collection centres.They will just keep the share sum and a hash is computed which is then signed by the private keys of the collection centre.During the result computation,it can be verified for the integrity and authenticity.This result declaration module, is managed by  higher officials and only they know the different  values used for each collection centre during the share generation.Any  pairs of this  values and the corresponding share sum, which is the  values, the polynomial interpolation can be done.The result computation can be done with different combination of the share sum from  different collection centres which adds reliability.The trust is maintained by the Shamir's scheme because less than  collection centres cannot get any information about the final result.At least  collection centres have to collate to get back the result.

\section{Proposed Algorithms}
The following algorithm only includes the core functionality required.Additional functionalities can be added depending on the requirement.Suitable hash algorithm and signature algorithm must be chosen for maintaining the integrity and authenticity.When the voting is finished the hash of final share sum of each collection centre  can be computed using SHA(Secure Hash Algorithm) \cite{pub2014draft} and is digitally signed by the previously issued private keys of the collection centre.The election official can verify this for integrity and authenticity.
\RestyleAlgo{boxruled}
\begin{algorithm}
	\begin{scriptsize}
		\KwIn{Vote casted by the voters}
		\KwOut {Sum of the shares of the votes}
		\BlankLine
		Let  denote the number of candidates and 
		 denote number of voters.\\
		Set  equals  bits
		for encoding the votes. \\
		Choose an appropriate field 
	
		\For {each vote   }{
			enc\textunderscore vote = bin\textunderscore decimal(set\textunderscore bit ()) \\
			\tcc{ is set according to the vote casted }
			\tcc{enc\_vote is the encoded vote in decimal }
			Pick  random numbers  from \\
			Construct the polynomial \\ =enc\textunderscore vote \\
		\For {}{
				Generate share  \\
				\tcc{ where  is the  share of  vote}
				Send the share  to  collection centre \\ through a secure communication channel\\
			 }
		\For {each Collection Centre  }{
			 	Sum of shares  }
			}
	\caption{E-Voting}
	\label{Alg:Evt}
	\end{scriptsize}
	\end{algorithm}
	
	\begin{algorithm}
	\begin{scriptsize}
		\KwIn{Share sum of the votes from collection centre}
		\KwOut {Votes obtained by each candidate}
		\BlankLine
		\For {each randomly chosen  Collection Centre }  {
			retrieve  } 
		Interpolate using  and corresponding  values to obtain the  polynomial Q(x)\\
		Obtain the secret value .\\
		Decode  and obtain the binary representation.\\
		Each  bits will represent each candidates vote.\\
		Publish the final results.
		\BlankLine
		\BlankLine
		\caption{Result Computation}
		\label{Alg:result}
	\end{scriptsize}
\end{algorithm}	
\section{E-voting Example}
Let us assume that three people Alice,Bob and Charles are contesting in an election and there are seven voters.So the maximum vote each contestant can get is seven.Three bits are hence required for the representation of votes gained by each candidate and a total of nine bits for the representation of encoded votes corresponds to each candidate.

\\

A sample voting scenario is given below where six voters made the vote out of seven.
\begin{table}[ht]
	\small
		\caption{Example E-voting}
\centering
\begin{tabular}{|c|c|c|c|c|c|} \hline
 \multirow{2}{*}{Vote} & \multirow{2}{*} {Alice} &\multirow{2}{*}  {Bob} & \multirow{2}{*}{Charles} & \multicolumn{2}{c|}{Encoding Vote}\\
\cline{5-6}
    &   & & & Binary & Decimal \\
    \hline
	1 & \checked & 		   &  & 001000000 & 64 \\ 
	\hline 
	 2& 		    & \checked &  & 000001000 & 8 \\ 
	\hline 
	 3&  		& \checked &  &000001000   &  8\\ 
	\hline  
	4&\checked &  		   &  & 001000000 & 64 \\ 
	\hline 
	 5&  		&			&\checked  & 000000001 & 1 \\ 
	\hline
	 6 &\checked  &  &  & 001000000 & 64 \\ 
	\hline 
\end{tabular} 
\end{table}

The votes are encoded as shown in Table I corresponds to each contesting candidate.Lets choose a field .We have considered a  secret sharing scheme where any two shares can be used to reconstruct the secret .Every time a vote is casted, a random polynomial  of degree 1 are constructed with constant term as the encoded vote and the other coefficient are
 chosen randomly from .Generate the three shares  and  with 's as 1,2 and 3.It is noted that the shares are random irrespective of the encoded vote.So the collection centre cannot derive any information about the secret from the shares they receive.The collection centre also compute the share sum  from the shares they receive.Table II shows the random polynomials constructed,the corresponding shares generated and also the share sum in the sample run of the algorithm corresponds to  secret sharing scheme.
\begin{table}[ht]
	\small
	\caption{Vote Sharing}
\centering
\begin{tabular}{|c|c|l|c|c|c|} \hline
	\multirow{2}{*}{Vote} & \multirow{2}{*} {enc \textunderscore vote} & \multirow{2}{*}  {q(x)} &  \multicolumn{3}{c|}{Shares}\\
	\cline{4-6}
   & & &  &  &  \\
	\hline
	1 & 64 & 233.x+64 & 40 & 16 & 249\\ 
	\hline 
  	2 & 8 & 157.x+8 & 165 & 65 & 222 \\ 
	\hline 
	3 & 8 & 78.x+8 & 86 & 164 & 242 \\ 
	\hline  
	4 & 64 & 255.x+64 & 62 & 60 & 58 \\ 
	\hline 
	5 & 1 & 217.x+1 & 218 & 178 & 138 \\ 
	\hline
	6 & 64 & 124.x+64& 188 & 55 & 179\\ 
	\hline 
	\multicolumn{3}{|c|}{share sum } & 245 & 24 & 60 \\
	\hline
\end{tabular} 
\end{table}

The election result can be computed from the sum of shares  maintained by  each collection centre using Lagrange interpolation.The polynomial
 can be obtained using any two shares in the example using the Lagrange Interpolation formula as follows.


The final result depends on  which is easily obtained by


Computation of results using different combination of shares  are shown in equation 1,2 and 3.The operations are carried out in .It is noted that the reconstructed values are consistent.


The final result can be obtained by decoding the reconstructed result 209 into
binary.It is noted that 3 bits will represent vote secured by each candidate.



The result can be published based on the obtained values which is shown below.

\begin{table}[ht]
	\small
	\caption{E-voting Result}
\centering
	\begin{tabular}{|c|c|} \hline
		 \textbf{Candidate}	& \textbf{Votes Secured} \\
		\hline
		Alice & 3 \\
		\hline
		Bob &   2 \\
		\hline
		Charles & 1 \\
		\hline
	\end{tabular} 
\end{table}

\section{Analysis}

Security in online election is a challenging task.Authenticating the voter is a major challenge along with the privacy of the vote.We have considered manual authentication and proposed a modification to the existing voting scheme which uses electronic voting machine.The voting machines are not reliable and also in certain situations where the number of candidates are more, more than one voting machine needs to be connected.The proposed scheme is cost effective and also reliable.

It is noted that the proposed algorithm mentioned is simple and effective and provides privacy to the vote casted.The shares are generated  by constructing a random polynomial and the share size is same as the encoded vote.The collection centers have no idea about how the votes are encoded, how many bits are used for encoding, which bits represents a particular candidate votes etc.The collection centres will receive a random value from the field  from which no information about the secret vote can be obtained.The coalition of  untrusted collection centers can obtain the result.But they doesn't have any knowledge about the number of collection centres,the threshold used and also what is the  values assigned to each collection centre.In the example we have considered 1,2 and 3 for simplicity, however different  values can be used and is kept secret.

Shamir's secret sharing scheme is information theoretically secure.It is perfect in the sense that no information can be obtained from less than the threshold number of participants.This adds trust to the existing E-voting scheme,because the computation of the result need participation of  centers.The computation of the shares and the reconstruction of the final result using the share sum can be done using polynomial evaluation and interpolation.Efficient  algorithms for polynomial evaluation and interpolation are mentioned in \cite{aho1974design},\cite{lloyd1982art}.Simple quadratic algorithms are sufficient because the number of shares generated is not too large.

The encoding and decoding of the votes can also be done easily.The codes for each candidates and also the number of bits required to represent the votes depends on the number of voters and number of contesting candidate.These setups are done by the election officials prior  to the election process.The decoding of votes is a simple binary conversion which can also be done easily.
The integrity of the share sum maintained by each centre is achieved by implementing a digital signature scheme.This can also be efficiently implemented
using any digital signature scheme \cite{atreya2002digital}. 

The algorithm is computationally efficient and the complexity involved  depends on the share generation during the voting and the communication with the Collection Centres .The number of shares are usually small and hence the share generation using polynomial evaluation is simple.The secure communication between the voting terminal and the collection centre is a more challenging.Separate communication module can be incorporated to do it efficiently.The collection centre must also be capable of handling requests from large number of voting terminals.Region wise collection centres can be incorporated to balance the load and update the top level collection centre data in a periodic manner.The result analysis needs the polynomial interpolation but is done only once and it doesnt add much complexity to the performance of the system.


\section{Conclusions}
\label{section:conclusions}
The E-voting scheme using Shamir's secret sharing homeomorphism is a first proposal which helps to obtain not only the election result but also the votes gained by each candidate with encoding and decoding of votes in a typical manner.The proposal of secret sharing homomorphism was suggested by several authors, however true or false voting mechanism is mentioned.The proposed algorithm generalize the use of secret sharing homomorphism to E-voting which provides secrecy, computational efficiency, trust and reliability.The system does not also leave any trace of the vote made by a voter.

The strong requirement of the scheme mentioned here is a secure channel for sending shares.The shares can be send through different channels to different collection centres.The intruder have to get access to  different channels for breaking the security of the scheme.For additional security, the shares can also be encrypted by using the public keys of the collection centre.There are several homomorphic encryptions which support this or ordinary encryption decryption can be used.

The system works efficiently for a moderate election with less number of voters.
If the number of voters and candidates are more, the encoded vote will have a large value and the system has to chose a field of large size.This will large share size and too much communication overhead.This can be avoided by breaking the encoded vote into smaller code and makes shares of it.However the complexity involved in the implementation will increase.

We have done a preliminary implementation of the scheme using java \cite{nair2015improved}.Additional modules are incorporated as per the requirement.Another feature that can be incorporated is the implementation of digital signature scheme,which ensures integrity and authenticity of the shares.Verifiable secret sharing techniques can also be incorporated which ensures the consistency of the shares, however it slow down the system performance. We are looking for a more sophisticated implementation guaranteeing authentication using mobile phones and OTP(One Time Password) for all the users using adhar details.Instead of voting terminals every one can vote using the registered mobile which is our future plan.

\begin{thebibliography}{50}

\bibitem{shamir1979} A. Shamir. How to share a secret.\emph{ Communications of the ACM}, 22(11):612-613, 1979.

\bibitem{blakley1979}G. R. Blakley et al. Safeguarding crypto-
graphic keys. In \emph{Proceedings of the national
	computer conference}, volume 48, pages 313-317, 1979.

\bibitem{mignotte1983}M. Mignotte. How to share a secret. In \emph{Cryptography}, pages 371-375. Springer, 1983.

\bibitem{asmuth1983}C. Asmuth and J. Bloom. A modular approach to key safeguarding. \emph{Information Theory, IEEE Transactions on}, 29(2):208-210,1983.

\bibitem{lloyd1982art}Lloyd, E. Keith. "The art of computer programming, vol. 2, seminumerical algorithms , Donald E. Knuth, Addison‐Wesley, Reading, Mass, 1981. No. of pages: xiv+ 688. Price:£ 17· 95. ISBN 0 20103822 6." Software: Practice and Experience 12.9 (1982): 883-884.

\bibitem{aho1974design}Aho, Alfred V., and John E. Hopcroft. Design \& Analysis of Computer Algorithms. Pearson Education India, 1974.

\bibitem{abe1999mix}Abe, Masayuki. "Mix-networks on permutation networks." Advances in cryptology-ASIACRYPT’99. Springer Berlin Heidelberg, 1999. 258-273.

\bibitem{asmuth1983modular}Asmuth, Charles, and John Bloom. "A modular approach to key safeguarding." IEEE transactions on information theory 30.2 (1983): 208-210.

\bibitem{benaloh1987secret}Benaloh, Josh Cohen. "Secret sharing homomorphisms: Keeping shares of a secret secret." Advances in Cryptology—CRYPTO’86. Springer Berlin Heidelberg, 1987.

\bibitem{benaloh1987verifiable}Benaloh, Josh Daniel Cohen. Verifiable secret-ballot elections. Yale University. Department of Computer Science, 1987.

\bibitem{benaloh1994receipt}Benaloh, Josh, and Dwight Tuinstra. "Receipt-free secret-ballot elections." Proceedings of the twenty-sixth annual ACM symposium on Theory of computing. ACM, 1994.

\bibitem{chaum2004secret}Chaum, David. "Secret-ballot receipts: True voter-verifiable elections." IEEE security \& privacy 2.1 (2004): 38-47.

\bibitem{camenisch1995blind}Camenisch, Jan L., Jean-Marc Piveteau, and Markus A. Stadler. "Blind signatures based on the discrete logarithm problem." Advances in Cryptology—EUROCRYPT'94. Springer Berlin Heidelberg, 1995.

\bibitem{chaum1983blind}Chaum, David. "Blind signatures for untraceable payments." Advances in cryptology. Springer US, 1983.


\bibitem{chor1985verifiable}Chor, Benny, et al. "Verifiable secret sharing and achieving simultaneity in the presence of faults." 2013 IEEE 54th Annual Symposium on Foundations of Computer Science. IEEE, 1985.

\bibitem{cramery1997secure}Cramery, Ronald, Rosario Gennaroz, and Berry Schoenmakersx. "A Secure and Optimally Efficient Multi-Authority Election Scheme." (1997).

\bibitem{cohen1985robust}Cohen, Josh D., and Michael J. Fischer. "A robust and verifiable cryptographically secure election scheme." 2013 IEEE 54th Annual Symposium on Foundations of Computer Science. IEEE, 1985.

\bibitem{chen2014secure}
Chin-Ling Chen, Yu-Yi Chen, Jinn-Ke Jan, and Chih-Cheng Chen.
\newblock A secure anonymous e-voting system based on discrete logarithm
problem.
\newblock {\em Applied Mathematics \& Information Sciences}, 8(5), 2014.

\bibitem{damgaard2010generalization}
Ivan Damg{\aa}rd, Mads Jurik, and Jesper~Buus Nielsen.
\newblock A generalization of paillier’s public-key system with applications
to electronic voting.
\newblock {\em International Journal of Information Security}, 9(6):371--385,
2010.

\bibitem{fujioka1993practical}Fujioka, Atsushi, Tatsuaki Okamoto, and Kazuo Ohta. "A practical secret voting scheme for large scale elections." Advances in Cryptology—AUSCRYPT'92. Springer Berlin Heidelberg, 1993.


\bibitem{gritzalis2002principles} Gritzalis, Dimitris A. "Principles and requirements for a secure e-voting system." Computers \& Security 21.6 (2002): 539-556.

\bibitem{gritzalis2003secure} Gritzalis, Dimitris, ed. Secure electronic voting. Dordrecht: Kluwer Academic Publishers, 2003.

\bibitem{hussien2013design}Hussien, Hanady, and Hussien Aboelnaga. "Design of a secured e-voting system." Computer Applications Technology (ICCAT), 2013 International Conference on. IEEE, 2013.

\bibitem{hirt2000efficient}
Martin Hirt and Kazue Sako.
\newblock Efficient receipt-free voting based on homomorphic encryption.
\newblock In {\em Advances in Cryptology—EUROCRYPT 2000}, pages 539--556.
Springer, 2000.

\bibitem{ibrahim2003secure}Ibrahim, Subariah, et al. "Secure E-voting with blind signature." Telecommunication Technology, 2003. NCTT 2003 Proceedings. 4th National Conference on. IEEE, 2003.

\bibitem{iftene2007general}
Sorin Iftene.
\newblock General secret sharing based on the chinese remainder theorem with
applications in e-voting.
\newblock {\em Electronic Notes in Theoretical Computer Science}, 186:67--84,
2007.

\bibitem{jakobsson1998practical}Jakobsson, Markus. "A practical mix." Advances in Cryptology—EUROCRYPT'98. Springer Berlin Heidelberg, 1998. 448-461.


\bibitem{kothari1985generalized}Kothari, Suresh C. "Generalized linear threshold scheme." Advances in Cryptology. Springer Berlin Heidelberg, 1985.


\bibitem{kuri2009xor}Kurihara, Jun, et al. "A new (k, n)-threshold secret sharing scheme and its extension." Information Security. Springer Berlin Heidelberg, 2008. 455-470.


\bibitem{lee2000receipt}
Byoungcheon Lee and Kwangjo Kim.
\newblock Receipt-free electronic voting through collaboration of voter and
honest verifier.
\newblock In {\em Proceeding of JW-ISC2000}. Citeseer, 2000.

\bibitem{malkhi2003voting}
Dahlia Malkhi, Ofer Margo, and Elan Pavlov.
\newblock E-voting without ‘cryptography’.
\newblock In {\em Financial Cryptography}, pages 1--15. Springer, 2003.

\bibitem{neff2001verifiable}
C~Andrew Neff.
\newblock A verifiable secret shuffle and its application to e-voting.
\newblock In {\em Proceedings of the 8th ACM conference on Computer and
	Communications Security}, pages 116--125. ACM, 2001.

\bibitem{okamoto1998receipt}Okamoto, Tatsuaki. "Receipt-free electronic voting schemes for large scale elections." Security Protocols. Springer Berlin Heidelberg, 1998.

\bibitem{pan2014enhanced}
Haijun Pan, Edwin Hou, and Nirwan Ansari.
\newblock Enhanced name and vote separated e-voting system: an e-voting system
that ensures voter confidentiality and candidate privacy.
\newblock {\em Security and Communication Networks}, 2014.

\bibitem{peng2005multiplicative}Peng, Kun, et al. "Multiplicative homomorphic e-voting." Progress in Cryptology-INDOCRYPT 2004. Springer Berlin Heidelberg, 2005. 61-72.

\bibitem{park1994efficient}Park, Choonsik, Kazutomo Itoh, and Kaoru Kurosawa. "Efficient anonymous channel and all/nothing election scheme." Advances in Cryptology—EUROCRYPT’93. Springer Berlin Heidelberg, 1994.


\bibitem{rivest1978method}Rivest, Ronald L., Adi Shamir, and Len Adleman. "A method for obtaining digital signatures and public-key cryptosystems." Communications of the ACM 21.2 (1978): 120-126.

\bibitem{stadler1996publicly}Stadler, Markus. "Publicly verifiable secret sharing." Advances in Cryptology—EUROCRYPT’96. Springer Berlin Heidelberg, 1996.

\bibitem{schoenmakers1999simple}Schoenmakers, Berry. "A simple publicly verifiable secret sharing scheme and its application to electronic voting." Advances in Cryptology—CRYPTO’99. Springer Berlin Heidelberg, 1999.

\bibitem{sako1994secure}Sako, Kazue, and Joe Kilian. "Secure voting using partially compatible homomorphisms." Advances in Cryptology—CRYPTO’94. Springer Berlin Heidelberg, 1994.

\bibitem{sako1995receipt}Sako, Kazue, and Joe Kilian. "Receipt-free mix-type voting scheme." Advances in Cryptology—EUROCRYPT’95. Springer Berlin Heidelberg, 1995.

\bibitem{wang2007ssboln}Wang, Daoshun, et al. "Two secret sharing schemes based on Boolean operations." Pattern Recognition 40.10 (2007): 2776-2785.

\bibitem{bjorck1970solution}Björck, Ȧke, and Victor Pereyra. "Solution of Vandermonde systems of equations." Mathematics of Computation 24.112 (1970): 893-903.

\bibitem{atreya2002digital}Atreya, Mohan, et al. Digital signatures. Osborne/McGraw-Hill, 2002.

\bibitem{nair2015improved}Nair, Divya G., V. P. Binu, and G. Santhosh Kumar. "An Improved E-voting scheme using Secret Sharing based Secure Multi-party Computation." arXiv preprint arXiv:1502.07469 (2015).

\bibitem{pub2014draft}Pub, N. F. "DRAFT FIPS PUB 202: SHA-3 standard: Permutation-based hash and extendable-output functions." Federal Information Processing Standards Publication (2014).

\end{thebibliography}
\end{document}
