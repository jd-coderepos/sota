\documentclass{article}
\usepackage{amsfonts, amsmath, amssymb}
\usepackage[T1]{fontenc} \usepackage[latin1]{inputenc} \usepackage[dvips]{graphicx}
\usepackage{color}
\newtheorem{lemma}{Lemma}
\newtheorem{theorem}{Theorem}
\newtheorem{coro}{Corollary}
\newtheorem{fac}{Fact}
\newtheorem{question}{Question}
\newtheorem{conj}{Conjecture}
\newtheorem{prop}{Proposition}
\newtheorem{proof}{Proof}
\newtheorem{claim}{Claim}
\newtheorem{obs}{Observation}
\newtheorem{problem}{Problem}
\def\RMO{R{\footnotesize{OOTED}} M{\footnotesize{AXIMUM}} L{\footnotesize{EAF}} O{\footnotesize{UTBRANCHING}} }
\def\RMOB{R{\footnotesize{OOTED}} M{\footnotesize{AXIMUM}} L{\footnotesize{EAF}} O{\footnotesize{UTBRANCHING}}. }
\def\RMOC{R{\footnotesize{OOTED}} M{\footnotesize{AXIMUM}} L{\footnotesize{EAF}} O{\footnotesize{UTBRANCHING}}: }
\def\RMOD{R{\footnotesize{OOTED}} M{\footnotesize{AXIMUM}} L{\footnotesize{EAF}} O{\footnotesize{UTBRANCHING}}, }
\def\MO{M{\footnotesize{AXIMUM}} L{\footnotesize{EAF}} O{\footnotesize{UTBRANCHING}}}
\def\ML{M{\footnotesize{AXIMUM}} L{\footnotesize{EAF}} S{\footnotesize{PANNING}} T{\footnotesize{REE}}}



\title{On finding directed trees with many leaves}

\author{Jean Daligault and St\'ephan Thomass\'e}
\begin{document}
\maketitle

\begin{abstract}
The \RMO problem consists in finding a spanning directed tree rooted at some prescribed vertex of a 
digraph with the maximum number of leaves. Its parameterized version asks if there 
exists such a tree with at least  leaves. We use the notion of  numbering studied in \cite{stnum}, \cite{stnumdir}, \cite{LLWemb} to exhibit combinatorial bounds on the existence of spanning directed trees with many leaves. These combinatorial bounds allow us to produce a constant factor approximation algorithm for finding directed trees with many leaves, whereas the best known approximation algorithm has a -factor \cite{DrescherMaxLeaf}. We also show that \RMO admits a quadratic kernel, improving over the cubic kernel given by Fernau et al \cite{FernauMaxLeaf}.
\end{abstract}


\section{Introduction}
An \emph{outbranching} of a digraph  is a spanning directed tree in . 
We consider the following problem:

\vspace{12pt}
{\bf \RMOC}


\vspace{12pt}
\textbf{Input}: A digraph , an integer , a vertex  of .


\textbf{Output}: TRUE if there is an outbranching of  rooted at  with at least  leaves, otherwise FALSE.
\vspace{12pt}

This problem is equivalent to finding a Connected Dominating Set of size at most , connected meaning in this setting that every vertex is reachable by a directed path from . Indeed, the set of internal nodes in an outbranching correspond to a connected dominating set.

Finding undirected trees with many leaves has many applications in the area of communication networks, see \cite{MaxLeafApplication2} or \cite{MaxLeafApplication} for instance. An extensive litterature is devoted to the paradigm of using a small connected dominating set as a backbone for a communication network.

\RMO is NP-complete, even restricted to acyclic digraphs \cite{AlonMaxLeaf2}, and MaxSNP-hard, even on undirected graphs \cite{ApxHardMaxLeaf}.

Two natural ways to tackle such a problem are, on the one hand, polynomial-time approximation algorithms, and on the other hand, parameterized complexity. Let us give a brief introduction on the parameterized approach.


\vspace{12pt}

An efficient way of dealing with NP-hard problems
is to identify a parameter which contains its 
computational hardness. For instance, instead 
of asking for a minimum vertex cover in a graph -
a classical NP-hard optimization question - one
can ask for an algorithm which would decide, in
 time for some fixed , if a graph 
of size  has a  vertex cover of size at 
most . If such an algorithm exists, the problem
is called {\it fixed-parameter tractable}, or FPT for 
short. An extensive literature is devoted to FPT, 
the reader is invited to read \cite{DowneyFellows}, \cite{FlumGrohe}
and \cite{NiedBookFPT}.

Kernelization is a natural way of proving that a 
problem is FPT. Formally, a {\it kernelization algorithm}
receives as input an instance  of the parameterized
problem, and outputs, in polynomial time  in the size of 
the instance, another instance  such that: , the size of  only depends of , and the instances  and  are both true or both false.



The reduced instance  is called a {\it kernel}.
The existence of a kernelization algorithm 
clearly implies the FPT character of the problem since 
one can kernelize the instance, and then solve the reduced
instance  using brute force, hence giving an 
 algorithm. 
A classical result asserts that being FPT is indeed equivalent to 
having kernelization. The drawback of this result is that 
the size of the reduced instance  is not necessarily 
small with respect to . A much more constrained condition
is to be able to reduce to an instance of polynomial 
size in terms of . Consequently, in the zoology of parameterized 
problems, the first distinction is done between three classes: 
W[1]-hard, FPT, polykernel. 

A kernelization algorithm can be used as a preprocessing step to reduce the size of the instance before applying some other parameterized algorithm. Being able to ensure that this kernel has actually polynomial size in  enhances the overall speed of the process. See \cite{GuoNiedFPT} for a recent review on kernelization.


\vspace{12pt}

An extensive litterature is devoted to finding trees with many leaves in undirected and directed graphs. The undirected version of this problem, \ML, has been extensively studied. There is a factor 2 approximation algorithm for the \ML ~problem \cite{SolisMaxLeafApprox}, and a  kernel \cite{EstivillMaxLeaf}. An  exact algorithm was designed in \cite{FominKratschMaxLeaf}. Other graph theoretical results on the existence of trees with many leaves can be found in \cite{SeymourMaxLeaf} and \cite{StorerMaxLeaf}.

The best approximation algorithm known for \MO ~is a factor  algorithm \cite{DrescherMaxLeaf}. From the Parameterized Complexity viewpoint, Alon et al showed that \MO ~restricted to a wide class of digraphs containing all strongly connected digraphs is FPT \cite{AlonMaxLeaf}, and Bonsma and Dorn extended this result to all digraphs and gave a faster parameterized algorithm \cite{BonsmaMaxLeaf}. Very recently, Kneis, Langer and Rossmanith \cite{KneisMaxLeaf} obtained an  algorithm for \MO, which is also an improvement for the undirected case over the numerous FPT algorithms designed for \ML. Fernau et al \cite{FernauMaxLeaf} proved that \RMO has a polynomial kernel, exhibiting a cubic kernel. They also showed that the unrooted version of this problem admits no polynomial kernel, unless polynomial hierarchy collapses to third level, using a breakthrough lower bound result by Bodlaender et al \cite{NoKernel}. A linear kernel for the acyclic subcase of \RMO and an  algorithm for \RMO were exhibited in \cite{GutinMaxLeaf}.

\vspace{12pt}
This paper is organized as follows. In Section~\ref{sbounds} we exhibit combinatorial bounds on the problem of finding an outbranching with many leaves. We use the notion of \emph{ numbering} introduced in \cite{stnum}. We next present our reduction rules, which are independent of the parameter, and in the following section we prove that these rules give a quadratic kernel. We finally present a constant factor approximation algorithm in Section~\ref{approx} for finding directed trees with many leaves.




\section{Combinatorial Bounds}\label{sbounds}
Let  be a directed graph. For an arc  in , we say that  is an \emph{in-neighbour} of , that  is an \emph{outneighbour} of , that  is an \emph{in-arc} of  and an \emph{out-arc of }. The \emph{outdegree} of a vertex is the number of its outneighbours, and its \emph{indegree} is the number of its in-neighbours.
An outbranching with a maximum number of leaves is said to be \emph{optimal}. Let us denote by  the number of leaves in an optimal outbranching of .

Without loss of generality, we restrict ourselves to the following. We exclusively consider loopless digraphs with a distinguished vertex of indegree 0, denoted by . \emph{We assume that there is no arc  in  with , and no arc  with  and  an outneighbour of , and that  has outdegree at least 2}. Throughout this paper, we call such a digraph a \emph{rooted digraph}. Definitions will be made exclusively with respect to rooted digraphs, hence the notions we present, like connectivity and resulting concepts, do slightly differ from standard ones.
Let  be a rooted digraph with a specified vertex . 

The rooted digraph  is \emph{connected} if every vertex of  is reachable by a directed path rooted at  in . A \emph{cut} of  is a set  such that there exists a vertex  endpoint of no directed path from  in . We say that  is \emph{2-connected} if  has no cut of size at most 1. A cut of size 1 is called a \emph{cutvertex}. Equivalently, a rooted digraph is 2-connected if there are two internally vertex-disjoint paths from  to any vertex besides  and its outneighbours.

We will show that the notion of  numbering behaves well with respect to outbranchings with many leaves. It has been introduced in \cite{stnum} for 2-connected undirected graphs, and generalized in \cite{stnumdir} by Cheriyan and Reif for digraphs which are 2-connected in the usual sense. We adapt it in the context of rooted digraphs.

Let  be a 2-connected rooted digraph. An  numbering of  is a linear ordering  of  such that, for every vertex , either  is an outneighbour of  
or there exist two in-neighbours  and  of  such that . An equivalent presentation of an  numbering of  is an injective embedding  of the graph  where  has been duplicated into two vertices  and , into the -segment of the real line, such that , , and such that the image by  of every vertex besides  and  lies inside the convex hull of the images of its in-neighbours. Such \emph{convex embeddings} have been defined and studied in general dimension by Lov\'asz, Linial and Wigderson in \cite{LLWemb} for undirected graphs, and in \cite{stnumdir} for directed graphs.

Given a linear order  on a finite set , we denote by  the linear order on  which is the reverse of . An arc  of  is a \emph{forward} arc
if  or if  appears before  in ;  is a \emph{backward} arc if  or if  appears after  in . A spanning out-tree  is \emph{forward} if all its arcs are forward. Similar
definition for \emph{backward} out-tree.

The following result and proof is just an adapted version of \cite{stnumdir},  given here for the sake of completeness.
 
\begin{lemma}\label{rrnum}
Let  be a 2-connected rooted digraph. There exists an  numbering of .
\end{lemma}
\emph{Proof}: By induction over . We first reduce to the case where the indegree of every vertex besides  is exactly 2. Let  be a vertex of indegree at least 3 in . Let us show that there exists an in-neighbour  of  such that the rooted digraph  is 2-connected.
Indeed, there exist two internally vertex disjoint paths from  to . Consider such two paths intersecting  only once each, and denote by  the rooted digraph obtained from  by removing one arc  not involved in these two paths. There are two internally disjoint paths from  to  in . Consider . Assume by contradiction that there exists a vertex  which cuts  from  in . As  does not cut  from  in  and the arc  alone is missing in ,  must cut  and not  from  in . Which is a contradiction, as there are two internally disjoint paths from  to  in . By induction,  has an  numbering, which is also an  numbering for .

Hence, let  be a rooted digraph, where every vertex besides  has indegree . As  has indegree 0, there exists a vertex  with outdegree at most 1 in  by a counting argument. If  has outdegree 0, then let  be an  numbering of , let  and  be the two in-neighbours of . Insert  between  and  in  to obtain an  numbering of .
Assume now that  has a single outneighbour . Let  be the second in-neighbour of . Let  be the graph obtained from  by contracting the arc  into a single vertex . As  is 2-connected, consider by induction an  numbering  of . Replace  by . It is now possible to insert  between its two in-neighbours in order to make it so that  lies between  and . Indeed, assume without loss of generality that  is after  in . Consider the smallest in-neighbour  of  in . As  is an  numbering of ,  lies before  in . We insert  just after  to obtain an  numbering of . 
\\


Note that an  numbering  of  naturally gives two acyclic covering subdigraphs of , the rooted digraph  consisting of the forward arcs of , and the rooted digraph  consisting of the backward arcs of . The intersection of these two acyclic digraphs is the set of out-arcs of .

\begin{coro}
Let  be a 2-connected rooted digraph. There exists an acyclic connected spanning subdigraph  of  which contains at least half of the arcs of .
\end{coro}
Let  be an undirected graph. A \emph{vertex cover} of  is a set of vertices covering all edges of . A \emph{dominating set} of  is a set  such that for every vertex ,  has a neighbour in . A \emph{strongly dominating set} of  is a set  such that every vertex has a neighbour in .

Let  be a rooted digraph. A \emph{strongly dominating set} of  is a set  such that every vertex besides  has an in-neighbour in .
We need the following folklore result:
\begin{lemma}\label{domi1}
Any undirected graph  on  vertices and  arcs has a vertex cover of size .
\end{lemma}
\emph{Proof}: By induction on . If there exists a vertex of degree at least 2 in , choose it in the vertex cover, otherwise choose any non-isolated vertex.


\begin{lemma}\label{domi2}
Let  be a bipartite graph over , with  for every . There exists a subset of  dominating  with size at most .
\end{lemma}
\emph{Proof}:
Let  be the graph which vertex set is , and where  is an arc if  and  share a common neighbour in . The result follows from Lemma~\ref{domi1} since  has  arcs and  vertices.



\begin{coro}\label{leaves}
Let  be an acyclic rooted digraph with  vertices of indegree at least 2 and with a root of outdegree . Then  has an outbranching with at least  leaves.
\end{coro}
\emph{Proof}: Denote by  the number of vertices of . For every vertex  of indegree at least 3, delete incoming arcs until  has indegree exactly 2. Since  is acyclic, it has a sink . 

Let  be the set of vertices of indegree 1 in , of size . Let  be the set of in-neighbours of vertices of , of size at most . Let  be the set of vertices of indegree 2 dominated by . Let . Let  be the set of vertices of indegree 2 not dominated by . Note that  cannot have the same size as . Indeed,  contains the outneighbours of , and hence  contains , which has outdegree at least 2. More precisely, . As  and , we have that . Moreover, as  has size at most , we have that . Consider a copy  of  and a copy  of . Let  be the bipartite graph with vertex bipartition , and where , with  and , is an edge if  is an arc in . By Lemma~\ref{domi2} applied to , there exists a set  of size at most  which dominates  in . The set  strongly dominates  in , and has size at most . As  and , this yields . As  is acyclic, any set strongly dominating  contains  and is a connected dominating set. Hence there exists an outbranching  of  having a subset of  as internal vertices.  has at least  leaves. 


\\

This bound is tight up to one leaf. The rooted digraph  depicted in Figure 1 is 2-connected, has  vertices of indegree at least 2,  and .


\vspace{12pt}

\begin{figure}
\centering
\includegraphics[angle=-90]{boloney6.eps}
\caption{The "boloney" graph }
\end{figure}

Finally, the following combinatorial bound is obtained:

\begin{theorem}\label{bound1}
Let  be a 2-connected rooted digraph with  vertices of indegree at least 3. Then  .
\end{theorem}
\emph{Proof}: Apply Corollary~\ref{leaves} to the rooted digraph with the larger number of vertices of indegree 2 among  and . \\



An arc is \emph{simple} if does not belong to a 2-circuit. A vertex 
is \emph{nice} if it is incident
to a simple in-arc. 

The second combinatorial bound is the following:
\begin{theorem}\label{bound2}
Let  be 2-connected rooted digraph. Assume that  has  nice vertices. Then  has an outbranching with at least  leaves.
\end{theorem}
\emph{Proof}:
By Lemma~\ref{rrnum}, we consider an  numbering  of .
For every nice vertex 
(incident
to some in-arc ) with indegree at least three, delete incoming arcs
of  different from  until  has only one incoming forward arc
and one incoming backward arc. For every other vertex of indegree at
least 3 in , delete incoming arcs of  until  has only one
incoming forward arc and one incoming backward arc. At the end of this
process,  is still an  numbering of the digraph , and
the number of nice vertices has not decreased.

Denote by  the set of forward arcs of , and by  the set of
backward arcs of . As  is an  numbering of , 
and  are spanning trees of  which partition the arcs of .

The crucial definition is the following: say that an arc  of 
(resp. of ), with , is \emph{transverse} if  and 
are \emph{incomparable} in  (resp. in ), that is if  is not
an ancestor of  in  (resp. in ). Observe that  cannot be
an ancestor of  in  (resp. in ) since  is backward (resp.  is forward) while 
is forward (resp. backward) and .

Assume without loss of generality that  contains more transverse
arcs than . Consider now any planar
drawing of the rooted tree . We will make use of this drawing to
define the following:
if two vertices  and  are incomparable in , then one of these
vertices is to the left of the other, with respect to our drawing.
Hence, we can
partition the transverse arcs of  into two subsets: the set 
of transverse arcs
 for which  is to the left of , and the set  of
transverse arcs
 for which  is to the right of . Assume without loss of
generality that .

The digraph  is an acyclic digraph by definition of .
Moreover,
it has  vertices of indegree two since the heads of the arcs of

are pairwise distinct. Hence, by Corollary~\ref{leaves}, 
has an outbranching
with at least  leaves, hence so does .

We now give a lower bound on the number of transverse arcs in  to
bound . Consider
a nice vertex  in , which is not an outneighbour of , and with a simple in-arc  belonging to, say,
. If  is not
a transverse arc, then  is an ancestor of  in . Let  be
the outneighbor of
 on the path from  to  in . Since  is simple, the
vertex  is distinct
from . No path in  goes from  to , hence  is a
transverse arc. Therefore,
we proved that  (and hence every nice vertex) is incident to a
transverse arc (either
an in-arc, or an out-arc). Thus there are at least 
transverse arcs in .

Finally, there are at least  transverse arcs in , and thus
. In all,
 has an outbranching with at least  leaves.











\vspace{12pt}



As a corollary, the following result holds for oriented graphs (digraphs with no 2-circuit):
\begin{coro}
Every 2-connected rooted oriented graph on  vertices has an outbranching with at least  leaves.
\end{coro}





\section{Reduction Rules}\label{srules}
 We say that , with , is a \emph{bipath of length } if the set of arcs adjacent to  in  is exactly .

To exhibit a quadratic kernel for \RMOD we use the following four reduction rules:
\begin{itemize}
\item[(0)] If there exists a vertex not reachable from  in , then reduce to a trivially FALSE instance. 
\item[(1)] Let  be a cutvertex of . Delete vertex  and add an arc  for every  and .
\item[(2)] Let  be a bipath of length 4. Contract two consecutive internal vertices of . 
\item[(3)] Let  be a vertex of . If there exists  such that  cuts  from , then delete the arc .
\end{itemize}

Note that these reduction rules are not parameter dependent. Rule (0) only needs to be applied once.

\begin{obs}\label{obs1}
Let  be a cutset of a rooted digraph . Let  be an outbranching of . There exists a vertex in  which is not a leaf in .
\end{obs}

\begin{lemma}\label{rules}
The above reduction rules are safe and can be checked and applied in polynomial time.
\end{lemma}
\emph{Proof}:
\begin{itemize}
\item[(0)] Reachability can be tested in linear time.
\item[(1)] Let  be a cutvertex of . Let  be the graph obtained from  by deleting vertex  and adding an arc  for every  and . Let us show that maxleaf  maxleaf. Assume  is an outbranching of  rooted at  with  leaves. By Observation~\ref{obs1},  is not a leaf of . Let  be the father of  in . Let  be the tree obtained from  by contracting  and .  is an outbranching of  rooted at  with  leaves. 

Let  be an outbranching of  rooted at  with  leaves.  is a cut in , hence by Observation~\ref{obs1} there is a non-empty collection of vertices  which are not leaves in . Choose  such that  is not an ancestor of  for every . Let  be the graph obtained from  by adding  as an isolated vertex, adding the arc , and for every , for every arc  with , delete the arc  and add the arc . As  is not reachable in  from any vertex , there is no cycle in . Hence  is an outbranching of  rooted at  with at least  leaves. Moreover, deciding the existence of a cut vertex and finding one if such exists can be done in polynomial time.
\item[(2)] Let  be a bipath of length 4. Let , , ,  and  be the vertices of  in this consecutive order. Let  be the rooted digraph obtained from  by contracting  and . Let  be an outbranching of . Let  be the rooted digraph obtained from  by contracting  with its father in .  is an outbranching of  with as many leaves as . Let  be an outbranching of . If the father of  in  is , then  is an outbranching of  with at least as many leaves as . If the father of  in  is , then  is an outbranching of  with at least as many leaves as .
\item[(3)] Let  be a vertex of . Let  be a vertex such that  cuts  from . Let  be the rooted digraph obtained from  by deleting the arc . Every outbranching of  is an outbranching of . Let  be an outbranching of  containing . There exists a vertex  which is an ancestor of . Thus  is an outbranching of  with at least as many leaves as .
\end{itemize}





We apply these rules iteratively until reaching a \emph{reduced instance}, on which none can be applied.
\begin{lemma}\label{largedegree}
Let  be a reduced rooted digraph with a vertex of indegree at least . Then  is a TRUE instance. 
\end{lemma}
\emph{Proof}: Assume  is a vertex of  with in-neighbourhood , with . For every ,  does not cut  from . Thus there exists a path  from  to  outside . The rooted digraph  is connected, and for every ,  has outdegree 0 in . Thus  has an outbranching with at least  leaves, and such an outbranching can be extended into an outbranching of  with at least as many leaves. 


\section{Quadratic kernel}\label{skernel}

In this section and the following, a vertex of a 2-connected rooted digraph  is said to be \emph{special} if it has indegree at least 3 or if one of its incoming arcs is simple. A non special vertex is a vertex  which has exactly two in-neighbours, which are also outneighbours of . A \emph{weak bipath} is a maximal connected set of non special vertices. If  is a weak bipath, then the in-neighbours of , for  in  are exactly  and . Moreover,  and  are each outneighbour of a special vertex. Denote by  the in-neighbour of  which is a special vertex.



This section is dedicated to the proof of the following statement:

\begin{theorem}\label{kernel}
A digraph  of size at least  reduced under the reduction rules of previous section has an outbranching with at least  leaves.
\end{theorem}

\emph{Proof}:
By Theorem~\ref{bound1} and Theorem~\ref{bound2}, if there are at least  special vertices, then  has an outbranching with at least  leaves. Assume that there are at most  special vertices in . 

As  is reduced under Rule (2), there is no bipath of length 4. We can associate to every weak bipath  of  of length  a set  of  out-arcs toward special vertices. Indeed,  let  be a weak bipath of . For every three consecutive vertices  of , ,  is not a bipath by Rule (2), hence there exists an arc  with  or  and . Moreover  must be a special vertex as arcs between non-special vertices lie within their own weak bipath. The set of these arcs  has the presribed size.

By Lemma~\ref{largedegree}, any vertex in  has indegree at most  as  is reduced under Rule (3), hence there are at most  non special vertices in .



\vspace{12pt}




To sum up, the kernelization algorithm is as follows: starting from a rooted digraph , apply the reduction rules. Let  be the obtained reduced rooted digraph. If  has size more than , then reduce to a trivially TRUE instance. Otherwise,  is an instance equivalent to  of size quadratic in .


Our analysis for this quadratic kernel for \RMO is actually tight up to a constant factor.
Indeed, the following graph  is reduced under the reduction rules stated on Section~\ref{srules} and has a number of vertices quadratic in its maximal number of leaves. Let  . For every ,  is an arc of . For every , ,  is a 2-circuit of . For every ,  is an arc of . For every , ,  is an arc of . This digraph  is reduced under the reduction rules of Section~\ref{srules}, and . Finally,  has  vertices. 


Note that this graph has many 2-circuits. We are not able to deal with them with respect to kernelization. For the approximation on the contrary, we are able to deal with the 2-circuits to produce a constant factor approximation algorithm.





\section{Approximation}\label{approx}
Let us first point out that the reduction rules described in Section~\ref{rules} directly give an approximation algorithm asymptotically as good as the best known approximation algorithm \cite{DrescherMaxLeaf}. Indeed, as these rules are independant of the parameter, and as our proof of the existence of a solution of size  when the reduced graph has size more than  is contructive, this yields a  approximation algorithm. Let us sketch this approximation algorithm. Start by applying the reduction rules described in Section~\ref{rules} to the input rooted digraph. This does not change the value of the problem. Let  be the size of the reduced graph. Exhibit an outbranching with at least  leaves as in the proof of Theorem~\ref{kernel}. Finally, undo the sequence of contractions yield by the application of reduction rules at the start of the algorithm, repairing the tree as in the proof of Lemma~\ref{rules}. The tree thus obtained has at least  leaves, while the tree with maximum number of leaves in the input graph has at most  leaves. Thus this algorithm is an  approximation algorithm.


Let us describe now our constant factor approximation algorithm for \RMOD being understood that this also gives an approximation algorithm of the same factor for \MO ~as well as for finding an out-tree (not necessarily spanning) with many leaves in a digraph.

Given a rooted digraph , apply exhaustively Rule (1) of Section~\ref{srules}. The resulting rooted digraph  is 2-connected. By Lemma~\ref{rules}, .

Let us denote by  the digraph  restricted to non special vertices. Recall that  is a dijoint union of bipaths, which we call \emph{non special components}. A vertex of outdegree 1 in  is called an \emph{end}. Each end has exactly one special vertex as an in-neighbour in .

 
\begin{theorem}\label{majbound} 
Let  be a 2-connected rooted digraph with  special vertices and  non special components. Then max.
\end{theorem}
\emph{Proof}: The upper bound is clear, as at most two vertices in a given non special component can be leaves of a given outbranching. The first term of the lower bound comes from Theorem~\ref{bound1} and Theorem~\ref{bound2}. To establish the second term, consider the digraph  which vertices are the special vertices of  and . For every non special component of , add an edge in  between the special in-neighbours of its two ends. Consider an outbranching of  rooted at . This outbranching uses  edges in , and directly corresponds to an out-tree  in . Extend  into an outbranching  of . Every non special component which is not used in  contributes to at least a leaf in , which concludes the proof.



\vspace{12pt}

Consider the best of the three outbranchings of  obtained in polynomial time by Theorem~\ref{bound1}, Theorem~\ref{bound2} and Theorem~\ref{majbound}. This outbranching has at least max leaves. The worst case is when . In this case, the upper bound becomes: , hence we have a factor  approximation algorithm for \RMOB


\section{Conclusion}
We have given a quadratic kernel and a constant factor approximation algorithm for \RMOC reducing the gap between the problem of finding trees with many leaves in undirected and directed graphs. \ML ~ has a factor 2 approximation algorithm, and \RMO now has a factor 92 approximation algorithm. Reducing this 92 factor into a small constant is one challenge. The gap now essentially lies in the fact that \ML ~has a linear kernel while \RMO has a quadratic kernel. Deciding whether \RMO has a linear kernel is a challenging question. Whether long paths made of 2-circuits can be dealt with or not might be key to this respect.



\bibliographystyle{plain}
\bibliography{maxleafarxiv}

\end{document}
