
\documentclass [10pt,a4paper]{article}
\pagestyle{empty} \topmargin 0 true cm \oddsidemargin 0pt
\headheight=16pt \headsep=20pt
\evensidemargin 0pt \textwidth=14.5 true cm \textheight=21 true cm
\def\ay{\arraycolsep=1.5pt}
\usepackage[numbers,sort&compress]{natbib}
\usepackage{amsfonts}
\usepackage{indentfirst}
\usepackage{amsmath}
\usepackage{amsfonts}
\usepackage{amssymb}
\usepackage{amsthm}
\usepackage{maplestd2e}
\usepackage{a4wide}
\usepackage[CJKbookmarks]{hyperref}
\usepackage{graphicx}
\usepackage{float}
\usepackage{subfigure}
\DeclareMathOperator{\sign}{sign}
\def\ssmsczi{\zihao{10}\ziju{0.135}}
\def\rmn#1{\romannumeral #1}
\def\Rmn#1{\expandafter\uppercase\expandafter{\romannumeral #1}}
\def\cases#1{\left\{\,\vcenter{\normalbaselines \openup\jot
  \ialign{\hfil&\quad{##}\hfil\crcr#1\crcr}}
    \right.}
\begin{document}
\newtheorem{theorem}{Theorem}[section]
\newtheorem{definition}{Def\mbox{}inition}[section]
\newtheorem{lemma}{Lemma}[section]
\newtheorem{corollary}{Corollary}[section]


\title{The weighted dif\mbox{}ference substitutions and Nonnegativity Decision of Forms\footnote{ Partially supported by a
National Key Basic Research Project of China (2004CB318000) and by
National Natural Science Foundation of China (10571095). Sponsored
by K.C.Wong Magna Fund in Ningbo University. }}
\date{}

\author{
 Xiaorong Hou~~ Song Xu~~Junwei Shao\\
 \textit{\small  College of Automation,
University of Electronic Science and Technology of China, Sichuan,
PRC}\\
 \textit{\small  Faculity of Science, Ningbo University, Ningbo,
Zhejiang, PRC}\\
\textit{\small E-mail:
\href{mailto:houxr@uestc.edu.cn}{houxr@uestc.edu.cn},
\href{mailto:xusong@nbu.edu.cn}{xusong@nbu.edu.cn} }}


 \maketitle
{\noindent\small  {\bf Abstract} In this paper, we study the
weighted dif\mbox{}ference substitutions from geometrical views.
First, we give the geometric meanings of the weighted
dif\mbox{}ference substitutions, and introduce the concept of
convergence of the sequence of substitution sets.    Then it is
proven that the
 sequence of the successive
weighted dif\mbox{}ference substitution sets is convergent.  Based
on the convergence
 of the sequence of the successive weighted
dif\mbox{}ference sets,  a new, simpler method to prove that if the
form   is positive def\mbox{}inite on , then the
sequence of  sets  is
positively terminating is presented,  which is dif\mbox{}ferent from
the one given in \cite{Yong:1}. That is, we can decide the
nonnegativity of a positive definite form by
 successively running the weighted
dif\mbox{}ference substitutions finite times. Finally, an algorithm
for deciding an indef\mbox{}inite form with a counter-example  is
obtained, and
 some examples are listed  by using the obtained algorithm.  }\2ex]




\section{ Introduction }
Theories and methods of nonnegative polynomials have been  widely
used in  robust control, non-linear control and non-convex
optimization \cite{P:1,P:2,J:1}, etc.   Some famous research works
on nonnegativity decision of polynomials without cell-decomposition
were given  by P\'{o}lya's Theorem \cite{Po:1,G:1} and papers
\cite{Ca:1,H:2}.

Few years ago,  Yang \cite{Yang:1,Yang:2,Yang:3} introduced a
heuristic method for nonnegativity  decision of polynomials, which
is now called Successive Dif\mbox{}ference Substitution (SDS). It
has been applied to prove a great many polynomial inequalities with
more variables and higher degrees. Yang recommended  further studies
on
 SDS and  put forward some open problems.

 A valuable progress on the
 topic was made by Yao \cite{Yong:1}.  He investigated a weighted dif\mbox{}ference
 substitution
 
instead of the  original dif\mbox{}ference
 substitution

 
and proved that, for a form (namely, a homogeneous polynomial) which
is positive definite on , the corresponding sequence
of SDS sets is positively terminating, where
  That
is,  we can decide the nonnegativity of a positive definite form by
 successively running SDS finite times.

 This paper is organized as follows. Section 2
introduces some preliminary notions of the weighted
dif\mbox{}ference substitutions. Section 3 provides a new
perspective  to study the weighted dif\mbox{}ference substitutions,
gives the geometric meanings  of them, and  proves that the sequence
of the successive  weighted dif\mbox{}ference
 substitution sets
is convergent.  A new, simpler method to prove that if the form  
is positive def\mbox{}inite on , then the sequence
of sets  is positively
terminating is presented in Section 4, and an algorithm for deciding
an indef\mbox{}inite form with a counter-example in Section 5. By
using the obtained algorithm, several examples are listed in Section
6.





\section{Preliminary  notions}

We first introduce some notations and def\mbox{}initions according
to \cite{Yong:1} (with some differences).

  Consider ,  where


Let  be a permutation of .
 is an  matrix for
which  and 0 in
all other positions (Permutation matrix).

\begin{definition}
\emph{  square matrix  is
defined as follows:
 
And the set that consists of all  is denoted by
 (there are  elements in ) and called the weighted
dif\mbox{}ference
 substitution matrix set.   Accordingly, the set of
linear transformations
 is called  the weighted dif\mbox{}ference  substitution
set, which consists of  substitutions, where , and ,  are
respectively the transposes of , . }
 \end{definition}

 \begin{definition}\label{zcyydh}
\emph{ The
 set of
linear transformations

is called the -times successive weighted dif\mbox{}ference
substitution set, which consists of   substitutions.}
\end{definition}


\begin{definition}\emph{Given the form , when  traverse all the permutations of  respectively, we def\mbox{}ine the set

which is called  the -times successive weighted dif\mbox{}ference
substitution set of the form .}
 \end{definition}



\begin{definition}
\emph{  We def\mbox{}ine the sequence of sets
 as follows}

 \end{definition}

It's time to def\mbox{}ine the termination of
, which is directly related
to the positive semi-definite property of the form .

Let ,
and let . Then we write
a form  with degree  as 

 \begin{definition}
\emph{ The form  is called trivially positive if the
coef\mbox{}f\mbox{}icients  of every term
 in   are
nonnegative.  If , then  is called  trivially
negative.}
\end{definition}

\begin{definition}
   \emph{ If the form
     for all ,  then  is called positive semi-def\mbox{}inite ; If
     for all ,
    is called positive def\mbox{}inite on
   ; If there are  and 
such that  and  ,
    is called indef\mbox{}inite on
   . }
\end{definition}

  \begin{lemma}  \label{zpft}
\emph{Given the form  on   , if the form  is
trivially positive, then   is positive semi-def\mbox{}inite on
   ; If the
form  is trivially negative, then   isn't positive
semi-def\mbox{}inite on
   . }
\end{lemma}


\begin{definition}\label{deft}
\emph{Given a form  on   , if there is a positive
integer  such that every element of the set
 is trivially positive, the sequence of sets
 is called positively
terminating;  If there is a positive integer  and a form  such
that  and  is trivially negative, the
sequence of sets  is called
negatively terminating;   The sequence of sets
 is neither positively
terminating nor negatively terminating, then it is called not
terminating.}
 \end{definition}

  By Definition 2.7, it's easy to get the following lemma.

    \begin{lemma}
\emph{Given the form  on   , if the sequence of
sets  is positively
terminating, then   is positive semi-def\mbox{}inite on
   ; If the
sequence of sets  is
negatively terminating, then   isn't positive
semi-def\mbox{}inite on
   . }
\end{lemma}


Next, we will give the def\mbox{}inition of the normalized
substitution.

\begin{definition}
   \emph{Let  be an  matrix. If ,  is called a normalized matrix.  And the corresponding substitution
   }
\emph{is called a normalized substitution.}
    \end{definition}

\begin{lemma}  \label{zc}
\emph{Let . If
 are normalized matrices, then  is a
normalized matrix , that is,
 
 And let
 ,
 then
  iff
 .}
\end{lemma}
 The proof of Lemma  \ref{zc} is very straightforward and is omitted.

The -dimensional simplex  is defined as follows


\begin{definition}
   \emph{ If the form
     for all ,  then  is called positive semi-definite on
   ;
    If
     for all , then
    is called positive definite on
   ; If there are  and 
such that  and  ,  then
    is called indefinite on
   . }
\end{definition}

     Obviously, we have the following conclusion.
 \begin{lemma}  \label{abcd}
\emph{The form  is positive semi-definite (positive definite,
indefinite) on
    iff  is positive semi-definite (positive definite,
indefinite) on
   .}
\end{lemma}

According to Lemma \ref{abcd}, for brevity, we suppose that the form
 is defined  on  in the remainder of this paper.

\section{ Convergence
 of the sequence of successive weighted dif\mbox{}ference substitution sets.}
 In this section, we'll consider the  weighted dif\mbox{}ference
substitutions from geometrical
 views.


 Let . Consider the weighted dif\mbox{}ference substitution
  
where  is denoted by (\ref{eee}).

 By (\ref{yangyaos}), if
, then ; If
, then ;
; If , then
. Moreover,  is the barycenter of the
(k-1)-dimentional proper face of  which contains the
points  for  .  According to
Lemma \ref{zc}, for all , we
have . Therefore,  is a
subsimplex of the first barycentric subdivision of ,
satisfying
.
And it is indicated that the weighted dif\mbox{}ference substitution
(\ref{yangyaos}) and  the subsimplex  correspond
to each other.

 Analogously,  the other  weighted dif\mbox{}ference substitutions correspond
 to  the other
  subsimplexes of the first barycentric subdivision of .

  Hence, from geometrical
 views, the weighted dif\mbox{}ference substitution set corresponds to
  the f\mbox{}irst barycentric subdivision of , that is,  there is a one-to-one correspondence between the weighted dif\mbox{}ference substitutions and the  subsimplexes of the first
barycentric subdivision of .
 \begin{figure}[H]
\begin{center}
\includegraphics[width=0.25\textwidth]{fig1}
 \caption{Barycentric subdivision}\label{Fig1}
\end{center}
\end{figure}

  For instance, when , the following six weighted dif\mbox{}ference substitution
  matrices correspond to subsimplexes labeled as 1-6 in Fig. 1,
  respectively,







By Lemma \ref{zc} and Def\mbox{}inition \ref{zcyydh}, we have that
the -times successive weighted dif\mbox{}ference  substitution
set corresponds to the -th barycentric subdivision of
.




  Next, we'll introduce the concept of
convergence of the sequence of  substitution sets.


\begin{definition}\label{sld1} \emph{ Let  be a subsimplex of ,  the maximum distance between
 vertexs of 
is called the diameter of .}
\end{definition}

\begin{definition}\label{sld0}
\emph{ Let , and  be the subdivision of
 for   If for all , there
exists  such that all the diameters of the
subsimplexes
 of  are less than , the subdivision sequence
 is called convergent.}
\end{definition}

\begin{definition}\label{sld5}
\emph{Suppose that the subdivision scheme through which  is
subdivided into  corresponds to the substitution set 
for
 . If the sequence  is convergent, the
 sequence of substitution sets
 is called convergent. If ,
, brief\mbox{}ly, we say the sequence of the
successive -substitution sets is convergent.}
\end{definition}
It's time to consider  the convergence of the sequence of the
successive weighted dif\mbox{}ference substitution sets.
\begin{lemma}\label{bs}
\emph{\cite{Edwin:1, James:1} Let  be a complex. If  is the
-th barycentric subdivision of , then for all ,
there exists  such that all the diameters of the
subsimplexes
 of  are less than . }
\end{lemma}

 By Lemma \ref{bs} , we have the following theorem,  which plays important
 roles
 in the proofs of Theorem \ref{zd} and Theorem \ref{ce} in the next sections.

\begin{theorem}
\emph{The   barycentric subdivision
 sequence  of  and the corresponding sequence of
the successive weighted dif\mbox{}ference substitution sets

 are convergent.} \label{sl}
\end{theorem}




\section{Nonnegativity decision of forms }
Given a form  on . We know that if the sequence of
sets  is negatively
terminating, then we can conclude that   is positive
semi-definite on . Thus there is a natural question,
that is, which kind of forms can be solved by the method? Yao
\cite{Yong:1} proved that, for a positive definite form  on
, the sequence of sets
 is positively terminating.

In this section, we present a new, simpler method to prove the
conclusion, which is based on the convergence
 of the sequence of the successive weighted dif\mbox{}ference substitution sets.

\begin{theorem} \label{zd}
\emph{Let the form   be positive def\mbox{}inite on
, then the sequence of  sets
 is positively terminating.}
\end{theorem}
\begin{proof} We only give the proof for the ternary form with
degree , and the multivariate form can be gotten by induction.

Suppose that
. An
arbitrary -times successive weighted dif\mbox{}ference
substitution can be written as

where 
and .


Let , then (\ref{eq11})  becomes

Thus

where


Obviously,

And  since   is positive def\mbox{}inite on
,  there exists  such that



 On the one hand,   the  vertexs of the subsimplex which corresponds to the  successive weighted dif\mbox{}ference
 substitution (\ref{eq11})  are respectively
 
By Theorem (\ref{sl}),
  can be suf\mbox{}f\mbox{}iciently small
 when  is  suf\mbox{}f\mbox{}iciently  large.


 On the other hand, for  is continuous on  and by
 (\ref{aaa})-(\ref{ccc}), we have  when  are suf\mbox{}f\mbox{}iciently
 small.

Putting together the above two aspects,  we have that  there exists
a  suf\mbox{}f\mbox{}iciently  large integer   such that 
becomes trivially positive by (\ref{eq11}).
 For the successive weighted dif\mbox{}ference substitution (\ref{eq11}) is arbitrary, the sequence of  sets  is
positively terminating. \end{proof}

According to the proof of Theorem  \ref{zd}, we obtain the following
conclusion.

\begin{corollary}\label{zd111}
\emph{Let the form  be positive definite on ,  then
by an arbitrary -times successive weighted dif\mbox{}ference
substitution,  when  is suf\mbox{}f\mbox{}iciently larger,  
can become a nonlacunary  trivially positive  form.} \label{gzd}
\end{corollary}

Theorem  \ref{zd} is somewhat analogous to  P\'{o}lya's Theorem.
However,  many examples show that P\'{o}lya's Theorem seems almost
useless to positive semi-definite forms except for few cases, while
SDS is demonstrated very helpful to positive semi-definite ones as
well.


\section{Decision of indef\mbox{}inite forms}

Many problems,  such  as the inequality disproving, are always
transformed into decision of indef\mbox{}inite forms.

 Given a form
 on . Suppose that there exists , such that . It is well-known to us that if
the sequence of sets  is
negatively terminating, then   is indef\mbox{}inite on
. Then it follows a question naturally: for an
indef\mbox{}inite form  on , is the
 negatively terminating? The
following theorem answers the question.

\begin{theorem} \label{ce}
\emph{Let the form  be indef\mbox{}inite on ,  then
the sequence of sets  is
negatively terminating.}
\end{theorem}

\begin{proof}  Since the form  is indef\mbox{}inite on
   ,  there exists  such that  .  And  is continuous on
,  so there exists a neighborhood
 of  (If  is on the
boundary of , then we take ) such that   for all .
For the  barycentric subdivision
 sequence of  is
convergent, then there exists a subsimplex   of the -th
barycentric subdivision of , which corresponds to the
-times successive weighted dif\mbox{}ference substitution

 
 satisfying , where  is a  suf\mbox{}f\mbox{}iciently
larger integer, and  is the weighted dif\mbox{}ference
 substitution matrix set.  Thus,  is positive definite on
 . By Theorem  \ref{zd},
the sequence of  sets  is
positively terminating, so there exists an
 -times successive
weighted dif\mbox{}ference substitution
 
 satisfying that 
 is trivially negative. Therefore, the sequence of sets  is
negatively terminating.\end{proof}






By the proving process of Theorem \ref{ce},  we obtain the following
algorithm,  which is used to decide the  nonnegativity of a form or
to decide an indef\mbox{}inite form with a
counter-example.\\
\textbf{Algorithm} \textbf{(YYS)}\\
Input: the form  , where   is positive def\mbox{}inite or indef\mbox{}inite on  .  \\
Output:  ``The form  is positive semi-def\mbox{}inite'', or ``, ''.\\
step1: Let  .\\
step2:  Compute ,

\qquad Let


\qquad where

 step3: Let .

\quad step31:  If  is null, then output ``the form 
is positive semi-definite'', and terminate.

\quad step32:  If there is a trivially negative form
, then output


 ~~~~~~~~~~~~  and terminate, where

   \quad \quad \quad \quad\qquad


 ~~~~~~~~~~~~   is the -th component of ,  is the -th component
 of , and  is the

  ~~~~~~~~~~~~  the total number of the components of .

\quad step33:  Else, Compute
.  Let




~~~~~~~~~~~~  where


~~~~~~~~~~~~   and  extracts operands from
. And let


~~~~~~~~~~~~ then go to step3.

By Algorithm YYS, we design  a Maple program called YYS, see
Appendix. To the program YYS,  there are some positive semi-definite
forms making the program do not terminate, that is, we cann't decide
these positive semi-definite forms by the method.


\section{Examples}


In this section, we demonstrate the program  YYS with some examples.


 \textbf{Example 1.} Show that the following form is positive semi-definite on
 ,
 

Utilize the program YYS and execute order YYS(,[x,y,z]).  The
procedure need successively run the weighted dif\mbox{}ference
substitutions 3 times, then outputs:``The form  is positive
semi-definite''.


\textbf{Example 2.} Show that the following form is
indef\mbox{}inite on .
 

   Executing order YYS(, [x,y,z]), we have a
counter-example by successively running the weighted
dif\mbox{}ference substitutions 2 times:
 
Obviously, , so the  form  is indef\mbox{}inite on
.


\textbf{Example 3.} Let , and . Try to decide whether the following inequality holds for all
.



Take of\mbox{}f denominators of the left polynomial, and denote the
new polynomials by ,,,  for , respectively. Execute order YYS(, [x,y,z]),
.  For , the procedure only need run
the weighted dif\mbox{}ference substitutions 1 time, then outputs:
``The form  is positive semi-definite''.  For ,  we have a
counter-example by successively running the weighted
dif\mbox{}ference substitutions 5 times:
  So
the  inequality cann't hold for .



\begin{thebibliography}{99}
\bibitem{P:1}
Parrilo P A. Structured semidefinite programs and semialgebraic
geometry methods in robustness and optimization. PhD Thesis, Calif.
Inst. Tech, Pasadena,  2000.

\bibitem{P:2}
Parrilo P A. Semidefinite programming relaxations for semialgebraic
problems. Math. Prog.  2003, Ser. B, 96(2), 293-320.

\bibitem{J:1}
Lasserre J  B.  Global optimization with polynomials and the problem
of moments. SIAM J. Opt. 2001, 11(3), 796-817.

\bibitem{Po:1}
P\'{o}lya G., Szego G. Problems and Theorems in Analysis(Vol.2),
 New York Berlin Heideberg: Springer-Verlag, 1972.

\bibitem{G:1}
Hardy G H., Littlcwood J E.,  P\'{o}lya G. Inequalities [M], Camb
Univ.  Press, 1952.

\bibitem{Ca:1}
 Catlin D W., D'Angelo J P.  Positivity conditions for
bihomogeneous polynomials.  Math. Res. Lett. 1997, 4, 555-567.

\bibitem{H:2}
 Handelman D. Deciding eventual positivity of polynomials. Ergod. Th. \& Dynam.  Sys. 1986, 6, 57-79.

\bibitem{Yang:1}
Yang L. Solving Harder Problems with Lesser Mathematic. Proceedings
of the 10th Asian Technology Conference in Mathematics, ATCM Inc,
2005, 37-46.

\bibitem{Yang:2}
Yang L. Difference Substitution and Automated Inequality Proving.
Journal of Guangzhou University, Natural Science Edition, 2006,
5(2), 1-7. (in Chinese)


\bibitem{Yang:3}Yang L., Xia B C.  Automated Proving and Discoverering on Inequalities. Science
Press, Beijing, 2008. (in Chinese)



\bibitem{Yong:1}
Yao Y. Termination of the Sequence of SDS Sets and Machine Decision
for Positive Semi-definite Forms.  arXiv: 0904.4030.






\bibitem{Edwin:1}
Edwin H. Spanier. Algebraic Topology. Springer-Verlag New York,
Inc., 1966.

\bibitem{James:1} James R. Munkres,   Elements of Algebraic Topology,
Addison Wesley Publishing Company, 1984.



\end{thebibliography}


\subsection*{ Appendix.  Maple Program YYS}
\begin{maplettyout}
 YYS:=proc(poly,var)
 local a,b,A,f,i,j,k,m,n,r,s,t,F,G,H,M,W,newvar,st,Var:
 uses combinat, LinearAlgebra:
 F:=[[poly,[0]]]:
 n:=nops(var):
 Var:=convert(var,Vector):
 W:=(n)->Matrix(n,n,(i,j)->`if`(i<=j,1/j,0)):
 b:=permute(n): a:=W(n):
 A:=[seq(<seq(a[b[i][j]],j=1..n)>,i=1..n!)]:
 for i to nops(A) do
     for j to n do
         newvar[i,j]:=op(j,convert(A[i].Var,list)):
     od:
 od:
 r:=100:
 for s to r do
    m:=nops(F):
    f:=[]:
    for k to m do
        G[k]:=[]:
        for i from 1 to nops(A) do
            st:={seq(Var[j]=newvar[i,j],j=1..n)}:
            G[k]:=[op(G[k]),[expand(subs(st,F[k][1])),[op(F[k][2]),i]]]:
        od:
    od:
    F:=[seq(op(G[u]),u=1..nops(F))]:
    for i to nops(F) do
        if max([coeffs(F[i][1])]) < 0 then
            print(F[i][2]):
            M:=IdentityMatrix(n):
            for j from 2 to nops(F[i][2]) do
                M:=M.A[F[i][2][j]]:
            od:
            print(convert(M.Vector[column](n,1/n),list)):
            return ("The form is indefinite"):
       elif min([coeffs(F[i][1])]) > 0 then
            f:=[op(f),i]:
        fi:
    od:
    if nops(f)>0 then
        F:=subs({seq(F[f[t]]=NULL,t=1..nops(f))},F):
    fi:
    if nops(F)=0 then
        print(s):
        return("The form is positive semi-definite"):
    fi:
 od:
 end proc:

\end{maplettyout}



 \clearpage
\end{document}
