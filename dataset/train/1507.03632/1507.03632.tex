\subsection{One dominant oscillation}\label{sec:oneOsc}



\begin{lemma}\label{lem: taylor}
Let $A, B, a, b, r \in \ra$ where $a, b, r > 0$. Let
$\varphi_1,\varphi_2\in\mathbb{R}$ be such that 
$e^{i\varphi_1}, e^{i\varphi_2}\in\lalg$. Suppose
also that $a, b$ are linearly dependent over $\rats$ and that
whenever $1-\cos(at + \varphi_1) = 0$, it holds that 
$A\cos(bt + \varphi_2) + B > 0$. Define the function
\[ f(t) = 1-\cos(at + \varphi_1) + e^{-rt}(A\cos(bt + \varphi_2) + B). \]
Then $f(t) = \Omega(e^{-rt})$, that is, there exist effective constants
$T\geq 0$ and $c > 0$ such that for $t \geq T$, we have $f(t)\geq ce^{-rt}$.
\end{lemma}
\begin{proof}
The case of $A=0$ is easy: by the premise of the Lemma, we
have $B>0$ and then $f(t)\geq Be^{-rt}$ for all $t$.
Thus, assume $A\neq 0$ throughout.
Let the linear dependence between $a, b$ be given by $an_1 - bn_2 = 0$
for $n_1,n_2\in\mathbb{N}$ coprime and let
$\mathcal{C}$ be the equivalence class of $-\varphi_1/a$ 
modulo $2\pi/a$, that is,
\[
\mathcal{C} \defn \left\{ \frac{-\varphi_1 + 2k\pi}{a} \, \middle| \, k\in\mathbb{Z} \right\}.
\]
We will refer to $\mathcal{C}$ as the set of \emph{critical points}
throughout. 

It is clear that at critical points, we have $1-\cos(at +
\varphi_1)=0$. Moreover, the linear dependence of $a,b$
entails that for each fixed value of $(\cos(at),\sin(at))$,
there are only finitely many possible values for $(\cos(bt),\sin(bt))$.
Indeed, we have
\[ e^{ibt}\in\{\omega e^{iatn_1} \, | \, \omega\mbox{ an $n_2$-th root of unity} \}, \]
so in particular, for $t\in\mathcal{C}$, we have
\[
e^{ibt}\in\{\omega e^{-in_1\varphi_1}\, |\, \omega\mbox{ an $n_2$-th root of unity} \}.
\]
Thus, the possible values of $(\cos(bt), \sin(bt))$
for $t$ critical are algebraic and effectively computable.
Let $M \defn \min\{ A\cos(bt+\varphi_2)+B\,|\,t\in\mathcal{C}\}$.
By the premise of the Lemma, we have $M>0$.

Let $t_1,t_2,\dots,t_j,\dots$ be the non-negative critical points.
Note that by construction we have $|t_j - t_{j-1}| = 2\pi/a$.
For each $t_j$, define the \emph{critical region} to be the interval
$[t_j-\delta, t_j+\delta]$, where
\[ \delta\defn \frac{M}{2|A|b}. \]

Let $g(t)\defn A\cos(bt+\varphi_2)+B$ and notice that $g'(t)\leq |A|b$
everywhere. We first prove the claim for $t$ inside critical
regions: suppose $t$ lies in a critical region and let $j$ minimise 
$|t-t_j|\leq\delta$. Then by the Mean Value Theorem, we have 
\[ 
|g(t) - g(t_j)| \leq |t-t_j| |A|b \leq \delta|A|b = \frac{M}{2},
\]
so 
\[ g(t)\geq g(t_j) - \frac{M}{2} \geq \frac{M}{2}, \]
whence $f(t) \geq e^{-rt}g(t) \geq Me^{-rt}/2 = \Omega(e^{-rt})$.


Now suppose $t$ is outside all critical regions and let $j$
minimise $|t-t_j|$. Since the distance between critical 
points is $2\pi/a$ by construction, we have $a|t-t_j|\leq \pi$. 
Therefore,
\begin{align*}
1-\cos(at + \varphi_1) & = 1-\cos(at - at_j) \geq 
\frac{|a(t-t_j)|^2}{2} \\
& > \frac{(a\delta)^2}{2} = \frac{a^2M^2}{8|A|^2b^2}>0.
\end{align*}
Thus, there exists a computable constant $D>0$ such that 
$f(t)=1-\cos(at+\varphi_1) + e^{-rt}g(t)\geq D$
for all large enough $t$ outside critical regions.

Combining the two results, we have $f(t) = \Omega(e^{-rt})$
everywhere.
\end{proof}



\begin{lemma}\label{lem: layered}
Let $C, D, a, b, r_1, r_2$ be real algebraic numbers such that 
$a, b, r_1, r_2 > 0$ and $C, D$ are not both $0$.
Let also $\varphi_1,\varphi_2\in\mathbb{R}$ be
such that
$e^{i\varphi_1},e^{i\varphi_2}\in\lalg$.  
Define the exponential polynomial $f$ by
\begin{align*}
f(t) = \; & 1-\cos(at + \varphi_1) \\
& + e^{-r_1t}(C\cos(bt+\varphi_2) + D) + e^{-(r_1+r_2)t}F(t).
\end{align*}
Here $F$ is an exponential polynomial whose
dominant characteristic roots are purely imaginary. Suppose also
that $f$ has order at most $7$.
Then it is decidable whether $f$ has infinitely many
zeros. 
\end{lemma}
\begin{proof}
Notice that the dominant term of $f$ is always non-negative, so the
function is positive for arbitrarily large $t$. Thus, $f(t)=0$ for
some $t$ if and only if $f(t)\leq 0$ for some $t$, and analogously,
$f$ has infinitely many zeros if and only if $f(t)\leq 0$
infinitely often. We can eliminate the 
case $|D|>|C|$, since then $f$ is clearly ultimately positive 
or oscillating, depending on the sign of $D$. 
Thus, we can assume $|D|\leq|C|$.


We now consider two cases, depending on whether $a/b \in \rats$.

\emph{Case I.}
Suppose first that $a, b$ are linearly independent over
$\mathbb{Q}$. By Lemma \ref{lem: density}, the trajectory
$(at+\varphi_1 \bmod 2\pi,bt + \varphi_2\bmod 2\pi)$ 
is dense in $[0,2\pi)^2$, and moreover the restriction
of this trajectory to $at + \varphi_1 \bmod 2\pi = 0$ is
dense in $\{0\}\times [0, 2\pi)$. 

If $|D| < |C|$, then we argue that $f$ is infinitely often 
negative, and hence has infinitely many zeros.
Indeed, $|D| < |C|$ entails the existence of a non-trivial interval 
$I\subseteq [0,2\pi)$ such that 
\[ t\bmod 2\pi\in I \Rightarrow C\cos(bt+\varphi_2) + D < 0. \]
What is more, we can in fact find $\epsilon>0$ and a subinterval 
$I' \subseteq I$ such that
\[ t\bmod 2\pi\in I' \Rightarrow C\cos(bt+\varphi_2)+D < -\epsilon. \] 
Thus, by density, 
$1 - \cos(at + \varphi_1)=0$ and $C\cos(bt+\varphi_2) + D < -\epsilon$
will infinitely often hold simultaneously. Then just take $t$ large enough to 
ensure, say, $|e^{-r_2t}F(t)| < \epsilon/2$ at these infinitely
many points, and the claim follows.

Thus, suppose now $|C|=|D|$. Replacing $\varphi_2$ by $\varphi_2+\pi$ 
if necessary, we can write the function as:
\begin{align*}
f(t) = \; & 1-\cos(at + \varphi_1) \\
& + De^{-r_1t}(1-\cos(bt+\varphi_2)) + e^{-(r_1+r_2)t}F(t).
\end{align*}
As $a,b$ are linearly independent, 
for all $t$ large enough, $1-\cos(at+\varphi_1)$ and $1-\cos(bt+\varphi_2)$
cannot simultaneously be `too small'. More precisely, by Lemma \ref{lem: twoCosBaker}, 
there exist effective constants $E, T, N>0$ such that for all $t\geq T$,
we have
\begin{align*}
1-\cos(at+\varphi_1) > E / t^N \mbox{ or }  
1-\cos(bt + \varphi_2) > E / t^N. \\
\end{align*}
Now, if $D<0$, it is easy
to show that $f$ has infinitely many zeros. Indeed, consider the
times $t$ where the dominant term $1-\cos(at + \varphi_1)$
vanishes. For all large enough such $t$, since $t^{-N}$ shrinks more 
slowly than $e^{-r_2t}$, we will have 
\begin{align*}
f(t) & = e^{-r_1t}D(1-\cos(bt + \varphi_2)) + e^{-(r_1+r_2)t}F(t)  \\
& < e^{-r_1t}( EDt^{-N} + e^{-r_2t}F(t) )  \\
& \leq e^{-r_1t}\frac{1}{2}EDt^{-N} \\
& < 0, \\
\end{align*}
so $f$ has infinitely many zeros.
Similarly, if $D>0$, we can show that $f$ is
ultimately positive.  Indeed, for all $t$ large enough, 
we have
\begin{align*}
f(t) & \geq e^{-r_1t}D(1-\cos(bt+\varphi_2)) + e^{-(r_1+r_2)t}F(t) \\
& > e^{-r_1t}DE t^{-N} + e^{-(r_1+r_2)t}F(t) \\
& > 0, \\
\end{align*}
or
\begin{align*}
f(t) & \geq 1-\cos(at+\varphi_1) + e^{-(r_1+r_2)t}F(t) \\
& > E t^{-N} + e^{-(r_1+r_2)t}F(t) \\
& > 0. \\
\end{align*}
Therefore, $f$ has only finitely many zeros.


\emph{Case II.} Now suppose $a,b$ are linearly dependent. By the premise of
the Lemma, the order of $F$ is at most $2$ (in fact, 
at most $1$ if $D\neq 0$). However, by Theorem \ref{thm: mtargument}, the claim follows immediately
for all cases in which the characteristic roots of $F$ are all real
or complex but with frequencies linearly dependent on $a$. Thus, the only remaining
case to consider is the function
\begin{align*} 
f(t) = \; & 1-\cos(at+\varphi_1) \\
& + e^{-r_1t}C\cos(bt+\varphi_2) + e^{-(r_1+r_2)t}H\cos(ct+\varphi_3), 
\end{align*}
where $H,c\in\ra$, $c > 0$ and $a/c\not\in\rats$.

As explained at the beginning of the proof of Lemma \ref{lem: taylor}, due to the linear
dependence of $a, b$ over $\rats$, when $1-\cos(at+\varphi_1)=0$, there are only finitely
many possibilities for the value of $C\cos(bt+\varphi_2)$, each algebraic, effectively
computable and occurring periodically. If at least one of these values is non-positive,
then by the linear independence of $a,c$ over $\rats$, we will simultaneously have 
$1-\cos(at+\varphi_1) = 0$, $C\cos(bt+\varphi_2)\leq 0$ and $H\cos(ct+\varphi_3) < 0$
infinitely often, which yields $f(t)<0$ infinitely often and entails the existence of
infinitely many zeros. On the other hand, if at the critical points $1-\cos(at+\varphi_1) = 0$
we always have $C\cos(bt+\varphi_2) > 0$, then by Lemma \ref{lem: taylor}, we have 
\[ 1-\cos(at+\varphi_1) + e^{-r_1t}C\cos(bt+\varphi_2) = \Omega(e^{-r_1t}), \]
whereas obviously 
\[ \left|e^{-(r_1+r_2)t}H\cos(ct+\varphi_3)\right| = \bigoh(e^{-(r_1+r_2)t}). \]
If follows that $f$ is ultimately positive and hence has only finitely many zeros.
\end{proof}


\begin{lemma}\label{lem: oneOscTwo}
Let $A, B, a, b, c, r$ be real algebraic numbers such that 
$a, b, c, r > 0$,  $A, B \neq 0$.
Let also $\varphi_1,\varphi_2,\varphi_3\in\re$ be
such that
$e^{i\varphi_1},e^{i\varphi_2},e^{i\varphi_3}\in\lalg$.  
Define the exponential polynomial $f$ by
\[
f(t) = 1-\cos(ct + \varphi_3) + e^{-rt}(A\cos(at+\varphi_1) + B\cos(bt+\varphi_2)).
\]
Then it is decidable whether $f$ has infinitely many
zeros. 
\end{lemma}
\begin{proof}
We argue the function is infinitely often positive and infinitely often negative
by looking at the values of $t$ for which the dominant term $1-\cos(ct+\varphi_3)$
vanishes. This happens precisely at the times $t = -(\varphi_3+2k\pi)/c$ for $k\in\zed$,
giving rise to a discrete restriction of $f$:
\begin{align*}
g(k) \defn e^{r\varphi_3}\left(e^{2\pi r}\right)^k
\left(
A\cos\left( k\frac{2\pi a}{c} - \frac{a\varphi_3}{c}+\varphi_1 \right) +\right.\\
\left.
B\cos\left( k\frac{2\pi b}{c} - \frac{b\varphi_3}{c}+\varphi_2 \right)
\right).
\end{align*}
This is a linear recurrence sequence over $\re$ of order $4$, 
with characteristic roots $e^{2\pi(r\pm ia/c)}$ and $e^{2\pi(r\pm ib/c)}$. 
In particular, it has no real dominant characteristic root. It is well-known 
that real-valued  linear recurrence sequences with no dominant real 
characteristic root are infinitely often positive and infinitely often negative:
see for example \cite[Theorem 7.1.1]{gyori}. Therefore, by continuity, 
$f$ must have infinitely many zeros.
\end{proof}


\begin{lemma}\label{lem: oneOscOneRep}
Let $A, B, a, b, r$ be real algebraic numbers such that 
$a, b, r > 0$,  $A \neq 0$.
Let also $\varphi_1,\varphi_2,\varphi_3\in\re$ be
such that
$e^{i\varphi_1},e^{i\varphi_2},e^{i\varphi_3}\in\lalg$.  
Define the exponential polynomial $f$ by
\[
f(t) = 1-\cos(at + \varphi_1)
+ e^{-rt}(At\cos(bt + \varphi_2) + B\cos(bt+\varphi_3)).
\]
Then it is decidable whether $f$ has infinitely many
zeros. 
\end{lemma}
\begin{proof}
If $a/b\in\rats$, then the claim follows immediately from Theorem
\ref{thm: mtargument}. If $a/b\not\in\rats$, then by Lemma \ref{lem: density},
it will happen infinitely often that $1-\cos(at+\varphi_1) = 0$ and
$At\cos(bt+\varphi_2) < -|A|t/2$. Then clearly $f(t) < 0$ infinitely often.
Since $f(t)>0$ infinitely often as well, due to the non-negative dominant
term $1-\cos(at+\varphi_1)$, it follows that $f$ has infinitely many zeros.
\end{proof}














