\subsection{One dominant oscillation}\label{sec:oneOsc}



\begin{lemma}\label{lem: taylor}
Let  where . Let
 be such that 
. Suppose
also that  are linearly dependent over  and that
whenever , it holds that 
. Define the function

Then , that is, there exist effective constants
 and  such that for , we have .
\end{lemma}
\begin{proof}
The case of  is easy: by the premise of the Lemma, we
have  and then  for all .
Thus, assume  throughout.
Let the linear dependence between  be given by 
for  coprime and let
 be the equivalence class of  
modulo , that is,

We will refer to  as the set of \emph{critical points}
throughout. 

It is clear that at critical points, we have . Moreover, the linear dependence of 
entails that for each fixed value of ,
there are only finitely many possible values for .
Indeed, we have

so in particular, for , we have

Thus, the possible values of 
for  critical are algebraic and effectively computable.
Let .
By the premise of the Lemma, we have .

Let  be the non-negative critical points.
Note that by construction we have .
For each , define the \emph{critical region} to be the interval
, where


Let  and notice that 
everywhere. We first prove the claim for  inside critical
regions: suppose  lies in a critical region and let  minimise 
. Then by the Mean Value Theorem, we have 

so 

whence .


Now suppose  is outside all critical regions and let 
minimise . Since the distance between critical 
points is  by construction, we have . 
Therefore,

Thus, there exists a computable constant  such that 

for all large enough  outside critical regions.

Combining the two results, we have 
everywhere.
\end{proof}



\begin{lemma}\label{lem: layered}
Let  be real algebraic numbers such that 
 and  are not both .
Let also  be
such that
.  
Define the exponential polynomial  by

Here  is an exponential polynomial whose
dominant characteristic roots are purely imaginary. Suppose also
that  has order at most .
Then it is decidable whether  has infinitely many
zeros. 
\end{lemma}
\begin{proof}
Notice that the dominant term of  is always non-negative, so the
function is positive for arbitrarily large . Thus,  for
some  if and only if  for some , and analogously,
 has infinitely many zeros if and only if 
infinitely often. We can eliminate the 
case , since then  is clearly ultimately positive 
or oscillating, depending on the sign of . 
Thus, we can assume .


We now consider two cases, depending on whether .

\emph{Case I.}
Suppose first that  are linearly independent over
. By Lemma \ref{lem: density}, the trajectory
 
is dense in , and moreover the restriction
of this trajectory to  is
dense in . 

If , then we argue that  is infinitely often 
negative, and hence has infinitely many zeros.
Indeed,  entails the existence of a non-trivial interval 
 such that 

What is more, we can in fact find  and a subinterval 
 such that
 
Thus, by density, 
 and 
will infinitely often hold simultaneously. Then just take  large enough to 
ensure, say,  at these infinitely
many points, and the claim follows.

Thus, suppose now . Replacing  by  
if necessary, we can write the function as:

As  are linearly independent, 
for all  large enough,  and 
cannot simultaneously be `too small'. More precisely, by Lemma \ref{lem: twoCosBaker}, 
there exist effective constants  such that for all ,
we have

Now, if , it is easy
to show that  has infinitely many zeros. Indeed, consider the
times  where the dominant term 
vanishes. For all large enough such , since  shrinks more 
slowly than , we will have 

so  has infinitely many zeros.
Similarly, if , we can show that  is
ultimately positive.  Indeed, for all  large enough, 
we have

or

Therefore,  has only finitely many zeros.


\emph{Case II.} Now suppose  are linearly dependent. By the premise of
the Lemma, the order of  is at most  (in fact, 
at most  if ). However, by Theorem \ref{thm: mtargument}, the claim follows immediately
for all cases in which the characteristic roots of  are all real
or complex but with frequencies linearly dependent on . Thus, the only remaining
case to consider is the function

where ,  and .

As explained at the beginning of the proof of Lemma \ref{lem: taylor}, due to the linear
dependence of  over , when , there are only finitely
many possibilities for the value of , each algebraic, effectively
computable and occurring periodically. If at least one of these values is non-positive,
then by the linear independence of  over , we will simultaneously have 
,  and 
infinitely often, which yields  infinitely often and entails the existence of
infinitely many zeros. On the other hand, if at the critical points 
we always have , then by Lemma \ref{lem: taylor}, we have 

whereas obviously 

If follows that  is ultimately positive and hence has only finitely many zeros.
\end{proof}


\begin{lemma}\label{lem: oneOscTwo}
Let  be real algebraic numbers such that 
,  .
Let also  be
such that
.  
Define the exponential polynomial  by

Then it is decidable whether  has infinitely many
zeros. 
\end{lemma}
\begin{proof}
We argue the function is infinitely often positive and infinitely often negative
by looking at the values of  for which the dominant term 
vanishes. This happens precisely at the times  for ,
giving rise to a discrete restriction of :

This is a linear recurrence sequence over  of order , 
with characteristic roots  and . 
In particular, it has no real dominant characteristic root. It is well-known 
that real-valued  linear recurrence sequences with no dominant real 
characteristic root are infinitely often positive and infinitely often negative:
see for example \cite[Theorem 7.1.1]{gyori}. Therefore, by continuity, 
 must have infinitely many zeros.
\end{proof}


\begin{lemma}\label{lem: oneOscOneRep}
Let  be real algebraic numbers such that 
,  .
Let also  be
such that
.  
Define the exponential polynomial  by

Then it is decidable whether  has infinitely many
zeros. 
\end{lemma}
\begin{proof}
If , then the claim follows immediately from Theorem
\ref{thm: mtargument}. If , then by Lemma \ref{lem: density},
it will happen infinitely often that  and
. Then clearly  infinitely often.
Since  infinitely often as well, due to the non-negative dominant
term , it follows that  has infinitely many zeros.
\end{proof}














