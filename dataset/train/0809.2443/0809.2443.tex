\documentclass[a4paper,12pt]{article}
\usepackage{enumerate,amsmath,amsfonts,amssymb,amsthm,mathrsfs}
\usepackage{epsfig,subfigure,graphicx,setspace,latexsym,color}
\usepackage{indentfirst,dsfont}
\usepackage{natbib}
\usepackage{setspace}
\usepackage{latexsym}
\usepackage[all]{xy}
\usepackage{graphicx}

\theoremstyle{plain}
\newtheorem{theorem}{Theorem}[section]
\newtheorem{lemma}[theorem]{Lemma}
\newtheorem{cons}[theorem]{Construction}
\newtheorem{proposition}[theorem]{Proposition}
\newtheorem{coro}[theorem]{Corollary}
\newtheorem{corollary}[theorem]{Corollary}
\theoremstyle{definition}
\newtheorem{definition}[theorem]{Definition}
\newtheorem{example}[theorem]{Example}
\newtheorem{examples}[theorem]{Examples}
\newtheorem{remark}[theorem]{Remark}
\theoremstyle{remark}
\newtheorem{notation}[theorem]{Notation}


\setlength{\textheight} {9.2 in} \setlength{\textwidth} {6.5 in}
\voffset -1 in \hoffset -0.5 in \topmargin .8 in
\setlength{\evensidemargin} {0.6 in} \setlength{\oddsidemargin}{0.6
in} \setlength {\columnsep}{6 mm} \baselineskip 8 mm





\begin{document}
\title{ NP-Completeness of Hamiltonian Cycle Problem on Rooted Directed Path Graphs}
\author{B. S. Panda\thanks{Computer Science and Application Group,
Department of Mathematics, Indian Institute of Technology Delhi,
Hauz Khas, New Delhi 110 016, INDIA, E-mail:
bspanda@maths.iitd.ac.in, dinabandhu.pradhan@mail2.iitd.ac.in}, D.
Pradhan\thanks{This author was supported by Council of Scientific \&
Industrial Research, INDIA.} \\Indian Institute of Technology Delhi}
\date{}
\maketitle

\abstract{ The Hamiltonian cycle problem is to decide whether a
given graph has a  Hamiltonian cycle.  Bertossi and Bonuccelli
(1986, Information Processing Letters, 23, 195-200) proved that the
Hamiltonian Cycle Problem is NP-Complete even for undirected path
graphs and left the Hamiltonian cycle problem open for directed path
graphs. Narasimhan (1989, Information  Processing Letters, 32,
167-170) proved that the Hamiltonian Cycle Problem is NP-Complete
even for directed path graphs and left the Hamiltonian cycle problem
open for rooted directed path graphs. In this paper we  resolve this
open problem by proving that the Hamiltonian Cycle Problem is also
NP-Complete for rooted directed path graphs.
}\\

\noindent {\bf Keywords: Intersection graph, undirected path graph,
directed path graph, rooted directed path graph, NP-Completeness,
Hamiltonian cycle}

\section{Introduction}
Let  be a graph. Let  denote the set of all neighbors
of  in . Let  denote the degree of . A subset
 of  is called a clique of  if , the subgraph of 
induced on , is a complete subgraph of .  is called a
maximal clique if  is a clique but no proper super set of  in
 is a clique in . By the maximum degree of a graph , we
mean the maximum of the degrees of the vertices in . A cycle 
of  is called a {\it Hamiltonian cycle} of  if  contains
all the vertices of . The problem of deciding whether a given
graph  has a Hamiltonian cycle is known as {\bf Hamiltonian cycle
problem}. This problem in general graph is well-known to be
NP-Complete \cite{garey}. It is known to be NP-Complete even when
the inputs are restricted to several classes of interesting special
classes of graphs such as {\it planar cubic -connected graphs}
\cite{dawes}, {\it bipartite graphs} \cite{bipartite}, {\it edge
graphs} (line graphs) \cite{booth}, and {\it chordal graphs}
\cite{stewart}. Bertossi and Bonuccelli \cite{bert} proved that the
Hamiltonian Cycle Problem is NP-Complete even for undirected path
graphs. The Hamiltonian cycle problem for directed path graphs and
circular arc graphs were left open by Bertossi and Bonuccelli
\cite{bert}. Narasimhan \cite{giri} later proved that the
Hamiltonian Cycle Problem is NP-Complete even for directed path
graphs. However, the Hamiltonian cycle problem for circular graphs
was solved   by Shih et al.\cite{hsu} in  time.
The Hamiltonian cycle problem on rooted directed path graph was left
open  by Narasimhan \cite{giri}. The Hamiltonian cycle problem on
rooted directed path graph is also mentioned to be open in
(\cite{spinrad}, page ).

In this paper we resolve this open problem. In fact, we prove that
the Hamiltonian cycle problem is also NP-Complete for rooted
directed path graphs. However, it is worth mentioning that the
Hamiltonian cycle problem can be solved in polynomial time for {\it
-sep rooted directed path graphs} \cite{panda}, a proper subclass
of rooted directed path graphs.

 The rest of the paper is organized as follows. In Section , we
introduce the rooted directed path graphs and present some results
on this class of graphs. In Section , we prove that the
Hamiltonian cycle problem is NP-Complete for  rooted directed path
graphs. We use the techniques similar to those used in
\cite{bert,giri}. The reduction is carried out from the Hamiltonian
Cycle Problem on bipartite graph with maximum  degree , which was
proved to be NP-Complete by Itai et al. \cite{itai}.

\section{Rooted directed path graphs}

Two paths, say  and , in a tree  are said to intersect
if .

Let  be a finite family of non-empty sets. An
undirected graph  is an intersection graph for  if
there is a one-to-one correspondence between the vertices of  and
the sets in  such that two vertices in  are adjacent
 if and only if their corresponding sets have non-empty intersection.
 If  is a family of paths in an undirected tree ,
then  is called an {\it undirected path} graph. If 
is a family of directed paths in a directed tree , i.e., a tree
in which each edge is oriented, then  is called a {\it directed
path} graph. Note that a directed tree may have more than one vertex
of in-degree zero. A rooted directed tree is a directed tree having
exactly one vertex of in-degree zero. If  is a family
of directed paths in a rooted directed tree , then  is called
a {\it rooted directed path} graph. Undirected path graphs, directed
path graphs, and rooted directed path graphs are also known as {\bf
undirected vertex graphs or UV graphs}, {\bf directed vertex graphs
or DV graphs}, and {\bf rooted directed vertex graphs or RDV
graphs}, respectively (see \cite{monma}).



Note that in the above definition of rooted directed path graph, the
rooted directed tree  is arbitrary. However, Gavril \cite{gav}
characterized the rooted directed path graphs  in terms of a tree
 such that  is the set of all maximal cliques of .

\begin{theorem}[Clique Tree Theorem,\cite{gav,monma}]\label{ctt}

Let  be a graph and let  be the set of all
maximal cliques of . For each vertex , let
 be the set of cliques of  containing
the vertex . Then  is a rooted directed path graph if and only
if there exists a rooted directed tree  with the vertex set
, such that for every , , the
subtree of  induced on , is a directed path in
.



\end{theorem}
The tree  in the above theorem is called the {\bf RDP clique
tree} for .

\section{NP-Completeness }

Consider the following  problem.

\noindent {\bf Problem  :}\\
{\bf Instance} : A bipartite graph  having maximum degree .\\
{\bf Question} : Does  contain a Hamiltonian cycle?

It has been shown in \cite{itai} that the  problem  is
NP-Complete.

\begin{lemma}[\bf Itai et. al \cite{itai}]\label{lemma3}
The problem  is NP-Complete.
\end{lemma}

We will show that the following problem  is NP-complete.

\noindent {\bf Problem  :}\\
{\bf Instance} : A rooted directed path graph .\\
{\bf Question} : Does  contain a Hamiltonian cycle?

The transformation of an instance of problem  to an instance of
the problem  is described below.

Let , a bipartite graph having  vertices with maximum
degree , be an instance of the problem .  Without loss of
generality, we assume that , the vertex sets  and
 both have  vertices each and that  has no vertex with
degree one, since otherwise  has  no Hamiltonian cycle. Let
 and . We
show how to construct an instance of the problem  by showing
how to construct a directed path graph  such that  has a
Hamiltonian cycle if and only if  has a Hamiltonian cycle. We
describe  by describing all its maximal cliques. Note that
describing all the maximal cliques of a graph fully defines the
graph itself.


\begin{cons}\label{cons1}
\noindent Corresponding to each  vertex , ,  construct the clique . Corresponding to each  vertex
 with ,  , construct two
cliques  and . Corresponding to each  vertex  with , ,  construct the  clique .





Note that the cliques mentioned above are the only maximal cliques
in . Hence it is clear that,   

\end{cons}

Figure \ref{1} illustrates the construction of the maximal cliques
of  from a given bipartite graph .

We now prove that the resulting graph  is a rooted directed path
graphs.

\begin{lemma}\label{lemma1}
The graph  constructed by Construction \ref{cons1}, is a rooted
directed path graph.

\end{lemma}


\begin{proof}

Let  be the set of all maximal cliques of . Hence,
 .  Let  be the
directed graph such that . Clearly,  is a rooted directed tree with root .
Figure \ref{1} contains a bipartite graph , the set of maximal
cliques of  constructed by Construction \ref{cons1} and an RDP
clique tree  of   constructed as above.






\begin{figure}
\centerline{\hbox{\epsfig{figure=fig1.eps,height=10cm,width=14cm}}}
\end{figure}

Let  be a vertex of . If  is either  or ,
then  consists of the only one vertex and hence is
a directed path of length zero. If  and , then
 consists of the directed path . If  and
, then  consists of the directed path
. Hence for each
vertex ,  is a directed path in .
Hence by the Theorem \ref {ctt},  is a rooted directed path
graph.
\end{proof}

\begin{lemma}\label{lemma2}

The bipartite graph  contains a Hamiltonian cycle if and only if
  contains a Hamiltonian cycle.

\end{lemma}

\begin{proof}
\noindent {\bf Necessity:}

If  has a Hamiltonian cycle , we obtain a Hamiltonian cycle
 for  as follows. If  are three consecutive
vertices in , we obtain  by substituting the sequence
 if
 ( in this case,  is the third vertex adjacent to
), or with  if
. This results in a Hamiltonian cycle for  since all
the vertices are covered.

\noindent {\bf Sufficiency:}

Let  be a Hamiltonian cycle for .

\noindent{\bf Claim :} A sequence of the form , where  if

\hspace{1cm}, or of the form , where  if ,

\hspace{1cm}must appear in 


\noindent{\bf Proof of Claim :} If , then
.
So, the sequence 
must appear in . If , then  must
appear in  as  and  are the only two neighbors
of  in . We call such a sequence  a {\it -block}. Now
each  such that  is contained in exactly one
-block. There are exactly  distinct 's each appearing in
exactly one clique and there are exactly  -blocks. So, each
-block must appear immediately after an  and must appear
immediately before an  in . If a -block, which starts
with  and ends with ,  appears immediately after
 and appears immediately before  in , then 
and . So, the claim is proved.

Now we obtain  from  as follows. If a -block appears
immediately after  and appears immediately before  in
, we obtain  by substituting the sequence -block, with the sequence . It is easy  to see that the resulting  is a
Hamiltonian cycle for .



\end{proof}
Next we show that the problem  is NP-Complete.

\begin{theorem}
The problem  is NP-Complete.
\end{theorem}
\begin{proof}
Clearly, the problem  is in NP. To show that  is
NP-hard, we use a transformation from the problem .

Consider an instance of , i.e., a bipartite graph 
with maximum degree . Construct the graph  from 
using {\bf Construction \ref{cons1}}. It can be easily verified that
the construction of  from  can be done in polynomial time. By
Lemma \ref{lemma1},  is a rooted directed path graph. Again by
Lemma \ref{lemma2}, there exists a Hamiltonian cycle in  if and
only if there exists a Hamiltonian cycle in . So,  is an `yes'
instance of  if and only if  is an `yes' instance of
. Since, by Lemma \ref{lemma3},  is NP-Complete, 
is also NP-Complete.

\end{proof}

\section{Conclusion} In this paper, we proved that the problem of
deciding whether a given rooted directed path graph  contains a
Hamiltonian cycle is NP-complete. This was an open problem and was
mentioned in \cite{giri} and in (\cite {spinrad}, page ).


 \bibliographystyle{plain}
\begin{thebibliography}{10}

\bibitem{bipartite}
T.~Akiyama, T.~Nishizeki, and N.~Saito.
\newblock N{P}-{C}ompleteness of the {H}amiltonian {C}ycle {P}roblem for
  bipartite graphs.
\newblock {\em J. Inform. Process.}, 3:73--76, 1980.

\bibitem{bert}
A.~A. Bertossi and M.~A. Bonuccelli.
\newblock Hamiltonian circuits in interval generalizations.
\newblock {\em Inform. Process. Lett.}, 23:195--200, 1986.

\bibitem{booth}
K.~S. Booth and G.~S. Lueker.
\newblock Testing for the consecutive ones property, interval graphs, and graph
  planarity using {PQ}-tree algorithms.
\newblock {\em J. Comput. System Sci.}, 13:335--379, 1976.

\bibitem{dawes}
E.~J. Cockayne, R.~M. Dawes, and S.~T. Hedetniemi.
\newblock Total domination in graphs.
\newblock {\em Networks}, 10:211--219, 1980.

\bibitem{stewart}
C.~J. Colburn and L.~K. Stewart.
\newblock Dominating cycles in series-parallel graphs.
\newblock {\em Ars Combin.}, 19A:107--112, 1985.

\bibitem{garey}
M.~R. Garey and D.~S. Johnson.
\newblock {\em Computers and intractability}.
\newblock W. H. Freeman and Co., San Francisco, Calif., 1979.
\newblock A guide to the theory of NP-{C}ompleteness, A Series of Books in the
  Mathematical Sciences.

\bibitem{gav}
F.~Gavril.
\newblock A recognition algorithm for intersection graphs of directed paths in
  directed trees.
\newblock {\em Discrete Mathematics}, 13:237--249, 1975.

\bibitem{itai}
A.~Itai, C.~H. Papadimitriou, and J.~L. Szwarcfiter.
\newblock Hamiltonian paths in grid graphs.
\newblock {\em SIAM J. Comput.}, 11:676--686, 1982.

\bibitem{monma}
C.~L. Monma and V.~K. Wei.
\newblock Intersection graphs of paths in a tree.
\newblock {\em J. Combin. Theory Ser. B}, 41(2):141--181, 1986.

\bibitem{giri}
G.~Narasimhan.
\newblock A note on the {H}amiltonian circuit problem on directed path graphs.
\newblock {\em Inform. Process. Lett.}, 32:167--170, 1989.

\bibitem{panda}
B.~S. Panda, V.~Natarajan, and S.~K. Das.
\newblock Parallel algorithms for {H}amiltonian 2-separator chordal graphs.
\newblock {\em Parallel Process. Lett.}, 12(1):51--64, 2002.

\bibitem{hsu}
W.~K. Shih, T.~C. Chern, and W.~L. Hsu.
\newblock An {} algorithm for the {H}amiltonian cycle problem
  on circular-arc graphs.
\newblock {\em SIAM J. Comput.}, 21(6):1026--1046, 1992.

\bibitem{spinrad}
J.~P. Spinrad.
\newblock {\em Efficient graph representations}, volume~19 of {\em Fields
  Institute Monographs}.
\newblock American Mathematical Society, Providence, RI, 2003.

\end{thebibliography}




\end{document}
