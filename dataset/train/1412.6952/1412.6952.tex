

\documentclass[10pt,twocolumn,twoside]{IEEEtran} 

\IEEEoverridecommandlockouts                              

\usepackage{graphics} \usepackage{epsfig} \usepackage{amsmath} \let\proof\relax
\let\endproof\relax
\usepackage{amssymb,amsthm}  \usepackage{amscd}
\usepackage[noadjust]{cite}
\usepackage{float}
\usepackage{times}


\newtheorem{Assumption}{Assumption}
\newtheorem{Definition}{Definition}
\newtheorem{theorem}{Theorem}
\newtheorem{Remark}{Remark}
\newtheorem{pro}{Proposition}
\newtheorem{lem}{Lemma}
\newtheorem{cor}{Corollary}
\newtheorem{fact}{Fact}
\newcommand{\G}{G}
\renewcommand{\qedsymbol}{}
\newcommand{\R}{\mathbb{R}}
\newcommand{\ol}{\overline}
\renewcommand{\cal}{\mathcal}
\renewcommand{}{\right )}
\newcommand{\diag}{\operatorname{diag}}
\newcommand{\rk}{\operatorname{rank}}
\renewcommand{\;}{\,;\,}
\newcommand{\N}{\mathbb{N}}


\title{\LARGE \bf
Swarm Aggregation under Fading Attractions}


\author{Xudong Chen\thanks{X. Chen is with the Coordinated Science Laboratory, University of Illinois at Urbana-Champaign. email: xdchen@illinois.edu.}}



\begin{document}


\maketitle
\thispagestyle{empty}
\pagestyle{empty}

\begin{abstract}
Gradient descent methods have been widely used for organizing multi-agent systems, in which they can provide decentralized control laws with provable convergence. 
Often, the control laws are designed so that two neighboring agents repel/attract each other at a short/long distance of separation. 
When the interactions between neighboring agents are moreover nonfading, the potential function from which they are derived is radially unbounded. Hence, the LaSalle's principle is sufficient to establish the system convergence. 
This paper investigates, in contrast, a more realistic scenario where interactions between neighboring agents have fading attractions. In such setting, the LaSalle type arguments may not be sufficient. To tackle the problem,  
we introduce a class of partitions, termed \emph {dilute partitions}, of formations which cluster agents according to the inter- and intra-cluster interaction strengths.  We then apply dilute partitions to trajectories of formations  generated by the multi-agent system, and show that each of the trajectories remains bounded along the evolution,  and converges to the set of equilibria.    
\end{abstract}

\section{Introduction}
The use of gradient descent for organizing a group of mobile autonomous agents has been widely appreciated in mathematics and in its real-world applications. Descent equations often provide the most direct demonstration of the existence of local minima, and provide easily implemented algorithm for finding the minima. Furthermore,  in the context of multi-agent control, gradient descent can be interpreted as providing decentralized  control laws for pairs of neighboring agents in the system. Specifically, we consider a class of multi-agent systems in which pairs of neighboring agents  attract/repel each other in a reciprocal way, depending {\it only} on the distances of separation. Then, the resulting dynamics of the agents evolve as a gradient flow over a Euclidean space. We describe below the model in precise terms:   
\vspace{3pt}
\\
{\bf Model}. 
Let  be an undirected connected graph of~ vertices, with  the vertex set, and  the edge set. We denote by  an edge of . 
Let  be the set of neighbors of~, i.e.,    To each vertex~, we assign an agent~, with  its coordinate. With a slight abuse of notation, we refer to agent~ as~. For every edge , we let  be the distance between  and , i.e., .  The equations of motion of the  agents  in  are given by

Each scalar function  is assumed to be continuously differentiable; we refer to~, for , the {\bf interaction functions} associated with system~\eqref{MODEL}.  
An important property associated with system~\eqref{MODEL} is that the dynamics of the agents evolve as a gradient flow. A direct computation yields that the associated potential function is given by 



Designing of the interaction functions that are necessary for organizing such multi-agent system has been widely investigated: questions about swarm aggregation and avoidance of collisions~\cite{GP,chu2003self,XC2014ACC}, questions about local/global stabilization of targeted configurations~\cite{chu2003self,krick2009,dimarogonas2008stability,xudongchen2015CDCtriangulatedformationcontrol,zhiyongsun2015ECC}, questions about robustness issues of control laws under perturbations~\cite{AB2012CDC,sun2014CDC,USZB,mou2014CDC}, questions about counting number of critical formations~\cite{BDO2014CT,UH2013E}  have all been treated to some degree. 
We also refer to~\cite{xudongchen2015CDCformationcontroltimevaryinggraph,XC2014CDC,JB2006cdc,cao2008control,baillieul2007combinatorial,AB2013TAC,AL2014ECC,lin2014distributed,chen2015decentralized} for other types of models for multi-agent control, as variants of system~\eqref{MODEL}. 

For the purpose of achieving swarm aggregation, the interaction functions 's are often designed so that neighboring agents 
attract each other at a long distance. In particular, we note here that if the underlying graph  is connected, and the interaction functions between neighboring agents have  non-fading attractions (as considered in most of the literatures: see, for example,~\cite{GP,chu2003self,zhiyongsun2015ECC,krick2009,dimarogonas2008stability}); then, for any initial condition, the resulting gradient flow will converge to the set of equilibria. In other words, there is no escape of agents to infinity along the evolution of the multi-agent system. Indeed, in any of such case, the  associated potential function~\eqref{POTENTIAL} is {\it radially unbounded}, i.e., it approaches to infinity as the size of a formation tends to infinity. So then, each trajectory  of system~\eqref{MODEL} has to remain bounded, and hence converges to the set of equilibria.  

On the other hand, it is more realistic  to assume that the magnitude of an attraction between two neighboring agents fades away as their mutual distance grows. We refer to~\cite{cucker2007emergent}, as an example, for modeling the flocking behavior with fading interactions. Specifically, the authors there considered a second order model: 

with the graph  being complete and without repulsions, i.e., the function  is positive at all distances .  
Also, we recall that the Lennard-Jones force, which describes the interaction between a pair of neutral molecules/atoms, has strong repulsion and fading attraction. 

 We note here that,  under the assumption of fading attraction, the potential function associated with system~\eqref{MODEL} may remain bounded as the size of a formation grows; indeed, one may find a continuous path of formations along which the potential function decreases while the size of formation approaches to infinity. In particular,  conventional techniques for proving convergence of gradient flows, such as using the potential function as a Lyapunov function and then appealing to the LaSalle's principle~\cite{lasalle1960some}, may not work in this case. 
Nevertheless, we are still able to show that all the trajectories generated by system~\eqref{MODEL} converge to the set of equilibria. The proof of the system convergence  relies on the use of a class of partitions, termed {\it dilute partitions}, of formations introduced in section~III. Roughly speaking, dilute partitions decompose formations into different clusters of agents according to certain combinatorial and metric conditions. We apply dilute partitions to trajectories of formations generated by system~\eqref{MODEL}, and investigate how clusters of agents evolve over time and interact with each other. In particular, we show that each trajectory generated by system~\eqref{MODEL} has to remain bounded, and hence converges to the set of equilibria.  
This approach, via the use of dilute partition, to multi-agent systems might be of independent interest for studying other problems that involve large sized formations.



This paper expands on some preliminary result presented in~\cite{XC2014ACC} by, among others, providing an analysis of system~\eqref{MODEL} with an arbitrary connected graph (whereas in~\cite{XC2014ACC}, we dealt only with the complete graph), a finer description of the dilute partitions and the associated properties, and a considerable amount of analyses and proofs that were left out. 
The remainder of the paper is organized as follows. In section~II, we introduce definitions and notations and describe some preliminary results about system~\eqref{MODEL}. We also state the main theorem of the paper. In particular, the main theorem states that  the equilibria of system~\eqref{MODEL} have bounded size, and moreover, all trajectories generated by system~\eqref{MODEL} converge to the set of equilibria under the assumption of fading attractions. Sections~III and~IV are devoted to establishing properties of system~\eqref{MODEL} that are needed for proving the main theorem. A detailed organization of these two sections will be given after the statement of the theorem. We provide conclusions in the last section. The paper ends with Appendices containing proofs of some technical results. 


\section{Backgrounds and Main Theorem}
In this section, we introduce the main definitions used in this work, describe some preliminary results, and state the main theorem of the paper. 

\subsection{Backgrounds and notations}
Let  be an undirected graph of  vertices. Let  be a subset of ; a subgraph  of  is said to be {\it induced} by  if the following condition is satisfied: an edge  is in  if and only if  is in .  

Given a formation of  agents in , with states ,  respectively, we set . We call  a {\bf configuration};  
a configuration  can be viewed as an {\it embedding} of the graph  in  by assigning vertex  to . We call the pair  a {\bf framework}. 
We define the  {\it configuration space}  , associated with the graph~, as follows:
 
Equivalently,  is the set of embeddings of the graph  in  whose neighboring vertices have distinct positions. 
Let  be a framework, with .  
 Let \label{eq:typicalexample}
g(d) = -\frac{\sigma_1}{d^{n_1}} + \frac{\sigma_2}{d^{n_2}} 
\lim_{d\to 0+} dg(d)=-\infty,\displaystyle \lim_{d\to 0+}\int^1_d sg(s)ds=-\infty.
g(d) > 0, \hspace{10pt} \forall \, d \ge \alpha_+,    
\lim_{d\to\infty} d g(d)=0.\label{eq:defx+}
g_{ij}(d) > 0, \hspace{10pt} \forall \, d \ge \alpha_+,   
 \label{eq:inducedsub-system}
\dot{ x}_{i} = \sum_{v_j\in V'_i}g_{ij}(d_{ij}) ( x_j- x_i),\hspace{10pt} \forall v_i\in V'

f_i(p):= \sum_{v_j \in V_i} g_{ij}(d_{ij}) ( x_j- x_i).

\inf\{\Psi(p) \mid p\in P_G\} > -\infty.

g_{ij}(d) < 0, \hspace{10pt} \forall \, d \le \alpha_- \mbox{ and } \forall\, (v_i,v_j)\in E.  

g_{ij}(d) > 0, \hspace{10pt} \forall \, d \ge \alpha_+ \mbox{ and } \forall\, (v_i,v_j)\in E.  

\min_{d\in [\alpha_-, \alpha_+]} \int^d_{1} sg_{ij}(s) ds = \inf_{d \in \R_+} \int^d_{1} sg_{ij}(s)ds. 
\label{eq:defpsi0}
\psi_0:= \min_{(v_i,v_j) \in E}\,\left\{ \min_{d\in [\alpha_-, \alpha_+]} \int^d_{1} s g_{ij}(s) ds \right\}; 

\Psi(p) \ge |E|\, \psi_0, \hspace{10pt} \forall\, p\in P_G,
\label{eq:defd-d+}
\left \{
\begin{array}{l}
d_-(p) := \min \left \{  \|x_j - x_i\| \mid  (v_i, v_j)\in E   \right \} \vspace{3pt} \\
d_+(p) := \max \left \{  \|x_j - x_i\| \mid  (v_i, v_j)\in E   \right \}, 
\end{array}
\right.

\inf \left\{  d_-(p(t))  \mid t\ge 0 \right \} > 0. 

\int^d_{1}  s g_{ij}(s) ds + \, \psi_0 > \Psi(p(0))

\Psi(p(t)) \ge  \int^d_{1}  sg_{ij}(s) ds + \, \psi_0 > \Psi(p(0))
\label{eq:potentialconferencepaper}
\lim_{d_+(p) \to \infty} \Psi(p) = \infty, 

g_{ij}(d) = - \frac{\sigma_{ij,1}}{d^{n_{ij,1}}} + \frac{\sigma_{ij,2}}{d^{n_{ij,2}}} 

\int^{\infty}_{1} sg_{ij}(s)ds = -\frac{\sigma_{ij,1}}{n_{ij,1} - 2} + \frac{\sigma_{ij,2}}{n_{ij,2} - 2} < \infty, 

D_- \le d_-(p) \le d_+(p) \le D_+
\label{eq:inducedpartitionofV}
V = \sqcup^m_{i=1} V_i.
\phi(p_i) : = \max \left \{\| x_k - x_j\| \mid v_j, v_k \in V_i \right\}. \label{eq:intradistance}
\cal{L}_-(\sigma): = \max\left \{ \phi(p_i) \mid 1\le i \le m  \right\}.

d(p_i,p_j): = \min\left\{\|x_{i'} - x_{j'}\| \mid  v_{i'} \in V_i, \,  v_{j'} \in V_j \right \}. 
 \label{eq:interdistance}
\cal{L}_+(\sigma) := \min_{(i,j)} \left \{ d(p_i,p_j) \right \}, 
\left \{\sigma_i\in \Sigma(l_i\; p(n_i)) \right \}_{ i\in\N},
\cal{L}_-(\sigma_i) \le L_0, \hspace{10pt} \forall i\in \mathbb{N}.   
\{i\}, \varnothing\label{eq:definingrule}
 \in E \hspace{10pt} \mbox{ and } \hspace{10pt} \| x_{j_k} - x_{j_{k+1}}\| \le l 
\label{eq:evaluatesize}
\phi(p_i)< l', \hspace{10pt} \mbox{ for } \hspace{5pt} l' := (N-1 )l.  
\label{eq:1mN}
1< m < N.
d(p_{j_k},p_{j_{k+1}})  \le l', \hspace{10pt}  \forall \, k = 1,\ldots, q'-1. \phi(p'_i) \le l'', \hspace{10pt} \mbox{ for } \hspace{5pt} l'' :=  (2m -1) \, l'.  \sigma_i = \{(G_j, p_j(i))\}^m_{j=1}.\left \{\sigma'_i \in \Sigma_1(l_i\; p_1(n_i)) \right\}_{ i\in \N }.
\cal{L}_-(\sigma'_i) \le L'_0 \hspace{10pt}
  \forall i\in\N.\sigma'_i = \{(G_{1_j}, p_{1_j}(n_i))\}^{m'}_{j = 1},
\sigma^*_i := \{G_{1_j}, p_{1_j}(n_i)\}^{m'}_{j = 1} \cup \{G_j, p_{j}(n_i)\}^{m}_{j = 2},

\cal{L}_-(\sigma^*_i) \le \max \left\{ L'_0, \, \phi(p_2(n_i)),\ldots, \phi(p_m(n_i))\right\}.
\label{D-dpD+}
D_- \le d_-(p) \le d_+(p) \le D_+.

g_{ij}(d) > 0, \hspace{10pt} \forall \, d \ge \alpha_+ \mbox{ and } \forall\, (v_i,v_j)\in E;  
\label{eq:eqfirstdefD+}
D_+ := (N - 1)\,\alpha_+.
\label{eq:condition1Nov1}
x^1_N - x^1_1 = d_+(p) > (N - 1) \alpha_+ .

x^1_{j} - x^1_{i}   > \alpha_+, \hspace{10pt} \forall \, v_i\in V' \mbox{ and } \forall\, v_j\in V''.

x^1_N - x^1_1 = \sum^{N-1}_{j=1}   \le (N - 1) \alpha_+,

E^*:= \left\{  (v_i, v_j) \in E \mid v_i\in V',\, v_j\in V'' \right \},

s'(p):= \sum_{v_i \in V'} x^1_i \hspace{10pt} \mbox{ and } \hspace{10pt} s''(p):= \sum_{v_i \in V''} x^1_i. 

\frac{d}{dt} s'(p) = -\frac{d}{dt} s''(p)  = \sum_{(v_i,v_j) \in E^*} g_{ij}(d_{ij}) (x^1_j - x^1_i ). 

\frac{d}{dt} s'(p) = -\frac{d}{dt} s''(p) > 0,
\label{eq:defbargij}
\ol g_{ij}(d) := dg_{ij}(d).

\ol g_{ij}^{-1}(S) := \left\{ d \in \R_+ \mid \ol g_{ij}(d) \in S  \right\}.

\lim_{\eta \to \infty} \sup \left \{d \in  \ol g^{-1}_{ij}  \right \} = 0.
\pm \, \eta is nonempty, and is contained in .  
\end{proof}


Let  be a positive number;  we define a subset of  as follows: 

Recall that  is the vector field of system~\eqref{MODEL} at . With Lemma~\ref{lem:lem2}, we establish the following fact:

\begin{pro}\label{pro:nmvecfld}
Let  be defined in~\eqref{eq:defZGd}. Then, 
\, 
\end{pro}

\begin{proof}
The proof will be carried out by induction on the number of vertices of . For the base case , we have 

We also have for any , 

From the condition of {\it strong repulsion}, we have
, which establishes the base case. 

For the inductive step, we assume that Proposition~\ref{pro:nmvecfld} holds for , and prove for . The proof will be carried out by contradiction: we assume that there exists a number  such that for any , there is a number  and a configuration \label{eq:defgamma}
 \| f'(p') \| = \omega\,  \eta; 
  \|f'_i(p') \| \ge \omega' \,   \eta, \hspace{10pt} \mbox{for } \omega' := \omega/ \sqrt{k - 1}.\label{eq:assumptiononf'_i}
f'_i(p) =  \in \R^n.

\|f(p)\| \ge \|f_i(p)\| = \omega'\, \eta > \eta
 \dot x^1_i =\| f'_i(p) \| + g_{ik}(d_{ik}) (x^1_k - x^1_i ).
g_{ik}(d_{ik}) (x^1_k - x^1_i ) \le  \| f(p) \| - \| f'_i(p) \|.
\label{eq:ineqforgik}
g_{ik}(d_{ik}) (x^1_k - x^1_i )  \le - (\omega' -1)\, \eta < 0,  

\dot x^1_k  = g_{ik}(d_{ik})( x^1_i - x^1_k ) + \sum_{v_j\in V_k - \{v_i\}} g_{jk}(d_{jk})(x^1_j - x^1_k ).
\label{eq:3:16pmNov1}
g_{jk}(d_{jk})( x^1_j - x^1_k )  \le -\omega'' \, \eta, \hspace{10pt} \mbox{for } \omega'' := \frac{\omega' - 2 } {k - 1}.
 x^1_i < x^1_k < x^1_j < x^1_{j'} < x^1_{j''} < \cdots. 
\left\{
\begin{array}{l}
\cal{L}_-(\sigma)= \max\left \{ \phi(p_i) \mid 1\le i \le m  \right\}, \\
\cal{L}_+(\sigma) = \min_{(i,j)} \left \{ d(p_i,p_j) \right \}
\end{array}
\right. 
\label{eq:defselfclustering}
\cal{L}_-(\sigma_t) < l_0 \hspace{10pt} \mbox{ and } \hspace{10pt} \cal{L}_+(\sigma_t) > l_1. 

\sup\{\phi(p(t)) \mid t \ge 0\} < \infty.

c(p(t)) := \sum_{v_i \in V} x_i(t)/ |V|.  
\label{eq:defsigmatforproposition5}
\sigma_t = \{(G_i, p_i(t))\}^m_{i = 1}, \hspace{10pt} \mbox{ with } \hspace{5pt} G_i = (V_i, E_i), 
G_{\cal{I}'} = (V_{\cal{I}'}, E_{\cal{I}'}), \hspace{10pt} \mbox{ with } \hspace{5pt} V_{\cal{I}'}: =\sqcup_{j\in \cal{I}'} V_j \label{eq:defPIkt}
\Pi(k\; t) := \max_{\cal{I}'}\{\pi(\cal{I}'\; t) \mid  |\cal{I}'| = k\}. 
\label{eq:Pi1thehe}
\Pi(1\; t) = \max\{\|c(p_i(t))\|  \mid i = 1,\ldots, m\}; 

\Pi(m\; t) = \|c(p(t))\| = 0.
\label{eq:1:30pm}
\Pi(1\; t) \ge \ldots \ge \Pi(m\; t) = 0.

\Pi(k+1\; t) \ge r - N (r - \Pi(k\; t) ) -  2 l_0. 
\label{eq:phiptcaonima}
 \phi(p(t)) <2(\Pi(1\; t) + l_0), \hspace{10pt} \forall\, t \ge 0.

\begin{array}{lll}
d_{ij}(t)  & \le & \|x_{i}(t) - c(p_{i'}(t))\|  + \| c(p_{j'}(t)) - x_{j}(t) \| \\
& & +   \|  c(p_{i'}(t)) - c(p_{j'}(t)) \|;
\end{array}

\left\{
\begin{array}{l}
 \|x_{i}(t) - c(p_{i'}(t))\| < \phi(p_{i'}(t)) < l_0, \\
 \| c(p_{j'}(t)) - x_{j}(t) \| < \phi(p_{j'}(t)) < l_0; 
\end{array}
\right.

\| c(p_{i'}(t)) - c(p_{j'}(t))\| \le   2 \Pi(1\; t).
\Pi(1\; t)  \le \Pi(1\; t_1) = r, \hspace{10pt} \forall\,  t \le t_1.
\Pi(2\; t_1) \ge r -  2 l_0. 

\Pi(2\; t)  \le \Pi(2\; t_2) = r -  2 l_0, \hspace{10pt} \forall\,  t \le t_2.

\Pi(3\; t_2) \ge r -  2(N + 1) l_0.

\Pi(k+1\; t_k) \ge r - 2 \sum^{k-1}_{i = 0} N^i  l_0.
\label{eq:8:24pm}
0 = \Pi(m\; t_{m-1}) \ge r - 2 \sum^{m - 2}_{i = 0} N^i  l_0,
\label{eq:defsigmat}
\sigma_t:= \{(G_i, q_i(t))\}^m_{i = 1};  
\label{eq:contradictioncondition1}
\sigma_t\in \Sigma(l_i\; q(t)) \hspace{5pt} \mbox{ and } \hspace{5pt}  \cal{L}_-(\sigma_t) <  L_0. 

\hat v :=
\left\{
\begin{array}{ll}
v /\|v\| & \mbox{if } v \neq 0, \\
0  & \mbox{otherwise}.
\end{array}
\right. 
\label{eq:idon'tsleepwelltoday}
\langle \hat c(p_{\cal{I}'}(t)), c(p_{i}(t)) \rangle \ge r - N(r - \Pi(k\; t)).

\pi(\cal{I}'\; t) =\langle \hat c(p_{\cal{I}'}(t)),\, \sum_{i\in \cal{I}'}w_i \, c(p_i(t)) \rangle. 
\langle \hat c(p_{\cal{I}'}(t)), c(p_j(t)) \rangle \le \|c(p_j(t))\| \le r,
 \langle \hat c(p_{\cal{I}'}(t)), c(p_i(t)) \rangle \ge  r - \frac{1}{w_i} .  

E_{\cal{I}'}:=\{(v_{a},v_{b}) \in E \mid v_{a}\in V_{\cal{I}'},\, v_{b} \notin V_{\cal{I}'} \}. 
\rho_{ab}(t) := \left\langle c(p_{\cal{I}'}(t)) , x_{b}(t) - x_{a}(t) \right\rangle.\label{eq:lajitong}
\frac{d}{dt}\pi(\cal{I}'\; t)^2 = 2\sum_{(v_{a},v_{b}) \in E_{\cal{I}'} } g_{ab}(d_{ab}(t))\, \rho_{ab}(t).
\langle \hat c(p_{\cal{I}'}(t)), c(p_j(t)) \rangle \le \| c(p_{\cal{I}'}(t))\|.\label{eq:12:22pm}
 \left\langle  \hat c(p_{\cal{I}'}(t)),\, c(p_{j}(t)) \right\rangle > \langle \hat c(p_{\cal{I}'}(t)), c(p_{i}(t)) \rangle - 2l_0. 
\left\{
\begin{array}{l}
\| x_{a}(t) - c(p_{i}(t))\| < \phi(p_{i}(t)) < l_0 \vspace{3pt}\\
\| x_{b}(t) - c(p_{j}(t))\| < \phi(p_{j}(t))  < l_0.
\end{array}
\right.
\label{eq:1:59pm}
\left\{
\begin{array}{l}
\langle  \hat c(p_{\cal{I}'}(t)),    c(p_{i}(t)) - x_{a}(t)  \rangle < l_0  \vspace{3pt}\\
\langle  \hat c(p_{\cal{I}'}(t)),   x_{b}(t) - c(p_{j}(t)) \rangle < l_0,
\end{array}
\right. 

\left\langle  \hat c(p_{\cal{I}'}(t)),\, c(p_{j}(t)) \right\rangle > \langle \hat  c(p_{\cal{I}'}(t)), c(p_{i}(t)) \rangle - 2l_0 + \rho_{ab}(t).
\label{eq:7:27pmatcaffebene}
 \left\langle  \hat c(p_{\cal{I}'}(t)),\, c(p_{j}(t)) \right\rangle \ge r - N (r - \Pi(k\; t) ) -  2 l_0.

\pi(\cal{I}''\; t) \ge  \left\langle  \hat c(p_{\cal{I}'}(t)),\, c(p_{j}(t)) \right\rangle. 
\pi(\cal{I}''\; t) =  \tilde w_{j} \langle \hat c(p_{\cal{I}''}(t)),c(p_j(t))\rangle +  \tilde w_{\cal{I}'} \langle \hat c(p_{\cal{I}''}(t)),c(p_{\cal{I}'}(t)) \rangle.
\langle \hat c(p_{\cal{I}''}(t)), c(p_j(t)) \rangle \ge \langle \hat c(p_{\cal{I}'}(t)),\, c(p_{j}(t)) \rangle,
 \|c(p_{\cal{I}''}(t))\| \le \Pi(k+1\; t) \le \Pi(k\; t) = \|c(p_{\cal{I}'}(t))\|,
\langle \hat c(p_{\cal{I}''}(t)), c(p_{\cal{I}'}(t)) \rangle \ge\langle \hat c(p_{\cal{I}'}(t)), c(p_{\cal{I}''}(t)) \rangle.
\langle \hat c(p_{\cal{I}'}(t)), c(p_j(t)) \rangle \le \| c(p_{\cal{I}'}(t))\|,
\langle \hat c(p_{\cal{I}'}(t)), c(p_{\cal{I}''}(t)) \rangle \ge  \langle \hat c(p_{\cal{I}'}(t)),\, c(p_{j}(t)) \rangle.
\label{eq:defXGijd}
X_{G,ij}(d) :=\left  \{ p\in P_G  \mid  \|x_j - x_i\| = d \right \}; 
\label{eq:defmud}
\mu(d) := \inf\{\|f(p)\| \mid p\in X_{G}(d) \};
\label{eq:defQd}
Q_G(d):= \left\{ p\in P_G \mid d_-(p) \ge d  \right \}.

\|f(p) - f(p')\|^2 = \sum_{v_i\in V}\|f_i(p) - f_i(p')\|^2, 
 \sum_{v_j\in V_i} \| g_{ij}(d_{ij})(x_j - x_i) - g_{ij}(d'_{ij})(x'_j - x'_i) \|. \|(x_j - x_i) - (x'_j - x'_i) \| < 2\delta.\label{eq:forcontinuity0}
\min\{ \|u\|,  \|u'\|\} \ge d \hspace{10pt} \mbox{ and } \hspace{10pt} \|u - u'\| < \delta', 
\label{eq:forcontinuity1}
\|g_{ij}(\|u\|) u - g_{ij}(\|u'\|) u' \| < \epsilon'. 

\|g_{ij}(\|u\|) u - g_{ij}(\|u'\|) u' \| \le \ol g_{ij}(\|u\|) + \ol g_{ij}(\|u'\|) < \epsilon'.

K:= \{u\in \R^n \mid d \le \|u\| \le d_* + 1\}.

\widetilde g_{ij}: u\mapsto g_{ij}(\|u\|) u
\label{eq:alihuiwoyoujian}
\inf\{ \|f(p)\| \mid p\in X_{G}(d') \cap Q_G(d_*) \} = \mu(d').

x'_1(i) := x_1(i) + (d'/ d - 1) (x_1(i) - x_2(i)),
 \|p'(i) - p(i)\| = \|x'_1(i) - x_1(i)\| = |d' - d| < \delta.\label{eq:boundedfpi}
\| f(p'(i)) - f(p(i)) \| \le \epsilon, \hspace{10pt} \forall\, i\in \N.

X_G(d) = \left \{  (x_1, x_2)\in \R^4  \mid \|x_2 - x_1\| = d \right \}.

X_{G', ij}(d) := \left\{ p'\in P_{G'} \mid  \|x_{j} - x_{i}\| = d \right\};

\mu_{G'}(d) := \inf\left\{ \|f'(p')\| \mid p' \in X_{G'}(d) \right\}.

\nu(d) := \min_{G'}\{ \mu_{G'}(d) \}, 
\label{eq:proveforsmallconfiguration}
\| f(p) \| > 1, \hspace{10pt} \mbox{ if } \hspace{5pt} p \in X_{G}(d) \mbox{ and }  d_-(p) < d_0.
\label{eq:proveforlargeconfiguration}
\|f(p)\| > \nu(d) /2, \hspace{10pt} \mbox{ if } \hspace{5pt} p \in X_{G}(d) \mbox{ and }  d_+(p) > d_1.

K := \left\{ p\in X_G(d) \mid d_0 \le \|x_j - x_i\| \le d_1, \, \forall (v_i,v_j) \in E\right\}.
\label{eq:verylargedistance}
0 < \ol g_{ij}(d') < \nu(d) / (2k^2), \hspace{10pt} \, \forall\, d' \ge l. 
h_1,\ldots, h_{k'}
h_i := \sum_{v_j\in V_i - V'} g_{ij}(d_{ij}) (x_j - x_i). 

\|h_i\| \le \sum_{v_j\in V_i - V'} \ol g_{ij}(d_{ij}) < \nu(d) / (2k), 

\|f(p)\| \ge \|f_{V'}(p)\| \ge \|f'(p')\| - \|h\| > \nu(d) /2.

X_{G,ij}(d) = \left\{ p\in P_G \mid \|x_j - x_i\| = d \right\}. 

\inf\left\{ \|p' - p''\| \mid p'\in X_{G,ij}(d'), \, p''\in X_{G,ij}(d'') \right\}. 

d(X_{G,ij}(d'), X_{G,ij}(d'')) = |d' - d''|/ \sqrt{2}. 
\label{eq:equalitysatisfied}
\|p' - p''\| =  |d' - d''|/ \sqrt{2}.

\left\{
\begin{array}{l}
x'_i  =  -x'_j   =  (d', 0, \ldots, 0) / 2, \vspace{1pt}\\
x''_i  =  - x''_j =  (d'',0,\ldots,0) / 2.  \vspace{1pt}\\
\end{array}
\right.
 \|p' - p''\| \ge |d' - d''|/ \sqrt{2}.\label{eq:lem81}
 \|x'_i - x''_i\|^2 + \|x'_j - x''_j\|^2 \ge \frac{1}{2} (d' - d'')^2.

\left\{
\begin{array}{ll}
y'_i := x'_i - x' & y'_j := x'_j - x', \vspace{3pt}\\
y''_i := x''_i - x'' & y''_j := x''_j - x''.  
\end{array}
\right.
\label{eq:lem82}
\|x'_i - x''_i\|^2 + \|x'_j - x''_j\|^2 = 2 \|y'_i - y''_i\|^2 + \|x' - x''\|^2.
\label{eq:lem83}
 \|y'_i - y''_i\| \ge | \|y'_i\| - \|y''_i \||  = |d' - d''|/ 2.  
\label{eq:defTepsilon}
\Psi - \Psi \le \epsilon.
\label{eq:Novsecond8:45am}
\sup \{d_{ij}(t)g_{ij}(d_{ij}(t)) \mid t \ge 0   \} < \infty.

\sup\{dg_{ij}(d) \mid d \ge d_*\} < \infty,
\label{eq:firstproof}
\cal{L}_-(\sigma_t) < L_0, \hspace{10pt} \forall\,  t\ge t_{j'_i}, 
\label{eq:secondproof}
\cal{L}_+(\sigma_t) > \max\{l_i, L_0\},  \hspace{10pt} \forall\, t \ge t_{j''_i}.  
\label{eq:evaluatept'}
q(t) \in X_{G,ij}(I_0), \hspace{10pt} \forall\, t \in  [t_0, t_0 +  \tau_0 ]. 

\Psi(q(t_0)) - \Psi(q(t_0 + \tau_0))   =  \int^{t_0+ \tau_0}_{t_0}\| f(q(t)) \|^2 \, dt. 
\label{eq:epsilon0decrease}
\Psi(q(t_0)) - \Psi(q(t_0 + \tau_0))   \ge \epsilon_0, \hspace{5pt}\mbox{for} \hspace{5pt} \epsilon_0 :=  \frac{\mu(L_0)^2\tau_0}{4}.
\label{eq:deltaisepsilon0}
\Psi(q(T_{\epsilon_0})) - \Psi(q(\infty)) = \epsilon_0. 

\Psi(q(t + \tau_0)) < \Psi(q(T_{\epsilon_0})) - \epsilon = \Psi(q(\infty)),

\Psi(q(t_1)) - \Psi(q(t_1 + \tau_1))   \ge \epsilon_1, \hspace{5pt} \mbox{for} \hspace{5pt} \epsilon_1 := \frac{\mu(L_1)^2\tau_1}{4}.

Appealing again to Lemma~\ref{lem:upperboundonvelocity}, we obtain an instant  such that 
. 
Since both sequences  and  monotonically increase, and approach to infinity, there is a~ such that 
 and  .  Then,~\eqref{eq:secondproof} holds for the choice of~ because otherwise, there will be an instant~, with , such that , and hence
, which is a contradiction. 
 This completes the proof.   
\end{proof}


\end{document}
