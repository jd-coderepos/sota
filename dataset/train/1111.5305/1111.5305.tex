\documentclass[final]{siamltex}


\usepackage[pdftex]{graphicx,color}
\usepackage{xspace}
\usepackage{amsmath,amssymb}
\usepackage{array}
\usepackage{fullpage}
\usepackage{picinpar}
\newcommand{\margin}[1]{\marginpar{\hspace*{0in}\parbox{0.8in}{\footnotesize \raggedright #1\par~}}}

\newcommand{\N}{\mathbb{N}}
\newcommand{\R}{\mathbb{R}}
\newcommand{\Z}{\mathbb{Z}}
\newcommand{\Np}{\mathbb{N}_{\ge 0}}
\newcommand{\Rp}{\mathbb{R}_{\ge 0}}
\newcommand{\Zp}{\mathbb{Z}_{\ge 0}}

\newcommand{\minimize}{\text{minimize }}
\newcommand{\st}{\text{ such that }}
\newcommand{\leftside}{\text{\sf left}}
\newcommand{\rightside}{\text{\sf right}}
\newcommand{\OPT}{\text{\sc opt}}

\newcommand{\refLP}{(\ref{eq:LP})\xspace}

\newcommand{\transposal}[2]{{#1}^{#2}}
\newcommand{\triangulated}[2]{{\widehat{#1}}^{#2}}

\newcommand{\LK}{\text{\sc lk}}
\newcommand{\MCP}{\text{\sc mcp}}
\newcommand{\MWT}{\text{\sc mwt}}
\newcommand{\mcp}{\MCP}
\newcommand{\mwt}{\MWT}
\newcommand{\qgc}{\text{\sc qgc}}
\newcommand{\qgt}{\text{\sc qgt}}
\newcommand{\dg}{^{\circ}}
\newcommand{\conv}{\text{\sc conv}}
\newcommand{\area}{\text{\rm area}}
\newcommand{\boundary}{\partial}

\renewcommand{\overline}[1]{{#1}}

\newcommand{\blanket}{B} \newcommand{\blanketSet}{{\cal B}} \newcommand{\convPart}{\text{\sc cp}} 
\newcommand{\cost}{c}
\newcommand{\edge}{e}
\newcommand{\edges}{E}
\newcommand{\face}{f}
\newcommand{\graph}{G}
\newcommand{\polygon}{P}
\newcommand{\region}{R}
\newcommand{\tri}{t}  \newcommand{\vertex}{v}
\newcommand{\vertices}{V}
\newcommand{\fracTriang}{X}

\newcommand{\sensitivity}{\sigma}

\newcommand{\YXY}{YXY\xspace} 









\newcommand{\glob}[1]{\margin{glob: \ensuremath{#1}}}
\newcommand{\loc}[1]{\margin{\ensuremath{#1}}}
\newcommand{\conflict}[1]{\margin{{\bf conflict}: #1}}

\newenvironment{linearprogram}[2]{
\samepage
\renewcommand{\arraystretch}{1.3}

}

\newcommand{\FULL}[1]{#1}



\newcommand{\Picture}[1]{\input{pictures/#1}}
\newcommand{\xfigpdf}[1]{\Picture{#1.pdf_t}}
\newcommand{\xfig}[1]{\xfigpdf{#1}}

\newcommand{\prepaid}{\text{\sc pp}}

\newcommand{\MS}{\raisebox{1.5pt}{}}
\newcommand{\ES}{\xi}

\newtheorem{fact}{Fact}[theorem] 
\newcommand{\qed}{\endproof}

\newenvironment{proofidea}{\par{\it Proof idea}. \ignorespaces}{}

\newenvironment{fullproof}{\par{\it Full proof}. \ignorespaces}{\endproof}

\title{On a Linear Program for Minimum-Weight Triangulation
\thanks{To appear in SICOMP. Extended abstract paper appeared in SODA 2012 \cite{yousefi2012linear,yousefi2013linear}.}}

\author{Arman Yousefi\thanks{University of California, Los Angeles, USA.
    Research partially funded by GAANN fellowship.
}
\and
Neal E. Young\thanks{University of California, Riverside, USA.
  Research partially funded by NSF grants 0729071 and 1117954.
}
}

\begin{document}



\maketitle

\begin{abstract}
Minimum-weight triangulation (MWT) is NP-hard.
It has a polynomial-time constant-factor approximation algorithm,
and a variety of effective polynomial-time heuristics that, for many instances, can find the exact MWT.
Linear programs (LPs) for MWT are well-studied,
but previously no connection was known between any LP
and any approximation algorithm or heuristic for MWT.
Here we show the first such connections:
for an LP formulation due to Dantzig et al.~(1985):
(i) the integrality gap is constant;
(ii) given any instance, if the aforementioned heuristics find the MWT, then so does the LP.
\end{abstract}




\section{Introduction}

In 1979, Garey and Johnson listed minimum-weight triangulation (MWT) 
as one of a dozen important problems neither known to be in P 
nor known to be NP-hard \cite{garey1979computers}.
In 2006 the problem was finally shown to be NP-hard \cite{mulzer2008minimum}.
The problem has a sub-exponential time exact algorithm \cite{smith1988studies},
as well as a polynomial-time approximation scheme (PTAS) 
for random inputs \cite{golin1996limit}.
It is not currently known whether, for some constant ,
finding a -approximation is NP-hard,
but this is unlikely as a quasi-polynomial-time approximation scheme exists \cite{remy2009quasi}.
MWT has an -approximation algorithm \cite{plaisted1987heuristic},
and, most important here,
an -approximation algorithm
called {\sc QuasiGreedy} \cite{krznaric1998quasi}.
The constant in the big-O upper bound from \cite{krznaric1998quasi}
is astronomically large.

If restricted to simple polygons, MWT
has a well-known -time dynamic-programming algorithm 
\cite{gilbert1979new,klincsek1980minimal}.
Polynomial-time algorithms also exist for instances with a constant number of ``shells''
\cite{anagnostou1993polynomial}
and for instances with only a constant number of vertices in the interior of the 
region  to be triangulated
\cite[\S 2.5.1]{Giannopoulos:2007cx},
\cite{hoffmann2006minimum,borgelt2008fixed,spillner5faster,knauer2006fixed}.


\paragraph{Linear program of Dantzig et al.~for MWT}
Linear programming (LP) methods are a primary paradigm
for the design of approximation algorithms.
For many hard combinatorial optimization problems, 
especially so-called packing and covering problems,
the polynomial-time approximation algorithm with the best approximation ratio
is based on linear programming,
either via randomized rounding or the primal-dual method.
The design of a good approximation algorithm 
is often synonymous with bounding the integrality gap
of an underlying LP.

MWT has several straightforward linear programming (LP) relaxations.
Studying their integrality gaps
may lead to better approximation algorithms,
or may widen our understanding 
of general methods and their limitations
(as standard randomized rounding and primal-dual approaches 
may be insufficient for MWT).

Dantzig et al.~(1985) introduce the following LP
(presented here as reformulated by \cite{de1996polytope}).
Below  denotes the set of empty triangles.\footnote
{That is, triangles lying in the region to be triangulated, 
whose vertices are in the given set of points,
but otherwise contain none of the given points.}
 denotes the region to be triangulated,
minus the sides of triangles in .
The LP asks to assign a non-negative weight  to
each triangle  so that, for each point  in the 
region, the triangles containing it are assigned total weight 1:

Above,  is the set of vectors of non-negative reals, 
indexed by the triangles in .
The cost  of triangle 
is the sum over the edges  in  of the cost  of the edge,
defined to be  (the length of ), unless  is on the boundary of ,
in which case the cost is .
(Internal edges are discounted by 1/2 since any internal edge occurs in either zero or two triangles in any triangulation.)
 as specified is infinite, 
but can easily be restricted to a polynomial-size set of points
without weakening the LP.  (E.g., let  contain, 
for each possible edge , two points  and , each on one side of  and very near .)

For the simple-polygon case, the above LP finds the exact MWT (every extreme point
has 0/1 coordinates, and so corresponds to a triangulation).
This was shown by Dantzig et al.~(1985)  \cite[Thm.~7]{dantzig1985triangulations},
then (apparently independently) by De Loera et al.~(1996) \cite[Thm.~4.1(i)]{de1996polytope}
and Kirsanov (2004) \cite[Cor.~3.6.2]{kirsanov2004minimal}.
For summaries of these results, see \cite[Ch.~8]{de2010triangulations} and \cite{takeuchi1998polytopes}.
Kirsanov describes an instance (a 13-gon with a point at the center) 
for which this LP has integrality gap just above 1,
as well as instances (50 random points equidistant from a center point)
that are solved by the LP but not by the LMT-skeleton heuristic.

Other authors have considered {\em edge-based} LP's, mainly for use in branch-and-bound
\cite{kyoda1996study,kyoda1997branch,ono1996package,tajima1998optimality,aurenhammer2000optimal}.
These edge-based LPs have unbounded integrality gaps.

LPs for maximal independent sets, which are well studied, are closely related to all the above LPs,
as triangulations can be defined as maximal independent sets of triangles (or of edges).
The above LPs enforce some, but not all, well-studied inequalities for maximal independent sets.

It is known to be NP-hard to determine whether there exists a triangulation
whose edge set is a subset of a given set 
\cite{lloyd77triangulations}.
For a given set , if we change the cost function in the above LP to
,
the LP will have a zero-cost integer solution iff there is such a triangulation.
Unless P=NP, this implies that the LP with that cost function has unbounded integrality gap.\footnote
{If the LP has bounded integrality gap,  it has a zero-cost fractional solution iff it has a zero-cost integer solution.}
Thus, any bound on the integrality gap of the LP 
with the MWT cost function must rely intrinsically on that cost function.
Similarly,  given an arbitrary fractional solution , it is NP-hard to determine whether there is an integer solution in the support of .\footnote
{Given an arbitrary subset  of the edges, the problem of determining whether 
contains a triangulation reduces to the problem of determining whether there is an integer solution in the support of a given fractional solution to the LP, as follows.  
Let set  consist of the empty triangles whose edges are in ,
so  contains a triangulation (by edges) 
iff  contains a triangulation (by triangles).
For each triangle , solve the LP with the cost function
that gives  cost zero, every other triangle in  cost one,
and all triangles not in  cost infinity.  
If (for any ) the LP for  has no finite-cost feasible solution,
then  contains no triangulation.  Otherwise, for each ,
let  denote an optimal fractional solution to the LP for .  
Let  be the average of these fractional solutions.
Because of the choice of the cost function, if a given  does not give positive weight to , then no (integer) triangulation in  contains .  Thus,  contains a triangulation iff there is a triangulation in the support of .}
These are obstacles to standard randomized-rounding methods.



\paragraph{First new result: integrality gap is constant}
We show that LP \refLP\ has constant integrality gap.  
This is the first non-trivial upper bound on the integrality gap of any MWT LP.
To show it, we revisit the analysis of {\sc QuasiGreedy} \cite{krznaric1998quasi},
which shows that {\sc QuasiGreedy} produces a triangulation
of cost , where  is the length of the MWT 
of the given instance 
(and also the cost of the optimal integer solution to the LP).
We generalize their arguments to show that there exists a triangulation
of cost , where  is the cost of the optimal
{\em fractional} solution to the LP.

Our analysis also reduces the approximation ratio in their analysis by an order of magnitude, 
but the approximation ratio remains a large constant.

\paragraph{MWT heuristics}
Much of the MWT literature concerns polynomial-time heuristics that, given an instance, 
find edges that must be in (or excluded from) any MWT.
Here is a summary.
Gilbert observes that the {\em shortest potential edge} is in every MWT \cite{gilbert1979new}.
Yang et al.~extend this result by proving that an edge  is in every MWT 
if, for any edge  that intersects ,
 \cite{yang1994chain}.
(We refer to the edges satisfying this property as the {\em  subgraph}.)
This subgraph includes every edge connecting two {\em mutual nearest neighbors}. 
Keil \cite{keil1994computing} defines another heuristic called {\em -skeleton} as follows. 
An edge  is in the -skeleton if and only if there does not exist a point 
in the point set such that . 
Thus, an edge  is in the -skeleton if and only if 
the interior of the two circles of diameter 
 passing through  and  do not contain any points.
Keil \cite{keil1994computing} then shows that for , an edge that is in 
-skeleton is in every MWT. 
Cheng et al.~strengthen this to  where  \cite{cheng1996approaching}.
Das and Joseph show that an edge  cannot be in any MWT if
both of the two triangles with base  and base angle  
contain other vertices \cite{das1989triangulations}. 
Drysdale et al.~strengthen this to angle  \cite{drysdale2001exclusion}.
This property of  is called the {\em diamond property}.
Dickerson et al.~describe a simple local-minimality property such that,
if an edge  lacks the property,
the edge cannot be in any MWT.
Using this, they show that the so-called
{\em LMT skeleton} must be {\em in} the MWT
\cite{dickerson1997large}.

A primary use of the heuristics is to solve
some instances of MWT exactly in polynomial time, as follows:
{\em Given an instance, use the heuristics to identify edges that are in the MWT.
If the regions left untriangulated by these edges are simple polygons
(equivalently, if the edges span the given points)
then find the MWT of each region independently
using the standard dynamic programming algorithm.}
(The MWT will be the union of the MWT's of the regions.)
According to \cite{dickerson1997large} (1997),
most random instances with 40,000 points are solvable in this way.


\paragraph{Second new result: LP generalizes heuristics}
We show that 
LP~\refLP\ generalizes these heuristics
in that {\em if the heuristics solve a given instance
as described above, then so does the LP}
(that is, the extreme points of the LP are integer solutions --- incidence vectors of optimal triangulations).
In this sense, the LP, whose formulation requires little explicit geometry,
generalizes all of these varied and generally incomparable heuristics.
(In fact the LP appears to be stronger than the heuristics, in that some natural instances
are solved by the LP, but not by the heuristics
\cite[\S 3.5]{kirsanov2004minimal}.\footnote
{Where  contains the center of a unit circle and  random points on the circle.})
This is the first connection we know of between the heuristics and any MWT LP.

Roughly, the heuristics are based on a combination of
(i) local-improvement arguments about the MWT and
(ii) logical closure (once the heuristic determines the status of one edge with respect to the MWT,
this in turn determines the status of other edges, and so on).
We extend these arguments to apply to the optimal fractional triangulation .
This is possible because
(i)   looks ``locally'' like a MWT and 
(ii) the LP enforces logical closure of linear constraints on .


After we finished the body of this work, we became aware of
and examined additional heuristics by Wang et al.~\cite{wang1997new} 
and Aichholzer et al.~\cite{aichholzer1996triangulations}.
We conjecture that the LP generalizes them as well.

\paragraph{An equivalent formulation of the LP}
The following constraints are equivalent to the last constraints in LP~\refLP
(see e.g.~\cite[Thm.~1.1(i), Prop.~2.5]{de1996polytope},
\cite{takeuchi1998polytopes}, or \cite[Thm.~3.4.1]{kirsanov2004minimal})
and are useful for reasoning about fractional triangulations.
For any fractional triangulation  and edge ,

Here  contains the triangles that contain  and lie on one side of ,
while  contains the triangles that contain  and lie on the other side of .
(If  is on the boundary, take .)
The notation  denotes 1 if  and 0 otherwise.

\paragraph{Practical considerations}
Using the  constraints~(\ref{constraint:edge})  
instead of the constraints in~\refLP gives an equivalent LP
with total size (i.e., non-zeros in the constraint matrix) 
proportional to the number of empty triangles.
The empty triangles can be identified, and the LP constructed, 
in time proportional to their number \cite{dobkin1990searching}.
Their number is always , 
but often smaller (e.g.~ in expectation for randomly distributed points).

The time to construct and solve the LP can be further reduced 
by a preprocessing step based on the heuristics
--- 
remove any variable  if the heuristics prove any edge of  to be excluded from every MWT,
and add a constraint
 
if they prove an interior edge  to be in every MWT.
For randomly distributed points,
only  edges (in expectation) have the diamond property,
forming  possible empty triangles,
from which the modified LMT skeleton can be computed in  time
\cite{dickerson1997large,dickerson1997fast}.
In our ad-hoc experiments on ``typical'' instances with  points,
only a small number of variables were left undetermined by the heuristics.
This allowed us to use standard LP solvers to quickly solve the remaining LP.
(This is in keeping with Dickerson et al's experiments, which found that most random
instances on 40,000 points were solvable by heuristics \cite{dickerson1997large}.)
Similarly, this preprocessing
should help integer-LP solvers to quickly find the MWT
(for instances for which the optimal solution is fractional).
It is known that, asymptotically, 
for  random points, the expected number of remaining variables
is , but the leading constant is apparently astronomically small
\cite{bose2002diamonds}.




\paragraph{Remarks} 
The results here suggest that the LP of Dantzig et al.\ captures
much of the structure of MWT.
This suggests a line of attack for improving the approximation ratio:
use systematic LP methods such as randomized rounding, the primal-dual method,
and lift-and-project \cite{balas2002lift} to study the integrality gap of the LP.
Success would yield an better approximation (conceivably, even a PTAS, using lift-and-project).
Failure would increase our understanding of the limitations of these techniques.

Implicit in our bound on the integrality gap
is a polynomial-time algorithm with matching approximation ratio.
Actually, there are two.
Both algorithms first compute Levcopoulos and Krznaric's convex partition 
of the point set
(see our Lemma~\ref{lemma:LK}) \cite{krznaric1998quasi},
then extend  by triangulating each face  of .
The triangulation of each  can be done either 
(a) using the standard dynamic program to find a minimum-weight triangulation of ,
or (b) as follows: compute the fractional solution  to the linear program,
then, for each face  of ,
{\em transpose}  into a fractional triangulation 
of 
(as described in Section~\ref{sec:part i}),
then use the cheapest triangulation of  implicit in .

That the first algorithm above is an -approximation algorithm
follows from Levcopoulos and Krznaric's previous work \cite{krznaric1998quasi}.
However, the bound we show here
--- , where  is a large constant per Lemma~\ref{lemma:LK} ---
is substantially smaller 
than their previous bound.
Roughly, we obtain a better bound by analyzing the transposal operation 
at the level of {\em triangles}, instead of edges.

\paragraph{Open problems}
The integrality gap is constant,
but there is still a huge gap between the best lower bound known (barely above 1.0) and the upper bound shown here (astronomically large).
The next step in improving our upper bound
would be to reduce the value of  in Lemma~\ref{lemma:LK}.
We suspect that a primal-dual analysis is implicit in the analysis here; 
making the dual solution explicit might be a step in this direction.

Many different cost functions 
(other than the total edge length) for triangulations
are studied in the literature.
The MWT LP extends naturally
by modifying the cost function or restricting the set of allowed triangles.
(For example, the integrality of the extreme points of the LP for the simple-polygon case
implies that the simple-polygon result generalizes to any linear cost function.)
We conjecture that results similar to those in this paper can be obtained for other cost functions.

If MWT heuristics can solve a given instance of MWT, then so can the LP.
However, the heuristics are also useful for instances that they don't completely solve:
on such instances, the heuristics can still identify some edges that are in (or excluded from) 
every MWT, even if these do not completely determine the triangulation.
Can some analogous property be shown for the LP?
That is, is there some condition (e.g., based on the optimal primal/dual solution to the LP) such that,
if the condition holds for an edge , then that edge must be in (or excluded from) every MWT?




\smallskip
\begin{definition}
The {\em interior} of a segment  is .
The {\em interior} of a polygon  consists of  minus its boundary.
Two sets {\em properly intersect} (or {\em overlap}, or {\em cross}) 
if the intersection of their interiors is non-empty.
The (Euclidean) length of line segment  is denoted .
For any set  of segments, 
 denotes the total length of segments in . 

A {\em planar straight-line graph} (PSLG) is an undirected graph 
along with a planar embedding that identifies each vertex with a planar point
and each edge with the line segment connecting its endpoints,
so that each edge intersects other edges (and ) only at its endpoints.
The {\em length} of  is the sum of the Euclidean lengths of its edges.
 partitions the plane into polygonal {\em faces}.\footnote
{Where two points are in the same face if there is a path between them that intersects no edge,
with the caveat that the term {\em face} excludes the single such unbounded region.}
A face or polygon is {\em empty} if its interior contains no vertex.

A {\em diagonal}, or~{\em potential edge}, of  
is any segment  connecting two vertices of a face,
and contained in that face, so that  is still a PSLG.
A {\em partition} of  is a PSLG that extends  by adding (non-crossing) diagonals;
equivalently, the faces of the partition refine the faces of .
A {\em convex partition} of  is a partition whose faces are empty and strictly convex.
The minimum-length convex partition of  is denoted .
A {\em triangulation} of  is a partition whose faces are empty triangles.
A {\em fractional triangulation}  is a feasible solution to the LP.
For any potential edge , the {\em weight of edge}  in , denoted , 
is  if  is on
the boundary of the region to be triangulated, and otherwise half this amount.



Formally, an instance of MWT is specified by a planar point set ,
implicitly defining a PSLG  where  contains the edges 
on the boundary of the convex hull of .
A solution is a minimum-length triangulation of .
\end{definition}


Throughout, we fix an instance  of MWT specified by a given point set .
Unless stated otherwise, every graph considered is a partition of .
Since the vertex set  is the same for all such graphs, we identify each particular graph by its edge set.



\section{Integrality gap is constant}  

This section proves our first new result:

\begin{theorem}\label{thm:gap}
  Given any instance  of MWT, 
  for any fractional triangulation ,
  there exists an integer solution of value .
  That is, LP~\refLP has constant integrality gap.
\end{theorem}
\proof
Fix the MWT instance  and an arbitrary fractional triangulation .
Fix a convex partition  of .
(Later, we will fix  to be a particular convex partition  with some particular properties.)

The idea of the proof is to define a ``rounding'' procedure that 
converts  into the desired integer solution.
The procedure fractures  into 
a separate fractional triangulation  
for each face  of 
(where  covers exactly ).
Then, independently within each face  of ,
the procedure replaces the fractional triangulation  
by the optimal integer triangulation of .
The final ``rounded'' solution is then the union
of these integer triangulations (one for each face  of ),
of total cost at most  (and, hopefully, ).

In the second step, since each  is a simple polygon, it follows from 
known results (e.g.~\cite[Thm.~7]{dantzig1985triangulations}; see the introduction)
that the cost of the optimal integer triangulation of  is at most the cost of .
Thus, the integrality gap will be 
as long as the first step triangulates the faces so that
.

The proof divides into two parts:
(i) describing a correct rounding procedure that fractures  
into a fractional triangulation  
for each face  of 
(we call this {\em transposing}  into )
and
(ii) bounding the cost  by .

\subsection{Part (i) --- fracturing  into the faces of }
\label{sec:part i} \label{sec:feasibility}
Fix any face  of  of the convex partition .
Our goal is to convert  into a fractional triangulation
 of .

We start with the observation that , restricted to triangles that cross ,
can be separated into independent layers, 
where each layer is a set of triangles that uniformly covers 
(and possibly some points outside ).
We say such a layer {\em blankets} :
\begin{definition}[blanket]\label{def:blanket}
A set  of empty polygons with endpoints in 
  {\em blankets} the face  
  if the union of the polygons contains 
  and no two of the polygons overlap within 
  (they may overlap outside ).
  {\em (In this subsection, the polygons in blankets are always triangles.)}
\end{definition}

The next lemma describes how to decompose  (over ) into blankets:
\begin{lemma}\label{lemma:blankets}
There exists a set  of blankets (each containing only triangles)
  and weights  for each ,
  such that  and, 
  for every triangle  crossing ,
  
  {\em (Recall ``'' is 1 if  is in , else 0.)}
\end{lemma}
\begin{proof}
  Recall that, for MWT instances consisting of a simple polygon, 
  the LP gives optimal 0/1 solutions (e.g.,~\cite[Thm.~7]{dantzig1985triangulations}).
  We adapt a proof of that property.

Choose any triangle  that crosses  and has .
  If  completely covers , then stop and take .
  Otherwise, some edge  of triangle  crosses the interior of .
  Since  has positive weight, there must be a positive-weight triangle  
  that has  as an edge and lies on 's opposite side
  (this is implied by Constraint (\ref{constraint:edge})).
  Glue  and  together to form a polygonal region.
  Continue in this way, growing the polygonal region by
  repeatedly gluing a new triangle to any boundary edge  that crosses .
  Stop when the region has no boundary edge that crosses .
  The triangles glued together in this way form the blanket .

Let  be the minimum weight of any triangle in .
  This gives the first blanket  and its weight .
  Subtract  from each  for .
  This reduces 's coverage of  uniformly by .
  To generate the remaining blankets in  (and their weights),
  iterate this process as long as  still covers  
  with positive (and necessarily uniform) weight.
  (The process does terminate, as each iteration brings some  to zero.)
\end{proof}

Fix the set  of blankets of  from Lemma~\ref{lemma:blankets}
and the corresponding weights .

We next describe how to convert any single blanket 
into a true triangulation  of .  
The final fractional triangulation
 will be the convex combination of these
triangulations, where the triangulation of  is given weight .

Recall that any blanket  consists of triangles
that together uniformly cover the convex face  (and may extend outside of ).
To define the triangulation ,
we start with {\em edge transposals} \cite[e.g.~Lemma 4.2]{krznaric1998quasi}.
For any edge  that crosses , 
{\em transposing}  in 
slides  to its {\em transposal}, denoted ,
a diagonal of  that has minimum length 
among four or fewer diagonals that are ``near'' .
We give the formal definition next,
and then extend that to define transposals 
of triangles and blankets .
\begin{definition}[transposing an edge \cite{krznaric1998quasi}]
  Fix any triangle edge  that crosses  
  (that is, that intersects the boundary of  in two points or along an edge).
  The {\em transposal of  in },
  denoted , is defined
  by the following operation:
\begin{window}[0,r,{\includegraphics[height=1.1in]{pictures/edge_transposal}},{}]
  Clip the edge  to chord  of .
  For each endpoint  of ,
  if the endpoint lies in the interior of an edge  of  
  (as opposed to being a vertex of ),
  then slide  along  to one of the endpoints of ,
  called the {\em destination} of .
  Otherwise (the endpoint is a vertex of ), take that vertex as the destination.
  Choose the destinations (for those where there is a choice) to minimize the length
  of the diagonal that connects the destinations.  (Break ties consistently.)
  The resulting diagonal is .
\end{window}
\end{definition}

Next we define what it means to transpose {\em triangles} and {\em blankets}.
We give a somewhat uninformative formal definition, 
then describe the important properties.
\begin{definition}
  For any triangle ,
  the {\em transposal of  in },
  denoted ,
  is the convex hull of the endpoints of the transposals 
  of the edges of  that cross .


  For any blanket ,
  the {\em transposal} of  in ,
  denoted , 
  is the set containing, for each triangle , 
  the transposal  of .
  That is, .
\end{definition}

\begin{window}[3,r,{\includegraphics[height=0.9in]{pictures/morph1}},{}]
  Consider a blanket .
  By definition, the edges of triangles in  don't cross within .
  But, a-priori, their transposals might.
  We next argue that this is not the case.
  In fact, we prove more:
  roughly, that transposing preserves 
  the topology of the partition that  induces on .
  More precisely, consider that partition, which comes from
  clipping the edges of the triangles in  into 
  as shown to the right.
  Consider any edge .
  Focus on just those chords that have an endpoint in the interior of .  
  Order these chords, as shown in the first of the three pictures,
  according to the order of their endpoints on  going from  to .
  For chords sharing an endpoint on ,
  break ties in favor of chords that lean closer to .
\end{window}

\begin{lemma}[transposing preserves order]\label{lemma:ordering}
  In the above ordering of chords along ,
  all chords whose endpoints have transposal destination  
  precede all chords whose endpoints have destination .
  (Informally, when transposing the edges, when we slide the endpoints
  to their destinations, the endpoints that slide to  precede
  the endpoints that slide to , so no crossings are introduced.)
\end{lemma}

\begin{proof}
Without loss of generality, assume that  lies (one vertex) clockwise of .
\begin{window}[2,r,{\includegraphics[width=0.9in]{pictures/morph2}},{}]
\noindent Focus on the chords  where  is in the interior of .
Let  contain those whose endpoint  has destination .
Let  contain those whose endpoint  has destination .
We show that, if we leave  and travel {\em counterclockwise} around the boundary to ,
we encounter the chords in  before we encounter the chords in .
This proves the claim, because, as chords in  don't cross, 
as we travel counterclockwise
we must encounter chords in the same order
that we would if traveling clockwise from  to .
\end{window}

\begin{window}[0,r,{\smallskip\includegraphics[width=0.9in]{pictures/morph3}},{}]
Consider the perpendicular bisector of .
Since  is convex, the bisector intersects the boundary of 
at a single point across from .
Suppose that this intersection point is in the interior of some edge 
as shown to the right.  
(If the intersection is a vertex of , take  to be that point
and follow similar reasoning.)

\hspace*{\parindent}As we travel counterclockwise from  to ,
until we pass the first endpoint of , every chord endpoint  that we encounter 
is in some edge  of  that lies entirely on the -side of the bisector.
Since both endpoints of  are closer to  than to ,
no matter which endpoint of  is the destination of ,
the destination of  will definitely be .
Thus, until we pass the first endpoint of , we encounter only chords in .
\end{window}

As we travel through the interior of edge  (or, if  is a point, through )
for all chord endpoints  that we encounter,
their chords  will have the {\em same} transposal
(since  is in the interior of  
and  is in the interior of , and transposing breaks ties consistently).
Thus, traveling through , either we encounter only chords in ,
or we encounter only chords in .

Once we pass reach the other endpoint of , until we reach ,
every chord that we encounter
is an edge of  that lies entirely on the -side of the bisector,
so, reasoning as before, we encounter only chords in .
\end{proof}

Because transposing preserves order, 
the topological structure of the {\em transposal} of any blanket 
is inherited from the the partition that  induces on .
Here is an example:

\noindent\includegraphics[width=\textwidth]{pictures/transposals_of_triangles}

Above are five copies of a face  (with gray background).
Copy 1 shows the face blanketed by six triangles.
In copy 2, the triangle edges are clipped to their chords in the face,
giving the partition that  induces on .
In copies 3 through 5, each chord is shifted to its edge transposal,
by sliding each endpoint to its destination.
Copy 5 shows the resulting edge transposals,
and the transposal of  in .

Clearly, in the partition that  induces on  (copy 2)
each region is of the form  for some .
Because transposing preserves order, moving the edges of that partition
to their transposals preserves the topological structure of the partition:
the transposal  of  (copy 5)
is a convex partition of 
whose edges are the transposals of the edges of ,
and whose regions are the transposals of the triangles in .
Also, for each triangle ,
the boundary of its transposal 
consists of the transposals of the edges of ,
together with up to three edges of .

The transposal of a blanket is a convex partition of the face,
but not quite a triangulation, because each of its regions may have up to six sides.
To get a triangulation, we simply triangulate each of its regions:

\begin{definition}
The {\em triangulated transposal of a triangle  in },
denoted ,
is the minimum-weight triangulation of the transposal ,
except that, if   has no area, then 
is the empty set.
The {\em triangulated transposal of a blanket  in },
denoted ,
is the union of the triangulated transposals of the triangles in the blanket.
\end{definition}

The transposal  is a convex partition of 
whose regions are the transposals of the triangles in ,
so the triangulated transposal of  indeed triangulates .

Finally, we define the fractional triangulation  of .
We start with the fractional triangulation .
We restrict  to triangles crossing .
We decompose this restriction of 
into a convex combination of blankets of 
(per Lemma~\ref{lemma:blankets}).
Then, in this convex combination, we replace each blanket 
by its triangulated transposal , a triangulation of .\footnote
{Alternatively, we could take  to be
the {\em cheapest} triangulation  
over all blankets .}
Here is the formal definition:
\begin{definition}
  \label{def:triangulation_transposal}
  Define the {\em transposal of  in },
  denoted , 
  to be the fractional triangulation of 
  formed by the convex combination 
  of the transposals of the blankets in , so that
  
\end{definition}

We now complete Part (i) of the proof of Thm.~\ref{thm:gap}:
\begin{lemma}\label{thm:feasibility}
  Fix any fractional triangulation  and any convex face .
  The transposal  of  in  
  defined above
  is a feasible fractional triangulation of .
  That is, it covers the points in  uniformly with weight 1.
\end{lemma}

\begin{proof}
As discussed, this holds because  is a convex combination
of triangulations of .
Indeed, it covers each point  in  with total weight

The first equality is by definition of .
The second just exchanges the order of summation.
The third holds because  triangulates 
(so exactly one  contains ).
The last follows by Lemma~\ref{lemma:blankets}.
\end{proof}

\subsection{Part (ii) --- bounding the cost}
Fix the convex partition  and fractional solution .
By Lemma~\ref{thm:feasibility}, for each face  of ,
the transposal  as defined in Part (i)
is a fractional triangulation of .
To complete the proof of Thm.~\ref{thm:gap},
we bound the sum of the costs of these fractional triangulations.

We start by observing that we can view 
as taking the weight of each triangle  in ,
and transferring that weight to (every triangle in) the triangulated transposal 
 of  in :
\begin{fact}\label{observation:cost}
  
\end{fact}

\begin{proof} 

The first equality is the definition of .
The second holds by definition of 
(namely, 
iff
 for some ).
The third just exchanges the order of summation.
The last follows from Lemma~\ref{lemma:blankets}.
\end{proof}

So far, we've considered how a blanket of triangles transposes into a single .
Next we consider how a single triangle  transposes across multiple faces.
Of course, a given triangle  can cross many faces,
but {\em in all but two} its transposal will have no area
(and thus will play no part in the triangulated transposal of  in ):

\begin{lemma}\label{lemma:transposals}
  Any given triangle  crosses at most two faces  in 
  in which its transposal  has positive area.
  Thus, for a given , only two faces  have
  .
\end{lemma}

\begin{proof}
Fix a triangle  and consider how the faces of  can overlap .
Say that a face  is {\em accommodating} if 's transposal  in  has positive area.

\begin{window}[0,r,{\scalebox{.5}{\xfig{transposal-configs}}},{}]
  In the two examples to the right,
  each dashed edge is an edge transposal of an edge of .
  Within each accommodating face, the (positive area) transposal of  is dark.

  \hspace*{\parindent}We claim that {\em every accommodating face touches all three edges of }.
  (A face ``touches'' an edge if the intersection of the face and the edge,
  including boundaries and endpoints, is non-empty.
  For example, the accommodating face 2 on the left of the figure, and 2 and 5 on the right, 
  touch all three edges of .  
  Each other face is non-accommodating and, 
  except for 3 and 4 on the right, touches only two edges of .)
\end{window}

The claim holds because, if a face  touches only two edges of ,
the third edge of  lies outside of ,
so the two edges cross the interior of a single edge of .
Thus, the two edges of  that touch  must have identical transposals,
forcing  to have no area.

Now consider the case that  has a face  that touches the {\em interior}
of all three edges of  (as in the figure to the left, above).
Since the faces are non-overlapping and convex,
no face other than  can touch all three edges of .
By the claim, then, only face  might be accommodating, so the lemma holds.

So assume that no face touches the interiors of all three edges of .

By the claim, any accommodating face  still has to touch all three edges of ,
but now there is at least one edge, say , of  whose interior  avoids.
Thus,  must touch  at an endpoint, say, .
(For example, consider the figure on the right above.
Faces 2, 3, 4, and 5 touch all three edges of , but not all three interiors.)
Since  touches  at , but does not touch the interior of ,
there must be an edge  of  that extends through the interior of .
Since  is not inside ,  must cut across  to the interior of the edge .
Thus, 


If there are two accommodating faces, they must extend an edge across  
from the {\em same} vertex , for otherwise the extending edges would cross inside .

Now consider all edges in  that extend from  across the interior of .
Let these edges be , rotating in order around .
(In the picture above, .)
 has  corresponding faces , 
also in order rotating around ,
where  and  share edge .
By the conclusion (\ref{eqn:quote}) of the paragraph before last,
only these  faces might be accommodating.

 To finish, we observe that  is not accommodating
unless  (the first or last face).
Indeed, for  edges  and  of  
extend from  across  to .
Since these edges touch at ,
the transposal of  in  is thus {\em just the point }.
Thus, the transposal of  in  has no area.
\end{proof}

Our goal is to show that transposing  across the faces
increases the cost of  by at most a constant factor.
For any triangle , 
by the lemma and Fact~\ref{observation:cost}, 
transposing 
transfers the weight  
to the triangulated transposals  of 
in {\em at most two} faces .

To proceed we bound the cost of each
 in terms of the cost of .
Recall that  is the minimum-weight triangulation of
its (non-triangulated) transposal ,
which has at most six sides (up to three edges of ,
and up to three transposals of edges of ).
We start by bounding the cost of .
Our bound depends on the {\em sensitivity} of the edges of the convex partition ,
defined as follows:

\begin{definition}[sensitivity]\label{def:sensitivity}
An edge  is {\em -sensitive} if, 
  for any potential edge  that crosses ,
  for each endpoint  of ,
  the distance from  to the closest endpoint of 
  is at most .
\end{definition}

(In other words, the circle of radius 
around each endpoint of  contains an endpoint of .)

For the rest of the section, fix  
such that all edges of  are -sensitive.

\begin{lemma}\label{lemma:sensitive}
For any face  of  and triangle ,
  the total length of the edges in 's transposal 
  that are not also edges of 
  is at most  times the length of 's edges.
\end{lemma}

\begin{proof}  Let  be any face of  and  be any edge that crosses . 

    We claim that {\em the length of the edge transposal 
      of  in  is at most  times the length of }.
    This claim implies the lemma,
    because each edge of the transposal of  (but not of )
    is the edge transposal 
    of a unique edge  of .
    To finish, we prove the claim.
  \begin{window}[0,r,{\scalebox{.5}{\xfig{transposal-length}}},{}]
    For an edge  that crosses , one of the following three cases holds:
    {\bf (1)}  is incident to one vertex of  and properly intersects one  side of 
    (as in the figure immediately to the right),
    {\bf (2)}  properly intersects two sides  and  of 
    (as in the figure to the far right),
    or
    {\bf (3)}  is incident to two vertices of .


\hspace*{\parindent}In case (1), let  be the vertex that  shares with  (and ).
 Since  is -sensitive, and  crosses ,
 the endpoint  of  is at most  
 from some endpoint of .
 Since  is the minimum distance
 between  and any endpoint of , this implies
 .

 \end{window}


 In case (2), let  be an endpoint of  and let  and  respectively be 
 the closest endpoints of  and  to .
 Because  is the shortest segment 
 from an endpoint of  to an endpoint of ,
 .
 By the triangle inequality,
 
 Because  and  are -sensitive,
  and  are each at most .

 In case (3), the transposal  of  is the same as , 
 so  the claim holds.
\end{proof}

It is straightforward to extend the bound to the {\em triangulated} transposal 
 of .
Recall that the cost of a triangle  is the sum of the costs of its edges,
where the cost of an edge is half its length, unless the edge is on the boundary
of the entire region, in which case the cost of the edge is its length.
The cost 
of a triangulation 
is the sum of the costs of the triangles 
in the triangulation.

\begin{lemma}\label{lemma:6-gon}
  For any face  and any triangle , 
  the cost 
  of the triangulated transposal of  in 
  is at most three times the cost  
  of the (non-triangulated) transposal of  in .
\end{lemma}

\begin{proof}
  Recall that  has at most six vertices,
  say, , ordered clockwise.
  Triangulate  by adding up to three
  interior diagonals connecting the odd vertices
  (e.g.~, , ).
  The total length of the added diagonals
  is at most the total length of the boundary.
  Likewise, the sum of costs of the added diagonals
  is at most the sum of the costs of the edges on the boundary of .

  Each added edge occurs in two triangles in this triangulation,
  whereas each boundary edge occurs in just one triangle.
  Thus, adding the diagonals gives a triangulation of cost
  at most three times the cost of .
  The lemma follows, as  is the minimum cost
  of any triangulation of .
\end{proof}

Next we gather the bounds in the previous lemmas 
to bound the total cost across all the faces.
We are not finished, 
as the bound depends on not only the cost of the fractional triangulation,
but also the total length of the edges in the convex partition 
and the sensitivity  of those edges:

\begin{lemma}\label{bound}
  The total cost  is at most
  

\end{lemma}

\begin{proof}
The total cost is

\end{proof}

To proceed further,
we need a convex partition whose edges have constant sensitivity
and total length .
Levcopoulos and Krznaric have shown the existence 
of something close: a convex partition  whose edges are 4.45-sensitive
and have total length 
(recall that  is the minimum-length convex partition of ):

\begin{lemma}[\cite{krznaric1998quasi}]\label{lemma:LK}
  For some constant , for any MWT instance ,
  there exists a convex partition  of ,
  whose edges are 4.45-sensitive,
  having total length .
\end{lemma}

\begin{proof}
  Levcopoulos and Krznaric 
  show that what they call the {\em quasi-greedy convex partition}
  has these properties:
  for Property (1), see their Lemma 5.4 and the discussion before it;
  for Property (2), see their Corollary 5.3   \cite{krznaric1998quasi}.
\end{proof}

This convex partition will work for us:
we prove next that .

(Note that  is trivially at most the cost of any {\em integer}
triangulation, but the bound here concerns the {\em fractional} triangulation,
so requires proof.)

The proof uses the constraints on  and leverages
a previous analysis of 
due to Plaisted and Hong \cite[Lemma 10]{plaisted1987heuristic}.
\begin{lemma}\label{lemma:fracmcp}
 
\end{lemma}

\begin{proof}
  For every vertex  in the interior of the convex hull of the vertex set , 
  define a {\em star} at  to be a subset of edges incident to 
  such that no two successive edges (around ) 
  are separated by an angle of 180 degrees or more.
  For every vertex  on the boundary of the convex hull of , 
  define the (only) star at  to consist
  of the two boundary edges  incident to .
Let  denote the minimum cost of any star at .
  Plaisted and Hong show 
  \cite[Lemma 10]{plaisted1987heuristic}.

  We claim , where  is  if  is on the boundary of the convex hull, 
  and otherwise half this amount.
  As ,
  the claim implies the lemma.
  We prove the claim.

  It's easy to see that, for any boundary vertex , , 
  so restrict attention to just an interior vertex  and its edges.

  \newcommand{\wrap}{\text{wrap}}

\begin{window}[0,r,{\includegraphics[width=1.3in]{pictures/helicoid}},{}]
    Because  satisfies Constraint (\ref{constraint:edge}),
    rotating clockwise around , 
    there is a sequence 
    of distinct vertices
    such that for each ,
    the triangle 
    has positive weight in .
    (To find the sequence, take any positive-weight triangle that has  as a vertex.
    Let  and  be the other vertices, in clockwise order.
    By Constraint (\ref{constraint:edge}), there is a positive-weight triangle 
    that shares edge  and lies clockwise of that edge.
    Let  be the other vertex of that triangle.
    By Constraint (\ref{constraint:edge}), there is a positive-weight triangle 
    that shares edge  and lies clockwise of that edge.
    Let   be the other vertex of that triangle.
    Continue, stopping when the next vertex 
    that would be added is  ---
    this must happen by Constraint (\ref{constraint:edge}).)
  \end{window}
  
  Let  and  denote edge  
  and triangle , respectively.
  Note that each edge, and each triangle, is distinct.
  Call the sequence of edges  a {\em helix}.
  Let  denote the number of times  wraps around .
  By a standard construction
the 's 
  can be expressed as a linear combination of incidence vectors of helices.
  (Similar to Lemma~\ref{lemma:blankets}'s proof,
  repeatedly find a helix , choose weight ,
  and subtract  from each triangle  in the helix,
  reducing coverage near  by .)
  This gives a probability distribution  on helices 
  such that each .

Now choose a helix  at random from the probability distribution .
  Partition  greedily into contiguous subsequences of edges such that each
  group  is maximal subject to the constraint that the total clockwise angle around 
  swept by the group's edges is at most .
  (In the figure, white triangles separate groups.)
  Consideration shows that each group contains a star,
  and (as neighboring groups are separated by at most ),
  there are at least  groups.

  From the randomly chosen , choose a group  uniformly at random 
  from 's first  groups.
  For any given edge , the probability that  is in  is at most
  .
  Thus, by linearity of expectation, the expected total length  of edges in  
  is at most .
  On the other hand, every  contains a star, so .
  This proves Lemma~\ref{lemma:fracmcp}.
\end{proof}

For the rounding procedure,
fix the (previously arbitrary) convex partition  
to be the partition  from Lemma~\ref{lemma:LK}.
The cost of the final triangulation is at most

Hence, the integrality gap is at most ,
completing the proof of Thm.~\ref{thm:gap}.
\qed





\section{LP generalizes heuristics}\label{sec:heuristics}

This section proves our second new result 
(the LP generalizes MWT heuristics).
Here is a summary of heuristics for determining that 
a given potential edge  of  is in every MWT 
of a given MWT instance :

\newcommand{\thing}[2]{\smallskip \par\item[#1]{#2}}

\begin{description}
  \thing{-skeleton:}
  {
    For  where ,
    there does not exist a point 
    in the point set such that .
    Equivalently, 
    the two disks of diameter  having  as a chord are empty
    of points.
    If this condition holds, then  is in every MWT of 
    \cite{keil1994computing,cheng1996approaching}.
  }
  \thing{-subgraph:}
  {
For every potential edge  that crosses ,
    its size  is at most .
    If this condition holds, then  is in every MWT of 
    \cite{yang1994chain,gilbert1979new}.
  }
  \thing{maximality:}
  {
    For every potential edge that crosses , that edge is known to be {\em excluded from} every MWT.
    If so, then  is in every MWT of 
    (see e.g.~\cite{dickerson1997large}).
  }
\end{description}
\smallskip

Here is a summary of heuristics for determining that 
a given potential edge  of  
(not on the boundary of the region to be triangulated) 
is excluded from every MWT of :

\begin{description}
  \thing{independence:}
  {
  Some potential edge that crosses  is known to be in every MWT.
  If this condition holds, then  is not in any MWT of 
  (see e.g.~\cite{dickerson1997large}).
  }
  \thing{diamond:}
  {
  Neither of the two triangles with base  and base angle  are empty.
  If this condition holds, then  is not in any MWT of 
  \cite{das1989triangulations,drysdale2001exclusion}.
  }
  \thing{LMT skeleton:}
  {
    For every two triangles  and  for which  is {\em locally minimal},
    one of the edges of  or  is known to be excluded from every MWT.
    If this condition holds, then  is not in any MWT of 
    \cite{dickerson1997large}.
    (Edge  is {\em locally minimal} 
    for two triangles  and 
    if  and
     and  together are a minimum-length triangulation of 
the quadrilateral  ---
    that is, either  is non-convex,
    or  is no longer than the other diagonal of .)
  }
\end{description}
\smallskip

\noindent
Let  denote the set of edges that can be deduced to be in every MWT
by applying the logical closure of the above six rules.
(Logical closure is necessary because the maximality, independence, and LMT-skeleton conditions
depend on the known statuses of edges other than .
For example, if one of the conditions implies that some edge  is excluded from every MWT,
then the LMT-skeleton condition may then in turn 
imply that some new edge  is excluded from every MWT,
because  lies on one of two triangles  or  in the pair for which  is locally minimal.)

Ideally, the set  gives a partition of  in which every face is empty.
If this happens, then the remaining edges in the MWT can be found
by triangulating each remaining face independently using the standard
dynamic-programming algorithm, and we say  is {\em solvable} via the heuristics.
According to \cite{dickerson1997large} (1997),
most random instances with as many as 40,000 points are solvable via the heuristics.\footnote
{\cite{dickerson1997large} define the modified LMT-skeleton to be
the set of edges that can be deduced to be in every MWT via 
(the logical closure of) just the 
diamond, LMT-skeleton, maximality, and independence conditions above.
The use of logical closure is crucial to the effectiveness of the LMT skeleton.
}  

Here is our second new result.  If an instance is solvable via the heuristics,
then LP \refLP solves the instance also:
\begin{theorem}\label{thm:heuristics}
  For any instance  of MWT,
  let  be the partition of  defined above.
  If every face of  is empty, then every optimal extreme point of the LP (for )
  is the incidence vector of a minimum-length triangulation.
\end{theorem}

The remainder of the section gives the proof. The first step is to show that each condition above that ensures that an edge is in
(or excluded from) every MWT also ensures that the LP gives the edge weight 1 (or 0)
in any optimal fractional solution.

Say that LP~\refLP\ {\em forces a potential edge  to } (where )
if, for every optimal fractional triangulation  of ,
the weight that  gives to  is .

\begin{lemma}\label{lemma:in}
  If any of the following conditions holds, the LP forces potential edge  of  to 1.
  \begin{enumerate}
  \item \label{in:beta}
    {\bf -skeleton:} The -skeleton condition above holds for .
\item \label{in:YXY}
    {\bf -subgraph:} The -subgraph condition above holds for .
\item \label{in:maximality}
    {\bf maximality:}
    The LP forces every potential edge that crosses  to 0.
  \end{enumerate}
\end{lemma}

\begin{proofidea}
  Part~(\ref{in:maximality}) is relatively straightforward:
  if  gives weight 0 to every edge that crosses ,
  then no triangle  that crosses  has positive , 
  so the only way  can cover points near  is with triangles that have  as a side.

  The original -skeleton and the -subgraph heuristics 
  are shown to be valid for MWT by local-improvement arguments:
  if the condition holds for an edge  that is {\em not} in the MWT,
 then a polygon  covering  within the MWT can be retriangulated at lesser cost,
  contradicting the optimality of the MWT
  \cite{keil1994computing,cheng1996approaching,yang1994chain,gilbert1979new}.
Here we extend those arguments to any optimal {\em fractional} triangulation :
  if the condition holds and  does not give  weight 1,
  then a polygon  covering  whose triangles have positive weight in 
  can be retriangulated (lowering the weight of those triangles by 
  and raising the weight of other triangles by ),
  giving a fractional triangulation that costs less than .

  The original arguments are intricate geometric case analyses, typically taking several
  pages.  The arguments do not extend completely to our setting for the following reason:
  in the MWT setting, the polygon  identified for re-triangulation 
  is the union of non-crossing triangles,
  whereas here, in the fractional setting, 
  the polygon  is the union of triangles that {\em can} cross
  (much as in Lemma~\ref{lemma:blankets}).
  If the triangles in  do not cross, then the original arguments apply,
  but in general additional analysis is needed.
\begin{window}[0,r,{\resizebox{2.5in}{1.25in}{\begin{picture}(0,0)\includegraphics{pictures/tsequence.pdf}\end{picture}\setlength{\unitlength}{4144sp}\begingroup\makeatletter\ifx\SetFigFont\undefined \gdef\SetFigFont#1#2#3#4#5{\reset@font\fontsize{#1}{#2pt}\fontfamily{#3}\fontseries{#4}\fontshape{#5}\selectfont}\fi\endgroup \begin{picture}(9952,6118)(1536,-7021)
\put(2341,-3661){\makebox(0,0)[lb]{\smash{{\SetFigFont{25}{30.0}{\rmdefault}{\mddefault}{\updefault}{\color[rgb]{0,0,0}}}}}}
\put(4816,-4156){\makebox(0,0)[lb]{\smash{{\SetFigFont{25}{30.0}{\rmdefault}{\mddefault}{\updefault}{\color[rgb]{0,0,0}}}}}}
\put(2001,-4111){\makebox(0,0)[lb]{\smash{{\SetFigFont{25}{30.0}{\rmdefault}{\mddefault}{\updefault}{\color[rgb]{0,0,0}}}}}}
\put(8361,-3901){\makebox(0,0)[lb]{\smash{{\SetFigFont{25}{30.0}{\rmdefault}{\mddefault}{\updefault}{\color[rgb]{0,0,0}}}}}}
\put(5376,-4501){\makebox(0,0)[lb]{\smash{{\SetFigFont{25}{30.0}{\rmdefault}{\mddefault}{\updefault}{\color[rgb]{0,0,0}}}}}}
\put(1551,-4486){\makebox(0,0)[lb]{\smash{{\SetFigFont{25}{30.0}{\rmdefault}{\mddefault}{\updefault}{\color[rgb]{0,0,0}}}}}}
\put(11473,-4501){\makebox(0,0)[lb]{\smash{{\SetFigFont{25}{30.0}{\rmdefault}{\mddefault}{\updefault}{\color[rgb]{0,0,0}}}}}}
\put(7584,-4486){\makebox(0,0)[lb]{\smash{{\SetFigFont{25}{30.0}{\rmdefault}{\mddefault}{\updefault}{\color[rgb]{0,0,0}}}}}}
\put(10621,-3886){\makebox(0,0)[lb]{\smash{{\SetFigFont{25}{30.0}{\rmdefault}{\mddefault}{\updefault}{\color[rgb]{0,0,0}}}}}}
\end{picture} }},{}]
    To illustrate, consider the -skeleton.
    Suppose for contradiction that the -skeleton condition
    holds for an edge  but  does not occur in the MWT.
    The previous work \cite{keil1994computing,cheng1996approaching} 
    shows that there must be
    a sequence  of empty triangles in the MWT whose union 
    covers  as shown in the left of the figure to the right.
    Using the -skeleton condition, they show that this union 
    has a triangulation that costs less than does , 
    contradicting the optimality of the MWT.

    \hspace*{\parindent}
    In the current context, if  has weight below 1 in ,
    then there must (similarly) exist a sequence  of empty triangles
    with positive weight in  covering , 
    but these triangles can {\em cross} (see the example to the right above).
    We extend their arguments to show that, even if such
    crossing occurs, a triangulation of lower cost can still be found.\end{window}\end{proofidea}




\begin{fullproof}
Here are the details of the proofs for Part \ref{in:beta} (-skeleton) and Part \ref{in:YXY} (-subgraph).
Part \ref{in:maximality} (maximality) is discussed already above, in the proof idea.




\medskip
\noindent
{\bf Part \ref{in:beta} (-skeleton):}
The original -skeleton heuristics 
are shown to be valid for MWT by local-improvement arguments:
if an edge  is in the -skeleton (for ) but {\em not} in the MWT,
then a polygon  covering  within the MWT can be retriangulated at lesser cost,
contradicting the optimality of the MWT \cite{keil1994computing,cheng1996approaching}.
We briefly sketch their argument and then extend it to any optimal {\em fractional} triangulation .

Assume for the remainder of this section that  goes horizontally from 
the point  on the left to the point  on the right.
If  is not in the MWT, there exists a set of MWT edges that intersect . 
Let , be the set of edges indexed in  
non-decreasing order of their length. 
If the edges are removed from the MWT, an empty polygonal region  results. 
In \cite{keil1994computing,cheng1996approaching}
it is shown that 
 can be retriangulated at lesser cost by a set of edges that contains . 
The idea is to generate a sequence of triangulated 
polygons  such that  is the degenerate polygon ,  is 
a triangulation of  and . To obtain ,  is expanded 
to include the endpoints  and  of . Assume  is above the line through 
 and  is below it. 
If both  and  already lie on the boundary of  then . 
Otherwise, at least one of them will not be on the boundary of .
Assume without loss of generality  is not on the boundary of  
(If  is also outside , it will be dealt with similarly).

Since  is not on the boundary of , edge  intersects a boundary edge  of .
Consider the sequence  of vertices on the path from  to  on the boundary of 
(there are two such paths, but the one above the line through  is intended).
On the sequence , vertex  is the last vertex before  that belongs to  and 
 is the first vertex after  that belongs to . 
This observation allows us to clearly define  in 
the fractional setting because in that setting, polygon  may be self-intersecting 
and  may intersect more than one boundary edge of  in the half-space above .
In general, the triangle  contains a
subsequence  of vertices on  from  to  
and another subsequence  from  to . 
The polygon  is then triangulated arbitrarily, 
and  is the union of  and the triangulated polygon  and 
possibly another triangulated polygon to include  if  is not on the boundary of .
\begin{window}[0,r,{\resizebox{2in}{1.8in}{\begin{picture}(0,0)\includegraphics{pictures/retriangulate2.pdf}\end{picture}\setlength{\unitlength}{4144sp}\begingroup\makeatletter\ifx\SetFigFont\undefined \gdef\SetFigFont#1#2#3#4#5{\reset@font\fontsize{#1}{#2pt}\fontfamily{#3}\fontseries{#4}\fontshape{#5}\selectfont}\fi\endgroup \begin{picture}(4925,5371)(1126,-5120)
\put(3781,-1726){\makebox(0,0)[lb]{\smash{{\SetFigFont{20}{24.0}{\rmdefault}{\mddefault}{\updefault}{\color[rgb]{0,0,0}}}}}}
\put(4096,-16){\makebox(0,0)[lb]{\smash{{\SetFigFont{20}{24.0}{\rmdefault}{\mddefault}{\updefault}{\color[rgb]{0,0,0}}}}}}
\put(2701,-5011){\makebox(0,0)[lb]{\smash{{\SetFigFont{20}{24.0}{\rmdefault}{\mddefault}{\updefault}{\color[rgb]{0,0,0}}}}}}
\put(4777,-1527){\makebox(0,0)[lb]{\smash{{\SetFigFont{20}{24.0}{\rmdefault}{\mddefault}{\updefault}{\color[rgb]{0,0,0}}}}}}
\put(5626,-2536){\makebox(0,0)[lb]{\smash{{\SetFigFont{20}{24.0}{\rmdefault}{\mddefault}{\updefault}{\color[rgb]{0,0,0}}}}}}
\put(2626,-2041){\makebox(0,0)[lb]{\smash{{\SetFigFont{20}{24.0}{\rmdefault}{\mddefault}{\updefault}{\color[rgb]{0,0,0}}}}}}
\put(2701,-1186){\makebox(0,0)[lb]{\smash{{\SetFigFont{20}{24.0}{\rmdefault}{\mddefault}{\updefault}{\color[rgb]{0,0,0}}}}}}
\put(1141,-2491){\makebox(0,0)[lb]{\smash{{\SetFigFont{20}{24.0}{\rmdefault}{\mddefault}{\updefault}{\color[rgb]{0,0,0}}}}}}
\end{picture} }},{}]
This construction is shown in the figure to the right.
The polygon with dashed boundary is , and  is a triangulation of the dark gray polygon.
The union of  and arbitrary triangulations of the light gray polygons is  
which includes the endpoints  and  of . 
The light gray polygon above  is .
The white vertices inside triangle  are
, and the black vertices inside the triangle are .

\hspace*{\parindent}
In \cite{keil1994computing,cheng1996approaching} it is shown by induction on  that 
for every  all edges in the triangulation of  are shorter than .
By induction all the edges of  are shorter than  
and . Thus it remains to show that all the new edges that are added to  to 
form  are shorter than . The new edges triangulate the polygon  
and they are all contained in triangle , so each new edge is shorter than .
Since , . It remains to show that .                                     
The argument for  is similar.
The following two facts are used for this part of the argument.  
\end{window}


\begin{fact}[{\cite[Lemma 2]{keil1994computing}}]\label{prop:length}
Let  be an edge in the -skeleton (where ). 
For any edge , if  intersects , then it has length  greater than 
.
\end{fact}

\begin{fact}[{\cite[Remote Length Lemma]{cheng1996approaching}}]
\label{prop:remote}
Let  be an edge in the -skeleton (where ).
Let  and  be four other distinct points of the point set 
such that  intersects ,  intersects ,  
does not intersect , and  and  lie on the same side of the line through . 
Then .
\end{fact}

The argument to show  is as follows.
If  lies in triangle , then
.
The second inequality holds based on Fact~\ref{prop:length}.
If  is outside , consider the convex hull of the path from 
 to  on . 
Vertex  must lie in some triangle  where  and  are hull vertices. Thus, 
. Since  and  are hull vertices, 
they were added in the growth process in the past. Thus, the edges  and  with endpoints 
 and  were processed before , so neither  nor  is longer
than . Combining this observation with Fact~\ref{prop:remote} gives . 
Using a similar argument, one can show that  and  are both shorter than . 
Thus, . This completes the proof that 
all the edges of  are shorter than , and thus the new triangulation of  costs less.

Next we extend the above arguments to any optimal {\em fractional} triangulation . 
If  does not give  weight 1,
there is a triangle  with positive  that properly intersects .
For any side  of  that intersects  there must be a triangle  
with positive  that has  as a side and lies on the other side of  from . 
The existence of a triangle  is a consequence of Constraints (\ref{constraint:edge}). 
Repeating the same argument for the new triangle(s) gives a set of triangles that cover . 

\begin{window}[0,r,{\resizebox{3.5in}{1.7in}{\begin{picture}(0,0)\includegraphics{pictures/tsequence.pdf}\end{picture}\setlength{\unitlength}{4144sp}\begingroup\makeatletter\ifx\SetFigFont\undefined \gdef\SetFigFont#1#2#3#4#5{\reset@font\fontsize{#1}{#2pt}\fontfamily{#3}\fontseries{#4}\fontshape{#5}\selectfont}\fi\endgroup \begin{picture}(9952,6118)(1536,-7021)
\put(2341,-3661){\makebox(0,0)[lb]{\smash{{\SetFigFont{25}{30.0}{\rmdefault}{\mddefault}{\updefault}{\color[rgb]{0,0,0}}}}}}
\put(4816,-4156){\makebox(0,0)[lb]{\smash{{\SetFigFont{25}{30.0}{\rmdefault}{\mddefault}{\updefault}{\color[rgb]{0,0,0}}}}}}
\put(2001,-4111){\makebox(0,0)[lb]{\smash{{\SetFigFont{25}{30.0}{\rmdefault}{\mddefault}{\updefault}{\color[rgb]{0,0,0}}}}}}
\put(8361,-3901){\makebox(0,0)[lb]{\smash{{\SetFigFont{25}{30.0}{\rmdefault}{\mddefault}{\updefault}{\color[rgb]{0,0,0}}}}}}
\put(5376,-4501){\makebox(0,0)[lb]{\smash{{\SetFigFont{25}{30.0}{\rmdefault}{\mddefault}{\updefault}{\color[rgb]{0,0,0}}}}}}
\put(1551,-4486){\makebox(0,0)[lb]{\smash{{\SetFigFont{25}{30.0}{\rmdefault}{\mddefault}{\updefault}{\color[rgb]{0,0,0}}}}}}
\put(11473,-4501){\makebox(0,0)[lb]{\smash{{\SetFigFont{25}{30.0}{\rmdefault}{\mddefault}{\updefault}{\color[rgb]{0,0,0}}}}}}
\put(7584,-4486){\makebox(0,0)[lb]{\smash{{\SetFigFont{25}{30.0}{\rmdefault}{\mddefault}{\updefault}{\color[rgb]{0,0,0}}}}}}
\put(10621,-3886){\makebox(0,0)[lb]{\smash{{\SetFigFont{25}{30.0}{\rmdefault}{\mddefault}{\updefault}{\color[rgb]{0,0,0}}}}}}
\end{picture} }},{}]
Let  be the sequence of triangles in the order they intersect 
 in the direction from  to . Triangle  is incident on  and  is incident on . 
All triangles have a positive weight in .
The triangles in the sequence may or may not cross each other as shown in the figure to the right.
Let  (the shaded area in the figure) be the polygon formed by the boundary edges of triangles in the sequence.
As shown in the right figure polygon  is self-intersecting if some of the triangles in the sequence cross.
Next, we consider both cases and derive a contradiction in each case.
\end{window}




\noindent{\em Case 1} --- no triangles in  cross.  For this case we apply directly the technique used in 
\cite{keil1994computing} and \cite{cheng1996approaching} to retriangulate the interior of  at lower cost.
Lowering the weight of those triangles in  by 
and raising the weight of new triangles by ,
gives a fractional triangulation of cost less than .



\noindent
{\em Case 2 ---} some triangles in  cross.  In this case the technique of \cite{keil1994computing} and 
\cite{cheng1996approaching} cannot be directly applied because,
in that setting, the polygon  identified for retriangulation 
is the union of non-crossing triangles,
whereas in this case,  is the union of triangles that cross.

Let  be the set of edges of triangles in  that intersect  
indexed in the order they intersect  (in the direction from  to ).
The only part of the argument used in \cite{keil1994computing} and 
\cite{cheng1996approaching} that doesn't go through concerns Fact~\ref{prop:remote}. 
Fact~\ref{prop:remote} holds for any pair of edges  and  that 
intersect  and do not intersect each other. Recall that in the MWT setting the edges intersecting  
do not intersect each other, so Fact~\ref{prop:remote} holds for any pair 
of those edges. However, in the current case some edges in  may intersect each other. 
Thus, Fact~\ref{prop:remote} does not automatically hold in this case.  
This issue is resolved by the following technical lemma.
\begin{fact}\label{obs:point}
Let  and  be two edges of triangles  such that 
 intersects ,  intersects ,  
intersects , and  and  lie on the same side of the line through .
Then  and  are both shorter than .  
\end{fact}



\begin{proof}
Assume that endpoints  and  are above the line through .
Let  be the intersection point of  and , and assume without loss of generality that 
point  is also above the line through .
Let  and  be the intersection points of  and  with .
Assume  so that . 
\begin{window}[2,r,{\resizebox{2.3in}{1.8in}{\begin{picture}(0,0)\includegraphics{pictures/beta3.pdf}\end{picture}\setlength{\unitlength}{4144sp}\begingroup\makeatletter\ifx\SetFigFont\undefined \gdef\SetFigFont#1#2#3#4#5{\reset@font\fontsize{#1}{#2pt}\fontfamily{#3}\fontseries{#4}\fontshape{#5}\selectfont}\fi\endgroup \begin{picture}(4900,4705)(1316,-5703)
\put(6201,-4038){\makebox(0,0)[lb]{\smash{{\SetFigFont{20}{24.0}{\rmdefault}{\mddefault}{\updefault}{\color[rgb]{0,0,0}}}}}}
\put(4186,-3796){\makebox(0,0)[lb]{\smash{{\SetFigFont{20}{24.0}{\rmdefault}{\mddefault}{\updefault}{\color[rgb]{0,0,0}}}}}}
\put(5086,-4696){\makebox(0,0)[lb]{\smash{{\SetFigFont{17}{20.4}{\rmdefault}{\mddefault}{\updefault}{\color[rgb]{0,0,0}}}}}}
\put(1331,-3993){\makebox(0,0)[lb]{\smash{{\SetFigFont{20}{24.0}{\rmdefault}{\mddefault}{\updefault}{\color[rgb]{0,0,0}}}}}}
\put(3004,-4681){\makebox(0,0)[lb]{\smash{{\SetFigFont{17}{20.4}{\rmdefault}{\mddefault}{\updefault}{\color[rgb]{0,0,0}}}}}}
\put(3575,-4685){\makebox(0,0)[lb]{\smash{{\SetFigFont{17}{20.4}{\rmdefault}{\mddefault}{\updefault}{\color[rgb]{0,0,0}}}}}}
\put(2460,-4521){\makebox(0,0)[lb]{\smash{{\SetFigFont{17}{20.4}{\rmdefault}{\mddefault}{\updefault}{\color[rgb]{0,0,0}}}}}}
\put(5286,-5518){\makebox(0,0)[lb]{\smash{{\SetFigFont{20}{24.0}{\rmdefault}{\mddefault}{\updefault}{\color[rgb]{0,0,0}}}}}}
\put(2406,-5203){\makebox(0,0)[lb]{\smash{{\SetFigFont{20}{24.0}{\rmdefault}{\mddefault}{\updefault}{\color[rgb]{0,0,0}}}}}}
\put(3732,-2527){\makebox(0,0)[lb]{\smash{{\SetFigFont{20}{24.0}{\rmdefault}{\mddefault}{\updefault}{\color[rgb]{0,0,0}}}}}}
\put(2812,-5607){\makebox(0,0)[lb]{\smash{{\SetFigFont{20}{24.0}{\rmdefault}{\mddefault}{\updefault}{\color[rgb]{0,0,0}}}}}}
\put(4492,-2357){\makebox(0,0)[lb]{\smash{{\SetFigFont{20}{24.0}{\rmdefault}{\mddefault}{\updefault}{\color[rgb]{0,0,0}}}}}}
\put(3286,-1428){\makebox(0,0)[lb]{\smash{{\SetFigFont{20}{24.0}{\rmdefault}{\mddefault}{\updefault}{\color[rgb]{0,0,0}}}}}}
\put(4231,-1265){\makebox(0,0)[lb]{\smash{{\SetFigFont{20}{24.0}{\rmdefault}{\mddefault}{\updefault}{\color[rgb]{0,0,0}}}}}}
\put(4441,-4194){\makebox(0,0)[lb]{\smash{{\SetFigFont{20}{24.0}{\rmdefault}{\mddefault}{\updefault}{\color[rgb]{0,0,0}}}}}}
\put(2687,-3856){\makebox(0,0)[lb]{\smash{{\SetFigFont{20}{24.0}{\rmdefault}{\mddefault}{\updefault}{\color[rgb]{0,0,0}}}}}}
\put(3357,-3857){\makebox(0,0)[lb]{\smash{{\SetFigFont{20}{24.0}{\rmdefault}{\mddefault}{\updefault}{\color[rgb]{0,0,0}}}}}}
\end{picture} }},{}]
We first show that triangle bounded by ,  and  contains a vertex from the boundary of polygon .
Consider a maximal contiguous subsequence of edges of  that starts with  and 
every edge in the subsequence intersects .
Let  be the last edge in this subsequence.
Note that such  is well-defined.
Let  and  be the endpoints of  below and above  respectively, 
and let  be the intersection point of  and .
Clearly  is between  and . 
Based on the maximality of the subsequence , edge  is the last edge that intersects , so  does not intersect .
Also  intersects  at a point between  and  and shares an endpoint with . 
Edge  cannot be incident on  because any edge connecting  to a point on line segment  
intersects . Therefore,  is incident on . Also,  intersects , and it doesn't intersect 
 or . Thus, the other endpoint of  must be in the triangle .
Let  be this endpoint.
\end{window}

\begin{window}[0,r,{\resizebox{2in}{1.8in}{\hspace*{0.5in}\begin{picture}(0,0)\includegraphics{pictures/remote1.pdf}\end{picture}\setlength{\unitlength}{4144sp}\begingroup\makeatletter\ifx\SetFigFont\undefined \gdef\SetFigFont#1#2#3#4#5{\reset@font\fontsize{#1}{#2pt}\fontfamily{#3}\fontseries{#4}\fontshape{#5}\selectfont}\fi\endgroup \begin{picture}(5129,5105)(5864,-7412)
\put(7590,-3786){\makebox(0,0)[lb]{\smash{{\SetFigFont{48}{57.6}{\rmdefault}{\mddefault}{\updefault}{\color[rgb]{0,0,0}}}}}}
\put(7223,-3279){\makebox(0,0)[lb]{\smash{{\SetFigFont{48}{57.6}{\rmdefault}{\mddefault}{\updefault}{\color[rgb]{0,0,0}}}}}}
\put(7649,-2934){\makebox(0,0)[lb]{\smash{{\SetFigFont{48}{57.6}{\rmdefault}{\mddefault}{\updefault}{\color[rgb]{0,0,0}}}}}}
\put(8647,-5824){\makebox(0,0)[lb]{\smash{{\SetFigFont{48}{57.6}{\rmdefault}{\mddefault}{\updefault}{\color[rgb]{0,0,0}}}}}}
\put(7664,-4061){\makebox(0,0)[lb]{\smash{{\SetFigFont{48}{57.6}{\rmdefault}{\mddefault}{\updefault}{\color[rgb]{0,0,0}}}}}}
\put(7044,-5936){\makebox(0,0)[lb]{\smash{{\SetFigFont{48}{57.6}{\rmdefault}{\mddefault}{\updefault}{\color[rgb]{0,0,0}}}}}}
\put(8041,-5964){\makebox(0,0)[lb]{\smash{{\SetFigFont{48}{57.6}{\rmdefault}{\mddefault}{\updefault}{\color[rgb]{0,0,0}}}}}}
\put(8723,-6961){\makebox(0,0)[lb]{\smash{{\SetFigFont{48}{57.6}{\rmdefault}{\mddefault}{\updefault}{\color[rgb]{0,0,0}}}}}}
\put(6677,-7168){\makebox(0,0)[lb]{\smash{{\SetFigFont{48}{57.6}{\rmdefault}{\mddefault}{\updefault}{\color[rgb]{0,0,0}}}}}}
\put(6486,-5815){\makebox(0,0)[lb]{\smash{{\SetFigFont{48}{57.6}{\rmdefault}{\mddefault}{\updefault}{\color[rgb]{0,0,0}}}}}}
\end{picture} }},{}]
The argument so far shows the existence of a point  in the triangle bounded by ,  and  (see the figure to the right). 
Thus, .
The first inequality holds because  is in triangle , 
and the second inequality holds for the following reason. From the definition of -skeleton, edge  is in the -skeleton 
if and only if there does not exist a point 
in the point set such that . For , this implies that
 for any point  in the point set. 

\hspace*{\parindent}
In triangle  the angle  is less than , so 
, 
and a similar argument on triangle  shows that .
\end{window}
This completes the proof of Fact~\ref{obs:point}.
\end{proof}

The remainder of the argument is similar to Case 1, where triangles do not cross.
The interior of polygon  can be retriangulated at lesser cost using the techniques
in the original -skeleton arguments. 
Finally, lowering the weight of triangles in  by 
and raising the weight of new triangles by ,
gives a fractional triangulation that costs less than .




\noindent
{\bf Part \ref{in:YXY} (-subgraph):}
 The argument used above for the -skeleton works for the the -subgraph as well.
The only parts of the argument for the -skeleton that use geometric properties of 
edges in the -skeleton are Facts~\ref{prop:length} and~\ref{prop:remote}.
Thus, it suffices to show that these two facts hold for the edges of the -subgraph too.
\begin{window}[0,r,{\resizebox{2in}{1.8in}{\hspace*{0.5in}\begin{picture}(0,0)\includegraphics{pictures/yxy.pdf}\end{picture}\setlength{\unitlength}{4144sp}\begingroup\makeatletter\ifx\SetFigFont\undefined \gdef\SetFigFont#1#2#3#4#5{\reset@font\fontsize{#1}{#2pt}\fontfamily{#3}\fontseries{#4}\fontshape{#5}\selectfont}\fi\endgroup \begin{picture}(2800,3012)(3206,-4914)
\put(4115,-2085){\makebox(0,0)[lb]{\smash{{\SetFigFont{12}{14.4}{\rmdefault}{\mddefault}{\updefault}{\color[rgb]{0,0,0}}}}}}
\put(4847,-4836){\makebox(0,0)[lb]{\smash{{\SetFigFont{12}{14.4}{\rmdefault}{\mddefault}{\updefault}{\color[rgb]{0,0,0}}}}}}
\put(4355,-2516){\makebox(0,0)[lb]{\smash{{\SetFigFont{12}{14.4}{\rmdefault}{\mddefault}{\updefault}{\color[rgb]{0,0,0}}}}}}
\put(4848,-4176){\makebox(0,0)[lb]{\smash{{\SetFigFont{12}{14.4}{\rmdefault}{\mddefault}{\updefault}{\color[rgb]{0,0,0}}}}}}
\put(3977,-3564){\makebox(0,0)[lb]{\smash{{\SetFigFont{12}{14.4}{\rmdefault}{\mddefault}{\updefault}{\color[rgb]{0,0,0}}}}}}
\put(5092,-3604){\makebox(0,0)[lb]{\smash{{\SetFigFont{12}{14.4}{\rmdefault}{\mddefault}{\updefault}{\color[rgb]{0,0,0}}}}}}
\end{picture} }},{}]
Let  be an edge of the -subgraph and  be any edge that intersects . 
By definition of the -subgraph,
 is at most , so
the union of two disks centered at  and  with radius  doesn't contain  
or  (see the figure to the right). 
If the angle  in triangle  and  in triangle 
are both greater than , we have ,
and proof of Fact~\ref{prop:length} is complete.
To show that , 
let  and  be the intersections of  with the circle centered at .
We have . The second inequality holds because
 is an inscribed angle (i.e. is an angle formed by two chords with a 
common endpoint in the circle centered at ) and its intercepted arc (i.e. the part of the circle 
which is "inside" the angle) is greater than . The argument to show 
is similar.
\end{window}
We next show that Fact~\ref{prop:remote} 
also holds for any edge  of the -subgraph.
Let  and  be four other distinct points of the point set 
such that  and  both intersect .
We first consider the case that  and  do not intersect.
Let  and  respectively denote the distance of  and  from the line through . 
Lemma 2 in \cite{yang1994chain} states that if , then . 
Similarly, if , then . Hence 
which proves Fact~\ref{prop:remote} in this case. 
If on the other hand,  and  intersect, we use Fact~\ref{obs:point}. The proof of 
Fact~\ref{obs:point} is almost the same whether  is an edge of -subgraph or an edge of
the -skeleton.
\end{fullproof}



\begin{lemma}\label{lemma:out}
  If any of the following conditions holds for a potential edge  of  
  (not on the boundary of the region to be triangulated),
  the LP forces  to 0.
  \begin{enumerate}
  \item \label{out:independence} {\bf independence:}
    The LP forces a potential edge that crosses  to 1.  
\item \label{out:diamond}
    {\bf diamond:} The diamond condition holds for .
\item \label{out:LMT} {\bf LMT skeleton:}
    For every two triangles  and  for which  is locally minimal,
    the LP forces one of the edges of  or  to 0.
 \end{enumerate}
\end{lemma}

\begin{proofidea}
  Part~(\ref{out:independence}) is straightforward: if potential edges  and  cross,
  then the LP covering constraint for a point near the intersection of  and 
  implies that the total weight of potential triangles that have  or  as sides is at most 1.

  Part~(\ref{out:LMT}), the LMT skeleton, is straightforward.
  If an optimal fractional triangulation  gives  positive weight,
  then (by constraint~(\ref{constraint:edge}) implied by the LP)
  there must be two triangles  and  with positive  and 
  whose intersection is .
  Edge  must be locally minimal for  and  
  (otherwise  could be improved by reducing  and 
  and raising the weights of the other two triangles that triangulate ).

  Part~(\ref{out:diamond}), the diamond condition, 
  is handled as the -skeleton and -subgraph
  are handled in the proof idea of Lemma~\ref{lemma:in}.
\end{proofidea}




\begin{fullproof}
As discussed in the proof idea, Part \ref{out:independence} (independence) 
and Part \ref{out:LMT} (LMT skeleton) are straightforward. We give the detailed proof of 
Part \ref{out:diamond} (diamond).



\bigskip
\noindent
{\bf Part \ref{out:diamond} (diamond):}
Like -skeleton and YXY subgraph, 
the original diamond heuristic for MWT is proved by local-improvement arguments:
if the condition holds for an edge  that {\em is} in the MWT,
then a polygon covering  within the MWT can be retriangulated at lesser cost,
contradicting the optimality of the MWT \cite{das1989triangulations,drysdale2001exclusion}.
We first give a summary of the results in \cite{drysdale2001exclusion} 
and then use them to extend the result to any optimal {\em fractional} triangulation .

\begin{window}[2,r,{\resizebox{2in}{1.8in}{\begin{picture}(0,0)\includegraphics{pictures/ABdefine.pdf}\end{picture}\setlength{\unitlength}{4144sp}\begingroup\makeatletter\ifx\SetFigFont\undefined \gdef\SetFigFont#1#2#3#4#5{\reset@font\fontsize{#1}{#2pt}\fontfamily{#3}\fontseries{#4}\fontshape{#5}\selectfont}\fi\endgroup \begin{picture}(3007,2714)(3226,-6188)
\put(6148,-4898){\makebox(0,0)[lb]{\smash{{\SetFigFont{12}{14.4}{\rmdefault}{\mddefault}{\updefault}{\color[rgb]{0,0,0}}}}}}
\put(4928,-3845){\makebox(0,0)[lb]{\smash{{\SetFigFont{12}{14.4}{\rmdefault}{\mddefault}{\updefault}{\color[rgb]{0,0,0}}}}}}
\put(4663,-3670){\makebox(0,0)[lb]{\smash{{\SetFigFont{12}{14.4}{\rmdefault}{\mddefault}{\updefault}{\color[rgb]{0,0,0}}}}}}
\put(4630,-6093){\makebox(0,0)[lb]{\smash{{\SetFigFont{12}{14.4}{\rmdefault}{\mddefault}{\updefault}{\color[rgb]{0,0,0}}}}}}
\put(4841,-6039){\makebox(0,0)[lb]{\smash{{\SetFigFont{12}{14.4}{\rmdefault}{\mddefault}{\updefault}{\color[rgb]{0,0,0}}}}}}
\put(3254,-4899){\makebox(0,0)[lb]{\smash{{\SetFigFont{12}{14.4}{\rmdefault}{\mddefault}{\updefault}{\color[rgb]{0,0,0}}}}}}
\put(3640,-4200){\makebox(0,0)[lb]{\smash{{\SetFigFont{12}{14.4}{\rmdefault}{\mddefault}{\updefault}{\color[rgb]{0,0,0}}}}}}
\put(5468,-5881){\makebox(0,0)[lb]{\smash{{\SetFigFont{12}{14.4}{\rmdefault}{\mddefault}{\updefault}{\color[rgb]{0,0,0}}}}}}
\put(5224,-4598){\makebox(0,0)[lb]{\smash{{\SetFigFont{12}{14.4}{\rmdefault}{\mddefault}{\updefault}{\color[rgb]{0,0,0}}}}}}
\put(5318,-5337){\makebox(0,0)[lb]{\smash{{\SetFigFont{12}{14.4}{\rmdefault}{\mddefault}{\updefault}{\color[rgb]{0,0,0}}}}}}
\end{picture} }},{}]
Suppose  is horizontal and  and  are its endpoints, and  is on the left of . 
Let  and  be the two isosceles triangles with base angle  
above and below  and  be the disk with diameter  as shown in the figure to the right.
Suppose that  contains a point  and  contains a point .
If  is in the MWT,  is not in the MWT, and there is a set of triangles in the MWT that 
intersect . Consider the sequence of triangles 
encountered when tracing  toward , starting from edge  and stopping with the 
first triangle that has a vertex inside disk , and let  be the polygon formed by the boundary
edges of triangles in the sequence.
Let  be the vertex found inside  --- if all else fails, then .
In the preceding figure,  is the dark gray area above .  
Similarly, consider the sequence of triangles 
encountered when tracing  toward , starting from edge  and stopping with the 
first triangle that has a vertex inside disk , and let  be the polygon formed by the boundary
edges of triangles in the sequence. In the figure,  is the dark gray area below .
The boundary edges of  are grouped naturally into two chains, one from  to  and 
one from  to . Vertex  doesn't belong to any of the two chains. 
 is a {\em fan} on  (or ) if all triangles in  are incident on  (or ). 
Similarly  can be a fan on  (or ).
\end{window}


Drysdale et al.~\cite{drysdale2001exclusion} prove the following two facts to show how  or  or 
their union can be triangulated at lesser cost.\footnote
{In summarizing the result of \cite{drysdale2001exclusion}, we use the same notations and names. 
The only exception is that \cite{drysdale2001exclusion} uses  and  instead of  and .}   
\begin{fact}[following {\cite[Lemma 8]{drysdale2001exclusion}}]\label{prop:diamond8}
If  (or ) is not a fan, it can be retriangulated at lower cost. 
\end{fact}

\begin{fact}[{\cite[Lemma 9]{drysdale2001exclusion}}]\label{prop:diamond9}
When both  and  are fans, then their union can be retriangulated at lower cost.
\end{fact}

The above facts contradict the optimality of the MWT. Thus,  cannot be in any MWT.
Next we extend the above arguments to any optimal {\em fractional} triangulation .
We find polygons  and  corresponding to  and  in the above argument and show that if  does not give 
 weight zero, then  can be retriangulated at lesser cost. Lowering the weight of those
triangles by  and raising the weight of other triangles by  gives a fractional triangulation
that costs less than . The details of the argument follow.
 
If  does not give  weight 0, there is a triangle  with positive  
that has  as a side. Triangle  intersects .
For any side  of  that intersects  there must be a triangle  with positive  
that has  as a side and lies on the other side of  from . By repeating the same argument 
for the new triangle(s), a sequence  of triangles can be obtained that completely covers . 
As shown in the figure below triangles in the sequence  may or may not cross each other.
The left figure shows the case that no triangles cross, while the right figure shows the case that some triangles cross. 
If triangles in the sequence do not cross, the original arguments from \cite{drysdale2001exclusion} apply.
However, if triangles in the sequence cross, additional analysis is needed.

\begin{center}
\scalebox{.65}
{
\xfig{sequence}
}
\end{center}
Consider the set of triangles in  
encountered when tracing  toward , starting from edge  and stopping with the 
first triangle that has a vertex inside disk , and let  be the polygon formed by the 
boundary edges of these triangles. Also let
 be the vertex found inside  --- if all else fails, then .
Similarly, define  to be the polygon formed by the boundary edges of the set of triangles encountered 
when tracing  from  toward  until a 
vertex is inside . The following figure shows  and  (the dark shaded regions) in two cases.
In the left figure there are no crossing triangles while in the right figure some triangles in  cross. 
\begin{center}
\scalebox{.7}
{
  \xfig{ABregion}
}
\end{center}


We consider the following cases and show how in each case  can be retriangulated at lower cost.



\begin{window}[0,r,{\resizebox{2.7in}{1.55in}{\begin{picture}(0,0)\includegraphics{pictures/case2small.pdf}\end{picture}\setlength{\unitlength}{4144sp}\begingroup\makeatletter\ifx\SetFigFont\undefined \gdef\SetFigFont#1#2#3#4#5{\reset@font\fontsize{#1}{#2pt}\fontfamily{#3}\fontseries{#4}\fontshape{#5}\selectfont}\fi\endgroup \begin{picture}(4704,2714)(2735,-2896)
\put(4327,-1607){\makebox(0,0)[lb]{\smash{{\SetFigFont{12}{14.4}{\rmdefault}{\mddefault}{\updefault}{\color[rgb]{0,0,0}}}}}}
\put(7354,-1606){\makebox(0,0)[lb]{\smash{{\SetFigFont{12}{14.4}{\rmdefault}{\mddefault}{\updefault}{\color[rgb]{0,0,0}}}}}}
\put(5723,-496){\makebox(0,0)[lb]{\smash{{\SetFigFont{12}{14.4}{\rmdefault}{\mddefault}{\updefault}{\color[rgb]{0,0,0}}}}}}
\put(5901,-371){\makebox(0,0)[lb]{\smash{{\SetFigFont{12}{14.4}{\rmdefault}{\mddefault}{\updefault}{\color[rgb]{0,0,0}}}}}}
\put(5807,-2800){\makebox(0,0)[lb]{\smash{{\SetFigFont{12}{14.4}{\rmdefault}{\mddefault}{\updefault}{\color[rgb]{0,0,0}}}}}}
\put(6013,-2747){\makebox(0,0)[lb]{\smash{{\SetFigFont{12}{14.4}{\rmdefault}{\mddefault}{\updefault}{\color[rgb]{0,0,0}}}}}}
\put(4291,-687){\makebox(0,0)[lb]{\smash{{\SetFigFont{12}{14.4}{\rmdefault}{\mddefault}{\updefault}{\color[rgb]{0,0,0}}}}}}
\put(4833,-669){\makebox(0,0)[lb]{\smash{{\SetFigFont{12}{14.4}{\rmdefault}{\mddefault}{\updefault}{\color[rgb]{0,0,0}}}}}}
\put(5983,-1080){\makebox(0,0)[lb]{\smash{{\SetFigFont{12}{14.4}{\rmdefault}{\mddefault}{\updefault}{\color[rgb]{0,0,0}}}}}}
\end{picture} }},{}] 
\noindent{\em Case 1} ---  is not a fan and some triangles in  overlap.
This case is shown in the figure to the right and its magnified version below.
In the sequence of triangles in  
encountered when tracing  toward , let  be the first triangle 
that crosses some of the previous triangles in the sequence (the light gray triangle labeled in the figure).
Let  be the polygon that covers the set of all triangles before  (the dark gray polygon labeled  in the figure).
Polygon  and triangle  share an edge. Let  be this edge such that  is to left of  (see the figure below).
The boundary edges of  can be grouped naturally into two chains, one from  to  and 
one from  to .
We use the following additional facts from \cite{drysdale2001exclusion} to show that  can be retriangulated at lower cost.
Note that the following facts can be applied to  because 
 is the union of triangles that do not cross.
\end{window}

\begin{fact}[{\cite[Lemma 7]{drysdale2001exclusion}}]\label{lemma:diamond7}
If the chain of boundary vertices from  to  
has three consecutive vertices ,  and  with  
and the internal angle  is less than , then  can be retriangulated 
to decrease its cost. 
(The same is true with  and  exchanging roles with  and , respectively.)
\end{fact}

We say the {\em clockwise limit \footnote{ We use the term {\em clockwise limit} to be consistent with the 
terminology of the original paper \cite{drysdale2001exclusion}. The word {\em limit} in this context has nothing 
to do with the well-known mathematical limit.} on the directions of the boundary edges on the chain} (from  to )
is perpendicular to , if for every two consecutive vertices  and  on the chain,
 is above or on the line through  that is perpendicular to .

\begin{fact}[{\cite[Lemma 6]{drysdale2001exclusion}}]\label{lemma:diamond6}
If the chain of boundary vertices from  to  
has no 
three consecutive vertices , , and  with  
that form an internal angle () of less than , 
then the clockwise limit on the directions
of the boundary edges is perpendicular to . 
(The same is true with  and  exchanging roles with  and , respectively,
and ``counter-clockwise'' replacing ``clockwise''.)
\end{fact}



Facts~\ref{lemma:diamond6} implies that if the clockwise limit on the direction 
of boundary edges (on the chain from  to ) is not perpendicular to the line through , then
there are three consecutive vertices , , and  with  
that form an internal angle () of less than , and the existence of 
such vertices by Fact~\ref{lemma:diamond7} implies that  can be retriangulated 
to decrease its cost. The remainder of the proof for Case 1 shows that one of the edges 
on the boundary of  violates the mentioned
limit on the direction of boundary edges, and thus,  can be retriangulated at lower cost.



In the following figure, the darker shaded area is . Triangles in  do not overlap. 
Triangle  is the first triangle that overlaps some of the previous triangles.\label{case2}
Triangle  and polygon  share edge ,
and  crosses some triangles covered by . Thus,  crosses some boundary 
edges of  either on the chain from  to  or on the chain from  to .
Assume without loss of generality that triangle  crosses an edge on the chain from  to .
Since edge  is a boundary edge of , the other two sides of triangle  ( and ) should intersect .
Let  be the first edge on the chain from  to  that is intersected by  when moving from  to .

\begin{center}
\scalebox{.5}
{
  \xfig{case2large}
}
\end{center}
The line through  divides the plane into a half-space above it (the half-space containing point ) and 
a half-space below it. It's easy to see that  is in the half-space above  and  is below it, 
so  cannot be above the line through  that is perpendicular to . This means that the clockwise 
limit on the direction of the boundary edges is not perpendicular to . 
This combined with Fact~\ref{lemma:diamond7} and Fact~\ref{lemma:diamond6} implies that the interior of  can 
be retriangulated at lower cost.



\noindent{\em Case 2 ---}  is not a fan and no two triangles in  cross. 
Since triangles in  do not overlap, Fact~\ref{prop:diamond8} directly implies that  can be retriangulated at a lesser cost.

The previous two cases show that if  is not a fan, we can retriangulate some triangles in  to reduce the cost of triangulation.
A similar argument applies to , so there remains the case where both  and  are fans. We consider this case next.



\noindent{\em Case 3 ---}  and  are both fans. 
 is a fan on  (or ) if all triangle in   are incident on  (or ), and 
thus no two triangle in  overlap. Similarly if  is a fan, no two triangle in 
overlap. Additionally, when both  and  are fans, triangles in  are all above the
line through  and triangles in  are below the line through , and no triangle in 
intersects a triangle in . Hence, no two triangle in  can cross, and 
by Fact~\ref{prop:diamond9}  can be retriangulated at lower cost.


In all the above cases,
by lowering the weight of some triangles in  by 
and raising the weight of some other triangles in the new triangulation by , 
a fractional triangulation costing less than  can be obtained, which contradicts the optimality of .
This completes the proof of Part \ref{out:diamond} (diamond property) and Lemma~\ref{lemma:out}. 
\end{fullproof}





Assume (as in the statement of Thm.~\ref{thm:heuristics}) that the set  of edges
that can be deduced to be in every MWT of  
gives a partition of  in which every face is empty.
It follows from Lemmas \ref{lemma:in} and \ref{lemma:out}
(by a simple inductive proof) that every edge that can be deduced to be excluded from every MWT
is forced to 0 by the LP, and every edge that can be deduced to be in every MWT is forced to 1.
Thus, in any optimal fractional triangulation ,
no potential triangle  that crosses an edge in  has positive weight .
Thus, the optimal fractional triangulations  are exactly those that,
for each face  of the partition, induce an optimal fractional triangulation 
of the simple polygon .
It is known
(e.g.~\cite[Thm.~7]{dantzig1985triangulations},
\cite[Thm.~4.1(i)]{de1996polytope},
\cite[Cor.~3.6.2]{kirsanov2004minimal})
that, for any simple polygon , each basic optimal fractional triangulation
is the incidence vector of an actual triangulation of .
Thus, each optimal extreme point of the LP for  is also the incidence vector
of a triangulation of , proving Thm.~\ref{thm:heuristics}.



































\section{Acknowledgements}

Thanks to two anonymous referees for suggestions on improving the presentation of the results.

\bibliographystyle{siam}
\bibliography{bib}

\end{document}
