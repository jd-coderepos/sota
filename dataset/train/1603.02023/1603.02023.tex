


Recall from Section III that, in the delay-free case, we had
plant ${\bf G} = {\bf Comp} ({\bf G}_1, ..., {\bf G}_N)$ over $\Sigma$,
specification $E \subseteq \Sigma^*$, prohibitible event set $\Sigma_{hib}$,
and forcible event set $\Sigma_{for}$. 

Now for $k,l \in \mathcal{N}$ let $\Sigma_{k,com,l}'$ be partitioned as
$\Sigma_{k,com,l}' = \Sigma_{k,com,l}^{bd} \dot{\cup} \Sigma_{k,com,l}^{ud}$, where
$\Sigma_{k,com,l}^{bd}$
is the subset of communication events with bounded
delay and $\Sigma_{k,com,l}^{ud}$ the subset of those with unbounded delay.
First, the new plant $\tilde{\bf G}$ including both the plant components of $\bf G$ and the channels is
\begin{align} \label{eq:newplant}
\tilde{\bf G} = {\bf Comp}(&{\bf G},\{{\bf CH}(k,\sigma,l)|\sigma
\in \Sigma_{k,com,l}^{bd}, k, l \in \mathcal{N}\}, \notag \\
\{&{\bf CH}(k,\sigma,l)|\sigma \in \Sigma_{k,com,l}^{ud}, k, l \in \mathcal{N}\}),
\end{align}
where ${\bf CH}(k,\sigma,l)$ is the ATG displayed in Fig. \ref{fig:channel}.
The event set $\tilde{\Sigma}$ of $\tilde{\bf G}$ is $\tilde{\Sigma}
= \Sigma ~\cup ~\{\sigma',\sigma''|\sigma \in \Sigma_{k,com,l}', k,l \in \mathcal{N}\}$.
Since none of the added events $\sigma'$ and $\sigma''$ is
forcible, or prohibitible, the new subset of forcible events and prohibitible
events are unchanged, i.e. $\tilde\Sigma_{for} = \Sigma_{for}$ and
$\tilde{\Sigma}_{hib} = \Sigma_{hib}$. So $\hat\Sigma_{for,k}$ and $\hat\Sigma_{hib,k}$
(as defined in (\ref{eq:disforce})) are also unchanged. Following the allocation policy for building distributed
control architecture, we choose $\hat\Sigma_{for,k}$ (resp. $\hat\Sigma_{hib,k}$)
to be the subset of forcible (resp. prohibitible) events for component ${\bf G}_k$ in
the new plant, i.e.,
\begin{align}
\tilde{\Sigma}_{for,k} &:= \hat\Sigma_{for,k} \label{eq:new_forci_event_k}\\
\tilde{\Sigma}_{hib,k} &:= \hat\Sigma_{hib,k} \label{eq:new_prohib_event_k}
\end{align}
Since $\hat\Sigma_{for,k}$ and $\hat\Sigma_{hib,k}$, ($k \in \mathcal{N}$)
are pairwise disjoint, so are  $\tilde\Sigma_{for,k}$ and $\tilde\Sigma_{hib,k}$.
Therefore, $\tilde\Sigma_{hib} = \dot{\bigcup}_{k \in \mathcal{N}} \tilde\Sigma_{hib,k}$
and $\tilde\Sigma_{for} = \dot{\bigcup}_{k \in \mathcal{N}} \tilde\Sigma_{for,k}$.








The specification imposed on $\bf G$ is not changed, but should be extended
to the new event set $\tilde\Sigma$, i.e. the specification
$\tilde{E} = \tilde{P}^{-1} E$,
where $\tilde{P}:\tilde\Sigma^*\rightarrow\Sigma^*$ is the natural projection.


As we have mentioned, a consequence of
introducing the communication channels is that the agents ${\bf G}_k$ ($k \in \mathcal{N}$)
have distinct observable event sets. Hence the local preemptors/controllers to be allocated to different
agents will be required to have different observable event sets. To address this, rather than synthesizing
a monolithic supervisor for a single observable event subset $\Sigma_o$, we propose to synthesize $N$ decentralized
supervisors one for each observable event set $\tilde\Sigma_{o,k}$ ($k \in \mathcal{N}$) given by
\begin{align*}  \tilde\Sigma_{o, k} := ({\Sigma_{o} \setminus \Sigma_{com, k}'} ) &\cup \{\sigma,\sigma''|\sigma \in \Sigma_{k,com,l}', l \in \mathcal{N}, l \neq k\} \\
&\cup \{\sigma'|\sigma \in \Sigma_{l,com,k}', l \in \mathcal{N}, l \neq k\}.
\end{align*}

 For the synthesis of decentralized supervisors,
it is proved in \cite{RudWon:1992,ParkChoi09} that a set of decentralized supervisors exists
which synthesizes a language $K\subseteq L_m({\bf G})$ if and only if $K$
is coobservable, controllable and $L_m({\bf G})$-closed. Like observability, coobservability is not closed under
set union; consequently when $K$ is not coobservable, there generally does not exist the supremal coobservable
(and controllable, $L_m({\bf G})$-closed) sublanguage of $K$, and there is no
existing algorithm that computes a coobservable sublanguage of $K$. In this paper, we employ the concept
of (timed) relative coobservability \cite{CaiZW16}, which is stronger
than coobservability (thus only a sufficient condition for existence of
decentralized supervisors), but the supremal (timed) relatively coobservable sublanguage always exists.
Let $\tilde P_k: \tilde\Sigma^* \rightarrow \tilde\Sigma^*_{o,k}$ and $C \subseteq L_m(\tilde{\bf G})$
be an ambient language. A sublanguage $K \subseteq C$ is {\it timed relatively coobservable} (with respect to $C$, $\tilde{\bf G}$
and $\tilde P_k$, $k \in \mathcal{N}$), or simply timed $C$-coobservable, if
for every $k \in \mathcal{N}$ and
every pair of strings $s,s' \in \Sigma^*$ with $\tilde P_k (s) = \tilde P_k (s')$
there holds
\begin{align*}
&(\forall \sigma \in{\Sigma_{k,act}\cup\{tick\}}) \\
 &~~~~~~~~~~~~~s\sigma \in \overline{K}, s' \in \overline{C},
s'\sigma \in L(\tilde{\bf G}) \Rightarrow s'\sigma \in \overline{K}.
\end{align*}
Namely, relative coobservability of $K$ requires that $K$ be relatively observable with
respect to each $\tilde P_{o,k}$ and $\Sigma_k$, $k \in \mathcal{N}$.
It is proved in
\cite{CaiZW16} that there always exists a unique supremal
relatively coobservable sublanguage of a given language, which may be effectively computed  by an algorithm in \cite{CaiZW16}. Since relative coobservability is stronger
than coobservability, the supremal relatively coobservable (and controllable, $L_m({\bf G})$-closed) sublanguage is guaranteed
to be coobservable (and controllable, $L_m({\bf G})$-closed), and
thereby ensures the existence of decentralized supervisors \cite{RudWon:1992,ParkChoi09}. 

For the new plant $\tilde{\bf{G}}$ and specification language $\tilde{E}$,
write $\mathcal{CCO}(\tilde{E} \cap L_m(\tilde{\bf{G}}))$ for the
family of relatively coobservable (and controllable, $L_m(\tilde{\bf
G})$-closed) sublanguages of $\tilde{E} \cap L_m(\tilde{\bf{G}})$.
Then $\mathcal{CCO}(\tilde{E} \cap L_m(\tilde{\bf{G}}))$ is nonempty
(the empty language $\emptyset$ belongs) and has a unique
supremal element
\[\sup\mathcal{CCO}(\tilde{E} \cap L_m(\tilde{\bf{G}})) = \bigcup\{K|K\in\mathcal{CCO}(\tilde{E} \cap L_m(\tilde{\bf{G}}))\}.\]
Let the generator ${\bf NSUP}$ be such that
\begin{equation} \label{eq:new_monosup}
L_m({\bf NSUP}) := \sup \mathcal {CCO}(\tilde{E} \cap L_m(\tilde{\bf{G}})).
\end{equation}
We call ${\bf NSUP}$ the {\it controllable and coobservable
behavior}, and assume that $L_m({\bf NSUP}) \neq \emptyset$.\footnote{The introduced
bounded/unbounded communication delays may cause $L_m({\bf NSUP}) = \emptyset$,
which means that the delay requirements are too strong to be satisfied. In that case,
we shall weaken the delay requirements by either decreasing delay bounds of bounded-delay channels
(when the delay bound of an event $\sigma$ needs to be decreased to 0, we do not employ a channel model
for $\sigma$, and consequently events $\sigma'$ and $\sigma''$ defined in the channel model are also removed from the alphabet)
or reducing the number of unbounded-delay channels, until we
obtain a nonempty $L_m({\bf NSUP})$. }

Next, for each observable event set $\tilde\Sigma_{o,k}$ ($k \in \mathcal{N}$),
we construct as in (\ref{eq:posup}) a {\it partial-observation decentralized supervisor}
${\bf NSUPO}_k$ defined over $\tilde\Sigma_{o,k}$. It is well-known \cite{RudWon:1992,LinWon95}
that such constructed decentralized supervisors ${\bf NSUPO}_k$
collectively achieve the same controlled behavior as $\bf NSUP$
does.
The control actions of the supervisor ${\bf NSUPO}_k$
include (i) preempting event $tick$ via forcible events in $\tilde\Sigma_{for,k}$ (as in (\ref{eq:new_forci_event_k}))
and (ii) disabling prohibitible events in $\tilde\Sigma_{hib,k}$ (as in (\ref{eq:new_prohib_event_k})).

Finally, we apply the localization procedure developed in Section IV to decompose,
one at a time, each decentralized supervisor ${\bf NSUPO}_k$, $k \in \mathcal{N}$.
The result is a set of partial-observation local preemptors ${\bf NLOC}_{\alpha}^P = (Y_{\alpha},
\Sigma_{\alpha},\eta_{\alpha}, y_{0,\alpha},Y_{m,\alpha})$, one for each forcible
event $\alpha \in \tilde\Sigma_{for}$, as well as a set of partial-observation local controllers ${\bf NLOC}_{\beta}^C = (Y_{\beta},
\Sigma_{\beta},\eta_{\beta}, y_{0,\beta},Y_{m,\beta})$, one for each $\beta \in \tilde\Sigma_{hib}$.
Owing to $\tilde\Sigma_{for} = \dot{\bigcup}_{k \in \mathcal{N}} \tilde\Sigma_{for,k}$
(resp. $\tilde\Sigma_{hib} = \dot{\bigcup}_{k \in \mathcal{N}} \tilde\Sigma_{hib,k}$),
one local preemptor ${\bf NLOC}_{\alpha}^P$ (resp. one local controller ${\bf NLOC}_{\beta}^C$)
will be owned by precisely one agent.

The following is the main result of this section, which asserts that the collective controlled
behavior of the resulting partial-observation local preemptors and local controllers, communicated
through the introduced channels with bounded/unbounded delays, is identical to that of {\bf NSUP}.

\begin{theorem} \label{thm:coobs_equ}
The set of partial-observation local preemptors $\{{\bf NLOC}_{\alpha}^P|\alpha \in
\tilde\Sigma_{for} \}$ and the set of partial-observation local
controllers $\{{\bf NLOC}_{\beta}^C|\beta \in \tilde\Sigma_{hib}\}$ derived above
are equivalent to the controllable and coobservable behavior ${\bf NSUP}$ in (\ref{eq:new_monosup}) with respect to
the plant $\tilde{\bf G}$, i.e.
\begin{align}
   L(\tilde{\bf G}) \cap L(\bf NLOC)  &= L(\bf NSUP) \label{eq:sub1:coobsequv}\\
   L_m(\tilde{\bf G}) \cap L_m(\bf NLOC)  &= L_m(\bf NSUP) \label{eq:sub2:coobsequv}
\end{align}
with
\begin{align}
L(\bf NLOC)~:=~ &\Big(\mathop \bigcap\limits_{\alpha \in \tilde\Sigma_{for}}P_{\alpha}'^{-1}L({\bf NLOC}^P_{\alpha}) \Big)\notag\\
           \cap~ &\Big(\mathop \bigcap\limits_{\beta \in \tilde\Sigma_{hib}}P_{\beta}'^{-1}L({\bf NLOC}^C_{\beta}) \Big) \label{eq:sub1:newloc}\\
L_m(\bf NLOC)~:=~ &\Big(\mathop \bigcap\limits_{\alpha \in \tilde\Sigma_{for}}P_{\alpha}'^{-1}L_m({\bf NLOC}^P_{\alpha})\Big)\notag\\
            \cap~ &\Big(\mathop \bigcap\limits_{\beta \in \tilde\Sigma_{hib}}P_{\beta}'^{-1}L_m({\bf NLOC}^C_{\beta}) \Big) \label{eq:sub2:newloc}
\end{align}
where $P_{\alpha}': \tilde\Sigma^* \rightarrow \Sigma_{\alpha}^*$
and $P_{\beta}': \tilde\Sigma^* \rightarrow \Sigma_{\beta}^*$.
\end{theorem}

The proof of Theorem~\ref{thm:coobs_equ}, presented below, is similar to that of Theorem
\ref{thm:equ}, which relies on the facts that (i) for each forcible
event, there is a corresponding partial-observation local preemptor that
preempts event $tick$ consistently with {\bf NSUP}, and (ii)
for each prohibitible event, there is a corresponding partial-observation
local controller that disables/enables it consistently with
${\bf NSUP}$. 

By the above localization approach, each agent ${\bf G}_k$ ($k \in \mathcal{N}$) acquires
a set of partial-observation local preemptors $\{{\bf NLOC}_{\alpha}^P | \alpha \in \tilde\Sigma_{for,k} \}$
and a set of partial-observation local controllers $\{{\bf NLOC}_{\beta}^C | \beta \in \tilde\Sigma_{hib,k} \}$.
Thus we obtain a distributed control architecture for multi-component TDES under partial observation and communication delay.

\vspace{2em}
\noindent {\it Proof of Theorem~\ref{thm:coobs_equ}}: The equality of (\ref{eq:sub2:coobsequv}) and the ($\supseteq$)
direction of (\ref{eq:sub1:coobsequv}) may be verified analogously
as in the proof of Theorem 1. Here we prove ($\subseteq$) of (\ref{eq:sub1:coobsequv})
by induction, i.e. $L(\tilde{\bf G}) \cap L({\bf NLOC}) \subseteq L({\bf NSUP})$.

For the {\bf base step}, note that none of $L(\tilde{\bf G})$, $L({\bf NLOC})$ and
$L({\bf NSUP})$ is empty; and thus the empty string $\epsilon$ belongs to all
of them. For the {\bf inductive step}, suppose that $s \in L(\tilde{\bf G}) \cap
L({\bf NLOC})$, $s \in L({\bf NSUP})$ and $s\sigma \in L(\tilde{\bf G}) \cap
L({\bf NLOC})$ for arbitrary event $\sigma \in \Sigma$; we must show that $s\sigma
\in L({\bf NSUP})$. Since $\tilde\Sigma = \tilde\Sigma_{uc}\dot\cup \tilde\Sigma_{hib}
\dot\cup\{tick\}$, we consider the following three cases.

(i) $\sigma \in \tilde\Sigma_{uc}$. Since $L({\bf NSUP})$ is controllable, and
$s\sigma \in L(\tilde{\bf G})$ (i.e. $\sigma \in Elig_{\tilde{\bf G}}(s)$), we have
$\sigma \in Elig_{L_m({\bf NSUP})}(s)$. That is, $s\sigma \in
\overline{L_m({\bf NSUP})} = L({\bf NSUP})$.

(ii) $\sigma = tick$. By the hypothesis that $s, s.tick \in L({\bf NLOC})$,
for every forcible event $\alpha \in \tilde\Sigma_{for,k}$, $k \in \mathcal{N}$,
$s, s.tick \in P_{\alpha}'^{-1}L({\bf NLOC}_{\alpha}^P)$, i.e. $P_{\alpha}'(s),
P_{\alpha}'(s).tick \in L({\bf NLOC}_{\alpha}^P)$. Let $y = \eta_\alpha(y_{0,\alpha},
P_{\alpha}'(s))$; then $\eta_\alpha(y,tick)!$. The rest of the proof is similar to
case (ii) of proving Theorem~\ref{thm:equ}, with ${\bf LOC}_{\alpha}^P$ and $P_\alpha$
replaced by ${\bf NLOC}_{\alpha}^P$ and $P_{\alpha}'$ respectively.


(iii) $\sigma \in \tilde\Sigma_{hib}$. There must exist a partial-observation local
controller ${\bf NLOC}_{\sigma}^C$ for $\sigma$. It follows from $s\sigma \in
L({\bf NLOC})$ that $s\sigma \in P_{\sigma}'^{-1}L({\bf NLOC}_{\sigma}^C)$ and
$s \in P_{\sigma}'^{-1}L({\bf NLOC}_{\sigma}^C)$. So $P_{\sigma}'(s\sigma)
\in L({\bf NLOC}_{\sigma}^C)$ and $P_{\sigma}'(s) \in L({\bf NLOC}_{\sigma}^C)$,
namely, $\eta_{\sigma}(y_0,P_{\sigma}'(s\sigma))!$ and $\eta_{\sigma}(y_0,
P_{\sigma}'(s))!$. Let $y := \eta_{\sigma}(y_0,P_{\sigma}'(s))$; then
$\eta_{\sigma}(y,\sigma)!$ (because $\sigma \in \Sigma_{\sigma}$).
The rest of the proof is similar to that
in \cite{ZhangCW17} for untimed DES. \hfill $\square$









