\documentclass[11pt,a4paper,oneside,onecolumn]{article}

\usepackage{cite}
\usepackage[english]{babel}
\usepackage[ansinew]{inputenc}
\usepackage{epsfig,float,latexsym,enumerate,amssymb,amsmath,amsfonts,color}
\usepackage{subfigure}

\setlength{\parindent}{0cm} \setlength{\parskip}{0.1cm} \setlength{\oddsidemargin}{0.5cm}\setlength{\evensidemargin}{0.5cm}\setlength{\textwidth}{15cm}\setlength{\textheight}{24cm}\setlength{\topmargin}{-1cm}

\newcommand{\x}[1]{x_{#1}}
\newcommand{\y}[1]{y_{#1}}
\def\eps{\varepsilon}

\newtheorem{theorem}{Theorem}[section]
\newtheorem{acknowledgement}[theorem]{Acknowledgement}
\newtheorem{algo}[theorem]{Algorithm}
\newtheorem{axiom}[theorem]{Axiom}
\newtheorem{case}[theorem]{Case}
\newtheorem{claim}[theorem]{Claim}
\newtheorem{conclusion}[theorem]{Conclusion}
\newtheorem{condition}[theorem]{Condition}
\newtheorem{conjecture}[theorem]{Conjecture}
\newtheorem{corollary}[theorem]{Corollary}
\newtheorem{property}[theorem]{Property}
\newtheorem{criterion}[theorem]{Criterion}
\newtheorem{definition}[theorem]{Definition}
\newtheorem{example}[theorem]{Example}
\newtheorem{exercise}[theorem]{Exercise}
\newtheorem{lemma}[theorem]{Lemma}
\newtheorem{notation}[theorem]{Notation}
\newtheorem{problem}[theorem]{Problem}
\newtheorem{proposition}[theorem]{Proposition}
\newtheorem{remark}[theorem]{Remark}
\newtheorem{solution}[theorem]{Solution}
\newtheorem{summary}[theorem]{Summary}


\def\QED{\ensuremath{{\square}}}
\def\markatright#1{\leavevmode\unskip\nobreak\quad\hspace*{\fill}{#1}}

\newenvironment{proof}
{\begin{trivlist}\item[\hskip\labelsep{\bf Proof.}]}
{\markatright{\QED}\end{trivlist}}

\begin{document}

\title{Locating a single facility and a high-speed line}

\author{
	J. M. D\'{\i}az-B\'{a}\~{n}ez
	\thanks{Departamento Matem\'{a}tica Aplicada II, Universidad de Sevilla, Espa\~na. 
	{\tt \{dbanez,iventura\}@us.es}} \and 
	M. Korman
	\thanks{Universitat Polit\`{e}cnica de Catalunya (UPC), Barcelona, Espa\~na.
	{\tt mkormanc@ulb.ac.be}} \and
	P. P\'{e}rez-Lantero
	\thanks{Escuela de Ingenier\'ia Civil en Inform\'atica, Universidad de Valpara\'{i}so, Chile.
	{\tt pablo.perez@uv.cl}} \and
	I. Ventura\footnotemark[1]
} 

\maketitle

\begin{abstract}
In this paper we study a facility location problem in the plane in
which a single point (facility) and a rapid transit line (highway)
are simultaneously located in order to minimize the total travel
time from the clients to the facility, using the  or Manhattan
metric. The rapid transit line is given by a segment with any length
and orientation, and is an alternative transportation line
that can be used by the clients to reduce their travel time to the
facility. We study the variant of the problem in which clients can
enter and exit the highway at any point. We provide an -time
algorithm that solves this variant, where  is the number
of clients. We also present a detailed characterization of the solutions,
which depends on the speed given in the highway.
\end{abstract}
    




\section{Introduction}\label{section:intro}

Suppose we are given a set of clients represented as a set of points in
the plane, and a service facility represented as a point to which
all clients have to move. Every client can reach the facility
directly, or use an alternative rapid transit line called highway in order to reduce the
travel time. The highway is a straight line segment of arbitrary
orientation.
If a client moves
directly to the facility, it moves at unit speed and the distance
traveled is the Manhattan or  distance to the facility. In the
case where a client uses the highway, it travels the  distance
at unit speed to one point of the highway, traverses with a speed
 the Euclidean distance to other highway point, and
finally travels the  distance from that point to the facility
at unit speed. All clients traverse the highway at the same speed.
The highway is used by a client point whenever it saves time to reach
the facility.
Given the set of points representing the clients, the facility
location problem consists in determining at the same time the
facility point and the highway in order to minimize the total
weighted travel time from the clients to the facility. The weighted
travel time of a client is its travel time multiplied by a weight
representing the intensity of its demand.

Recent papers have dealt with geometric problems considering
travelling distances as a combination of planar and network distances.
Carrizosa and Rodr\'iguez-Ch\'ia~\cite{carrisoza1997} introduced the
-facility min-sum location problem on the plane with a metric 
induced by a gauge and a finite set of rapid transit lines giving the network distance.
This problem was further developed by 
Gugat and Pfeiffer~\cite{gugat2007}, and Pfeiffer and Klamroth~\cite{pfeiffer2008}.
Brimberg et al.~\cite{brimberg2003,brimberg2005} studied 
the location of a new single facility
considering given regions of distinct distance measures.
All these papers consider the well-known
Weber problem under a new metric.
Other papers can be found in the context of the combination of the 
 distance with the network distance.
Abellanas et al.~\cite{abellanas03}
introduced the time metric model: Given an
underlying metric, the user can travel at speed  when moving
along a highway  or unit speed elsewhere. The particular case in
which the underlying metric is the  metric and all highways are
axis-parallel segments of the same speed, is called the {\em city
metric}~\cite{aichholzer02}. 
In the scenario of setting up an optimal distance network
that minimizes the maximum travel time among a
set of points, several problems have been recently investigated in detail~\cite{ahn07,aloupis10,cardinal08}. Other similar and more general models
were studied by Korman and Tokuyama~\cite{korman08}.
See~\cite{phdkorman} and~\cite{dbanezMS04} for surveys on highway location 
and extensive facility location problems, respectively.

A similar problem consisting in simultaneously locating a
service facility point and a high-speed line (i.e.\ highway) 
of fixed length was recently studied by
Espejo and Rodr\'{i}guez-Ch\'{i}a~\cite{espejo11} and D\'iaz-B\'a\~nez et
al.~\cite{diaz-banez11}. The authors considered that highway is a 
\emph{turnpike}~\cite{korman08}, that is, clients can enter
and exit the highway only at the endpoints. 
A first solution was introduced by Espejo and Rodr\'{i}guez-Ch\'ia, and
after that an improved solution was given by D\'iaz-B\'a\~nez et al. 
The problem aims to minimize
the total weighted travel time from the demand points to the facility
service, and can be solved in  time~\cite{diaz-banez11}, where 
 is the number of clients.
D\'iaz-B\'a\~nez et al.~\cite{diaz-banez11-3}
continued the study of this variant by considering the min-max
optimization criterion. They minimize the maximum time distance from
the clients to the facility point.

In this paper we study a related problem in which the length of the
highway is variable, that is, it is not fixed in advance as part
of the input of the problem, and clients can enter and
exit the highway at any point. Due to the latter condition, highway is called
\emph{freeway}~\cite{korman08}. Since both entering and leaving the highway are
allowed at all its points, then the structure (i.e.\ highway) is continuously
integrated in the plane.
We minimize the total weighted
transportation time from the demand points to the facility. The problem
of locating a min-max freeway of fixed length was solved by
D\'iaz-B\'a\~nez et al.~\cite{diaz-banez11-3} in  time.

The following notation is introduced in order to formulate the problem.
Let  be the set of  demand points,  be the service facility
point,  be the highway, and  be the speed in which demand
points move along . Given a demand point ,  denotes the
weight of . The travel time between a demand point  and the
service facility , denoted by , is equal to:


The problem can be formulated as follows:

\medskip

{\bf The Freeway and Facility Location problem (FFL-problem)} Given
a set  of  demand points, the weight  of each point  of
, and fixed speed , locate the facility point  and the
highway  in order to minimize the next function:



\paragraph{Our results.}
We first show that there exist
optimal solutions of the FFL-problem in which the highway  has
infinite length and the facility point  is located on . We
then consider only optimal solutions satisfying these properties. We
second show that for all demand points  the shortest path
from  to the facility point  has one of three possible shapes:
(a)  moves directly to , (b)  first moves vertically to
reach  and after that moves along  to reach , and (c) 
first moves horizontally to reach  and after that moves along 
to reach . For each demand point, the shape of its shortest time
path to  depends on both the speed  in which demand points
moves along  and the slope of . This discretization on the
shortest path shapes allows us to simplify the expression of
 and then to obtain a clear
expression of the objective function . 
Using geometric obervations, we reduce the search space of the optimal solutions. This is
done by considering
the grid  defined by all axis-parallel lines passing through the
demand points. 
We prove the existence of optimal solutions 
that satisfy one of the following two properties: (1)  passes
through a demand point and  belongs to a line of grid , and
(2)  is a vertex of . The discretization of the search space
permits us to obtain the main result of this paper, a general
-time algorithm to solve the FFL-problem. Our algorithm
divides the search into two cases that correspond to the above two
properties. As a surprising result, we prove that when speed  is
greater than  the algorithm
can avoid the search of optimal solutions satisfying property (2)
because in that case there always exists an optimal solution which
holds property (1). This result simplifies the algorithm when speed
exceeds that bound. We finally present three examples,
two of them showing that when speed is increased and we keep the
same configuration of demand points the shapes of the shortest time
paths can change. A third example shows that when speed is less
than , there exist configurations in which the
optimal solution satisfies property (2).

\paragraph{Outline.}
The discretization on the shapes of the shortest paths from the
demand points to the facility is stated in Section~\ref{section:preliminaries}.
In Section~\ref{section:discretization} we show how the search
space of optimal solutions can be reduced.
In Section~\ref{section:algorithm-free} the algorithm to solve 
the FFL-problem is presented and in Section~\ref{section:refinement} we give the
refinement of it. In Section~\ref{section:experimental}, the examples are presented.
Finally, in Section~\ref{section:conclu}, we present the conclusions
and further research.


\section{Discretization of the shortest paths}
\label{section:preliminaries}

Any solution to our problem will be encoded by a pair of elements
, where  is the facility point and  is the highway.
Given  and , we say that a demand point  does not use 
(or goes directly to ) if  is equal to .
Otherwise we say that  uses . Given any point  of the plane,
let  and  denote the  and coordinates of ,
respectively.
\begin{claim}\label{claim1}
Let  and  be demand points using the highway  such that 
they move in contrary directions along . Let segments 
denote the portions of  traversed by  and , respectively.
Segments  and  have disjoint interiors.
\end{claim}
\newpage
\begin{proposition}\label{prop:fac-in-highway}
There exists an optimal solution of the
FFL-problem in which the facility point is located on the highway.
\end{proposition}
\begin{proof}
Let  denote an optimal solution of the FFL-problem and
suppose that  does not belong to . Let  be a translation of 
such that  belongs to . We select  so that to satisfy a condition that
will be stated later. Let  be a demand
point. If  does not use  then:

Otherwise, if  uses , let  be points such that

Let  be the point , which belongs to the line
containing . Observe from
Claim~\ref{claim1}
that we can select  so that point  belongs to  for every
demand point  using . Then, by using the triangular inequality with the
 metric, we obtain:

From equations (\ref{eq1}) and (\ref{eq2}) we have
. Then the pair  must be an
optimal solution and the result thus follows.
\end{proof}

Results similar to Propostion~\ref{prop:fac-in-highway}, stating that
the facility point belongs to the corresponding highway, can be 
found in~\cite{diaz-banez11-3,espejo11}. 
Observe from equations~(\ref{eqnobj}) and~(\ref{eq1})
that there always exists an
optimal solution  to the FFL-problem in which the length of
 is infinite.
We then assume from this
point forward that every solution satisfies that the highway is a straight
line and the facility point belongs to the highway.
Observe that this assumption does not have negative consequences,
due to the fact that in practice a highway of infinite length is not
possible. In fact, if a solution  to the problem is such that
 is a straight line, then  is also an optimal solution,
where  is the segment of minimum length 
such that every demand point both enter and exit  on a point of .

Let  always denote the non-negative angle of the highway
with respect to the positive direction of the -axis. Unless
otherwise specified, we assume .
Observe that if  we can, by properties of
 and  metrics, modify the coordinate system so that angle
 satisfies .

Given highway  and a demand point , let  be the
intersection point between  and the vertical line passing through
. Similarly, let  be the intersection point between  and
the horizontal line passing through . Let  denote the
half-line contained in  that emanates from  and does not
contain , and  denote the half-line contained in 
that emanates from  and does not contain . Notice from the
assumption  that given  and a
demand point ,  is the nearest point to  on  under the
 metric.

The next lemma characterizes the way in which demand points move
optimally to the facility.
Let . Since 
we have .

\begin{lemma}\label{lemma:way-of-move}
Given the highway  and the facility point  located on , the
following holds for all demand points .
If  then  moves first to  and after
that moves to  using .
Otherwise, if , the next
statements are true:
\begin{itemize}
  \item[a] If  then  moves first to  and after that
moves to  using .
  \item[b] If  then  moves first to  and after that
moves to  using .
  \item[c] If  then  moves
directly to  without using .
\end{itemize}
\end{lemma}

\begin{proof}
Let  be a demand point and assume w.l.o.g.\ that  is below
. Consider the function  for
all . Notice that  is convex because it is a sum of two
convex functions. Then any local minimum of  is a global minimum.
Let .

Suppose . Given  small enough, let
 and  be the points such that
. Refer to
Fig.~\ref{fig:movement-2}.
\begin{figure}[h]
  \centering
  \includegraphics[width=7.0cm]{movement-2.eps}
  \caption{\small{Proof of Lemma~\ref{lemma:way-of-move}. The boundary of the square
    represented with solid lines is the set of points  such that
    , and the perimeter of the square represented with dotted
    lines is the set of points  such that .}}
  \label{fig:movement-2}
\end{figure}

Then we have the following:


From equations (\ref{eq3}) and (\ref{eq4}) we conclude that 
is the minimum of . Therefore we have  and the
first part of the lemma thus follows.

Suppose now . Then we have three
cases:

{\em Case 1}: . On one hand we have  for
all points . On the other hand, if 
is small enough and  is the point
such that , then
. Therefore,  and statement (a) follows.

{\em Case 2}: . Let  be a small enough value.
On one hand, if  is the point
such that , then
. 
On the other hand, if  is the point
such that , then

Therefore,  and statement (b)
follows.

{\em Case 3}: . If 
then  is one of the nearest points to  on  by considering
the  metric. Thus
.
Otherwise, if , we proceed as follows. On one hand we have
 for all points  to the left of . On the other
hand, if  is small enough and  is the nearest
point to  satisfying , then
.
Therefore,  and statement (c) follows.
\end{proof}
Fig.~\ref{fig:movement} illustrates Lemma~\ref{lemma:way-of-move}.
\begin{figure}[h]
  \centering
  \includegraphics[width=15cm]{movement.eps}
  \caption{\small{a) If  then all points
  move vertically to .
  b) if  then some points move
  vertically to , some other points move horizontally, and the rest of the points move
  directly to .}}
  \label{fig:movement}
\end{figure}
Because of Lemma~\ref{lemma:way-of-move}, the travel time  between a
demand point  and facility  simplifies to:


Given a solution  to the
FFL-problem, we can always partition the set  of demand points
into there sets , , and
 as follows. Set  contains the points 
such that  and  and the points 
such that  and . Set  contains the
points  such that  and  is either on or below
, and the points  such that  and  is either
on or above . Set  is equal to .
It is straightforward to obtain what follows (refer to Fig.~\ref{fig:expression-detail}).
\begin{figure}[h]
  \centering
  \includegraphics[width=15cm]{objective-function.eps}
  \caption{\small{Points , , and  belong to , , and ,
  respectively. In case a) we have .
  In case b) we have .}}
  \label{fig:expression-detail}
\end{figure}

If  then  is equal to:

and the objective function  equals:

Otherwise, if , then
 is equal to:

and  equals:


\section{Reducing the search space}\label{section:discretization}

Let  be the grid defined by the set of all axis-parallel lines
passing through the elements of .
\begin{figure}[h]
  \centering
  \includegraphics[width=11.0cm]{grid-and-discretization.eps}
  \caption{\small{a) Lemma~\ref{lemma:discretization-FFL}~(a). b)
Lemma~\ref{lemma:discretization-FFL}~(b)}}
  \label{fig:grid-and-disc}
\end{figure}
\begin{lemma}\label{lemma:discretization-FFL}
There always exists an optimal solution  to the FFL-problem
satisfying one of the next statements:   contains a point of
 and  is on a line of grid , and   is a vertex of
grid . Refer to Fig.~\ref{fig:grid-and-disc}.
\end{lemma}
\begin{proof}
Let  be an optimal solution of the FFL-problem satisfying
neither condition (a) nor condition (b). Using local linear
perturbations, we will transform  into other optimal solution
that satisfies at least one of these conditions.

Assume first the case where . Let
 (resp. ) be the smallest value such
that if we translate both  and  with vector 
(resp. ) then  belongs to a vertical line of  or
 contains a point of . Given ,
let  and  be  and  translated with vector
, respectively. Using Lemma~\ref{lemma:way-of-move}, we
partition  into four sets , , , and  as
follows. Set  (resp. ) contains the demand points doing a
rightwards (resp. leftwards) movement to reach . Set  (resp.
) contains the demand points doing only a downwards (resp.
upwards) movement to reach . Observe that:
 where .
Let  denote , . Thus, for any
, the variation of objective function
when we translate both  and  with vector  is the
following:

Since  is optimal we must have
 for all
. It implies
 and
 for all
. Therefore, by translating both 
and  with vector either  or , we
ensure that  is on a vertical line of  or  passes through a
point of , or both conditions. If it holds only that  is on a
vertical line of , then we repeat the same operation in the
vertical direction in order to ensure that  is on a horizontal
line of  or  passes through a point of , and condition (a)
or condition (b) holds. Otherwise, if it holds only that  passes
through a point of , then it is straightforward to prove, by
using similar arguments, that  can be translated along  in
order to ensure that  belongs to a line of , that is either vertical or
horizontal, and condition (a) holds. In fact, for some demand point
 of ,  will coincide with the point in which  enters
, that is,  or .

In the case where  every demand point 
moves vertically to  (Lemma~\ref{lemma:way-of-move}), and we can
proceed as follows by using arguments similar to the above ones. We
first translate vertically both  and  with the same vector in
order to ensure that  contains a point of . After that,  is
translated along  if necessary in order to  belongs to a
vertical line of  and condition (a) holds. The lemma thus
follows.
\end{proof}
\begin{corollary}\label{cor:discretization-FFL}
If  then there is an optimal solution
satisfying Lemma~\ref{lemma:discretization-FFL}~.
\end{corollary}



\section{The algorithm to solve the
FFL-problem}\label{section:algorithm-free}

\begin{theorem}\label{theorem:free/var/sum}
The FFL-problem can be solved in  time.
\end{theorem}
\begin{proof}
We find an optimal solution by solving two cases separately. The
first case is when solution satisfies
Lemma~\ref{lemma:discretization-FFL}~(a), and the second case is
when solution satisfies Lemma~\ref{lemma:discretization-FFL}~(b).

In order to solve the first case we find for each demand point 
and each line  of  an optimal angle  such that
 is minimized, where  is
the line passing through  whose angle with respect to the
positive direction of the -axis is equal to , and
 is the intersection point between  and
. Assume w.l.o.g.\ that  is vertical and  is
located to the left of . It is easy to observe from
equations~(\ref{eq7}) and~(\ref{eq6}) that for any  the
expression of  has the form
,
where  are constants. Furthermore, if we progressively
increase the value of  from  to , that
expression changes whenever sets , , and  change,
that is, when  crosses a demand point, 
crosses a horizontal line of , or . Then
consider the sequence
 of 
angles, where each angle   is such that
either  contains a demand point, 
belongs to a horizontal line of grid , or .
Notice then that the expression of  is
the same for all  .
If we preprocess the demand points  by constructing the dual
arrangement of ~\cite{o-rourke98}, such a sequence can be
obtained in  time by using both the Zone
Theorem~\cite{o-rourke98} and the order of the demand points with
respect to the -coordinate. Observe that if for a value of  we
know the expression of  in the interval
, then  can be
minimized in constant time in that interval. Furthermore, if
 contains a demand point or 
belongs to a horizontal line of , then the expression of
 in the interval
 can be obtained in constant time from
the expression of  in the interval
. It is easy to see now that
, , can
be minimized in  time by minimizing
 in  for
. Since there are  demand points and  has
 lines, then an overall -time algorithm is obtained.

We can proceed similarly in order to solve the second case. We find
an optimal solution  for each vertex  of the grid  as
follows. Let  be a vertex of . Given an angle , let
 be the line passing through , whose angle with
respect to the positive direction of the -axis is equal to
. Then, by Corollary~\ref{cor:discretization-FFL}, we look
for an angle  such that the
objective function  is minimized. It follows
from equation~(\ref{eq6}) that the expression of
 has the form
 for any
, where  are
constants. If we progressively increase  from  to
 the expression of  keeps
unchanged as long as  does not cross a demand point. The
sorted sequence of values of  in which it happens can be
obtained in linear time by using duality~\cite{o-rourke98}. That
sequence of values induces a partition of the interval
 into intervals where in each of them the
expression of  is constant. We can now continue
as was done above to solve the first case. Since  has 
vertices then an overall -time algorithm is thus obtained.
The result thus follows.
\end{proof}

\section{A refinement of the
algorithm}\label{section:refinement}

 Theorem~\ref{theorem:free/var/sum} shows an algorithm
that solves the FFL-problem by dividing the search of optimal
solutions into two steps. It first looks in  time for an
optimal solution that satisfies
Lemma~\ref{lemma:discretization-FFL}~(a), and after that looks
within the same time complexity for an optimal solution satisfying
Lemma~\ref{lemma:discretization-FFL}~(b). In the following we show
that for ``reasonable'' values of speed  we can simplify the
algorithm of Theorem~\ref{theorem:free/var/sum} by finding only
optimal solutions that hold
Lemma~\ref{lemma:discretization-FFL}~(a). We will use the next
technical lemma.
\begin{lemma}\label{lemma:function}
Let , , , and  be non-negative
constants and  be a
function so that

for all . Then next statements are true:
\begin{itemize}
  \item[a] If  and  then  is constant.
  \item[b] If  and  then  is monotone decreasing.
  \item[c] If  and  then  is monotone increasing.
  \item[d] If  and  then  has no minima.
\end{itemize}
\end{lemma}
\begin{proof}
Statement (a) is immediate. Let  be the first derivative of 
and observe that:

If  and  then ,
, and equation
 has no solution in  because .
Therefore,  is a monotone decreasing function and statement
(b) thus holds.

If  and  then ,
, and equation
 has no solution in  because .
Therefore,  is a monotone increasing function and statement (c) thus
holds.

Consider  and . Since  it
suffices to prove that equation  has only one solution
which must be a global maximum of .
Equation  is equivalent to equation

We will prove that equation  has no solution in
, which implies that equation 
has a unique solution in  because
. This will complete the proof.

It is straightforward to see that equation  is equivalent
to equation , where  is equal to:

Consider the function  so
that

for all . Since 
and  for all , we have
 for all . We now show that
the minimum of  in  is greater than zero,
implying equation  has no solution.

If  then there is no  such
that  because  is
positive for all . Suppose
. Then we have:

and thus there is no  such that
. Therefore,  if and
only if , that is, . Let us prove
that .
 The
roots of polynomial 
are respectively equal to
 and
. 
Then we conclude that
 because the main coefficient of
 is positive,  is the greatest root of
, and . Since  and
 at the unique extreme point , then
 for all . Therefore,  for
all , equation  has no solution in
, and then  has only one extreme point in
 which is a global maximum. The lemma
follows.
\end{proof}

\begin{lemma}\label{lemma:discretization-FFL-2}
If speed  is greater than , then there always exists an optimal solution 
to the FFL-problem satisfying
Lemma~\ref{lemma:discretization-FFL}~, that is,  contains a
point of  and  is on a line of grid .
\end{lemma}
\begin{proof}
It suffices to show that for every solution , there
exists another solution  satisfying
Lemma~\ref{lemma:discretization-FFL}~(a) and
. The proof is as follows.

Let  be a solution of the FFL-problem so that 
contains no demand point. Assume here angle  satisfies
. If  is such that
, then the result follows from
Corollary~\ref{cor:discretization-FFL}. Therefore, assume
.
We proceed to prove that there exists another solution
 such that  contains a demand point and
. Consider the sets  and
 as defined before.

Given an angle , let 
be the line passing through  so that the angle between
 and the positive direction of the -axis is equal to
. Observe then by equation~(\ref{eq6}) that  is equal to:

where , , and  are non-negative constants that depend on
the coordinates of the demand points and the speed . Then we can
argue what follows by both noting that  (resp. ) if and only if
 (resp. ) is not empty and using
Lemma~\ref{lemma:function}.

We can rotate  with center  by either increasing or decreasing
 to the value  in such
a way solution  is obtained, where either
 or  contains a demand point of
, and . If
 then the result follows from
Corollary~\ref{cor:discretization-FFL}. Otherwise, if 
contains a demand point of , then
 is the desired solution, and to
finalize the proof, we translate  along , if necessary, as
was done in the proof of Lemma~\ref{lemma:discretization-FFL}, in
order to ensure  belongs to a line of grid . The result thus
follows.
\end{proof}

\section{Examples}\label{section:experimental}

In Fig.~\ref{fig:experiment1} and Fig.~\ref{fig:experiment2} we show
the same example, consisting of nine demand points
. Each demand point is represented by a solid
dot, and labeled with a triple, the first two components are the
coordinates and the third component is its weight. Facility point
 is represented by a cross. In both examples optimal solutions
satisfy Lemma~\ref{lemma:discretization-FFL}~(a), that is, highway
contains a demand point and facility point belongs to a line of grid
. As expected, when we increase the highway's speed from the
example in Fig.~\ref{fig:experiment1} to the one in
Fig.~\ref{fig:experiment2}, the shortest paths to the facility
point change according to the claims of
Lemma~\ref{lemma:way-of-move}.
\begin{figure}[h]
  \centering
  \includegraphics[width=10cm]{experiment-1.eps}
  \caption{\small{The highway contains a demand point and facility point belongs
  to a line of grid . Some points perform an horizontal movement to reach the
highway.}}
  \label{fig:experiment1}
\end{figure}

\begin{figure}[h]
  \centering
  \includegraphics[width=10cm]{experiment-2.eps}
  \caption{\small{The highway contains a demand point and facility point belongs
  to a line of grid . All points perform only a vertical movement to reach the
highway.}}
  \label{fig:experiment2}
\end{figure}

In Fig.~\ref{fig:experiment1} speed  is equal to  and
highway  of the optimal solution contains point , facility
point  is on the vertical line passing through
, and there are some demand points moving horizontally to reach
facility point . The value of the objective function 
is equal to .

In Fig.~\ref{fig:experiment2} speed  is equal to , highway
 of the optimal solution contains point , facility
 is on the vertical line containing , and all
demand points move vertically only to reach the highway. Since speed
is greater than speed in Fig.~\ref{fig:experiment1}, the value of
the objective function  reduces to .

In Fig.~\ref{fig:experiment3} we present a different example
consisting of nine demand points , with the aim of
showing the existence of configurations for which optimal solutions
satisfy only Lemma~\ref{lemma:discretization-FFL}~(b). In this
example speed  is equal to . Highway
 of the optimal solution contains no demand point and facility
 is on a vertex of grid , in fact, it is located on both
the horizontal line through  and the vertical line through
. The value of  is equal to .

\begin{figure}[h]
  \centering
  \includegraphics[width=9cm]{experiment-3.eps}
  \caption{\small{The highway contains no demand point and facility point
  is a vertex of grid .}}
  \label{fig:experiment3}
\end{figure}

\section{Conclusions and further research}\label{section:conclu}

We have solved in  time the problem of locating at the same
time a facility point and a freeway of variable
length,
among a set of demand points, in order to minimize the total
weighted travel time from the demand points to the facility.
Some examples are presented to show that there exist optimal
solutions corresponding to each type of solutions that our
algorithm considers.

A natural restriction to be considered in further research
of this problem is to upper bound the length of the highway,
that is, to consider that highway has fixed length.
In this case, there also exist optimal solutions in which the facility
point belongs to the highway. This was in fact showed in
Proposition~\ref{prop:fac-in-highway} because in the proof
we did not change the lenght of the highway.
It is not hard to see that when the highway's length is fixed,
the shortest paths from the demand points to the facility point can be
discretized as follows. If , then we distinguish
three regions , , and  as depicted in
Fig.~\ref{fig:disc-fixed-len}~a). Points belonging to 
move to the nearest endpoint of ,
and points of  move vertically to . Otherwise,
if , then eight regions
 can be identified as 
shown in Fig.~\ref{fig:disc-fixed-len}~b). Points of
 move to the nearest endpoint of , 
points of  move directly to , points of 
move horizontally to , and points of  move vertically
to . 
\begin{figure}[h]
  \centering
  \includegraphics[width=15cm]{fixed-len-disc.eps}
  \caption{\small{Discretization when the highway's length is fixed.}}
  \label{fig:disc-fixed-len}
\end{figure}

We believe that with the above discretization, the search space of
optimal solutions can be simplified by using a similar (and more detailed)
statement as Lemma~\ref{lemma:discretization-FFL}. This will permit
to obtain an algorithm similar to the one presented 
in Theorem~\ref{theorem:free/var/sum}.

Other variant to be considered in further research is the problem of locating at the
same time the facility point and a turnpike of variable length. This will
extend~\cite{diaz-banez11,espejo11}.


\section*{Acknowledgement}

Authors D\'iaz-B\'a\~nez and Ventura were partially supported by 
project FEDER MEC MTM2009-08652 and ESF EUROCORES programme 
EuroGIGA - ComPoSe IP04 - MICINN 
Project EUI-EURC-2011-4306. 
Korman had the support of the Secretary for 
Universities and Research of the Ministry of Economy and Knowledge of the 
Government of Catalonia and the European Union.
P\'erez-Lantero was partially supported by project FEDER MEC MTM2009-08652 and
grant FONDECYT 11110069.

\nocite{*}

\small

\bibliographystyle{plain}
\bibliography{freeway}

\end{document}
