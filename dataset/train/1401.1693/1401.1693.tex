
\newcommand{\Pos}{\mathit{Pos}}

A \emph{parity game} is given by the following data:
  \begin{itemize}
  \item a (finite or infinite) set of positions $\Pos$
           partitioned into proponent's (Player 0) and opponent's (Player 1) positions:
           $\Pos = \Pos_0 + \Pos_1$\@;
   \item a total edge relation $\rightarrow   \,\subseteq\, \Pos \cross \Pos$\@;\footnote{ 
             total means forall $p\in \Pos$ there exists $p' \in\Pos$ with  $p \rightarrow p'$\@.}
    \item a function $\Omega \in \Pos \rightarrow \Nat$ with a finite range; 
            we call $\Omega(p)$ the priority of position $p$\@. 
  \end{itemize}
The players move a token along the edge relation $\rightarrow$\@. When the
token is on a position in $\Pos_0$ then proponent decides where to move
next and likewise for opponent.

In order to formalize the notion of ``to decide'' we must introduce
strategies.  Formally, a strategy for a player $i\in {0,1}$ is a
function $\sigma$ that for any nonempty string $\vec p =
p(0)\ldots p(n)$ over $\Pos$ and such that $p(k)\rightarrow p(k+1)$
for $k=0 \ldots n-1$ and $p(n)\in Pos_i$ associates a position
$\sigma(\vec p) \in\Pos$ such that $p(n) \rightarrow
\sigma(\vec p)$\@.

Given a starting position $p$ and strategies $\sigma_0$ and $\sigma_1$ for
the two players one then obtains an infinite sequence of positions (a
``play'') $p(0),p(1),p(2),\ldots$  by 
\begin{eqnarray*}
   p(0) &=& p \\
p(n+1) &=& \sigma_i(p(0) \ldots p(n))   ~~\mathit{where}~ p(n) \in\Pos_i
\end{eqnarray*}
We denote this sequence by $\mathit{play}(p,\sigma_0,\sigma_1)$\@.  

The play is won by proponent (Player 0) if the largest number that
occurs infinitely often in the sequence $\Omega(\mathit{play}(p,\sigma_0,\sigma_1))$
is even and it is won by opponent if that number is odd. Note that
$\Omega(\_)$ is applied component-wise and that a largest priority indeed
exists since $\Omega$ has finite range.

Player $i$ wins from position $p$ if there exists a strategy $\sigma_i$ for
player $i$ such that for any strategy $\sigma_{1-i}$ for the other player
(Player $1-i$) player $i$ wins $\mathit{play}(p,\sigma_0,\sigma_1)$\@.  
We write $W_i$ for the set of positions from which Player $i$ wins.

A strategy $\sigma$ is \emph{positional} if $\sigma(p(0)..p(n))$ only
depends on $p(n)$\@.  Player $i$ \emph{wins positionally} from $p$
when the above strategy $\sigma_i$ can be chosen to be positional.

The following is a standard result.
\begin{theorem}\label{theorem.positional.wins}
Every position $p$ is either in $W_0$ or in $W_1$ and player $i$ wins
positionally from every position in $W_i$\@. 
\end{theorem}
In view of this theorem we can now confine attention to positional
strategies. A strategy that wins against all positional strategies is
indeed a winning strategy (against all strategies) since the optimal
counterstrategy is itself positional. 
\begin{example}
  Fig.~\ref{parisp} contains a graphical display of a parity game. 
Positions in $\Pos_0$ and $\Pos_1$ are
  represented as circles and boxes, respectively, and labelled with
  their priorities. Formally, $\Pos=\{a,b,c,d,e,f,g,h,i\}$ and $\Pos_0=\{b,d,f,h\}$ and $\Pos_1=\{a,c,e,g,i\}$ and $\Omega(a)=3$, \dots, and $\rightarrow=\{(a,b), (b,f),\dots\}$. 

  In the right half of Fig.~\ref{parisp} the winning sets are
  indicated and corresponding positional winning strategies are given
  as fat arrows. The moves from positions that are not in the
  respective winning set are omitted but can of course be filled in in
  an arbitrary fashion.
\end{example}
\begin{figure}
\begin{tabular}{cc}
\includegraphics[scale=0.4]{parisp.pdf} \hspace{3em} & \includegraphics[scale=0.4]{parisp2.pdf} 
\end{tabular}
\caption{\label{parisp}
A parity game and its decomposition into winning sets.}
 \end{figure}
\subsection{Certification of winning strategies}




Given a parity game with finitely many positions, presented explicitly as a finite labelled graph, and 
a partition of $\Pos$ into $V_0$ and $V_1$ we are now looking for an
easy to verify certificate as to the fact that $V_0 = W_0$ and $V_1 = W_1$\@.

In essence, such a certificate will consist of a positional strategy
$\sigma_i$ for each player $i$ such that $i$ wins using $\sigma_i$
from every position $p$ in $V_i$\@. Clearly this implies $V_i=W_i$ and
the above theorem asserts that in principle such certificates always
exist when $V_i=W_i$. However, it remains to explain how we can check
that a given positional strategy $\sigma_i$ wins from a given position
$p$\@.

We first note that for this it is enough that it wins against any
adversarial positional strategy because the ``optimal''
counterstrategy, i.e., the one that wins from all adversarial winning
positions is positional (by theorem~\ref{theorem.positional.wins})\@. 
Thus, given a positional strategy $\sigma_i$ for player $i$ we can remove all edges from
positions $p'\in\Pos_i$ that are not chosen by the strategy and in the
remaining game graph look for a cycle whose largest priority has
parity $1-i$ and is reachable from $p$\@. If there is such a cycle
then the strategy was not good and otherwise it is indeed a winning
strategy for Player $i$\@. 

Algorithmically, the absence of such a cycle can be checked by
starting a depth-first search from every position of adversary
priority (parity $1-i$) after removing all positions of favourable and
higher priority (parity $i$). More efficiently, one can decompose
the reachable (from the purported winning set) part of the remaining
graph into nontrivial strongly connected components (SCC). If such an
SCC only contains positions whose priority has parity $1-i$ then,
clearly, the strategy is bad. Otherwise, one may remove the positions
with the largest priority of parity $i$, decompose the remaining graph
into SCCs and continue recursively. Essentially, this is the standard
algorithm for nonemptiness of Strett automata described in
\cite{DBLP:journals/fmsd/BloemGS06}. This paper also describes an
efficient algorithm for SCC decomposition that could be used here.
\begin{example}
After removing the edges not taken by Player 0 according to his purported winning strategy we obtain the following graph: 

\includegraphics[scale=0.4]{parisp3.pdf}

We see that the two reachable SCC from $W_0$ are $\{a,b,f\}$ and
$\{g,h\}$. The first one contains the cycles $a,f$ and $a,b,f$ which
both have largest priority $4$. The other one is itself a cycle with
largest priority $2$.

Likewise, adopting the viewpoint of Player 1, after removing the edges
not taken by his strategy we obtain

\includegraphics[scale=0.4]{parisp4.pdf}

and find the reachable (from $W_1$) SCCs to be $\{c,d,i\}$. The only cycles therein are $d,e$ and $d,e,i$. Both are good for Player 1. 
\end{example}
\subsection{Game-theoretic characterization}

The game $G(T,\eta)$ associated with the LTS $T$ and $\eta$ as above
is the defined as follows.
Positions are pairs $(s,\phi)$ where $\FV(\phi) \subseteq  \mathit{dom}(\eta)$ and $s\in S$\@. 
In positions of the form $(s,\psi)$ where $\psi$ starts with $\Lor$ or $\may{a}$, it is
proponent's (Player 0) turn. The possible moves (for proponent to choose from)
are:
   \begin{eqnarray*}
(s, \psi_1 \Lor \psi_2) &\leadsto& (s, \psi_1) \\
(s,\psi_1 \Lor  \psi_2) &\leadsto& (s, \psi_2) \\
     (s,\may{a}\psi) &\leadsto& (t,\psi) ~\mathit{where}~(s~\stackrel{a}{\longrightarrow}~t) \in T\mbox{.}
   \end{eqnarray*}
In positions of the form $(s,\psi)$ where $\psi$ starts with $\Land$ or $\must{a}$ it is
the opponent's turn. The possible moves (for opponent to choose from)
 are: 
 \begin{eqnarray*}
  (s,\psi_1 \Land \psi_2)   &\leadsto& (s, \psi_1)  \\
    (s,\psi_1 \Land \psi_2) &\leadsto& (s, \psi_2) \\
            (s,\must{a}\psi)  &\leadsto& (t,\psi) ~\textit{where}~(s~\stackrel{a}{\longrightarrow} t) \in T\mbox{.}
 \end{eqnarray*}
In positions of the form $(s,\Q X.\phi)$ the proponent is to move, but
there is only one possible move:
 \begin{eqnarray*}
(s,\mu X.\phi) &\leadsto& (s,\phi[X := \mu X.\phi]) \\
(s,\nu X.\phi) &\leadsto& (s,\phi[X := \nu X.\phi])
 \end{eqnarray*}

In positions of the form $(s,X)$, $(s,P)$, the proponent is to move, but
there is only one possible move: $(s,X)$ (respectively $(s,P)$) itself. 
So, de facto, the game ends in such a position.

Each position $(s,\phi)$ is assigned a natural number $\Omega(s,\phi)$, its
priority, as follows:
\begin{eqnarray*}
\Omega(s,\mu X.\phi) &=& 2 * \nd(\mu X.\phi) + 1  \\
\Omega(s,\nu X.\phi) &=& 2 * \nd(\nu X.\phi)\\
\Omega(s,P) &=& 0 ~~\mathit{if}~P \in T(s)\\
\Omega(s,P) &=& 1  ~~\mathit{if}~P \not\in T(s)\\
\Omega(s,X) &=& 0  ~~\mathit{if}~s \in \eta(X)\\
\Omega(s,X) &=& 1  ~~\mathit{if}~s \not\in \eta(X)\\
\Omega(s,\phi) &=& 0 ~~\mathit{in~all~other~cases}. 
\end {eqnarray*}




For any position $(s,\phi)$ we can consider the subgame consisting
of the positions reachable from $(s,\phi)$\@. 
Even if $T$ is infinite this subgame has only finitely many priorities because
the only second components occurring in reachable positions are subformulas of 
$\phi$ and one-step unwindings thereof. This subgame is therefore a parity
game to which the previous section applies. 
\begin{example}\label{noten}
Let $\phi = \mu X.\, P  \Lor \may{a}X $ which 
asserts that a state where $P$ is true can be reached.

Define the transition system $T$ by $|T|=\{s,t\}$ and
$\stackrel{a}{\longrightarrow}_T = \{(s,s), (s,t), (t,t)\}$ and
$T(s)=\emptyset$ and $T(t)=\{P\}$.  The associated game graph is as
follows 

\medskip

\begin{picture}(0,0)\includegraphics[scale=0.5]{spiel1.pdf}\end{picture}\setlength{\unitlength}{2072sp}\begingroup\makeatletter\ifx\SetFigFont\undefined \gdef\SetFigFont#1#2#3#4#5{\reset@font\fontsize{#1}{#2pt}\fontfamily{#3}\fontseries{#4}\fontshape{#5}\selectfont}\fi\endgroup \begin{picture}(5649,3399)(1564,-3898)
\put(3901,-3616){\makebox(0,0)[lb]{\smash{{\SetFigFont{10}{24.0}{\rmdefault}{\mddefault}{\updefault}{\color[rgb]{0,0,0}$P\Lor\may{a}\phi,t$}}}}}
\put(6436,-3601){\makebox(0,0)[lb]{\smash{{\SetFigFont{10}{24.0}{\rmdefault}{\mddefault}{\updefault}{\color[rgb]{0,0,0}$P,t$}}}}}
\put(6421,-916){\makebox(0,0)[lb]{\smash{{\SetFigFont{10}{24.0}{\rmdefault}{\mddefault}{\updefault}{\color[rgb]{0,0,0}$P,s$}}}}}
\put(3886,-916){\makebox(0,0)[lb]{\smash{{\SetFigFont{10}{24.0}{\rmdefault}{\mddefault}{\updefault}{\color[rgb]{0,0,0}$P\Lor\may{a}\phi,s$}}}}}
\put(1816,-3616){\makebox(0,0)[lb]{\smash{{\SetFigFont{10}{24.0}{\rmdefault}{\mddefault}{\updefault}{\color[rgb]{0,0,0}$\phi,t$}}}}}
\put(1786,-931){\makebox(0,0)[lb]{\smash{{\SetFigFont{10}{24.0}{\rmdefault}{\mddefault}{\updefault}{\color[rgb]{0,0,0}$\phi,s$}}}}}
\put(4006,-2266){\makebox(0,0)[lb]{\smash{{\SetFigFont{10}{24.0}{\rmdefault}{\mddefault}{\updefault}{\color[rgb]{0,0,0}$\may{a}\phi,s$}}}}}
\end{picture} 
\medskip 

The priorities of the positions labelled $(\phi,s), (\phi,t), (P,s)$ are 1; the priorities of the three other positions are 0. 

Player 0 wins from every position except $(P,s)$. The winning strategy
moves to $(\may{a}\phi,s)$ and then $(\phi,t)$ and then $(P,t)$. Note
that a strategy that moves from $(P\Lor \may{a}\phi,s)$ to $(\phi,s)$
looses even though it never leaves the winning set $W_0$. Thus, in
order to compute winning strategies it is not enough to choose any
move that remains in the winning set.
\end{example}

\begin{theorem}\label{theorem.game.characterization}
If $s \in \Sem{\phi}{\eta}$ then proponent wins $G(T,\eta)$ from $(s,\phi)$\@.
\end{theorem}
Before proving this, we note that the converse is in this case actually a relatively simple consequence. 
\begin{corollary}\label{corollary.game.characterization}
If proponent wins $G(T,\eta)$ from $(s,\phi)$ then $s \in \Sem{\phi}{\eta}$\@. 
\end{corollary}
\begin{proof}
Suppose that proponent wins $G(T,\eta)$ from $(s,\phi)$ and $s \not\in \Sem{\phi}{\eta}$\@.
We then have $s \in \Sem{\phi^*}{\eta'}$ using Lemma~\ref{lemma:dual} for the
formal dualisation for formulas and complementation for environments\@.
Thus, by the theorem, proponent wins $G(T,\eta')$ from $(s,\phi^*)$\@. 
However, it is easy to see that a winning strategy for proponent in $G(T,\eta')$ from
$(s,\phi^*)$ is tantamount to a winning strategy for opponent in $G(T,\eta)$
from $(s,\phi)$\@; so we get a contradiction using theorem~\ref{theorem.positional.wins}\@. 
\end{proof}
\begin{proof}[of Theorem~\ref{theorem.game.characterization}]
The proof of Theorem~\ref{theorem.game.characterization} now works by
 induction
on $\phi$\@. The cases where $\phi$ is a formula with an outermost fixpoint are the interesting ones. 

In case $\phi$ is of the form $\mu X.\psi$\@, 
define 
  $$U := \{t \,|\, \mathit{proponent~wins~} G(T,\eta) \mathit{~from~}  (t,\mu X.\psi)\}.$$  
We must show that $\Sem{\phi}{\eta} \subseteq U$\@. 
By definition of $\Sem{\phi}{\eta}$ it suffices to show that $\Sem{\psi}{\eta[X \mapsto U]} \subseteq U$\@.
Thus, suppose that $t \in \Sem{\psi}{\eta[X \mapsto U]}$\@.
By the induction hypothesis this means that proponent wins $G(T,\eta[X \mapsto U])$ from $(t,\psi)$\@.
Call the corresponding winning strategy $\sigma$\@.  
We should prove that proponent also wins from $(t,\mu X.\psi)$\@. 
We move to $(t,\psi[X:= \mu X.\psi])$ and play
according to $\sigma$ until we reach a state $(t',\mu X.\psi)$ at which point we
know that $t'\in U$ so we can then continue to play according to the
strategy embodied in that statement. 
Of course, if we never reach such a position then $\sigma$ 
will win the whole game.

In case $\phi$ is of the form $\nu X.\psi$ define $ U := \Sem{\nu
  X.\psi}{\eta}$\@.  We define a winning strategy for positions of the
form $(t,\nu X.\psi)$ where $t\in U$ as follows.  First, we move
(forcedly) to $(t,\psi[X:=\nu X.\psi])$\@.  We know that $t \in
\Sem{\psi}{\eta[X \mapsto U]}$ by unwinding so that, inductively, we
have a strategy that allows us to either win rightaway, or move to
another position $(t',\nu X.\psi)$ where $t' \in U$ and all priorities
encountered on the way are smaller than the one of $\nu X.\psi$ due to
the definition of nesting depth and in particular
Lemma~\ref{nedele}. We start over and unless we eventually do win
rightaway at some point we would have seen the priority of $\nu
X.\psi$ infinitely often which is the largest and even.
\end{proof}
We remark that while the previous result is well-known 
the proof presented here 
is quite different from the ones in the standard literature, e.g.\  \cite{Bradfield01modallogics} 
which uses the order-theoretic concept of 
signatures and are less compositional than ours in the sense that the proof is not directly by structural induction on formulas but rather on the global development of all the fixpoints. 