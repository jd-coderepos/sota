
\newcommand{\Pos}{\mathit{Pos}}

A \emph{parity game} is given by the following data:
  \begin{itemize}
  \item a (finite or infinite) set of positions 
           partitioned into proponent's (Player 0) and opponent's (Player 1) positions:
           \@;
   \item a total edge relation \@;\footnote{ 
             total means forall  there exists  with  \@.}
    \item a function  with a finite range; 
            we call  the priority of position \@. 
  \end{itemize}
The players move a token along the edge relation \@. When the
token is on a position in  then proponent decides where to move
next and likewise for opponent.

In order to formalize the notion of ``to decide'' we must introduce
strategies.  Formally, a strategy for a player  is a
function  that for any nonempty string  over  and such that 
for  and  associates a position
 such that \@.

Given a starting position  and strategies  and  for
the two players one then obtains an infinite sequence of positions (a
``play'')   by 

We denote this sequence by \@.  

The play is won by proponent (Player 0) if the largest number that
occurs infinitely often in the sequence 
is even and it is won by opponent if that number is odd. Note that
 is applied component-wise and that a largest priority indeed
exists since  has finite range.

Player  wins from position  if there exists a strategy  for
player  such that for any strategy  for the other player
(Player ) player  wins \@.  
We write  for the set of positions from which Player  wins.

A strategy  is \emph{positional} if  only
depends on \@.  Player  \emph{wins positionally} from 
when the above strategy  can be chosen to be positional.

The following is a standard result.
\begin{theorem}\label{theorem.positional.wins}
Every position  is either in  or in  and player  wins
positionally from every position in \@. 
\end{theorem}
In view of this theorem we can now confine attention to positional
strategies. A strategy that wins against all positional strategies is
indeed a winning strategy (against all strategies) since the optimal
counterstrategy is itself positional. 
\begin{example}
  Fig.~\ref{parisp} contains a graphical display of a parity game. 
Positions in  and  are
  represented as circles and boxes, respectively, and labelled with
  their priorities. Formally,  and  and  and , \dots, and . 

  In the right half of Fig.~\ref{parisp} the winning sets are
  indicated and corresponding positional winning strategies are given
  as fat arrows. The moves from positions that are not in the
  respective winning set are omitted but can of course be filled in in
  an arbitrary fashion.
\end{example}
\begin{figure}
\begin{tabular}{cc}
\includegraphics[scale=0.4]{parisp.pdf} \hspace{3em} & \includegraphics[scale=0.4]{parisp2.pdf} 
\end{tabular}
\caption{\label{parisp}
A parity game and its decomposition into winning sets.}
 \end{figure}
\subsection{Certification of winning strategies}




Given a parity game with finitely many positions, presented explicitly as a finite labelled graph, and 
a partition of  into  and  we are now looking for an
easy to verify certificate as to the fact that  and \@.

In essence, such a certificate will consist of a positional strategy
 for each player  such that  wins using 
from every position  in \@. Clearly this implies  and
the above theorem asserts that in principle such certificates always
exist when . However, it remains to explain how we can check
that a given positional strategy  wins from a given position
\@.

We first note that for this it is enough that it wins against any
adversarial positional strategy because the ``optimal''
counterstrategy, i.e., the one that wins from all adversarial winning
positions is positional (by theorem~\ref{theorem.positional.wins})\@. 
Thus, given a positional strategy  for player  we can remove all edges from
positions  that are not chosen by the strategy and in the
remaining game graph look for a cycle whose largest priority has
parity  and is reachable from \@. If there is such a cycle
then the strategy was not good and otherwise it is indeed a winning
strategy for Player \@. 

Algorithmically, the absence of such a cycle can be checked by
starting a depth-first search from every position of adversary
priority (parity ) after removing all positions of favourable and
higher priority (parity ). More efficiently, one can decompose
the reachable (from the purported winning set) part of the remaining
graph into nontrivial strongly connected components (SCC). If such an
SCC only contains positions whose priority has parity  then,
clearly, the strategy is bad. Otherwise, one may remove the positions
with the largest priority of parity , decompose the remaining graph
into SCCs and continue recursively. Essentially, this is the standard
algorithm for nonemptiness of Strett automata described in
\cite{DBLP:journals/fmsd/BloemGS06}. This paper also describes an
efficient algorithm for SCC decomposition that could be used here.
\begin{example}
After removing the edges not taken by Player 0 according to his purported winning strategy we obtain the following graph: 

\includegraphics[scale=0.4]{parisp3.pdf}

We see that the two reachable SCC from  are  and
. The first one contains the cycles  and  which
both have largest priority . The other one is itself a cycle with
largest priority .

Likewise, adopting the viewpoint of Player 1, after removing the edges
not taken by his strategy we obtain

\includegraphics[scale=0.4]{parisp4.pdf}

and find the reachable (from ) SCCs to be . The only cycles therein are  and . Both are good for Player 1. 
\end{example}
\subsection{Game-theoretic characterization}

The game  associated with the LTS  and  as above
is the defined as follows.
Positions are pairs  where  and \@. 
In positions of the form  where  starts with  or , it is
proponent's (Player 0) turn. The possible moves (for proponent to choose from)
are:
   
In positions of the form  where  starts with  or  it is
the opponent's turn. The possible moves (for opponent to choose from)
 are: 
 
In positions of the form  the proponent is to move, but
there is only one possible move:
 

In positions of the form , , the proponent is to move, but
there is only one possible move:  (respectively ) itself. 
So, de facto, the game ends in such a position.

Each position  is assigned a natural number , its
priority, as follows:
U := \{t \,|\, \mathit{proponent~wins~} G(T,\eta) \mathit{~from~}  (t,\mu X.\psi)\}.  
We must show that \@. 
By definition of  it suffices to show that \@.
Thus, suppose that \@.
By the induction hypothesis this means that proponent wins  from \@.
Call the corresponding winning strategy \@.  
We should prove that proponent also wins from \@. 
We move to  and play
according to  until we reach a state  at which point we
know that  so we can then continue to play according to the
strategy embodied in that statement. 
Of course, if we never reach such a position then  
will win the whole game.

In case  is of the form  define \@.  We define a winning strategy for positions of the
form  where  as follows.  First, we move
(forcedly) to \@.  We know that  by unwinding so that, inductively, we
have a strategy that allows us to either win rightaway, or move to
another position  where  and all priorities
encountered on the way are smaller than the one of  due to
the definition of nesting depth and in particular
Lemma~\ref{nedele}. We start over and unless we eventually do win
rightaway at some point we would have seen the priority of  infinitely often which is the largest and even.
\end{proof}
We remark that while the previous result is well-known 
the proof presented here 
is quite different from the ones in the standard literature, e.g.\  \cite{Bradfield01modallogics} 
which uses the order-theoretic concept of 
signatures and are less compositional than ours in the sense that the proof is not directly by structural induction on formulas but rather on the global development of all the fixpoints. 