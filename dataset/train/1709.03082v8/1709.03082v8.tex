\section{Introduction}

By 2019, the cost to the global economy due to cybercrime is projected to reach \\approxX\mu\sigma[0, n-1]10^{th}20^{th}90^{th}100^{th}n-1f(x) = sign(\vec{w}\vec{x}+b)yxargmaxargmax\{\vec{x}_{i}\ |\ \vec{x}_{i} \in \mathbb{R}^{m}\}\vec{x}_{i}f(x) = sign(\vec{w}\vec{x} + b)\approx\approx\approx\approx\approx\approx\approx\approx\approx\approx\approx\approx\approx\approx\approx\approx\approx\approx\approx\approx\approx\approx\approx\approx\approx\approx\approx\approx\approx\approx\approx\approx\approx\approx\approx\sigma\sigma\sigmayxO(1)O(n)$. As results have shown, the GRU-SVM model also outperformed the GRU-Softmax model in both training time and testing time. Thus, it corroborates the respective algorithm complexities of the classifiers.\\

\section{Conclusion and Recommendation}

We proposed an amendment to the architecture of GRU RNN by using SVM as its final output layer in a binary/non-probabilistic classification task. This amendment was seen as viable for the fast prediction time of SVM compared to Softmax. To test the model, we conducted an experiment comparing it with the established GRU-Softmax model. Consequently, the empirical data attests to the effectiveness of the proposed GRU-SVM model over its comparator in terms of predictive accuracy, and training and testing time.\\
\indent	Further work must be done to validate the effectiveness of the proposed GRU-SVM model in other binary classification tasks. Extended study on the proposed model for a faster multinomial classification would prove to be prolific as well. Lastly, the theory presented to explain the relatively low performance of the Softmax function as a binary classifier might be a pre-cursor to further studies.

\section{Acknowledgment}
An appreciation to the open source community (Cross Validated, GitHub, Stack Overflow) for the virtually infinite source of information and knowledge; to the Kyoto University for their intrusion detection dataset from their honeypot system.