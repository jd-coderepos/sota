

\documentclass[11pt,a4paper]{article}
\usepackage[hyperref]{emnlp2020}
\usepackage{times}
\usepackage{latexsym}
\renewcommand{\UrlFont}{\ttfamily\small}


\usepackage{url}

\usepackage{graphicx}
\usepackage{CJKutf8}
\usepackage{multirow}
\usepackage{array}
\usepackage{tabularx}
\usepackage{makecell}
\usepackage{amsmath} 
\usepackage{amssymb}
\usepackage{tablefootnote}
\usepackage{subcaption}
\usepackage{diagbox}
\usepackage{graphicx}
\usepackage{algorithm}
\usepackage[noend]{algpseudocode}
\usepackage{booktabs}
\usepackage{xcolor}
\usepackage{color, soul} \usepackage{colortbl}
\usepackage{makecell}
\definecolor{orange}{rgb}{1,0.5,0}
\definecolor{green}{rgb}{0,0.5,0.5}



\usepackage{microtype}

\newcommand\BibTeX{B\textsc{ib}\TeX}
\newcolumntype{L}[1]{>{\raggedright\arraybackslash}p{#1}}
\newcolumntype{C}[1]{>{\centering\arraybackslash}p{#1}}
\newcolumntype{R}[1]{>{\raggedleft\arraybackslash}p{#1}}

\DeclareSymbolFont{extraup}{U}{zavm}{m}{n}
\DeclareMathSymbol{\varheart}{\mathalpha}{extraup}{86}
\DeclareMathSymbol{\vardiamond}{\mathalpha}{extraup}{87}


\usepackage{microtype}

\aclfinalcopy \def\aclpaperid{3298} 




\title{Named Entity Recognition for Social Media Texts \\
with Semantic Augmentation
}



  


\author{
    Yuyang Nie, \hspace{0.1cm}
    Yuanhe Tian, \hspace{0.1cm}
    Xiang Wan, \hspace{0.1cm}
    Yan Song, \hspace{0.1cm}
    Bo Dai \\
University of Electronic Science and Technology of China\\
    University of Washington \hspace{0.1cm} Shenzhen Research Institute of Big Data\\
    The Chinese University of Hong Kong (Shenzhen)\\
    \tt  nyy207@gmail.com \hspace{0.1cm} yhtian@uw.edu \hspace{0.1cm} wanxiang@sribd.cn\\
    \tt songyan@cuhk.edu.cn \hspace{0.1cm} daibo@uestc.edu.cn
}

\date{}










\begin{document}
\maketitle

\def\thefootnote{*}\footnotetext{Equal contribution.}
\def\thefootnote{\dag}\footnotetext{Corresponding author.}

\def\thefootnote{\arabic{footnote}}



\begin{abstract}
\textcolor{black}{
Existing approaches for named entity recognition suffer from data sparsity problems when conducted on short and informal texts, especially user-generated social media content.
Semantic augmentation is a potential way to alleviate this problem.
Given that rich semantic information is implicitly preserved in pre-trained word embeddings, they are potential ideal resources for semantic augmentation.
In this paper, we propose a neural-based approach to NER for social media texts where both local (from running text) and augmented semantics are taken into account.
In particular, we obtain the augmented semantic information from a large-scale corpus, and propose an 
attentive semantic
augmentation module and a gate module to encode and aggregate such information, respectively.
Extensive experiments are performed on three benchmark datasets collected from English and Chinese social media platforms,
where the results demonstrate the superiority of our approach to previous studies across all three datasets.\footnote{The code and the best performing models are available at \url{https://github.com/cuhksz-nlp/SANER}.}
}

\end{abstract}

\section{Introduction}

\textcolor{black}{
The increasing popularity of microblogs results in a large amount of user-generated data, in which texts are usually short and informal.
How to effectively understand these texts remains a challenging task since the insights are hidden in unstructured forms of social media posts.
Thus, named entity recognition (NER) is a critical step for detecting proper entities in texts and providing support for downstream natural language processing (NLP) tasks \cite{DBLP:conf/aaai/PangLGXSC19,DBLP:conf/acl/MartinsMM19}.
}

\begin{figure}
    \centering
    \includegraphics[width=0.4\textwidth, trim=0 35 0 0]{figure/figure_example.pdf}
    \caption{\textcolor{black}{An example shows that an NE tagged with ``\textit{PER}'' (Person) is suggested by its similar words.}}
    \label{fig:example}
    \vskip -1.5em
\end{figure}


However, NER in social media remains a challenging task because (i) it suffers from the data sparsity problem since entities usually represent a small part of proper names, which makes the task hard to be generalized; (ii) social media texts do not follow strict syntactic rules \cite{DBLP:conf/emnlp/RitterCME11}.
To tackle these challenges,
previous studies tired to leverage domain information (e.g., existing gazetteer and embeddings trained on large social media text) and external features (e.g., part-of-speech tags) to help with social media NER \cite{DBLP:conf/emnlp/PengD15,DBLP:conf/aclnut/AguilarMLS17}.
However, these approaches rely on extra efforts to obtain such extra information and suffer from noise in the resulted information.
For example, training embeddings for social media domain could bring a lot unusual expressions to the vocabulary.
Inspired by studies using semantic augmentation (especially from lexical semantics) to improve model performance on many NLP tasks \cite{song-xia-2013-common,song-etal-2018-joint,kumar-etal-2019-closer,amjad-etal-2020-data}, it is also a potential promising solution to solving social media NER.
\textcolor{black}{
Figure \ref{fig:example} shows a typical case. ``\textit{Chris}'', supposedly tagged with ``\textit{Person}'' in this example sentence, is tagged as other labels in most cases. 
Therefore, in the predicting process, it is difficult to label ``\textit{Chris}'' correctly.
}
A sound solution is to augment the semantic space of ``\textit{Chris}'' through its similar words, such as ``\textit{Jason}'' and ``\textit{Mike}'', which can be obtained by existing pre-trained word embeddings from the general domain.




In this paper, we propose an effective approach to NER for social media texts with semantic augmentation. 
In doing so, we augment the semantic space for each token from pre-trained word embedding models, such as GloVe \cite{DBLP:conf/emnlp/PenningtonSM14} and Tencent Embedding \cite{DBLP:conf/naacl/SongSLZ18}, and encode semantic information through an 
attentive semantic
augmentation module.
Then we apply a gate module to weigh the contribution of the augmentation module and context encoding module in the NER process. 
To further improve NER performance, we also attempt multiple types of pre-trained word embeddings for feature extraction, which has been demonstrated to be effective in previous studies \cite{DBLP:conf/coling/AkbikBV18,DBLP:conf/emnlp/JieL19,DBLP:conf/naacl/KasaiFFRR19,DBLP:conf/aaai/KimKK19,DBLP:journals/corr/abs-1911-04474}.
To evaluate our approach, we conduct experiments on three benchmark datasets, where the results show that our model outperforms the state-of-the-arts with clear advantage across all datasets.







\section{The Proposed Model}

The task of social media NER is conventionally regarded as sequence labeling task, where an input sequence  with  tokens is annotated with its corresponding NE labels  in the same length.
Following this paradigm, we propose a neural model with semantic augmentation for the social media NER.
Figure \ref{fig:model} shows the architecture of our model, where the backbone model and the semantic augmentation module are illustrated in white and yellow backgrounds, respectively.
For each token in the input sentence, we firstly extract the most similar words of the token according to their pre-trained embeddings. 
Then, the augmentation module use an attention mechanism to weight the semantic information carried by the extracted words.
Afterwards, the weighted semantic information is leveraged to enhance the backbone model through a gate module.

In the following text, we firstly introduce the encoding procedure for augmenting semantic information.
Then, we present the gate module to incorporate augmented information into the backbone model.
Finally, we elaborate the tagging procedure for NER with the aforementioned enhancement. 






\begin{figure}
    \centering
    \includegraphics[width=0.48\textwidth, trim=0 20 0 0]
{figure/model_social.png}
    \caption{\textcolor{black}{
    The overall architecture of our proposed model with semantic augmentation. 
    An example sentence and its output NE labels are given, where the augmented semantic information for the word ``\textit{Chris}'' are also illustrated with the processing through the augmentation module and the gate module. 
    }}
    \label{fig:model}
    \vskip -1.5em
\end{figure}


\subsection{Attentive Semantic Augmentation}


The high quality of text representation is the key to obtain good model performance for many NLP tasks \cite{song2017learning,sileo-etal-2019-mining}.
However, obtaining such text representation is not easy in the social media domain because of data sparsity problem.
Motivated by this fact,
we propose semantic augmentation mechanism for social media NER by enhancing the representation of each token in the input sentence with the most similar words in their semantic space, which can be measured by pre-trained embeddings.


In doing so, 
for each token ,
we use pre-trained word embeddings (e.g., GloVe for English and Tencent Embedding for Chinese) to extract the top  words that are most similar to  based on cosine similarities and denote them as

Afterwards, we use another embedding matrix to map all extracted words  to their corresponding embeddings .
Since not all  are helpful for predicting the NE label of  in the given context, it is important to distinguish the contributions of different words to the NER task in that context.
Consider that the attention and weight based approaches are demonstrated to be effective choices to selectively leverage extra information in many tasks \cite{kumar-etal-2018-knowledge,margatina-etal-2019-attention,tian-etal-2020-joint,tian2020improving,tian-etal-2020-suppertagging,tian-etal-2020-constituency}, we propose an attentive semantic augmentation module (denoted as ) to weight the words according to their contributions to the task in different contexts.
Specifically, for each token , the augmentation module assigns a weight to each word  by

where  is the hidden vector for  obtained from the context encoder with its dimension matching that of the embedding (i.e., ) of . 
Then, we apply the weight  to the word  to compute the final augmented semantic representation by

where  is the derived output of , and contains the weighted semantic information.
Therefore, the augmentation module ensures that the augmented semantic information are weighted based on their contributions and important semantic information is distinguished accordingly.





\subsection{The Gate Module}


We observe that the contribution of the obtained augmented semantic information to the NER task could vary in different contexts and
a gate module (denoted by ) is naturally desired to weight such information in the varying contexts.
Therefore, to improve the capability of NER with the semantic information, we propose a gate module to aggregate such information to the backbone NER model.
Particularly,
we use a  gate to control the information flow by

where  and  are trainable matrices and  the corresponding bias term.
Afterwards, we use

to balance the information from context encoder and the augmentation module, where  is the derived output of the gate module;
 represents the element-wise multiplication operation and  is a 1-vector with its all elements equal to .


\subsection{Tagging Procedure}


\textcolor{black}{
To provide  to the augmentation module, we adopt a context encoding module (denoted as ) proposed by \citet{DBLP:journals/corr/abs-1911-04474}.
Compared with vanilla Transformers, this encoder additionally models the direction and distance information of the input, which has been demonstrated to be useful for the NER task. 
Therefore, the encoding procedure of the input text can be denoted as

where  and  are lists of hidden vectors and embeddings of , respectively. 
In addition, since pre-trained word embeddings contain substantial context information from large-scale corpus, and different types of them may contain diverse information, a straightforward way of incorporating them is to concatenate their embedding vectors by

where  is the final word embedding for  and  the set of all embedding types.
Afterwards, a trainable matrix  is used to map  obtained from the gate module to the output space by .
Finally, a conditional random field (CRF) decoder is applied to predict the labels  (where  is the set with all NE labels) in the output sequence  by

where  and  are the trainable parameters to model the transition for  to .
}

\section{Experiments}

\vspace{-0.2cm}

\begin{table}[t]
\scalebox{0.84}{
    \begin{tabular}{L{1.6cm}|L{1.5cm}L{1.0cm}|R{0.73cm}R{0.85cm}R{0.85cm}}
\toprule
    \textbf{Language}           & \textbf{Dataset}          &           & \textbf{Train} & \textbf{Dev} & \textbf{Test} \\
    \midrule
    \multirow{6}{*}{English}    & \multirow{3}{*}{W16}   & Sent. & 2,394 & 1,000 & 3,850 \\
                                &                           & Ent.  & 1,496 & 661   & 3,473 \\
                                &                           & Uns.  & -     & 52.1  & 80.0  \\
                                \cmidrule{2-6}
                                & \multirow{3}{*}{W17}   & Sent. & 3,394 & 1,008 & 1,287 \\
                                &                           & Ent.  & 1,975 & 835   & 1,079 \\
                                &                           & Uns.  & -     & 34.8  & 84.5   \\
    \midrule
    \multirow{3}{*}{Chinese}    & \multirow{3}{*}{WB}   & Sent. & 1,350 & 270 & 270 \\
                                &                           & Ent.  & 1,885 & 389   & 414 \\
                                &                           & Uns.  & -     & 51.4  & 45.2      \\
    \bottomrule
    \end{tabular}
    }
    \vspace{-0.2cm}
    \caption{The statistics of all benchmark datasets w.r.t. the number of sentences ( Sent.), named entities ( Ent.) and the percentage of unseen entities ( Uns.).
    }
    \label{tab:dataset}
    \vskip -1.5em
\end{table}


\subsection{Settings}


In our experiments, we use three social media benchmark datasets, including WNUT16 (W16) \cite{DBLP:conf/aclnut/StraussTRMX16}, WNUT17 (W17) \cite{DBLP:conf/aclnut/DerczynskiNEL17}, and Weibo (WB) \cite{DBLP:conf/emnlp/PengD15}, 
where W16 and W17 are English datasets constructed from Twitter, and WB is built from Chinese social media platform (Sina Weibo).
For all three datasets, we use their original splits and report the statistics of them in Table \ref{tab:dataset} (e.g., the number of sentences (Sent.), entities (Ent.), and the percentage of unseen entities (\%Uns.) with respect to the entities appearing in the training set).








\begin{table*}[t]
\begin{subtable}[t]{0.45\textwidth}
    \centering
    \small
\begin{tabular}{C{0.35cm}|C{0.5cm}C{0.5cm}|C{1.35cm}C{1.35cm}C{1.2cm}}
        \toprule
        \textbf{ID} & \textit{\textbf{SE}} & \textbf{\textit{GA}} & \textbf{W16} & \textbf{W17} & \textbf{WB} \\
        \midrule
        1 &   &           & 54.79            & 48.41            & 65.36 \\
        2 & &          & 55.03            & 48.36            & 65.01 \\
        3 & &          & 56.28            & 48.98            & 66.24 \\
        4 & &            & 56.86            & 49.26            & 68.21 \\
        5 & &            & \textbf{57.94}   & \textbf{50.02}   & \textbf{69.32} \\
        \bottomrule
    \end{tabular}
\caption{Development Set}
    \label{tab:dev}
\end{subtable}
\hspace{0.8cm}
\begin{subtable}[t]{0.45\textwidth}
    \centering
    \small
\begin{tabular}{C{0.35cm}|C{0.5cm}C{0.5cm}|C{1.35cm}C{1.35cm}C{1.2cm}}
        \toprule
        \textbf{ID} & \textit{\textbf{SE}} & \textbf{\textit{GA}} & \textbf{W16} & \textbf{W17} & \textbf{WB} \\
        \midrule
        1 &   &                   & 52.98        & 48.82        & 66.02 \\
        2 & &                  & 53.11        & 48.71        & 65.78 \\
        3 & &                  & 54.02        & 49.56        & 67.52 \\
        4 & &                  & 54.29        & 49.81        & 68.46 \\
        5 & &                  & \textbf{55.01}& \textbf{50.36}& \textbf{69.80} \\
        \bottomrule
    \end{tabular}
\caption{\textcolor{black}{Test Set}}
\label{tab:self}
\end{subtable}
\vspace{-0.2cm}
\caption{\textcolor{black}{ scores of the baseline model and ours enhanced with semantic augmentation (``'') and the gate module (``'') on the development (a) and test (b) sets. ``'' and ``'' represent the direct summation and attentive augmentation module, respectively.  and  denote the use and non-use of corresponding modules.}}
\label{tab: results}
\vskip -1.5em
\end{table*}


For model implementation, we follow \citet{DBLP:conf/naacl/LampleBSKD16} to use the BIOES tag schema to represent the NE labels of tokens in the input sentence.
For the text input, we try two types of embeddings for each language.\footnote{We report the results of using each individual type of embeddings in Appendix A.}
Specifically, for English, we use 
ELMo \cite{DBLP:conf/naacl/PetersNIGCLZ18} and BERT-cased large  \cite{DBLP:conf/naacl/DevlinCLT19};
for Chinese, we use 
Tencent Embedding \cite{DBLP:conf/naacl/SongSLZ18}, and ZEN \cite{DBLP:journals/corr/zen}.\footnote{
We obtain the pre-trained BERT model from \url{https://github.com/google-research/bert},
Tencent Embeddings from \url{https://ai.tencent.com/ailab/nlp/embedding.html},
and ZEN from \url{https://github.com/sinovation/ZEN}.
Note that we use ZEN because it achieves better performance than BERT on different Chinese NLP tasks.
For reference, we report the results of using BERT in Appendix B.}
In the context encoding module, we use a two-layer transformer-based encoder proposed by \citet{DBLP:journals/corr/abs-1911-04474} with 128 hidden units and 12 heads.
To extract similar words carrying augmented semantic information, we use the pre-trained word embeddings from GloVe for English and those embedding from Tencent Embeddings for Chinese to extract the most similar 10 words (i.e., ) \footnote{The results of using other embeddings as sources to extract similar words are reported in the Appendix C.}.
In the augmentation module, we randomly initialize the embeddings of the extracted words (i.e.,  for ) to represent the semantic information carried by those words.\footnote{We also try other ways (e.g., GloVe for English and Tencent Embedding for Chinese) to initialize the word embeddings, but do not find significant differences.}


During the training process, we fix all pre-trained embeddings in the embedding layer and use Adam \cite{DBLP:journals/corr/KingmaB14} to optimize negative log-likelihood loss function with the learning rate set to ,  and . 
We train 50 epochs for each method with the batch size set to  and tune the hyper-parameters on the development set\footnote{We report the details of hyperparameter settings of different models in the Appendix D.}.
The model that achieves the best performance on the development set is evaluated on the test set with the  scores obtained from the official conlleval toolkits\footnote{The script to evaluate all models in the experiments is obtained from \url{https://www.clips.uantwerpen.be/conll2000/chunking/conlleval.txt}.}.







\subsection{Overall Results}



To explore the effect of the proposed attentive semantic augmentation module () and the gate module (), we run different settings of our model with and without the modules.
In addition, we also try baselines that use direct summation () to leverage the semantic information carried by the similar words, where the embeddings of the words are directly summed without weighting through attentions.
The experimental results () of the baselines and our approach on the development and test sets of all datasets are reported in Table \ref{tab: results}(a) and (b), respectively.


There are some observations from the results on the development and test sets.
First, compared to the baseline without semantic augmentation (ID=1), models using direct summation (, ID=2) to incorporate different semantic information undermines NER performance on two of three datasets, namely, W17 and WB;
on the contrary, the models using the proposed attentive semantic augmentation module (, ID=4) consistently outperform the baselines (ID=1 and ID=2) on 
all datasets.
It indicates that  could distinguish the contributions of different semantic information carried by different words in the given context and leverage them accordingly to improve NER performance.
Second, comparing the results of models with and without the gate module () (i.e. ID=3 vs. ID=2 and ID=5 vs. ID=4), we find that the models with gate module achieves superior performance to the others without it.
This observation suggests that the importance of the information from the context encoder and  varies, and the proposed gate module is effective in adjusting the weights according to their contributions.





\textcolor{black}{
Moreover, we compare our model under the best setting with previous models on all three datasets in Table \ref{tab:comparison}, where our model outperforms others on all datasets.
We believe that the new state-of-the-art performance is established.
The reason could be that compared to previous studies, our model is effective to alleviate the data sparsity problem in social media NER with the augmentation module to encode augmented semantic information.
Besides, the gate module can distinguish the importance of information from the context encoder and  according to their contribution to NER. 
}



\begin{table}[t]
    \centering
\begin{small}
    \begin{tabular}{l | C{1.0cm}C{1.0cm}C{1.0cm}}
    \toprule
    \textbf{Model} & \textbf{W16}  & \textbf{W17} & \textbf{WB} \\
    \midrule
    \newcite{DBLP:conf/acl/ZhangY18}                    & -                 & -                     & 58.79         \\
    \newcite{DBLP:journals/corr/abs-1911-04474}& 54.06                 & 48.98              & 65.03         \\
    \newcite{DBLP:conf/naacl/ZhuW19}                    & -                 & -                     & 59.31         \\
    \newcite{DBLP:conf/ijcai/GuiM0ZJH19} & - & - & 59.92 \\
    \newcite{DBLP:conf/emnlp/SuiCLZL19}                 & -                 & -                     & 63.09         \\
    \newcite{DBLP:conf/naacl/AkbikBV19}                         & -                 & 49.59 & - \\
    \newcite{DBLP:conf/acl/ZhouZJZFGK19}                        & 53.43                 & 42.83        & - \\
    \newcite{DBLP:conf/naacl/DevlinCLT19}       & 54.36 & 49.52 & 67.33 \\
    \newcite{DBLP:conf/nips/MengWWLNYLHSL19}    & - & - & 67.60 \\
    \newcite{DBLP:conf/cikm/XuWHL19}         & - & - & 68.93 \\
    \midrule
    Ours   & \textbf{55.01}& \textbf{50.36}& \textbf{69.80} \\
    \bottomrule
    \end{tabular}
    \end{small}
\vspace{-0.2cm}
    \caption{\textcolor{black}{Comparison of  scores of our best performing model (the full model with augmentation module and gate module) with that reported in previous studies on all English and Chinese social media datasets. }}
    \label{tab:comparison}
    \vskip -1.0em
\end{table}

\section{Analysis}



\subsection{Performance on Unseen Named Entities}





Since this work focuses on addressing the data sparsity problem in social media NER, where the unseen NEs
are one of the important factors that hurts model performance.
To analyze whether our approach with attentive semantic augmentation () and the gate module () can address this problem, we report the recall of our approach (i.e., ``++'') to recognize the unseen NEs on the test set of all datasets in Table \ref{tab:unseen}.
For reference, we also report the recall of the baseline without  and , as well as our runs of previous studies (marked by ``'').
It is clearly observed that our approach outperforms the baseline and previous studies on unseen NEs on all datasets, which shows that it can appropriately leverage semantic information carried by similar words and thus alleviate the data sparsity problem. 





\subsection{Case Study}


To demonstrate how the augmented semantic information improves NER with the attentive augmentation module and the gate module, 
we show the extracted augmented information for the word ``\textit{Chris}'' and visualize the weights for each augmented term in
Figure \ref{fig:case}, where deeper color refers to higher weight.
In this case, the words ``\textit{steve}'' and ``\textit{jason}'' have higher weights in .
The explanation could be that in all cases, these two words are a kind of ``\textit{Person}''. 
Thus, higher attention to these terms helps our model to identify the correct NE label. 
On the contrary, the term ``\textit{anderson}'' and ``\textit{andrew}'' never occur in the dataset, and therefore provide no helpful effect in this case and eventually end with the lower weights in . 
In addition, a model can also mislabel ``\textit{Chris}'' as ``\textit{Music-Artist}'', because ``\textit{Chris}'' belongs to that NE type in most cases and there is a word ``\textit{filming}'' in its context.
However, our model with the gate module can distinguish that the information from semantic augmentation is more important and thus make correct prediction.




\begin{table}[t]
    \centering
\scalebox{0.9}{
\begin{tabular}{L{3.1cm}|C{1.25cm}C{1.25cm}C{1.1cm}}
        \toprule
        \textbf{Model} & \textbf{W16} & \textbf{W17} & \textbf{WB} \\
        \midrule
        \# of Unseen NEs & 2778 & 912 & 189 \\
        \midrule
        \citet{DBLP:conf/naacl/DevlinCLT19}       & 49.02            & 46.73            & 45.98 \\
        \citet{DBLP:journals/corr/abs-1911-04474} & 48.97            & 46.89            & 45.71 \\
        \midrule
Baseline                                        & 49.04            & 46.72            & 45.79 \\
        Ours (+ +)                              & \textbf{51.27}   & \textbf{49.45}   & \textbf{48.81} \\
        \bottomrule
    \end{tabular}
    }
\vspace{-0.2cm}
    \caption{
    The recall of our models with and without the attentive semantic augmentation () and the gate module () on unseen named entities (whose numbers are also reported at the first row) on all three datasets. 
    The results of our runs of previous models (marked with ``'') are also reported for references.
}
    \label{tab:unseen}
\end{table}



\begin{figure}[t]
    \centering
    \includegraphics[width=0.48\textwidth, trim=0 20 0 0]{figure/figure_visualization.pdf}
    \caption{\textcolor{black}{An example of helping recognize the NE ``\textit{Chris}'' by augmented semantic information (darker color refers to greater value). ``'' and ``'' represent the context encoder and attentive augmentation module, respectively. }}
    \label{fig:case}
    \vskip -1.0em
\end{figure}





\section{Conclusion}
\textcolor{black}{
In this paper, we proposed a neural-based approach to enhance social media NER with semantic augmentation to alleviate data sparsity problem. 
Particularly, an 
attentive semantic
augmentation module is suggested to encode semantic information and a gate module is applied to aggregate such information to tagging process. 
Experiments conducted on three benchmark datasets in English and Chinese show that our model outperforms previous studies and achieves the new state-of-the-art result. 
}




\bibliography{emnlp2020}
\bibliographystyle{acl_natbib}








\vspace{0.4cm}






\appendix



\section*{Appendix A: Effect of Using Different Embeddings} \label{app: embedding}





\vspace{-0.2cm}

\begin{table}[h]
    \centering
    \small
\begin{tabular}{l L{0.9cm} | C{1.1cm}C{1.1cm}C{0.9cm}}
        \toprule
        \textbf{Model} & \textbf{Emb.} & \textbf{W16} & \textbf{W17} & \textbf{WB} \\
        \midrule
        Baseline            & \multirow{2}{*}{ELMo} & 52.16             & 47.31             & -             \\
        \ \ ++   &                          & 54.31             & 48.76             & -             \\
        \cmidrule{1-5}
        Baseline            & \multirow{2}{*}{BERT} & 52.09             & 48.33             & -             \\
        \ \ ++   &                          & \textbf{54.16}    & \textbf{49.57}    & -             \\
        \midrule
        Baseline            & \multirow{2}{*}{Tencent}& -                 & -               & 60.54              \\
        \ \ ++   &                          & -                 & -                 & 63.12         \\
        \cmidrule{1-5}
        Baseline            & \multirow{2}{*}{ZEN}   & -                 & -                 & 66.09              \\
        \ \ ++   &                          & -                 & -                 & \textbf{68.96}\\
        \bottomrule
    \end{tabular}
\vspace{-0.2cm}
    \caption{
Experimental results ( scores) of our approach with semantic augmentation () and gate module () on all datasets, where only one type of embeddings is used in the embedding layer to represent the input sentence.
The results of their corresponding baseline without  and  are also reported.
}
    \label{tab:embedding}
    \vskip -1.0em
\end{table}








In our main experiments, we use two types of embeddings for each language: ELMo \cite{DBLP:conf/naacl/PetersNIGCLZ18} and BERT-cased large  \cite{DBLP:conf/naacl/DevlinCLT19} for English, and Tencent Embedding \cite{DBLP:conf/naacl/SongSLZ18} and ZEN \cite{DBLP:journals/corr/zen} for Chinese.
In Table \ref{tab:embedding}, we report the results ( scores) of our model with the best setting (i.e. the full model with semantic augmentation () and gate module ()) as well as the baselines without  and , where either one of the two types of embedding is used to represent the input sentence.
From the results, it is found that our model with  and  can consistently outperforms the baseline models with different settings of embeddings.



\section*{Appendix B: Comparison Between BERT and ZEN on WB}

\vspace{-0.2cm}
\begin{table}[h]
    \centering
    \begin{tabular}{L{3.8cm} | C{2.4cm}}
        \toprule
        \textbf{Embeddings} & \textbf{WB}\\
        \midrule
        BERT + Tencent      & 69.56         \\
        ZEN + Tencent       & \textbf{69.80}\\
        \bottomrule
    \end{tabular}
    \vspace{-0.2cm}
    \caption{Experimental results ( scores) of our model with  and  on the WB dataset, where BERT or ZEN is used as one of the two types of embeddings (the other one is Tencent Embedding) to represent the input sentence for the embedding layer.}
    \label{tab:bert}
    \vskip -1em
\end{table}

\noindent
In our main experiments, we use ZEN \cite{DBLP:journals/corr/zen} instead of BERT \cite{DBLP:conf/naacl/DevlinCLT19} as the embedding to represent the input for Chinese.
The reason is that ZEN achieves better performance compared with BERT, which is confirmed by
Table \ref{tab:bert} with its results ( scores) showing the performance of our approach with the best settings (i.e. two types of embeddings with  and .) on WB dataset. Either BERT or ZEN is used as one of the two types of embeddings (the other type of embedding is Tencent Embedding).





\section*{Appendix C: Effect of Using Different Embeddings to Extract Similar Words} \label{app: source}

\vspace{-0.2cm}
\begin{table}[h]
    \centering
    \small
\begin{tabular}{l L{1.5cm} | C{0.8cm}C{0.8cm}C{0.8cm}}
        \toprule
        \textbf{Model} & \textbf{Source} & \textbf{W16} & \textbf{W17} & \textbf{WB} \\
        \midrule
        \multicolumn{2}{l|}{Baseline} & 49.56             & 49.11             & 66.02             \\
        \midrule
           & Word2vec  & 54.94             & 50.22             & -             \\
        Ours  & GloVe     & \textbf{55.01}    & \textbf{50.36}    & -             \\
        \addlinespace[0.05cm]
        \cline{2-5}
        \addlinespace[0.05cm]
        (++)   & Giga      & -                 & -                 & 69.68         \\
           & Tencent   & -                 & -                 & \textbf{69.80}\\
        \bottomrule
    \end{tabular}
\vspace{-0.3cm}
    \caption{
Experimental results ( scores) of our best performing models (i.e., the ones with  and ) using different types of pre-trained embeddings as the source to extract similar words.
The results of baseline (the one without  and ) are also reported.}
\label{tab:source}
    \vskip -1em
\end{table}

In addition to use embeddings for input sentence representation,
we also try different embedding sources (i.e. pre-trained word embeddings) to extract similar words for each token in the input sentence.
For English, we use Word2vec \cite{DBLP:journals/corr/word2vec} and Glove \cite{DBLP:conf/acl/ManningSBFBM14}; for Chinese, we use Giga \cite{DBLP:conf/acl/ZhangY18} and Tencent Embedding \cite{DBLP:conf/naacl/SongSLZ18}.\footnote{We obtain Word2vec from \url{https://code.google.com/archive/p/word2vec/}, GloVe from \url{https://nlp.stanford.edu/projects/glove/}, Giga from \url{https://github.com/jiesutd/LatticeLSTM}.}
The experimental results of our model with the best setting (i.e., the one with  and ) using different sources are reported in Table \ref{tab:source}.
The result of the baseline model without  and  is also reported for reference.
The results show that our approach can consistently outperforms the baseline with different sources to find similar words, which demonstrates the robustness of our approach.






\section*{Appendix D: Hyper-parameter Settings}
\label{app: hyper-para}

\vspace{-0.2cm}
\begin{table}[th]
    \centering
    \small
\begin{tabular}{l | l | r}
        \toprule
& \textbf{Values} & \textbf{Best} \\
        \midrule
        Dropout rate            & , , ,       &  \\
        Learning rate           & , ,   &   \\
Batch size              & , ,                &  \\
        Number of layers        & , ,                  &  \\
        Number of head          & , ,                 &  \\
        Hidden units            & , ,             &  \\
        \# of similar of words ()                  &  , ,               &  \\  
        \bottomrule
    \end{tabular}
\vspace{-0.2cm}
    \caption{
    All values of different hyper-parameters as well as the best one used in our experiments.
    }
    \label{tab:hyper}
    \vskip -1em
\end{table}

\noindent
We report all values of the hyper-parameters tried for our models in Table \ref{tab:hyper},
where
we try different combinations of them and find the best hyper-parameter configurations (which is also reported in Table \ref{tab:hyper}) on the development set of each dataset.




\end{document}