







\documentclass[11pt]{article}
\usepackage{fullpage}

\usepackage{graphicx}
\usepackage{algorithmic}
\usepackage{algorithm}
\usepackage{amssymb}
\usepackage{amsmath}
\usepackage{amsthm}

\usepackage[linktocpage=true,pagebackref=true]{hyperref}


\newtheorem{theorem}{Theorem}
\newtheorem{lemma}[theorem]{Lemma}
\newtheorem{claim}[theorem]{Claim}
\newtheorem{proposition}[theorem]{Proposition}
\newtheorem{corollary}[theorem]{Corollary}
\newtheorem{fact}[theorem]{Fact}
\newtheorem{example}[theorem]{Example}
\newtheorem{notation}[theorem]{Notation}
\newtheorem{observation}[theorem]{Observation}
\newtheorem{conjecture}[theorem]{Conjecture}

\theoremstyle{definition}
\newtheorem{definition}[theorem]{Definition}

\theoremstyle{remark}
\newtheorem{remark}[theorem]{Remark}



\renewcommand{\Pr}{\mathrm{\sc Pr}}
\newcommand{\E}{\mathbb{E}}

\newcommand{\OPT}{\mathsf{OPT}}
\newcommand{\opt}{\mathsf{opt}}
\newcommand{\x}{{\bf x}}
\newcommand{\y}{{\bf y}}
\newcommand{\f}{{\bf f}}
\newcommand{\e}{{\bf e}}
\newcommand{\p}{{\bf p}}
\newcommand{\Int}{\mathbb{Z}}
\newcommand{\Real}{\mathbb{R}}
\newcommand{\poly}{\mathrm{poly}}
\newcommand{\calG}{\mathcal{G}}
\newcommand{\calV}{\mathcal{V}}
\newcommand{\calE}{\mathcal{E}}
\newcommand{\calF}{\mathcal{F}}
\newcommand{\calT}{\mathcal{T}}
\newcommand{\calH}{\mathcal{H}}
\newcommand{\calQ}{\mathcal{Q}}
\newcommand{\calR}{\mathcal{R}}
\newcommand{\calZ}{\mathcal{Z}}
\newcommand{\hatG}{\widehat{G}}
\newcommand{\hatE}{\widehat{E}}
\newcommand{\hatV}{\widehat{V}}
\newcommand{\cost}{\mathsf{cost}}
\newcommand{\len}{\mathsf{length}}
\newcommand{\calP}{\mathcal{P}}
\newcommand{\boldP}{\mathbb{P}}

\renewcommand{\deg}{\mathsf{deg}}
\newcommand{\indeg}{\mathsf{indeg}}
\newcommand{\outdeg}{\mathsf{outdeg}}

\renewcommand{\setminus}{-}

\newcommand{\NP}{\mathrm{NP}}
\newcommand{\ZPP}{\mathrm{ZPP}}
\newcommand{\polylog}{\mathrm{polylog}}

\pagestyle{plain}

\begin{document}

\title{Approximating Directed Steiner Problems via Tree Embedding}

\author{Bundit Laekhanukit\thanks{
The Weizmann Institute of Science, Israel,
email:\url{bundit.laekhanukit@weizmann.ac.il}.
}.\footnote{
The work was partly done while the author was at 
McGill University, Simons Institute for the Theory of Computing
and the Swiss AI Lab IDSIA.
Partially supported by the ERC Starting Grant NEWNET 279352 and
by Swiss National Science Foundation project 200020\_144491/1.
}}






















\maketitle


\begin{abstract}
Directed Steiner problems are fundamental problems in 
Combinatorial Optimization and Theoretical Computer Science.
An important problem in this genre is
the {\em -edge connected directed Steiner tree} (-DST) problem. 
In this problem, we are given a directed graph  on  vertices
with edge-costs, a root vertex , a set of  terminals  
and an integer .
The goal is to find a min-cost subgraph 
that connects  to each terminal  by
 edge-disjoint -paths.
This problem includes as special cases the well-known 
{\em directed Steiner tree} (DST) problem (the case )
and
the {\em group Steiner tree} (GST) problem. 
Despite having been studied and mentioned many times in literature, 
e.g., by Feldman~et~al. [SODA'09, JCSS'12], 
by Cheriyan~et~al. [SODA'12, TALG'14]
and by Laekhanukit [SODA'14],
there was no known non-trivial approximation algorithm for -DST
for  even in the special case that an input graph is 
directed acyclic and has a constant number of layers.
If an input graph is not acyclic, 
the complexity status of -DST is not known even for 
a very strict special case that  and .

In this paper, we make a progress toward developing a non-trivial
approximation algorithm for -DST.
We present an -approximation 
algorithm for -DST on directed acyclic graphs (DAGs) with 
layers, which can be extended to a special case of -DST 
on ``general graphs'' when an instance has 
a {\em -shallow} optimal solution, i.e., 
there exist  edge-disjoint -paths, each of length at most ,
for every terminal .
For the case  (DST), our algorithm yields an approximation ratio
of , thus implying an -approximation
algorithm for DST that runs in quasi-polynomial-time
(due to the height-reduction of Zelikovsky [Algorithmica'97]).
Our algorithm is based on an LP-formulation that allows us to embed a
solution to a tree-instance of GST, 
which does not preserve connectivity. 
We show, however, that one can randomly extract a solution of -DST
from the tree-instance of GST.

Our algorithm is almost tight when  are constants since the
case that  and  is NP-hard to approximate to within a factor
of , and our algorithm archives the same approximation 
ratio for this special case. 
We also remark that the -hardness instance of
-DST is a DAG with  layers, and our algorithm gives 
-approximation for this special case.
Consequently, as our algorithm works for general graphs,
we obtain an -approximation
algorithm for a -shallow instance of 
the {\em  edge-connected directed Steiner subgraph} problem, where
we wish to connect every pair of terminals by  edge-disjoint paths.
\end{abstract}


\section{Introduction}
\label{sec:intro}

Network design is an important class of problems in Combinatorial
Optimization and Theoretical Computer Science as 
it formulates scenarios that appear in practical settings. 
In particular, we might wish to design an overlay network that
connects a server to clients, and this can be formulated as 
the {\em Steiner tree} problem.
In a more general setting, we might have an additional constraint
that the network must be able to function after 
link or node failures, leading to 
the formulation of the {\em survivable network design} problem. 
These problems are well-studied in symmetric case
where a network can be represented by an undirected graph. 
However, in many practical settings,
links in networks are not symmetric.
For example, we might have different upload and download bandwidths in
each connection, and sometimes, transmissions are only allowed in one
direction.
This motivates the study of network design problems in directed
graphs, in particular, {\em directed Steiner} problems.

One of the most well-known directed network design problem 
is the {\em directed Steiner tree} problem (DST), which asks to find a
minimum-cost subgraph that connects a given root vertex to each
terminal.
DST is a notorious problem as there is no known polynomial-time
algorithm that gives an approximation ratio better than polynomial. 
A polylogarithmic approximation can be obtained only when an algorithm
is allowed to run in quasi-polynomial-time
\cite{CharikarCCDGGL99,Rothvoss11,FriggstadKKLST14}.
A natural generalization of DST, namely, 
the {\em  edge-connected directed Steiner tree} (-DST) problem,
where we wish to connect a root vertex to each terminal by 
edge-disjoint paths, is even more mysterious as there is no known
non-trivial approximation algorithm, despite having been studied and
mentioned many times in literature, e.g., 
by Feldman~et~al.~\cite{FeldmanKN12}, 
by Cheriyan~et~al.~\cite{CheriyanLNV14}
and by Laekhanukit~\cite{Laekhanukit14}. 


The focus of this paper is in studying the approximability of -DST.
Let us formally describe -DST. 
In -DST, we are given a directed graph  with edge-costs
, a root vertex  and 
a set of terminals .
The goal is to find a min-cost subgraph  such that
 has a  edge-disjoint directed -paths from the root  to
each terminal . 
Thus, removing any  edges from  leaves at least one path from
the root  to each terminal ,
and DST is the case when  (i.e., we need only one path).
The complexity status of -DST tends to be negative. 
It was shown by Cheriyan~et~al. \cite{CheriyanLNV14} that
the problem is at least as hard as the {\em label cover} problem.
Specifically, -DST admits no
-approximation, for any ,
unless .
Laekhanukit~\cite{Laekhanukit14}, subsequently, showed that
-DST admits no -approximation 
unless .
The integrality gap of a natural LP-relaxation for -DST is
 which holds even for a special case of 
{\em connectivity-augmentation} where we wish to increase 
a connectivity of a graph by one.
All the lower bound results are based on the same construction
which are directed acyclic graphs (DAGs) with diameter , i.e.,
any path in an input graph has length (number of edges) at most 
(we may also say that it has {\em  layers}).
Even for a very simple variant of -DST, namely -DST, 
where we have two terminals,
one terminal requires one path from the root and 
another terminal requires  edge-disjoint paths, 
it was not known whether the problem is NP-hard or 
polynomial-time solvable.
To date, the only known positive result for -DST is 
an -time (exact) algorithm  for -DST on DAGs
\cite{CheriyanLNV14}, which thus runs in polynomial-time when  is  
constant,
and a folk-lore -approximation algorithm, which can
be obtained by computing min-cost -flow for  times, 
one from the root  to each terminal 
and then returning the union as a solution.
We emphasize that there was no known non-trivial approximation
algorithm even when an input graph is ``directed acyclic'' and 
has ``constant number of layers''.
Also, in contrast to DST, in which an -time
(exact) algorithm exists for general graphs,
it is not known whether -DST for  and  is
polynomial-time solvable if an input graph is not acyclic.
This leaves challenging questions whether ones can design a non-trivial 
approximation algorithm for -DST on DAGs with at most  layers,
and whether ones can design a non-trivial approximation algorithm
when an input graph is not acyclic.

In this paper, we make a progress toward developing a non-trivial
approximation algorithm for -DST.
We present the first ``non-trivial'' approximation algorithm for
-DST on DAGs with  layers that achieves an approximation ratio
of .
Our algorithm can be extended to a special case of -DST 
on ``general graphs'' where an instance has 
a {\em -shallow} optimal solution, i.e., 
there exist  edge-disjoint -paths, each of length at most ,
for every terminal .
Consequently, as our algorithm works for a general graph,
we obtain an -approximation
algorithm for a -shallow instance of 
the {\em  edge-connected directed Steiner subgraph} problem, where  
we wish to connect every pair of terminals by  edge-disjoint paths,
i.e., the set of terminal  is required to be -edge connected in
the solution subgraph (there is no root vertex in this problem).


Our algorithm is almost tight when  and  are constants
because the case that  and  is essentially 
the {\em set cover} problem, which is NP-hard to approximate to within
a factor of  \cite{LundY94,Feige98},
and our algorithm achieves the same approximation ratio.
We also remark that the -hardness instance of
-DST is a DAG with  layers, and our algorithm gives 
-approximation for this special case.
For , we obtain a slightly better bound of ,
thus giving an LP-based -approximation algorithm for DST
as a by product. 

The key idea of our algorithm is to formulate an LP-relaxation
with a special property that a fractional solution
can be embedded into a tree instance of 
the {\em group Steiner tree} problem (GST).
Thus, we can apply the GKR Rounding algorithm in \cite{GargKR00}
for GST on trees to round the fractional solution.
However, embedding of an LP-solution to a tree instance of GST does
not preserve connectivity.
Also, it does not lead to a reduction from -DST to 
the  edge-connected variant of GST, namely, -GST. 
Hence, our algorithm is, although simple, not straightforward. 


\subsection{Our Results}

Our main result is an -approximation
algorithm for -DST on a -shallow instance,
which includes a special case that an input graph 
is directed acyclic and has at most  layers. 

\begin{theorem}
\label{thm:approx-k-dst}
Consider the  edge-connected directed Steiner tree problem.
Suppose an input instance has an optimal solution  in which,
for every terminal , 
 has  edge-disjoint -paths 
such that each path has length at most . 
Then there exists an -approximation
algorithm.
In particular, there is an 
-approximation
algorithm for -DST on a directed acyclic graph with  layers.
\end{theorem}

For the case , our algorithm yields a slightly better guarantee
of . 
Thus, we have as by product an LP-based approximation algorithm 
for DST.
Applying Zelikovsky's height-reduction theorem
\cite{Zelikovsky97,HelvigRZ01},
this implies an LP-based quasi-polynomial-time
-approximation algorithm for DST.
(The algorithm runs in time  and has approximation ratio
.)

Theorem~\ref{thm:approx-k-dst} also implies an algorithm of the same 
(asymptotic) approximation ratio for a -shallow instance of 
the  edge-connected directed Steiner subgraph problem, where we
wish to find a subgraph  such that the set of terminal  is
-edge-connected in . 
To see this, we invoke the algorithm in
Theorem~\ref{thm:approx-k-dst} as follows.
Take any terminal  as a root vertex of a -DST instance.
Then we apply the algorithm for -DST to find a subgraph 
such that every terminal is  edge-connected from .
We apply the algorithm again to find a subgraph  
such that every terminal is  edge-connected to .
Then the set of terminal  is -edge connected in the graph
 by transitivity of edge-connectivity.
The cost incurred by this algorithm is at most twice that of 
the algorithm in Theorem~\ref{thm:approx-k-dst}.
Thus, we have the following theorem as a corollary of 
Theorem~\ref{thm:approx-k-dst}

\begin{theorem}
\label{thm:approx-k-conn-subgraph}
\label{thm:approx-k-dst}
Consider the  edge-connected directed Steiner subgraph problem.
Suppose an input instance has an optimal solution  in which,
for every pair of terminals , 
 has  edge-disjoint -paths 
such that each path has length at most . 
Then there exists an -approximation
algorithm.
\end{theorem}

\paragraph{Overview of our algorithm}

The key idea of our algorithm is to embed an LP solution 
for -DST to a standard LP of GST on a tree.
(We emphasize that we embed the LP solution of -DST to that of GST
not -GST.) 
At first glance, a reduction from -DST to GST on trees
is unlikely to exist because any such reduction 
would destroy all the connectivity information.
We show, however, that such tree-embedding exists,
but we have to sacrifice running-time and cost to obtain such
embedding.

The reduction is indeed the same as a folk-lore reduction from DST
to GST on trees. 
That is, we list all rooted-paths (paths that start from the root
vertex) of length at most  in an input graph and form a suffix
tree. 
In the case of DST, if there is an optimal solution which is a tree of
height , then it gives an approximation preserving reduction from
GST to DST which blows up the size (and thus the running time) of the 
instance to .  
Unfortunately, for the case of -DST with , 
this reduction does not give an equivalent reduction 
from -DST to -GST on trees. 
The reduction is valid in one direction, i.e., any feasible solution
to -DST has a corresponding feasible solution to the tree-instance
of -GST. 
However, the converse is not true as a feasible solution to the
tree-instance of -GST might not give a feasible solution to -DST.
Thus, our reduction is indeed an ``invalid'' reduction from -DST to
a tree instance of ``GST'' (the case ).

To circumvent this problem, we formulate an LP that provides a
connection between an LP solution on an input -DST instance 
and an LP solution of a tree-instance of GST. 
Thus, we can embed an LP solution to an LP-solution of GST on a 
(very large) tree.
We then round the LP solution using the GKR Rounding algorithm for GST on
trees \cite{GargKR00}.
This algorithm, again, does not give a feasible solution to -DST as
each integral solution we obtain only has ``connectivity one'' 
and thus is only feasible to DST.
We cope with this issue by using a technique 
developed by Chalermsook~et~al. in \cite{ChalermsookGL15}.
Specifically, we sample a sufficiently large number of
DST solutions and show that the union of all these solutions
is feasible to -DST using cut-arguments.


Each step of our algorithm and the proofs are mostly standard,
but ones need to be careful in combining each step.
Otherwise, the resulting graph would not be feasible to -DST.

\medskip\noindent{\bf Organization.}
We provide definitions and notations in Section~\ref{sec:prelim}. 
We start our discussion by presenting a reduction from DST to GST
in Section~\ref{sec:DST-to-GST}.
Then we discuss properties of minimal solutions for -DST
in Section~\ref{sec:props-min-solution}.
We present standard LPs for -DST and GST in
Section~\ref{sec:standard-LPs}
and formulate a stronger LP-relaxation for -DST in
Section~\ref{sec:strong-LP-k-DST}.
Then we proceed to present our algorithm in
Section~\ref{sec:algo-kDST}. 
Finally, we provide some discussions in Section~\ref{sec:conclusion}.

\section{Preliminaries}
\label{sec:prelim}

We use standard graph terminologies. 
We refer to a directed edge , shortly, by 
(i.e.,  and  are head and tail of , respectively),
and we refer to an undirected edge by .
For a (directed or undirected) graph , we denote by  and 
the sets of vertices and edges of , respectively.
If a graph  is associated with edge-costs , 
then we denote the cost of any subgraph  by
.
For any path , 
we use {\em length} to mean the number of edges in a path  and
use {\em cost} to mean the total costs of edges in .


In the {\em directed Steiner tree} problem (DST), 
we are given a directed graph  with edge-costs , 
a root vertex  and a set of terminals .
The goal is to find a min-cost subgraph  such that
 has a directed path from the root  to each terminal .
A generalization of DST is the 
{\em  edge-connected directed Steiner tree} problem (-DST).
In -DST, we are given the same input as in DST plus an integer . 
The goal is to find a min-cost subgraph  that has  edge-disjoint
paths from the root  to each terminal .
The {\em  edge-connected directed Steiner subgraph} problem
is a variant of -DST, where there is no root vertex, and
the goal is to find a min-cost subgraph  such that
the set of terminals  is  edge-connected in . 

The problems on undirected graphs that are closely related to 
of DST and -DST are the 
{\em group Steiner tree} problem (GST) and 
the {\em  edge-connected group Steiner tree} problem (-GST).
In GST, we are given an undirected graph  with edge-costs
, a root vertex  and a collection of subset of
vertices  called groups. The goal is to find a
a min-cost subgraph  that connects  to each group
. In -GST, the input consists of an additional integer ,
and the goal is to find a min-cost subgraph  with  edge-disjoint
-paths for every group .

Consider an instance of DST (resp., -DST).
We denote by  the set of all paths in  that start from the root .
The set of paths in  that end with a particular pattern, say
, is denoted by . This pattern 
can be a vertex , an edge  or a path  in
. For example,  consists of paths  of the form
. We say that a path  ends at a vertex 
(resp., an edge ) if  (resp., ) is 
the last vertex (resp., edge) of .

We may consider only paths with particular length, say .
We denote by  the set of paths that start at  and has length
at most . The notation for  is analogous to , e.g.,
 is the set of paths in  that end at an
edge .
A concatenation of a path  with an edge  or a vertex 
are denoted by  and , respectively.
For example, .

Given a subset of vertices , 
the set of edges entering  is denoted by 

The indegree of  is denoted by .
Analogously, we use  and  for
the set of edges leaving .
For undirected graphs, we simply use the notations  and .

We say that a feasible solution  to -DST is {\em -shallow}
if, for every terminal , there exists a set of 
 edge-disjoint -paths in  such that every path has length at most
.
An instance of -DST that has an optimal -shallow solution
is called a -shallow instance.
We also use the term -shallow analogously for -GST
and the  edge-connected Steiner subgraph problem.




To distinguish between the input of -DST (which is a directed graph)
and -GST (which is an undirected graph), we use script fonts, 
e.g., , to denote the input of -GST.
Also, we use  to denote the set of all paths from the root 
to any vertex  in the graph . 
The cost of a set of edges  (or a graph) is defined by a function 
. 
At each point, we consider only one instance of -DST 
(respectively, -GST).
So, we denote the cost of the optimal solution to -DST by 
 (respectively, ).

\section{Reduction from Directed Steiner Tree to Group Steiner Tree}
\label{sec:DST-to-GST}

In this section, we describe a reduction  from DST to GST.
We recall that  denotes all the -paths in a DST instance . 
The reduction is by simply listing paths in the directed graph 
as vertices in a tree  and
joining each path  to  if  is a path in . 
In fact,  is a {\em suffix tree} of paths in .
To be precise,


We set the cost of edges of  to be .
Since the root  has no incoming edges in ,
 maps to a unique vertex , and we define  as
the root vertex of the GST instance.
We will abuse  to mean both the root of DST and 
its corresponding vertex of GST. 
For each terminal , define a group of the GST instance as


It is easy to see that the reduction  produces a tree,
and there is a one-to-one mapping between 
a path in the tree  and a path in
the original graph .
Thus, any tree in  corresponds to a subtree of  
(but not vice versa),
which implies that the reduction  is approximation-preserving 
(i.e., ). 
Note, however, that the size of the instance blows up 
from  to , 
where  is the length of the longest path in .
The reduction holds for general graphs, but
it is approximation-preserving only if the DST instance
is -shallow, i.e., it has an optimal solution  
such that any -path in  has length at most , 
for all terminals .
However, Zelikovsky~\cite{Zelikovsky97,HelvigRZ01} showed
that the {\em metric completion} of  always contains a -shallow 
solution with cost at most  of 
an optimal solution to DST. 
(This is now known as Zelikovsky's height reduction theorem.)
Thus, we may list only paths of length at most  
from the metric completion.
We denote the reduction that lists only paths of length at most  by
.






\section{Properties of Minimal Solutions to -DST}
\label{sec:props-min-solution}

In this section, we provide structural lemmas which are building
blocks in formulating a strong LP-relaxation for -DST. 
These lemmas characterize properties of 
a minimal solution to -DST.

\begin{lemma}
\label{lem:rv-paths-in-minimal-kDST}
Let  be any minimal solution to -DST.
Then  has at most  edge-disjoint -paths,
for any vertex .
\end{lemma}
\begin{proof}
Suppose to a contrary that  has  edge-disjoint -paths,
for some vertex .
Then  must have indegree at least  in .
We take one of the  edges entering , namely, .
By minimality of , removing  results in a graph 
 that has less than 
 edge-disjoint -paths for some terminal .
Thus, by Menger's theorem, 
there must be a subset of vertices 
such that ,  and 
. 
Observe that we must have  in  because
 is a feasible solution to -DST, 
which means that .
Since we remove only one edge  from , 
the graph  must have  edge-disjoint -paths. 
But, this implies that ,
a contradiction.
\end{proof}

\begin{lemma}
\label{lem:paths-eq-indeg}
Let  be any minimal solution to -DST.
Any vertex  has indegree exactly , 
where  is the maximum number of 
edge-disjoint -paths in .
\end{lemma}

\begin{proof}
The proof follows a standard uncrossing argument.
Assume a contradiction that 
 has indegree at least  in .
By Menger's theorem, there is a subset of 
vertices  such that 
,  and  
that separates  from .
We assume that  is a minimum such set. 
Since , there is an edge
 that is not contained in ,
i.e., . 

By minimality of , removing  results in 
the graph  such that  has less than
 edge-disjoint -path for some terminal .
Thus, by Menger's theorem, 
there is a subset of vertices 
such that , ,  
and .
(The latter is because  is a feasible solution to -DST.)

Now we apply an uncrossing argument to  and .
By submodularity of , we have

Observe that , 
and . 
So, by the edge-connectivity of  and ,  

The sum of the left-hand side of Eq~\eqref{eq:submod} is 

So, we conclude that 

Consequently, we have the set  such that 
,  and  that
separates  from .
Since , we know that  is strictly smaller than .
This contradicts to the minimality of .
\end{proof}

The following is a corollary of Lemma~\ref{lem:rv-paths-in-minimal-kDST}
and Lemma~\ref{lem:paths-eq-indeg}
\begin{corollary}
\label{cor:kdst-maxdeg-k}
Let  be a minimal solution to -DST.
Then any vertex  has indegree at most .
\end{corollary}

The next lemma follows from Corollary~\ref{cor:kdst-maxdeg-k}.

\begin{lemma}
\label{lem:no-of-paths-k-DST}
Consider any minimal solution  to -DST
(which is a simple graph).
For any edge  and , 
there are at most  paths in  with length at most 
that start at the root  and ends at .
That is, 
 for all ,
where  is 
the set of -paths of length  in . 
\end{lemma}
\begin{proof}
We prove by induction.
The base case  is trivial because any rooted path of
length at most  cannot have a common edge.

Assume, inductively, that  
for some . 
Consider any edge .
By Corollary~\ref{cor:kdst-maxdeg-k},  has indegree at most .
Thus, there are at most  edges entering , namely,
, where . 
By the induction hypothesis, each edge is the last edge of at most
 paths in . 
Thus, we have at most  paths that
end at . That is,

\end{proof}




\section{Standard LPs for -DST and GST}
\label{sec:standard-LPs}

In this section, we describe standard LPs for -DST and GST. 
Each LP consists of two sets of variables, 
a variable  on each edge  and 
a variable  on each path  and a terminal .
The variable  indicates whether we choose an edge 
in a solution.
The variable  is a flow-variable on each path and 
thus can be written in a compact form 
using a standard flow formulation.



The standard LP for GST is similar to LP-k-DST.



\section{A Strong LP-relaxation for for -DST}
\label{sec:strong-LP-k-DST}

In this section, we formulate a strong LP-relaxation for -DST that
allows us to embed a fractional solution into an LP solution of LP-GST
on a tree. 



For -shallow instances of -DST, we replace  by 
to restrict length of paths to be at most . 
The next lemma shows that LP-k-DST* is an LP-relaxation for -DST.

\begin{lemma} \label{lem:valid-LP-k-DST*}
LP-k-DST* is an LP-relaxation for -DST.
Moreover, replacing  by  gives an LP-relaxation for -DST on
-shallow instances.
\end{lemma}

\begin{proof}
LP-k-DST* is, in fact, obtained from LP-k-DST (which is a standard LP)
by adding a new variable  and two constraints.
\begin{itemize}
\item[(1)] {\bf Subflow-Capacity:}
  .
\item[(2)]{\bf Aggregating -Flow:}
  .
\end{itemize}

To show that these two constraints are valid for -DST,
we take a minimal feasible (-shallow) solution
 of -DST.
We define a solution  to LP-k-DST as below.

By construction,  implies that .
Thus,  satisfies the Subflow-Capacity constraint.
By minimality of , Corollary~\ref{lem:no-of-paths-k-DST} 
implies that even if we list all the paths of length  in
, at most  of them end at the same edge, and we know
that rooted paths of length one share no edge 
(given that  is a simple graph).
Thus,  satisfies the Aggregating -Flow constraint.
Consequently, these two constraints are valid for -DST.

On the other hand, any integral solution that is not feasible to
-DST could not satisfy the constraints of LP-k-DST* simply
because LP-k-DST* contains the constraints of LP-k-DST, which is a
standard LP for -DST.
Thus, LP-k-DST* is an LP-relaxation for -DST.

The proof for the case of -shallow instances is the same as above  
except that we take  as a minimal -shallow solution
and replace  by . 
\end{proof}


\section{An Approximation Algorithm for -DST}
\label{sec:algo-kDST}

In this section, we present an approximation algorithm
for -DST on a -shallow instance. 
Our algorithm is simple.
We solve LP-k-DST* on an input graph  
and then embed an optimal fractional solution  to 
an LP-solution  of LP-GST on the tree . 
We lose a factor of  in this process.
As we now have a tree-embedding of an LP-solution,
we can invoke the GKR Rounding algorithm \cite{GargKR00} to round
an LP-solution on the tree .
Our embedding guarantees that any edge-set of size  
in the original graph  never maps to 
an edge-set in the tree  that 
separates  and  in . 
So, the rounding algorithm still outputs a feasible solution to GST
with constant probability even if we remove edges in the tree
 that correspond to a subset of  edges in .
Consequently, we only need to run the algorithm for 
 times
to boost the probability of success so that,
for any subset of  edges and any terminal ,
we have at least one solution that 
contains an -path using none of these  edges.
In other words, the union of all the solutions
satisfies the connectivity requirement.
Our algorithm is described in Algorithm~\ref{algo:kdst}.

\begin{algorithm}
\caption{Algorithm for -DST}
\begin{algorithmic}
\label{algo:kdst}
\STATE Solve LP-k-DST* and obtain an optimal solution .
\STATE Map  to a solution  to LP-GST on .
\FOR{ \TO }
   \STATE Run GKR Rounding on  to get a solution . 
   \STATE Map  back to a subgraph  of .
\ENDFOR
\RETURN  as a solution to -DST.
\end{algorithmic}
\end{algorithm}

We map a solution  of LP-k-DST* on  to 
a solution  of LP-GST on the tree 
as below.
Note that there is a one-to-one mapping between a path in  and 
a path in the tree .


\subsection{Cost Analysis}
\label{sec:cost-analysis}

We show that .

\begin{lemma}
\label{lem:cost-kdst-mapping}
Consider a solution  to LP-GST, which is mapped from 
a solution  of LP-k-DST* when 
an input -DST instance is -shallow,
for .
The cost of  is at most
.
\end{lemma}
\begin{proof}
By the constraint 
,
we have that

\end{proof}

It can be seen from Algorithm~\ref{algo:kdst} and 
Lemma~\ref{lem:cost-kdst-mapping} that 
the algorithm outputs a solution  with cost 
at most .
Thus,  is an -approximate solution.
It remains to show that  is feasible to -DST.

\subsection{Feasibility Analysis}
\label{sec:feasibility-analysis}

Now we show that  is feasible to -DST 
with at least constant probability.
To be formal, consider any subset  of  edges.
We map  to their corresponding edges  in the tree .
Thus, .

Observe that no vertices in  correspond
to paths that contain an edge in .
Thus, we can define an LP solution  for LP-GST
on the graph  as follows.


We show that  is feasible to LP-GST on .
\begin{lemma}
\label{lem:feasibility-of-calG-minus-F}
For any subset of edges , 
define  from  as above. 
Then  is feasible to LP-GST on .
\end{lemma}

\begin{proof}
First, observe that  only if a path  contains no edges in .
So, by construction,  satisfies

for all .
Hence,  satisfies the capacity constraint.

Next we show that  satisfies the connectivity constraint.
Consider the solution  to LP-k-DST*.
By the feasibility of  and the Max-Flow-Min-Cut theorem, 
the graph  with capacities 
can support a flow of value one from  to any terminal . 
This implies that
.
Consequently, we have 


All the other constraints are satisfied 
because  is constructed from .
Thus,  is feasible to LP-GST on .
\end{proof}

Lemma~\ref{lem:feasibility-of-calG-minus-F} implies that
we can run the GKR Rounding algorithm on .
The following is the property of GKR Rounding.

\begin{lemma}[\cite{GargKR00}]
\label{lem:prop-of-GKR}
There exists a randomized algorithm such that,
given a solution  to LP-GST on a tree 
with height ,
the algorithm outputs a subgraph 
so that the probability that any subset of vertices 
 is connected to the root is at least

Moreover, the probability that each edge is chosen is at most .
That is, .
The running time of the algorithm is . 
\end{lemma}
 
Since  (coordinate-wise),
we can show that running GKR Rounding 
on  simulates 
the runs on  for all 
with , simultaneously.

\begin{lemma}
\label{lem:backward-feasible-kDST}
Let  be a subgraph of  obtained by
running GKR Rounding on ,
and let  be a subgraph of  corresponding to .
Then, for any subset of edges  with 
and for any terminal , 

\end{lemma}
\begin{proof}
Let us briefly describe the work of GKR Rounding.
The algorithm marks each edge  in the tree with
probability , where  is the parent of
an edge  in , which is unique.
Then the algorithm picks an edge  if all of its ancestors
are marked. 
Clearly, removing any set of edges  from 
only affects paths that contain an edge in .

Let  be defined from  as above.
This LP solution is defined on a graph .
Thus, the probability of success is not affected by removing 
from the graph. 
By Lemma~\ref{lem:feasibility-of-calG-minus-F},
we can run GKR Rounding on  and obtain
a subgraph  of .
Since  for all paths 
and  for all ,
we have from Lemma~\ref{lem:prop-of-GKR}
and Lemma~\ref{lem:feasibility-of-calG-minus-F} that

\end{proof}

Finally, we recall that Algorithm~\ref{algo:kdst} employs
GKR Rounding on  for  times.
So, for any subset of  edges 
and for any terminal ,
there exists one subgraph that has an -path 
that contains no edge in  with large probability.
In particular, the union is a feasible solution to -DST
with at least constant probability.
\begin{lemma}
Consider the run of Algorithm~\ref{algo:kdst}.
The solution subgraph  
is a feasible solution to -DST
with probability at least .
\end{lemma}
\begin{proof}
For , let  be a subgraph of  
obtained by running GKR Rounding on 
and mapping the solution back to a subgraph of 
as in Algorithm~\ref{algo:kdst}.
By Lemma~\ref{lem:backward-feasible-kDST},  
has an -path with probability .
Since each  is sampled independently, we have

We have at most  such sets  and
at most  terminals.
So, there are at most  bad events where 
there exists an edge-set of size  that 
separates the root  and some terminal .
Therefore, by union bound,  is a feasible solution 
to -DST with probability at least .
\end{proof}

This completes the proof of Theorem~\ref{thm:approx-k-dst}.
Note that, for the case of DST (), we only need to run GKR
Rounding for  times, thus implying an approximation ratio of
. 

\section{Conclusion and Discussion}
\label{sec:conclusion}

We presented the first non-trivial approximation algorithm for -DST
in a special case of a -shallow instance, which exploits
the reduction from DST to GST. 
We hope that our techniques will shed some light in designing 
an approximation algorithm for -DST in general case and
perhaps lead to a bi-criteria approximation algorithm in the same manner
as in \cite{ChalermsookGL15}.

One obstruction in designing an approximation algorithm in directed
graphs is that there is no ``true'' (perhaps, probabilistic) tree-embedding for
directed graphs.
Both devising a tree-embedding for directed graphs and designing an
approximation algorithm for -DST with  are big open
problems in the area.
Another open problem, which is considered as the most challenging one
by many experts, is whether there exists a polynomial-time algorithm
for DST that yields a sub-polynomial approximation ratio.

\subparagraph*{Acknowledgements.}
Our work was inspired by the works of 
Rothvo{\ss}~\cite{Rothvoss11} and 
Friggstad~et~al.~\cite{FriggstadKKLST14}
and by discussions with Joseph Cheriyan and Lap Chi Lau.
We also thank Zachary Friggstad for useful discussions.

\bibliographystyle{alpha}
\bibliography{kdst}





































\end{document}
