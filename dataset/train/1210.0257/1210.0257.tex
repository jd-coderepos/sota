\documentclass[11pt]{article}
 \usepackage{amssymb,amsmath}
\usepackage{amsthm}
\usepackage{graphicx,epsfig,mathrsfs}
 \usepackage{graphicx}
\usepackage{boxedminipage}
\usepackage{color}

\def\comm#1{\marginpar{\textcolor{red}{*\raggedright{\small #1}*}}}


\usepackage{vmargin}
 
\newcommand{\defparprob}[4]{
  \vspace{1mm}
\noindent\fbox{
  \begin{minipage}{0.96\textwidth}
  \begin{tabular*}{\textwidth}{@{\extracolsep{\fill}}lr} #1  & {\bf{Parameter:}} #3 \\ \end{tabular*}
  {\bf{Input:}} #2  \\
  {\bf{Question:}} #4
  \end{minipage}
  }
  \vspace{1mm}
}



\def\rem#1{{\marginpar{\raggedright\scriptsize #1}}}



\usepackage{algorithm}
\usepackage{algorithmic}


\usepackage{graphicx}
\usepackage{amsfonts}
\usepackage{amssymb}
\usepackage{amsmath}
\usepackage{amssymb}
\usepackage{latexsym}

\pagestyle{plain}

\newtheorem{theorem}{Theorem}
\newtheorem{observation}{Observation}
\newtheorem{proposition}{Proposition}
\newtheorem{corollary}{Corollary}
\newtheorem{lemma}{Lemma}
\newtheorem{claim}{Claim}
\newtheorem{fact}{Fact}
\newtheorem{define}{Definition}
\newtheorem{definition}{Definition}
\newtheorem{Definition}{Definition}
\newtheorem{tlemma}{Technical Lemma}
\newtheorem{remark}{Remark}
\newtheorem{conclusion}{Conclusion}
\newtheorem{conjecture}{Conjecture}

\newcommand{\PP}{{\bf P}}
\newcommand{\FF}{{\cal F}}
\newcommand{\GG}{{\cal G}}
\newcommand{\HH}{{\cal H}}
\newcommand{\Ok}{ \tilde{O}(\sqrt k) }
\newcommand{\tw}{{\mathbf{tw}}}
\newcommand{\cwd}{{\mathbf{cwd}}}
\newcommand{\sfc}{-face}
\newcommand{\tHmf}{-topological-minor-free}
\newcommand{\Hmf}{-minor-free}

\newcommand{\tDS}{{\texttt{\sc DS}}}
\newcommand{\tCDS}{{\texttt{\sc CDS}}}
\let \ol \overline
\newcommand{\rp}{-protrusion}
\newcommand{\rpp}{-protrusion}

\newcommand{\pmin}{{\sc -min-CMSO}}
\newcommand{\peq}{{\sc -eq-CMSO}}
\newcommand{\pmax}{{\sc -max-CMSO}}
\newcommand{\pmem}{{\sc -min/eq/max-CMSO}}
\newcommand{\pmm}{{\sc -min/max-CMSO}}
\newcommand{\pem}{{\sc -eq/max-CMSO}}
\newcommand{\pmeq}{{\sc -min/eq-CMSO}}
\newcommand{\eg}{{\bf eg}}
\newcommand{\twoc}{-connected}
\newcommand{\threec}{-connected}
\newcommand{\s}{\mbox{}}
\newcommand{\defeq}{:=}
\newcommand{\eqdef}{=:}
\newcommand{\Grid}[1]{{\sf grid}_{#1}}
\newcommand{\varGrid}[1]{{\sf vargrid}_{#1}}
\newcommand{\vc}{\mbox{\bf vc}}
\newcommand{\w}{\mbox{\bf w}}
\newcommand{\fvs}{{\bf fvs}}
\newcommand{\fc}{\mbox{\bf -fc}}
\newcommand{\cp}{{\bf cp}}
\newcommand{\ds}{\mbox{\bf -ds}}
\newcommand{\red}{{\bf red}}
\newcommand{\rds}[1]{\mbox{{\bf rds}}}
\newcommand{\rfc}[1]{\mbox{{\bf rfc}}}
\newcommand{\bds}[1]{\mbox{{\bf bds}}}
\newcommand{\rbc}[1]{\mbox{{\bf bfc}}}
\newcommand{\dist}{\mbox{{\bf dist}}}
\newcommand{\rep}{\mbox{{\bf rep}}}
\newcommand{\rdist}{\mbox{{\bf rdist}}}
\newcommand{\bdist}{\mbox{{\bf bdist}}}
\newcommand{\cO}{\mathcal{O}}
\newcommand{\cU}{\mathcal{U}}
\newcommand{\cS}{\mathcal{S}}
\newcommand{\cN}{\mathcal{N}}
\newcommand{\dsp}{\displaystyle}  



\newcommand{\mm}{\mbox{\bf mm}}
\newcommand{\bor}{\mbox{\bf bor}}
\newcommand{\cov}{\mbox{\bf cov}}
\newcommand{\pack}{\mbox{\bf pack}}
\newcommand{\pw}{\mbox{\bf pw}}
\newcommand{\dset}[1]{{\left\{\,#1\,\right\}}}
\newcommand{\tset}[1]{{\{\,#1\,\}}}
\newcommand{\sset}[2]{{#1\{\,#2\,#1\}}}
\newcommand{\set}[1]{\mathchoice {\dset{#1}}{\tset{#1}}{\tset{#1}}{\tset{#1}}}

\newcommand{\nats}{{\ensuremath{\mathbb{N}}}}
\newcommand{\Ptime}{{\ensuremath{\mathbf{P}}}}
\newcommand{\degen}[1]{\delta^{#1}}
\newcommand{\mdegen}[1]{\delta^{#1}_{m}}
\newcommand{\fandeg}[1]{\mbox{\bf deg}^{#1}}
\newcommand{\mfandeg}[1]{\mbox{\bf mdeg}^{#1}}
\newcommand{\h}[1]{\end{document}}

\newcommand{\rdegen}{{\bf rdegen}}
\newcommand{\param}{{\bf p}}
\newcommand{\edist}{\mbox{\sc e-dist}}
\newcommand{\optw}{\operatorname{tw}}
\newcommand{\ttree}{\mathcal{T}}

\newcommand{\sed}[1]{\noindent {\tt SED> #1}}


\DeclareMathOperator{\operatorClassNP}{\sf NP}
\newcommand{\classNP}{\ensuremath{\operatorClassNP}}
\DeclareMathOperator{\minor}{\preceq}
\DeclareMathOperator{\tminor}{\preceq_{\mathrm{t}}}
\newcommand{\trminor}[1]{\preceq_{\mathrm{tp}}^{#1}}
\newcommand{\bw}{{\bf bw}}
\newcommand{\mids}{\omega}

\def\rem#1{{\marginpar{\raggedright\scriptsize #1}}}

\usepackage{todonotes}
\newcommand{\mar}[1]{#1}
\newcommand{\term}[1]{#1}

\def\baselinestretch{1}





\begin{document}


\title{Kernels for (connected) Dominating Set on  graphs with \\ Excluded Topological subgraphs\thanks{Preliminary versions of this paper appeared in SODA 2012 and STACS 2013. }}
\author{Fedor V. Fomin\thanks{University of Bergen, Norway, \texttt{fomin@ii.uib.no}.  The research was supported by the European Research Council through ERC Grant Agreement n. 267959.}
\and 
Daniel Lokshtanov\thanks{University of Bergen, Norway, \texttt{daniello@ii.uib.no}. The research was supported by the Bergen Research Foundation and the University of Bergen through project ``BeHard'.}
\and
Saket Saurabh\thanks{The Institute of Mathematical Sciences, CIT Campus, Chennai, India, \texttt{saket@imsc.res.in}. 
The research was supported by the European Research Council through Starting Grant 306992 ``Parameterized Approximation'.}
\and 
Dimitrios M. Thilikos\thanks{Department of Mathematics, National and Kapodistrian University of Athens, Athens, Greece,  \texttt{sedthilk@thilikos.info}. Co-financed by the E.U. (European Social Fund - ESF) and Greek national funds through the Operational Program ``Education and Lifelong Learning'' of the National Strategic Reference Framework (NSRF) - Research Funding Program: ``Thales. Investing in knowledge society through the European Social Fund''.}~\thanks{AlGCo project-team, CNRS, LIRMM, Montpellier, France}}
\date{}
\maketitle
\thispagestyle{empty}

\begin{abstract}
\vspace{1mm}

\noindent We give the first linear kernels for the  {\sc Dominating Set}  and  {\sc Connected Dominating Set}  problems on graphs excluding a fixed graph  as a topological minor.  In other words, we prove the existence of polynomial time algorithms that, for a  given  \tHmf\,  graph  and  a positive integer , 
output an \tHmf\,  graph    on   vertices such that  has a (connected) dominating set of size  if and only if  has one.


Our results extend the known classes of graphs on which the {\sc Dominating Set}  and  {\sc Connected Dominating Set}  problems admit linear kernels.  Prior to our work, it was known that these problems admit linear kernels on graphs excluding a fixed  apex graph  as a minor.   Moreover, for   {\sc Dominating Set},  a kernel of size  , where  is a  constant depending  on the size of , follows from a more general result on the kernelization of {\sc Dominating Set} on graphs of bounded degeneracy.  Alon and Gutner explicitly asked  whether one can obtain a linear kernel for {\sc Dominating Set} on \Hmf\,  graphs. We answer this question in the affirmative and in fact prove a more general result. 
For  {\sc Connected Dominating Set}  no  polynomial kernel  even on \Hmf\,  graphs was known prior to our work.  On the negative side, it is known that {\sc Connected Dominating Set}  on -degenerated graphs does not admit a polynomial kernel unless \textsf{coNP}   \textsf{NP/poly}. 


Our kernelization algorithm is based on a non-trivial combination of the following  ingredients
 \begin{itemize} 
\item The structural theorem of Grohe and Marx [STOC 2012] for graphs excluding a fixed graph  as a topological minor;
\item A novel notion of protrusions, different than the one defined in [FOCS 2009];
\item Our results are based on  a  generic reduction rule that produces an equivalent instance (in case the input graph is \Hmf)  of the problem, with treewidth 
. 
The application of this rule  in a divide-and-conquer fashion, together with the new notion of protrusions,  gives us the linear kernels. 

\end{itemize}
A protrusion in a graph [FOCS 2009] is a subgraph of constant treewidth which is separated from the rest of the graph by at most a constant number of vertices. In our variant of protrusions, instead of stipulating that the subgraph be of constant 
{\em treewidth}, we ask that it contains a {\em constant number of vertices from a solution}. We believe that this new take on protrusions would be useful for other graph problems and in different algorithmic settings. 
\end{abstract}



\noindent{\bf Keywords:}{ Kernelization, Connected Dominating Set, topological minor free graphs.}




\maketitle



\section{Introduction}
{\em Kernelization} is well established subarea of parameterized complexity. 
 A parameterized problem is said to admit a {\em polynomial kernel} if there is a polynomial time algorithm (the degree of polynomial being independent of the parameter ), called a {\em kernelization} algorithm, that reduces the input instance down to an instance with size bounded by a polynomial  in , while preserving the answer. This reduced instance is called a {\em  kernel} for the problem. If the size of the kernel is , then we call it a {\em linear kernel} (for a more formal definition, see Section~\ref{sec:defs_and_nots}). Kernelization has turned out to  be an interesting computational approach both from practical and theoretical perspectives. There are many real-world applications where even very simple preprocessing can be surprisingly effective, leading to significant reductions  in the size of the input. Kernelization is a natural tool not only for  measuring the quality of preprocessing rules proposed for  specific problems but also for designing new powerful preprocessing algorithms. From the theoretical perspective, kernelization provides a deep insight into the hierarchy of parameterized problems in  {\sf FPT}, the most interesting class of parameterized problems.  There are also interesting links  
  between lower bounds on the sizes of kernels and classical computational complexity  \cite{BDFH08,Dell:2010sh,DruckerA12}. 
 
 
 
 
The {\sc Dominating Set} (\tDS) problem
 together with its numerous variants, is one of the most  classical and well-studied problems in algorithms and combinatorics~\cite{HaynesHS98}. 
In the {\sc Dominating Set} (\tDS) problem,
we are given a graph  and a non-negative integer , and the question is 
whether  contains a set of  vertices whose closed neighborhood contains all the vertices of .  
The connected variant of the problem,  {\sc Connected Dominating Set} (\tCDS) asks, given a graph  and a non-negative 
integer , whether  contains a dominating set  of at most  vertices such that 
for every connected component  of , we have that  is connected.  
This definition of \tCDS\ differs slightly from the established one where one just  demands that  
the subgraph induced by the dominating set  be connected. Our definition generalizes the established 
one  to  include disconnected graphs. 
A  considerable part of the algorithmic  study of these {\sf NP}-complete  problems has been focused on the  design of parameterized and kernelization algorithms. In general,  \tDS \,  is {\sf W}[2]-complete 
and therefore it cannot be solved by a parameterized algorithm, 
unless an unexpected collapse occurs in the 
Parameterized Complexity hierarchy (see~\cite{DowneyF98,FlumGrohebook,Niedermeierbook06}) and thus also does not admit a kernel.  
However, there are  interesting graph classes where  {\em fixed-parameter tractable} {\sf  (FPT)}  algorithms exist
for the \tDS \, problem. The project of  widening the families of graph classes, on which such algorithms exist, inspired a multitude of  ideas that made \tDS \,  the test bed for some of the most cutting-edge techniques of parameterized algorithm design. For example, the initial study of parameterized subexponential algorithms for \tDS \, on planar graphs \cite{AlberBFKN02,DemaineFHT05talg,FominT06} resulted in the creation of bidimensionality theory
characterizing a broad range of graph problems  that admit efficient approximation schemes, fixed-parameter algorithms or kernels on a broad range of graphs \cite{DemaineFHT05sub,DemaineH07-CJ,DornFLRS10,FominLRS10,F.V.Fomin:2010oq,FominLS12}. 
 
    
  One of  the  first  results  on linear kernels is the celebrated work of Alber et al.  on  \tDS \,  on planar graphs \cite{AlberFN04}. This work  augmented significantly the interest in  proving polynomial (or preferably linear) 
 kernels for other parameterized problems.   
  The result of Alber et al.~\cite{AlberFN04}, see also  \cite{ChenFKX07}, has been extended to much more general graph classes like graphs of bounded genus \cite{H.Bodlaender:2009ng} and apex-minor-free graphs \cite{F.V.Fomin:2010oq}.
An important step in this direction was made by  Alon and Gutner \cite{AG08TechReport,Gutner09}  who obtained a kernel of size  for  \tDS \,  on \Hmf\,  and \tHmf\,  graphs, where the constant  depends on the excluded graph . Later, Philip et al.~\cite{PhilipRS09} obtained a kernel of size  on  -free and -degenerate graphs, where  depends on  and  respectively.  In particular, for -degenerate graphs, a subclass of -free graphs,  the algorithm of   Philip et al.~\cite{PhilipRS09} produces a kernel of size
 . Similarly, the sizes of   the kernels in~\cite{AG08TechReport,Gutner09,PhilipRS09} are bounded by  polynomials in  with degrees depending on the size of the excluded minor . 
  Alon and Gutner \cite{AG08TechReport} mentioned as a  challenging question  whether one can  characterize the families of graphs for which the dominating set problem admits a linear kernel, i.e. a kernel  of size , where the function  depends {\em exclusively} on the 
graph family. 
In this direction, there are already results for more restricted graph classes.
According to
the meta-algorithmic results on kernels introduced in~\cite{H.Bodlaender:2009ng},  \tDS \,  has a kernel 
of size  on graphs of genus . An alternative meta-algorithmic 
framework, based on bidimensionality theory \cite{DemaineFHT05sub}, was introduced in~\cite{F.V.Fomin:2010oq}, implying the existence of a kernel of size  for \tDS \, on graphs excluding an { apex\footnote{An {\em apex} graph is a graph that can be made planar by the removal of a single vertex.}} graph  as a minor. While apex-minor-free graphs form much more general class of graphs than  graphs of bounded genus, \Hmf\,  graphs  and \tHmf\, graphs form much larger classes than apex-minor-free graphs. For example, the class of graphs excluding , the complete  graph on  vertices, as a minor, contains all apex graphs. Alon and Gutner in ~\cite{AG08TechReport} and Gutner  in~\cite{Gutner09} posed as an open problem  
whether one can obtain a linear kernel for \tDS \,  on \Hmf\,  graphs.
Prior to our work, the only result on linear kernels for \tDS \, on graphs excluding a fixed graph  as a topological minor, was the result of  
 Alon and Gutner  in~\cite{AG08TechReport}  for the  special case where .
See Fig.~\ref{fig:graph_classes} for the relationship between these classes.  





\begin{figure}[t]
\begin{center}
\includegraphics[scale=0.287]{graphs}
\caption{Kernels for DS and CDS on classes of sparse graphs. Arrows represent inclusions of classes.
In the diagram, [J.ACM 04] refers to the paper of Albers et al. \cite{AlberFN04}, [FOCS 09]   to the paper of Bodlaender et al.~\cite{H.Bodlaender:2009ng},
[SODA 10] and [SODA 12] to the papers of Fomin et al. \cite{F.V.Fomin:2010oq,F.V.Fomin:2012},  [ESA 09] to the paper of Philip et al. \cite{PhilipRS09}, and  [WG 10] to the paper of Cygan et al. \cite{Cygan:2010bv}. }
\label{fig:graph_classes}
\end{center}
\end{figure}

 
 It is tempting to conjecture that similar improvements on kernel sizes are possible for more general graph classes like -degenerate graphs. For example, for graphs of bounded vertex degree, a subclass of -degenerate graphs,  has a trivial linear kernel. Unfortunately,  for -degenerate graphs the existence of a linear kernel, or even a polynomial kernel with the exponent of the polynomial being independent of , is very unlikely. 
   By the recent work of   Cygan et al. \cite{CyganGH12}, the kernelization algorithm of Philip et al.~\cite{PhilipRS09}  is essentially  tight---the existence of a kernel of size  
  for   on -degenerate  graphs would imply that  {\sf coNP} is contained in {\sf NP/poly}.  
 
 In this work we show how to generalize the linearity of kernelization  for \tDS \, from 
 bounded-degree graphs and  apex minor free graphs to the class of graphs excluding a fixed graph  as a topological minor.
  Moreover, a   modification of the ideas for \tDS \, kernelization can be used to obtain a linear kernel for \tCDS, which is usually a much more difficult problem to handle due to the connectivity constraint.  For example, \tCDS \, does not have a polynomial kernel on -degenerate graphs  unless {\sf coNP} is in {\sf NP/poly}    \cite{Cygan:2010bv}. We must {\em emphasize} that our linear kernels are existential. That is, we just show the mere existence of polynomial time algorithms computing linear kernels. 
  
  The class of graphs excluding a fixed graph  as a topological minor  is a wide class of graphs containing \Hmf\, graphs and graphs of constant vertex degrees. The existence of a linear kernel for \tDS \, on this class of graphs significantly extends and improves  previous works  
  \cite{AG08TechReport,F.V.Fomin:2012,Gutner09}.
The extension of the results for planar graphs from \cite{AlberFN04} and apex-minor-free graphs from \cite{F.V.Fomin:2010oq} to the more general family of  \Hmf \, graphs requires several new ideas. Similar difficulties in 
  generlizing algorithmic techniques from apex-minor free to \Hmf \, graphs were observed in approximation   
\cite{Demaine:2009pd}  and parameterized algorithms  \cite{DemaineFHT05sub,DraganFG08}.  The basic idea behind kernelization algorithms on  apex-minor-free  graphs is the bidimensionality of \tDS. Roughly speaking, the treewidth of these graphs with dominating set of size  is  .  In other words, 
excluding an apex graph 
makes it possible to bound the tree-decomposability  of the input 
graph by a {\em sublinear} function of the size of a dominating set which is not the case for more general classes of \Hmf \, graphs or a family of graphs excluding a fixed graph  as a topological minor. 
  
  
A main ingredient of our kernelization algorithms are new reduction rules that allow us to obtain the desired kernels on \Hmf \, graphs.  This is an important step for our kernel on  the class of graphs excluding a fixed graph  as a topological minor.   The main idea behind our algorithm is to identify and remove ``irrelevant" vertices  without changing the solution such that in the reduced graph one can select  vertices whose removal leaves protrusions, that is, subgraphs of constant treewidth separated from the remaining vertices by a constant number of vertices. If we are  able to 
obtain such a graph, we can use   the techniques from \cite{F.V.Fomin:2010oq} to construct the linear kernel. 
Roughly speaking, our rule to  identify  ``irrelevant"  vertices works as follows: 
we try specific vertex subsets of constant size, for each subset we try all ``feasible" scenarios how dominating sets can  interact with the subset, and find neighbours of theses 
subsets whose removal does not change the outcome of any feasible scenario. 
The main difference of this new reduction  rule in comparison to other rules for  \tDS \,   \cite{AlberFN04,ChenFKX07} is that instead of  reducing the size of the graph to , it reduces the treewidth of the graph to  . Thus idea-wise, it is closer  to 
 the ``irrelevant vertex'' approach  developed  by Robertson and Seymour  for 
  disjoint paths  and minor checking problems \cite{RobertsonS-XIII}. 
However, the significant difference with this technique is that in all applications of ``irrelevant vertex''   the bounds on the treewidth are exponential or even worse
\cite{KawarabayashiK08,Kawarabayashi:2010cs,Kobayashi:2009jt}. Moreover,  Adler et al.~\cite{Adler11} provide instances  of the  disjoint paths problem on planar graphs, for which the irrelevant vertex approach of  Robertson and Seymour produces graphs of treewidth .  Our rule provides a reduced graph with   \emph{sublinear}    treewidth for .


The proof that after deletion of all irrelevant vertices the treewidth of the graph becomes sublinear is non-trivial.  For this proof we 
  need  the   theorem of Robertson and Seymour \cite{RobertsonS03}    on decomposing a graph into a set  of torsos connected via clique-sums. By making use of this theorem, we show that by applying the rule for all subsets of apex vertices of  each torso, it is possible to  reduce the treewidth  of each torso to . This implies that the treewidth of the reduced graph is also . However,  the number of torsos can be  and the sublinear treewidth of the reduced graph  still does not bring us directly to the kernel. To  overcome this obstacle, we have to implement the irrelevant vertex rule  in a divide and conquer manner, and only after doing this can we guarantee that the reduced graph admits a linear kernel.  The idea of using divide and conquer in kernelization is our first conceptual contribution. 
  









The second main step of our kernelization algorithm for , on  the class of graphs excluding a fixed graph  as a topological minor, is to design reduction rules for graphs of bounded degree. 
The ideas introduced for \Hmf \, graphs  can hardly  work on graphs of bounded degree, and hence on   graphs excluding a fixed graph  as a topological minor. The reason is that the bound  on the treewidth of such graphs   would imply that  \tDS
\, is solvable in subexponential time on graphs of bounded degree, which in turn can be shown to contradict the  Exponential Time Hypothesis       
\cite{ImpagliazzoPZ01}. This is why the kernelization techniques developed for \Hmf \, graphs do not seem to be applicable directly in our case.


\paragraph{High level overview of the main  ideas.}
Our kernelization algorithm has two main phases. In the first phase we partition the input graph  into subgraphs , such that 
 ; for every , the neighbourhood ,  and .   In the second phase, we replace these graphs by smaller equivalent graphs.  Towards this, we treat 
  graphs , ,  as -boundaried graphs with boundary . Our second  conceptual contribution is a  polynomial time algorithmic procedure for replacing a -boundaried graph by an equivalent graph of size .   Observe that as a result of such replacements, the size of the new graph is 
   
  and thus we obtain a linear kernel. 
  Kernelization techniques based on replacing a -boundaried graph by an equivalent instance or, more specifically, protrusion replacement were used before in~\cite{H.Bodlaender:2009ng,F.V.Fomin:2010oq,FominLMPS11, abs-1207-0835}. At this point it is also important to mention earlier works done in~\cite{FellowsL89,ArnborgCPS93,BodlaendervA01a,Fluiter97,BodlaenderH98}  
  on protrusion replacement in the algorithmic setting on graphs of bounded treewidth. 
  The substantial differences  with our replacement 
  procedure  and the ones used before in the kernelization setting are the following. 
  \begin{itemize}
  \item In the protrusion replacement procedure it is assumed that the size of the boundary  and the treewidth  of the replaced graph are constants. In our case neither the treewidth, nor the boundary size are  bounded. In particular, the boundary size  could be a {\em linear} function of .
  \item In earlier protrusion replacements, the size of the equivalent replacing graph is bounded by some (non-elementary) function of . In our case this is a {\sl linear} function of .  
  \end{itemize} 
Our new  replacement procedure strongly exploits the fact that graphs  possess 
 a set of desired properties allowing us to apply the irrelevant vertex technique explained above. 
However, not every graph  excluding some fixed graph   as a topological minor can be partitioned into graphs with the desired properties.  We show that, in this case, there is  another polynomial time procedure transforming  into an equivalent graph, which in turn can be partitioned. The procedure is based on a generalized notion of protrusion, which is the third conceptual contribution of this paper. In the new notion of protrusion we relax  the requirement that protrusions are of bounded treewidth by the condition   that they  have a bounded size dominating set. Let us remark, that a similar notion of a generalized protrusion, bounded by the size of a certificate, can be used  for a variety of graph problems. We show that
either   a graph does not have the desired partition, or  it   contains a  sufficiently large generalized protrusion, which can be replaced by a smaller equivalent subgraph.  
The construction of the partitioning  is heavily based on the recent work of Grohe and Marx on the structure of  such graphs  \cite{GroheM12}.


   As a byproduct of our results we  obtain the first subexponential time  algorithms for {\sc Connected Dominating Set}, a deterministic algorithm  solving the problem on an -vertex \Hmf \, graph in  time .  For {\sc Dominating Set} our results imply a significant 
simplification and refinement  of  a  algorithm on \Hmf \, graphs due to Demaine et al. \cite{DemaineFHT05sub}. 
Also our kernels can be used to obtain, subexponential, polynomial-space parameterized algorithms for these problems.    

\paragraph{Organization of the paper.}
The remaining part of this paper is organized as follows.
In Section~\ref{sec:defs_and_nots}, we provide definitions and  state known results used in the paper. In Section~\ref{sec:gens:protrs}, we introduce the new notion of ``generalized protrusions" and build a theory of  replacements for such protrusions. 
We provide a decomposition lemma in Section~\ref{sec:slicedecs}, which will be used  for kernelization algorithms. In Sections~\ref{sec:domset_kernel} and~\ref{sec:CDSkernel} we give the two main results of the paper, linear kernels for \tDS\ and \tCDS\ on  the class of graphs excluding a fixed graph  as a topological minor. In Section~\ref{sec:concludes}  we conclude with questions for further research and give a  short overview of some of the developments which happened since the conference versions of this paper were published, including work on  kernelization of \tDS\ and \tCDS\ on graphs of bounded expansion and on nowhere-dense graphs. 


\section{Preliminaries}
\label{sec:defs_and_nots}


In this section we give various definitions which we make use of in the paper. We refer to Diestel's book 
\cite{diestelbook} for standard definitions from Graph Theory.
 Let~ be a graph with vertex set  and edge set .  A graph~ is a
 \emph{subgraph} of~ if~ and~.
 For a subset , the subgraph~ of  is called the   \emph{subgraph induced  by }  if~.
By 
we denote the (open) neighborhood of  in graph , that is, the set of all vertices
adjacent to  and by . 
Similarly, for a subset , we define 
and .  Given a set , we define  as the set of vertices in 
 that have a neighbor in .  We omit the subscripts when they are clear from the context. A subset of vertices  is called a {\em dominating set} of  if 
. A subset of vertices   is called a {\em connected dominating set} if it is a dominating set and for every connected component  of  we have that  is connected. Throughout the paper, given a graph  and vertex subsets  and , whenever we say that a subset  
{\em dominates all but (everything but)}  then we mean that . Observe that a vertex of  can also be dominated by the set . 


\medskip


We denote by  the complete graph on  vertices. Also for a given graph  and a vertex subset  by  we mean a clique on 
the vertex set . 
For an integer  and  vertex subsets , we say that a subset  is \emph{-dominated} by , if for every   there is  such that the distance between  and  is at most . For , we simply say that  is dominated by .  We denote by   the set of vertices -dominated by .  

Throughout this paper we use ,    and    for the sets of  integers,  non-negative and 
non-positive integers respectively.   Finally,  we use  for the set of positive integers.
\paragraph{Minors and Contractions.}
Given an edge   of a graph , the graph   is
obtained from   by contracting the edge , that is,
the endpoints   and  are replaced by a new vertex 
which  is  adjacent to the old neighbors of  and  (except
from  and ).  A graph  obtained by a sequence of
edge-contractions is said to be a \emph{contraction} of .  We denote it by .
A graph  is a {\em minor} of a graph  if  is the contraction of some subgraph
of  and we denote it by .
We say that a graph  is {\Hmf \,} when it does not
contain  as a minor. We also say that a graph class 
is {\Hmf \,} (or, excludes  as a minor)  when
all its members are \Hmf.
An \emph{apex graph} is a graph obtained from a planar graph 
by adding a vertex and making it adjacent to some of the vertices of .
A graph class  is \emph{apex-minor-free} if 
excludes a fixed apex graph  as a minor.





A \emph{subdivision} of a graph  is obtained by replacing each edge of  by a non-trivial path. We say that  is a \emph{topological minor} of  if some subgraph of  is isomorphic to a subdivision of  and denote it by . 
A graph  \emph{excludes a graph  as a (topological) minor} if  is not a (topological) minor of .  For a graph , 
by , we denote all graphs that exclude  as  topological minors.



















\paragraph{ Tree-Decompositions.} A \emph{tree-decomposition} of a graph  is a pair  where  
is a rooted tree and , such that :

\begin{enumerate}
\item .
\item For each edge , there is a  such that both  and  belong to .
\item For each , the nodes in the set  form a subtree of .
\end{enumerate}
If  is a path then we call the pair    as {\em path-decomposition}.

\noindent
The following notations are the same as that in \cite{GroheM12}. Given a tree-decomposition of a graph 
, 
we define mappings   and . 
 For all ,\\
\begin{center}
tMstM\end{center}

\\



For all , . 

For a subgraph  of  by  we denote . 
\medskip
\noindent



Let  be a tree-decomposition of a graph . The {\em width} of  is 
 and the {\em adhesion}  of the tree-decomposition is 
 
We use  to denote the treewidth of the input graph. For every node , the {\em torso} at  is the graph 

\begin{center}
.
\end{center}

We take the graph induced by , turn  into a clique, and make vertices  adjacent if they appear together in the separator  of some child  of .


\paragraph{Parameterized graph problems.}
A parameterized graph problem   is usually defined as a subset of 
where, in each instance  of   encodes a graph and  is the parameter (we denote by  the set of all non-negative integers). In this paper we use an extension of this definition (also used by Bodlaender et al.~\cite{H.Bodlaender:2009ng}) that permits the parameter  to be negative 
with the additional constraint that either all pairs with non-positive values of the parameter 
are in  or that no such pair is in . Formally, a parametrized problem 
is a subset of  where for all 
with  it holds that  if and only if  .
This extended definition encompasses the traditional one and is needed for technical reasons  
(see Subsection~\ref{subsec:finiinteginde}).
In an instance of a parameterized problem  the integer  is called the parameter. Now we formally define the 
 and  problems. 

\defparprob{}{An undirected graph  and a positive integer .}{}{Does there exists  of size at most  such that ?}

\medskip

\defparprob{}{An undirected graph  and a positive integer .}{}{Does there exists  of size at most  such that  and  is connected?}






\paragraph{Kernels   and Protrusions.}
A central notion in 
parameterized complexity is {\em fixed parameter tractability}, which means, 
for a given instance  
solvability in time  where  is an arbitrary function of  and  is a polynomial function in the input size. 
The notion of {\em kernelization} is formally defined as follows. 
 
 
 
\begin{definition}
A {\em{kernelization algorithm}}, or simply a {\em{kernel}}, for a parameterized problem  is an algorithm  that, given an instance  of , works in polynomial-time and returns an equivalent instance  of . Moreover, 
there exists a computable function  such that whenever  is the output for an instance , then it holds that . If the upper bound  is a polynomial (linear) function of the parameter, then we say that  admits a {\em{polynomial (linear) kernel}}.
\end{definition}

We often abuse the notation and call the output of a kernelization algorithm, the ``reduced'' equivalent instance, also a kernel. 
 







 \begin{definition}
Given a graph , we say that a set  is an {\em -protrusion} of  if 
   and the number of vertices in  with a neighbor in  is at most . 
  \end{definition}

\subsection{Known Decomposition Theorems} We start with the definition of nearly embeddable graphs. 
\begin{definition}[-nearly embeddable graphs]
Let  be a surface with boundary cycles , i.e.\ each cycle
 is the border of a disc in . A graph  is
{\em -nearly embeddable} in , if  has a subset  of size at most ,
called {\em apices}, such that there are (possibly empty) subgraphs
 of  such that
\begin{itemize}
\setlength{\itemsep}{-2pt}
\item ,
\item  is embeddable in , we fix an embedding of ,
\item graphs  (called \emph{vortices}) are pairwise disjoint,
\item for , let ,   has a path decomposition 
, of width at most  such that

\begin{itemize}
\item for  and for  we have 
\item for , we have  and the points  appear on  in this order (either if we walk clockwise or anti-clockwise).
\end{itemize}
\end{itemize}
\end{definition}












\noindent
The decomposition theorem that we use extensively for our proofs is given in the next theorem. 
\begin{theorem}  [\cite{abs-1209-0129,GroheM12,RobertsonS03}]
\label{thm:structure theorem}  
For every graph , there exists a constant , 
depending only on the size of , such that every graph   with , there is a tree-decomposition 
 of adhesion at most  such that for all , one of the following conditions is satisfied: 
\begin{enumerate}
\item    is -nearly embedded in a surface  in which  cannot be embedded.
\item  has at most  vertices of degree larger than .
\end{enumerate}
Moreover, if  is an \Hmf \,   then nodes of second type do not exist. 
\noindent
Furthermore, there is an algorithm that, given graphs ,  on  and  vertices, respectively, computes such a tree-decomposition in time 
 for some computable function ,  and moreover computes an apex set  of size at most   
for every bag of the first type.
\end{theorem}
One of the main consequence of Theorem~\ref{thm:structure theorem} we need for our purposes is  that (in the case when 
 is \Hmf) for every 
 there exist  constants  and  such that  for every torso   of the decomposition from 
Theorem~\ref{thm:structure theorem}, there exists a set of vertices  of size at most , called apices, such 
that the graph obtained from  after deleting the apices does not contain some  apex graph  of size 
 as a minor. See, e.g.~\cite[Theorem ]{Grohe03}. 

 Furthermore we can assume that in  , for any , . That is, no bag is contained in other. 
 See~\cite[Lemma 11.9]{FlumGrohebook} for the proof.  
 






  



\subsection{Known Approximation Algorithms}
Recall that by  we denote the class of graphs that exclude a fixed graph  as a topological minor. 
In this subsection we state known polynomial-time constant factor approximation algorithms  for \tDS \, and \tCDS\ on . It is well known that graphs in 
 has bounded 
degeneracy. The following is known about the approximation of \tDS. 

\begin{lemma}[\cite{Dvorak13}]
\label{lemma:approximation}
Let  be a   graph. Then there exists a constant  depending only on  such that 
{\sc Dominating Set}  admits  a -factor approximation algorithm on  . 
\end{lemma}


For \tCDS \,  we  need the following proposition attributed to \cite{Duchet82}.
\begin{proposition}
\label{lem:bb}
Let  be a connected graph and let  be a dominating set of  such that  has at most  connected components. Then 
there exists a set  of size  at most   
 such that  is a connected dominating set in  and, given , we can find such a set  in polynomial time. 
\end{proposition}

Combining Lemma~\ref{lemma:approximation} and Proposition~\ref{lem:bb} we arrive at the following. 
\begin{lemma}
\label{lemma:approximationcds}
Let  be a   graph and  the constant from  Lemma~\ref{lemma:approximation}. Then
\tCDS \,  admits  a -factor approximation algorithm on  . 
\end{lemma}





\section{A New Algorithm for Protrusion Replacement}
\label{sec:gens:protrs}


In the next section we introduce the notion of a ``generalized protrusion''.  Recall that a protrusion in a graph  is a subgraph of constant treewidth which is separated from the rest of the graph by at most a constant number of vertices. In our variant of protrusions, instead of stipulating that the subgraph be of constant treewidth, we ask that it contains a  constant number of vertices from a solution. In this section we show that even if we have a generalized protrusion then we can find a replacement for it efficiently.  
We first start with some relevant definitions and concepts. 




\subsection{Boundaried Graphs} 
\label{subsec:boungrap}
Here we define the notion of {\em boundaried graphs} and various operations on them.
\begin{definition}{\rm [\bf Boundaried Graphs]}\label{def:boungraph}
A \term{boundaried graph} is a graph  with a set  
of  distinguished vertices and an injective labelling  
from   to the set . The set  is called the \term{{\em boundary}} of  and  the vertices in   are called  {\em boundary vertices} or \term{{\em terminals}}. 
Given a boundaried graph  we denote its boundary by 
we denote its labelling by , 
and we define its {\em label set} by .
Given a finite set , we define 
  to denote the class of all boundaried graphs whose label set is . 
We also denote by  the class of all boundaried graphs.
Finally we say that a boundaried graph is a {\em -boundaried} graph if .
\end{definition}





\begin{definition}{\rm [\bf Gluing by ]} Let  and  be two  boundaried graphs. We denote by  the  graph 
(not boundaried) obtained by taking the disjoint union of  and  and identifying equally-labeled vertices of the boundaries of  and  In  there is an edge between two vertices if there is  an edge between them either in  or in , or both.  
\end{definition}

We remark that if  has a label which is not present in , or vice-versa, then in  we just forget that label. 

\begin{definition} {\rm [\bf Gluing by ]}
The {\em boundaried gluing operation}  is similar to the normal gluing operation, but results in a boundaried graph rather than a graph. Specifically  results in a boundaried graph where the graph is  and a vertex is in the boundary of  if it was in the boundary of  or  of . Vertices in the boundary of  keep their label from  or . 
\end{definition}





Let  be a class of (not boundaried)  graphs.
By slightly abusing notation we say that a boundaried graph {\em belongs to a graph class } if the underlying graph belongs to 


\begin{definition}{\rm [\bf Replacement]}\label{defn:replacement}
Let  be a -boundaried graph containing a set 
such that  Let  be a -boundaried graph. The result of {\em replacing  with } is the graph 
where  is treated as a -boundaried graph with  
\end{definition}













\subsection{Finite Integer Index}
\label{subsec:finiinteginde}
\begin{definition}{\rm [\bf Canonical equivalence on boundaried graphs.]}
Let  be a parameterized graph problem whose instances are pairs of the form 
 Given two boundaried graphs  we say that \term{} if 

 and there exists a \term{{\em transposition constant}}
 such that 

Here,  is a function of the two graphs  and . 
\end{definition}
Note that  the relation   is
an equivalence relation. Observe that  could be negative in the above definition. This is the reason we allow the parameter in parameterized problem instances to take negative values.  


Next  we define a notion of ``transposition-minimality'' for the members 
of  each equivalence class of 


\begin{definition}{\rm [\bf Progressive representatives]}
\label{def:progrepr}
Let  be a parameterized graph problem whose instances are pairs of the form 
and let  be some equivalence class of . We say that  is a \term{{\em progressive 
representative}}
of  if for every 
there exists  such that 

\end{definition}

The following lemma guarantees the existence of a progressive representative for each equivalence class of 
. 



\begin{lemma}[\cite{H.Bodlaender:2009ng}]
\label{lem:existprog}
Let  be a parameterized graph problem whose instances are pairs of the form .
Then each  equivalence class of  has a progressive representative.
\end{lemma}




Notice that two  boundaried graphs with different label sets belong to 
different equivalence classes of  Hence for every equivalence 
class  of  there exists some finite set  such that 
. We are now in position  to give the following definition.

\begin{definition}{\rm [\bf Finite Integer Index]}
\label{def:deffii}
A parameterized graph problem  whose instances are pairs of the form 
has {\em Finite Integer Index} (or  is \term{{\em FII}}), if and only if for every finite 
the number of equivalence classes of   that are subsets of 
is finite. For each  we define  to be
a set containing exactly one progressive representative of each equivalence class of 
that is a subset of . We also define . 
\end{definition} 


\subsection{Replacement lemma}




We first define a notion of monotonicity for parameterized problems. 

\begin{definition}
We say that a parameterized graph problem  is {\em positive monotone} if for every graph  
there exists a unique  such that for all  and ,  and for all 
 and , .  A parameterized graph problem  is {\em negative monotone} if for every graph  
there exists a unique  such that for all  and ,  and for all 
 and , .  is monotone if it is either positive monotone or negative monotone.  
We denote the integer  by {\sc Threshold()} (in short  {\sc Thr()}). 
\end{definition}

We first give an intuition for the next definition.  We are considering monotone functions and thus for every graph  
there is an integer  where the answer flips. However, for our purpose we need a corresponding notion for 
boundaried graphs.   If we think of the representatives as some ``small perturbation'' , then it is the max threshold over all small perturbations (``adding a representative = small perturbation''). This leads to the following definition. 

\begin{definition}
Let  be a monotone parameterized graph problem that has FII. Let    be
a set containing exactly one progressive representative of each equivalence class of  that is a subset of 
, where .  
For a -boundaried graph , we define   

\end{definition}



The next lemma says the following. Suppose we are dealing with some FII problem and we are given a boundaried graph with constant size boundary.  We know it has some constant size representative and we want to find this representative. 
Now in general finding a representative for a boundaried graph is more difficult than solving the corresponding problem
The next lemma says basically that if ``OPT'' of a boundaried graph is low, then we can efficiently find its representative. 
Here by ``OPT''  we mean , which is a robust version of the threshold function (under adding a representative). 
And by efficiently we mean as efficiently as solving the problem on normal (unboundaried) graphs if we know that ``OPT'' is low. Specifically, the main result of this section is as follows. 

\begin{lemma}
\label{lem:red2finiteindex}
Let  be a monotone parameterized graph problem that has FII. Furthermore, let  be an 
algorithm for  that, given a pair , decides whether it is in  in time . 
Then for every  there exists a  (depending on  and ), and 
an algorithm that, given a -boundaried graph   with  outputs, in   steps,
a -boundaried graph   such that  and  . Moreover we can compute the translation 
constant   from  to  in the same time.
\end{lemma}

\begin{proof}
We give prove the claim for positive monotone problems ; the proof for negative monotone problems is identical. 
Let    be
a set containing exactly one progressive representative of each equivalence class of  that is a subset of 
, where , and  let   The set   is hardwired in the description of the algorithm. 
 Let  be the set of progressive representatives in . Let . Our objective is to find 
  a representative   for   such that 
 
Here,  is a constant  that depends on  and .  Towards this   
we define the following matrix for the set of representatives. Let 

The matrix  has constant size and is also hardwired in the description of the algorithm. 


Now given  we find its representative as follows. 
\begin{itemize}
\item Compute the following row vector G\oplus Y_1,\PiG\oplus Y_\rho,\Pi. For each  we decide whether  using the assumed algorithm for deciding 
,  letting  increase from  until the first time . Since  is positive monotone this will happen for some 
. Thus the total time to compute the vector  is . 

\item Find a translate row in the matrix . That is, find an integer  and a representative 
 such that  

Such a row must exist since  is  a set of representatives for ; the representative  for the equivalence class to which  belongs, satisfies the condition.  
\item Set  to be  and the translation constant to be .
\end{itemize}
From here it easily follows that . This completes the proof.  
\end{proof}
 We remark that the algorithm whose existence is guaranteed by the Lemma~\ref{lem:red2finiteindex} assumes that the set   of representatives  are hardwired in the algorithm.  In its full generality we currently don't known of a procedure that for problems having FII outputs such a representative set. The application of Lemma~\ref{lem:red2finiteindex}  makes our kernelization algorithm non-constructive.  












\section{Generalized  Protrusions and Slice-Decomposition}\label{sec:slicedecs}


In this section our objective is to show that in polynomial time we can  
partition the graph   into parts that satisfy certain properties.  
To obtain  our decomposition we need to use a more general  notion of 
protrusion. Recall that a protrusion in a graph  is a subgraph of constant treewidth which is separated from the rest of the graph by at most a constant number of vertices. In our variant of protrusions, instead of stipulating that the subgraph be of constant {\em treewidth}, we ask that it contains a {\em constant number of vertices from a solution}. More precisely, we need the following kind of protrusions. 
\begin{definition}{\rm [\bf -{\sc DS}-protrusion]} 
 Given a graph , we say that a set  is an {\em -{\sc DS}-protrusion} of  if 
   the number of vertices in  with a neighbor in  is at most  and there exists a 
   subset  of size at most  such that   is a dominating set of . 
\end{definition}





The notion of a -{\sc DS}-protrusion  differs from a protrusion in the following way. In a 
protrusion   is at most , while in a -{\sc DS}-protrusion we require that the dominating set of the graph induced by   is small. In the case of a -{\sc DS}-protrusion, the treewidth could be unbounded. One can similarly define the notion of a --protrusion for other graph problems . Next we define a 
-{\sc CDS}-protrusion. 

\begin{definition}{\rm [\bf -{\sc CDS}-protrusion]} 
 Given a graph , we say that a set  is an {\em -{\sc CDS}-protrusion} of  if 
   the number of vertices in  with a neighbor in  is at most  and there exists a 
   subset  of size at most  such that  for every connected component  of  we have that 
     is connected and is a dominating set for . 
    \end{definition}

A natural question is what can we do if we get a large -{\sc DS}-protrusion (or -{\sc CDS}-protrusion).  The next lemma shows that in that case we can replace it with an equivalent small graph. 
We will also need the following. Let  be a graph class.  Given a  parameterized graph problem   and a graph class  
we denote by  the problem obtained by 
removing from  all instances that 
encode graphs that do not belong to   Our result is as follows. 






\begin{lemma}
\label{lem:fiidomset}
Let  be a fixed graph. For  every  there exist a  (depending on \tDS \, (\tCDS),  and ),  and 
an algorithm  such that given a -{\sc DS}-protrusion  (-{\sc CDS}-protrusion) with boundary  
,  and , 
 outputs in    time ( time),  
a -boundaried graph   such that   () and   () and . Moreover in the same time 
we can also find the translation constant  from   to . 



\end{lemma}

\begin{proof}
Let  be the class of  graphs that excludes  as a topological minor. 
For every  let  be the constant as defined in Lemma~\ref{lem:red2finiteindex}. It is known 
that  \tDS (\tCDS ) is FII\cal~\cite{H.Bodlaender:2009ng} and monotone (see~\cite[Lemmas 7.3 and 7.4]{H.Bodlaender:2009ng}). Furthermore, \tDS \, and \tCDS \, can be solved in time  ~\cite[Theorem 4]{AlonG09} 
and  ~\cite[Theorem 1]{GolovachV08}  respectively.  Here,  and  is the parameter in the definitions of  \tDS \, and \tCDS. We  use these algorithms in Lemma~\ref{lem:red2finiteindex} with the parameter value being . That is, . 
Thus, if   then 
by  Lemma~\ref{lem:red2finiteindex} in time    (), we can obtain a 
-boundaried graph   such that  (),   and 
.   The last assertion that   follows from the fact that  \tDS is FII and thus all the graphs in the set of representatives with respect to  belong to . 
Moreover, in the same time 
we can also find the translation constant  from   to  as done in Lemma~\ref{lem:red2finiteindex}. 

Let   be the class of  graphs that excludes a fixed graph  as a minor. It is known 
that  \tDS (\tCDS ) is FII\cal~\cite{H.Bodlaender:2009ng} and monotone. Thus, as in the case of , we can  obtain a 
-boundaried graph   such that  (),   and 
.
\end{proof}







\begin{center}
\fbox{
\parbox{.95\textwidth}{
   Throughout this section we work on a graph   that excludes a fixed graph  as a topological minor. Here,  will denote . 
   
   Furthermore, we assume that we have a (connected) dominating set  such that the size of  is at most -factor away (-factor away) from the size of an optimal (connected) dominating set of , obtained by applying Lemma~\ref{lemma:approximation}  (Lemma~\ref{lemma:approximationcds}) on the input graph .   }
}
  \end{center}

Let  be a tree-decomposition of a graph . For a subtree  of , we define  as the set of edges in  such that it has exactly one endpoint in . Furthermore we define  and 
\begin{center}
.
\end{center}
In plain words,  denotes the union of bags corresponding to the nodes in  and  is the graph induced on   with ``external adhesions'' being torsoed. 

Our main objective in this section is to obtain the following {\em-slice decomposition} for . 

\begin{definition}{\rm [{\bf -slice decomposition]}}
Let  be a fixed graph and let  be a graph with .  Let   be the 
tree-decomposition given by Theorem~\ref{thm:structure theorem}. An {\em-slice decomposition} of a graph 
is a collection  of pairwise vertex disjoint  subtrees   of  such that the 
following hold. 

\begin{itemize}
\item 
\item There exists a graph  whose size only depends on , such that each  is either -minor-free
or  has at most  vertices of degree at least .

\item 


\end{itemize}
We refer to the sets    as the {\em slices} of 
\end{definition}

Essentially, the slice decomposition allows us to  partition the input graph  into subgraphs , such that 
 ; for every , the neighbourhood ,  and . To see this consider an instance  of , where  excludes a fixed graph  as a topological minor.  Now obtain an   -slice decomposition for  for . We take 
 
and . One can easily verify that this partition of  satisfies the stated properties.  
 This is the decomposition we were talking about in the introduction. 
 
 Now we give a definitions that is useful in our procedure to find  the slice decomposition. 
\begin{definition}
Let  be the tree-decomposition of a graph  given by  Theorem~\ref{thm:structure theorem}. For a subset 
 and a subtree  of  we define .  
\end{definition}

Let  be the tree-decomposition of a graph  given by  Theorem~\ref{thm:structure theorem}.   
If we delete an edge  from the tree  then 
we get two trees. We call the {\em trees as  and  based on whether they contain  or . } 

\begin{definition}
Let  be the tree-decomposition of a graph  given by  Theorem~\ref{thm:structure theorem} and  be the assumed dominating (connected) set of .  We call a tree edge  
{\em heavy} if  and . We use  to denote the set of heavy edges.  
\end{definition}







Our main lemma in this section shows that in polynomial time we can find an -slice decomposition or 
a large -{\sc DS}-protrusion (or -{\sc CDS}-protrusion) or a large protrusion. In the latter cases we can apply either 
Lemma~\ref{lem:fiidomset} or a similar lemma developed in~\cite[Lemma~7]{H.Bodlaender:2009ng} for protrusions and reduce the graph.  






Before we prove the main result of this section, we prove some combinatorial properties of the set .  Given , by {\em subgraph of  formed by the edges in } we mean a subgraph of  whose vertex set consists of the end points of edges in  and the edge set is . 

\begin{lemma}
Let  be the subgraph of  formed by the edges in . Then  is a subtree of .
\end{lemma}
\begin{proof}
Clearly,  is a forest as it is a subgraph of .  To complete the proof we need to show that it is connected. We prove 
this using contradiction.  Suppose  is a forest and   and , , are two maximal subtrees in . Then we know that there exists a path  in  
such that the first and the last edges are heavy and the path  contains a light edge. 
Furthermore, we can assume that the first edge, say
, belongs to  and the last edge, say   belongs to . Let a light edge on the path be . Now when we delete the 
edge  from  we get two trees  and . We know that either  and  or vice versa. 
Suppose  and . Since  contains the heavy edge  we have that 
. Similarly we can show that . This shows that  is a heavy edge, contradicting that  is light. One can similarly argue that  is a heavy edge  when   and . This contradicts our assumption that  is not a subtree of . This completes the proof of the lemma.
\end{proof}



For our next proof we first give some well known observations about trees. Given a tree , 
we call a node {\em leaf}, {\em link} or {\em branch} if its degree 
in  is ,  or  respectively. Let
 be the set of branch nodes, 
be the set of link nodes and  be the set of leaves in the
tree . Let  be the set of maximal paths
consisting entirely of link nodes.

\begin{fact}
\label{fact:simplecounta}
 .
\end{fact}


\begin{fact}
\label{fact:simplecountb}
 .
\end{fact}
\begin{proof}
Root the tree at an arbitrary node of degree at least . If there is no node of degree  or more in  
then we know that  is a path  and the assertion follows. Consider  which is the disjoint union of paths
. With every path , we
associate the unique child in  of the last node of this path (furtherest from the root)  in
. Observe that this association is injective and the associated
node is either a leaf or a branch node. Hence 

follows from Fact .
\end{proof}

\begin{lemma}
\label{lemma:boundingleavesandpaths}
Let  be the subgraph formed by the edges in . 
If  is a yes instance of \tDS \, (\tCDS) then (a) ;  
(b) ; and 
(c)  .  
Here  is the factor of approximation in Lemma~\ref{lemma:approximation} (Lemma~\ref{lemma:approximationcds}). 
\end{lemma}
\begin{proof}
Root the tree at an arbitrary node  of degree at least  in . If there is no node of degree  
or more in  then we know that   is a path,  and the proof follows. 
We call a pair of nodes  and  {\em siblings} if they do not 
belong to the same path from the root  in . Observe that all the leaves of  are siblings. 

 Let  be an approximate solution to  () returned by applying Lemma~\ref{lemma:approximation} (Lemma~\ref{lemma:approximationcds}) on . Since   is a yes instance we have that . 
Let  be the leaves of  and let  be the  edges in  incident 
with   , respectively.  To prove our first statement we will show that for every , we have a vertex 
   such that  and for  all , . This will establish an 
 injection from the set of leaves to the dominating set  and thus the bound will follow. Towards this observe that 
 since the edge  is heavy,  we have that .  
Furthermore, for every pair of 
 vertices , , we have that 
 . The last assertion  follows from the fact that for a pair of siblings  and  
  the only vertices that  can be in the intersection of   and  must belong to both  and 
  . However, the size of any  is upper bounded by . This together with the fact that 
   implies that for every , we have a vertex   such that 
  and for  all , .  This implies that . However since  is a yes instance to \tDS \, we have that . This completes the proof of part (a)  of the lemma. 
Proofs for part (b) and part (c) of the lemma follow from Facts~\ref{fact:simplecounta} and \ref{fact:simplecountb}. 
\end{proof}


Before we prove our next lemma we show a lemma about dominating sets of subgraphs of .
\begin{lemma}
\label{lem:smalldominatingset}
Let  be a fixed graph and let  be a graph with .  
Let  be the tree-decomposition of  given by 
Theorem~\ref{thm:structure theorem} and let  be a dominating set of . If  is  a subtree of ,  then 
 is 
a dominating set for . 
\end{lemma}
\begin{proof}
The proof follows from the fact that  dominates all the vertices in  except possibly the ones that have 
neighbors in . Thus,  is 
a dominating set for . 
\end{proof}




Let  be the  paths in . We use  and  to denote the first  and the last vertices, respectively,  
of the path . Since  is a path consisting of link vertices,  we have that  and  have unique 
neighbors  and  respectively in .  Observe that since  is a subtree of , we have that for every 
,  is also a path in . 
If we delete the edges  and  from the tree , then there is a subtree of  that contains the path ; we call this subtree 
.  For any two vertices  and  on the path  we use  to denote the subpath between  and  in . 
Furthermore for  any subpath  , if we delete the edges incident to   and  on  and not present in   from the tree , then there is a subtree of  that contains the path ; we call this subtree  .   


Now we recall the definition of . Let  be a  monotone parameterized graph problem  that is FII. Then for every  there exists a  (depending on  and ), such that, given a -boundaried graph   with   there exists a -boundaried graph   such that  and  . 
In the next lemma we show that if any of the paths is ``too long'' then using a simple application of pigeonhole principle we can get a -{\sc DS}-protrusion. We use  to denote the number of vertices in the path .  

\begin{lemma}
\label{lem:boundingpaths}
Let  be an instance of  \tDS\ (\tCDS) and let  be the  paths in . Further, let  be a dominating set of . 
Then,  for some path , , if  then  contains a -{\sc DS}-protrusion (-{\sc CDS}-protrusion) of size at least .  Here,  . Furthermore, we can find such a  -{\sc DS}-protrusion (-{\sc CDS}-protrusion) in polynomial time. 
\end{lemma}
\begin{proof} 
Let  be the the path such that . 
Let 
. For 
every vertex 

we mark two vertices of the path . We mark the first and 
the last vertices on , say  and , such that  and .  That is, 
 and  and for all  or  we have that .  
This way we will only mark at most  vertices of the path . However the path is longer than 
  and thus by the pigeonhole principle we have that there exists a subpath of , say , such that no vertex of this subpath is marked and . 
 Let . Let  and  be the neighbors of   and  respectively 
that are not present on  
. Clearly, the only vertices in  that have neighbors in  belong to 
.  
Thus the vertices in  that have neighbors in   is upper bounded by . Furthermore, since no vertex on the path  is marked, we have that all the vertices in  belonging to  also belong to . Then by Lemma~\ref{lem:smalldominatingset}, we have that 
 dominates all the vertices in . Furthermore, in , 
no bag is contained in another and thus  (see discussion after Theorem~\ref{thm:structure theorem}). This shows that  is a -{\sc DS}-protrusion of the desired size.
\end{proof}


The final result of this section is the following decomposition lemma. 



\begin{lemma}
\label{lem:slicedeco}
Let  be a fixed graph and  be the class of graphs that excluds a fixed graph  as a topological minor.  
Then there exist two constants  and  (depending on \tDS \, (\tCDS)) 
such that given a yes instance  of \tDS \, (\tCDS), in polynomial time,  we can either find 
\begin{itemize}
\item a -slice decomposition; or 
\item a -{\sc DS}-protrusion (or -{\sc CDS}-protrusion) of size more than  or;
\item an -protrusion of size more than  where  depends only on .
\end{itemize}
\end{lemma}
\begin{proof}
Let  be a yes instance of \tDS\  (\tCDS). This implies that the size of the (connected) dominating set  returned by Lemma~\ref{lemma:approximation} (Lemma~\ref{lemma:approximationcds}) is at most . 
Let  be the subtree of  formed by  heavy edges. By Lemma~\ref{lemma:boundingleavesandpaths}, we know that 
\begin{itemize}
\item[(a)] ; 
\item[(b)] ; and
\item[(c)] .  
 \end{itemize}
 Recall that for every path  , we defined . If for any path  we have that  then by Lemma~\ref{lem:boundingpaths}   contains a -{\sc DS}-protrusion of size at least , and we can find this protrusion in polynomial time. Thus we assume that for all paths  we have that . 

Let  denote the number of vertices in  that are not present in any other 
 for . Furthermore, for all  we have that 

The last assertion is based on the following arguments. The sets  and   can be separated by a separator of size at most  and the vertices of  that appear in  both sets are present in this separator. 
Observe that . 
This implies that 

Let . 
This implies that the number of heavy edges is upper bounded by . Let  be the subtrees of  obtained by deleting all the edges in , that is, by deleting all the edges in ,  see Fig.~\ref{illus-decompos} for an illustration. Note that   We now argue that either the collection  forms a -slice decomposition of  or we have found a -protrusion or a -{\sc DS}-protrusion of size more than  in . 


\begin{figure}[t]
\begin{center}
\includegraphics[scale=0.33]{slicedeompillus}
\end{center}
 \caption{\label{illus-decompos} An illustration of the decomposition. The heavy edges are shown in red.}
\end{figure}

First we show that  
Note that by construction, each   is a heavy edge. Now observe that each   belongs to at most  distinct edge sets , we have that  
We set , and . Since  we have that  is a constant; indeed . 



Since  is connected we have that for every tree  there is a unique node in  that is incident with  edges in . We denote this special node by . We root the tree  at . Let  be a child of  in  and let   and  be the subtrees of  obtained after deleting the edge . Since at least one edge incident with   is heavy we have that . However the edge  is not heavy and thus it must be the case that . Let . Then  by Lemma~\ref{lem:smalldominatingset}, we have that   is a dominating set of size at most  for . Furthermore, the only vertices in  that have neighbors in  belong to  and thus its size is also upper bounded by . This implies that if  then it is a  -{\sc DS}-protrusion of size at least . Thus  from now onwards we assume that this is not the case. This implies that for every subtree rooted at  and every child  of   we have that 
 . Next we look at   and based on its type.  Recall from Theorem~\ref{thm:structure theorem} that they are of the following types. 
 
 \medskip
 
 \noindent 
 {\bf Case 1:    has at most  vertices of degree larger than .} 
In the case 
we show that there exists an  depending only on  such that either  has at most  vertices of degree larger than ,  or  contains an -protrusion of size more than . Here, . Suppose some vertex  in  has degree at most  in  , but has degree at least  in . Let  be the closed neighbourhood of  in  and  be the neighborhood of  in . 
Each vertex in  must lie in a connected component  of  on at most  vertices. Towards this, observe that no vertex in  sees any vertex outside  even in the graph .  Thus, if  we will get  -{\sc DS}-protrusion. Let  be  plus the union of all such components. By assumption  and hence . Finally, the only vertices in  that have neighbors outside of  in  are in , and . The treewidth of  is at most  since removing  from  leaves components of size . Thus  is an -protrusion of size more than . If no such  exists it follows that every vertex of degree at most  in  has degree at most  in . The vertices of   that are not in  have degree at most . Thus    has at most  vertices of degree at least .

\medskip
 \noindent 
 {\bf Case 2:    is -nearly embedded in a surface  in which  cannot be embedded.} 
In the case  we have that  excludes some graph  depending only on  as a minor. The graph  can be obtained from   by joining constant size graphs (of size at most ) to vertex sets that form cliques in . Thus there exists a graph  depending only on  such that  excludes  as a minor. This completes the proof of this lemma. 
\end{proof}








\section{Kernelization Algorithm for \tDS}
\label{sec:domset_kernel}
In this section we use the slice decomposition obtained in the last section 
to obtain linear kernels for  \tDS \ and in the next section outline an algorithm for \tCDS.



Given an instance  of \tDS \,  we first apply Lemma~\ref{lemma:approximation} and find  a dominating set  of . 
If  we return that  is a {\sc no} instance of \tDS. Else, 
we apply Lemma~\ref{lem:slicedeco} and
\begin{itemize}
\item  either find  a -slice decomposition; or 
\item a -{\sc DS}-protrusion  of  
of size more than ; or
\item a -protrusion of size more than  where  depends only on .
\end{itemize}
In the second case we apply Lemma~\ref{lem:fiidomset}. Given , by making use of  Lemma~\ref{lem:fiidomset}, we obtain a boundaried graph  such that  and .  
We also compute the translation constant  between  
and .  Now we replace the graph  with  and obtain a new equivalent instance . See  Definition~\ref{defn:replacement} for the notion of replacement. (Recall that  is a non-positive integer.) In the third case we apply the protrusion replacement lemma of~\cite[Lemma~7]{H.Bodlaender:2009ng} to obtain a new equivalent instance  for  with . We repeat this process until Lemma~\ref{lem:slicedeco}  returns a slice decomposition. 
For simplicity we denote by  itself the graph on which Lemma~\ref{lem:slicedeco} returns the slice decomposition. Since the number of times this process can be repeated is upper bounded by , we can obtain a -slice decomposition for  in polynomial time.   




Let  be the pairwise vertex disjoint  subtrees   of  coming 
from the slice decomposition of . Recall that .
Let  ,  and .  
In this section we will treat   as a  graph with boundary .  
Observe that by Lemma~\ref{lem:smalldominatingset}, it follows that 
 is a dominating set for . 

We have two kinds of graphs . In one case we have that  is -minor-free for a graph  whose size  depends only on . In the other case we have that the graph  has at most  vertices of degree at least . To obtain our kernel we will show the following  two lemmas. 

\begin{lemma}
\label{lem:newperspectivequasi}
There exists a constant  such that if    is a graph with boundary  such that  is a dominating set for  and  has at most  vertices of degree at least , then in polynomial time, we can obtain a graph  with boundary  
such that  
Furthermore we can also compute the translation constant  of  and  in polynomial time. 
\end{lemma}

The second lemma is for \Hmf \, graphs. 


\begin{lemma}
\label{lem:newperspectiveHminor}
There exists a constant  such that  given an \Hmf \, graph   with boundary  
such that  is a dominating set for  we can obtain, in polynomial time, a graph  with boundary  such that 
 
Furthermore we can also compute the translation constant  of  and  in polynomial time. 
\end{lemma}

Once we have proved Lemmas~\ref{lem:newperspectivequasi} and \ref{lem:newperspectiveHminor}, 
we construct the linear sized kernel for \tDS \, as follows. Given the graph  we 
obtain the slice decomposition and check if any of  has size more than . (Recall that 
 and .) 
 If yes then we either apply  
Lemma~\ref{lem:newperspectivequasi} or Lemma~\ref{lem:newperspectiveHminor} based on the type of  
and obtain a graph  such that  . We think 
, where  as a -boundaried graph with 
boundary . Then we obtain  a smaller equivalent graph  and .  After this we can 
repeat the whole process once again. This implies that when we cannot apply Lemmas~\ref{lem:newperspectiveHminor} or  
\ref{lem:newperspectivequasi}  on  we have that each of 
.  Furthermore notice that by the definition of the slice decomposition we have that . This implies that in this case we have the 
following

The last inequality follows from the fact that  is upper bounded by the second component of the slice decomposition and   is upper bounded by the size of the approximate dominating set . This brings us to the following theorem. 
\begin{theorem}
\label{thm:lineardomsettopo}
\tDS \, admits a linear kernel on graphs excluding a fixed graph  as a topological minor. 
\end{theorem}

It only remains to prove Lemmas~\ref{lem:newperspectivequasi} and~\ref{lem:newperspectiveHminor} to complete the proof of 
Theorem~\ref{thm:lineardomsettopo}. 


\subsection{Irrelevant Vertex Rule and proofs for  Lemmas ~\ref{lem:newperspectivequasi} and \ref{lem:newperspectiveHminor}}
For the proofs of  Lemmas~\ref{lem:newperspectivequasi} and~\ref{lem:newperspectiveHminor} we will introduce a reduction rule that removes irrelevant vertices. 
If the graph  is -minor-free then the irrelevant vertex rule  will be used in a recursive fashion. 
In each recursive step it is used in order to reduce the treewidth of torsos and hence also the entire graph. Then the graph 
is split in two pieces and the procedure is applied recursively to the two pieces. In the leaf of the recursion tree when the graph becomes 
smaller but still big enough then we apply Lemma~\ref{lem:fiidomset} on it and obtain an equivalent instance. 




Let  be  a graph given with its  tree-decomposition  
as described in Theorem~\ref{thm:structure theorem}, and  be one of its torsos. Let  be a dominating set of , and , , be the set of apices of . The reduction rule essentially ``preserves'' all dominating sets of size at most  in , without introducing any new ones. 
To describe the reduction rule we need several definitions. 
The first step in our reduction rule is to classify different subsets  of  into feasible and infeasible sets. The intuition behind the definition is that a subset  of  is feasible if there exists a set  in  of size at most   such that  dominates all but  and   . However, we cannot test in polynomial time whether such a set  exists. We will therefore say that a subset  of  is {\em feasible} if the -approximation for \tDS \, (Lemma~\ref{lemma:approximation}) 
outputs a set  of size at most  such that   dominates  and . Observe that if such a set  of size at most  exists then  is surely feasible in the first sense, while if no such set  of size at most  exists, then  is surely not feasible (again in the first sense). We will frequently use this in our arguments. Let us remark that there always exists a feasible set . In particular,  is feasible since  dominates . For feasible sets  we will denote by  the set  output by the approximation algorithm.

For every subset , we select a vertex  of  such that . If such a vertex exists,  we call it a \emph{representative} of . 
Let us remark that some sets can have no representatives and some distinct subsets of  may have the same representative. 
We define  to be the set of representative vertices for subsets of . The size of  is at most . For ,  the set of {\em dominated vertices} (by ) is . We say that a  vertex  is {\em fully dominated} by  if . A vertex  is {\em irrelevant with respect to } if , ,  and  is fully dominated by . 

Now we are ready to state the irrelevant vertex rule. 
\begin{description}
\item[Irrelevant Vertex Rule:] If a vertex  is irrelevant with respect to every feasible , then delete  from . 
\end{description}



 \begin{lemma}
 \label{lem:domseteqiv1}
Let  be a dominating set in , and  be the graph obtained by applying the Irrelevant Vertex Rule on , where  was the deleted vertex. Then .
\end{lemma}
 \begin{proof}
 We view  and  as graphs with boundary . 
 Let the transposition constant be . To prove that , we show that given a -boundaried graph  and a positive 
 integer  we have that . 
  Let   be a dominating set for  of size at most .  Let . If  then 
   is a smaller dominating set for . Therefore we assume that .  Let , and observe that  is feasible because   dominates all but .  If , then  is a dominating set of size at most  for 
  . So assume . Observe that  and  and therefore all the neighbors of  lie in . Since  is irrelevant with respect to all feasible subsets of  and  is feasible, we have that  is irrelevant with respect to . Hence 
  . There is a representative ,  (since ), such that . Hence   is a dominating set of 
   of size at most . 
  
Now, let   be a dominating set of size at most  for .  Let  . As in the forward direction we can assume that . We show that  also dominates  in . Specifically  is a set dominating all but  in  of size at most  so  is feasible. Since  is irrelevant with respect to , we have  and thus  is a dominating set for  of size at most . This concludes the proof. 
 \end{proof}

For a graph  and its dominating set , we apply the Irrelevant Vertex Rule exhaustively on all torsos of , obtaining an induced subgraph  of . By Lemma~\ref{lem:domseteqiv1} and transitivity of  we have that . We now prove that a graph  which can not be reduced by the irrelevant vertex rule has a property that each of its torso has a small -dominating set. 



\begin{lemma}
\label{lem:dstwbound} Let  be a graph which is irreducible by the Irrelevant Vertex Rule and  be a dominating set of . 
For every torso  of the tree-decomposition  of , we have that  
has a -dominating set of size . Furthermore if  is a \Hmf \, graph then . 
\end{lemma}
\begin{proof}
Let , where  are the apices of . We will obtain a -dominating set of size  in . Towards this end, consider the following set,  
The number of   representative vertices  and the number of feasible subsets  
 is at most , where  is a constant depending only on . The size of  is at most  for every . Thus  . We prove that  is a -dominating set of . Let . If  or , then  dominates . So suppose . Then, since  is not irrelevant, we have that  there is a feasible subset  of  such that  is relevant with respect to . Hence  is not fully dominated by  and so  has a neighbour . But  is dominated by , and thus  is -dominated by  in . Hence  has a -dominating set of size .

The graph  can be obtained from  by contracting all edges in  and adding all edges in . Since contracting and adding edges does  not increase the size of a minimum -dominating set of a graph,  has a -dominating set of size . This completes the proof for the first part.

Now assume that  is a \Hmf \, graph. It is well known  that the treewidth of a \Hmf \, graph is at most the maximum treewidth of its torsos, see e.g.\cite{DemaineFHT05sub}. Thus to show that  it is sufficient to show that its torsos have small treewidth.  To conclude,  excludes an apex graph as a minor  (see, e.g.~\cite[Theorem ]{Grohe03}) and it has a -dominating set of size . By the bidimensionality of -dominating set, we have that ~\cite{DemaineFHT05sub,FominGT09con}. Now we add all the apices of  to all the bags of the tree-decomposition  of  to obtain a tree-decomposition for  of width   .  
\end{proof}

Let us also remark that Irrelevant Vertex Rule is based on the performance of a polynomial time approximation algorithm. Thus by 
Lemmas~\ref{lemma:approximation}, \ref{lem:domseteqiv1} and ~\ref{lem:dstwbound}, and the fact that the treewidth of a graph is at most the maximum treewidth of its torsos, see e.g.\cite{DemaineFHT05sub}, we obtain the following lemma.
\begin{lemma}
\label{lem:sumreductiondomset}
There is a polynomial time algorithm that for  a given graph  and a dominating set  of , outputs graph  such that  and for every torso  of the tree-decomposition  of , we have that  has a -dominating set of size . Furthermore if  is a \Hmf \, graph then . 
\end{lemma}

Before we proceed further, we show the power of Lemma~\ref{lem:sumreductiondomset} by deriving a simple subexponential time algorithm for \tDS \, on \Hmf \, graph. This is one of the cornerstone results in~\cite{DemaineFHT05sub} and is based on a non-trivial two-layer dynamic programming over clique-sum decomposition tree of a \Hmf \, graphs. 
Lemma~\ref{lem:sumreductiondomset} can be used to obtain much simpler algorithm. Given a graph  and a positive integer  we first apply a factor -approximation algorithm given in~\cite{DemaineHaj05,FominLRS10} for \tDS \, on  and obtain a set . If the size of  is more than  then we return that  does not have a dominating set of size at most . Otherwise,  we apply Lemma~\ref{lem:sumreductiondomset} and obtain an equivalent graph  such that . Now applying a constant factor approximation algorithm developed in~\cite{DemaineFHT05sub} for computing the treewidth on  we get a tree-decomposition of width  . It is well known that checking whether a graph with treewidth  has a dominating set of size at most  can be done in time  ~\cite{RooijBR09}. This together with the above bound on the treewidth, gives us an alternative proof of the following theorem.
\begin{theorem}[\cite{DemaineHaj05}]\label{THM:Demaine}
Given an -vertex  graph  excluding a fixed graph  as a minor, one can check whether  has a dominating set of size at most  in time . 
\end{theorem}



Having Lemma \ref{lem:sumreductiondomset} proving Lemma~\ref{lem:newperspectivequasi} becomes simple.

\begin{proof}[Proof of Lemma~\ref{lem:newperspectivequasi}]
We apply Lemma~\ref{lem:sumreductiondomset} to  with a decomposition that has a single bag containing the entire graph and the apices  of the bag being the vertices of degree at least . By Lemma~\ref{lem:sumreductiondomset},  has a -dominating set of size . Since all vertices of  have degree at most  it follows that . 
\end{proof}

We need the following well known lemma, see e.g.\ \cite{Bodlaender98}, on separators in graphs of bounded treewidth for the proof of Lemma~\ref{lem:newperspectiveHminor}. 
\begin{lemma}
\label{lemma:balsep1}
Let  be a graph given with a tree-decomposition of width at most  and 
 
be a weight function. 
Then  in polynomial time we can find a bag  of the given tree-decomposition 
such that for every connected component  of , . Furthermore, 
the connected components  of  can be grouped into two sets 
 and  such that 
, for . \end{lemma}

\begin{proof}[Proof of Lemma~\ref{lem:newperspectiveHminor}]
By  we denote the graph with boundary . 
By Lemma~\ref{lem:sumreductiondomset},  we may assume that . We prove the lemma using induction on . 
If  we are done, as in this case we know that  is a -{\sc DS} protrusion. Thus, if  then 
 we can apply Lemma~\ref{lem:fiidomset} and in polynomial time obtain a graph  such that  and 
 . In the same time we can compute the translation constant depending on  and  
 and return it. Thus, we return   and the translation constant . 
 



Otherwise, using a constant factor approximation of treewidth on \Hmf \, graphs~\cite{FeigeHajLee08}, we compute a tree-decomposition of  of width , for some constant . Now, by applying Lemma~\ref{lemma:balsep1} on this decomposition, we find a partitioning of  into ,  and  such that there are no edges from  to , , and  for . Let 
. Observe that  is also a dominating set. 


Let  and .  Let  and . 
We now apply the algorithm recursively on 
 and  and obtain graphs 
,   such that for ,  . Let  and  be the translation 
constants returned by the algorithm.  Since , we have that  is   a dominating set of  and hence we actually can run the algorithm recursively on the two subcases. The algorithm returns  and  and translation constants  and . Let 
 and .  We will show that . Let  be a graph with boundary  and  be a positive integer.  Then

This proves that .  Now we will show that  . 

Let  be the largest possible
size of the set  output by the algorithm when run on a graph  with a
dominating set . We
upper bound  by the following recursive formula.
 
Using simple induction one can show that the above solves to . See for an example~\cite[Lemma~]{FominLRS10}. Hence we conclude that . This completes the proof of the lemma. 
\end{proof}

The algorithm of Demaine et al. \cite{DemaineHaj05} computing a dominating set of size  in an  -vertex  \Hmf \, graph uses exponential (in ) space
. 
Theorem~\ref{thm:lineardomsettopo} implies almost directly the following refinement of Theorem~\ref{THM:Demaine}.
  \begin{theorem} 
 Given an -vertex  graph  excluding a fixed graph  as a minor,  one  can check whether  has a dominating set of size at most  in time 
 and space . 
\end{theorem}
\begin{proof} 


Our algorithm first applies Theorem~\ref{thm:lineardomsettopo} to obtain a graph with  vertices.
Now we are assuming that the number of vertices in  is . 
We solve a slightly more general version of domination, where we are   given a subset  and the requirement is to find a set  of size at most    such that for every , . When , the set  is a dominating set of size .
By the separator theorem of Alon et al.  \cite{AlonST90}
 for \Hmf \, graphs, one can find in polynomial time a partition of  into ,  and  such that , there are no edges from  to  and  for . The algorithm finds such a partition and guesses how  interacts with . 

In particular, first the algorithm correctly guesses  (by looping over all subsets of ). For each guess, it puts  into  and removes  and  from  (these vertices are already dominated and will not be used in the future to dominate even more vertices). For every remaining vertex  in , the algorithm guesses whether it will be dominated by a vertex in , in which case the algorithm deletes all edges from  to vertices in , or by a vertex in , in which case the algorithm deletes all edges from  to vertices in . Let  be  plus all the vertices in  that we guessed were dominated from . At this point  and  are distinct components of the instance and can be solved independently. The running time is governed by the following recurrence.

The space used is clearly polynomial. This concludes the proof.
\end{proof}




\section{Kernelization algorithm for \tCDS}\label{sec:CDSkernel}
The kernelization algorithm for \tCDS \, is   similar to \tDS---we also  use slice decomposition to obtain a 
linear kernel.  However, the irrelevant vertex rule is a bit different. 
The  kernelization algorithm for \tCDS \, follows from the  results 
analogous to  Lemmas~\ref{lem:newperspectiveHminor} and \ref{lem:newperspectivequasi} for \tDS.  For completeness we spell out all the steps. 


In particular given an instance  of \tCDS \,  we first apply Lemma~\ref{lemma:approximation} and find  a dominating set  of . 
If  we return that  is a {\sc no} instance of \tCDS. Else, 
we apply Lemma~\ref{lem:slicedeco} and
\begin{itemize}
\item  either find  -slice decomposition; or 
\item a -{\sc CDS}-protrusion of size more than ; or
\item a -protrusion of size more than  where  depends only on .
\end{itemize}
In the second case we apply Lemma~\ref{lem:fiidomset}. For a given , we apply Lemma~\ref{lem:fiidomset} and construct a boundaried graph  such that  and . We also compute the translation constant  between  
and .  Now we replace the graph  with  and obtain a new equivalent instance , here we remind that  is a  non-positive integer. In the third case we apply the protrusion replacement lemma of~\cite[Lemma~7]{H.Bodlaender:2009ng} to obtain a new equivalent instance  for  with . We repeat this process until Lemma~\ref{lem:slicedeco}  returns a slice decomposition. 
For simplicity we denote by  itself the graph on which Lemma~\ref{lem:slicedeco} returns the slice decomposition. The number of times this process can be repeated  does not exceed   and a  -slice decomposition for  is constructed  in polynomial time.   

 The pairwise disjoint connected subtrees   of  coming 
from the slice decomposition of  is denoted by   and we put  .
We define   ,  and .  
As in the previous section,  we   treat   as a  graph with boundary .  
Then by Lemma~\ref{lem:smalldominatingset},  
 is a dominating set for . 

For  two kinds of graphs , we use different reductions. 
 In the first case we have that the graph  has at most  vertices of degree at least .
\begin{lemma}
\label{lem:newperspectivequasicds}
There exists a constant  such that if    is a graph with boundary   such that  is a dominating set for  and  has at most  vertices of degree at least , then in polynomial time, we can obtain a graph  with boundary  
such that  
Furthermore we can also compute the translation constant  of  and  in polynomial time. 
\end{lemma}
  In the other  case we have that  is -minor-free for a graph  whose size only depends on .  \begin{lemma}
\label{lem:newperspectiveHminorcds}
There exists a constant  such that  given an \Hmf \, graph   with boundary  
such that  is a dominating set for , in polynomial time, we can obtain a graph  with boundary  such that 
 
Furthermore we can also compute the translation constant  of  and  in polynomial time. 
\end{lemma}


In order to obtain the linear sized kernel for \tCDS \, the proof of  
 Lemmas~\ref{lem:newperspectivequasicds} and \ref{lem:newperspectiveHminorcds} suffices. 
 Indeed, for  graph  we 
obtain the slice decomposition and check if any   has size more than . If yes then we either apply  
Lemma~\ref{lem:newperspectivequasicds} or Lemma~\ref{lem:newperspectiveHminorcds} based on the type of  
and obtain a graph  such that  . We view 
, where  as a -boundaried graph with 
boundary . Then we obtain  a smaller equivalent graph  and .  After this we can 
repeat the whole process once again. This implies that when we can not apply Lemmas~\ref{lem:newperspectiveHminorcds} or  
\ref{lem:newperspectivequasicds}  on  we have that each of 
.  Furthermore notice that . This implies that  


Thus (subject to the proof of two lemmas) we have the following theorem. 
\begin{theorem}
\label{thm:lineardomsettopocds}
\tCDS \, admits a linear kernel on graphs excluding a fixed graph  as a topological minor. 
\end{theorem}


\subsection{Irrelevant Vertex Rule and proofs for  Lemmas~\ref{lem:newperspectivequasicds} and \ref{lem:newperspectiveHminorcds} }


As with \tDS, we will reduce  the treewidth of a torso not only in the beginning of the procedure but also when we apply it recursively. 
Let  be an \Hmf \,  graph,  be a dominating set of  (not necessarily connected),  be one of its torsos, and , ,  be the set of apices of , where  is some constant depending only on . We will define a reduction rule that essentially ``preserves" all dominating sets of size at most  with ``good enough'' connectivity properties, without introducing new such sets. Just as for \tDS{} we will say that a subset  of  is feasible if the factor -approximation for \tDS{} (Lemma~\ref{lemma:approximation})  concludes that there exists a set  of size 
at most  which   dominates  and . 
If such a set exists and  is feasible we denote this set by . 




Recall, that for \tDS \, we had the notion of a representative element for every subset . The representative vertex was crucially used in establishing  Lemma~\ref{lem:domseteqiv1}, where we used it to simulate all the domination properties of the deleted vertex ``''.  We need a similar notion of representatives for \tCDS, however here the representatives will be vertex subsets rather than single vertices. With vertex subsets we will be able to simulate not only   domination properties, but also the connectivity properties of an  irrelevant vertex. More precisely, for every subset , we compute a minimum size vertex set  such that  is connected and . If the size of such a minimum set is at most , then we say that  is a {\em representative} of , and add all the vertices in  to the set . Note that .
For each    we can test whether a representative exists in time  by making a modification of the algorithm for the  Steiner tree problem from~\cite{BjorklundHKK07}. Alternatively we can test it in time  by brute force. Let  denote the set of vertices in 
. Here  is the set of vertices at distance at most  from  in the graph  (not in ). 
The set of vertices \emph{covered} by  is . Note that a vertex in  is never covered by a set .  Let  denote the set of vertices   in  such that  has more connected components than . Observe that if  will be connected  then  is essentially the set of cut vertices. However, for a disconnected graph it is the union of cut vertices for each connected component. 

The definition of an irrelevant vertex with respect to  is   different than for \tDS{}. A vertex
 
  is called {\em irrelevant with respect to }, if .  The irrelevant vertex rule for \tCDS \, is exactly the same as in Section~\ref{sec:domset_kernel} for \tDS \, but the correctness proof and analysis is more complicated. Recall that a subset  of  is feasible if the factor -approximation for \tDS{} (Lemma~\ref{lemma:approximation})  concludes that there exists a set  of size 
at most  which dominates all but , such that .

\begin{description}
\item[Irrelevant Vertex Rule:] If a vertex  is irrelevant with respect to every feasible  then delete  from . 
\end{description}


 \begin{lemma}
 \label{lem:condomseteqiv1}
Let  be a dominating set in , and  be the graph obtained by applying the Irrelevant Vertex Rule on , where  was the deleted vertex.  Then . 
\end{lemma}
 \begin{proof}
 We view  and  as graphs with boundary . Let the transposition constant be . To show that , we show that given any boundaried graph  and a positive 
 integer  we have that .  Let   be a connected dominating set for  of size at most .  Observe that since  is a dominating set of , we have that there exists a connected dominating set   such that  (Proposition~\ref{lem:bb}). Let . If  then 
   is a smaller connected dominating set for . Thus, we assume that 
  .  Let , and observe that  is feasible since  dominates all but  and has size at  most . If , then  is a connected dominating set of size  for .  So assume . Since  is irrelevant with respect to  we have that . 

  



Let  be the connected component of  that contains .  Since,  is not a cut vertex of , we have the following easy observation. 

\begin{observation}
\label{claim:wisokay}
 is connected. 
\end{observation}


Let  be the connected dominating set of , . 
We will show that  has a connected dominating set of size at most  and that will show that 
.   Observe that since  and the only vertices that are common between  and  belong to , we have that 
. 

Let  be the vertex set of the connected component of  that contains . 
If  then there is a subset  such that , ,  is connected and 
.  Furthermore, since  we have that every connected component of  contains a vertex of . This implies that   is connected. Since ,  and  we have that . This implies that 
 and thus  is a connected dominating set of size at most  of  that avoids  and thus by Observation~\ref{claim:wisokay}, it  is also a connected dominating set of . This implies that in this case .   







Now suppose that . Let . The vertex set  is a dominating set of size at most  in the connected graph  and so  has a connected dominating set  that contains  of size at most . Let  be the connected component of  that contains . Notice that  and so there is a connected set  such that  and . Finally, let  be the set of vertices in  that are at distance exactly  from  in . Note that  (as every path from  to  a vertex in  has length at least ) and that . Set 
, and . We have that  while . Hence .  Note that  is connected. Furthermore by our choice of 
 we have that every connected component of  contains a vertex of 
 and hence a vertex of . However,  and  (or )  is connected and thus  
 is connected.  Observe that . This implies that 
 and thus  is a connected dominating set of size at most  of  that avoids  and thus by Observation~\ref{claim:wisokay} is also a connected dominating set of . This implies that in this case .   













Now we prove the reverse direction. Let    be a connected dominating set for  of size at most .  By Observation~\ref{claim:wisokay} we know that  and  are connected and thus  is also a connected dominating set of size at most  for . This concludes the proof.  
\end{proof}











 Next we prove an auxiliary lemma that upper bounds the number of cut vertices in terms of the dominating set of the graph. 
 
 \paragraph{Cuts and Blocks.} 
 A maximal connected subgraph without a cut vertex is called a \textbf{block}. Every block of a graph  is either a maximal -connected subgraph, or a bridge or an isolated vertex. By maximality, different blocks of  overlap in at most one vertex, which is then a cut vertex of . Therefore, every edge of  lies in a unique block and  is the union of its blocks.

\begin{definition} Let  denote the set of cut vertices of  and  the set of its blocks. The bipartite graph on  where  and  are adjacent when  is called the block graph of .
\end{definition}

\begin{proposition}[\cite{diestelbook}] 
\label{prop:blockgraphtree}
The block graph of a connected graph is a tree.
\end{proposition}

 \begin{lemma}
 \label{lem:cutvertexcount}
 Let  be a graph and  be a dominating set of , then the number of cut vertices in  is upper bounded by . That is, . 
 \end{lemma} 
\begin{proof}
Let  denote the set of cut vertices of  and  the set of its blocks. Consider the block graph  on . By Proposition~\ref{prop:blockgraphtree} we know that  is a tree. Now we root this tree at some vertex in . Observe that there is unique association of cut vertices to its parent -- which is a block.  
We also know that for every  cut vertex  that either  is in  or a vertex in its parent  block. However, the blocks are pairwise disjoint except for the vertices in . Thus, this implies that there is an injective map from  to  and hence . 
\end{proof}

Now we are ready to prove the treewidth bounding lemma of this section. 
 Just as for \tDS, it is possible to prove that after removing all irrelevant vertices, the treewidth of each torso in the reduced graph is . The most important difference is that instead of -dominating set we construct a -dominating set in the proof.   We start with the following auxiliary lemma that will be useful for the proof. 

 

\begin{lemma}
\label{lem:sumreductioncondomset}
There is a polynomial time algorithm that for  a given graph  and a dominating set  of , outputs graph  such that  and for every torso  of the tree-decomposition  of , we have that  has a -dominating set of size . Furthermore if  is a \Hmf \, graph then . 
\end{lemma}
\begin{proof}
Let , where  are the apices of . Also, let  denote the set of cut vertices of . 
We will obtain a -dominating set of size  in . Towards this end, consider the following set, 

The size of the set of representative vertices, , is at most . The number of feasible subsets  is at most , where  is a constant depending only on . The size of  is at most  for every . By Lemma~\ref{lem:cutvertexcount} we have that . Thus  . We prove that  is a -dominating set of . Let . If  or  or  then  dominates . So suppose . Then, since  is not irrelevant there is a feasible subset  of  such that  is relevant with respect to . Hence there exists a vertex  in  which is not in . If ,  denotes the set of vertices in , 
 then  is -dominated by a vertex   in . Otherwise  is dominated by some  in  and hence  is -dominated by  in . Hence  has a -dominating set of size .

The graph  can be obtained from  by contracting all edges in  and adding all edges in . Since contracting and adding edges can not increase the size of a minimum -dominating set of a graph,  has a -dominating set of size .
This completes the proof for the first part.

Now assume that  is a \Hmf \, graph. It is well known  that the treewidth of a \Hmf \, graph is at most the maximum treewidth of its torsos, see e.g.\cite{DemaineFHT05sub}. Thus to show that  it is sufficient to show that its torsos have small treewidth.  
To conclude,  excludes an apex graph as a minor  (see discussions after Theorem~\ref{thm:structure theorem}) and it has a -dominating set of size . By the bidimensionality of -dominating set, we have that ~\cite{DemaineFHT05sub,FominGT09con}. Now we add all the apices of  to all the bags of the tree-decomposition  of  to obtain a tree-decomposition for .  Thus . 

Let us also remark that Irrelevant Vertex Rule is based on the performance of a polynomial time approximation algorithm and thus the whole procedure can be implemented in polynomial time. This concludes the proof. 
\end{proof}







Having Lemma \ref{lem:sumreductioncondomset} proving Lemma~\ref{lem:newperspectivequasicds} becomes simple.

\begin{proof}[Proof of Lemma~\ref{lem:newperspectivequasicds}]
We apply Lemma~\ref{lem:sumreductioncondomset} to  with a decomposition that has a single bag containing the entire graph and the apices  of the bag being the vertices of degree at least . By Lemma~\ref{lem:sumreductioncondomset},  has a -dominating set of size . Since all vertices of  have degree at most  it follows that .  
\end{proof}
 



Proof for Lemma~\ref{lem:newperspectiveHminorcds} is identical to the proof of Lemma~\ref{lem:newperspectiveHminor}, except that we need to use Lemma~\ref{lem:sumreductioncondomset} in  
place of Lemma~\ref{lem:sumreductiondomset}. Thus we omit it. 
 



Recently, Bodlaender et al.~\cite{BodlaenderCKN13} obtained an algorithm solving \tCDS \ on graphs of treewidth  in time . Theorem~\ref{thm:lineardomsettopocds} combined with this implies that  \tCDS \,  on \Hmf \, graphs is solvable in time . To our knowledge, this is the first subexponential parameterized algorithm for \tCDS \, on \Hmf \, graphs.




\begin{theorem}
\label{thm:subexpcondomset}
Given an -vertex  graph  excluding a fixed graph  as a minor,  one  can check whether  has a connected dominating set of size at most  in time 
. 
\end{theorem}






\section{Conclusions}\label{sec:concludes}
In this paper we give linear kernels for two widely studied parameterized problems, namely \tDS\ and  \tCDS,
for every graph class that excludes some graph as a topological minor. The emerging questions are the following two:



\begin{enumerate}
\item Can our kernelization results for  \tDS\ and  \tCDS\  be extended to more general sparse graph classes?
\item Can our techniques be applied to more general families of parameterized problems?
\end{enumerate}




Very recently,  the first question was answered both positively and negatively by Drange et al. \cite{Drange2015}.  In particular, \tDS \ admits a vertex-linear kernel on graphs of bounded expansion and an almost vertex-linear kernel on nowhere-dense graphs. On the other hand  
  \tCDS  \ admits no polynomial kernel on graphs of bounded expansion unless \textsf{coNP} 
    \textsf{NP/poly}.  It is important to point out that methods used by Drange et al. \cite{Drange2015} is entirely different than ours. Their algorithm is completely combinatorial and do not rely on topological arguments. Our kernelization algorithm for \tCDS\ is still the best known. It would be interesting to see if the combinatorial methods developed in 
  Drange et al. \cite{Drange2015} could be used to design an explicit kernelization algorithm for \tCDS\ on graph classes excluding a fixed graph  as a topological minor. 
 


\paragraph{Acknoweldgements}
Thanks to Marek Cygan for sending us a copy of \cite{CyganGH12}. We sincerely thank all the reviewers for their insightful comments and suggestions. 



\begin{thebibliography}{10}

\bibitem{Adler11}
Isolde Adler, Stavros~G. Kolliopoulos, Philipp~Klaus Krause, Daniel Lokshtanov,
  Saket Saurabh, and Dimitrios~M. Thilikos.
\newblock Irrelevant vertices for the planar disjoint paths problem.
\newblock {\em J. Comb. Theory, Ser. {B}}, 122:815--843, 2017.

\bibitem{AlberBFKN02}
J.~Alber, H.~L. Bodlaender, H.~Fernau, T.~Kloks, and R.~Niedermeier.
\newblock Fixed parameter algorithms for dominating set and related problems on
  planar graphs.
\newblock {\em Algorithmica}, 33(4):461--493, 2002.

\bibitem{AlberFN04}
Jochen Alber, Michael~R. Fellows, and Rolf Niedermeier.
\newblock Polynomial-time data reduction for dominating sets.
\newblock {\em J. ACM}, 51:363--384, 2004.

\bibitem{AG08TechReport}
Noga Alon and Shai Gutner.
\newblock Kernels for the dominating set problem on graphs with an excluded
  minor.
\newblock Technical Report TR08-066, ECCC, 2008.

\bibitem{AlonG09}
Noga Alon and Shai Gutner.
\newblock Linear time algorithms for finding a dominating set of fixed size in
  degenerated graphs.
\newblock {\em Algorithmica}, 54(4):544--556, 2009.

\bibitem{AlonST90}
Noga Alon, Paul Seymour, and Robin Thomas.
\newblock A separator theorem for nonplanar graphs.
\newblock {\em J. Amer. Math. Soc.}, 3(4):801--808, 1990.

\bibitem{ArnborgCPS93}
Stefan Arnborg, Bruno Courcelle, Andrzej Proskurowski, and Detlef Seese.
\newblock An algebraic theory of graph reduction.
\newblock {\em Journal of the ACM}, 40:1134--1164, 1993.

\bibitem{BjorklundHKK07}
Andreas Bj\"{o}rklund, Thore Husfeldt, Petteri Kaski, and Mikko Koivisto.
\newblock {F}ourier meets {M}\"{o}bious: Fast subset convolution.
\newblock In {\em Proceedings of the 39th annual ACM Symposium on Theory of
  Computing (STOC 2007)}, pages 67--74. ACM Press, 2007.

\bibitem{Bodlaender98}
Hans~L. Bodlaender.
\newblock A partial {}-arboretum of graphs with bounded treewidth.
\newblock {\em Theoret. Comput. Sci.}, 209(1-2):1--45, 1998.

\bibitem{BodlaenderCKN13}
Hans~L. Bodlaender, Marek Cygan, Stefan Kratsch, and Jesper Nederlof.
\newblock Deterministic single exponential time algorithms for connectivity
  problems parameterized by treewidth.
\newblock {\em Inf. Comput.}, 243:86--111, 2015.

\bibitem{BDFH08}
Hans~L. Bodlaender, Rodney~G. Downey, Michael~R. Fellows, and Danny Hermelin.
\newblock On problems without polynomial kernels.
\newblock {\em J. Comput. Syst. Sci.}, 75(8):423--434, 2009.

\bibitem{H.Bodlaender:2009ng}
Hans~L. Bodlaender, Fedor~V. Fomin, Daniel Lokshtanov, Eelko Penninkx, Saket
  Saurabh, and Dimitrios~M. Thilikos.
\newblock (meta) kernelization.
\newblock {\em J. {ACM}}, 63(5):44:1--44:69, 2016.

\bibitem{BodlaenderH98}
Hans~L. Bodlaender and Torben Hagerup.
\newblock Parallel algorithms with optimal speedup for bounded treewidth.
\newblock {\em SIAM J. Comput.}, 27:1725--1746, 1998.

\bibitem{BodlaendervA01a}
Hans~L. Bodlaender and Babette van Antwerpen-de Fluiter.
\newblock Reduction algorithms for graphs of small treewidth.
\newblock {\em Information and Computation}, 167:86--119, 2001.

\bibitem{ChenFKX07}
Jianer Chen, Henning Fernau, Iyad~A. Kanj, and Ge~Xia.
\newblock Parametric duality and kernelization: Lower bounds and upper bounds
  on kernel size.
\newblock {\em SIAM J. Comput.}, 37:1077--1106, 2007.

\bibitem{CyganGH12}
Marek Cygan, Fabrizio Grandoni, and Danny Hermelin.
\newblock Tight kernel bounds for problems on graphs with small degeneracy.
\newblock {\em {ACM} Trans. Algorithms}, 13(3):43:1--43:22, 2017.

\bibitem{Cygan:2010bv}
Marek Cygan, Marcin Pilipczuk, Michal Pilipczuk, and Jakub~Onufry Wojtaszczyk.
\newblock Kernelization hardness of connectivity problems in d-degenerate
  graphs.
\newblock {\em Discrete Applied Mathematics}, 160(15):2131--2141, 2012.

\bibitem{Fluiter97}
Babette de~Fluiter.
\newblock {\em Algorithms for Graphs of Small Treewidth}.
\newblock PhD thesis, Utrecht University, 1997.

\bibitem{Dell:2010sh}
Holger Dell and Dieter van Melkebeek.
\newblock Satisfiability allows no nontrivial sparsification unless the
  polynomial-time hierarchy collapses.
\newblock {\em J. {ACM}}, 61(4):23:1--23:27, 2014.

\bibitem{DemaineFHT05talg}
Erik~D. Demaine, Fedor~V. Fomin, Mohammadtaghi Hajiaghayi, and Dimitrios~M.
  Thilikos.
\newblock Fixed-parameter algorithms for (k, r)-center in planar graphs and map
  graphs.
\newblock {\em ACM Trans. Algorithms}, 1(1):33--47, 2005.

\bibitem{DemaineFHT05sub}
Erik~D. Demaine, Fedor~V. Fomin, Mohammadtaghi Hajiaghayi, and Dimitrios~M.
  Thilikos.
\newblock Subexponential parameterized algorithms on bounded-genus graphs and
  {}-minor-free graphs.
\newblock {\em J. ACM}, 52(6):866--893, 2005.

\bibitem{Demaine:2009pd}
Erik~D. Demaine, Mohammad~Taghi Hajiaghayi, and Ken-ichi Kawarabayashi.
\newblock Approximation algorithms via structural results for apex-minor-free
  graphs.
\newblock In {\em Proceedings of the 36th International Colloquium on Automata,
  Languages and Programming (ICALP 2009)}, volume 5555 of {\em Lecture Notes in
  Comput. Sci.}, pages 316--327. Springer, 2009.

\bibitem{DemaineHaj05}
Erik~D. Demaine and Mohammadtaghi Hajiaghayi.
\newblock Bidimensionality: new connections between {FPT} algorithms and
  {PTAS}s.
\newblock In {\em Proceedings of the 16th Annual ACM-SIAM Symposium on Discrete
  Algorithms (SODA 2005)}, pages 590--601, New York, 2005. ACM-SIAM.

\bibitem{DemaineH07-CJ}
Erik~D. Demaine and Mohammadtaghi Hajiaghayi.
\newblock The bidimensionality theory and its algorithmic applications.
\newblock {\em The Computer Journal}, 51(3):332--337, 2007.

\bibitem{diestelbook}
Reinhard Diestel.
\newblock {\em Graph Theory, 4th Edition}, volume 173 of {\em Graduate texts in
  mathematics}.
\newblock Springer, 2012.

\bibitem{DornFLRS10}
Frederic Dorn, Fedor~V. Fomin, Daniel Lokshtanov, Venkatesh Raman, and Saket
  Saurabh.
\newblock Beyond bidimensionality: Parameterized subexponential algorithms on
  directed graphs.
\newblock {\em Inf. Comput.}, 233:60--70, 2013.

\bibitem{DowneyF98}
Rodney~G. Downey and Michael~R. Fellows.
\newblock {\em Parameterized Complexity}.
\newblock Springer, 1998.

\bibitem{DraganFG08}
Feodor~F. Dragan, Fedor~V. Fomin, and Petr~A. Golovach.
\newblock Spanners in sparse graphs.
\newblock {\em J. Comput. Syst. Sci.}, 77(6):1108--1119, 2011.

\bibitem{Drange2015}
P{\aa}l~Gr{\o}n{\aa}s Drange, Markus~Sortland Dregi, Fedor~V. Fomin, Stephan
  Kreutzer, Daniel Lokshtanov, Marcin Pilipczuk, Michal Pilipczuk, Felix Reidl,
  Fernando~S{\'{a}}nchez Villaamil, Saket Saurabh, Sebastian Siebertz, and
  Somnath Sikdar.
\newblock Kernelization and sparseness: the case of dominating set.
\newblock In {\em 33rd Symposium on Theoretical Aspects of Computer Science,
  {STACS} 2016, February 17-20, 2016, Orl{\'{e}}ans, France}, volume~47 of {\em
  LIPIcs}, pages 31:1--31:14. Schloss Dagstuhl - Leibniz-Zentrum fuer
  Informatik, 2016.

\bibitem{DruckerA12}
Andrew Drucker.
\newblock New limits to classical and quantum instance compression.
\newblock In {\em Proceedings of the 53rd Annual Symposium on Foundations of
  Computer Science (FOCS)}, pages 609--618. IEEE, 2012.

\bibitem{Duchet82}
P.~Duchet and H.~Meyniel.
\newblock On {H}adwiger's number and the stability number.
\newblock In {\em Graph theory ({C}ambridge, 1981)}, volume~62 of {\em
  North-Holland Math. Stud.}, pages 71--73. North-Holland, Amsterdam, 1982.

\bibitem{abs-1209-0129}
Zdenek Dvorak.
\newblock A stronger structure theorem for excluded topological minors.
\newblock {\em CoRR}, abs/1209.0129, 2012.

\bibitem{Dvorak13}
Zdenek Dvorak.
\newblock Constant-factor approximation of the domination number in sparse
  graphs.
\newblock {\em Eur. J. Comb.}, 34(5):833--840, 2013.

\bibitem{FeigeHajLee08}
Uriel Feige, MohammadTaghi Hajiaghayi, and James~R. Lee.
\newblock Improved approximation algorithms for minimum weight vertex
  separators.
\newblock {\em SIAM J. Comput.}, 38(2):629--657, 2008.

\bibitem{FellowsL89}
Michael~R. Fellows and Michael~A. Langston.
\newblock An analogue of the {M}yhill-{N}erode theorem and its use in computing
  finite-basis characterizations (extended abstract).
\newblock In {\em Proceedings of the 30th Annual Symposium on Foundations of
  Computer Science (FOCS)}, pages 520--525. IEEE, 1989.

\bibitem{FlumGrohebook}
J{\"o}rg Flum and Martin Grohe.
\newblock {\em Parameterized Complexity Theory}.
\newblock Texts in Theoretical Computer Science. An EATCS Series.
  Springer-Verlag, Berlin, 2006.

\bibitem{F.V.Fomin:2010oq}
F.~V. Fomin, D.~Lokshtanov, S.~Saurabh, and D.~M. Thilikos.
\newblock Bidimensionality and kernels.
\newblock In {\em Proceedings of the 21st Annual ACM-SIAM Symposium on Discrete
  Algorithms (SODA 2010)}, pages 503--510. ACM-SIAM, 2010.

\bibitem{F.V.Fomin:2012}
F.~V. Fomin, D.~Lokshtanov, S.~Saurabh, and D.~M. Thilikos.
\newblock Linear kernels for (connected) dominating set on {H}-minor-free
  graphs.
\newblock In {\em Proceedings of the 23rd Annual ACM-SIAM Symposium on Discrete
  Algorithms (SODA 2012)}, pages 82--93. ACM-SIAM, 2010.

\bibitem{FominGT09con}
Fedor~V. Fomin, Petr~A. Golovach, and Dimitrios~M. Thilikos.
\newblock Contraction obstructions for treewidth.
\newblock {\em J. Comb. Theory, Ser. B}, 101(5):302--314, 2011.

\bibitem{FominLMPS11}
Fedor~V. Fomin, Daniel Lokshtanov, Neeldhara Misra, Geevarghese Philip, and
  Saket Saurabh.
\newblock Hitting forbidden minors: Approximation and kernelization.
\newblock {\em {SIAM} J. Discrete Math.}, 30(1):383--410, 2016.

\bibitem{FominLRS10}
Fedor~V. Fomin, Daniel Lokshtanov, Venkatesh Raman, and Saket Saurabh.
\newblock Bidimensionality and {EPTAS}.
\newblock In {\em Proceedings of the 22nd Annual ACM-SIAM Symposium on Discrete
  Algorithms (SODA 2011)}, pages 748--759. SIAM, 2010.

\bibitem{FominLS12}
Fedor~V. Fomin, Daniel Lokshtanov, and Saket Saurabh.
\newblock Bidimensionality and geometric graphs.
\newblock In {\em Proceedings of the 23rd Annual ACM-SIAM Symposium on Discrete
  Algorithms (SODA 2012)}, pages 1563--1575. SIAM, 2012.

\bibitem{FominT06}
Fedor~V. Fomin and Dimitrios~M. Thilikos.
\newblock Dominating sets in planar graphs: Branch-width and exponential
  speed-up.
\newblock {\em SIAM J. Comput.}, 36:281--309, 2006.

\bibitem{GolovachV08}
Petr~A. Golovach and Yngve Villanger.
\newblock Parameterized complexity for domination problems on degenerate
  graphs.
\newblock In {\em Proceedings of the 34th International wokshop on
  Graph-Theoretic Concepts in Computer Scienc (WG 2008)}, volume 5344 of {\em
  Lecture Notes in Comput. Sci.}, pages 195--205. Springer, Berlin, 2008.

\bibitem{Grohe03}
Martin Grohe.
\newblock Local tree-width, excluded minors, and approximation algorithms.
\newblock {\em Combinatorica}, 23:613--632, 2003.

\bibitem{GroheM12}
Martin Grohe and D{\'{a}}niel Marx.
\newblock Structure theorem and isomorphism test for graphs with excluded
  topological subgraphs.
\newblock {\em {SIAM} J. Comput.}, 44(1):114--159, 2015.

\bibitem{Gutner09}
Shai Gutner.
\newblock Polynomial kernels and faster algorithms for the dominating set
  problem on graphs with an excluded minor.
\newblock In {\em Proceedings of the 4th Workshop on Parameterized and Exact
  Computation (IWPEC 2009)}, Lecture Notes in Comput. Sci., pages 246--257.
  Springer, 2009.

\bibitem{HaynesHS98}
Teresa~W. Haynes, Stephen~T. Hedetniemi, and Peter~J. Slater.
\newblock {\em Fundamentals of domination in graphs}.
\newblock Marcel Dekker Inc., New York, 1998.

\bibitem{ImpagliazzoPZ01}
Russell Impagliazzo, Ramamohan Paturi, and Francis Zane.
\newblock Which problems have strongly exponential complexity?
\newblock {\em J. Comput. Syst. Sci.}, 63(4):512--530, 2001.

\bibitem{KawarabayashiK08}
Ken-ichi Kawarabayashi and Yusuke Kobayashi.
\newblock The induced disjoint path problem.
\newblock In {\em Proceedings of the 13th Conference on Integer Programming and
  Combinatorial Optimization (IPCO 2008)}, volume 5035 of {\em Lecture Notes in
  Comput. Sci.}, pages 47--61. Springer, Berlin, 2008.

\bibitem{Kawarabayashi:2010cs}
Ken-ichi Kawarabayashi and Bruce Reed.
\newblock Odd cycle packing.
\newblock In {\em Proceedings of the 42nd ACM Symposium on Theory of Computing
  (STOC 2010)}, pages 695--704, New York, NY, USA, 2010. ACM.

\bibitem{abs-1207-0835}
Eun~Jung Kim, Alexander Langer, Christophe Paul, Felix Reidl, Peter Rossmanith,
  Ignasi Sau, and Somnath Sikdar.
\newblock Linear kernels and single-exponential algorithms via protrusion
  decompositions.
\newblock {\em {ACM} Trans. Algorithms}, 12(2):21, 2016.

\bibitem{Kobayashi:2009jt}
Yusuke Kobayashi and Ken-ichi Kawarabayashi.
\newblock Algorithms for finding an induced cycle in planar graphs and bounded
  genus graphs.
\newblock In {\em Proceedings of the twentieth Annual ACM-SIAM Symposium on
  Discrete Algorithms (SODA 2009)}, pages 1146--1155. ACM-SIAM, 2009.

\bibitem{Niedermeierbook06}
Rolf Niedermeier.
\newblock {\em Invitation to fixed-parameter algorithms}, volume~31 of {\em
  Oxford Lecture Series in Mathematics and its Applications}.
\newblock Oxford University Press, Oxford, 2006.

\bibitem{PhilipRS09}
Geevarghese Philip, Venkatesh Raman, and Somnath Sikdar.
\newblock Polynomial kernels for dominating set in graphs of bounded degeneracy
  and beyond.
\newblock {\em {ACM} Trans. Algorithms}, 9(1):11, 2012.

\bibitem{RobertsonS-XIII}
Neil Robertson and P.~D. Seymour.
\newblock Graph minors. {XIII}. {T}he disjoint paths problem.
\newblock {\em J. Comb. Theory Series B}, 63(1):65--110, 1995.

\bibitem{RobertsonS03}
Neil Robertson and P.~D. Seymour.
\newblock Graph minors. {XVI}. {E}xcluding a non-planar graph.
\newblock {\em J. Combin. Theory Ser. B}, 89(1):43--76, 2003.

\bibitem{RooijBR09}
Johan M.~M. van Rooij, Hans~L. Bodlaender, and Peter Rossmanith.
\newblock Dynamic programming on tree decompositions using generalised fast
  subset convolution.
\newblock In {\em Proceedings of the 17th Annual European Symposium on
  Algorithms (ESA)}, volume 5757 of {\em Lecture Notes in Comput. Sci.}, pages
  566--577. Springer, 2009.

\end{thebibliography}

\end{document}