\begin{proof}
The proof proceeds by induction on the size of the proof, as in Defn.~\ref{definition:size}. 
In each of the following base cases, the conditions for splitting are immediately satisfied.
For the base case for the \textit{tidy name} rule, the bottommost rule of a proof is of the form 
, where .
For the base case for the \textit{tidy} rule, the bottommost rule is of the form 
, such that .
For the base case for  and ,  and  hold.

\begin{enumerate}[label=\textbf{\Alph*},ref=\Alph*,leftmargin=*]
\item \textbf{Principal cases for wen.}
There are principal cases for \textit{wen} where the rules \textit{close}, \textit{suspend}, \textit{left wen}, \textit{right wen} and \textit{fresh} interfere directly with \textit{wen} at the root of a principal formula. Three representative cases are presented.

\begin{enumerate}[label*=\textbf{.\arabic*}]
\item The first principal case for  is when the bottommost rule of a proof is an instance of the \textit{close} rule of the form 
,
where  and .
By the induction hypothesis,
there exist  and  such that  and  and either  or , and also we have derivation .
Since , if  then .
Furthermore, the size of the proof of  is no larger than the size of the proof of ; hence strictly bounded by the size of the proof of .
If  then by the \textit{close} rule .
If  then, since , by the \textit{extrude new} rule,
.
Hence in either case 
and thereby the derivation 

can be constructed,
meeting the conditions for splitting for \textit{wen}.


\item Consider the second principal case for \textit{wen} where the bottommost rule of a proof is an instance of the \textit{suspend} rule of the form
, where  and .
By the induction hypothesis,
there exist  and  such that and  and  and either  or , and also .
Furthermore, the size of the proof of  is no larger than the size of the proof of ; hence strictly bounded by the size of the proof of .
Since , if  then, by the \textit{new wen} and \textit{extrude new} rules, 
.
If  then, by the \textit{close} rule, .
So in either case, ,
and hence the derivation

can be constructed, as required.
The principal cases for \textit{left wen} and \textit{right wen} are similar.


\fullproof{
Another similar principal cases for  is where the first the bottommost rule of a proof is of the form
, where  and .
By induction,
there exist  and  such that and  and either  or , and also .
Furthermore, the size of the proof of  is no larger than the size of the proof of ; hence strictly bounded by the size of the proof of .
If  define , and if  define .
In the case , since , .
Hence the following derivation
 can be constructed, as required.
}


\item Consider the principal case for  when the bottommost rule of a proof is an instance of the \textit{fresh} rule of the form 
, 
where . Notice that  is required to handle the effect of \textit{equivariance}.
By applying the induction hypothesis inductively on the length of , there exist  and  such that  and  and , and also . Furthermore, the size of the proof of  is bounded above by the size of the proof of .
By the induction hypothesis, there exist  and  such that ,  and either  or , and also
.
There are two cases to consider. If  then let ; and if  then let , in which case, since  we have 
. 
In either case . Thereby we can construct the derivation
.
Furthermore, appealing to Lemma~\ref{lemma:close}, the proof

can be constructed and, furthermore, , since by Lemma~\ref{lemma:bound}  and the \textit{wen} count strictly decreases.
\end{enumerate}

\begin{comment}
Consider the principal case for  when  is the form of the bottommost rule of a proof, where  and .
By induction,
there exist  and  such that  and either  or , for , and -ary killing context  such that 
.
There are two cases to consider. If  then we are done. 
Otherwise, consider  in which case , as required.
\end{comment}



\item \textbf{Principal cases for new.}
The principal cases for \textit{new} are where the rules \textit{close}, \textit{extrude new}, \textit{medial new} and \textit{new wen} rules interfere directly with the  quantifier at the root of the principal formula.
Three cases are presented.

\begin{enumerate}[label*=\textbf{.\arabic*}]
\item The first principal case for  is when the bottommost rule of a proof is an instance of the \textit{close} rules of the form
,
where .
By the induction hypothesis, there exist formulae  and  such that  and  and either  or , and also we have derivation
.
Furthermore, the size of the proof of  is no larger than the size of the proof of ; hence strictly bounded by the size of the proof of .
In the case , we have , since .
In the case , we have
.
Hence, by applying one of the above cases the following derivation
 can be constructed as required.
The principal case where the bottommost rule in a proof is the \textit{extrude new} rule follows a similar pattern.
\begin{comment}
There are four principal cases for .
This first is when the bottommost rule of a proof is
,
where  and . Observe that the proof of  has proof one step shorter than  and,
by Lemma~\ref{lemma:bound}, 
thereby strictly bounding the size of the proof.
N.B.\ this argument is used implicitly in every case for splitting.

By induction, there exist formulae  and  and  and  and either  or , for , and -ary killing context such that the following holds.
.
Furthermore, the size of the proof of  is no larger than the size of the proof of ; hence strictly bounded by the size of the proof of .

In the case that , the following derivation can be constructed, since ,
.

In the case that , we have derivation 
.

Hence, by applying one of the above cases for each , the following derivation can be constructed as required.

\end{comment}


\fullproof{
The second principal case for \textit{new} is when the bottommost rule of a proof is an instance of the \textit{extrude new} rule as follows, where .

where  is provable.
By, the induction hypothesis, there exist formulae  and  where  and either  or , and also
.
Furthermore, the size of the proof of  is bounded above by the size of the proof of ; thereby strictly bounded by the size of the proof of .
If , define , and if  define .
In the cases where , since , 
, where the premise equals .
In the cases where , .
Hence the derivation 

can be constructed, as required.
}
\begin{comment}
The second principal case for \textit{new} is when the bottommost rule of a proof begins as follows, where .

where  is provable.
By, the induction hypothesis, there exist formulae  and  where  and either  or , for , and -ary killing context such that the following holds:
.
Furthermore, the size of the proof of  is bounded above by the size of the proof of ; thereby strictly bounded by the size of the proof of .

If , define  and if  define  .
In the cases where , since , .
In the cases where , .
Hence the following derivation can be constructed, as required.

\end{comment}


\item Consider the second principal case for \textit{new} where the \textit{medial new} rule is the bottommost rule of a proof of the form

The  is required to handle cases induced by equivariance.
By applying the induction hypothesis repeatedly, 
there exists  and  such that  and
 and , and also .
Furthermore, the size of the proof of  is bounded above by the size of the proof of .
By the induction hypothesis, there exist  and  such that  and , for , and -ary killing context such that 
.
Furthermore, the size of the proofs of  and  are bounded above by the size of the proof of .
By the induction hypothesis again, there exist  and  such that  and  and either  or ,
and also
.
Also by the induction hypothesis, there exist  and  such that  and  and either  or , 
and also
.
Now define  and  such that

and, 
if for all ,  and , then ;
otherwise .
Hence for each , one of the following derivations holds.
\begin{itemize}
\item 
 and  hence
.

\item
If  and , hence , by the \textit{left wen} rule
.

\item
If , hence , and , by the \textit{right wen} rule
.

\item
Otherwise by the \textit{suspend} rule 
\end{itemize}
If for all  such that ,  and  then . Otherwise, by Lemma~\ref{lemma:commute},
, where the premise is equialent to .
Thereby the  derivation below left can be constructed, and furthermore, using Lemma~\ref{lemma:close}, 
the proof below right can also be constructed.




By Lemma~\ref{lemma:bound},
; hence  since the \textit{new count} strictly decreases, as required.
\begin{comment}
Consider the third principal case for \textit{new} where the \textit{medial new} rule is the bottommost rule of a proof of the form

such that  and .

By induction, there exist  and  such that  and , for , and -ary killing context such that 
.
Furthermore, the size of the proofs of  and  are bounded above by the size of the proof of .
Observe that by -conversion,  and  also hold, where  and .

By induction, there exist  and  such that  and  and either  or , for , and -ary killing context  such that 
.
Also by induction, there exist  and  such that  and  and either  or , for , and -ary killing context  such that 
.

Now define  and  such that

and, if  and  for all  and , then ;
otherwise . 
In cases where for some , , by Lemma~\ref{lemma:commute} ;
In cases where for some , , by Lemma~\ref{lemma:commute} .
Hence for each , one of the of the following derivations holds.
\begin{itemize}
\item 
 and  for all  and  hence
.

\item
If  for all , hence  for all , and hence  thereby , hence by using the \textit{left wen} rule
.

\item
If  for all , hence  for all , thereby , hence by using the \textit{right wen} rule
.

\item
Otherwise by using the \textit{suspend} rule the following derivation holds.

\end{itemize}
Thereby the following derivation can be constructed.

Furthermore, the following proof can be constructed.

By Lemma~\ref{lemma:bound} and since the \textit{new count} strictly decreases,
; thereby  as required.
\end{comment}
\smallskip

\item Consider the third principal  case for \textit{new} where the bottommost rule of a proof is the \textit{new wen} rule of the form

By applying the induction hypothesis repeatedly, 
there exist  and  such that 
and  and , and also . Furthermore, the size of the proof of  is bounded above by the size of the proof of .
By the induction hypothesis, there exist  and  such that  and  and either  or , and also 
.
Furthermore, the size of the proof of  is bounded above by the size of the proof of , hence strictly bounded above by the size of the proof of  enabling the induction hypothesis.
By the induction hypothesis again, there exist  and  such that  and  and either  or , and also
.

Let  and  be defined such that, if , then ;
or, if , then .
If  then define .
If , then define .
There are four scenarios for constructing a derivation with premise  and conclusion .
\begin{itemize}
\item In the case  and  then .

\item If  and  then .

\item If both  and  hold, then 
we have 


\item If both  and  then 
,
where the premise is equivalent to 
.
\end{itemize}
Thereby, by applying one of the above cases,  we have 

In the case that ,  the left most derivation below holds. 
In the case,  and  the middle derivation below holds.  
Hence in either case, appealing to Lemma~\ref{lemma:close}, the proof below right can be constructed:


\noindent Furthermore, by Lemma~\ref{lemma:bound}, .
Hence  since the \textit{new} count strictly decreases.
\begin{comment}
Consider the fourth principal  case for \textit{new} where the bottommost rule of a proof is the \textit{new wen} rule of the form

where ,  and .

By induction, there exist  and  such that  and either  or , for ,
and -ary killing context  such that
.
Furthermore, the size of the proof of  is bounded above by the size of the proof of  enabling the induction hypothesis.
By induction again, there exist  and  such that  and either  or , for , and -ary killing context  such that 
.

Let  and  be defined such that, if , then ;
or, if , then .
If  for all , then define .
If for some  , then define .

In the following fix some  such that . There are four scenarios for .
In the case  for all , if  then  and if  then .


If, for some , , then , by Lemma~\ref{lemma:commute}, and one of the following hold.
\begin{itemize}
\item If, further to  for some ,  then 
.
\item
If, further to  for some ,   then 
.
\end{itemize}
Thereby, by applying one of the above cases for each , the following derivation can be constructed.
.

Now observe that the following proof can always be constructed for every  such that .

By using the proof above, a proof can be constructed for each  in the following two cases.

In the first case where , a proof can be constructed such that

Furthermore, by Lemma~\ref{lemma:bound}, . Hence 
Hence .

In the second case,  and, since , by Lemma~\ref{lemma:nfv} we have .
Hence the following proof can be constructed, using the above proof:
.
Furthermore, by Lemma~\ref{lemma:bound}, .
Hence .
\end{comment}

\end{enumerate}



\item \textbf{Principal cases for seq.} There are two forms of principal cases for \textit{seq}.
The first case, induced by the \textit{sequence} rule, is the case that forces the \textit{medial}, \textit{medial1} and \textit{medial new} rules.
The other cases are induced by the \textit{suspend}, \textit{left wen} and \textit{right wen} rules (which are forced as a knock on effect of the \textit{medial new} rule).

\begin{enumerate}[label*=\textbf{.\arabic*}]
\item Consider the first principal case for \textit{seq}.
The difficulty in this case is that, due to associativity of \textit{seq}, the \rseq rule may be applied in several ways when there are multiple occurrences of \textit{seq}.
Consider a principal formula of the form , where we aim to split the formula around the second \textit{seq} operator. The difficulty is that the bottommost rule may be an instance of the \rseq rule applied between  and . Symmetrically, the principal formula may be of the form  but the bottommost rule may be an instance of the \rseq rule applied between  and .
In the following analysis, only the former case is considered; the symmetric case follows a similar pattern.
The principal formula is  and the bottommost rule is an instance of the \textit{sequence} rule of the form
 
where ,  (otherwise splitting is trivial), and either  or  (otherwise the \rseq rule cannot be applied);
and also .
By the induction hypothesis, there exist  and  such that  and  hold, for , and an -ary killing context  such that 

Furthermore,
the size of the proof of formula

is bounded above by the size of the proof of
, hence the induction hypothesis is enabled.
By the induction hypothesis, there exists  and  such that  and , for , and -ary killing context  such that

Furthermore, by Lemma~\ref{lemma:medial} there exist killing contexts  and  and sets of integers ,  such that

Thereby, the following derivation can be constructed.

Furthermore, the following two proofs can be constructed.

By Lemma~\ref{lemma:bound}, 

which are also upper bounds for

and
.
Furthermore,  and  both  and  
Hence the sizes of the above proofs of 
 
and 

 are strictly less than the size of the proof of .



\item Consider the principal case for \textit{seq} where the bottommost rule of a proof is an instance of the \textit{suspend} rule of the form
 By induction, there exist  and  such that  and  hold, for ,
and -ary killing context  such that 
.
Furthermore the size of the proof of  is bounded above by the size of the proof of .
By induction again, there exist  and  such that  and , for , and -ary killing context  such that the following derivation holds.
.
Furthermore, the size of the proof of  is bounded by the size of the proof of . 
By applying the induction hypothesis again, there exist  and  such that  and  and either  or , and also
.
Furthermore, the size of the proof of  is bounded above by the size of the proof of .
By a fourth induction, there exist  and  such that both  and  hold, for , and -ary killing context  such that the following derivation holds:

By Lemma~\ref{lemma:medial}, there exists some  and  and killing contexts  and  such that

Define  and  as follows.
If , then 

and hence, we can construct the derivation 
 where the premise equals .
If however , then define

and hence, the derivation 

can be constructed.
By Lemma~\ref{lemma:medial}, for some ,  and killing contexts  and , we obtain the following derivation:

By using the above derivations we can construct the following derivation:

\begin{comment}
We aim now to establish  and
.
\end{comment}

\noindent Consider whether the judgement  holds.
We have two cases: in the first,  and ; in the second . In each case, one of the following derivations can be respectively constructed.

Similarly, consider whether judgement  holds.
Either we have 

or we have . In each case, one of the following derivations holds, respectively.

Thereby, by applying one of the above cases for each  and , 
the following two proofs exist.
\begin{comment}

\end{comment}

Furthermore, by Lemma~\ref{lemma:bound}, 

Hence, sizes

are strictly bounded above by 
, as required.
Cases for \textit{left wen} and \textit{right wen} rules are similar.
\begin{comment}
There are three similar principal case for \textit{seq} that do not correspond to cases in \textsf{MAV}~\cite{Horne2015}, induced by the \textit{suspend}, \textit{left wen} and \textit{right wen} rules.
We present here only the case induced by \textit{suspend}.
Consider the case where the bottommost rule of a proof is a \textit{suspend} rule that interferes with the \textit{seq} connective to which splitting is applied as follows.

In the above, it is assumed that  and also ,  and .
Hence, by induction, there exist  and  such that  and , for ,
and -ary killing context  such that the following derivation holds.
.
Furthermore the size of the proof of  is bounded by the size of the proof of .

By induction again, there exist  and  such that  and , for , and -ary killing context  such that the following derivation holds.
.
Furthermore, the size of the proof of  is bounded by the size of the proof of . 

Since , by Lemma~\ref{lemma:nfv}, . By appling the induction hypothesis again, there exist  and  such that  and either  or , for , and -ary killing context  such that 
.
Furthermore, the size of the proof of  is bounded above by the size of the proof of .

By a fourth induction, there exist  and  such that  and , for , and -ary killing context  such that the following derivation holds.

In the case , by Lemma~\ref{lemma:medial}, the following derivation holds, where  and  and killing contexts  and .

In the case , by Lemma~\ref{lemma:medial}, the following derivation holds, where  and  and killing contexts  and .

Define the formulae
 and 
.
If , then define 
and .
	If , then define 
and .
Hence we construct the following derivation, using Lemma~\ref{lemma:medial} again, for some  and  and killing contexts  and .

We aim now to establish  and
.
Firstly, observe that  then
, since , by Lemma~\ref{lemma:nfv}, .
Also, if  then 
.
Furthermore, the following proof can be constructed.

Thereby, in either case .
A similar argument establishes 
.

By applying the above proofs appropriately, the following two proofs can be constructed.


Furthermore, by Lemma~\ref{lemma:bound}, the following inequalities holds.

Hence, by Lemma~\ref{lemma:multisets}, the size of the formulae
 and

  are strictly bounded above by the size of 
, as required.
\end{comment}


 
\end{enumerate}


\item \textbf{Principal case for times.}
There is only one principal case for \textit{times}, which does not differ significantly from the corresponding case in \textsf{BV} and its extensions. A proof may begin with an instance of the \textit{switch} rule of the form

such that  and  (otherwise the \textit{switch} rule cannot be applied), and also  and  (otherwise splitting holds trivially).
By the induction hypothesis, there exist  and  such that  and  hold, for , and an -ary killing context  such that 
derivation  holds.
Furthermore  and  are bounded above by .
Hence, by the induction hypothesis twice there exist formulae , ,  and  such that , ,  and , for  and , and -ary killing context  and -ary killing context  such that derivations
 
can be constructed.
Thereby the following derivation can be constructed.

Now observe that the following two proofs can be constructed.

Furthermore,

and 
,
since  and .
Also, by Lemma~\ref{lemma:bound}, the following inequality holds.

Hence both

and  hold.
Thereby the size of each of the above proofs is strictly bounded above by the size of the proof of .




\item \textbf{Principal cases for with.}
There are three forms of principal case where the \textit{with} operator is directly involved in the bottommost rules.
Note that in \textsf{MAV} the \textit{with} operator is separated from the core splitting lemma, much like universal quantification in this paper.
However, in the case of \textsf{MAV1} the \textit{left name} and \textit{right name} rules introduce inter-dependencies between nominals and \textit{with}, forcing cases for \textit{with} to be checked in this lemma.

\begin{enumerate}[label*=\textbf{.\arabic*}]

\item Consider the principal case involving the \textit{extrude} rule.
In this case, the bottommost rule is of the form 

Now, by the induction hypothesis, 
since  holds, we have that 
 and  hold, as required.



\item Consider the principal case involving the \textit{left name} rule.
In this case, the bottommost rule is of the form 
 
By the induction hypothesis, there exist  and  such that
 and  and  
and either  or . Furthermore, the size of the proof of  is strictly less than the size of the proof of , since the \textit{wen} count strictly decreases, and by Lemma~\ref{lemma:bound}, .
By the induction hypothesis again,  and  hold.


Now if  then  and  holds immediately, whereas 
 is proved as below left. Otherwise,  and
 is proved in the middle derivation below, whereas  is proved in the right derivation below.




Hence, in either case,  
and since ,
we have that  holds.
Thereby  and  hold, as required.
The case for the \textit{left name} rule, where  replaces  is similar; as are the cases for the \textit{right name} and \textit{with name} rules.


\begin{comment} Case for \textit{with name}, omitted for readability.
Consider the principal case involving the \textit{with name} rule.
In this case, the bottommost rule is of the form  such that .
By the induction hypothesis, there exist  and  such that  and  and  
and either  or . Furthermore, the size of the proof of  is strictly less than the size of the proof of .
By the induction hypothesis again,  and  hold.
Now if  then  and so .
Otherwise  so .
Hence .
By a similar argument, .
Thereby we can construct the following two proofs as follows.

Furthermore 
and , hence the size of proofs strictly decrease as requried.
\smallskip
\end{comment}


\item Consider the principal case involving the \textit{medial} rule.
In this case, the bottommost rule of a proof is of the form 

By the induction hypothesis, for  there exists  and  such that  and  hold, and -ary killing context  such that 
.
Furthermore, the size of the proofs of  and  are strictly less than the size of the proof of . Hence by the induction hypothesis again, , ,  and .
Hence we can construct the following two proofs, as required.



\end{enumerate}


\item \textbf{Commutative cases induced by equivariance.}
There are certain commutative cases induced by the \textit{equivariance} rule for nominal quantifiers.
These are the cases that force the rules \textit{all name}, \textit{with name}, \textit{left name} and \textit{right name} to be included.
Notice also that \textit{equivariance} for \textit{new} is required when handling the case induced by \textit{equivariance} for \textit{wen}; hence \textit{equivariance} for both nominal quantifiers must be explicit structural rules rather than properties derived from each other.


\begin{enumerate}[label*=\textbf{.\arabic*}]
\item Consider the commutative case for \textit{wen} where the bottommost rule of a proof is an instance of the \textit{close} rule of following form

Notice that  is the principal connective but the \textit{close} rule is applied to  behind the principal connective.
Thus we desire some formula  such that 
and  and either  or there exists  such that  and , and the size of  is strictly smaller than .
By the induction hypothesis, there exist  and  such that  and  and either  or  and the derivation
 holds.
Furthermore the size of the proof of  is bounded above by the size of the proof of ; hence strictly bounded by the size of the proof of .
Hence, by induction, there exist  and  such that  and  and either  or  the derivation 
 holds.
Observe that if , then 
, since .
If  then 
.
Thereby the following derivation can be constructed, where if  then  and if  then , and also the premise is equivalent to  by \textit{equivariance} for \textit{new}:
.
Furthermore, the following proof can be constructed 

and, by Lemma~\ref{lemma:bound},  hence 
, as required.


\item Consider a commutative case for \textit{new} induced by \textit{equivariance} for \textit{new}, where the bottommost rule is an instance of \textit{extrude new} of the form

By the induction hypothesis, there exist  and  such that  and  and either  or , where
.
Furthermore, the size of the proof of  is bound above by the size of the proof of , hence strictly bound above by the size of the proof of .
Hence, by induction again, there exist  and  such that  and  and either  or , and also .
Now define  and  as follows.
If  then let .
If  then let .
If  then let .
If  then let .
Now observe if  then
 and .
For  observe
, since ,
and if  then ,
while if  then , by \textit{equivariance} for \textit{wen}.
Hence in all cases 
and, since  and , we can arrange that .
Now, for the cases where , we have ,
and hence .
Also if , then . Hence in either case we can construct the proof .
Furthermore, ,
since by Lemma~\ref{lemma:bound} .



\item Similar commutative cases for \textit{wen} and \textit{new} as principal formulae are induced by \textit{equivariance} where the bottommost rule in a proof is an instance of the \textit{close}, \textit{right wen} or \textit{suspend} rules.
In each case, the quantifier involved in the bottommost rule appears behind the principal connective and is propagated in front of the principal connective using \textit{equivariance}.
\end{enumerate}

\fullproof{
A similar commutative case for \textit{wen} is induced where the bottommost rule in a proof is an instance of the \textit{left wen} rule of the form 

where  and .
By the induction hypothesis, there exist  and  such that  and  and either  or  such that .
Furthermore, the size of the proof of  is bounded by the size of the proof of  hence strictly bounded by the size of the proof of .
Hence, by the induction hypothesis again, there exist  and  such that  and  and either  or , and derivation  holds.
If  let  otherwise . If  let , otherwise .
Thereby the following derivations hold:
\begin{itemize}
\item If  and  then
.

\item If  and  then , where the premise is equivalent to .

\item If  then 
and .
\end{itemize}
Hence in any of the above cases, .
Now, if , then ;
and if  then , hence , since .
Clearly , hence  holds.  Furthermore, , since by Lemma~\ref{lemma:bound} .


The third commutative case for \textit{wen} induced by \textit{equivariance} is where the bottommost rule is an instance of the \textit{suspend} rule of the form 
, where .
By the induction hypothesis, there exist  and  where  and  and either  or  such that .
Furthermore, the size of the proof of  is bounded above by the size of the proof of , hence strictly bounded above by the size of the proof of .
Hence by the induction hypothesis again, there exist  and  where  and  and either  or , such that .
If  then let , and if  then  let .
Also observe that whether we have  or , we have .
Thereby the following derivation can be constructed:
.
Furthermore, the following proof can be constructed ;
and the size of the proof of , since  by Lemma~\ref{lemma:bound}.


Consider the first commutative case for \textit{new} induced by equivariance, where the bottommost rule is of the form
,
where .
By the induction hypothesis, there exist  and  such that  and  and either  or , where
.
Furthermore, the size of the proof of  is bound above by the size of the proof of , hence strictly bound above by the size of the proof of .
Hence, by induction again, there exist  and  such that  and  and either  or , and also .
If  then let , and if  then let . 
Also, regardless of whether  or , . Hence derivation  can be constructed, where the premise is equivalent to .
Furthermore, ,
and ,
since by Lemma~\ref{lemma:bound} .

}




 


\item \textbf{Regular commutative cases.} As in every splitting lemma, there are numerous \textit{commutative} cases where the bottommost rule in a proof does not directly involve the principal connective.
For each principal formula handled by this splitting lemma (\textit{new}, \textit{wen}, \textit{with}, \textit{seq} and \textit{times}) there are commutative cases induced by \textit{new}, \textit{wen}, \textit{all}, \textit{with} and \textit{times} and also two commutative cases induced by \textit{seq}.
Thus there are 35 similar commutative cases to check, that all follow a pattern, hence only a representative selection of four cases are presented that make special use of -conversion and the rules \textit{new wen},
\textit{all name}, \textit{with name}, \textit{left name} and \textit{right name}.
Further, representative cases appear in the proof for existential quantifiers.


\begin{enumerate}[label*=\textbf{.\arabic*}]

\item Consider the commutative case where the principal formula is  and the bottommost rule is an instance of \textit{extrude new} but applied to a distinct \textit{new} quantifier , as in the following rule instance 
 
Also assume, by -conversion, that .
By induction, there exist  and  such that ,  and either  or , and also
.
Furthermore, the size of the proof of  is bounded above by the size of the proof of  and hence strictly bounded above by the size of the proof of , enabling the induction hypothesis.
Hence, by the induction hypothesis, there exist formulae  and  such that  and  and either  or , and also .
Define  such that if  then  and if  then .
Hence if  then  by applying the \textit{new wen} rule, where the premise equals .
If  then . In both cases, .
Now observe that either  and , hence the derivation  below holds;
or , hence the derivation  below holds. Given these, the derivation  can be constructed:

Since  and , we have ;
thereby the proof  above can be constructed.
Furthermore,  since by Lemma~\ref{lemma:bound} 
and the \textit{wen count} strictly decreases.



\item Consider the commutative case for principal formula  where the bottommost rule is \textit{external}: 

where  holds.
By the induction hypothesis, we have that both  and  hold;
and furthermore the multiset inequalities 

hold.
Hence, by the induction hypothesis, there exist  and  such that ,  and either  or .
Also, by the induction hypothesis, there exist  and  such that
,  and either  or .
Furthermore the two derivations 

and
 hold.
Now define  such that if  and  then , and 
 otherwise, observing that in either case .
In the case  and , by the \textit{with name} rule,
.
In the case  and , by the \textit{left name} rule,
.
In the case that  and , by the \textit{right name} rule,
.
Thereby the following derivation and proof can be constructed:

Furthermore, by Lemma~\ref{lemma:bound}, ; and,
since the \textit{wen} count strictly decreases,
.




\item Consider the commutative case where the principal formula is  and the bottommost rule is an instance of the
\textit{extrude1} rule of the form

 assuming 
and  holds.
By Lemma~\ref{lemma:universal}, for every variable ,  holds.
Furthermore, since , we have equivalence .
The strict multiset inequality  holds.
Hence, by the induction hypothesis, for every variable , there exist formulae  and  such that  and  and either  or , and also
.
Define  such that if  then  and if  then . Hence if  then, since  permutes with any quantifier using the \textit{all name} rule, .
Hence, for a fresh  such that  and , 
the following derivations can be constructed:

Furthermore,  by Lemma~\ref{lemma:bound}; hence 
 
since the wen count strictly decreases.


\item Consider the commutative case where the principal connective is  and the bottommost rule is an instance of the extrude new rule of the form 

 where  and also , where the second condition can be achieved by -conversion.
By the induction hypothesis, there exist  and  such that ,  and either  or , and also
.
Furthermore, the size of the proof of  is bounded above by the size of the proof of  and hence strictly bounded above by the size of the proof of , enabling the induction hypothesis.
Hence, by the induction hypothesis, there exist formulae  and  such that  and  and either  or , and also .
Define  such that if  then  and if  then .
Now observe that either we have that  and  and hence
the derivation  below left holds;  or we have that  and hence the derivation  belw  holds. 
Hence,
by applying one of these cases, we have the derivation  below, where the premise is equivalent to .

Since  and , we have ;
thereby the proof  above can be constructed.
Furthermore,  since by Lemma~\ref{lemma:bound} 

and the \textit{wen count} strictly decreases.
\end{enumerate}

\fullproof{
Consider the commutative case when the \textit{wen} quantifier commutes with another \textit{wen} quantifier at the root of the principal formula.
In this case, the bottommost rule of a proof is of the form , where  and , where the second condition can be achieved by -conversion.
By the induction hypothesis, there exist  and  such that ,  and either  or , and also
.
Furthermore, the size of the proof of  is bounded above by the size of the proof of  and hence strictly bounded above by the size of the proof of , enabling the induction hypothesis.
Hence, by the induction hypothesis, there exist formulae  and  such that  and  and either  or , and also .

Define  such that if  then  and if  then . In either case, .
Now observe that either  and  hence
;
or  hence .
Hence the following derivation can be constructed, by applying one of these cases:
, where the premise is equivalent to .
Since  and  we have ,
hence the following proof can be constructed:
.
Furthermore,  since by Lemma~\ref{lemma:bound} 
and the \textit{wen count} strictly decreases.
\smallskip











Consider the commutative case when the \textit{new} quantifier commutes with  as the principal formula.
In this case the bottommost rule of a proof is of the form ,
assuming that .
By induction, there exist  and  such that ,  and either  or , and also
.
Furthermore, the size of the proof of  is bounded above by the size of the proof of  and hence strictly bounded above by the size of the proof of , enabling the induction hypothesis.
Hence, by the induction hypothesis, there exist formulae  and  such that  and , for , and also -ary killing context  such that
.
Furthermore, the size of the proofs of  and  are bounded above by the size of the proof of .
Now observe that either  and  hence
;
or  hence .
Hence the following derivation can be constructed, by the above observations:
.
Observe that  is a -ary killing context as required.
\smallskip


Consider the commutative case when the \textit{wen} quantifier commutes with  as the principal formula.
In this case the bottommost rule of a proof is of the form , where .
By induction, there exist  and  such that ,  and either  or , and also
.
Furthermore, the size of the proof of  is bounded above by the size of the proof of  and hence strictly bounded above by the size of the proof of , enabling the induction hypothesis.
Hence, by the induction hypothesis, there exist formulae  and  such that  and , for , and also -ary killing context  such that
.
Furthermore, the size of the proofs of  and  are bounded above by the size of the proof of .
Now observe that either  and  hence
;
or  hence .
Hence the following derivation can be constructed, by the above observations:
.
Observe that  is a -ary killing context as required.
\smallskip



Consider the commutative case where the \textit{new} quantifier commutes with  as the principal formula.
In this case the bottommost rule of a proof is of the form ,
where .
By induction, there exist  and  such that ,  and either  or , and also
.
Furthermore, the size of the proof of  is bounded above by the size of the proof of  and hence strictly bounded above by the size of the proof of , enabling the induction hypothesis.
Hence, by the induction hypothesis, there exist formulae  and  such that  and , for , and also -ary killing context  such that
.
Furthermore, the size of the proofs of  and  are bounded above by the size of the proof of .
Now observe that either  and  hence
;
or  hence .
Hence the following derivation can be constructed:
.
Observe that  is a -ary killing context as required.
\smallskip


Consider the commutative case when the \textit{wen} quantifier commutes with  as the principal formula.
In this case the bottommost rule of a proof is of the form . where .
By induction, there exist  and  such that ,  and either  or , and also
.
Furthermore, the size of the proof of  is bounded above by the size of the proof of  and hence strictly bounded above by the size of the proof of , enabling the induction hypothesis.
Hence, by the induction hypothesis, there exist formulae  and  such that  and , for , and also -ary killing context  such that
.
Furthermore, the size of the proofs of  and  are bounded above by the size of the proof of .
Now observe that either  and  hence
;
or  hence .
Hence the following derivation can be constructed:
.
Observe that  is a -ary killing context as required.
\smallskip




Consider commutative cases where the principal formula moves entirely to the left hand side of a \textit{seq} operator. 
For principal formula , the bottommost rule in a proof is of the form

such that  holds.
By the induction hypothesis, there exist  and  such that  and , for , and -ary killing context  such that the derivation
 holds, 
and furthermore the size of the proof of  is bounded above by the size of the proof of  hence strictly bounded above by the size of the proof of  enabling the induction hypothesis.
By the induction hypothesis again, there exist formulae  and  such that  and  and either  or , and also the derivation  holds.
Hence the following derivation can be constructed.

Furthermore, by Lemma~\ref{lemma:bound},

and hence , since the \textit{wen} count strictly decreases.
The cases where the principal formula moves entirely to the right hand side of the \textit{seq} operator, and the analogous case for \textit{times}, are similar to the above case.






\item \textbf{Commutative cases involving all and with.}
Consider the commutative case for \textit{with} where  is the principal formula. In this case the bottommost rule is the following form, such that  holds:
.
By the induction hypothesis,  and ; and furthermore strict multiset inequalities 

and

hold.
Hence, by the induction hypothesis, there exist  and  such that  and , for ; and  and  such that
 and , for ;
and also -ary killing context  and -ary killing context 
such that the two derivations 

and
 hold.
Furthermore, the size of the proofs of , ,  and  are bounded above by the size of the proof of .
Thereby the following derivation can be constructed.




Consider the commutative case where universal quantification commutes with   as the principal formula. Suppose the bottommost rule is of the form
, 
assuming 
where  holds.
By Lemma~\ref{lemma:universal}, for every variable ,  holds.
Furthermore, since , .
Since  is removed,  holds.
Pick a fresh  such that .
Hence, by the induction hypothesis, there exist formulae  and  such that  and , for ; and also -ary killing context  such that
.
Furthermore, the size of the proof of  and  are bounded above by the size of the proof of .
Since  was chosen such that  the following derivation can be constructed, as required.
.
\smallskip


Consider the commutative case for \textit{with} where  is the principal formula. The bottommost rule is the form

where  holds.
By the induction hypothesis,  and ; and furthermore the strict multiset inequalities

and
 hold.

Hence, by the induction hypothesis, there exist  and  such that  and , for ; and  and  such that
 and , for ;
and also -ary killing context  and -ary killing context 
such that the two derivations 

and
 hold.
Furthermore, the size of the proofs of , ,  and  are bounded above by the size of the proof of .
Thereby the following derivation can be constructed.

\smallskip



Consider the commutative case where universal quantification commutes with   as the principal formula. Suppose the bottommost rule is of the form
,
 assuming 
where  holds.
By Lemma~\ref{lemma:universal}, for every variable ,  holds.
Furthermore, since , .

The strict multiset inequality  holds.
Pick a fresh  such that .
Hence, by the induction hypothesis, there exist formulae  and  such that  and , for ; and also -ary killing context  such that
.
Furthermore, the size of the proof of  and  are bounded above by the size of the proof of .

Since  was chosen such that  the following derivation can be constructed, as required.
.
\smallskip


Consider the commutative case for \textit{sequence} rule in the presence of principal formula , where the \textit{seq} connective in the principal formula is not active on the \textit{sequence} rule. In this case, the bottommost rule in a proof is an instance of the \textit{sequence} rule of the form
,
where  and  and  holds.
By the induction hypothesis, there exists ,  such that  and , for , and -ary killing context  such that the following derivation holds:
.
Furthermore,  hence the induction hypothesis is enabled again.
By the induction hypothesis, for , there exist formulae  and  such that  and  hold, for , and killing contexts  such that the following derivation holds.

Furthermore the following strict multiset inequalities hold.

Hence the following derivation can be constructed, as required.

The case for the \rseq rule commuting with the principal formula  is similar to the above case. Also the cases for the \textit{switch} rule commuting with seq and times as the principal formula, follow a similar pattern.
\smallskip

}  




\item \textbf{Commutative cases deep in contexts.}
In many commutative cases, the bottommost rule does not interfere with the principal formula either directly or indirectly. Two such cases are presented for \textit{wen} as the principal connective. Other such cases use almost identical reasoning.

\begin{enumerate}[label*=\textbf{.\arabic*}]

\item Consider when a rule is applied outside the scope of the principal formula.
In this case, the bottommost rule in a proof is of the form

By the induction hypothesis, there exist formulae  and  such that  and  and either  or , and also .
Hence clearly derivation 
 holds.
Furthermore, by Lemma~\ref{lemma:bound},  
and .


\fullproof{
Assume that the following application of any rule

is the bottommost rule in a proof,
such that .
By the induction hypothesis, there exist -ary killing context  and formulae  and  such that  and , for , such that .
Hence, the derivation 

holds, satisfying the induction invariant.



Alternatively, the bottommost rule may appear inside the context of principal formula without affecting the root connective of the principal formula. 
Consider the case where \textit{seq} is the principal formula.
Assume that the following application of any rule is the bottommost rule in a proof

such that  has a proof of length . 
Hence by induction, there exist -ary killing context  and formulae  and  such that  and  hold and have a proof no longer than , for , and also 
.
Hence we can construct the following proof of length no longer than , for all , as required:
.
}

\item Consider the case where the following application of any rule in a derivation of the form 

is the bottommost rule is a proof of length , where  has a proof of length .
Hence, by induction, there exist formulae  and  such that  and  and either  or , and also
.
Furthermore, the size of the proof of  is bounded above by the size of the proof of ; hence either  or  and the length of the proof of  is bound by .
The proof 
 can be constructed as required.
Furthermore, if  then , by Lemma~\ref{lemma:bound}. Otherwise, 
hence  by Lemma~\ref{lemma:bound} and the length of the proof of  is .
Thereby in either case, the size of the proof of  is bounded above by the size of the proof of .
\end{enumerate}
 
\end{enumerate}

This covers all scenarios for the bottommost rule, hence splitting follows by induction over the size of the proof.
\end{proof}
