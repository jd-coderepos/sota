\begin{proof}
The proof proceeds by induction on the size of the proof, as in Defn.~\ref{definition:size}. 
In each of the following base cases, the conditions for splitting are immediately satisfied.
For the base case for the \textit{tidy name} rule, the bottommost rule of a proof is of the form 
$\vcenter{\infer[]{\new x.\new \vec{y}. \cunit \cpar P}{\new \vec{y}.\cunit \cpar P}}$, where $\nfv{\vec{y}}{P}$.
For the base case for the \textit{tidy} rule, the bottommost rule is of the form 
$\vcenter{\infer[]{\left(\cunit \wwith \cunit\right) \cpar P}{\cunit \cpar P}}$, such that $\vdash \cunit \cpar P$.
For the base case for $\textit{times}$ and $\textit{seq}$, $\vdash \left(\cunit \tensor \cunit\right) \cpar \cunit$ and $\vdash \left(\cunit \cseq \cunit\right) \cpar \cunit$ hold.

\begin{enumerate}[label=\textbf{\Alph*},ref=\Alph*,leftmargin=*]
\item \textbf{Principal cases for wen.}
There are principal cases for \textit{wen} where the rules \textit{close}, \textit{suspend}, \textit{left wen}, \textit{right wen} and \textit{fresh} interfere directly with \textit{wen} at the root of a principal formula. Three representative cases are presented.

\begin{enumerate}[label*=\textbf{.\arabic*}]
\item The first principal case for $\textit{wen}$ is when the bottommost rule of a proof is an instance of the \textit{close} rule of the form 
$
\vcenter{\infer[]{\wen x. P \cpar \new x. Q \cpar R
}{
\mathopen{\new x.} \left( P \cpar Q \right) \cpar R}}
$,
where $\vdash \mathopen{\new x.} \left( P \cpar Q \right) \cpar R$ and $\nfv{x}{R}$.
By the induction hypothesis,
there exist $S$ and $T$ such that $\vdash P \cpar Q \cpar T$ and $\nfv{x}{S}$ and either $S = T$ or $S = \wen x. T$, and also we have derivation $\infer{R}{S}$.
Since $\nfv{x}{S}$, if $S = T$ then $\vcenter{\infer[]{\new x. Q \cpar S}{\mathopen{\new x.}\left( Q \cpar T \right)}}$.
Furthermore, the size of the proof of $P \cpar Q \cpar T$ is no larger than the size of the proof of $\mathopen{\new x.} \left( P \cpar Q \right) \cpar R$; hence strictly bounded by the size of the proof of $\wen x. P \cpar \new x. Q \cpar R$.
If $S = \wen x. T$ then by the \textit{close} rule $\vcenter{\infer[]{\new x. Q \cpar \wen x. T}{\mathopen{\new x.}\left( Q \cpar T \right)}}$.
If $S = T$ then, since $\nfv{x}{S}$, by the \textit{extrude new} rule,
$\vcenter{\infer[]{\new x. Q \cpar T}{\mathopen{\new x.}\left( Q \cpar T \right)}}$.
Hence in either case $\vcenter{\infer[]{\new x. Q \cpar S}{\mathopen{\new x.}\left( Q \cpar T \right)}}$
and thereby the derivation 
$
\vcenter{
\infer[]{
\new x. Q \cpar R
}{
\infer[]{
\new x. Q \cpar S
}{
\mathopen{\new x.} \left(Q \cpar T \right)
}}}
$
can be constructed,
meeting the conditions for splitting for \textit{wen}.


\item Consider the second principal case for \textit{wen} where the bottommost rule of a proof is an instance of the \textit{suspend} rule of the form
$
\vcenter{
\infer[]{
\wen x. P \cpar \wen x. Q \cpar R
}{
\mathopen{\wen x.} \left( P \cpar Q \right) \cpar R
}}
$, where $\vdash \mathopen{\wen x.} \left( P \cpar Q \right) \cpar R$ and $\nfv{x}{R}$.
By the induction hypothesis,
there exist $S$ and $T$ such that and $\vdash P \cpar Q \cpar T$ and $\nfv{x}{S}$ and either $S = T$ or $S = \new x. T$, and also $\vcenter{\infer[]{R}{S}}$.
Furthermore, the size of the proof of $P \cpar Q \cpar T$ is no larger than the size of the proof of $\mathopen{\wen x.} \left( P \cpar Q \right) \cpar R$; hence strictly bounded by the size of the proof of $\wen x. P \cpar \wen x. Q \cpar R$.
Since $\nfv{x}{S}$, if $S = T$ then, by the \textit{new wen} and \textit{extrude new} rules, 
$\vcenter{\infer[]{\wen x. Q \cpar T}{\infer[]{\new x. Q \cpar T}{\mathopen{\new x.} \left( Q \cpar T \right)}}}$.
If $S = \new x. T$ then, by the \textit{close} rule, $\vcenter{\infer[]{\wen x. Q \cpar \new x. T}{\mathopen{\new x.}\left( Q \cpar T \right)}}$.
So in either case, $\vcenter{\infer[]{\wen x. Q \cpar S}{\mathopen{\new x.} \left( Q \cpar T \right)}}$,
and hence the derivation
$
\vcenter{
\infer[]{
\wen x. Q \cpar R
}{
\infer[]{
\wen x. Q \cpar S
}{
\mathopen{\new x.}\left(Q \cpar T \right)
}}}
$
can be constructed, as required.
The principal cases for \textit{left wen} and \textit{right wen} are similar.


\fullproof{
Another similar principal cases for $\textit{wen}$ is where the first the bottommost rule of a proof is of the form
$
\infer[]{
\wen x. P \cpar Q \cpar R
}{
\mathopen{\wen x.} \left( P \cpar Q \right) \cpar R
}
$, where $\vdash \mathopen{\wen x.} \left( P \cpar Q \right) \cpar R$ and $\nfv{x}{Q \cpar R}$.
By induction,
there exist $S$ and $T$ such that and $\vdash P \cpar Q \cpar T$ and either $S = T$ or $S = \new x. T$, and also $\vcenter{\infer[]{R}{S}}$.
Furthermore, the size of the proof of $P \cpar Q \cpar T$ is no larger than the size of the proof of $\mathopen{\wen x.} \left( P \cpar Q \right) \cpar R$; hence strictly bounded by the size of the proof of $\wen x. P \cpar Q \cpar R$.
If $S = T$ define $U = Q \cpar T$, and if $S = \new x. T$ define $U = \mathopen{\new x.} \left( Q \cpar T \right)$.
In the case $S = \new x. T$, since $\nfv{x}{Q}$, $\infer[]{Q \cpar \new x. T}{ \mathopen{\new x.} \left( Q \cpar T \right) }$.
Hence the following derivation
$
\infer[]{
Q \cpar R
}{
\infer[]{
Q \cpar S
}{
U
}}
$ can be constructed, as required.
}


\item Consider the principal case for $\textit{wen}$ when the bottommost rule of a proof is an instance of the \textit{fresh} rule of the form 
$\vcenter{\infer[]{\wen x. \wen \vec{y}. P \cpar Q}{ \wen \vec{y}. \new x. P \cpar Q }}$, 
where $\vdash \wen \vec{y}. \new x. P \cpar Q$. Notice that $\vec{y}$ is required to handle the effect of \textit{equivariance}.
By applying the induction hypothesis inductively on the length of $\vec{y}$, there exist $\vec{z}$ and $\hat{Q}$ such that $\vec{z} \subseteq \vec{y}$ and $\nfv{\vec{y}}{\new \vec{z} \hat{Q}}$ and $\vdash \new x. P \cpar \hat{Q}$, and also $\vcenter{\infer[]{ Q }{ \new \vec{z}. \hat{Q} }}$. Furthermore, the size of the proof of $\new x. P \cpar \hat{Q}$ is bounded above by the size of the proof of $\wen \vec{y}. \new x. P \cpar Q$.
By the induction hypothesis, there exist $R$ and $S$ such that $\nfv{x}{R}$, $\vdash P \cpar S$ and either $R = S$ or $R = \wen x. S$, and also
$
\vcenter{ \infer[]{\hat{Q}}{R}}
$.
There are two cases to consider. If $R = S$ then let $T = \new \vec{z}. S$; and if $R = \wen x. S$ then let $T = \new x. \new \vec{z}. S$, in which case, since $\new \vec{z}. \new x. S \equiv \new x. \new \vec{z}. S$ we have 
$\vcenter{\infer[]{\new \vec{z}. R}{T}}$. 
In either case $\nfv{x}{T}$. Thereby we can construct the derivation
$
\vcenter{
\infer[]{
Q
}{
\infer[]{
\new \vec{z}. \hat{Q}
}{
\infer[]{
\new \vec{z}. R
}{
T
}}}}
$.
Furthermore, appealing to Lemma~\ref{lemma:close}, the proof
$
\vcenter{
\infer[]{
\wen \vec{y}. P \cpar \new \vec{z}. S
}{
\infer[]{
\mathopen{\new \vec{y}.} \left( P \cpar S \right)
}{
\infer[]{
\new \vec{y}. \cunit
}{
\cunit
}}}}
$
can be constructed and, furthermore, $\size{ \wen \vec{y}. P \cpar \new \vec{z}. S } \prec \size{ \wen x. \wen \vec{y}. P \cpar Q }$, since by Lemma~\ref{lemma:bound} $\size{ \new \vec{z}. S } \preceq \size{ Q }$ and the \textit{wen} count strictly decreases.
\end{enumerate}

\begin{comment}
Consider the principal case for $\textit{wen}$ when $\wen x P \cpar Q \longrightarrow \new x P \cpar Q$ is the form of the bottommost rule of a proof, where $\vdash \new x P \cpar Q$ and $\nfv{x}{Q}$.
By induction,
there exist $R_i$ and $S_i$ such that $\vdash P \cpar S_i$ and either $R_i = S_i$ or $R_i = \wen x S_i$, for $1 \leq i \leq n$, and $n$-ary killing context $\tcontext{}$ such that 
$
 Q \longrightarrow \tcontext{ R_1, R_2, \hdots, R_n }
$.
There are two cases to consider. If $R_i = S_i$ then we are done. 
Otherwise, consider $R_i = \wen x S_i$ in which case $\wen x S_i \longrightarrow \new x S_i$, as required.
\end{comment}



\item \textbf{Principal cases for new.}
The principal cases for \textit{new} are where the rules \textit{close}, \textit{extrude new}, \textit{medial new} and \textit{new wen} rules interfere directly with the $\textit{new}$ quantifier at the root of the principal formula.
Three cases are presented.

\begin{enumerate}[label*=\textbf{.\arabic*}]
\item The first principal case for $\textit{new}$ is when the bottommost rule of a proof is an instance of the \textit{close} rules of the form
$
\vcenter{
\infer[]{
\new x. P \cpar \wen x. Q \cpar R
}{
\mathopen{\new x.} \left( P \cpar Q \right) \cpar R
}}
$,
where $\vdash \mathopen{\new x.} \left( P \cpar Q \right) \cpar R$.
By the induction hypothesis, there exist formulae $U$ and $V$ such that $\vdash P \cpar Q \cpar V$ and $\nfv{x}{U}$ and either $U = V$ or $U = \wen x. V$, and also we have derivation
$\vcenter{\infer[]{R}{U}}$.
Furthermore, the size of the proof of $P \cpar Q \cpar V$ is no larger than the size of the proof of $\mathopen{\new x.} \left( P \cpar Q \right) \cpar R$; hence strictly bounded by the size of the proof of $\mathopen{\new x.} P \cpar \wen x. Q \cpar R$.
In the case $U = V$, we have $
\vcenter{\infer[]{
\wen x. Q \cpar V
}{
\mathopen{\wen x.} \left( Q \cpar V \right)
}}
$, since $\nfv{x}{U}$.
In the case $U = \wen x. V$, we have
$
\vcenter{\infer[]{
\wen x. Q \cpar \wen x. V
}{
\mathopen{\wen x.} \left( Q \cpar V \right)
}}
$.
Hence, by applying one of the above cases the following derivation
$
\vcenter{
\infer[]{
\wen x. Q \cpar R
}{
\infer[]{
\wen x. Q \cpar U
}{
\mathopen{\wen x.} \left( Q \cpar V \right)
}}}
$ can be constructed as required.
The principal case where the bottommost rule in a proof is the \textit{extrude new} rule follows a similar pattern.
\begin{comment}
There are four principal cases for $\textit{new}$.
This first is when the bottommost rule of a proof is
$
\new x P \cpar \wen x Q \cpar R
\longrightarrow
\new x \left( P \cpar Q \right) \cpar R
$,
where $\vdash \new x \left( P \cpar Q \right) \cpar R$ and $\nfv{x}{R}$. Observe that the proof of $\new x \left( P \cpar Q \right) \cpar R$ has proof one step shorter than $\new x P \cpar \wen x Q \cpar R$ and,
by Lemma~\ref{lemma:bound}, $\size{ \new x \left( P \cpar Q \right) \cpar R } \lsim \size{\new x P \cpar \wen x Q \cpar R}$
thereby strictly bounding the size of the proof.
N.B.\ this argument is used implicitly in every case for splitting.

By induction, there exist formulae $U_i$ and $V_i$ and $\vdash P \cpar Q \cpar V_i$ and $\nfv{x}{U_i}$ and either $U_i = V_i$ or $U_i = \wen x V_i$, for $1 \leq i \leq n$, and $n$-ary killing context such that the following holds.
$R \longrightarrow \tcontext{ U_1, U_1, \hdots, U_n }$.
Furthermore, the size of the proof of $P \cpar Q \cpar V_i$ is no larger than the size of the proof of $\new x \left( P \cpar Q \right) \cpar R$; hence strictly bounded by the size of the proof of $\new x P \cpar \wen x Q \cpar R$.

In the case that $U_i = V_i$, the following derivation can be constructed, since $\nfv{x}{U_i}$,
$
\wen x Q \cpar V_i
\longrightarrow
\wen x \left( Q \cpar V_i \right)
$.

In the case that $U_i = \wen x V_i$, we have derivation 
$
\wen x Q \cpar \wen x V_i
\longrightarrow 
\wen x \left( Q \cpar V_i \right)
$.

Hence, by applying one of the above cases for each $i$, the following derivation can be constructed as required.
\[
\begin{array}{rl}
\wen x Q \cpar R
\longrightarrow
\wen x Q \cpar \tcontext{ U_1, U_2, \hdots, U_n }
\longrightarrow&
\tcontext{ \wen x Q \cpar U_i \colon 1 \leq i \leq n }
\\
\longrightarrow&
\tcontext{ \wen x \left( Q \cpar V_i \right) \colon 1 \leq i \leq n }
\end{array}
\]
\end{comment}


\fullproof{
The second principal case for \textit{new} is when the bottommost rule of a proof is an instance of the \textit{extrude new} rule as follows, where $\nfv{x}{Q}$.
$
\vcenter{
\infer[]{
\new x. P \cpar Q \cpar R
}{
\mathopen{\new x.} \left( P \cpar Q \right) \cpar R
}}
$
where $\vdash \mathopen{\new x.} \left( P \cpar Q \right) \cpar R$ is provable.
By, the induction hypothesis, there exist formulae $U$ and $V$ where $\vdash P \cpar Q \cpar V$ and either $U = V$ or $U = \wen x. V$, and also
$\vcenter{\infer[]{R}{U}}$.
Furthermore, the size of the proof of $P \cpar Q \cpar V$ is bounded above by the size of the proof of $\mathopen{\new x.} \left( P \cpar Q \right) \cpar R$; thereby strictly bounded by the size of the proof of $\new x. P \cpar Q \cpar R$.
If $U = \wen x. V$, define $W = \mathopen{\wen x.} \left( Q \cpar V \right)$, and if $U = V$ define $W = Q \cpar V$.
In the cases where $U = \wen x. V$, since $\nfv{x}{Q}$, 
$\vcenter{\infer[]{Q \cpar \wen x. V}{\mathopen{\wen x.} \left( Q \cpar V \right)}}$, where the premise equals $W$.
In the cases where $U = V$, $Q \cpar U = W$.
Hence the derivation 
$
\vcenter{
\infer[]{
Q \cpar R
}{
\infer[]{
Q \cpar U
}{
W
}}}
$
can be constructed, as required.
}
\begin{comment}
The second principal case for \textit{new} is when the bottommost rule of a proof begins as follows, where $\nfv{x}{Q \cpar R}$.
$
\new x P \cpar Q \cpar R
\longrightarrow
\new x \left( P \cpar Q \right) \cpar R
$
where $\vdash \new x \left( P \cpar Q \right) \cpar R$ is provable.
By, the induction hypothesis, there exist formulae $U_i$ and $V_i$ where $\vdash P \cpar Q \cpar V_i$ and either $U_i = V_i$ or $U_i = \wen x V_i$, for $1 \leq i \leq n$, and $n$-ary killing context such that the following holds:
$R \longrightarrow \tcontext{ U_1, U_2, \hdots, U_n }$.
Furthermore, the size of the proof of $P \cpar Q \cpar V_i$ is bounded above by the size of the proof of $\new x \left( P \cpar Q \right) \cpar R$; thereby strictly bounded by the size of the proof of $\new x P \cpar Q \cpar R$.

If $U_i = \wen x V_i$, define $W_i = \wen x \left( Q \cpar V_i \right)$ and if $U_i = V_i$ define $W_i = Q \cpar V_i$ .
In the cases where $U_i = \wen x V_i$, since $\nfv{x}{Q}$, $Q \cpar \wen x V_i \longrightarrow \wen x \left( Q \cpar V_i \right) = W_i$.
In the cases where $U_i = V_i$, $Q \cpar U_i = W_i$.
Hence the following derivation can be constructed, as required.
\[
Q \cpar R
\longrightarrow
\tcontext{ U_1, U_2, \hdots, U_n } 
\longrightarrow
\tcontext{ Q \cpar U_i \colon 1 \leq i \leq n } 
\longrightarrow
\tcontext{ W_i \colon 1 \leq i \leq n } 
\]
\end{comment}


\item Consider the second principal case for \textit{new} where the \textit{medial new} rule is the bottommost rule of a proof of the form
\[
\infer[\mbox{such that $\vdash \mathopen{\new \vec{y}.} \left(\new x. P \cseq \new x. Q\right) \cpar R$.}]{
\mathopen{\new x. \new \vec{y}.}\left( P \cseq Q \right) \cpar R
}{
\mathopen{\new \vec{y}.} \left( \new x. P \cseq \new x. Q \right) \cpar R
}
\]
The $\vec{y}$ is required to handle cases induced by equivariance.
By applying the induction hypothesis repeatedly, 
there exists $\vec{z}$ and $\hat{R}$ such that $\vec{z} \subseteq \vec{y}$ and
$\nfv{\vec{y}}{\wen \vec{z}. \hat{R} }$ and $\vdash \left( \new x. P \cseq \new x. Q \right) \cpar \hat{R}$, and also $\vcenter{\infer[]{R}{\hat{R}}}$.
Furthermore, the size of the proof of $\left( \new x. P \cseq \new x. Q \right) \cpar \hat{R}$ is bounded above by the size of the proof of $\mathopen{\new \vec{y}.} \left( \new x. P \cseq \new x. Q \right) \cpar R$.
By the induction hypothesis, there exist $S_i$ and $T_i$ such that $\vdash \new x. P \cpar S_i$ and $\vdash \new x. Q \cpar T_i$, for $1 \leq i \leq n$, and $n$-ary killing context such that 
$\vcenter{
\infer[]{\hat{R}}{\tcontext{ S_1 \cseq T_1, S_2 \cseq T_2, \hdots, S_n \cseq T_n }}}
$.
Furthermore, the size of the proofs of $\new x. P \cpar S_i$ and $\new x. Q \cpar T_i$ are bounded above by the size of the proof of $\left(\new x. P \cseq \new x. Q\right) \cpar R$.
By the induction hypothesis again, there exist $U^i$ and $\hat{U}^i$ such that $\vdash P \cpar \hat{U}^i$ and $\nfv{x}{U^i}$ and either $U^i = \hat{U}^i$ or $U^i = \wen x. \hat{U}^i$,
and also
$
\vcenter{\infer[]{S_i}{U^i}}
$.
Also by the induction hypothesis, there exist $V^i$ and $\hat{V}^i$ such that $\vdash Q \cpar \hat{V}^i$ and $\nfv{x}{V^i}$ and either $V^i = \hat{V}^i$ or $V^i = \wen x. \hat{V}^i$, 
and also
$
\vcenter{\infer[]{T_i}{V^i}}
$.
Now define $W$ and $\hat{W}$ such that
$
\hat{W} = \wen \vec{z}. \tcontext{ \hat{U}^i \cseq \hat{V}^i \colon 1 \leq i \leq n }
$
and, 
if for all $1 \leq i \leq n$, $U^i = \hat{U}^i$ and $V^i = \hat{V}^i$, then $W = \hat{W}$;
otherwise $W = \wen x. \hat{W}$.
Hence for each $i$, one of the following derivations holds.
\begin{itemize}
\item 
$U^i = \hat{U}^i$ and $V^i = \hat{V}^i$ hence
$U^i \cseq V^i = \hat{U}^i \cseq \hat{V}^i$.

\item
If $U^i = \wen x. \hat{U}^i$ and $V^i = \hat{V}^i$, hence $\nfv{x}{V^i}$, by the \textit{left wen} rule
$
\vcenter{
\infer[]{
\wen x. \hat{U}^i \cseq \hat{V}^i
}{
\mathopen{\wen x.} \left( \hat{U}^i \cseq \hat{V}^i \right)
}}
$.

\item
If $U^i = \hat{U}^i$, hence $\nfv{x}{\hat{U}^i}$, and $V^i = \wen x. \hat{V}^i$, by the \textit{right wen} rule
$
\vcenter{
\infer[]{
U^i \cseq \wen x. \hat{V}^i
}{
\mathopen{\wen x.} \left( \hat{U}^i \cseq \hat{V}^i \right)
}}
$.

\item
Otherwise by the \textit{suspend} rule $
\vcenter{\infer[]{
\wen x. \hat{U}^i \cseq \wen x. \hat{V}^i
}{
\mathopen{\wen x.} \left( \hat{U}^i \cseq \hat{V}^i \right)
}}
$
\end{itemize}
If for all $i$ such that $1 \leq i \leq n$, $U^i = \hat{U}^i$ and $V^i = \hat{V}^i$ then $W = \hat{W}$. Otherwise, by Lemma~\ref{lemma:commute},
$
\vcenter{
\infer[]{
\wen \vec{z}. \tcontext{
 U^i \cseq V^i \colon 1 \leq i \leq n
}
}{
\wen \vec{z}. \wen x. \tcontext{
 \hat{U}^i \cseq \hat{V}^i \colon 1 \leq i \leq n
}
}}$, where the premise is equialent to $W$.
Thereby the  derivation below left can be constructed, and furthermore, using Lemma~\ref{lemma:close}, 
the proof below right can also be constructed.
\[
\infer[]{
R
}{
\infer[]{
\wen \vec{z}. \hat{R}
}{
\infer[]{
\wen \vec{z}. \tcontext{ S_i \cseq T_i \colon 1 \leq i \leq n }
}{
\infer[]{
\wen \vec{z}. \tcontext{ U^i \cseq V^i \colon 1 \leq i \leq n }
}{
W
}}}}
\qquad
\infer[]{
\mathopen{\new \vec{y}.} \left( P \cseq Q \right) \cpar \hat{W}
}{
\infer[]{
\mathopen{\new \vec{y}.} \left(\left( P \cseq Q \right) \cpar
\tcontext{
 \hat{U}^i \cseq \hat{V}^i \colon 1 \leq i \leq n
}
\right)
}{
\infer[]{
\mathopen{\new \vec{y}.} 
\tcontext{
 \left( P \cseq Q \right) \cpar \left(\hat{U}^i \cseq \hat{V}^i\right) \colon 1 \leq i \leq n
}
}{
\infer[]{
\mathopen{\new \vec{y}.} 
\tcontext{
 \left( P \cpar \hat{U}^i \right) \cseq \left( Q \cpar \hat{V}^i\right) \colon 1 \leq i \leq n
}
}{
\infer[]{
\mathopen{\new \vec{y}.} 
\tcontext{
 \cunit \colon 1 \leq i \leq n
}
}{
\cunit
}}}}}
\]



By Lemma~\ref{lemma:bound},
$\size{\hat{W}} \preceq \size{R}$; hence $\size{ \mathopen{\new \vec{y}.} \left( P \cseq Q \right) \cpar \hat{W} } \prec \size{ \mathopen{\new x. \new \vec{y}.} \left( P \cseq Q \right) \cpar R }$ since the \textit{new count} strictly decreases, as required.
\begin{comment}
Consider the third principal case for \textit{new} where the \textit{medial new} rule is the bottommost rule of a proof of the form
$
\new x \left( P \cseq Q \right) \cpar R
\longrightarrow 
\left( \new x P \cseq \new x Q \right) \cpar R
$
such that $\vdash \left(\new x P \cpar \new x Q\right) \cpar R$ and $\nfv{x}{R}$.

By induction, there exist $S_i$ and $T_i$ such that $\vdash \new x P \cpar S_i$ and $\vdash \new x Q \cpar T_i$, for $1 \leq i \leq n$, and $n$-ary killing context such that 
$
R \longrightarrow \tcontext{ S_1 \cseq T_1, S_2 \cseq T_2, \hdots, S_n \cseq T_n }
$.
Furthermore, the size of the proofs of $\new x P \cpar S_i$ and $\new x Q \cpar T_i$ are bounded above by the size of the proof of $\left(\new x P \cseq \new x Q\right) \cpar R$.
Observe that by $\alpha$-conversion, $\vdash \new y \left(P\sub{x}{y}\right) \cpar S_i$ and $\vdash \new y \left(Q\sub{x}{y}\right) \cpar T_i$ also hold, where $\nfv{y}{\new x P \cpar S_i}$ and $\nfv{y}{\new x Q \cpar T_i}$.

By induction, there exist $U^i_j$ and $\hat{U}^i_j$ such that $\vdash P\sub{x}{y} \cpar \hat{U}^i_j$ and $\nfv{y}{U^i_j}$ and either $U^i_j = \hat{U}^i_j$ or $U^i_j = \wen y \hat{U}^i_j$, for $1 \leq j \leq m_i$, and $m_i$-ary killing context $\tcontextmn{i}{0}{}$ such that 
$
S_i \longrightarrow \tcontextmn{i}{0}{ U^i_1, U^i_2, \hdots, U^i_{m_i} }
$.
Also by induction, there exist $V^i_k$ and $\hat{V}^i_k$ such that $\vdash Q\sub{x}{y} \cpar \hat{V}^i_k$ and $\nfv{y}{V^i_k}$ and either $V^i_k = \hat{V}^i_k$ or $V^i_k = \wen y \hat{V}^i_k$, for $1 \leq k \leq \ell_i$, and $\ell_i$-ary killing context $\tcontextmn{i}{1}{}$ such that 
$
T_i \longrightarrow \tcontextmn{i}{1}{ V^i_1, V^i_2, \hdots, V^i_{\ell_i} }
$.

Now define $W_i$ and $\hat{W}_i$ such that
$
\hat{W}_i = \tcontextmn{i}{0}{ \hat{U}^i_j \colon 1 \leq j \leq m_i } \cseq \tcontextmn{i}{1}{ \hat{V}^i_k \colon 1 \leq k \leq \ell_i }
$
and, if $U^i_j = \hat{U}^i_j$ and $V^i_k = \hat{V}^i_k$ for all $j$ and $k$, then $W_i = \hat{W}_i$;
otherwise $W_i = \wen x \hat{W}_i$. 
In cases where for some $1 \leq j \leq m_i$, $U^i_j = \wen x \hat{U}^i_j$, by Lemma~\ref{lemma:commute} $\tcontextmn{i}{0}{ U^i_1, U^i_2, \hdots, U^i_{m_i} } \longrightarrow \wen x \tcontextmn{i}{0}{ \hat{U}^i_1, \hat{U}^i_2, \hdots, \hat{U}^i_{m_i} }$;
In cases where for some $1 \leq k \leq \ell_i$, $V^i_k = \wen x \hat{V}^i_k$, by Lemma~\ref{lemma:commute} $\tcontextmn{i}{1}{ V^i_1, V^i_2, \hdots, V^i_{\ell_i} } \longrightarrow \tcontextmn{i}{1}{ \hat{V}^i_1, \hat{V}^i_2, \hdots, \hat{V}^i_{\ell_i} }$.
Hence for each $i$, one of the of the following derivations holds.
\begin{itemize}
\item 
$U^i_j = \hat{U}^i_j$ and $V^i_k = \hat{V}^i_k$ for all $j$ and $k$ hence
$\tcontextmn{i}{0}{ U^i_j \colon 1 \leq j \leq m_i } \cseq \tcontextmn{i}{1}{ V^i_k \colon 1 \leq k \leq \ell_i } = W_i$.

\item
If $U^i_j = \hat{U}^i_j$ for all $j$, hence $\nfv{y}{U^i_j}$ for all $j$, and hence $\nfv{y}{Q}$ thereby $\nfv{y}{\tcontextmn{i}{0}{ \hat{U}^i_j \colon 1 \leq j \leq m_i }}$, hence by using the \textit{left wen} rule
$
\tcontextmn{i}{0}{ U^i_j \colon 1 \leq j \leq m_i } \cseq \tcontextmn{i}{1}{ V^i_k \colon 1 \leq k \leq \ell_i }
\longrightarrow
\wen x \tcontextmn{i}{0}{ \hat{U}^i_j \colon 1 \leq j \leq m_i } \cseq \tcontextmn{i}{1}{ \hat{V}^i_k \colon 1 \leq k \leq \ell_i }
\longrightarrow
W_i
$.

\item
If $V^i_k = \hat{V}^i_k$ for all $k$, hence $\nfv{y}{V^i_j}$ for all $k$, thereby $\nfv{y}{\tcontextmn{i}{1}{ \hat{V}^i_k \colon 1 \leq k \leq \ell_i }}$, hence by using the \textit{right wen} rule
$
\tcontextmn{i}{1}{ U^i_j \colon 1 \leq j \leq m_i } \cseq \tcontextmn{i}{1}{ V^i_k \colon 1 \leq k \leq \ell_i }
\longrightarrow
\tcontextmn{i}{1}{ \hat{U}^i_j \colon 1 \leq j \leq m_i } \cseq \wen x \tcontextmn{i}{1}{ \hat{V}^i_k \colon 1 \leq k \leq \ell_i }
\longrightarrow
W_i
$.

\item
Otherwise by using the \textit{suspend} rule the following derivation holds.
$
\tcontextmn{i}{0}{ U^i_j \colon 1 \leq j \leq m_i } \cseq \tcontextmn{i}{1}{ V^i_k \colon 1 \leq k \leq \ell_i }
\longrightarrow
\wen x \tcontextmn{i}{0}{ \hat{U}^i_j \colon 1 \leq j \leq m_i } \cseq \wen x \tcontextmn{i}{1}{ \hat{V}^i_k \colon 1 \leq k \leq \ell_i }
\longrightarrow
W_i
$
\end{itemize}
Thereby the following derivation can be constructed.
\[
\begin{array}{rl}
R
\longrightarrow&
\tcontext{ S_i \cseq T_i \colon 1 \leq i \leq n }
\\
\longrightarrow&
\tcontext{ \tcontextmn{i}{0}{ U^i_j \colon 1 \leq j \leq m_i } \cseq \tcontextmn{i}{1}{ V^i_k \colon 1 \leq k \leq \ell_i } \colon 1 \leq i \leq n }
\longrightarrow
\tcontext{ W_i }
\end{array}
\]
Furthermore, the following proof can be constructed.
\[
\begin{array}{rl}
\left( P \cseq Q \right)\sub{x}{y} \cpar \hat{W}_i
\longrightarrow&
\left( P\sub{x}{y} \cpar \tcontextmn{i}{0}{ \hat{U}^i_j \colon 1 \leq j \leq m_i } \right) \cseq
\left( Q\sub{x}{y} \cpar \tcontextmn{i}{1}{ \hat{V}^i_k \colon 1 \leq k \leq \ell_i } \right)
\\
\longrightarrow&
\tcontextmn{i}{0}{ P\sub{x}{y} \cpar \hat{U}^i_j \colon 1 \leq j \leq m_i } \cseq
\tcontextmn{i}{1}{ Q\sub{x}{y} \cpar \hat{V}^i_k \colon 1 \leq k \leq \ell_i }
\\
\longrightarrow&
\tcontextmn{i}{0}{ \cunit \colon 1 \leq j \leq m_i } \cseq
\tcontextmn{i}{1}{ \cunit \colon 1 \leq k \leq \ell_i }
\longrightarrow \cunit
\end{array}
\]
By Lemma~\ref{lemma:bound} and since the \textit{new count} strictly decreases,
$\size{\hat{W}_i} \preceq \size{R}$; thereby $\size{ \left( P \cseq Q \right)\sub{x}{y} \cpar \hat{W}_i } \prec \size{ \new x \left( P \cseq Q \right) \cpar R }$ as required.
\end{comment}
\smallskip

\item Consider the third principal  case for \textit{new} where the bottommost rule of a proof is the \textit{new wen} rule of the form
\[
\infer[\mbox{, where $\vdash \new \vec{z}. \wen y. \new x. P \cpar Q$.} ]{
\new x. \new \vec{z}. \wen y. P \cpar Q
}{
\new \vec{z}. \wen y. \new x. P \cpar Q
}
\]
By applying the induction hypothesis repeatedly, 
there exist $\vec{w}$ and $\hat{Q}$ such that $\vec{w} \subseteq \vec{z}$
and $\nfv{\vec{z}}{\wen \vec{w} .\hat{Q}}$ and $\vdash \wen y. \new x. P \cpar \hat{Q}$, and also $\vcenter{\infer[]{Q}{\wen \vec{w}. \hat{Q}}}$. Furthermore, the size of the proof of $\wen y. \new x. P \cpar \hat{Q}$ is bounded above by the size of the proof of $\new \vec{z}. \wen y.\new x. P \cpar Q$.
By the induction hypothesis, there exist $R$ and $S$ such that $\nfv{x}{R}$ and $\vdash \new x. P \cpar S$ and either $R = S$ or $R = \new y. S$, and also 
$
\vcenter{
\infer[]{\hat{Q}}{R}}
$.
Furthermore, the size of the proof of $\new x. P \cpar S$ is bounded above by the size of the proof of $\wen y. \new x. P \cpar Q$, hence strictly bounded above by the size of the proof of $\new x. \wen y. P \cpar Q$ enabling the induction hypothesis.
By the induction hypothesis again, there exist $U$ and $V$ such that $\nfv{x}{U}$ and $\vdash P \cpar V$ and either $U = V$ or $U = \wen x. V$, and also
$
\vcenter{
\infer[]{S}{U}}
$.

Let $W$ and $\hat{W}$ be defined such that, if $R = \new y. S$, then $\hat{W} = \new y. V$;
or, if $R = S$, then $\hat{W} = V$.
If $V = U$ then define $W = \wen \vec{w}. \hat{W}$.
If $U = \wen x. V$, then define $W = \wen x. \wen \vec{w}. \hat{W}$.
There are four scenarios for constructing a derivation with premise $W$ and conclusion $\wen \vec{w}. R$.
\begin{itemize}
\item In the case $V = U$ and $R = \new y. S$ then $\wen \vec{w}. \new y. U = W$.

\item If $V = U$ and $R = S$ then $\wen \vec{w}. U = W$.

\item If both $U = \wen x. V$ and $R = \new y. S$ hold, then 
we have 
\[
\infer[\mbox{, where the premise is $W$.}]{
\wen \vec{w}.R
}{
\infer[]{
\wen \vec{w}. \new y. \wen x. V }{
\wen x. \wen \vec{w}. \new y. V
}}
\]

\item If both $U = \wen x. V$ and $R = S$ then 
$
\vcenter{
\infer[]{
\wen \vec{w}.R
}{
 \wen \vec{w}. U
}}
$,
where the premise is equivalent to 
$W$.
\end{itemize}
Thereby, by applying one of the above cases,  we have $\vcenter{
\infer[.]{
\hat{Q}
}{
\infer[]{
\wen \vec{w}. Q
}{
\infer[]{
\wen \vec{w}. R
}{
W
}}}}
$

In the case that $\hat{W} = \new y. V$,  the left most derivation below holds. 
In the case, $\hat{W} = V$ and $\nfv{y}{V}$ the middle derivation below holds.  
Hence in either case, appealing to Lemma~\ref{lemma:close}, the proof below right can be constructed:
\[
\infer[]{
\wen y. P \cpar \new y. V
}{
\mathopen{\new y.} \left( P \cpar V \right)
}
\qquad
\infer[]{
\wen y. P \cpar \hat{W}
}{
\infer[]{
\mathopen{\wen y.} \left( P \cpar V \right)
}{
\mathopen{\new y.} \left( P \cpar V \right)
}}
\qquad
\infer[]{
\new \vec{z}. \wen y. P \cpar \wen \vec{w}. \hat{W}
}{
\infer[]{
\mathopen{\new \vec{z}.} \left( \wen y. P \cpar \hat{W} \right)
}{
\infer[]{
\mathopen{\new \vec{z}. \new y.} \left( P \cpar V \right)
}{
\infer[]{
\new \vec{z}. \new y. \cunit
}{
\cunit
}}}}
\]

\noindent Furthermore, by Lemma~\ref{lemma:bound}, $\size{ \wen \vec{w}. \hat{W} } \preceq \size{ Q }$.
Hence $\size{\wen y. P \cpar \wen \vec{w}. \hat{W}} \prec \size{ \new x. \new \vec{z}. \wen y. P \cpar Q }$ since the \textit{new} count strictly decreases.
\begin{comment}
Consider the fourth principal  case for \textit{new} where the bottommost rule of a proof is the \textit{new wen} rule of the form
$
\new x \wen y P \cpar Q
\longrightarrow
\wen y \new x P \cpar Q
$
where $\vdash \wen y \new x P \cpar Q$, $\nfv{x}{Q}$ and $\nfv{y}{Q}$.

By induction, there exist $R_i$ and $S_i$ such that $\vdash \new x P \cpar S_i$ and either $R_i = S_i$ or $R_i = \new y S_i$, for $1 \leq i \leq n$,
and $n$-ary killing context $\tcontext{}$ such that
$
Q \longrightarrow \tcontext{ R_1, R_2, \hdots, R_n }
$.
Furthermore, the size of the proof of $\new x P \cpar S_i$ is bounded above by the size of the proof of $\wen y \new x P \cpar Q$ enabling the induction hypothesis.
By induction again, there exist $U^i_j$ and $V^i_j$ such that $\vdash P \cpar V^i_k$ and either $U^i_j = V^i_j$ or $U^i_j = \wen x V^i_j$, for $1 \leq j \leq m_i$, and $n$-ary killing context $\tcontextn{i}{}$ such that 
$
S_i \longrightarrow
\tcontextn{i}{ U^i_1, U^i_2, \hdots, U^i_{m_i} }
$.

Let $W_i$ and $\hat{W}_i$ be defined such that, if $R_i = \new y S_i$, then $\hat{W}_i = \new y \tcontextn{i}{ V^i_j \colon 1 \leq j \leq m_i }$;
or, if $R_i = S_i$, then $\hat{W}_i = \tcontextn{i}{ V^i_j \colon 1 \leq j \leq m_i }$.
If $V^i_j = U^i_j$ for all $1 \leq j \leq m_i$, then define $W_i = \hat{W}_i$.
If for some $1 \leq j \leq m_i$ $U^i_j = \wen x V^i_j$, then define $W_i = \wen x \hat{W}_i$.

In the following fix some $i$ such that $1 \leq i \leq n$. There are four scenarios for $R_i \longrightarrow W_i$.
In the case $V^i_j = U^i_j$ for all $1 \leq j \leq m_i$, if $R_i = \new y S_i$ then $\new y \tcontextn{i}{ U^i_j \colon 1 \leq j \leq m_i } = W_i$ and if $R_i = S_i$ then $\tcontextn{i}{ U^i_j \colon 1 \leq j \leq m_i } = W_i$.


If, for some $1 \leq j \leq m_i$, $U^i_j = \wen x V^i_j$, then $\tcontextn{i}{ U^i_j \colon 1 \leq j \leq m_i } \longrightarrow \wen x \tcontextn{i}{V^i_j \colon 1 \leq j \leq m_i}$, by Lemma~\ref{lemma:commute}, and one of the following hold.
\begin{itemize}
\item If, further to $U^i_j = \wen x V^i_j$ for some $1 \leq j \leq m_i$, $R_i = \new y S_i$ then 
$
R_i \longrightarrow \new y \tcontextn{i}{ U^i_j \colon 1 \leq j \leq m_i } \longrightarrow \new y \wen x \tcontextn{i}{ V^i_j \colon 1 \leq j \leq m_i } \longrightarrow \wen x \new y \tcontextn{i}{ V^i_j \colon 1 \leq j \leq m_i } =  W_i
$.
\item
If, further to $U^i_j = \wen x V^i_j$ for some $1 \leq j \leq m_i$,  $R_i = S_i$ then 
$
R_i \longrightarrow \tcontextn{i}{ U^i_1, U^i_2, \hdots, U^i_{m_i} }
\longrightarrow \wen x \tcontextn{i}{ V^i_1, V^i_2, \hdots, V^i_{m_i} } = W_i
$.
\end{itemize}
Thereby, by applying one of the above cases for each $i$, the following derivation can be constructed.
$
Q
\longrightarrow
 \tcontext{ R_1, R_2, \hdots, R_n }
\longrightarrow
 \tcontext{ W_1, W_2, \hdots, W_n }
$.

Now observe that the following proof can always be constructed for every $i$ such that $1 \leq i \leq n$.
\[
\begin{array}{rl}
\new y \left( P \cpar \tcontextn{i}{ V^i_j \colon 1 \leq j \leq m_i } \right)
\longrightarrow
\new y \tcontextn{i}{ P \cpar V^i_j \colon 1 \leq j \leq m_i }
\longrightarrow
\new y \tcontextn{i}{ \cunit \colon 1 \leq j \leq m_i }
\longrightarrow
\cunit
\end{array}
\]
By using the proof above, a proof can be constructed for each $i$ in the following two cases.

In the first case where $\hat{W}_i = \new y \tcontextn{i}{ V^i_j \colon 1 \leq j \leq m_i }$, a proof can be constructed such that
$
\wen y P \cpar \hat{W}_i
\longrightarrow
\new y \left( P \cpar \tcontextn{i}{ V^i_j \colon 1 \leq j \leq m_i } \right)
\longrightarrow
\cunit
$
Furthermore, by Lemma~\ref{lemma:bound}, $\size{ \new y \tcontextn{i}{ V^i_j \colon 1 \leq j \leq m_i } } \preceq \size{ Q }$. Hence 
Hence $\size{ \wen y P \cpar \new y \tcontextn{i}{ V^i_j \colon 1 \leq j \leq m_i }} \prec \size{ \new x \wen y P \cpar Q }$.

In the second case, $\hat{W}_i = \tcontextn{i}{ V^i_j \colon 1 \leq j \leq m_i }$ and, since $\nfv{y}{R}$, by Lemma~\ref{lemma:nfv} we have $\nfv{y}{\hat{W}_i}$.
Hence the following proof can be constructed, using the above proof:
$
\wen y P \cpar \hat{W}_i
\longrightarrow
\new y \left( P \cpar \tcontextn{i}{ V^i_j \colon 1 \leq j \leq m_i } \right)
\longrightarrow
\cunit
$.
Furthermore, by Lemma~\ref{lemma:bound}, $\size{ \tcontextn{i}{ V^i_j \colon 1 \leq j \leq m_i } } \preceq \size{ Q }$.
Hence $\size{\wen y P \cpar \tcontextn{i}{ V^i_j \colon 1 \leq j \leq m_i }} \prec \size{\new x \wen y P \cpar Q}$.
\end{comment}

\end{enumerate}



\item \textbf{Principal cases for seq.} There are two forms of principal cases for \textit{seq}.
The first case, induced by the \textit{sequence} rule, is the case that forces the \textit{medial}, \textit{medial1} and \textit{medial new} rules.
The other cases are induced by the \textit{suspend}, \textit{left wen} and \textit{right wen} rules (which are forced as a knock on effect of the \textit{medial new} rule).

\begin{enumerate}[label*=\textbf{.\arabic*}]
\item Consider the first principal case for \textit{seq}.
The difficulty in this case is that, due to associativity of \textit{seq}, the \rseq rule may be applied in several ways when there are multiple occurrences of \textit{seq}.
Consider a principal formula of the form $\left(T_0 \cseq T_1\right) \cseq T_2$, where we aim to split the formula around the second \textit{seq} operator. The difficulty is that the bottommost rule may be an instance of the \rseq rule applied between $T_0$ and $T_1 \cseq T_2$. Symmetrically, the principal formula may be of the form $T_0 \cseq \left(T_1 \cseq T_2\right)$ but the bottommost rule may be an instance of the \rseq rule applied between $T_0 \cseq T_1$ and $T_2$.
In the following analysis, only the former case is considered; the symmetric case follows a similar pattern.
The principal formula is $\left(T_0 \cseq T_1\right) \cseq T_2$ and the bottommost rule is an instance of the \textit{sequence} rule of the form
\[
\infer[]{
\left( T_0 \cseq T_1 \cseq T_2 \right) \cpar \left( U \cseq V \right) \cpar W
}{
\left(\left( T_0 \cpar U \right) \cseq \left( \left(T_1 \cseq T_2\right) \cpar V \right)\right) \cpar W
}
\] 
where $T_0 \not\equiv \cunit$, $T_2 \not\equiv \cunit$ (otherwise splitting is trivial), and either $U \not\equiv \cunit$ or $V \not\equiv \cunit$ (otherwise the \rseq rule cannot be applied);
and also $\vdash \left(\left( T_0 \cpar U \right) \cseq \left( \left(T_1 \cseq T_2\right) \cpar V \right)\right) \cpar W$.
By the induction hypothesis, there exist $P_i$ and $Q_i$ such that $\vdash T_0 \cpar U \cpar P_i$ and $\vdash \left(T_1 \cseq T_2\right) \cpar V \cpar Q_i$ hold, for $1 \leq i \leq n$, and an $n$-ary killing context $\tcontext{}$ such that 
\[
\infer[.]{W}{
\tcontext{ P_1 \cseq Q_1, \hdots, P_n \cseq Q_n }
}
\]
Furthermore,
the size of the proof of formula
$
\left(T_1 \cseq T_2\right) \cpar V \cpar Q_i
$
is bounded above by the size of the proof of
$\left(\left( T_0 \cpar U \right) \cseq \left( \left(T_1 \cseq T_2\right) \cpar V \right)\right) \cpar W$, hence the induction hypothesis is enabled.
By the induction hypothesis, there exists $R^i_j$ and $S^i_j$ such that $\vdash T_1 \cpar R^i_j$ and $\vdash T_2 \cpar S^i_j$, for $1 \leq j \leq m_i$, and $m_i$-ary killing context $\tcontextn{i}{}$ such that
\[
\infer[.]{
V \cpar Q_i
}{
\tcontextn{i}{R^i_1 \cseq S^i_1, \hdots, R^i_{m_i} \cseq S^i_{m_i}}
}
\]
Furthermore, by Lemma~\ref{lemma:medial} there exist killing contexts $\tcontextmn{i}{0}{}$ and $\tcontextmn{i}{1}{}$ and sets of integers $J^i \subseteq \left\{1, \hdots, n\right\}$, $K^i \subseteq \left\{1, \hdots, n\right\}$ such that
\[
\infer[.]{
\tcontextn{i}{R^i_1 \cseq S^i_1, \hdots, R^i_{m_i} \cseq S^i_{m_i}}
}{
\tcontextmn{i}{0}{R^i_j \colon j \in J^i}
 \cseq
\tcontextmn{i}{1}{S^i_k \colon k \in K^i}
}
\]
Thereby, the following derivation can be constructed.
\[
\infer[]{
\left( U \cseq V \right) \cpar W
}{
\infer[]{
\left( U \cseq V \right) \cpar \tcontext{ P_1 \cseq Q_1, \hdots, P_n \cseq Q_n } 
}{
\infer[]{
 \tcontext{ \left( U \cseq V \right) \cpar \left( P_1 \cseq Q_1 \right), \hdots, \left( U \cseq V \right) \cpar \left( P_n \cseq Q_n \right) } 
}{
\infer[]{
\tcontext{ \left( U \cpar P_1 \right) \cseq \left( V \cpar Q_1 \right), \hdots, \left( U \cpar P_n \right) \cseq \left( V \cpar Q_n \right) } 
}{
\infer[]{
 \tcontext{ \left( U \cpar P_i \right) \cseq \tcontextn{i}{R^i_j \cseq S^i_j \colon 1 \leq j \leq m_i } \colon 1 \leq i \leq n } 
}{
\tcontext{
 \left( U \cpar P_i \right) \cseq \tcontextmn{i}{0}{R^i_j \colon j \in J^i} \cseq \tcontextmn{i}{1}{S^i_k \colon k \in K^i}
 \colon 1 \leq i \leq n 
} 
}}}}}
\]
Furthermore, the following two proofs can be constructed.
\[
\infer[]{
T_2 \cpar \tcontextn{i}{ S^i_j \colon 1 \leq j \leq m_i }
}{
\infer[]{
\tcontextn{i}{ T_2 \cpar S^i_j \colon 1 \leq j \leq m_i }
}{
\infer[]{
\tcontextn{i}{ \cunit \colon 1 \leq j \leq m_i }
}{
\cunit
}}}
\qquad\qquad
\infer[]{
\left(T_0 \cseq T_1\right) \cpar \left(\left( U \cpar P_i \right) \cseq \tcontextn{i}{R^i_j \colon 1 \leq j \leq m_i }\right)
}{
\infer[]{
\left(T_0 \cpar U \cpar P_i\right) \cseq \left(T_1 \cpar \tcontextn{i}{R^i_j \colon 1 \leq j \leq m_i }\right)
}{
\infer[]{
T_1 \cpar \tcontextn{i}{R^i_j \colon 1 \leq j \leq m_i }
}{
\infer[]{
\tcontextn{i}{T_1 \cpar R^i_j \colon 1 \leq j \leq m_i }
}{
\infer[]{
\tcontextn{i}{\cunit \colon 1 \leq j \leq m_i }
}{
\cunit
}}}}}
\]
By Lemma~\ref{lemma:bound}, 
\[\size{ \tcontext{
 \left( U \cpar P_1 \right) \cseq \tcontextmn{i}{0}{R^i_j \colon j \in J^i} \cseq \tcontextmn{i}{1}{S^i_k \colon k \in K^i}
 \colon 1 \leq i \leq n 
} }  \preceq \size{ \left( U \cseq V \right) \cpar W }
\]
which are also upper bounds for
$
\size{ \tcontextmn{i}{0}{R^i_j \colon j \in J^i} 
}$
and
$\size{
\tcontextmn{i}{1}{S^i_k \colon k \in K^i}
}$.
Furthermore, $T_0 \not\equiv \cunit$ and $T_2 \not\equiv \cunit$ both $\occ{ T_0 } \mstrict \occ{ T_0 \cseq T_1 \cseq T_2 }$ and  $\occ{ T_2 } \mstrict \occ {T_0 \cseq T_1 \cseq T_2 }$
Hence the sizes of the above proofs of 
$T_2 \cpar \tcontextn{i}{ S^i_j \colon 1 \leq j \leq m_i }$ 
and 
\[
\left(T_0 \cseq T_1\right) \cpar \left(\left( U \cpar P_i \right) \cseq \tcontextn{i}{R^i_j \colon 1 \leq j \leq m_i }\right)
\]
 are strictly less than the size of the proof of $\left( T_0 \cseq T_1 \cseq T_2 \right) \cpar \left( U \cseq V \right) \cpar W$.



\item Consider the principal case for \textit{seq} where the bottommost rule of a proof is an instance of the \textit{suspend} rule of the form
\[
\infer[\mbox{, where $\vdash \left(P_0\cseq \mathopen{\wen x.} \left( P_1\cseq P_2  \right) \cseq P_3 \right) \cpar Q$ holds.}]{
\left(P_0\cseq \wen x. P_1\cseq \wen x. P_2  \cseq P_3 \right) \cpar Q
}{
\left(P_0\cseq \mathopen{\wen x.} \left( P_1\cseq P_2  \right) \cseq P_3 \right) \cpar Q
}
\] By induction, there exist $U^0_i$ and $U^1_i$ such that $\vdash P_0\cpar U^0_i$ and $\vdash \left( \mathopen{\wen x.} \left( P_1\cseq P_2  \right) \cseq P_3 \right) \cpar U^1_i$ hold, for $1 \leq i \leq n$,
and $n$-ary killing context $\tcontext{}$ such that 
$
\vcenter{
\infer[]{Q}{
\tcontext{ U^0_i \cseq U^1_i \colon 1 \leq i \leq n }
}}
$.
Furthermore the size of the proof of $\left( \mathopen{\wen x.} \left( P_1\cseq P_2  \right) \cseq P_3 \right) \cpar U^1_i$ is bounded above by the size of the proof of $\left(P_0\cseq \wen x. P_1\cseq \wen x. P_2  \cseq P_3 \right) \cpar Q$.
By induction again, there exist $V^i_j$ and $W^i_j$ such that $\vdash \mathopen{\wen x.} \left( P_1\cseq P_2  \right) \cpar V^i_j$ and $\vdash P_3 \cpar W^i_j$, for $1 \leq j \leq m_i$, and $m_i$-ary killing context $\tcontextn{i}{}$ such that the following derivation holds.
$
\vcenter{
\infer[]{
U^1_i
}{
\tcontextn{i}{ V^i_j \cseq W^i_j \colon 1 \leq j \leq m_i }
}}
$.
Furthermore, the size of the proof of $\mathopen{\wen x.} \left( P_1\cseq P_2  \right) \cpar V^i_j$ is bounded by the size of the proof of $\left( \mathopen{\wen x.} \left( P_1\cseq P_2  \right) \cseq P_3 \right) \cpar U^1_i$. 
By applying the induction hypothesis again, there exist $R^i_j$ and $\hat{R}^i_j$ such that $\nfv{x}{R^i_j}$ and $\vdash \left(P_1\cseq P_2 \right) \cpar \hat{R}^i_j$ and either $R^i_j = \hat{R}^i_j$ or $R^i_j = \new x. \hat{R}^i_j$, and also
$
\vcenter{
\infer[]{
V^i_j
}{
 R^i_j
}}
$.
Furthermore, the size of the proof of $\left(P_1\cseq P_2 \right) \cpar \hat{R}^i_j$ is bounded above by the size of the proof of $\left( \mathopen{\wen x.} \left( P_1\cseq P_2  \right) \cseq P_3 \right) \cpar U^1_i$.
By a fourth induction, there exist $S^{i,j}_k$ and $T^{i,j}_k$ such that both $\vdash P_1\cpar S^{i,j}_k$ and $\vdash P_2  \cpar T^{i,j}_k$ hold, for $1 \leq k \leq \ell^{i,j}$, and $\ell^{i,j}$-ary killing context $\tcontextn{i,j}{}$ such that the following derivation holds:
\[
\infer[]{
\hat{R}^i_j
}{
\tcontextn{i,j}{ S^{i,j}_1 \cseq T^{i,j}_1, S^{i,j}_2 \cseq T^{i,j}_2, \hdots, S^{i,j}_{\ell^{i,j}} \cseq T^{i,j}_{\ell^{i,j}} }
}.
\]
By Lemma~\ref{lemma:medial}, there exists some $I^i_j \subseteq \left\{ 1 \hdots \ell^{i,j} \right\}$ and $J^i_j \subseteq \left\{ 1 \hdots \ell^{i,j} \right\}$ and killing contexts $\tcontextmn{i,j}{0}{}$ and $\tcontextmn{i,j}{1}{}$ such that
\[
\infer[]{
\hat{R}^i_j
}{
\infer[]{
\tcontextn{i,j}{ S^{i,j}_k \cseq T^{i,j}_k \colon 1 \leq k \leq \ell^{i,j} }
}{
\tcontextmn{i,j}{0}{S^{i,j}_k \colon k \in I^i_j } \cseq
\tcontextmn{i,j}{1}{T^{i,j}_k \colon k \in J^i_j }
}}.
\]
Define $\hat{S}^i_j$ and $\hat{T}^i_j$ as follows.
If $R^i_j = \hat{R}^i_j$, then 
\[
\hat{S}^i_j = \tcontextmn{i,j}{0}{S^{i,j}_k \colon k \in I^i_j }
\mbox{ and }
\hat{T}^i_j = \tcontextmn{i,j}{1}{T^{i,j}_k \colon k \in J^i_j };
\]
and hence, we can construct the derivation 
\[
\infer[]{
R^i_j
}{
\tcontextmn{i,j}{0}{S^{i,j}_k \colon k \in I^i_j } \cseq
\tcontextmn{i,j}{1}{T^{i,j}_k \colon k \in J^i_j }
}
\] where the premise equals $\hat{S}^i_j  \cseq \hat{T}^i_j$.
If however $R^i_j = \new x. \hat{R}^i_j$, then define
\[
\hat{S}^i_j = \new x. \tcontextmn{i,j}{0}{S^{i,j}_k \colon k \in I^i_j }
\mbox{ and }
\hat{T}^i_j = \new x. \tcontextmn{i,j}{1}{T^{i,j}_k \colon k \in J^i_j };
\]
and hence, the derivation 
\[
\infer[]{
R^i_j
}{
\infer[]{
\mathopen{\new x.} \left(
 \tcontextmn{i,j}{0}{S^{i,j}_k \colon k \in I^i_j } \cseq
 \tcontextmn{i,j}{1}{T^{i,j}_k \colon k \in J^i_j }
\right)
}{
\hat{S}^i_j \cseq
\hat{T}^i_j
}}
\]
can be constructed.
By Lemma~\ref{lemma:medial}, for some $K^i \subseteq \left\{ 1\hdots m_i \right\}$, $L^i \subseteq \left\{ 1\hdots m_i \right\}$ and killing contexts $\tcontextmn{i}{0}{}$ and $\tcontextmn{i}{1}{}$, we obtain the following derivation:
\[
\infer[]{
\tcontextn{i}{
    \hat{S}^i_j \cseq 
    \hat{T}^i_j \cseq 
    W^i_j \colon 1 \leq j \leq m_i
}
}{
\tcontextmn{i}{0}{
    \hat{S}^i_j
    \colon j \in K^i
} \cseq
\tcontextmn{i}{1}{
    \hat{T}^i_j \cseq 
    W^i_j \colon j \in L^i
}
}
\]
By using the above derivations we can construct the following derivation:
\[
\infer[]{
Q
}{
\infer[]{
\tcontext{ U^0_i \cseq U^1_i \colon 1 \leq i \leq n }
}{
\infer[]{
\tcontext{ U^0_i \cseq \tcontextn{i}{ V^i_j \cseq W^i_j \colon 1 \leq j \leq m_i } \colon 1 \leq i \leq n }
}{
\infer[]{
\tcontext{ U^0_i \cseq \tcontextn{i}{ R^i_j \cseq W^i_j \colon 1 \leq j \leq m_i
                                    } \colon 1 \leq i \leq n }
}{
\infer[]{
\tcontext{
           U^0_i \cseq \tcontextn{i}{ \hat{S}^i_j \cseq 
                                                                        \hat{T}^i_j 
\cseq W^i_j \colon 1 \leq j \leq m_i
                                    }  \colon 1 \leq i \leq n }
}{
\tcontext{ 
           U^0_i \cseq
             \tcontextmn{i}{0}{
    \hat{S}^i_j 
    \colon j \in K^i
} \cseq
\tcontextmn{i}{1}{
    \hat{T}^i_j \cseq 
    W^i_j \colon j \in L^i
}
           \colon 1 \leq i \leq n  }
}}}}}
\]
\begin{comment}
We aim now to establish $
\vdash \left( P_0\cpar \wen x Q\right) \cpar
U^0_i \cseq
             \tcontextmn{i}{0}{
    \hat{S}^i_j 
    \colon j \in K^i
}
$ and
$\vdash \left( \wen x P_2  \cpar P_3 \right) \cpar
\tcontextmn{i}{1}{
    \hat{T}^i_j \cseq 
    W^i_j \colon j \in L^i
}
$.
\end{comment}

\noindent Consider whether the judgement $\vdash \wen x. P_1 \cpar \hat{S}^i_j$ holds.
We have two cases: in the first, $\hat{S}^i_j = \tcontextmn{i,j}{0}{S^{i,j}_k \colon k \in I^i_j }$ and $\nfv{x}{\hat{S}^i_j}$; in the second $\hat{S}^i_j = \new x. \tcontextmn{i,j}{0}{S^{i,j}_k \colon k \in I^i_j }$. In each case, one of the following derivations can be respectively constructed.
\[
\infer[]{
\wen x. P_1\cpar \tcontextmn{i,j}{0}{S^{i,j}_k \colon k \in I^i_j }
}{
\infer[]{
\new x. P_1\cpar \tcontextmn{i,j}{0}{S^{i,j}_k \colon k \in I^i_j }
}{
\mathopen{\new x.}\left( P_1\cpar \tcontextmn{i,j}{0}{S^{i,j}_k \colon k \in I^i_j } \right)
}}
\qquad\qquad
\infer[]{
\wen x. P_1\cpar \new x. \tcontextmn{i,j}{0}{S^{i,j}_k \colon k \in I^i_j }
}{
\mathopen{\new x.} \left( P_1\cpar \tcontextmn{i,j}{0}{S^{i,j}_k \colon k \in I^i_j } \right)
}
\]
Similarly, consider whether judgement $\vdash \wen x. P_2 \cpar \hat{T}^i_j$ holds.
Either we have 
\[
\hat{T}^i_j = \tcontextmn{i,j}{1}{T^{i,j}_k \colon k \in J^i_j } \mbox{ and } \nfv{x}{\hat{T}^i_j}; 
\]
or we have $\hat{T}^i_j = \new x. \tcontextmn{i,j}{1}{T^{i,j}_k \colon k \in J^i_j }$. In each case, one of the following derivations holds, respectively.
\[
\infer[]{
\wen x. P_2\cpar \tcontextmn{i,j}{1}{T^{i,j}_k \colon k \in J^i_j }
}{
\infer[]{
\new x. P_2\cpar \tcontextmn{i,j}{1}{T^{i,j}_k \colon k \in J^i_j }
}{
\mathopen{\new x.} \left( P_2\cpar \tcontextmn{i,j}{1}{T^{i,j}_k \colon k \in J^i_j } \right)
}}
\qquad\qquad
\infer[]{
\wen x. P_2\cpar 
\hat{T}^i_j
}{
\mathopen{\new x.} \left( P_2\cpar \tcontextmn{i,j}{1}{T^{i,j}_k \colon k \in J^i_j } \right)
}
\]
Thereby, by applying one of the above cases for each $i$ and $j$, 
the following two proofs exist.
\begin{comment}
\[
\begin{array}{rl}
\wen x. P_1\cpar \hat{S}^i_j
\longrightarrow&
\mathopen{\new x.} \left( P_1\cpar \tcontextmn{i,j}{0}{S^{i,j}_k \colon k \in I^i_j } \right)
\\
\longrightarrow&
\new x. \tcontextmn{i,j}{0}{ P_1\cpar S^{i,j}_k  \colon k \in I^i_j }
\longrightarrow
\new x. \tcontextmn{i,j}{0}{ \cunit \colon k \in I^i_j }
\longrightarrow
\cunit
\end{array}
\]
\end{comment}
\[
\infer[]{
\left( P_0 \cseq \wen x. P_1\right) \cpar
\left(
    U^0_i \cseq
    \tcontextmn{i}{0}{
         \hat{S}^i_j 
         \colon j \in K^i
    }
\right)
}{
\infer[]{
\left( P_0\cpar U^0_i \right) \cseq
\left(
    \wen x. P_1 \cpar
    \tcontextmn{i}{0}{
             \hat{S}^i_j 
      \colon j \in K^i
    }
\right)
}{
\infer[]{
    \wen x. P_1 \cpar
    \tcontextmn{i}{0}{
             \hat{S}^i_j 
      \colon j \in K^i
    }
}{
\infer[]{
    \tcontextmn{i}{0}{
         \wen x. P_1 \cpar
         \hat{S}^i_j 
      \colon j \in K^i
    }
}{
\infer[]{
    \tcontextmn{i}{0}{
         \mathopen{\new x.} \left( P_1\cpar \tcontextmn{i,j}{0}{S^{i,j}_k \colon k \in I^i_j } \right)
      \colon j \in K^i
    }
}{
\infer[]{
    \tcontextmn{i}{0}{
         \mathopen{\new x.} \tcontextmn{i,j}{0}{ P_1\cpar S^{i,j}_k  \colon k \in I^i_j }
      \colon j \in K^i
    }
}{
\infer[]{
    \tcontextmn{i}{0}{
         \new x. \tcontextmn{i,j}{0}{ \cunit \colon k \in I^i_j }
      \colon j \in K^i
    }
}{
\cunit
}}}}}}}
\qquad
\infer[]{
\left( \wen x. P_2  \cseq P_3 \right) \cpar
\left(
\tcontextmn{i}{1}{
    \hat{T}^i_j \cseq 
    W^i_j \colon j \in L^i
}\right)
}{
\infer[]{
\tcontextmn{i}{1}{
    \left( \wen x. P_2  \cseq P_3 \right) \cpar
    \left( \hat{T}^i_j \cseq W^i_j \right)
    \colon j \in L^i
}
}{
\infer[]{
\tcontextmn{i}{1}{
    \left( \wen x. P_2  \cpar \hat{T}^i_j\right) \cseq
    \left( P_3 \cpar W^i_j \right)
    \colon j \in L^i
}
}{
\infer[]{
\tcontextmn{i}{1}{
    \wen x. P_2  \cpar \hat{T}^i_j
    \colon j \in L^i
}
}{
\infer[]{
\tcontextmn{i}{1}{
    \mathopen{\new x.} \left( P_2\cpar \tcontextmn{i,j}{1}{T^{i,j}_k \colon k \in J^i_j } \right)
    \colon j \in L^i
}
}{
\infer[]{
\tcontextmn{i}{1}{
    \new x. \tcontextmn{i,j}{1}{ P_2\cpar T^{i,j}_k  \colon k \in J^i_j }
    \colon j \in L^i
}
}{
\infer[]{
\tcontextmn{i}{1}{
    \new x. \tcontextmn{i,j}{1}{ \cunit \colon k \in J^i_j }
    \colon j \in L^i
}
}{
\cunit
}}}}}}}
\]
Furthermore, by Lemma~\ref{lemma:bound}, 
\[
\size{ 
    U^0_i \cseq
    \tcontextmn{i}{0}{
         \hat{S}^i_j 
         \colon j \in K^i
    }
} \preceq \size{ Q }
\mbox{ and }
\size{ 
 \tcontextmn{i}{1}{
    \hat{T}^i_j \cseq 
    W^i_j \colon j \in L^i
} } \preceq \size{Q}.
\]
Hence, sizes
\[
\size{
\left( P_0 \cseq \wen x. P_1\right) \cpar
\left(
    U^0_i \cseq
    \tcontextmn{i}{0}{
         \hat{S}^i_j 
         \colon j \in K^i
    }
\right)
}
\mbox{ and }
\size {
\left( \wen x. P_2  \cseq P_3 \right) \cpar
\left(
\tcontextmn{i}{1}{
    \hat{T}^i_j \cseq 
    W^i_j \colon j \in L^i
}\right)
}
\]
are strictly bounded above by 
$\size{ \left(P_0\cseq \wen x. P_1\cseq \wen x. P_2  \cseq P_3 \right) \cpar Q }$, as required.
Cases for \textit{left wen} and \textit{right wen} rules are similar.
\begin{comment}
There are three similar principal case for \textit{seq} that do not correspond to cases in \textsf{MAV}~\cite{Horne2015}, induced by the \textit{suspend}, \textit{left wen} and \textit{right wen} rules.
We present here only the case induced by \textit{suspend}.
Consider the case where the bottommost rule of a proof is a \textit{suspend} rule that interferes with the \textit{seq} connective to which splitting is applied as follows.
\[
\left(P_0\cseq \wen x P_1\cseq \wen x P_2  \cseq P_3 \right) \cpar T
\longrightarrow
\left(P_0\cseq \wen x \left( P_1\cseq P_2  \right) \cseq P_3 \right) \cpar T
\]
In the above, it is assumed that $\vdash \left(P_0\cseq \wen x \left( P_1\cseq P_2  \right) \cseq P_3 \right) \cpar T$ and also $\nfv{x}{P}$, $\nfv{x}{S}$ and $\nfv{x}{T}$.
Hence, by induction, there exist $U^0_i$ and $U^1_i$ such that $\vdash P_0\cpar U^0_i$ and $\vdash \left( \wen x \left( P_1\cseq P_2  \right) \cseq P_3 \right) \cpar U^1_i$, for $1 \leq i \leq n$,
and $n$-ary killing context $\tcontext{}$ such that the following derivation holds.
$
T \longrightarrow \tcontext{ U^0_i \cseq U^1_i \colon 1 \leq i \leq n }
$.
Furthermore the size of the proof of $\left( \wen x \left( P_1\cseq P_2  \right) \cseq P_3 \right) \cpar U^1_i$ is bounded by the size of the proof of $\left(P_0\cseq \wen x P_1\cseq \wen x P_2  \cseq P_3 \right) \cpar T$.

By induction again, there exist $V^i_j$ and $W^i_j$ such that $\vdash \wen x \left( P_1\cseq P_2  \right) \cpar V^i_k$ and $\vdash P_3 \cpar W^i_j$, for $1 \leq j \leq m_i$, and $m_i$-ary killing context $\tcontextn{i}{}$ such that the following derivation holds.
$
U^1_i
\longrightarrow
\tcontextn{i}{ V^i_j \cseq W^i_j \colon 1 \leq j \leq m_i }
$.
Furthermore, the size of the proof of $\wen x \left( P_1\cseq P_2  \right) \cpar V^i_k$ is bounded by the size of the proof of $\left( \wen x \left( P_1\cseq P_2  \right) \cseq P_3 \right) \cpar U^1_i$. 

Since $\nfv{x}{T}$, by Lemma~\ref{lemma:nfv}, $\nfv{x}{V^i_k}$. By appling the induction hypothesis again, there exist $R^i_j_k$ and $\hat{R}^i_j_k$ such that $\vdash \left(P_1\cseq P_2 \right) \cpar \hat{R}^i_j_k$ and either $R^i_j_k = \hat{R}^i_j_k$ or $R^i_j_k = \new x \hat{R}^i_j_k$, for $1 \leq k \leq g^i_j$, and $g^i_j$-ary killing context $\tcontextmn{i}{j}{}$ such that 
$
V^i_j \longrightarrow \tcontextmn{i}{j}{ R^i_j_1, R^i_j_2, \hdots, R^i_j_{g^i_j} }
$.
Furthermore, the size of the proof of $\left(P_1\cseq P_2 \right) \cpar \hat{R}^i_j_k$ is bounded above by the size of the proof of $\left( \wen x \left( P_1\cseq P_2  \right) \cseq P_3 \right) \cpar U^1_i$.

By a fourth induction, there exist $S^{i,j,k}_k$ and $T^{i,j,k}_k$ such that $\vdash P_1\cpar S^{i,j,k}_k$ and $\vdash P_2  \cpar T^{i,j,k}_k$, for $1 \leq k \leq \ell^{i,j}_k$, and $\ell^{i,j}_k$-ary killing context $\tcontextmn{i,j}{k}{}$ such that the following derivation holds.
\[
\hat{R}^i_j_k
\longrightarrow
\tcontextmn{i,j}{k}{ S^{i,j,k}_1 \cseq T^{i,j,k}_1, S^{i,j,k}_2 \cseq T^{i,j,k}_2, \hdots, S^{i,j,k}_{\ell^{i,j}_k} \cseq T^{i,j,k}_{\ell^{i,j}_k} }
\]
In the case $R^i_j_k = \hat{R}^i_j_k$, by Lemma~\ref{lemma:medial}, the following derivation holds, where $I^i_j_k \subseteq \left\{ 1 \hdots \ell^{i,j}_k \right\}$ and $J^i_j_k \subseteq \left\{ 1 \hdots \ell^{i,j}_k \right\}$ and killing contexts $\tcontextmn{i,j}{k,0}{}$ and $\tcontextmn{i,j}{k,1}{}$.
\[
\hat{R}^i_j_k
\longrightarrow
\tcontextmn{i,j}{k}{ S^{i,j,k}_k \cseq T^{i,j,k}_k \colon 1 \leq k \leq \ell^{i,j}_k }
\longrightarrow
\tcontextmn{i,j}{k,0}{S^{i,j,k}_k \colon k \in I^i_j_k } \cseq
\tcontextmn{i,j}{k,1}{T^{i,j,k}_k \colon k \in J^i_j_k }
\]
In the case $R^i_j_k = \new x \hat{R}^i_j_k$, by Lemma~\ref{lemma:medial}, the following derivation holds, where $I^i_j_k \subseteq \left\{ 1 \hdots \ell^{i,j}_k \right\}$ and $J^i_j_k \subseteq \left\{ 1 \hdots \ell^{i,j}_k \right\}$ and killing contexts $\tcontextmn{i,j}{k,0}{}$ and $\tcontextmn{i,j}{k,1}{}$.
\[
\begin{array}{rl}
\new x \hat{R}^i_j_k
 \longrightarrow&
\new x \tcontextmn{i,j}{k}{ S^{i,j,k}_k \cseq T^{i,j,k}_k \colon 1 \leq k \leq \ell^{i,j}_k }
\\
\longrightarrow&
\new x
\left(
\tcontextmn{i,j}{k}{S^{i,j,k,0}_k \colon k \in I^i_j_k } \cseq
\tcontextmn{i,j}{k}{T^{i,j,k,1}_k \colon k \in J^i_j_k }
\right)
\\
\longrightarrow&
\new x
\tcontextmn{i,j}{k}{S^{i,j,k,0}_k \colon k \in I^i_j_k } \cseq
\new x
\tcontextmn{i,j}{k}{T^{i,j,k,1}_k \colon k \in J^i_j_k }
\end{array}
\]
Define the formulae
$
\hat{S}^i_j_k = \tcontextmn{i,j}{k,0}{S^{i,j,k}_k \colon k \in I^i_j_k }
$ and 
$
\hat{T}^i_j_k = \tcontextmn{i,j}{k,1}{T^{i,j,k}_k \colon k \in J^i_j_k }
$.
If $R^i_j_k = \hat{R}^i_j_k$, then define $\hat{S}^i_j_k = \hat{S}^i_j_k $
and $\hat{T}^i_j_k = \hat{T}^i_j_k$.
	If $R^i_j_k = \new x \hat{R}^i_j_k$, then define $\hat{S}^i_j_k = \new x \hat{S}^i_j_k$
and $\hat{T}^i_j_k = \new x \hat{T}^i_j_k$.
Hence we construct the following derivation, using Lemma~\ref{lemma:medial} again, for some $K^i_j \subseteq \left\{ 1\hdots g^i_j \right\}$ and $L^i_j \subseteq \left\{ 1\hdots g^i_j \right\}$ and killing contexts $\tcontextmn{i}{j,0}{}$ and $\tcontextmn{i}{j,1}{}$.
\[
\begin{array}{rl}
T \longrightarrow&
\tcontext{ U^0_i \cseq U^1_i \colon 1 \leq i \leq n }
\\
\longrightarrow&
\tcontext{ U^0_i \cseq \tcontextn{i}{ V^i_j \cseq W^i_j \colon 1 \leq j \leq m_i } \colon 1 \leq i \leq n }
\\
\longrightarrow&
\tcontext{ U^0_i \cseq \tcontextn{i}{ \tcontextmn{i}{j}{ R^i_j_k \colon 1 \leq k \leq g^i_j } 
                                                     \cseq W^i_j \colon 1 \leq j \leq m_i
                                    } \colon 1 \leq i \leq n }
\\
\longrightarrow&
\tcontext{
           U^0_i \cseq \tcontextn{i}{ \tcontextmn{i}{j}{ 
                                                                        \hat{S}^i_j_k \cseq 
                                                                        \hat{T}^i_j_k 
                                                                        \colon 1 \leq k \leq g^i_j } 
                                                     \cseq W^i_j \colon 1 \leq j \leq m_i
                                    }  \colon 1 \leq i \leq n }
\\
\longrightarrow&
\tcontext{ \begin{array}{l}
           U^0_i \cseq \tcontextn{i}{ \tcontextmn{i}{j,0}{ \hat{S}^i_j_k \colon k \in K^i_j } \colon 1 \leq j \leq m_i } \cseq \\
           \tcontextn{i}{ \tcontextmn{i}{j,1}{ \hat{T}^i_j_k \colon k \in L^i_j } \cseq W^i_j \colon 1 \leq j \leq m_i }
           \end{array}
           \colon 1 \leq i \leq n  }
\end{array}
\]
We aim now to establish $\vdash \left( P_0\cpar \wen x Q\right) \cpar U^0_i \cseq \tcontextn{i}{ \tcontextmn{i}{j,0}{ \hat{S}^i_j_k \colon k \in K^i_j } \colon 1 \leq j \leq m_i }$ and
$\vdash \left( \wen x P_2  \cpar P_3 \right) \cpar \tcontextn{i}{ \tcontextmn{i}{j,1}{ \hat{T}^i_j_k \colon k \in L^i_j } \cseq W^i_j \colon 1 \leq j \leq m_i }$.
Firstly, observe that $\hat{S}^i_j_k = \hat{S}^i_j_k$ then
$
\wen x P_1\cpar \hat{S}^i_j_k
\longrightarrow 
\new x P_1\cpar \hat{S}^i_j_k
\longrightarrow 
\new x \left( P_1\cpar \hat{S}^i_j_k \right)
$, since $\nfv{x}{T}$, by Lemma~\ref{lemma:nfv}, $\nfv{x}{U^{i,j}_k}$.
Also, if $\hat{S}^i_j_k = \new x \hat{S}^i_j_k$ then 
$
\wen x P_1\cpar \new x \hat{S}^i_j_k
\longrightarrow
\new x \left( P_1\cpar \hat{S}^i_j_k \right)
$.
Furthermore, the following proof can be constructed.
\[
P_1\cpar \tcontextmn{i,j}{k}{S^{i,j,k}_k \colon 1 \leq k \leq \ell^{i,j}_k }
\longrightarrow
\tcontextmn{i,j}{k}{ P_1\cpar S^{i,j,k}_k \colon 1 \leq k \leq \ell^{i,j}_k }
\longrightarrow
\tcontextmn{i,j}{k}{ \cunit \colon 1 \leq k \leq \ell^{i,j}_k }
\longrightarrow
\cunit
\]
Thereby, in either case $\vdash \wen x P_1\cpar \hat{S}^i_j_k$.
A similar argument establishes 
$
\vdash \wen x P_2  \cpar \hat{T}^i_j_k
$.

By applying the above proofs appropriately, the following two proofs can be constructed.
\[
\begin{array}{l}
\left( P_0\cseq \wen x P_1\right)  \cpar \left( U^0_i \cseq \tcontextn{i}{ \tcontextmn{i}{j,0}{ \hat{S}^i_j_k \colon k \in K^i_j } \colon 1 \leq j \leq m_i } \right)
\\\qquad\qquad\qquad\qquad\qquad
\begin{array}{rl}
\longrightarrow&
\left( P_0\cpar U^0_i  \right)  \cseq 
\left(  \wen x P_1\cpar \tcontextn{i}{ \tcontextmn{i}{j}{ \hat{S}^i_j_k \colon  k \in K^i_j } \colon 1 \leq j \leq m_i } \right)
\\
\longrightarrow&
\left(  \wen x P_1\cpar \tcontextn{i}{ \tcontextmn{i}{j}{ \hat{S}^i_j_k \colon  k \in K^i_j } \colon 1 \leq j \leq m_i } \right)
\\
\longrightarrow&
\left(  \tcontextn{i}{ \tcontextmn{i}{j}{ \wen x P_1\cpar \hat{S}^i_j_k \colon  k \in K^i_j } \colon 1 \leq j \leq m_i } \right)
\\
\longrightarrow&
\left(  \tcontextn{i}{ \tcontextmn{i}{j}{ \cunit \colon  k \in K^i_j } \colon 1 \leq j \leq m_i } \right)
\longrightarrow
\cunit
\end{array}
\end{array}
\]
\[
\begin{array}{l}
\left( \wen x P_2  \cseq P_3 \right) \cpar \tcontextn{i}{ \tcontextmn{i}{j}{ \hat{T}^i_j_k \colon k \in L^i_j } \cseq W^i_j \colon 1 \leq j \leq m_i }
\\\qquad\qquad\qquad\qquad\qquad
\begin{array}{rl}
\longrightarrow&
\tcontextn{i}{ \left( \wen x P_2  \cseq P_3 \right) \cpar \tcontextmn{i}{j}{ \hat{T}^i_j_k \colon k \in L^i_j } \cseq W^i_j \colon 1 \leq j \leq m_i }
\\
\longrightarrow&
\tcontextn{i}{ 
               \left( \wen x P_2  \cpar \tcontextmn{i}{j}{ \hat{T}^i_j_k \colon k \in L^i_j } \right)
                \cseq \left(P_3 \cpar W^i_j\right) \colon 1 \leq j \leq m_i }
\\
\longrightarrow&
\tcontextn{i}{ \left( \wen x P_2  \cpar \tcontextmn{i}{j}{ \hat{T}^i_j_k \colon k \in L^i_j } \right) \colon 1 \leq j \leq m_i }
\\
\longrightarrow&
\tcontextn{i}{ \tcontextmn{i}{j}{ \wen x P_2  \cpar \hat{T}^i_j_k \colon k \in L^i_j } \colon 1 \leq j \leq m_i }
\\
\longrightarrow&
\tcontextn{i}{ \tcontextmn{i}{j}{ \cunit \colon k \in L^i_j } \colon 1 \leq j \leq m_i }
\longrightarrow
\cunit
\end{array}
\end{array}
\]
Furthermore, by Lemma~\ref{lemma:bound}, the following inequalities holds.
\[
\begin{array}{rl}
\size{ \left( U^0_i \cseq \tcontextn{i}{ \tcontextmn{i}{j,0}{ \hat{S}^i_j_k \colon k \in K^i_j } \colon 1 \leq j \leq m_i } \right) } &\preceq \size{ T }
\\
\size{ \tcontextn{i}{ \tcontextmn{i}{j}{ \hat{T}^i_j_k \colon k \in L^i_j } \cseq W^i_j \colon 1 \leq j \leq m_i } } &\preceq \size{T}
\end{array}
\]
Hence, by Lemma~\ref{lemma:multisets}, the size of the formulae
$\left( \wen x P_2  \cseq P_3 \right) \cpar \tcontextn{i}{ \tcontextmn{i}{j}{ \hat{T}^i_j_k \colon k \in L^i_j } \cseq W^i_j \colon 1 \leq j \leq m_i }$ and
$\left( P_0\cseq \wen x P_1\right)  \cpar \left( U^0_i \cseq \tcontextn{i}{ \tcontextmn{i}{j,0}{ \hat{S}^i_j_k \colon k \in K^i_j } \colon 1 \leq j \leq m_i } \right)$
  are strictly bounded above by the size of 
$\left(P_0\cseq \wen x P_1\cseq \wen x P_2  \cseq P_3 \right) \cpar T$, as required.
\end{comment}


 
\end{enumerate}


\item \textbf{Principal case for times.}
There is only one principal case for \textit{times}, which does not differ significantly from the corresponding case in \textsf{BV} and its extensions. A proof may begin with an instance of the \textit{switch} rule of the form
\[
\infer[\mbox{where $\vdash \left(T_0 \tensor U_0 \tensor \left( \left(T_1 \tensor U_1\right) \cpar V \right)\right) \cpar W$,}]{
\left( T_0 \tensor T_1 \tensor U_0 \tensor U_1 \right) \cpar V \cpar W
}{
\left(T_0 \tensor U_0 \tensor \left( \left(T_1 \tensor U_1\right) \cpar V \right)\right) \cpar W
}
\]
such that $T_0 \tensor U_0 \not\equiv \cunit$ and $V \not\equiv \cunit$ (otherwise the \textit{switch} rule cannot be applied), and also $T_0 \tensor T_1 \not\equiv \cunit$ and $U_0 \tensor U_1 \not\equiv \cunit$ (otherwise splitting holds trivially).
By the induction hypothesis, there exist $R_i$ and $S_i$ such that $\vdash \left(T_0 \tensor U_0\right) \cpar R_i$ and $\vdash \left(T_1 \tensor U_1\right) \cpar V \cpar S_i$ hold, for $1 \leq i \leq n$, and an $n$-ary killing context $\tcontext{}$ such that 
derivation $\vcenter{\infer{W}{\tcontext{ R_1 \cpar S_1, \hdots, R_n \cpar S_n }}}$ holds.
Furthermore $\size{\left(T_0 \tensor U_0\right) \cpar R_i}$ and $\size{\left(T_1 \tensor U_1\right) \cpar V \cpar S_i}$ are bounded above by $\size{\left(T_0 \tensor U_0 \tensor \left( \left(T_1 \tensor U_1\right) \cpar V \right)\right) \cpar W}$.
Hence, by the induction hypothesis twice there exist formulae $P^{i,0}_j$, $Q^{i,0}_j$, $P^{i,1}_k$ and $Q^{i,1}_k$ such that $\vdash T_0 \cpar P^{i,0}_j$, $\vdash U_0 \cpar Q^{i,0}_j$, $\vdash T_1 \cpar P^{i,1}_k$ and $\vdash U_1 \cpar Q^{i,1}_k$, for $1 \leq j \leq m^0_i$ and $1 \leq k \leq m^1_i$, and $m^0_i$-ary killing context $\tcontextmn{0}{i}{}$ and $m^1_i$-ary killing context $\tcontextmn{1}{i}{}$ such that derivations
\[
\vcenter{
\infer[]{
R_i}{
\tcontextmn{0}{i}{P^{i,0}_j \cpar Q^{i,0}_j \colon 1 \leq j \leq m^0_i}
}
}
\quad
\mbox{ and }
\quad
\vcenter{
\infer[]{
V \cpar S_i
}{
\tcontextmn{1}{i}{P^{i,1}_k \cpar Q^{i,1}_k \colon 1 \leq k \leq m^1_i}
}}
\] 
can be constructed.
Thereby the following derivation can be constructed.
\[
\infer[]{
V \cpar W
}{
\infer[]{
V \cpar \tcontext{ R_i \cpar S_i \colon 1 \leq i \leq n } 
}{
\infer[]{
\tcontext{ R_i \cpar V \cpar S_i \colon 1 \leq i \leq n }
}{
\infer[]{
\tcontext{
\tcontextmn{0}{i}{P^{i,0}_j \cpar Q^{i,0}_j \colon 1 \leq j \leq m^0_i}
   \cpar
  \tcontextmn{1}{i}{P^{i,1}_k \cpar Q^{i,1}_k \colon 1 \leq k \leq m^1_i}
\colon 1 \leq i \leq n }
}{
\infer[]{
 \tcontext{ \tcontextmn{1}{i}{
\tcontextmn{0}{i}{P^{i,0}_j \cpar Q^{i,0}_j \colon 1 \leq j \leq m^0_i}
  \cpar
  P^{i,1}_k \cpar Q^{i,1}_k \colon 1 \leq k \leq m^1_i
} \colon 1 \leq i \leq n }
}{
 \tcontext{ \tcontextmn{1}{i}{
\tcontextmn{0}{i}{
P^{i,0}_j \cpar P^{i,1}_k \cpar Q^{i,0}_j \cpar Q^{i,1}_k
  \colon 1 \leq j \leq m^0_i
}
 \colon 1 \leq k \leq m^1_i
} \colon 1 \leq i \leq n }
}}}}}
\]
Now observe that the following two proofs can be constructed.
\[
\infer[]{
\left(T_0 \tensor T_1\right)
\cpar
P^{i,0}_j \cpar P^{i,1}_k
}{
\infer[]{
\left(T_0 \cpar P^{i,0}_j\right) \tensor \left(T_1 \cpar P^{i,1}_k\right)
}{
\cunit
}}
\qquad\qquad\qquad\qquad
\infer[]{
\left(U_0 \tensor U_1\right)
\cpar
Q^{i,0}_j \cpar Q^{i,1}_k
}{
\infer[]{
\left(U_0 \cpar Q^{i,0}_j\right) \tensor \left(U_1 \cpar Q^{i,1}_k\right)
}{
 \cunit
}}
\]
Furthermore,
$\occ{ T_0 \tensor T_1 } \mstrict \occ{ T_0 \tensor T_1 \tensor U_0 \tensor U_1 }$
and 
$\occ{ U_0 \tensor U_1 } \mstrict \occ{ T_0 \tensor T_1 \tensor U_0 \tensor U_1 }$,
since $T_0 \tensor T_1 \not\equiv \cunit$ and $U_0 \tensor U_1 \not\equiv \cunit$.
Also, by Lemma~\ref{lemma:bound}, the following inequality holds.
\[
\size{
 \tcontext{ \tcontextmn{1}{i}{
\tcontextmn{0}{i}{
P^{i,0}_j \cpar P^{i,1}_k \cpar Q^{i,0}_j \cpar Q^{i,1}_k
  \colon 1 \leq j \leq m^0_i
}
 \colon 1 \leq k \leq m^1_i
} \colon 1 \leq i \leq n }
} \preceq \size{ V \cpar W }
\]
Hence both
$
\size{ P^{i,0}_j \cpar P^{i,1}_k }
\preceq \size{ V \cpar W }$
and $
\size{ Q^{i,0}_j \cpar Q^{i,1}_k }
\preceq \size{ V \cpar W }$ hold.
Thereby the size of each of the above proofs is strictly bounded above by the size of the proof of $\left(T_0 \tensor T_1 \tensor U_0 \tensor U_1 \right) \cpar V \cpar W$.




\item \textbf{Principal cases for with.}
There are three forms of principal case where the \textit{with} operator is directly involved in the bottommost rules.
Note that in \textsf{MAV} the \textit{with} operator is separated from the core splitting lemma, much like universal quantification in this paper.
However, in the case of \textsf{MAV1} the \textit{left name} and \textit{right name} rules introduce inter-dependencies between nominals and \textit{with}, forcing cases for \textit{with} to be checked in this lemma.

\begin{enumerate}[label*=\textbf{.\arabic*}]

\item Consider the principal case involving the \textit{extrude} rule.
In this case, the bottommost rule is of the form 
\[
\infer[\mbox{where $\vdash \left(P  \cpar R\right)\wwith\left( Q \cpar R\right) \cpar S$ holds.}]{
\left(P \wwith Q\right) \cpar R \cpar S
}{
\left(P  \cpar R\right)\wwith\left( Q \cpar R\right) \cpar S
}
\]
Now, by the induction hypothesis, 
since $\vdash \left(P  \cpar R\right)\wwith\left( Q \cpar R\right) \cpar S$ holds, we have that 
$\vdash P  \cpar R \cpar S$ and $\vdash Q \cpar R \cpar S$ hold, as required.



\item Consider the principal case involving the \textit{left name} rule.
In this case, the bottommost rule is of the form 
\[
\infer[\mbox{, where $\nfv{x}{Q}$, such that $\vdash \mathopen{\wen x.}\left( P \wwith Q \right) \cpar R$.}]{
\left( \wen x.P \wwith Q \right) \cpar R
}{
\mathopen{\wen x.}\left( P \wwith Q \right) \cpar R
}
\] 
By the induction hypothesis, there exist $S$ and $\hat{S}$ such that
$\vcenter{\infer[]{R}{S}}$ and $\nfv{x}{S}$ and $\vdash \left(P \wwith Q\right) \cpar \hat{S}$ 
and either $S = \hat{S}$ or $S = \mathopen{\new x.}\hat{S}$. Furthermore, the size of the proof of $\left(P \wwith Q\right) \cpar \hat{S}$ is strictly less than the size of the proof of $\left( \wen x.P \wwith Q \right) \cpar R$, since the \textit{wen} count strictly decreases, and by Lemma~\ref{lemma:bound}, $\size{\hat{S}} \leq \size{R}$.
By the induction hypothesis again, $\vdash P \cpar \hat{S}$ and $\vdash Q \cpar \hat{S}$ hold.


Now if $S = \hat{S}$ then $\nfv{x}{\hat{S}}$ and $\vdash Q \cpar S$ holds immediately, whereas 
$\vdash \wen x.P \cpar R$ is proved as below left. Otherwise, $S = \new x.\hat{S}$ and
$\vdash \wen x.P \cpar R$ is proved in the middle derivation below, whereas $\vdash  Q \cpar S$ is proved in the right derivation below.
\[
\infer[]{
\wen x.P \cpar R 
}{
\infer[]{
\wen x.P \cpar \hat S
}{
\infer[]
{\wen x.\left(P \cpar \hat S \right)}
{
\infer[]{
\mathopen{\new x.}\left(P \cpar \hat{S} \right)
}{
\infer[]{
\mathopen{\new x.}\cunit
}{
\cunit
}}}}}
\qquad \qquad
\infer[]{
\wen x.P \cpar R 
}{
\infer[]{
\wen x.P \cpar \new x. \hat S
}{
\infer[]{
\mathopen{\new x.}\left(P \cpar \hat{S} \right)
}{
\infer[]{
\mathopen{\new x.}\cunit
}{
\cunit
}}}}
\qquad
\qquad
\infer[.]{
Q \cpar \wen x.\hat{S}
}{
\infer[]{
\mathopen{\wen x.}\left( Q \cpar \hat{S} \right)
}{
\infer[]{
 \mathopen{\new x.}\left( Q \cpar \hat{S} \right)
}{
\infer[]{
 \new x.\cunit 
}{
 \cunit
}}}}
\]



Hence, in either case, $\vdash Q \cpar S$ 
and since $
\vcenter{
\infer[]{
Q \cpar R
}{
Q \cpar S
}}$,
we have that $\vdash Q \cpar R$ holds.
Thereby $\vdash \wen x.P \cpar R$ and $\vdash Q \cpar R$ hold, as required.
The case for the \textit{left name} rule, where $\new$ replaces $\wen$ is similar; as are the cases for the \textit{right name} and \textit{with name} rules.


\begin{comment} Case for \textit{with name}, omitted for readability.
Consider the principal case involving the \textit{with name} rule.
In this case, the bottommost rule is of the form $\left( \wen x.P \wwith \wen x.Q \right) \cpar R \longrightarrow \mathopen{\wen x.}\left( P \wwith Q \right) \cpar R$ such that $\vdash \mathopen{\wen x.}\left( P \wwith Q \right) \cpar R$.
By the induction hypothesis, there exist $S$ and $\hat{S}$ such that $R \longrightarrow S$ and $\nfv{x}{S}$ and $\vdash \left(P \wwith Q\right) \cpar \hat{S}$ 
and either $S = \hat{S}$ or $S = \mathopen{\new x.}\hat{S}$. Furthermore, the size of the proof of $\left(P \wwith Q\right) \cpar \hat{S}$ is strictly less than the size of the proof of $\left(P \wwith Q\right) \cpar R \cpar S$.
By the induction hypothesis again, $\vdash P \cpar \hat{S}$ and $\vdash Q \cpar \hat{S}$ hold.
Now if $S = \hat{S}$ then $\nfv{x}{\hat{S}}$ and so $\wen x. P \cpar \hat{S} \longrightarrow \mathopen{\wen x.}\left( P \cpar \hat{S} \right) \longrightarrow \mathopen{\new x.}\left( P \cpar \hat{S} \right)$.
Otherwise $S = \new x.\hat{S}$ so $\wen x. P \cpar \new x.\hat{S} \longrightarrow \mathopen{\new x.} \left( P \cpar \hat{S} \right)$.
Hence $\wen x. P \cpar S \longrightarrow \mathopen{\new x.} \left( P \cpar \hat{S} \right)$.
By a similar argument, $\wen x. Q \cpar S \longrightarrow \mathopen{\new x.} \left( Q \cpar \hat{S} \right)$.
Thereby we can construct the following two proofs as follows.
\[
\begin{array}{rl}
\wen x.P \cpar R 
\longrightarrow&
\wen x.P \cpar S
\\
\longrightarrow&
\mathopen{\new x.}\left(P \cpar \hat{S} \right)
\longrightarrow
\mathopen{\new x.}\cunit
\longrightarrow 
\cunit
\end{array}
\begin{array}{rl}
\wen x.Q \cpar R 
\longrightarrow&
\wen x.Q \cpar S
\\
\longrightarrow&
\mathopen{\new x.}\left(Q \cpar \hat{S} \right)
\longrightarrow
\mathopen{\new x.}\cunit
\longrightarrow
\cunit
\end{array}
\]
Furthermore $\occ{\wen x.P \cpar R} \prec \occ{\left(P \wwith Q\right) \cpar R \cpar S}$
and $\occ{Q \cpar R} \prec \occ{\left(P \wwith Q\right) \cpar R \cpar S}$, hence the size of proofs strictly decrease as requried.
\smallskip
\end{comment}


\item Consider the principal case involving the \textit{medial} rule.
In this case, the bottommost rule of a proof is of the form 
\[
\infer[\mbox{such that $\vdash \left(\left( P \wwith R \right) \cseq \left( Q \wwith S \right)\right) \cpar W$ holds.}]{
\left(\left( P \cseq Q \right) \wwith \left( R \cseq S \right)\right) \cpar W
}{
\left(\left( P \wwith R \right) \cseq \left( Q \wwith S \right)\right) \cpar W
}
\]
By the induction hypothesis, for $1 \leq i \leq n$ there exists $U_i$ and $V_i$ such that $\vdash \left( P \wwith R \right) \cpar U_i$ and $\vdash \left( Q \wwith S \right) \cpar V_i$ hold, and $n$-ary killing context $\tcontext{}$ such that 
$
\vcenter{\infer[]{W}{\tcontext{U_i \cseq V_i \colon 1 \leq i \leq n}}}$.
Furthermore, the size of the proofs of $\left( P \wwith R \right) \cpar U_i$ and $\left( Q \wwith S \right) \cpar V_i$ are strictly less than the size of the proof of $\left(\left( P \wwith R \right) \cseq \left( Q \wwith S \right)\right) \cpar W$. Hence by the induction hypothesis again, $\vdash P \cpar U_i$, $\vdash R \cpar U_i$, $\vdash Q \cpar V_i$ and $\vdash S \cpar V_i$.
Hence we can construct the following two proofs, as required.
\[
\infer[]{
\left( P \cseq Q \right) \cpar W
}{
\infer[]{
\left( P \cseq Q \right) \cpar \tcontext{U_i \cseq V_i \colon 1 \leq i \leq n}
}{
\infer[]{
\tcontext{\left( P \cseq Q \right) \cpar \left(U_i \cseq V_i\right) \colon 1 \leq i \leq n}
}{
\infer[]{
\tcontext{\left( P \cpar U_i \right) \cseq \left(Q \cpar V_i\right) \colon 1 \leq i \leq n}
}{
\infer[]{
\tcontext{\cunit  \colon 1 \leq i \leq n}
}{
\cunit
}}}}}
\qquad
\infer[]{
\left( R \cseq S \right) \cpar W
}{
\infer[]{
\left( R \cseq S \right) \cpar \tcontext{U_i \cseq V_i \colon 1 \leq i \leq n}
}{
\infer[]{
\tcontext{\left( R \cseq S \right) \cpar \left(U_i \cseq V_i\right) \colon 1 \leq i \leq n}
}{
\infer[]{
\tcontext{\left( R \cpar U_i \right) \cseq \left(S \cpar V_i\right) \colon 1 \leq i \leq n}
}{
\infer[]{
\tcontext{\cunit \colon 1 \leq i \leq n}
}{
\cunit
}}}}}
\]


\end{enumerate}


\item \textbf{Commutative cases induced by equivariance.}
There are certain commutative cases induced by the \textit{equivariance} rule for nominal quantifiers.
These are the cases that force the rules \textit{all name}, \textit{with name}, \textit{left name} and \textit{right name} to be included.
Notice also that \textit{equivariance} for \textit{new} is required when handling the case induced by \textit{equivariance} for \textit{wen}; hence \textit{equivariance} for both nominal quantifiers must be explicit structural rules rather than properties derived from each other.


\begin{enumerate}[label*=\textbf{.\arabic*}]
\item Consider the commutative case for \textit{wen} where the bottommost rule of a proof is an instance of the \textit{close} rule of following form
\[
\infer[\mbox{,where $\vdash \mathopen{\new y.}\left( \wen x. P \cpar Q \right) \cpar R$, $\nfv{y}{R}$ and $\nfv{x}{R}$.} ]{
\wen x. \wen y. P \cpar \new y. Q \cpar R
}{
\mathopen{\new y.}\left( \wen x. P \cpar Q \right) \cpar R
}
\]
Notice that $\wen x$ is the principal connective but the \textit{close} rule is applied to $\wen y$ behind the principal connective.
Thus we desire some formula $R'$ such that $
\vcenter{
\infer[]{
\new y. Q \cpar R
}{R'}}$
and $\nfv{x}{R'}$ and either $\vdash \wen y. P \cpar R'$ or there exists $Q'$ such that $R' = \new x.Q'$ and $\vdash \wen y. P \cpar Q'$, and the size of $ \wen y. P \cpar R'$ is strictly smaller than $\wen x. \wen y. P \cpar \new y. Q \cpar R$.
By the induction hypothesis, there exist $S$ and $T$ such that $\nfv{y}{S}$ and $\vdash \wen x. P \cpar Q \cpar T$ and either $S = T$ or $S = \wen y. T$ and the derivation
$
\vcenter{
\infer{R}{S}}
$ holds.
Furthermore the size of the proof of $\wen x. P \cpar Q \cpar T$ is bounded above by the size of the proof of $\mathopen{\new y.}\left( \wen x.P \cpar Q \right) \cpar R$; hence strictly bounded by the size of the proof of $\wen x. \wen y. P \cpar \new y.Q \cpar R$.
Hence, by induction, there exist $U$ and $V$ such that $\vdash P \cpar V$ and $\nfv{x}{U}$ and either $U = V$ or $U = \new x. V$ the derivation 
$
\vcenter{
\infer[]{
Q \cpar T
}{
U
}}
$ holds.
Observe that if $S = T$, then 
$
\vcenter{
\infer[]{
\new y. Q \cpar S
}{
\mathopen{\new y.} \left( Q \cpar T \right)
}}$, since $\nfv{y}{S}$.
If $S = \wen y. T$ then 
$
\vcenter{
\infer[]{
\new y. Q \cpar \wen y. T
}{
\mathopen{\new y.} \left( Q \cpar T \right)
}}$.
Thereby the following derivation can be constructed, where if $U = V$ then $W = \new y. V$ and if $U = \new x. V$ then $W = \new x. \new y. V$, and also the premise is equivalent to $W$ by \textit{equivariance} for \textit{new}:
$
\vcenter{
\infer[]{
\new y. Q \cpar R
}{
\infer[]{
\new y. Q \cpar S
}{
\infer[]{
\mathopen{\new y.} \left( Q \cpar T \right)
}{
\new y. U
}}}}
$.
Furthermore, the following proof can be constructed 
$
\vcenter{
\infer[]{
\wen y. P \cpar \new y. V
}{
\infer[]{
\mathopen{\new y.} \left(P \cpar V\right)
}{
\infer[]{
 \new y. \cunit
}{
\cunit
}}}}$
and, by Lemma~\ref{lemma:bound}, $\size{\new y. V} \preceq \size{\new y. Q \cpar R}$ hence 
$\size{ \wen y. P \cpar \new y. V } \prec \size{ \wen x. \wen y. P \cpar \new y. Q \cpar R }$, as required.


\item Consider a commutative case for \textit{new} induced by \textit{equivariance} for \textit{new}, where the bottommost rule is an instance of \textit{extrude new} of the form
\[
\infer[, \mbox{where $\nfv{y}{Q}$ and $\vdash \mathopen{\new y.} \left( \new x. P \cpar Q\right) \cpar R$.}]{
\new x. \new y. P \cpar Q \cpar R
}{
\mathopen{\new y.} \left( \new x. P \cpar Q\right) \cpar R
}
\]
By the induction hypothesis, there exist $S$ and $T$ such that $\nfv{y}{S}$ and $\vdash \new x. P \cpar Q \cpar T$ and either $S = T$ or $S = \wen y. T$, where
$\vcenter{\infer[]{R}{S}}$.
Furthermore, the size of the proof of $\new x. P \cpar Q \cpar T$ is bound above by the size of the proof of $\mathopen{\new y.} \left( \new x. P \cpar Q\right) \cpar R$, hence strictly bound above by the size of the proof of $\new x. \new y. P \cpar Q \cpar R$.
Hence, by induction again, there exist $U$ and $V$ such that $\nfv{x}{U}$ and $\vdash P \cpar V$ and either $U = V$ or $U = \wen x. V$, and also $\infer[]{Q \cpar T}{U}$.
Now define $\hat{W}$ and $W$ as follows.
If $S = T$ then let $\hat{W} = V$.
If $S = \wen y. T$ then let $\hat{W} = \wen y. V$.
If $U = V$ then let $W = \hat{W}$.
If $U = \wen x. V$ then let $W = \wen x. \hat{W}$.
Now observe if $S = T$ then
$
\vcenter{
\infer[]{
Q \cpar R
}{
\infer[]{
Q \cpar T
}{
U
}}}$ and $U = W$.
For $S = \wen y. T$ observe
$
\vcenter{
\infer[]{
Q \cpar R
}{
\infer[]{
Q \cpar \wen y. T
}{
\infer[]{
\mathopen{\wen y.} \left( Q \cpar T \right)
}{
\mathopen{\wen y.} U
}}}}
$, since $\nfv{y}{Q}$,
and if $U = V$ then $\mathopen{\wen y.} U = \hat{W}$,
while if $U = \wen x. V$ then $\wen y.U \equiv \wen x.\hat{W}$, by \textit{equivariance} for \textit{wen}.
Hence in all cases $\vcenter{\infer[]{Q \cpar R}{W}}$
and, since $\nfv{y}{Q}$ and $\nfv{y}{T}$, we can arrange that $\nfv{y}{W}$.
Now, for the cases where $\hat{W} = V$, we have $\nfv{y}{V}$,
and hence $\vcenter{\infer[]{\new y. P \cpar V}{\mathopen{\new y.}\left( P \cpar V \right)}}$.
Also if $\hat{W} = \wen y.V$, then $
\vcenter{
\infer[]{
\new y. P \cpar \wen y.V
}{
\mathopen{\new y.}\left( P \cpar V \right)
}}$. Hence in either case we can construct the proof $
\vcenter{
\infer[]{
\new y. P \cpar \hat{W}
}{
\infer[]{
\mathopen{\new y.}\left( P \cpar V \right)
}{
\infer[]{
 \new y.\cunit
}{ \cunit
}}}}$.
Furthermore, $\size{ \new y. P \cpar \hat{W} } \prec \size{ \new x. \new y. P \cpar Q \cpar R }$,
since by Lemma~\ref{lemma:bound} $\size{ \hat{W} } \preceq \size{ Q \cpar R }$.



\item Similar commutative cases for \textit{wen} and \textit{new} as principal formulae are induced by \textit{equivariance} where the bottommost rule in a proof is an instance of the \textit{close}, \textit{right wen} or \textit{suspend} rules.
In each case, the quantifier involved in the bottommost rule appears behind the principal connective and is propagated in front of the principal connective using \textit{equivariance}.
\end{enumerate}

\fullproof{
A similar commutative case for \textit{wen} is induced where the bottommost rule in a proof is an instance of the \textit{left wen} rule of the form 
$
\infer[]{
\wen x. \wen y. P \cpar Q \cpar R
}{
\mathopen{\wen y.} \left( \wen x. P \cpar Q \right) \cpar R
}
$
where $\nfv{y}{Q}$ and $\vdash \mathopen{\wen y.} \left( \wen x. P \cpar Q \right) \cpar R$.
By the induction hypothesis, there exist $S$ and $T$ such that $\nfv{y}{S}$ and $\vdash \wen x. P \cpar Q \cpar T$ and either $S = T$ or $S = \new y. T$ such that $
\infer[]{R}{S}$.
Furthermore, the size of the proof of $\wen x. P \cpar Q \cpar T$ is bounded by the size of the proof of $\mathopen{\wen y.} \left( \wen x. P \cpar Q \right) \cpar R$ hence strictly bounded by the size of the proof of $\wen x. \wen y. P \cpar Q \cpar R$.
Hence, by the induction hypothesis again, there exist $U$ and $V$ such that $\nfv{x}{U}$ and $\vdash P \cpar V$ and either $U = V$ or $U = \new x. V$, and derivation $
\infer[]{Q \cpar T}{U}$ holds.
If $S = \new y. T$ let $\hat{W} = \new y. V$ otherwise $\hat{W} = V$. If $U = \new x. V$ let $W = \new x. \hat{W}$, otherwise $W = \hat{W}$.
Thereby the following derivations hold:
\begin{itemize}
\item If $S = \new y. T$ and $U = \new x. V$ then
$
\infer[]{
Q \cpar S
}{
\infer[]{
\mathopen{\new y.} \left( Q \cpar T \right)
}{
 \new y. U \equiv W
}}$.

\item If $S = \new y. T$ and $U = V$ then $
\infer[]{Q \cpar S
}{
\infer[]{
\mathopen{\new y.} \left( Q \cpar T \right) 
}{
 \new y. U
}}$, where the premise is equivalent to $W$.

\item If $S = T$ then $Q \cpar S = Q \cpar T$
and $
\infer[]{
Q \cpar T
}{
 U \equiv W
}$.
\end{itemize}
Hence in any of the above cases, $
\infer[]{
Q \cpar R
}{
\infer[]{
 Q \cpar S
}{ W
}}$.
Now, if $\hat{W} = \new y. V$, then $
\infer[]{
\wen y. P \cpar \hat{W}
}{
 \new y. \left(P \cpar V\right)
}$;
and if $\hat{W} = V$ then $S = T$, hence $
\infer[]{
\wen y. P \cpar \hat{W}
}{
\infer[]{
 \new y. P \cpar \hat{W}
}{
 \mathopen{\new y.} \left( P \cpar V\right)
}}$, since $\nfv{y}{V}$.
Clearly $
\infer[]{
\mathopen{\new y.} \left( P \cpar V\right)
}{
\infer[]{
\new y. \cunit
}{
 \cunit
}}
$, hence $\vdash \wen y.P \cpar \hat{W}$ holds.  Furthermore, $\size{\wen y. P \cpar \hat{W}} \prec \size{\wen x. \wen y. P \cpar Q \cpar R}$, since by Lemma~\ref{lemma:bound} $\size{\hat{W}} \preceq \size{ Q \cpar R }$.


The third commutative case for \textit{wen} induced by \textit{equivariance} is where the bottommost rule is an instance of the \textit{suspend} rule of the form 
$
\infer[]{
\wen x. \wen y. P \cpar \wen y. Q \cpar R
}{
\mathopen{\wen y.} \left( \wen x. P \cpar Q \right) \cpar R
}
$, where $\vdash \mathopen{\wen y.} \left( \wen x. P \cpar Q \right) \cpar R$.
By the induction hypothesis, there exist $S$ and $T$ where $\nfv{y}{S}$ and $\vdash \wen x. P \cpar Q \cpar T$ and either $S = T$ or $S = \new y. T$ such that $\infer{R}{S}$.
Furthermore, the size of the proof of $\wen x. P \cpar Q \cpar T$ is bounded above by the size of the proof of $\mathopen{\wen y.} \left( \wen x. P \cpar Q \right) \cpar R$, hence strictly bounded above by the size of the proof of $\wen x. \wen y. P \cpar \wen y. Q \cpar R$.
Hence by the induction hypothesis again, there exist $U$ and $V$ where $\nfv{x}{U}$ and $\vdash P \cpar V$ and either $U = V$ or $U = \new x. V$, such that $\infer{Q \cpar T}{U}$.
If $U = V$ then let $W = \new y. V$, and if $U = \new x. V$ then  let $W = \new x. \new y. V$.
Also observe that whether we have $S = T$ or $S = \new y. T$, we have $
\infer[]{
\wen y. Q \cpar S
}{
\mathopen{\wen y.} \left( Q \cpar T \right)
}$.
Thereby the following derivation can be constructed:
$\infer[]{
\wen y. Q \cpar R
}{
\infer[]{
\mathopen{\wen y.} \left( Q \cpar T \right)
}{
\wen y. U \equiv W
}}$.
Furthermore, the following proof can be constructed $
\infer[]{
\wen y. P \cpar \new y. V
}{
\infer[]{
\mathopen{\new y.} \left( P \cpar V \right)
}{
\infer[]{
\new y. \cunit
}{
\cunit
}}}$;
and the size of the proof of $\size{ \wen y. P \cpar \new y. V } \prec \size{ \wen x. \wen y. P \cpar \wen y. Q \cpar R}$, since $\size{ \new y. V } \preceq \size{ \wen y. Q \cpar R }$ by Lemma~\ref{lemma:bound}.


Consider the first commutative case for \textit{new} induced by equivariance, where the bottommost rule is of the form
$
\infer[]{
\new x. \new y. P \cpar \wen y. Q \cpar R
}{
\mathopen{\new y.} \left( \new x. P \cpar Q\right) \cpar R
}
$,
where $\vdash \mathopen{\new y.} \left( \new x. P \cpar Q\right) \cpar R$.
By the induction hypothesis, there exist $S$ and $T$ such that $\nfv{y}{S}$ and $\vdash \wen x. P \cpar Q \cpar T$ and either $S = T$ or $S = \wen y .T$, where
$\infer[]{R}{S}$.
Furthermore, the size of the proof of $\wen x. P \cpar Q \cpar T$ is bound above by the size of the proof of $\mathopen{\new y.} \left( \new x. P \cpar Q\right) \cpar R$, hence strictly bound above by the size of the proof of $\new x. \new y. P \cpar \wen y. Q \cpar R$.
Hence, by induction again, there exist $U$ and $V$ such that $\nfv{x}{U}$ and $\vdash P \cpar V$ and either $U = V$ or $U = \wen x. V$, and also $\infer[]{Q \cpar T}{U}$.
If $U = V$ then let $W = \wen y. V$, and if $U = \wen x. V$ then let $W = \wen x. \wen y. V$. 
Also, regardless of whether $S = T$ or $S = \wen y. T$, $\infer[]{\wen y. Q \cpar S}{\mathopen{\wen y.} \left( Q \cpar S \right)}$. Hence derivation $
\infer[]{
\wen y. Q \cpar R
}{
\infer[]{
\wen y. Q \cpar S
}{
\infer[]{
\mathopen{\wen y.} \left( Q \cpar T \right)
}{
\wen y. U
}}}$ can be constructed, where the premise is equivalent to $W$.
Furthermore, $
\infer[]{
\new y. P \cpar \wen y. V
}{
\infer[]{
\mathopen{\new y.} \left( P \cpar V \right)
}{
\infer[]{
\new y. \cunit 
}{ \cunit
}}}$,
and $\size{ \new y. P \cpar \wen y. V } \prec \size{ \new x. \new y. P \cpar \wen y. Q \cpar R }$,
since by Lemma~\ref{lemma:bound} $\size{ \wen y. V } \preceq \size{ \wen y. Q \cpar R }$.

}




 


\item \textbf{Regular commutative cases.} As in every splitting lemma, there are numerous \textit{commutative} cases where the bottommost rule in a proof does not directly involve the principal connective.
For each principal formula handled by this splitting lemma (\textit{new}, \textit{wen}, \textit{with}, \textit{seq} and \textit{times}) there are commutative cases induced by \textit{new}, \textit{wen}, \textit{all}, \textit{with} and \textit{times} and also two commutative cases induced by \textit{seq}.
Thus there are 35 similar commutative cases to check, that all follow a pattern, hence only a representative selection of four cases are presented that make special use of $\alpha$-conversion and the rules \textit{new wen},
\textit{all name}, \textit{with name}, \textit{left name} and \textit{right name}.
Further, representative cases appear in the proof for existential quantifiers.


\begin{enumerate}[label*=\textbf{.\arabic*}]

\item Consider the commutative case where the principal formula is $\new x. P$ and the bottommost rule is an instance of \textit{extrude new} but applied to a distinct \textit{new} quantifier $\new y. Q$, as in the following rule instance 
\[
\infer[\mbox{, where $\nfv{y}{\new x. P \cpar R}$.}]{
\new x. P \cpar \new y. Q \cpar R \cpar S
}{
\mathopen{\new y.} \left( \new x. P \cpar Q \cpar R\right) \cpar S
}
\] 
Also assume, by $\alpha$-conversion, that $x \not= y$.
By induction, there exist $T$ and $U$ such that $\vdash \new x. P \cpar Q \cpar R \cpar U$, $\nfv{y}{T}$ and either $T = U$ or $T = \wen y. U$, and also
$\vcenter{\infer{S}{T}}$.
Furthermore, the size of the proof of $\new x. P \cpar Q \cpar R \cpar U$ is bounded above by the size of the proof of $\mathopen{\new y.} \left( \new x. P \cpar Q \cpar R\right) \cpar S$ and hence strictly bounded above by the size of the proof of $\new x. P \cpar \new y. Q \cpar R \cpar S$, enabling the induction hypothesis.
Hence, by the induction hypothesis, there exist formulae $V$ and $\hat{V}$ such that $\vdash P \cpar \hat{V}$ and $\nfv{x}{V}$ and either $V = \hat{V}$ or $V = \wen x. \hat{V}$, and also $
\vcenter{
\infer[]{
Q \cpar R \cpar U
}{V}}$.
Define $W$ such that if $V = \hat{V}$ then $W = \new y. \hat{V}$ and if $V = \wen x. \hat{V}$ then $W = \wen x. \new y. \hat{V}$.
Hence if $V = \wen x. \hat{V}$ then $
\vcenter{
\infer[]{
\new y. V
}{
\wen x. \new y. \hat{V}
}}$ by applying the \textit{new wen} rule, where the premise equals $W$.
If $V = \hat{V}$ then $\new y. V = W$. In both cases, $\nfv{x}{W}$.
Now observe that either $T = U$ and $\nfv{y}{U}$, hence the derivation $(a)$ below holds;
or $T = \wen y. U$, hence the derivation $(b)$ below holds. Given these, the derivation $(c)$ can be constructed:
\[
\begin{array}{cccc}
\infer[]{
\new y. Q \cpar R \cpar T
}{
\mathopen{\new y.} \left(Q \cpar R \cpar U\right)
}
&
\infer[]{
\new y. Q \cpar R \cpar \wen y. U
}{
\infer[]{
 \mathopen{\new y.} \left(Q \cpar R\right) \cpar \wen y. U
}{
 \mathopen{\new y.} \left(Q \cpar R \cpar U\right)
}}
& 
\infer[]{
\new y. Q \cpar R \cpar S
}{
\infer[]{
\new y. Q \cpar R \cpar T
}{
\infer[]{
\mathopen{\new y.} \left( Q \cpar R \cpar U \right)
}{
\infer[]{
\new y. V
}{
W
}}}} &
\infer[]{
P \cpar \new y. \hat{V}
}{
\infer[]{
 \mathopen{\new y.} \left( P \cpar \hat{V} \right) 
}{
\infer[]{
 \new y. \cunit
}{
 \cunit
}}}\\
(a) & (b) & (c) & (d)
\end{array}
\]
Since $\nfv{y}{\new x. P \cpar R}$ and $x \not= y$, we have $\nfv{y}{P}$;
thereby the proof $(d)$ above can be constructed.
Furthermore, $\size{ P \cpar \new y. \hat{V} } \prec \size{ \wen x. P \cpar \new y. Q \cpar R \cpar S }$ since by Lemma~\ref{lemma:bound} $\size{ \new y. \hat{V} } \preceq \size{\new y. Q \cpar R \cpar S}$
and the \textit{wen count} strictly decreases.



\item Consider the commutative case for principal formula $\wen x. T$ where the bottommost rule is \textit{external}: 
\[
\infer[]{
\wen x. T \cpar \left( U \wwith V \right) \cpar W \cpar P
}{
\left(\left(\wen x. T \cpar U \cpar W\right) \wwith
\left(\wen x. T \cpar V \cpar W\right)\right)
\cpar P
}
\]
where $\vdash \left(\left(\wen x. T \cpar U \cpar W\right) \wwith \left(\wen x. T \cpar V \cpar W\right)\right) \cpar P$ holds.
By the induction hypothesis, we have that both $\vdash \wen x. T \cpar U \cpar W \cpar P$ and $\vdash \wen x. T \cpar V \cpar W \cpar P$ hold;
and furthermore the multiset inequalities 
\[
\begin{array}{rcl}
\occ{\wen x. T \cpar U \cpar W \cpar P} & \mstrict & \occ{ \wen x. T \cpar \left( U \wwith V \right) \cpar W \cpar P } \mbox{ and }\\
\occ{\wen x. T \cpar V \cpar W \cpar P} & \mstrict & \occ{ \wen x. T \cpar \left( U \wwith V \right) \cpar W \cpar P }
\end{array}
\]
hold.
Hence, by the induction hypothesis, there exist $Q$ and $\hat{Q}$ such that $\vdash T \cpar \hat{Q}$, $\nfv{x}{Q}$ and either $Q = \hat{Q}$ or $Q = \new x. \hat{Q}$.
Also, by the induction hypothesis, there exist $R$ and $\hat{R}$ such that
$\vdash T \cpar \hat{R}$, $\nfv{x}{R}$ and either $R = \hat{R}$ or $R = \new x. \hat{R}$.
Furthermore the two derivations 
$
\vcenter{
\infer[]{
U \cpar W \cpar P
}{ Q }}
$
and
$
\vcenter{
\infer[]{
V \cpar W \cpar P
}{ R }
}
$ hold.
Now define $S$ such that if $Q = \hat{Q}$ and $R = \hat{R}$ then $S = \hat{Q} \wwith \hat{R}$, and 
$S = \mathopen{\wen x.} \left( \hat{Q} \wwith \hat{R} \right)$ otherwise, observing that in either case $\nfv{x}{S}$.
In the case $Q = \wen x. \hat{Q}$ and $R = \wen x. \hat{R}$, by the \textit{with name} rule,
$
\vcenter{
\infer[]{
\wen x. \hat{Q} \wwith \wen x. \hat{R}
}{
\mathopen{\wen x.} \left( \hat{Q} \wwith \hat{R} \right)
}}$.
In the case $Q = \wen x. \hat{Q}$ and $R = \hat{R}$, by the \textit{left name} rule,
$
\vcenter{
\infer[]{
\wen x. \hat{Q} \wwith \hat{R}
}{
\mathopen{\wen x.} \left( \hat{Q} \wwith \hat{R} \right)
}}$.
In the case that $Q = \hat{Q}$ and $R = \wen x. \hat{R}$, by the \textit{right name} rule,
$
\vcenter{
\infer[]{
\hat{Q} \wwith \wen x. \hat{R}
}{
\mathopen{\wen x.} \left( \hat{Q} \wwith \hat{R} \right)
}}$.
Thereby the following derivation and proof can be constructed:
\[
\infer[]{
\left( U \wwith V \right) \cpar W \cpar P
}{
\infer[]{
\left(U \cpar W \cpar P\right) \wwith \left(V \cpar W \cpar P\right) 
}{
\infer[]{
Q \wwith R
}{
S
}}}
\qquad
\infer[.]{
T \cpar \left(\hat{Q} \wwith \hat{R}\right)
}{
\infer[]{
\left(T \cpar \hat{Q}\right) \wwith \left( T \cpar \hat{R}\right)
}{
\infer[]{
\cunit \wwith \cunit
}{
\cunit
}}}
\]
Furthermore, by Lemma~\ref{lemma:bound}, $\size{ S } \preceq \size{ \left( U \wwith V \right) \cpar W \cpar P }$; and,
since the \textit{wen} count strictly decreases,
$\size{ T \cpar \hat{Q} \wwith \hat{R} } \prec \size{\wen x. T \cpar \left( U \wwith V \right) \cpar W \cpar P}$.




\item Consider the commutative case where the principal formula is $\wen x. T$ and the bottommost rule is an instance of the
\textit{extrude1} rule of the form
\[
\infer[]{
\wen x. T \cpar \forall y. U \cpar V \cpar W
}{
\mathopen{\forall y.}\left( \wen x. T \cpar U \cpar V \right) \cpar W
}\]
 assuming $\nfv{y}{(\wen x. T \cpar V)}$
and $\vdash \mathopen{\forall y.}\left( \wen x. T \cpar U \cpar V \right) \cpar W$ holds.
By Lemma~\ref{lemma:universal}, for every variable $z$, $\vdash \left( \wen x. T \cpar U \cpar V \right)\sub{y}{z} \cpar W$ holds.
Furthermore, since $\nfv{y}{(\wen x. T \cpar V)}$, we have equivalence $\left( \wen x. T \cpar U \cpar V \right)\sub{y}{z} \cpar W \equiv \wen x. T \cpar U\sub{y}{z} \cpar V \cpar W$.
The strict multiset inequality $
\occ{ \wen x. T \cpar U\sub{y}{z} \cpar V \cpar W } \mstrict \occ{ \wen x. T \cpar \forall y. U \cpar V \cpar W }
$ holds.
Hence, by the induction hypothesis, for every variable $z$, there exist formulae $P^z$ and $Q^z$ such that $\vdash T \cpar Q^z$ and $\nfv{x}{P^z}$ and either $P^z = Q^z$ or $P^z = \new x. Q^z$, and also
$
\vcenter{
\infer[]{
U\sub{y}{z} \cpar V \cpar W
}{
P^z
}}
$.
Define $W^z$ such that if $P^z = Q^z$ then $W^z = \forall z. Q^z$ and if $P^z = \new x. Q^z$ then $W^z = \new x. \forall z. Q^z$. Hence if $P^z = \new x. Q^z$ then, since $\forall$ permutes with any quantifier using the \textit{all name} rule, $
\vcenter{
\infer[]{
\forall z. \new x. Q^z
}{
 \new x. \forall z. Q^z
}}$.
Hence, for a fresh $z$ such that $\nfv{z}{(\forall y. U \cpar V \cpar W)}$ and $\nfv{z}{T}$, 
the following derivations can be constructed:
\[
\infer[]{
\forall y. U \cpar V \cpar W
}{
\infer[]{
\mathopen{\forall z.} \left( U\sub{y}{z} \cpar V \cpar W \right) 
}{
\infer[]{
\forall z. P^z
}{
W^z
}}}
\qquad
\infer[]{
T \cpar \forall z. Q^z
}{
\infer[]{
 \mathopen{\forall z.} \left( T \cpar  Q^z\right) 
}{
\infer[]{
 \mathopen{\forall z.} \cunit 
}{
 \cunit
}}}
\]
Furthermore, $\size{ W^z } \preceq \size{ \forall y. U \cpar V \cpar W }$ by Lemma~\ref{lemma:bound}; hence 
\[
\size{ T \cpar \forall z. Q^z } \prec \size{ \wen x. T \cpar \forall y. U \cpar V \cpar W }
\] 
since the wen count strictly decreases.


\item Consider the commutative case where the principal connective is $\textit{wen}$ and the bottommost rule is an instance of the extrude new rule of the form 
\[
\infer[,]{
\wen x. P \cpar \new y. Q \cpar R \cpar S
}{
\mathopen{\new y.} \left( \wen x. P \cpar Q \cpar R\right) \cpar S
}
\]
 where $\nfv{y}{\wen x. P \cpar R}$ and also $x \not= y$, where the second condition can be achieved by $\alpha$-conversion.
By the induction hypothesis, there exist $T$ and $U$ such that $\vdash \wen x. P \cpar Q \cpar R \cpar U$, $\nfv{y}{T}$ and either $T = U$ or $T = \wen y. U$, and also
$\vcenter{\infer{S}{T}}$.
Furthermore, the size of the proof of $\wen x. P \cpar Q \cpar R \cpar U$ is bounded above by the size of the proof of $\mathopen{\new y.} \left( \wen x. P \cpar Q \cpar R\right) \cpar S$ and hence strictly bounded above by the size of the proof of $\wen x. P \cpar \new y. Q \cpar R \cpar S$, enabling the induction hypothesis.
Hence, by the induction hypothesis, there exist formulae $V$ and $\hat{V}$ such that $\vdash P \cpar \hat{V}$ and $\nfv{x}{V}$ and either $V = \hat{V}$ or $V = \new x. \hat{V}$, and also $
\vcenter{
\infer[]{
Q \cpar R \cpar U}{
 V
}}$.
Define $W$ such that if $V = \hat{V}$ then $W = \new y. \hat{V}$ and if $V = \new x. \hat{V}$ then $W = \new x. \new y. \hat{V}$.
Now observe that either we have that $T = U$ and $\nfv{y}{U}$ and hence
the derivation $(a)$ below left holds;  or we have that $T = \wen y. U$ and hence the derivation $(b)$ belw  holds. 
Hence,
by applying one of these cases, we have the derivation $(c)$ below, where the premise is equivalent to $W$.
\[
\begin{array}{cccc}
\infer[]{
\new y. Q \cpar R \cpar T
}{
 \mathopen{\new y.} \left(Q \cpar R \cpar U\right)
}
&
\infer[]{
\new y. Q \cpar R \cpar \wen y. U 
}{
\infer[]{
\mathopen{\new y.} \left(Q \cpar R\right) \cpar \wen y. U
}{
 \mathopen{\new y.} \left(Q \cpar R \cpar U\right)
}}
&
\infer[]{
\new y. Q \cpar R \cpar S
}{
\infer[]{
\new y. Q \cpar R \cpar T
}{
\infer[]{
\mathopen{\new y.} \left( Q \cpar R \cpar U \right)
}{
\new y. V
}}} &
\infer[.]{
P \cpar \new y. \hat{V}
}{
\infer[]{
 \mathopen{\new y.} \left( P \cpar \hat{V} \right)
}{
\infer[]{
 \new y. \cunit
}{
 \cunit
}}}
\\
(a)  & (b) & (c) & (d)
\end{array}
\]
Since $\nfv{y}{\wen x. P}$ and $x \not= y$, we have $\nfv{y}{P}$;
thereby the proof $(d)$ above can be constructed.
Furthermore, $\size{ P \cpar \new y.\hat{V} } \prec \size{ \wen x. P \cpar \new y. Q \cpar R \cpar S }$ since by Lemma~\ref{lemma:bound} 
\[
\size{ \new y. \hat{V} } \preceq \size{\new y. Q \cpar R \cpar S}
\]
and the \textit{wen count} strictly decreases.
\end{enumerate}

\fullproof{
Consider the commutative case when the \textit{wen} quantifier commutes with another \textit{wen} quantifier at the root of the principal formula.
In this case, the bottommost rule of a proof is of the form $\wen x. P \cpar \wen y. Q \cpar R \cpar \infer[]{
S}{
 \mathopen{\wen y.} \left( \wen x. P \cpar Q \cpar R\right) \cpar S
}$, where $\nfv{y}{\wen x. P \cpar R}$ and $x \not= y$, where the second condition can be achieved by $\alpha$-conversion.
By the induction hypothesis, there exist $T$ and $U$ such that $\vdash \wen x. P \cpar Q \cpar R \cpar U$, $\nfv{y}{T}$ and either $T = U$ or $T = \new y. U$, and also
$\infer{S}{T}$.
Furthermore, the size of the proof of $\wen x. P \cpar Q \cpar R \cpar U$ is bounded above by the size of the proof of $\mathopen{\wen y.} \left( \wen x. P \cpar Q \cpar R\right) \cpar S$ and hence strictly bounded above by the size of the proof of $\wen x. P \cpar \wen y. Q \cpar R \cpar S$, enabling the induction hypothesis.
Hence, by the induction hypothesis, there exist formulae $V$ and $\hat{V}$ such that $\vdash P \cpar \hat{V}$ and $\nfv{x}{V}$ and either $V = \hat{V}$ or $V = \new x. \hat{V}$, and also $
\infer[]{
Q \cpar R \cpar U 
}{ V }$.

Define $W$ such that if $V = \hat{V}$ then $W = \new y. \hat{V}$ and if $V = \new x. \hat{V}$ then $W = \new x. \new y. \hat{V}$. In either case, $\nfv{x}{W}$.
Now observe that either $T = U$ and $\nfv{y}{U}$ hence
$\infer[]{
\wen y. Q \cpar R \cpar T
}{
\infer[]{
 \mathopen{\wen y.} \left(Q \cpar R \cpar U\right) 
}{
 \mathopen{\new y.} \left(Q \cpar R \cpar U\right)
}}$;
or $T = \new y. U$ hence $
\infer[]{
\wen y. Q \cpar R \cpar \new y. U
}{
\infer[]{
 \mathopen{\wen y.} \left(Q \cpar R\right) \cpar \mathopen{\new y.} U 
}{
\mathopen{\new y.} \left(Q \cpar R \cpar U\right)
}}$.
Hence the following derivation can be constructed, by applying one of these cases:
$
\infer[]{
\wen y. Q \cpar R \cpar S
}{
\infer[]{
 \wen y. Q \cpar R \cpar T
}{
\infer[]{
\mathopen{\new y.} \left( Q \cpar R \cpar U \right)
}{
\new y. V
}}}
$, where the premise is equivalent to $W$.
Since $\nfv{y}{\wen x. P}$ and $x \not= y$ we have $\nfv{y}{P}$,
hence the following proof can be constructed:
$
\infer[]{
P \cpar \new y. \hat{V}
}{
\infer[]{
\mathopen{\new y.}\left( P \cpar \hat{V} \right)
}{
\infer[]{
\new y. \cunit 
}{
\cunit
}}}$.
Furthermore, $\size{ P \cpar \new y. \hat{V} } \prec \size{ \wen x. P \cpar \new y. Q \cpar R \cpar S }$ since by Lemma~\ref{lemma:bound} $\size{ \new y. \hat{V} } \preceq \size{\new y. Q \cpar R \cpar S}$
and the \textit{wen count} strictly decreases.
\smallskip











Consider the commutative case when the \textit{new} quantifier commutes with $P_0 \tensor P_1$ as the principal formula.
In this case the bottommost rule of a proof is of the form $
\infer[]{
\left( P_0 \tensor P_1 \right) \cpar \new y. Q \cpar R \cpar S
}{
\mathopen{\new y.} \left( \left( P_0 \tensor P_1 \right) \cpar Q \cpar R\right) \cpar S
}$,
assuming that $\nfv{y}{\left( P_0 \tensor P_1 \right) \cpar R}$.
By induction, there exist $T$ and $U$ such that $\vdash \left( P_0 \tensor P_1 \right) \cpar Q \cpar R \cpar U$, $\nfv{y}{T}$ and either $T = U$ or $T = \wen y. U$, and also
$\infer{S}{T}$.
Furthermore, the size of the proof of $\left( P_0 \tensor P_1 \right) \cpar Q \cpar R \cpar U$ is bounded above by the size of the proof of $\mathopen{\new y.} \left( \left( P_0 \tensor P_1 \right) \cpar Q \cpar R\right) \cpar S$ and hence strictly bounded above by the size of the proof of $\left( P_0 \tensor P_1 \right) \cpar \new y. Q \cpar R \cpar S$, enabling the induction hypothesis.
Hence, by the induction hypothesis, there exist formulae $V_i$ and $W_i$ such that $\vdash P_0 \cpar V_i$ and $\vdash P_1 \cpar W_i$, for $1 \leq i \leq n$, and also $n$-ary killing context $\tcontext{}$ such that
$
\infer[]{
Q \cpar R \cpar U 
}{
\tcontext{ V_i \cpar W_i \colon 1 \leq i \leq n }
}$.
Furthermore, the size of the proofs of $P_0 \cpar V_i$ and $P_1 \cpar W_i$ are bounded above by the size of the proof of $\left( P_0 \tensor P_1 \right) \cpar \new y. Q \cpar R \cpar S$.
Now observe that either $T = U$ and $\nfv{y}{U}$ hence
$\infer[]{
\new y. Q \cpar R \cpar T
}{
 \mathopen{\new y.} \left(Q \cpar R \cpar U\right)
}$;
or $T = \wen y. U$ hence $
\infer[]{
\new y. Q \cpar R \cpar \wen y. U
}{
\infer[]{
 \mathopen{\new y.} \left(Q \cpar R\right) \cpar \wen y. U
}{
 \mathopen{\new y.} \left(Q \cpar R \cpar U\right)
}}$.
Hence the following derivation can be constructed, by the above observations:
$
\infer[]{
\new y. Q \cpar R \cpar S
}{
\infer[]{
\new y. Q \cpar R \cpar T
}{
\infer[]{
\mathopen{\new y.} \left( Q \cpar R \cpar U \right)
}{
\new y. \tcontext{ V_i \cpar W_i \colon 1 \leq i \leq n }
}}}
$.
Observe that $\new y. \tcontext{ }$ is a $n$-ary killing context as required.
\smallskip


Consider the commutative case when the \textit{wen} quantifier commutes with $P_0 \tensor P_1$ as the principal formula.
In this case the bottommost rule of a proof is of the form $
\infer[]{
\left( P_0 \tensor P_1 \right) \cpar \wen y. Q \cpar R \cpar S
}{
 \mathopen{\wen y.} \left( \left( P_0 \tensor P_1 \right) \cpar Q \cpar R\right) \cpar S
}$, where $\nfv{y}{\left( P_0 \tensor P_1 \right) \cpar R}$.
By induction, there exist $T$ and $U$ such that $\vdash \left( P_0 \tensor P_1 \right) \cpar Q \cpar R \cpar U$, $\nfv{y}{T}$ and either $T = U$ or $T = \new y. U$, and also
$\infer{S}{T}$.
Furthermore, the size of the proof of $\left( P_0 \tensor P_1 \right) \cpar Q \cpar R \cpar U$ is bounded above by the size of the proof of $\mathopen{\wen y.} \left( \left( P_0 \tensor P_1 \right) \cpar Q \cpar R\right) \cpar S$ and hence strictly bounded above by the size of the proof of $\left( P_0 \tensor P_1 \right) \cpar \wen y. Q \cpar R \cpar S$, enabling the induction hypothesis.
Hence, by the induction hypothesis, there exist formulae $V_i$ and $W_i$ such that $\vdash P_0 \cpar V_i$ and $\vdash P_1 \cpar W_i$, for $1 \leq i \leq n$, and also $n$-ary killing context $\tcontext{}$ such that
$\infer[]{
Q \cpar R \cpar U
}{
\tcontext{ V_i \cpar W_i \colon 1 \leq i \leq n }
}$.
Furthermore, the size of the proofs of $P_0 \cpar V_i$ and $P_1 \cpar W_i$ are bounded above by the size of the proof of $\left( P_0 \tensor P_1 \right) \cpar \wen y. Q \cpar R \cpar S$.
Now observe that either $T = U$ and $\nfv{y}{U}$ hence
$\infer[]{
\wen y. Q \cpar R \cpar T
}{
\infer[]{
\mathopen{\wen y.} \left(Q \cpar R \cpar U\right)
}{
 \mathopen{\new y.} \left(Q \cpar R \cpar U\right)
}}$;
or $T = \new y. U$ hence $
\infer[]{
\wen y. Q \cpar R \cpar \new y. U
}{
\infer[]{
\mathopen{\wen y.} \left(Q \cpar R\right) \cpar \new y. U
}{
 \mathopen{\new y.} \left(Q \cpar R \cpar U\right)
}}$.
Hence the following derivation can be constructed, by the above observations:
$
\infer[]{
\wen y. Q \cpar R \cpar S
}{
\infer[]{
\wen y. Q \cpar R \cpar T
}{
\infer[]{
\mathopen{\new y.} \left( Q \cpar R \cpar U \right)
}{
\mathopen{\new y.} \tcontext{ V_i \cpar W_i \colon 1 \leq i \leq n }
}}}
$.
Observe that $\new y. \tcontext{ }$ is a $n$-ary killing context as required.
\smallskip



Consider the commutative case where the \textit{new} quantifier commutes with $P_0 \cseq P_1$ as the principal formula.
In this case the bottommost rule of a proof is of the form $
\infer[]{
\left( P_0 \cseq P_1 \right) \cpar \new y. Q \cpar R \cpar S 
}{
\mathopen{\new y.} \left( \left( P_0 \cseq P_1 \right) \cpar Q \cpar R\right) \cpar S
}$,
where $\nfv{y}{\left( P_0 \cseq P_1 \right) \cpar R}$.
By induction, there exist $T$ and $U$ such that $\vdash \left( P_0 \cseq P_1 \right) \cpar Q \cpar R \cpar U$, $\nfv{y}{T}$ and either $T = U$ or $T = \wen y. U$, and also
$\infer{S}{T}$.
Furthermore, the size of the proof of $\left( P_0 \cseq P_1 \right) \cpar Q \cpar R \cpar U$ is bounded above by the size of the proof of $\mathopen{\new y.} \left( \left( P_0 \cseq P_1 \right) \cpar Q \cpar R\right) \cpar S$ and hence strictly bounded above by the size of the proof of $\left( P_0 \cseq P_1 \right) \cpar \new y. Q \cpar R \cpar S$, enabling the induction hypothesis.
Hence, by the induction hypothesis, there exist formulae $V_i$ and $W_i$ such that $\vdash P_0 \cpar V_i$ and $\vdash P_1 \cpar W_i$, for $1 \leq i \leq n$, and also $n$-ary killing context $\tcontext{}$ such that
$\infer[]{
Q \cpar R \cpar U
}{
\tcontext{ V_i \cseq W_i \colon 1 \leq i \leq n }
}$.
Furthermore, the size of the proofs of $P_0 \cpar V_i$ and $P_1 \cpar W_i$ are bounded above by the size of the proof of $\left( P_0 \cseq P_1 \right) \cpar \new y. Q \cpar R \cpar S$.
Now observe that either $T = U$ and $\nfv{y}{U}$ hence
$
\infer[]{
\new y. Q \cpar R \cpar T
}{
\mathopen{\new y.} \left(Q \cpar R \cpar U\right)
}$;
or $T = \wen y. U$ hence $
\infer[]{
\new y. Q \cpar R \cpar \wen y. U
}{
\infer[]{
 \mathopen{\new y.} \left(Q \cpar R\right) \cpar \wen y. U 
}{ \mathopen{\new y.} \left(Q \cpar R \cpar U\right)
}}$.
Hence the following derivation can be constructed:
$
\infer[]{
\new y. Q \cpar R \cpar S
}{
\infer[]{
\new y. Q \cpar R \cpar T
}{
\infer[]{
\mathopen{\new y.} \left( Q \cpar R \cpar U \right)
}{
\new y. \tcontext{ V_i \cseq W_i \colon 1 \leq i \leq n }
}}}
$.
Observe that $\new y. \tcontext{ }$ is a $n$-ary killing context as required.
\smallskip


Consider the commutative case when the \textit{wen} quantifier commutes with $P_0 \cseq P_1$ as the principal formula.
In this case the bottommost rule of a proof is of the form $
\infer[]{
\left( P_0 \cseq P_1 \right) \cpar \wen y. Q \cpar R \cpar S
}{
\mathopen{\wen y.} \left( \left( P_0 \cseq P_1 \right) \cpar Q \cpar R\right) \cpar S
}$. where $\nfv{y}{\left( P_0 \cseq P_1 \right) \cpar R}$.
By induction, there exist $T$ and $U$ such that $\vdash \left( P_0 \cseq P_1 \right) \cpar Q \cpar R \cpar U$, $\nfv{y}{T}$ and either $T = U$ or $T = \new y. U$, and also
$\infer{S}{T}$.
Furthermore, the size of the proof of $\left( P_0 \cseq P_1 \right) \cpar Q \cpar R \cpar U$ is bounded above by the size of the proof of $\mathopen{\wen y.} \left( \left( P_0 \cseq P_1 \right) \cpar Q \cpar R\right) \cpar S$ and hence strictly bounded above by the size of the proof of $\left( P_0 \cseq P_1 \right) \cpar \wen y. Q \cpar R \cpar S$, enabling the induction hypothesis.
Hence, by the induction hypothesis, there exist formulae $V_i$ and $W_i$ such that $\vdash P_0 \cpar V_i$ and $\vdash P_1 \cpar W_i$, for $1 \leq i \leq n$, and also $n$-ary killing context $\tcontext{}$ such that
$
\infer[]{
Q \cpar R \cpar U 
}{
 \tcontext{ V_i \cseq W_i \colon 1 \leq i \leq n }
}$.
Furthermore, the size of the proofs of $P_0 \cpar V_i$ and $P_1 \cpar W_i$ are bounded above by the size of the proof of $\left( P_0 \cseq P_1 \right) \cpar \wen y. Q \cpar R \cpar S$.
Now observe that either $T = U$ and $\nfv{y}{U}$ hence
$
\infer[]{
\wen y. Q \cpar R \cpar T
}{
\infer[]{
 \wen y. \left(Q \cpar R \cpar U\right) 
}{ \mathopen{\new y.} \left(Q \cpar R \cpar U\right)
}}$;
or $T = \new y. U$ hence $
\infer[]{
\wen y. Q \cpar R \cpar \new y. U
}{
\infer[]{
 \wen y. \left(Q \cpar R\right) \cpar \new y. U 
}{
 \mathopen{\new y.} \left(Q \cpar R \cpar U\right)
}}$.
Hence the following derivation can be constructed:
$
\infer[]{
\wen y. Q \cpar R \cpar S
}{
\infer[]{
\wen y. Q \cpar R \cpar T
}{
\infer[]{
\mathopen{\new y.} \left( Q \cpar R \cpar U \right)
}{
\mathopen{\new y.} \tcontext{ V_i \cseq W_i \colon 1 \leq i \leq n }
}}}
$.
Observe that $\new y. \tcontext{ }$ is a $n$-ary killing context as required.
\smallskip




Consider commutative cases where the principal formula moves entirely to the left hand side of a \textit{seq} operator. 
For principal formula $\wen x. U$, the bottommost rule in a proof is of the form
$
\infer[]{
\wen x. U \cpar \left( V \cseq P \right) \cpar W \cpar Q
}{
\left( \left( \wen x. U \cpar V \cpar W\right) \cseq P \right) \cpar Q
}
$
such that $\vdash \left( \left( \wen x. U \cpar V \cpar W\right) \cseq P \right) \cpar Q$ holds.
By the induction hypothesis, there exist $R_i$ and $S_i$ such that $\vdash \wen x. U \cpar V \cpar W \cpar R_i$ and $\vdash P \cpar S_i$, for $1 \leq i \leq n$, and $n$-ary killing context $\tcontext{}$ such that the derivation
$
\infer[]{
 Q
}{
 \tcontext{ R_1 \cseq S_1, \hdots, R_n \cseq S_n }
}
$ holds, 
and furthermore the size of the proof of $\wen x. U \cpar V \cpar W \cpar R_i$ is bounded above by the size of the proof of $\left( \left( \wen x. U \cpar V \cpar W\right) \cseq P \right) \cpar Q$ hence strictly bounded above by the size of the proof of $\wen x. U \cpar \left( V \cseq P \right) \cpar W \cpar Q$ enabling the induction hypothesis.
By the induction hypothesis again, there exist formulae $T_i$ and $\hat{T}_i$ such that $\vdash U \cpar \hat{T}_i$ and $\nfv{x}{T_i}$ and either $T_i = U_i$ or $T_i = \new x. U_i$, and also the derivation $
\infer[]{
V \cpar W \cpar R_i
}{
 T_i
}$ holds.
Hence the following derivation can be constructed.
\[
\infer[]{
\left( V \cseq P \right) \cpar W \cpar Q
}{
\infer[]{
\left( V \cseq P \right) \cpar W \cpar \tcontext{ R_1 \cseq S_1, \hdots, R_n \cseq S_n } \\
}{
\infer[]{
\tcontext{ \left( V \cseq P \right) \cpar W \cpar R_i \cseq S_i \colon 1 \leq i \leq n } \\
}{
\infer[]{
\tcontext{ \left( V \cpar W \cpar R_i \right) \cseq \left( P \cpar S_i \right) \colon 1 \leq i \leq n } 
}{
\infer[]{
\tcontext{ V \cpar W \cpar R_i \colon 1 \leq i \leq n } 
}{
\tcontext{ T_i \colon 1 \leq i \leq n }
}}}}}
\]
Furthermore, by Lemma~\ref{lemma:bound},
$\size{T_i} \preceq \size{\left( V \cseq P \right) \cpar W \cpar Q}$
and hence $\size{ U \cpar \hat{T}_i } \prec \size{\wen x. U \cpar \left( V \cseq P \right) \cpar W \cpar Q}$, since the \textit{wen} count strictly decreases.
The cases where the principal formula moves entirely to the right hand side of the \textit{seq} operator, and the analogous case for \textit{times}, are similar to the above case.






\item \textbf{Commutative cases involving all and with.}
Consider the commutative case for \textit{with} where $T_0 \tensor T_1$ is the principal formula. In this case the bottommost rule is the following form, such that $\vdash \left(\left(\left(T_0 \tensor T_1\right) \cpar U \cpar W\right) \wwith \left(\left(T_0 \tensor T_1\right) \cpar V \cpar W\right)\right) \cpar P$ holds:
$
\infer[]{
\left(T_0 \tensor T_1\right) \cpar \left( U \wwith V \right) \cpar W \cpar P
}{
\left(\left(T_0 \tensor T_1\right) \cpar U \cpar W \wwith
\left(T_0 \tensor T_1\right) \cpar V \cpar W\right)
\cpar P
}
$.
By the induction hypothesis, $\vdash \left(T_0 \tensor T_1\right) \cpar U \cpar W \cpar P$ and $\vdash \left(T_0 \tensor T_1\right) \cpar V \cpar W \cpar P$; and furthermore strict multiset inequalities 
$\occ{\left(T_0 \tensor T_1\right) \cpar U \cpar W \cpar P} \mstrict \occ{ \left(T_0 \tensor T_1\right) \cpar \left( U \wwith V \right) \cpar W \cpar P }$
and
$\occ{\left(T_0 \tensor T_1\right) \cpar V \cpar W \cpar P} \mstrict \occ{ \left(T_0 \tensor T_1\right) \cpar \left( U \wwith V \right) \cpar W \cpar P }$
hold.
Hence, by the induction hypothesis, there exist $Q^0_i$ and $Q^1_i$ such that $\vdash T_0 \cpar Q^0_i$ and $\vdash T_1 \cpar Q^1_i$, for $1 \leq i \leq m$; and $R_j^0$ and $R_j^1$ such that
$\vdash T_0 \cpar R^0_j$ and $\vdash T_1 \cpar R^1_j$, for $1 \leq j \leq n$;
and also $m$-ary killing context $\tcontextn{0}{}$ and $n$-ary killing context $\tcontextn{1}{}$
such that the two derivations 
$
\infer[]{
U \cpar W \cpar P 
}{
 \tcontextn{0}{Q^0_i \cpar Q^1_i \colon 1 \leq i \leq m }
}
$
and
$
\infer[]{
V \cpar W \cpar P
}{
 \tcontextn{1}{R^0_j \cpar R^1_j \colon 1 \leq j \leq n }
}
$ hold.
Furthermore, the size of the proofs of $T_0 \cpar R^0_j$, $T_1 \cpar R^1_j$, $T_0 \cpar Q^0_i$ and $T_1 \cpar Q^1_i$ are bounded above by the size of the proof of $\left(T_0 \tensor T_1\right) \cpar \left( U \wwith V \right) \cpar W \cpar P$.
Thereby the following derivation can be constructed.
\[
\infer[]{
\left( U \wwith V \right) \cpar W \cpar P
}{
\infer[]{
\left(U \cpar W \cpar P\right) \wwith \left(V \cpar W \cpar P\right) 
}{
\tcontextn{0}{Q^0_i \cpar Q^1_i \colon 1 \leq i \leq m }
\wwith
\tcontextn{1}{R^0_j \cpar R^1_j \colon 1 \leq j \leq n }
}}
\]



Consider the commutative case where universal quantification commutes with $T_0 \tensor T_1$  as the principal formula. Suppose the bottommost rule is of the form
$
\infer[]{
\left(T_0 \tensor T_1\right) \cpar \forall y. U \cpar V \cpar W
}{
\mathopen{\forall y.}\left( \left(T_0 \tensor T_1\right) \cpar U \cpar V \right) \cpar W
}$, 
assuming $\nfv{y}{(\left(T_0 \tensor T_1\right) \cpar V)}$
where $\vdash \mathopen{\forall y.}\left( \left(T_0 \tensor T_1\right) \cpar U \cpar V \right) \cpar W$ holds.
By Lemma~\ref{lemma:universal}, for every variable $z$, $\vdash \left( \left(T_0 \tensor T_1\right) \cpar U \cpar V \right)\sub{y}{z} \cpar W$ holds.
Furthermore, since $\nfv{y}{(\left(T_0 \tensor T_1\right) \cpar V)}$, $\left( \left(T_0 \tensor T_1\right) \cpar U \cpar V \right)\sub{y}{z} \cpar W \equiv \left(T_0 \tensor T_1\right) \cpar U\sub{y}{z} \cpar V \cpar W$.
Since $\forall$ is removed, $
\occ{ \left(T_0 \tensor T_1\right) \cpar U\sub{y}{z} \cpar V \cpar W } \mstrict \occ{ \left(T_0 \tensor T_1\right) \cpar \forall y. U \cpar V \cpar W }
$ holds.
Pick a fresh $z$ such that $\nfv{z}{(\forall y. U \cpar V \cpar W)}$.
Hence, by the induction hypothesis, there exist formulae $P_i$ and $Q_i$ such that $\vdash T_0 \cpar P_i$ and $\vdash T_1 \cpar Q_i$, for $1 \leq i \leq n$; and also $n$-ary killing context $\tcontext{}$ such that
$
\infer[]{
U\sub{y}{z} \cpar V \cpar W
}{
\tcontext{ P_i \cpar Q_i \colon 1 \leq i \leq n }
}
$.
Furthermore, the size of the proof of $T_0 \cpar P_i$ and $T_1 \cpar Q_i$ are bounded above by the size of the proof of $\left(T_0 \tensor T_1\right) \cpar \forall y. U \cpar V \cpar W$.
Since $z$ was chosen such that $\nfv{z}{ \forall y. U \cpar V \cpar W }$ the following derivation can be constructed, as required.
$
\infer[]{
\forall y. U \cpar V \cpar W
}{
\infer[]{
\mathopen{\forall z.} \left( U\sub{y}{z} \cpar V \cpar W \right) 
}{
\mathopen{\forall z.} \tcontext{ P_i \cpar Q_i \colon 1 \leq i \leq n }
}}
$.
\smallskip


Consider the commutative case for \textit{with} where $T_0 \cseq T_1$ is the principal formula. The bottommost rule is the form
$
\infer[]{
\left(T_0 \cseq T_1\right) \cpar \left( U \wwith V \right) \cpar W \cpar P
}{
\left(\left(T_0 \cseq T_1\right) \cpar U \cpar W \wwith
\left(T_0 \cseq T_1\right) \cpar V \cpar W\right)
\cpar P
}
$
where $\vdash \left(\left(\left(T_0 \cseq T_1\right) \cpar U \cpar W\right) \wwith \left(\left(T_0 \cseq T_1\right) \cpar V \cpar W\right)\right) \cpar P$ holds.
By the induction hypothesis, $\vdash \left(T_0 \cseq T_1\right) \cpar U \cpar W \cpar P$ and $\vdash \left(T_0 \cseq T_1\right) \cpar V \cpar W \cpar P$; and furthermore the strict multiset inequalities
$\occ{\left(T_0 \cseq T_1\right) \cpar U \cpar W \cpar P} \mstrict \occ{ \wen x. T \cpar \left( U \wwith V \right) \cpar W \cpar P }$
and
$\occ{\left(T_0 \cseq T_1\right) \cpar V \cpar W \cpar P} \mstrict \occ{ \wen x. T \cpar \left( U \wwith V \right) \cpar W \cpar P }$ hold.

Hence, by the induction hypothesis, there exist $Q^0_i$ and $Q^1_i$ such that $\vdash T_0 \cpar Q^0_i$ and $\vdash T_1 \cpar Q^1_i$, for $1 \leq i \leq m$; and $R_j^0$ and $R_j^1$ such that
$\vdash T_0 \cpar R^0_j$ and $\vdash T_1 \cpar R^1_j$, for $1 \leq j \leq n$;
and also $m$-ary killing context $\tcontextn{0}{}$ and $n$-ary killing context $\tcontextn{1}{}$
such that the two derivations 
$
\infer[]{
U \cpar W \cpar P
}{
 \tcontextn{0}{Q^0_i \cseq Q^1_i \colon 1 \leq i \leq m }
}
$
and
$
\infer[]{
V \cpar W \cpar P 
}{
  \tcontextn{1}{R^0_j \cseq R^1_j \colon 1 \leq j \leq n }
}
$ hold.
Furthermore, the size of the proofs of $T_0 \cpar R^0_j$, $T_1 \cpar R^1_j$, $T_0 \cpar Q^0_i$ and $T_1 \cpar Q^1_i$ are bounded above by the size of the proof of $\left(T_0 \cseq T_1\right) \cpar \left( U \wwith V \right) \cpar W \cpar P$.
Thereby the following derivation can be constructed.
\[
\infer[]{
\left( U \wwith V \right) \cpar W \cpar P
}{
}{
\infer[]{
\left(U \cpar W \cpar P\right) \wwith \left(V \cpar W \cpar P\right) 
}{
\tcontextn{0}{Q^0_i \cseq Q^1_i \colon 1 \leq i \leq m }
\wwith
\tcontextn{1}{R^0_j \cseq R^1_j \colon 1 \leq j \leq n }
}}
\]
\smallskip



Consider the commutative case where universal quantification commutes with $T_0 \cseq T_1$  as the principal formula. Suppose the bottommost rule is of the form
$
\infer[]{
\left(T_0 \cseq T_1\right) \cpar \forall y. U \cpar V \cpar W
}{
\mathopen{\forall y.}\left( \left(T_0 \cseq T_1\right) \cpar U \cpar V \right) \cpar W
}$,
 assuming $\nfv{y}{(\wen x. T \cpar V)}$
where $\vdash \mathopen{\forall y.}\left( \left(T_0 \cseq T_1\right) \cpar U \cpar V \right) \cpar W$ holds.
By Lemma~\ref{lemma:universal}, for every variable $z$, $\vdash \left( \left(T_0 \cseq T_1\right) \cpar U \cpar V \right)\sub{y}{z} \cpar W$ holds.
Furthermore, since $\nfv{y}{(\left(T_0 \cseq T_1\right) \cpar V)}$, $\left( \left(T_0 \cseq T_1\right) \cpar U \cpar V \right)\sub{y}{z} \cpar W \equiv \left(T_0 \cseq T_1\right) \cpar U\sub{y}{z} \cpar V \cpar W$.

The strict multiset inequality $
\occ{ \left(T_0 \cseq T_1\right) \cpar U\sub{y}{z} \cpar V \cpar W } \mstrict \occ{ \left(T_0 \tensor T_1\right) \cpar \forall y. U \cpar V \cpar W }
$ holds.
Pick a fresh $z$ such that $\nfv{z}{(\forall y. U \cpar V \cpar W)}$.
Hence, by the induction hypothesis, there exist formulae $P_i$ and $Q_i$ such that $\vdash T_0 \cpar P_i$ and $\vdash T_1 \cpar Q_i$, for $1 \leq i \leq n$; and also $n$-ary killing context $\tcontext{}$ such that
$
\infer[]{
U\sub{y}{z} \cpar V \cpar W
}{
\tcontext{ P_i \cseq Q_i \colon 1 \leq i \leq n }
}
$.
Furthermore, the size of the proof of $T_0 \cpar P_i$ and $T_1 \cpar Q_i$ are bounded above by the size of the proof of $\left(T_0 \cseq T_1\right) \cpar \forall y. U \cpar V \cpar W$.

Since $z$ was chosen such that $\nfv{z}{ \forall y. U \cpar V \cpar W }$ the following derivation can be constructed, as required.
$
\infer[]{
\forall y. U \cpar V \cpar W
}{
\infer[]{
\mathopen{\forall z.} \left( U\sub{y}{z} \cpar V \cpar W \right) 
}{
\forall z. \tcontext{ P_i \cseq Q_i \colon 1 \leq i \leq n }
}}
$.
\smallskip


Consider the commutative case for \textit{sequence} rule in the presence of principal formula $T \cseq U$, where the \textit{seq} connective in the principal formula is not active on the \textit{sequence} rule. In this case, the bottommost rule in a proof is an instance of the \textit{sequence} rule of the form
$
\infer[]{
\left( T \cseq U \right) \cpar \left( V \cseq P \right) \cpar W \cpar Q
}{
\left( \left( \left( T \cseq U\right) \cpar V \cpar W\right) \cseq P \right) \cpar Q
}
$,
where $T \cseq U \not\equiv \cunit$ and $P \not\equiv \cunit$ and $\vdash \left( \left( \left( T \cseq U\right) \cpar V \cpar W\right) \cseq P \right) \cpar Q$ holds.
By the induction hypothesis, there exists $R_i$, $S_i$ such that $\vdash  \left( T \cseq U\right) \cpar V \cpar W \cpar R_i$ and $\vdash P \cpar S_i$, for $1 \leq i \leq n$, and $n$-ary killing context $\tcontext{}$ such that the following derivation holds:
$
\infer[]{
 Q
}{
 \tcontext{ R_1 \cseq S_1, \hdots, R_n \cseq S_n }
}
$.
Furthermore, $\size{\left( T \cseq U\right) \cpar V \cpar W \cpar R_i} \mleq \size{\left( \left( \left( T \cseq U\right) \cpar V \cpar W\right) \cseq P \right) \cpar Q}$ hence the induction hypothesis is enabled again.
By the induction hypothesis, for $1 \leq i \leq n$, there exist formulae $P^i_j$ and $Q^i_j$ such that $\vdash T \cpar P^i_j$ and $\vdash U \cpar Q^i_j$ hold, for $1 \leq j \leq m_i$, and killing contexts $\tcontextn{i}{}$ such that the following derivation holds.
\[
\infer[]{
V \cpar W \cpar R_i
}{
 \tcontextn{i}{P^i_1 \cseq Q^i_1, \hdots, P^i_{m_i} \cseq Q^i_{m_i}}
}
\]
Furthermore the following strict multiset inequalities hold.
\[
\occ{T \cpar P^i_j} \mstrict \occ{\left( T \cseq U \right) \cpar \left( V \cseq P \right) \cpar W \cpar Q
}
\qquad
\mbox{and}
\qquad
\occ{U \cpar Q^i_j} \mstrict \occ{\left( T \cseq U \right) \cpar \left( V \cseq P \right) \cpar W \cpar Q
}
\]
Hence the following derivation can be constructed, as required.
\[
\infer[]{
\left( V \cseq P \right) \cpar W \cpar Q
}{
\infer[]{
\left( V \cseq P \right) \cpar W \cpar \tcontext{ R_1 \cseq S_1, \hdots, R_n \cseq S_n }
}{
\infer[]{
\tcontext{ \left( V \cseq P \right) \cpar W \cpar \left(R_1 \cseq S_1\right), \hdots, \left( V \cseq P \right) \cpar W \cpar \left(R_n \cseq S_n\right) }
}{
\infer[]{
\tcontext{ \left( V \cpar W \cpar R_1 \right) \cseq \left( P \cpar S_1 \right), \hdots, \left( V \cpar W \cpar R_n \right) \cseq \left( P \cpar S_n \right) }
}{
\infer[]{
\tcontext{ V \cpar W \cpar R_1, \hdots, V \cpar W \cpar R_n }
}{
\tcontext{ \tcontextn{i}{P^i_j \cseq Q^i_j \colon 1 \leq j \leq m_i } \colon 1 \leq i \leq n }
}}}}}
\]
The case for the \rseq rule commuting with the principal formula $T \tensor U$ is similar to the above case. Also the cases for the \textit{switch} rule commuting with seq and times as the principal formula, follow a similar pattern.
\smallskip

}  




\item \textbf{Commutative cases deep in contexts.}
In many commutative cases, the bottommost rule does not interfere with the principal formula either directly or indirectly. Two such cases are presented for \textit{wen} as the principal connective. Other such cases use almost identical reasoning.

\begin{enumerate}[label*=\textbf{.\arabic*}]

\item Consider when a rule is applied outside the scope of the principal formula.
In this case, the bottommost rule in a proof is of the form
\[
\infer[\mbox{, such that $\vdash \wen x. U \cpar \context{ W }$.}]{
\wen x. U \cpar \context{ V }
}{
 \wen x. U \cpar \context{ W }
}
\]
By the induction hypothesis, there exist formulae $P$ and $Q$ such that $\vdash U \cpar Q$ and $\nfv{x}{P}$ and either $P = Q$ or $P = \new x. Q$, and also $
\vcenter{\infer[]{
\context{ W }
}{
 P
}}
$.
Hence clearly derivation $
\vcenter{\infer[]{
\context{ V }
}{
\infer[]{
 \context{ W } 
}{
  P
}}}
$
 holds.
Furthermore, by Lemma~\ref{lemma:bound}, $\size{ \wen x. U \cpar \context{ W }} \prec \size{ U \cpar \context{ W }}$ 
and $\size{ U \cpar \context{ W }} \preceq \size{\wen x. U \cpar \context{ V }}$.


\fullproof{
Assume that the following application of any rule
$
\infer[]{
\left( T \cseq U \right) \cpar \context{ V }
}{
 \left( T \cseq U \right) \cpar \context{ W }
}$
is the bottommost rule in a proof,
such that $\vdash \left( T \cseq U \right) \cpar \context{ W }$.
By the induction hypothesis, there exist $n$-ary killing context $\tcontext{}$ and formulae $Q_i$ and $R_i$ such that $\vdash T \cpar Q_i$ and $\vdash U \cpar R_i$, for $1 \leq i \leq n$, such that $
\infer[]{
\context{ W } 
}{
\tcontext{ Q_1 \cseq R_1, \ldots, Q_n \cseq R_n }
}
$.
Hence, the derivation 
$
\infer[]{
\context{ V }
}{
\infer[]{
 \context{ W } 
 \tcontext{ Q_1 \cseq R_1, \ldots, Q_n \cseq R_n }
}}
$
holds, satisfying the induction invariant.



Alternatively, the bottommost rule may appear inside the context of principal formula without affecting the root connective of the principal formula. 
Consider the case where \textit{seq} is the principal formula.
Assume that the following application of any rule is the bottommost rule in a proof
$
\infer[]{
\left( \context{T} \cseq V \right) \cpar W 
}{
 \left( \context{U} \cseq V \right) \cpar W
}
$
such that $\vdash \left( \context{U} \cseq V \right) \cpar W$ has a proof of length $n$. 
Hence by induction, there exist $n$-ary killing context $\tcontext{}$ and formulae $P_i$ and $Q_i$ such that $\vdash \context{U} \cpar P_i$ and $\vdash V \cpar Q_i$ hold and have a proof no longer than $n$, for $1 \leq i \leq n$, and also 
$
\infer[]{
W
}{
 \tcontext{ P_1 \cseq Q_1, \ldots, P_n \cseq Q_n }
}
$.
Hence we can construct the following proof of length no longer than $n+1$, for all $i$, as required:
$
\infer[]{
\context{T} \cpar P_i
}{
\infer[]{
   \context{U} \cpar P_i 
}{
 \cunit
}}
$.
}

\item Consider the case where the following application of any rule in a derivation of the form 
\[
\infer[]{
\wen x. \context{T} \cpar W
}{
 \wen x. \context{U} \cpar W
}
\]
is the bottommost rule is a proof of length $k+1$, where $\vdash \wen x. \context{U} \cpar W$ has a proof of length $k$.
Hence, by induction, there exist formulae $P$ and $Q$ such that $\vdash \context{U} \cpar Q$ and $\nfv{x}{P}$ and either $P = Q$ or $P = \new x. Q$, and also
$
\vcenter{
\infer[]{
W
}{
P
}}
$.
Furthermore, the size of the proof of $\context{U} \cpar Q$ is bounded above by the size of the proof of $\wen x. \context{U} \cpar W$; hence either $\size{\context{U} \cpar Q} \prec \size{\wen x. \context{U} \cpar W}$ or $\size{\context{U} \cpar Q} = \size{\wen x. \context{U} \cpar W }$ and the length of the proof of $U \cpar Q$ is bound by $k$.
The proof 
$
\vcenter{
\infer[]{
\context{T} \cpar Q
}{
\infer[]{
 \context{U} \cpar Q 
}{
   \cunit
}}}
$ can be constructed as required.
Furthermore, if $\size{\context{U} \cpar Q} \prec \wen x. \size{\context{U} \cpar W}$ then $\size{\context{U} \cpar Q} \prec \size{\wen x. \context{U} \cpar \context{ V }}$, by Lemma~\ref{lemma:bound}. Otherwise, $\size{\context{U} \cpar Q} = \size{\wen x. \context{U} \cpar W }$
hence $\size{U \cpar Q} \preceq \size{\wen x. U \cpar \context{ V }}$ by Lemma~\ref{lemma:bound} and the length of the proof of $\vdash \context{T} \cpar Q$ is $k+1$.
Thereby in either case, the size of the proof of $\context{T} \cpar Q$ is bounded above by the size of the proof of $\wen x. \context{T} \cpar W$.
\end{enumerate}
 
\end{enumerate}

This covers all scenarios for the bottommost rule, hence splitting follows by induction over the size of the proof.
\end{proof}
