

\documentclass[11pt,a4paper]{article}
\usepackage[hyperref]{emnlp-ijcnlp-2019}
\usepackage{times}
\usepackage{latexsym}
\usepackage{graphicx}
\usepackage{multirow}
\usepackage{float}
\usepackage{amssymb}
\usepackage{url}
\usepackage{booktabs}


\usepackage{helvet} \usepackage{courier} 

\usepackage{color,soul}
\usepackage{amsmath,mathtools}
\usepackage{amsfonts}
\usepackage[capitalize]{cleveref}
\usepackage{multirow,multicol}
\usepackage{MnSymbol}
\usepackage{array}
\usepackage{subcaption}
\usepackage{pifont}\newcommand{\cmark}{\ding{51}}\newcommand{\xmark}{\ding{55}}

\newcolumntype{L}[1]{>{\raggedright\let\newline\\\arraybackslash\hspace{0pt}}m{#1}}
\newcolumntype{C}[1]{>{\centering\let\newline\\\arraybackslash\hspace{0pt}}m{#1}}
\newcolumntype{R}[1]{>{\raggedleft\let\newline\\\arraybackslash\hspace{0pt}}m{#1}}
\newcommand\nnfootnote[1]{\begin{NoHyper}
  \renewcommand\thefootnote{}\footnote{#1}\addtocounter{footnote}{-1}\end{NoHyper}
}



\aclfinalcopy 





\newcommand{\slfrac}[2]{\left.#1\middle/#2\right.}
\newcommand*\rot{\rotatebox{90}}
\newcommand\BibTeX{B{\sc ib}\TeX}
\newcommand\confname{EMNLP-IJCNLP 2019}
\newcommand\conforg{SIGDAT}

\DeclareMathOperator*{\ReLU}{\text{ReLU}}
\DeclareMathOperator*{\softmax}{\text{softmax}}
\DeclareMathOperator*{\argmax}{\text{argmax}}
\newcommand\norm[1]{\left\lVert#1\right\rVert}
\newcommand\disparagraph[1]{\paragraph{\small ~~#1}}

\makeatletter
\def\thanks#1{\protected@xdef\@thanks{\@thanks
        \protect\footnotetext{#1}}}
\makeatother



\title{DialogueGCN: A Graph Convolutional Neural Network for Emotion~Recognition~in~Conversation} 


\author{Deepanway Ghosal, Navonil Majumder, Soujanya Poria\thanks{ Corresponding author},\\ \textbf{Niyati Chhaya and Alexander Gelbukh}\\
  Singapore University of Technology and Design, Singapore \\
 Instituto Polit\'ecnico Nacional, CIC, Mexico \\
  Adobe Research, India\\
\\ {\tt \{1004721@mymail.,sporia@\}sutd.edu.sg}, {\tt navo@nlp.cic.ipn.mx},\\ {\tt nchhaya@adobe.com}, {\tt gelbukh@gelbukh.com} 
}


\date{}

\begin{document}

\maketitle

\begin{abstract}

Emotion recognition in conversation (ERC) has received much attention, lately, from researchers due to its potential widespread applications in diverse areas, such as health-care, education, and human resources. In this paper, we present Dialogue Graph Convolutional Network (DialogueGCN), a graph neural network based approach to ERC. We leverage self and inter-speaker dependency of the interlocutors to model conversational context for emotion recognition. Through the graph network, DialogueGCN addresses context propagation issues present in the current RNN-based methods. We empirically show that this method alleviates such issues, while outperforming the current state of the art on a number of benchmark emotion classification datasets.
\end{abstract}

\section{Introduction}

Emotion recognition has remained an active research topic for decades \cite{dmello,iemocap,strapparava2010annotating}. However, the recent proliferation of open conversational data on social media platforms, such as Facebook, Twitter, Youtube, and Reddit, has warranted serious attention~\cite{poria2019emotion,dialoguernn,huang2019ana} from researchers towards emotion recognition in conversation (ERC). ERC is also undeniably important in affective dialogue systems (as shown in \cref{fig:affective-dialogue}) where bots understand users' emotions and sentiment to generate emotionally coherent and empathetic responses. 

\begin{figure}[t]
    \centering
    \includegraphics[width=\linewidth]{affective_dialogue.pdf}
    \caption{Illustration of an affective conversation where the emotion depends on the context. Health assistant understands affective state of the user in order to generate affective and empathetic responses.}
    \label{fig:affective-dialogue}
\end{figure}

Recent works on ERC process the constituent utterances of a dialogue in sequence, with a recurrent neural network (RNN). Such a scheme is illustrated in \cref{fig:intent-modelling}~\cite{poria2019emotion}, that relies on propagating contextual and sequential information to the utterances. Hence, we feed the conversation to a bidirectional gated recurrent unit (GRU)~\cite{gru}. However, like most of the current models, we also ignore intent modelling, topic, and personality due to lack of labelling on those aspects in the benchmark datasets. In theory, RNNs like long short-term memory (LSTM)~\cite{hochreiter1997long} and GRU should propagate long-term contextual information. However, in practice it is not always the case~\cite{bradbury2016quasi}. This affects the efficacy of RNN-based models in various tasks, including ERC.

To mitigate this issue, some variants of the state-of-the-art method, DialogueRNN~\cite{dialoguernn}, employ attention mechanism that pools information from entirety or part of the conversation per target utterance. However, this pooling mechanism does not consider speaker information of the utterances and the relative position of other utterances from the target utterance. Speaker information is necessary for modelling inter-speaker dependency, which enables the model to understand how a speaker coerces emotional change in other speakers. Similarly, by extension, intra-speaker or self-dependency aids the model with the understanding of emotional inertia of individual speakers, where the speakers resist the change of their own emotion against external influence. On the other hand, consideration of relative position of target and context utterances decides how past utterances influence future utterances and vice versa. While past utterances influencing future utterances is natural, the converse may help the model fill in some relevant missing information, that is part of the speaker's background knowledge but explicitly appears in the conversation in the future.
\begin{figure}[t]
    \centering
    \includegraphics[width=\linewidth]{bayesian_dialogue.pdf}
    \caption{Interaction among different controlling variables during a dyadic conversation between persons A and B. Grey and white circles represent hidden and observed variables, respectively.  represents personality,  represents utterance,  represents interlocutor state,  represents interlocutor intent,  represents emotion and \emph{Topic} represents topic of the conversation. This can easily be extended to multi-party conversations.}
    \label{fig:intent-modelling}
\end{figure}
We leverage these two factors by modelling conversation using a directed graph. The nodes in the graph represent individual utterances. The edges between a pair of nodes/utterances represent the dependency between the speakers of those utterances, along with their relative positions in the conversation. By feeding this graph to a graph convolution network (GCN)~\cite{NIPS2016_6081}, consisting of two consecutive convolution operations, we propagate contextual information among distant utterances. We surmise that these representations hold richer context relevant to emotion than DialogueRNN. This is empirically shown in \cref{sec:results}.

The remainder of the paper is organized as follows --- \cref{sec:related-works} briefly discusses the relevant and related works on ERC; \cref{sec:method} elaborates the method; \cref{sec:experiments} lays out the experiments; \cref{sec:results} shows and interprets the experimental results; and finally, \cref{sec:conclusion} concludes the paper.

\section{Related Work}
\label{sec:related-works}
Emotion recognition in conversation is a popular research area in natural language processing~\cite{kratzwald2018decision, colneric2018emotion} because of its potential applications in a wide area of systems, including opinion mining, health-care, recommender systems, education, etc. 

However, emotion recognition in conversation has attracted attention from researchers only in the past few years due to the increase in availability of open-sourced conversational datasets~\cite{chen2018emotionlines,zhou2018emotional,poria2018meld}. A number of models has also been proposed for emotion recognition in multimodal data i.e. datasets with textual, acoustic and visual information. Some of the important works include ~\citep{poria-EtAl:2017:Long,chen2017multimodal,AAAI1817341,zadatt,hazarika2018icon,hazarika-EtAl:2018:N18-1}, where mainly deep learning-based techniques have been employed for emotion (and sentiment) recognition in conversation, in only textual and multimodal settings. The current state-of-the-art model in emotion recognition in conversation is~\cite{dialoguernn}, where authors introduced a party state and global state based recurrent model for modelling the emotional dynamics.


Graph neural networks have also been very popular recently and have been applied to semi-supervised learning, entity classification, link prediction, large scale knowledge base modelling, and a number of other problems ~\cite{kipf2016semi,schlichtkrull2018modeling,bruna2013spectral}. Early work on graph neural networks include ~\cite{scarselli2008graph}. Our graph model is closely related to the graph relational modelling work introduced in ~\cite{schlichtkrull2018modeling}.

\section{Methodology}
\label{sec:method}

One of the most prominent strategies for emotion recognition in conversations is contextual modelling. We identify two major types of context in ERC -- sequential context and speaker-level context. Following \citet{poria-EtAl:2017:Long}, we model these two types of context through the neighbouring utterances, per target utterance.

Computational modeling of context should also consider emotional dynamics of the interlocutors in a conversation. Emotional dynamics is typically subjected to two major factors in both dyadic and multiparty conversational systems --- inter-speaker dependency and self-dependency. Inter-speaker dependency refers to the emotional influence that counterparts produce in a speaker. This dependency is closely related to the fact that speakers tend to mirror their counterparts to build rapport during the course of a dialogue \cite{navarretta2016mirroring}. However, it must be taken into account, that not all participants are going to affect the speaker in identical way. Each participant 
generally affects each
other participants in unique ways. In contrast, self-dependency, or emotional inertia, deals with the aspect of emotional influence that speakers have on themselves during conversations. Participants in a conversation are likely to stick to their own emotional state due to their emotional inertia, unless the counterparts invoke a change. Thus, there is always a major interplay between the inter-speaker dependency and self-dependency with respect to the emotional dynamics 
in the conversation.


We surmise that combining these two distinct yet related contextual information schemes (sequential encoding and speaker level encoding) would create enhanced context representation leading to better understanding of emotional dynamics in conversational systems.
\subsection{Problem Definition}
\label{sec:problem-definition}
Let there be  speakers/parties  
in a conversation. 
The task is to predict the emotion labels (\textit{happy}, \textit{sad}, \textit{neutral}, \textit{angry}, \textit{excited}, \textit{frustrated}, \textit{disgust}, and \textit{fear}) of the constituent utterances , where utterance  is uttered by speaker ,
while  being the mapping between utterance and index of its corresponding speaker. 
We also represent  as the feature representation of the utterance,
obtained using the feature extraction process described below.





\subsection{Context Independent Utterance-Level Feature Extraction}
\label{sec:text-feat-extr}
A convolutional neural network~\cite{kim2014convolutional} is used to extract textual features from the transcript of the utterances. 
We use a single convolutional layer followed by max-pooling and a fully connected layer to obtain 
the feature representations for the utterances. The input to this network is the 300 dimensional pretrained 840B GloVe vectors~\cite{pennington2014glove}. We use filters of size 3, 4 and 5 with 50 feature maps in each. The convoluted features are then max-pooled with a window size of 2 followed by the ReLU activation~\cite{nair2010rectified}.
These are then concatenated and fed to a  dimensional fully connected layer, whose activations form the representation of the utterance. 
This network is trained at utterance level with the emotion labels.









\subsection{Model}
We now present our Dialogue Graph Convolutional Network (DialogueGCN\footnote{Implementation available at \url{https://github.com/SenticNet/conv-emotion}}) framework for emotion recognition in conversational setups. DialogueGCN consists of three integral components --- Sequential Context Encoder, Speaker-Level Context Encoder, and Emotion Classifier. An overall architecture of the proposed framework is illustrated in \cref{fig:model}.  

\begin{figure*}[t]
    \centering
    \includegraphics[width=\textwidth]{DialogueGCN.pdf}
    \caption{Overview of DialogueGCN, congruent to the illustration in \cref{example}.}
    \label{fig:model}
\end{figure*}

\subsubsection{Sequential Context Encoder} 
Since, conversations are sequential by nature, contextual information flows along that sequence. We feed the conversation to a bidirectional gated recurrent unit (GRU) to capture this contextual information:
  = , \text{for }i = 1, 2,\dots, N, where  and  are context-independent and sequential context-aware utterance representations, respectively.


Since, the utterances are encoded irrespective of its speaker, this initial encoding scheme is speaker agnostic, as opposed to the state of the art, DialogueRNN~\cite{dialoguernn}.

\subsubsection{Speaker-Level Context Encoder}
\label{sec:spealer-level}
We propose the Speaker-Level Context Encoder module in the form of a graphical network to capture speaker dependent contextual information in a conversation. Effectively modelling speaker level context requires capturing the inter-dependency and self-dependency among participants.
We design a directed graph from the sequentially encoded utterances to capture this interaction between the participants. Furthermore, we propose a local neighbourhood based convolutional feature transformation process to create the enriched speaker-level contextually encoded features. The framework is detailed here. 

First, we introduce the following notation: a conversation having  utterances is represented as a directed graph , with vertices/nodes , labeled edges (relations) 
 where  is the relation type of the edge between  and  
and  is the weight of the labeled edge , with , where  and . 

\paragraph{\textbf{Graph Construction:}} The graph is constructed from the utterances in the following way,

  \textbf{Vertices:} Each utterance in the conversation is represented as a vertex  in . 
  Each vertex  is initialized with the corresponding sequentially encoded feature vector , for all . 
  We denote this vector as the vertex feature. Vertex features are subject to change downstream, when the neighbourhood based transformation process is applied to encode speaker-level context.


  \textbf{Edges:} Construction of the edges  depends on the context to be modeled. 
  For instance, if we hypothesize that each utterance (vertex) is contextually dependent on all the other utterances in a conversation (when encoding speaker level information), then a fully connected graph would be constructed. That is each vertex is connected to all the other vertices (including itself) with an edge. 
  However, this results in  number of edges, which is computationally very expensive for graphs with large number of vertices. 
  A more practical solution is to construct the edges by keeping a past context window size of  and a future context window size of . 
  In this scenario, each utterance vertex  has an edge with the immediate  utterances of the past: ,  utterances of the future:  and itself: . For all our experiments in this paper, we consider a past context window size of 10 and future context window size of 10.
  
  As the graph is directed, two vertices can have edges in both directions with different relations. 


  \textbf{Edge Weights:} 
  The edge weights are set using a similarity based attention module. The attention function is computed in a way such that, for each vertex, the incoming set of edges has a sum total weight of 1. Considering a past context window size of  and a future context window size of , the weights are calculated as follows,
  
  This ensures that, vertex  which has incoming edges with vertices  (as speaker-level context) receives a total weight contribution of 1. 

  
  \textbf{Relations:} The relation  of an edge  is set depending upon two aspects:
  
      \textbf{\textit{Speaker dependency --- }} The relation depends on both the speakers of the constituting vertices:  (speaker of ) and  (speaker of ).
      
      \textbf{\textit{Temporal dependency --- }} The relation also depends upon the relative position of occurrence of  and  in the conversation: whether  is uttered before  or after.
  If there are  distinct speakers in a conversation, there can be a maximum of  (speaker of )  (speaker of )  ( occurs before  or after)  distinct relation types  in the graph .
  
  Each speaker in a conversation is uniquely affected by each other speaker, hence we hypothesize that explicit declaration of such relational edges in the graph would help in capturing the inter-dependency and self-dependency among the speakers in play, which in succession would facilitate speaker-level context encoding. 
  
  As an illustration, let two parties  participate in a dyadic conversation having 5 utterances, where  are uttered by  and  are uttered by . 
  If we consider a fully connected graph, the edges and relations will be constructed as shown in \cref{example}.
\subparagraph{Feature Transformation:}
We now describe the methodology to transform the sequentially encoded features using the graph network. 
The vertex feature vectors () are initially speaker independent and thereafter transformed into a speaker dependent feature vector using a two-step graph convolution process. 
Both of these transformations can be understood as special cases of a basic differentiable message passing method~\cite{gilmer2017neural}.

In the first step, a new feature vector  is computed for vertex  by aggregating local neighbourhood information (in this case neighbour utterances specified by the past and future context window size) using the relation specific transformation inspired from \cite{schlichtkrull2018modeling}:

where,  and  are the edge weights,  denotes the neighbouring indices of vertex  under relation . 
 is a problem specific normalization constant which either can be set in advance, such that, , or can be automatically learned in a gradient based learning setup. 
 is an activation function such as ReLU,  and  are learnable parameters of the transformation.
\begin{table}[b]
\small
\begin{center}
 	\centering
	\begin{tabular}{cccC{2.5cm}}
		\toprule
		Relation & ,  &  &  \\
		\midrule
		1 & ,  & Yes & (1,3), (1,5), (3,5) \\
		2 & ,  & No & (1,1), (3,1), (3,3) (5,1), (5,3), (5,5) \\
		3 & ,  & Yes & (2,4)\\
		4 & ,  & No & (2,2), (4,2), (4, 4)\\
		5 & ,  & Yes & (1,2), (1,4), (3,4)\\
		6 & ,  & No & (3,2), (5,2), (5,4)\\
		7 & ,  & Yes & (2,3), (2,5), (4,5)\\
		8 & ,  & No & (2,1), (4,1), (4,3)\\
		\bottomrule
	\end{tabular}
	\caption{ and  denotes the speaker of utterances  and . 2 distinct speakers in the conversation implies  distinct relation types. 
	The rightmost column denotes the indices of the vertices of the constituting edge which has the relation type indicated by the leftmost column. \label{example}}
\end{center}
\end{table}
In the second step, another local neighbourhood based transformation is applied over the output of the first step,

where,   and  are parameters of these transformation and  is the activation function.

This stack of transformations, \cref{eq-ref1,eq-ref2}, effectively accumulates normalized sum of the local neighbourhood (features of the neighbours) i.e. the neighbourhood speaker information for each utterance in the graph. 
The self connection ensures self dependent feature transformation.


\paragraph{Emotion Classifier:} The contextually encoded feature vectors  (from sequential encoder) and  (from speaker-level encoder) are concatenated and a similarity-based attention mechanism is applied to obtain the final utterance representation:

 Finally, the utterance is classified using a fully-connected network:
 

\paragraph{Training Setup:}

We use categorical cross-entropy along with L2-regularization as the measure of loss () during training:

where  is the number of samples/dialogues,  is the number of utterances in sample ,  is the probability distribution of emotion labels for utterance  of dialogue ,  is the expected class label of utterance  of dialogue ,  is the L2-regularizer weight, and  is the set of all trainable parameters. 


We used stochastic gradient descent based Adam~\cite{DBLP:journals/corr/KingmaB14} optimizer to
train our network. Hyperparameters were optimized using grid search. 





\section{Experimental Setting}
\label{sec:experiments}

\subsection{Datasets Used}
\label{sec:dataset-details}
We evaluate our DialogueGCN model on three benchmark datasets --- IEMOCAP \cite{iemocap}, AVEC \cite{Schuller:2012:ACA:2388676.2388776}, and MELD~\cite{poria2018meld}. All these three datasets are multimodal datasets containing textual, visual and acoustic information for every utterance of each conversation.
However, in this work we focus on conversational emotion recognition only from the textual information. Multimodal emotion recognition is outside the scope of this paper, and is left as future work.


\paragraph{IEMOCAP}~\cite{iemocap} dataset contains videos of two-way conversations of ten unique speakers, where only the first eight speakers from session one to four belong to the train-set. 
Each video contains a single dyadic dialogue, segmented into utterances. 
The utterances are annotated with one of six emotion labels, which are happy, sad, neutral, angry, excited, and frustrated.

\paragraph{AVEC}~\cite{Schuller:2012:ACA:2388676.2388776} dataset is a modification of
SEMAINE database~\cite{5959155} containing interactions between humans and
artificially intelligent agents. 
Each utterance of a dialogue is annotated with four real valued affective attributes: valence (), arousal (), expectancy (), and power (). 
The annotations are available every 0.2 seconds in the original database. 
However, in order to adapt the annotations to our need of utterance-level annotation, we averaged the attributes over the span of an utterance.

\paragraph{MELD}~\cite{poria2018meld} is a multimodal emotion/sentiment classification dataset which has been created by the extending the EmotionLines dataset \cite{chen2018emotionlines}.
Contrary to IEMOCAP and AVEC, MELD is a multiparty dialog dataset.
MELD contains textual, acoustic and visual information for more than 1400 dialogues and 13000 utterances from the Friends TV series.
Each utterance in every dialog is annotated as one of the seven emotion classes: anger, disgust, sadness, joy, surprise, fear or neutral.

\begin{table}[ht!]
\centering
\resizebox{\linewidth}{!}{
        \begin{tabular}{l|c|c|c|c|c|c}
            \toprule
            \multirow{2}{*}{Dataset}&\multicolumn{3}{c|}{ dialogues}&\multicolumn{3}{c}{ utterances}\\
            &train&val&test&train&val&test\\
            \hline \hline
            IEMOCAP&\multicolumn{2}{c|}{120}&31&\multicolumn{2}{c|}{5810}&1623\\
            AVEC&\multicolumn{2}{c|}{63}&32&\multicolumn{2}{c|}{4368}&1430\\
            MELD&1039&114&280& 9989& 1109& 2610\\
\bottomrule
        \end{tabular}
        }
    \caption{Training, validation and test data distribution in the datasets. No predefined train/val split is provided in IEMOCAP and AVEC, hence we use 10\% of the training dialogues as validation split.}
        \label{table:data1}
\end{table}






\subsection{Baselines and State of the Art}
\label{sec:baselines}

For a comprehensive evaluation of DialogueGCN, we compare our model with the following baseline methods:

\paragraph{CNN~\cite{kim2014convolutional}}
This is the baseline convolutional neural network based model which is identical to our utterance level feature extractor network (\cref{sec:text-feat-extr}). This model is context independent as it doesn't use information from contextual utterances.


\paragraph{Memnet~\cite{Sukhbaatar:2015:EMN:2969442.2969512}}
This is an end-to-end memory network baseline \cite{hazarika-EtAl:2018:N18-1}. Every utterance is fed to the network and the memories, which correspond to the previous utterances, is continuously updated in a multi-hop fashion. Finally the output from the memory network is used for emotion classification.

\paragraph{c-LSTM~\cite{poria-EtAl:2017:Long}}

Context-aware utterance representations are generated by capturing the contextual content from the surrounding utterances using a Bi-directional LSTM~\cite{hochreiter1997long} network. The context-aware utterance representations are then used for emotion classification. The contextual-LSTM model is speaker independent as it doesn't model any speaker level dependency.

\paragraph{c-LSTM+Att~\cite{poria-EtAl:2017:Long}}

In this variant of c-LSTM, an attention module is applied to the output of c-LSTM at each timestamp by following \cref{eq:beta,eq:beta-2}. Generally this provides better context to create a more informative final utterance representation.










\paragraph{CMN~\cite{hazarika-EtAl:2018:N18-1}}

CMN models utterance context from dialogue history
using two distinct GRUs for two speakers. 
Finally, utterance representation is obtained by feeding the current utterance as query to two distinct memory networks for both speakers. However, this model can only model conversations with two speakers. 

\paragraph{ICON~\cite{hazarika-EtAl:2018:N18-1}}
ICON which is an extension of CMN, connects outputs of individual speaker GRUs in CMN using another GRU for explicit inter-speaker modeling. This GRU is considered as a memory to track the overall conversational flow. Similar to CMN, ICON can not be extended to apply on multiparty datasets e.g., MELD.
\paragraph{DialogueRNN~\cite{dialoguernn}}

This is the state-of-the-art method for ERC. It is a recurrent network that uses two GRUs to track individual speaker states and global context during the conversation. Further, another GRU is employed to track emotional state through the conversation. DialogueRNN claims to model inter-speaker relation and it can be applied on multiparty datasets.

\section{Results and Discussions}
\label{sec:results}









\begin{table*}[t]
  \centering
  \resizebox{\linewidth}{!}{
\begin{tabular}{l||c@{~~}c|c@{~~}c|c@{~~}c|c@{~~}c|c@{~~}c|c@{~~}c|c@{~~}c}
    \hline
    \multirow{3}{*}{Methods} & \multicolumn{14}{c}{IEMOCAP}  \\
    \cline{2-15} & \multicolumn{2}{c|}{Happy} & \multicolumn{2}{c|}{Sad} &
                                        \multicolumn{2}{c|}{Neutral} & \multicolumn{2}{c|}{Angry} & \multicolumn{2}{c|}{Excited} & \multicolumn{2}{c|}{Frustrated} & \multicolumn{2}{c}{\textbf{Average(w)}}\\
    \cline{2-15} & Acc. & F1 & Acc. & F1 & Acc. & F1 & Acc. & F1 & Acc. & F1 & Acc. & F1 & Acc. & F1 \\
    \hline
    \hline
  CNN &27.77&29.86&57.14&53.83&34.33&40.14&61.17&52.44&46.15&50.09&62.99&55.75&48.92&48.18 \\
    Memnet &25.72&33.53&55.53&61.77&58.12&52.84&59.32&55.39&51.50&58.30&67.20&59.00&55.72&55.10\\
    bc-LSTM &29.17&34.43&57.14&60.87&54.17&51.81&57.06&56.73&51.17&57.95&67.19&58.92&55.21&54.95\\
    bc-LSTM+Att &30.56&35.63&56.73&62.90&57.55&53.00&59.41&59.24&52.84&58.85&65.88&59.41&56.32&56.19\\
    CMN &25.00&30.38&55.92&62.41&52.86&52.39&61.76&59.83&55.52&60.25&71.13&60.69&56.56&56.13 \\
    ICON &22.22&29.91&58.78&64.57&62.76&57.38&64.71&63.04&58.86&63.42&67.19&60.81&59.09&58.54\\
    DialogueRNN &25.69&33.18&75.10&{78.80}&58.59&{59.21}&64.71&{\bf 65.28} &80.27&{\bf 71.86}&61.15&58.91&63.40&{62.75}\\
    \hline
    \textbf{DialogueGCN} &40.62 &{\bf 42.75} &89.14 &{\bf 84.54} &61.92 &{\bf 63.54} &67.53 &64.19 &65.46 &63.08 &64.18 &{\bf 66.99} &65.25 &{\bf 64.18} \\
    \hline
   \end{tabular}
  }
  \caption{Comparison with the baseline methods on IEMOCAP dataset; Acc. = Accuracy; bold font denotes the best performances. Average(w) = Weighted average.\label{results:IEMOCAP}}
\end{table*}






\subsection{Comparison with State of the Art and Baseline}
We compare the performance of our proposed DialogueGCN framework with the state-of-the-art DialogueRNN and baseline methods in \cref{results:IEMOCAP,results:AVEC}. We report all results with average of 5 runs. Our DialogueGCN model outperforms the SOTA and all the baseline models, on all the datasets, while also being statistically significant under the paired t-test (p \textless 0.05).

\paragraph{IEMOCAP and AVEC:} On the IEMOCAP dataset, DialogueGCN achieves new state-of-the-art average F1-score of 64.18\% and accuracy of 65.25\%, which is around 2\% better than DialogueRNN, and at least 5\% better than all the other baseline models. 
Similarly, on AVEC dataset, DialogueGCN outperforms the state-of-the-art on all the four emotion dimensions: valence, arousal, expectancy, and power.

To explain this gap in performance, it is important to understand the nature of these models. DialogueGCN and DialogueRNN both try to model speaker-level context (albeit differently), whereas, none of the other models encode speaker-level context (they only encode sequential context). This is a key limitation in the baseline models, as speaker-level context is indeed very important in conversational emotion recognition. 

As for the difference of performance between DialogueRNN and DialogueGCN, we believe that this is due to the different nature of speaker-level context encoding. DialogueRNN employs a gated recurrent unit (GRU) network to model individual speaker states. Both IEMOCAP and AVEC dataset has many conversations with over 70 utterances (the average conversation length is 50 utterances in IEMOCAP and 72 in AVEC). As recurrent encoders have long-term information propagation issues, speaker-level encoding can be problematic for long sequences like those found in these two datasets. In contrast, DialogueGCN tries to overcome this issue by using neighbourhood based convolution to model speaker-level context. 
\begin{figure*}[t]
    \centering
    \begin{subfigure}[t]{0.47\textwidth}
        \centering
        \includegraphics[width=\linewidth]{dgcn_inters.pdf}
        \caption{}
        \label{fig:dgcn_inters}
    \end{subfigure}
    ~
    \begin{subfigure}[t]{0.47\textwidth}
        \centering
        \includegraphics[width=\linewidth]{dgcn_short.pdf}
        \caption{}
        \label{fig:dgcn_short}
    \end{subfigure}
        \caption{Visualization of edge-weights in \cref{eq:edge_weights} --- (a) Target utterance attends to other speaker's utterance for correct context; (b) Short utterance attends to appropriate contextual utterances to be classified correctly. }
\end{figure*}
\paragraph{MELD:} The MELD dataset consists of multiparty conversations and we found that emotion recognition in MELD is considerably harder to model than IEMOCAP and AVEC - which only consists of dyadic conversations. Utterances in MELD are much shorter and rarely contain emotion specific expressions, which means emotion modelling is highly context dependent. Moreover, the average conversation length is 10 utterances, with many conversations having more than 5 participants, which means majority of the participants only utter a small number of utterances per conversation. This makes inter-dependency and self-dependency modeling difficult.
Because of these reasons, we found that the difference in results between the baseline models and DialogueGCN is not as contrasting as it is in the case of IEMOCAP and AVEC. Memnet, CMN, and ICON are not suitable for this dataset as they exclusively work in dyadic conversations. Our DialogueGCN model achieves new state-of-the-art F1 score of 58.10\% outperforming DialogueRNN by more than 1\%. We surmise that this improvement is a result of the speaker dependent relation modelling of the edges in our graph network which inherently improves the context understanding over DialogueRNN.
\begin{table*}[t]
\centering
   \begin{tabular}{l||c@{~}c@{~}c@{~}c@{~}|c}
    \hline
    \multirow{3}{*}{Methods} & \multicolumn{4}{c|}{AVEC} & \multirow{2}{*}{MELD}\\
    \cline{2-5} &\multicolumn{1}{c}{Valence}& \multicolumn{1}{c}{Arousal}& \multicolumn{1}{c}{Expectancy} & \multicolumn{1}{c|}{Power}\\
\hline
    \hline
  CNN &0.545&0.542&0.605&8.71 & 55.02 \\
    Memnet &0.202 &0.211&0.216&8.97 & -\\
    bc-LSTM &0.194&0.212&0.201&8.90 & 56.44\\
    bc-LSTM+Att &0.189&0.213&0.190&8.67 & 56.70\\
    CMN &0.192&0.213&0.195&8.74 & -\\
    ICON & 0.180&0.190&0.180&8.45 & -\\
    DialogueRNN &0.168&0.165& 0.175&7.90 & 57.03\\
    \hline
    \textbf{DialogueGCN} &{\bf 0.157}&{\bf 0.161}&{\bf 0.168}&{\bf 7.68} & {\bf 58.10}\\
    \hline
   \end{tabular}
\caption{Comparison with the baseline methods on AVEC and MELD dataset; MAE and F1 metrics are user for AVEC and MELD, respectively.\label{results:AVEC}}
  \end{table*}
\subsection{Effect of Context Window}
We report results for DialogueGCN model in \cref{results:IEMOCAP,results:AVEC} with a past and future context window size of (10, 10) to construct the edges. We also carried out experiments with decreasing context window sizes of (8, 8), (4, 4), (0, 0) and found that performance steadily decreased with F1 scores of 62.48\%, 59.41\% and 55.80\% on IEMOCAP. DialogueGCN with context window size of (0, 0) is equivalent to a model with only sequential encoder (as it only has self edges), and performance is expectedly much worse. 
We couldn't perform extensive experiments with larger windows because of computational constraints, but we expect performance to improve with larger context sizes.
\subsection{Ablation Study}
We perform ablation study for different level of context encoders, namely sequential encoder and speaker-level encoder, in \cref{table:abl1}. We remove them one at a time and found that the speaker-level encoder is slightly more important in overall performance. This is due to speaker-level encoder mitigating long distance dependency issue of sequential encoder and DialogueRNN. Removing both of them results in a very poor F1 score of 36.7 \%, which demonstrates the importance of contextual modelling in conversational emotion recognition.
\begin{table}[ht!]
\centering
	\begin{tabular}{C{2cm}C{2.5cm}c}
		\toprule
		Sequential Encoder & Speaker-Level Encoder & F1  \\
		\midrule
		\cmark & \cmark &  64.18 \\
		\cmark & \xmark &  55.30 \\
		\xmark & \cmark &  56.71 \\
		\xmark & \xmark &  36.75 \\
\bottomrule
	\end{tabular}
	\caption{Ablation results w.r.t the contextual encoder modules on IEMOCAP dataset.\label{table:abl1}}
\end{table}

\begin{table}[ht!]
\centering
	\begin{tabular}{C{2.5cm}C{2.5cm}c}
		\toprule
		Speaker Dependency Edges & Temporal Dependency Edges & F1  \\
		\midrule
		\cmark & \cmark &  64.18 \\
		\cmark & \xmark &  62.52 \\
		\xmark & \cmark &  61.03 \\
		\xmark & \xmark &  60.11 \\
\bottomrule
	\end{tabular}
	\caption{Ablation results w.r.t the edge relations in speaker-level encoder module on IEMOCAP dataset. \label{table:abl2}}
\end{table}

Further, we study the effect of edge relation modelling. As mentioned in \cref{sec:spealer-level}, there are total  distinct edge relations 
for a conversation with  distinct speakers. First we removed only the temporal dependency (resulting in  distinct edge relations), and then only the speaker dependency (resulting in  distinct edge relations) and then both (resulting in a single edge relation all throughout the graph). The results of these tests in \cref{table:abl2} show that having these different relational edges is indeed very important for modelling emotional dynamics. These results support our hypothesis that each speaker in a conversation is uniquely affected by the others, and hence, modelling interlocutors-dependency is rudimentary. \cref{fig:dgcn_inters} illustrates one such instance where target utterance attends to other speaker's utterance for context. This phenomenon is commonly observable for DialogueGCN, as compared to DialogueRNN.

\subsection{Performance on Short Utterances}
Emotion of short utterances, like ``okay'', ``yeah'', depends on the context it appears in. For example, without context ``okay'' is assumed `neutral'. However, in \cref{fig:dgcn_short}, DialogueGCN correctly classifies ``okay'' as `frustration', which is apparent from the context. We observed that, overall, DialogueGCN correctly classifies short utterances, where DialogueRNN fails.



\subsection{Error Analysis}
We analyzed our predicted emotion labels and found that misclassifications are often among similar emotion classes. In the confusion matrix, we observed that our model misclassifies several samples of `frustrated' as `angry' and `neutral'. This is due to subtle difference between frustration and anger. Further, we also observed similar misclassification of `excited' samples as `happy' and `neutral'. All the datasets that we use in our experiment are multimodal. A few utterances e.g., `ok. yes' carrying \emph{non-neutral} emotions were misclassified as we do not utilize audio and visual modality in our experiments. In such utterances, we found audio and visual (in this particular example, high pitched audio and frowning expression) modality providing key information to detect underlying emotions (\emph{frustrated} in the above utterance) which DialogueGCN failed to understand by just looking at the textual context.


\section{Conclusion}
\label{sec:conclusion}
In this work, we present Dialogue Graph Convolutional Network (DialogueGCN), that models inter and self-party dependency to improve context understanding for utterance-level emotion detection in conversations. On three benchmark ERC datasets, DialogueGCN outperforms the strong baselines and existing state of the art, by a significant margin. Future works will focus on incorporating multimodal information into DialogueGCN, speaker-level emotion shift detection, and conceptual grounding of conversational emotion reasoning. We also plan to use DialogueGCN in dialogue systems to generate affective responses.
\bibliography{emnlp-ijcnlp-2019}
\bibliographystyle{acl_natbib}
\appendix
\end{document}