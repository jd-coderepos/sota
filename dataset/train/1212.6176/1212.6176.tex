\documentclass[12pt]{article}
\usepackage{amsmath, amssymb, graphics}
\usepackage{floatflt}
\usepackage{algorithm}
\usepackage{algorithmic}
\usepackage{wrapfig}
\usepackage{xcolor}
\usepackage[dvips]{epsfig}
\usepackage{amsfonts}
\usepackage{moreverb}
\usepackage{fullpage}
\usepackage{graphicx}
\usepackage{amsmath}



\setlength{\textheight}{8.7in}
\setlength{\textwidth}{6.3in}
\setlength{\evensidemargin}{-0.1in}

\setlength{\oddsidemargin}{-0.1in}
\setlength{\headheight}{10pt}
\setlength{\headsep}{10pt}
\setlength{\topsep}{0.1in}
\setlength{\topmargin}{0.0in}
\setlength{\itemsep}{0in}
\renewcommand{\baselinestretch}{1.2}
\parskip=0.08in



\newtheorem{theorem}{Theorem}[section]
\newtheorem{lemma}[theorem]{Lemma}
\newtheorem{observation}[theorem]{Observation}
\newtheorem{corollary}[theorem]{Corollary}
\newtheorem{proposition}[theorem]{Proposition}
\newtheorem{claim}{Claim}[section]
\newtheorem{fact}[theorem]{Fact}
\newtheorem{problem}{Problem}


\def\deg{\mbox{\tt deg}}
\def\Port{\mbox{\tt Port}}
\def\Cost{\mbox{\tt Cost}}
\def\origpath{{\cal P}}
\def\shortcutpath{{\cal P}^*}
\def\TC{\mbox{\sf Tree\_Cons}}
\def\FP{\mbox{\sf Find\_Paths}}
\def\ME{\mbox{\sf Merge}}
\def\MAIN{\mbox{\sf Main}}
\def\DistributedCapacity{\mbox{DistributedCapacity}}
\def\inline#1:{\par\vskip 7pt\noindent{\bf #1:}\hskip 10pt}
\def\Proof{\par\noindent{\bf Proof:~}}
\def\blackslug{\hbox{\hskip 1pt \vrule width 4pt height 8pt
    depth 1.5pt \hskip 1pt}}
\def\QED{\quad\blackslug\lower 8.5pt\null\par}
\def\inQED{\quad\quad\blackslug}


\newcommand{\R}{\mathbb{R}}
\def\eps{\epsilon}
\newcommand{\BFS}[0]{BFS}
\newcommand{\LEVEL}[0]{LAYER}
\newcommand{\MAXDEG}[0]{\Delta}
\newcommand{\DIFF}[0]{\mbox{Diff}}
\newcommand{\EllipsoidRunTime}[0]{T_{ellips}}
\newcommand{\PS}[0]{ 
}
\newcommand{\Terminals}[0]{\mathcal{S}}
\newcommand{\PP}[0]{ }
\newcommand{\DegThreeConst}[0]{4}
\newcommand{\DegAnyConst}[0]{\MAXDEG+1}
\newcommand{\MLP}[0]{\mbox{\tt Minimum Labeled Path}}
\newcommand{\MBP}[0]{\mbox{\tt Maximal Breach Path}}
\newcommand{\MLST}[0]{\mbox{\tt Minimum Labeled Spanning Tree}}
\newcommand{\MLC}[0]{\mbox{\tt Minimum Labeled Cut}}
\newcommand{\LPM}[0]{\mbox{\tt Labelled Prefect Matching}}
\newcommand{\SuperIndex}[0]{\widetilde{I}}
\newcommand{\TD}[0]{\mathcal{T}}
\newcommand{\Steiner}[0]{\mbox{\tt Steiner Tree}}
\newcommand{\NodeSteiner}[0]{\mbox{\tt Node} \mbox{\tt Weighted}
\mbox{\tt Steiner} \mbox{\tt Tree}}
\newcommand{\PSDeg}[0]{\mbox{\tt Min Degree Steiner Tree}}
\newcommand{\RBSC}[0]{  }
\newcommand{\SC}[0]{\mbox{\tt Set Cover}}
\newcommand{\VC}[0]{\mbox{\tt Vertex Cover}}
\newcommand{\MINSAT}[0]{3-MIN-SAT}
\newcommand{\PMINSAT}[0]{Planar-3-MIN-SAT}
\newcommand{\SelectionVec}[0]{\bold{S}}

\newcommand{\Suff}[0]{\mathrm{Suff}}
\newcommand{\First}[0]{\mathrm{First}}
\newcommand{\Last}[0]{\mathrm{Last}}
\newcommand{\dist}{\mbox{\rm dist}}
\newcommand{\PPBTWAlg}[0]{ComputePrivatePath}


\def\cG{{\cal G}}
\def\Cost{\mbox{\tt Cost}}
\def\DegCost{\mbox{\tt DegCost}}
\def\HighDeg{\MAXDEG^{*}}
\def\OptDegCost{\DegCost^{*}}






\title{Secluded Connectivity Problems}


\author{
Shiri Chechik
\thanks{
\hbox{Microsoft Research Silicon Valley Center, USA. Email:}
{\tt schechik@microsoft.com}}
\and
M. P. Johnson
\thanks{
\hbox{Department of Electrical Engineering, UCLA, Los Angeles, USA. Email:}
{\tt mpjohnson@gmail.com}}
\and
Merav Parter
\thanks{The Weizmann Institute of Science, Rehovot, Israel.
Email: {\tt \{merav.parter,david.peleg\}@ weizmann.ac.il}.}
\thanks{Recipient of the Google Europe Fellowship in distributed computing;
 research supported in part by this Google Fellowship.}
\and
David Peleg 
\thanks{Supported in part by the Israel Science Foundation
(grant 894/09), the United States-Israel Binational Science Foundation
(grant 2008348), the Israel Ministry of Science and Technology
(infrastructures grant), and the Citi Foundation.}
} 
\begin{document}

\maketitle

\begin{abstract}
Consider a setting where possibly sensitive information sent over a path in a network is visible to every {neighbor} of the path, i.e., every neighbor of some node on the path, thus including the nodes {\em on} the path itself. The {\em exposure} of a path  can be measured as the number of nodes adjacent to it, denoted by . A path is said to be {\em secluded} if its exposure is small. A similar measure can be applied to other connected subgraphs, such as Steiner trees connecting a given set of terminals. Such subgraphs may be relevant due to considerations of privacy, security or revenue maximization.
This paper considers problems related to minimum exposure connectivity structures such as paths and Steiner trees. It is shown that on unweighted undirected -node graphs, the problem of finding the minimum exposure path connecting a given pair of vertices is strongly inapproximable, i.e., hard to approximate within a factor of  for any  (under an appropriate complexity assumption), but is approximable with ratio , where  is the maximum degree in the graph. One of our main results concerns the class of bounded-degree graphs, which is shown to exhibit the following interesting dichotomy. On the one hand, the minimum exposure path problem is NP-hard on \emph{node-weighted} or \emph{directed} bounded-degree graphs (even when the maximum degree is 4). On the other hand, we present a polynomial algorithm (based on a nontrivial dynamic program) for the problem on unweighted undirected bounded-degree graphs. Likewise, the problem is shown to be polynomial also for the class of  (weighted or unweighted) bounded-treewidth graphs.
Turning to the more general problem of finding a minimum exposure Steiner tree connecting a given set of  terminals, the picture becomes more involved. In undirected unweighted graphs with unbounded degree, we present an approximation algorithm with ratio . On unweighted undirected bounded-degree graphs, the problem is still polynomial when the number of terminals is fixed, but if the number of terminals is arbitrary, then the problem becomes NP-hard again.
\end{abstract}


\section{Introduction}
\vskip .1cm \noindent \textbf{The problem.}
Consider a setting where possibly sensitive information sent over a path in a network is visible to every {neighbor} of the path, i.e., every neighbor of some node on the path, thus including the nodes {\em on} the path itself. The {\em exposure} of a path  can be measured as the size (possibly node-weighted) of its \emph{neighborhood} in this sense, denoted by . A path is said to be {\em secluded} if its exposure is small. A similar measure can be applied to other connected subgraphs, such as Steiner trees connecting a given set of terminals.
Our interest is in finding connectivity structures with exposure as low as possible. This may be motivated by the fact that in real-life applications, a connectivity structure operates normally as part of the entire network   (and is not ``extracted'' from it), and so controlling the effect of its operation on the other nodes in the network may be of interest, in situations in which any ``activation'' of a node (by taking it as part of the structure) leads to an activation of its neighbors as well. In such settings, to minimize the set of total active nodes, we aim toward finding secluded or sufficiently private connectivity structures. Such subgraphs may be important in contexts where privacy is an important concern, or in settings where security measures must be installed on any node from which the information is visible, making it desirable to minimize their number. Another context  where minimizing exposure may be desirable is when the information transferred among the participants has commercial value and overexposure to ``free viewers'' implies revenue loss.

This paper considers the problem of minimizing the exposure of subgraphs that satisfy some desired connectivity requirements.  Two fundamental connectivity problems are considered, namely, single-path connectivity and Steiner trees, formulated as the \PP\ and \PS\ problems, respectively, as follows. Given a graph  and an  pair (respectively, a terminal set ), it is required to find an  path (respectively, a Steiner tree) of minimum exposure.

\vskip .1cm \noindent \textbf{Related Work.}
The problems considered in this paper are variations of the classical
shortest path and Steiner tree problems. In the standard versions of
these problems, a cost measure is associated with edges or vertices, e.g., representing length or weight and the task is to identify a minimum cost subgraph satisfying the relevant connectivity requirement. Essentially, the cost of the solution subgraph is a \emph{linear} sum of the solution's \emph{constituent parts}, i.e., the sum of the weights of the edges or vertices chosen.

In contrast, in the setting of \emph{labeled connectivity} problems, edges (and occasionally vertices) are associated with \emph{labels} (or \emph{colors}) and the objective is to identify a subgraph  that satisfies the connectivity requirements while minimizing the number of used labels. In other words, costs are now assigned to labels rather than to single edges. Such labeling schemes incorporate grouping constraints, based on partitioning the set of available edges into classes, each of which can be purchased in its entirety or not at all. These grouping constraints are motivated by applications from telecommunication networks, electrical networks, and multi-modal transportation networks. Labeled connectivity problems have been studies extensively from complexity-theoretic and algorithmic points of view \cite{DS99,YuanVJ05,HassinMS07,FellowsGK10}. The optimization problems in this category include, among others, the  problem \cite{HassinMS07,YuanVJ05}, the  problem \cite{KrumkeW98,HassinMS07}, the  problem \cite{ZhangCTZ11}, and the  problem \cite{Monnot05}.

In both the traditional setting and the labeled connectivity setting, only edges or nodes that are explicitly part of the selected output structure are ``paid for'' in solution cost. That is, the cost of a candidate structure is a pure function of its components, ignoring the possible effects of ``passive'' participants, such as nodes that are ``very close'' to the structure in the input graph .  In contrast, in the setting considered in this paper, the cost of a connectivity structure  is a function not only of its components but also of their immediate surroundings, namely, the manner in which  is embedded in  plays a role as well. (Alternatively, we can say that the cost is a not necessarily a linear function of its components.)

To the best of our knowledge, secluded connectivity problems have not been considered before in the literature. The \PP\ and \PS\  problems are related to several existing combinatorial optimization problems. These include the \RBSC\ problem \cite{CarrDKM00,Peleg07}, the   problem \cite{HassinMS07,YuanVJ05} and the  \cite{Kar72} and \NodeSteiner\ problems \cite{KleinR95}. A prototypical example is the \RBSC\ problem, in which we are given a set  of red elements, a set  of blue elements and a family  of subsets of blue and red elements, and the objective is to find a subfamily  covering all blue elements that minimizes the number of red elements covered. This problem is known to be strongly inapproximable.

Finally, turning to geometric settings, similarly motivated problems have been studied in the networking and sensor networks communities, where sensors are often modeled as unit disks. For example, the \MBP\ problem
\cite{MeguerdichianKPS01} is defined in the context of traversing a region
of the plane that contains sensor nodes at predetermined points,
and its objective is to maximize the minimum distance between the points
on the path and the the sensor nodes.
(The solution uses edges of the sensor nodes' Voronoi diagram.)
A dual problem studied extensively is {\em barrier coverage},
i.e., the (deterministic or stochastic) placement of sensors
in order to make it difficult for an adversary to cross the
region unseen (see \cite{LiuDWS08} and the references therein).
\cite{ChenKL10} studies -barrier coverage, in which the task is to
position the sensors so that any path across the region will intersect
with at least  sensor disks.
Similarly motivated problems have been studied in the context of path planning
in AI. ``Stealth'' path planning problems, in which the task is to find a
minimum ``visibility'' path from source to destination, have been considered in
\cite{Johansson:2010:KPM:1948395.1948440,MarzouqiJ06,MarzouqiJ11}.
Although the motivation is similar, such problems are technically
quite different from the graph-based problems studied here;
those problems are typically posed in the geometric plane, amid obstacles
that cause occlusion, and visibility is defined in terms of line-of-sight.
\vskip .1cm \noindent \textbf{Contributions.}
In this paper, we introduce the concept of \emph{secluded connectivity} and study some of its complexity and algorithmic aspects. We first show that the \PP\ (and hence also \PS) problem is strongly inapproximable on unweighted undirected graphs with unbounded degree (more specifically, is hard to approximate with ratio , where  is the number of nodes in the graph , assuming ).
Conversely, we devise a  approximation algorithm for the \PP\ problem and a  approximation algorithm for the \PS\ problem, where  is the maximum degree in the graph and  is the number of terminals.

One of our key results concerns bounded-degree graphs and reveals an interesting dichotomy. On the one hand, we show that \PP\ is NP-hard on the class of \emph{node-weighted} or \emph{directed} bounded-degree graphs, even if the maximum degree is 4. In contrast, we show that on the class of unweighted undirected bounded-degree graphs, the \PP\ problem admits an \emph{exact} polynomial-time algorithm, which is based on a complex dynamic programming and requires some nontrivial analysis. Likewise, the \PS\ problem with fixed size terminal set is in P as well.

Finally, we consider some specific graph classes. We show that the \PP\ and \PS\ problems are polynomial for
bounded-treewidth graphs. We also show that the \PP\ (resp., \PS) problem can be approximated with ratio  (resp., ) in polynomial time for hereditary graph classes of bounded density. As an example,
the \PP\ problem has a 6 approximation on planar graphs.
(A more careful direct analysis of the planar case yields ratio 3.)

\vskip .1cm \noindent \textbf{Preliminaries.}
Consider a node-weighted graph ,
for some weight function ,
with  nodes and maximum degree .
For a node , let  be the set
of 's neighbors and let  be 's
{\em closed neighborhood}, i.e.,
including  itself.
A {\em path} is a sequence , oriented from left
to right, also termed a  path.
Let  for .
Let  and .
For a path  and nodes  on it, let  be the  subpath of 
from  to . For a connected subgraph  and for
, let  be the distance between 
and  in . Let  be the nodes that are strictly neighbors of  nodes and
 be the set of nodes in the 1-neighborhood of .
Define the cost of  as

Note that if  is unweighted, then the cost of a subgraph  is simply
the cardinality of the set of  nodes and their neighbors,
.

We sometimes consider the neighbors of node  in different subgraphs. To avoid confusion, we denote  the neighbors of  restricted to graph .

For a subgraph , let  denote the sum of the degrees of the nodes of . If  is a path, then this key parameter  is closely related to our problem. It is not hard to see that for any given path , . The problem of finding an  path  with minimum  is polynomial, making it a convenient starting-point for various heuristics for the problem.

In this paper we consider two main connectivity problems. In the \PP\ problem we are given an unweighted graph , a source node  and target node , and the objective is to find an  path  with minimum neighborhood size.
A generalization of this problem is the \PS\ problem, in which instead of two terminals  and  we are given a set of  terminal nodes  and it is required to find a tree  in  covering , of minimum neighborhood size.
If the given graph  is weighted, then the \emph{weighted} \PP\ and \PS\ problems require minimizing the neighborhood cost as given in Eq. (\ref{eq:wcost}).
We now define these tasks formally. For an  pair, let  be the set of all  paths and let  be the minimum cost among these paths. Then the objective of the \PP\ problem is to find a  that attains this minimum, i.e., such that .
For the \PS\ problem, let T be the set of all trees in  covering , and let  be the minimum cost among these trees, i.e.,

Then the solution for the
problem is a tree  such that .

\vskip .1cm \noindent \textbf{Notation}
For a graph  and a set  of  terminals, let  be some Steiner tree (spanning all terminals). Consider a decomposition of the tree into  subpaths  defined as follows.
Let  be the set of terminals and nodes with more than three neighbors in  (thus branching points in ). For , let  be some maximal path in  such that , i.e.,  is a maximal subpath in  whose internal section is \emph{free} of  nodes. Since any internal node in  is of degree at most 2,  is indeed a path in . Let  be the collection of these maximal subpaths. Note that .


\section{Unweighted Undirected Graphs with Unbounded Degree}

\vskip .1cm \noindent \textbf{Hardness of approximation.}
\begin{theorem}
\label{thm:hardness}
On unweighted undirected graphs with unbounded degree, the \PP\ problem (and hence also the \PS\ problem) is strongly inapproximable. Specifically, unless , the \PP\ problem cannot be approximated to within a factor  for any .
\end{theorem}
We show this for the simple case where . The theorem is shown by a gap-preserving reduction from the \RBSC\ (RBSC) problem \cite{Peleg07}. An instance of \RBSC\ consists of a set  of blue elements, a set  of red elements, and a collection  of subsets of . The task is to choose a family of sets covering all blue elements but a minimum number of red elements. In \cite{CarrDKM00} it is shown that the \RBSC\ problem cannot be approximated to within an  ratio unless , where .
Given a \RBSC\ instance, we construct a \PP\ instance  as follows.
For each , let  be the sets containing .
For every , add to  a node  corresponding to .
Define  and
.
Add edges so that the sets  and  form a complete bipartite graph for every .
In addition, we connect a node  to all nodes in , and connect a node  to all nodes in .
So far, our construction contains  nodes (see Fig. \ref{fig:hardness}).

We now turn to describing the representation of the red elements . For every , let  be a ``supernode'' consisting of  nodes (with no edges among them), and let  be the set of all supernodes. All  nodes of  are connected to node  if , for every , , and .  Finally, let  be a ``hypernode'' consisting of  nodes (with no edges among them), all of which are connected to all the constituent nodes of all supernodes.
This completes the construction of . Overall, .

By ``visiting'' a supernode or the hypernode, we mean visiting any of their constituent nodes.
We now make an immediate observation which allows us to restrict ourselves to  paths  not visiting any supernode or the hypernode, i.e., with .
\begin{claim}
\label{cl:approx_1_n}
No -approximate secluded path solution will visit any supernodes or the hypernode, i.e., will satisfy .
\end{claim}
\Proof
For every path  that goes through a node  it holds that . On the other hand, a path  that visits only  nodes, i.e.,  costs only . The claim follows.
\QED

Let  be the indices of the supernodes in the neighborhood of path . Then

We first show the correctness of the reduction and then consider gap-preservation.
\begin{claim}
\label{cl:rbcs_pp}
There exists an  path of cost  iff there exists an \RBSC\ solution of cost .
\end{claim}
\Proof
() By Claim \ref{cl:approx_1_n}, we restrict ourselves to paths not visiting any supernodes. Thus, by the structure of the graph, any such  path visits one of the nodes  for each , corresponding to the selection of  to cover . Since all  ordinary nodes will be in ,  of the cost is due to the size- supernodes, which correspond to  red elements covered in the \RBSC\ solution.

() Conversely, given a cost- \RBSC\ solution, we construct the path as follows. Starting from , for every , add to  a node  such that  is in the \RBSC\ solution, and then finally add . The cost due to the ordinary nodes is  and due to the  supernodes is .
\QED

We now show that the reduction is gap-preserving.
By Claim \ref{cl:rbcs_pp} it follows that , where  is the optimal \RBSC\ cost (the number of red elements  covered by the optimal solution).
Assume that there exists an  approximation algorithm for the \PS\ problem. This would result in path a  such that

Thus , implying that
 is a  approximation to \RBSC. Since , and as \RBSC\ is inapproximable within a factor of  for every fixed , we get that \PP\ is inapproximable within a factor of , where  for some appropriate function .
This complete the proof of Thm. \ref{thm:hardness}
\QED
\begin{corollary}
\label{cor:dag}
The \PP\ problem (and hence also the \PS\ problem) is strongly inapproximable in directed acyclic graphs.
\end{corollary}
\Proof
The proof is by gap-preserving reduction from \RBSC, very similar to the general case. The  key modifications from the general case are that the hypernode of  nodes is not included and we add directions to the edges; directing the edges from  toward the nodes of , namely, , from the nodes of  to the nodes of  for every , and from the nodes of  to the node . We also direct all the supernodes edges towards the supernode elements. It can be verified that this directed graph is acyclic. The rest of the analysis is as in the undirected case. The graph  contains  nodes, thus the \PP\ problem cannot be approximated within an  ratio. The corollary follows.
\QED
\begin{figure}[h!]
\begin{center}
\includegraphics[scale=0.3]{reduc_fig.pdf}
\caption{ \label{fig:hardness}
\sf
Example of the gap-preserving reduction from \RBSC\ to \PP.
The \RBSC\ instance is given by .  The optimal solution for \RBSC\ is , which covers  red elements. The corresponding  path is of cost . Note that an alternative solution of  is of higher cost, covering 4 red elements, and the corresponding  path is of cost .
}
\end{center}
\end{figure}

\subsection{Approximation}
\begin{theorem}
\label{thm:delta_approx}
The \PP\ problem in unweighted undirected graphs can be approximated within a ratio of .
\end{theorem}
\Proof
Given an instance of the \PP\ problem, let  be an  path that minimizes . Note that we may assume without loss of generality that for every node  in , the only neighbors of  in  from among the nodes of  are the nodes adjacent to  in . To see this, note that otherwise, if  had an edges to some neighbor  such that  is not on , we could have shortened the path  (by replacing the subpath from  to  with the edge ) and obtained a shorter path with at most the same cost as .

Recall that  denotes the sum of the node degrees of the path , and that  for any . Recall also that the problem of finding an  path  with minimum  is polynomial. We claim that the algorithm that returns the path  minimizing  yields a  approximation ratio for the \PP\ problem.

In order to prove this, we show there exists a path  such that .
This implies that , as required.

The path  is constructed by the following iterative process.
Initially, all nodes are unmarked and we set .
While there exists a node with more than  unmarked neighbors on the path , pick such a node .
Let  be the first (closest to ) neighbor of  in  and let  be the last (closest to ) neighbor of  in .
Replace the subpath  in  with the path  and mark the node .
We now show that .
Let  be the set of marked nodes in , where  is the node marked in iteration .
Note that  is the number of iterations in the entire process.

For the sake of analysis, partition  into two sets: , those that have no unmarked neighbor in , and , those that do.
We claim that .

For each , let  be the path that was replaced by the process of constructing  in iteration , and let  be the path  in the beginning of that iteration.
Since  has more than  unmarked neighbors in , we get that .
Let  be the path obtained by removing the first two nodes and last two nodes from .
Note that .
In addition, the sets  for  are pairwise disjoint, and moreover, none of the nodes in  has a neighbor in .
We thus get that .
Note that  and that .
We get that .
\QED

In addition, for the \PS\ problem with  terminals, we establish the following.
\begin{theorem}
\label{thm:steiner_delta_approx}
The \PS\ problem in unweighted undirected graphs can be approximated within a ratio of

\end{theorem}
Recall that  is the sum of degrees in the tree  and
let . Clearly, . \\The  problem is defined as follows. Given a graph  and a set of terminals , the objective is to find a Steiner tree  of minimal , i.e., such that .
\begin{claim}
\label{cl:degcost_approx}
The  problem is -hard for arbitrary  and can be approximated within a ratio of . If  is constant, then  is polynomial.
\end{claim}
\Proof
We first present an  approximation algorithm.
In the \NodeSteiner\ problem, nonnegative costs are assigned to nodes as well as to edges. The cost of subgraph  that contains the terminals  is the sum of the costs of its nodes and edges. While constant factor approximations exists for the standard  problem (of edge-weighted graphs), the \NodeSteiner\ problem cannot be approximated to within less than a logarithmic factor  assuming that , by a reduction from  \cite{Feige98,KleinR95}.
Using the -approximation algorithm of  \cite{Feige98}, the  problem can be solved as follows. Set the weight of each vertex to be its degree, that is , for every , and set zero weights to the edges.
The optimal \NodeSteiner\ tree with these weights corresponds to the optimal  tree. Since \NodeSteiner\ is polynomial for fixed , so is  \cite{KleinR95}.

We now show that this approximation is tight.
For brevity, let  denote the special case of the \NodeSteiner\ problem in which edges have zero weights and  denote the  task.
The approximation-preserving reduction from  presented in \cite{KleinR95} assigns zero weights to the edges, and hence  is inapproximable within logarithmic factor.  By reducing from  to  we will show that  is inapproximable within logarithmic factor as well. Given a  instance graph  and terminal set , where , we transform it into an instance  of . We assume that the weights are polynomial. To obtain , the weights  are scaled (multiplied by a common large enough factor ) to weights  such that  for every node . Next, define the residual of vertex  as . The scaling step guarantees that .
Next, add to each vertex  a disjoint set  consisting of -weight  neighbors. Formally, , where  and . Finally,   for every  and  for every . It is easy to see that optimal solution of  in  corresponds to an optimal solution for  in . Since the weights  are polynomial (hence so is ) it follows that a  approximation for   in  would imply an  approximation for   in , in contradiction to the logarithmic inapproximability of  due to \cite{KleinR95}.
\QED
We now show the following.
\begin{lemma}
\label{cl:degcost}
.
\end{lemma}
\Proof
Let  be an optimal secluded Steiner tree, meaning . Starting with  we will perform a set of operations on it, ending with a spanning  tree  for which . Since  this would establish that . We begin by describing the procedure for obtaining  from  and then analyze its cost, i.e.,  . Note that as we do not know ; this procedure is only theoretical to aid the analysis and is not meant for implementation.

Let . Throughout, we assume that . Else, any -approximation algorithm for the \Steiner\ problem is a -approximation algorithm to the \PS\ problem.
Initially, let all nodes in  be unmarked and set .
A node  is \emph{high-degree with respect to}  if it has at least  unmarked neighbors in . While there is a high-degree node  with respect to , do the following. For each subpath  such that , i.e.,  has at least two neighbors in , replace the following subpaths. Let  (respectively, ) be the neighbor of  closest to  (respectively, ) on . Replace the subpath  with the subpath  and mark the node .

We now show that . Let  be the set of marked nodes in . This implies that the procedure consists of  iterations. Let  and  be the spanning  subgraph at the end of iteration  for  (thus ). The node  is the high-degree node with respect to  that was observed at iteration . Let  be the unmarked nodes in the final subgraph . Note that  might not be a tree (due to cycle introduced by the marked nodes), but clearly, a subgraph of it is a  tree spanning all terminals. By definition, the  of  is given by

where , for , is given by . Let  be the set of marked nodes at the end of iteration . Given a path , let  denote its internal subpath. For every marked node  and a subpath  for which  has at least  unmarked neighbors in  at iteration , let  be the subpath in  that was shortcut in  by introducing ; otherwise let .
Let  be the unmarked neighbors of  at iteration  and define
, the set of ``internal'' unmarked neighbors of  in  (the nodes that are removed from  due to path replacements).

We now make two observations. (A)  for every node , and (B)  are pairwise disjoint. We start with (A). Note that for every  on which  has unmarked neighbors, we ignore at most  nodes in  (corresponding to the endpoints of the replaced path, ). Since there are at most  subpaths in , there are at most  unmarked neighbors of  in  that are ignored in . As  has  unmarked neighbors in , we get that
, which establishes (A). Note that (B) holds since each set  is part of the tree in iteration  but is not part of the tree in iteration , i.e.  (due to the subpath replacement step).
We now refer to Eq. (\ref{eq:degcost_t}) and bound  and . First, observe that . This holds since
every node  corresponds to a disjoint set  of at least  nodes of the optimal tree . Thus,  and . Since the degree of each marked node is bounded by , it follows that

We now bound the degree of the unmarked nodes  of . Let .
Since , it follows that  and hence .
By the stopping criteria, it follows that any node in  has at most  neighbors in .
Hence


Combining Eqs. (\ref{eq:degcost_t}), (\ref{eq:degcost_M}), and (\ref{eq:degcost_Mhat}), it follows that .
Finally, we show that there exists a graph  and a set of terminals  for which . Set , graph   consists of  nodes. Let . The skeleton for graph  is a path . Then, for every , connect  to the terminal . In addition, connect each of the  nodes to the set of nodes . The graph  is presented in Fig. \ref{fig:degcost_gap}. Clearly, the optimal tree  under both measures, i.e.,  and , is given by the subgraph induced by . Since every  node is counted  times in  instead of one time in , it follows that
 and . Since , we get that . The lemma follows.
\QED
\begin{claim}
\label{cl:steiner_sqrtn}
The \PS\ problem on unweighted undirected graphs can be approximated within .
\end{claim}
\Proof
Let  be a guess for , the solution of the optimal secluded Steiner tree. A Steiner tree  is constructed for each for each guess  as follows. Remove from  all vertices with degree  and let  be a -approximation for the optimal Steiner tree (with minimum number of edges). Let  for , i.e., the 2-approximation Steiner tree (where the objective is to minimize the number of edges) computed when considering the correct guess. We show that .  Let  be the cost of the optimal Steiner tree (i.e., number of edges in the tree). Then clearly .
In addition, . This follows, since our approximated Steiner tree  contains at most twice the edges of the optimal one, and the degree of each vertex is bounded by .
Finally, since the cost of any secluded Steiner tree is bounded by the number of nodes , we get that .
\QED
We are now ready to complete the proof for Thm. \ref{thm:delta_approx}.
\Proof
First note that any -approximation algorithm to the \Steiner\ problem is an -approximation algorithm to \PS. Next, note that obtaining an  approximation is trivial, since  for any spanning tree  and .
The approximation of  is thanks to Claim \ref{cl:steiner_sqrtn} and that of  is due to Claims \ref{cl:degcost} and \ref{cl:degcost_approx}. Specifically, let  be an  approximation algorithm for the  problem, and let  be the resulting spanning  tree output by algorithm  given graph  and  as input. Due to Claim \ref{cl:degcost_approx}, such an  exists, and hence . Combining this with \ref{cl:degcost}, we get that , as required. The theorem follows.
\QED

\begin{figure}[h!]
\begin{center}
\includegraphics[scale=0.3]{fig_degcost.pdf}
\caption{ \label{fig:degcost_gap}
\sf
Illustration of the gap between  and .  The light purple nodes correspond to the set of terminals .
}
\end{center}
\end{figure}

\section{Bounded-Degree Graphs}
In this section we show that the ~\PP\ problem (and thus also the \PS\ problem) is -hard on both directed graphs and weighted graphs, even if the maximum node degree is . We have the following.
\begin{lemma}
\label{lem:privatepath_vc}
The \PP\ problem is NP-complete even for graphs of maximum degree 
if they are either (a) node-weighted, or (b) directed.
\end{lemma}
We start with (a).
We present a reduction from  of the weighted \PS\ problem.
Throughout we consider the special case of the weighted \PS\ problem, where , namely, the \PP\ problem.
In \cite{VCDeg3} it was shown that the  problem is NP-complete even for planar graphs with maximum degree . Given a  instance in the form of a bounded-degree graph , the instance  to the weighted \PP\ is constructed as follows. We add a ``heavy'' weight neighbor  to every node . In addition, every edge  is replaced by a gadget  consisting of four nodes,  forming a diamond graph. Let  and . The set of nodes of  is given by . The weight function  is defined as follows:  if , and  otherwise.

We now describe the gadget construction. Each edge  is replaced by a diamond graph  consisting of the edges  used to connect  and  in . (See Fig \ref{fig:vc}(a).) In addition, the new node  (respectively, )
is connected to the node  (respectively, ) for some arbitrary edges , for .
Finally, fix some ordering  on the edges of  that starts with  and ends with . Now, for every  connect  with , creating a ``tour'' from  to  via the gadgets. This completes the description of the instance .
For a path  and set of nodes , denote the set of neighbors of 's nodes in  by
.
We now show that  is an optimal vertex cover for  iff  is
an optimal  secluded path in the weighted graph .
We begin by making an immediate observation.
\begin{observation}
\label{obs:vc_pp}
For every optimal  secluded path , .
\end{observation}
\Proof
Assume the contrary, and let  be an optimal secluded path with at least one node . Since  has a heavy neighbor , it holds that .
Consider next an alternative path  that goes strictly through  the nodes in , i.e., follows the tour through the gadgets . Since , we get that , in contradiction to the optimality of .
\QED
Consequently, in the rest of the proof, we consider only secluded paths  that do not visit nodes of . Note that every such  path is of length  and the set of  nodes is fully contained in neighbors of this path, i.e., . It therefore holds that,

The main factor determining the quality of an efficient path, among all paths that do not visit , is the number of path neighbors in , i.e., . That is, by choosing between the -side (respectively, -side) in every gadget , for , the resulting path has  (respectively, ) in its neighborhood, i.e., . Overall, there are  such decisions to be made. Let  be an optimal vertex cover for . We show the following.
\begin{claim}
\label{cl:vc}
.
\end{claim}
\Proof
We begin by showing that  for every  path  that does not go through the nodes of .
Let . We now show that  is a legal vertex cover of , hence  and by Eq. (\ref{eq:p_notin_v}),  as required. Assume for contradiction that there exists an edge  not covered by the nodes of , i.e., both . Consider the gadget . Since  does not go through the nodes of , it passes through  and . In particular, when reaching , a decision is made between passing through the  side  or through the  side . Without loss of generality assume that  passes through , i.e., . Then, since  is a neighbor of , it is in  and we end with contradiction.

Conversely, we now show that for every legal vertex cover  of , there exists an
 secluded path  whose cost satisfies . The path  is constructed as follows. Start from  and traverse the gadgets  in order, from  to . Within each gadget , for  such that  and ,
move from  to  through  if  and through  otherwise. One of these choices must hold, since  is a legal vertex cover, so it must contain at least one of the nodes  and . Hence, the path  ``pays'' for a node  in  only if . Formally, we get that . Hence, combining with Eq. (\ref{eq:p_notin_v}), . The claim follows.
\QED
Part (a) of the lemma follows.
Finally, we now turn consider Lemma \ref{lem:privatepath_vc}(b).
It is sufficient to show a directed version for the reduction such that Obs. \ref{obs:vc_pp} still holds. Note that the main essence is to prevent any optimal path from visiting the nodes of  (and hence to visit every ). In the weighted case, this goal is achieved by adding a ``heavy'' unique neighbor to every node in  (the black nodes in Fig. \ref{fig:vc}(a)). In the directed case, this property is achieved by directing the edges  in the following manner.
Recall the ordering  on the edges , which imposes a direction for each edge in our construction as follows.
We include the arcs , , , , , , and in addition,  for every . Finally, the edges between  and  to the nodes of  are directed towards the nodes of  (see Fig. \ref{fig:vc}(b)).  This forces any  path in  to go through strictly the nodes in . Obs. \ref{obs:vc_pp} now holds for the directed case, and the rest follows as in the weighted case. The lemma follows.
\QED
\begin{figure}[h!]
\begin{center}
\includegraphics[scale=0.3]{fig_vc_w.pdf}
\hfill
\includegraphics[scale=0.3]{fig_vc_d.pdf}
\caption{ \label{fig:vc}
\sf
Illustration of the reduction from . (a) The weighted case. Top: the graph  with the diamond graph gadgets . The black nodes correspond to the heavy neighbor attached to each node . Bottom: zoom into a single gadget. (b) The directed case. Directionality enforces visiting the gadgets in order, precluding the tour in  nodes.
}
\end{center}
\end{figure}

\subsection{Polynomial-time algorithm for the unweighted undirected case.}
In contrast, for unweighted undirected case of bounded-degree graphs, we show that the \PP\ problem is polynomial.
For two subpaths , define their asymmetric difference as

\begin{observation}
\label{obs:delta}
(a)  for every .\\
(b) .
\end{observation}
\Proof
Part (a) simply follows as . Part (b) follows by definition.
\QED
We have the following.
\begin{theorem}
\label{thm:bounded_deg_poly}
The \PP\ problem is \emph{polynomial} on unweighted undirected degree-bounded graphs.
\end{theorem}
Note that in the previous section we showed that if the graph is either weighted or directed then the \PP\ problem (i.e., the special case of \PS\ with two terminals) is NP-hard. In addition, it is noteworthy that the related problem of  problem \cite{YuanVJ05, HassinMS07}  is NP-hard even for unweighted planar graph with max degree 4 (this follows from a straightforward reduction from \VC).

We begin with notation and couple of key observations in this context.
For a given path , let  be the distance in edges between  and  in the path . Recall, that  is the maximum degree in graph .
\begin{observation}
\label{obs:opt_path_common}
Let  be two nodes in some optimal  path  that share a common neighbor, i.e., . Then .
\end{observation}
\Proof
Assume for contradiction that there exists an optimal  path  such that
 where  and ,  for .
Recall that due to the optimality of , for every node  in , the only neighbors of  in  from among the nodes of  are the nodes adjacent to  in  (otherwise the path can be shortcut). Let  be the mutual neighbor of  and  and consider the alternative  path  obtained from  by replacing the subpath  by the subpath . Let  be an length- internal subpath of  where . Then since the  degree of  is at most , it follows that

where  is an upper bound on the number  neighbors other than  and .
In addition, note that by the optimality of  it contains no shortcut and hence

(i.e., for node  the only neighbors on the path  are  and ).
Thus,

where the last inequality follows from Eq. (\ref{eq:path_bounded_new}), contradicting the optimality of . The observation follows.
\QED
In other words, the observation says that two vertices on the optimal path at distance  (which is constant for bounded-degree graphs) or more have no common neighbors. This key observation is at the heart of our dynamic program, as it enables the necessary subproblem independence property. The difficulty is that this observation applies only to optimal paths, and so a delicate analysis is needed to justify why the dynamic program works.
Note that the main difficulty of computing the optimal secluded path  is that the cost function  is not a linear function of path's  components as in the related  measure. Instead, the residual cost of the th vertex in the path \emph{depends} on the neighborhood of the length- prefix of . This dependency implies that the secluded path computation cannot be simply decomposed into independent subtasks.
However, in contrast to suboptimal  paths, the dependency (due to mutual neighbors) between the components of an optimal path is \emph{limited} by the maximum degree  of the graph. The \emph{limited dependency} exhibited by any  optimal path facilitates the correctness of the dynamic programming approach. Essentially, in the dynamic program, entries that correspond to subsolution  of \emph{any} optimal  path enjoy the limited dependency, and hence the values computed for these entries correspond to the \emph{exact} cost of an optimal path that starts with  and ends with . In contrast, entries of subpaths  that do not participate in any optimal  path, correspond to an \emph{upper bound} on the cost of some path  that starts with  and ends with . This is due to the fact that the value of the entry is computed under the limited dependency assumption, and thus does not take into account the possible double counting of mutual neighbors between \emph{distant} vertices in the path. Therefore, the possibly ``falsified'' entries cannot compete with the exact values, which are guaranteed to be computed for the entries that hold the subpaths of the optimal path. This informal intuition is formalized below.

For a path  of length , let  be the -suffix of .
\begin{lemma}
\label{lem:opt_path_common}
Let  be an optimal  path of length . Let  be some partition of  into two subpaths such that . Then .
\end{lemma}
\Proof
For ease of notation, let , , and . By definition, . In the same manner, .  Assume for contradiction that the lemma does, namely, . Then by Obs. \ref{obs:delta}(a) we have that . This implies that . Let . There are two cases to consider: (a) , and (b)  has a neighbor  in . We handle case (a) by further dividing it into two subcases: (a1) , i.e.,  occurs at least twice in , once in  and once in , and (a2)  has a neighbor  in . Note that in both subcases, there exists a shortcut of , obtained in subcase (a1) by cutting the subpath between the two duplicates of  and in subcase (a2) by shortcutting from  to .  This shortcut results in a strictly lower cost path, in contradiction to the optimality of . We proceed with case (b).
Let  be such that . If , then clearly the path can be shortcut by going from  directly to , resulting in a lower cost path, in contradiction to the optimality of . If , then  (since ) and , which in contradiction to Obs. \ref{obs:opt_path_common}. The Lemma follows.
\QED
\begin{corollary}
\label{cor:opt_path_common}
Let  be an optimal  path of length . Let  be some partition of  into two subpaths such that . Then .
\end{corollary}
For clarity of representation, we describe a polynomial algorithm for the \PP\ problem, i.e., where  and in addition , in which case .
The general case of  is immediate by the description for the special case of . The case of \PS\ with a fixed number of terminals  is described in Subsection \ref{subsec:multiterminals}. 
\par The algorithm we present is based on dynamic programming.
For each , and every length- subpath given by a quartet of nodes , it computes a length- path  that  starts with  and ends with  (if such exists) and an upper bound  on the cost of this path, . These values are computed inductively, using the values previously computed for other 's and . In contrast to the general framework of dynamic programming, the interpretation of the computed values  and , namely, the relation between the dynamic programming values  and  and some ``optimal'' counterparts  is more involved. In general, for arbitrary  and , the path  is not guaranteed to be optimal in any sense and neither is its corresponding value  (as ).
However, quite interestingly, there is a subset of quartets for which a useful characterization of  and  can be established. Specifically, for every , there is a subclass of quartets , for which the computed values are in fact ``optimal'', in the sense that for every , the length- path  is of minimal cost among all other length- paths that start with  and end with . We call such a path a \emph{semi-optimal} path, since it is optimal only restricted to the length and specific suffix requirements. It turns out that for the special class of quartets , every semi-optimal path of  is also a prefix of some optimal  path. This property allows one to apply Cor. \ref{cor:opt_path_common}, which constitutes the key ingredient in our technique. In particular, it allows us to establish that . This is sufficient for our purposes since the set  contains \emph{any} quartet that occurs in \emph{some} optimal secluded  path . Specifically, for every  optimal path , it holds that the quartet  satisfies . The correctness of the dynamic programming is established by the fact that the values computed for quartets that occur in optimal paths in fact correspond to optimal values as required (despite the fact the these values are ``useless'' for other quartets).

\subsubsection{The algorithm}
If the shortest path between  and  is less than , then the optimal \PP\ can be found by an exhaustive search, so we assume throughout that . Let  be the set of all length-\DegThreeConst~ subpaths  in . That is, for every , we have  and  for every .
For , define the collection of \emph{shifted successor} of  as . For every pair , where  and , the algorithm computes a value  and length- path  ending with   (i.e., ). These values are computed inductively. For , let  and

Once the algorithm has computed  for every  and every , in step  it computes

Note that

Let  such that  and  achieves the minimum value in  Eq. (\ref{eq:cost_dynamic}). Then define  and let

Let , and set  for  such that .
Note that there are at most  entries , each computed in constant time, and so the overall running time is .
\vskip .2cm \noindent \textbf{Analysis.}
Throughout this section we make use of two notions of optimality for secluded paths. The two notions are parameterized by  and by an index .  Let  be the set of length- paths  starting at  and ending with  for .
Denote the minimum cost of a path in  by 
and denote the set of paths attaining this minimum cost by .
We refer to paths in  as \emph{semi-optimal} paths, since they are optimal only among length- paths ending with .
Note that in general, a semi-optimal path  is not necessarily an optimal secluded path from  to ,
nor is it a prefix of an optimal secluded path from  to any , and thus Cor. \ref{cor:opt_path_common} cannot be applied.
However, we next argue that there exists a subset of quartets  for which every path  is  a prefix of an optimal  secluded path,
and since Obs. \ref{obs:opt_path_common} holds for any subpath of some optimal  path, we can apply it for this subset of sequences.
Let  be the set of all optimal  paths for  .
A path  that is a \emph{prefix} of some optimal  path is hereafter referred to as a \emph{pref-optimal} path.
Let .
That is,  is the set of all optimal secluded paths in which the subpath in positions  to  is equal to . Note that in contrast to the set , which is nonempty for every , and might contain suboptimal  secluded paths, the set   might be empty for many choices of , but every path  is indeed optimal.
\par Define the subset of subpaths for which  is nonempty, for , as
 We next establish some useful properties for . Let  correspond to the set of paths in  but truncated at position , that is, . Clearly, every path  in  is of length .
Moreover, since  contains optimal paths,  consists of pref-optimal paths.
It is important to understand the distinction between the set  and the set . Both sets contain length- paths that start with  and end with . However, while  (if it exists) corresponds to a prefix of some  optimal path, a path  is not necessarily a prefix of some optimal path; it is only optimal among paths of length  ending with . In particular, this difference implies that Cor. \ref{cor:opt_path_common} can be safely applied to paths in . \\
\par We first provide a general observation regarding the computed values  and .
\begin{observation}
\label{lem:start}
For every  and , if  then .
\end{observation}
\Proof
The left inequality follows immediately, since both  and  are in  and  is a semi-optimal path, i.e., it has the lowest cost among these paths. The right inequality is proved by induction on . for , the claim holds trivially. Assume it holds for every  and every , and consider a fixed  and . Let . Then

where Ineq. (\ref{eq:ineqd}) follows from Obs. \ref{obs:delta}(b), Ineq. (\ref{eq:ineqsig}) follows from the inductive assumption, Eq. (\ref{eq:ineqsig1}) follows from Eq. (\ref{eq:delta}) and Eq. (\ref{eqn:start}) follows from Eq. (\ref{eq:cost_dynamic}). The observation follows.
\QED
It turns out that for the set of special quartets that occur on some optimal  path, a stronger characterization of  and  can be established.
\begin{lemma}
\label{lem:in_optimal}
For every  and ~,\\
(a) , and \\
(b) .
\end{lemma}
Let  be some optimal secluded path and let .
Since , by Lemma \ref{lem:in_optimal} we get that the computed path  is indeed an optimal  path of cost . Our remaining goal is therefore to establish Lemma \ref{lem:in_optimal}.
\vskip .1cm \noindent \textbf{Proof Sketch.}
We prove that the inequalities of Obs. \ref{lem:start} are in fact \emph{equalities} for every  and every . The proof, by induction on , is not immediate and requires several properties to be established and to come into play together. To begin, assume that
both parts of  Lemma \ref{lem:in_optimal} hold for every  for every  and consider some . In addition to the inductive assumption, we assume that  (which would be established later on). Given these assumptions, for part (a) of the lemma we show that . Next, to show that part (b)  also holds, i.e., that , we would like to show that  is a pref-optimal path and thus Cor. \ref{cor:opt_path_common} can be applied.
Our argumentation can be sketched as follows. We prove that for  every   it holds that (a) every semi-optimal path  is pref-optimal, and (b) there is a set of quartets denoted by , for which the following holds for every : (b1)  for every , and hence  is pref-optimal, and (b2)  and thus the inductive hypothesis can be established.
\paragraph{Detailed analysis.}
We proceed by showing that every semi-optimal length- path  for ,  is a pref-optimal path.
\begin{lemma}
\label{lem:con_subopt}
For every , .
\end{lemma}
\Proof
We first show that . To do that, we consider a path  for which ,
and show that . Thus .
Let  and , thus  is an optimal  path.
Assume for contradiction that . Let , i.e., .  Consider an alternative  path . Note that  is indeed a legal  path since . We now compute its cost and compare it to that of .

where Ineq. (\ref{ineq:0}) and (\ref{ineq:1}) follow by parts (b) and (a) of Obs. \ref{obs:delta} and Ineq. (\ref{ineq:2}) is by the contradiction assumption. Eq. (\ref{ineq:3}) follows from the fact that  is an optimal  path and thus Cor. \ref{cor:opt_path_common} can be safely applied.
We therefore get that , in contradiction to the optimality of .
\par Conversely, we now prove that .
Let  and . Let  and . Consider the path  obtained by replacing the length- prefix of  with . We show that  has the same cost, , hence  is also an optimal   path. This would imply that  and , establishing the lemma.

where Ineq. (\ref{eq:con1}) follows from the fact that both  and  are in , i.e., both are length- paths starting at  and ending with  and  is semi-optimal. Ineq. (\ref{eq:con2}) follows from the fact that  is an optimal  path, thus Cor. \ref{cor:opt_path_common} can be safely applied. The fact that  is optimal necessitates equality. The lemma follows.
\QED
For  and , define 
That is,  iff there exist some  and an optimal  path  such that .
Note that by the definition of , there exists at least one pref-optimal path  of length  that ends with . In addition, since  there exists at least one  corresponding to . In other words,  for every  and .
Lemma \ref{lem:con_subopt} implies that the following holds for every .
\begin{observation}
\label{cl:parent_opt}
If  then \\
(a) for every  there exists a semi-optimal path , which is also a pref-optimal path, such that .\\
(b) .
\end{observation}
We first provide an auxiliary claim that holds for every . Let  be such that (a)  and
(b) .
\begin{claim}
\label{cl:nn}
For every , 
\end{claim}
\Proof
Let  such that . Note that since , by Obs. \ref{cl:parent_opt}(a), such a  is guaranteed to exist. Let  and let .
We now prove that  and therefore .

where Ineq. (\ref{eq:sig0}) follows from Obs. \ref{obs:delta}(a), Ineq. (\ref{eq:sig1}) follows from the fact that both  and , being semi-optimal, has the minimal cost in this set. Eq. (\ref{eq:sig2}) follows from the fact that  and since , Lemma \ref{lem:con_subopt} implies that  is a pref-optimal, i.e., a prefix of some optimal secluded path, i.e., . Therefore, Cor. \ref{cor:opt_path_common} can be applied.
\QED
We now turn to proving Lemma \ref{lem:in_optimal}, showing that for every , and every , the inequalities of Obs. \ref{lem:start} become equalities.
\inline Proof of Lemma \ref{lem:in_optimal}:
The lemma is proven by induction on . For , both claims clearly hold. Assume the claims hold for  and consider . Let . To be able to apply the inductive hypothesis to , we first show the following.
\begin{claim}
\label{cl:parent}
.
\end{claim}
\Proof
Note that since , it holds that , and also  by Obs. \ref{cl:parent_opt}(b).
Let , thus .
By Lemma \ref{lem:con_subopt},   and .
Let us pick three arbitrary representatives , , . Recall that due to Lemma \ref{lem:con_subopt},  and  are pref-optimal paths and therefore Cor. \ref{cor:opt_path_common} can be safely applied for them. We first show that 
To see this, let . Then by Claim \ref{cl:nn}, . In addition, by Lemma \ref{lem:con_subopt},  and therefore Cor. \ref{cor:opt_path_common} can be safely applied, so Eq. (\ref{eq:parent_opt}) is established. We next show that .  Clearly, , so it suffices to show that  as  is semi-optimal. Indeed,

where Ineq. (\ref{eq:ineq4}) follows from Obs. \ref{obs:delta}(a) and Ineq. (\ref{eq:ineq5}) follows from Lemma \ref{lem:start} and Eq. (\ref{eq:delta}). Ineq. (\ref{eq:ineq6}) follows from the fact that  and  attains the minimum value of Eq. (\ref{eq:cost_dynamic}). Eq. (\ref{eq:ineq7}) follows from part (b) of the inductive assumption for . In Eq. (\ref{eq:ineq8}) the first equality follows from the fact that both  and  are in , and thus , and the second equality follows from Eq.  (\ref{eq:parent_opt}).
\par We therefore get that  and since , by Lemma \ref{lem:con_subopt} it holds that , i.e.,  is also a pref-optimal path. This implies that , as required.
\QED
By Claim \ref{cl:parent} and Obs. \ref{cl:parent_opt}(b), . It follows that the inductive hypothesis can be applied to . By part (a) of the inductive assumption for , it holds that . Finally, by Claim \ref{cl:nn} we get that the path  chosen by the algorithm satisfies , which establishes (a).
We now consider (b). Note that

where Eq. (\ref{eq:eqeq1}) follows from the inductive hypothesis for  and Eq. (\ref{eq:eqeq2}) follows from the fact that  and thus Cor. \ref{cor:opt_path_common} can be applied.
This completes the proof of Lemma \ref{lem:in_optimal}.
We are now ready to complete the correctness proof of the algorithm. Let  be an optimal  path of length . Then . Therefore, by Lemma \ref{lem:in_optimal}, , and so , and by Lemma \ref{lem:start}, the cost of any other  such that , and every , satisfies .  Thm. \ref{thm:bounded_deg_poly} is established.
\begin{corollary}
\label{cor:deg_pp}
The \PP\ problem can be solved in  time.
\end{corollary}
This follows since there are  distinct length-() subpaths, and the length of the path is bounded by . Hence, overall there are  entries and the value for each entry is computed in  time.
For the \PS\ problem, we show the following.
\begin{theorem}
\label{thm:steiner_bounded_deg_poly}
On unweighted undirected degree-bounded graphs, we have the following:
(a) for arbitrary , the \PS\ problem is NP-hard; (b) for , the \PS\ problem is \emph{polynomial}.
\end{theorem}
\Proof
We start with (a).
We prove this by reduction from \VC. By \cite{VCDeg3}, we know that \VC~ is NP-complete even for planar graphs with maximum degree .
Given an instance of the \VC~ problem consists of a bounded-degree graph , we transform it into an instance of  the \PS\ problem that consists of a bounded-degree graph  and a set of terminal . Let  be a complete binary tree with  leaves .
Consider some ordering on  and connect  to , for every . In addition, for each vertex , add a single neighbor  in . Next, for each edge , add a vertex . Every edge  is replaced by the two edges  and  in . In sum, the graph  is defined by  and  
Note that the maximum degree  in  is at most . Hence  is a bounded-degree graph as well.
The terminal set  includes the  nodes and the nodes of the binary tree . We now show that  has vertex cover of size  iff there exists a secluded Steiner tree  in  of cost . Recall that  is some optimal vertex cover in  and  is the cost of the optimal secluded Steiner tree in .
We now show that

We begin by showing that .
Given a vertex cover  for , we show that there exists a Steiner tree  of cost .
The tree  is given by 
and .
Since  is a legal vertex cover in , it follows that every terminal  has at least one neighbor  (corresponding to the endpoints of  in ) that connects it to the tree . Hence,  is a valid Steiner tree. We now analyze its cost. The cost of  consists of the set  of the terminals and their neighbors, and the set . Therefore,  and
thus  as required.

It remains to show that .
Given a Steiner tree  in , we show how to obtain  a legal vertex cover  such that .
Let . First, note that  is a vertex cover. Assume for contradiction that it is not, and let  be a non-covered edge by .
Since neither  nor  in , we get that  is disconnected in , in contradiction to the fact that  is a Steiner tree. Finally, observe that
. Since ,
part (a) of the theorem is established.
We now turn to prove Thm. \ref{thm:bounded_deg_poly}(b) and show that there exists a polynomial-time algorithm if the number of terminals is bounded by a constant.
\subsection{Polynomial-time algorithm for \PS\ with .}
\label{subsec:multiterminals}
For simplicity we present the algorithm for maximum degree three (and note that the proof works for any fixed bounded degree).
Let . We note that the main difference with \PS\ problem is that here we also need to take into account the ``skeleton'' of the tree.
We iterate over possible skeletons of trees, where by skeleton of a tree we mean the following.
Consider some tree  and contract every path  such that all nodes in  but the endpoints are of degree 2 and are not terminal into a single edge. The resulting graph is the skeleton of the tree . Note that it is enough to look on trees  such that all leaves are terminals (as otherwise we can take a subtree of the tree spanning all terminals).
Note that the skeleton may contain at most  nodes that are not terminals.
Given the (at most ) nodes that are not terminals, an upper bound on the number of different skeletons is the number of different trees with at most  nodes, which is . There are  options to choose the nonterminal nodes in the skeleton.
We thus get that there are at most  different skeletons.

A subtree  is an extended skeleton of the skeleton  if it obtained by replacing each edge  in the skeleton  with a path  of the following form. Either the path  is a path of  of length at most 9 (and contains at most 8 internal nodes), or the prefix of the path  is a subpath of length 4 in  from  to some node  and the suffix of  is a subpath of length 4 in  from some node  to , the nodes  and  are connected by an imaginary edge (which will be replaced later by a path from ).

The algorithm for finding the tree with the lowest cost  is as follows.
Iterate over all possible skeletons  and for each skeleton  iterate over all possible extended skeleton .
For each  do the following. Let  be the set of nodes in .
For each imaginary edge  in , consider the graph  that is obtained by removing all nodes  from .
Find the lowest cost \PP\  from  to  in .
For each imaginary edge  in , replace the edge with the path . Let  be the tree obtained through this process.
Finally, return the tree  with the lowest cost.
Let  be the returned tree.

We call nodes in a tree  splitting nodes if their degree in  is greater than 2.
We call a node special in  if it is either a splitting node or a terminal.
Consider the optimal tree  and let  be its skeleton and  be its extended skeleton.
We claim that , this will imply the correctness of the algorithm.

As in the \PP\ problem we may restrict ourselves only to trees such that all nodes that their parent are not special do not have edges to nodes on the tree that are not in their subtree, similarly, every node that none of whose children is special has no edges to nodes in its subtree (otherwise we can find a different tree with a subset of the nodes of ). In other words, all nodes in  that are neither special nor have a neighbor in  that is special have no edges in  to other nodes in  that are not adjacent to them in .

The following observation is an extension of Obs. \ref {obs:opt_path_common} and is crucial for the algorithm.
\begin{observation}
\label{obs:tree-neighbors}
Consider an edge  in the skeleton  and their corresponding path .
The node  for  cannot have a common neighbor with any other node .
\end{observation}





\begin{lemma}

\end{lemma}
\Proof
The proof follows from Obs. \ref{obs:tree-neighbors}.
Note that in the optimal solution , .
Consider an imaginary edge  in . Let  be the subpath from  to  in .
Note that the subpath  are disjoint for all imaginary edges  in  .
Let .
Note that .
Note also that .
We get that .
\QED

\section{Secluded Connectivity for Specific Graph Families}
\vskip .1cm \noindent \textbf{Bounded-treewidth graphs.}
For a graph , let  denote the \emph{treewidth} of .
In this section, we consider graphs  of constant treewidth, i.e. . A \emph{separation} of a graph  is a triplet  where ,  and  for every  and . The set  then separates  and  in .
The concept of treewidth was introduced by Robertson and Seymour \cite{RobertsonS86} using tree-decompositions. (See \cite{TWBodlaender93, Bodlaender07} for an in-depth introduction to this topic.)
For a graph , a tree decomposition is a pair  consisting of a tree  and a collection  of vertex subsets (called \emph{bags}) with the following properties:
(T1) Every vertex  is contained in the least one bag . For every edge  there is at least one bag  containing both vertices . (T2) For every vertex , the nodes  of  with  form a subtree of .
\par To avoid confusion, the elements of  are referred to as \emph{nodes} and the elements of  are referred to as vertices. In addition, we may informally refer to the node  by its index . Let  be the set containing the vertices of node  itself, and the vertices in the bags of its descendants in . Let  (respectively, ) be the set of terminals in  (respectively, ).
Let  be some upper bound on  such that  for every . Let  be the set of terminals in .
We show the following.
\begin{theorem}
\label{thm:tw_correctness}
The \PS\ problem (and hence also the \PP~problem) can be solved in \emph{linear} time for graphs with fixed treewidth. In addition, given the tree decomposition of , the \PS\ problem is solvable in  if . This holds even for weighted and directed graphs.
\end{theorem}
If the tree decomposition is not known, then for fixed treewidth graphs it can be computed in linear time \cite{Bodlaender93}.
For simplicity, we consider the unweighted undirected case. The weighted and directed cases are considered by the end of the analysis. Given a tree  in  and a separation  in , let  (respectively, ) be the forest obtained by considering the edges of  restricted to  (respectively, ) nodes. Let 
and . Finally, let .
Since  is a separation, it follows that

Such decomposition of the cost function  holds for any connectivity construct . Indeed, the correctness of the dynamic programming applied on the tree-decomposition of  is established due to this cost independency property of Eq. (\ref{eq:treewidth_cost}).
\subsubsection{Algorithm}
The \PS\ algorithm is an extension of the algorithm of \cite{ChimaniMZ11} for computing an optimal  in a bounded-treewidth graph.
The input to the algorithm is a connected graph , a set of terminals  and a tree decomposition  of .
For ease of analysis, we assume that the tree decomposition  is \emph{nice} in the sense that every node  has at most two children  in the tree decomposition  (as in \cite{ChimaniMZ11}).
\par Initially choose any root node  that  contains at least one terminal, i.e. , and direct the tree  such that  is the root.
For each  we maintain a table . Each entry  in the table represents a configuration, implicitly corresponding to a subsolution. Such a subsolution is a forest  that when imposed on the nodes of , resulting in , spans the terminals  in , and  may be extended into a tree that spans all the terminals. Specifically, the configurations define a collection of disjoint subsets of the vertices of  and a collection of edges between vertices in . In addition, it specifies the vertices  that are neighbors of the forest  (note that for such , its neighbor in  is not necessarily in ; it might be in an ancestor of  in ) . We follow the representation scheme of \cite{ChimaniMZ11} where each subsolution is represented by a color assignment  to the vertices of , and extend it by adding an additional color  that represents the status of being a neighbor. We refer to this coloring as a \emph{configuration}. The interpretation of the colors is representing connected components in  (the forest  restricted to the vertices of ) and vertices in  that have neighbors in , i.e., vertices  with the same color  belong to the same connected component in the subsolution forest . The vertices  such that  do not participate in the forest but have at least one neighbor in . Finally, vertices  with  neither belong to the subsolution represented by  nor have a neighbor that belongs to it.
The idea behind this representation is the following. We assume that there is a Steiner tree  in . The forest  is obtained by restricting the -neighborhood of the tree  to the vertices in . This restriction of  is encoded by the configuration . In particular, for a Steiner tree , there is a unique configuration  for each bag  that is consistent with  (as formally defined later). Overall, each table  has at most  entries, for , where  is the \emph{Bell-number} \cite{ChimaniMZ11}, corresponding to the number of partitions of at most  elements into  classes.
\vskip .1cm \noindent \textbf{Legal Configuration.}
Let  denote the set of active vertices in the configuration . A configuration  is \emph{legal} if it satisfies (L1) every terminal  is assigned a color in the range , and (L2) .
\par The algorithm proceeds by computing, for each legal configuration , a value  that represents the minimum number of vertices in  that belongs to the neighborhood of a forest  that respects the configuration.
These values are computed bottom-up. Before we proceed, we first define the notion of compatible configurations.
\vskip .1cm \noindent \textbf{Compatible configurations.}
Let  be parent of  in the tree .
Then  is \emph{compatible} with , denoted as   if the following four properties hold:\\
(Q1)  and  agree on their mutual vertices . For any ,  iff . In addition, for any pair  that are in the same connected component, , it holds that  and  are connected in  as well. (Note that some components in the child  might be connected in the parent , and thus the converse does not hold.)
(Q2) If  connects two connected components of , then this is supported by the graph  edges (i.e., the the edges required for the connectivity exist). \\
(Q3) Any terminal of  is connected to some connected component in the forest represented by . Hence,  contains at most  connected components and for each terminal  there exists some  such that  and .\\
(Q4) .
\par The value  of the coloring  is computed bottom-up. Let  be a leaf. Then for every \emph{legal} , set  and  if the configuration is invalid. Consider a non-leaf node  with children  and . Let  be a triple of compatible configurations, such that  and .
Define

where, for 

Since  is a separation in the graph , the correctness of Eq. (\ref{eq:cost_tw_mid}) follows from Eq. (\ref{eq:treewidth_cost}). Finally, for each entry , maintain the minimum cost over any set of compatible triples.  Define,

The optimal solution value for the whole graph can be found in the root bag  of , identifying a cheapest solution where all vertices with color  are contained in the same connected component (i.e., have the same color).
Computing the optimal solution (which is sufficiently represented by the set of nodes in the final tree) is possible by backtracking or by storing the set of edges for each row and each bag.
\subsubsection{Analysis}
For a given configuration entry , the forest  \emph{agrees} with  if the following four properties hold:\\
(Z1)  has  connected components where   is the number of distinct colors in the coloring . \\
(Z2) All vertices  of the same color,  are in the same connected component in .\\
(Z3) The  neighbors of the forest are active, .\\
(Z4) The terminals  are connected in  in the following manner.
Any terminal  is in \emph{non-singleton} component in . Specifically, it is connected to some node . In addition, the terminals of , are in some connected component in .
\par For a legal configuration  define the family of all forests in  that agree with  as

Let

and  be the minimum value of any forest .
\begin{lemma}
\label{lem:cor}
.
\end{lemma}
\Proof
This can be shown by a straightforward inductive proof on the decomposition tree. The base cases are leaf nodes where the hypothesis clearly holds. Consider  and assume the induction assumption holds for all descendants of bag .
Let  be the forest that attains the minimum value in the family , i.e., .
Let  be the children of  in  and let  be the two compatible entries such that , hence  for .
For a forest  and , define


Since  is a separation in  and by Property (Q4) we have the following.

\begin{claim}
\label{cl:tw1}
 for every , where .
\end{claim}

In addition, since  is a separation in  by Eq. (\ref{eq:treewidth_cost}), it follows that

For , let  be the forest that attains the minimum value of  in their family, i.e.,
.
By induction assumption, .
By Eq. (\ref{eq:cost_tw_mid_2}) and Claim \ref{cl:tw1}, we get that

where Ineq. (\ref{ineq:tw1}) follows from Eq. (\ref{eq:cost_tw_mid_2}),
Ineq. (\ref{ineq:tw2}) follows from induction assumption and
Ineq. (\ref{ineq:tw3}) follows from Claim \ref{cl:tw1}.
Thus since the triple  consists of compatible configurations, by Eq.  (\ref{eq:cost_tw_mid}), Eq. (\ref{eq:cost_tw_mid_2}), Eq. (\ref{eq:alg_cost}) and Eq. (\ref{eqn:tw_val12})
we get that

Assume for contradiction that . Then, by
Eq. (\ref{eq:alg}) and Eq. (\ref{eq:tw_val_f_star}), we get that
there exists  such that

Without loss of generality let .
Combining Claim \ref{cl:tw1} and Ineq. (\ref{ineq:delta}), we get that , and since , we end with contradiction to the induction assumption.
We therefore get that  and . The lemma follows.
\QED
We now complete the proof of Thm. \ref{thm:tw_correctness}
\Proof
Let  be set of configurations in the root  of , in which all vertices are in a single connected component. Note that there exists a configuration 
such that the optimal secluded tree .  Let   be the configuration that attains the minimum value  in the set. It then follows, by Lemma \ref{lem:cor}, .
The Steiner tree can be computed by a brute-force backtracking procedure.
\par We now turn to running time.
The size of any table  is bounded by .
During the bottom-up traversal of  we consider all possible row combinations for at most 3 tables. The merge operation can be done in  time, and hence requires overall . All other operations for entry processing  are linear in the size of the bag, .
Note that computing the values of subsolution  are linear in the treewidth. Thus our time complexity is of the same as order as that of \cite{ChimaniMZ11}. In particular, it is linear for graphs with fixed treewidth, in this case the tree decomposition can also be computed in linear time and hence the total complexity is linear. If  then given the tree decomposition  for , the running time is .
\QED

Note that the algorithm extends also to directed and node-weighted graphs.
We now outline the modifications needed for these extensions.
The extension to weighted graph is immediate. The only modification needed
is to
define ,  and 
according to the weighted cost as in Eq. (\ref{eq:wcost})
instead of the unweighted cost,
and apply a similar analysis.


The extension to directed graphs is a bit more involved.
Note that in the directed case it is not enough to store configurations that represent the different connected components since in order to construct a valid tree we need to make sure that every node has only one parent.
Therefore it is important to ``keep track'' of the roots of these connected components.
For each connected component  we store as part of the configuration  the root  of this connected component.
Namely, we add to each configuration a root for each color.
In order to merge two connected components  and  to single tree with root , we only consider edges
from  to . However, note that the root of a component does not necessarily have to be from the current bag and hence the number of different configurations might be too large. To overcome this issue we use the following observation.  Consider a bag  and a configuration .
Note that it cannot be that two different connected components in  would have both roots that are not from . To see this, consider a subtree whose root  is not in , since  is a separator then  does not have edges to nodes in  forcing  to be the root of the final tree. Therefore, there could be at most one such component. We thus restrict ourselves to configurations where there is at most one connected component with a root not in  (otherwise the configuration is invalid). Notice that the number of valid configurations might increase by at most a factor of  (for all options of the root not in ). Hence, the running time increases by a factor of .
The rest of the analysis is similar and thus is omitted from this version.
\vskip .1cm \noindent \textbf{Bounded Density Graphs.}
Let .
We show the following.
\begin{proposition}
\label{prop:bounded_density}
Let  be a hereditary class of graphs with a linear number of edges, i.e., a set of graphs such that for each ,  for some constant , and where  implies that  for each subgraph  of .
Then .
\end{proposition}
\Proof
Note that for every Steiner tree  in , it holds that

The left inequality is immediate. We now prove the right inequality. Let  be the neighborhood of  and let  be the subgraph of  induced by  and  be its edge set. We have that

where the first equality follows by definition, the second inequality follows from the hereditariness of  and the third equality follows by the definition of .
Hence, Eq. \ref{eq:bounded_den} holds. In particular, for the optimal secluded tree , we get that  The proposition holds.
\QED
An example of such a class of graphs is the family of planar graphs. For this family, the above proposition yields a 6-approximation for the \PP\ problem. (A more careful direct analysis for planar graphs yields a 3-approximation, however) Finally, note that whereas computing  for a constant number of terminals  is polynomial, for arbitrary , computing  is NP-hard but can be approximated to within a ratio of  thanks to Claim \ref{cl:degcost_approx}. Thus, for the class of bounded density graphs, by Prop. \ref{prop:bounded_density}, the \PP\ problem has a constant ratio approximation and the \PS\ problem has an  ratio approximation.
\begin{theorem}
For the class of bounded-density graphs, the \PP\ (respectively, \PS) problem
is approximated within a ratio of  (resp., ).
\end{theorem}







\bigskip\bigskip\bigskip
{\small


\begin{thebibliography}{10}

\bibitem{Bodlaender93}
H.L. Bodlaender.
\newblock A linear time algorithm for finding tree-decompositions of small
  treewidth.
\newblock In {\em STOC}, 226--234, 1993.

\bibitem{TWBodlaender93}
H.L. Bodlaender.
\newblock A tourist guide through treewidth.
\newblock {\em Acta Cybern.}, 11:1--22, 1993.

\bibitem{Bodlaender07}
H.L. Bodlaender.
\newblock Treewidth: Structure and algorithms.
\newblock In {\em SIROCCO}, 11--25, 2007.

\bibitem{CarrDKM00}
R.D. Carr, S. Doddi, G. Konjevod, and M.V. Marathe.
\newblock On the red-blue set cover problem.
\newblock In {\em SODA}, 345--353, 2000.

\bibitem{ChenKL10}
A. Chen, S. Kumar, and T.-H. Lai.
\newblock Local barrier coverage in wireless sensor networks.
\newblock {\em IEEE Tr. Mob. Comput.}, 9:491--504, 2010.

\bibitem{ChimaniMZ11}
M. Chimani, P. Mutzel, and B. Zey.
\newblock Improved steiner tree algorithms for bounded treewidth.
\newblock In {\em IWOCA}, 374--386, 2011.

\bibitem{DS99}
I. Dinur and S. Safra.
\newblock On the hardness of approximating label-cover.
\newblock {\em IPL}, 89:247--254, 2004.

\bibitem{Feige98}
U. Feige.
\newblock A threshold of ln {\it n} for approximating set cover.
\newblock {\em J. ACM}, 45:634--652, 1998.

\bibitem{FellowsGK10}
M.R. Fellows, J. Guo, and I.A. Kanj.
\newblock The parameterized complexity of some minimum label problems.
\newblock {\em JCSS}, 76:727--740, 2010.

\bibitem{VCDeg3}
M.R. Garey and D.S. Johnson.
\newblock {The Rectilinear Steiner Tree Problem is NP-Complete}.
\newblock {\em SIAM J. Appl. Math.}, 32:826--834, 1977.

\bibitem{HassinMS07}
R. Hassin, J. Monnot, and D. Segev.
\newblock Approximation algorithms and hardness results for labeled
  connectivity problems.
\newblock {\em J. Comb. Optim.}, 14:437--453, 2007.

\bibitem{Johansson:2010:KPM:1948395.1948440}
A. Johansson and P. Dell'Acqua.
\newblock Knowledge-based probability maps for covert pathfinding.
\newblock In {\em MIG}, 339--350, 2010.

\bibitem{Kar72}
R.M. Karp.
\newblock Reducibility among combinatorial problems.
\newblock In R.E. Miller and J.W. Thatcher, Eds, {\em Complexity of
  Computer Computations},  85--103. Plenum Press, NY, 1972.

\bibitem{KleinR95}
P.N. Klein and R. Ravi.
\newblock A nearly best-possible approximation algorithm for node-weighted
  steiner trees.
\newblock {\em J. Algo.}, 19:104--115, 1995.

\bibitem{KrumkeW98}
S.O. Krumke and H.-C. Wirth.
\newblock On the minimum label spanning tree problem.
\newblock {\em IPL}, 66:81--85, 1998.

\bibitem{LiuDWS08}
B. Liu, O. Dousse, J. Wang, and A. Saipulla.
\newblock Strong barrier coverage of wireless sensor networks.
\newblock In {\em MobiHoc}, 411--420, 2008.

\bibitem{MarzouqiJ06}
M. Marzouqi and R. Jarvis.
\newblock New visibility-based path-planning approach for covert robotic
  navigation.
\newblock {\em Robotica}, 24:759--773, 2006.

\bibitem{MarzouqiJ11}
M. Marzouqi and R. Jarvis.
\newblock Robotic covert path planning: A survey.
\newblock In {\em RAM}, 77--82, 2011.

\bibitem{MeguerdichianKPS01}
S. Meguerdichian, F. Koushanfar, M. Potkonjak, and M.B. Srivastava.
\newblock Coverage problems in wireless ad-hoc sensor networks.
\newblock In {\em INFOCOM}, 1380--1387, 2001.

\bibitem{Monnot05}
J. Monnot.
\newblock The labeled perfect matching in bipartite graphs.
\newblock {\em IPL}, 96:81--88, 2005.

\bibitem{Peleg07}
D. Peleg.
\newblock Approximation algorithms for the label-cover and
  red-blue set cover problems.
\newblock {\em J. Discrete Algo.}, 5:55--64, 2007.

\bibitem{RobertsonS86}
N. Robertson and P.D. Seymour.
\newblock Graph minors. ii. algorithmic aspects of tree-width.
\newblock {\em J. Algo.}, 7:309--322, 1986.

\bibitem{YuanVJ05}
S. Yuan, S. Varma, and J.P. Jue.
\newblock Minimum-color path problems for reliability in mesh networks.
\newblock In {\em INFOCOM}, 2658--2669, 2005.

\bibitem{ZhangCTZ11}
P. Zhang, J.Y. Cai, L. Tang, and W. Zhao.
\newblock Approximation and hardness results for label cut and related
  problems.
\newblock {\em J. Comb. Optim.}, 21:192--208, 2011.

\end{thebibliography}

}

\end{document}
