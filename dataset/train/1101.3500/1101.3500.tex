\documentclass[12pt,draftcls,onecolumn]{IEEEtran}


\IEEEoverridecommandlockouts                              \overrideIEEEmargins
\usepackage{tipa}
\usepackage{pifont}
\usepackage{cases}
\usepackage{txfonts}
\usepackage{graphicx}
\usepackage{array}
\usepackage{mathrsfs}
\usepackage{amsfonts}
\usepackage{setspace}
\usepackage{subfigure}


\textwidth=17cm \textheight=23.5cm \topmargin=-1cm
\evensidemargin=0.3cm \oddsidemargin=-0.3cm \arraycolsep=1pt

\usepackage{setspace}
\doublespacing


\newtheorem{Theorem}{Theorem}
\newtheorem{Lemma}{Lemma}
\newtheorem{Proposition}{Proposition}
\newtheorem{Remark}{Remark}
\newtheorem{Definition}{Definition}
\newtheorem{Example}{Example}













\title{\LARGE \bf
Computation for Supremal Simulation-Based Controllable and Strong
Observable Subautomata }


\author{Yajuan Sun, Hai Lin, Fuchun Liu\thanks{Yajuan Sun, Hai Lin are with Electrical and Computer Engineering Dept., National
University Of Singapore, 117576, Singapore (email: \{sunyajuan, elelh\}@nus.edu.sg)}\thanks{Fuchun Liu is with Faculty of Computer, Guangdong University of Technology, Guangzhou 510006, China (email: liufch8@gmail.com)}}


\begin{document}

\maketitle
\thispagestyle{empty}




\begin{abstract}
Bisimulation relation has been successfully applied to computer
science and control theory. In our previous work, simulation-based
controllability and simulation-based observability are proposed,
under which the existence of bisimilarity supervisor is
guaranteed. However, a given specification automaton may not
satisfy these conditions, and a natural question is how to compute
a maximum permissive sub-specification. This paper aims to answer
this question and investigate the computation of the supremal
simulation-based controllable and strong observable subautomata
with respect to given specifications by the lattice theory. In
order to achieve the supremal solution, three monotone operators,
namely simulation operator, controllable operator and strong
observable operator, are proposed upon the established complete
lattice. Then, inequalities based on these operators are
formulated, whose solution is the simulation-based controllable
and strong observable set. In particular, a sufficient condition
is presented to guarantee the existence of the supremal
simulation-based controllable and strong observable subautomata.
Furthermore, an algorithm is proposed to compute such subautomata.
\end{abstract}





\section{INTRODUCTION}
Bisimulation relation was introduced in \cite{milner} as a
behavioral equivalence relationship between two dynamical systems,
and since then it has been used widely in the study of discrete
event systems (DESs) \cite{f}, linear systems \cite{pa},
probabilistic systems \cite{pro}, and hybrid systems \cite{hyb}.
Bisimulation provides a stronger equivalence than the extensively
studied language equivalence \cite{cc}. It is known that the
language generated by two bisimilar systems are equivalent, but
the systems possessing the same language might not be bisimilar.
Moreover, two bisimilar systems have equivalent reachability
properties, or more generally, preserve properties specified in
terms of temporal logic such as CTL* \cite{tl}. Therefore, the
bisimilarity control that aims to achieve a bisimulation
equivalence between controlled system and specification has
attracted lots of attentions these years.


Komenda and Schuppen characterized the language controllability
and observability in terms of partial bisimulation by using
coalgebra for supervisory control of DESs under partial
observation \cite{k1}. Tabuada investigated the controller
synthesis problem of affine systems for bisimulation equivalence
\cite{TabA} and extended it to various systems including
discrete-event systems, nonlinear control systems, behavioral
systems, and hybrid systems by means of category theory
\cite{TabC}. In Zhou's work \cite{cbis3} and our previous work
\cite{liu}, the problem addressed is to design a supervisor to
execute the control action to achieve the bisimulation relation
between supervised system and specification, where plant and
specification are generally described as nondeterministic
automata. In Zhou's work \cite{cbis3}, a small model theorem is
established to show that the supervisor exists if and only if it
exists over the power set of Cartesian product of system and
specification state spaces.

In our previous work \cite{liu}, a different framework is proposed
to characterize the existence of the supervisor. The supervisor
exists if and only if the specification is simulation-based
controllable under full observation. As for the partial
observation case, the specification should be both
simulation-based controllable and simulation-based observable to
ensure the existence of the supervisor. However, in most
situations, a given specification does not satisfy those
conditions. Then, a natural question is how to compute a maximum
permissive sub-specification. Here, we would like to calculate the
supremal simulation-based controllable and strong observable
subautomata. Please note that the existing work for the
calculation of supremal controllable/normal sublanguages are all
based on the language controllability/normality \cite{ke},
\cite{kf}. To our best knowledge, there is no work considering the
computation of the supremal subautomata under simulation-based
controllability and simulation-based observability, where the
specifications are given as automata instead of languages.


This paper aims to answer this question and investigate the
computation of the supremal simulation-based controllable and
strong observable subautomata with respect to given specifications
by the lattice theory. Some preliminary results on the computation
of the supremal simulation-based controllable subautomata under
full observations were presented in \cite{sun}. In this paper, we
will calculate the supremal simulation-based controllable and
strong observable subautomata for the partial observation case. In
order to achieve the supremal solution, three monotone operators,
namely simulation operator, controllable operator and strong
observable operator, are proposed upon the established complete
lattice. Then, inequalities based on these three operators are
formulated, whose solution is the simulation-based controllable
and strong observable set. In particular, a sufficient condition
is presented to guarantee the existence of the supremal
simulation-based controllable and strong observable subautomata.
Furthermore, an algorithm is proposed to compute such subautomata.


This note is organized as follows. Section 2 gives the
preliminary. Section 3 reviews the works that have been done under
full observation. Section 4 studies the computation of the
supremal simulation-based controllable and strong observable
subautomata under partial observation. An illustrative example is
provided in Section 5. The note concludes with section 6.


\section{Preliminary}
\subsection{Discrete Event System}
A DES is modeled as an automaton , where  is the set of states,  is a finite set
of events,  is the
transition function,  is the initial state, 
is the set of marked states.  is
the active function and  is the active event set at
state . Let  be the set of all finite strings over
, including the empty string . Then the
transition function  can be extended to  in the nature way \cite{cc}. The
language generated by  is defined as  is defined. The event set can be
partition into  = ,
where  is the set of uncontrollable events and
 is the controllable event set. It can be also
partitioned into  = ,
where  is the set of unobservable events and
 is the set of observable events. Given an event
string ,  is the length of the string and
 is the  event of this string, where . When a string of events occurs, the sequence of observable
events is filtered by a projection : , which is defined inductively as follows:
, for  and ,  if ,
otherwise, . The accessible operator  is used
to remove the states which are not accessible from the initial
state, and it is defined as below.






\begin{Definition}
Given an automaton , the
accessible operator on G is defined as:

where , where
 \}, , :
 is a transition function,
and for any  and , .
\end{Definition}


Further, the concept of subautomaton is introduced and a
subautomaton operator is proposed to construct a subautomaton from
a given state set.


\begin{Definition}
Given an automaton , the
subautomaton of  is defined as , where , ,
and  = .
\end{Definition}
The notation 
means that we are restricting  to the smaller domain of
the states . The subautomaton of  picks its states and
marked states from the corresponding sets in .

\begin{Definition}
Given an automaton , the
subautomata operator is defined as:

where Z ,  = \{\},  = , and .
\end{Definition}
By this subautomata operator, we can construct a subautomata of
the original automata  from a set , whose elements are the
state pairs of  and . In addition, the state set  of
this subautomata is a subset of the corresponding state set  of
 and the transition function of this subautomata restricts
 to a smaller domain of the states .


Then, simulation relation is used to describe the equivalence
between automata as follows.
\begin{Definition}
Let  and  be two automata. 
is said to be simulated by , denoted by , if there is a binary relation    such that  and for each
,

(1) , where 
such that .

(2) , then .

\end{Definition}
If , , and
 is symmetric,  is a bisimulation relation between
 and , denoted by . We
sometimes omit the subscript  from  or
 when it is clear from the context. Moreover, the
main result of \cite{liu} is as below.

\begin{Theorem}\label{tliu}
Given a plant , a specification
 and a projection , assume
that  is language controllable and language observable.
Then, there exists a simulation relation 
and a -supervisor  such that 
and  is -consistent if and only if
 is simulation-based controllable and simulation-based
observable.
\end{Theorem}

The simulation-based controllability and simulation-based
observability are defined as below.





\begin{Definition}
Given a plant  and a
specification ,  is
simulation-based controllable with respect to  and
 if it satisfies:

(1) (Simulation Condition) There is a simulation relation 
such that .

(2) (Controllable Condition) ( )( 
 )( )[.
\end{Definition}



The set  is said to be a
simulation-based controllable set if  is a
simulation relation from  to  and 
satisfies the controllable condition.

\begin{Definition}
Given a plant  and a
specification .  is said to
be simulation-based observable with respect to , 
and , if it satisfies:

(1) (Simulation Condition) There is a simulation relation 
such that .

(2) (Observable Condition) ,  with     and ].
\end{Definition}

Simulation-based controllability and simulation-based
observability implies language controllability and language
observability, but the reverse does not hold.

\subsection{Lattice Theory}






\begin{Definition}
Consider a set  and a relation  over .
 is reflexive if for each , ; it is
antisymmetric if  and  implies ; it
is transitive if  and  implies that . The partial order relation, denoted by , over 
is a reflexive, antisymmetric and transitive relation. The pair  is a poset.
\end{Definition}

\begin{Definition}
Consider a set  and .  is said to be the
supremal of , denoted by  or , if it satisfies :
(1) : , (2) .  is said to be the infimal of ,
denoted by  and , if it satisfies: (1) :  (2)  : . The poset  is called a
lattice if ,  for any finite . If  for arbitrary , then  is called a
complete lattice.
\end{Definition}

A poset may be a lattice, but it may have a set  of infinite
size for which  or  may not exist. However,  and
 exist for any  on a complete lattice.
Moreover, monotone functions and disjunctive functions are defined
over a complete lattice .

\begin{Definition}
A function  is said to be monotone if for any
 : .  is
said to be disjunctive if for any .
\end{Definition}




Furthermore, the following lemmas are introduced to obtain the
supremal solution of the system of inequalities \cite{kb}.

\begin{Lemma}\label{l1}

Consider the system of inequalities \{\} over a compete lattice (X, ). Let Y
= \{\} be
the set of all solutions of the system of inequalities and 
= \{\} be the set of all fixed points
of , where  and  is the
supremal solution of . If  is
disjunctive and  is monotone, then , , and .
\end{Lemma}





\begin{Lemma}\label{l2}
Consider the inequalities \{
and \}. If  is disjunctive and  is monotone,
 can be obtained by iterative computation: ,
 until .



\end{Lemma}


In this note, we focus on the computation of the supremal
simulation-based controllable and strong observable subautomaton
for the specification, which is not simulation-based controllable
and observable.


\section{Full Observation}
In this section, we establish a complete lattice over which the
constructed simulation operator, controllable operator and their
properties are reviewed \cite{sun}.



\begin{Definition} Given a plant  and
a specification , the poset is
defined as ().
\end{Definition}

It can be seen that this power set lattice () is built upon the state pairs from  and  and it
is a complete lattice \cite{kb}. Thus, supremal and infimal
defined with respect to a compete lattice are unique.

\begin{Remark}
An alternative poset can be a prelattice ,
where := \{~is~simulation-based controllable and strong observable) \} is
a set of automata and  is a simulation relation. However,
the supremal solution with respect to the prelattice
 is not unique because this simulation
relation over  is a preorder, which is transitive,
reflexive but not anti-symmetric.






\end{Remark}

Next, we introduce several operators defined over ().



\begin{Definition}
The simulation operator  defined by
, for , if the following
conditions are satisfied:

1. .

2.    .

3.   .
\end{Definition}

The simulation operator evolves from a similar operator in
\cite{pov} and it has following properties. Their proofs can be
found in \cite{sun}.


\begin{Proposition}\label{ss}
Given a plant  and a specification
,  is a simulation relation
from  to  if and only if  and .
\end{Proposition}

\begin{Proposition}\label{sm}
Given a plant , a specification
 and the sets ,  if .
\end{Proposition}

















\begin{Theorem} \label{st}
Given a plant  and a
specification , the supremal
simulation relation is the maximal fixed-point  of the operator
 if , where .
Moreover,

where  is an identity function,
and for each , .
\end{Theorem}






Before presenting the controllable operator, we introduce the
following concepts.
\begin{Definition}
Given an automaton  and a state , the string set of , denoted
by , is defined as . The nondeterministic state set of , denoted
by , is defined as . Further, we define the
nondeterministic active event set of the state , denoted by
, as .
\end{Definition}

We can obtain all the strings that can reach  from  through
 and all the states that are reachable from  with the
strings in  by . Besides,  is a union
of the active event set of the states in . Next, we propose
the following notion to guarantee the existence of the supremal
simulation-based controllable subautomata.
\begin{Definition}
Given a plant  and a specification
,  is said to be calculable for
the supremal simulation-based controllable subautomaton with respect
to  if it satisfies:

where .
\end{Definition}

Before presenting the controller operator, the simulation-based
controllable product is established.
\begin{Definition}
Given a plant  and a specification
, the simulation-based
controllable product of  and  is an automaton:

where

\end{Definition}
According to the definition of simulation-based controllable
product, a transition that leads to the new states through event
 is allowed if the active event sets of this state pair
 share the event . Besides, there will be a
transition to  if the state ,  which is
reachable from initial state  of  along , does not
include the uncontrollable event , where  is
defined at a certain state of  reachable from its initial state
 through . Moreover, the state pairs that are not
reachable from  are removed by the accessible
operator. Next, the controllable operator is built upon complete
lattice .
\begin{Definition}
Given a plant , a specification  and an automaton  for
, the controllable operator  defined by 
if it satisfies:

where for any ,  =
\{ s.t.  \}.
\end{Definition}

Moreover, this controllable operator satisfies following properties.






\begin{Proposition} \label{cs}
Given a plant , a specification   and a set ,
 satisfies the controllable condition if 
and there is  such that .
\end{Proposition}


\begin{proof}
Assume that  violates the controllable condition when  and there is  such that ,
where , then there exists  and
 such that  and  with . As , there is  with . Moreover,  belongs to the state
set of  because it is reachable from  by the string . Furthermore, we have  as  and . Thus,  in  by
the definition of the simulation-based controllable product. We
obtain , therefore, . On
the other hand, we have  as . Then, we obtain  by the
definition of the controllable operator. Thus, there is a
contradiction. Therefore,  satisfies the controllable
condition.
\end{proof}


\begin{Proposition}\label{cm}
Given a plant , a specification  and a set ,
 if  and  is calculable
for the supremal simulation-based controllable subautomaton with
respect to .
\end{Proposition}

\begin{proof}
For any , we have  and . Then,  since . Further,  because of the
definition of  and the calculability of 
for the supremal simulation-based controllable subautomaton with
respect to . Thus, . Therefore, we have
.
\end{proof}




\section{Partial Observation}
In this section, we establish a monotone strong observable operator
over complete lattice . Combine it with
the simulation operator and the controllable operator, the
inequalities whose solution is the simulation-based controllable and
strong observable set are set up. Then, an algorithm is proposed for
the computation of simulation-based controllable and strong
observable subautomata.





\subsection{Strong Observable Operator}



\begin{Definition}
Given a plant  and a
specification , the
simulation-based observable product of  and  is defined as:

where  and for any
, ,

The transition 

\end{Definition}

In particular, the transition can be extended from domain  to
domain  in the
following recursive manner: 
   if , otherwise,
.

The simulation-based observable product  satisfies
the following proposition.


\begin{Proposition} \label{all}
Given a plant , a specification
 and their simulation-based
observable product ,  iff there exists  with
 such that  and .
\end{Proposition}


\begin{proof}
The induction method is adopted to prove this proposition.
(Necessity) 1. , then . (1) , that is,
. Let . Obviously, we have ,  and . (2) Let
 with . For any , we have
,  and . (3) Assume that , the necessity
of this proposition holds. (4) . For any , where
, there exists  with ,  and  s.t.  since the
necessity of this proposition holds when  and
. Then, ,  and . 2. Let  with . (1) , then . Obviously, the
necessity holds. (2) Let  with . Any

satisfies the following cases. Case : there exists  with ,  and
 s.t. , then
,  and . Or case : there
exists  with ,  and  s.t. , then
,  and . Or case : there
exists  then , , ,  and . (3) Assume that
, the necessity of this proposition holds when .
(4) . For any , where , we have following cases. Case 1: there
exists  with ,  and  s.t.  since  and
. Then,
,  and . Or case 2: there exists  with ,  and
 s.t.  since it is similar to 1.(4) when
 and . Then,
,  and
. Or case 3: there exists  with ,  and  s.t.

since it satisfies the case 1.(3) when  and
. Then,
,  and . 3. Assume that , , the
necessity of this proposition holds. (4) Let  and
. For any , it satisfies the following cases. Case 1: there
exists  with ,  and  s.t.  since the
necessity of this proposition holds when  and . Then, ,  and . Or case 2:
there exists  with ,  and  s.t.
 since it satisfies 3 when  and . Then, ,  and . Or case 3:
there exists  s.t. .
Similarly, we obtain ,  and
. (Sufficiency) 1. , then
. Let . (1)  and
. For any , it is
obvious that . (2) . Let  with
. For any ,
we have . (3) Assume that the sufficiency of this proposition
holds when  and . (4) . For any  with
, we obtain . Because
the sufficiency of this proposition holds when  and
 from above assumption,
there exists  with  s.t. , then . 2.
. Let  and . (1)
 and . For any  with , we have
.
(2) . Let  with ,
 and . Then, for
any , we have . (3) Assume that
the sufficiency of this proposition holds when  and
. (4) . Let  with
.
If , then . There exists  with  s.t.  as the
sufficiency of this proposition holds when  and
. Moreover, . Then, . If
, we have
. Then,
there are two cases. Case 1: If
, we obtain . Obviously, the sufficiency of
the proposition holds. Case 2: If , there
exists  with  s.t.
 and .
Futher,  if ;  if . Thus, there is
 with  s.t.  as . Then,  because of . Therefore, we
have  by
the definition of the simulation-based observable product. 3.
Assume that , , the sufficiency of this
proposition holds. (4)  and . Let ,  and . If
, then  with  and
. Thus, there exists  with  s.t.
. Therefore, . Then, . If , we have .
Then, we have two cases. Case 1: If , we
obtain . There is  with  and  with  s.t. 
because the sufficiency of the proposition holds when  and . Then, .
Therefore,  by
the definition of the simulation-based observable product. Case 2:
If . There exists  with  s.t.  with . Moreover,  if  and
 if . Thus, there is  with  s.t.  as
 and  satisfying the assumption 3. Then,  because of the definition of
the simulation-based observable product.
\end{proof}

Based on the simulation-based observable product, the following
concepts are introduced.

\begin{Definition}
Given a simulation-based observable product  and , the equivalent projection string set
of  with respect to the plant  is defined as
 s.t.
.
\end{Definition}

It can be seen that all the strings of plant  that have the same
projection as the string  of specification are included in
. In order to guarantee the existence of the supremal
simulation-based strong observable subautomata, we propose the
following concept.

\begin{Definition}
Given a plant  and a specification
,  is said to be calculable
for the supremal simulation-based strong observable subautomaton
with respect to  if it satisfies:

where .
\end{Definition}

The specification  is said to be calculable for
simulation-based controllable and strong observable subautomaton
with respect to  if it is calculable for both supremal
simulation-based controllable subautomaton and supremal
simulation-based strong observable subautomaton.

Because the simulation-based observability is not closed under
state union, the supremal simulation-based observable subautomaton
does not exist. Here, we introduce the simulation-based strong
observability which implies simulation-based observability and it
is also closed under state union under certain conditions.
\begin{Definition}
Given a plant , a specification
 and their simulation-based
observable product ,  is said to be
simulation-based strong observable with respect to ,
 and  if it satisfies:

(1) (Simulation Condition) There is a simulation relation 
such that .

(2) (Strong Observable Condition)  and  
[.
\end{Definition}

The  is said to be a
simulation-based strong observable set if  is a
simulation relation from  to  and 
satisfies the strong observable condition. Furthermore, the set
 is a simulation-based controllable and strong
observable set if it is a simulation-based controllable set and
also a simulation-based strong observable set.

The relationship between simulation-based strong observability and
simulation-based observability as below.
\begin{Proposition}
Given a plant  and a
specification ,  is
simulation-based observable with respect to ,  and
 if  is simulation-based strong observable with respect to
,  and .
\end{Proposition}

\begin{proof}
Because  is simulation-based strong observable with respect to
,  and , we have that  is simulated by .
Assume that  satisfies the strong observable condition but not
the observable condition, then there exists  with
 s.t. there is  with  if  and ,
where . Let , we have the
following cases. (1) . Since , . (2) 
with . We have  and . Moreover,  and  because  implies . Thus,
 according to the strong
observable condition. Let , we
have . In addition, . Thus,
 because  satisfies the
strong observable condition. Therefore, all the cases contradict
the assumption. As a result,  satisfies the observable
condition. Hence,  is simulation-based observable with respect
to ,  and .
\end{proof}

Based on the simulation-based strong observability, we propose the
following notion.

\begin{Definition}
Let  be a plant,  be a specification,  be their simulation-based observable product and  be a subautomaton
for  . For any ,  and , the state failure set of
 for the strong observability, denoted by , is
defined as:

\end{Definition}

Then, we construct the strong observable operator based on the
complete lattice .
\begin{Definition}
Given a plant , a specification
 and a subautomaton  for , the strong observable operator  defined by  if
it satisfies:

\end{Definition}

The strong observable operator satisfies following propositions.

\begin{Proposition}\label{ns}
Given a plant , a specification
 and a set ,  satisfies the strong observable condition if  and there exists  such that .
\end{Proposition}


\begin{proof}
Let  be a
subautomaton for . Assume that  violates the strong
observable condition when  and , where , then there exists 
with  such that  with
 and . Thus, . Then, . Since , we have . By the definition of the strong observable
operator , , which
introduces a contradiction. Then,  satisfies the strong
observable condition.
\end{proof}

\begin{Proposition}\label{nm}
Given a plant , a specification  and the sets ,  if  and
 is calculable for the supremal simulation-based strong
observable subautomaton with respect to .
\end{Proposition}

\begin{proof}
Let  be a
subautomaton for  and
,
\\
 be a subautomaton for . For any 
and , we have . If , then  for any . Thus, we obtain  because of
. Hence, . Since , we also have  with  for any  such
that  and any  such that
 and . Moreover,  as . Assume that there exists ,  and  with  and 
such that  and . If , we have the following
cases: (1) . Obviously, . (2) , then
 since  is calculable
for supremal simulation-based strong observable subautomaton with
respect to . On the other side, there are two cases if . (1). Because  and , we obtain
. Then, . (2) . Because  is
calculable for the supremal simulation-based strong observable
subautomaton with respect to , we have . Thus, we get  from all above cases, which contradicts the
assumption that . Therefore,
. Hence, . Similarly, we can prove that  when . As a result, .
\end{proof}






From definition of , we have .
Then, the supremal state set  satisfying 
is a fixed point of  from lattice theory. As  is
monotone by Proposition \ref{nm}, the maximal fixed point of
 can be obtained by iterating , and it will be
discussed in next subsection.









\subsection{Supremal Simulation-based Strong Observable Subautomata}
A sufficient condition is proposed to guarantee the existence of
the supremal simulation-based strong observable set. Further, an
algorithm is presented to such subautomaton.






\begin{Proposition}\label{supsn}
Let  be a plant,  be a specification,  = \{  \} and  be a set of
fixed points of . For any 
and identify function , the
function  is
defined as:

Then, any  is a simulation-based strong
observable set and  if  and  is calculable for the supremal simulation-based
strong observable subautomaton with respect to .
\end{Proposition}


\begin{proof}
As  and , we obtain that  is a
simulation relation from  to  by Proposition \ref{ss}.
Moreover,  satisfies the strong observable
condition by Proposition \ref{ns} because . Hence,  is a
simulation-based strong observable set. From lattice theory,
(, ) is a compete lattice over which we
definite the simulation operator  and the strong observable
operator  which are monotone by Proposition \ref{sm} and
Proposition \ref{nm}. The identity function  =
 and  =  are
disjunctive. Hence,  by Lemma 1.
\end{proof}

{\it Algorithm 1: } Given a plant 
and a specification , the
algorithm for computing the supremal simulation-based strong
observable subautomaton with respect to , , and  is
as follows:

Step 1. Check whether  is calculable for the supremal
simulation-based strong observable with respect to . If not,
the supremal simulation-based strong observable subautomaton does
not exist, otherwise, go to step 2.

Step 2. Let , ,  until .

Step 3. If , the supremal simulation-based
strong observable subautomaton does not exist, otherwise, if
  ,  is the supremal
simulation-based strong observable subautomaton with respect to
, , and .



\begin{Remark}
Since  and  are nondeterministic, their number of
transitions are  and  respectively. So the complexity of the simulation-based
observable product is . Then, the complexity of checking the
calculability of specification  for the supremal
simulation-based strong observable subautomaton with respect to
 is . Further, the complexity of the simulation
operator is  and the most
iterative times is , the complexity of Algorithm 1
is .
\end{Remark}



\begin{Theorem}
Algorithm 1 is correct.
\end{Theorem}
\begin{proof}
We have  by Lemma 2 and Proposition \ref{supsn}.
Further,  is a simulation-based strong observable set if
 and  is calculable for the supremal
simulation-based strong observable subautomaton w.r.t  by
Proposition \ref{supsn}. Therefore,  is the supremal
simulation-based strong observable set. Base on it, we build the
subautomton . Therefore,  is the supremal
simulation-based strong observable subautomaton w.r.t. ,
, and .
\end{proof}



\subsection{Supremal Simulation-based Controllable and Strong Observable Subautomata}
Further, we propose a sufficient condition to guarantee the
existence of the supremal simulation-based controllable and strong
observable set and an algorithm to calculate such subautomaton.
\begin{Proposition}\label{supscn}
Let  be a plant,  be a specification,  = \{  \} and  is a set of
fixed points of . For any 
and identify function , the
function  is
defined as:

Then, any  is a simulation-based
controllable and strong observable set and  if
 and  is calculable for supremal
simulation-based controllable and strong observable subautomaton
with respect to .
\end{Proposition}


{\it Algorithm 2:}  Given a plant 
and a specification , the
algorithm for computing the supremal simulation-based controllable
and strong observable subautomaton is as follows:

Step 1. Check whether  is calculable for the supremal
simulation-based controllable and strong observable subautomaton
with respect to . If not, the supremal simulation-based
controllable and strong observable subautomaton does not exist,
otherwise, go to step 2.

Step 2. Let , ,  until .

Step 3. If , the supremal simulation-based
controllable and strong observable subautomaton does not exist,
otherwise,  is the supremal simulation-based controllable
and strong observable subautomaton if .



\begin{Remark}
The complexity of checking calculability of specification  for
the supremal simulation-based controllable subautomaton is
. Further,
the complexity of the Algorithm 1 and the simulation-based
controllable product are  and 
respectively, the complexity of Algorithm 2 is .
\end{Remark}



\begin{Theorem}
Algorithm 2 is correct.
\end{Theorem}

The proofs for Propositions \ref{supscn} and Theorem 4 are similar
to Proposition \ref{supsn} and Theorem 3.

\begin{Remark}
Since simulation-based strong observability implies
simulation-based observability, the supremal simulation-based
controllable and strong observable subautomaton is
simulation-based controllable and observable. Further, its
language is controllable and observable because simulation-based
controllability and observability implies language controllability
and observability \cite{liu}.
\end{Remark}


Further, this supremal controllable and strong observable
subautomaton satisfies the following property.



\begin{Proposition}\label{supa}
Given a specification  and a
plant  such that , the subautomaton 
obtained by Algorithm 2 is a supremal element of automata set
:= \{ ( )  ( ~is
~simulation-based controllable and strong observable) \} based on
the prelattice .
\end{Proposition}

\begin{proof}
Let  be an automaton
satisfies that  and  is simulation-based
controllable. We need to prove that there exists a simulation
relation  between  and  such that  when . Because
, there is  such
that  for any . Assume that , there are two cases
according to Algorithm 2: (1)(). Then  because  with
. Thus,  is not
simulation-based controllable w.r.t.  and , which
introduces a contradiction. (2) For any  such that , we have any  such that
 with  and  violates the
controllable condition. Then, we have  and , where  and
. Thus, there is 
such that  and . Then , which implies
that  does not satisfy the controllable condition. Hence, we
obtain a contradiction. Therefore, the assumption does not hold.
That is, . Thus, . Similarly, we can
prove that  if  and  is
simulation-based strong observable. As a result,  is a
supremal element of .
\end{proof}

\begin{Remark}
The assumption requiring that , can be satisfied
at the most cases because the descried specification should not be
out of the range of the behavior of the plant. This is similar to
the precondition  in Ramadge-Wonham's
framework.
\end{Remark}




\section{EXAMPLE}
\begin{figure}[!htb]
\begin{center}
\includegraphics*[scale=.50]{wp.pdf}
\caption{Manufacturing System (Left) and Plant (Right)} \label{all3}
\end{center}
\end{figure}

Consider a manufacturing system that consists of two workstations,
three rooms and a robot as shown in Fig. \ref{all3} (Left).
Initially, the robot is in workstation 1. By choosing rail 1 (event
), this robot nondeterministically goes to room 2 and room 3 and
by choosing rail 2 (event ), it can go to room 1. If the robot is
in room 2 and it hears the alarm (event ), it can go to the
workstation 2 (event ). Or it can take a video (event ) when
it is in room 2 and after that it has two choices : to go to
workstation 2 (event ) or to receive the message from the host
computer (event ). After the message has been received, the robot
can active an energy-saving mode (event ) and then go to
workstation 2 (event ). If the robot is in room 3, its behavior
is similar to what it does in room 2 except that it can pick up a
box from room 3 (event ) and then go to workstation 2 (event
). If it is in room 1, it also has two choices: to pick up a
box from room 1 (event ) then go to workstation 2 (event )
or to take a video (event ) and after then go to workstation 2
(event ). In this model, we assume that the event 
describing that the robot hears the alarm is uncontrollable, the
event  describing that the robot receives a message from the host
computer is uncontrollable and unobservable and all the rest events
are controllable and observable.



\begin{figure}[!htb]
\begin{center}
\includegraphics*[scale=.50]{s.pdf}
\caption{Specification (Left) and Supremal Simulation-Based
Controllable and Strong Observable Subautomata (Right)} \label{all4}
\end{center}
\end{figure}






The automata model  of the robot in manufacturing system is shown
in Fig. \ref{all3} (Right). The specification  is in Fig.
\ref{all4} (Left) to restrict the behavior of , which requires
that the robot can go to the workstation 2 after hearing the alarm
or go to workstation 2 after taking the video if it is in room 2. It
can be seen that . Thus, if we use language equivalence
as a notion of behavioral equivalence, there is no need to control.
However, as mentioned above,  can exhibit some undesired
behaviors, which motivates us to design a supervisor  such that
the controlled system  is bisimilar to . In \cite{liu}, such
a supervisor  exists if and only if  is simulation-based
controllable and observable under partial observation. However, 
in this example is not simulation-based controllable and observable.
In this paper, we want to calculate the supremal simulation-based
controllable and strong observable subautomaton of . By Algorithm
2, we obtain that  is calculable for such kind of subautomaton.
Next, we have ,  and  in the first iteration.
Further,  and . Hence, the
supremal simulation-based controllable and strong observable
subautomata is achieved in Fig. \ref{all4} (Right).







\section{CONCLUSIONS}

By resorting to lattice theory, we proposed a computational approach
to solve the supremal simulation-based controllable and strong
observable subautomata, where both plant and specification are
modeled as nondeterministic automata. The obtained solution provides
a sufficient condition of the existence of the supremal
simulation-based controllable and strong observable subautomta and
an explicit algorithm to calculate such subautomta. Further, an
example is generated to illustrate the proposed techniques.













\begin{thebibliography}{99}











\bibitem{milner}R. Milner, {\it Communication and Concurrency}. Prentice Hall, New
York, 1989.



\bibitem{f}J. Fernandez, ``An implementation
of an efficient algorithem for bisimulation equivalence," {\it Sci.
Comput. Programming}, vol. 13, pp. 219-236, 1990.

\bibitem{pa}P. Tabuada and G. J. Pappas, ``Linear temporal logic control of discrete-time linear
systems," {\it IEEE Transactions on Automatic Control}, vol. 51, pp.
1862-1877, December 2006.


\bibitem{pro}V. Danos, J. Desharnais, F. Laviolette, ``Bisimulation and concongruence for probabilistic
systems," {\it Inform. Comput. Programming }, vol. 204, pp. 503-523,
2006.

\bibitem{hyb}E. Haghverdi, P. Tabuada, and G. J. Pappas, ``Bisimulation relation for dynamical, control, and hybrid
systems," {\it Theoret. Comp. Sci.}, vol. 342, pp. 229-261, 2005.

\bibitem{k1}J. Komenda and J. H. van Schuppen, ``Control of discrete-event systems with partial observations using coalgebra and
coinduction," {\it Discrete Event Dynamical Systems: Theory and
Applications}, vol. 15, pp. 257-315, 2005.




\bibitem{TabA}P. Tabuada and G. J. Pappas, ``Bisimilar control affine
systems," {\it Systems  Control Letters},  vol. 52, pp. 49-58,
2004.

\bibitem{TabC}P. Tabuada, ``Controller synthesis for bisimulation
equivalence," {\it Systems  Control Letters}, vol. 57, pp.
443-452, 2008.






\bibitem{cc}C. G. Cassandtras and S. Lafortune, {\it Introduction to
Discrete Event Systems}. Boston, MA: Kluwer, 1999.

\bibitem{sun}Y. Sun, H. Lin, F. Liu, Ben M. Chen,
``Computation for Supremal Simulation-Based Controllable
Subautomata," {\it In Proceedings of 8th IEEE International
Conference on Control and Automation}, pp. 1450-1455, Xiamen,
China, June 9-11, 2010.


\bibitem{kb}R. Kumar and V. K. Garg, {\it Modeling and Control of Logical Discrete
Event Systems}. Kluwer Academic Publishers, Boston, MA, 1995.







\bibitem{tl} E. A. Emerson, `` Temporal and Modal Logic", In J. van leeuwen, editor, {\it Handboook of Theoretical Computer Science: Formal
Models and Semantics}, North-Holland Pub. Co./MIT Press, volume B,
pp. 995-1072.

\bibitem{cbis3}C. Zhou, R. Kumar, and S. Jiang, ``Control of Nondeterministic
Discrete Event Systems for Bisimulation Equivalence," {\it IEEE
Transactions on Automatic Control}, vol. 51, pp. 754-765, 2006.


\bibitem{liu}F. Liu, D. Qiu, and H. Lin, ``Bisimilarity control of nondeterministic discrete event systems," {\it submitted for publication}, 2010.








\bibitem{pov}Paulo Tabuada, {\it Verification and Control of Hybrid
Systems}, Springer, 2009.

\bibitem{ke}R. Kumar and V. K. Garg, ``Extremal Solutions of Inequations over
Lattices with Applications to Supervisory Control," {\it
Theoretical Computer Science}, vol. 148, pp. 67-92, November 1995.


\bibitem{kf}R. D. Brandt, V. K. Garg, R. Kumar, F. Lin, S. I. Marcus, and W.
M. Wonham, ``Formulas for Calculating Supremal Controllable and
Normal Sublanguages," {\it Systems and Control Letters}, vol. 15,
pp. 111-117, August 1990.



\end{thebibliography}

\end{document}
