\documentclass[11pt]{article}
\usepackage{amstext}
\usepackage{amsthm}
\usepackage{amsmath}
\usepackage{amssymb}
\usepackage{setspace}
\usepackage{enumerate}
\usepackage{graphicx}
\usepackage[toc,page]{appendix}
\usepackage{array}

\setcounter{secnumdepth}{3} 

\setlength{\textheight}{8.9in}
\setlength{\textwidth}{6.6in}
\setlength{\evensidemargin}{-0.18in}
\setlength{\oddsidemargin}{-0.18in}
\setlength{\headheight}{0pt}
\setlength{\headsep}{0pt}
\setlength{\topsep}{0in}
\setlength{\topmargin}{0.0in}
\setlength{\itemsep}{0in}
\renewcommand{\baselinestretch}{1.04}
\parskip=0.08in

\def\Proof{\par\noindent{\bf Proof:~}}
\def\blackslug{\hbox{\hskip 1pt \vrule width 4pt height 8pt
    depth 1.5pt \hskip 1pt}}
\def\QED{\quad\blackslug\lower 8.5pt\null\par}
\def\inQED{\quad\quad\blackslug}
\def\dnsitem{\vspace{-7pt}\item}
\def\dnssubitem{\vspace{-5pt}\item}

\newtheorem{theorem}{Theorem}[section]
\newtheorem{proposition}[theorem]{Proposition}
\newtheorem{definition}[theorem]{Definition}
\newtheorem{algorithm}[theorem]{Algorithm}
\newtheorem{claim}[theorem]{Claim}
\newtheorem{lemma}[theorem]{Lemma}
\newtheorem{notation}[theorem]{Notation}
\newtheorem{corollary}[theorem]{Corollary}
\newtheorem{fact}[theorem]{Fact}
\newtheorem{comment}[theorem]{Comment}
\newtheorem{remark}[theorem]{Remark}
\newtheorem{observation}[theorem]{Observation}
\newtheorem{assumption}[theorem]{Assumption}
\newtheorem{property}[theorem]{Property}
\newtheorem{invariant}[theorem]{Invariant}
\newtheorem{conjecture}{Conjecture}
\theoremstyle{definition}
\newtheorem{condition}[theorem]{Condition}
\newtheorem{procedure}[theorem]{Procedure}
\newtheorem{execution}{Execution}

\def\propt1{\mathcal{P}_0}
\newtheorem*{tpropt1*}{Property }
\def\propbr{\mathcal{P}_1}
\def\propbbr{\mathcal{P}_2}
\def\formcto{\mathcal{F}_1}
\def\formctt{\mathcal{F}_2}
\def\regtwo{\mathcal{R}_1}
\def\regclean{\mathcal{R}_2}
\def\regct{\mathcal{R}_3}
\def\regcolored{\mathcal{R}_4}
\def\decomponereg{\mathcal{A}_1}
\def\decomponecorr{\mathcal{A}_2}
\def\decomproot{\mathcal{A}_3}
\def\decompsize{\mathcal{A}_4}
\def\decompcleanl{\mathcal{A}_5}
\def\decompedges{\mathcal{A}_6}
\def\boxpropnocorhigh{\mathcal{C}_1}
\def\boxpropnocorlow{\mathcal{C}_2}
\def\boxpropcor{\mathcal{C}_3}

\def\propres{\mathcal{P}_1}
\def\propdom{\mathcal{P}_2}
\def\propstaller{\mathcal{P}_3}

\newcolumntype{C}[1]{>{\centering}m{#1}}

\def\invboxes{\mathcal{I}}
\newtheorem*{tinvboxes*}{Invariant }
\def\invboxesdom{\mathcal{I}_D}
\def\invboxesstaller{\mathcal{I}_S}

\def\varexcess{\tt \Psi}

\def\performmatch{\mbox{\sf Perform\_Match}}
\def\play{\mbox{\sf Play}}
\def\update{\mbox{\sf Update}}
\def\densify{\mbox{\sf Densify}}
\def\interpart{\mbox{\sf InterPart}}
\def\joinhigh{\mbox{\sf JoinHigh}}
\def\discblue{\mbox{\sf DisconnectExtBlue}}
\def\fixbwparent{\mbox{\sf FixBWParent}}
\def\convhigh{\mbox{\sf ConvHigh}}

\def\labelitemi{--}

\begin{document}
\onehalfspacing

\title{The Domination Game: \\
Proving the 3/5 Conjecture on Isolate-Free Forests}
\date{\today}
\author{Neta Marcus
\thanks{The Weizmann Institute of Science, Rehovot, Israel.
{\tt \{neta.marcus,david.peleg\}@weizmann.ac.il}.}
\and
David Peleg  
\thanks{Supported in part by the Israel Science Foundation (grant 1549/13)
and the I-CORE program of the Israel PBC and ISF (grant 4/11).}
}
\maketitle

\begin{abstract}
\pagenumbering{arabic} 
We analyze the \emph{domination game}, where two players, {\em Dominator} and {\em Staller}, construct together a dominating set  in a given graph, by alternately selecting vertices into . Each move must increase the size of the dominated set. The players have opposing goals: Dominator wishes  to be as small as possible, and Staller has the opposite goal. 
Kinnersley, West and Zamani conjectured in \cite{kinnersley2013extremal} that when both players play optimally on an isolate-free forest, there is a guaranteed upper bound for the size of the dominating set that depends only on the size  of the forest. This bound is  when the first player is Dominator, and  when the first player is Staller. 
The conjecture was proved for specific families of forests in \cite{kinnersley2013extremal} and extended by Bujt\'as in \cite{bujtas2015domination}. 
Here we prove it for all isolate-free forests, by supplying an algorithm for Dominator that guarantees the desired bound.
\end{abstract}


\section{Introduction}
\label{chapter:intro}

We analyze a two-party game on graphs called the \emph{domination game}, in which two players with opposing goals construct together a dominating set for a given graph.
The game was introduced by Bre\u{s}ar, Klav\u{z}ar and Rall in \cite{brevsar2010domination}.
One setting in which such a problem may be of interest is the following scenario:

{\em New city regulations state that a house is only fire-safe if it is a short distance from a trained firefighter.
In order to make sure all houses are fire-safe, a list of citizens that should be trained and hired as firefighters must be made.
Two people volunteer for the task of making the list:
The city treasurer, who wishes to minimize the costs and therefore wants the number of firefighters to be as small as possible,
and 
the head of the firefighters union, who benefits from adding new members and therefore wishes to maximize the number of firefighters.
The mayor, a seasoned politician, decides to let both volunteers add names to the list 
in turns, each adding a single firefighter that would improve the safety of the city by making additional houses fire-safe, until the new regulations are met.
What strategy should the treasurer adopt? What can be guaranteed about the outcome?
}

We study the possible outcomes of such a selection process under some specific settings.

\paragraph{The problem.}
Throughout, we consider an undirected graph  of size .
We assume  is \emph{isolate-free}, i.e., it has no isolated vertices.
A \emph{dominating set} is a set  such that for each , either  or  has a neighbor in .

For a graph  and a subset , denote the \emph{closed neighborhood} of  by , that is,
. 
For a single vertex , define 
.
The \emph{open neighborhood} of  is the set of its neighbors in , and it is denoted by .
Whenever  is clear from the context, we omit it and write simply ,  or .
The size of the smallest dominating set in  is denoted by .
For any set , a vertex  is said to be \emph{dominated by }.

In the \emph{domination game}, two players construct together a dominating set, . 
The players alternate in taking turns, and in each turn, the current player picks a single vertex and adds it to .
The two players are referred to as \emph{Dominator} and \emph{Staller}, and they have opposing goals regarding  - Dominator wants to minimize , while Staller wants to maximize it.

The chosen vertex at step  is referred to as the player's \emph{move} in step  or the 'th move, and is denoted by .
The partial dominating set constructed at the end of step  is denoted by . Define .
A move  is considered \emph{legal} if the dominated set increases, that is, .
The players must make legal moves at all steps. 
The game ends when all vertices of  are dominated by , that is, when  is a dominating set of .
Hence, the game is a maximal sequence of legal moves, that is, a sequence  such that  is a dominating set but  is not.

The domination game has two variants:
It is called a \emph{Dominator-start} game when the first move is taken by Dominator, 
and a \emph{Staller-start} game when Staller makes the first move.
Hence, in a Dominator-start game, the odd moves are decided by Dominator and the even moves are decided by Staller.
In a Staller-start game, it is the opposite.
When both players play optimally, we call the size of the resulting dominating set the \emph{game domination number} of , 
and denote it by  for the Dominator-start variant, and by  for the Staller-start variant.

We wish to study the following conjecture, introduced in \cite{kinnersley2013extremal}.
\begin{conjecture}
\label{conjecture:3_5}
If  is an isolate-free -vertex forest (i.e., it has no singleton vertices), then 
 
and 
.
\end{conjecture}
The conjecture was later extended to general isolate-free graphs in \cite{bujtas2015domination}, but here we focus on forests.

Since our goal is to prove the conjecture, we introduce modified goals for the players.
We say Dominator wins in a Dominator-start (respectively, Staller-start) game if the game ends within at most  (resp., ) moves, and otherwise Staller wins.

\begin{procedure}
\label{proc:game_outline}
Given an isolate-free -vertex forest , the Dominator-start variant of the game can be described by the following algorithm.
\begin{enumerate}
	\dnsitem ;
              ;
	      
	\dnsitem While :
	\begin{enumerate}
		\dnsitem 
		\dnssubitem \emph{current player}  Dominator if  is odd and Staller otherwise
		\dnssubitem Receive a legal move  from the current player
		\dnssubitem .
	\end{enumerate}
	\dnsitem If , Dominator wins. Otherwise, Staller wins.
\end{enumerate}
\end{procedure}

Hereafter, the total number of moves in a specific execution of the game is denoted by .

A similar algorithm can be used to describe the Staller-start variant, except that then  is set to , and the odd moves are performed by Staller.

\paragraph{Previous approaches.}
As mentioned earlier, Conjecture \ref{conjecture:3_5} was introduced by Kinnersley, West and Zamani in \cite{kinnersley2013extremal}.
In the paper, the conjectured bound of  moves is achieved for specific types of forests, and a weaker bound of  moves is proved for arbitrary isolate-free -vertex forests.

In \cite{bujtas2015domination}, Bujt\'as proves the conjecture for isolate-free forests in which no two leaves are at distance , 
and improves the bound for arbitrary isolate-free -vertex forests from  to .
The proofs in \cite{bujtas2015domination} use a method for coloring and evaluating vertices according to their state, and creating intermediate graphs, in order to choose moves and to prove the desired bound.


\paragraph{Motivating example.}
We start with a simple example that illustrates some of the difficulties that any algorithm for Dominator must face.
Consider a Dominator-start game which is played on the graph shown in Figure \ref{fig:locality_example_intro}.
The graph contains  vertices, therefore Dominator wins if and only if the game ends within  moves or less.
Even though the neighborhoods of the vertices  and  are similar, 
the reader can verify that Dominator can win by playing  or  as the first move, 
whereas if Dominator plays  or , Staller can win the game by playing  in the following move.

\begin{figure}[thbp]
  \caption{\sf Motivating example.}
  \medskip
  \centering
    \fbox{\includegraphics[width=7cm]{locality_example_intro.jpg}}
  \label{fig:locality_example_intro}
\end{figure}

We believe this example can be extended to graphs of arbitrarily large size, in which choosing between moves that appear to be the same locally may determine the outcome of the game. 


\paragraph{Our contributions.}
We provide an algorithm for Dominator that guarantees that the game ends within the number of moves required by Conjecture \ref{conjecture:3_5} on all isolate-free forests, which proves the conjecture.
We rely on the general method used in \cite{bujtas2015domination} and extend it, by separating the value of each vertex from its color, as well as fine-tuning additional aspects of the intermediate graphs. 
We start with Section \ref{section:dom_game_notations}, where we lay the foundations for the analysis by formalizing various aspects of the game.
We then introduce the algorithm in Section \ref{section:algorithm_outline}, and prove that it achieves the bound of Conjecture \ref{conjecture:3_5} in Section \ref{section:analysis}.
We conclude the analysis of Dominator's strategy in Section \ref{section:implementation}, where we discuss a possible implementation of the algorithm and describe our tests.
Finally, Section \ref{section:conclusions} contains some concluding remarks.

\section{Notation and Preliminaries}
\label{section:dom_game_notations}

Before describing the algorithm, we introduce some definitions and properties used to analyze the game.

\paragraph{Graph notions and vertex labeling.}
The graphs on which the game is played are undirected and unrooted forests that have no isolated (singleton) vertices.
We label all vertices in the initial graph  with distinct even indices, .
The motivation for this is that we later introduce virtual vertices,
and we want an easy way to tell apart real (non-virtual) vertices from virtual ones - a real vertex  always has an even index, while a virtual vertex  has an odd index.

When we refer to \emph{components} of a graph, we always mean maximal connected components.
For a component , define the size of  to be the number of vertices in  and denote it by .
The degree of a vertex  in a graph  is denoted by . When  is clear from the context, it is denoted by .


\paragraph{Vertex color, vertex value and legal moves.}
Recall that the players construct together a set , until it becomes a dominating set.
We use a variation of the grading system introduced in \cite{bujtas2015domination}.
During the game, each vertex has one of three possible \emph{colors}, and one of three possible \emph{values} 
(and these may change between steps).

\begin{definition}
Let  be a vertex in the graph.
The \emph{color}, or \emph{type}, of  at the end of step , denoted by , is defined by the following properties:
\begin{itemize}
	\dnsitem  is called \emph{white} or  if  is not dominated, that is, .
	\dnsitem  is called \emph{blue} or  if  is dominated but has an undominated neighbor, that is,  but .
	\dnsitem  is called \emph{red} or  if  is dominated and all its neighbors are dominated, that is, .
\end{itemize}
\end{definition}

When  is clear from the context, we denote the color by  instead of .

We define ,  and  to be the sets of vertices of type ,  and  (respectively) at the end of step .

Even though the first step of the game is step , we use (the end of) step  to denote the state of the graph before the first move.

\begin{observation}
For all steps , , and these sets are disjoint.
\end{observation}

\begin{observation}
For every  and , .
Also,  and , namely,  and  for every .
\end{observation}

\begin{observation}
If  then  for all .
If  then  for all .
\end{observation}

\begin{claim}
\label{claim:legal_moves}
At the beginning of step , the set of legal moves consists of exactly the vertices of , 
or in other words, the only vertices that a player cannot choose at step  are those in .
This also implies that every move is either on or adjacent to some white vertex.
\end{claim}
\Proof
Let .
If  is a red vertex, then  and therefore , so  is an illegal move.
On the other hand,
if  is a blue vertex, then  and therefore , so  is a legal move.
Similarly, 
if  is a white vertex, then  and therefore  is a legal move.
\QED

\begin{observation}
\label{obs:edges_no_r}
If for some step ,  and , then  
(that is, white and red vertices cannot be neighbors).
\end{observation}

\begin{definition}
For any step  and for any , 
let  be the color of  assuming the st move was , that is,  if .
\end{definition}

In addition to its color, each vertex also has a value.

\begin{definition}
A function  is called a \emph{value} function if it satisfies the following requirements for all .
\begin{itemize}
	\dnsitem If  then ,
	\dnsitem If  then ,
	\dnsitem If  then .
\end{itemize}
If , we say that at step ,  is \emph{worth}  points, or \emph{has value} .
For a set of vertices , define . 
\end{definition}

When  is clear from the context, we may omit it, and denote the value of  by .

\begin{definition}
For any step , a vertex  is called \emph{high}, 
and its type is generically referred to as , if .
Let  denote the set of high vertices at the end of step .
\end{definition}

\begin{definition}
For any step  and for any vertex , if  and , we say that  is a  vertex (at the end of step ).
Similarly, if  and , then  is called a  vertex.
Note that saying that a vertex is of type  is synonymous to saying that it is of type  or .
\end{definition}

In the graphical illustrations to appear hereafter, vertices are of type  except where specifically labeled otherwise.

\begin{definition}
For any step  and for any  and ,
let  be the value of  at the end of step  assuming the st move was ,
that is,  if .

\end{definition}

\begin{observation}
For any step , for every  and for any value function , .
Also, for any  and for any value function,  and , and consecutively  and .
\end{observation}

Let us remark that the value function defined later on for the algorithm will ensure that  is monotonically decreasing in .

\paragraph{Gain.}
The \emph{gain} of a vertex  under a given value function is the number of points gained when the current player chooses it. Formally, 
given the value function , the corresponding gain function \\
 is defined by
.

Again, whenever  is clear from the context, we omit it.

\begin{claim}
\label{claim:3pts}
For any  and for any , the following properties hold.
\begin{enumerate}[(a)]
{\setlength\itemindent{10pt} 
	\dnsitem 
	\label{claim:3pts:b}
	If  for all , then .
	
	\dnsitem 
	\label{claim:3pts:w}
	If  then  for any gain function .
}
\end{enumerate}
\end{claim}
\Proof
If , then for any value function it holds that  and . 
Therefore  (for any gain function), establishing (\ref{claim:3pts:w}).
It remains to prove (\ref{claim:3pts:b}) in case . Then by the definition  has some white neighbor, , at the end of step . 
Since , the value of  decreases by at least . 
Also, since , we conclude that if  assigns a value of  to all new blue vertices then the value of  decreases by at least . 
Hence .
\QED

\begin{corollary}
\label{cor:3pts_def}
It is always possible to define the value function such that at least  points are gained in every legal move.
\end{corollary}
\Proof
Consider a move . If , then  since .
Otherwise,  and therefore  has a neighbor  such that .
The value of  itself decreases by at least . Hence if , then  for any value function.
Otherwise , and we can choose  such that , gaining , and then .
\QED
In fact, the algorithm will define the value function in such a way, namely, it will ensure that every move (including Staller moves) gains at least  points.

We now formulate a useful condition on strategies. 
Denote the average gain per move (over the entire game) by

\smallskip
\par\noindent {\bf The average gain condition:}
The average gain per move satisfies .
\begin{claim}
\label{claim:avg_5}
In a Dominator-start game, the average gain condition is equivalent to Conjecture \ref{conjecture:3_5}.
\end{claim}
\Proof
As  and , we have . 
Therefore,  is equivalent to 
,
 which yields .
\QED

Next, denote the average gain over steps  by 
~~~~~ 
\smallskip
\par\noindent {\bf The shifted average gain condition:}
Excluding the first move, the average gain satisfies .
\begin{claim}
\label{claim:staller_start_cond}
In a Staller-start game, the shifted average gain condition implies
Conjecture \ref{conjecture:3_5}.
\end{claim}
\Proof
By Claim \ref{claim:3pts}, , and if we use a value function  satisfying  for all , then 
 for all .
Assume .
Then  and , and 

As  and , we have
, or 
,
establishing the claim.
\QED

\paragraph{Removing vertices and edges.}
Recall that red vertices are illegal moves and cannot be played, and are also worth  points. Therefore we have the following.

\begin{observation}
Red vertices can be removed from the graph along with their edges, without changing the outcome of the game.
\end{observation}

By definition, each blue vertex  has at least one white neighbor. 
Moreover,  is converted from blue to red exactly when its last white neighbor is converted to blue or to red, regardless of the states of its blue neighbors. 
Therefore we have the following.
\begin{observation}
Edges between two blue vertices can be removed from the graph without changing the outcome of the game.
\end{observation}

However, it may sometimes be useful for our algorithm to keep edges that have a  vertex as one of their endpoints.
The decision on whether to remove these edges or not will be made by the algorithm.

The algorithm maintains a graph called the \emph{underlying graph}, 
which contains only vertices and edges that may affect the outcome of the game.
This data structure also stores the decisions made by the algorithm about deleting edges between blue vertices,
and contains only the edges that were not deleted.
In particular, the algorithm ensures the following property, throughout the execution.

\begin{property}
\label{prop:underlying_graph}
The \emph{underlying graph} at the end of step , denoted , satisfies the following conditions.
\begin{enumerate}
	\dnsitem , and the vertices of  are labeled with the labeling  defined in Section \ref{section:dom_game_notations}.
	\dnsitem  = .
	\dnsitem  contains only edges that have at least one endpoint in  (this guarantees that both endpoints are in  by Observation \ref{obs:edges_no_r}),
	and contains all edges that have at least one endpoint in .
	\dnsitem .
\end{enumerate}
\end{property}

The following observation is an immediate result of the fact that edges are not removed as long as one of their endpoints is white.
\begin{observation}
\label{obs:w_neighborhood}
If , then .
That is, the neighborhood of a white vertex does not change as long as it is  white (except maybe for some of its white neighbors turning blue).
\end{observation}

\begin{corollary}
\label{cor:last_5}
The last move on a component gains at least  points.
\end{corollary}
\Proof
Since  is isolate-free, and the last move is either on or adjacent to some white vertex (by Claim \ref{claim:legal_moves}), 
we conclude from Observation \ref{obs:w_neighborhood} that the underlying graph contains at least one additional vertex (that is not red) adjacent to the move. 
Therefore the total gain is at least  points.
\QED

We want to define a single algorithm that will serve to prove the conjecture for both variants of the game.
The following corollary explains how this can be done.

\begin{corollary}
\label{cor:dom_stall_start}
Given an algorithm , which guarantees that the game ends within at most  moves in the Dominator-start variant of the game given any initial isolate-free forest where all vertices are high (and not necessarily white), it is possible to construct an algorithm  which guarantees that the game ends within at most  moves in the Staller-start variant of the game.
\end{corollary}
\Proof
The desired goal can be achieved by an algorithm  that sets the value function at the end of the first step as described in the proof of Claim \ref{claim:staller_start_cond}, and then invokes  for all the following moves (so that move  is considered by  as move  for all ).
This holds since the underlying graph  contains only high vertices, so the corollary follows from Claim \ref{claim:staller_start_cond}.
\QED

Hereafter we focus on finding an algorithm which achieves the desired gain for the Dominator-start variant of the game, and the conjecture will follow from Corollary \ref{cor:dom_stall_start}.


\paragraph{Structural notations.}
We use the following definitions.

\begin{definition}[White, blue, high subgraph]
We say a subgraph of  is white (respectively, blue, high) if all its vertices are white (respectively, blue, high).
Specifically,  is high.
\end{definition}

\begin{definition}[Tail, Subtail] 
Let , , be a path in ,
that is,  for every .
 is called a \emph{tail} if 
	 (i.e.,  is a leaf), and  for all .
We call  the \emph{tail lead}, and we say that  has a tail.
If , we say that  is a \emph{subtail}.
\end{definition}

\begin{figure}[thbp]
  \caption{\sf Graphical conventions used in our illustrations.}
  \medskip
  \centering
    \fbox{\includegraphics[width=9cm]{notations.jpg}}
  \label{fig:notations}
\end{figure}

\begin{figure}[thbp]
  \caption{\sf Split vertex  in general form.
Note that by our graphical conventions, the vertex  has degree  or higher, and  has degree  or higher, but  and  have degree exactly .}
  \medskip
  \centering
    \fbox{\includegraphics[height=1.5cm]{split_vertex.jpg}}
  \label{fig:splitP}
\end{figure}
	
\begin{definition}[Split vertex] 
\label{def:split_vertex}
A vertex of degree at least  is called a \emph{split} vertex if it has at least two tails. See Figure \ref{fig:splitP} (our graphical conventions are summarized in Figure \ref{fig:notations}).
\end{definition}
	
\begin{definition}[Path component] 
A component is called a \emph{path component} if all its vertices have degree  or  
(since the graph is a forest, there cannot be cycles).
Vertices on a path component that have degree  are called \emph{internal vertices} of the component.

Path components may be described by a sequence of the colors of their vertices. 
For example, when we refer to ``\emph{a path of the form }" we mean a path component of size  with vertices  such that  is ,  is white and  is high.
Specifically, we use the term ``\emph{ component}" to describe a component of size  containing one blue vertex and one white vertex. 
\end{definition}
	
\begin{definition}[Complex component] 
A component containing at least one split vertex is called a \emph{complex component}.
\end{definition}


\begin{claim}
\label{claim:every_tree_has_split}
Let  be a component and let  and  be vertices in  (not necessarily distinct).
If, when  is rooted at , the subtree  rooted at  is not a subtail, then  contains a split vertex.
\end{claim}
\begin{figure}[thbp]
  \caption{\sf A sample subtree  containing split vertices.}
  \medskip
  \centering
    \fbox{\includegraphics[width=7cm]{subtree_splits.jpg}}
  \label{fig:subtree_splits}
\end{figure}
\Proof
Let  be such a subtree. Let  be a vertex on  such that  (guaranteed to exist since  is not a subtail).
Let  be all the leaves of the subtree rooted at , and for each , let  be the first vertex of degree at least  on the (unique) path from  to , including the endpoints (see an example in Figure \ref{fig:subtree_splits}). 
Since ,  is guaranteed to exist for every .
Notice that not all  are distinct.
Let  be the  farthest from . 
Since , there are at least two leaves,  and , in its subtree.
Since  is the first vertex on the path from  to  that has degree at least , we conclude that  and therefore  has a tail towards .
Similarly, we conclude that  has another tail towards .
Therefore  has at least two tails, which means it is a split vertex.
\QED

\begin{corollary}
Every tree containing a vertex of degree  or more has at least one split vertex.
\end{corollary}

\begin{corollary}
Each (maximal connected) component is either a path component or a complex component.
\end{corollary}


\section{The algorithm}
\label{section:algorithm_outline}

\subsection{Outline}
\label{sub:alg_outline}

In order to prove Conjecture \ref{conjecture:3_5}, we show a possible course of action for every move that guarantees the average gain condition, namely, an average gain of  points or more in the Dominator-start variant of the game.
By Corollary \ref{cor:dom_stall_start}, if Dominator uses an algorithm guaranteeing this gain, then Dominator wins both in the Dominator-start variant and in the Staller-start variant.

We do not describe a specific algorithm in this section, but rather show that such an algorithm exists.
In Section \ref{sub:algorithmic_details} we present a concrete naive algorithm resulting from this outline, and in Section \ref{section:implementation} we discuss better implementations.
Section \ref{sub:simplified_alg_no_distance_4} contains a simplified version of this algorithm, that can be used on isolate-free forests in which no two leaves are at distance .

The suggested algorithm outline consists of several parts, performed for each move.
Suppose  moves () were already played, and the algorithm needs to decide on the st move (if it is a Dominator move), or preprocess for step  (if  is a Staller move).
\begin{enumerate}
	\dnsitem At the end of step , the current underlying graph, denoted by , undergoes a \emph{simulation} process consisting of two phases, each of which is described in detail later.
	\begin{itemize}
		\dnsitem {\bf Phase 1:}
		The graph is simplified by replacing subtrees of certain specific forms by virtual vertices (i.e., vertices that were not in ).
		The resulting (possibly smaller) graph is called the \emph{dense graph} and is denoted by .

		\dnssubitem {\bf Phase 2:}
		The resulting dense graph  is separated into \emph{boxes}, each of which is a connected subcomponent satisfying one of several properties.
		The process of separating the dense graph into boxes is called \emph{box decomposition}, and each vertex of the dense graph is assigned into a single box.
		We define Invariant  which must be satisfied by the box decompositions used by the algorithm.
		A box decomposition satisfying this invariant is called a \emph{valid box decomposition}. 
		
		As becomes clear later, a dense graph may have more than one valid box decomposition, 
		and we show in the analysis that it is possible to maintain the underlying graph such that the corresponding dense graph has at least one valid box decomposition.
		We say that the underlying graph  and the corresponding dense graph  are \emph{good} if  has a valid box decomposition, 
		and similarly we say that a component  of the dense or underlying graph is \emph{good} if a graph containing only this component is good.
	\end{itemize}

	\dnsitem If move  is performed by Staller, then the new underlying graph  is generated from  in a way that guarantees that at least  points are gained 
	by Staller's move , 
	and that the corresponding dense graph has a valid box decomposition.
	In the analysis, we show that an underlying graph satisfying these requirements can be generated from any good underlying graph and for any Staller move.

	\dnsitem 
	\label{alg_outline:choose_dom}
	Otherwise (move  is a Dominator move),
	move  is chosen (along with a corresponding underlying graph) greedily for Dominator from the vertices of ,
	such that the gain is maximal among all such moves which result in a good underlying graph .
	
	If several potential moves achieve the (same) maximal gain, ties are broken by choosing a move maximizing the minimal cumulative gain in the next three moves, 
	i.e., maximizing 
	
	where  is the maximal gain that can be achieved by Dominator in its following move (with a good underlying graph), and we define  for all .
	
	If there are still several such maximizing moves, then the tie is broken arbitrarily.
\end{enumerate}

It remains to describe the two phases of the simulation process.

\subsection{Phase  of the simulation: Creating the dense graph}
\label{sub:dense_graph}

\begin{figure}[thbp]
  \caption{\sf All unlabeled vertices are white.
	      (a) A triplet subtree on the underlying graph. 
	      (b) The corresponding subtree on the dense graph. The triplet vertices in the set  are the vertices that have numbers next to them. These numbers are their triplet depths.
	      The vertex  is the only triplet head in (a), and its triplet witnesses are ,  and .
	      The vertices in  in (a) are ,  and all other vertices that are adjacent to leaves.
	      Note that we assume that  does not have another neighbor in  except for  and .
	     }
  \medskip
  \centering
    \fbox{\includegraphics[height=4cm]{triplet_vertex.jpg}}
  \label{fig:t1_vertex}
\end{figure}

The dense graph is the result of removing subtrees called \emph{triplet witnesses} and replacing them with \emph{virtual leaves}.
The subtrees are constructed by the following process. 
Initially, set 


Next,
the family  of \emph{triplet vertices}, 
and the family  of \emph{potential triplet witnesses},
are constructed using the following iterative rule.
For every , 
we construct in parallel the sets  of triplet vertices and  of potential triplet witnesses.
For each vertex  we also define its \emph{triplet depth}, , and its \emph{triplet subtree}.
We define  and  iteratively as follows.
\bigskip
\par\noindent
Initially,
; ~~
.


\bigskip
\par\noindent
After defining  and :
\begin{itemize}
	\dnsitem Add to  every (blue or white) vertex  that has at least three neighbors in : \\
	.
			
	\dnsitem Add to  all vertices from  that are white and have degree exactly , and all vertices from : \\
	.

	\dnsitem
	For each  do:
	\begin{enumerate}
		\dnsitem Set the triplet depth of , .
		\dnssubitem Choose three \emph{triplet witnesses}, , from the vertices of  (which is guaranteed to contain at least three vertices),
		according to the following priorities (in this order of significance):
		\begin{enumerate}
			\dnssubitem Prefer witnesses from .
			\dnssubitem Prefer witnesses having lower triplet depth .
			\dnssubitem Prefer witnesses with higher  labels.
		\end{enumerate}

		The subtree containing  and its witnesses (and no other neighbors of ) is called the \emph{triplet subtree rooted at }.
	\end{enumerate}

\end{itemize}
Note that there is a maximal triplet depth  in the graph, and for all ,  (while ).
This is true since the graph's diameter upper bounds  for every .
If , then set .

See illustration in Figure \ref{fig:t1_vertex}.

\begin{definition}
Let  be a triplet vertex.
If  is not a triplet witness, then it is called a \emph{triplet head}
(note that  may still be a potential triplet witness that was not chosen as a witness).
\end{definition}

\begin{observation}
\label{obs:triplet_subtrees_white}
Let  be a triplet vertex. 
All vertices in the triplet subtree rooted at  are white, except (possibly) for  itself, which is either white or blue.
If  is not a triplet head, then  is white as well.
\end{observation}



\begin{claim}
\label{claim:triplet_vertex_implies_head}
Let  be a tree. If  contains a triplet vertex, then it contains a triplet head.
\end{claim}
\Proof
Consider the set , and let  be a vertex in  with maximal triplet depth (among the vertices of ).
By the way we define  we know that , 
and since  for all , we conclude that  is not a triplet witness, and therefore it is a triplet head.
\QED

\begin{definition}
A \emph{virtual vertex} or \emph{virtual leaf} is a white leaf with odd label  that exists only on the dense graph, and is adjacent to a vertex  that is a triplet head on the underlying graph.
A vertex that is not virtual is called \emph{real}, and each real vertex has at most one virtual neighbor.
\end{definition}

The dense graph is created by replacing all triplet witnesses of each triplet head, along with their entire subtrees, 
with a single virtual vertex colored white (see Figure \ref{fig:t1_vertex}(b)), thus converting each triplet subtree into a subtail of length .
This operation can be performed as follows:
\smallskip
\par\noindent{\bf Procedure \densify}:
\begin{enumerate}
	\dnsitem
	Calculate .
	
	/* \textit{Note that  does not contain virtual vertices.} */ 
	\dnsitem For each :
	\begin{enumerate}
		\dnsitem Disconnect the edges between  and its triplet witnesses, and remove the components containing the triplet witnesses.
		\dnssubitem Create a new (virtual) white leaf  and add an edge between  and .
	\end{enumerate}
	\dnsitem Return the resulting graph.
\end{enumerate}
The dense graph  results from invoking the procedure  on .


\subsection{Phase  of the simulation: Box decomposition}
\label{sub:alg_decomp}

In the second phase of the simulation, the algorithm decomposes the dense graph  into boxes, so that each vertex belongs to exactly one box.
We start by defining the boxes and their possible types.

\subsubsection{Box types}
\label{sub:box_types}

We now define a box, and the four possible box types.

\begin{definition}
\label{def:box_general}
Let  be the set of vertices in , and let  be a connected subset of vertices in the dense graph.
 is a \emph{box} in  if it satisfies the following requirements.
\begin{enumerate}
	\dnsitem  is of (at least) one of four types: \emph{regular}, \emph{dispensible}, \emph{high leftover} and \emph{corrupted}, which are defined below.
	\dnsitem  contains at most two  vertices.
	\dnsitem If  is not regular, then it has a blue vertex  called the \emph{box root}, and  does not have a neighbor in  that is a (white) leaf. 
\end{enumerate}
For a vertex , we define the \emph{internal neighbors} of  to be its neighbors inside the box, and the \emph{internal degree} of  to be the number of internal neighbors it has.
\end{definition}

From now on,
whenever we consider the degree or the neighbors of a vertex in a specific box, 
we mean its internal degree and its internal neighbors, 
except where specifically noted otherwise.

\begin{figure}[thbp]
  \caption{\sf (a) Dispensible box of type . 
	      (b) Dispensible box of type  where condition \ref{def:dispensible:d2:p1} holds.
	      (c) Dispensible box of type  where condition \ref{def:dispensible:d2:p2} holds.
	      Box roots are marked as .}
  \medskip
  \centering
    \fbox{\includegraphics[height=4cm]{dispensible.jpg}}
  \label{fig:dispensible}
\end{figure}

\begin{definition}
\label{def:dispensible}

There are two types of dispensible boxes.
\begin{enumerate}
	\dnsitem A box  is called \emph{dispensible of type }, denoted by , if it is a path  of the form , and the box root is .

	\dnsitem A box  of size  is called \emph{dispensible of type }, denoted by , if the following conditions hold.
	\begin{enumerate}
		\dnsitem The box root  is a  vertex of internal degree .
		\dnssubitem  has a high subtail of length .
		\dnssubitem The neighbor  of  that is not on the high subtail satisfies exactly one of the following conditions.
		\begin{enumerate}
			\dnsitem 
			\label{def:dispensible:d2:p1}
			 has internal degree , and it has two additional neighbors in ,  and , such that  is a high leaf, 
			and  is the () lead of a tail of the form  (note that this implies that  could be the root of a  box).
			\dnssubitem 
			\label{def:dispensible:d2:p2}
			 has internal degree , and it is the lead of a subtail of the form  (in this case as well, the  vertex on the tail could be the root of a  box).
		\end{enumerate}
	\end{enumerate}
\end{enumerate}
 
A box is called \emph{dispensible}, denoted by , if it is dispensible of type  or .
See Figure \ref{fig:dispensible} for illustrations.
\end{definition}

\begin{definition}
A box  in  is called a \emph{high leftover box} if all its vertices are high and it has a  root,
and additionally, it does not contain triplet subtrees. 
\end{definition}

\paragraph{}
There are several types of regular boxes, defined below.

\begin{figure}[thbp]
  \caption{\sf Boxes corresponding to the different types of regular colored boxes (not all requirements are illustrated).
		(a) A box satisfying Property :(\ref{invnp:1leaf}).
		(b) A box satisfying Property :(\ref{invnp:1tail}).
		(c) A box satisfying Property :(\ref{invnp:d1}).
		(d) A box satisfying Property :(\ref{invnp:2leaves}).
		(e) A box satisfying Property :(\ref{invnp:2tail}).}
  \medskip
  \centering
    \fbox{\includegraphics[width=14cm]{regular_colored.jpg}}
  \label{fig:regular_complex}
\end{figure}


\begin{definition}
\label{def:regular_colored}

A box  of size  or more is called a \emph{regular colored} box if it satisfies exactly one of the following two properties,  and , and additionally it satisfies Property  defined below.

\begin{description}
	\dnsitem[.]  contains a single  vertex, , such that at least one of the following conditions is satisfied. 
	\begin{enumerate}[(a)]
		\dnsitem \label{invnp:1leaf}
		 is a leaf on a subtail of a vertex , and  has a high subtail of length  or more and does not have white subtails of length  or . 
		\dnssubitem \label{invnp:1tail}
		 has a (high) subtail of length  or more, and no leaf neighbors. 
		\dnssubitem \label{invnp:d1}
		 is a leaf and  (i.e.,  is a dispensible box of type ).
	\end{enumerate}

	\dnsitem[.]  contains two  vertices,  and , 
	such that the internal degree of  is not greater than the internal degree of ,
	and at least one of the following conditions holds. 
	\begin{enumerate}[(a)]
		\dnsitem \label{invnp:2leaves}
		 and  are leaves of subtails of the same vertex , and  does not have a white leaf.
		\dnssubitem \label{invnp:2tail}
		 is a leaf of a subtail of , and  does not have leaf neighbors.
	\end{enumerate}
\end{description}

See Figure \ref{fig:regular_complex} for illustrations, and note that regular colored boxes do not necessarily contain a split vertex. 

\begin{tpropt1*}
Let  be a triplet vertex of depth  in a box  of the dense graph. 
Then for every three white tails of length  of  whose tail leads 
are not all in  (i.e., not all three tail leads are potential triplet witnesses in the underlying graph ), 
at least one vertex  in one of these tails is the parent of a box  whose box root has internal degree at most 
(i.e.,  is either a dispensible box of type , or a high leftover or corrupted box whose box root has at most one internal neighbor).
Note that this relates to all white tails of length  of , and not only the tails lead by the current triplet witnesses.
\end{tpropt1*}
\end{definition}

\begin{figure}[thbp]
  \caption{\sf (a)  box of Form . 
	      (b)  box of Form . 
	}
  \medskip
  \centering
    \fbox{\includegraphics[height=4cm]{c12.jpg}}
  \label{fig:c12_box}
\end{figure}

\begin{definition}
A box  is called a \emph{ box} if it contains exactly  vertices and is of one of the forms  or  (see Figure \ref{fig:c12_box}):
\begin{description}
	\dnsitem[.]
	 contains two high split vertices,  and , which are neighbors, and have the following tails:
	\begin{enumerate}[(a)]
		\dnsitem  has a  tail and two high tails of length .
		\dnssubitem  has a high leaf, and a  tail.
	\end{enumerate}

	\dnsitem[.]
	 contains a single high split vertex with exactly four tails of the following forms:
	\begin{enumerate}[(a)]
		\dnsitem Two high tails of length two.
		\dnssubitem A  tail.
		\dnssubitem A tail of length  of the form .

	\end{enumerate}
\end{description}
\end{definition}


\begin{definition}
\label{def:regular_box}
A \emph{regular box} is a box  that does not have a box root and satisfies exactly one of the following properties.
\begin{description}
	\dnsitem[.]  is of size .
	\dnsitem[.]  is high, i.e., does not contain  vertices, and it contains at least  vertices and satisfies Property .
	\dnsitem[.]  is a  box.
	\dnsitem[.]  is a regular colored box.
\end{description}

If  contains a split vertex and is not a  box, it is called a \emph{regular complex} box.
If  does not contain split vertices, it is called a \emph{regular path} box.
\end{definition}

\begin{definition}
A box  that is not dispensible, high leftover or regular is called a \emph{corrupted box}. 
\end{definition}


\subsubsection{The decomposition}

\begin{figure}[thbp]
  \caption{\sf A component of the dense graph and a valid box decomposition. Boxes are marked by dotted rectangles.}
  \medskip
  \centering
    \fbox{\includegraphics[height=8cm]{box_decomp.jpg}}
  \label{fig:box_decomp}
\end{figure}

In the second phase of the simulation, each maximal connected component  in the dense graph  is decomposed into boxes according to the following definition.
\begin{definition}
\label{def:box_decomp}
Let  be the set of vertices in , and let  be a partition of  into boxes, i.e., a collection of subsets of  satisfying Definition \ref{def:box_general}, such that:
\begin{enumerate}[(a)]
	{\setlength\itemindent{10pt} \dnsitem .}
	{\setlength\itemindent{10pt} \dnsitem  for every , .}
\end{enumerate}
 is called a \emph{box decomposition} of  if it satisfies the following properties.
\begin{description}
	\dnsitem[.] Each connected component of  contains at most one regular box.
	\dnsitem[.]  contains at most one corrupted box.
	\dnsitem[.] For each box  that is not regular, the box root  of  has at most one neighbor outside the box, and if such a neighbor  exists, then it is not a box root.
	The vertex  (if exists) is called 's \emph{parent}, and the box  containing  is called the \emph{parent box} of . 
	The single box in each component that does not have a parent is called the \emph{root box} of the component.
	\dnsitem[.] All parent boxes are of size  or more. 
	\dnsitem[.] The parent box of a high leftover box is not high.
	\dnsitem[.] Edges between boxes always connect a box root to its parent, and are called \emph{external edges}.  
\end{description}

\end{definition}

Component types are defined according to their root boxes. 
For example, a component whose root box is dispensible is called a dispensible component. 
One exception is that a component is called a \emph{corrupted component} if it contains a corrupted box anywhere in it (regardless of the type of its root box).

\paragraph{}
Finally, we define a semi-corrupted component.

\begin{definition}
A component  of the dense graph  is called \emph{semi-corrupted} if every one of its box decompositions contains a corrupted box, 
but there exists some  such that if , then there exist an underlying graph , a corresponding dense graph  and a box decomposition  of  that does not contain corrupted boxes, and the gain from playing  on  is at least  points.
\end{definition}

We are now ready to introduce the invariant that must be maintained by the algorithm.

\begin{tinvboxes*}
Let  be any step in the game.
\begin{description}
	\dnsitem[.] If move  is played by Dominator, then there exists a box decomposition  of the dense graph  that does not contain corrupted boxes.
	\dnsitem[.] If move  is played by Staller, then there exists a box decomposition  of the dense graph  that contains at most one corrupted box, 
	and if such a box exists then it is in a semi-corrupted component.
\end{description}
\end{tinvboxes*}

\begin{definition}
\label{def:good_graph_valid_box_decomp}

\begin{enumerate}
	\dnsitem The underlying graph  and the corresponding dense graph  are \emph{good} if  has a \emph{valid} box decomposition, i.e., a box decomposition satisfying Invariant .
	\dnsitem A component  of the dense graph  or the underlying graph  is \emph{good} if a graph containing only this component is good.
\end{enumerate}
\end{definition}

See Figure \ref{fig:box_decomp} for an example component on the dense graph and a valid box decomposition.

In Section \ref{section:analysis} we show that there exists an algorithm following the described outline, such that for every , if at the end of step  the dense graph is good (i.e., it satisfies Invariant ), and the average gain up to (and including) step  is at least  points, then the algorithm guarantees that the average gain at the end of 
some future step  
is at least  points.
This, in turn, guarantees also that the average gain at the end of the game is at least  points.


\subsection{Algorithmic details}
\label{sub:algorithmic_details}
The outline described in the previous subsections gives rise to the following naive (and highly inefficient) implementation.

\smallskip
\par\noindent{\bf Procedure :}
\begin{enumerate}
	\dnsitem Initialize the state:
	\begin{enumerate}
		\dnsitem  with a fixed vertex labeling (see Section \ref{section:dom_game_notations}.
/*  \textit{the underlying graph} */
		\dnssubitem . /* \textit{the constructed dominating set} */
		\dnssubitem  for every . /* \textit{the value function} */
		\dnssubitem . /* \textit{counter of moves} */
	\end{enumerate}
	
	\dnsitem While :
	\begin{enumerate}
		\dnsitem .
		
		\dnssubitem If  is odd: /* \textit{Dominator's turn} */
		\begin{enumerate}
			\dnsitem Create the dense graph  from  using Procedure  of Section \ref{sub:dense_graph}.
			\dnsitem . /* \textit{choose a move for Dominator and update the underlying graph and value function} */ 
		\end{enumerate}
		
		\dnssubitem Else: /* \textit{Staller's turn} */
		\begin{enumerate}
			\dnsitem Receive a legal move  from Staller.
			\dnsitem . /* \textit{update the underlying graph and value function} */
		\end{enumerate}
		\dnssubitem .
	\end{enumerate}
\end{enumerate}

The procedure  performs the following actions:
\begin{enumerate}
	\dnsitem For each choice of a pair  of an underlying graph  satisfying Property \ref{prop:underlying_graph} and a value assignment , 
	which may result from the move ,
	check if  with the value assignment  has a valid box decomposition.
	\dnsitem From the collection of pairs  which have a valid box decomposition,
	choose the pair  achieving the highest excess gain, i.e., maximizing  (breaking ties arbitrarily),
	and update  and  accordingly.
\end{enumerate}

The procedure  performs the following actions:
\begin{enumerate}
	\dnsitem .
	\dnsitem For each vertex :
	\begin{itemize}
		\dnsitem[] For each choice of a pair  of an underlying graph  satisfying Property \ref{prop:underlying_graph} and a value assignment , 
		which may result from the move :
		\begin{itemize}
			\dnssubitem[] If  with  has a valid box decomposition,
			then add  to .
		\end{itemize}
	\end{itemize}
	\dnsitem From the collection , 
	choose the sequence  achieving the highest gain, i.e., maximizing , breaking ties as described in Step \ref{alg_outline:choose_dom} in Section \ref{sub:alg_outline} (by checking all possible choices for the next three moves and choosing the move which maximizes the minimal gain over these moves).
	\dnsitem Update  and  according to  and .
	\dnsitem Return  as the selected move for step .
\end{enumerate}

\paragraph{}
Section \ref{section:implementation} contains a short discussion of more efficient implementations for the strategy outlined in Sections \ref{sub:alg_outline} through \ref{sub:alg_decomp}.



\subsection{Simplified algorithm for forests in which no two leaves are at distance }
\label{sub:simplified_alg_no_distance_4}

Conjecture \ref{conjecture:3_5} was proved in \cite{bujtas2015domination} for isolate-free forests in which no two leaves are at distance exactly , using a much simpler algorithm than the one described in this thesis.
We note that our algorithm can also take a simpler form when used on this family of graphs.
It may be instructive to consider this variant, in order to pinpoint the aspects of our algorithm that were needed in order to handle the possible existence of pairs of leaves at distance 4.

Observe that if no two leaves are at distance  from each other, then  does not contain triplet subtrees.
Therefore, there is no difference between the underlying graph  and the dense graph  (and since Property  refers to triplet subtrees, this property also becomes irrelevant).
Additionally, it is always possible for Dominator to make a move gaining at least  points on any regular colored or high box whose size is greater than  (this fact follows from Claims \ref{claim:b2_dom} and \ref{claim:dom_split_7} which appear later in the analysis of Dominator moves), and any Staller move on such boxes gains at least  points (by Claim \ref{claim:staller_regular} in the analysis of Staller moves),
and in both cases all resulting boxes can be regular boxes that are not  boxes.
As a result, there is no need to perform the box decomposition, and the following simpler algorithm suffices (compare this to the algorithm in Section \ref{sub:alg_outline}):

Suppose  moves () were already played, and the algorithm needs to decide on the st move (if it is a Dominator move), or preprocess for step  (if  is a Staller move).
\begin{enumerate}
	\dnsitem If move  is performed by Staller, then the new graph  is generated from  in a way that guarantees that at least  points are gained 
	by Staller's move , 
	and that all components of  are regular complex or regular path boxes.

	\dnsitem 
	Otherwise (move  is a Dominator move),
	move  is chosen greedily for Dominator from the vertices of ,
	such that the gain is maximal among all such moves which result in a graph  all of whose components are regular complex or regular path boxes. 
	If there are ties, they are broken arbitrarily.	
\end{enumerate}
Note that when all components are of size , all moves (by both players) gain at least  points.

\paragraph{}
The complex form of the algorithm for general isolate-free forests results from the fact that in some cases, Dominator must play on graphs that only contain split vertices with two or three tails, and all these tails are white tails of length . 
This prompted the creation of the dense graph (for handling triplet subtrees), as well as the addition of dispensible boxes, which contain subtrees that can be ignored by Dominator (i.e., that cannot reduce the gain of playing on their parent boxes. See Lemma \ref{claim:ana_box} for details). 
High leftover and corrupted boxes were added in order to handle Staller moves on these ``hidden" subtrees, while Property  and  boxes were added in order to make sure Dominator can always make moves that achieve the desired gain.
The analysis section that follows covers all possible moves on all types of boxes, but the core of the algorithm remains this simplified version.


\section{Analysis}
\label{section:analysis}

We separate the analysis into several parts, and show that Invariant  is always satisfied, and that an average gain of  points per move is achieved.
In Section \ref{sub:an_gain} we introduce sufficient conditions for guaranteeing Dominator's win.
In Section \ref{sub:an_simplify}  we show properties that simplify the case analysis,
and Section \ref{sub:special_subtrees} contains an analysis of two special subtrees that appear in many of the cases. 
We then analyze all possible moves and the resulting graphs.
The possible outcomes of Staller moves are analyzed in Section \ref{sub:an_staller}, 
and those of Dominator moves in Section \ref{sub:an_dom}.
Lastly, in Section \ref{sub:an_conclusion} we combine all the results to conclude that the algorithm outline proves Conjecture \ref{conjecture:3_5}.

\subsection{A policy for ensuring high average gain}
\label{sub:an_gain}

We have seen that it suffices to guarantee an average gain of  points per move (Claim \ref{claim:avg_5}), 
and also that the last move on any component gains at least  points (Corollary \ref{cor:last_5}).
Therefore, it suffices to guarantee that each pair of consecutive Dominator and Staller moves gains at least  points in order to make sure that Dominator wins the game.

\begin{definition}
\label{def:excess_gain}
Define the \emph{excess gain} of move  at step , denoted by , as follows.
\begin{itemize}
	\dnsitem If  is odd (Dominator plays ), then .
	\dnsitem If  is even (Staller plays ), then .
\end{itemize}
Additionally, define the \emph{cumulative excess gain} at step  to be the sum of excess gains in steps  through , and denote it by 

\end{definition}

\begin{observation}
\label{obs:sufficient_overhead}
Each of the following conditions is sufficient in order for Dominator to win.
\begin{enumerate}
	\dnsitem .
	\dnsitem .
	\dnsitem  is odd (i.e., Dominator plays ), and .
	\dnsitem  is even (i.e., Staller plays ), and . 
\end{enumerate}
\end{observation}
\Proof
 
\begin{enumerate}
	\dnsitem  is the condition in Conjecture \ref{conjecture:3_5}.

	\dnsitem If , then the game ended with an average gain of at least  points, which is a sufficient condition according to Claim \ref{claim:avg_5}.
	
	\dnsitem If  is odd and , then  is even and  points are gained in  moves. Therefore an average gain of  points is achieved throughout the game, which is a sufficient condition by Claim \ref{claim:avg_5}.
	
	\dnsitem If  is even and  then , since Corollary \ref{cor:last_5} guarantees that . Therefore this is a sufficient condition.
\QED
\end{enumerate}

\paragraph{}
The following guarantees are maintained throughout the execution, as will be shown in the analysis.
\begin{enumerate}
	\dnsitem Every Staller move gains at least  points, i.e.,  for all .
	\dnsitem  for all .
	\dnsitem The algorithm never relies on past gains, but rather on future gains. Namely, it guarantees that if  is negative at some point, then there will be positive excess gain in future moves to make up for it.
Therefore, if  for some , we can use the cumulative excess gain to convert  vertices to .

\end{enumerate}

\subsection{Preliminary properties simplifying the analysis}
\label{sub:an_simplify}

In this section we prove properties which will allow us to calculate a lower bound for the gain of playing a (real) vertex from the dense graph directly on the box containing it in the dense graph.
First, we describe the difference between playing on the dense graph and playing on the underlying graph.
Then, we prove that it suffices to analyze moves on the dense graph by analyzing them directly on the box that contains them.
Since Dominator moves are chosen from vertices of the dense graph, this includes all Dominator moves and all Staller moves that are on the dense graph, i.e., all moves that are not under triplet subtrees.
Staller moves that are not on the dense graph will be analyzed separately in Section \ref{sub:an_staller}.

\begin{figure}[thbp]
  \caption{\sf The graphs of Lemma \ref{claim:ana_dense}.}
  \medskip
  \centering
    \fbox{\includegraphics[width=7cm]{dense_graph_claim.jpg}}
  \label{fig:dense_graph_claim}
\end{figure}

\begin{lemma}
\label{claim:ana_dense}
Let  be a graph, and let  be the corresponding dense graph.
Let  be a (real) vertex which exists both in  and in .
If  points are gained when  is played directly on  (i.e., without invoking ) and the resulting graph (again, without invoking ) is , 
then  points can be gained when  is played on , yielding , and the dense graph  resulting from  is the same as  except that it may have (at most three) additional  components.
See Figure \ref{fig:dense_graph_claim} for an illustration of the graphs in the claim.
\end{lemma}
\Proof
Consider ,  and  as in the claim. 
Since , the only difference between the graphs  and  is the replacement of triplet subtrees from  with white leaves in .
Therefore, if  does not contain triplet vertices, then .
Assume that  contains at least one triplet vertex, and let  be a triplet head (guaranteed to exist by Claim \ref{claim:triplet_vertex_implies_head}).
Let  and  be the triplet witnesses of  in , and let  be the white leaf adjacent to  in  that is not in  
(the procedure  guarantees that exactly one such leaf exists, since it creates a single virtual leaf next to each triplet head).
Denote by  the set of vertices in the triplet subtree rooted at  in , excluding .
Observation \ref{obs:triplet_subtrees_white} guarantees that all vertices in  are white, and we know that  is white.
 is not in , and all vertices of  are not in , therefore by the claim's assumption,  cannot be any of these vertices.

If , then the above implies that all vertices of  (in ) and  (in ) remain white. Therefore  is a triplet head in  with the same triplet witnesses (this results from the way we choose triplet witnesses when there are more than three potential witnesses), which implies that  has one virtual (white) leaf neighbor in .
Note that  itself has the same color (and value) in  and in .

\begin{figure}[thbp]
  \caption{\sf (a) An example triplet subtree rooted at a vertex  in , with triplet witnesses  and . All unlabeled vertices are white.
		(b) The components of  resulting from the triplet subtree following the move .}
  \medskip
  \centering
    \fbox{\includegraphics[width=9cm]{triplet_subtree_move.jpg}}
  \label{fig:triplet_subtree_move}
\end{figure}

If , then  becomes red when  is played on , and adds exactly  points to the gain (since it is white in ).
In , each of the triplet witnesses  and  becomes a blue vertex in a separate component, and may be set as a  or  vertex.
For each , if  in , then  is in a  component in , and otherwise (i.e., if  is a triplet vertex in ),  is a blue triplet head of degree  in a component of . See Figure \ref{fig:triplet_subtree_move} for an example.
In both cases,  is a blue vertex in a  component in , and can be converted to  without violating Invariant .
Therefore exactly  points can be gained from the vertices of .

The above shows that each virtual leaf in  has a corresponding white leaf in , except possibly for three  components, and that each white leaf in  has a corresponding white leaf in .
The claim follows from the fact (mentioned above) that all other vertices (except for the vertices in the three discussed  components) are the same in both graphs.
\QED

We now prove that in some cases, it is possible to calculate a lower bound for the gain of a move  by calculating its gain on another, simpler graph.
Specifically, this simpler graph can be any graph that contains only the vertices of the box  that contains , in some valid box decomposition  of .
We use this important property later in the analysis of Dominator and Staller moves.

\begin{lemma}
\label{claim:ana_box}
Suppose  is good.
Let  be a vertex in a box  of a valid box decomposition  of  such that all boxes in  are not corrupted, 
and consider the graph  that contains a single component with a box decomposition  containing a single box  that is identical to .
Suppose  points can be gained by playing  in  and the resulting underlying graph  is good. 
Then it is possible to play  in  so that it leads to a good .
More precisely, the following properties hold.
\begin{description}
	\dnsitem[.]
	If  does not contain a semi-corrupted component, and one or more of the following conditions is satisfied:
	\begin{enumerate}
		\dnsitem 
		\label{claim:ana_box:nocorhigh:rootb}
		 is a root box in .
		\dnssubitem 
		\label{claim:ana_box:nocorhigh:boxr}
		 is the box root of .
		\dnssubitem 
		\label{claim:ana_box:nocorhigh:nodense}
		The box root  of  does not exist in  (i.e.,  becomes red. Note that this may occur even if ).
		\dnssubitem 
		\label{claim:ana_box:nocorhigh:yesroot}
		The box root  of  exists in  (i.e.,  does not become red).
	\end{enumerate}
	Then at least  points are gained by playing  in ,
	 and the resulting graph  is good (namely, it has a valid box decomposition) and does not contain semi-corrupted components.
	
	\dnsitem[.]
	If  does not contain a semi-corrupted component, and none of the above conditions (\ref{claim:ana_box:nocorhigh:rootb}) - (\ref{claim:ana_box:nocorhigh:yesroot}) are satisfied,
	then at least  points are gained by playing  in , 
	and the resulting graph is good and does not contain semi-corrupted components.

	\dnsitem[.]
	If  contains a semi-corrupted component  and one or more of the following conditions holds:
	\begin{enumerate}
		\dnsitem The corrupted box of  (under some valid box decomposition) is a root box of (a valid box decomposition of) . Note that this includes the case that  is a root box in .
		\dnssubitem Playing  on , where  is the box root of , gains at least  points in , and the resulting graph is good.
			\end{enumerate}
	Then at least  points are gained by playing  in ,
	 and the resulting graph is good (and may contain up to one semi-corrupted component).
\end{description}
\end{lemma}
Note that the classification of Lemma \ref{claim:ana_box} does not cover all cases in which  contains a semi-corrupted component, because not all cases are needed for the rest of the analysis.
\Proof
Assume all conditions of the claim are satisfied, and let  be a valid box decomposition of .
Consider an intermediate partition  of the vertices of  which is constructed by the following procedure.

\smallskip
\par\noindent{\bf Procedure }:
\begin{enumerate}
	\dnsitem For each box  whose vertices exist in  (i.e., such that none of its vertices become red in move ),
	add  to .
	Note that this includes all boxes of  excluding , and possibly excluding its parent box  (if exists) 
	as well as some high leftover boxes of size  from  whose root (and only) vertex becomes red.
	\dnsitem Add all boxes of  to .

	\dnsitem If  is not a root box in , and its box root  is in  but not in ,
	then add the box  of size  containing  to .
	Note that  is either high leftover (if  is high) or corrupted (otherwise).

	\dnsitem If  is not a root box in , and not all vertices of its parent box  are in , 
	then add the sets  containing all maximal connected subsets of  to .
	\dnsitem For every component  of size  in , 
	add the box  containing all vertices of  to  (if it is not already in ). 
\end{enumerate}
Note that  is not necessarily a valid box decomposition.
Also observe that  is a partition of the vertices of , and that all the sets added to  by the above five steps are boxes.
After performing Procedure , we invoke Operation  on the resulting graph  and disconnect all external edges in  that connect two blue vertices.
\smallskip
\par\noindent
{\bf Operation :}
\begin{itemize}
\dnsitem[] For every edge  that is external in , do:
	\begin{itemize}
		\dnsitem[] If  and  are both blue, remove .
	\end{itemize}
\end{itemize}
\bigskip
Note that after performing Operation  there are no parent boxes of size  in  (since box roots are blue).

We now consider the partition  under the three different settings specified in the claim, and describe in each setting how  can be modified into a valid box decomposition of , so  is good, while achieving the desired properties (i.e., a gain of at least  points with no corrupted boxes in Case , a gain of at least  points with no corrupted boxes in Case , and a gain of at least  points with at most one semi-corrupted component in Case ). This will imply the claim.
We rely on the following three claims.

\begin{claim}
\label{claim:ana_box:p1_bw}
All parent boxes of size  in  are of the form .
\end{claim}
\Proof
Let  be a parent box of size  in , 
and assume towards contradiction that  is of the form .
Recall that Observation \ref{obs:w_neighborhood} guarantees that all neighbors of each white vertex are still in the graph.
Also recall that Definitions \ref{def:box_general} and \ref{def:box_decomp} guarantee that each box that is not a root box contains a blue vertex (the box root), and that a parent box is of size  or more (Property ).
Since  is in  and  was constructed using Procedure , we conclude that  was added to  for one of the following three reasons.
The first option is that  was in . This is impossible, since then  would be a parent box of size , which contradicts Property  of Definition \ref{def:box_decomp}.
The second option is that  is in . This is impossible when  is white, since a box of the form  cannot be a parent box and cannot have a parent, which means that it must be in a component of size  in . This contradicts the fact that the component containing  in  must contain a blue vertex by Observation \ref{obs:w_neighborhood}.
The third option is that , where  is the parent box of  in . This is also impossible when  is white since, as in the previous case,  must contain a blue vertex by Observation \ref{obs:w_neighborhood}.
The claim follows.
\QED

\begin{claim}
\label{claim:ana_box:p1_root}
All boxes of size  in  are root boxes.
\end{claim}
\Proof
Let  be a box of size  in .
Denote by  and  the blue and white vertices of , respectively.
If  was a root box in , then it is clearly also a root box in .
Otherwise, we know that  was not a box root:
Assume towards contradiction that  was a box root in , and denote the box which contained  and  in  by .
Definition \ref{def:box_general} guarantees that box roots do not have neighbors that are white leave, therefore  was not a leaf in , and had a neighbor, .
Since  is white in , we conclude from Observation \ref{obs:w_neighborhood} that  is in  (and Procedure  guarantees that it is in ), in contradiction to the assumption that  is of size .
Therefore  was not a box root in , which means that it does not have a parent in .
In all cases, we conclude that  is a root box in .
\QED


\begin{claim}
\label{claim:ana_box:fix_p}
Assume that  points are gained, and the described partition  
does not contain corrupted boxes and satisfies all properties of Definition \ref{def:box_decomp} except, possibly, for Properties  and .
Then  is good, and  can be modified into a valid box decomposition of  that does not contain corrupted boxes, while gaining at least  points.
\end{claim}
\Proof
We start by handling the case that  does not satisfy Property , i.e., it contains a high box that is the parent of at least one high leftover box.
Consider the following operation.
\smallskip
\par\noindent
{\bf Operation :}
\begin{itemize}
\dnsitem[] While there are high boxes  and  such that  is the parent of , do:
	\begin{itemize}
		\dnsitem[] Remove both boxes  and  from  and replace them with the single box  that contains all vertices of  and .
	\end{itemize}
\end{itemize}
\bigskip
After performing Operation ,  satisfies Property , and the gain does not change.
It is possible that now  is a valid box decomposition of .
If this is not the case, it remains to modify  so that it satisfies Property .
From Claims \ref{claim:ana_box:p1_bw} and \ref{claim:ana_box:p1_root} we conclude that all parent boxes of size  in  are root boxes of the form , and we have seen earlier that after performing Operation   does not contain parent boxes of size .
Let  be a (root) parent box of size  in , and let  be another box in  such that  is the parent of .
Consider the following operation. See Figure \ref{fig:ana_box_to_graph} for illustrations.

\begin{figure}[thbp]
  \caption{\sf Examples of the results of Operation  for each pair of boxes  and .
		(a) The resulting box , if  is dispensible of type  or high leftover.
		(b) An example of the resulting boxes  and  when  is dispensible of type .}
  \medskip
  \centering
    \fbox{\includegraphics[width=9cm]{ana_box_to_graph.jpg}}
  \label{fig:ana_box_to_graph}
\end{figure}		

\smallskip
\par\noindent
{\bf Operation :}
\begin{itemize}
\dnsitem[] For every parent box  in , such that  is blue and  is white, do:
	\begin{enumerate}
		\dnsitem Find a box  in  such that  is the parent of , and denote the box root of  by .
		\dnsitem Remove both boxes  and  from  and replace them with the single box  that contains all vertices of  and .
		\dnsitem Convert  and  to  (if they are high).
		\dnsitem If  was a dispensible box of type , then:
		\begin{enumerate}
			\dnsitem Denote by  the  vertex of  that is not , and denote by  and  the two vertices in  that are on its high subtail.
			\dnssubitem Remove  from , and replace it with the following two boxes:
			The box , which contains all vertices of  except for  and , 
			and the box , which contains ,  and  with box root . 
		\end{enumerate}
	\end{enumerate}
\end{itemize}
\bigskip

Observe that Operation  may only increase the gain (if  or  is high).
We make the following two observations.
First, if  was a dispensible box of type  or a high leftover box, then  is a regular colored box, since it satisfies Property :(\ref{invnp:2tail}) of Definition \ref{def:regular_colored}
(see Case (a) in Figure \ref{fig:ana_box_to_graph}).
Second, if  was a dispensible box of type , then the box  is a regular colored box (again by Property :(\ref{invnp:2tail})), and the box  is dispensible of type  (see an example in Case (b) of Figure \ref{fig:ana_box_to_graph}).
We conclude that after performing Operation , the resulting  satisfies Definition \ref{def:box_decomp} and does not contain corrupted boxes, and therefore it is a valid box decomposition of , and that at least  points are gained.
\QED


For Cases  and , it remains to show that in each setting, the gain is as described in the theorem and  satisfies Properties , ,  and  of Definition \ref{def:box_decomp} before performing Operations  and , and then the theorem will follow from Claims \ref{claim:ana_box:p1_root} and \ref{claim:ana_box:fix_p}.
In Case , we cannot use Claim \ref{claim:ana_box:fix_p} directly.

\bigskip
\par\noindent
{\bf Case ():}
Let us first consider the setting of Case (), in which  does not contain a semi-corrupted component, and one of the four conditions (1)-(4) is satisfied.
We examine each of these conditions.
\smallskip
\par\noindent
{\bf Subcase ():}
	First, suppose Condition (\ref{claim:ana_box:nocorhigh:rootb}) holds, namely,  is a root box in . 
	Since box roots are always blue, external edges always connect a box root to its parent, and root boxes do not have parents (see Definitions \ref{def:box_general} and \ref{def:box_decomp}),
	we conclude that all external neighbors of vertices of  (i.e., neighbors that are in  but not in ) are blue box roots.
	This implies that if a vertex in  does not have internal white neighbors (i.e., white neighbors that are also in ), then it does not have any white neighbor.
	We conclude that  does not contain vertices that are in  but are not in  (i.e., vertices that are red in  but blue in ). 
	Additionally, if an external neighbor  does not have internal white neighbors in its box in , then it must be in a high leftover box of size , 
	which means (since box roots cannot be parents) that it does not have additional external neighbors except its parent.
	Therefore, the only case in which an external neighbor's color changes as a result of playing  is when a vertex in a high leftover box of size  becomes red, and in such a case  more points are gained and no additional boxes are modified.
	
	We make the following four observations regarding the properties of Definition \ref{def:box_decomp}.
	First, each connected component contains at most one regular box, i.e., Property  is satisfied.
	This is because all boxes from  that were not in the same component with  are in , 
	all boxes of  that were in the same component with  were not regular,
	and  is a valid box decomposition and therefore satisfies this condition.	
	Second,  does not contain corrupted boxes, and therefore Property  is satisfied.
	This is because the only potentially corrupted box is , generated in step  of Procedure , and from the previous paragraph we conclude that it does not exist when  is a root box. 
	Third, Property  holds, since all box roots have at most one external neighbor, and if such a neighbor exists then it is not another box root.
	This is because all external edges connecting blue vertices were disconnected in Operation  (and box roots are blue).
	Fourth, all external edges connect a box root to its parent, 	
	because all external edges of  are either external edges in , or were external edges in .  
	Therefore Property  holds.

	We conclude from Claim \ref{claim:ana_box:fix_p} that  can be modified into a valid box decomposition of  while preserving a gain of  points or more, and that  does not contain semi-corrupted components, which is what we wanted to prove.
	
\smallskip
\par\noindent
{\bf Subcase ():}
	Next, suppose Condition () of Case () holds, namely,
	 is not a root box, and  is the box root of .
	Since only the box root  had a parent in , and  is red, we conclude that all boxes from  are root boxes in .
	Therefore all components except, possibly, for components which contain vertices from the parent box , 
	satisfy Properties , ,  and , for the same reasons as in the previous subcase.
	It remains to handle the components that contain vertices from .
	Consider the boxes  in  which resulted from the parent box  in .
	Exactly one of the following cases occurs.
	\begin{enumerate}
		\dnsitem All vertices of  are in .
		Then the box  remains as it was, except that maybe the vertex  that was the parent of the box root  of  was converted from white to blue.
		Whether  was a regular, dispensible or high leftover box, it is still of the same type that it was, since all these properties are still satisfied if a single vertex is converted from white to  (see Section \ref{sub:box_types}). 
		Therefore in this case  satisfies Properties , ,  and . 
		
		\dnsitem Some vertex of  became red and is not in .
		Then at least  additional points were gained, and they can be used to convert all remaining vertices of  to high vertices (since each box contains at most two  vertices).
		Claim \ref{claim:ana_box:p1_root} guarantees that all resulting boxes of size  are root boxes, and we conclude that all properties of Definition \ref{def:box_decomp} are satisfied on these components as well.
	\end{enumerate}

	From Claim \ref{claim:ana_box:fix_p}, we conclude that  can be modified into a valid box decomposition of  while preserving a gain of  points or more, and that  does not contain semi-corrupted components.

\smallskip
\par\noindent
{\bf Subcase ():}
	Next, consider Condition () of Case (), namely,
	the box root  of  does not exist in  (i.e.,  becomes red), and assume that .
	Then all boxes of  are root boxes in  (since only the box root  could have a parent), and all components that do not contain vertices from  remain as they were in .
	As in the previous subcase, 
	we conclude that Properties , ,  and  are satisfied by , and therefore from Claim
	\ref{claim:ana_box:fix_p} we conclude that  is good and does not contain semi-corrupted components, and at least  points are gained.
	
\smallskip
\par\noindent
{\bf Subcase ():}
	Finally, suppose Condition () of Case () holds, i.e.,
	 is not a root box and the box root  of  exists in  (that is, it does not become red). Denote by  the box containing  in .
	Since box roots do not have internal neighbors that are white leaves, and Observation \ref{obs:w_neighborhood} guarantees that the neighborhood of a white vertex remains as it was, 
	we conclude that  is not of size .
	Therefore Properties  and  are satisfied in . 
	Properties  and  are satisfied for the same reasons as in Subcase ().
	We conclude from Claim \ref{claim:ana_box:fix_p} that  has a valid box decomposition that does not contain corrupted boxes, and that at least  points are gained.

\smallskip
\par\noindent
In all the above subcases, the gain in  is at least as high as the gain in  and the resulting graph is good (with no semi-corrupted components).
Hence Case () follows.

\bigskip
\par\noindent
{\bf Case ():}
We now turn to Case  of the claim, in which  is not a root box, and the box root  is in  but not in .
Consider the following operation.
\smallskip
\par\noindent
{\bf Operation :}
\begin{itemize}
\dnsitem[] Convert the root box  of  to .
\end{itemize}
The difference between the gain in , and the gain in  after performing Operation  (if  was ), is at most  points.
Since  was a box root and box roots cannot be parents, the box  containing  in  is a high leftover box of size  that is not the parent of another box, 
and its parent box  remains as it was in .
We conclude that the component  containing  satisfies all properties of Definition \ref{def:box_decomp} except, possibly, for Property  (i.e.,  may also be a high box).
All other components can be analyzed as in Case () above, and therefore we conclude from Claim \ref{claim:ana_box:fix_p} that  can be converted into a valid box decomposition of  while gaining at least  points, and that  does not contain semi-corrupted components.

\bigskip
\par\noindent
{\bf Case ():}
Finally, we consider Case . 
If  contains a semi-corrupted component  and one of the specified conditions holds, then at least one of the conditions of Case  above is satisfied (except that now one of the resulting components may contain a corrupted box).
All components except, possibly, for the component  containing vertices from , satisfy the conditions of Claim \ref{claim:ana_box:fix_p}, and therefore  can be modified so that its restriction to these components is valid, while preserving the number of points gained on them.
It remains to handle .
Property  of Definition \ref{def:box_decomp} guarantees that the root box of  that is in  is a box  of size at least  in ,
therefore Property  is also satisfied by . 
Since  contains a single corrupted box, we conclude that  satisfies Properties , ,  and  on  for the same reasons as in Case () above.
Therefore, after performing Operation ,  is a box decomposition of  and at least  points are gained. 

It remains to check whether  is semi-corrupted.
In order for  to be semi-corrupted (and for  to be a valid box decomposition of ), we need to check if it contains a move  gaining at least  points with a good resulting graph .
Since  is semi-corrupted, it contains a move  gaining at least  points when played in .
Under each of the conditions of the claim, one of the conditions of Case  is satisfied for step  (since the move  in question is either a box root, or on a root box), and we conclude that at least  points can be gained when playing on  so that  is good. Therefore  is semi-corrupted, as desired.

This concludes the proof of Lemma \ref{claim:ana_box}.
\QED

\subsection{Two special subtrees}
\label{sub:special_subtrees}

In this section we focus attention on two types of subtrees that occur frequently in subsequent analysis, 
and describe how the algorithm may cope with such subtrees, and what moves can be used in the analysis.
We start by defining the subtrees.

\begin{figure}[thbp]
  \caption{\sf An example of a fix vertex (marked as ) and its fix subtree.}
  \medskip
  \centering
    \fbox{\includegraphics[height=1.5cm]{fix_vertex.jpg}}
  \label{fig:tu_vertex}
\end{figure}
\begin{definition}
A vertex  is called a \emph{fix} vertex if the following conditions hold.
\begin{enumerate}
	\dnsitem  has (at least) one neighbor that is a  leaf.
	\dnsitem  has (at least) two high tails of length .
	\dnsitem There is a valid box decomposition where none of the vertices in these tails are box roots. 
		Note that this implies that  is white.
\end{enumerate}
If  has at most one additional white neighbor, and this neighbor (if exists) is not the lead of a white tail of length  or , 
we say that  is a \emph{strong fix} vertex. 

The subtree containing  and the three specified tails is called the \emph{fix subtree rooted at }.
See Figure \ref{fig:tu_vertex}.
\end{definition}

\begin{figure}[thbp]
  \caption{\sf Semi-triplet split vertices (marked as ) and their semi-triplet subtrees.}
  \medskip
  \centering
    \fbox{\includegraphics[width=8cm]{semi_triplet_split.jpg}}
  \label{fig:semi_triplet_split}
\end{figure}
\begin{definition}
A high split vertex  is called a \emph{semi-triplet} vertex if the following conditions hold. 
\begin{enumerate}
	\dnsitem  has three high tails of length , and there is a valid box decomposition where none of the vertices in these tails are box roots. 
	\dnsitem  is not a triplet vertex (i.e., not all vertices in the tails are white).
\end{enumerate}

If  has at most one additional white neighbor, and this neighbor (if exists) is not the lead of a white tail of length  or , 
we say that  is a \emph{strong semi-triplet} vertex.

The subtree containing  and the three specified tails is called the \emph{semi-triplet subtree rooted at }. 
See Figure \ref{fig:semi_triplet_split}.
\end{definition}


\begin{lemma}
\label{lemma:fix_subtree_gain}
Let  be a box in a valid box decomposition of the dense graph .
If  contains a fix subtree before Dominator's move, then Dominator gains at least  points in the following move, 
where  is the number of points needed in order to convert the resulting graph to a good graph, and .
If  contains a strong fix subtree, then Dominator gains at least  points.
\end{lemma}
\Proof
Let  be a box in  as described.
If  contains a fix subtree, consider the move  which is a lead of a high tail of length  adjacent to .
Playing  converts at least  vertices ( and its adjacent leaf, and the  leaf) to red, 
which gains at least  points, and no new  vertices are created.
Since  contains at most two  vertices, at most two points 
are needed in order to convert all resulting boxes to high, and therefore .

For a strong fix subtree, if  points are gained then no additional vertices became red (otherwise the gain would be greater than ). 
Therefore, if  is converted to , one of the following cases occurs.

\smallskip
\par\noindent
{\bf Case ():}
 is a strong fix vertex with a white neighbor that is the lead of a tail of length  that is not white. 
Then the resulting box is a path of the form , and it can be converted to a regular colored path box of the form  (if it is not already so).

\smallskip
\par\noindent
{\bf Case ():}
Either  does not have additional neighbors, or its neighbors are not leads of tails of length .
After converting all  vertices except  in the box containing  to , and possibly disconnecting edges between  and its blue neighbors (so that  has at most one neighbor), 
 and the remaining subtail from the fix subtree can be separated into a dispensible box of type , with  as the box root.

\smallskip
\par\noindent
In both cases, at least  points are gained.
\QED

\begin{lemma}
\label{lemma:semi_triplet_subtree_gain}
Let  be a box in a valid box decomposition of the dense graph .
If  contains a semi-triplet subtree before Dominator's move, then the following properties hold.
\begin{enumerate}
	\dnsitem If the semi-triplet subtree has a  leaf, then at least  points are gained in the following move,
	where  is the number of points needed in order to convert the resulting graph to a good graph, and .
	\dnsitem If the semi-triplet subtree is strong, then at least  points are gained in the next move.
	\dnsitem Otherwise, at least  points are gained.
\end{enumerate}
\end{lemma}
\Proof
Let  be a box in  as described that contains a semi-triplet subtree rooted at a vertex .
We analyze the different cases.
\smallskip
\par\noindent
{\bf Case ():}
The semi-triplet subtree rooted at  contains a  leaf, .
If  is played, then at least  points are gained from the vertices that become red (,  and the neighbor of ) and from converting  vertices in resulting components of size  to . We note that if less than two  components are created as a result of playing , then at least one additional vertex was converted to red and therefore at least  points are gained.
The claim follows, since  contains at most two  vertices and therefore . 

\smallskip
\par\noindent
{\bf Case ():}
The semi-triplet subtree rooted at  does not contain a  leaf.
Then it must contain a  tail lead, . 
Let  be another tail lead. As a result of playing , at least  points can be gained from the resulting red vertices ( and the adjacent leaf), and from disconnecting  to a component of size  and converting it to . 
As before, , and at least  points are gained.

If the semi-triplet subtree is strong, then an additional point can be gained by converting  to :
If  does not have an additional white neighbor that is not in the triplet subtree, 
or if  has such a neighbor and it is not the lead of a subtail of length  or , 
then  can be a box root of a dispensible box of type  (possibly after disconnecting edges between  and its blue neighbors). 
Otherwise,  has a white neighbor that is the lead of a subtail of length , and this subtail is not white (note that  cannot have a white leaf neighbor by the definition of strong semi-triplet). 
We conclude that  is in a box of the form , and therefore it can be converted to a regular colored path box by converting all  vertices to .
\QED


\subsection{Results of Staller moves}
\label{sub:an_staller}

In this section we analyze all possible Staller moves, i.e., the result of Staller playing any vertex  of the underlying graph.
Notice that  may be in the dense graph, or in a triplet subtree in the underlying graph.
Theorem \ref{thm:staller_good} summarizes all the possible outcomes of Staller moves.

\begin{theorem}
\label{thm:staller_good}
If Staller plays on a vertex  in  and  is good, then at least  points are gained and the resulting underlying graph, , is good.
\end{theorem}

We separate the proof of the theorem into several claims, and note that Lemma \ref{claim:ana_box} guarantees that it suffices to analyze each move inside the box containing it (and calculate the gain on the underlying graph accordingly).

\paragraph{}

First, we extend the definition of box decomposition to the underlying graph in the following natural way:
\begin{definition}
A decomposition  of the set  of vertices of the underlying graph  is called a \emph{box decomposition}, if the decomposition  which results from  by replacing (without repetitions) each vertex of the underlying graph  that is not on the dense graph  with the virtual leaf that replaces them on  (i.e., the vertex  adjacent to the nearest triplet head) is a box decomposition of .
\end{definition}


\begin{claim}
\label{claim:staller_triplet}
If Staller plays on  a vertex  that is not on the dense graph , then at least  points are gained and the resulting dense graph  is good.
\end{claim}
\begin{figure}[thbp] 
  \caption{\sf All possible Staller moves on  (marked as ) on triplet witnesses as described in the proof of Claim \ref{claim:staller_triplet}.
	Each tail represents the subtree of a triplet witness, which is either a real tail (of the witness is in ), or a triplet subtree. 
	In all cases, (a) is the graph  before the move and (b) is the resulting graph .}
  \medskip
  \centering
    \fbox{\includegraphics[width=14cm]{triplet_staller.jpg}}
  \label{fig:triplet_staller}
\end{figure}
\Proof
If  is not on the dense graph , then it is in a triplet subtree rooted at a vertex , and  is in some box  on .
There are three cases to consider, illustrated in Figure \ref{fig:triplet_staller}.
\bigskip
\par\noindent
{\bf Case ():}
 is not a leaf.
Then at least  points can be gained from  and its triplet witnesses or its adjacent leaf, 
and all resulting components on  are  components (if  was in ), and the component containing .
Since  contains at most two  vertices, at least  points can be gained while converting the boxes resulting from  (that are not  components) to high, 
and therefore Invariant  is satisfied, so the resulting dense graph  is good.

\bigskip
\par\noindent
{\bf Case ():}
 is a leaf at distance  from , and  is blue.
Then at least  points are gained since  and its neighbor become red. Therefore, as before,
at least  points can be gained while satisfying Invariant .

\bigskip
\par\noindent
{\bf Case ():}
 is a leaf and the nearest split vertex  is white.
Therefore  points are gained, and the resulting box  contains a fix vertex ().
If  is corrupted, then at least one of the following three subcases occurs.

\smallskip
\par\noindent
{\bf Subcase ():}
 was not a triplet head in . Then  is a strong fix vertex in , and Lemma \ref{lemma:fix_subtree_gain} guarantees that at least  
points can be gained while converting  to a high box, and therefore the component is semi-corrupted. Therefore Invariant  is satisfied by the resulting dense graph .

\smallskip
\par\noindent
{\bf Subcase ():}
 contained a single  vertex. Then Lemma \ref{lemma:fix_subtree_gain} guarantees that at least  points can be gained while converting  to a high box, and as in the previous item, this implies that Invariant  is satisfied by the resulting dense graph .

\smallskip
\par\noindent
{\bf Subcase ():}
 contained two  vertices.
This subcase splits further into three.
\begin{description}
	\dnsitem[Subcase ():] If  was a dispensible box of type , then  contains a strong fix vertex 
	(since all white vertices in the dense graph had at most one white neighbor, and if this neighbor was a virtual leaf then they have no additional white neighbors). 
	From Lemma \ref{lemma:fix_subtree_gain} we conclude that  is semi-corrupted, therefore Invariant  is satisfied by the resulting dense graph .
	\dnsitem[Subcase ():] If  was a  box, then again  contains a strong fix vertex (since all neighbors of white leaves in the dense graph, i.e., all vertices that could be triplet vertices in the underlying graph, have at most one white neighbor, and this neighbor is not the lead of a white tail of length  or ). 
	\dnsitem[Subcase ():]If  was a regular colored box with two  vertices, then  is not corrupted, and therefore this case can be ignored.
\end{description}
Note that the above subcases cover all graphs in which  is corrupted, for the following reasons:
If  was a high leftover or high regular box then  is not corrupted.
Additionally, if  was a regular colored box satisfying Property  (Subcase ()), then  also satisfies it.
Finally, we know that  was not corrupted because Invariant  guarantees that there are no corrupted boxes before Staller's move.
\QED

\begin{claim}
\label{claim:staller_dispensible}
If Staller plays on  a vertex  that is in a dispensible box, then at least  points are gained and the resulting dense graph  is good.
\end{claim}
\begin{figure}[thbp]
  \caption{\sf All possible moves on dispensible boxes.
		(a) Dispensible box of type .
		(b), (c) Dispensible box of type .}
  \medskip
  \centering
    \fbox{\includegraphics[width=12cm]{dispensible_staller.jpg}}
  \label{fig:dispensible_staller}
\end{figure}

\begin{figure}[hbtp]
  \caption{\sf Possible boxes resulting from Staller moves on dispensible boxes.}
  \medskip
  \centering
    \fbox{\includegraphics[width=7cm]
{dispensible_staller_sc.jpg}}
  \label{fig:dispensible_staller_sc}
\end{figure}
\Proof

Let  be a dispensible box.
The possible moves are shown in Figure \ref{fig:dispensible_staller}.

\smallskip
\par\noindent
{\bf Case (a):}
 is dispensible of type .
Then Staller has two choices.
\begin{enumerate}
	\dnsitem If Staller plays , then at least  points are gained from  and , and a single  component results. From Lemma \ref{claim:ana_box} we know that Invariant  is satisfied on . 
	\dnsitem If Staller plays  or , then from Lemma \ref{claim:ana_box} we know that at least  points are gained and the resulting graph  is good.
\end{enumerate}

\smallskip
\par\noindent
{\bf Cases (b) and (c):}
 is dispensible of type .
Then Staller has three types of choices.
\begin{enumerate}
	\dnsitem If Staller plays a high vertex, i.e., , , , , , , , , , ,  or ,
	then at least  points are gained and the resulting boxes are high, and therefore Invariant  is satisfied.
	\dnsitem If Staller plays ,  or , then at least  points are gained, and the resulting boxes are dispensible, regular of size , and if  is played then regular colored path (see Case (1) in Figure \ref{fig:dispensible_staller_sc}).
	\dnsitem If Staller plays , then at least  points are gained and the resulting box is a path of the form  (see Case (2) in Figure \ref{fig:dispensible_staller_sc}). The containing component is semi-corrupted since Dominator can play on the middle  vertex of this box (which is the box root) and gain at least  points. 
\QED
\end{enumerate}

\begin{claim}
\label{claim:staller_clean_leftover}
If Staller plays on  a vertex  that is in a high leftover box, then at least  points are gained and the resulting dense graph  is good.
\end{claim}
\Proof
If  is in a high leftover box, then at least  points are gained from  becoming red, and all resulting boxes are high and do not contain triplet subtrees.
Therefore there is a box decomposition satisfying Invariant  on the resulting dense graph .
\QED

\begin{claim}
\label{claim:staller_regular}
If Staller plays on  a vertex  that is in a regular box, then at least  points are gained and the resulting dense graph  is good.
\end{claim}
\Proof
Let  be a regular box.
There are four cases to consider, corresponding to the four categories in Definition \ref{def:regular_box}.

\bigskip
\par\noindent
{\bf Case :}
 is of size .
Then any move on  gains at least  points, and Lemmas \ref{claim:ana_dense} and \ref{claim:ana_box} guarantee that the resulting graph is good.

\bigskip
\par\noindent
{\bf Case :}
 is a high regular box.
Then any move gains at least  points and all resulting boxes can be high.

\begin{figure}[t]
  \caption{\sf All possible moves on a  box. The dotted edges correspond to the two types of  boxes, and exactly one of them exists.}
  \medskip
  \centering
    \fbox{\includegraphics[width=5cm]{c12_staller.jpg}}
  \label{fig:c12_staller}
\end{figure}

\bigskip
\par\noindent
{\bf Case :}
 is a  box.
The possible moves are described in Figure \ref{fig:c12_staller}.

If Staller plays , 
then the remaining box is semi-corrupted since Dominator can play  in the next move and gain at least  points from the red vertices and from converting  to , 
and the remaining box is a dispensible box of type . 
See Case (a) in Figure \ref{fig:c12_staller_sc}.

\begin{figure}[thbp]
  \caption{\sf Possible semi-corrupted boxes resulting from Staller moves on  boxes, and moves  gaining at least  points.}
  \medskip
  \centering
    \fbox{\includegraphics[width=10cm]{c12_staller_sc.jpg}}
  \label{fig:c12_staller_sc}
\end{figure}

If Staller plays , then at least  points are gained and, after separating the vertices ,  and  to a dispensible box of type  rooted at ,
the resulting box is high and satisfies Property , and therefore it is regular.

If Staller plays , then at least  points are gained and all resulting boxes are regular colored path boxes and regular boxes of size .

If Staller plays one of the vertices , ,  or , then at least  points can be gained while converting the  vertices to high, and the resulting boxes are regular.

If Staller plays  or , then at least  points are gained and the resulting box (after separating ,  and  to a dispensible box of type  rooted at )
contains a strong fix vertex and a single  vertex (see Case (b) in Figure \ref{fig:c12_staller_sc}), and therefore it is in a semi-corrupted component. 

If Staller plays , then at least  points are gained from converting  to a  vertex in a  box, and  becomes a strong semi-triplet vertex in a corrupted box.
From Lemma \ref{lemma:semi_triplet_subtree_gain} we conclude that the box is semi-corrupted, since it is possible to gain at least  points in the following move (since ).

\bigskip
\par\noindent
{\bf Case :}
 is a regular colored box.
Case :(\ref{invnp:d1}) (i.e.,  is a dispensible box of type ) was already analyzed in Claim \ref{claim:staller_dispensible}, and therefore we ignore it.
Therefore exactly one of the following cases occurs (see Figure \ref{fig:regular_complex}).
Note that in all the following cases, we analyze the subtree rooted at some split vertex, and show that all the resulting boxes are regular. 
If an unexpected vertex becomes red, then at least two additional points are gained, and therefore the box containing this vertex can be converted to a high box. 
We therefore ignore this possibility in the case analysis.
We split the analysis into two subcases, as follows.

\smallskip
\par\noindent
{\bf Subcase :}
 is a  vertex. Then at least one of the following cases occurs.
	\begin{enumerate}
		\dnsitem  is the only  vertex and  satisfies Case :(\ref{invnp:1leaf}) of Definition \ref{def:regular_colored}. 
		Then at least  points are gained by converting the vertex  adjacent to  to . 
		If the resulting box does not satisfy Case :(\ref{invnp:1leaf}), then it must satisfy Case :(\ref{invnp:1tail}), and therefore it is a regular colored box.
		
		\dnsitem  is the only  vertex and  satisfies Case :(\ref{invnp:1tail}).
		Then at least  points are gained from  and from an adjacent subtail lead, and the resulting boxes are a high box, 
		and regular colored path boxes satisfying case :(\ref{invnp:1tail}) or :(\ref{invnp:d1}) 
		(and possibly additional  components). 
		
		\dnsitem  is a  leaf and  satisfies Case :(\ref{invnp:2leaves}) or Case :(\ref{invnp:2tail}). 
		Then, similarly to Case :(\ref{invnp:1leaf}), at least  points are gained from  and its neighbor, and if the resulting box cannot be converted to a high box while gaining at least  points, 
		then it satisfies one of the cases :(\ref{invnp:2leaves}) and :(\ref{invnp:2tail}).
		
		\dnsitem  is a non-leaf vertex and  satisfies Case :(\ref{invnp:2tail}).
		Then, similarly to Case :(\ref{invnp:1tail}), at least  points are gained from  and from an adjacent tail lead, and the resulting boxes are a high box, and regular colored path boxes 
		(and possibly additional  components). 
	\end{enumerate}

\smallskip
\par\noindent
{\bf Subcase :}
 is a high vertex. Then at least  points are gained from , and therefore no additional  vertices need to be created. At least one of the following cases occurs.
	\begin{enumerate}
		\dnsitem  contains a single  vertex, , satisfying Case :(\ref{invnp:1leaf}) of Definition \ref{def:regular_colored}.
		Let  be the vertex closest to  that has a subtail ending at  and another subtail (guaranteed to exist by the definition).
		Exactly one of the following cases occurs.
		\begin{enumerate}
			\dnsitem  is on the subtail of  that contains , or .
			Then at least  points are gained, 
			and the resulting boxes are high regular boxes, and possibly a regular colored path box with two  leaves.
			\dnsitem  is on another subtail of .
			Then at least  points can be gained, 
			and the resulting boxes are a regular colored box satisfying Case :(\ref{invnp:2leaves}) or Case :(\ref{invnp:2tail}), and possibly another regular colored path with a single  leaf.
			\dnsitem  is not on a subtail of .
			Then at least  points are gained and the resulting box that is not high can satisfy Case :(\ref{invnp:1leaf}).
		\end{enumerate}
		
		\dnsitem  contains a single  vertex, , satisfying Case :(\ref{invnp:1tail}). Then exactly one of the following cases occurs.
		Notice that  is blue and therefore does not have neighbors that are blue leaves, which implies that it does not have leaf neighbors.
		\begin{enumerate}
			\dnsitem  is on a subtail of , and is at distance  or  from . 
			Then at least  points can be gained from  and its neighbors, which means that at least  points can be gained while converting the box containing  to a high box.
			\dnsitem  is on a subtail of , and is at distance  or more from .
			Then at least  points can be gained, and the resulting boxes are a regular colored box satisfying Case :(\ref{invnp:2tail}), and possibly a regular colored path with a single  leaf, or a  component.
			\dnsitem  is not on a subtail of . Then at least  points can be gained, and the resulting boxes can be high boxes and a regular colored box satisfying Case :(\ref{invnp:1tail}).
		\end{enumerate}
		
		\dnsitem  contains two  vertices,  and , that are leaves of subtails of a vertex , corresponding to Case :(\ref{invnp:2leaves}).
		Then exactly one of the following cases occurs.
		\begin{enumerate}
			\dnsitem  is on a subtail of  that contains a  vertex, or . Then at least  points can be gained on the subtail, 
			and at least  points can be gained while converting the other  leaf to .
			In this case, all resulting boxes are high boxes and possibly a regular colored path with two  leaves.
			\dnsitem Otherwise, at least  points are gained and the resulting box that is not high still satisfies Case :(\ref{invnp:2leaves}).
		\end{enumerate}
		
		\dnsitem  contains two  vertices,  and , such that  is a leaf on a subtail of , corresponding to Case :(\ref{invnp:2tail}).
		Then exactly one of the following cases occurs.
		\begin{enumerate}
			\dnsitem  is on the subtail of  that contains . Then at least  points can be gained on the subtail, 
			which means that at least  points can be gained while converting  to .
			In this case, all resulting boxes are high, and possibly a regular colored path with two  leaves.
			\dnsitem Otherwise, at least  points are gained and the resulting box that is not high still satisfies Case :(\ref{invnp:2leaves}).
\QED
		\end{enumerate}
	\end{enumerate}

Theorem \ref{thm:staller_good} follows from Claims \ref{claim:staller_triplet}, \ref{claim:staller_dispensible}, \ref{claim:staller_clean_leftover} and \ref{claim:staller_regular}.

\subsection{Dominator moves}
\label{sub:an_dom}

We have seen that if Staller plays on a good graph , 
then the resulting graph is also good.
Our goal in this section is to prove the following theorem.

\begin{theorem}
\label{thm:dom_gain}
Let .
If Dominator plays on a vertex  in  and  is good, and Dominator chooses all moves greedily according to the guidelines in Section \ref{section:algorithm_outline},
then the resulting graph  is good, and at least one of the following properties holds.
\begin{enumerate}
	\dnsitem .
	\dnsitem  and .
	\dnsitem  and .
\end{enumerate}
\end{theorem}

Notice that we do not make requirements about the last move (step ) because of Corollary \ref{cor:last_5}, and therefore in all the following claims we only consider  such that .

The definition of semi-corrupted components guarantees that if a semi-corrupted component is created, then in the following Dominator move , and therefore we focus on the case that there is a box decomposition that does not contain corrupted boxes.
Let  be a box decomposition of  that does not contain corrupted boxes. 

\begin{claim}
\label{claim:c12_dom}
If  contains a  box, then  and the resulting graph  is good.
\end{claim}
\Proof
Recall that all possible moves appear in Figure \ref{fig:c12_staller}. 
If Dominator plays , then at least  points are gained and the resulting boxes, after disconnecting the edge between  and , are a  box and a dispensible box of type .
\QED

\begin{claim}
\label{claim:b2_dom}
If  contains a regular colored box (including a dispensible component of type ),
then  and the resulting graph  is good.
\end{claim}
\begin{figure}[thbp]
  \caption{\sf Possible moves (marked as ) gaining at least  points on regular colored boxes containing  vertices, corresponding to the different cases in the proof of Claim \ref{claim:b2_dom}.
		(a) Case (:\ref{invnp:1leaf}). 
		(b) Case (:\ref{invnp:1tail}). 
		(c) Case (:\ref{invnp:d1}).
		(d) Case (:). 
		(e) Case (:).}
  \medskip
  \centering
    \fbox{\includegraphics[width=14cm]{regular_colored_b2_dom.jpg}}
  \label{fig:regular_colored_b2_dom}
\end{figure}
\Proof
Let  be a regular colored box.
The following cases cover all possibilities for .

Except where noted otherwise, we assume all high vertices are white. If an unexpected vertex becomes red, then at least two additional points are gained, and the resulting box can be converted to a high box. Therefore we ignore this possibility in the analysis.

See Figure \ref{fig:regular_colored_b2_dom} for illustrations, and recall that all possible regular colored boxes are illustrated in Figure \ref{fig:regular_complex}.
We separate the analysis into cases according to the different properties of Definition \ref{def:regular_colored}, as follows.
\bigskip
\par\noindent
{\bf Case (:\ref{invnp:1leaf}):}
 satisfies Case :(\ref{invnp:1leaf}) of Definition \ref{def:regular_colored}.
Let  be a vertex with a subtail containing a  leaf and a high subtail of length  or more.
By playing on the neighbor of the leaf on the high subtail,
at least  points can be gained and the resulting box satisfies Case :(\ref{invnp:2leaves}).

\bigskip
\par\noindent
{\bf Case (:\ref{invnp:1tail}):}
 satisfies Case :(\ref{invnp:1tail}). 
Let  be the  vertex. 
By playing on the neighbor of the leaf on a subtail of  of length  or more, exactly one of the following cases can result.
\begin{enumerate}
	\dnsitem At least  points are gained, and the resulting box satisfies Case :(\ref{invnp:2tail}).
	\dnsitem At least  points are gained after converting  to , and the resulting box is high.
\end{enumerate}

\smallskip
\par\noindent
{\bf Case (:\ref{invnp:d1}):}
 satisfies Case :(\ref{invnp:d1}). 
By playing on the middle vertex , 
all vertices of the box are eliminated and at least  points are gained.

\bigskip
\par\noindent
{\bf Case ():}
 satisfies Case :(\ref{invnp:2leaves}) or Case :(\ref{invnp:2tail}).
This splits further into the following two subcases.

\smallskip
\par\noindent
{\bf Subcase ():}
	There is a  leaf on a subtail of length  of some vertex . Since the leaf is a  vertex, its neighbor must be white.
	Exactly one of the following cases results from playing .
	\begin{enumerate}
		\dnsitem  is high. Then at least  points can be gained while converting the remaining  vertex in the box to high.
		\dnsitem  is not high. Then at least  points are gained and the resulting boxes do not contain  vertices.
	\end{enumerate}

\par\noindent
{\bf Subcase ():}
	Otherwise, there must be two  leaves that are neighbors of the same vertex, .
	Playing  gains at least  points, and the resulting boxes are high.
\QED

\begin{claim}
\label{claim:dom_split_7}
If  contains a high regular box  of size  or more (including a high leftover root box), and  is either a path or contains a split vertex  satisfying one of the following requirements:
\begin{enumerate}
	\dnsitem 
	\label{claim:dom_split_7:4tails}
	 has four tails or more.
	\dnsitem 
	\label{claim:dom_split_7:not2}
	 has a tail that is not of length .
	\dnsitem
	\label{claim:dom_split_7:b3}
	 has a tail of length  containing a  vertex.
\end{enumerate}
Then  and the resulting graph  is good.
\end{claim}
\begin{figure}[thbp]
  \caption{\sf Possible moves (marked as ) gaining at least  points on high regular complex boxes, corresponding to the different cases in the proof of Claim \ref{claim:dom_split_7}.
		When there are two items, (a) is before playing  and (b) is the result.
		(1) Case (a).
		(2) Case (b). 
		(3) Case (c:1). 
		(4) Case (c:2). 
		(5) Case (d:1). 
		(6) Case (d:2).} 
  \medskip
  \centering
    \fbox{\includegraphics[width=11cm]{dom_clean.jpg}}
  \label{fig:dom_clean}
\end{figure}
\Proof
First, observe that if  contains a high path of length  or more, then playing on the neighbor of a leaf on this path can gain at least  points, 
and if a box remains, it is a path with a  leaf and no other  vertices, and therefore it is a regular box. 
Otherwise, let  be a high regular complex box, and  a split vertex in , as described.
As before, we assume all high vertices are white, since if an unexpected vertex becomes red, at least two additional points are gained and the box containing the red vertex can be converted to a high box.
We separate the analysis into cases according to the different conditions of the claim.
See Figure \ref{fig:dom_clean} for illustrations.

\bigskip
\par\noindent
{\bf Case (a):}
 has  tails or more.
Then playing  gains at least  points, and the resulting boxes are high boxes, and paths with a  leaf and no internal  vertices, i.e., regular colored path boxes and boxes of size .

\smallskip
\par\noindent
{\bf Case (b):}
 has a leaf neighbor and an additional tail.
Then playing  can gain at least  points from the red vertices and the other tail lead, and as in the previous case, the resulting boxes are high boxes, 
regular colored path boxes and boxes of size .

\smallskip
\par\noindent
If Cases (a) and (b) are not satisfied, then all tails are of length  or more.

\smallskip
\par\noindent
{\bf Case (c):}
 has a tail of length  or more.
If Dominator plays the vertex  that is the neighbor of the leaf on the shortest tail, then exactly one of the following two subcases occurs.
\smallskip
\par\noindent
{\bf Subcase (1):}
	The shortest tail is of length . 
	Then playing  gains  points from the red vertices and from , and the resulting box satisfies Case :(\ref{invnp:1tail}) of Definition \ref{def:regular_colored}.
\smallskip
\par\noindent
{\bf Subcase (2):}
	The shortest tail is of length  or more.
	Then playing  gains at least  points from  and its neighbors, and the resulting box satisfies Case :(\ref{invnp:1leaf}) of Definition \ref{def:regular_colored}.

\bigskip
\par\noindent
{\bf Case (d):}
All tails are of length exactly , and  has a tail of length  containing a  vertex.
Then at least one of the following two subcases occurs.
\smallskip
\par\noindent
{\bf Subcase (1):}
	There is a  leaf on a tail of .
	Then playing  gains at least  points from the vertices that become red and the tail leads, and the resulting boxes are high boxes and boxes of size .
\smallskip
\par\noindent
{\bf Subcase (2):}
	 has a  tail lead, .
	Then playing a vertex  that is another tail lead adjacent to  can gain at least  points from the red vertices and from , 
	and the resulting boxes are high boxes and boxes of size .
\QED

\begin{claim}
\label{claim:t4_dom}
If  contains a dispensible component and , then , and additionally,  and all resulting graphs are good.
\end{claim}
\Proof
First observe that if there is a dispensible box of type  that is a root box, then  by Claim \ref{claim:b2_dom} (in fact, ).
Therefore the dispensible component is of type .
Recall that all possible moves appear in Figure \ref{fig:dispensible_staller}.
If Dominator plays  or  (according to the type of  box), 
then two of the resulting components are dispensible components of type  in some valid box decomposition  of .
Therefore, one of the following cases occurs.

\smallskip
\par\noindent
{\bf Case (a):}
After Staller's move a semi-corrupted box is created.
Then .

\smallskip
\par\noindent
{\bf Case (b):}
Staller does not create a semi-corrupted box.
Then at least one dispensible component of type  remains, 
and Dominator can play move  on a high vertex in a dispensible component of type  and gain at least  points.

\smallskip
\par\noindent
Since Theorem \ref{thm:staller_good} guarantees that , we get the claim.
\QED

\begin{claim}
\label{claim:minus_dom}
If  contains a high regular box of size  or more and , then , and additionally,  and all resulting graphs are good.
\end{claim}
\begin{figure}[thbp]
  \caption{\sf Subtrees in  corresponding to the different cases in the proof of Claim \ref{claim:minus_dom}.
		(1) Case (a). 
		(2) Case (b:1). 
		(3) Case (b:2).} 
  \medskip
  \centering
    \fbox{\includegraphics[width=14cm]{dom_6.jpg}}
  \label{fig:dom_6}
\end{figure}
\Proof
From Claims \ref{claim:c12_dom}, \ref{claim:b2_dom} and \ref{claim:dom_split_7} we conclude that all regular boxes of size  or more are high complex boxes, all split vertices in these boxes have two or three tails of length , and all vertices in these tails are white.

Let  be a high regular box of size  or more as described.
Let  be a split vertex in  that has at most one neighbor that is not a tail lead (from Claim \ref{claim:every_tree_has_split} we know that such a vertex exists), and such that all vertices in the tails of  are white, and assume that .
See illustrations in Figure \ref{fig:dom_6}.

First, observe that if there is such a split vertex  with exactly two tails, then playing one of the tail leads gains at least  points and the resulting box can be separated into a dispensible box of type  rooted at , and a high box.
Therefore,  is a triplet vertex of depth .
Additionally, note that Property  guarantees that there is a vertex  on a tail of  that is the parent of a dispensible box of type , since if all three tail leads were potential triplet witnesses then this would imply that  has a (virtual) white leaf, contradicting our assumption that all tails of  are of length  
(and since a high box cannot be a parent of a high leftover box, and  does not contain corrupted boxes). 
Let  be a tail lead adjacent to  on another tail. Playing 
could gain  points from converting  to a  vertex that is the box root of a dispensible box of type .
Since the other resulting box would be high, and , we conclude that doing so would violate Property .
Therefore, we conclude that for each such split vertex  there exists a split vertex  that would become a triplet vertex in this case. 
See Cases (2) and (3) in Figure \ref{fig:dom_6} ( corresponds to  and  corresponds to ).

Let  be a leaf on , 
and let  to be the split vertex farthest from .
Let  be the split vertex that is closest to .
We analyze the results of playing , and separate them into cases according to the structure of the graph.
Observe that at least  points are gained by this move, therefore .

\bigskip
\par\noindent
{\bf Case (a):}
There is a dispensible box of type  that is adjacent to a leaf of a tail of .
Then playing  gains at least  points, and the resulting boxes are  boxes, a high box containing the semi-triplet vertex , and a path  of the form . 
See Case (1) in Figure \ref{fig:dom_6}.
For all , 
if Staller plays , then at least  points are gained 
and in the following Dominator move there is a valid box decomposition containing a high box with a semi-triplet vertex, therefore in this case  by Lemma \ref{lemma:semi_triplet_subtree_gain}. 
If Staller plays elsewhere, 
then either a semi-corrupted component is created in  (in which case at least  points are gained in step ), 
or Dominator can play  and gain at least  points in step .
We conclude that in this case . 	
\bigskip
\par\noindent
{\bf Case (b):}
After Dominator plays , the resulting graph  contains 
a dispensible component of type  and a high box with a semi-triplet subtree rooted at . This splits further into the following two subcases.

\smallskip
\par\noindent
{\bf Subcase (1):}
The resulting semi-triplet subtree has a  leaf (Case (2) in Figure \ref{fig:dom_6}).
Lemma \ref{lemma:semi_triplet_subtree_gain} guarantees that in this case, if Staller does not play on the semi-triplet subtree then at least  points are gained.
If Staller does play on the semi-triplet subtree, then Dominator can play on the  component and gain at least  points.
If Staller creates a semi-corrupted component, then at least  points are gained in step  as well.
In all cases, .

\smallskip
\par\noindent
{\bf Subcase (2):}
The semi-triplet subtree rooted at  has a  tail lead (Case (3) in Figure \ref{fig:dom_6}).
We conclude that the internal degree of  in  is exactly , for the following reasons:
First, assume towards contradiction that the internal degree of  is . Then  is a split vertex with two tails of length , in contradiction to the assumption that .
Next, assume towards contradiction that the internal degree of  is  or more.
Then  has at least two additional neighbors, besides the tail lead  adjacent to  and the two other tail leads.
Since  was chosen to be the split vertex farthest from , and all tails of all split vertices are of length exactly , at least one of the other neighbors of , , must be one of the following:
\begin{enumerate}
	\dnsitem A lead of a white tail of length . 
	\dnsitem A vertex of internal degree  that has a white leaf and a neighbor  that is a triplet vertex of depth .
\end{enumerate}
In both cases, Dominator could play  and gain at least  points from , the tail leads and  and , 
and the resulting boxes in  would be  boxes and  boxes (since every triplet subtree has a vertex that is the parent of a  box by Property , as none of the split vertices have leaf neighbors).
This contradicts the assumption that , and we conclude that the internal degree of  is less than .

Since the internal degree of  is exactly , and it does not have another white tail of length  
or a white tail of length , 
we conclude that
 is a strong semi-triplet vertex. From Lemma \ref{lemma:semi_triplet_subtree_gain} we conclude that if Staller does not play  on the semi-triplet subtree, then .
If Staller does play on the semi-triplet subtree, then as before, Dominator can play on the  component and gain at least  points, so either way .

\smallskip
\par\noindent
	If Staller plays elsewhere and creates a semi-corrupted component, then  as well.
\bigskip
\par\noindent
This concludes the proof, since all resulting graphs are good.
\QED

\begin{claim}
\label{claim:t1_dom}
If all root boxes in  are of size , then  and the resulting graph  is good, and at least one of the following properties holds.
\begin{enumerate}
	\dnsitem .
	\dnsitem .
\end{enumerate}
\end{claim}
\Proof
Recall that boxes of size  cannot be parent boxes, and therefore all components of the dense graph are of size  (i.e., components of the forms ,  and ).
Dominator can play on any real vertex and gain at least  points, and the resulting dense graph  contains only components of size .
If Staller plays on a real vertex of the dense graph, then at least  points are gained and therefore .
Otherwise, since the box containing Staller's move contains at most one  vertex, the proof of Claim \ref{claim:staller_triplet} guarantees that one of the following cases occurs.
\smallskip
\par\noindent
{\bf Case (a):}
At least  points are gained in Staller's move, and the resulting box is high.
Therefore , and the resulting graph  is good.
\smallskip
\par\noindent
{\bf Case (b):}
At least  points are gained in Staller's move (i.e., ), and the resulting box is semi-corrupted and contains a strong fix vertex. 
Lemma \ref{lemma:fix_subtree_gain} guarantees that in this case, at least  points are gained in the following Dominator move (i.e., ), and the resulting graph  is good.
\QED

Theorem \ref{thm:dom_gain} follows from Claims 
\ref{claim:c12_dom}, \ref{claim:b2_dom}, \ref{claim:t4_dom}, \ref{claim:minus_dom} and \ref{claim:t1_dom}.



\subsection{Analysis conclusion}
\label{sub:an_conclusion}

We conclude by showing that if Dominator plays according to the algorithm, then the game ends with an average gain of at least  points per move, and therefore Dominator wins.

\begin{theorem}
\label{cor:dom_wins}
If Dominator plays greedily according to the guidelines in Section \ref{section:algorithm_outline}, then the average gain in a Dominator-start game is at least  points.
\end{theorem}
\Proof
We first note that if  is good for some  and , then there exists  such that at least one of the following properties holds.
We observe that for even , these properties are guaranteed by Theorem \ref{thm:dom_gain} and Corollary \ref{cor:last_5},
and for odd , they are guaranteed by Theorem \ref{thm:staller_good} (and the definition of semi-corrupted components).

\begin{enumerate}
	\dnsitem \label{cor:dom_wins:odd}
	 is odd and at least  points are gained in steps  through , and  is good.
	
	\dnsitem \label{cor:dom_wins:even}
	 is even and at least  points are gained in steps  through , and  is good.
	
	\dnsitem \label{cor:dom_wins:last}
	At least  points are gained in steps  through  and  is empty, i.e., the game is over.
\end{enumerate}

We note that for ,  is high and therefore good,
and , and therefore there exists some  satisfying one of the above cases.

The theorem follows by induction, since  in Cases \ref{cor:dom_wins:odd} and \ref{cor:dom_wins:even}, and the game ends when the graph is empty, and therefore  must satisfy Case \ref{cor:dom_wins:last}.
\QED

This concludes the analysis.

\section{Implementing the algorithm}
\label{section:implementation}

The greedy algorithm described in Section \ref{section:algorithm_outline} often achieves stronger results than what is required in order to prove Conjecture \ref{conjecture:3_5}.
Specifically, it would suffice if Dominator's move was chosen such that  is non-negative when possible while preserving the invariant, 
and when no such move is possible, chose a move which guarantees that the excess gain at the end of the next Staller move or the next Dominator move is non-negative (if the current Dominator move is not the last move of the game).
The analysis shown in the previous section guarantees that Dominator always has such a move.

We have implemented a variant of the algorithm in order to verify the correctness of the algorithm and the analysis, 
and ran it successfully on all trees up to size  (using the tree generation algorithm described in \cite{wright1986constant}),
as well as on some specifically constructed intermediate underlying graphs (containing components which consist of several boxes in all valid box decompositions).
In each test, all possible games resulting from the tested initial graph were checked, i.e., Dominator's moves were chosen according to the algorithm, and all possible legal moves were tested for each Staller move.

For efficiency reasons, the implementation differs from the algorithm described in Section \ref{section:algorithm_outline} in the following ways.
\begin{enumerate}
	\dnsitem Not all possible underlying graphs and value functions are tested, but rather a small subset which is closely related to the previous underlying graph and value function.
	\dnsitem The implementation uses a deterministic box decomposition process, rather than checking all possible box decompositions. 
	\dnsitem Dominator's move is always chosen from the vertices in the root boxes of the dense graph, and leaves are not considered except when all components of the dense graph are of size .
	Note that the analysis of Dominator's moves refers only to the root boxes, and therefore there is always a move on a root box.
	\dnsitem When a semi-corrupted component is created at the end of Staller's move, the following Dominator move is chosen from the vertices of the corrupted box.
	\dnsitem If it is impossible to gain more than  points on Dominator's move, and if there are several moves gaining  points, ties are broken according to the following priorities:
	\begin{enumerate}
		\dnsitem Prefer to play on a white vertex of a dispensible component.
		\dnsitem Prefer a move where the resulting dense graph contains a root box of the form .
		\dnsitem Prefer a move where the resulting dense graph contains a strong semi-triplet subtree or a semi-triplet subtree with a  leaf.
		\dnsitem Prefer a move where the resulting dense graph contains a dispensible component of type . 
		\dnsitem Break additional ties arbitrarily.
	\end{enumerate}
\end{enumerate}

Because we used the implementation to verify parts of the analysis as well, we did not make additional improvements, and also verified that the excess gain for Staller moves is never negative, 
and that if the excess gain of a move played by Dominator is negative, then the sum of excess gains over at most three moves starting from this move is not negative (see Theorem \ref{thm:dom_gain} for details).

The efficiency of the algorithm can be further improved using additional modifications, such as choosing the first move achieving non-negative excess gain (as described above), 
and choosing moves in a deterministic manner imitating the proofs in the analysis.


\section{Conclusions}
\label{section:conclusions}

The algorithm described for Dominator achieves the desired bound of  on all isolate-free forests, which proves Conjecture \ref{conjecture:3_5}.
The variant of the conjecture that relates to general isolate-free graphs remains open, however an upper bound of  is proved in \cite{kinnersley2013extremal}, and an improved bound of  is shown in \cite{bujtas2015game}. 
In \cite{bujtas2015game}, Bujt\'as further improves these results (to bounds below ) for graphs with minimum degree  or more.

We note that the algorithm introduced here does not perform optimally (i.e., does not achieve the game domination number) on all graphs, 
and it may be interesting to optimize the solutions and find strategies that achieve the game domination number.
Constructing a strategy for Staller may also be of interest, whether it is an optimal strategy or a strategy that performs optimally against a specific Dominator strategy.





\begin{thebibliography}{1}

\bibitem{brevsar2010domination}
Bo{\v{s}}tjan Bre{\v{s}}ar, Sandi Klav{\v{z}}ar, and Douglas~F Rall.
\newblock Domination game and an imagination strategy.
\newblock {\em SIAM Journal on Discrete Mathematics}, 24(3):979--991, 2010.

\bibitem{bujtas2015domination}
Csilla Bujt{\'a}s.
\newblock Domination game on forests.
\newblock {\em Discrete Mathematics}, 338(12):2220--2228, 2015.

\bibitem{bujtas2015game}
Csilla Bujt{\'a}s.
\newblock On the game domination number of graphs with given minimum degree.
\newblock {\em The Electronic Journal of Combinatorics}, 22(3):P3--29, 2015.

\bibitem{kinnersley2013extremal}
William~B Kinnersley, Douglas~B West, and Reza Zamani.
\newblock Extremal problems for game domination number.
\newblock {\em SIAM Journal on Discrete Mathematics}, 27(4):2090--2107, 2013.

\bibitem{wright1986constant}
Robert~Alan Wright, Bruce Richmond, Andrew Odlyzko, and Brendan~D McKay.
\newblock Constant time generation of free trees.
\newblock {\em SIAM Journal on Computing}, 15(2):540--548, 1986.

\end{thebibliography}

\end{document}
