\subsection{Stuttering theorem for $FO_{\varprop}$}
In this section, we will prove that if in an $\omega$-sequence $\aword$, a subword $\mathbf{s}$ is repeated consecutively a large number of times then this $\omega$-word and other $\omega$-words obtained by removing some of the repetitions of $\mathbf{s}$ satisfy the same set of $FO_{\varprop}$ formulae, this is what we call the stuttering theorem for $FO_{\varprop}$. Such a result will allow us to bound the repetition of iteration of loops in path schema and thus to obtain a model-checking algorithm for the logic $FO_{\varprop}$ optimal in complexity. In order to prove the stuttering theorem, we will use EF games.

In the sequel we consider a natural $N \geq 2$ and two $\omega$-words over $\powerset{\varprop}$ of the following form $\aword = \aword_1 \mathbf{s}^M \aword_2, \aword'= \aword_1 \mathbf{s}^{M+1} \aword_2 \in (\Sigma)^{\omega}$ with $M > 2^{N+1}$ and $\mathbf{s}$ a non-empty finite word. We will now show that the game $EF_N(\astruct,\bstruct)$ is winning. The strategy for Duplicator will work as follows: at the $i$-th round (for $i \leq N$), if the point chosen by the Spoiler is close (upto a distance of $2^{N-i}$) to another previously chosen position then the Duplicator will chose a point in the other word at the exact same distance from the corresponding position and if the point is far (at a distance greater thatn $2^{N-i}$) from any other position  then in the other word the Duplicator will chose a position also far away from any other position. 

Before to provide a winning strategy for the Duplicator we define define some invariants on any $i$-round play (with $i \leq N$) that will be maintained by the Duplicator's strategy. In order to define this invariant and the Duplicator's strategy, let us mark the following positions $a_{-1}=b_{-1}=|\aword_1|$,
$a_{0}=|\aword_1\mathbf{s}^{M}|$ and $b_{0}=|\aword_1\mathbf{s}^{M+1}|$. Note that these are the positions where the repetition of $\mathbf{s}$ starts
respectively ends in each word.
 Given a play $i$-round play $Pl_i=(p_1,a_1,b_1)(p_2,a_2,b_2)\cdots(p_i,a_i,b_i)$, for each $j \in \interval{1}{i}$ such that $a_{-1} < a_j < a_{0}$, we define $\lef{a_j}=\max(a_k \mid k \in \interval{-1}{i} \mbox{ and } a_k < a_j)$ and $\rig{a_j}=\min(a_k \mid k \in \interval{-1}{i} \mbox{ and } a_j < a_k)$ (ie $\lef{a_j}$ and $\rig{a_j}$ are the closest neighbor of $a_j$). We define similarly $\lef{b_j}$ and $\rig{b_j}$. We will then say that the $i$-round play $Pl_i$ respects the invariant $\aninv$ iff the following conditions are satisfied for all $j,k \in \interval{1}{i}$:
\begin{enumerate}
  \item $a_{j} \leq a_{-1}$ iff  $b_{j} \leq b_{-1}$, and in this case  $a_{j}=b_{j}$ 
  \item $a_0 \leq a_j$ iff $b_0 \leq b_j$, and in this case $a_{j}-a_0=b_{j}-b_0$
  \item $a_{j}<a_{k}$ if and only if $b_{j}<b_{k}$
  \item if $a_{-1} < a_j < a_{0}$ (note that we have necessarily $b_{-1} < b_j < b_{0}$ )then:
    \begin{enumerate}
      \item  $a_{j}-a_{-1}=b_{j}-b_{-1}$ or $a_{0}-a_{j}=b_{0}-b_{j}$.
      \item $\rig{a_j}-a_j < 2^{N+1-i}.|\mathbf{s}|$ iff $\rig{b_{j}}-b_j< 
        2^{N+1-i}.|\mathbf{s}|$  and in this case $\rig{a_j}-a_{j} = \rig{b_j}-b_j$
      \item $ a_j-\lef{a_j} < 2^{N+1-i}.|\mathbf{s}|$ iff $b_j - \lef{b_{j}} <
        2^{N+1-i}.|\mathbf{s}|$  and in this case $a_j-\lef{a_{j}} = b_j-\lef{b_{j}}$ 


    \end{enumerate}
\end{enumerate}
First we remark the invariant $\aninv$ is a sufficient condition for a play to be winning as stated by the following lemma.
\begin{lemma}
\label{lem-inv}
If a $N$-round play over $\aword$ and $\aword'$ respects $\aninv$, then it is a winning play for Duplicator.
\end{lemma}
\begin{proof}
Let $(p_1,a_1,b_1)(p_2,a_2,b_2)\cdots(p_N,a_N,b_N)$ be a $N$-round play over $\aword$ and $\aword'$ respecting $\aninv$. Let $i,j \in \interval{1}{N}$. First by $\aninv.3.$ we have $a_i=a_j$ iff $b_i=b_j$ and also $a_i < a_j$ iff $b_i < b_j$. We will now prove that $a_i+1=a_j$ iff $b_i+1=b_j$. First, if $a_i < a_{-1}$  or if  $a_0 \leq a_i$ this is obvious thanks to $\aninv.1$ and $\aninv.2$. Now assume $a_{-1} < a_i < a_0$, then $a_j=\rig{a_i}$ and $\rig{a_i}-{a_i} < 2.|\mathbf{s}|$, then thanks to $\aninv.4.b$ we have $a_i+1=a_j$ iff $b_i+1=b_j$.  For the last case, if $a_{-1} = a_i$, then $a_i=\lef{a_j}$ and we can conclude similarly thanks to $\aninv.4.c$. It remains now to prove that for all $i \in \interval{1}{N}$ and for $c\in \varprop$ $c\in \astruct(a_i)$ iff $c\in \bstruct(b_i)$, but this is obvious since either $a_i$ and $b_i$ are pointing at the same position in $\aword_1$ (thanks to $\aninv.1$) or at the same position in the word $\aword_2$ (thanks to $\aninv.2$) or at the same position in the word $\mathbf{s}$ (thanks to  $\aninv.4.a$). This allows us to conclude that $(p_1,a_1,b_1)(p_2,a_2,b_2)\cdots(p_N,a_N,b_N)$ is winning for Duplicator. \qed
\end{proof}
We define now a strategy $\hat\astrategy_D$ for Duplicator which will respect at each round the invariant $\aninv$ and this no matter what the Spoiler plays. Using the previous lemma, this will allows us to deduce that this strategy is a winning strategy for the Duplicator.Let $i \in \interval{1}{N}$ and $Pl_{i-1}=(p_1,a_1,b_1)(p_2,a_2,b_2)\cdots(p_{i-1},a_{i-1},b_{i-1})$ be a $(i-1)$-round play. First we define $b_i=\astrategy_D(Pl_i,\tuple{0,a_i})$ that is what Duplicator answers if the Spoiler chooses position $a_i$ in the $\omega$-word $\aword$. We have $b_i=\hat\astrategy_D(Pl_i,\tuple{0,a_i})$ defined as follows:
\begin{itemize}
\item If $a_i\leq a_{-1}$, then $b_i=a_i$
\item If $a_0 \leq a_{i}$, then $b_i=a_i+|\mathbf{s}|$ (same positions in word $\aword_2$)
\item If $a_{-1} < a_i < a_0$ and $a_l=\lef{a_i}$ and $a_r=\rig{a_i}$, then:
\begin{itemize}
\item  If $a_i-a_l \leq a_r-a_i$, we have $b_i=b_l+(a_i-a_l)$
\item  If $a_r-a_i < a_i-a_l$, we have $b_i=b_r-(a_r-a_i)$
\end{itemize}
\end{itemize}
Similarly we have $a_i=\hat\astrategy_D(Pl_i,\tuple{1,b_i})$ defined as follows:
\begin{itemize}
\item If $b_i\leq b_{-1}$, then $a_i=b_i$
\item If $b_0 \leq b_{i}$, then $a_i=b_i-|\mathbf{s}|$ (same positions in word $\aword_2$)
\item If $b_{-1} < b_i < b_0$ and $b_l=\lef{b_i}$ and $b_r=\rig{b_i}$, then:
\begin{itemize}
\item  If $b_i-b_l \leq b_r-b_i$, we have $a_i=a_l+(b_i-b_l)$
\item  If $b_r-b_i < b_i-b_l$, we have $a_i=a_r-(b_r-b_i)$
\end{itemize}
\end{itemize}


\begin{lemma}
\label{lem-strat-win}
For any Spoiler's strategy $\astrategy_S$ and for all $i \in \interval{0}{N}$, we have that $\play{i}{\astrategy_S,\hat\astrategy_D}{\aword,\aword'}$ respects $\aninv$.
\end{lemma}
\begin{proof}
The proof proceeds by induction on $i$. The base case for $i=0$ is obvious since the empty play respects $\aninv$. Let $\astrategy_S$ be a Spoiler's strategy and for $i \in \interval{1}{N-1}$ we assume that $\play{i-1}{\astrategy_S,\hat\astrategy_D}{\aword,\aword'}$ respects $\aninv$. Suppose that $\astrategy_S(\play{i-1}{\astrategy_S,\hat\astrategy_D}{\aword,\aword'})=\tuple{0,a_i}$ and let $b_i=\hat\astrategy_D(\play{i-1}{\astrategy_S,\hat\astrategy_D}{\aword,\aword'},\tuple{0,a_i})$. Before to probe that $\play{i}{\astrategy_S,\hat\astrategy_D}{\aword,\aword'}$ respects $\aninv$, we state some properties.

First, we prove that if $a_{-1} < a_i < a_0$ and $a_l=\lef{a_i}$ and $a_r=\rig{a_i}$, then we have $b_{-1}<b_i <b_0$ and $b_l=\lef{b_i}$ and $b_r=\rig{b_i}$ and $a_{i}-a_{-1}=b_{i}-b_{-1}$ or $a_{0}-a_{i}=b_{0}-b_{i}$. First note that by induction hypothesis and using $\aninv.2$, since $a_r < a_l$ and since there is no $m \in \interval{-1}{i-1}$ such that $a_r < a_m < a_l$, we deduce that $b_r < b_l$ and there is no $m \in \interval{-1}{i-1}$ such that $b_r < b_m < b_l$. We then proceed by the following case analysis:

\begin{enumerate}
\item \textbf{Case $a_i-a_l<  2^{N+1-i}.|\mathbf{s}|$ and $a_r-a_i < 2^{N+1-i}$}. Then we have $a_r-a_l < 2.2^{N+1-i}.|\mathbf{s}|$, ie $a_r-a_l < 2^{N+1-(i-1)}.|\mathbf{s}|$. By induction hypothesis over $\play{i-1}{\astrategy_S,\hat\astrategy_D}{\aword,\aword'}$, since $a_l=\lef{a_r}$ and $a_r=\rig{a_l}$ (note that if $a_l=a_{-1}$ or $a_r=a_0$ a similar reasoning can be done),  we also have that  $b_r-b_l < 2^{N+1-(i-1)}.|\mathbf{s}|$ and $a_r-a_l=b_r-b_l$ [using $\aninv.4.b$ and $\aninv.4.c$]. Since $a_r-a_l=b_r-b_l$, from $\hat\astrategy_D$, no matter if $a_i-a_l \leq a_r-a_i$ or $a_r-a_i < a_i-a_l$, we will have $b_r-b_i=a_r-a_i$ and $b_i-b_l=a_l-a_i$ and also $b_{-1} \leq b_l < b_i < b_r \leq b_0$. We also directly have $b_l=\lef{b_i}$ and $b_r=\rig{b_i}$. Finally since by induction hypothesis we have $a_{r}-a_{-1}=b_{r}-b_{-1}$ or $a_{0}-a_{r}=b_{0}-b_{r}$, we deduce also that $a_{i}-a_{-1}=b_{i}-b_{-1}$ or $a_{0}-a_{i}=b_{0}-b_{i}$.
\item \textbf{Case $a_i-a_l < 2^{N+1-i}.|\mathbf{s}|$ and $a_r-a_i \geq 2^{N+1-i}$ and $a_r-a_l \leq 2^{N+1-(i-1)}.|\mathbf{s}|$}. Then $b_i=b_l+(a_i-a_l)$.  By induction hypothesis over $\play{i-1}{\astrategy_S,\hat\astrategy_D}{\aword,\aword'}$, since $a_l=\lef{a_r}$ and $a_r=\rig{a_l}$ (note that if $a_l=a_{-1}$ or $a_r=a_0$ a similar reasoning can be done),  we also have that  $b_r-b_l < 2^{N+1-(i-1)}.|\mathbf{s}|$ and $a_r-a_l=b_r-b_l$ [using $\aninv.4.b$ and $\aninv.4.c$].This allows us to deduce that $b_r-b_i=a_r-a_i$ and $b_i-b_l=a_l-a_i$ and  $b_{-1} \leq b_l < b_i < b_r \leq b_0$ and $b_r-b_i \geq 2^{N+1-i}$. We also directly have $b_l=\lef{b_i}$ and $b_r=\rig{b_i}$. Finally since by induction hypothesis we have $a_{l}-a_{-1}=b_{l}-b_{-1}$ or $a_{0}-a_{l}=b_{0}-b_{l}$, we deduce also that $a_{i}-a_{-1}=b_{i}-b_{-1}$ or $a_{0}-a_{i}=b_{0}-b_{i}$.
\item \textbf{Case $a_i-a_l < 2^{N+1-i}.|\mathbf{s}|$ and $a_r-a_i \geq 2^{N+1-i}$ and $a_r-a_l \geq 2^{N+1-(i-1)}.|\mathbf{s}|$}. Then $b_i=b_l+(a_i-a_l)$.  By induction hypothesis over $\play{i-1}{\astrategy_S,\hat\astrategy_D}{\aword,\aword'}$, since $a_l=\lef{a_r}$ and $a_r=\rig{a_l}$ (note that if $a_l=a_{-1}$ or $a_r=a_0$ a similar reasoning can be done),  we also have that  $b_r-b_l \geq  2^{N+1-(i-1)}.|\mathbf{s}|$. Hence we have that $(b_r-b_i) \geq 2^{N+1-(i-1)}.|\mathbf{s}| - (b_i-b_l) \geq 2^{N+1-(i-1)}.|\mathbf{s}| -  2^{N+1-i}.|\mathbf{s}| $. From which we deduce $b_r-b_i \geq 2^{N+1-i}.|\mathbf{s}|$ This allows us to deduce that  $b_{-1} \leq b_l < b_i < b_r \leq b_0$ and $b_i-b_l=a_i-a_l$ and $b_r-b_i \geq 2^{N+1-i}$. We also directly have $b_l=\lef{b_i}$ and $b_r=\rig{b_i}$. Finally since by induction hypothesis we have $a_{l}-a_{-1}=b_{l}-b_{-1}$ or $a_{0}-a_{l}=b_{0}-b_{l}$, we deduce also that $a_{i}-a_{-1}=b_{i}-b_{-1}$ or $a_{0}-a_{i}=b_{0}-b_{i}$.
\item \textbf{Case $a_i-a_l \geq 2^{N+1-i}.|\mathbf{s}|$ and $a_r-a_i < 2^{N+1-i}$ and $a_r-a_l < 2^{N+1-(i-1)}.|\mathbf{s}|$}. Then $b_i=b_r-(a_r-a_i)$.  By induction hypothesis over $\play{i-1}{\astrategy_S,\hat\astrategy_D}{\aword,\aword'}$, since $a_l=\lef{a_r}$ and $a_r=\rig{a_l}$ (note that if $a_l=a_{-1}$ or $a_r=a_0$ a similar reasoning can be done),  we also have that  $b_r-b_l < 2^{N+1-(i-1)}.|\mathbf{s}|$ and $a_r-a_l=b_r-b_l$ [using $\aninv.4.b$ and $\aninv.4.c$].This allows us to deduce that $b_r-b_i=a_r-a_i$ and $b_i-b_l=a_l-a_i$ and  $b_{-1} \leq b_l < b_i < b_r \leq b_0$ and $b_i-b_l \geq 2^{N+1-i}$. We also directly have $b_l=\lef{b_i}$ and $b_r=\rig{b_i}$. Finally since by induction hypothesis we have $a_{l}-a_{-1}=b_{l}-b_{-1}$ or $a_{0}-a_{l}=b_{0}-b_{l}$, we deduce also that $a_{i}-a_{-1}=b_{i}-b_{-1}$ or $a_{0}-a_{i}=b_{0}-b_{i}$.
\item \textbf{Case $a_i-a_l \geq 2^{N+1-i}.|\mathbf{s}|$ and $a_r-a_i < 2^{N+1-i}.|\mathbf{s}|$ and $a_r-a_l \geq 2^{N+1-(i-1)}.|\mathbf{s}|$}. Then $b_i=b_r-(a_r-a_i)$.  By induction hypothesis over $\play{i-1}{\astrategy_S,\hat\astrategy_D}{\aword,\aword'}$, since $a_l=\lef{a_r}$ and $a_r=\rig{a_l}$ (note that if $a_l=a_{-1}$ or $a_r=a_0$ a similar reasoning can be done),  we also have that  $b_r-b_l \geq  2^{N+1-(i-1)}.|\mathbf{s}|$ . Hence we have that $(b_i-b_l)\geq 2^{N+1-(i-1)}.|\mathbf{s}| - (b_r-b_i) \geq 2^{N+1-(i-1)}.|\mathbf{s}| -  2^{N+1-i}.|\mathbf{s}| $. From which we deduce $b_i-b_l\geq 2^{N+1-i}.|\mathbf{s}|$ This allows us to deduce that  $b_{-1} \leq b_l < b_i < b_r \leq b_0$ and $b_r-b_i=a_r-a_i$ and $b_i-b_l\geq 2^{N+1-i}$. We also directly have $b_l=\lef{b_i}$ and $b_r=\rig{b_i}$. Finally since by induction hypothesis we have $a_{r}-a_{-1}=b_{r}-b_{-1}$ or $a_{0}-a_{r}=b_{0}-b_{r}$, we deduce also that $a_{i}-a_{-1}=b_{i}-b_{-1}$ or $a_{0}-a_{i}=b_{0}-b_{i}$.
\item \textbf{Case $a_i-a_l = 2^{N+1-i}.|\mathbf{s}|$ and $a_r-a_i = 2^{N+1-i}.|\mathbf{s}|$}. Then $b_i=b_l+(a_i-a_l)$. Then we have $a_r-a_l= 2^{N+1-(i-1)}.|\mathbf{s}|$. By induction hypothesis over $\play{i-1}{\astrategy_S,\hat\astrategy_D}{\aword,\aword'}$, since $a_l=\lef{a_r}$ and $a_r=\rig{a_l}$ (note that if $a_l=a_{-1}$ or $a_r=a_0$ a similar reasoning can be done),  we also have that  $b_r-b_l \geq  2^{N+1-(i-1)}.|\mathbf{s}|$ and $|(a_r-a_l)-(b_r-b_l)| \leq |\mathbf{s}|$ [using $\aninv.4.a$]. From this we deduce that $b_r-b_l \geq a_r-a_l$ (otherwise we would have $b_r-b_l < 2^{N+1-(i-1)}.|\mathbf{s}|$, which allow us to deduce immediately that  $b_r-b_i \geq a_r-a_i$ and hence  $b_r-b_i \geq 2^{N+1-i}.|\mathbf{s}|$. Hence we also have  $b_{-1} \leq b_l < b_i < b_r \leq b_0$. We also directly have $b_l=\lef{b_i}$ and $b_r=\rig{b_i}$. Finally since by induction hypothesis we have $a_{l}-a_{-1}=b_{l}-b_{-1}$ or $a_{0}-a_{l}=b_{0}-b_{l}$, we deduce also that $a_{i}-a_{-1}=b_{i}-b_{-1}$ or $a_{0}-a_{i}=b_{0}-b_{i}$.
\item \textbf{Case $a_i-a_l = 2^{N+1-i}.|\mathbf{s}|$ and $a_r-a_i > 2^{N+1-i}.|\mathbf{s}|$}. Then $b_i=b_l+(a_i-a_l)$. Then we have $a_r-a_l \geq 2^{N+1-(i-1)}.|\mathbf{s}|$. By induction hypothesis over $\play{i-1}{\astrategy_S,\hat\astrategy_D}{\aword,\aword'}$, since $a_l=\lef{a_r}$ and $a_r=\rig{a_l}$ (note that if $a_l=a_{-1}$ or $a_r=a_0$ a similar reasoning can be done),  we also have that  $b_r-b_l \geq  2^{N+1-(i-1)}.|\mathbf{s}|$. This allow us to deduce immediately that  $b_r-b_i \geq 2^{N+1-i}.|\mathbf{s}|$. Hence we also have  $b_{-1} \leq b_l < b_i < b_r \leq b_0$. We also directly have $b_l=\lef{b_i}$ and $b_r=\rig{b_i}$. Finally since by induction hypothesis we have $a_{l}-a_{-1}=b_{l}-b_{-1}$ or $a_{0}-a_{l}=b_{0}-b_{l}$, we deduce also that $a_{i}-a_{-1}=b_{i}-b_{-1}$ or $a_{0}-a_{i}=b_{0}-b_{i}$.
\item \textbf{Case $a_i-a_l > 2^{N+1-i}.|\mathbf{s}|$ and $a_r-a_i = 2^{N+1-i}.|\mathbf{s}|$}. Then $b_i=b_r-(a_r-a_i)$. Then we have $a_r-a_l \geq 2^{N+1-(i-1)}.|\mathbf{s}|$. By induction hypothesis over $\play{i-1}{\astrategy_S,\hat\astrategy_D}{\aword,\aword'}$, since $a_l=\lef{a_r}$ and $a_r=\rig{a_l}$ (note that if $a_l=a_{-1}$ or $a_r=a_0$ a similar reasoning can be done),  we also have that  $b_r-b_l \geq  2^{N+1-(i-1)}.|\mathbf{s}|$. This allow us to deduce immediately that  $b_i-b_l \geq 2^{N+1-i}.|\mathbf{s}|$. Hence we also have  $b_{-1} \leq b_l < b_i < b_r \leq b_0$. We also directly have $b_l=\lef{b_i}$ and $b_r=\rig{b_i}$. Finally since by induction hypothesis we have $a_{l}-a_{-1}=b_{l}-b_{-1}$ or $a_{0}-a_{l}=b_{0}-b_{l}$, we deduce also that $a_{i}-a_{-1}=b_{i}-b_{-1}$ or $a_{0}-a_{i}=b_{0}-b_{i}$.
\item \textbf{Case $a_i-a_l > 2^{N+1-i}.|\mathbf{s}|$ and $a_r-a_i > 2^{N+1-i}.|\mathbf{s}|$ and $a_i-a_l <a_r-a_i$ }. Then $b_i=b_l+(a_i-a_l)$. Then we have $a_r-a_l \geq 2^{N+1-(i-1)}.|\mathbf{s}|$. By induction hypothesis over $\play{i-1}{\astrategy_S,\hat\astrategy_D}{\aword,\aword'}$, since $a_l=\lef{a_r}$ and $a_r=\rig{a_l}$ (note that if $a_l=a_{-1}$ or $a_r=a_0$ a similar reasoning can be done),  we also have that  $b_r-b_l \geq  2^{N+1-(i-1)}.|\mathbf{s}|$ . This allow us to deduce immediately that  $b_i-b_l \geq 2^{N+1-i}.|\mathbf{s}|$ $|(a_r-a_l)-(b_r-b_l)| \leq |\mathbf{s}|$ [using $\aninv.4.a$]. Consequently we deduce that $b_r-b_i \geq a_r-a_i-|\mathbf{s}| \geq 2^{N+1-i}.|\mathbf{s}|$. Hence we have  $b_{-1} \leq b_l < b_i < b_r \leq b_0$. We also directly have $b_l=\lef{b_i}$ and $b_r=\rig{b_i}$. Finally since by induction hypothesis we have $a_{l}-a_{-1}=b_{l}-b_{-1}$ or $a_{0}-a_{l}=b_{0}-b_{l}$, we deduce also that $a_{i}-a_{-1}=b_{i}-b_{-1}$ or $a_{0}-a_{i}=b_{0}-b_{i}$.
\item \textbf{Case $a_i-a_l > 2^{N+1-i}.|\mathbf{s}|$ and $a_r-a_i > 2^{N+1-i}.|\mathbf{s}|$ and $a_r-a_i <a_i-a_l$ }. Then $b_i=b_r-(a_r-a_i)$. Then we have $a_r-a_l \geq 2^{N+1-(i-1)}.|\mathbf{s}|$. By induction hypothesis over $\play{i-1}{\astrategy_S,\hat\astrategy_D}{\aword,\aword'}$, since $a_l=\lef{a_r}$ and $a_r=\rig{a_l}$ (note that if $a_l=a_{-1}$ or $a_r=a_0$ a similar reasoning can be done),  we also have that  $b_r-b_l \geq  2^{N+1-(i-1)}.|\mathbf{s}|$ . This allow us to deduce immediately that  $b_i-b_l \geq 2^{N+1-i}.|\mathbf{s}|$ $|(a_r-a_l)-(b_r-b_l)| \leq |\mathbf{s}|$ [using $\aninv.4.a$]. Consequently we deduce that $b_i-b_l \geq a_i-a_l-|\mathbf{s}| \geq 2^{N+1-i}.|\mathbf{s}|$. Hence we have  $b_{-1} \leq b_l < b_i < b_r \leq b_0$. We also directly have $b_l=\lef{b_i}$ and $b_r=\rig{b_i}$. Finally since by induction hypothesis we have $a_{r}-a_{-1}=b_{r}-b_{-1}$ or $a_{0}-a_{r}=b_{0}-b_{r}$, we deduce also that $a_{i}-a_{-1}=b_{i}-b_{-1}$ or $a_{0}-a_{i}=b_{0}-b_{i}$.
 \end{enumerate}
 Thanks to the previous case analysis, it is then easy to check that $\play{i}{\astrategy_S,\hat\astrategy_D}{\aword,\aword'}$ respects $\aninv$. The case where $\astrategy_S(\play{i-1}{\astrategy_S,\hat\astrategy_D}{\aword,\aword'})=\tuple{1,b_i}$ can be treated in an equivalence manner. \qed



\end{proof}
Using Lemma \ref{lem-inv} and \ref{lem-strat-win}, we deduce that Duplicator has a winning strategy against any strategy of the Spoiler in $EF_N(\astruct,\bstruct)$, so by Theorem \ref{theorem:EF}, we can conclude the main theorem of this subsection.
\begin{theorem}[Stuttering Theorem]
\label{theorem:stuttering}
Let $\aword = \aword_1 \mathbf{s}^M \aword_2, \aword'= \aword_1 \mathbf{s}^{M+1} \aword_2 \in (\Sigma)^{\omega}$  such that $N \geq 2$, $M> 2^{N+1}$ and $\mathbf{s}$ is a non-empty finite word. Then $\astruct\equivrel{N}\bstruct$.
\end{theorem}

