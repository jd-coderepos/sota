\subsection{Stuttering theorem for }
In this section, we will prove that if in an -sequence , a subword  is repeated consecutively a large number of times then this -word and other -words obtained by removing some of the repetitions of  satisfy the same set of  formulae, this is what we call the stuttering theorem for . Such a result will allow us to bound the repetition of iteration of loops in path schema and thus to obtain a model-checking algorithm for the logic  optimal in complexity. In order to prove the stuttering theorem, we will use EF games.

In the sequel we consider a natural  and two -words over  of the following form  with  and  a non-empty finite word. We will now show that the game  is winning. The strategy for Duplicator will work as follows: at the -th round (for ), if the point chosen by the Spoiler is close (upto a distance of ) to another previously chosen position then the Duplicator will chose a point in the other word at the exact same distance from the corresponding position and if the point is far (at a distance greater thatn ) from any other position  then in the other word the Duplicator will chose a position also far away from any other position. 

Before to provide a winning strategy for the Duplicator we define define some invariants on any -round play (with ) that will be maintained by the Duplicator's strategy. In order to define this invariant and the Duplicator's strategy, let us mark the following positions ,
 and . Note that these are the positions where the repetition of  starts
respectively ends in each word.
 Given a play -round play , for each  such that , we define  and  (ie  and  are the closest neighbor of ). We define similarly  and . We will then say that the -round play  respects the invariant  iff the following conditions are satisfied for all :
\begin{enumerate}
  \item  iff  , and in this case   
  \item  iff , and in this case 
  \item  if and only if 
  \item if  (note that we have necessarily  )then:
    \begin{enumerate}
      \item   or .
      \item  iff   and in this case 
      \item  iff   and in this case  


    \end{enumerate}
\end{enumerate}
First we remark the invariant  is a sufficient condition for a play to be winning as stated by the following lemma.
\begin{lemma}
\label{lem-inv}
If a -round play over  and  respects , then it is a winning play for Duplicator.
\end{lemma}
\begin{proof}
Let  be a -round play over  and  respecting . Let . First by  we have  iff  and also  iff . We will now prove that  iff . First, if   or if   this is obvious thanks to  and . Now assume , then  and , then thanks to  we have  iff .  For the last case, if , then  and we can conclude similarly thanks to . It remains now to prove that for all  and for   iff , but this is obvious since either  and  are pointing at the same position in  (thanks to ) or at the same position in the word  (thanks to ) or at the same position in the word  (thanks to  ). This allows us to conclude that  is winning for Duplicator. \qed
\end{proof}
We define now a strategy  for Duplicator which will respect at each round the invariant  and this no matter what the Spoiler plays. Using the previous lemma, this will allows us to deduce that this strategy is a winning strategy for the Duplicator.Let  and  be a -round play. First we define  that is what Duplicator answers if the Spoiler chooses position  in the -word . We have  defined as follows:
\begin{itemize}
\item If , then 
\item If , then  (same positions in word )
\item If  and  and , then:
\begin{itemize}
\item  If , we have 
\item  If , we have 
\end{itemize}
\end{itemize}
Similarly we have  defined as follows:
\begin{itemize}
\item If , then 
\item If , then  (same positions in word )
\item If  and  and , then:
\begin{itemize}
\item  If , we have 
\item  If , we have 
\end{itemize}
\end{itemize}


\begin{lemma}
\label{lem-strat-win}
For any Spoiler's strategy  and for all , we have that  respects .
\end{lemma}
\begin{proof}
The proof proceeds by induction on . The base case for  is obvious since the empty play respects . Let  be a Spoiler's strategy and for  we assume that  respects . Suppose that  and let . Before to probe that  respects , we state some properties.

First, we prove that if  and  and , then we have  and  and  and  or . First note that by induction hypothesis and using , since  and since there is no  such that , we deduce that  and there is no  such that . We then proceed by the following case analysis:

\begin{enumerate}
\item \textbf{Case  and }. Then we have , ie . By induction hypothesis over , since  and  (note that if  or  a similar reasoning can be done),  we also have that   and  [using  and ]. Since , from , no matter if  or , we will have  and  and also . We also directly have  and . Finally since by induction hypothesis we have  or , we deduce also that  or .
\item \textbf{Case  and  and }. Then .  By induction hypothesis over , since  and  (note that if  or  a similar reasoning can be done),  we also have that   and  [using  and ].This allows us to deduce that  and  and   and . We also directly have  and . Finally since by induction hypothesis we have  or , we deduce also that  or .
\item \textbf{Case  and  and }. Then .  By induction hypothesis over , since  and  (note that if  or  a similar reasoning can be done),  we also have that  . Hence we have that . From which we deduce  This allows us to deduce that   and  and . We also directly have  and . Finally since by induction hypothesis we have  or , we deduce also that  or .
\item \textbf{Case  and  and }. Then .  By induction hypothesis over , since  and  (note that if  or  a similar reasoning can be done),  we also have that   and  [using  and ].This allows us to deduce that  and  and   and . We also directly have  and . Finally since by induction hypothesis we have  or , we deduce also that  or .
\item \textbf{Case  and  and }. Then .  By induction hypothesis over , since  and  (note that if  or  a similar reasoning can be done),  we also have that   . Hence we have that . From which we deduce  This allows us to deduce that   and  and . We also directly have  and . Finally since by induction hypothesis we have  or , we deduce also that  or .
\item \textbf{Case  and }. Then . Then we have . By induction hypothesis over , since  and  (note that if  or  a similar reasoning can be done),  we also have that   and  [using ]. From this we deduce that  (otherwise we would have , which allow us to deduce immediately that   and hence  . Hence we also have  . We also directly have  and . Finally since by induction hypothesis we have  or , we deduce also that  or .
\item \textbf{Case  and }. Then . Then we have . By induction hypothesis over , since  and  (note that if  or  a similar reasoning can be done),  we also have that  . This allow us to deduce immediately that  . Hence we also have  . We also directly have  and . Finally since by induction hypothesis we have  or , we deduce also that  or .
\item \textbf{Case  and }. Then . Then we have . By induction hypothesis over , since  and  (note that if  or  a similar reasoning can be done),  we also have that  . This allow us to deduce immediately that  . Hence we also have  . We also directly have  and . Finally since by induction hypothesis we have  or , we deduce also that  or .
\item \textbf{Case  and  and  }. Then . Then we have . By induction hypothesis over , since  and  (note that if  or  a similar reasoning can be done),  we also have that   . This allow us to deduce immediately that    [using ]. Consequently we deduce that . Hence we have  . We also directly have  and . Finally since by induction hypothesis we have  or , we deduce also that  or .
\item \textbf{Case  and  and  }. Then . Then we have . By induction hypothesis over , since  and  (note that if  or  a similar reasoning can be done),  we also have that   . This allow us to deduce immediately that    [using ]. Consequently we deduce that . Hence we have  . We also directly have  and . Finally since by induction hypothesis we have  or , we deduce also that  or .
 \end{enumerate}
 Thanks to the previous case analysis, it is then easy to check that  respects . The case where  can be treated in an equivalence manner. \qed



\end{proof}
Using Lemma \ref{lem-inv} and \ref{lem-strat-win}, we deduce that Duplicator has a winning strategy against any strategy of the Spoiler in , so by Theorem \ref{theorem:EF}, we can conclude the main theorem of this subsection.
\begin{theorem}[Stuttering Theorem]
\label{theorem:stuttering}
Let   such that ,  and  is a non-empty finite word. Then .
\end{theorem}

