

\documentclass[11pt,a4paper]{article}
\usepackage[hyperref]{emnlp2020} \usepackage{makecell}
\usepackage{hyperref}
\usepackage{graphicx}
\usepackage{caption}
\usepackage{subcaption}
\usepackage{latexsym}
\renewcommand{\UrlFont}{\ttfamily\small}
\usepackage{bbm}
\usepackage{amsmath,amsfonts,amsthm,amssymb} \usepackage{bm}
\usepackage{multirow}
\usepackage{microtype}
\usepackage{graphicx}

\aclfinalcopy 



\newcommand\BibTeX{B\textsc{ib}\TeX}
\title{Hierarchical Pre-training for Sequence Labelling in Spoken Dialog}

\author{Emile Chapuis\textsuperscript{\rm 1}\footnote[1]{Equal contribution}, \textbf{Pierre Colombo\textsuperscript{\rm 1,2}\thanks{\rm stands for equal contribution}, Matteo Manica\textsuperscript{\rm 3}}\\ \textbf{Matthieu Labeau\textsuperscript{\rm 1}, Chloe Clavel\textsuperscript{\rm 1}}\\
\textsuperscript{\rm 1}LTCI, Telecom Paris, Institut Polytechnique de Paris, \textsuperscript{\rm 2}IBM GBS France, \textsuperscript{\rm 3}IBM Research Zurich\\
\textsuperscript{\rm 1}firstname.lastname@telecom-paris.fr, \textsuperscript{\rm 3}tte@zurich.ibm.com 
}


\date{}

\begin{document}
\maketitle
\begin{abstract}
Sequence labelling tasks like Dialog Act and Emotion/Sentiment identification are a key component of spoken dialog systems. In this work, we propose a new approach to learn generic representations adapted to spoken dialog, which we evaluate on a new benchmark we call Sequence labellIng evaLuatIon benChmark fOr spoken laNguagE benchmark (\texttt{SILICONE}). \texttt{SILICONE}\footnote{Benchmark can be found in the dataset library from HuggingFace \cite{2020HuggingFace-datasets} at \url{https://huggingface.co/datasets/silicone}} is model-agnostic and contains 10 different datasets of various sizes. We obtain our representations with a hierarchical encoder based on transformer architectures, for which we extend two well-known pre-training objectives. Pre-training is performed on OpenSubtitles: a large corpus of spoken dialog containing over  billion of tokens.
We demonstrate how hierarchical encoders achieve competitive results with consistently fewer parameters compared to state-of-the-art models and we show their importance for both pre-training and fine-tuning.
\end{abstract}

\section{Introduction}











The identification of both Dialog Acts (\texttt{DA}) and Emotion/Sentiment (\texttt{E/S}) in spoken language is an important step toward improving model performances on spontaneous dialogue task.
Especially, it is essential to avoid the generic response problem, i.e., having an automatic dialog system generate an unspecific response --- that can be an answer to a very large number of user utterances~\cite{generic,generic_emo}. \texttt{DA} and emotion identification~\cite{classif,heavy_tailed} are done through sequence labelling systems that are usually trained on large corpora (with over  labelled utterances) such as Switchboard~\cite{datase_swda}, \texttt{MRDA}~\cite{dataset_mrda} or Daily Dialog Act~\cite{dataset_dailydialog}.
Even though large corpora enable learning complex models from scratch (e.g., seq2seq~\cite{colombo2020guiding}), those models are very specific to the labelling scheme employed. Adapting them to different sets of emotions or dialog acts would require more annotated data.\\Generic representations \cite{mikolov2013distributed,pennington2014glove,peters2018deep,bert,xlnet,roberta} have been shown to be an effective way to adapt models across different sets of labels. Those representations are usually trained on large written corpora such as OSCAR~\cite{oscar}, Book Corpus~\cite{book} or Wikipedia~\cite{wiki}. Although achieving state-of-the-art (SOTA) results on written benchmarks~\cite{glue}, they are not tailored to spoken dialog (\texttt{SD}). Indeed, \citet{tran2019role} have suggested that training a parser on conversational speech data can improve results, due to the discrepancy between spoken and written language (e.g., disfluencies~\cite{disfluency}, fillers~\cite{fillers,dinkar2020importance}, different data distribution).
Furthermore, capturing discourse-level features, which distinguish dialog from other types of text~\cite{context}, e.g., capturing multi-utterance dependencies, is key to embed dialog that is not explicitly present in pre-training objectives~\cite{bert,xlnet,roberta}, as they often treat sentences as a simple stream of tokens. \\
The goal of this work is to train on \texttt{SD} data a generic dialog encoder capturing discourse-level features that produce representations adapted to spoken dialog.
We evaluate these representations on both \texttt{DA} and \texttt{E/S} labelling through a new benchmark \texttt{SILICONE} (Sequence labellIng evaLuatIon benChmark fOr spoken laNguagE) composed of datasets of varying sizes using different sets of labels. 
We place ourselves in the general trend of using smaller models to obtain lightweight representations~\cite{tiny,albert} that can be trained without a costly computation infrastructure while achieving good performance on several downstream tasks~\cite{efficient}.
Concretely, since hierarchy is an inherent characteristic of dialog~\cite{context}, we propose the first hierarchical generic multi-utterance encoder based on a hierarchy of transformers. This allows us to factorise the model parameters, getting rid of long term dependencies and enabling training on a reduced number of GPUs.
Based on this hierarchical structure, we generalise two existing pre-training objectives. As embeddings highly depend on data quality~\cite{flaubert} and volume~\cite{roberta}, we preprocess OpenSubtitles~\cite{open}: a large corpus of spoken dialog from movies. This corpora is an order of magnitude bigger than corpora~\cite{budzianowski2018multiwoz,ubuntu,cornell} used in previous works~\cite{mehri2019pretraining,emotion_transfert}. Lastly, we evaluate our encoder along with other baselines on \texttt{SILICONE}, which lets us draw finer conclusions of the generalisation capability of our models\footnote{Upon publication, we will release the code, models and especially the preprocessing scripts to replicate our results.}.
%
 
\section{Method}\label{sec:model}
We start by formally defining the Sequence Labelling Problem. At the highest level, we have a set  of conversations composed of utterances, i.e.,  with  being the corresponding set of labels (e.g., \texttt{DA}, \texttt{E/S}). At a lower level each conversation  is composed of utterances , i.e  with  being the corresponding sequence of labels: each  is associated with a unique label . At the lowest level, each utterance  can be seen as a sequence of words, i.e . Concrete examples with dialog act can be found in \autoref{tab:comp_ex}.

\begin{table}[!htb]\label{tab:kysymys}
\centering
\resizebox{.47\textwidth}{!}{\begin{tabular}{l|c} 
\hline
Utterances & \texttt{DA}  \\
\hline
How long does that take you to get to work? &	\texttt{qw}\\
Uh, about forty-five, fifty minutes.&	\texttt{sd} \\
\makecell[l]{How does that work, work out with, uh,\\storing your bike and showering and all that?}
&	\texttt{qw} \\
Yeah , &	\texttt{b}\\
It can be a pain .	&\texttt{sd}\\
\hline
\makecell[l]{It's, it's nice riding to school because\\ it's all along a canal path, uh,}&	\texttt{sd} \\ 
\makecell[l]{Because it's just, \\it's along the Erie Canal up here.}&	\texttt{sd} \\ 
So, what school is it?&	\texttt{qw}  \\ 
Uh, University of Rochester.&	\texttt{sd} \\ 
Oh, okay.& \texttt{bk} \\
\hline
\end{tabular}}
\caption{Examples of dialogs labelled with \texttt{DA} taken from \texttt{SwDA}. The labels  \texttt{qw}, \texttt{sd}, \texttt{b}, \texttt{bk} respectively correspond to wh-question, statement-non-opinion, backchannel and response acknowledgement.}
\label{tab:comp_ex}
\end{table}


\subsection{Pre-training Objectives}\label{ssec:notations}
Our work builds upon existing objectives designed to pre-train encoders: the Masked Language Model (\texttt{MLM}) from~\citet{bert,roberta,albert,zhang2019hibert} and the Generalized Autoregressive Pre-training (\texttt{GAP}) from \citet{xlnet}.


\noindent\textbf{\texttt{MLM} Loss}: The \texttt{MLM} loss corrupts sequences (or in our case, utterances) by masking a proportion  of tokens. The model learns bidirectional representations by predicting the original identities of the masked-out tokens. Formally, for an utterance , a random set of indexed positions  is selected and the associated tokens are replaced by a masked token \texttt{[MASK]} to obtain a corrupted utterance . The set of parameters  is learnt by maximizing : 
 
where  is the corrupted utterance,    and  is the proportion of masked tokens.

\noindent\textbf{\texttt{GAP} Loss}: the \texttt{GAP} loss consists in computing a classic language modelling loss across different factorisation orders of the tokens. In this way, the model will learn to gather information across all possible positions from both directions. The set of parameters  is learnt by maximising:
 where  is the set of permutations of length  and  represent the first  tokens of  when permuting the sequence according to . 

\subsection{Hierarchical Encoding}
Capturing dependencies at different granularity levels is key for dialog embedding.
Thus, we choose a hierarchical encoder~\cite{crf_chen, sota_swda_1}. It is composed of two functions  and , satisfying:

where  is the embedding of  and  the embedding of . The structure of the hierarchical encoder is depicted in~\autoref{fig:pretraining_fig}.

\subsection{Hierarchical Pre-training}
\subsubsection{General Motivation}
\begin{figure}[ht]
    \centering
    \includegraphics[width=0.46\textwidth]{schema/pretraining_schema.pdf}
    \caption{General structure of our proposed hierarchical dialog encoder, with a decoder: ,  and the sequence label decoder () are colored respectively in green, blue and red.}
    \label{fig:pretraining_fig}\vspace{-.3cm}
\end{figure}
Current self-supervised pre-training objectives such as \texttt{MLM} and \texttt{GAP} are trained at the sequence level, which for us translates to only learning .
In this section, we extend both the \texttt{MLM} and \texttt{GAP} losses at the dialog level in order to pre-train .
Following previous work on both multi-task learning~\cite{multi_task_1,multi_task_2} and hierarchical supervision~\cite{garcia2019token,hierarchy_loss}, we argue that optimising simultaneously at both levels rather than separately improves the quality of the resulting embeddings.
Thus, we write our global hierarchical loss as:

where  is either the \texttt{MLM} or \texttt{GAP} loss at the utterance level and  is its generalisation at the dialog level.



\subsubsection{\texttt{MLM} Loss}
The \texttt{MLM} loss at the utterance level is defined in~\autoref{eq:mlm_loss}. Our generalisation at the dialog level masks a proportion  of utterances and generates the sequences of masked tokens (a concrete example can be found in \autoref{sec:sup_model}).  
Thus, at the dialog level the \texttt{MLM} loss is defined as: 
 
where    is the set of positions of masked utterances in the context ,  is the corrupted context, and  is the proportion of masked utterances.  

\subsubsection{\texttt{GAP} Loss} 
The \texttt{GAP} loss at the utterance level is defined in~\autoref{eq:gap_loss}. A possible generalisation of the \texttt{GAP} at the dialog level is to compute the loss of the generated utterance across all factorization orders of the context utterances. Formally, the \texttt{GAP} loss is defined at the dialog level as: 

where   denotes the first -th tokens of the permuted -th utterance when permuting the context according to   and  the first  utterances of  when permuting the context according to .

\subsection{Architecture}
Commonly, The functions  and  are either modelled with recurrent cells~\cite{hred} or Transformer blocks~\cite{attention_is}. Transformer blocks are more parallelizable, offering shorter paths for the forward and backward signals and requiring significantly less time to train compared to recurrent layers. To the best of our knowledge this is the first attempt to pre-train a hierarchical encoder based only on transformers\footnote{Although it is possible to relax the fixed size imposed by transformers~\cite{dai2019transformer} in this paper we follow~\cite{colombo2020guiding} and fix the context size to  and the max utterance length to  --- these choices are made to work with OpenSubtitles, since the number of available dialogs drops when considering a number of utterances greater than .}. \\
The structure of the model can be found in~\autoref{fig:pretraining_fig}. In order to optimize dialog level losses as described in~\autoref{eq:h_loss}, we generate (through ) the sequence with a Transformer Decoder ().
For downstream tasks, the context embedding  is fed to a simple MLP (simple classification), or to a CRF/GRU/LSTM (sequential prediction) --- see~\autoref{sec:sup_model} for more details. In the rest of the paper, we will name our hierarchical transformer-based encoder  and the hierarchical RNN-based encoder .
We use  to refer to the set of model parameters learnt using the pre-training objective  (either \texttt{MLM} or \texttt{GAP}) at the level \footnote{if  solely utterance level training is used, if  solely dialog level is used and if  multi level supervision is used ( according to the case.)}.



\subsection{Pre-training Datasets}
Datasets used to pre-train dialog encoders~\cite{emotion_transfert,mehri2019pretraining} are often medium-sized (e.g. Cornell Movie Corpus~\cite{cornell}, Ubuntu~\cite{ubuntu}, MultiWOz~\cite{multiwoz}).
In our work, we focus on OpenSubtitles~\cite{opensub}\footnote{http://opus.nlpl.eu/OpenSubtitles-alt-v2018.php} because (1) it contains spoken language, contrarily to the Ubuntu corpus~\cite{ubuntu} based on logs; (2) as Wizard of Oz~\cite{multiwoz} and Cornell Movie Dialog Corpus~\cite{cornell}, it is a multi-party dataset; and (3) OpenSubtitles is an order of magnitude larger than any other spoken language dataset used in previous work.
We segment OpenSubtitles by considering the duration of the silence between two consecutive utterances. Two consecutive utterances belong to the same conversation if the silence is shorter than \footnote{We choose }. Conversations shorter than the context size  are dropped\footnote{Using pre-training method based on the next utterance proposed by \citet{mehri2019pretraining} requires dropping conversation shorter than  leading to a non-negligible loss in the preprocessing stage.}. After preprocessing, Opensubtitles contains subtitles from  movies or series which represent  conversations and over  billion of words. 

\subsection{Baseline Encoder}
We compare the different methods we presented with two different types of baseline encoders: pre-trained encoders, and hierarchical encoders based on recurrent cells. The latter, achieve current SOTA performance in many sequence labelling tasks~\cite{sota_swda_1,colombo2020guiding,self_attention}.

\noindent\textbf{Pre-trained Encoder Models}.
We use BERT~\cite{bert} through the pytorch implementation provided by the Hugging Face transformers library~\cite{Wolf2019HuggingFacesTS}. The pre-trained model is fed with a concatenation of the utterances. Formally given an input context  the concatenation  is fed to BERT. 

\noindent\textbf{Hierarchical Recurrent Encoders}. In this work we rely on our own implementation of the model based on . Hyperparameters are described in~\autoref{sec:sup_model}. 
\section{Evaluation of Sequence Labelling}
\subsection{Related Work}
Sequence labelling tasks for spoken dialog mainly involve two different types of labels: \texttt{DA} and \texttt{E/S}. Early work has tackled the sequence labelling problem as an independent classification of each utterance. Deep neural network models that currently achieve the best results~\cite{bayesian_dialog,svm_dialog,hmm_dialog} model both contextual dependencies between utterances~\cite{colombo2020guiding,concurrent_mrda} and labels~\cite{crf_chen,crf_kumar,crf_li}.

The aforementioned methods require large corpora to train models from scratch, such as: Switchboard Dialog Act (\texttt{SwDA})~\cite{datase_swda},  Meeting Recorder Dialog Act (\texttt{MRDA})~\cite{dataset_mrda}, Daily Dialog Act~\cite{dataset_dailydialog}, HCRC Map Task Corpus~(\texttt{MT})~\cite{dataset_maptask}.
This makes harder their adoption to smaller datasets, such as: Loqui human-human dialogue corpus (\texttt{Loqui})~\cite{dataset_loquihuman}, BT Oasis Corpus (\texttt{Oasis})~\cite{dataset_gtoasis}, Multimodal Multi-Party Dataset (\texttt{MELD})~\cite{dataset_meld}, Interactive emotional dyadic motion capture database (\texttt{IEMO}), SEMAINE database (\texttt{SEM})~\cite{dataset_semaine}. 

\subsection{Presentation of \texttt{SILICONE}}
Despite the similarity between methods usually employed to tackle \texttt{DA} and \texttt{E/S} sequential classification, studies usually rely on a single type of label. Moreover, despite the variety of small or medium-sized labelled datasets, evaluation is usually done on the largest available corpora (e.g., \texttt{SwDA}, \texttt{MRDA}).
We introduce \texttt{SILICONE}, a collection of sequence labelling tasks, gathering both \texttt{DA} and \texttt{E/S} annotated datasets. \texttt{SILICONE} is built upon preexisting datasets which have been considered by the community as challenging and interesting.
Any model that is able to process multiple sequences as inputs and predict the corresponding labels can be evaluated on \texttt{SILICONE}. We especially include small-sized datasets, as we believe it will ensure that well-performing models are able to both distil substantial knowledge and adapt to different sets of labels without relying on a large number of examples.
The description of the datasets composing the benchmark can be found in the following sections, while corpora statistics are gathered in~\autoref{tab:presentation_of_silicon}.


\subsubsection{\texttt{DA} Datasets}

\noindent\textbf{Switchboard Dialog Act Corpus (\texttt{SwDA})} is a telephone speech corpus consisting  of two-sided telephone conversations with provided topics. This dataset includes additional features such as speaker id and topic information. The SOTA model, based on a seq2seq architecture with guided attention, reports an accuracy of ~\cite{colombo2020guiding} on the official split.

\noindent\textbf{ICSI MRDA Corpus (\texttt{MRDA})} has been introduced by~\citet{dataset_mrda}. It contains transcripts of multi-party meetings hand-annotated with \texttt{DA}. It is the second biggest dataset with around  utterances. The SOTA model reaches an accuracy of  \cite{sota_swda_1} and uses Bi-LSTMs with attention as encoder as well as additional features, such as the topic of the transcript. 

\noindent\textbf{DailyDialog Act Corpus ({})} has been produced by~\citet{dataset_dailydialog}. It contains multi-turn dialogues, supposed to reflect daily communication by covering topics about daily life. The dataset is manually labelled with dialog act and emotions. It is the third biggest corpus of \texttt{SILICONE} with  utterances. The SOTA model reports an accuracy of ~\cite{sota_swda_1}, using Bi-LSTMs with attention as well as additional features.
We follow the official split introduced by the authors.

\noindent\textbf{HCRC MapTask Corpus (\texttt{MT})} has been introduced by~\cite{dataset_maptask}. To build this corpus, participants were asked to collaborate verbally by describing a route from a first participant’s map by using the map of another participant. This corpus is small ( utterances). As there is no standard train/dev/test split\footnote{We split according to the code in https://github.com/NathanDuran/Maptask-Corpus.} performances depends on the split. \citet{tran2017generative} make use of a Hierarchical LSTM encoder with a GRU decoder layer and achieves an accuracy of .

\noindent\textbf{Bt Oasis Corpus (\texttt{Oasis})} contains the transcripts of live calls made to the BT and operator services. This corpus has been introduced by~\cite{dataset_gtoasis} and is rather small ( utterances). There is no standard train/dev/test split \footnote{We use a random split from https://github.com/NathanDuran/BT-Oasis-Corpus.} and few studies use this dataset. 

\subsubsection{\texttt{S/E} Datasets}
In \texttt{S/E} recognition for spoken language, there is no consensus on the choice the evaluation metric (e.g., \citet{weighted_preproc,weighted_no} use a weighted F-score while \citet{accuracy_fscore} report accuracy). For \texttt{SILICONE}, we choose to stay consistent with the \texttt{DA} research and thus follow~\citet{accuracy_fscore} by reporting the accuracy. Additionally, emotion/sentiment labels are neither merged nor prepossessed\footnote{Comparison with concurrent work is more difficult as system performance heavily depends on the number of classes and label processing varies across studies \cite{variability}.}.


\noindent\textbf{DailyDialog Emotion Corpus ()} has been previously introduced and contains eleven emotional labels. The SOTA model~\cite{de2019joint} is based on BERT with additional Valence Arousal and Dominance features and reaches an accuracy of 85\% on the official split.

\noindent\textbf{Multimodal EmotionLines Dataset (\texttt{MELD})} has been created by enhancing and extending EmotionLines dataset~\cite{emotion_lines} where multiple speakers participated in the dialogues. There are two types of annotations  and  : three sentiments (positive, negative and neutral) and seven emotions (anger, disgust, fear, joy,neutral, sadness and surprise). The SOTA model with text only is proposed by \citet{accuracy_fscore} and is inspired by quantum physics. On the official split, it is compared with a hierarchical bi-LSTM, which it beats with an accuracy of \% () and \% () against  and .

\noindent\textbf{IEMOCAP database (\texttt{IEMO})} is a multimodal database of ten speakers. It consists of dyadic sessions where actors perform improvisations or scripted scenarios. Emotion categories are: anger, happiness, sadness, neutral, excitement, frustration, fear, surprise, and other. There is no official split on this dataset. One proposed model is built with bi-LSTMs and achieves , with text only \cite{accuracy_fscore}.

\noindent\textbf{SEMAINE database (\texttt{SEM})} comes from the Sustained Emotionally coloured Machine human Interaction using Nonverbal Expression project~\cite{dataset_semaine}. This dataset has been annotated on three sentiments labels: positive, negative and neutral by \citet{barriere2018attitude}. It is built on Multimodal Wizard of Oz experiment where participants held conversations with an operator who adopted various roles designed to evoke emotional reactions. There is no official split on this dataset.
\begin{table*}
\begin{center}
\resizebox{.7\textwidth}{!}{\begin{tabular}{c  c c c c c c c} 
 \hline
 Corpus  &  &  &  & Utt. &  & Task& Utt./ \\ 
  \hline
    &  1k  & 100 & 11 & 200k   & 42 & \texttt{DA} & 4.8k\\
   &  56 & 6 &  12 &110k& 5& \texttt{DA} & 2.6k\\
      &  11k  & 1k & 1k & 102k  & 4 & \texttt{DA}& 25.5k \\
   &  121 & 22 & 25 &  36k & 12& \texttt{DA} & 3k\\
   &  508 & 64 & 64 & 15k & 42 & \texttt{DA} & 357\\
 \hline
     &   11k  & 1k & 1k & 102k  & 7 &\texttt{E}&2.2k \\
     &  934 & 104  &  280 & 13k  & 3 & \texttt{S} & 4.3k\\
        & 934 & 104  &  280 & 13k  & 7 & \texttt{S} & 1.8k\\
    \texttt{IEMO}&  108 & 12 & 31 & 10k & 6 & \texttt{E}&1.7k\\
      \texttt{SEM}  & 62 & 7 & 10 & 5,6k & 3 & \texttt{S}& 1.9k\\
\end{tabular}}
\end{center}
\caption{Statistics of datasets composing  \texttt{SILICONE}. \texttt{E} stands for emotion label and \texttt{S} for sentiment label;  stands for datasets with available official split. Sizes of Train, Val and Test are given in number of conversations. }\label{tab:presentation_of_silicon}
\end{table*}
























\section{Results on \texttt{SILICONE}}
This section gathers experiments performed on the \texttt{SILICONE} benchmark. 
We first analyse an appropriate choice for the decoder, which is selected over a set of experiments on our baseline encoders: a pre-trained BERT model and a hierarchical RNN-based encoder ().
Since we focus on small-sized pre-trained representations, we limit the sizes of our pre-trained models to \texttt{TINY} and \texttt{SMALL} (see \autoref{tab:model_size}). We then study the results of the baselines and our hierarchical transformer encoders () on \texttt{SILICONE} along three axes: the accuracy of the models, the difference in performance between the \texttt{E/S} and the \texttt{DA} corpora, and the importance of pre-training.
As we aim to obtain robust representations, we do not perform an exhaustive grid search on the downstream tasks.
\subsection{Decoder Choice}
Current research efforts focus on single label prediction, as it seems to be a natural choice for sequence labelling problems (\autoref{ssec:notations}).
Sequence labelling is usually performed with CRFs \cite{crf_chen, crf_kumar} and GRU decoding \cite{colombo2020guiding}, however, it is not clear to what extent inter-label dependencies are already captured by the contextualised encoders, and whether a plain MLP decoder could achieve competitive results.
As can be seen in \autoref{tab:baseline}, we found that in the case of \texttt{E/S} prediction there is no clear difference between CRFs and MLPs, while GRU decoders exhibit poor performance, probably due to a lack of training data.
It is also important to notice, that training a sequential decoder usually requires thorough hyper-parameter fine-tuning.
As our goal is to learn and evaluate general representations that are decoder agnostic, in the following, we will use a plain MLP decoder for all the models compared.
\begin{table}[]
\begin{center}
 \resizebox{.4\textwidth}{!}{\begin{tabular}{ c |c c c } 
 \hline
 &Avg  & Avg \texttt{DA} &   Avg \texttt{E/S} \\ \hline
BERT (+MLP) & 72,8& 81.5  & 64.0 \\
BERT (+GRU) &69.9& 80.4 & 59.3 \\ 
BERT (+CRF) &72.8&81.5 & 64.1   \\ \hline
 (+MLP)  &69.8&  79.1 & 60.4   \\
 (+GRU) &67.6& 79.4& 55.7   \\
 (+CRF) & 70.5&80.3 & 60.7   \\
\end{tabular}}
\caption{Experiments comparing decoder performances. Results are given on \texttt{SILICONE} for two types of baseline encoders (pre-trained BERT models and hierarchical recurrent encoders ).}\label{tab:baseline}
\end{center}
\end{table}
\begin{table*}
\setlength{\tabcolsep}{4pt}
\begin{center}
 \resizebox{\textwidth}{!}{
 \begin{tabular}{c || c || c c c  c c | c c c c c } 
 \hline
 & \textbf{Avg} & \texttt{SwDA}  & \texttt{MRDA} &  & &   &    &   &  &  &  \\ \hline
    BERT-4layers& 70.4 & 77.8 & 90.7  & 79.0 &88.4 & 66.8 & 90.3 & 55.3  &53.4 & 43.0 &58.8\\
    BERT& 72.8 & 79.2  &90.7  & \textbf{82.6} & 88.2   &66.9& 91.9 & 59.3  &\textbf{61.4} & \textbf{45.0} &62.7\\
     & 69.8 & 77,5  &90,9 & 80,1 & 82,8 & 64,3 & 91.5 &59,3 & 59.9 & 40.3& 51.1 \\
      &73.3& \textbf{79.3}  &92.0& 80.1 & 90.0   &68,3  & 92.5 &62.6 & 59.9 &42.0&66.6\\
      & 71.6 & 78.6  & 91.8 & 78.1 & 89.3   & 64.1 & 91.6 & 60.5 & 55.7 & 42.2 & 63.9 \\
     & \textbf{74.3} & 79.2  & \textbf{92.4}& 81.5 & \textbf{90.6}   &\textbf{69.4}  & \textbf{92.7} & \textbf{64.1} &60.1& \textbf{45.0} & \textbf{68.2}
\end{tabular}
}
\caption{Performances of different encoders when decoding using a MLP on \texttt{SILICONE}. The datasets are grouped by label type (\texttt{DA} vs \texttt{E/S}) and ordered by decreasing size.  stands for ,  for  and  for .}
\label{tab:results}
\end{center}
\end{table*}

\subsection{General Performance Analysis}
\autoref{tab:results} provides an exhaustive comparison of the different encoders over the \texttt{SILICONE} benchmark.
As previously discussed, we adopt a plain MLP as a decoder to compare the different encoders.
We show that \texttt{SILICONE} covers a set of challenging tasks as the best performing model achieves an average accuracy of .
Moreover, we observe that despite having half the parameters of a BERT model, our proposed model achieves an average result that is  higher on the benchmark.
\texttt{SILICONE} covers two different sequence labelling tasks: \texttt{DA} and \texttt{E/S}.
In \autoref{tab:results} and \autoref{tab:baseline}, we can see that all models exhibit a consistently higher average accuracy (up to ) on \texttt{DA} tagging compared to \texttt{E/S} prediction. This performance drop could be explained by the different sizes of the corpora (see \autoref{tab:presentation_of_silicon}). 
Despite having a larger number of utterances per label (), \texttt{E/S} tasks seem generally harder to tackle for the models. For example, on \texttt{Oasis}, where the  is inferior than those of most \texttt{E/S} datasets (, , \texttt{IEMO} and \texttt{SEM}), models consistently achieve better results.


\subsection{Importance of Pre-training for \texttt{SILICONE}} Results reported in \autoref{tab:results} and \autoref{tab:baseline} show that pre-trained transformer-based encoders achieve consistently higher accuracy on \texttt{SILICONE}, even when they are not explicitly considering the hierarchical structure.
This difference can be observed both in small-sized datasets (e.g. \texttt{MELD} and \texttt{SEM}) and in medium/large size datasets (e.g \texttt{SwDA} and \texttt{MRDA}).
To validate the importance of pre-training in a regime of low data, we train different  (with random initialisation) on different portions of \texttt{SEM} and .
Results shown in \autoref{fig:exp_split_semaine} illustrate the importance of pre-trained representations.
\begin{figure}
  \centering
  \includegraphics[width=0.45\textwidth]{schema/exp_split_semaine.pdf}
    \caption{A comparison of pre-trained encoders being fine-tuned on different percentage the training set of \texttt{SEM}. Validation and test set are fixed over all experiments, reported scores are averaged over 10 different random split.}
    \label{fig:exp_split_semaine}
\end{figure}
\section{Model Analysis}
In this section, we dissect our hierarchical pre-trained models in order to better understand the relative importance of each component. We show how a hierarchical encoder allows us to obtain a light and efficient model. Additional experiments can be found in \autoref{supp:results}.
\subsection{Pre-training on Spoken vs Written Data}
First, we explore the differences in training representations on spoken and written corpora. 
Experimentally, we compare the predictions on \texttt{SILICONE} made by  and the one made by ().
The latter is a hierarchical encoder where utterance embeddings are obtained with the hidden vector representing the first token [CLS] (see \cite{bert}) of the second layer of BERT.
In both cases, predictions are performed using an MLP\footnote{We consider the two first layer for a fair comparison based on the number of model parameters.}.
Results in \autoref{tab:spoken_language} show higher accuracy when the pre-training is performed on spoken data.
Since \texttt{SILICONE} is a spoken language benchmark, this result might be due to the specific features of colloquial speech (e.g. disfluencies, sentence length, vocabulary, word frequencies).
\begin{table}
\begin{center}
 \begin{tabular}{ c | c c } 
 \hline
 &Avg \texttt{DA} & Avg \texttt{E/S} \\ \hline
 BERT (4 layers)&80.5 & 60.2  \\
() & 80.5& 61.1\\
 () &\textbf{80.8} & \textbf{64.0}\\ 
\end{tabular}
\caption{Results of ablation studies on \texttt{SILICONE}}
 \label{tab:spoken_language}
\end{center}
\end{table}
\subsection{Hierarchy and Multi-Level Supervision}
We study the relative importance of three aspects of our hierarchical pre-training with multi-level supervision. We first show that accounting for the hierarchy increases the performance of fine-tuned encoders, even without our specific pre-training procedure.
We then compare our two proposed hierarchical pre-training procedures based on the \texttt{GAP} or \texttt{MLM} loss. Lastly, we look at the contribution of the possible levels of supervision on reduced training data from \texttt{SEM}.
\subsubsection{Importance of hierarchical fine-tuning}
We compare the performance of BERT-4layers with the () previously described.
Results reported in \autoref{tab:spoken_language} demonstrate that fine-tuning on downstream tasks with a hierarchical encoder yields to higher accuracy, with fewer parameters, even when using already pre-trained representations.
\subsubsection{\texttt{MLM} vs \texttt{GAP}} 
In this experiment, we compare the different pre-training objectives at utterance and dialog level.
As a reminder () and () are respectively trained using the standard \texttt{MLM} loss~\cite{bert} and the standard \texttt{GAP} loss~\cite{xlnet}.
In \autoref{tab:pretraining_comparizon} we report the different pre-training objective results.
We observe that pre-training at the dialog level achieves comparable results to the utterance level pre-training for \texttt{MLM} and slightly worse for \texttt{GAP}.
Interestingly, we observe that () compared to  () achieves worse results, which is not consistent with the performance observed on other benchmarks, such as GLUE \cite{glue}.
The lower accuracy of the models trained using a \texttt{GAP}-based loss could be due to several factors (e.g., model size, pre-training using the \texttt{GAP} loss could require a finer choice of hyper-parameters).
Finally, we see that supervising at both dialog and utterance level helps for \texttt{MLM}\footnote{We investigate a similar setting for \texttt{GAP} which lead to poor results, the loss hit a plateau suggesting that objectives are competing against each other. More advanced optimisations techniques \cite{multi_loss_opt} are left for future work.}.
\subsubsection{Multi level Supervision for pre-training}\label{sssec:multi_sup}
In this section, we illustrate the advantages of learning using several levels of supervision on small datasets. We fine-tune different model on \texttt{SEM} using different size of the training set. Results are shown in \autoref{fig:exp_split_semaine}. Overall we see that introducing sequence level supervision induces a consistent improvement on \texttt{SEM}. Results on  are provided in \autoref{supp:results}.
\begin{table}[]
\begin{center}
 \begin{tabular}{ c | c c } 
 \hline
 & Avg \texttt{DA} &   Avg \texttt{E/S} \\ \hline
 & 80.8 & 64.0 \\ & 80.8 & 64.0 \\\hline  & 80.7 & 62.0 \\ & 80.4 & 62.8 \\\hline   & \textbf{81.9} & \textbf{64.7} \\
\end{tabular}
\caption{Comparison of \texttt{GAP} and \texttt{MLM} with a comparable number of parameters. For all models a MLP decoder is used on top of a \texttt{TINY} pre-trained encoder.}
\label{tab:pretraining_comparizon}
\end{center}
\end{table} 
\subsection{Other advantages of hierarchy}
\begin{table}
\begin{center}
\begin{tabular}{ c |c c c c} 
 \hline
 & Emb. & Word &Seq & Total  \\ \hline
  
    BERT & \multirow{5}{*}{23}  & \multicolumn{2}{c}{87}  & 110 \\
        BERT (4-layer) &  & \multicolumn{2}{c}{43}  & 66 \\
    HMLP &   &8.6 & 7.8  & 40  \\
    (\texttt{TINY}) &   &2.9 & 2.8 & 28.7 \\
     (\texttt{SMALL}) &   &10.6 & 10.6 & 45  \\
\end{tabular}
\caption{Number of parameters for the encoders. Sizes are given in million of parameters.}
\label{tab:model_size}
\end{center}
\end{table}
Introducing a hierarchical design in the encoder allows to break dialog into utterances and to consider inputs of size  instead of size . First, it allows parameters sharing, reducing the number of model parameters. The different model sizes are reported in \autoref{tab:model_size}.
Our \texttt{TINY} model contains half the parameters of BERT (4-layers). Furthermore, modelling long-range dependencies hierarchically makes learning faster and allows to get rid of learning tricks (e.g., partial order prediction~\cite{xlnet}, two-stage pre-training based on sequence length~\cite{bert}) required for non-hierarchical encoders.
Lastly, original BERT and XLNET are pre-trained using respectively 16 and 512 TPUs. Pre-training lasts several days with over  iterations. Our \texttt{TINY} hierarchical models are pre-trained during  iterations (1.5 days) on 4 NVIDIA V100. \section{Conclusions}
In this paper, we propose a hierarchical transformer-based encoder tailored for spoken dialog. We extend two well-known pre-training objectives to adapt them to a hierarchical setting and use OpenSubtitles, the largest spoken language dataset available, for encoder pre-training. Additionally, we provide an evaluation benchmark dedicated to comparing sequence labelling systems for the NLP community, \texttt{SILICONE}, on which we compare our models and pre-training procedures with previous approaches.
By conducting ablation studies, we demonstrate the importance of using a hierarchical structure for the encoder, both for pre-training and fine-tuning.
Finally, we find that our approach is a powerful method to learn generic representations on spoken dialog, with less parameters than state-of-the-art transformer models. 

These results open new future research directions: (1) to investigate new pre-training objectives leveraging the hierarchical framework in order to achieve better results on \texttt{SILICONE} while keeping light models (2) to provide multilingual models using the whole pre-training corpus (OpenSubtitles) available in 62 languages, (3) investigate robust methods \cite{robust_staerman} and the application of our embedding to different anomaly detection settings \cite{anomaly_1,anomaly_2}.
We hope that the \texttt{SILICONE} benchmark, experimental results, and publicly available code encourage further research to build stronger sequence labelling systems for NLP.

\section*{Acknowledgement}
This work was supported by a grant overseen from the French National Research Agency (ANR-17-MAOI).


\newpage
\bibliography{emnlp2020}
\bibliographystyle{acl_natbib}
\appendix
\clearpage


\section{Additional Details on data composing \texttt{SILICONE}}\label{supp:copora}
In this section, we illustrate the diversity of the dataset composing \texttt{SILICONE}. In~\autoref{fig:lght_utt}, we plot two histograms representing the different utterance lengths for \texttt{DA} and \texttt{E/S}. As expected, for spoken dialog, lengths are shorter than for written benchmarks (e.g., GLUE).

\begin{figure*}
    \centering
    \begin{subfigure}[b]{0.45\textwidth}   
        \centering 
        \includegraphics[width=\textwidth]{schema/utterance_length_DA.pdf}
        \caption{\texttt{SILICONE} \texttt{DA}}       
        \label{fig:mean and std of net34}
    \end{subfigure}
    \quad
    \begin{subfigure}[b]{0.45\textwidth}   
        \centering 
        \includegraphics[width=\textwidth]{schema/utterance_length_SE.pdf}
        \caption{\texttt{SILICONE} \texttt{S/E}}       
        \label{fig:mean and std of net44}
    \end{subfigure}
    \vskip\baselineskip
    \caption{Histograms showing the utterance length for each dataset of \texttt{SILICONE}.} 
    \label{fig:lght_utt}
\end{figure*}





\section{Additional Details for Models}
\label{sec:sup_model}
In this section we report model hyper-parameters and as well as additional descriptions of our baselines. For all models we use a tokenizer based on WordPiece~\cite{wordpiece}.\\
We also provide a concrete example of corrupted context for the \texttt{MLM} Loss.

\subsection{Hierarchical pre-training}
We report in~\autoref{tab:archi_hyper} the main hyper-parameters used fo our model pre-training. We used GELU~\cite{gelu} activations and the dropout rate \cite{dropout} is set to .
\begin{table}[]
    \centering
    \begin{tabular}{c|cc}\hline
     & \texttt{TINY}& \texttt{SMALL}  \\\hline
     Nbs of heads   & 1& 6  \\
        & 2&4 \\
       &2 & 4\\
       & 50 & 50\\
       & 5 & 5\\
       nbs of heads  &6 &6 \\
      Inner dimension  &768 &768 \\
  Model Dimension   &768 &768 \\
Vocab length  &32000 &32000 \\
  : Emb. size& 768 &768 \\
  :&64 & 64\\
  : &64 & 64
    \end{tabular}
    \caption{Architecture hyperparameters used for the hierarchical pre-training.}
    \label{tab:archi_hyper}
\end{table}
\subsection{\texttt{MLM} Loss example}
In this section we propose a visual illustration of the corrupted context \autoref{fig:corrup_exp} by the \texttt{MLM} Loss.
\begin{figure*}
    \centering
            \begin{subfigure}[b]{0.45\textwidth}   
        \centering 
        \includegraphics[width=\textwidth]{schema/example1.pdf}
        \caption{Initial context composed by 5 utterances.}       
        \label{fig:corrup_exp_initial_ctx}
    \end{subfigure}
    \vskip\baselineskip
    \begin{subfigure}[b]{0.45\textwidth}   
        \centering 
        \includegraphics[width=\textwidth]{schema/example2.pdf}
        \caption{ is chosen to be masked.}       
        \label{fig:corrup_exp_pick_utt_1}
    \end{subfigure}
\quad     
    \begin{subfigure}[b]{0.45\textwidth}  
        \centering 
        \includegraphics[width=\textwidth]{schema/example3.pdf}
        \caption{Corrupted context with utterance  masked.}    
        \label{fig:corrup_exp_mask_utt_1}
    \end{subfigure}
    \vskip\baselineskip
    \begin{subfigure}[b]{0.45\textwidth}   
        \centering 
        \includegraphics[width=\textwidth]{schema/example4.pdf}
        \caption{ is chosen to be masked.}       
        \label{fig:corrup_exp_pick_utt_2}
    \end{subfigure}
\quad       
    \begin{subfigure}[b]{0.45\textwidth}  
        \centering 
        \includegraphics[width=\textwidth]{schema/example5.pdf}
        \caption{Corrupted context with utterance  masked.}    
        \label{fig:corrup_exp_mask_utt_2}
    \end{subfigure}
\caption{This figure shows an example of corrupted context. Here  is randmoly set to 2 meaning that two utterances will be corrupted.  and  are randomly picked in \ref{fig:corrup_exp_pick_utt_1}, \ref{fig:corrup_exp_pick_utt_2} and then masked in  \ref{fig:corrup_exp_mask_utt_1}, \ref{fig:corrup_exp_mask_utt_2}.
    }
    \label{fig:corrup_exp}
\end{figure*}



\subsection{Experimental Hyper-parameters for \texttt{SILICONE}}
For all models, we use a batch size of  and automatically select the best model on the validation set according to its loss.
We do not perform exhaustive grid search either on the learning rate (that is set to ), nor on other hyper-parameters to perform a fair comparison between all the models. We use ADAMW~\cite{adam,adamW} with a linear scheduler on the learning rate and the number of warm-up steps is set to .

\subsection{Additional Details on Baselines}
A representation for all the baselines can be found in~\autoref{fig:baselines_schema_powe}. For all models, both hidden dimension and embedding dimension is set to  to ensure fair comparison with the proposed model. The MLP used for decoding contains 3 layers of sizes . We use RELU~\cite{relu} to introduce non linearity inside our architecture.

\begin{figure*}
    \centering
            \begin{subfigure}[b]{0.475\textwidth}   
        \centering 
        \includegraphics[width=\textwidth]{schema/h_mlp.pdf}
        \caption{Hierarchical encoder with MLP decoder performing single label prediction.}       
        \label{fig:h_mlp}
    \end{subfigure}
    \quad
    \begin{subfigure}[b]{0.41\textwidth}   
        \centering 
        \includegraphics[width=\textwidth]{schema/h_seq.pdf}
        \caption{Hierarchical encoder with sequential decoder (either GRU or CRF).}       
        \label{fig:h_seq}
    \end{subfigure}
            \vskip\baselineskip
            
    \begin{subfigure}[b]{0.475\textwidth}  
        \centering 
        \includegraphics[width=\textwidth]{schema/bert_mlp.pdf}
        \caption{BERT encoder with MLP decoder performing single label prediction.}    
        \label{fig:b_mlp}
    \end{subfigure}
    \hfill
    \begin{subfigure}[b]{0.475\textwidth}
        \centering
        \includegraphics[width=\textwidth]{schema/bert_seq.pdf}
        \caption{BERT encoder with sequential decoder (either GRU or CRF)}    
        \label{fig:b_seq}
    \end{subfigure}



    \caption{Schema of the different models evaluated on \texttt{SILICONE}. In this figure ,  and the sequence label decoder () are respectively colored in green, blue and red for the hierarchical encoder (see \autoref{fig:h_mlp} and \autoref{fig:b_seq}). For BERT there is no hierarchy and embedding is performed through  colored in grey (see \autoref{fig:b_mlp}, \autoref{fig:b_seq})} 
    \label{fig:baselines_schema_powe}
\end{figure*}





\section{Additional Experimental Results}
\label{supp:results}
In this section we report the detailed results on \texttt{SILICONE}, including the ones presented in~\autoref{tab:results}. We report results on two new experiments: importance of pre-training time for both a TINY and SMALL model, we report the convergence time of a TINY model and finally we extend~\autoref{sssec:multi_sup} by reporting results on \texttt{IEMO}.

\subsection{Detailed Results on \texttt{SILICONE}}
We show in~\autoref{tab:full_results} the results on the \texttt{SILICONE} benchmark for all the models mentioned in the paper.

\begin{table*}
\begin{center}
 \resizebox{\textwidth}{!}{\begin{tabular}{c || c || c c c  c c | c c c c c } 
 \hline
 & \textbf{Avg} & \texttt{SwDA}& \texttt{MRDA} &   & \texttt{MT} & \texttt{Oasis} &   &   &  & \texttt{IEMO} &  \texttt{SEM}  \\ \hline
      BERT-4layers (+MLP)& 69.45 & 77.8 & 90.7 & 79.0  &88.4 & 66.8 & 90.3 & 49.3  &50.4 & 43.0 &58.8\\
    BERT (+MLP)& 72.79 & 79.2  &90.7 & 82.6 & 88.2   &66.9 & 91.9 & 59.3  &61.4 & 45.0 &62.7\\
    BERT (+GRU)& 69.84 & 78.2  &90.4 & 80.8 & 88.7   &63.7 & 90 & 50.4  &48.9 & 45.0 &62.3\\
        BERT (+CRF)& 72.8 & 79.0  &90.8 & 88.3   &67.2  &   81.9& 91.5 & 59.4  &61.0 & 44.2 &61.5\\ \hline
      (+MLP)& 69.77 & 77,5  &90,9 & 80,1 & 82,8 & 64,3 & 91.5  &59,3 & 59.9 & 40.3& 51.1\\
      (+GRU)&67.54 & 78.2 & 90.9 & 79,9 & 84,4   &63,5 & 91.5& 50,7 & 50.4 & 35.2 & 50.7  \\
      (+CRF)& 70.5 & 77.8  &91,3 & 79,7 & 87,5   & 65,3  & 91,1  &62,1 & 57,4  & 42.1 &50.7\\ \hline
      (TINY)  &73.3& 79.3  &92.0 & 80.1 & 90.0   &68,3  & 92.5  &62.6 & 59.9 &42.0&66.6\\
     (TINY)& 72.4 & 78.5  &91.8  & 78.0 & 89.8   &66.0  & 92.5  &62.6 & 59.3 &42.0&63.5\\
     (TINY)& 72.4 & 78.6  &91.8 & 79.0 & 89.8   &65.0  & 91.8  &61.8 & 58.1 &39.2&68.9\\
        HBERT (w)  (TINY)& 70.8 & 77.6  &91.4 & 79.3 & 88.3   &65.8  & 91.9 &58.0 & 56.3 &40.0&59.1 \\ 
            (SMALL)& 74.32 & 79.2  & 92.4 & 81.5 & 90.6   &69.4  &92.7 & 64.1 &60.1& 45.0 & 68.2 \\ \hline

       (TINY) & 71.58 & 78.6  & 91.8 & 78.1 & 89.3   & 64.1 & 91.6 & 60.5 & 55.7 & 42.2 & 63.9 \\ (TINY)&71.52 & 78.5  & 90.9 &79.0  & 88.9 & 66.3  &92.0 & 59.2 &57.5& 39.9 & 63.0 \\\hline


\end{tabular}}

\caption{Performances of all mentioned model with different decoders such as MLP, GRU, CRF  \texttt{SILICONE}. The datasets are grouped by
label type (\texttt{DA} vs \texttt{E/S}) and order by decreasing size.}
\label{tab:full_results}
\end{center}
\end{table*}

\subsection{Improvement over pre-training}
In this experiment we illustrate how pre-training improves performance on \texttt{SEM} (see~\autoref{fig:SEMAINE}). As expected accuracy improves when pre-training.  

\begin{figure}
  \centering
  \includegraphics[width=\linewidth]{schema/itterations.pdf}
\caption{Illustration of improvement of accuracy during pre-training stage on \texttt{SEM} for both a \texttt{TINY} and \texttt{SMALL} model.}
  \label{fig:SEMAINE}
\end{figure}

\subsection{Multi level Supervision for pre-training \texttt{MELD}} In this experiment we report results of the experiment mentioned in~\autoref{sssec:multi_sup}. In this experiment we see that the training process seems to be noisier for fractions lower than 40\%. For larger percentages, we observe that including higher supervision (at the dialog level) during pre-training leads to a consistent improvement.
\begin{figure}
  \centering
  \includegraphics[width=0.4\textwidth]{schema/exp_split_meld_s.pdf}
    \caption{A comparison of different parameters initialisation on . Training is performed using a different percentage of complete training set. Validation and test set are fixed over all experimentation. Each score  is the averaged accuracy over 10 random runs.}
    \label{fig:exp_split_meld_s}
\end{figure}




\section{Negative Results on \texttt{GAP}}
\label{sec:sup_neg_res}
We briefly describe few ideas we tried to make \texttt{GAP} works at both the utterance and dialog level. We hypothesise that:
\begin{itemize}
    \item giving the same weight to the utterance level and the dialog level (see \autoref{eq:multi_obj}) was responsible of the observed plateau. Different combinations lead to fairly poor improvements.
    \item the limited model capacity was part of the issue. Larger models does not give the expected results.
\end{itemize}
 




\end{document}
