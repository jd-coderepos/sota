\documentclass[letterpaper, 10pt]{article}

\usepackage{geometry}
\usepackage{graphics} \usepackage{epsfig} \usepackage{mathptmx} \usepackage{times} \usepackage{amsthm}
\usepackage{amsmath} \usepackage{amssymb, color}  \usepackage{stmaryrd}
\usepackage{soul}
\newtheorem{theo}{Theorem}
\newtheorem{lem}{Lemma}
\newtheorem{claim}{Claim}
\newtheorem{assert}{Assertion}
\newtheorem{defi}{Definition}
\newtheorem{propo}{Proposition}
\newtheorem{remark}{Remark}
\newtheorem*{pb}{Problem (BHD)}
\newtheorem{coro}{Corollary}

\usepackage[dvipsnames]{xcolor}

\newcommand{\identity}{\mathbb{I}}
\newcommand{\rref}[1]{(\ref{#1})}
\newcommand{\ical}{\mathcal{I}}
\newcommand{\scal}{\mathcal{S}}
\newcommand{\gcal}{\mathcal{G}}
\newcommand{\ucal}{\mathcal{U}}
\newcommand{\kk}{\mathcal{K}}
\newcommand{\norme}[1]{\left\Vert #1\right\Vert}
\newcommand{\abs}[1]{\left| #1 \right|}
\newcommand{\reels}{\mathbb{R}}
\newcommand{\complexe}{\mathbb{C}}
\newcommand{\entiers}{\mathbb{N}}
\newcommand{\sign}{\text{sign}}
\newcommand{\siss}{\text{SISS}}

\newcommand{\pcal}{\mathcal{P}}
\title{\Large\bf
Global stabilization of linear systems with bounds on the feedback and its successive derivatives
}


\author{Jonathan Laporte, Antoine Chaillet and Yacine Chitour \thanks{This research was partially supported by a public grant overseen by the French ANR as part of the “Investissements d'Avenir” program, through the iCODE institute, research project funded by the IDEX Paris-Saclay, ANR-11-IDEX-0003-02.}\thanks{J. Laporte, A. Chaillet  and  Y. Chitour  are with L2S - Univ. Paris Sud - CentraleSup\'elec. 3, rue Joliot-Curie. 91192 - Gif sur Yvette, France.
        {\tt\small jonathan.laporte, antoine.chaillet, yacine.chitour@l2s.centralesupelec.fr}}}


\begin{document}
\newcommand{\AC}[1]{\textcolor{blue}{#1}}
\newcommand{\JL}[1]{\textbf{\textcolor{orange}{#1}}}
\newcommand{\ACC}[1]{\textcolor{BlueViolet}{#1}}

\date{}
\maketitle


\section*{Abstract}
We address the global stabilization of linear time-invariant (LTI) systems when the magnitude of the control input and its successive time derivatives, up to an order , are bounded by prescribed values. We propose a static state feedback that solves this problem for any admissible LTI systems, namely for stabilizable systems whose internal dynamics has no eigenvalue with positive real part. This generalizes previous work done for single-input chains of integrators and rotating dynamics. 

\section{Introduction}
The study of control systems subject to input constraints is motivated by the fact that signals delivered by physical actuators may be limited in amplitude, and may not evolve arbitrarily fast. An a priori bound on the amplitude of the control signal is usually referred to as \emph{input saturation} whereas a bound on the variation of control signal is referred to as \emph{rate saturation} (e.g \cite{saberi2012}).


Stabilization of linear time-invariant systems (LTI for short) with input saturation has been widely studied in the literature. Such a system is given by  where , 
 belongs to a bounded subset of ,  is an  matrix and  is an  one. Global stabilization of  can be achieved if and only if the LTI system is asymptotically null controllable with bounded controls, i.e., it can be stabilized in the absence of input constraint and the eigenvalues of  have non positive real parts. Saturating a linear feedback law may fail at globally stabilizing  as it was observed first in \cite{FULLER69} and then \cite{SY91} for the special case of integrator chains (i.e., when  is the -th Jordan block and ). As shown for instance in \cite{OptCRyan}, optimal control can be used to define a globally stabilizing feedback  for  but, when the dimension is greater than , deriving a closed form for this stabilizer becomes extremely difficult. The first globally stabilizing feedback with rather simple closed form (nested saturations) was provided in \cite{Teel92} for chains of integrators and then in \cite{SSY} for the general case. In \cite{Lin95control}, a global feedback stabilizer for  was built by relying on control Lyapunov functions arising from a mere existence result. Other globally stabilizing feedback laws for  have been proposed with an additional property of robustness with respect to perturbations. In \cite{Saberi:2002ux}, using low-and-high gain techniques, a robust stabilizer was proposed to ensure semiglobal stability, meaning that the control gains can be tuned in such a way that the basin of attraction contains any prescribed compact subset of . This restriction has been removed in \cite{saberi2000}, where the authors provided a global feedback stabilizer for  which is robust with respect to perturbations, based on an earlier idea due to Megretsky \cite{Megretski96bibooutput}. Nonetheless, the feedback laws of \cite{saberi2000} and \cite{Megretski96bibooutput} require to solve a nonlinear optimization problem at every point , which makes its practical implementation questionable. In \cite{chitour2015}, an easily implementable global feedback stabilizer for  which is robust with respect to perturbations was proposed but it only covers the multiple integrator case and it is discontinuous since it is based on sliding mode techniques. Robust stabilization of  was also addressed in \cite{AZCHCHGR15} by relying on the control Lyapunov techniques developed in \cite{Lin95control}.

In contrast to stabilization of LTI systems subject to input saturation, there are much less results available in the literature regarding global stabilization under rate saturation, i.e., when the first time derivative of the control signal is also {\it a priori} bounded. In \cite{Freeman:1998tp}, the authors rely on a backstepping procedure to build a bounded globally stabilizing feedback with a bounded rate, but the methodology does not allow to {\it a priori} impose a prescribed rate. In \cite{SFbound}, a dynamic feedback law inspired from \cite{Megretski96bibooutput} is constructed
and can even be generalized to take into account constraints on higher time derivatives of the control signal. However, as mentioned previously, the numerical efficiency of such feedbacks is definitely questionable.  A rather involved global feedback stabilizer for  achieving amplitude and rate saturations was also obtained in \cite{SoSuAL} for continuous time affine systems with a stable free dynamics. This corresponds in our setting to requiring that the matrix  is stable, i.e.,  (up to similarity). Finally, let us mention the references \cite{lauvdal97}, \cite{lin1997semi} for semiglobal stabilization results and \cite{SilvaTarbouch03} for local stabilization results using LMIs and anti-windup design. One should also mention \cite{teel1996nonlinear} where a nonlinear small gain theorem is given for the behaviour analysis of control systems with saturation.

The results presented here encompass input and rate saturations as special cases. More precisely, given an integer , we construct a globally stabilizing feedback for  such that the control signal and its  first time derivatives, are bounded by arbitrary prescribed positive values, along all trajectories of the closed-loop system. This problem has already been solved by the authors in \cite{LCC1} for the multiple integrator and skew-symmetric cases. The solution given in that paper for the multiple integrator case consisted in considering appropriate nested saturation feedbacks. We also indicated in \cite{LCC1} that these feedbacks fail at ensuring global stability in the skew-symmetric case and we then provided an {\it ad hoc} feedback law for this specific case. Here, we solve the general case with a unified strategy.


The paper should be seen as a first theoretical step towards the global stabilization of an LTI system when the input signal is delivered by a dynamical actuator that limits the control action in terms of magnitude and  first time derivatives. Further developments are needed to explicitly take into account the dynamics of such an actuator. Possible extensions of this work may also address the question of global stabilization by smooth feedback laws (i.e.,  with respect to time) when \emph{all} successive derivatives need to be bounded by prescribed values.


The paper is organized as follows. In Section \ref{sec: main}, we precisely state the problem we want to tackle, the needed definitions as well as the main results we obtain, namely Theorem~\ref{th:main} for the single input case and Theorem~\ref{th:mi} for the multiple input case. Section \ref{sec:proofs} contains the proof of the main results. In section \ref{sec:red:th1} we show that the proof of Theorem \ref{th:main} is a consequence of two propositions. The first one (cf. Proposition~\ref{prop:SISSl}), we show that the feedback proposed in Theorem~\ref{th:main} is indeed a globally stabilizing feedback for . We actually prove a stronger result dealing with robustness properties of this feedback, as it is required in \cite{Teel92} and \cite{SSY}. The second proposition (cf. Proposition~\ref{prop:bound:U}) specifically deals with bounding the  first derivatives of the control signal by relying on delicate estimates. Section \ref{sec:red:th2} contains the proof of Theorem \ref{th:mi} which is a consequence of Proposition \ref{prop:SISSl} and Proposition \ref{prop:bound:U:mi}, the latter providing estimates on the successive time derivatives of the control signal. We close the paper by an Appendix, where we gather several technical results used throughout the paper.



\paragraph{Notations :} We use  and  to denote the sets of real numbers and the set of non negative integers respectively. Given a set  and a constant , we let . Given , we define . For a given set ,  the boundary of  is denoted by . The factorial of  is denoted by  and the binomial coefficient is denoted  .

Given  and , we say that a function  is of class  if its differentials up to order  exist and are continuous, and we use  to denote the -th order differential of . By convention, . 

Given ,  denotes the set of  matrices with real coefficients. The transpose of a matrix  is denoted by . The identity matrix of dimension  is denoted by . We say that an eigenvalue of  is {\it critical} if it has zero real part and we set  where  is the number of conjugate pairs of nonzero purely imaginary eigenvalues of  (counting multiplicity), and  is the multiplicity of the zero eigenvalue of . We define , and . 

We use  to denote the Euclidean  norm of an arbitrary vector . Given  and , we say that  is eventually bounded by , and we write , if there exists  such that  for all .
\section{Problem statement and main results}\label{sec: main}


Given  and ,  consider the LTI system defined by

where , , , and . Assume that the pair  is stabilizable and that all the eigenvalues of  have non positive real parts. Recall that these assumptions on  are necessary and sufficient for the existence of a bounded continuous state feedback  which globally asymptotically stabilizes the origin of (\ref{sys:linear}): see \cite{SSY}. 

Given an integer  and a -tuple of positive real numbers , we want to derive a feedback law whose magnitude and -first time derivatives are bounded by , .
\begin{defi}[\emph{feedback law -bounded by }]
Given ,  and , let  be a -tuple of positive real numbers. We say that  is a \emph{feedback law -bounded by   for system \rref{sys:linear}} if it is of class  and, for every trajectory of the closed-loop system , the control signal ,  satisfies  for all . The function  is said to be a feedback law -bounded for system \rref{sys:linear}, if there exist -tuple of positive real numbers  such that  is a feedback law -bounded by   for system \rref{sys:linear}.
\end{defi}

Based on this definition, we can write our stabilization problem of Bounded Higher Derivatives as follows.
\begin{pb}
Given  and a -tuple of positive real numbers , design a feedback law  such that the origin of the closed-loop system  is globally asymptotically stable (GAS for short) and the feedback  is a feedback law -bounded by  for system \rref{sys:linear}.
\end{pb}

Our construction to solve Problem (BHD) will often use the property of \textit{Small Input Small State with linear gain} ( for short) developed in \cite{SSY}. We recall below its definition



\begin{defi}[\emph{, \cite{SSY}}]
Given  and , the control system , with  and , is said to be \emph{} if, for all  and all bounded measurable signal   eventually bounded by , every solution of  is eventually bounded by .
A system is said to be \emph{} if it is  for some . An input-free system  is called , if the control system  is . 
\end{defi}
 
 \begin{remark}\label{rem1}
It follows readily from this definition that if  is , then all solutions  converge to the origin. Note, however, that the  property does not necessarily ensure GAS in the absence of input, as it does not imply stability of its origin.
\end{remark}

When a feedback law ensures both global asymptotic stability and , we refer to is an -stabilizing feedback.

\begin{defi}[\emph{-stabilizing feedback}]\label{def:sissfeedback}
Given a control system  with  and , we say that a feedback law  is \emph{stabilizing} if the origin of the closed-loop system  is globally asymptotically stable. If, in addition, this closed-loop system is , then we say that  is \emph{ -stabilizing}.
\end{defi}




As mentioned before the feedback law given in  \cite{LCC1}, which solves Problem (BHD) for the special case of multiple integrators, simply made use of nested saturations with carefully chosen saturation functions. We recall next why this feedback construction cannot work in general. For that purpose it is enough to consider the 2D simple oscillator case which is the control system given by , with ,  and . This system is one of the two basic systems to be stabilized by means of a bounded feedback, as explained in \cite{SSY}. One must then consider a stabilizing feedback law , where  is a fixed vector in  and  is a saturation function, i.e., a bounded, continuously differentiable function satisfying  for  and . Note that  is chosen so that the linearized system at  is Hurwitz. In particular it implies that . Pick now the following sequence of initial conditions . A straightforward computation yields that the first time derivative of the control along each trajectory satisfies , which grows unbounded as  tends to infinity. Therefore this feedback can not be a -bounded feedback.

In order to solve Problem (BHD) for the  oscillator, we showed in \cite{LCC1} that a feedback law of the type  with  and  does the job and it also solves Problem (BHD) in case the matrix  in \eqref{sys:linear} is stable. However, we are not able to show whether  stabilizes or not the system in the case where . It turns out that the previous issue is as difficult as asking if a saturated linear feedback stabilizes or not the abovementioned 4D case, which is an open problem.  It is therefore not immediate how to address the general case. This is why Theorem~\ref{th:main} is a non trivial extension of the solution of Problem (BHD) provided for the two-dimensional oscillator.


\subsection{Single input case}

For the case of single input systems the solution of Problem (PHB) is given by the following statement.

\begin{theo}[Single input]
\label{th:main}
Given , consider a single input system  where ,  and . Assume that  has no eigenvalue with positive real part and that the pair  is stabilizable. Then, given any  and any -tuple  of positive real numbers, there exist vectors  and matrices , , such that the feedback law  defined as
 
is a feedback law -bounded by  and -stabilizing for system .
\end{theo} 

In view of Definition \ref{def:sissfeedback}, the feedback law \rref{th:feed} globally asymptotically stabilizes the origin of \rref{sys:linear}, and thus solves Problem (BHD). We stress that, even though the exact computation of the control gains  is quite involved (see proof in Section \ref{sec:proofs}), the structure of the proposed feedback law \rref{th:feed} is rather simple. It should also be noted that, unlike the results developed in \cite{LCC1}, this feedback law applies to any admissible single-input systems in a unified manner.



\subsection{Multiple input case}
To give the main result for LTI system with multiple input we need this following definition.
\begin{defi}[Reduced controllability form]
Given  and , a LTI system is said to be in \emph{reduced controllability form} if it reads
 where, for some -tuple  in  with ,  is Hurwitz, for every  all the eigenvalues of  are critical,  and the pairs  are controllable. 
\end{defi}
From Lemma  in \cite{SSY}, it is then clear that without loss of generality, in our case, we can consider that system \rref{sys:linear}  is already given in the reduced controllability form. We can now establish the solution of Problem (BHD) for the multiple input case.

\begin{theo}[Multiple input]
\label{th:mi}
Let  and -tuple  of positive real numbers. Given  and , consider system \rref{lin:sys:mi}. Then, there exist  feedback laws  such that:
\begin{itemize}
\item[i)] for every ,  is a feedback law -bounded and -stabilizing for ;
\item[ii)] the feedback law  given by 
 
is a feedback law -bounded by  and -stabilizing for system \rref{lin:sys:mi}.
\end{itemize}
\end{theo}

This statement provides a unified control law solving Problem (BHD) for all admissible LTI systems. It allows in particular multi-input systems, which was not covered in \cite{LCC1}.

\section{Proof of the main results}\label{sec:proofs}

\subsection{Proof of Theorem \ref{th:main}}
In this section, we prove Theorem \ref{th:main}. For that purpose, we first reduce the argument to establishing of Propositions ~\ref{prop:SISSl} and \ref{prop:bound:U} given below. The first one indicates that the feedback given in Theorem \ref{th:main} is  stabilizing for  in the case of single input. The second proposition provides an estimate of the successive time derivatives of the control signal.
\subsubsection{Reduction of the proof of Theorem \ref{th:main} to the proofs of Propositions~\ref{prop:SISSl} and \ref{prop:bound:U}}\label{sec:red:th1}

Let ,  and  be a -tuple of positive real numbers. Define . Consider a single input linear system  where ,  and  are  and  matrices respectively. We assume that the pair  is stabilizable and that all the eigenvalues of  have non positive real parts. As observed in \cite{SSY}, it is sufficient to consider the case where the pair  is controllable and all eigenvalues of  are critical. Indeed, since  is stabilizable there exists a linear change of coordinates transforming  and  into  and ,  where  is Hurwitz, the eigenvalues of  are critical and the pair  is controllable. Then, it is immediate to see that we only have to treat the case where  has only critical eigenvalues. From now on, we therefore assume that  has only eigenvalues with zero real parts, and that the pair  is controllable.

Our construction uses the following linear change of coordinates given by \cite[Lemma 5.2]{SSY}. This decomposition puts the original system in a triangular form made of one-dimensional integrators and two-dimensional oscillators.
\begin{lem}[Lemma  in \cite{SSY}]
\label{lem:SSY}
Let , , , be a controllable single input linear system. Assume that all the eigenvalues of  are critical. Let  be the nonzero eigenvalues of . Let  be a family of positive numbers. Define 

Then there is a linear change of coordinates that puts  in the form
 where  for  , and  for .
\end{lem}

With no loss of generality, we prove Theorem~\ref{th:main} for system \rref{lem:sys}, where the positive constants  will be fixed later. Let  be a positive constant. We rely on a candidate feedback  under the form 
 
with 
 
It therefore remains to choose the positive constants  such that the feedback law \rref{feed:Pj} is a feedback law -bounded by , and  -stabilizing for system \rref{lem:sys}. For that aim, we rely on the next two propositions, respectively proven in Sections \ref{sec:proofprop1} and \ref{sec:proofprop2}. 

\begin{propo}[]
\label{prop:SISSl}
Let , , , be a controllable single input linear system. Assume that all the eigenvalues of  are critical. Let  be the nonzero eigenvalues of . Then, there exist  functions ,  such that for any constants  satisfying

 the feedback law \rref{feed:Pj} is -stabilizing for system \rref{lem:sys}.
\end{propo}

\begin{propo}[]
\label{prop:bound:U}
Let , , , be a controllable single input linear system. Assume that all the eigenvalues of  are critical. Let  be the nonzero eigenvalues of . Let , , be positive constants in .
Then, there exist a positive constant , and continuous functions , , such that for any trajectory of the closed-loop system \rref{lem:sys} with the feedback law \rref{feed:Pj}, the control signal  defined by  for all  satisfies, for all ,

\end{propo}

Pick  in such a way that
 
Choose recursively , , such that
 
where the functions  appearing above are defined in Proposition~\ref{prop:bound:U}.
By Proposition~\ref{prop:SISSl}, the feedback law \rref{feed:Pj} is -stabilizing for system \rref{lem:sys}. Moreover, as a consequence of Proposition~\ref{prop:bound:U}, for any trajectory of the closed-loop system \rref{lem:sys} with the feedback law \rref{feed:Pj}, the control signal  defined by  for all  satisfies  for all . Thus, the feedback law \rref{feed:Pj} is a  feedback law -bounded by  for system \rref{lem:sys}. Since there is a linear change of coordinate () that puts \rref{lem:sys} into the original form , the feedback law defined given in \rref{th:feed} can be picked as
 
and it is a feedback law -bounded by , and  -stabilizing for \eqref{sys:linear}. To sum up, the proof of Theorem~\ref{th:main} boils down to establishing Propositions~\ref{prop:SISSl} and \ref{prop:bound:U}.



\subsubsection{Proof of Proposition \ref{prop:SISSl}}\label{sec:proofprop1}
Proposition \ref{prop:SISSl} is proved by induction on . More precisely, we show that the following property holds true for every positive integer .
\begin{itemize}
\item[ : ] Given any , let  be such that  and  be positive constants. Then there exist  functions ,  such that for any constants  satisfying
 
the feedback law \rref{feed:Pj} is -stabilizing for system \rref{lem:sys}, with , , and . Moreover the linearization of this closed-loop system around the origin is asymptotically stable.
\end{itemize}
In order to start the argument, we give intermediate results whose proofs are given in Appendix
and which will be used for the initialization step of the induction and the inductive step. The first statement establishes  for the one-dimensional integrator.
\begin{lem}
\label{lem:SISS:int}
Let . For every , the scalar system given by

is  , its origin is  and its linearisation around zero is . 
 \end{lem}
The next lemma guarantees that the two-dimensional oscillator is .
\begin{lem}
\label{Lem:osci}
For every ,  there exist  such that for any  the two-dimensional system given by
 
is , its origin is  and its linearisation around zero is . 
\end{lem}
We now start the inductive proof of . For , we have to consider two cases. Either  and  corresponding to the simple integrator
 or  and  corresponding to the simple oscillator 
for some .
In both cases,  can be readily deduced by invoking Lemma \ref{lem:SISS:int} and \ref{Lem:osci} respectively. Given , assume that  holds. In order to establish , it is sufficient to consider the following two cases:
\begin{itemize}
\item[{\bf case i)}] , i.e, all the eigenvalues of  are zero (multiple integrator);
\item[{\bf case ii)}]  , i.e some eigenvalues of  have non zero imaginary part (multiple integrator with rotating modes).
\end{itemize} In both cases we reduce our problem to the choice of only one constant  using the inductive hypothesis.
\paragraph{case i)} Let  be a set of positive numbers to be chosen later. Consider the multiple integrator given by
 where  for . Let . We then can rewrite this system as 
 for some matrices  and  of appropriate dimensions. From the inductive hypothesis, there exist  functions  for  such that for any set of positive constants  satisfying  satisfying  and   , for each ,
the feedback law  defined by 
 is -stabilizing for . Choose  satisfying the above conditions. The feedback law \rref{feed:Pj} is then given by
 Since  for all (see \rref{def:theta} and \rref{def:consA}), the closed-loop system can be rewritten as

 with

We now move to the other case where the dynamics involves multiple integrators with rotating modes.
\paragraph{case ii)} Let  be a set of positive constants to be chosen later. Let , and  be such that . Let  be a set of non zero real numbers. Consider the following linear control system
 where  for  , and  for .  Let . We then can rewrite this system as follows 
 From the inductive hypothesis, there exist  functions  for   such that for any set of positive constant  satisfying  and   , for each ,
the feedback law  defined by
 is -stabilizing for . Choose  satisfying the above conditions. The feedback law \rref{feed:Pj} is then given by
 By noticing that  for all   (see \rref{def:theta} and \rref{def:consA}), the closed-loop system can be rewritten as
 with 


\paragraph{}
In both cases, it remains to show that there exists a function  such that if  then the closed-loop systems \rref{pr:th:SYS:casei} and \rref{pr:th:SYS:caseii}  are , globally asymptotically stable with respect to the origin, and theirs respective linearizations at zero are asymptotically stable. It is sufficient to prove that the closed-loop systems are  and their linearization at zero are asymptotically stable. Indeed, from Remark \ref{rem1}, the  property guarantees the convergence of any solution of the closed-loop with no input. If moreover the linearized system is asymptotically stable, then the globally asymptotic stability of zero follows readily.

For any , the linearization at zero of the -subsystem in \rref{pr:th:SYS:casei} (respectively  \rref{pr:th:SYS:caseii}) is asymptotically stable since it is given by  (respectively ). Moreover, the linearization at zero of the -subsystem in \rref{pr:th:SYS:casei} (respectively  \rref{pr:th:SYS:caseii}) is given by  (respectively ). Due to the inductive hypothesis, the origin of  is asymptotically stable. Thus, local asymptotic stability of \rref{pr:th:SYS:casei} and  \rref{pr:th:SYS:caseii} follows easily.






It remains to prove that systems \rref{pr:th:SYS:casei} and \rref{pr:th:SYS:caseii} are . In both cases, using that  for all , it holds from \eqref{def:g1:cii} and \eqref{g_1:est} that
 and from \rref{def:ro1:cii} and \rref{ro_1:est} that




Recall that, due to the inductive hypothesis,  is  for some  and . We next prove the  property for {\bf case ii)}.

Let 
 From Lemma \ref{Lem:osci} (with ), there exist  such that for any  the system
 is . Define
and choose .  Let 

Given , let  and  be two bounded measurable functions, eventually bounded by . Consider any trajectory  of the following system
In view of \rref{pr:th:SYS:caseii}, \rref{ro_1:est}, \rref{g_1:est}, \rref{f_1:est} and \rref{feed_int} the above system is clearly forward complete. We next show that there exists a constant  such that  and . 
From \rref{f_1:est} and recalling that , a straightforward computation yields
 Since , it follows that
Moreover from \rref{ch:a1:cii}, \rref{ch:delta:cii} and  it follows that
 where  is defined in \rref{def:C:cii}. Using the  property of System , it follows  that the solution of \rref{proof:prop1:SYS:caseii:pert} satisfies 
 
Consequently, using \rref{SISS:est:ro(y)} and \rref{SISS:est:G(y)}, it follows that

Using \rref{ch:a1:cii}, we have . Moreover \rref{ch:delta:cii} ensures that . So it follows that
 The  property of  ensures that


Now let  be defined as
 Then . There are two cases to consider, either  or . In the case when , we have
 So invoking again the  property of , one gets that the solution of \rref{proof:prop1:SYS:caseii:pert} satisfies

 In the case when , the estimate \rref{est:y_1:cii:2} follows readily from \rref{est:y_1:cii:1}. Exploiting again the  property of System , it follows that
 It then follows from \rref{ch:a1:cii} that
 Taking the limsup of the above estimate, we get from \rref{eq:proof:prop1:limsup} that
 Consequently, we obtain that
 which finishes to establish  for the case . Proceeding as in case , it can be shown that system \rref{pr:th:SYS:casei} is . This end the inductive proof of . 
\newline



\subsubsection{Proof of Proposition \ref{prop:bound:U}}\label{sec:proofprop2}
Fix . Let  and  be two integers such that ,  be positive constant numbers, and  be positive numbers less than or equal to 1. Consider the system \rref{lem:sys} with the feedback law \rref{feed:Pj}, where ,  and . We establish Proposition \ref{prop:bound:U} by induction on . More precisely we prove the following statement:
\begin{itemize}
\item[ :] For each , there exist a positive constant  and continuous functions , , such that for any trajectory  of the closed-loop system \rref{lem:sys} with the feedback law \rref{feed:Pj}, the control signal  defined by  for all  satisfies, for all ,

\end{itemize}

For , this statement () holds trivially. Indeed, it is easy to see that for any trajectory of the closed-loop system \rref{lem:sys} with the feedback law \rref{feed:Pj} we have
 
Now, assume that  holds true for some . We next prove that  also holds true. To that aim, let  be any trajectory of the closed-loop system \rref{lem:sys} with the feedback law \rref{feed:Pj}, and the control signal , . By the induction hypothesis, there exist a positive constant  and continuous functions , , such that for every   it holds that 

It is sufficient to show that there exist a positive constant  and continuous functions , , such that

Indeed, the desired results will be obtained by setting , and  for . In order to establish \eqref{fedbound:ind:H:p+1}, we start by defining the following auxiliary functions: 

and, for all ,

Then, we can rewrite  as 

where, for every ,
 
where  for all  and  otherwise, and  is defined in \rref{def:consA}. The -th time derivative of the control signal  is given, for all , by . Therefore to prove , it is sufficient to show that, for each , there exists continuous functions  , , such that, for all ,

 is actually a constant independent of , we write it as  for the sake of notation homogeneity.


For , we apply Leibniz's rule to \rref{def:U_i_t_1} with respect to  and  and obtain that the -th time derivative of  is given, for all , by 
 To obtain \rref{bound_U_i_p+1}, it is sufficient to prove that for each , and  there exist continuous functions  for  such that, for all ,


In order to get \rref{est:int1} we next provide,  for each , estimates of ,  and  for . One can observe that, for each ,  depends on the constants , the states  and the feedback . By an induction argument using differentiation of system \rref{lem:sys}, one can obtain the following statement: for any , , there exist continuous functions 
 such that, for all positive times, it holds that
 
where, by convention,  are constant functions independent of  for  and . Using \rref{fedbound:ind:H} in the above estimate, one gets that, for any  and , there exist functions , for , and  such that, for all ,
 
Setting, for , 
 
one can obtain that,  for all , all , and all ,

It follows that \rref{est:int1} for  holds true. For any  and , the -th time derivative of , defined in \rref{f_i}, is given, for all , by 
 Thus, one can get that

From \rref{est:y_i_k}, and using the fact that , one can obtain that for each  and  it holds that, for all ,
 
Since the right-hand side of \rref{est:y_i_k_2} is independent of , and  for all , one can gets that there exist continuous functions 
 such that, for any  and all , it holds
 
A trivial estimate for any , any , and all  is given by




By the Fa\`a di Bruno's formula (given in Lemma \ref{lem:fa_di} in Appendix), for each , and , the -th time derivative of  is given, for all , by
 
where  denotes the set of tuples   of positive integers satisfying  and .
Observe that the -th derivative of the function  defined in \eqref{g} reads

with . Using \rref{eq:dev_g}, and taking the absolute value, one can get, for all ,

Using \rref{est:f_i_k}, one can obtain that, for any , any  and for all , 

It follows that, for all ,  ,


Thus, it can be seen that, for every  and , there exist continuous functions  and , , such that, for all ,

Then, from \rref{est:gfi} and \rref{est:y_i_k} it follows that \rref{est:int1} holds true for any . This ends the inductive proof of .
\subsection{Proof of Theorem \ref{th:mi}}
\subsubsection{Reduction of the proof of Theorem \ref{th:mi} to the proof of Propositions \ref{prop:SISSl} and \ref{prop:bound:U:mi}}\label{sec:red:th2}
We prove Theorem \ref{th:mi} by induction on the number of inputs . We show that the inductive step reduces to Proposition \ref{prop:SISSl} and Proposition \ref{prop:bound:U:mi} which is proven in Section \ref{sss:Pr:3}.

For , the conclusion follows from Theorem \ref{th:main}. For a given  assume that Theorem \ref{th:mi} holds. We show that Theorem \ref{th:mi} then holds for LTI systems given in the reduced controllability form with  inputs. Let  and  be a -tuple of positive real numbers. Define . Given  consider a LTI system given in the reduced controllability form with  inputs by

where  and  for each ,  is Hurwitz, for every  all the eigenvalues of  are critical, and the pairs  are controllable. 

Since  is Hurwitz, if we find a  feedback law -bounded by , and -stabilizing for subsystem then, clearly, this feedback does the job for the complete system. From now on, we only consider the subsystem and we rewrite it compactly as 

\dot{x}_1 &  = A_{11} x_1 + b_{11} u_1 + \tilde{A}z+ \tilde{B}\overline{u}, \label{sys:mi:x} \\
\dot{z} & =  \overline{A} z + \overline{B} \overline{u}, \label{sys:mi:z}
 where , .


We next provide a key technical lemma.
\begin{lem}
\label{lem:struct:l}
Let , , , be a controllable single input linear system. Assume that all the eigenvalues of  are critical. Let  be the nonzero eigenvalues of ,  be a sequence of positive numbers and  be such that the linear change of coordinate  transforms  into system \rref{lem:sys} compactly written as . Rewrite  as  where  if  otherwise . 
Then  has the following property
\begin{itemize}
\item[ :]   is independent of , and each  depend only on .
\end{itemize}
 Moreover, given , let  be independent of the constants , then the matrices  and  satisfy property . 
\end{lem} The proof of Lemma \ref{lem:struct:l} follows from a careful examination of the proofs of Lemmas  and  in \cite{SSY}.


\paragraph{}

Let  be a sequence of positive numbers (to be chosen later). Let  be the linear change of coordinate that transforms  into the form of system \rref{lem:sys} compactly written as . We now make the following changes of coordinates , and system \rref{sys:mi} is then given by

\dot{y}&  = J y +b  u_1 + T \tilde{A}z+ T \tilde{B}\overline{u}, \label{sys:mi:x:g} \\
\dot{z} & =  \overline{A} z + \overline{B} \overline{u}. \label{sys:mi:z:g}

Let  be a feedback law -bounded feedback law by , and -stabilizing for subsystem \rref{sys:mi:z:g}, for some  (thanks to the inductive hypothesis, we know that this feedback exists). Let , to be chosen later.  We seek the following feedback:

\label{feed:mi_1}
u_1(y,z)& :=\frac{\mu(y)}{(1+\norme{z}^2)^{p}}, \\ \label{feed:mi_2}
\overline{u}(z) & := \kappa(z), 
 where  is defined in \rref{feed:Pj}. We now show that there exist positive constants   such that the feedback law \rref{feed:mi:g} is a feedback law -bounded and -stabilizing for system \rref{sys:mi:g}. This choice is based on Proposition \ref{prop:SISSl} and the following statement which is proven in Section \ref{sss:Pr:3}.
\begin{propo}[\emph{-bounded feedback}]
\label{prop:bound:U:mi}
Let , for , be positive constants in . Consider system \rref{sys:mi:g} with the feedback law \rref{feed:mi:g}. Assume that  is a feedback law -bounded by , and -stabilizing for subsystem \rref{sys:mi:z:g}. 
Then, there exist a positive constant , and continuous functions , , such that for any trajectory of the closed-loop system \rref{sys:mi:g} with the feedback law \rref{feed:mi:g}, the control signal  defined by  for all  satisfies, for all ,

\end{propo}
Pick  in such a way that

Choose recursively , , such that
 
where the functions  appearing above are defined in Proposition~\ref{prop:bound:U:mi} and the functions  are defined in Proposition \ref{prop:SISSl}.
By Proposition~\ref{prop:SISSl}, the feedback law  is -stabilizing for system . We now prove that the closed-loop system \rref{sys:mi:g} with the feedback \rref{feed:mi:g} is  (now, all the coefficients have been chosen). To that aim, first notice that there exist  such that, for all ,
  
Let 
Given , let  be two bounded measurable functions of the appropriate dimension, eventually bounded by . Consider any trajectory  of the following system
 From the  property of -subsystem it follows that . Thus, using the above estimate, it is immediate to see that  Therefore, invoking the  property of
, it follows that . So, the closed-loop system \rref{sys:mi:g} with the feedback \rref{feed:mi:g} is .
Moreover, as a consequence of Proposition~\ref{prop:bound:U:mi} and of the inductive hypothesis, for any trajectory of the closed-loop system \rref{lem:sys} with the feedback law \rref{feed:mi:g}, the control signal , defined by  with  and  for all , satisfies  for all . Thus, the feedback law \rref{feed:mi:g} is a  feedback law -bounded by  for system \rref{sys:mi:g}.

\subsubsection{Proof of Proposition \ref{prop:bound:U:mi}}
\label{sss:Pr:3}
For the sake of notation compactness let . To prove Proposition \ref{prop:bound:U:mi}, we establish by induction on  that the following property holds, for all :
\begin{itemize}
\item[ :]There exist a positive constant , and continuous functions , , such that for any trajectory of the closed-loop system \rref{sys:mi:g} with the feedback law \rref{feed:mi:g}, the control signal  defined by  for all  satisfies, for all ,

\end{itemize}

For , the statement () holds trivially. Now, assume that  holds true for some . We next prove that  also holds true. Let  be any trajectory of the closed-loop system \rref{sys:mi:g} with the feedback law \rref{feed:mi:g}, and the control signal  and , . As in the proof of Proposition \ref{prop:bound:U}, it is sufficient to prove that there exist a positive constant  and continuous functions , , such that
Let , for all . Define , for all . With the same notation given in the proof of Proposition \ref{prop:bound:U}, one can write  as

where, for every ,
 
As in the proof of Proposition \ref{prop:bound:U}, we next show that  for each , there exist continuous functions  , , such that, for all ,

 is actually a constant independent of , we write it as  for the sake of notation homogeneity. For , we apply Leibniz's rule to \rref{def:U_i_t_1:mi} and obtain that the -th time derivative of  is given, for all , by 
 Then, to get \rref{bound_U_i_k+1:mi}, it is sufficient to show that :
\begin{itemize}
\item[a)] there exists  such that, for any  and for all , 

\item[b)] for each , there exist , and  for  such that, for any  and for all , 

\item[c)] for each , there exist , and  for  such that, for any  and for all , 

\end{itemize}

We now establish . From an argument of induction using differentiation of -subsystem \rref{sys:mi:z:g} coupled with the fact that  is -bounded feedback law, it can easily be shown that there exist  such that for any  and for any ,   Using the Leibniz rule, it can be establish that there exist  such that, for any ,  for all . Thanks to Fa\'a Di Bruno Formula (Lemma \ref{lem:fa_di}) applied to , item  follows.


 
We now deal with item . From Lemma \ref{lem:struct:l} and an induction argument using differentiation of system \rref{sys:mi:x:g}, one can obtain the following statement: for any , , there exist continuous functions ,  , ,  , and , such that, for all , 
 So, using the inductive hypothesis and the fact that  is a -bounded feedback law, one can obtain item .


Proceeding as in Proposition \ref{prop:bound:U}, one can get item . This ends the proof of Proposition \ref{prop:bound:U:mi}.

\section{Appendix}
\subsection{Proof of Lemma \ref{lem:SISS:int} }

Let  and . We first prove forward completeness of 

in response to any locally bounded function . For this, let . Its derivative along trajectories of \rref{sys:int:pp} satisfies

Then, a straightforward computation leads to 
and forward completeness follows using classical comparison results. Moreover when , \rref{Lyap:D:int:lem} ensures that the origin of \rref{sys:int:pp} is G.A.S.

We then prove  the  property of the system \rref{sys:int:pp} with respect to .
Given , let  be a bounded measurable function on  eventually bounded by . Since the system is forward complete, we can consider without loss of generality that
 for all . From \rref{Lyap:D:int:lem} and the fact that , one can obtain that

Observing that 

it follows that
 
Consequently,  whenever . It follows that every trajectory of \rref{sys_int_pert_2} eventually enters and remains in the set  (indeed,  for all  and ). Thus Lemma \ref{lem:SISS:int} can be easily established.



\subsection{Proof of Lemma \ref{Lem:osci}}

Let . Given any , let , which is Hurwitz since  is skew-symmetric and  is controllable. Therefore there exists a symmetric positive definite matrix  satisfying the following Lyapunov equation 
 A simple computation gives
 
The smallest and largest eigenvalues of  denoted by  and  respectively are given by
 with


Define  as 
 
Given , let  and  be class  functions given by
 

There exists  such that
 Moreover, there exists a constant , independent of , such that 




Proceeding as in the proof of Lemma~\ref{lem:SISS:int}, forward completeness of 
 
can easily be derived in response to any locally measurable bounded function . 
We next show that  the system \rref{sys:osci:ppp} is  with respect to , for some  and with

Since \rref{sys:osci:ppp} is forward complete, we can assume without loss of generality that  satisfies , for some . Consider the Lyapunov function  defined in \rref{Lyap_osci}. By noticing that \rref{sys:osci:ppp} can be rewritten as
 
one gets that the time derivative of  along trajectories of \rref{sys:osci:ppp} satisfies 
 Since  is a symmetric matrix satisfying the Lyapunov equation \rref{eq:lyap:osci}, it follows that

By completing the squares it holds that, for all ,

Therefore, one can get that

Using the fact that  for all , and exploiting \rref{pr:elm:osci:Delta}, it follows that
 
Consequently, it holds that  whenever . Let  and set . Define .  If  then . Consequently, any trajectory eventually enters and stay in . Moreover, we have that  . From \rref{assert:lyap:osci}, it follows that . Moreover, one can see that there exists a constant  such that for any  we have . So we obtain 
 for some , which concludes the proof.


\subsection{Fa\`a Di Bruno's Formula}

\begin{lem}[Fa\`a Di Bruno's formula, \cite{fdb}, p. 96]
\label{lem:fa_di}
For  , let  and . Then the -th order derivative of the composite function  is given by

where  is the Bell polynomial given by

where  denotes the set of tuples   of positive integers satisfying 

\end{lem}



\bibliographystyle{IEEEtran}
\bibliography{IEEEabrv,biblio}
\end{document}
