\documentclass[english,runningheads,a4paper]{llncs}
\usepackage[T1]{fontenc}
\usepackage[latin9]{inputenc}
\usepackage[b5j]{geometry}
\usepackage{float}
\usepackage{graphicx}

\makeatletter

\floatstyle{ruled}
\newfloat{algorithm}{tbp}{loa}
\providecommand{\algorithmname}{Algorithm}
\floatname{algorithm}{\protect\algorithmname}

\newtheorem{thm}{\protect\theoremname}
\newtheorem{defn}[thm]{\protect\definitionname}
\newtheorem{lem}[thm]{\protect\lemmaname}
\newtheorem{prop}[thm]{\protect\propositionname}

\makeatother

\usepackage{babel}
\providecommand{\definitionname}{Definition}
\providecommand{\lemmaname}{Lemma}
\providecommand{\propositionname}{Proposition}
\providecommand{\theoremname}{Theorem}

\begin{document}

\title{A Parameterized Approximation Algorithm for The Shallow-Light Steiner
Tree Problem
\thanks{This project was supported by the Natural Science
Foundation of Fujian Province (2012J05115), Doctoral Fund of Ministry
of Education of China for Young Scholars (20123514120013) and Fuzhou
University Development Fund (2012-XQ-26). }
}
\author{{Longkun Guo, Kewen Liao}, XiuJun Wang}
\institute{{School of Mathematics and Computer Science,
Fuzhou University, China}\\
{School of Computer Science, University of
Adelaide, Australia}}

\maketitle
\begin{abstract}
For a given graph  with a terminal set  and a selected
root , a positive integer cost and a delay on every edge
and a delay constraint , the shallow-light Steiner tree
(\emph{SLST}) problem is to compute a minimum cost tree spanning the
terminals of , in which the delay between root and every vertex
is restrained by . This problem is NP-hard and very hard to approximate.
According to known inapproximability results, this problem admits
no approximation with ratio better than factor 
unless  \cite{khandekar2013some},
while it admits no approximation ratio better than 
for  unless  \cite{bar2001generalized}.
Hence, the paper focus on parameterized algorithm for \emph{SLST}.
We firstly present an exact algorithm for \emph{SLST} with time complexity
, where  and
 are the number of terminals and vertices respectively. This
is a pseudo polynomial time parameterized algorithm with respect to
the parameterization: ``number of terminals''. Later, we improve
this algorithm such that it runs in polynomial time 
, and computes a Steiner tree with delay bounded by 
and cost bounded by the cost of an optimum solution, where 
is any small real number. To the best of our knowledge, this is the
first parameterized approximation algorithm for the \emph{SLST} problem. \end{abstract}
\begin{keywords}
Shallow light Steiner tree, parameterized approximation algorithm,
directed Steiner tree, exact algorithm, auxiliary graph, pseudo-polynomial
time complexity.
\end{keywords}

\section{Introduction}

The well-known shallow-light Steiner tree problem (or namely the delay
restrained minimum Steiner tree problem) is defined as below:
\begin{defn}
For a graph  with a terminal set , a root vertex
, a cost function , a delay function
, and a delay bound , the\emph{
shallow-light Steiner tree (SLST) problem} is to compute a minimum
cost Steiner tree  spanning all terminals of , such that
the delay from  to every terminal in  is not larger than
.
\end{defn}
For notation briefness, we assume  in graph
, and use \emph{SLST }and\emph{ } to denote the shallow-light
Steiner tree problem and an optimal shallow-light Steiner tree respectively.
 For the \emph{SLST} problem, bifactor approximation algorithms have
been developed.
\begin{defn}
An algorithm  is a bifactor -approximation
for the \emph{SLST} problem, if and only if for every instance of
\emph{SLST, } computes a Steiner tree  in polynomial time,
such that the delay from  to every terminal in  is bounded
by  and the cost of  is bounded by  times
of the cost of the optimal solution.

Noting that single factor -approximation is identical to bifactor
-approximation for \emph{SLST}, we use them
interchangeably in the text.
\end{defn}
\textbf{Related Work.} It is known that the \emph{SLST} problem is
NP-hard, and can not be approximated better than factor 
unless  \cite{khandekar2013some}.
This is because the group Steiner tree problem can be embedded into
this problem. Furthermore, no polylogarithmic approximation within
polynomial time complexity has been developed. The best work is a
long standing result due to Charikar et al, which is a polylogarithmic
approximation in quasi-polynomial time, i.e. factor-
approximation within time complexity \cite{charikar1998approximation}.
Due to the difficulty in single factor approximation algorithm design,
bifactor approximation has been investigated. Hajiaghayi et al presented
an -approximation algorithm that
runs in polynomial time \cite{hajiaghayi2006approximating}. Besides,
Kapoor and Sarwat gave an approximation with bifactor ,
where  is an input parameter \cite{kapoor2007bounded}. The last
algorithm is an approximation that improves the cost of the tree,
and is with bifactor  when  \cite{kapoor2007bounded}.

The \emph{SLST }problem remains hard to approximate even when .
In that case, this problem becomes the shallow light spanning tree
(SLT) problem, which has broad applications in network design, VLSI
and etc. For computational complexity, the SLT problem is claimed
to be with inapproximability hardness of  \cite{naor1997improved}.
For approximation, Charikar et al's  ratio with time
complexity \cite{charikar1998approximation} is still
the best single factor result. Naor and Schieber gave an approximation
bifactor of , i.e. with delay and cost bounded
by 2 times and  times of that of the optimal solution
respectively \cite{naor1997improved}. To the best of our knowledge,
these are the best long standing approximation ratios. Some special
cases of the SLT problem are also interesting. If edge cost is equal
to the delay for each edge, the SLT problem remains NP-hard and admit
no approximation algorithms with bifactor  for
any  and  \cite{khuller1995balancing},
while the best possible result for \emph{SLST} is a -approximation
\cite{elkin2011steiner}.  Moreover, the SLT problem remains NP-hard
when all edge delays are equal, but polynomially solvable when all
edge costs are equal \cite{salama1997delay}. For the equal-delay
case, namely the hop constrained minimum spanning tree problem, Althaus
et al have presented an approximation with a ratio of 
in \cite{althaus2005approximating}.

Another two important special cases of the \emph{SLST} problem is
when  is constant or when all edge delays are equal. Unfortunately,
for the former case, \emph{SLST} can not be approximated better than
a factor of  for even  unless 
\cite{bar2001generalized}, since the Set Cover problem can be embedded
into this case. Bar-Ila et al also developed a factor-
approximation for the cases of  in the same paper, achieving
the best possible ratio. When all edge delays are equal, namely the
hop constrained minimum Steiner tree problem, it is open that if there
exists factor- approximation for this problem,
as the spanning case.

\textbf{Our Contribution. }The first result of this paper is an exact
algorithm, with time complexity ,
for the \emph{SLST} problem. This result indicates that if the number
of terminal and the delay constraint are bounded, the \emph{SLST}
problem is polynomial solvable. Our technique is mainly based on constructing
an auxiliary graph, where every Steiner tree satisfies the delay constraint,
i.e. in the auxiliary graph, we only need to compute Steiner tree
without considering the delay constraint. Though its time complexity
seems terrible, the exact algorithm is efficient for real-world applications
for , particularly when  or all edge delays are
equal (the hop constrained minimum Steiner tree problem).

On the theoretical side, we note that this algorithm runs in pseudo
polynomial time (for constant ), since  appears in the formula
of the time complexity. The second result is to improve this time
complexity to polynomial time 
following a similar line of polynomial-time approximation scheme (PTAS)
design, such that it computes a Steiner tree with delay bounded by
 and cost bounded by the cost of an optimum solution.


\section{A Parameterized Approximation Algorithm for the Shallow Light Steiner
Problem}

In this section we shall approximate the shallow-light Steiner tree
(\emph{SLST}) problem. Firstly and intuitively, our main observation
is that the difficulty of computing a \emph{} comes from obeying
the given delay constraint. Therefore, our key idea is to construct
an auxiliary directed graph  where there exists only cost (i.e.
no delay) on edges, such that every Steiner tree (spanning the same
terminal set) in  corresponds to a Steiner tree that satisfies
the given delay constraint  in . Secondly since the directed
Steiner tree problem is known parameterized tractable with respect
to the parameterization: ``number of terminals''\cite{guo2011parameterized,ding2007finding},
an exact algorithm is immediately obtained; then an approximation
algorithm with ratio  can be derived from the exact
algorithm by a method of shrinking the value of . The approximation
algorithm computes a  with delay bounded by 
and cost bounded by the cost of an optimum .


\subsection{Construction of the Auxiliary Graph}

Though different in technique details, the key idea to construct the
auxiliary graph is similar to the auxiliary graph used to balance
the cost and delay of  disjoint shortest paths in \cite{guo2013improved}:
using layer graphs. For a given graph  with positive integer
cost and delay on every edge, and a delay constraint , the layer
graph , i.e. the auxiliary graph to be constructed, contains vertices,
terminals and edges roughly as in the following:
\begin{enumerate}
\item  vertices  corresponding to every
vertex  ;
\item  edges 
corresponding to every edge 
and with ;
\item one terminal , corresponding to every terminal ,
together with cost-0 edges 
that connect auxiliary vertices of  to the auxiliary terminal;
\end{enumerate}
Therefore,  has  vertices,  edges, and 
terminals. The construction is formerly as in Algorithm \ref{alg:Construction-of-auxiliary}
(An example of such construction is as depicted in Figure \ref{fig:Construction-of-acyclic}).

\begin{algorithm}
\textbf{Input}: Graph , a set of terminals ,
a root vertex , cost and delay
 on every edge , and a delay constraint
;

\textbf{Output}: Auxiliary acyclic graph  and the terminal set
therein, .
\begin{enumerate}
\item , ;
\item \textbf{For }each  \textbf{do}

\begin{enumerate}
\item ;
\item \textbf{If}  \textbf{then}

\begin{enumerate}
\item ,
and set  for
each ;
\item ;
\end{enumerate}
\end{enumerate}
\item \textbf{For} each 
that  \textbf{do}


\quad{},
and set 
for each ;

\item \textbf{For} each  \textbf{do}


\quad{},
and set  .

\item Return  and .
\end{enumerate}
\caption{\label{alg:Construction-of-auxiliary}Construction of auxiliary graph
.}
\end{algorithm}


\begin{figure}
\begin{centering}
\includegraphics[width=0.7\textwidth]{acyclic}
\par\end{centering}

\caption{\label{fig:Construction-of-acyclic}Construction of acyclic graphs:
(a) is the original graph, in which  are terminals;
(b) is the constructed auxiliary graph, in which 
are terminals.}
\end{figure}


It remains to show that the -rooted minimum cost directed Steiner
tree in  corresponds to a -rooted minimum  in .
\begin{lem}
\label{lem:degree}A minimum cost directed Steiner tree rooted at
 in  contains at most one vertex of 
for each .\end{lem}
\begin{proof}
Let  be a -rooted minimum cost directed Steiner tree in .
Suppose  contains  and . Then we
show that  is not minimum and get a contradiction. Let  be
 except removing the edge entering  and replacing
every edge in the subtree of  that roots at ,
say 
by edge . Apparently,
 spanning the same terminal set as . That is, there exists
a directed Steiner tree  with less cost than  in . This
contradicts with the fact that  is minimum.\end{proof}
\begin{thm}
\label{thm:acyclic}Let  be the set of terminal vertices 
in . Then there exists a -rooted directed Steiner tree spanning
 of minimum cost  in  iff there exists a Steiner tree
spanning  of minimum cost  with delay between  and every
terminal restrained by  in .\end{thm}
\begin{proof}
Let  be a minimum cost directed Steiner tree rooted at  in
. Let  be a subgraph of , in which 
if and only if there exists .
Then because , we
have . It remains to show  is a Steiner tree. From
Lemma \ref{lem:degree}, 
holds for every . So a path connecting  to a terminal in 
corresponds to a path connecting  to a terminal in .  Then
since every terminal of  is reachable from  in , all
terminals of  are connected to  in . Besides, because
 is a tree,  contains no loops or parallel edges. Therefore,
 is a Steiner tree of .

Let  be a Steiner tree in . Then there is a unique path from
root  to every other vertex of . Hence, every vertex of 
has a unique delay from . Let  contains edge 
for every , and edge 
if and only if , where  is the delay
from  to  in  and  the delay from
 to . Since the delay of from  to every vertex
in  is not larger than , edge 
belongs to , and hence . Then because every 
is reachable from  and no loop or parallel edge exists following
the construction of ,  is a Steiner tree in  with cost
. This completes the proof.
\end{proof}

\subsection{A parameterized Approximation Algorithm for Shallow-Light Steiner
Tree }

This subsection shall give an exact algorithm and a parameterized
approximation algorithm for the \emph{SLST} problem. From Theorem
\ref{thm:acyclic}, an algorithm for the \emph{SLST} problem can be
obtained by computing a minimum cost directed Steiner tree in .
Unfortunately, it is known that the (minimum) directed Steiner tree
problem is NP-hard and maybe even more difficult to approximate than
\emph{SLST}, i.e. only a quasi-polynomial time algorithm with a polylogarithmic
approximation factor has been developed\cite{charikar1998approximation}.
However, when the number of the terminals is a constant, the directed
Steiner tree problem is polynomial solvable, as stated in the proposition
below:
\begin{prop}
\label{prop:An-optimum-solution}\cite{guo2011parameterized}An optimum
solution to the directed Steiner tree problem can be computed within
, where  and  are the number
of terminals and vertices respectively.
\end{prop}
Following Algorithm \ref{alg:Construction-of-auxiliary}, Theorem
\ref{thm:acyclic} and Proposition \ref{prop:An-optimum-solution},
we could now state the exact algorithm for the \emph{SLST} problem
as in the following:

\begin{algorithm}
\textbf{Input}: Graph , , , cost
function and delay function 
, a delay constraint , and auxiliary graph  with ,

\textbf{Output}: , an optimum solution to the \emph{SLST} problem.
\begin{enumerate}
\item ;
\item Compute a minimum cost Steiner tree in , say  spanning the
terminal of  by the method of \cite{ding2007finding};
\item \textbf{For} every  \textbf{do}


\quad{}\textbf{If}  \textbf{then} ;

\item Return .
\end{enumerate}
\caption{\label{alg:An-exact-algorithm}An exact algorithm for \emph{SLST}.}


\end{algorithm}


Following Theorem \ref{thm:acyclic} and Proposition \ref{prop:An-optimum-solution},
we immediately have the correctness of Algorithm \ref{alg:An-exact-algorithm}.
For time complexity, since  contains  edges, 
vertices and  terminals, it takes 
time to compute a minimum Steiner tree in . Hence, we have:
\begin{thm}
\label{thm:best-1}Algorithm \ref{alg:An-exact-algorithm} solved
the SLST problem correctly, and runs in time .
\end{thm}
We note that Algorithm \ref{alg:An-exact-algorithm} runs in pseudo-polynomial
time, since the formula of the time complexity contains . However,
following the technique of polynomial-time approximation scheme (PTAS)
design, a parameterized approximation algorithm for the \emph{SLST}
problem could proceed as: firstly compute , which is  except
the delay of every edge  is sat to ,
such that the value of delay constraint is shrunken from  to a
polynomial on ; secondly construct graph  with the new delay
on edges; and finally run Algorithm \ref{alg:An-exact-algorithm}
on the auxiliary graph  of the new delay. Formally, the parameterized
approximation algorithm for the\emph{ SLST }problem is as in the following:

\begin{algorithm}
\textbf{Input}: A given parameter , graph , ,
, cost and delay 
on every edge , and a delay constraint ;

\textbf{Output}: , an approximation solution to the \emph{SLST}
problem.
\begin{enumerate}
\item \textbf{For} every edge of  \textbf{do}


\quad{};


/{*} Compute .{*}/

\item Construct auxiliary graph  and compute  using Algorithm
\ref{alg:Construction-of-auxiliary};
\item Compute a minimum cost Steiner tree  subjected to the new delay
constraint by applying
Algorithm \ref{alg:An-exact-algorithm} on  and  with respect
to the new delay;
\item Return .
\end{enumerate}
\caption{\label{alg:appro-algorithm-1}A parameterized approximation algorithm
for \emph{SLST}.}
\end{algorithm}


Following Algorithm \ref{alg:appro-algorithm-1}, the delay constraint
in  is . Then
from Lemma \ref{thm:best-1}, the time complexity of the algorithm
is 
after shrinking  to . Hence, we have
\begin{lem}
\label{lm:best}Algorithm \ref{alg:appro-algorithm-1} runs in time
.
.
\end{lem}
It remains to show the approximation of the algorithm, which is given
by the following theorem:
\begin{thm}
\label{thm:ratio}Algorithm \ref{alg:appro-algorithm-1} computes
a Steiner tree spanning all terminals of  in  with cost bounded
by the cost of an optimum , and delay bounded by .\end{thm}
\begin{proof}
Clearly, an optimum  in  will satisfy the new delay constraint
 in . Then since
 is a optimum solution to \emph{SLST} in , it is with cost
not larger than the cost of an optimum  in .

It remains to show the delay of  in . Let  be an arbitrary
path in . Then since the delay of  in  is bounded by
, we have:




Following the definition of , 
holds, and hence:




Combining Inequality (\ref{eq:1}) and (\ref{eq:2}) yields:




Therefore, following Inequality (\ref{eq:3}), the delay of 
in  is:



This completes the proof.
\end{proof}

\section{Conclusion}

This paper investigated exact algorithms and then parameterized approximation
algorithms for the SLST problem. The first result is an exact algorithm
that computes optimum  in time ,
and the second result is a factor- approximation
algorithm with time complexity .
A problem remained open is whether design of algorithms for the SLST
problem with polylogarithmic approximation ratio is possible.

\bibliographystyle{plain}
\bibliography{drmst}

\end{document}
