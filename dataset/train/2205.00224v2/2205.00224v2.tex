\documentclass[10pt,twocolumn,letterpaper]{article}

\usepackage{iccv}
\usepackage{times}
\usepackage{epsfig}
\usepackage{graphicx}
\usepackage{amsmath}
\usepackage{amssymb}



\usepackage[pagebackref=true,breaklinks=true,letterpaper=true,colorlinks,bookmarks=false]{hyperref}



\def\iccvPaperID{2812} \def\httilde{\mbox{\tt\raisebox{-.5ex}{\symbol{126}}}}

\ificcvfinal\pagestyle{empty}\fi

\begin{document}

\title{Entropy Regularization Series: A Simple Method for Differentiated Unsupervised Learning}

\author{Sue Sin Chong\\
Tsinghua University\\
Institution1 address\\
{\tt\small zhangsx18@mails.tsinghua.edu.cn}


}

\maketitle
\ificcvfinal\thispagestyle{empty}\fi






\begin{abstract}
   Is it possible to train several classifiers to perform meaningful crowdsourcing to produce a better prediction label set without any ground-truth annotation? A recent work which has achieved state-of-the-art results on unsupervised classification, by combining end-to-end and contrastive methods. In this paper, we will attempt to modify the contrastive learning objectives to train several neural network to coordinate to produce a state-of-the-art prediction on the CIFAR10 and CIFAR100-20 task. This paper will present a remarkably simple method to modify a single unsupervised classification pipeline to generate a series of neural networks with to learn different features of a same dataset. With a set of neural networks that converge on different set of features on the same dataset, instead of painting a class as a generic template, it is now possible to characterize a class with a set of more specific and finer features. The Entropy Regularization Series (ERS), when added upon the contrastive learning objective functions, gives users a knob to tweak the entropy state of the output space of unsupervised learning, thereby diversifying the types of neural networks that can be generated by the same pipeline. Exploiting the multiple converging points, the ERS can serve as a self-filter for noisy and uncertain samples, it is able to separate certain samples from sample where there are confusion and separate certain classes from classes where there are confusion. The ERS is a sensitive knob to tweak decision boundaries, and has proven to be able to produce classifiers that beat state-of-the-art at contrastive learning stage. Experiments show that ERS can produce different neural networks, that each have accuracy that is comparable to the state-of-the-art, yet have each have learnt a different set of features from the original dataset. It allows users of unsupervised classification to in essence perform search in the output space entropy of contrastive learning. 
\end{abstract}


\begin{IEEEkeywords}
Unsupervised Learning, Diversified Feature Set, Entropy Regularization, Convergence

\end{IEEEkeywords}






\IEEEpeerreviewmaketitle



\section{Introduction}

How to make unsupervised algorithms to converge and learn different features at will at the same time has always been difficult. Recently, a combination of representation learning method and end-to-end learning  has given promising results on the coarse classification of reasonably large datasets such as CIFAR100 \cite{wvangansbeke2020scan}. \cite{wvangansbeke2020scan}'s three-step process involves 2 contrastive learning stages  and a fine-tuning stage.  

Recently, \cite{JunGuo_AAAI_2020} works on feature selection in unsupervised learning. \cite{pretextInvariant2020}, \cite{pretrain2019Uncurated} works on making the pretext training phase of unsupervised learning more robust. 
The properties of similarity search vectors and application in unsupervised image classification task, has also been much studied in  \cite{alex2018spreading}, \cite{spreadLocal2019} . The integration of noise and feature into the unsupervised learning pipeline has also proven to be useful \cite{bojanowski2017unsupervised}. \cite{jain2016approximate} \cite{Zhang2015SparseCQ} has studied the possibility of searching in a quantized sparse space representation. \cite{he2019momentum} has studied momentum of the unsupervised clustering process. Most recently, \cite{chen2020simple} introduced a simple contrastive learning framework. Whereas \cite{chuang2020debiased} makes the contrastive learning stage robust. And \cite{asano2020self} is a fine-tuning technique that has proven to be very effective in the final stages of unsupervised learning. Whereas \cite{vo2019unsupervised} attempts to find new objects or features as an optimization objective. 

Entropy regularization is an important technique in machine learning, and has many applications. \cite{pereyra2017regularizing}, \cite{cuturi2013sinkhorn}, \cite{Liong2015DeepHF} all utilized entropy regularization enhance unsupervised learning, entropy regularization can either maximize marginal entropy of bits or speed-up classification. In searching, \cite{subic2017} also utilized entropy regularization on one-hot codes.  Whereas \cite{entropy2015Spring} has studied the mass-spring-damper system without a singular kernel, which might be able to play the role of controller in unsupervised learning. 

\subsection{Related Work}
Out of Distribution (OOD) Detection is an increasing prominent field in machine learning. \cite{pidhorskyi2018generative} is a method for generating state-of-the-art OOD detector with adversarial method. \cite{lee2018simple} used a method based on Gaussian to extract features from a trained neural network to train an OOD detector with a relatively simple tools. Via \cite{lee2018simple}, one is able to link the certainty of features learnt to OOD classifier accuracy. 

\subsection{Motivation}
The ability to check for feature that has not been detected by an unsupervised learner but also an important feature of the dataset is important. It serves as a fill-in-the-blank check for an unsupervised classifier. It allows us to paint a class as a set of more specific and finer and at the same time correct features, instead of having to accept a class as a very generic template of features.

Secondly, the ability to separate weakly-classified samples from well-classified samples only by using similarly unsupervised-trained neural network or pipeline, allows the unsupervised learner to self-check, and improve. When it is hard to differentiate, it will be very helpful to be able to have neural networks trained with different feature emphasis to vote on a particular sample. This motivates us to modify entropy of the output space to train neural networks which has different correctness emphasis to help partition the dataset into easy-to-classify and possibly-hard-to-classify samples. 

\section{Preliminaries}
We attempt to introduce the possibility of searching in the entropy space in the unsupervised image classification task. This work attempts to combine the best of representation learning, end-to-end learning and the properties of spreading vector and entropy control to give users a simple yet intuitive way to manage the entropy in output space of unsupervised learning. 

In this paper, we present a remarkably simple method for training reasonably good unsupervised classifiers which automatically learns different features of the dataset. We view the problem of improving unsupervised classification as the unsupervised and automatic extraction of different features. We present a framework where it is possible to generate series of neural networks which have different and complementing feature emphasis simply by tweaking constants. Our objective is to maximize the number of possibilities in output space entropy distribution of converging neural networks, which allows for a large variety of decision boundaries and therefore a highly diversified set of learnt features. 

To tackle the task of unsupervised entropy management, we developed an approach of involving entropy and entropy regularization terms in objective functions throughout the contrastive learning stage, called the entropy regularization series (ERS). The ERS is a set of regularization terms to be added on unsupervised learning objective functions . ERS offers a simple-to-implement yet sensitive knob to unsupervised classification entropy management, which allows for differentiated unsupervised learning. This work presents the array of rich modification options presented by adding entropy regularization term, and giving insight to the possibility of automatically identify various features of different classes by training multiple neural networks. We show that the probability of finding a neural network that has learnt different feature set is high, and that generating a stable neural network which has learnt a different feature set is very convenient.

The main contributions of our paper are as follows:
\begin{itemize} 
\item We identified a method to manage entropy in the unsupervised classification problem, the ERS method.
\item ERS is simple to implement and use, and almost always guarantee to learn different feature sets from the same data. Without ERS, unsupervised learning always converges on to the same set of feature, reproducably. 
\item ERS method can produce networks that have classification accuracy which compares to state-of-the-art, but with different feature emphasis. It can also serve as a uniqueness of convergence check on an unsupervised classification pipeline.
\item Series of neural network trained with ERS can self-check to filter for samples where there are second opinions. 
\item We conduct experiments on CIFAR100-20 task. And shown that finding a neural network which learnt the features of a variety of subclasses is easy.
\item ERS can mine for neural networks which can be better trained as out-of-distribution detectors.
\end{itemize}

In the following section, we will discuss the definition and implications of solving the problem of diversified unsupervised learning with ERS. 

\section{Problem Definition}
Given a unsupervised classification pipeline , how to automatically train series of neural networks which detect different features and are each independently a reasonable classifier. 

\subsection{Applications} ERS is a natural filter for weakly-classified samples in unsupervised learning. With a series of neural network which each converge to a different set of correct features, it can immediately identify weakly-classified samples. Secondly, ERS checks for convergence in features learnt in unsupervised classification pipelines. When the dataset has multiple ways of matching different features to the same set of classes, it is possible for ERS to list the different mappings between features and classes. It serves as a neighborhood exploration tool which allows for grid-searching with constants on some arbitrary entropy structure in the output space. 

\subsection{Implications} When a   is sufficient to produce a reasonable classifier by unsupervised learning on a dataset. By using ERS, we can exhaust the feature discovery possibilities of the target . If there exist any subclass within a super-class, the ERS will effectively mine for sub-class representations in the original dataset as there will very likely exist a neural network which converge to an alternative set of features, it serves as a checker for the uniqueness of feature convergence in a unsupervised classification process. This is particularly useful when the actual number of classes is unknown. 


The first section will discuss the ERS method on Unsupervised Semantic Clustering pipeline. Whereas the second section will discuss reasoning about entropy regularization, and the convergence of neural network trained with ERS. The third section will discuss the implications of finding complementing neural networks which has learnt different features of the same dataset. The fourth section will discuss the application of combinations of neural network with different feature emphasis. This paper will conclude with applications and implications of using ERS as a output space entropy controlling tool. 

\section{Entropy Management via Entropy Regularization Series (ERS) }

The Entropy Regularization Series (ERS) is a set of entropy regularization terms to be added to contrastive learning optimization functions, as follows. ERS is to be added on the contrastive learning stages of unsupervised classification to encourage different decision boundary and entropy distribution in the output space, thereby enforcing different feature emphasis. Experiments show that training neural networks with different confusion matrix is not possible without the ERS. The implementation of ERS merely requires an additional tens of lines of code to the loss function. 



Entropy terms were previously added in semantic clustering to encourage uniform prediction amongst classes in the contrastive learning stage. Higher order entropy terms in ERS plays the role of tweaking and modifying prediction amongst classes. Our experiment shows that ERS in unsupervised classification objective functions are indispensable for identification of different features, ERS improve the feature diversity of the neural network set in general.

 Higher order entropy terms play the role of controller in controlling the distances between clusters in the output space, hence it is possible to produce neural networks with different confusion matrices. It is possible to both maximize or minimize for between-cluster spaces and the smoothing of the clustering process. 

\subsection{ERS Acting Upon Unsupervised Semantic Clustering}

There are 3 portions to unsupervised classification in the \cite{wvangansbeke2020scan}, SimCLR, Semantic Clustering(SCAN), and Selflabel(SLL). SimCLR and SCAN being contrastive learning stages. Adding entropy terms in the objective functions of contrastive learning stages can improve the diversity of neural network trained. Regularization term in SimCLR is as follows.



Regularization term in SCAN portion of the pipeline is as follows.  and  plays the role of a spring and damper for between-clusters entropy.



The scalar terms  , is a function of number of classes. Below is a table of scalar terms and the classification accuracy of different lambda combinations. Together, the new optimization function presents a control function related to number of classes to modify the entropy in the input space. It is hence possible to specify a specific behavior desired of output neural network, and then simply by modifying relative values of   train a converging network with the desired property.
\begin{table}[h]
\caption{Experiment Results}
\label{tab:tabela1}
\begin{tabular}{|c|c|c|c|c|c|c|}
\hline
Desc &  &  &  &  & Scan & SLL  \\ \hline
SCAN & 0 & 5 & 0 & 0 & 0.44 & 0.507       \\
1 & 0 & 5 & 4 & 0 & 0.439 & 0.503 \\
2 & 2 & 5 & 4 &  & 0.4335 & 0.5012 \\
3 & 2 & 5 & 4 &  & 0.4005 & 0.4408 \\
4 & 2 & 5 &  &  & 0.456 & 0.4767 \\
\hline
\end{tabular}
\end{table}

Reflecting the sensitivity of spring constant and dampening ratio in a physical spring system, neural networks trained with objective functions with regularization terms that are scalar multiples of each other can also have significantly different behaviors. It is also possible to improve the performance of a single classifier in the contrastive learning stage.


\subsubsection{Reasoning About Entropy State}
 By optimizing for entropy throughout, we are giving the constraint of number of classes throughout the output space, forcing different values and structures of classes in the trained neural network. The regularization of entropy throughout the pipeline, encourages us to introduce an abridged notation to reason about the state of entropy that the training has result in. It will allow for the convenience of solving for deducing about the stability of training process and the properties of the output neural network. The 3-step unsupervised classification process is essentially a cascading function of the entropy in the input set. 

\begin{table}[h]
\centering
\caption{Entropy Regularization State Notation}
\label{tab:tabela1}
\begin{tabular}{|c|c|c|}
\hline
 & Value & Notation \\ \hline
 & 0 &  \\
 &  &  \\
\hline
\end{tabular}
\end{table}

Let  be the pretext regularization function. Similarly, we define the following for SCAN. 



Let  be the SCAN regularization function. Then compounding the two regularization functions, the state of entropy as a result of the training process can be expressed as . Hence, the contrastive training process is maximizing similarity together with a second order differential equation of entropy, offering fine-grained knobs to the desired output entropy state, and the decision boundary formation within clusters. The ERS serves as a control function for entropy in the contrastive learning stage. 
\subsection{Second Order Entropy Control Functions}

 is a function of number of classes, aka . Expressing the entropy control function as an inner product between a series of functions of number of classes, we have the following. 



Self-labeling which attempts to correctly classify noisy neighbors near clusters by minimizing cross-entropy loss, is the only step in the pipeline which only optimizes for cross-entropy. ERS makes it possible to systematically produce converging neural networks with different decision boundaries simply by modifying . Experiments show that neural networks trained with different objective functions are most confident about a wide variety of different prototype images for each superclass. This offers a special edge to the task of CIFAR100-20, as it allows for the mining of more subclass prototypes within a super-class, thereby improving overall accuracy of classification.

\subsection{Grid-Searching for Optimization Functions to Produce the Best Performing Neural Network in Contrastive Learning Stage}
On the contrastive learning stages, ERS has consistently been able to mine for neural networks that beat state-of-the-art by  at the end of contrastive learning stage on the CIFAR10 and CIFAR100-20 classification task. However, the slight edge gained doesn't persist through the self-labelling stage. ERS is also able to produce neural networks which in turn train better out-of-distribution classifiers on the same dataset. 

\section{A Sensitive Knob}
The ERS is necessary to train neural networks to learn different features. Experiment shows that simply by tweaking the model without any regularization term, the confusion matrix remains largely similar. The terms forces different structure on the neural network, which results in differences in confusion matrix. 

ERS is a sensitive knob for tweaking the entropy in output space. In this section, we will discuss an example neural network series, trained with . To demonstrate the fact that ERS is able to produce neural networks which learn different set of feature at ease, we introduce the notion of n guess accuracy.  

\subsubsection{N Guesses Accuracy}
Given n neural networks, the probability that at least one neural network has learnt the features of a class and matched the feature correctly to a class can be approximated by n guess accuracy.  

The accuracy of N guesses is calculated as follows. Given n neural networks, if any of the n neural network predicts the label correctly, then it is calculated as a correct prediction. There is no hierarchy or preference to the set of prediction label produced by the set of n neural networks. 

\subsection{Self-complementing Series}
The  term is a sensitive knob for controlling the fine-grained cluster formation process. It is easy to train a set of neural networks simply by multiplying the  with a geometric series. By tweaking , we are almost certain to find a neural network which has learnt a different set of correct features.

\begin{table}[h]
\caption{2 Guess Of a  Series}
\begin{center}
\begin{tabular}{c|c|c|c|c|c}
    \hline
      2 guess ACC & Agreement &  &  &  &  \\
     \hline
     \multirow  63.79 & 44.90  & 2 & 5 & 4 & 4 \\
   & & 2 & 5 & 4 & 8\\ \hline
    \multirow  61.82 & 55.81 & 2 & 5 & 4 & 8 \\
   & & 2 & 5 & 4 & 16\\
    \hline
    \multirow 61.8 & 50.59 &2 & 5 & 4 & 16 \\
    & & 2 & 5 & 4 & 32 \\
    \hline
    \multirow  61.68& 42.64  & 2 & 5 & 4 & 4 \\
    & & 2 & 5 & 4 & 32\\
\end{tabular}
\end{center}
\label{tab:multicol}
\end{table}

The neural network in the series agrees samples in which they are both certain, and only disagrees on samples where there are confusion. With the ERS, we are able to identify classes and samples where there are second opinions. The ERS can serve as a filtering method for noisy samples and noisy classes.

\begin{table}[h]
\caption{3/4 Guess Of a  Series}
\begin{center}
\begin{tabular}{c|c|c|c|c|c}
    \hline
      3/4 guess ACC & 2-agreement & &  &  &  \\
     \hline
    \multirow  69.35 & 75.86 & 2 & 5 & 4 & 4 \\
    && 2 & 5 & 4 & 8\\
      & & 2 & 5 & 4 & 32\\
    \hline
    \multirow  67.80 & 83.97  & 2 & 5 & 4 & 4 \\
    && 2 & 5 & 4 & 8\\
      & & 2 & 5 & 4 & 16\\
       \hline
       \multirow  67.49  & 82.50 & 2 & 5 & 4 & 4 \\
    && 2 & 5 & 4 & 16\\
      & & 2 & 5 & 4 & 32\\
      \hline
       \multirow  71.92  & NA & 2 & 5 & 4 & 4 \\
    && 2 & 5 & 4 & 8\\
    && 2 & 5 & 4 & 16\\
      & & 2 & 5 & 4 & 32\\

\end{tabular}
\end{center}
\label{tab:multicol}
\end{table}

Within the same series, it is possible to sieve for confident samples with majority votes for at least three quarters of the samples. We can further clamp down on samples where all of the networks do not agree on, aka samples with very high confusion. Below is an image of the most confident prototypes of the series of neural network. This series of neural network has learnt the many sub-classes within the 20 super-classes of the CIFAR100 dataset. For instance, in the class of super-class of large carnivores, this series has learnt bear, tiger, leopard and lion respectively as their most confident prototype for the same superclass. 

\includegraphics[width = \linewidth]{serieslabels.png}

ERS can mine for a wider set of representation of a class, when there are many sub-classes within a super-class, the advantage will be prominent.

\section{Different Classes of ERS Functions}


\subsection{Extended Stability}
Although it is possible to simply train for instance, 4 neural networks and achieve the aim of filtering for weakly-classified samples. It is also possible to explore the entropy space further with even larger classes of ERS functions. It has been proven via experimentation that the following lambda bounds will produce optimization functions that can produce a converging neural network. The number of possible ERS functions which can train converging neural networks in the contrastive learning stage is very large. 

\begin{table}[h]
\centering
\caption{Big O Bounds for }
\label{tab:tabel3}
\begin{tabular}{|c|c|}
\hline
 & Big O Bound  \\ \hline
 &  \\
 &  \\
 &  \\
 &  \\
\hline
\end{tabular}
\end{table}

\begin{table}[h]
\centering
\caption{Stable Templates}
\label{tab:tabela1}
\begin{tabular}{|c|c|c|c|}
\hline
 &  &  &   \\ \hline
0 & O(1) & 0 & 0 \\
0 & O(1) & O(1) & 0 \\
O(1) & O(1) & O(1) & 0 \\
O(1) & O(1) & O(1) & O(1) \\
O(1) & O(1) &  & 0 \\
O(1) & O(1) & 0 &  \\
O(1) & O(1) &  &  \\
O(1) & O(1) &  &   \\
O(1) & O(n) &  &  \\
\hline
\end{tabular}
\end{table}

The decision boundaries of neural network trained is very sensitive to small changes in . It is reasonable to expect that one can find neural networks that can identify all other subclass prototypes in the super-class by grid-searching the set of s. With the ERS, it is possible to have multiple neural networks which is confident about more sub-classes within the super-class to learn the sub-classes within a super-class. 

It is observed that the accuracy of a single classifier in cifar20 can hardly surpass 0.5 due to both the limit of number of classes and lack of labels. Having multiple networks learning the 20 classes in CIFAR100-20, makes it possible to have different neural networks learn the features of smaller classes, which result in a better superclass classification accuracy.   

\subsection{Compounding Effect of Cross-Type ERS}
On the CIFAR100 dataset, we have trained 25 Res18Nets for classification into 20 coarse classes. Each of the 25 Res18Nets has been trained with a different lambda series. Results in the following table is obtained by combining 2 to 4 of the 25 neural networks. 

\begin{table}[h]
\centering
\caption{N-guess Performances}
\label{tab:tabela1}
\begin{tabular}{|c|c|c|c|}
\hline
N Guess & Best ACC & Mean ACC & Median ACC  \\ \hline
1 nn & 50.7 & NA & NA \\
2 guess & 64.50 & 58.20 & 59.24 \\
3 guess & 70.57 & 63.70 & 64.70 \\
4 guess & 73.99 & 68.90 & 69.10 \\
\hline
\end{tabular}
\end{table}
Each of the classifier is independently an at least  accuracy classifier. And  of the classifiers show at least  improvement in classification guess. When we compound ERS of different classes, it is much more likely to make the n guess of the neural networks more robust. Below is a listing of some of the best combinations of neural networks.  Results show that some of the best combinations often involve neural networks with non-zero , demonstrating the indispensability of the entropy regularization terms in training neural networks to learn different features. 
\begin{table}[ht]
\caption{Some 2 Guess Combinations}
\begin{center}
\begin{tabular}{c|c|c|c|c}
    \hline
      2 guess ACC &  &  &  &  \\
     \hline
    \multirow  64.50  & 0 & 5 & 0 & 0 \\
    & 2 & 5 & 4 & 0\\
    \hline
    \multirow 64.12 & 0 & 5 & 0 & 0 \\
    & 2 & 5 &  &  \\
    \hline
    \multirow 63.79 & 2 & 5 & 4 & 8 \\
    & 2 & 5 & 4 & 4\\
\end{tabular}
\end{center}
\label{tab:multicol}
\end{table}

\begin{table}[h]
\caption{3 Guess Combinations}
\begin{center}
\begin{tabular}{c|c|c|c|c}
    \hline
      3 guess ACC &  &  &  &  \\
     \hline
    \multirow  70.79  & 0 & 5 & 0 & 0 \\
    & 2 & 5 & 4 & 0\\
    & 2 & 5 &  &  \\
    \hline
    \multirow 70.66 & 0 & 5 & 0 & 0 \\
    & 2 & 5 & 4& 0 \\
    &2 & 5 & 4 & 4 \\
    \hline
    \multirow 70.60 & 0 & 5 & 0 & 0 \\
    & 2 & 5 &  &  \\
     & 2 & 5 &  &  \\
\end{tabular}
\end{center}
\label{tab:multicol}
\end{table}

In the CIFAR100-20 task, there are 5 sub classes in each super-class, there are multiple ways of mapping sub-classes to a super class. ERS mines the input dataset for multiple sub-classes within the super-class. By compounding  2 to 3 neural networks whereby each network learns a subclass within a super-class, the n guess predictions will in expectation make good guesses which will include each subclass. 

\includegraphics[width = \linewidth]{best3combi.png}
Best 3 combinations often have confident prototypes drawn from a larger variety of sub-classes. 
\includegraphics[width = \linewidth]{best3combi2.png}
While there may be confusion across super-classes, each subclass is only represented once in each classifier confident prototypes. Combinations in aggregation learn to be confident about a large set of sub-classes. This will allow us to make a move towards classifying the dataset into more sub-classes, and eventually approach 100 classes in the CIFAR100 task. 

\subsubsection{Learning Sub-classes Within Super-class}
With the ERS, we are able to mine for different prototypical images within a super-class in the CIFAR100-20 task. Instead of being forced to view a super-class as a set non-descriptive generic feature, ERS is able to mine for specific subclass features within a super-class. And together, paint a super-class as a collection of specific subclass features. 

\section{Majority Vote}
By training several neural network, and have them voting on a classes of a sample can produce state-of-the-art prediction accuracy. N guess accuracy is the upper bound of majority vote accuracy. We are able to improve upon the accuracy of a single prediction set by compounding the accuracy of several classifiers.  Our best combinations which involves 27 different neural networks voting for the label of every single sample, reaches a new state-of-the-art accuracy rate of 0.58. 

\begin{table}[h]
\caption{State-of-the-Art Results on CIFAR100-20 by Majority Voting}
\begin{center}
\begin{tabular}{c|c}
    \hline
       Number of Classifiers & Accuracy by Majority Vote  \\  \hline
       State-of-the-Art & 50.7 \\ 
       3 & 55.9 \\
       4 & 56.1 \\

\end{tabular}
\end{center}
\label{tab:multicol}
\end{table}

This again demonstrates that the fuzziness introduced by ERS allows for meaningful verification check across neural network trained from the same process. We also observed that by dividing the classifiers into three tiers according to its accuracy rate, and organize a voting amongst neural networks which are drawn from each of the tiers more easily produce an accuracy rate which is higher than randomly selected neural networks. Although 

\section{Conclusion}
Neural networks trained with ERS have different feature extraction emphasis. The knobs presented by  is fine-grained decision boundaries knob to neural networks trained is very sensitive to small changes in the series.  This presents the possibility of using simple techniques on s to automatically mine-train neural networks which learn different features by exploiting the differences in a wide array of of correct features learnt about the original dataset. The ability to generate multiple converging neural network at ease will allow us to characterize the feature set of input data in greater detail in the unsupervised context.
{\small
\bibliographystyle{ieee_fullname}
\bibliography{egbib}
}

\end{document}
