

\subsection{Proof of Lemma~\ref{lem:key-lemma-R}}  \label{sec:proof-R}

Suppose \R\ wins the Slope Game from $((q,q'), (\rho, \rho'))$ using a strategy of segment depth $d$.
A position in the Slope Game contains a positive vector $(\rho,\rho')$,
while a position in the Simulation Game contains a pair $(n,n') \in \N\x\N$ 
of counter values, that can also be interpreted as a positive vector.
We will derive a strategy for \R\ in the Simulation Game that is winning from all positions
$(qn,q'n')$ where $(n,n')$ is $(\qq \cdot d)$-below $(\rho,\rho')$.
The crucial idea of the proof is to consider the segments of the supposed winning strategy in the
Slope Game separately.
Each such segment is a strategy for one phase and as such, describes how to move in
the Simulation Game until the next lasso is observed. Afterwards, \R\ can
chose to continue playing according to the next lower segment, or ``roll back'' the
cycle and continue playing according to the current segment.
By the rules of the Slope Game we observe that after sufficiently many such rollbacks
the difference between the ratio $n/n'$ of the actual counters and the slope of the next
lower segment is negligible, i.e., these vectors are equivalent in the sense
of Definition~\ref{def:vector-equivalence} in Section~\ref{ssec:belt-theorem-proof}.
Then, \R\ can safely continue to play according to the next lower segment at level
$d-1$.


To safely play such a strategy in the Simulation Game, \R\ needs to ensure that her own
counter does not decrease too much as that could restrict her ability to move.
We observe however, that any partial play that ``stays in some segment'' at height $d$,
can be decomposed into a single acyclic prefix plus a number of cycles.
Such a play therefore preserves the invariant that all visited points are
$\qq\cdot (d-1)$-below the slope of the phase. In particular, this means that \R's
counter is always $\ge\qq\cdot(d-1)$.

Formally, the proof of Lemma~\ref{lem:key-lemma-R} proceeds by induction on the segment depth $d$.

\subparagraph*{Case $d = 1$.} 
This means that \R\ has a strategy to win the Slope Game in the first phase,
and hence to enforce that the effect of all cycles is behind $(\rho,\rho')$ but not
positive. Denote this strategy by $T$.
In the Simulation Game, \R\ will re-use this strategy as we describe below.
At every position $(qn,q'n')$ in the Simulation Game, 
\R\ keeps a record of the \emph{corresponding position} $(\pi, (\rho,\rho'))$ in the Slope Game, enforcing the
invariant that $(q,q')$ are the ending states of the path $\pi$.

From the initial position  $(qn,q'n')$ with corresponding position $((q,q'), (\rho,\rho'))$,
\R\ starts playing the Simulation Game according to $T$,
until the path in the corresponding position of the Slope Game, say $\pi_1$,
describes a lasso (this must happen after at most $\qq$ rounds).
Thus $\pi_1$ splits into:
\begin{equation}
  \pi_1 = \widetilde\pi_1 \, \bar\pi_1
\end{equation}
where the suffix $\bar\pi_1$ is a  cycle. Denote by $(\widetilde\alpha_1, \widetilde\alpha'_1)$
and $(\bar\alpha_1, \bar\alpha'_1)$ the effects of $\widetilde\pi_1$ and
$\bar\pi_1$, respectively. 
The current values of counters are clearly
\begin{equation}
 n + \widetilde\alpha_1 + \bar\alpha_1 \qquad \text{and }\quad
 n' + \widetilde\alpha'_1 + \bar\alpha'_1
\end{equation}
assuming that the play did not end by now with \R's win.
As the length of path $\pi_1$ is at most $\qq$ and
$(n,n')$ is assumed to be $\qq$-below $(\rho,\rho')$, we know that
all positions visited by now in the Simulation Game were below $(\rho,\rho')$.
In particular, \R's counter value was surely non-negative by now.

Now \R\ \emph{rolls back} the cycle $\bar\pi_1$, namely changes the corresponding
position in the Slope Game from $(\pi_1, (\rho,\rho'))$ to $(\widetilde\pi_1,(\rho,\rho'))$ 
and continues playing according to $T$. 
The play continues until \R\ wins or the path in the corresponding position of the
Slope Game, say $\pi_2$, is a lasso again. Again, we split the path into an acyclic
prefix and a cycle:
\begin{equation}
  \pi_2 = \widetilde\pi_2 \, \bar\pi_2.
\end{equation}
Denote the respective effects by $(\widetilde\alpha_2, \widetilde\alpha'_2)$
and $(\bar\alpha_2, \bar\alpha'_2)$.
A crucial but simple observation is that, assuming that the play did not end by now with \R's win,
the current values of counters are now
\begin{equation}
    n + \widetilde\alpha_2 + \bar\alpha_1 + \bar\alpha_2 \qquad \text{and }\quad
    n' + \widetilde\alpha'_2 + \bar\alpha'_1 + \bar\alpha'_2,
\end{equation}
i.e. the effect $(\widetilde\alpha_1, \widetilde\alpha'_1)$ of $\widetilde\pi_1$ does not contribute any more.
As $(\bar\alpha_1, \bar\alpha'_1)$ is behind $(\rho,\rho')$ we may
apply Lemma~\ref{lem:preserve-above} to $(\bar\alpha_1, \bar\alpha'_1)$ and $c = 0$
in order to deduce, similarly as before, that all positions by now were below $(\rho,\rho')$.
Now \R\ rolls back $\bar\pi_2$ by establishing $(\widetilde\pi_2, (\rho,\rho'))$ as
the new corresponding position in the Slope Game.
Continuing in this way, after $k$ rollbacks the counter values are:
\begin{align}
\begin{aligned}
 &n\ + \widetilde\alpha_k + (\bar\alpha_1 + \bar\alpha_2 + \ldots \ + \bar\alpha_{k-1}) + \bar\alpha_k
 \qquad \text{and}\\
 &n' + \widetilde\alpha'_k + (\bar\alpha'_1 + \bar\alpha'_2 + \ldots + \bar\alpha'_{k-1}) + \bar\alpha'_k,
 \end{aligned}
\end{align}
assuming that \R\ did not win earlier. 
All the vectors $(\bar\alpha_i, \bar\alpha'_i)$, and thus also the sum
\begin{equation}\label{eq:vectorsum}
  (\bar\alpha_1 + \bar\alpha_2 + \ldots \ + \bar\alpha_{k-1}, \bar\alpha'_1 + \bar\alpha'_2 + \ldots + \bar\alpha'_{k-1})
\end{equation}
are behind $(\rho,\rho')$, hence similarly as before all positions by now have been below $(\rho,\rho')$,
by Lemma~\ref{lem:preserve-above} applied to the vector~\eqref{eq:vectorsum} above.

This in particular means that \R's counter remains above value $c$.
However, as by assumption all observed cycles come from a final segment in her Slope Game strategy,
the vector~\eqref{eq:vectorsum} cannot be positive for any $k$. Thus, every rollback strictly
decreases \V's counter value.
We conclude that after sufficiently many rollbacks, \V's counter must eventually drop below $0$ and
hence \R\ eventually wins.
\subparagraph*{Case $d > 1$.} 
By assumption, \R\ has a strategy $T$ for the Slope Game, which has segment depth $d>0$.
As before, \R's strategy in the Simulation Game will re-use the strategy $T$
from the Slope Game, using rollbacks.

\R\ plays according to the initial segment of this strategy, that allows her to win or at least
guarantee that the effect of the first observed lasso's circle is less steep than $(\rho,\rho')$.
After $l$ rollbacks, the counter values will be of the form:
\begin{align} \label{eq:countervalues}
\begin{aligned}
  &n + \widetilde\alpha + (\bar\alpha_1 + \ldots \ + \bar\alpha_{m})  + (\bar\gamma_1 +\ldots \ + \bar\gamma_{l})\quad\text{and}\\
  &n' + \widetilde\alpha' + (\bar\alpha'_1 + \ldots + \bar\alpha'_{m}) + (\bar\gamma'_1 + \ldots + \bar\gamma'_{l}),
 \end{aligned}
\end{align}
where the absolute values of $\widetilde\alpha$ and $\widetilde\alpha'$ are at most $\qq$,
the vectors $(\bar\gamma_i,\bar\gamma'_i)$ are behind $(\rho,\rho')$ and positive,
and the vectors $(\bar\alpha_i,\bar\alpha'_i)$ are behind $(\rho,\rho')$ and non-positive.
We apply Lemma~\ref{lem:preserve-above} to $c = \qq\cdot(d-1)$ and learn
that all the positions by now have been $(\qq\cdot(d-1))$-below $(\rho,\rho')$.

In general \R\ has no power to choose whether a effect of a cycle at a next rollback is positive or not.
However, if from some point on all effects are non-positive then \V's counter
eventually drops below $0$ and \R\ wins.
Thus w.l.o.g\., we focus on positions in the Simulation Game immediately after a rollback of a
cycle with positive effect.
Using the notation from~\eqref{eq:countervalues}, suppose $(\gamma_l,\gamma'_l)$ 
is the effect of the last rolled back cycle.
We need the following claim in order to apply the induction assumption:
\begin{claim}
  After sufficiently many rollbacks the vector $(\bar n, \bar n')$ of current counter values
  in the Simulation Game is $(\qq\cdot(d-1))$-below some vector $(\gamma,\gamma')$
  which is equivalent to the positive effect $(\gamma_l,\gamma'_l)$ of the last rolled back cycle.
\end{claim}
\begin{proof}
  By an easy geometric argument. Ignore vectors $(\alpha_i,\alpha'_i)$ as they
  preserve being $(\qq\cdot(d-1))$-below all positive vectors that are less steep
  than $(\rho,\rho')$.  If \V\ wants to falsify the condition, he would need to
  increase the steepness of the rolled back cycle infinitely often, which is
  clearly impossible as there are only finitely many simple cycles.
\end{proof}

Let $(\bar q \bar n, \bar q' \bar n')$ be a position of the Simulation Game satisfying the claim.
We know that \R\ has a winning strategy in the Slope Game from $((\bar q, \bar q'),
(\gamma_l,\gamma'_l))$, of segment depth at most $d-1$.
Because $(\gamma_l,\gamma'_l)$ is equivalent to $(\gamma,\gamma')$, we can apply
Lemma~\ref{lem:constant-winner} and know that the same strategy is winning in the
Slope Game from $((\bar q, \bar q'), (\gamma, \gamma'))$.
By the induction assumption we conclude that \R\ wins the Simulation Game
from $(\bar q \bar n, \bar q' \bar n')$, which completes the proof of
Lemma~\ref{lem:key-lemma-R}.\qed



\subsection{Proof of Lemma~\ref{lem:key-lemma-V}}  \label{sec:proof-V}

Suppose \V\ wins the Slope Game from $((q,q'), (\rho, \rho'))$ using a strategy of segment depth $d$.
We will show that \V\ wins the Simulation Game from every position $(q n, q' n')$ where
$(n,n')$ is $(\qq \cdot d)$-above $(\rho,\rho')$.
We will again build on the concept of rollbacks and proceed by induction on $d$.

\subparagraph*{Case $d = 1$.} 
In this case, \V\ has a strategy to win the Slope Game immediately after the first phase.
This means he can enforce that the effects of the cycles of all observed lassos are not behind
$(\rho,\rho')$.
By a straightforward induction using part 2 of Lemma~\ref{lem:preserve-above}
one can show that \V\ can preserve the invariant that all visited points
are $K$-above $(\rho,\rho')$. This in particular means that
his counter value stays positive and he wins by enforcing an infinite play.

\subparagraph*{Case $d > 1$.} 
Let $T$ denote the initial segment of \V's strategy in the Slope Game.
Every effect of a cycle in $T$ is either not behind $(\rho,\rho')$ or behind
$(\rho,\rho')$, but positive.

In the Simulation Game, \V\ will play according to this initial segment $T$, using
rollbacks, as long as the effect of the rolled back cycle is not behind
$(\rho,\rho')$.
Just as in the previous case, we can apply part 2 of Lemma~\ref{lem:preserve-above}
for $c = \qq\cdot d$ and derive that in this way, \V\ is able to keep the current
counter values $(\qq\cdot d)$-above $(\rho,\rho')$.
 
Suppose that after a few iterations, the effect $(\alpha,\alpha')$ of 
the last cycle \emph{is} behind $(\rho,\rho')$ and let $(\bar q \bar n, \bar q' \bar n')$ be the
position in the Simulation Game directly afterwards.
In this case, $(\alpha, \alpha')$ is clearly positive and less steep than
$(\rho,\rho')$.
Now the point described by the counter values before this last cycle was $(\qq\cdot
d)$-above $(\rho,\rho')$ and because the cycle is no longer than $K$, we know that the
point $(\bar n, \bar n')$ of current counter values (after the cycle) is still
$(\qq\cdot(d-1))$-above $(\rho,\rho')$.
This means, as $(\alpha, \alpha')\lesssteep(\rho,\rho')$,
that $(\bar n, \bar n')$ is also $(\qq\cdot(d-1))$-above $(\alpha,\alpha')$.

Knowing that \V\ has a winning strategy in the Slope Game from $((\bar q, \bar q'),
(\alpha,\alpha'))$ of segment depth at most $d-1$, by induction assumption we obtain 
a winning strategy for \V\ in the Simulation Game from $(\bar q \bar n, \bar q' \bar n')$. 
This completes the description of \V's winning strategy from $(q n, q' n')$ and hence also
the proof of Lemma~\ref{lem:key-lemma-V}.\qed



\subsection{Proof of Lemma~\ref{lem:simul-periodic}}\label{proof:lemma:x}
For technical convenience we assume w.l.o.g.~that no belt contains the upper right corner of $\is$
(this can always be achieved by minimally extending $\is$, if necessary.)
Thus every belt intersects either the horizontal, or the vertical border of $\is$, but not both.

Recall that the non-parallel belts do not overlap/interfere with each other outside $\is$, hence 
we can consider them separately. 
For the rest of the proof fix states $q, q'$ and let $(\rho, \rho') = \slope(q,q')$.
W.l.o.g.~suppose that $\belt(q,q')$ intersects the horizontal border of $\is$ 
(if it intersects the vertical border of $\is$ the proof is analogous).

For simplicity we assume that no other belt is parallel to $\belt(q,q')$.
The proof below may be easily adapted to the general case by considering a bunch of parallel belts
jointly, instead of just the single one $\belt(q,q')$.

By a \emph{cross-section} at level $n'$ we mean the intersection of $\simul_{q,q'}$
with two consecutive horizontal lines at that level,
i.e.~with $\{ (n,n'), (n,n'+1) : n \in \N \}$.
We may assume that cross-sections are always non-empty (this can always be ensured by
slightly widening $\belt(q,q')$ if necessary).
We say that two cross-sections $s_1$ and $s_2$ are \emph{equal} if
one of them is obtained by a shift of the other by a multiple of $(\rho, \rho')$; formally, we require 
for some $k \in \N$,
\begin{align} \label{eq:equalcs}
s_1 + k \cdot (\rho, \rho') \  \ = \ \ s_2.
\end{align}
Choose two cross-sections $s_1, s_2$ at levels $n'_1$ and $n'_2$ respectively,
and $k > 0$ that satisfies~\eqref{eq:equalcs}.
Let $P$ be the restriction of $\simul_{q,q'}$ to the area between $s_1$ and $s_2$, 
and $A$ be the restriction of $\simul_{q,q'}$ to the area below $s_1$:
\begin{align*}
A \ & = \  \{ (n,n') \in \  \simul_{q,q'} \ : \ n' < n'_1 \} &
P \ & = \ \{ (n,n') \in \ \simul_{q,q'} \ : \ n'_1 \leq n' < n'_2 \}.
\end{align*}
Recall that $A$ and $P$, similarly as $\simul_{q,q'}$, are subsets of $\belt(q,q')$. We claim: 
\begin{lemma} \label{lem:periodicbelt}
For every $s_1, s_2$ and $k > 0$ satisfying~\eqref{eq:equalcs},
\[
\simul_{q,q'} \ = \ A \ \cup \ P^*, \quad \text{ where } P^* \ = \ \bigcup_{i \in \N} (P + i \cdot k \cdot (\rho,\rho')).
\]
\end{lemma}
Before proving this lemma note that it implies Lemma~\ref{lem:simul-periodic}.
Indeed, by Theorem~\ref{thm:belt-theorem}, a cross-section contains polynomially many points, and therefore there are at most exponentially many non-equal
cross sections.
Thus, by the pigeonhole principle, there are surely two equal cross-sections
at exponentially bounded levels $n'_1$ and $n'_2$.
 


Now we prove Lemma~\ref{lem:periodicbelt}. The proof strongly relies on the locality of the simulation condition.
We first claim one inclusion of Lemma~\ref{lem:periodicbelt}, namely:
\begin{claim} \label{cl:incl}
$A \ \cup \ P^* \subseteq \ \simul_{q,q'}$.
\end{claim}
\begin{proof}
We show  that the following relation is a simulation:
\[
R \ \ = \ \ \simul \ \setminus \ \{ (q n, q' n') : (n,n') \in \belt(q,q')\}
\ \cup \ \{ (q n , q' n') : (n,n') \in A \ \cup \ P^* \}.
\]
(Roughly speaking, $R$ is obtained from $\simul$ by replacing $\simul_{q,q'}$ with $A \ \cup \ P^*$.)
We claim that $R$ is a simulation, relying on the locality of the simulation condition.
Formally, we define the \emph{relative $R$-neighborhood} of a point $(n,n')$ as 
\[
\{ (p l, p' l') : (p (n+l), p' (n'+l')) \in R, \ (p, p') \in Q\times Q', \ l, l' \in \{-1,0,1\} \} .
\] 
Note that the simulation condition for a pair of configurations $(q n, q' n')$ with respect to the relation $R$ 
only depends on the relative $R$-neighborhood of $(n, n')$. 
Similarly, one defines the relative $\simul$-neighborhood of a point $(n, n')$.

By the definition of cross-section and of the sets $A$ and $P$, 
the relative $R$-neighborhood of a point $(n,n') \in R$ equals the relative $\simul$-neighborhood of some 
(possibly other) point in $\simul_{q,q'}$.
Thus we deduce that every pair in $R$ satisfies the simulation condition wrt.~$R$, i.e.~$R$ is a simulation.
As $\simul$ is the largest simulation, the claim follows.
\end{proof}

In order to show the other inclusion of Lemma~\ref{lem:periodicbelt}, extend $n'_1$ and $n'_2$ to an infinite arithmetic progression
\[
n'_1, \ n'_2, \ n'_3, \ \ldots,
\]
i.e.~$n_{i+1} = n'_i + k \cdot \rho'$ for $i \geq 1$,
and consider the ``segments'' $P_i$ of $\simul_{q,q'}$ defined by the corresponding cross-sections:
\[
P_i \ = \  \{ (n,n')  \in \ \simul_{q,q'} \ : \ n'_i \leq n' < n'_{i+1} \} \qquad \text{ for } i \geq 1. 
\]
Clearly, $P = P_1$ and 
$\simul_{q,q'} \ = \ A \ \cup \ \bigcup_{i \geq 1} P_i$.
By Claim~\ref{cl:incl} it follows that $P_1 + k \cdot (\rho,\rho') \subseteq P_2$, or equivalently
$P_1 \subseteq P_2 - k \cdot (\rho,\rho')$. Analogously one shows:
\begin{align}  \label{eq:incl}
P_i  \ \subseteq \ P_{i+1} - k\cdot (\rho,\rho') \qquad \text{ for every } i \geq 1.
\end{align}
We claim that the inclusions are actually equalities:

\begin{claim} \label{cl:eq}
$P_i \ = \ P_{i+1} - k \cdot (\rho, \rho')$, for every $i \geq 1$.
\end{claim}
\begin{proof}
Due to Equation~\eqref{eq:incl}, it suffices to show the inclusions $P_{i+1} - k \cdot (\rho, \rho') \ \subseteq \ P_i$.
The inclusions follow, similarly as in the proof of Claim~\ref{cl:incl}, from the observation that the following relation
is a simulation:
\[
R \ \ =  \ \ \simul \ \setminus\  \{ (q n, q' n') : (n, n') \in \belt(q,q') \} 
\ \ \cup \ \ \{ (q n, q' n') : (n,n') \in A 
\ \cup \ \bigcup_{i \geq 2} P_i - k \cdot (\rho,\rho') \}.
\]
The relation $R$ is obtained from $\simul$, roughly speaking, by removing the first segment $P_1$ and shifting all other segments
$P_i$ by vector $- k \cdot (\rho, \rho')$. To prove that $R$ is a simulation, we exploit locality of the simulation condition 
exactly as before.
Additionally, we use the observation that the simulation condition is monotonic with respect to inclusion of relative neighborhoods,
together with the inclusions~\eqref{eq:incl}.
\end{proof}

\noindent
Claim~\ref{cl:eq} immediately implies Lemma~\ref{lem:periodicbelt}
and thus Lemma~\ref{lem:simul-periodic}.

























