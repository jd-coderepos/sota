

\subsection{Proof of Lemma~\ref{lem:key-lemma-R}}  \label{sec:proof-R}

Suppose \R\ wins the Slope Game from  using a strategy of segment depth .
A position in the Slope Game contains a positive vector ,
while a position in the Simulation Game contains a pair  
of counter values, that can also be interpreted as a positive vector.
We will derive a strategy for \R\ in the Simulation Game that is winning from all positions
 where  is -below .
The crucial idea of the proof is to consider the segments of the supposed winning strategy in the
Slope Game separately.
Each such segment is a strategy for one phase and as such, describes how to move in
the Simulation Game until the next lasso is observed. Afterwards, \R\ can
chose to continue playing according to the next lower segment, or ``roll back'' the
cycle and continue playing according to the current segment.
By the rules of the Slope Game we observe that after sufficiently many such rollbacks
the difference between the ratio  of the actual counters and the slope of the next
lower segment is negligible, i.e., these vectors are equivalent in the sense
of Definition~\ref{def:vector-equivalence} in Section~\ref{ssec:belt-theorem-proof}.
Then, \R\ can safely continue to play according to the next lower segment at level
.


To safely play such a strategy in the Simulation Game, \R\ needs to ensure that her own
counter does not decrease too much as that could restrict her ability to move.
We observe however, that any partial play that ``stays in some segment'' at height ,
can be decomposed into a single acyclic prefix plus a number of cycles.
Such a play therefore preserves the invariant that all visited points are
-below the slope of the phase. In particular, this means that \R's
counter is always .

Formally, the proof of Lemma~\ref{lem:key-lemma-R} proceeds by induction on the segment depth .

\subparagraph*{Case .} 
This means that \R\ has a strategy to win the Slope Game in the first phase,
and hence to enforce that the effect of all cycles is behind  but not
positive. Denote this strategy by .
In the Simulation Game, \R\ will re-use this strategy as we describe below.
At every position  in the Simulation Game, 
\R\ keeps a record of the \emph{corresponding position}  in the Slope Game, enforcing the
invariant that  are the ending states of the path .

From the initial position   with corresponding position ,
\R\ starts playing the Simulation Game according to ,
until the path in the corresponding position of the Slope Game, say ,
describes a lasso (this must happen after at most  rounds).
Thus  splits into:

where the suffix  is a  cycle. Denote by 
and  the effects of  and
, respectively. 
The current values of counters are clearly

assuming that the play did not end by now with \R's win.
As the length of path  is at most  and
 is assumed to be -below , we know that
all positions visited by now in the Simulation Game were below .
In particular, \R's counter value was surely non-negative by now.

Now \R\ \emph{rolls back} the cycle , namely changes the corresponding
position in the Slope Game from  to  
and continues playing according to . 
The play continues until \R\ wins or the path in the corresponding position of the
Slope Game, say , is a lasso again. Again, we split the path into an acyclic
prefix and a cycle:

Denote the respective effects by 
and .
A crucial but simple observation is that, assuming that the play did not end by now with \R's win,
the current values of counters are now

i.e. the effect  of  does not contribute any more.
As  is behind  we may
apply Lemma~\ref{lem:preserve-above} to  and 
in order to deduce, similarly as before, that all positions by now were below .
Now \R\ rolls back  by establishing  as
the new corresponding position in the Slope Game.
Continuing in this way, after  rollbacks the counter values are:

assuming that \R\ did not win earlier. 
All the vectors , and thus also the sum

are behind , hence similarly as before all positions by now have been below ,
by Lemma~\ref{lem:preserve-above} applied to the vector~\eqref{eq:vectorsum} above.

This in particular means that \R's counter remains above value .
However, as by assumption all observed cycles come from a final segment in her Slope Game strategy,
the vector~\eqref{eq:vectorsum} cannot be positive for any . Thus, every rollback strictly
decreases \V's counter value.
We conclude that after sufficiently many rollbacks, \V's counter must eventually drop below  and
hence \R\ eventually wins.
\subparagraph*{Case .} 
By assumption, \R\ has a strategy  for the Slope Game, which has segment depth .
As before, \R's strategy in the Simulation Game will re-use the strategy 
from the Slope Game, using rollbacks.

\R\ plays according to the initial segment of this strategy, that allows her to win or at least
guarantee that the effect of the first observed lasso's circle is less steep than .
After  rollbacks, the counter values will be of the form:

where the absolute values of  and  are at most ,
the vectors  are behind  and positive,
and the vectors  are behind  and non-positive.
We apply Lemma~\ref{lem:preserve-above} to  and learn
that all the positions by now have been -below .

In general \R\ has no power to choose whether a effect of a cycle at a next rollback is positive or not.
However, if from some point on all effects are non-positive then \V's counter
eventually drops below  and \R\ wins.
Thus w.l.o.g\., we focus on positions in the Simulation Game immediately after a rollback of a
cycle with positive effect.
Using the notation from~\eqref{eq:countervalues}, suppose  
is the effect of the last rolled back cycle.
We need the following claim in order to apply the induction assumption:
\begin{claim}
  After sufficiently many rollbacks the vector  of current counter values
  in the Simulation Game is -below some vector 
  which is equivalent to the positive effect  of the last rolled back cycle.
\end{claim}
\begin{proof}
  By an easy geometric argument. Ignore vectors  as they
  preserve being -below all positive vectors that are less steep
  than .  If \V\ wants to falsify the condition, he would need to
  increase the steepness of the rolled back cycle infinitely often, which is
  clearly impossible as there are only finitely many simple cycles.
\end{proof}

Let  be a position of the Simulation Game satisfying the claim.
We know that \R\ has a winning strategy in the Slope Game from , of segment depth at most .
Because  is equivalent to , we can apply
Lemma~\ref{lem:constant-winner} and know that the same strategy is winning in the
Slope Game from .
By the induction assumption we conclude that \R\ wins the Simulation Game
from , which completes the proof of
Lemma~\ref{lem:key-lemma-R}.\qed



\subsection{Proof of Lemma~\ref{lem:key-lemma-V}}  \label{sec:proof-V}

Suppose \V\ wins the Slope Game from  using a strategy of segment depth .
We will show that \V\ wins the Simulation Game from every position  where
 is -above .
We will again build on the concept of rollbacks and proceed by induction on .

\subparagraph*{Case .} 
In this case, \V\ has a strategy to win the Slope Game immediately after the first phase.
This means he can enforce that the effects of the cycles of all observed lassos are not behind
.
By a straightforward induction using part 2 of Lemma~\ref{lem:preserve-above}
one can show that \V\ can preserve the invariant that all visited points
are -above . This in particular means that
his counter value stays positive and he wins by enforcing an infinite play.

\subparagraph*{Case .} 
Let  denote the initial segment of \V's strategy in the Slope Game.
Every effect of a cycle in  is either not behind  or behind
, but positive.

In the Simulation Game, \V\ will play according to this initial segment , using
rollbacks, as long as the effect of the rolled back cycle is not behind
.
Just as in the previous case, we can apply part 2 of Lemma~\ref{lem:preserve-above}
for  and derive that in this way, \V\ is able to keep the current
counter values -above .
 
Suppose that after a few iterations, the effect  of 
the last cycle \emph{is} behind  and let  be the
position in the Simulation Game directly afterwards.
In this case,  is clearly positive and less steep than
.
Now the point described by the counter values before this last cycle was -above  and because the cycle is no longer than , we know that the
point  of current counter values (after the cycle) is still
-above .
This means, as ,
that  is also -above .

Knowing that \V\ has a winning strategy in the Slope Game from  of segment depth at most , by induction assumption we obtain 
a winning strategy for \V\ in the Simulation Game from . 
This completes the description of \V's winning strategy from  and hence also
the proof of Lemma~\ref{lem:key-lemma-V}.\qed



\subsection{Proof of Lemma~\ref{lem:simul-periodic}}\label{proof:lemma:x}
For technical convenience we assume w.l.o.g.~that no belt contains the upper right corner of 
(this can always be achieved by minimally extending , if necessary.)
Thus every belt intersects either the horizontal, or the vertical border of , but not both.

Recall that the non-parallel belts do not overlap/interfere with each other outside , hence 
we can consider them separately. 
For the rest of the proof fix states  and let .
W.l.o.g.~suppose that  intersects the horizontal border of  
(if it intersects the vertical border of  the proof is analogous).

For simplicity we assume that no other belt is parallel to .
The proof below may be easily adapted to the general case by considering a bunch of parallel belts
jointly, instead of just the single one .

By a \emph{cross-section} at level  we mean the intersection of 
with two consecutive horizontal lines at that level,
i.e.~with .
We may assume that cross-sections are always non-empty (this can always be ensured by
slightly widening  if necessary).
We say that two cross-sections  and  are \emph{equal} if
one of them is obtained by a shift of the other by a multiple of ; formally, we require 
for some ,

Choose two cross-sections  at levels  and  respectively,
and  that satisfies~\eqref{eq:equalcs}.
Let  be the restriction of  to the area between  and , 
and  be the restriction of  to the area below :

Recall that  and , similarly as , are subsets of . We claim: 
\begin{lemma} \label{lem:periodicbelt}
For every  and  satisfying~\eqref{eq:equalcs},

\end{lemma}
Before proving this lemma note that it implies Lemma~\ref{lem:simul-periodic}.
Indeed, by Theorem~\ref{thm:belt-theorem}, a cross-section contains polynomially many points, and therefore there are at most exponentially many non-equal
cross sections.
Thus, by the pigeonhole principle, there are surely two equal cross-sections
at exponentially bounded levels  and .
 


Now we prove Lemma~\ref{lem:periodicbelt}. The proof strongly relies on the locality of the simulation condition.
We first claim one inclusion of Lemma~\ref{lem:periodicbelt}, namely:
\begin{claim} \label{cl:incl}
.
\end{claim}
\begin{proof}
We show  that the following relation is a simulation:

(Roughly speaking,  is obtained from  by replacing  with .)
We claim that  is a simulation, relying on the locality of the simulation condition.
Formally, we define the \emph{relative -neighborhood} of a point  as 
 
Note that the simulation condition for a pair of configurations  with respect to the relation  
only depends on the relative -neighborhood of . 
Similarly, one defines the relative -neighborhood of a point .

By the definition of cross-section and of the sets  and , 
the relative -neighborhood of a point  equals the relative -neighborhood of some 
(possibly other) point in .
Thus we deduce that every pair in  satisfies the simulation condition wrt.~, i.e.~ is a simulation.
As  is the largest simulation, the claim follows.
\end{proof}

In order to show the other inclusion of Lemma~\ref{lem:periodicbelt}, extend  and  to an infinite arithmetic progression

i.e.~ for ,
and consider the ``segments''  of  defined by the corresponding cross-sections:

Clearly,  and 
.
By Claim~\ref{cl:incl} it follows that , or equivalently
. Analogously one shows:

We claim that the inclusions are actually equalities:

\begin{claim} \label{cl:eq}
, for every .
\end{claim}
\begin{proof}
Due to Equation~\eqref{eq:incl}, it suffices to show the inclusions .
The inclusions follow, similarly as in the proof of Claim~\ref{cl:incl}, from the observation that the following relation
is a simulation:

The relation  is obtained from , roughly speaking, by removing the first segment  and shifting all other segments
 by vector . To prove that  is a simulation, we exploit locality of the simulation condition 
exactly as before.
Additionally, we use the observation that the simulation condition is monotonic with respect to inclusion of relative neighborhoods,
together with the inclusions~\eqref{eq:incl}.
\end{proof}

\noindent
Claim~\ref{cl:eq} immediately implies Lemma~\ref{lem:periodicbelt}
and thus Lemma~\ref{lem:simul-periodic}.

























