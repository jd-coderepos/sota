\documentclass[10pt]{IEEEtran}
\usepackage{cite}
\usepackage{graphicx,subfigure}
\usepackage{amsmath,amssymb}
\usepackage{algorithm, algorithmic}
\usepackage{subfigure}
\usepackage{psfrag}
\usepackage{paralist}
\usepackage{authblk}
\usepackage{yhmath}
\usepackage{multirow}
\usepackage{balance}



\title{Architecture and Algorithms for an Airborne Network
\thanks{This research is supported in part by a grant from the U.S. Air Force Office of Scientific Research under grant number FA9550-09-1-0120.}}

\author[1]{Arunabha Sen}
\author[1]{Pavel Ghosh}
\author[1]{Tiffany Silva}
\author[1]{Nibedita Das}
\author[2]{Anjan Kundu}
\affil[1]{Department of Computer Science and Engineering, Arizona State University, Tempe, AZ, 85281 \authorcr \{Pavel.Ghosh, asen, tsilva, nmaulik\}@asu.edu}
\affil[2]{Saha Institute of Nuclear Physics, Kolkata 700064, India \authorcr anjan.kundu@saha.ac.in}

\begin{document}

\maketitle
\begin{abstract}
The U.S. Air Force currently is in the process of developing an Airborne Network (AN) to provide support to its combat aircrafts on a mission. The reliability needed for continuous operation of an AN is difficult to achieve through completely infrastructure-less mobile ad hoc networks. In this paper we first propose an architecture for an AN where airborne networking platforms (ANPs - aircrafts, UAVs and satellites) form the backbone of the AN. In this architecture, the ANPs can be viewed as mobile base stations and the combat aircrafts on a mission as mobile clients. Availability of sufficient control over the movement pattern of the ANPs, enables the designer to develop a topologically stable backbone network. The combat aircrafts on a mission move through a space called {\em air corridor}. The goal of the AN design is to form a backbone network with the ANPs with two properties: (i) the backbone network remains {\em connected at all times}, even though the topology of the network changes with the movement of the ANPs, and (ii) the entire three dimensional space of the air corridor is under {\em radio coverage at all times} by the continuously moving ANPs.

In addition to proposing an architecture for an AN,  the contributions of the paper include, (i) development of an algorithm that finds the velocity and transmission range of the ANPs so that the dynamically changing backbone network remains connected at all times, (ii) development of a routing algorithm that ensures a connection between the source-destination  node pair with the fewest number of path switching, (iii) given the dimensions of the air corridor and the radius of the {\em coverage sphere} associated with an ANP, development of an algorithm that finds the fewest number of ANPs required to provide complete coverage of the air corridor at all times, (iv) development of an algorithm that provides connected-coverage to the air corridor at all times, and (v) results of experimental evaluations of our algorithms, (vi) development of a  visualization tool that depicts the movement patterns of the ANPs and the resulting dynamic graph and the coverage volume of the backbone network.
\end{abstract}

\section{Introduction}
\label{sec:intro}
Efforts are currently underway in the U.S. Air Force to utilize a heterogeneous set of physical links (RF, Optical/Laser and SATCOM) to interconnect a set of terrestrial, space and highly mobile airborne platforms (satellites, aircrafts and Unmanned Aerial Vehicles (ANPs)) to form an Airborne Network (AN).
The design, development, deployment and management of a network where the nodes are mobile are considerably more complex and challenging than a network of static nodes. This is evident by the elusive promise of the Mobile Ad-Hoc Network (MANET) technology where despite intense research activity over the last fifteen years, mature solutions are yet to emerge \cite{Burbank06,Con07}. One major challenge in the MANET environment is the unpredictable movement pattern of the mobile nodes and its impact on the network structure. In case of an Airborne Network (AN), there exists considerable control over the movement pattern of the mobile platforms. A senior Air Force official can specify the controlling parameters, such as the {\em location, flight path and speed} of the ANPs to realize an AN with desired functionalities. Such control provides the designer with an opportunity of develop a topologically stable network, even when the nodes of the network are highly mobile. We view the AN as an infrastructure (a wireless mesh network) in the sky formed by mobile platforms such as aircrafts, satellites and UAVs to provide communication support to its clients such as combat aircrafts on a mission. Just as an Airborne Warning and Control System (AWACS) aircraft plays a role in a mission by providing communication support to fighter aircrafts directly engaged in combat, we believe that the aircrafts and ANPs forming the AN will provide similar support to the combat aircrafts over a much larger area. As shown in Fig.~\ref{fig:airCorridor1}, the combat air crafts on a mission fly through a zone referred to as an {\em air corridor}. In addition to forming a connected backbone network, the ANPs are also required to provide complete {\em radio coverage} in the air corridor so that the combat aircrafts, irrespective of their locations within the air corridor, have access to at least one backbone node (i.e., an ANP) and through it, the entire network. Accordingly, the AN is required to have two distinct properties: (1) the backbone network formed by the ANPs must remain {\em connected at all times}, even though the topology of the network changes with the movement of the ANPs, and (2) the entire three dimensional space of the air corridor is {\em covered at all times} by the continuously moving ANPs. To the best of our knowledge this is the first paper that proposes an architecture for an AN and provide solutions for the {\em all time connected-coverage problem of a three-dimensional space with mobile nodes}.

\begin{center}
\begin{figure*}[tbh]
    \begin{minipage}[tbh]{0.6\linewidth}
        \centering
       	\includegraphics[width=\textwidth, keepaspectratio]{Sudhi4airborne.pdf}
	\caption{A schematic view of the Airborne Network }
	\label{fig:airborne}
    \end{minipage}
    \hfill
    \begin{minipage}[tbh]{0.35\linewidth}
        \centering
       	\subfigure[Air Corridor and the combat aircrafts on a mission with planned flight paths]{\includegraphics[width=\textwidth, keepaspectratio]{airCorridor1.pdf}\label{fig:airCorridor1}}
	\hfill
	\subfigure[A section of air corridor]{\includegraphics[width=0.5\textwidth, keepaspectratio]{airCorridor2.pdf}
	\label{fig:airCorridor2}}
	\hfill
	\subfigure[Circular orbits of ANPs (black dots) placed at the top surface of air corridor]{\includegraphics[width=0.4\textwidth, keepaspectratio]{airCorridor4.pdf}
	\label{fig:airCorridor4}}
	\caption{Air Corridor, rectangular parallelopiped section, client airplanes, ANPs in circular orbits}
	\label{fig:airCorridor}
    \end{minipage}
\end{figure*}
\end{center}

One of the pioneering results in three dimensional coverage problem for sensor networks was presented by Haas {\em et al.} \cite{HAAS08,HAAS06} in which they concluded that the truncated octahedron has the highest volumetric quotient (the ratio of the volume of a polyhedron to the volume of its circumsphere) among all the space filling polyhedrons and utilized this to develop placement strategies for three dimensional underwater sensor networks. Their scheme is a centralized one. Distributed protocol of achieving three dimensional space coverage is found in the research of Tezcan {\em et al.} \cite{TEZ04}. Poduri {\em et al.} \cite{POD06} later on introduced the notion of  graphs and used it to obtain three dimensional sensor coverage. Similar research aiming at the coverage problem in 3D was also presented by \cite{CHEN08, LEI07, HUANG04}. However none of these researchers put any emphasis on the problem of obtaining coverage while the constituting nodes are mobile in a three dimensional space. The mobile nature of the ANPs in airborne networks add yet another dimension of difficulty to the 3D coverage problem.

In this paper we first propose an architecture for an AN where airborne networking platforms (ANPs - aircrafts, UAVs and satellites) form the backbone or mobile base stations of the AN, and the combat aircrafts on a mission function as mobile clients. We then proceed to determine the the number and initial location of the ANPs, their velocity and transmission range, so that the dynamically changing network retains properties (1) and (2) mentioned in the previous paragraph. The rest of the paper is organized as follows. In Section \ref{sec:sysModel}, we provide the system model and an architecture of an Airborne Network. Section \ref{sec:probFormulation} formally states the connectivity problem for an AN. In Section \ref{sec:connectivity}, we provide an algorithm that finds the velocity and the transmission range of the ANPs so that the dynamically changing  network remains connected at all times. Section \ref{sec:routing} presents a routing algorithm that ensures a connection between the source-destination  node pair with the fewest number of path switching.  Given the dimensions of the air corridor and the radius of the coverage sphere associated with an ANP, Section \ref{sec:coverageProbFormulation} formulates the coverage problem for the air corridor. Section \ref{sec:coverageSolution} presents an algorithm that finds the fewest number of ANPs required to provide complete coverage of the air corridor at all times. The Section \ref{sec:connCoverage} combines results of Sections \ref{sec:connectivity} and \ref{sec:coverageSolution} and presents an algorithm to provide connected-coverage to the air corridor at all times. In Section \ref{sec:visualization} we briefly describe a visualization tool that we developed to demonstrate the movement patterns of the ANPs and its impact on the resulting dynamic graph and the coverage volume of the backbone network. The results of experimental evaluations of our algorithms and related discussion is presented in Section \ref{sec:experiments}. Section \ref{sec:conclusion} concludes the paper.

\section{System Model and Architecture}
\label{sec:sysModel}
A schematic diagram of our view of an AN is shown in Fig.~\ref{fig:airborne}. In the diagram, the black aircrafts are the Airborne Network Platforms (ANP), the aircrafts that form the infrastructure of the AN (although in Fig.~\ref{fig:airborne}, only aircrafts are shown as ANPs,  the UAVs and satellites can also be considered as ANPs). We assume that the ANPs follow a circular flight path. The circular flight paths of the ANPs and their coverage area (shaded spheres with ANPs at the center) are also shown in Fig.~ \ref{fig:airborne}. Thick dashed lines indicate the communication links between the ANPs.  The figure also shows three fighter aircrafts on a mission passing through space known as {\em air corridor}, where network coverage is provided by ANPs 1 through 5. When the fighter aircrafts are at point P1 on their flight path, they are connected to ANP4 because point P1 is covered by ANP4 only. As the fighter aircrafts move along their flight trajectories, they pass through the coverage area of multiple ANPs and there is a smooth hand-off from one ANP to another when the fighter aircrafts leave the coverage area of one ANP and enter the coverage area of another. The fighter aircrafts are connected to an ANP as long as they are within the coverage area of that ANP. At points P1, P2, P3, P4, P5 and P6 on their flight path in Fig.~\ref{fig:airborne}, the fighter aircrafts are connected to the ANPs (4), (2, 4), (2, 3, 4), (3), (1, 3) and (1), respectively.

One major difference between the wireless mesh networks deployed in many U.S. cities \cite{Wifi08} and the ANs is the fact that, while the nodes of the wireless mesh networks deployed in the U.S.
cities are static, the nodes of an AN are highly mobile. However, as noted earlier, the AN designer has considerable control over the movements of the mobile platforms forming the AN. She can decide on the {\em locality} where the aircraft/ANPs should fly, its {\em altitude, flight path} and {\em speed of movement}. Control over these four important parameters, together with the knowledge of the {\em transmission range} of the transceivers on the flying platforms, provides the designer with an opportunity for creating a fairly stable network, even with highly mobile nodes. In this paper, we make a simplifying assumptions that two ANPs can communicate with each other whenever the distance between them does not exceed the specified threshold (transmission range of the onboard transmitter).  We are well aware of the fact that successful communication between two airborne platforms depends not only on the distance between them, but also on various other factors such as (i) the line of sight between the platforms \cite{TIW08}, (ii) changes in the atmospheric channel conditions due to turbulence, clouds and scattering, (iii) the  banking angle,  the wing obstruction and the dead zone  produced by the wake vortex of the aircraft \cite{EPS04} and (iv) Doppler effect \cite{Doppler} . Moreover, the transmission range of a link is not a constant and is impacted by various factors, such as transmission power, receiver sensitivity, scattering loss over altitude and range, path loss over propagation range, loss due to turbulence and the transmission aperture size \cite{EPS04}. However, the distance between the ANPS remains a very important parameter in determining whether communication between the ANPs can take place, and as the goal of this research is to understand the basic and fundamental issues of designing an AN with twin invariant properties of coverage and connectivity, we feel such simplifying assumptions are necessary and justified. Once the fundamental issues of the problem are well understood, factors (i) through (iv) can be incorporated into the model to obtain a more accurate solution.

\section{Design for Connectivity - Problem Formulation}
\label{sec:probFormulation}
It is conceivable that even if the network topology changes due to movement of the nodes, some underlying structural properties of the network may still remain invariant.
A structural property of prime interest in this context is the {\em connectivity} of the dynamic graph formed by the ANPs. We want the ANPs to fly in such a way, that even though the links between them are established and disestablished over time, the underlying graph remains connected at all times. Although we give connectedness of the graph as an example of a structural property, many other graph theoretic properties  can be specified as design requirements for the network. The problem can be described formally in the following way.

Consider  nodes (flying platforms) in an -dimensional space  (for ANP network scenario ). We denote by  the coordinates of the node  at time , where by convention  is considered a  column vector, and by , the  vector resulting from stacking the coordinates of the
nodes in a single vector. Suppose that the dynamics of node  (for all ), is given by , where  is the control vector taking values in some set . In vector notation, the system dynamics become


where  and  are  vectors, respectively. The network of flying platforms described by system (1), gives rise to a {\em dynamic graph} .

 is a dynamic graph consisting of

\begin{itemize}
\item a set of nodes  indexed by the set of flying platforms, and\vspace{-0.0in}
\item a set of edges  with  as the Euclidean distance between the platforms  and  and  is a constant.
\end{itemize}

Since we have control over the node dynamics, the question that naturally arises is whether we can control the motion of the ANPs so that  retains graph-theoretic properties of interest  for all time . A graph  is connected if there exists a path between any two nodes of the graph . Often times the property  will correspond to the requirement that the graph  remains connected at all times.  Formally the problem can be stated as follows. Suppose that  is the set of all graphs on  nodes with property . Is it possible to find a control law  such that if  then  for all ?

Although a few researchers have studied problems in this domain \cite{Mesbahi051, Mesbahi052, Zav07}, many important questions still remain unanswered. For example, in our study of the movement pattern of the ANPs to create a connected network, we assume that the flight paths of the mobile platforms are already known and we want to find out the speed at which these platforms should move, so that the resulting dynamic graph remains connected at all times. The studies undertaken in \cite{Mesbahi051,Mesbahi052,Zav07} do not address such issues. Although the movement of the airborne platforms will be in a three dimensional space, in a  simplified version of the problem in two dimension (i.e., when all the aircrafts are flying at the same altitude) the problem can be stated as follows:

\noindent
{\em Mobility Pattern for Connected Dynamic Graph  (MPCDG)}: This problem has five controlling parameters:\\
(i) a set of points \{ on a two (or three) dimensional space (representing the centers of circular flight paths of the platforms),\\ (ii) a set of radii \{\} representing the radii of circular flight paths, \
s^{2}_{ij}(t)=(\vec R_{i}(t)-\vec R_{j}(t))^{2}= R_{i}^{2}(t) +  R_{j}^{2}(t) - 2 \vec R_{i}(t) \cdot \vec R_{j}(t)
 \label{eq:vs}
 
s_{ij}(t) \leq D
\label{eq:sD}

tan~\beta_{i} = \frac{ R_{i}(0)cos~\theta_{i}(0) - r_{c_{i}}cos~\alpha_{c_{i}} }{ R_{i}(0)sin~\theta_{i}(0) - r_{c_{i}}sin~\alpha_{c_{i}}}
\label{eq:beta}

\vec R_{i}(t)=\vec r_{c_i} + \vec r_{i}(t)
\label{eq:vR}

R_{i}^{2}(t)= r_{c_i}^2 + r_{i}^{2} + 2r_{c_i} r_{i} \cos~(\beta_{i} - \alpha_{c_{i}} + \omega_{i}t )
\label{eq:R}

R_{i}(t)\cos \theta_{i}(t) & = & r_{c_i}\cos~\alpha_{c_i} + r_{i}\cos~(\beta_{i} + \omega_{i}t),
\label{eq:Rx}
\\
R_{i}(t)\sin~\theta_{i}(t) & = & r_{c_i}\sin~\alpha_{c_{i}} + r_{i}\sin~(\beta_{i} + \omega_{i}t)~~~
\label{eq:Ry}

&&R_{i}(t)R_{j}(t)\cos (\theta_{i}(t)-\theta_{j}(t)) = r_{c_i} r_{c_j} \cos~\alpha_{c_{i}c_{j}}\nonumber \\
&& + r_{i} r_{j} \cos(\beta_{ij}+(\omega_{i}-\omega_{j})t)+r_{c_i}r_{j} \cos(\alpha_{c_{i}} - \beta_{j} - \omega_{j}t) \nonumber \\
&& + r_{c_j}r_{i} \cos(\alpha_{c_{j}} - \beta_{i} - \omega_{i}t)
\label{eq:cosij}

s_{ij}^{2}(t) &= & r_{c_{i}}^{2} + r_{i}^{2} + 2r_{c_{i}}r_{i}\cos(\beta_{i} - \alpha_{c_{i}}+ \omega_{i}t) \nonumber\\
& + & r_{c_{j}}^{2} + r_{j}^{2} + 2r_{c_{j}}r_{j}\cos(\beta_{j} - \alpha_{c_{j}} + \omega_{j}t) \nonumber \\
& + & r_{c_i} r_{c_j} \cos~\alpha_{c_{i}c_{j}} + r_{i} r_{j} \cos(\beta_{ij} +(\omega_{i}-\omega_{j})t) \nonumber \\
& + & r_{c_i}r_{j} \cos(\alpha_{c_{i}} - \beta_{j} - \omega_{j}t) \nonumber \\
& + & r_{c_j} r_{i} \cos(\alpha_{c_{j}} - \beta_{i} - \omega_{i}t)
\label{eq:finals1}

s_{ij}^{2}(t) &= & r_{c_{i}}^{2} + r^{2} + 2r_{c_{i}}r\cos(\beta_{i} - \alpha_{c_{i}}+ \omega t) \nonumber\\
& + & r_{c_{j}}^{2} + r^{2} + 2r_{c_{j}}r\cos(\beta_{j} - \alpha_{c_{j}} + \omega t) \nonumber \\
& + & r_{c_i} r_{c_j} \cos~\alpha_{c_{i}c_{j}} + r^{2} \cos\beta_{ij} \nonumber \\
& + & r_{c_i}r \cos(\alpha_{c_{i}} - \beta_{j} - \omega t) \nonumber \\
& + & r_{c_j}r \cos(\alpha_{c_{j}} - \beta_{i} - \omega t)
\label{eq:finals2}

	\overline{TO} = \overline{TQ} - \overline{OQ} = h_{l} - y

	\angle{POC_{2}} = \angle{C_{2}OM} = \frac{\theta}{2} = \frac{\pi}{n}

\label{eq:hlwl}
\overline{PT}^{2} & = & \overline{PC_{2}}^{2} - \overline{C_{2}T}^{2} \nonumber\\
i.e., h_{l}^{2} & = & r_{s}^{2} - (\overline{C_{2}S} - \overline{TS})^{2} \nonumber\\
i.e., h_{l}^{2} & = & r_{s}^{2} - (r_{s}-w_{l})^{2} \nonumber\\
h_{l} & = & \sqrt{w_{l}(2r_{s}-w_{l})}

\label{eq:rswl}
\overline{C_{2}T} & = & \overline{TO} tan\frac{\theta}{2} \nonumber\\
r_{s} - w_{l} & = & (h_{l}-y) tan\frac{\pi}{n}\\
\label{eq:rsro}
\overline{C_{2}T} & = & \overline{C_{2}O} sin\frac{\theta}{2} \nonumber \\
r_{s} - w_{l} & = & r_{o}sin\frac{\theta}{2} \nonumber \\
w_{l} & = & r_{s} - r_{o}sin\frac{\pi}{n} 

\label{eq:rcy}
\overline{VA} & = & \overline{OU} \nonumber\\
& = & 2 (\overline{TQ} - \overline{OQ}) \nonumber\\
i.e., r_{c} & = & 2(h_{l}-y)

\label{eq:hcy}
\overline{VO}^{2} & = & \overline{VT}^{2} - \overline{TO}^{2}\nonumber \\
h_{c}^{2} & = & h_{l}^{2} - (h_{l}-y)^{2} \nonumber\\
h_{c} & = & \sqrt{y(2h_{l}-y)}

\label{eq:rc}
r_{c} & = & 2(h_{l}-y) ~~~~~~~~~~~~~~~~~(\textnormal{from equation \ref{eq:rcy}}) \nonumber\\
& = & 2(r_{s}-w_{l}) cot \frac{\theta}{2} ~~~~~~~~~~(\textnormal{from equation \ref{eq:rswl}}) \nonumber\\
& = & 2 r_{o} sin \frac{\theta}{2} cot \frac{\theta}{2} ~~~~~~~~~~~~(\textnormal{from equation \ref{eq:rsro}}) \nonumber\\
r_{c} & =  & 2 r_{o} cos \frac{\theta}{2} = 2 r_{o} cos \frac{\pi}{n}

h_{c}^{2} & = & h_{l}^{2} - (h_{l}-y)^{2} \nonumber\\
& = & (r_{s}^{2} - (r_{s}-w_{l})^{2}) - ((r_{s}-w_{l})cot \frac{\theta}{2})^{2}\nonumber \\
& = & r_{s}^{2} - ((r_{s} - w_{l})\frac{1}{sin \frac{\theta}{2}})^{2} \nonumber\\
& = & r_{s}^{2} - r_{o}^{2} \nonumber\\
h_{c} & = & \sqrt{r_{s}^{2} - r_{o}^{2}}\label{eq:hc}

\label{eq:vc}
v_{c} & = & \pi r_{c}^{2} h_{c}

minimize~mn  =  \lceil \frac{L_{ac}}{\sqrt{2}r_{c}}\rceil \times \lceil \frac{W_{ac}}{\sqrt{2}r_{c}}\rceil \times n &&\nonumber \\
= \lceil \frac{L_{ac}}{2\sqrt{2}r_{o}cos\frac{\pi}{n}}\rceil \times \lceil \frac{W_{ac}}{2\sqrt{2}r_{o}cos\frac{\pi}{n}}\rceil \times n &&\label{eq:s1obj} \\
subject~to:~~h_{c}  = \sqrt{r_{s}^{2} - r_{o}^{2}} \geq  H_{ac} && \label{eq:s1cons}

\frac{a}{b} = \frac{L_{ac}}{W_{ac}} \textnormal{    and     }(2r_{c})^{2} = a^{2} + b^{2}

a = \frac{2r_{c}L_{ac}}{\sqrt{L_{ac}^{2} + W_{ac}^{2}}},~~b = \frac{2r_{c}W_{ac}}{\sqrt{L_{ac}^{2} + W_{ac}^{2}}}

m & = & \lceil \frac{L_{ac}}{a} \rceil \times \lceil \frac{W_{ac}}{b} \rceil \\
& = & \lceil \frac{\sqrt{L_{ac}^{2} + W_{ac}^{2}}}{4r_{o}cos\frac{\pi}{n}} \rceil \times \lceil \frac{\sqrt{L_{ac}^{2} + W_{ac}^{2}}}{4r_{o}cos\frac{\pi}{n}} \rceil\\

minimize~mn = \lceil \frac{\sqrt{L_{ac}^{2} + W_{ac}^{2}}}{4r_{o}cos\frac{\pi}{n}} \rceil^{2} \times n &&\label{eq:s2obj} \\
subject~to:~~h_{c}  = \sqrt{r_{s}^{2} - r_{o}^{2}} \geq  H_{ac} && \label{eq:s2cons}

The diagrams for the placement of orbits, and hence the cylindrical regions following the above two strategies, are shown in Fig.~\ref{fig:illStrat1} and \ref{fig:illStrat2}, respectively. Hence, the location of he center of the orbits  can be easily determined.
\begin{figure}[!t]
\centering
\subfigure[Placement using Strategy 1]{\includegraphics[width=0.24\textwidth, keepaspectratio]{illStrat1.pdf}\label{fig:illStrat1}}
\hfill
\subfigure[Placement using Strategy 2]{\includegraphics[width=0.24\textwidth, keepaspectratio]{illStrat2.pdf}\label{fig:illStrat2}}
\caption{Covering  plane with circles of radius  using strategy 1 and 2}
\label{fig:strats}
\end{figure}
It may be observed both placement strategy 1 and 2 formulates the coverage problem as a non-linear optimization problem. We used the non-linear constrained program solver Nimbus \cite{nimbus} to solve optimization problems following strategies 1 and 2. The results obtained from Nimbus is discussed in Section \ref{sec:experiments}.

\section{Design for Connected Coverage}
\label{sec:connCoverage}

In Section \ref{sec:coverageProbFormulation}, we discussed the three dimensional air corridor coverage problem with the ANPs. As the ANPs are mobile, the coverage volume associated one ANP is continuously changing with time. In Section \ref{sec:coverageSolution} we described techniques to find least cost solution to the air corridor coverage problem with mobile nodes. Earlier in Section \ref{sec:connectivity}, we discussed how to determine the velocity and subsequently the transmission range of the of the ANPs, so that resulting backbone network formed by the ANPs remain connected at all times. In this section, we discuss how to design a network of ANPs, so that (i) it remains connected at all times and (ii) it provides 100\% coverage to the air corridor at all times.

We provide a two phase solution to the connected coverage problem. In the first phase, using the techniques described in Section \ref{sec:coverageSolution}, we determine the number and orbit of the ANPs that will provide 100\% coverage to air corridor at all times. Once that is accomplished, in the second phase, using the techniques described in Section \ref{sec:connectivity}, we determine the the velocity and the transmission range of the ANPs so that  the backbone network formed by the ANPS remains connected at all times. 

\section{Visualization Tool for Airborne Network Design}
\label{sec:visualization}
The visualization tool was designed for observing the movement of objects along circular orbits in a 3D plane. The Euclidean distance between every pair of objects keeps changing due the their movement. Each pair of objects has a threshold value specified. If the Euclidean distance between this pair of objectss is within the threshold, then they are connected by a link. As soon as the pairwise distance goes beyond the threshold, the link is broken. The tool was designed using OpenGL and C++. OpenGL is a 3D graphics API that works with C++ that provides dynamic interaction with the user. Mostly it is used for game programming and creating 3D scenes. For this program, we utilized some of the basic features of the API to create an interactive application to control the variables of each particular orbiting objects. One of the features of OpenGL is the ability to alter the camera view. This allows us to see every angle of the orbiting objects and rotate around the scene. A snapshot of the visualization tool with three moving points is shown in Fig.~\ref{fig:viz}.

\begin{figure}
\centering
\includegraphics[width=0.45\textwidth, keepaspectratio]{visual.pdf}
\caption{Snapshot of Visualization Tool with Three Moving Objects}
\label{fig:viz}
\end{figure}

The floor is based on a  grid. The center of each orbit has the ability to be moved along the  axis, ,  and the  axis . The -axis, which represents the height above the floor, allows the object to increase in height in the range . The radius of each circular orbit can be modified in the range . Each object has a connectivity threshold to each other. Each threshold, , , and  has the following range . Using the standard distance equation between two points in  space, a link will appear to declare if the objects are within the given threshold limit. The speed of each object is based on the system clock, which will vary between each computer. The speed of each object can be increased up to  units. Prior to each object being set in motion, they can be positioned strategically around their own orbit. The motion of all objects can be paused to analyze a specific situation. 

\section{Experimental Results and Discussion}
\label{sec:experiments}

\begin{figure*}
 \centering
 \subfigure[]{\includegraphics[width=0.32\textwidth, keepaspectratio]{100_70_10_Rs.pdf}\label{fig:rs1}}
 \hfill
 \subfigure[]{\includegraphics[width=0.32\textwidth, keepaspectratio]{100_180_10_Rs.pdf}\label{fig:rs2}}
 \hfill
 \subfigure[: Strategy 1 becomes the same as Strategy 2]{\includegraphics[width=0.32\textwidth, keepaspectratio]{100_100_10_Rs.pdf}\label{fig:rs3}}
 \caption{Variation of the ,  and the objective function for Strategy 1 and 2 (all values on  axis) with variable  and fixed value of  and }
 \label{fig:rs}
\end{figure*}

\begin{figure*}
 \centering
 \subfigure[]{\includegraphics[width=0.32\textwidth, keepaspectratio]{100_70_Hac_20.pdf}\label{fig:h1}}
 \hfill
 \subfigure[]{\includegraphics[width=0.32\textwidth, keepaspectratio]{100_180_Hac_20.pdf}\label{fig:h2}}
 \hfill
 \subfigure[: Strategy 1 becomes the same as Strategy 2]{\includegraphics[width=0.32\textwidth, keepaspectratio]{100_100_Hac_20.pdf}\label{fig:h3}}
 \caption{Variation of the ,  and the objective function for Strategy 1 and 2 (all values on  axis) with variable  and fixed value of  and }
 \label{fig:h}
\end{figure*}

\begin{figure*}
 \centering
 \subfigure[ versus ]{\includegraphics[width=0.27\textwidth, keepaspectratio]{ncosn.pdf}\label{fig:cos}}
 \hfill
 \subfigure[Change of objective function with  and ]{\includegraphics[width=0.33\textwidth, keepaspectratio]{strat2Obj.pdf}\label{fig:obj}}
 \hfill
 \subfigure[Objective function in log scale]{\includegraphics[width=0.35\textwidth, keepaspectratio]{strat2ObjLog.pdf}\label{fig:objLog}}
 \caption{The change of objective function and its components with  and }
 \label{fig:objs}
\end{figure*}

In this section we present the experimental evaluation results of two strategies proposed in Section \ref{sec:coverageSolution}.  The goal of these experiments were to find the impact of change of (i) radius of the coverage sphere () and (ii) height of the air corridor () on (a) radius  of the circular orbit of the flying ANPs (), (b) the number of ANPs in each orbit (), and (c) the total number of ANPs () needed to provide complete coverage for the air corridor, specified by its length, width and height parameters , respectively.  Fig.~\ref{fig:rs} show impact of changing  on  and  for different sets of values for  and for two different strategies 1 and 2. Fig.~\ref{fig:h} show impact of changing  on ,  and  for different set of values for ,  and for two different strategies 1 and 2. The parameter values for  and  used for the experimentation are indicated in the figures.

Since the optimal coverage problem turned out to be a non-linear optimization problem (equations (\ref{eq:s1obj}) to (\ref{eq:s2cons})), we used the non-linear constrained program solver Nimbus \cite{nimbus} to solve it using two different ANP orbit placement strategies 1 and 2. In the following we discuss some experimental results, some of which are intuitive, some others are not.

\medskip
\noindent
{\em Observation 1: }From Fig.~\ref{fig:rs}, it can be seen that increase in  results in increase in  and decrease in  for both the strategies 1. This is somewhat intuitive as it is only natural to expect that as the radius of the coverage sphere increases, the radius of the circular orbit of the ANPs will increase and the total number of ANPs needed to cover the entire air corridor will decrease. It may also be noted that when  is too small compared to , there may not be a feasible solution.
    
\medskip
\noindent
{\em Observation 2: }From Fig.~\ref{fig:h}, it can be seen that increase in  results in decrease in  and increase in  for both the strategies 1 and 2. This is also somewhat intuitive,  as the height of the air corridor increases, the radius of the circular orbit of the ANPs has to decrease (please see discussion in Section \ref{sec:coverageSolution}) and the total number of ANPs needed to cover the entire air corridor must increase. 

\medskip
\noindent
{\em Observation 3: }From Figs.~\ref{fig:rs} and \ref{fig:h}, it can be seen that, , the number of ANPs in an orbit remains a constant irrespective of changes in  and . This result is not at all obvious. However, on closer examination of the objective function in equations (\ref{eq:s1obj}) and (\ref{eq:s2obj}), one can find an explanation for this phenomenon. The plot of  versus  is shown in Fig.~\ref{fig:cos}. This factor is present in the objective function for both the strategies. From Fig.~\ref{fig:cos},  reaches its minimum value when  . Therefore the objective functions in equations (\ref{eq:s1obj}) and (\ref{eq:s2obj}) are minimized when . This nature of  also explains the fact in observation 1, where  decreases when  decreases (i.e., when  increases). Similarly, it also explains the fact in observation 2, where  increases when  increases (i.e., when  decreases).
 
\medskip          
\noindent
 {\em Observation 4: }From the Figs.~\ref{fig:rs1},~\ref{fig:rs2},~\ref{fig:h1},~\ref{fig:h2}, it can be seen that the cost of the solution (i.e., the number of ANPs needed to provide complete coverage of the air corridor) using strategy 1 is less than that of strategy 2. Although, the reason for this phenomenon may not be obvious at a first glance, on closer examination, we can explain the phenomenon.  Given the fact that  and presence of these two terms in objective functions of strategies 1 and 2 (equations (\ref{eq:s1obj}) and (\ref{eq:s2obj}) in page 8), it is not surprising that cost of the solution strategy 1 is less than that of strategy 2.

From our experiments we learn that (i)strategy 1 performs better than strategy 2 in all cases, except where , for which both the strategies are identical, (ii)the number of ANPs in an orbit remains a constant () irrespective of the values of  and , when the objective function is specified  by equations (\ref{eq:s1obj}) or (\ref{eq:s2obj}), and (iii)to optimize the objective function, the radius of the circular orbit of the ANPs should be made as large as possible subject to the constraint that the height of the corresponding invariant coverage cylinder is at least as large as the height of the air corridor .

\section{Conclusion}
\label{sec:conclusion}
\balance Existence of sufficient control over the movement pattern of the mobile platforms in Airborne Networks opens the avenue for designing topologically stable hybrid networks. In this paper, we discussed the system model and architecture for Airborne Networks (AN). We studied the problem of maintaining the connectivity in the underlying dynamic graphs of airborne networks with control over the mobility parameters and developed an algorithm to solve the problem. 

\bibliographystyle{IEEEtran}
\begin{thebibliography}{10}
\providecommand{\url}[1]{#1}
\csname url@samestyle\endcsname
\providecommand{\newblock}{\relax}
\providecommand{\bibinfo}[2]{#2}
\providecommand{\BIBentrySTDinterwordspacing}{\spaceskip=0pt\relax}
\providecommand{\BIBentryALTinterwordstretchfactor}{4}
\providecommand{\BIBentryALTinterwordspacing}{\spaceskip=\fontdimen2\font plus
\BIBentryALTinterwordstretchfactor\fontdimen3\font minus
  \fontdimen4\font\relax}
\providecommand{\BIBforeignlanguage}[2]{{\expandafter\ifx\csname l@#1\endcsname\relax
\typeout{** WARNING: IEEEtran.bst: No hyphenation pattern has been}\typeout{** loaded for the language `#1'. Using the pattern for}\typeout{** the default language instead.}\else
\language=\csname l@#1\endcsname
\fi
#2}}
\providecommand{\BIBdecl}{\relax}
\BIBdecl

\bibitem{Burbank06}
J.~L. Burbank, P.~H. Chimento, B.~K. Haberman, and W.~T. Kasch, ``Key
  {C}hallenges of {M}ilitary {T}actical {N}etworking and the {E}lusive
  {P}romise of {M}{A}{N}{E}{T} {T}echnology,'' \emph{IEEE Communication
  Magazine}, November 2006.

\bibitem{Con07}
M.~Conti and S.~Giardano, ``Multihop ad-hoc {N}etworking: the {R}eality,''
  \emph{IEEE Communications Magazine}, April 2007.

\bibitem{HAAS08}
S.~M. Alam and Z.~J. Haas, ``Coverage and {C}onnectivity in
  {T}hree-{D}imensional {U}nderwater {S}ensor {N}etworks,'' \emph{Wireless
  Communications and Mobile Computing}, vol.~8, no. 995-1009, 2008.

\bibitem{HAAS06}
------, ``Coverage and {C}onnectivity in {T}hree-{D}imensional {N}etworks,'' in
  \emph{International Conference on Mobile Computing and Networking}, 2006.

\bibitem{TEZ04}
H.~Tezcan, E.~Cayirci, and V.~Coskun, ``A {D}istributed {S}cheme for 3{D} space
  {C}overage in {T}actical {U}nderwater {S}ensor {N}etworks,'' in
  \emph{MILCOM}, 2004.

\bibitem{POD06}
S.~Poduri, S.~Pattem, B.~Krishnamachari, and G.~S. Sukhatme, ``Sensor {N}etwork
  {C}onfiguration and the {C}urse of {D}imensionality,'' in \emph{EmNets},
  2006.

\bibitem{CHEN08}
F.~Chen, P.~Jiang, and A.~Xue, ``An {A}lgorithm of {C}overage {C}ontrol for
  {W}ireless {S}ensor {N}etworks in 3{D} {U}nderwater {S}urveillance
  {S}ystems,'' \emph{Advanced Intelligent Computing Theories and Applications},
  vol. 5226, no. 1206-1213, 2008.

\bibitem{LEI07}
R.~Lei, L.~Wenyu, and G.~Peng, ``A {C}overage {A}lgorithm for
  {T}hree-{D}imensional {L}arge-{S}cale {S}ensor {N}etwork,'' in
  \emph{Intelligent Signal Processing and Communication Systems}, 2007.

\bibitem{HUANG04}
C.~Huang, Y.~Tseng, and L.~Lo, ``The {C}overage {P}roblem in
  {T}hree-{D}imensional {W}ireless {S}ensor {N}etwork,'' in \emph{GLOBECOM
  2004}, 2004.

\bibitem{Wifi08}
\BIBentryALTinterwordspacing
``City-wide {W}i-fi {P}rojects.'' [Online]. Available:
  \url{http://www.seattlewireless.net,http://www.wirelessphiladelphia.org,
  http://www.waztempe.com}
\BIBentrySTDinterwordspacing

\bibitem{TIW08}
A.~Tiwari, A.~Ganguli, and A.~Sampath, ``Towards a {M}ission {P}lanning
  {T}oolbox for {A}irborne {N}etworks: {O}ptimizing {G}round {C}overage {Under}
  {C}onnectivity {C}onstraints,'' in \emph{IEEE Aerospace Conference}, March
  2008, pp. 1--9.

\bibitem{EPS04}
B.~Epstein and V.~Mehta, ``Free {S}pace {O}ptical {C}ommunications {R}outing
  {P}erformance in {H}ighly {D}ynamic {A}irspace {E}nvironments,'' in
  \emph{IEEE Aerospace Conference Proceedings}, 2004.

\bibitem{Doppler}
\BIBentryALTinterwordspacing
``Doppler {E}ffect.'' [Online]. Available:
  \url{http://en.wikipedia.org/wiki/Doppler\_effect}
\BIBentrySTDinterwordspacing

\bibitem{Mesbahi051}
M.~Mesbahi, ``On {S}tate-dependent {D}ynamic {G}raphs and their
  {C}ontrollability {P}roperties,'' \emph{IEEE Transactions on Automatic
  control}, vol.~50, no.~3, pp. 2473--2478, 2005.

\bibitem{Mesbahi052}
------, ``Controlling {C}onnectivity of {D}ynamic {G}raphs,'' in \emph{44th
  IEEE Conference on Decision and Control}, December 2005.

\bibitem{Zav07}
M.~M. Zavlanos and G.~J. Papas, ``Potential {F}ields for {M}aintaining
  {C}onnectivity of {M}obile {N}etworks,'' \emph{IEEE Transactions on
  Robotics}, vol.~23, no.~4, August 2007.

\bibitem{Cormen}
T.~H. Cormen, C.~E. Leiserson, R.~L. Rivest, and C.~Stein, \emph{Introduction
  to {A}lgorithms}.\hskip 1em plus 0.5em minus 0.4em\relax McGraw Hill, 2001.

\bibitem{Diestel05}
R.~Diestel, \emph{Graph Theory}.\hskip 1em plus 0.5em minus 0.4em\relax
  Springer, 2005.

\bibitem{nimbus}
\BIBentryALTinterwordspacing
``Nimbus: Interactive multi-objective optimization system.'' [Online].
  Available: \url{http://wwwnimbus.it.jyu.fi/N4/index.html}
\BIBentrySTDinterwordspacing

\end{thebibliography}


\end{document}
