\documentclass{article}


\usepackage{arxiv}

\usepackage[utf8]{inputenc} \usepackage[T1]{fontenc}    \usepackage{hyperref}       \usepackage{url}            \usepackage{booktabs}       \usepackage{amsfonts}       \usepackage{nicefrac}       \usepackage{microtype}      \usepackage{lipsum}
\usepackage{graphicx}
\graphicspath{ {./images/} }
\usepackage{multirow}
\usepackage{lineno}
\usepackage{amsmath}
\usepackage{amssymb}

\title{MMBAttn: \textbf{M}ax-\textbf{M}ean and \textbf{B}it-wise \textbf{Att}e\textbf{n}tion for CTR Prediction}

\author{
 Hasan Saribas \\
  AI Enablement\\
  Huawei Türkiye R\&D Center\\
  Istanbul, Turkey \\
  \texttt{hasan.saribas1@huawei.com} \\
\And
 Cagri Yesil \\
  AI Enablement\\
  Huawei Türkiye R\&D Center\\
  Istanbul, Turkey \\
  \texttt{cagri.yesil1@huawei.com} \\
  \And
 Serdarcan Dilbaz \\
  AI Enablement\\
  Huawei Türkiye R\&D Center\\
  Istanbul, Turkey \\
  \texttt{serdarcan.dilbaz@huawei.com} \\
   \And
 Halit Orenbas \\
  AI Enablement\\
  Huawei Türkiye R\&D Center\\
  Istanbul, Turkey \\
  \texttt{halit.orenbas@huawei.com} \\
}

\begin{document}
\maketitle
\begin{abstract}
With the increasing complexity and scale of click-through rate (CTR) prediction tasks in online advertising and recommendation systems, accurately estimating the importance of features has become a critical aspect of developing effective models. In this paper, we propose an attention-based approach that leverages max and mean pooling operations, along with a bit-wise attention mechanism, to enhance feature importance estimation in CTR prediction. Traditionally, pooling operations such as max and mean pooling have been widely used to extract relevant information from features. However, these operations can lead to information loss and hinder the accurate determination of feature importance. To address this challenge, we propose a novel attention architecture that utilizes a bit-based attention structure that emphasizes the relationships between all bits in features, together with maximum and mean pooling. By considering the fine-grained interactions at the bit level, our method aims to capture intricate patterns and dependencies that might be overlooked by traditional pooling operations. To examine the effectiveness of the proposed method, experiments have been conducted on three public datasets. The experiments demonstrated that the proposed method significantly improves the performance of the base models to achieve state-of-the-art results.
\end{abstract}

\keywords{CTR prediction \and attention \and feature importance \and recommendation}


\section{Introduction}
Click-through rate (CTR) prediction is a critical problem in web search, recommendation systems and online advertisement displaying.
CTR prediction is important because it helps service providers attract and engage their users by ranking a small number of items from a large number of candidates.
Accurate CTR prediction allows for better optimization of advertising campaigns, improved user targeting, and enhanced revenue generation.
Improving the accuracy of advertising CTR can not only make the user experience better but also give more benefits to advertising platforms and advertisers.
Since many platforms rely primarily on advertisement revenues as their primary mode of income, the user base for advertisements is very large. Traditional approaches to CTR prediction have relied on statistical methods and hand-engineered features. While these methods have achieved moderate success, they often struggle to capture the complex and nonlinear relationships inherent in user behavior and contextual factors. Moreover, they require extensive manual feature engineering, making them less scalable and adaptable to dynamic advertising landscapes. Machine Learning (ML) techniques have been able to automatically discover relevant interactions for CTR but with rapid innovations in deep learning (DL) architectures, DL models have been able to better learn high-order feature interactions in sparse spaces. In recent years, deep learning has emerged as a powerful paradigm for solving complex prediction tasks by automatically learning hierarchical representations from raw data. This approach has shown remarkable success in various domains, such as computer vision and natural language processing. Consequently, researchers have directed their attention towards developing deep learning models for CTR prediction, aiming to leverage their capacity to capture intricate patterns and dependencies within large-scale datasets. Early approaches in CTR prediction leveraged logistic regression but these approaches lacked the ability capture higher-order interactions \cite{logistic2013trenches, logistic2007}. Later, the factorization machine (FM) structure gained popularity since FM projects sparse features onto low-dimensional space and learns feature interactions via inner products \cite{fm}. Over time, as deep learning approaches improved, a plethora of deep learning models such as Wide\&Deep, DeepFM, DCN, AutoInt, FiBiNET have seen widespread use for CTR prediction \cite{wide&deep, guo2017deepfm, deep&cross, song2019autoint, huang2019fibinet}. The attention mechanism has proven very powerful in deep learning and has been utilized heavily in state-of-the-art models in many domains \cite{attn2021review}. Attention has been applied for CTR prediction as well \cite{huang2019fibinet, zhang2022fibinet++, song2019autoint}.
In this article, we present a novel deep learning model where \textbf{M}ax-\textbf{M}ean and \textbf{B}it-wise \textbf{Att}e\textbf{n}tion (MMBAttn) are leveraged. Our model addresses the limitations of traditional methods and builds upon the state-of-the-art deep learning techniques by efficiently leveraging the attention mechanism with the lightweight maximum and mean operations.
The main contributions of this paper are summarized as follows:
\begin{itemize}
  \item We propose a new attention architecture (MMBAttn) for CTR prediction to emphasize important features.
  \item The proposed method is designed as a plug-and-play module and is therefore model-agnostic. It can be easily applied to any CTR prediction model.
  \item We performed extensive experiments on three public datasets, namely Criteo, Avazu, and Frappe. Additionally, we conducted ablation studies on Criteo and Avazu datasets to evaluate the performance of each component proposed by our method. The results indicate that the inclusion of each additional component leads to improved performance.
\end{itemize}

\begin{figure}[t]
\begin{center}
\includegraphics[width=\textwidth]{Attention.pdf}
\caption{The general architecture of the proposed attention mechanism.}\label{fig:architecture}
\end{center}
\end{figure}

\subsection{Related Works}

\subsubsection{Click-through Rate Prediction}
Logistic regression (LR) was one of the early proposed models for CTR prediction and it has proven to be a strong baseline \cite{logistic2013trenches, logistic2007}. Nonetheless, LR cannot accurately learn nonlinear interactions, so factorization machines (FM) were introduced. Opposed to LR, FM embeds the features to a dense so that the feature interactions can be modelled via inner products \cite{fm}. Multilayer perceptron was one of the earlier deep learning models to be used in CTR prediction \cite{covington2016deep}. Later, Wide\&Deep model improved on the multilayer perceptron structure by combining a wide network with a deep network to leverage the advantages of both approaches \cite{wide&deep}. DeepFM model built on the FM and Wide\&Deep models to learn second-order feature interactions \cite{guo2017deepfm}. Deep\&Cross Network (DCN) learns to model an even higher order of interactions to perform better CTR prediction \cite{deep&cross}.

\subsubsection{Attention Networks}
The attention mechanism was first proposed for neural machine translation where attention overcomes the information bottleneck created by recurrent neural network architectures \cite{bahdanau2014neural}. The seminal work from \cite{vaswani2017attention} showed that the attention mechanism can be used on its own to learn powerful representations from sequential data to tackle problems ranging from text summarization \cite{el2021automatic} to question answering \cite{yu2019deep}.  The success of attention in natural language processing led to the widespread usage of attention in other domains. Many of the seminal works in computer vision utilized the attention structure to obtain architectures that learn better representations. Attention was initially added to existing architectures where commonly used vision blocks such as convolution and pooling were combined with attention \cite{wang2020eca, hu2018squeeze, woo2018cbam}. Models that primarily rely on attention were able to achieve state-of-the-art results in the vision domain \cite{khan2022transformers, dosovitskiy2021image}. Based on this widespread success, attention mechanisms have also been used for CTR prediction.

\subsubsection{Attention Networks for CTR}
In this paper, we propose a new attention architecture (MMBAttn) for CTR prediction to emphasize important features. Attention mechanisms have shown great promise in improving the performance of recommendation systems by allowing them to attend to the most informative user-item interactions. For instance, the Attentional Factorization Machine (AFM) model \cite{xiao2017attentional} applies the attention mechanism to the intermediate output of the factorization machine layer to focus on the pairwise interaction of feature embeddings. The Deep Interest Evolution Network (DIEN) and Deep Session Interest Network (DSIN) models \cite{zhou2019deep, feng2019deep} do not apply attention to all the features. They utilize self-attention mechanisms to only sequential features to capture user behavior changes. Moreover, there are models that apply attention mechanisms not just to sequential features but also to categorical and numeric feature embeddings or feature interactions. For instance, the Autoint model \cite{song2019autoint} uses a multi-head self-attentive neural network with residual connections to explicitly model feature interactions in low-dimensional space. The InterHAt model \cite{li2020interpretable} applies a transformer with multi-head self-attention to input embeddings to capture interactive relations between features without using the Multi-Layer Perceptron (MLP) structure. After the multi-head self-attention layer, a hierarchical attention mechanism is utilized to benefit from high-order interactions. In contrast to Autoint and InterHAt, the Gated Attention Transformer (GAT) model \cite{long2021efficient} uses vanilla attention mechanisms and Gated Attention Transformer instead of self-attention to overcome the drawbacks of Autoint and InterHAt.
On the other hand, the Fibinet and Fibinet++ models \cite{huang2019fibinet, zhang2022fibinet++} learn the importance of features with the Squeeze-Excitation (SENET) and SENET+ mechanisms which are initially used in computer vision domain \cite{wang2020eca}. The SENET and SENET+ mechanisms consist of three components: squeeze, excitation, and re-weighting. In the squeeze part of SENET, the i-th feature is summarized with mean or max pooling, and the squeezed value represents the global information about the i-th feature embedding. In the excitation part, a weight is learned for each feature embedding by applying an MLP to the squeezed value vector created in the squeeze part. In the re-weight part, new feature embeddings are created by multiplying the learned weights and the original feature embeddings. The SENET+ updates the squeeze parts of SENET by segmenting each feature embedding into  groups. Then, both the max and mean values of each group are calculated separately, and the resulting vector is forwarded to the excitation part. Finally, Fibinet and Fibinet++ combine the output of SENET and SENET+ layers with a bilinear interaction layer and forward it to MLP layers for prediction.
The FAT-DeepFFM model \cite{zhang2019fat} introduces the Compose-Excitation network (CENet) field attention mechanism, which is an enhanced version of SENET to highlight feature importance. CENet combines SENET with the DeepFFM model, which is based on the Field-aware factorization machines (NFM) \cite{juan2017field}. This study also applies CENet on features before explicit feature interaction and on cross features after explicit feature interaction for comparison. Unlike SENET, CENet carries out the pooling operation with 1x1 convolution instead of max and mean pooling in the excitation part. Then, two fully connected (FC) layers are used for the excitation part, as in SENET.
The ECANFM model \cite{shi2021ctr} proposes the ECANET concept, which is based on the ECA module used in the computer vision domain, and combines it with the NFM model \cite{he2017neural}. The ECANFM model aims to improve feature embeddings by dynamically learning the importance of feature embeddings. ECANET uses mean pooling for the squeeze part. But in the excitation part, it uses one-dimensional convolution instead of fully connected layers.
Lastly Masknet model \cite{wang2021masknet} uses a kind of attention mechanism that learns weights for value in feature embeddings to highlight the important parts. Unlike SENET and its variants, Masknet does not use a squeeze module. It normalizes the feature embeddings and directly applies fully connected layers to the normalized features to learn the weights.
This study presents a novel approach to feature importance estimation that differs from existing works. Specifically, we propose an attention structure that combines max and mean pooling operations with multiple MLPs to obtain feature importances. To the best of our knowledge, there are only two studies that use max and mean pooling together including Fibinet++ \cite{zhang2022fibinet++} and CBAM \cite{woo2018cbam}. In Fibinet++ max and mean pooling are used together in a group-wise structure that employs shared MLPs in the excitation step. CBAM uses max and mean pooling operations with shared MLPs in the vision domain. In contrast, our approach uses Max-Mean pooling operations with separate MLPs. Unlike using shared MLPs employed in previous studies, our approach uses Max-Mean pooling operations with two distinct MLPs to emphasize feature importance that comes from max and mean pooling branches separately. Unlike previous methods that rely solely on max and mean pooling operations, our approach also employs a bit-wise structure that captures attention between all bits using a distinct MLP. This allows us to utilize information that may be lost during pooling operations. Importantly, our proposed attention structure integrates all these components to provide a comprehensive and effective feature importance estimation method. Overall, our study contributes to the field by introducing a novel approach that improves upon existing methods and provides a more accurate and comprehensive understanding of feature importance.

\section{Method}
In this study, an attention-based method has been proposed to better highlight the importance of features. Fig.~\ref{fig:architecture} illustrates the attention-based method. In the proposed approach, mean and max pooling operations are applied to the features to obtain feature importance. However, these pooling operations may result in information loss. To address this, a bit-wise attention structure is additionally utilized. With the bit-wise structure, we can emphasize the relationships between all bits across all features and achieve an architecture that highlights these relationships. The proposed attention-based models are designed as a plug-and-play module, hence they can be applied to all existing CTR prediction models.



\noindent\textbf{Embedding layer.} CTR prediction models often deal with high-dimensional and sparse input features, such as categorical variables representing user information, item details, and contextual factors. To effectively capture the relationships and interactions among these features, feature embedding layers are utilized. The embedding of a categorical variable  in an embedding layer can be calculated using the following equation:



\noindent where  is the one-hot representation of the categorical variable  at index . It is a binary vector of size , where  is the vocabulary size (number of unique values).
 is the embedding matrix. Each row in the matrix represents the embedding vector of a unique category.  is the dimensionality of the embedding, which is a hyperparameter specified prior to training.  is the embedding vector of every  to which it belongs.


\noindent\textbf{Max-Mean and Bit-wise attention (MMBAttn) module.} In this part, all components of the designed attention mechanism including, max, mean and bit-wise attention will be explained in detail.
Accordingly, the embedding weights in the max and mean attention blocks can be calculated as follows




\noindent where  is a Sigmoid function and  is embedding-wise max and mean pooling and can be computed as  and , respectively, where  and  is a scalar value and it represents the global information about the  embedding vector.  is MLP and can be formulate as 





\noindent where  and  are MLP weights used to learn feature importance and activation function ReLU comes after .  is reduction ratio. Output of max and mean attentions as



The re-weighted embedding vector obtained using max and mean attentions can be expressed as follows,



\noindent where  is an element-wise multiplication.

Similarly, the bit-wise embedding attention vector  can be calculated as



\noindent where  is concatenated embedding vector.

\noindent The re-weighted embedding vector obtained using bit-wise attentions can be formulated by



\noindent The final re-weighted embedding vector  can be expressed as follows,



\noindent Finally, this re-weighted embedding vector can be fed to any CTR prediction model.

\noindent\textbf{Training phase.} During the training phase, the widely used loss function for binary classification problems, also known as Binary Cross Entropy (BCE) or log loss, was employed. The formula for log loss is as follows:



In this equation,  \{0, 1\} represents the label value,  (0, 1) denotes the model's prediction, and  denotes the total number of examples used during training.



\section{Experiments}

In this section, the datasets used in the experiments, the evaluation metrics for comparing the methods, and the neural network-based methods employed are discussed. 


\subsection{Datasets}


\begin{table}[tbh]
\caption{The statistics of public datasets.}
\label{table:datasets}
\begin{center}
\begin{tabular}{|c c c c|} 
\hline
 Datasets & \# Instances & \# Fields & \# Features\\ [0.5ex] 
 \hline \hline
 Avazu & 40,428,967 & 22 & 1,544,250 \\
 \hline
  Criteo & 45,840,617 & 39 & 2,086,936 \\
 \hline
 Frappe & 288,609 & 10 & 5,382 \\  [0.25ex]   
 \hline
\end{tabular}
\end{center}
\end{table}

The experiments in this study were conducted on three publicly available datasets: Avazu, Criteo and Frappe. The statistical information for each dataset is listed in Table \ref{table:datasets}.

\textbf{Avazu\footnote{\url{https://www.kaggle.com/c/avazu-ctr-prediction/data}}}: Avazu dataset was published in a CTR prediction competition on Kaggle. It consists of chronologically ordered ad click data. The dataset includes an ID, click information, and 22 categorical features.

\textbf{Criteo\footnote{\url{http://labs.criteo.com/2013/12/download-terabyte-click-logs}}}: The Criteo dataset is widely used in many CTR modeling studies and is considered a benchmark for CTR prediction. The Criteo dataset comprises 26 anonymous categorical fields and 13 numerical feature fields.

\textbf{Frappe\footnote{\url{https://www.baltrunas.info/context-aware}}}: The Frappe dataset comprises a log of mobile app usage by users in different contexts. It includes entries for apps used by users in various situations. The target variable indicates whether the user has used a particular app in a specific context. This dataset comprises 10 fields.

\subsection{Evaluation Metrics}

Two evaluation metrics were used in the experiments. These metrics are AUC (Area Under ROC) and log loss (Cross entropy).

\textbf{AUC}: AUC represents the area under the ROC curve and is a commonly used evaluation metric for measuring the performance of classification problems. It measures the probability of assigning a higher score to a randomly chosen positive item than to a randomly chosen negative item for the CTR predictor. A higher AUC represents better performance.

\textbf{Log loss}: Log loss is a commonly used metric in binary classification problems. It measures the difference between the actual value and the predicted value. A smaller log loss value represents better performance.





\subsection{Implementation Details}
We implement our method based on the repository of FuxiCTR \cite{zhu2021open}, an open-source CTR prediction benchmark \footnote{\url{https://fuxictr.github.io}}. We experiment and report on reproducing the results of state-of-the-art models using the hyperparameters provided by this benchmark. To ensure the reliability of the results, we run the all base models 5 times. All the experiments are carried out with Python programming language (3.10.9) and Tensorflow (2.11.0) library on a system that has NVIDIA A40 GPU, AMD EPYC 7513 32-Core Processor, 100 GB RAM. 

\begin{table}[t]
\centering
\caption{Ablation studies for DNN model on Criteo and Avazu datasets.}
\label{table:performance}
\begin{tabular}{lcccccc}
\hline
\multirow{2}{*}{Model}                      & \multicolumn{3}{c}{Criteo} & \multicolumn{3}{c}{Avazu} \\
\cline{2-4} \cline{5-7}
                            & AUC    & Impr.   & LogLoss   & AUC     & Impr.    & LogLoss   \\
\hline \hline
DNN                         & 0.813  & Base    & 0.4381    & 0.7622  & Base    & 0.3681    \\
DNN + Mean                  & 0.8134 & 0.05\%  & 0.4381    & 0.7623  & 0.01\%  & 0.3683    \\
DNN + Max                   & 0.8136 & 0.07\%  & 0.438     & 0.76402 & 0.23\%  & 0.3673    \\
DNN + Bit-wise              & 0.8138 & 0.10\%  & 0.4379    & 0.76273 & 0.07\%  & 0.3681    \\
DNN + Max + Mean            & 0.8137 & 0.09\%  & 0.438     & 0.76489 & 0.34\%  & 0.3671    \\
DNN + Max + Mean + Bit-wise & 0.8143 & 0.15\%  & 0.4378    & 0.76499 & 0.37\%  & 0.3669    \\
\hline
\end{tabular}
\end{table}

\subsection{Baseline}
We compared our proposed model with several state-of-the-art models designed for CTR prediction and implemented in the FuxiCTR repository \cite{zhu2021open}. These models include Fibinet \cite{huang2019fibinet}, MaskNet \cite{wang2021masknet}, and AutoInt+ \cite{song2019autoint}, which all use attention mechanisms. Details about these models are provided in the introduction section. Although EDCN, DCN, and DeepFM \cite{deep&cross, chen2021enhancing, guo2017deepfm} do not utilize attention mechanisms, we included them in our comparison since they achieved the best results in the BarsCTR benchmark. Specifically, DCN and EDCN use MLP and Cross Networks with different fusion strategies, while DeepFM combines MLP and FM layers. FinalMLP \cite{mao2023finalmlp} is a two-stream model where each stream consists of MLP layers. One stream is fed with user-related features and the other with item-related features, and then a bilinear group aggregation strategy is used to fuse the two streams. The proposed attention mechanism has been applied to the baselines of DeepFM, FinalMLP, and EDCN models, as well as to a DNN model with 3 MLP layers of size 400 and ReLU activation function. This model is then compared to the same DNN model that uses the same hyperparameters but without attention.

\begin{figure}[ht]
\begin{center}
\includegraphics[width=\textwidth]{Results.png}
\caption{Performance of ablations for different reduction ratios and embedding sizes on the Criteo and Avazu datasets.}
\label{fig:reduction}
\end{center}
\end{figure}

\subsection{Ablation Studies}
The results in Table ~\ref{table:performance} showcase the performance of all components of the proposed attention mechanism on the Criteo and Avazu datasets. The DNN model represents the baseline performance on both datasets. In both datasets, the max pooling attention mechanism significantly increased the AUC scores, while the mean pooling mechanism achieved comparable scores to the baseline. Especially in the Avazu dataset, max pooling improved the baseline model's result by 0.23\%. The bit-wise component improves the AUC scores by 0.1\% and 0.07\% on Criteo and Avazu datasets, respectively. When the maximum and mean pooling attention mechanisms are used together, it results in an improvement in the AUC scores for both datasets. Particularly, there is a significant improvement of approximately 0.34\% in the AUC score for the Avazu dataset. Additionally, the log loss value also decreases on this dataset.  When all the components are used together (MMBAttn), the highest improvement in AUC and log loss metrics is achieved. The proposed attention mechanism particularly provides a 0.37\% improvement in AUC for the Avazu dataset, reducing the log loss value from 0.3681 to 0.3669. This indicates that the combined use of maximum and mean pooling with the bit-wise attention mechanism effectively captures the intricate relationships between features and enhances the overall predictive performance of the CTR prediction model. The results demonstrate the effectiveness of the proposed approach in improving both accuracy and log loss metrics, especially on the Avazu dataset.
Fig. ~\ref{fig:reduction} displays an analysis of the reduction ratio and embedding size for the DNN + MMBAttn model on the Criteo and Avazu datasets. It is evident that both benchmarks have similar performance in terms of the AUC across various reduction ratios. Notably, the highest performances on both datasets are achieved with a reduction size of 3. Comparing the performance of the models for different embedding sizes, we can observe that embedding sizes 8 and 10 achieved the highest AUC values of 0.81425\% and 0.76499\% in the Criteo and Avazu datasets, respectively.

\subsection{State-of-the-art Comparison}

Table ~\ref{table:sota} shows the results of the proposed attention structure applied to DNN, DeepFM, EDCN and FinalMLP methods and its comparison with the results of other state-of-the-art methods. We used best hyperparameters obtained by ablation studies.
In all three datasets, the model with MMBAttn (Ours) improved model performances compared to the DNN baseline in terms of both AUC and Logloss metrics. Notably, the AUC is improved by 0.37\% on Avazu dataset while by 0.15\% on both Criteo and Frappe datasets. When we apply the proposed attention mechanism (MMBAttn) to the DeepFM model, the AUC scores are improved by 0.07\%, 0.092\% and 0.15\% on the Criteo, Avazu and Frappe datasets, respectively, while the log loss values decrease. Similarly, when we plug MMBAttn into the EDCN and FinalMLP models, the performance of the models showed a slight improvement compared to the DNN version model. Finally, several studies have shown that even a small improvement in terms of AUC metric in CTR prediction performance leads to large increases in online CTR scores and hence revenue \cite{wide&deep, ling2017ensemble, deep&cross}. Overall, even with the version of the MMBAttn mechanism plugged into the simple DNN version (DNN + MMBAttn) that does not utilized any complex interaction layers such as FM, CrossNet, Inner or Outer products etc., it achieved the second-best scores after the FinalMLP model which has state-of-the-art scores on all three datasets.

\begin{table}[ht]
\centering
\caption{State-of-the-art comparison on Criteo, Avazu and Frappe datasets.}
\label{table:sota}
\resizebox{\textwidth}{!}{
\begin{tabular}{lccccccccc}
\hline
\multirow{2}{*}{Model}                        & \multicolumn{3}{c}{Criteo}   & \multicolumn{3}{c}{Avazu}      &  \multicolumn{3}{c}{Frappe} \\
\cline{2-4} \cline{5-7} \cline{8-10}
                            & AUC    & Impr.   & LogLoss   & AUC     & Impr.   & LogLoss    & AUC     & Impr.   & LogLoss\\\hline \hline
Fibinet                     & 0.813  & -       & 0.4388    & 0.7645  & -       & 0.3673     & 0.9832  & -       & 0.1941\\
MaskNet                     & 0.8139 & -       & 0.438     & 0.7643  & -       & 0.3674     & 0.9837  & -       & 0.1696\\
DCN                         & 0.8138 & -       & 0.4381    & 0.7652  & -       & 0.3665     & 0.9839  & -       & 0.1527\\
AutoInt+                    & 0.8139 & -       & 0.4379    & 0.7645  & -       & 0.3668     & 0.9848  & -       & 0.149\\ \hline
DNN                         & 0.813  & Base    & 0.4381    & 0.7622  & Base    & 0.3681     & 0.9835  & Base    & 0.1605\\
DNN + MMBAttn (Ours)        & 0.8143 & 0.15\%  & 0.4378    & 0.765   & 0.37\%  & 0.3669     & 0.985   & 0.15\%  & 0.1441\\
DeepFM                      & 0.8137 & Base    & 0.4381    & 0.7648  & Base    & 0.3667     & 0.9842  & Base    & 0.1566\\
DeepFM + MMBAttn (Ours)     & 0.8143 & 0.07\%  & 0.4377    & 0.7655  & 0.092\% & 0.3666     & 0.9857  & 0.15\%  & 0.1507\\
EDCN                        & 0.8143 & Base    & 0.43766   & 0.7627  & Base    & 0.3681     & 0.9847  & Base    & 0.1576\\
EDCN + MMBAttn (Ours)       & 0.8145 & 0.02\%  & 0.43751   & 0.7637  & 0.13\%  & 0.3672     & 0.9852  & 0.051\% & 0.1476\\
FinalMLP                    & 0.81486 & Base   & 0.4372    & 0.7663  & Base    & 0.3659     & 0.9855  & Base    & 0.1487\\
FinalMLP + MMBAttn (Ours)   & 0.81497 & 0.01\% & 0.4371    & 0.7666  & 0.04\%  & 0.3657     & 0.9861  & 0.061\% & 0.1475\\
\hline
\end{tabular}}
\end{table}
\section{Conclusion}

In this work, we presented an attention-based approach, MMBAttn, for enhancing feature importance estimation in CTR prediction tasks in online advertising and recommendation systems. By leveraging max and mean pooling operations along with a bit-wise attention mechanism, the proposed method addresses the challenge of accurately determining feature importance, which is crucial for developing effective models in complex and large-scale CTR prediction scenarios. Traditional pooling operations, such as max and mean pooling, are widely used to extract relevant information from features. However, these operations can lead to information loss and hinder the accurate estimation of feature importance. To overcome this limitation, MMBAttn introduces a novel attention architecture that uses a bit-based attention structure that emphasizes the relationships between all bits in features, together with maximum and mean pooling. Extensive experiments were conducted on three public datasets to evaluate the effectiveness of the proposed method. The results demonstrate significant improvements in the performance of the base models, achieving state-of-the-art results. The proposed method is also designed as a plug-and-play module and is therefore model-agnostic, making it easily applicable to any CTR prediction model.
Future work can explore the application of the proposed attention mechanism in other domains and further investigate optimization techniques.

\bibliographystyle{unsrt}
\bibliography{references}

\end{document}
