



\documentclass{article}

\usepackage{spconf,amsmath,theorem,epsfig,amssymb}
\usepackage{amsfonts}
\usepackage{pst-all}

\newcommand{\F}{\mathbb{F}}
\newtheorem{thm}{Theorem}















\title{Efficient FPGA-based multipliers for  and }

\threeauthors
  {Jamshid Shokrollahi
	\sthanks{Partially funded by the German Research Foundation (DFG) under
project RU 477/8}}
  {B-IT\\ University of Bonn \\ Germany \\
   jamshid@bit.uni-bonn.de}
{Elisa Gorla
  \sthanks{Partially funded by the Swiss National Science Foundation under
grant no. 107887}}
  {Department of Mathematics \\ University of Z\"urich\\ Switzerland \\
   elisa.gorla@math.unizh.ch}
  {\setcounter{footnote}{1}Christoph Puttmann}
  {Heinz Nixdorf Institute\\ University of Paderborn \\ Germany \\
   puttmann@hni.upb.de}




\hyphenation{Tam-pe-re micro-soft}




\begin{document}


\maketitle




\begin{abstract}
In this work we present a new structure for multiplication in finite fields. 
This structure is based on a digit-level LFSR (Linear Feedback Shift
Register) multiplier in which the area of digit-multipliers are
reduced using the Karatsuba method. We compare our results with the
other works of the literature for . We also propose new
formulas for multiplication in . These new
formulas reduce the number of -multiplications from 
to . The finite fields  and  are
important fields for pairing based cryptography. 
\end{abstract}


{\bf Keywords}: finite field multiplication, FPGA, pairing based
cryptography.

\section{Introduction}
\label{sec:intro}

Efficient multiplication in finite fields is a central task in the
implementation of most public key cryptosystems. A great amount of
work has been devoted to this topic (see \cite{knu98} or
\cite{gatger03} for a comprehensive list).
The two types of finite fields which are mostly used in cryptographic
standards are binary finite fields of type  and prime
fields of type , where  is a prime (cf. \cite{dss00}).
Efforts to efficiently fit finite field arithmetic into commercial
processors resulted into applications of medium characteristic finite
fields like those reported in \cite{baipaa98} and \cite{avamih03}.
Medium characteristic finite fields are fields of type ,
where  is a prime slightly smaller than the word size of the
processor, and has a special form that simplifies the modular
reduction. Mersenne prime numbers constitute an example of primes
which are used in this context. 
The security parameter is given by the length of the binary
representations of the field elements, and the extension degree  is
selected appropriately.
Due to security considerations, the extension degree for fields of
characteristic  or medium characteristic is usually chosen to be
prime.

With the introduction of the method of Duursma and Lee for the
computation of the Tate pairing (cf. \cite{duulee03}), fields of type
 for  prime have attracted special attention.
Computing the Tate pairing on elliptic curves defined over
 requires computations both in  and in
.
In \cite{kermar05} calculations are implemented using the tower of
extensions  and the
inherent parallelism of multiplication in extension fields is used to
accelerate the operations. 
Hardware designs and especially FPGA-based ones are suitable platforms
for parallel implementation of algorithms.
In that work multiplications in the first and the second field
extensions are computed via  and  multiplications in the ground
fields, respectively, requiring  multiplications in .

In our current work, which is mostly based on \cite{kermar05}, on the
one hand, we use asymptotically fast methods to improve the
performance of multiplication in , and on the other hand,
we propose new multiplication formulas to speedup multiplication in
.
Using the new formulas, multiplication in  is
done with only  multiplications instead of . 
We use the same extension tower, using  multiplications in
 to multiply elements in , but only
 multiplications in  for .
Our proposed method has a slightly increased number of
additions in comparison to the Karatsuba method. Notice however that a
multiplication in  requires many more resources than an
addition, therefore the overall resource consumption will be reduced.
The details of our method to generate the new formulas have been
omitted to limit the complexity and diversity of materials in this
paper, and have been submitted as another paper for CHES 2007.

A consistent amount of work has been done on hardware-based
multiplication in finite fields, especially those of characteristic
. The authors of \cite{bergua03} propose a least significant
digit-element (LSDE) multiplier for . 
This multiplier divides the input polynomials into digits of length D. Whereas the digits of one input polynomial are processed in parallel, the digits of the other input polynomial are handled serially. Then the result is reduced modulo the
irreducible polynomial. 
The same structure has also been used in \cite{kermar05} for
multiplication in . 
Our multiplier, on the other hand, is based on the digit-serial
implementation of LFSR (Linear Feedback Shift Register) multiplier
which is widely used in the literature (see \cite{mce87} or
\cite{sho06}), and performs the modular reduction during the multiplication. The first contribution of our current work is
the application of the Karatsuba multiplier inside the
digit-multipliers, which results in smaller area for these
multipliers. 
Our results demonstrate the efficiency of this design compared to other works. 
The second contribution is the application of a method using only 
multiplications in  for multiplication in
. This results in an area-saving of almost
 compared to the Karatsuba method which is used in
\cite{kermar05}.
   
Our work is organized as follows. Section~\ref{sec:multf397} is
devoted to the general structure of our multiplier for . In
Section~\ref{sec:circuit} we describe some improvements on the
traditional LFSR multiplier and compare our results with other works
from the literature. 
In Section~\ref{sec:multf3697} the new formulas for  together with suggestions for a new multiplier are
presented, and Section~\ref{sec:conclusion} concludes the paper.

\section{Multiplication in }
\label{sec:multf397}

The finite field  can be represented as a vector space
over . In this representation, elements of  are
vectors of length  over . Addition of elements is computed
by adding corresponding vectors. Multiplication is more complicated,
and depends on the selected basis for . There are two
popular bases which are used often in the literature, namely
polynomial and normal bases. 
A polynomial basis is generally more suitable for multiplication,
hence we choose this basis in our work. 

In the polynomial basis, elements of  are represented as
polynomials of degree at most  over . Two elements are added
by adding of the corresponding polynomials. Multiplication is based on polynomial
multiplication followed by reduction modulo the irreducible polynomial,
which generates the polynomial basis. In our case the irreducible
polynomial, which we denote by , is 

In the next sections we show the details of polynomial arithmetic in
our designs.

\subsection{Arithmetic in }

The element  is represented using the vector  of two bits such that the elements , , and  are
, , , respectively. In this representation the
operations addition, multiplication, and negation (multiplication by
) are done, as shown in \cite{grapag05}, using
Equations~\ref{equ:bitadd}, \ref{equ:bitmult}, and \ref{equ:bitneg},
respectively. 

The implementation of Equations~\ref{equ:bitadd} and \ref{equ:bitmult}
is done using 2 LUTs in the FPGA, whereas (\ref{equ:bitneg}) is only a
permutation of bits.

\subsection{Structure of the multiplier for }

The structure of a digit-level LFSR multiplier is shown in
Figure~\ref{fig:lfsr}. In this figure the two input polynomials
, and  are loaded into registers  and ,
respectively, and divided into digits of length . In each clock
cycle the most significant digit of  is multiplied by the words of
, through digit-multipliers specified by M, and added to the
content of the register in the feedback circuit.
Inputs to the digit multipliers are two polynomials of degree  in .
The product is a polynomial of degree . Powers  to
 of each multiplier must be added to the powers  to
 of the next multiplier. This is done by the overlap circuit.
In each clock cycle the register  and LFSR are shifted by  bits to the right.
Shifting LFSR to right is equivalent to multiplication by  which
generates the powers  to . These powers are reduced
modulo  of (\ref{equ:irreducible}) using the feedback circuit. 
The name Linear Feedback Shift Register descends from these feedback structures. 
 For more information about the digit-level LFSR multiplier and its
 costs for classical methods see \cite{sho06}. In the next section we
 discuss our improvements to the traditional LFSR multiplier.  

\begin{figure}
\includegraphics[width=8cm]{lfsr.eps}
\caption{Structure of a digit-level LFSR multiplier}
\label{fig:lfsr}
\end{figure}


\section{The Karatsuba method}
\label{sec:circuit}

In this section we use asymptotically fast methods to reduce the size of digit-multipliers. 
We use a similar approach to \cite{gatsho06} and combine the classical
and the Karatsuba methods to build small digit-multipliers.
Two linear polynomials  and  are multiplied
classically using the formula

with  multiplications and  addition. 
The same product can also be computed via

The new formula is called the Karatsuba method (see
\cite{karofm63}). It requires  operations instead of , but only
 multiplications, and uses fewer resources when the coefficients
  are replaced by polynomials. 
The classical method for multiplication of two polynomials of degree
 requires  operations. Recursive application of
the Karatsuba method reduces the cost of a multiplication to
 operations.
We represent the classical multiplication of two polynomials of degree
 by  and the method of (\ref{equ:karatsuba}) by
.
The methods  for , and 
constitute a set of polynomial multiplication methods. We call this set
. Using the elements of  we define the set of
recursive multiplication methods  which contains the
elements of  and all recursive combinations of elements of
. The recursive combination of the two methods
 and , for polynomials of lengths  and
, respectively, is the multiplication method
 for polynomials of length . Let

be given polynomials. In order to apply , we
write these polynomials as 

where  and  are
polynomials of degree .
If the polynomials  and  were coefficients,
the two polynomials  and  would be multiplied using
. The product using the method 
consists of several multiplications of the polynomials  and
, which are performed using .
We implement the digit-multipliers using the elements of 
to reduce their size. Our approach is similar to \cite{gatsho06}.

In Table~\ref{tab:lfsr397costs} we show the results of implementing
 multipliers on a XC2VP20-6FF896 FPGA. In this
table the first column is the digit-size . In a digit-level
multiplier with digit-size , inputs are preceded by enough zeros
so that their length becomes a multiple of . Hence it is natural to
choose a value of  such that the difference  is as small as possible. Our values for  are selected using
this criteria and hence differ from other standard values like
multiples of  in other works (see \cite{bergua03} and
\cite{kermar05}). 
The string in the second column shows the recursive combination of the
Karatsuba and classical methods which is applied.
It is important to notice that the method ,
which we used for polynomials of degree , applies to polynomials of
length . 
Therefore, we add a zero in front of the polynomial and then remove all
the gates containing an operation with the coefficients which are
known to be zero.
Hence this multiplier requires fewer resources than a complete
. This point distinguishes our
approach from that in \cite{gatsho06}. In the third, fourth, and fifth
columns are the number of slices, maximum working frequency of the
multiplier, and the required clock cycles for our designs.   

\begin{table}
\caption{Timing and area costs of digit-level LFSR multipliers in
   for different values of digit-size }
\begin{scriptsize}
\begin{tabular}{|c|c|c|c|c|}\hline
{} & {\bf Multiplication} & {\bf \# of slices} & {\bf Maximum} & {\bf \# of clock}\\
 & & & {\bf frequency (MHz)}& {\bf cycles} = \\ \hline
 &  &  &  & \\
 &  &  &  & \\
 &  &  &  & \\
 &  &  &  & \\
 &  &  &  & \\ \hline
\end{tabular}
\end{scriptsize}
\label{tab:lfsr397costs}
\end{table}
 
The results of comparing our results with those in \cite{kermar05} are
shown in Figure~\ref{fig:lfsrcomp}. Different digit-levels result in
different circuits, which we compare with respect to both time and
area. Area is the number of slices, whereas time is the product of
clock cycles and minimum period. Both designs are on the same
technology, but the speed grade of the FPGA in \cite{kermar05} is not
available.
As it is shown, our designs have better area-time performance. 
These improvements result, on the one hand, by using asymptotically
faster methods, and on the other hand, by integrating the modular
reduction stage into the LFSR. When a small digit-serial multiplier is
used even the small size of a modular reduction must be taken into account.
\begin{figure}

\def\dataKerMar05{1006 664 1821 298 2655 217 4335 180
    }
  \def\dataOurs{327 323 800 281 1716 200 2954 126 4006 97
  }
\psset{xunit=0.0012cm,yunit=0.005cm}
  \centering
\begin{pspicture}(5000,-200)(-1200,1200)
  \psaxes[Dx=1000,Dy=250,linewidth=1.2pt]{->}(0,0)(5000,1000)
  \listplot{\dataOurs}
  \listplot[linestyle=dashed]{\dataKerMar05}
  \rput(2500,-150){\scriptsize Area [Number of Slices]}
  \rput{90}(-1100,500){\scriptsize Time [ns]}
  \rput{90}(-900,500){\scriptsize (Number of clock cycles 
  Minimum clock-period)}
  \rput[l](2000,850){\psline(0,0)(1000,0)}
  \rput[l](3100,850){\scriptsize Our designs}
  \rput[l](2000,900){\psline[linestyle=dashed](0,0)(1000,0)}
  \rput[l](3100,900){\scriptsize Multipliers from \cite{kermar05}}
\end{pspicture}\hspace{0.75em}

\caption{Time vs. area comparisons of our multipliers with those in \cite{kermar05}}
\label{fig:lfsrcomp}
\end{figure} 

\section{Multiplication in }
\label{sec:multf3697}

Multiplication in  is done in the same way as in
\cite{kermar05}, as a tower of extensions of degrees  and , i.e.

The elements of  are polynomials of degree  in
 over , for  a root of  in .
The polynomials are multiplied by applying (\ref{equ:karatsuba}) and then
reduced modulo . The elements of  are
polynomials of degree  in , a root of  in . They are multiplied using the formulas
(\ref{equ:poly4formulas}) and then reduced modulo .

Combining (\ref{equ:karatsuba}), (\ref{equ:poly4formulas}) we have the
following theorem.

\begin{thm}
Let  be given as:

Let further their product  be 

Then the coefficients  of the product can be computed
using only  multiplications in .
\end{thm}

Closed-form formulas for this multiplication are shown in Appendix
\ref{sec:appendix}. Scalar multiplications are particularly simple
using these formulas. Scalar multiplications are multiplications by
, , and . Negation of coefficients and consequently of
polynomials is only a permutation of bits, as seen in
Section~\ref{sec:multf397}.
Indeed multiplication of an element in  by  is
a permutation, too. 
Let , then
 

All of the -multiplications can be done in parallel. This
property allows designers to implement as many of these multipliers as
possible, according to their time-area constraints. On the other hand,
these multipliers are used for other computations such as point
addition and doubling on elliptic curves for pairing-based
cryptography. Reading and writing intermediate values into register
files in such applications is time-consuming. 
To solve this problem we propose a new multiplier which is shown in Figure~\ref{fig:proposedstructure}. 
The new multiplier consists of three pipeline stages, namely, input, multiplication, and output. 
During the time of each multiplication in , the input stage loads the coefficients  and  from memory for the next multiplication, and computes the linear combinations in (\ref{equ:formulasinput}) to compute s.
In this time the output stage adds the last computed product  to memory variables according to (\ref{equ:formulasoutput}). 
In this structure the hatched multiplexers can select either one of their
inputs or the sum of the inputs. In this way all possible multiples of
input polynomials can be selected and added to the accumulators.

\begin{figure}
\centering
\psset{unit=0.3cm}
\begin{pspicture}(-2,-2)(24,28)
\rput[b](5,25){\scriptsize from memory}
\rput[b](17,25){\scriptsize from memory}
\rput[lb](0,15){
	\psframe(2,0)(8,2)
	\rput(5,1){\scriptsize accumulator}
	\psline(5,2)(5,3)
	\pspolygon[fillstyle=hlines](4,3)(6,3)(7,4)(3,4)
	\pspolygon(1,5)(3,5)(4,6)(0,6)
	\psline(2,5)(2,4.5)(4,4.5)(4,4)
	\pspolygon(7,5)(9,5)(10,6)(6,6)
	\psline(8,5)(8,4.5)(6,4.5)(6,4)
	\psframe(0.25, 7)(1.75, 8.5)
	\rput(1,7.75){\scriptsize }
	\psline(1,7)(1,6)
	\psframe(2.25, 7)(3.75, 8.5)
	\rput(3,7.75){\scriptsize }
	\psline(3,7)(3,6)
	\psframe(6.25,7)(7.75,8.5)
	\rput(7,7.75){\scriptsize }
	\psline(7,7)(7,6)
	\psline(9,6)(9,9)(1,9)(1,8.5)
	\psline(3,8.5)(3,9)
	\psline(7,8.5)(7,9)
	\psline(5,9)(5,10)
}
\rput[lb](12,15){
	\psframe(2,0)(8,2)
	\rput(5,1){\scriptsize accumulator}
	\psline(5,2)(5,3)
	\pspolygon[fillstyle=hlines](4,3)(6,3)(7,4)(3,4)
	\pspolygon(1,5)(3,5)(4,6)(0,6)
	\psline(2,5)(2,4.5)(4,4.5)(4,4)
	\pspolygon(7,5)(9,5)(10,6)(6,6)
	\psline(8,5)(8,4.5)(6,4.5)(6,4)
	\psframe(0.25, 7)(1.75, 8.5)
	\rput(1,7.75){\scriptsize }
	\psline(1,7)(1,6)
	\psframe(2.25, 7)(3.75, 8.5)
	\rput(3,7.75){\scriptsize }
	\psline(3,7)(3,6)
	\psframe(6.25,7)(7.75,8.5)
	\rput(7,7.75){\scriptsize }
	\psline(7,7)(7,6)
	\psline(9,6)(9,9)(1,9)(1,8.5)
	\psline(3,8.5)(3,9)
	\psline(7,8.5)(7,9)
	\psline(5,9)(5,10)
}
\rput[lb](7,0){
	\psframe(2,0)(8,2)
	\rput(5,1){\scriptsize accumulator}
	\psline(5,2)(5,3)
	\pspolygon[fillstyle=hlines](4,3)(6,3)(7,4)(3,4)
	\pspolygon(1,5)(3,5)(4,6)(0,6)
	\psline(2,5)(2,4.5)(4,4.5)(4,4)
	\pspolygon(7,5)(9,5)(10,6)(6,6)
	\psline(8,5)(8,4.5)(6,4.5)(6,4)
	\psframe(0.25, 7)(1.75, 8.5)
	\rput(1,7.75){\scriptsize }
	\psline(1,7)(1,6)
	\psframe(2.25, 7)(3.75, 8.5)
	\rput(3,7.75){\scriptsize }
	\psline(3,7)(3,6)
	\psframe(6.25,7)(7.75,8.5)
	\rput(7,7.75){\scriptsize }
	\psline(7,7)(7,6)
	\psline(9,6)(9,9)(1,9)(1,8.5)
	\psline(3,8.5)(3,9)
	\psline(7,8.5)(7,9)
	\psline(5,9)(5,10)
	\psline(5,0)(5,-1)
	\psline(7,2)(7,2.5)(12,2.5)(12,-1)
	\rput[t](5,-1){\scriptsize to memory}
	\rput[t](12,-1){\scriptsize from memory}
}
\psframe(2,10)(20,14)
\psframe[linestyle=dotted](-1,14.5)(23.5,24.5)
\rput[b]{90}(-1.2,19.5){\scriptsize input stage}
\psframe[linestyle=dotted](-1,-0.5)(23.5,9.5)
\rput[b]{90}(-1.2,4.5){\scriptsize output stage}
\rput(11, 12){\scriptsize -multiplier}
\psline(5,15)(5,14)
\psline(17,15)(17,14)
\end{pspicture}

 \caption{The proposed structure block for implementing the formulas of
  Appendix~\ref{sec:appendix}}
\label{fig:proposedstructure}
\end{figure}

\section{Conclusion}
\label{sec:conclusion}

In this paper we proposed a new structure for multiplication in
. This structure is based on digit-level LFSR multipliers, 
where the area of digit-multipliers are reduced using the Karatsuba
method. 
Another advantage of this approach is performing the modular reduction
during the multiplication. Our synthesis results showed the
performance improvement compared to other designs in the
literature. We have also presented new formulas for multiplication in
 using only  multiplications in
.
When the Karatsuba method is applied 18 multiplications are required.
Furthermore, we have introduced a feasible hardware structure for realizing our proposed formulas. Our formulas are for the case that  is
constructed from  using the irreducible
polynomial . In case that the finite field is constructed using
, the formulas require slight modifications.

\small

\begin{thebibliography}{10}
\providecommand{\url}[1]{#1}
\csname url@rmstyle\endcsname
\providecommand{\newblock}{\relax}
\providecommand{\bibinfo}[2]{#2}
\providecommand\BIBentrySTDinterwordspacing{\spaceskip=0pt\relax}
\providecommand\BIBentryALTinterwordstretchfactor{4}
\providecommand\BIBentryALTinterwordspacing{\spaceskip=\fontdimen2\font plus
\BIBentryALTinterwordstretchfactor\fontdimen3\font minus
  \fontdimen4\font\relax}
\providecommand\BIBforeignlanguage[2]{{\expandafter\ifx\csname l@#1\endcsname\relax
\typeout{** WARNING: IEEEtran.bst: No hyphenation pattern has been}\typeout{** loaded for the language `#1'. Using the pattern for}\typeout{** the default language instead.}\else
\language=\csname l@#1\endcsname
\fi
#2}}

\bibitem{knu98}
D.~E. Knuth, \emph{The Art of Computer Programming, vol. 2, Seminumerical
  Algorithms}, 3rd~ed.\hskip 1em plus 0.5em minus 0.4em\relax Reading~MA:
  Addison-Wesley, 1998, first edition 1969.

\bibitem{gatger03}
J.~von~zur Gathen and J.~Gerhard, \emph{Modern Computer Algebra}, 2nd~ed.\hskip
  1em plus 0.5em minus 0.4em\relax Cambridge,~UK: Cambridge University Press,
  2003, first edition 1999.

\bibitem{dss00}
\BIBentryALTinterwordspacing
\emph{Digital Signature Standard (DSS)}, U.S. Department of Commerce / National
  Institute of Standards and Technology, January 2000, federal Information
  Processings Standards Publication 186-2. [Online]. Available:
  \url{http://csrc.nist.gov/publications/fips/fips186-2/fips186-2.pdf}
\BIBentrySTDinterwordspacing

\bibitem{baipaa98}
D.~V. Bailey and C.~Paar, ``Optimal extension fields for fast arithmetic in
  public-key algorithms,'' in \emph{Advances in Cryptology: Proceedings of
  CRYPTO~'98, {\rm Santa Barbara~CA}}, ser. Lecture Notes in Computer Science,
  H.~Krawczyk, Ed., no. 1462.\hskip 1em plus 0.5em minus 0.4em\relax
  Springer-Verlag, 1998, pp. 472--485.

\bibitem{avamih03}
R.~M. Avanzi and P.~Mih{\u{a}}ilescu, ``Generic efficient arithmetic algorithms
  for {{PAFF}}s (processor adequate finite fields) and related algebraic
  structures (extended abstract),'' in \emph{Selected Areas in Cryptography
  (SAC 2003)}.\hskip 1em plus 0.5em minus 0.4em\relax Springer-Verlag, 2003,
  pp. 320--334.

\bibitem{duulee03}
\BIBentryALTinterwordspacing
I.~Duursma and H.~Lee, ``Tate-pairing implementations for tripartite key
  agreement.'' [Online]. Available:
  \url{citeseer.ist.psu.edu/duursma03tatepairing.html}
\BIBentrySTDinterwordspacing

\bibitem{kermar05}
T.~Kerins, W.~P. Marnane, E.~M. Popovici, and P.~S. L.~M. Barreto, ``Efficient
  hardware for the tate pairing calculation in characteristic three,'' in
  \emph{Cryptographic Hardware and Embedded Systems, CHES2005}, ser. Lecture
  Notes in Computer Science, J.~R. Rao and B.~Sunar, Eds., vol. 3659.\hskip 1em
  plus 0.5em minus 0.4em\relax Springer-Verlag, 2005, pp. 412--426.

\bibitem{bergua03}
G.~Bertoni, J.~Guajardo, S.~Kumar, G.~Orlando, C.~Paar, and T.~Wollinger,
  ``Efficient {} arithmetic architectures for cryptographic
  applications,'' in \emph{{Topics in cryptology, {CT-RSA 2003}: the
  Cryptographers' Track at the {RSA} Conference 2003, San Francisco, {CA},
  {USA}, April 13--17, 2003: Proceedings}}, ser. Lecture Notes in Computer
  Science, M.~Joye, Ed., vol. 2612.\hskip 1em plus 0.5em minus 0.4em\relax
  Springer-Verlag, 2003, pp. 158--175.

\bibitem{mce87}
R.~J. McEliece, \emph{{F}inite {F}ields for {C}omputer {S}cientists and
  {E}ngineers}.\hskip 1em plus 0.5em minus 0.4em\relax Kluwer Academic
  Publishers, 1987.

\bibitem{sho06}
\BIBentryALTinterwordspacing
J.~Shokrollahi, ``Efficient implementation of elliptic curve cryptography on
  fpgas,'' Ph.D. dissertation, Bonn University, Bonn, December 2006. [Online].
  Available:
  \url{http://hss.ulb.uni-bonn.de/diss\_online/math\_nat\_fak/2007/shokrollahi\_jamshid/ index.htm}
\BIBentrySTDinterwordspacing

\bibitem{grapag05}
\BIBentryALTinterwordspacing
R.~Granger, D.~Page, and M.~Stam, ``Hardware and software normal basis
  arithmetic for pairing-based cryptogaphy in characteristic three,''
  \emph{IEEE Transactions on Computers}, vol.~54, no.~7, pp. 852--860, 2005.
  [Online]. Available: \url{http://}
\BIBentrySTDinterwordspacing

\bibitem{gatsho06}
J.~von~zur Gathen and J.~Shokrollahi, ``{Fast arithmetic for polynomials over
   in hardware},'' in \emph{IEEE Information Theory Workshop
  (2006)}.\hskip 1em plus 0.5em minus 0.4em\relax Punta del Este, Uruguay:
  IEEE, March 2006, pp. 107--111.

\bibitem{karofm63}
A.~Karatsuba and Y.~Ofman, ``Multiplication of multidigit numbers on
  automata,'' \emph{Soviet Physics--Doklady}, vol.~7, no.~7, pp. 595--596,
  January 1963, translated from Doklady Akademii Nauk SSSR, Vol.~145, No.~2,
  pp.~293--294, July, 1962.

\end{thebibliography}


\appendix

\section{Multiplication formulas for }
\label{sec:appendix}

Let  be given as:

where  and ,  are roots of  and
, respectively.
Let their product  be 

Then the coefficients  of the product
can be computed using the following formulas.


\newpage


\end{document}
