\documentclass[runningheads,a4paper]{llncs}


\usepackage{mathrsfs}
\usepackage[USenglish]{babel}
\usepackage{theoremref} 
\usepackage{extarrows}
\usepackage{color}
\usepackage{xspace}
\usepackage{tikz}
	\usetikzlibrary{arrows,shapes,automata,backgrounds,decorations}
\usepackage{graphicx}
\usepackage{mdwlist,paralist}\setcounter{tocdepth}{3}
\usepackage{url}
\usepackage{xifthen}
\usepackage{hyperref}
\usepackage{DynamicCausality}

\usepackage{wasysym,amsmath,mathtools,amssymb}

\begin{document}

\title{Dynamic Causality in Event Structures (Technical Report)\thanks{Supported by the DFG Research Training Group SOAMED.}}
\author{Youssef Arbach, David Karcher, Kirstin Peters, Uwe Nestmann}
\institute{Technische Universit\"at Berlin, Germany\\
\texttt{\{youssef.arbach, david.s.karcher, kirstin.peters, uwe.nestmann\}@tu-berlin.de}}
\maketitle

\begin{abstract}
	In \cite{dynamicCausality15} we present an extension of Prime Event Structures by a mechanism to express dynamicity in the causal relation. More precisely we add the possibility that the occurrence of an event can add or remove causal dependencies between events and analyse the expressive power of the resulting Event Structures \wrt to some well-known Event Structures from the literature. This technical report contains some additional information and the missing proofs of \cite{dynamicCausality15}.
\end{abstract}

\setcounter{definition}{19}
\setcounter{lemma}{1}
\setcounter{theorem}{10}

\begin{figure}[t]
	\centering
	\begin{tikzpicture}[bend angle=30]
\event{b2}{0.3}{1.0}{left}{};
		\event{b3}{1.3}{1.0}{right}{};
		\event{b4}{0.8}{0}{left}{};
\draw[enablingPES] (b2) edge (b4);
		\draw[enablingPES] (b3) edge (b4);
		\draw[thick] (0.55, 0.5) -- (1.05, 0.5);
		\draw[conflictPES] (b2) edge (b3);
		\node (g) at (-0.5, 1) {};
		
\event{a1}{3}{1}{left}{};
		\event{a2}{4}{1}{right}{};
		\event{a3}{3}{0}{left}{};
		\node (a) at (2.2, 1) {};
		\draw[enablingPES] (a1) edge (a2);
		\draw[dropping]	(3.4, 1) -- (a3);
		
\event{b1}{5.6}{0.5}{above}{};
		\event{b2}{6.6}{0.5}{above}{};
		\node (b) at (5, 1) {};
		\draw[conflictEBES] (b1) edge (b2);
		
\event{ba}{8.3}{1}{above right}{};
		\event{bb}{8.3}{0}{below left}{};
\node (gs) at (7.5, 1) {};
		\draw[enablingAbsent] (bb) .. controls (8.8, -0.5) and (8.8, 0.5) .. (bb);
		\draw[adding] (ba) .. controls (8.6, 1) .. (8.6, 0.3);
		
\event{ca}{9.8}{0}{below left}{};
		\event{cb}{10.8}{0}{below right}{};
		\event{cc}{10.3}{1}{above}{};
\node (gx) at (9.4, 1) {};
		\draw[enablingAbsent] (ca) -- (cb);
		\draw[adding] (cc) -- (10.3, 0);
	\end{tikzpicture}
	\caption{Counterexamples.}
	\label{fig:counterExamples}
\end{figure}

\section{Event Structures for Resolvable Conflict}

For a transition based ES with a few additional properties, there is a natural
embedding into RCESs.
\begin{definition}\label{def:translationIntoRCES}
	Let  be an ES with a transition relation  defined on configurations such that
 implies  and
 implies that 
for all configurations  of .	Then .
\end{definition}
Note that the SESs and GESs satisfy these properties. We show that the resulting
structure  is indeed an RCES and that it is transition equivalent to .
\begin{lemma}	\label{lma:TransInRCES}
	Let  be an ES that satisfies the conditions of \defi\ref{def:translationIntoRCES}.	Then  is a RCES and .
\end{lemma}
\begin{proof}
	By Def. 9 in \cite{dynamicCausality15},  is a RCES.
	
	Assume . Then, by \defi\ref{def:translationIntoRCES},  and  for all .Then, by Def. 10 in \cite{dynamicCausality15}, .
	
	Assume . Then, by Def. 10 in \cite{dynamicCausality15},  and there is some  such that . By \defi\ref{def:translationIntoRCES} for , then  for all . So there is a set  such that  and  for each . Then, by \defi\ref{def:translationIntoRCES} for , it follows  and . Finally, by the second property of \defi\ref{def:translationIntoRCES}, .
\end{proof}

\section{Shrinking Causality}

In SESs both notions of configurations, traced-based and transition-based,
coincide; and in different situations, the more suitable one can be used.

\begin{lemma}\label{lem:SESconf}
	Let  be a SES. Then .
\end{lemma}

\begin{proof}
	Let .	By Def. 13 in \cite{dynamicCausality15},  implies that there is some  such that , , , and .	Hence, by Def. 13,  for all  and .	Thus, by Def. 13, .
	
	By Def. 13,  implies that there are  such that  and .
	Then, by Def. 13, we have:
	
	Let  and  for all .	Then, by Def. 13,  is a trace such that  (because of \eqref{eq:C1}),  for all  (because of \eqref{eq:C1} and \eqref{eq:C2}), for all  and all  we have  (because of \eqref{eq:C3} and \eqref{eq:C4}), and  (because ).
	Thus .
\end{proof}

Moreover the following technical Lemma relates transitions and the extension of traces by causally independent events.

\begin{lemma}
	\label{lem:SEStransTraces}
	Let  be a SES and .	Then  iff there are  such that , , and .
\end{lemma}

\begin{proof}
	By Def. 13 in \cite{dynamicCausality15} and \lem\ref{lem:SESconf},  implies that there is a trace  such that .

	If  then, by Def. 13, , , and . Then, by Def. 13,  and  for an arbitrary linearization  of the events in , \ie with  such that  whenever  and .
	
	If there is a trace  such that  and  then . Moreover, by Def. 13,  implies . Thus, by Def. 13, .
\end{proof}

Note that the condition  states that the events in  are causally independent from each other.

As mentioned above DESs and SESs have the same expressive power. To show this fact we define mutual encodings and show that they result into structures with equivalent behaviors. To translate a SES into a DES we create a bundle for each initial causal dependence and add all its droppers to the the bundle set.

\begin{definition}
	\label{def:SESintoDES}
	Let  be a SES.
	Then , where  iff , , and .
\end{definition}

The above translation from SES into DES shows that for each SES there is a DES with exactly the same traces and configurations.

\begin{lemma}
	\label{lem:SEStoDES}
	For each SES  there is a DES , namely , such that  and .
\end{lemma}

\begin{proof}[Proof  of Lemma~\ref{lem:SEStoDES}]
	Let  be a SES. By \defs12 and 1 in \cite{dynamicCausality15},  is irreflexive and symmetric. Hence, by \defs{7 in \cite{dynamicCausality15}} and \ref{def:SESintoDES},  is a DES.
	
	Let .
	By Def. 13 in \cite{dynamicCausality15},  iff , , and  for all .
	Since  and , we have  iff  for all .
	By \defi\ref{def:SESintoDES}, then  iff , , and  for all  and all .
	Hence, by the definition of traces in \S~2.2 in \cite{dynamicCausality15},  iff , \ie .
	
	By \lem\ref{lem:SESconf}, \S~2.2, and Def. 13, then also .
\end{proof}

The most discriminating behavioral semantics of DESs used in literature are
families of posets. Thus the translation should also preserve posets.

\begin{theorem}
	\label{thm:SESintoDES}
	For each SES  there is a DES , such that .
\end{theorem}

\begin{proof}
	Let  be a SES.\\
	By \lem\ref{lem:SEStoDES},  is a DES such that  and .\\
	Let , , and the bundles  all bundles pointing to . For  to be a cause for  \defi13 in \cite{dynamicCausality15} requires . Since  and , this condition holds iff the condition  holds for all .
	By \defi\ref{def:SESintoDES}, then .
	So, by \defs8 and 13 in \cite{dynamicCausality15}, .
\end{proof}

In the opposite direction we map each DES into a set of similar SESs such that each SES in this set has the same behavior as the DES. Therefore for each bundle  we choose a fresh event  as initial cause , make it impossible by a self-loop , and add all events  of the bundle  as droppers \drops{x_i}{d}{e}.

\begin{definition}
	\label{def:DESintoSES}
	Let  be a DES,  an enumeration of its bundles, and  a set of fresh events, \ie .
	Then  with , , and .
\end{definition}

Of course it can be criticized that the translation adds events (although they are fresh and impossible). But as the following example---with more bundles than events---shows it is not always possible to translate a DES into a SES without additional impossible events.

\begin{lemma}
	\label{lem:DESninSES}
	There are DESs , as \eg  with ,
	that cannot be translated into a SES  such that .
\end{lemma}

\begin{proof}[Proof of Lemma~\ref{lem:DESninSES}]
	Assume a SES  such that   and .
	According to \S~2.2 in \cite{dynamicCausality15},  contains all
	sequences of distinct events of  such that  is not the first, second, or third event, \ie for  to occur in a trace it has to be preceded by at least three of the other events.
	Since by Def. 13 in \cite{dynamicCausality15} conflicts cannot be dropped,  implies .
	Moreover, since  has to be preceded by at least three other events that can occur in any order,  has to contain at least three initial causes for . \WLOG let , , and .
	Because of the traces , we need the droppers \drops{b}{d}{e} and \drops{c}{d}{e}. Then  but .
	In fact if we fix  there only finitely many different SESs  and for none of them  holds.
\end{proof}

Note that the above lemma implies that no translation of the above DES can result into a SES with the same events such that the DES and its translation have same configurations or posets.
However, because the  are fresh, there are no droppers for the self-loops  in . So the translation ensures that all events in  remain impossible forever in the resulting SES. In fact we show again that the DES and its translation have the exactly same traces and configurations.

\begin{lemma}
	\label{lem:DEStoSES}
	For each DES  there is a SES , namely , such that  and .
\end{lemma}

\begin{proof}[Proof of Lemma~\ref{lem:DEStoSES}]
	Let  be a DES. By Def. 7 in \cite{dynamicCausality15},  is irreflexive and symmetric. Hence, by \defs12, 1 in \cite{dynamicCausality15}, and \ref{def:DESintoSES},  is a SES.	
	
	Let .	Then, by Def. 13 in \cite{dynamicCausality15},  iff , , and  for all . Note that we have  instead of , because all events in  have to be distinct and for all events in  there is an initial self-loop but no dropper.
	Since  and , we have   iff  for all .
	By \defi\ref{def:DESintoSES}, then  iff , , and  for all  and all .
	Hence, by the definition of traces in \S~2.2 in \cite{dynamicCausality15},  iff , \ie .
	
	By \lem\ref{lem:SESconf}, the definition of
	configurations in \S~2.2, and Def. 13, then
	also .
\end{proof}

Moreover the DES and its translation have exactly the same posets.

\begin{theorem}
	\label{thm:DESintoSES}
	For each DES  there is a SES , such that .
\end{theorem}

\begin{proof}[Proof of Theorem~\ref{thm:DESintoSES}]
	Let  be a DES. By \lem\ref{lem:DEStoSES},  is a SES such that  and .\\
	Let , , and the bundles  all bundles pointing to .
	For  to be a cause for  \defi13 in \cite{dynamicCausality15} requires .
	Since  and , this condition holds iff the condition  holds for all .
	By \defi\ref{def:DESintoSES}, then  iff .
	So, by \defs8 and 13 in \cite{dynamicCausality15}, .
\end{proof}

Thus SESs and DESs have the same expressive power.

\begin{proof}[Proof of Theorem~1 in \cite{dynamicCausality15}]
	By \ths\ref{thm:SESintoDES} and \ref{thm:DESintoSES}.
\end{proof}

\cite{Langerak97causalambiguity} proves that for DESs equivalence \wrt posets
based on early causality coincides with trace equivalence. Since SESs are as expressive as DESs \wrt families of posets based on early causality, the same correspondence holds for SESs.

\begin{corollary}
	\label{col:SESequiv}
	Let  be two SES. Then  iff .
\end{corollary}

Then Theorem~2 in \cite{dynamicCausality15} states:
\begin{quote}
	Let  be two SES.\\
	Then  iff  iff .
\end{quote}

\begin{proof}[Proof of Theorem~2 in \cite{dynamicCausality15}]
	By \cor\ref{col:SESequiv},  iff .
	
	If  then, by \lem\ref{lem:SESconf} and \defs13 in \cite{dynamicCausality15},  and . Hence assume .
	Note that, by Def. 13 and \lem\ref{lem:SESconf}, for all  there is a trace  such that . Moreover for every trace  except the empty trace there is a sub-trace  and a sequence of events  such that  and .
	Thus, by \lem\ref{lem:SEStransTraces},  iff .
\end{proof}

Theorem~3 in \cite{dynamicCausality15} states:
\begin{quote}
	SESs and EBESs are incomparable.
\end{quote}

\begin{proof}[Proof of Theorem~3 in \cite{dynamicCausality15}]
	\\
	Let  be the SES that is depicted in \fig\ref{fig:counterExamples}.
	Assume there is some EBES  such that .
	By Def. 13 in \cite{dynamicCausality15}, , \ie  cannot occur first.
	By \defi6 in \cite{dynamicCausality15}, a disabling  implies that  can never precedes .
	Thus we have , because within  each pair of events of  occur in any order.
	Similarly we have , because  implies that  always has to precede .
	Moreover, by \defi6, adding impossible events as causes
	or using them within the disabling relation does not influence the set of
	traces.
	Thus there is no EBES  with the same traces as . By
	\defi6 and the definition of posets in EBESs, then there is no EBES  with the same configurations or posets as .
	
	Let  be the EBES that is depicted in \fig\ref{fig:counterExamples}.
	Assume there is some SES  such that .
	According to \S~2.2 in \cite{dynamicCausality15}, .
	By Def. 13 and because of the traces  and , there are no initial causes for  and f, \ie .
	Moreover, , because of the trace  and because conflicts cannot be dropped.
	Thus  but , \ie
	there is no SES  with the same traces as . Then by
	\defi13, there is no SES  with the same configurations or families of posets as .
\end{proof}


\begin{lemma}\label{lma:SESinRCES}
	For each SES  there is a RCES , such that .
\end{lemma}

\begin{proof}
	By \defi13 in \cite{dynamicCausality15},  implies   for all .
	
	Assume .
	Then, by Def. 13,  implies  and .
	Then  implies . Then  and , because of .
	By Def. 13, then .
	
	Thus  satisfies the conditions of
	\defi\ref{def:translationIntoRCES}. Then by \lem\ref{lma:TransInRCES},
	 is a RCES such that .
\end{proof}

\begin{lemma}
\label{lma:SESinRCESstrictly} 
	There is no transition-equivalent SES to the RCES , where .
\end{lemma}

\begin{proof}
	Assume a SES  such that . Then .
	By Def. 13 in \cite{dynamicCausality15} and \lem\ref{lem:SESconf} and because of
	the configuration , the
	events  and  cannot be in conflict with each other, \ie .
	Moreover, because of the configurations , there are no initial causes for  and , \ie .
	Note that the relation  cannot disable events.
	Thus we have  and .
	But then, by Def. 13, .
	Since  does not hold, this violates our
	assumption, \ie there is no SES which is transition equivalent to .
\end{proof}

Theorem~4 in \cite{dynamicCausality15} states:
\begin{quote}
	SESs are strictly less expressive than RCESs.
\end{quote}

\begin{proof}[Proof of Theorem~4 in \cite{dynamicCausality15}]
	By \lems\ref{lma:SESinRCESstrictly} and \ref{lma:SESinRCES}.
\end{proof}

\section{Alternative Partial Order Semantics in DES and SES}
\label{app:partialOrderSemantics}

To show that DES and SES are not only behavioral equivalent ES models but are also very closely related at the structural level we consider the remaining four intentional partial order semantics for DES of \cite{Langerak97causalambiguity}.

Liberal causality is the least restrictive notion of causality in \cite{Langerak97causalambiguity}. Here each set of events from bundles pointing to an event  that satisfies all bundles pointing to  is a cause.

\begin{definition}[Liberal Causality]
	Let  be a DES,  one of its traces, , and  all bundles pointing to .
	A set  is a cause of  in  if
	\begin{compactitem}
		\item ,
		\item , and
		\item .
	\end{compactitem}
	Let  be the set of posets obtained this way for a trace .
\end{definition}

Bundle satisfaction causality is based on the idea that for an event  in a trace each bundle pointing to  is satisfies by exactly one event in a cause of .

\begin{definition}[Bundle Satisfaction Causality]
	Let  be a DES,  one of its traces, , and  all bundles pointing to .
	A set  is a cause of  in  if
	\begin{compactitem}
		\item  and
		\item there is a surjective mapping  such that  for all .
	\end{compactitem}
	Let  be the set of posets obtained this way for a trace .
\end{definition}

Minimal causality requires that there is no subset which is also a cause.

\begin{definition}[Minimal Causality]
	Let  be a DES and let  be  one of its traces, , and  all bundles pointing to .
	A set  is a cause of  in  if
	\begin{compactitem}
		\item ,
		\item , and
		\item there is no proper subset of  satisfying the previous two conditions.	
	\end{compactitem}
	Let  be the set of posets obtained this way for a trace .
\end{definition}

Late causality contains the latest causes of an event that form a minimal set.

\begin{definition}[Late Causality]
	Let  be a DES,  one of its traces, , and  all bundles pointing to .
	A set  is a cause of  in  if
	\begin{compactitem}
		\item ,
		\item ,
		\item there is no proper subset of  satisfying the previous two conditions, and
		\item  is the latest set satisfying the previous three conditions.	
	\end{compactitem}
	Let  be the set of posets obtained this way for a trace .
\end{definition}

As derived in \cite{Langerak97causalambiguity}, it holds that

for all traces .
Moreover a behavioral partial order semantics is defined and it is shown that two DESs have the same posets \wrt to the behavioral partial order semantics iff they have the same posets \wrt to the early partial order semantics iff they have the same traces.

Bundle satisfaction causality is---as the name suggests---closely related to the
existence of bundles. In SESs there are no bundles. Of course, as shown by the translation  in \defi\ref{def:SESintoDES}, we can transform the initial and dropped causes of an event into a bundle. And of course if we do so an SES  and its translation  have exactly the same families of posets. But, because bundles are no native concept of SESs, we cannot directly map the definition of posets \wrt to bundle satisfaction to SESs.

To adapt the definitions of posets in the other three cases we have to replace
the condition  by  and replace the condition  by  (as in \defi13 in \cite{dynamicCausality15}). The remaining conditions remain the same with respect to traces as defined in Def. 13.
Let , , and  denote the sets of posets obtained this way for a trace  of a SES  \wrt liberal, minimal, and late causality. Moreover, let  and  for all .

Since again the definitions of posets in DESs and SESs are very similar the translations  and  preserve families of posets. The proof is very similar to the proofs of \ths\ref{thm:SESintoDES} and \ref{thm:DESintoSES}.

\begin{theorem}
	For each SES  there is a DES , namely , and for each DES  there is a SES , namely , such that  for all .
\end{theorem}

\begin{proof}
	The definitions of posets in DESs and SESs \wrt to minimal and late causality
	differ in exactly the same condition and its replacement as the definitions of
	posets in DESs and SESs \wrt early causality. Thus the proof in these two cases is similar to the proofs of \ths\ref{thm:SESintoDES} and \ref{thm:DESintoSES}.
	
	If  is a SES then, by \lem\ref{lem:SEStoDES},  is a DES such that  and .
	If  is a DES then, by \lem\ref{lem:DEStoSES},  is a DES such that  and .
	In both cases let , , and  be all bundles pointing to .
	
	In the case of liberal causality, for  to be a cause for  the
	definition of posets in SESs requires  and .
	The second condition holds iff  as shown in the proofs of \ths\ref{thm:SESintoDES} and \ref{thm:DESintoSES}.
	By \defs\ref{def:SESintoDES} and \ref{def:DESintoSES}, the first conditions holds iff .
	So, by the definitions of posets in DESs and SESs \wrt to liberal causality, .
\end{proof}

\section{Growing Causality}
As in SESs, both notions of configurations of GESs, traced-based and
transition-based; coincide and in different situations, the more suitable one
can be used.
\begin{lemma}\label{lma:GESConfigEquivalence}
	Let  be a GES. Then .
\end{lemma}
\begin{proof}
	Let .
	
	By \defi15 in \cite{dynamicCausality15},  implies that there is some  such that , , , and . Hence, by \defi15,  for all  and .	Thus, by \defi15, .
	
	By \defi15,  implies that there are  such that  and .
	Then, by \defi15, we have:
	
	Let  and  for all .	Then, by \defi15,  is a trace such that  (because of \eqref{eq:D1}),  for all  (because of \eqref{eq:D1} and \eqref{eq:D2}), for all  and all  we have  (because of \eqref{eq:D3}, \eqref{eq:D4}, and, by \eqref{eq:D5}, ), and  (because ).
	Thus .
\end{proof}

For the incomparability result between GESs and EBESs we consider two
counterexamples, and show that there is no equivalent EBES or GES respectively.
\begin{lemma}\label{lma:EBESninGES}
	There is no configuration-equivalent GES to  (\cf \fig\ref{fig:counterExamples}).
\end{lemma}

\begin{proof}
	Assume a GES  such that .	According to \S~2.2 in \cite{dynamicCausality15}, . Because , , and by \defi15 in \cite{dynamicCausality15} and \lem\ref{lma:GESConfigEquivalence},  has to be an initial cause of  in , \ie .	But then, by \defi15 and \lem\ref{lma:GESConfigEquivalence},  although . This violates our assumption, \ie no GES can be configuration equivalent to .
\end{proof}
\begin{lemma}\label{lma:GESninEBES}
	There is no trace-equivalent EBES to  (\cf \fig\ref{fig:counterExamples}).
\end{lemma}
\begin{proof}
	Assume a EBES  such that . By \defi15 in \cite{dynamicCausality15},  and . Because of  and by \defi6 in \cite{dynamicCausality15},  and  have to be initially enabled in , \ie .	Moreover, because of ,  cannot disable , \ie . But then . This violates our assumption, \ie there is no trace-equivalent EBES to .
\end{proof}
Theorem~5 in \cite{dynamicCausality15} states:
\begin{quote}
	GESs are incomparable to BESs and EBESs.
\end{quote}

\begin{proof}[Proof of Theorem~5 in \cite{dynamicCausality15}]
	By \lem\ref{lma:EBESninGES}, there is no GES that is configuration equivalent
	to the BES . Thus no GES can have the same families of posets
	as the BES , because two BES with different configurations
	cannot have the same families of posets (\cf \S~2.2 in \cite{dynamicCausality15}). Moreover, by \defs3 and 5 in \cite{dynamicCausality15}, each BES is also an EBES. Thus no GES can have the same families of posets as the EBES .
	
	By \lem\ref{lma:GESninEBES}, there is no EBES and thus also no BES that is trace-equivalent to the GES . By \defi15 in \cite{dynamicCausality15}, two GES with different traces cannot have the same transition graphs. Thus no EBES or BES can be transition-equivalent to .
\end{proof}

For the incomparability between GESs and SESs, we study a GES counterexample, such that no SES is trace-equivalent.
\begin{lemma}\label{lma:GESninSES}
There is no trace-equivalent SES to  (\cf \fig\ref{fig:counterExamples}).
\end{lemma}
\begin{proof}
	Assume a SES  such that .	By \defi15 in \cite{dynamicCausality15}, .	Because of the trace  and by Def. 13 in \cite{dynamicCausality15},  and  cannot be in conflict, \ie  and .	Moreover, because of the traces , there are no initial cases for  or , \ie .	Thus, by Def. 13,  but .	This violates our assumption, \ie no SES can be trace equivalent to .
\end{proof}
Theorem~6 in \cite{dynamicCausality15} states:
\begin{quote}
	GESs and SESs are incomparable.
\end{quote}
\begin{proof}[Theorem~6 in \cite{dynamicCausality15}]
	By \lem\ref{lma:GESninSES}, no SES is trace-equivalent to the GES . By \defi15 in \cite{dynamicCausality15}, two GES with different traces cannot have the same transition graphs. Thus no SES is transition-equivalent to the GES .
	
	By \cite{Langerak:Thesis}, BESs are less expressive than EBESs and by \cite{Langerak97causalambiguity}, BESs are less expressive than DESs.	By \theo5 in \cite{dynamicCausality15}, BESs and GESs are incomparable an by \theo1 in \cite{dynamicCausality15} DESs are as expressive as SESs.	Thus GESs and SESs are incomparable.
\end{proof}
To show that GESs are strictly less expressive than RCESs, we give a translation
for one direction and a counterexample for the other.
\begin{lemma}\label{lma:GESinRCES}
	For each GES  there is an RCES , such that .
\end{lemma}
\begin{proof}
	Let .	By \defi15 in \cite{dynamicCausality15},  implies .
	
	Assume  and .
	By \defi15, then we have ,	, and .	Moreover, because ,  for all .	Hence  and .	Thus, by \defi15, .
	
	By \lem\ref{lma:TransInRCES},  is an RCES and .
\end{proof}
\begin{lemma}\label{lma:GESinRCESstrictly}
	There is no transition-equivalent GES to  (\cf \fig3 in \cite{dynamicCausality15}).
\end{lemma}
\begin{proof}
	Assume a GES  such that . Then .	By \defi15 in \cite{dynamicCausality15} and because of the configuration , the events , , and  cannot be in conflict with each other, \ie .	Moreover, because of the configurations , there are no initial causes for , , or , \ie . Finally, because of the configurations , neither  nor  can add a cause (except of themselves) to , \ie  and  for all .	Thus we have  and .	But then, by \defi15, . Since , this violates our assumption, \ie there is no GES that is transition equivalent to .
\end{proof}
Theorem~7 in \cite{dynamicCausality15} states:
\begin{quote}
	GESs are strictly less expressive than RCESs.
\end{quote}
\begin{proof}[Proof of Theorem~7 in \cite{dynamicCausality15}]
	By \lems\ref{lma:GESinRCES} and \ref{lma:GESinRCESstrictly}.
\end{proof}
\section{Dynamic Causality}
In order to justify our approach of state transition equivalence, we need a notion of (configuration) transition equivalence, and show that the new equivalence is needed.
\begin{definition}
\label{def:DCESConfigs}
Let  be a DCES. The set of its (reachable) configurations is
; the projection on the first component of the states.
\end{definition}
Lemma~1 in \cite{dynamicCausality15} states:
\begin{quote}
	There are DCESs that are transition equivalent but not state transition equivalent.
\end{quote}
\begin{proof}[Proof of Lemma~1 in \cite{dynamicCausality15}]
We consider two DCESs
	 and
	.
In  there is a transition 
by \defi18 in \cite{dynamicCausality15}. Initially the causality function is the
constant empty set function (\cf definition of ) in
\defi17 in \cite{dynamicCausality15}). After  occurs
 is updated according to
\defi18 Condition~6. Next we have
, where 
 according to
Condition~4. Now there is a possible transition to
. But if we
proceed from  to  with
 according to
Condition~4 and then to  with
 according to Condition~6 there is no transition to , because  needs  according to Condition~3.
In   both sequences of state transitions are possible by \defi18. Thus  and  are not state transition equivalent.

On the other hand, if we only consider the configurations of  and  saying there is a transition from  to  whenever , then  and  are transition equivalent.
\end{proof}

To compare DCESs to other ESs we define the Single State Dynamic Causality ESs
(SSDCs) as a subclass of DCESs.

\begin{definition}
\label{def:SingleStateDC}
Let SSDC be a subclass of DCESs such that  is a SSDC iff
.
\end{definition}
Since there are no adders and droppers for the same causal dependency, the order
of modifiers does not matter and thus there are no two different states sharing
the same configuration, \ie each configuration represents a state. Thus it is
enough for SSDC to consider transition equivalence with respect to
configurations, \ie .

\begin{lemma}
\label{lma:SingleCausalState}
Let  be a SSDC. Then for the causal-state function  of any
state  it holds .
\end{lemma}

\begin{proof}
If  the equation follows directly from the definitions of , , , and .

Assume . By induction, we have .
We prove for each  by a doubled case distinction . Let us first assume  but , then by Condition~6 in \cite{dynamicCausality15} we have  and since  we have , because  is a SSDC it follows  and because  it follows . Then in this case  holds. Let now still  but , then we have by contra-position of Condition~7 we have , and so  . Let us now consider the case  and here first . Then by Condition~5 it follows . Then  and because  it follows  and so . In the last case we consider  and  . By Condition~4 we have  and so . Thus . So in each case  holds.
\end{proof}


In SSDC Conditions~4, 5, 6, and 7 in \cite{dynamicCausality15} hold whenever .

\begin{lemma}
\label{lma:SSDCstateProp}
Let  be a SSDC and let  and  be two states of  with , then Conditions~4, 5, 6, and 7 in \cite{dynamicCausality15} of  hold for those two states.
\end{lemma}

\begin{proof}
Let  with .\\
Since ,  and , we have  and  and by the previous \lem\ref{lma:SingleCausalState} we have , so  and . Then  and so , thus  and so , but , which yields , so Condition~4 in \cite{dynamicCausality15} holds. Let now  with , then  so  follows, which is exactly 5. 

Conditions~6 and 7 are proven similarly.
\end{proof}

\begin{lemma}
\label{lma:SSDCConfInTrans}
Let  be a SSDC and  a transition in . Then for all  with , there is a transition  in , where  and .
\end{lemma}

\begin{proof}
By assumption Conditions~1 and 2 of \defi18 in \cite{dynamicCausality15} of the
transition relation holds for the two states  and
. Conditions~4,
5, 6, and
7 follow from \lem\ref{lma:SSDCstateProp}.
Condition~8 holds because of \defi\ref{def:SingleStateDC} and  is a SSDC. Condition~9 holds because it is a special case of the same conditions for  and . Let now , such that , then there is  and a  with , but this is a contradiction with Condition~9 of , so Condition~3 holds. Thus  holds.
\end{proof}

\begin{definition}
\label{def:InclusionIntoDCES}
Let  be a SES. Then its embedding is . Similarly let  be a GES. Then its embedding is .
\end{definition}

For each embedding the causal state coincides with a condition on the initial, added, and dropped causes, that are enforced in the transition relations of SESs and GESs.

\begin{lemma}
\label{lma:explicitSEScs}
Let  be a SES and  its embedding. Then we have for each state  of , .
\end{lemma}

\begin{proof}
By \lem\ref{lma:SingleCausalState} and because  in  for all configurations  and events .
\end{proof}

\begin{lemma}
\label{lma:explicitGEScs}
Let  be a GES and  its embedding. Then we have for each state  of ,  .
\end{lemma}

\begin{proof}
By \lem\ref{lma:SingleCausalState} and because  in  for all configurations  and events .
\end{proof}
SESs (resp. GESs) and their embeddings are transition equivalent.
\begin{lemma}
\label{lma:SESinDCES}
\label{lma:GESinDCES}
Let  be a GES or SES, then we have .
\end{lemma}



\begin{proof}
Let  be a SES and  a transition in , we define for a configuration  a causality state function  as . 
 is conflict free and , because  and Def. 13 in \cite{dynamicCausality15}, so Conditions~
1 and 2 of \defi18 in \cite{dynamicCausality15} are satisfied. Moreover in the configuration  we have , so Conditions~3, 4, and 5 are fulfilled. Conditions~6, 8, 7, and 9 are trivially satisfied, because , so .
Let now  in , then by \defs18, 13 in combination with \lem\ref{lma:explicitSEScs} there is a transition  in .\\
Let now  be a GES and  a transition in , we define for a configuration  a causality state function  as .  is conflict free and , because  and \defi15 in \cite{dynamicCausality15}, so Conditions~2, 1,  and 9 of \defi18 are satisfied. Moreover in the configuration  we have , so Conditions~3, 6, and 7 are fulfilled. Conditions~4, 5, and 8 are trivially satisfied, because .
Let now  in , then by \defs18, 15 in combination with \lem\ref{lma:explicitGEScs} there is a transition  in . 
\end{proof}

For the incomparability result between DCESs and SESs, we give an RCES
counterexample, which cannot be modeled by a DCES.

\begin{lemma}\label{lma:RCESNotinDCES}
There is no transition-equivalent DCES to  (\cf \fig\ref{fig:counterExamples}).
\end{lemma}

\begin{proof}[Proof of Lemma~\ref{lma:RCESNotinDCES}]
Assume  such that . Then .
	By \defi18 in \cite{dynamicCausality15} and because of the configuration , the events , , and  cannot be in conflict with each other, \ie .
	Moreover, because of the configurations , there are no initial causes for , , or , \ie . Note that the relation  cannot disable events.
	Finally, because of the configurations , neither  nor  can add a cause (except of themselves) to , \ie  and  for all .
	Thus we have  and in the state  it follows .
	But then, by \defi18,  for some causal state functions  and .
	Since , this violates our assumption, \ie there is no DCES that is transition equivalent to .
\end{proof}

Theorem~8 in \cite{dynamicCausality15} states:
\begin{quote}
DCESs and RCESs are incomparable.
\end{quote}

\begin{proof}[Proof of Theorem~8 in \cite{dynamicCausality15}]
It follows from \lems\ref{lma:RCESNotinDCES} and 1 in \cite{dynamicCausality15}, and because  (for  and  as in the proof of \lem1), then no two RCESs  and , with  can distinguish between  and .
\end{proof}

Theorem~9 in \cite{dynamicCausality15} states:
\begin{quote}
DCESs are strictly more expressive than GESs and SESs.
\end{quote}

\begin{proof}[Proof of Theorem~9 in \cite{dynamicCausality15}]
By \ths8, 7, and 4 in \cite{dynamicCausality15} and \lem\ref{lma:GESinDCES}.
\end{proof}


\section{Comparing DCESs with EBESs}
\label{sec:EBESsWithDCESs}
To compare with EBESs, we define a sub-class of DCESs, where posets could be
defined and used for semantics.

\begin{definition}
\label{def:EBDC}
Let EBDC denotes a subclass of SSDC with the additional requirements:
	\begin{inparaenum}
		\item \label{eq:EBDCOnlyDisabling} 
		\item \label{eq:NoCausalAmbiguity} 
	\end{inparaenum}
\end{definition}

The first condition translates disabling into  and ensures that
disabled events cannot be enabled again. The second condition reflects causal
unambiguity by  such that either the initial
cause or one of its droppers can happen.

We adapt the notion of precedence.

\begin{definition}
\label{def:EBDCLposets}
Let  be a EBDC and , then we define the
precedence relation  as . Let  be the reflexive and
transitive closure of .
\end{definition}

The relation  indeed represents a precedence relation, and its
reflexive transitive closure is a partial order.

\begin{lemma}
\label{lma:EBESPrecedence}
Let  be a EBDC, , and let . Let also  with  and  be the
transition sequence of ,
then .
\end{lemma}

\begin{proof}
Let  be the first occurrence of  in the sequence , so according to
Condition~1 of \defi18 in \cite{dynamicCausality15} it is enough to prove that
.
First, assume that , then 
according to the definition of . Then
according to \defi18 the only situation where  is that
there is a dropper  for it according to
Condition~4, but that is impossible
since  and  will be in conflict according to
Condition~\ref{eq:NoCausalAmbiguity} of \defi\ref{def:EBDC}. So  and thus  according to Condition~3 of \defi18.

Second assume that .
If  then according to
Condition~7 of \defi18  which
means  according to
Condition~3 of \defi18, which is a contradiction to the definition of . 
Then according to Condition~9 of \defi18, if , it follows , which again contradicts the definition of . So because , there is an , such that  but .

Third, assume . Then since EBDC are a subclass of SSDC we have
 according to \defi\ref{def:SingleStateDC}. Then  according to
Condition~1 of \defi16 in \cite{dynamicCausality15}, which means  according to definition of  in \defi18.
Let us assume that  then either  or another dropper  occurred before , which is impossible because of the
mutual conflict in Condition~\ref{eq:NoCausalAmbiguity} of \defi\ref{def:EBDC}. So .
\end{proof}

\begin{lemma}
	\label{lma:EBESPrecIsOrder}
	 is a partial order over . 
\end{lemma}

\begin{proof}
Let  and let  be the transition sequence of . Let also  be the configurations
where  first occur, respectively, then according to \lem\ref{lma:EBESPrecedence}, . Since  is the reflexive and transitive
closure of , then . For
anti-symmetry, assume that  also then according to
\lem\ref{lma:EBESPrecedence}: , but , then . The only
possibility for  is that  because otherwise  and ,
which is a contradiction.
\end{proof}

Let 
denotes the set of posets of the EBDC . We show that the
transitions of a EBDC  can be extracted from its posets.

\begin{theorem}
\label{thm:EBDCTrFromPosets}
Let  be a EBDC and 
with .\\
Then
 holds iff .
\end{theorem}

\begin{proof}
First, because  is a configuration it is conflict free. Now let us assume
, we
now show that all the conditions of \defi18 in \cite{dynamicCausality15} hold
for  and . Condition~1,
, holds by assumption. Conditions~4,
5, 6, and
7 follow immediately from \lem\ref{lma:SSDCstateProp}. Condition~8 follows from 
\defi\ref{def:SingleStateDC} of SSDC, since  is an EBDC 
which is a subclass of SSDC. To prove Condition~3, let
, then we have from \lem\ref{lma:SingleCausalState}
. 
Assume , \ie . So either  or . We can ignore the case that , because in EBDC the added
causality for  can only be , which would make 
impossible, but this cannot be the case since .
So let us consider the remaining option: . Then  by the definition of . Then by assumption,  and
therefore , which is a contradiction. Then . For Condition~9 
we show . 
The only growing causality is of the form  and according to 
\defi\ref{def:EBDCLposets},  means ,
then .

Let us now assume , and  with  and , so by \lem\ref{lma:EBESPrecedence} it follows .
\end{proof}

The following defines a translation from an EBESs into an EBDC, which
is proved in \lem\ref{lma:EmdeddingIsEBDC} to be an EBDC. Furthermore this
translation preserves posets.
Figure~\ref{fig:exampleEBES2DCES} provides an example, where conflicts with
impossible events are dropped for simplicity.

\begin{definition}
\label{def:EBES2DCES}
Let  be an EBES. Then 
  such that:
\begin{inparaenum}
	\item  are
defined as in \ref{def:DESintoSES}
	\item 
	\item \label{eq:EBES2DCESGrowing} .
\end{inparaenum}
\end{definition}

\begin{figure}[tb]
	\centering
	\begin{tikzpicture}
\event{a1}{-1}{1}{above}{};
		\event{a2}{-0.2}{1}{above}{};
		\event{a3}{-0.2}{0}{below left}{};
		\event{a5}{.6}{0}{right}{};
		\draw[enablingPES] (a1) edge (a3);
		\draw[enablingPES] (a2) edge (a3);
		\draw[enablingPES] (a3) .. controls (-0.9, 0.3) .. (a1);
		\draw[thick]	(-0.5, 0.37) -- (-0.2, 0.5);
		\draw[conflictPES] (a1) edge (a2);
		\draw[conflictEBES] (a3) edge (a5);
		\node (a) at (-1.7, 1) {(a)};
\event{b1}{3}{1}{above}{};
		\event{b2}{3.8}{1}{above}{};
		\event{b3}{4.3}{0.7}{below right}{};
		\event{b4}{3.8}{0}{below left}{};
		\event{b6}{5}{0}{below}{};
		\event{b5}{3}{0}{left}{};
		\draw[enablingPES] (b5) edge (b1);
		\draw[enablingPES] (b3) edge (b4);
		\draw[enablingPES] (b3) .. controls (4.8, 1.2) and (3.8, 1.2) .. (b3);
		\draw[enablingPES] (b4) .. controls (4.3, .3) and (4.3, -.5) .. (b4);
		\draw[enablingPES] (b5) .. controls (2.5, -.5) and (3.5, -.5) .. (b5);
		\draw[adding]		(b6) -- (4.2 , 0);
		\draw[conflictPES]  (b1) edge (b2);
		\draw[dropping]		(3.1, .4) -- (b4);
		\draw[dropping]		(4, .4) -- (b1);
		\draw[dropping]		(4, .4) -- (b2);
		\node (b) at (2.3, 1) {(b)};
	\end{tikzpicture}
	\caption{An EBES and its poset-equivalent DCES.}
	\label{fig:exampleEBES2DCES}
\end{figure}

\begin{lemma}
\label{lma:EmdeddingIsEBDC}
Let  be an EBES. Then  is an EBDC.
\end{lemma}

\begin{proof}
First  is a DCES. The definition of  in \defi\ref{def:DESintoSES} ensures Conditions~
16(1) and
16(2) in \cite{dynamicCausality15}. According to the definition of
 in \defi\ref{def:DESintoSES}, the only dropped causes are the
fresh events, which cannot be added by  according to
\defi\ref{def:EBES2DCES}(\ref{eq:EBES2DCESGrowing}). So Condition~16(3) also holds.

Second,  is a SSDC, since the only dropped events are the
fresh ones which are never added by , so \defi\ref{def:SingleStateDC} holds.

Third,
 is a EBDC. \defi\ref{def:EBDC}(\ref{eq:EBDCOnlyDisabling})
holds by definition. 
Bundle members in  mutually disable each other, then according to
the definition of  Condition~\ref{def:EBDC}(\ref{eq:NoCausalAmbiguity})
holds. Therefore  is a EBDC.
\end{proof}

Before comparing an EBES with its translation according to posets, we make use
of the following lemma.

\begin{lemma}
\label{thm:EBESandEBDCconfigEqui}
Let  be an EBES. Then .
\end{lemma}

\begin{proof}
First, . According to \S~2.2 in \cite{dynamicCausality15},
 means there is a trace  in 
such that . Let us prove that  corresponds to a transition
sequence in  leading to . \ie let us prove that there exists a
transition sequence  such that  for , and
 is defined according to \lem\ref{lma:SingleCausalState}. This means we have
to prove that  for .

 is conflict-free since it is a configuration in  which means that it does not contain any mutual
disabling according to trace definition in \S~2.2. Second, it is clear
that  by definition. Next, let us prove that
, \ie
 according
to \lem\ref{lma:SingleCausalState}.
 contains only fresh events according to the definition of ,
and the members of bundles  are droppers of these fresh events.
But since each of these bundles is satisfied, then each of these fresh events in
 is dropped. Furthermore, there cannot be added causality in 
for  except for  itself which makes it an impossible event, but it is not an impossible event since it occurs in a
configuration. Therefore  for
all  and all . On the other hand, conditions
4, 5, 6, and 7 of
Def. 18 in \cite{dynamicCausality15} hold according to \lem\ref{lma:SSDCstateProp}.
Condition 8 of \defi18 holds by \defi\ref{def:SingleStateDC}.
Since in the transition , only one
event --namely -- occurs, then \defi18(9) also holds.

In that way we proved that . In a similar way, and with the help of
\lem\ref{lma:SSDCConfInTrans}, we can prove that
 which means that .
\end{proof}

\begin{lemma}
\label{lma:EBESintoEBDC}
For each EBES  there is a DCES, namely , such that
.
\end{lemma}

\begin{proof}
First, . Let , then  by the definition of posets of EBESs.
Then according to \theo\ref{thm:EBESandEBDCconfigEqui}: . On the other hand, let  be the partial
order defined for  in  as in \defi\ref{def:EBDCLposets}. This
means that we should prove that . But since 
are the reflexive and transitive closures of  respectively, then it is
enough to prove that . In other words we have to prove
.

Let us start with . According to 
\S~2.2 in \cite{dynamicCausality15}  means
. If  then
 by the definition of  \defi\ref{def:EBES2DCES}. This means  according to the definition of  \defi\ref{def:EBDCLposets}.
If  then  since otherwise  are in conflict. This means
 according to \defi\ref{def:EBES2DCES}, which means  according to \defi\ref{def:EBDCLposets}.

Let us consider the other direction: .
 means  according to the definition of  in \defi\ref{def:EBDCLposets}. The third option where  is rejected
since the only initial causes that exist in  are the fresh
impossible events. If  then  according to the definition of  in . This
means  by the definition of  in \S~2.2. If on the
other hand  then  according to the definition of
, which means  in \S~2.2.

In that way we have proved that , which means that . In a similar way we can prove that , which means .
\end{proof}


\begin{lemma}
\label{lma:DCESninEBES}
There is a DCES such no EBES with the same configurations exits.
\end{lemma}
\begin{proof}
We consider the embedding  (\cf \fig\ref{fig:counterExamples}) of the SES , which models disjunctive causality. According to \defi13, because  and , it holds  and so . Further there is no transition , because , but there are transitions  and , because  ( resp.). The transitions are translated to the embedding according to \lem\ref{lma:SESinDCES} and \defi\ref{def:DCESConfigs} the same holds for the configurations. 

If we now assume there is a EBES  with the configurations  and  then according to \defi5 in \cite{dynamicCausality15} because there is no configuration  there must be a non-empty bundle  and caused by the the configurations  this bundle  must contain  and . Now the stability condition of \defi5 implies  and , so  and  are in mutual conflict contradicting to the assumption . Thus there is no EBES with the same configurations as 
\end{proof}

Theorem~10 in \cite{dynamicCausality15} states:
\begin{quote}
DCESs are strictly more expressive than EBESs.
\end{quote}

\begin{proof}[Proof of Theorem~10 in \cite{dynamicCausality15}]
Follows directly from \lems\ref{lma:DCESninEBES} and \ref{lma:EBESintoEBDC}.
\end{proof}

\bibliographystyle{plain}
\bibliography{dynamicCausality}

\end{document}
