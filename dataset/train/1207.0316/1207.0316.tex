\documentclass[11pt]{article}

\usepackage{amssymb}
\usepackage{epsfig}
\usepackage{amsmath}

 \usepackage{geometry}
 \geometry{a4paper, top=1in, bottom=1.5in, left=1in, right=1in}

 \newtheorem{theorem}{Theorem}[section]
 \newtheorem{lemma}{Lemma}[section]
 \newtheorem{corollary}{Corollary}[section]
 \newtheorem{claim}{Claim}
 \newtheorem{definition}{Definition}[section]
 \newtheorem{property}{Property}[section]

 \newcommand{\proof}{{\bf Proof:\ }}
 \newcommand{\pf}{\vspace{0.0em} \noindent {\bf Proof:\ }}
 \newcommand{\qed}{\vrule height4pt width4pt depth2pt}

\newcounter{algleo}
\newlength{\lefttab}
\newlength{\numberoffset}
\setlength{\numberoffset}{-1em}
\newenvironment{algleo}{\trivlist
   \topsep=0pt\parsep=0pt\itemsep=0pt
   \def\li{\item\refstepcounter{algleo}\makebox[0.8em][r]{\thealgleo\hspace{\numberoffset}}
       \hangafter1\hangindent1.8em\noindent}\def\linonumber{\item\makebox[0.8em][r]{\hspace{\numberoffset}}
       \hangafter1\hangindent1.8em\noindent}\addtolength{\lefttab}{1.25em}
   \addtolength{\numberoffset}{1.25em}
   \leftskip=\lefttab}{\endtrivlist}




\begin{document}


\title{Algorithmic Aspects of Homophyly of Networks}

\author{
Angsheng Li
\thanks{Institute of Software, Chinese Academy of Sciences, China.
  Email: \texttt{angsheng@ios.ac.cn}.
  The author is supported by the hundred talent program of the Chinese Academy of Sciences,
  and the grand challenge program, {\it Network Algorithms and Digital Information},
  Institute of Software, Chinese Academy of Sciences.}
\and
Peng Zhang
\thanks{Corresponding author.
  School of Computer Science and Technology, Shandong University, China
  Email: \texttt{algzhang@sdu.edu.cn}.
  The author is supported by the National Natural Science Foundation of China (60970003),
  the Special Foundation of Shandong Province Postdoctoral Innovation Project (200901010),
  and the Independent Innovation Foundation of Shandong University (2012TS072).}
}


\maketitle


\begin{abstract}
We investigate the algorithmic problems of the {\it homophyly
phenomenon} in networks. Given an undirected graph  and a vertex
coloring  of , we say that
a vertex  is {\it happy} if  shares the same color with all its
neighbors, and {\it unhappy}, otherwise, and that an edge  is
{\it happy}, if its two endpoints have the same color, and {\it unhappy},
otherwise. Supposing  is a {\it partial vertex coloring} of ,
we define the Maximum Happy Vertices problem (MHV, for
short) as to color all the remaining vertices such that the number
of happy vertices is maximized, and the Maximum Happy Edges problem
(MHE, for short) as to color all the remaining vertices such that
the number of happy edges is maximized.

Let  be the number of colors allowed in the problems.
We show that both MHV and MHE can be solved in polynomial time
if , and that both MHV and MHE are NP-hard if .
We devise a -approximation algorithm
for the MHV problem, where  is the maximum degree of vertices in
the input graph, and a -approximation algorithm for the MHE problem.
This is the first theoretical progress of these two natural and fundamental
new problems.
\end{abstract}




\section{Introduction}
\label{sec - introduction}
Networks or at least social networks heavily depend on human or
social behaviors. It is believed that {\em homophyly}~\cite[Chapter
4]{EK10} is one of the most basic notions governing the structure of
social networks. It is a common sense principle that people are more
likely to connect with people they like, as what says in the proverb
``birds of a feather flock together''.

Li and Peng in \cite{LP11,LP12} gave a mathematical definition of
{\it community}, and {\it small community phenomenon} of networks,
and showed that networks from some classic models do satisfy the
small community phenomenon. A. Li and J. Li et al. \cite{LLPP11} proposed
a homophyly model by introducing a color for every vertex in the classical
preferential attachment
model such that networks generated from this model satisfy
simultaneously the following properties: 1) power law degree
distribution, 2) small diameter property, 3) vertices of the same color
naturally form a small community, and 4) almost all vertices are
contained in some small communities, i.e., the small community
phenomenon of networks. This result implies the {\it homophyly law}
of networks that the mechanism of the small community phenomenon is
homophyly, and that vertices within a small community share
remarkable common features.

A. Li and J. Li et al. \cite{LLPP12} showed that many real networks
satisfy exactly the homophyly law, in which an interesting application is
the prediction and confirmation of keywords from a paper citation network
of high energy physics
theory\footnote{\texttt{http://snap.stanford.edu/data/cit-HepTh.html}.}.
The network contains  vertices (i.e., papers) and  edges
(i.e., citations). All the papers have titles and abstracts, but only 
papers have keywords listed by their authors.
We interpret the keywords of a paper to be a {\it function} of the paper.
By the homophyly law, vertices within a small community of the network must
share remarkable common features (keywords here).
The prediction is as follows: 1) to find a
small community from each vertex, if any, 2) to extract
the most popular  keywords from the known keywords in a community,
as the remarkable common features of this community,
3) to predict that (all or part of) the  remarkable common
keywords are keywords of a paper in the community, 4) to confirm a
prediction of keyword  for a paper , if  appears in either
the title or the abstract of paper . It is a surprising result
that this simple prediction confirms keywords for  papers in
the network. This experiment implies that real networks do satisfy
the homophyly law, and that the homophyly law is the principle for
prediction in networks.

The keywords can be viewed as the attributes of vertices in a network.
The above experimental result suggests a natural theoretical problem that,
given a network in which some vertices have their attributes unfixed,
how to assign attributes to these vertices such that the resulting
network reflects the homophyly law in the most degree?
Some attributes of a vertex cannot be changed, such as nationality, sex,
color and language, but some other attributes can be changed,
such as interest, job, income and working place.
For simplicity, we consider the case that each vertex contains only one
alterable attribute, i.e., the network is a -dimensional network.
Consider the following scenario. Suppose in a company there are many employees
which constitutes a friendship network.
Some employees have been assigned to work in some departments of the
company, while the remaining employees are waiting to be assigned.
An employee is {\em happy}, if s/he works in the same department with all of
(or  fraction of for some , or at least  for some
integer ) her/his friends; otherwise s/he is {\em unhappy}.
Similarly, a friendship is {\em happy} (or lucky) if the two
related friends work in the same department; otherwise the friendship is
{\em unhappy}. Our goal is to achieve the greatest social benefits, that is,
to maximize the number of {\em happy vertices} (similarly, {\em happy edges})
in the network.



We can easily express the above problems as graph coloring problems,
just identifying each attribute value with a different color.
Hence we get two specific graph coloring problems, as defined below.


\begin{definition}[The MHV problem]
(Instance) In the Maximum Happy Vertices (MHV) problem, we are given
an undirected graph , a color set ,
and a partial vertex coloring function . We say that 
is a partial function in the sense that  assigns colors to part of
vertices in .

(Query) A vertex is {\em happy} if it shares the same color with all its
neighbors, otherwise it is {\em unhappy}. The task is to extend  to a total
function  such that the number of happy vertices is maximized.
\end{definition}

\begin{definition}[The MHE problem]
(Instance) The input of the Maximum Happy Edges (MHE) problem is the same
as that of the MHV problem.

(Query) An edge is {\em happy} if its two endpoints have the same color,
otherwise it is {\em unhappy}. The goal is to extend  to a total function 
such that the number of happy edges is maximized.
\end{definition}

The vertex coloring defined by the total function 
in MHV and MHE is called a {\em total vertex coloring}.
In general, a (partial or total) vertex coloring can be denoted by
, where  is the set of all vertices having
color . A total vertex coloring is a partition of , while a partial
vertex coloring may not. Therefore, the MHV and MHE problems are two extension
problems from a partial vertex coloring to a total vertex coloring.
We remark that the coloring for our case is completely different from the
well-known Graph Coloring problem, which requires that the two endpoints of
an edge must be colored differently and asks to color a graph in such a way
by using the minimized number of colors. We use the notion of color just
for intuition.

If in the MHV problem the color number  is a constant, the problem is
denoted by -MHV. For the specific values of , we have the 2-MHV problem,
the 3-MHV problem, and so on. Note that in the original MHV problem  is
given as a part of the input. Similarly, we have the -MHE problem for
constant , with 2-MHE, 3-MHE, etc. being its specific problems.

We remark that both the MHV and MHE problems are natural and fundamental
algorithmic problems, and that they have not appeared yet in literature.
The reasons could be two folds. On the one hand, we ask the questions from
our network applications which did not happen before; on the other hand,
the meaning of coloring has been specified previously so that the two
endpoints of an edge must have different colors.
We notice that the current version of our problems may not really help
network applications much because of their simplicity.
For real network applications, probably the experimental method \cite{LLPP12}
introduced at the beginning of this section is fine enough.
However, this has no theoretical guarantee, owing to different structures
of networks. Our problems seem essentially new and fundamental algorithmic
problems. Theoretical analysis of the problems are always helpful to
understand the nature of the problems, and hence are very welcome.




\subsection{Our Results}
We investigate algorithms to solve the MHV and MHE problems.
It is easy to see that the partial function  plays an important role
in the MHV and MHE problems. If none of the vertices in the input graph has
a pre-specified color, then the MHV and MHE problems are trivial. The optimal
solution just assigns one arbitrary color to all the vertices. This will
make all vertices and all edges happy.

We prove that the MHV and MHE problems are NP-hard. Interestingly, the
complexity of -MHV and -MHE dramatically changes when  changes
from 2 to 3. Specifically, we prove that both 2-MHV and 2-MHE can be solved
in polynomial time, while both -MHV and -MHE are actually NP-hard for
any constant . We thus seek approximation algorithms for the MHV
and MHE problems, and their variants -MHV and -MHE ().

We design two approximation algorithms
{\sc Greedy-MHV} (Subsection \ref{subsec - greedy approxalg for MHV}) and
{\sc Growth-MHV} (Subsection \ref{subsec - subset-growth approxalg for MHV})
for the MHV problem and its variant -MHV.
Algorithm {\sc Greedy-MHV} is a simple greedy algorithm with approximation
ratio . Algorithm {\sc Growth-MHV} is an algorithm based on the
subset-growth technique with approximation ratio ,
where  is the maximum degree of vertices in the input graph.
In real networks,  is usually , implying that
the ratio  is reasonable.
As Algorithm {\sc Growth-MHV} is executing, more and more vertices are
colored. According to the current vertex coloring for the input graph,
we define several types for the vertices. (Note that the types here are not
colors.) Algorithm {\sc Growth-MHV} works based on carefully classifying
all the vertices into several types.

We can extend our algorithms for MHV to deal with two more natural variants
SoftMHV and HardMHV.
In the SoftMHV problem, a vertex  is happy if  shares the same color with
at least  neighbors, where  (that is, the soft threshold)
is a number in  and  is the degree of vertex .
In the HardMHV problem, a vertex  is happy if  shares the same color with
at least  neighbors, where  (that is, the hard threshold) is an integer.
We show that the SoftMHV and HardMHV problems can also be approximated within
.
The approximation algorithms for SoftMHV and HardMHV, given in the Appendix
for completeness, are similar to that for MHV.

For the MHE problem and its variant -MHE, we devise a simple
approximation algorithm based on a division strategy, namely,
Algorithm {\sc Division-MHE} (Section \ref{sec - algorithms for MHE}).
The approximation ratio is proved to be 1/2.




\subsection{Related Work and Relation to Other Problems}
\label{sec - related works}
The MHV and MHE problems are two quiet natural vertex classification problems
arising from the homophyly phenomenon in networks.
Classification is a fundamental problem and has wide applications in
statistics, pattern recognition, machine learning, and many other fields.
Given a set of objects to be classified and a set of colors, a classification
problem can be depicted as from a very high level assigning a color to
each object in a way that is consistent with some observed data or structure
that we have about the problem \cite{BFOS84,KT02}.
In our problems, the observed strucute is homophyly.
Since the MHV and MHE problems are essentially new, in the following we just
show some closely related problems and results.

Thomas Schelling \cite{S72,S78}, the Nobel economics prize winner, showed
by experiments how global patterns of spatial segregation arise from the
effect of homophyly operating at the local level.
The experiments in \cite{S72} are given in one-dimensional and two-dimensional
geometric models.
From a more general viewpoint of graph theory, Schelling's experiments,
although given in geometric models, can be viewed as how to remove and add
edges from/to a graph whose vertices are all colored by some colors
such that the resulting graph possesses the homophyly property.
In contrast, the MHV and MHE problems are how to color the vertices
in a given graph whose part of vertices are already colored such that
the resulting graph possesses the homophyly property.

The Multiway Cut problem \cite{EL92,DJP+94,CKR00,KKS+04} should be
the traditional optimization problem that is most related to MHV and MHE.
Given an undirected graph  with costs defined on edges and
a terminal set , the Multiway Cut problem asks for a set
of edges (called a {\em multiway cut}, or simply a {\em cut}) with the minimum
total cost such that its removal from graph  separates all terminals in 
from one another. The Multiway Cut problem in general graphs is NP-hard even
the terminal set contains only three terminals and each edge has a unit
cost \cite{DJP+94}. The current best approximation ratio known for this
problem is 1.3438 \cite{KKS+04}.

Removing a minimum multiway cut from a graph breaks the graph into several
components such that each component contains exactly one terminal.
From the viewpoint of graph coloring, this is equivalent to coloring the
uncolored vertices in a graph in which each terminal has a distinct
pre-specified color, such that the number of happy edges is maximized.
Therefore, the MHE problem is actually the dual of the Multiway Cut problem.
See Figure \ref{fig - multiway cut and vertex coloring} for an example.
(More precisely, the dual of Multiway Cut is only a special case of MHE,
since in MHE there may be more than one vertices having the same pre-specified
color.) However, Multiway Cut and MHE are quite different in terms of
approximation, since one is a maximization problem while the other is
a minimization problem.

\begin{figure}
\begin{center}
\includegraphics*[width=0.45\textwidth]{vertex-coloring.eps}
\end{center}
\caption{An instance of Multiway Cut and the induced vertex coloring.
The square vertices are terminals and have pre-specified colors,
while the round vertices are non-terminal vertices.
The hollow vertices are border vertices.}
\label{fig - multiway cut and vertex coloring}
\end{figure}

For a vertex subset  of graph , we define the {\em border}
of  to be the set of vertices in  that has a neighbor not in .
Given a vertex coloring  of graph ,
the vertices in the border of each  are obviously unhappy.
The MHV problem, which finds a vertex coloring that maximizes the number of
happy vertices, is actually equivalent to finding a vertex coloring
 for a graph in which some vertices are already
colored, such that the total number of vertices in borders of all 's
is minimized. Please refer to Figure \ref{fig - multiway cut and vertex coloring}
for an example. The latter problem we just introduce is a new minimization
problem; the MHV problem and this new problem are dual to each other.

From the above analysis, one can see that the partial function  in
the MHE problem (and the MHV problem), which assigns colors to part of
vertices of the input graph, actually simulates and generalizes
the {\em terminal set} part in the Multiway Cut problem.

Kann and Khanna et al. \cite{KKL+97} studied the Max -Cut problem \cite{FJ97}
and its dual, that is, the Min -Partition problem \cite{KKL+97}.
Given an undirected graph , the Min -Partition problem asks
to find a vertex coloring 
such that the number of edges whose two endpoints have the same color
(i.e., the happy edges in our setting) is minimized.

According to the way of definitions in \cite{KKL+97}, we can define
the dual of the Min -Cut problem \cite{SV95} as follows:
Given an undirected graph  and an integer ,
finding an edge subset whose removal breaks graph  into {\em exactly}
 components, such that the number of remaining edges is maximized.
Let's call this problem the Max -Partition problem.
In other words, Max -Partition asks for a total vertex coloring
 such that the number of
happy edges is maximized, where  should be a surjective function
(that is, for each color  there exists a vertex whose color is ).

The Max -Partition problem defined as above is close to the MHE problem,
but they are still different in the obvious way:
In Max -Partition there is no any vertex having a pre-specified
color and the required vertex coloring function  must be surjective,
while in MHE there must be some vertices having the pre-specified colors
and the required vertex coloring function  may not be surjective.




\bigskip
{\bf Notations.}
Let  be a graph. Let  and .
Suppose  is a vertex. Denote by  the set of neighbors of .
As usual,  means the degree of , i.e, .
Denote by  the set of neighbors of neighbors of  (not including
 itself), i.e., the vertices within distance 2 of  (assume each edge
has unit distance).

Given a vertex coloring , for a (colored or uncolored) vertex , define
 as the set of vertices in  that has not yet been colored.
For a colored vertex , define  as the set of vertices in 
having the {\em same} color as ,  as the set of vertices in
 having colors {\em different} to .

Given an instance  of some optimization problem ,
we use  ( for short) to denote the optimum (that is,
the value of an optimal solution) of the instance.
Let  be an algorithm for problem .
We use  ( for short) to denote the value of the solution
found by algorithm  on instance  of problem .
In addition,  and  also denote the corresponding solutions,
abusing notations slightly.




\bigskip
{\bf Organization of the paper.}
The remaining of the paper is organized as follows.
In Section \ref{sec - algorithms for MHV}, we show that 2-MHV is
polynomial-time solvable, and give the greedy approximation algorithm
and the subset-growth approximation algorithm for the MHV and -MHV
() problems.
In Section \ref{sec - algorithms for MHE}, we show that 2-MHE is
polynomial-time solvable, and give the division-strategy based approximation
algorithm for the MHE -MHE () problems.
In Section \ref{sec - Hardness results}, we prove the NP-hardness for the MHE,
-MHE (), MHV, and -MHV () problems.
In Section \ref{sec - conclusions} we conclude the paper by
introducing some future work. In the Appendix, we give approximation
algorithms for the SoftMHV and HardMHV problems.




\section{Algorithms for MHV}
\label{sec - algorithms for MHV}
In Subsection \ref{subsec - 2-MHV Is in P}, we give the polynomial time exact
algorithm for the 2-MHV problem. In Subsection \ref{subsec - Approxalgs for MHV},
we give the approximation algorithms {\sc Greedy-MHV} and {\sc Growth-MHV} for
the MHV problem.




\subsection{2-MHV Is in P}
\label{subsec - 2-MHV Is in P}
Let  be a finite set. Recall that a function
 is said to be submodular if
 holds for all .
Given a vertex subset , define function  to be
the number of vertices in  that has neighbors outside of , i.e.,
 is the size of the border (see Subsection \ref{sec - related works})
of . It is easy to verify that  is a submodular function.

Consider the 2-MHV problem, in which the color set  contains only two
colors 1 and 2. This problem can be solved in polynomial
time.

\begin{theorem}
The 2-MHV problem can be solved in  time.
\end{theorem}
\begin{proof}
Let  be the set of vertices
that are colored by color 1 by the partial function , and 
be the analogous vertex subset corresponding to color 2.
Then the 2-MHV problem is equivalent to finding a cut 
such that  for  and 
is minimized. We can do this by merging all vertices in  to
a single vertex , all vertices in  to a single vertex ,
and finding an - cut  on the resulting graph such that
 is minimized.
As pointed out by \cite[Lemma 3]{ZNI05}, finding such a cut can be done
by an algorithm in \cite{IFF01} for minimizing submodular functions
in   time, where  is the time to compute
the submodular function .
When the input graph is stored by a collection of adjacency lists,
 can be computed in  time in a straightforward way
(assuming the input graph contains no isolated vertex).
The proof of the theorem is finished.
\qed
\end{proof}




\subsection{Approximation Algorithms for MHV}
\label{subsec - Approxalgs for MHV}
The approximation algorithms for MHV work based on the types defined for
vertices, as shown in Definition \ref{def - types of vertices in MHV}.

\begin{definition}[Types of vertices in MHV]
\label{def - types of vertices in MHV}
Fix a (partial or total) vertex coloring. Let  be a vertex. Then,
\begin{enumerate}
\item  is an {\em -vertex} if  is colored and happy (i.e., );
\item  is a {\em -vertex} if  is colored and destined to be unhappy
(i.e., );
\item  is a {\em -vertex} if
\begin{enumerate}
    \item  is colored,
    \item  has not been happy (i.e., ), and
    \item  may become happy in the future (i.e., );
\end{enumerate}
\item  is an {\em -vertex} if  has not been colored.
\end{enumerate}
\end{definition}

See Figures \ref{fig - process a P-vertex}, \ref{fig - process a Lh-vertex},
\ref{fig - process a Lu-vertex} for examples of the vertex types.
Note that by a type name we also mean the set of vertices of that type.
Conversely, by a set name we also mean that each element in the set
is of that type. For example,  is the set of all -vertices;
each vertex in the set  is an -vertex.




\subsubsection{Greedy Approximation Algorithm for MHV}
\label{subsec - greedy approxalg for MHV}
{\bf Algorithm {\sc Greedy-MHV}.}
The approximation algorithm {\sc Greedy-MHV} for MHV is quiet simple.
We just color all uncolored vertices by the same color.
Since there are  colors in , we can obtain  vertex colorings for
graph . Finally we output the coloring that has the most number of
happy vertices.

\begin{theorem}
\label{th - 1/k-approximation for MHV}
Algorithm {\sc Greedy-MHV} is a -approximation algorithm for the MHV
problem, where  is the number of colors given in the input.
\end{theorem}
\begin{proof}
Let the partial function  be the vertex coloring used in Definition
\ref{def - types of vertices in MHV}.
We partition -vertices further into two subsets  and .
 is the set of uncolored vertices that can become happy (i.e., whose
neighbors have at most one color).  is the set of uncolored vertices
that are destined to be unhappy (i.e., whose neighbors already have
at least two distinct colors). Then  is a partition of
. Obviously, in the best case  can make all vertices in , 
and  happy, implying .

Let  be the number of happy vertices when Algorithm {\sc Greedy-MHV}
colors all uncolored vertices by color .
Then we have .
By the greedy strategy, , which is the number of happy vertices found
by {\sc Greedy-MHV}, is at least .
The theorem follows by observing that {\sc Greedy-MHV} obviously runs in
polynomial time.
\qed
\end{proof}





\subsubsection{Subset-Growth Approximation Algorithm for MHV}
\label{subsec - subset-growth approxalg for MHV}
The subset-growth algorithm starts with the partial vertex coloring
 defined by the partial function .
From a high level point of view, the algorithm iteratively augments
the subsets in  by satisfying the vertices that
can become happy easily at the current time, until 
becomes a partition of  and thus a vertex coloring is obtained.
This strategy is based on the following further classification of
-vertices, according to the type of their neighbors.
Recall that by Definition \ref{def - types of vertices in MHV},
-vertex means uncolored vertex.

\begin{definition}[Subtypes of -vertex in MHV]
Let  be an -vertex in a vertex coloring. Then,
\begin{enumerate}
\item  is an {\em -vertex} if  is adjacent to a -vertex;
\item  is an {\em -vertex} if
\begin{enumerate}
    \item  is not adjacent to any -vertex,
    \item  can become happy, that is,  is adjacent to -vertices with only
    one color;
\end{enumerate}
\item  is an {\em -vertex} if
\begin{enumerate}
    \item  is not adjacent to any -vertex,
    \item  is destined to be unhappy, that is,  is adjacent to -vertices
    with more than one colors;
\end{enumerate}
\item  is an {\em -vertex} if  is not adjacent to any colored vertex.
\end{enumerate}
\end{definition}

See Figures \ref{fig - process a P-vertex}, \ref{fig - process a Lh-vertex},
\ref{fig - process a Lu-vertex} for examples of the subtypes of -vertex.

The subset-growth algorithm {\sc Growth-MHV} is as follows.


\setcounter{algleo}{0}
\begin{algleo}
\linonumber {\bf Algorithm} {\sc Growth-MHV}
\linonumber {\em Input:} A connected undirected graph  and a partial
    coloring function .
\linonumber {\em Output:} A total vertex coloring for .

\li , .
\li \label{step - Growth-MHV - beginning of the main loop}
    {\bf while} there exist -vertices {\bf do}
\begin{algleo}
    \li \label{step - Growth-MHV - process an P-vertex}
        {\bf if} there exists a -vertex  {\bf then}
        \begin{algleo}
            \li .
            \li Add all the -neighbors of  to .
                The types of all affected vertices (including 
                and vertices in ) are changed accordingly.
        \end{algleo}
    \li \label{step - Growth-MHV - process an Lh-vertex}
        {\bf elseif} there exists an -vertex  {\bf then}
        \begin{algleo}
            \li Let  be any -vertex adjacent to ,
                then .
            \li Add  and all its -neighbors to .
                The types of all affected vertices
                (including  and vertices in ) are changed accordingly.
        \end{algleo}
    \li \label{step - Growth-MHV - process an Lu-vertex}
        {\bf else}
        \begin{algleo}
            \linonumber {\em Comment:} There must be an -vertex.
            \li Let  be any -vertex,  be the any -vertex
                adjacent to , then .
            \li Add  to .
                The types of all affected vertices
                (including  and vertices in ) are changed accordingly.
        \end{algleo}
    \li {\bf endif}
\end{algleo}
\li {\bf endwhile}
\li {\bf return} the vertex coloring .
\end{algleo}

When there are still -vertices (i.e., uncolored vertices), Algorithm
{\sc Growth-MHV} works in the following way. It first colors a -vertex's
neighbors to make this -vertex happy
(see Figure \ref{fig - process a P-vertex}).
When there is no any -vertex, it colors an -vertex and its neighbors
to make the -vertex happy (see Figure \ref{fig - process a Lh-vertex}).
When there is no any -vertex or -vertex, it colors an -vertex
by the color of its any -vertex neighbor
(see Figure \ref{fig - process a Lu-vertex}).
Note that coloring a vertex may generate new -vertices, or -vertices,
or -vertices.


\begin{figure}
\begin{center}
\includegraphics*[width=0.55\textwidth]{process-Pv2.eps}
\end{center}
\caption{Process a -vertex. The hollow vertex  in graph (a) is
the -vertex to be processed. The square vertices mean colored vertices,
while the round vertices mean uncolored vertices.}
\label{fig - process a P-vertex}
\end{figure}

\begin{figure}
\begin{center}
\includegraphics*[width=0.6\textwidth]{process-Lh.eps}
\end{center}
\caption{Process an -vertex. The hollow vertex  in graph (a) is
the -vertex to be processed. Note that when an -vertex is
to be processed, there is no -vertex in the current graph (a).}
\label{fig - process a Lh-vertex}
\end{figure}

\begin{figure}
\begin{center}
\includegraphics*[width=0.55\textwidth]{process-Lu.eps}
\end{center}
\caption{Process an -vertex. The hollow vertex  in graph (a) is
the -vertex to be processed. Note that when an -vertex is
to be processed, there is no any -vertex or -vertex in the current
graph (a).}
\label{fig - process a Lu-vertex}
\end{figure}


When there exist -vertices, it is impossible that there are only
-vertices but no any -vertex, -vertex or -vertex,
since by assumption  is a connected graph and by definition
-vertex is not adjacent to any colored vertex.
So, when there isn't any -vertex or -vertex, there must be
at least one -vertex.
As a result, in step \ref{step - Growth-MHV - process an Lu-vertex}
we don't need an {\bf if} statement like that in
steps \ref{step - Growth-MHV - process an P-vertex} and
\ref{step - Growth-MHV - process an Lh-vertex}.

We use a type name with the superscript ``org'' (means ``original'') to
denote the set of vertices of that type which is determined by
the partial function , and a type name with the superscript ``new'' to
denote the set of vertices of that type which is determined in the execution
of Algorithm {\sc Growth-MHV}. For example,  is the set of -vertices
that are determined by the partial function , and  is the set of
-vertices that are newly generated by Algorithm {\sc Growth-MHV}.


Let  be the maximum degree of vertices in the input graph.
We first bound the number of -vertices.

\begin{lemma}
\label{lm - |L_u^new| <= Delta (Delta-2) |H^new|}
.
\end{lemma}
\begin{proof}
Algorithm {\sc Growth-MHV} iteratively processes three types of vertices,
that is, the -vertices, the -vertices and the -vertices.
We will prove the lemma by proving the following three points:
(1) When Algorithm {\sc Growth-MHV} processes a -vertex, at most
 -vertices are generated,
(2) When Algorithm {\sc Growth-MHV} processes an -vertex, at most
 -vertices are generated, and
(3) When Algorithm {\sc Growth-MHV} processes an -vertex,
no -vertex is generated.

Consider the first point. Let  be a -vertex to be processed.
Suppose  has an -neighbor , which is adjacent to a -vertex.
Only if there is an -vertex  which is the neighbor of ,
 will become a newly generated -vertex when the -vertex 
is processed. See Figure \ref{fig - process a P-vertex} for an example.
Since the maximum vertex degree is ,  has at most 
-neighbors, and  has at most  -neighbors.
This implies that when  is processed, at most 
-vertices can be generated.

Then consider the second point. Suppose the -vertex to be processed
is . Suppose  has an -neighbor  ( can be an -vertex or
an -vertex), which is adjacent to a -vertex. Similarly,
only if there is an -vertex  which is the neighbor of ,
 will become a newly generated -vertex when the -vertex 
is processed. See Figure \ref{fig - process a Lh-vertex} for an example.
Since the maximum vertex degree is ,  has at most 
-neighbors, and  has at most  -neighbors.
This implies that when  is processed, at most 
-vertices can be generated.

Finally consider the third point. When Algorithm {\sc Growth-MHV} processes
an -vertex, there is no any -vertex (or -vertex) in the
current graph. So, adding an -vertex to some subset 
does not generate any new -vertex.
See Figure \ref{fig - process a Lu-vertex} for an example.

When Algorithm {\sc Growth-MHV} processes a -vertex or an -vertex,
at least one vertex becomes an -vertex. So we can charge the number of
newly generated -vertices to this newly generated -vertex.
This finishes the proof of the lemma.
\qed
\end{proof}


The following Lemma \ref{lm - upper bound on OPT} gives an upper bound
on , the number of happy vertices in an optimal solution to
the -MHV problem.

\begin{lemma}
\label{lm - upper bound on OPT}
.
\end{lemma}
\begin{proof}
By the partial function , all vertices in the original graph
(i.e., the input graph that has not been colored by Algorithm {\sc Growth-MHV})
are partitioned into four vertex subsets , , 
and . Subset  is further partitioned into four subsets
, ,  and .
By definition, all vertices in  are unhappy.
And, all vertices in  are destined to be unhappy
since each of them is adjacent to at least two vertices with different
colors. So, in the best case all vertices in  and 
except those in  would be happy.
Noticing that the vertices in  are already happy, we have


Since each -vertex must be adjacent to some -vertex, and each
-vertex can be adjacent to at most  -vertices, the number
of -vertices is at most .
Since , we get that

concluding the lemma.
\qed
\end{proof}


\begin{lemma}
\label{lm - lower bound on |H^new|}
.
\end{lemma}
\begin{proof}
Recall that there are four subtypes of an -vertex, i.e., -vertex,
-vertex, -vertex and -vertex.
Among them only -vertex and -vertex will (directly) contribute
to generating -vertices.
For an -vertex, it will ultimately become one of the other three
types of -vertex.
For each -vertex, although it may become an -vertex and hence
can contribute to generating -vertices, in the worst case we may
assume that it is added to some subset 
and contribute nothing to the generation of -vertex.

By step \ref{step - Growth-MHV - process an P-vertex} and
step \ref{step - Growth-MHV - process an Lu-vertex},
each time an -vertex is generated, at most  -vertices or
-vertices are consumed (i.e., colored). Furthermore, once
an -vertex is colored, it will never be re-colored or de-colored.
So we have


By Lemma \ref{lm - |L_u^new| <= Delta (Delta-2) |H^new|},
we have

Therefore, .
The lemma follows.
\qed
\end{proof}


\begin{theorem}
The MHV problem can be approximated within a factor of 
in polynomial time.
\end{theorem}
\begin{proof}
Algorithm {\sc Growth-MHV} obviously runs in polynomial time.
Let  be the number of happy vertices found by Algorithm {\sc Growth-MHV}.
Then we have

The theorem follows.
\qed
\end{proof}





\section{Algorithms for MHE}
\label{sec - algorithms for MHE}




\subsection{2-MHE Is in P}
For 2-MHE, the partial function  can only use two colors, to say,
color 1 and color 2. Given such an instance, merge all vertices with color 1
assigned by  into a single vertex , and all vertices with color 2
into a single vertex . (The edges whose two endpoints are merged disappear
in the procedure.) Then compute a minimum - cut  on
the resulting instance. Suppose  and . Assign color 1
to all vertices (including the merged vertices) in , and color 2 to
all vertices in . Since  is a minimum - cut,
the number of happy edges in the resulting vertex coloring is maximized.
By the work of \cite{ET75}, a maximum flow (and hence a minimum -)
in a unit capacity network can be computed in 
time. So we have

\begin{theorem}
\label{th - 2-MHE is in P}
The 2-MHE problem can be solved in  time.
\qed
\end{theorem}




\subsection{Approximation Algorithm for MHE}
The MHE problem admits a simple division-strategy based algorithm which
yields a -approximation. The algorithm is designed to
deal with more general graphs with nonnegative weights  defined
on edges. We thus denote by  the total weight of edges in an edge
subset .

\setcounter{algleo}{0}
\begin{algleo}
\linonumber {\bf Algorithm} {\sc Division-MHE}
\linonumber {\em Input:} An undirected graph  and a partial coloring
    function .
\linonumber {\em Output:} A total vertex coloring for .
\li .
\li Let  be the set of edges in  that has exactly one endpoint not
    colored by function . Define graph ,
    which is a subgraph of .
\li \label{step - Division-MHE - Graph G' and its stars}
    For each star  in  centered at an uncolored vertex , color 
    by a color in 
    such that the total weight of happy edges in  is maximized.
\li Color all vertices in  still having not been colored by just one
    arbitrary color. Denote by  the vertex coloring of .
\li .
\li Color all uncolored vertices in  by just one arbitrary color.
    Denote by  the vertex coloring of .
\li {\bf return} the better one among  and .
\end{algleo}

Algorithm {\sc Division-MHE} computes two independent solutions
 and  to graph , and then outputs the better one,
where the better one means the solution making more edges happy.
For an illustration of graph  and its stars in
step \ref{step - Division-MHE - Graph G' and its stars},
please refer to Figure \ref{fig - stars in graph G'}.

\begin{figure}
\begin{center}
\includegraphics*[width=0.6\textwidth]{division-mes.eps}
\end{center}
\caption{An example of graph . Each edge in  has its one endpoint
colored and the other endpoint uncolored. The square vertices mean colored
vertices, while the round vertices mean uncolored vertices. Each star
(marked with dashed circle) is centered at an uncolored vertex.
Two stars (e.g.,  and ) may share common colored vertices.}
\label{fig - stars in graph G'}
\end{figure}


\begin{theorem}
\label{th - algorithm Division-MHE is a 1/2-approxalg}
Algorithm {\sc Division-MHE} is a -approximation algorithm for
the MHE problem.
\end{theorem}
\begin{proof}
First, the algorithm obviously runs in polynomial time.

Let  be the total weight of edges already being happy by
the partial coloring function . This weight can be trivially obtained
by any solution.

Let  be the total weight of happy edges found by Algorithm
{\sc Division-MHE} on graph . Note that  is the maximum
total weight that can be obtained from graph .
Let  be the set of edges that has both of its two endpoints uncolored
by function , and  be its total weight.
Then we have .

By the algorithm, we know  and
. Then the approximation ratio  of
{\sc Division-MHE} is obvious
since .
\qed
\end{proof}





\section{Hardness Results}
\label{sec - Hardness results}




\subsection{NP-hardness of MHE}
\label{subsec - NP-hardness of MHE}
The NP-hardness of the -MHE problem is proved by a reduction from
the Multiway Cut problem \cite{DJP+94}.

\begin{theorem}
\label{th - 3-MHE is NP-hard}
The 3-MHE problem is NP-hard.
\end{theorem}
\begin{proof}
Given an undirected graph  and a terminal set ,
the 3-Terminal Cut problem (i.e., the Multiway Cut problem with
3 terminals), which is NP-hard \cite{DJP+94}, asks for a minimum cardinality
edge set such that its removal from  disconnects the three terminals
from one another. Given an instance  of 3-Terminal Cut, we construct
the instance  of 3-MHE as follows.
Graph  is just . Color set  is set to be .
The partial function  assigns colors 1, 2, 3 to vertices ,
respectively. Let  be the cardinality of an optimal 3-way cut for
, and  be the number of happy edges of an optimal vertex
coloring for . Then one can easily find that ,
where  (). This shows the 3-MHE problem is NP-hard.
\qed
\end{proof}

\begin{corollary}
The MHE problem is NP-hard.
\end{corollary}
\begin{proof}
In the input of MHE, just set  to be 3. \qed
\end{proof}



\begin{theorem}
\label{th - k-MHE is NP-hard}
The -MHE problem is NP-hard for any constant .
\end{theorem}
\begin{proof}
By Theorem \ref{th - 3-MHE is NP-hard}, we need only focus on .
Let  be such a constant.

Given a 3-MHE instance , we construct a -MHE instance 
as follows. Build  vertices 
and  edges , .
Vertices  and  are colored by color , for .
Let  be a vertex in  whose color given by  is 1.
Then put  edges , .
This is our new instance .

Obviously for , each edge  is happy
whereas each edge  is unhappy.
So, the optimum of  is just equal to the optimum of 
minus , concluding the theorem.
\qed
\end{proof}




\subsection{NP-hardness of MHV}
\label{subsec - NP-hardness of MHV}
\begin{theorem}
The -MHV problem is NP-hard for any constant .
\end{theorem}
\begin{proof}
By Theorem \ref{th - k-MHE is NP-hard}, -MHE is NP-hard ().
We thus reduce -MHE to -MHV.

Let  be a -MHE instance. The instance  of -MHV
is constructed as follows.
Add  vertices  and put an edge between 
and , for each  and each .
Vertex  is colored by , for .
For every edge , add a vertex  and replace the edge
by two edges  and .
This is our new instance .

Since in graph  there are vertices with pre-specified colors,
each  () cannot become happy no matter how the remaining
vertices are colored. Every original vertex  also cannot become happy
since it is adjacent to all 's.
Let  be any edge in . Since the degree of vertex  is 2,
it is happy iff its two neighbors have the same color.
This shows that the optimum of the -MHE instance  is equal to
the optimum of the -MHV instance .
The theorem follows.
\qed
\end{proof}




\section{Conclusions}
\label{sec - conclusions}
The MHV problem and the MHE problem are two natural graph coloring problems
arising in the homophyly phenomenon of networks. In this paper we prove
the NP-hardness of the MHV problem and the MHE problem, and give several
approximation algorithms for these two problems.

Since our algorithms {\sc Greedy-MHV}, {\sc Growth-MHV} and {\sc Division-MHE}
actually do not care whether the color number  is given in the input or
whether  is a constant, the -MHV and -MHE problems can also be
approximated within  and , respectively.

To improve the approximation ratios for MHV and MHE remains an immediate
open problem. It is also interesting to study the MHV and MHE problems
in random graphs generated from the classical network models, and in the
real-world large networks.


\section*{Acknowledgements}
We are grateful for fruitful discussions on this paper with
Dr. Mingji Xia at Institute of Software, Chinese Academy of Sciences.





\begin{thebibliography}{99}

\bibitem{BFOS84}
L. Breiman, J. H. Friedman, R. A. Olshen, C. J. Stone.
{\em Classification and Regression Trees.}
Wadsworth and Brooks, Monterey, CA, USA, 1984.

\bibitem{CKR00}
Gruia Calinescu, Howard J. Karloff, Yuval Rabani.
An improved approximation algorithm for multiway cut.
{\em Journal of Computer and System Sciences},
60(3):564--574, 2000.

\bibitem{DJP+94}
Elias Dahlhaus, David S. Johnson, Christos H. Papadimitriou, Paul D. Seymour,
Mihalis Yannakakis.
The complexity of multiterminal cuts.
{\em SIAM Journal on Computing},
23(4):864--894, 1994.

\bibitem{EK10}
David Easley, Jon Kleinberg.
{\em Networks, Crowds, and Markets: Reasoning About a Highly Connected World.}
Cambridge University Press, 2010.

\bibitem{EL92}
P\'eter L. Erd\"os, L\'aszl\'o A. Sz\'ekely.
Evolutionary trees: an integer multicommodity max-flow-min-cut theorem.
{\em Advances in Applied Mathematics},
13(4):375--389, 1992.

\bibitem{ET75}
S. Even, R. E. Tarjan.
Network flow and testting graph connectivity.
{\em SIAM Journal on Computing},
4:507--518, 1975.

\bibitem{FJ97}
A. Frieze, M. Jerrum.
Improved approximation algorithms for max -cut and max bisection.
{\em Algorithmica}, 18:67--81, 1997.

\bibitem{IFF01}
Satoru Iwata, Lisa Fleischer, Satoru Fujishige.
A combinatorial strongly polynomial algorithm for minimizing submodular functions.
{\em Journal of the ACM}, 48(4):761--777, 2001.
arXiv:math/0004089v1.

\bibitem{KKL+97}
Viggo Kann, Sanjeev Khanna, Jens Lagergren, Alessandro Panconesi.
On the hardness of approximating max -cut and its dual.
{\em Chicago Journal of Theoretical Computer Science}, vol. 1997, 1997.

\bibitem{KKS+04}
David R. Karger, Philip N. Klein, Clifford Stein, Mikkel Thorup, Neal E. Young.
Rounding algorithms for a geometric embedding of minimum multiway cut.
{\em Mathematics of Operations Research},
29(3):436--461, 2004.
arXiv:cs/0205051.

\bibitem{KT02}
Jon M. Kleinberg, \'Eva Tardos.
Approximation algorithms for classification problems with pairwise relationships:
metric labeling and Markov random fields.
{\em Journal of the ACM},
49(5):616--639, 2002.

\bibitem{LLPP11}
Angsheng Li, Jiankou Li, Yicheng Pan, Pan Peng.
Homophyly law of networks: Principle, method and experiments.
{\em Manuscript}, 2011. To appear.

\bibitem{LLPP12}
Angsheng Li, Jiankou Li, Yicheng Pan, Pan Peng.
Small community phenomenon in networks: Mechansims, roles and characteristics.
{\em Manuscript}, 2012. To appear.

\bibitem{LP11}
Angsheng Li, Pan Peng.
Community structures in classical network models.
{\em Internet Mathematics}, 7(2):81--106, 2011.

\bibitem{LP12}
Angsheng Li, Pan Peng.
The small community phenomenon in networks.
{\em Mathematical Structures in Computer Science},
22:1--35, 2012.
arXiv:1107.5786.

\bibitem{SV95}
H. Saran, V. Vazirani.
Finding -cuts within twice the optimal.
{\em SIAM Journal on Computing}, 24:101--108, 1995.

\bibitem{S72}
Thomas Schelling.
Dynamic models of segregation.
{\em Journal of Mathematical Sociology}, 1:143--186, 1972.

\bibitem{S78}
Thomas Schelling.
{\em Micromotives and Macrobehavior}, Norton, 1978.

\bibitem{ZNI05}
Liang Zhao, Hiroshi Nagamochi, Toshihide Ibaraki.
Greedy splitting algorithms for approximating multiway partition problems.
{\em Mathematical Programming}, 102(1):167--183, 2005.
\end{thebibliography}




\appendix

\section*{Appendix}




\section{Variants of MHV}
\label{sec - variants of MHV}
For a vertex  in the MHV problem, instead of requiring that {\em all}
neighbors of  have the same color as that of , to make  happy
we may only require at least  neighbors have the same
color as that of , or only require at least  neighbors have the color
identical to that of , for some global number . This leads to two
natural variants of the MHV problem, that is, the SoftMHV problem and
the HardMHV problem. Similarly, we can define the corresponding varints
for the -MHV problem, and our results in this section naturally extends
to these variants. For simplicity, we only consider approximation algorithms
for the SoftMHV and HardMHV problems.

Fix a vertex coloring, and let  be a (colored or uncolored) vertex.
Define  to be the set of vertices in  which has color ,
for .




\section{MHV with Soft Threshold}
Let  be a number in . In the soft-threshold extension of
the MHV problem (SoftMHV for short), a vertex  is happy if  is
colored and .
Given a connected undirected graph , a partial coloring function ,
the SoftMHV problem asks for a total vertex coloring extended from  that
maximizes the number of happy vertices.
(The number  can be given as a part of the input or be a constant.
We do not distinguish between these two cases for simplicy.)




\subsection{Algorithm for SoftMHV}
As what is done in Definition \ref{def - types of vertices in MHV},
we define the types of vertices according to the given vertex coloring.

\begin{definition}[Types of vertex in SoftMHV]
\label{def - types of vertex - SoftMHV}
Fix a (partial or total) vertex coloring. Let  be a vertex. Then,
\begin{enumerate}
\item  is an -vertex if  is colored and happy;

\item  is a -vertex if
\begin{enumerate}
\item  is colored, and
\item  is destined to be unhappy, (i.e., );
\end{enumerate}

\item  is a -vertex if
\begin{enumerate}
\item  is colored,
\item  has not been happy (i.e., ), and
\item  can become an -vertex (i.e., );
\end{enumerate}

\item  is an -vertex if  has not been colored.
\end{enumerate}
\end{definition}

We note that Algorithm {\sc Greedy-MHV} is also a -approximation
algorithm for the SoftMHV problem. To see this, we just define 
in Theorem \ref{th - 1/k-approximation for MHV} as the set of uncolored
vertices  such that ,
and .

\begin{theorem}
The SoftMHV problem can be approximated within a factor of 
in polynomial time. \qed
\end{theorem}

Below we give the subset-growth approximation algorithm {\sc Growth-SoftMHV}
for the SoftMHV problem. First we define the subtypes of -vertex.

\begin{definition}[Subtypes of -vertex in SoftMHV]
Let vertex  be an -vertex in a vertex coloring. Then,
\begin{enumerate}
\item  is an -vertex if  is adjacent to a -vertex,

\item  is an -vertex if
\begin{enumerate}
\item  is not adjacent to any -vertex,
\item  is adjacent to an -vertex or a -vertex, and
\item  can become happy (that is,
),
\end{enumerate}

\item  is an -vertex if
\begin{enumerate}
\item  is not adjacent to any -vertex,
\item  is adjacent to an -vertex or a -vertex, and
\item  is destined to be unhappy (that is,
),
\end{enumerate}

\item  is an -vertex if  is not adjacent to any colored
vertex.
\end{enumerate}
\end{definition}


\setcounter{algleo}{0}
\begin{algleo}
\linonumber {\bf Algorithm} {\sc Growth-SoftMHV}
\linonumber {\em Input:} A connected undirected graph  and a partial
    coloring function .
\linonumber {\em Output:} A total vertex coloring for .

\li , .
\li \label{step - Growth-SoftMHV - beginning of the main loop}
    {\bf while} there exist -vertices {\bf do}
\begin{algleo}
    \li {\bf if} there exists a -vertex  {\bf then}
        \begin{algleo}
            \li .
            \li \label{step - Growth-SoftMHV - process P-vertex}
                Add its any 
                -neighbors to vertex subset .
                The types of all affected vertices (including  and vertices
                in ) are changed accordingly.
        \end{algleo}
    \li {\bf elseif} there exists an -vertex  {\bf then}
        \begin{algleo}
            \li Let  be the vertex subset in which  has
                the maximum colored neighbors.
            \li \label{step - Growth-SoftMHV - process Lh-vertex}
                Add vertex  and its any 
                -neighbors to vertex subset .
                The types of all affected vertices (including  and vertices
                in ) are changed accordingly.
        \end{algleo}
    \li {\bf else}
        \begin{algleo}
            \linonumber {\em Comment:} There must be an -vertex.
            \li Let  be any -vertex, and  be any vertex subset in which
                 has colored neighbors.
            \li Add vertex  to subset .
                The types of all affected vertices (including  and vertices
                in ) are changed accordingly.
        \end{algleo}
    \li {\bf endif}
\end{algleo}
\li {\bf endwhile}
\li {\bf return} the vertex coloring .
\end{algleo}


In step \ref{step - Growth-SoftMHV - process P-vertex},
the algorithm adds the least number
(that is, )
of 's neighbors to subset  to make  happy. The same thing is done
in step \ref{step - Growth-SoftMHV - process Lh-vertex}.


\begin{lemma}
\label{lm - lower bound on |H^new|, Growth-SoftMHV}
.
\end{lemma}
\begin{proof}
Suppose Algorithm {\sc Growth-SoftMHV} is to process a -vertex ,
which is already colored by color .
When  is processed, at most  -neighbors
of  are added to . Each of the -neighbors has at most
 -neighbors.
In the worst case, all these -neighbors,
plus the remaining -neighbors of , could become -vertices
when  is processed. So, at most

-vertices can be generated in this case.

Then suppose the algorithm is to process an -vertex . Let
 be the vertex subset in which  has the maximum colored neighbors.
When  is processed, at most  -neighbors
of  are added to . Each of these -neighbors can have at most
 -neighbors. In the worst case,
all these -neighbors, plus the remaining -neighbors of ,
could become -vertices when  is processed.
So, at most

-vertices can be generated in this case.

When the algorithm processes an -vertex, there are only
-vertices or -vertices (if any) in the current graph.
So, coloring an -vertex does not generate any new -vertex.

By charging the number of newly generated -vertices to the newly
generated -vertex, we finish the proof of the lemma.
\qed
\end{proof}


\begin{theorem}
The SoftMHV problem can be approximated within a factor of
 in polynomial time.
\end{theorem}
\begin{proof}
Each time an -vertex is generated, at most 
-vertices are consumed (i.e., colored).
So, for the number of newly generated -vertices we have
.
By Lemma \ref{lm - lower bound on |H^new|, Growth-SoftMHV}, we get


Let  be the number of happy vertices in an optimal solution to
the problem. By the same reason as in Lemma \ref{lm - upper bound on OPT},
we obtain


Let  be the number of happy vertices found by Algorithm
{\sc Growth-SoftMHV}. Then we have


Finally, notice that Algorithm {\sc Growth-SoftMHV} obviously runs
in polynomial time. This gives the theorem.
\qed
\end{proof}




\subsection{NP-Hardness of SoftMHV}
\begin{theorem}
For any real number , there exist infinitely many integers
, such that the corresponding SoftMHV problem is NP-hard.
\end{theorem}
\begin{proof}
Reduce from 3-MHE. Let  be an instance of 3-MHE, and  be any
real constant in . We shall construct a SoftMHV instance
 in the following, in which the color number  is an integer
that depends only on . The value of  will be given later.

Let  be an integer constant depending on  and , which will be
fixed later.
For every edge , add  vertices  (called
-vertex), , , ,  (called -vertices),
, , ,  (called -vertices).
Replace edge  by two consecutive edges 
and . For each vertex 
, connect it to  via an edge
.
For , vertex  has a pre-specified color .

For every vertex , add  vertices ,
, , , , , ,
, , , , ,
 (called -vertices), where  is the maximum vertex
degree of .
For each  and each , connect vertex
 to  via an edge . Vertex  is
colored in advance by color , , .
This is our graph  in the new instance of SoftMHV.

Next we determine constants  and .
To enable the reduction to work,  and  should satisfy

Let  be any edge in . Consider vertex  in .
No matter how  is colored (recall that the color set is
), there is exactly one vertex in
 having the same color as that of .
Note that .
So, inequality (\ref{eqn - h + 3 >= rho(h + k + 2)}) guarantees that
if all vertices in  have the same
color as that of ,  will be happy, and,
inequality (\ref{eqn - h + 2 < rho (h + k + 2)}) guarantees that
if there is one vertex in  having
different color to that of ,  will be unhappy.

By inequality (\ref{eqn - h + 3 >= rho(h + k + 2)}) and
inequality (\ref{eqn - h + 2 < rho (h + k + 2)}), the value of integer 
should satisfy

Since  and ,
there must be at least one integer in the interval of
(\ref{eqn - interval of h}).

Of course, the left end of the interval of (\ref{eqn - interval of h})
should be at least 1. This gives


For each vertex  that comes from , we want to guarantee that
no matter how the vertices in  are colored,  will never be happy.
Note that no matter what color vertex  is colored by,
there are exactly  vertices in
 having the
same color as that of .
Since  and ,
to make vertex  unsatisfiable, we just need .
Since , this will be
guaranteed by letting


Since we start our reduction from the 3-MHE problem, naturally we need


By inequalities (\ref{eqn - k >= 4/rho - 3}), (\ref{eqn - k >= 2/rho}) and
(\ref{eqn - k >= 3}), we can set  as {\em any} integer such that

Once  is fixed, we can fix  according to (\ref{eqn - interval of h}).

We have completed our new instance  of SoftMHV.

Let  and .
Denote by  the number of happy edges in an optimal solution to
the 3-MHE instance , and by  the number of happy vertices
in an optimal solution to the SoftMHV instance .
We shall prove the following claim, which will finish the proof of the
theorem.

\begin{claim}
.
\end{claim}
\begin{proof}
()
Let  be an optimal solution to instance .
First we color every vertex  such that  is also in 
by color . For each edge , color 
by color , (actually coloring  by either  or
 is ok.) and color all vertices , , 
by the color of . This is our vertex coloring for instance .

By similar arguments before inequality
(\ref{eqn - k >= 2/rho}), for every vertex  and its
corresponding -vertices in ,  itself is unhappy and
there are exactly  happy vertices in
.
So we obtain  happy vertices from all -vertices in .

Let  be an edge in . In its corresponding
-vertices  and
-vertices ,
there are exactly  vertices that are happy by our coloring.
So we obtain  happy vertices from all the -vertices
and -vertices in .

Next let us consider vertex . If  is happy by ,
then  has  neighbors having the same color as that of .
So the fraction of happy neighbors of  is

where the first inequality is due to 
(by inequality (\ref{eqn - h + 3 >= rho(h + k + 2)})),
and hence  is happy.
Since , we can obtain  happy vertices from
all -vertices in .

Summing all, the number of happy vertices in  by our coloring
is at least .

()
Let  be an optimal solution to instance  of SoftMHV.
By the arguments before inequality (\ref{eqn - k >= 2/rho}),
every vertex  is unhappy by ,
and there are exactly  happy -vertices by .

Let  be any edge in . Since  is an optimal coloring,
we can assume that all vertices , ,  have
color . Taking into account the one more happy vertex
 (where ) for each ,
there are exactly  happy vertices by  from all
-vertices and -vertices.

Now only -vertices in  remain unconsidered.
Since , there must be at least 
happy -vertices. Let  be any such vertex.
Since  is happy, the number of neighbors of 
that have the color as that of  is at least

where the first inequality is due to inequality
(\ref{eqn - h + 2 < rho (h + k + 2)}).
This shows that the number of neighbors of  having color
 is at least . So, vertices  and  must have
the {\em same} color (as that of ).

Let us color every vertex  by color .
If there are vertices in  colored by colors in ,
then color all of them by color 1 (note that  is part of the instance of
the 3-MHE problem). This will never decrease the number of happy edges
in . By the above analysis, the number of happy edges in  is
at least .
\qed
\end{proof}

The proof of the theorem is finished.
\qed
\end{proof}




\section{MHV with Hard Threshold}
In the hard-threshold variant of the -MHV problem (HardMHV for short),
a vertex  is happy if , where  is an input
parameter.
Given a connected undirected graph , a partial coloring function ,
and an integer , the HardMHV problem asks for a total vertex coloring
extended from  that maximizes the number of happy vertices.
It is reasonable to assume , since otherwise
there is no feasible solution to the problem.




\subsection{Algorithm for HardMHV}
The following type definition of vertices is similar to
Definition \ref{def - types of vertex - SoftMHV}.

\begin{definition}[Types of vertex in HardMHV]
Fix a (partialor total) vertex coloring. Let  be a vertex. Then,
\begin{enumerate}
\item  is an -vertex if  is colored and happy,

\item  is a -vertex if
\begin{enumerate}
\item  is colored, and
\item  is destined to be unhappy (i.e., ),
\end{enumerate}

\item  is a -vertex if
\begin{enumerate}
\item  is colored,
\item  has not been happy (that is, ), and
\item  can become happy (i.e., ),
\end{enumerate}

\item  is an -vertex if  has not been colored.
\end{enumerate}
\end{definition}

Similar as the case of SoftMHV, Algorithm {\sc Greedy-MHV} is also
a -approximation algorithm for the HardMHV problem. To prove this
we only need to define  in Theorem \ref{th - 1/k-approximation for MHV}
as the set of uncolored vertices  such that
, and .

\begin{theorem}
There is a -approximation algorithm for the HardMHV problem. \qed
\end{theorem}


In the MHV and SoftMHV problems, for an -vertex , if  is
too large, then  may be destined to be unhappy.
In contrast, in the HardMHV problem, an -vertex  may be destined to
be unhappy even if : This will happen when .
Based on this observation, the -vertex type is divided into the following
four subtypes.

\begin{definition}[Subtypes of -vertex in HardMHV]
\label{def - subtypes of L-vertex - HardMHV}
Let vertex  be an -vertex in a vertex coloring. Then,
\begin{enumerate}
\item  is an -vertex if  is adjacent to a -vertex,

\item  is an -vertex if
\begin{enumerate}
\item  is not adjacent to any -vertex,
\item  is adjacent to an -vertex or a -vertex, and
\item  can become happy (i.e.,
),
\end{enumerate}

\item  is an -vertex if
\begin{enumerate}
\item  is not adjacent to any -vertex, and
\item  is destined to be unhappy (i.e.,
),
\end{enumerate}

\item  is an -vertex if
\begin{enumerate}
\item  is not adjacent to any colored vertex, and
\item  can become happy.
\end{enumerate}
\end{enumerate}
\end{definition}

One can verify that the subtypes in
Definition \ref{def - subtypes of L-vertex - HardMHV}
really form a partition of all -vertices. Note that the -vertex
not only refers to the destined-to-be-unhappy -vertex that is adjacent
to an -vertex or a -vertex (like the -vertex in MHV and
the -vertex in SoftMHV), but also refers to the
destined-to-be-unhappy -vertex that is not adjacent to any colored vertex,
as discussed before Definition \ref{def - subtypes of L-vertex - HardMHV}.


Below is the subset-growth approximation algorithm {\sc Growth-HardMHV}
for the HardMHV problem.

\setcounter{algleo}{0}
\begin{algleo}
\linonumber {\bf Algorithm} {\sc Growth-HardMHV}
\linonumber {\em Input:} A connected undirected graph , a partial
    coloring function , and an integer .
\linonumber {\em Output:} A total vertex coloring for .

\li , .
\li {\bf while} there exist -vertices {\bf do}
\begin{algleo}
    \li {\bf if} there exists a -vertex  {\bf then}
        \begin{algleo}
            \li .
            \li Add its any  -neighbors to .
                The types of all affected vertices (including  and vertices
                in ) are changed accordingly.
        \end{algleo}
    \li {\bf elseif} there exists an -vertex  {\bf then}
        \begin{algleo}
            \li Let  be the vertex subset in which  has the maximum
                colored neighbors.
            \li Add vertex  and its any  -neighbors
                to .
                The types of all affected vertices (including  and vertices
                in ) are changed accordingly.
        \end{algleo}
    \li {\bf else}
        \begin{algleo}
            \linonumber {\em Comment:} There must be an -vertex.
            \li Let  be any -vertex. If  has colored neighbors,
                then let  be any vertex subset containing a colored neighbor of .
                Otherwise let  be .
            \li Add vertex  to subset .
                The types of all affected vertices (including  and vertices in )
                are changed accordingly.
        \end{algleo}
    \li {\bf endif}
\end{algleo}
\li {\bf endwhile}
\li {\bf return} the vertex coloring .
\end{algleo}


\begin{lemma}
\label{lm - lower bound on |H^new|, Growth-HardMHV}
.
\end{lemma}
\begin{proof}
The proof of the lemma is similar to that of
Lemma \ref{lm - lower bound on |H^new|, Growth-SoftMHV}.
Only one point needs to pay attention.
When the algorithm processes an -vertex, there are only
-vertices or -vertices (if any) in the current graph.
Each time Algorithm {\sc Growth-HardMHV} processes an -vertex ,
it processes only {\em one} such vertex. So, if  has an -neighbor
,  will become an -vertex after the processing.
This means that coloring an -vertex does not generate any
new -vertex. We omit the other details of the proof.
\qed
\end{proof}

\begin{theorem}
The HardMHV problem can be approximated within a factor of
 in polynomial time.
\end{theorem}
\begin{proof}
Each time an -vertex is generated, at most  -vertices are consumed
(i.e., colored). So, for the number of newly generated -vertices we have
.
By Lemma \ref{lm - lower bound on |H^new|, Growth-HardMHV}, and noticing
that , we get


Let  be the number of happy vertices in an optimal solution to
the problem. By the same reason as in Lemma \ref{lm - upper bound on OPT},
we obtain


Let  be the number of happy vertices found by Algorithm
{\sc Growth-HardMHV}. Then we have

by the above two inequalities.
As Algorithm {\sc Growth-HardMHV} obviously runs in polynomial time,
the theorem follows.
\qed
\end{proof}




\subsection{NP-Hardness of HardMHV}
\begin{theorem}
The HardMHV problem is NP-hard for any constant , where  is
the color number in the problem.
\end{theorem}
\begin{proof}
We prove the theorem by reducing -MHE (see Theorem \ref{th - k-MHE is NP-hard})
to HardMHV.

Given an instance  of -MHE, we construct an instance 
of HardMHV as follows. For each edge , do the following.
Add a vertex  and  vertices
, , , , where  is
the maximum vertex degree of . The vertices 's are called
{\em satellite vertices}.
Replace edge  by two edges  and .
Connect each vertex  to  via an edge .
Finally, let . We thus get our HardMHV instance
.

Since , each original vertex  and each newly
added satellite vertex cannot be happy no matter how the vertices in 
are colored. For each edge , since its corresponding vertex
 is of degree ,  is happy iff its
two neighbors  and  have the same color. This shows that
the optimum of  is equal to that of , finishing
the proof of the theorem.
\qed
\end{proof}




\end{document}
