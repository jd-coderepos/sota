
In this section, we describe two rate-based serving algorithms.  The
first is the {\em High Water Mark Algorithm} (HWM).  The second is
an implementation of the dual-based method of~\cite{EC}, which we
will refer to as {\em DUAL}.

Both algorithms involve two phases. The offline phase, described in
Section~\ref{sec:offline}, takes as input the demand-supply forecast
graph and produces an allocation plan.  This allocation plan is just
a few numbers per contract, independent of the number of
impressions.
The online phase, described in
Section~\ref{sec:online}, repeatedly takes as input a user visit and
the set of eligible contracts for that user visit, and decides which
ad to serve based on the allocation plan.

The HWM algorithm is lightweight and fast, even in the offline
phase.  The DUAL algorithm, while lightweight and fast during the
online phase, is not nearly as fast in the offline phase.  (In our
experience, it is more than an order of magnitude slower.)  However,
it has the advantage of yielding provably optimal allocation during
serving, assuming that forecasts are perfect.  Of course, forecasts
are far from perfect in practice.  Thus, the benefits of allocation
using DUAL are somewhat mitigated.


The key to having a compact allocation plan is having a way to infer the the 
values at serving time without explicitly storing them.  This can be done by creating
an  fraction which represents the fraction of demand to be give to the contract,
and  which is the same for all neighboring supply nodes of a contract.  The mathematical basis
of this representation comes from the fact that an optimal solution can be represented
using these  values~\cite{EC}.  Although Linear Programming solvers can be
used to find an optimal solution, due to the size of the problem and time constraints,
we describe a much faster heuristic.

\subsection{HWM Algorithm}
\label{sec:offline} Offline optimization for HWM uses a simple, but
surprisingly effective heuristic for generating
the allocation plan.
The algorithm generates a serving rate for each contract , denoted , together with an
{\em allocation order}.  The allocation order is set so that a contract  with less eligible supply (i.e.
) has higher priority than those with more eligible supply.  By labeling each
contract with just two numbers, the HWM algorithm creates a compact and robust allocation plan.

During online serving, every contract gets an  fraction of
every impression, unless this would mean giving more than 100\% of
the impression away.  In this case, ties are broken in allocation
order: a contract  coming first in the allocation order will get
its  fraction, while a contract  coming later will get
whatever fraction is left, up to   (possibly getting
nothing for some impressions).

The algorithm itself labels every contract  with its eligible supply, denoted , which is used to
determine the allocation order; contracts with smaller  values come earlier in the allocation order.
In order to determine the serving rate,  for each , the HWM algorithm does the following steps.
\begin{enumerate}\itemsep=0in
\item Initialize remaining supply  for all .
\item For each contract , in allocation order, do:
    \begin{enumerate}\itemsep=0in
    \item Solve the following for :
        
        setting  if there is no solution.
    \item Update  for all .
    \end{enumerate}
\end{enumerate}
Notice that it is a simple manner to choose the allocation order using a different heuristic.  In our experience, the available
supply works quite well.  Contracts that have very little time left will also have very little supply and naturally become high
priority.  Likewise, highly targeted contracts also tend to have much less inventory.  This generally means they have less flexibility
in which user visits they can be served, and they are treated with higher priority.
Contrast this with a large, untargeted contract that can easily forego certain types of users visits when other highly targeted (and generally
more expensive) contracts may need those visits in order to satisfy their demands.

The compact allocation plan produced by the HWM algorithm is shown in Figure~\ref{fig:plan}.
Note that only the right hand side of the graph is sent to the Ad Servers.
In this example the allocation plan results in a feasible solution.
We note that the allocation plan is highly dependent on the supply forecast,
and an incorrect supply forecast may yield suboptimal results. We will explore this issue in detail in Section \ref{sec:robust}.

\begin{figure}[t]
\centering
\includegraphics[width=3in]{hwm_alloc.pdf}
\caption{An example compact allocation plan with the complete supply-demand graph. Note that only the data on the RHS of the figure is sent to the ad servers.}
\label{fig:plan}
\end{figure}

\subsubsection{Online Evaluation for HWM} \label{sec:online} Given an
allocation plan, which consists of the allocation order and serving
rate  for each contract , the role of the ad server is
to select one of the matching contracts. We present the algorithm
that the ad server follows below:
\begin{enumerate}\itemsep=0in
\item Given an impression , let  be the set of matching contracts listed in the allocation order.
\item If , let  be the maximum value so that . Finally, let . Note that by definition of , .
\item Select a contract  with probability  and the contract  with probability . Note that in the case that  with some probability no contract is selected.
\end{enumerate}

Consider the running example from Figure \ref{fig:plan}.
In the case an impression of type \texttt{\{CA, Age = 5\}} arrives, 
and the middle contract with allocation order  is always selected, since its serving rate is .
On the other hand, suppose an impression of type \texttt{\{Male, Age = 5\}} arrives.  Then with probability  it is allocated to contract 1, and with probability  it is allocated to contract 3, with the remaining probability () it is left unallocated.
(In case no contract is selected, the impression is then auctioned off in the Non-Guaranteed Marketplace.)

\subsection{The DUAL Algorithm}
The DUAL algorithm replaces the demand constraints described in
Section~\ref{sec:prob statement} with the following:

where we think of  as the under-delivery for contract .  We
additionally add the constraint that .  Our objective is
then

where  and  are fixed.  The first term attempts to
minimize the distance of the allocation from some target,
.  We set  to be the total demand of contract
 divided by its total (forecasted) supply.  In this way, we
attempt to make the overall mix of impressions as uniform as
possible.  (See~\cite{Informs,CIKM,Ghosh,KDD} for more discussion
on this.)

The second term represents the under-delivery penalty for the
contracts.  We set  (the penalty value per impression for
contract ) to be 10 for each contract.  This heuristically chosen
value performs well in practice.

In order to produce an allocation plan, DUAL simply reports the dual
values of the modified demand constraints for an optimal solution
--- in practice, these are actually the duals for an approximately
optimal solution.  Note that there is one dual value per contract.

\subsubsection{Online Evaluation for DUAL}
For each impression, we find the set of eligible contract.  This,
together with the dual values for their respective demand
constraints, is enough to compute the value of the primal solution
on the corresponding edges (i.e. the  values) computed
during the offline phase.  In fact, if the set of impressions seen
during serving is the same as the set of impressions used during the
offline phase, this method will faithfully reproduce exactly the
same allocation. Mathematically, define .  For an impression , let  be the set of
eligible contracts, and let  be the dual of the modified
demand constraint for each .  Then we first solve the
following for :

Then, we set .  (It turns out that  is
the dual value for the supply constraint for impression .)
Finally, we set  for all ; note that  for all .
As shown in~\cite{EC}, this reproduces the primal solution.

As with HWM, the fractional solution is interpreted as a probability
for serving.  That is, contract  is served to impression  with
probability .

\subsection{Robustness to Forecast Errors}
\label{sec:robust} Since rate-based algorithms produce serving rates
rather than absolute goals, it becomes much more robust to load
balancing, server failures, and so on.  However, it becomes more
sensitive to forecast errors.  For example, if the forecast is
double the true value, the serving rate is half what it should be,
roughly speaking. In this section, we show that in an idealized
setting, frequent re-optimization mitigates this problem greatly.
Note that this simple analysis applies to both HWM and DUAL.

Let us take a simple example in which our forecast is much higher
than reality for all contracts. Initially, all of the contracts are
underdelivering, since the actual supply is lower than the
forecasted supply, the  given in the allocation plan is too
low. As time passes, however, the rate begins to increase. This
increase is not because the supply forecast errors are recognized,
rather, as the expiration date for the contract approaches, the
urgency with which impressions should be served to the particular
contract increases.


\begin{table}[t]
\begin{center}
\caption{Increasing serving rate due to supply forecast errors}
\vspace{2pt}
{\begin{tabular}{|l|c|c|c|c|}
\hline
\multirow{2}{*}{Day} & Predicted & Remaining & Serving & Actual \\ & Supply & Demand & Rate & Delivery \\
\hline
1 & 5M & 2.5M & 0.50 & 0.4M \\
2 & 4M &    2.1M &  0.525 &     0.42M \\
3 & 3M & 1.68M &    0.56 &  0.45M \\
4 & 2M &    1.23M & 0.62 &  0.49M \\
5 & 1M & 0.74M &    0.74 &  0.59M\\
6 & &   0.15M & &  \\
\hline
\end{tabular}}
\label{tab:reopt}
\end{center}
\end{table}

To illustrate this point, consider a five day contract  with forecasted supply of  impressions per day, and
total demand of 2.5M impressions. If this is the only contract in the system, we should serve it at the rate of
. If the supply forecasts are perfect, then the contract will finish delivering exactly at the end of the last day. On the other hand, suppose that the actual supply is only 800K impressions per day (a 20\% supply forecasting error rate).  Then at the end of the first day the new serving rate would be recomputed to = .  Continuing with the example, we observe the behavior in Table \ref{tab:reopt}, which shows that after 5 days the contract would have underdelivered by .  Note that without this re-optimization, the underdelivery would be exactly 20\%, thus frequent reoptimizations help mitigate supply forecast errors.

We formalize the example above in the following theorem.
\begin{theorem}
Suppose we have  optimization cycles, the supply forecast error rate (defined
to be 1 minus the ratio of real supply to predicted supply) is , and the contract is never infeasible.
In the case that , the underdelivery is positive and bounded above by .
In the case that , the overdelivery is positive and bounded above by .
\end{theorem}
\begin{proof}
In the -th round, the optimizer will allocate a  fraction of
supply, of which a  fraction will be delivered. Therefore, the desired demand
will decrease by a factor of . Therefore, the supply available
 in the last round (i.e. the underdelivery) is:

Notice that when , this value is positive (i.e. meaning we underdeliver), while
when , this value is negative (i.e. meaning we overdeliver).
First, consider the case that .  We use the fact that
.
We have

Now,
consider the case that .  Here, we use the fact that .
Hence, we have

\end{proof}
So even a large 2X forecast error, leading to delivery rate of 0.5 of what it should be, does not result in delivering only 50\% of
the required demand.  On a week-long contract, re-optimizing every two hours, we have  cycles, and our under-delivery
is only .  Forecast errors in the other direction (which would lead to over-delivery) are even less severe.
A 0.5X forecast error, leading to delivery rate of twice what it should be, does not result on 100\% over-delivery.
Instead, the over-delivery is much less than 1\%.

\subsection{Feedback-Based Correction}
\label{sec:feedback}
In the previous section we showed how the supply forecast errors are mitigated by frequent reoptimization.
Although this guards against large underdeliveries, the offline optimization is slow to recognize the forecast error,
and the delivery rate increases very slowly in the beginning, even when faced with very large errors. An alternative solution is to
employ a feedback system that would recognize the errors and correct the allocations accordingly.

Such a feedback system falls squarely in the realm of control theory, and one can imagine many feedback controllers that would achieve this task.
We present a very simple protocol, which we will show performs quite well in practice.  The protocol is parameterized by two variables: , which
controls the allowed slack in delivery, and a pair  which control the boost to the serving rate.

If, during the lifetime of a contract,  the contract is delivering
more than  hours behind, the feedback system increases the
reported demand left for the contract by a factor of . For
example, let , and consider a 7 day contract for 70M
impressions. Ideally, the contract should deliver 3M impressions by
the end of day 3. If, say, only 2M impressions were still not
delivered by noon on day 4, the feedback system would increase the
demand left (5M = 7M-2M) by a factor of , making it
5M.
 If on the other hand, the contract had already delivered 3M impressions by noon on day 3, the feedback system would decrease
 its demand left by a factor of . We will show experimentally in Section \ref{sec:exp} that this, admittedly simple,
 system performs very well in real world scenarios.
