

\documentclass{sig-alternate-2013}
\usepackage{graphicx}
\usepackage{subfigure}
\usepackage{amsmath}
\usepackage{tabularx}
\usepackage{amssymb}
\usepackage{mathrsfs}
\usepackage{algorithm}
\usepackage{algorithmic}
\usepackage{multirow}
\usepackage{bm}

\newfont{\mycrnotice}{ptmr8t at 7pt}
\newfont{\myconfname}{ptmri8t at 7pt}
\let\crnotice\mycrnotice \let\confname\myconfname 

\permission{Permission to make digital or hard copies of all or part of this work for personal or classroom use is granted without fee provided that copies are not made or distributed for profit or commercial advantage and that copies bear this notice and the full citation on the first page. Copyrights for components of this work owned by others than ACM must be honored. Abstracting with credit is permitted. To copy otherwise, or republish, to post on servers or to redistribute to lists, requires prior specific permission and/or a fee. Request permissions from Permissions@acm.org.}
\conferenceinfo{MM'15,}{October 26--30, 2015, Brisbane, Australia.} 
\copyrightetc{\copyright~2015 ACM. ISBN \the\acmcopyr}
\crdata{978-1-4503-3459-4/15/10\ ...\^1^3^2^2^4^2^1^1^2^3^4\frac{5}{6}\frac{7}{6}105 \times105105 \times 105NKN\mathbf{C_u}K\mathbf{C_u}\mathbf{C_s}N-K\mathbf{C_u}\mathbf{C_s}\mathbf{C_u}\mathbf{C_s}N=8K=2\mathbf{C_u}\mathbf{C_s}\mathbf{C_u}K\mathbf{C_u}\mathbf{C_s}\mathbf{C_u}\mathbf{C_u}\mathbf{C_s}105 \times 105K=N-1K=1K=1K=N-1\mathbf{C_u}\mathbf{C_s}KN-KW \in R^{m \times n}U \in R^{m \times m}V \in R^{n \times n}S \in R^{m \times n}W\widetilde{U}\widetilde{V}\widetilde{S}kUVkSmnk\frac{k(m + n + 1)}{mn}m, n \gg kSWkW\mathcal{T}_kkSW_kkWkKNNKC_uKNKKKKKKKKKKKKC_sC_uKC_uC_sN-KC_uKNC_sNKKNK\mathbf{F}\mathbf{C_u}\mathbf{C_u}\mathbf{C_s}4096 \times 1Wkkkkkkk$ = 10) to the fc6 parameter matrix. The obtained ``mini'' model, with only 9, 477, 066 parameters and a high compression ratio of 18.73, becomes less than 40 megabytes in storage. Being portable even on mobiles, It manages to keep a top-5 error rate around 22\%. 



\section{Conclusion}

In the paper, we develop the DeepFont system to remarkably advance the state-of-the-art in the VFR task. A large set of labeled real-world data as well as a large corpus of unlabeled real-world images is collected for both training and testing, which is the first of its kind and will be made publicly available soon. While relying on the learning capacity of CNN, we need to combat the mismatch between available training and testing data. The introduction of SCAE-based domain adaption helps our trained model achieve a higher than 80\% top-5 accuracy. A novel lossless model compression is further applied to promote the model storage efficiency. The DeepFont system not only is effective for font recognition, but can also produce a font similarity measure for font selection and suggestion. 


\bibliographystyle{abbrv}
\normalsize
\bibliography{sigproc}  


\end{document}
