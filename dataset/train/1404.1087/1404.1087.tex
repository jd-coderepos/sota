

In this section, we present a reallocation algorithm that guarantees  reallocations amortized on rounds.
The details of the protocol can be found in Algorithm~\ref{alg:O(1)realloc}. 
Bounds on channel usage and reallocations are proved in Theorem~\ref{thm:O(1)realloc}. 
Corollary~\ref{corollary} establishes these bounds in our objective function. 
For convenience, for any number , we define the \defn{hyperceiling} of , denoted as , to be the smallest power of  that is not smaller than .
For this algorithm, we restrict the input to laxities that are powers of . The study of inputs with arbitrary laxities is left for future work.

The intuition of the protocol is the following. 
When a new client  arrives and there are already  clients in the system, for , we distinguish two cases.
If the laxity of  is at least , assign  to a special channel called \defn{big channel}. All clients allocated to the big channel transmit with period  so, because there are  clients in the system, one big channel is enough. All the other channels being used are called \defn{small channels}.
Otherwise, if the laxity is , assign client  to a channel reserved for laxities , we call it \defn{-channel}. If such channel does not exist or all -channels are full, reserve a new one.
For any laxity , all clients allocated to a -channel transmit with period . That is, a maximum of  clients can be allocated to a -channel.
When a client  of laxity  leaves a channel , if  is the big channel do nothing.
Otherwise, reallocate a client from the -channel of minimum load (if any other) to the slot left by .

With each arrival or departure the number of clients  change.
If, upon an arrival,  becomes larger than the laxity of some clients allocated to a big channel, reallocate those clients to other channels according to laxity, reserving new channels if necessary.
Because  was doubled since the allocation of these clients, these reallocations are amortized by the arrivals that doubled .
If, upon a departure,  becomes smaller than the laxity of some clients, reallocate those clients to a big channel, releasing the reservation of the channels that become empty.
Because  was halved since the allocation of these clients, these reallocations are amortized by the departures that halved .



\begin{algorithm}[tbp]
\label{alg:O(1)realloc}
\caption{ reallocations.  is the largest power of  that is not greater than . For any laxity , all clients allocated to a -channel are scheduled to transmit with period .}
\small
\SetKwFor{Upon}{upon}{do}{endupon}
\SetKwFor{Task}{Task}{}{endtask}
\DontPrintSemicolon
\label{init}\tcp*[f]{active clients count}\;
   \label{init}\tcp*[f]{big channel threshold}\;
   \textbf{start} tasks  and \;
\Task{1}{
\Upon{arrival of client }{
   \label{arrival}\;
   \If(\tcp*[f]{consolidate the big channel}){}{ \label{line:bigoutbegin}
      \;
         \ForEach{client  in the big channel such that }{\label{line:bigoutmid}
            \lIf{all -channels are full}{reserve a new -channel}
            reallocate  to the -channel of minimum load\;\label{line:bigoutend}\;
         }
         \ForEach{client  in the big channel}{
            re-schedule  to transmit with period \;
         }
   }
   \If(\tcp*[f]{allocate new client}){}{ \label{line:biguponarrival}
      allocate  to the big channel to transmit with period \;
   }\Else{\label{line:smalluponarrival}
      \lIf{all -channels are full}{reserve a new -channel}
      allocate  to the -channel of minimum load\; 
   }
}
}
\Task{2}{
\Upon{departure of client  from channel }{
   \label{departure}\;
   \If(\tcp*[f]{consolidate -channels}){ is not the big channel}{
      \lIf{ is empty}{release }
      \ElseIf{there is a -channel  that is not full}{ \label{line:simplebegin}
            reallocate a client from  to \;\label{line:simpleend}
      }
   }
   \If(\tcp*[f]{consolidate the big channel}){}{ 
      \;
         \ForEach{client  in the big channel}{
            re-schedule  to transmit with period \;
         }
      \ForEach{-channel  such that }{\label{line:biguponconsolidation}
            \ForEach{client  in }{
               reallocate  to the big channel to transmit with period \;\label{line:biguponconsolidationmid}
            }
            release \;\label{line:biguponconsolidationend}
      }
   }
}
}
\end{algorithm}



The following theorem bounds the number of reallocations and the number of channels used by the \constant algorithm.


\begin{theorem}
\label{thm:O(1)realloc}
Given a set of clients , the schedule  obtained by the \constant algorithm requires at most  reallocations up to round . Additionally, for any round  such that , the number of reserved channels is at most 

where  is the minimum number of channels required to allocate the clients in .
\end{theorem}



\begin{proof}
The bound on the number of channels follows from the algorithm. Specifically, for any round , there are at most  -channels full, and there is at most one big channel. With respect to the not-full -channels, the maximum  is either  or , whatever is smaller. Given that all laxities are powers of  the bound follows.

To bound the reallocations, we map reallocations to arrivals or departures. Given that rounds are defined by arrivals and departures, the amortized bound follows. The mapping is the following. Clients are reallocated due to one of three possible events as follows. 

\begin{enumerate}
\item\label{event1}
Some client  with laxity  departed from a -channel  that was full. Then, if there is some other -channel  that is not full, some client  allocated to  is reallocated to the slot left by  in  (see Lines~\ref{line:simplebegin}-\ref{line:simpleend} in Algorithm~\ref{alg:O(1)realloc}). 
\item\label{event2}
Upon the arrival of some client, the total number of clients  increases making  larger than the big-channel threshold. Then, any client  allocated to the big channel whose laxity is  is reallocated to a -channel (see Lines~\ref{line:bigoutbegin}-\ref{line:bigoutend} in Algorithm~\ref{alg:O(1)realloc}). 
\item\label{event3}
Upon the departure of some client, the total number of clients  decreases making  smaller than the big-channel threshold. Then, all clients in all -channels such that  are reallocated to the big channel (see Lines~\ref{line:biguponconsolidation}-\ref{line:biguponconsolidationend} in Algorithm~\ref{alg:O(1)realloc}). 
\end{enumerate}
No other event triggers a reallocation. 

Now, we define the mapping. For Event~\ref{event1}, the reallocation of  is mapped to the departure of . To define the mapping for Event~\ref{event2}, we need the following lemmas.

\begin{lemma}
\label{claim:doubled}
Consider a client  that has to be reallocated from the big channel in Lines~\ref{line:bigoutmid}-\ref{line:bigoutend} of Algorithm~\ref{alg:O(1)realloc}. 
Let  be the number of clients in the system at the time of reallocation in Line~\ref{line:bigoutend}, and
 be the number of clients in the system at the time of the last allocation of .
Then, it is .
That is, after the last allocation of  to the big channel, the total number of clients in the system at least has doubled.
\end{lemma}
\begin{proof}
In the following, all line numbers refer to Algorithm~\ref{alg:O(1)realloc}.
Clients are reallocated from the big channel because their laxities are strictly smaller than  (see Line~\ref{line:bigoutmid}).
In order to be allocated to the big channel,  must have a laxity at least  if it was upon arrival (see Line~\ref{line:biguponarrival}), or at least  if it was upon consolidation of the big channel (see Line~\ref{line:biguponconsolidation}). In either case, it must be , which implies that .
\qed\end{proof}

\begin{lemma}
\label{claim:bigout}
In any given round, at most half of the clients in the system are reallocated from the big channel after executing Lines~\ref{line:bigoutmid}-\ref{line:bigoutend} of Algorithm~\ref{alg:O(1)realloc}.
\end{lemma}
\begin{proof}
Out of the set of clients being reallocated from the big channel at a given round , consider the client  that was allocated last, say, in some round .
From Lemma~\ref{claim:doubled}, we know that between  and  the number of clients in the system has at least doubled. Furthermore, we know that none of the clients that arrived after  has to be reallocated in round , because  was the last one. Hence, it cannot be that more than half of the clients in the system are reallocated from the big channel in round .
\qed\end{proof}

Now we define the mapping for Event~\ref{event2}. Let  be the number of clients in the system at the time of Event~\ref{event2}. As shown in Lemma~\ref{claim:bigout}, \emph{at most}  clients have to be reallocated. And, as shown in Lemma~\ref{claim:doubled}, after the last allocation to the big channel of any client that has to be reallocated, the total number of clients in the system \emph{at least} has doubled. Hence, the at most  reallocations are mapped to the at least  arrivals.

To define the mapping for Event~\ref{event3}, we need the following lemma.

\begin{lemma}
\label{claim:halved}
Consider a client  that has to be reallocated to the big channel in Lines~\ref{line:biguponconsolidation}-\ref{line:biguponconsolidationend} of Algorithm~\ref{alg:O(1)realloc}. 
Let  be the number of clients in the system at the time of reallocation in Line~\ref{line:biguponconsolidationmid}, and
 be the number of clients in the system at the time of the last allocation of .
Then, it is .
That is, after the last allocation of , the total number of clients in the system at least has halved.
\end{lemma}
\begin{proof}
In the following, all line numbers refer to Algorithm~\ref{alg:O(1)realloc}.
Clients are reallocated to the big channel because their laxities are strictly larger than  (see Line~\ref{line:biguponconsolidation}).
Because  was not in the big channel,  must have a laxity strictly smaller than  if the last allocation was upon arrival (see Line~\ref{line:smalluponarrival}), or strictly smaller than  if it was upon consolidation of the big channel (see Line~\ref{line:bigoutmid}). 
In either case, it must be , which implies that .
\qed\end{proof}

Finally, the mapping for Event~\ref{event3} is the following. Let  be the number of clients in the system at the time of Event~\ref{event3}. As shown in Lemma~\ref{claim:halved}, after the last allocation of any client that has to be reallocated, the total number of clients in the system at least has halved. Hence, the at most  reallocations are mapped to the at least  departures.

In the mapping above, there is at most one reallocation for each arrival and at most two reallocations for each departure. Given that there cannot be more departures than arrivals, the number of reallocations up to round  are at most . Hence, the claim follows.
\qed\end{proof}



The following corollary is a direct consequence of Theorem~\ref{thm:O(1)realloc} and the fact that .

\begin{corollary}
\label{corollary}
Given a set of clients , the schedule  obtained by the \constant algorithm achieves, for any round , is
.
\end{corollary}




