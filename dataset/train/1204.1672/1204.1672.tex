\documentclass{llncs}

\usepackage{latexsym,amsmath,amssymb}
\usepackage{graphicx,epsfig}
\usepackage{url}
\usepackage{multirow}
\usepackage[nocompress]{cite}
 
\newcommand{\Suff}{\textit{Suff}}
\newcommand{\Pref}{\textit{Pref}}
\newcommand{\Fact}{\textit{Fact}}
\renewcommand{\alph}{\textit{alph}}
\newcommand{\St}{\textit{St}}
\newcommand{\LS}{\textit{LS}}
\newcommand{\RS}{\textit{RS}}
\newcommand{\BS}{\textit{BS}}
\newcommand{\SBS}{\textit{SBS}}
\newcommand{\WBS}{\textit{NBS}}
\newcommand{\MF}{\textit{MFSt}}

\renewcommand{\arraystretch}{1.4}

\renewcommand{\epsilon}{\varepsilon}

\newcommand{\todo}[1]{\marginpar{\small #1}}

\sloppy

\begin{document}

\title{A Characterization of Bispecial \\ Sturmian Words}


\author{Gabriele Fici}

\authorrunning{G. Fici}

\institute{I3S, CNRS \& Universit\'e Nice Sophia Antipolis, France \\ \email{fici@i3s.unice.fr} }

\maketitle

\begin{abstract}
A finite Sturmian word  over the alphabet  is left special (resp.~right special)  if  and  (resp.~ and ) are both Sturmian words. A bispecial Sturmian word is a Sturmian word that is both left and right special. We show as a main result that bispecial Sturmian words are exactly the maximal internal factors of Christoffel words, that are words coding the digital approximations of segments in the Euclidean plane. This result is an extension of the known relation between central words and primitive Christoffel words. Our characterization allows us to give an enumerative formula for bispecial Sturmian words. We also investigate the minimal forbidden words for the set of Sturmian words.
\end{abstract}

\keywords Sturmian words, Christoffel words, special factors, minimal forbidden words, enumerative formula.


\section{Introduction}\label{sec:intro}


Sturmian words are non-periodic infinite words of minimal factor complexity. They are characterized by the property of having exactly  distinct factors of length  for every  (and therefore are binary words) \cite{MoHe40}. As an immediate consequence of this property, one has that in any Sturmian word there is a unique factor for each length  that can be extended to the right with both letters into a factor of length . These factors are called \emph{right special factors}. Moreover, since any Sturmian word is recurrent (every factor appears infinitely often) there is a unique factor for each length  that is left special, i.e., can be extended to the left with both letters into a factor of length .

The set  of finite factors of Sturmian words coincides with the set of binary \emph{balanced} words, i.e., binary words having the property that any two factors of the same length have the same number of occurrences of each letter up to one. These words are also called (finite) Sturmian words and have been extensively studied because of their relevant role in several fields of theoretical computer science.

If one considers extendibility within the set , one can define \emph{left special Sturmian words} (resp.~\emph{right special Sturmian words}) \cite{DelMi94} as those words  over the alphabet  such that  and  (resp.~ and ) are both Sturmian words. For example, the word  is left special since  and  are both Sturmian words, but is not right special since  is not a Sturmian word.

Left special Sturmian words are precisely the binary words having suffix automaton\footnote{The suffix automaton of a finite word  is the minimal deterministic finite state automaton accepting the language of the suffixes of .} with minimal state complexity (cf.~\cite{SciZa07,Fi10b}). From combinatorial considerations one has that right special Sturmian words are the reversals of left special Sturmian words.

The Sturmian words that are both left special and right special are called \emph{bispecial Sturmian words}. They are of two kinds: \emph{strictly bispecial Sturmian words}, that are the words  such that , ,  and  are all Sturmian words, or \emph{non-strictly bispecial Sturmian words} otherwise. Strictly bispecial Sturmian words have been deeply studied (see for example~\cite{DelMi94,CarDel05}) because they play a central role in the theory of Sturmian words. They are also called \emph{central words}. Non-strictly bispecial Sturmian words, instead, received less attention.

One important field in which Sturmian words arise naturally is discrete geometry. Indeed, Sturmian words can be viewed as digital approximations of straight lines in the Euclidean plane. It is known that given a point  in the discrete plane , with , there exists a unique path that approximates from below (resp.~from above) the segment joining the origin  to the point . This path, represented as a concatenation of horizontal and vertical unitary segments, is called the \emph{lower (resp.~upper) Christoffel word} associated to the pair . If one encodes horizontal and vertical unitary segments with the letters  and  respectively, a lower (resp.~upper) Christoffel word is always a word of the form  (resp.~), for some . If (and only if)  and  are coprime, the associated Christoffel word is primitive (that is, it is not the power of a shorter word). It is known that a word  is a strictly bispecial Sturmian word if and only if  is a primitive lower Christoffel word (or, equivalently, if and only if  is a primitive upper Christoffel word). As a main result of this paper, we show that this correspondence holds in general between bispecial Sturmian words and Christoffel words. That is, we prove (in Theorem \ref{theor:main}) that  is a bispecial Sturmian word if and only if there exist letters  in  such that  is a Christoffel word. 

This characterization allows us to prove an enumerative formula for bispecial Sturmian words (Corollary \ref{cor:formula}): there are exactly  bispecial Sturmian words of length , where  is the Euler totient function, i.e.,  is the number of positive integers smaller than or equal to  and coprime with . It is worth noticing that enumerative formulae for left special, right special and strictly bispecial Sturmian words were known \cite{DelMi94}, but to the best of our knowledge we exhibit the first proof of an enumerative formula for non-strictly bispecial (and therefore for bispecial) Sturmian words.

We then investigate minimal forbidden words for the set  of finite Sturmian words. The set of \emph{minimal forbidden words} of a factorial language  is the set of words of minimal length that do not belong to  \cite{MiReSci02}. Minimal forbidden words represent a powerful tool to investigate the structure of a factorial language (see~\cite{BeMiRe96}).
We give a characterization of minimal forbidden words for the set of Sturmian words in Theorem \ref{theor:mf}. We show that they are the words of the form  such that  is a non-primitive Christoffel word, where . This characterization allows us to give an enumerative formula for the set of minimal forbidden words (Corollary \ref{cor:formulamf}):  there are exactly  minimal forbidden words of length  for every .

The paper is organized as follows. In Sec.\ \ref{sec:wsf} we recall standard definitions on words and factors. In Sec.\ \ref{sec:StCh} we deal with Sturmian words and Christoffel words, and present our main result. In Sec.\ \ref{sec:En} we give an enumerative formula for bispecial Sturmian words. Finally, in Sec.\ \ref{sec:MF}, we investigate minimal forbidden words for the language of finite Sturmian words. 

\section{Words and special factors}\label{sec:wsf}


Let  be a finite alphabet, whose elements are called letters. A word over  is a finite sequence of letters from . A right-infinite word over  is a non-ending sequence of letters from . The set of all words over  is denoted by . The set of all words over  having length  is denoted by . The empty word has length zero and is denoted by . For a subset  of , we note . Given a non-empty word , we let  denote its -th letter. The reversal of the word , with  for , is the word . We set . A palindrome is a word  such that . A word is called a power if it is the concatenation of copies of another word; otherwise it is called primitive. For a letter ,  is the number of 's appearing in .  A word  has period , with , if  for every . Since  is always a period of , every non-empty word has at least one period. 

A word  is a factor of a word  if  for some . In the special case  (resp.~), we call  a prefix (resp.~a suffix) of . We let ,  and  denote, respectively, the set of prefixes, suffixes and factors of the word . The factor complexity of a word  is the integer function ,  . 

A factor  of a word  is left special (resp.~right special) in  if there exist , , such that  (resp.~). A factor  of  is bispecial in  if it is both left and right special. In the case when , a bispecial factor  of  is said to be strictly bispecial in  if  are all factors of ; otherwise  is said to be non-strictly bispecial in . For example, let . The left special factors of  are , , ,  and . The right special factors of  are , ,  and . Therefore, the bispecial factors of  are , ,  and . Among these, only  is strictly bispecial.

In the rest of the paper we fix the alphabet .

\section{Sturmian words and Christoffel words}\label{sec:StCh}


A right-infinite word  is called a Sturmian word if  for every , that is, if  contains exactly  distinct factors of length  for every . Sturmian words are non-periodic infinite words of minimal factor complexity \cite{CoHe73}.
A famous example of infinite Sturmian word is the Fibonacci word  obtained as the limit of the substitution , . 

A finite word is called Sturmian if it is a factor of an infinite Sturmian word. Finite Sturmian words are characterized by the following balance property \cite{DuGB90}: a finite word  over  is Sturmian if and only if for any  such that  one has  (or, equivalently, ). We let  denote the set of finite Sturmian words. The language  is factorial (if , then ) and extendible (for every  there exist letters  such that ).

Let  be a finite Sturmian word. The following definitions are in \cite{DelMi94}. 

\begin{definition}
A word   is a  left special (resp.~right special) Sturmian word if  (resp.~if ). A bispecial Sturmian word is a Sturmian word that is both left special and right special. Moreover, a bispecial Sturmian word is strictly bispecial if  and  are all Sturmian word; otherwise it is non-strictly bispecial. 
\end{definition}

We let , , ,  and  denote, respectively, the sets of left special, right special, bispecial, strictly bispecial and non-strictly bispecial Sturmian words. Hence, .

The following lemma is a reformulation of a result of de Luca \cite{Del97}.

\begin{lemma}\label{lem:prefsuf}
Let  be a word over . Then  (resp.~) if and only if  is a prefix (resp.~a suffix) of a word in .
\end{lemma}

Given a bispecial Sturmian word, the simplest criterion to determine if it is strictly or non-strictly bispecial  is provided by the following nice characterization \cite{DelMi94}:

\begin{proposition}\label{prop:sturmstrispe}
 A bispecial Sturmian word is strictly bispecial if and only if it is a palindrome.
\end{proposition}

Using the results in \cite{DelMi94}, one can derive the following classification of Sturmian words with respect to their extendibility.

\begin{proposition}\label{prop:bisp}
 Let  be a Sturmian word. Then:
 
\begin{itemize}
\item   if and only if  is strictly bispecial;
\item   if and only if  is non-strictly bispecial;
\item   if and only if  is left special or right special but not bispecial;
\item   if and only if  is neither left special nor right special.
\end{itemize}
\end{proposition}

\begin{example}
The word  is a strictly bispecial Sturmian word, since , ,  and  are all Sturmian words, so that . The word  is a bispecial Sturmian word since , ,  and  are Sturmian words. Nevertheless,  is not Sturmian, since it contains  and  as factors. So  is a non-strictly bispecial Sturmian word, and . The Sturmian word  is left special but not right special, and . Finally, the Sturmian word  is neither left special nor right special, the only word in  being .
\end{example}

We now recall the definition of central word \cite{DelMi94}.

\begin{definition}
A word over  is central if it has two coprime periods  and  and length equal to .
\end{definition}

A combinatorial characterization of central words is the following (see~\cite{Del97}):

\begin{proposition}\label{prop:charcentral}
A word  over  is central if and only if  is the power of a single letter or there exist palindromes  such that , for different letters . Moreover, if , then  is the longest palindromic suffix of .
\end{proposition}

Actually, in the statement of Proposition \ref{prop:charcentral}, the requirement that the words  and   are palindromes is not even necessary \cite{CarDel05}.

We have the following remarkable result \cite{DelMi94}:

\begin{proposition}\label{prop:sbscen}
A word over  is a strictly bispecial Sturmian word if and only if it is a central word.
\end{proposition}

Another class of finite words, strictly related to the previous ones, is that of Christoffel words.

\begin{definition}
Let  and  be integers such that . The lower Christoffel word  is the word defined for  by

\end{definition}

\begin{example}
 Let  and . We have  mod. Hence, .
\end{example}

Notice that for every , there are exactly  lower Christoffel words , corresponding to the  pairs  such that  and . 

\begin{remark}
In the literature, Christoffel words are often defined with the additional requirement that  (cf.~\cite{Book08}). We call such Christoffel words primitive, since a Christoffel word is a primitive word if and only if .
\end{remark}

If one draws a word in the discrete grid  by encoding each  with a horizontal unitary segment and each  with a vertical unitary segment, the lower Christoffel word  is in fact the best grid approximation from below of the segment joining  to , and has slope , that is,  and   (see~\figurename~\ref{fig:GC}).

\begin{figure}
\begin{center}
\begin{minipage}{5.7cm}
\includegraphics[height=40mm]{GenChristoffel.pdf}
\end{minipage}
\begin{minipage}{5.7cm}
\includegraphics[height=40mm]{GenChristoffel2.pdf}
\end{minipage}
\end{center}
\caption{The lower Christoffel word  (left) and the upper Christoffel word  (right).\label{fig:GC}}
\end{figure}

Analogously, one can define the upper Christoffel word  by

Of course, the upper Christoffel word  is the best grid approximation from above of the segment joining  to  (see \figurename~\ref{fig:GC}).

\begin{example}
 Let  and . We have  mod. Hence, .
\end{example}

The next result follows from elementary geometrical considerations.

\begin{lemma}\label{lem:rev}
 For every pair  the word  is the reversal of the word .
\end{lemma}

If (and only if)  and  are coprime, the Christoffel word  intersects the segment joining  to  only at the end points, and is a primitive word. Moreover, one can prove that  and  for a palindrome . Since  is a bispecial Sturmian word and it is a palindrome,   is a strictly bispecial Sturmian word (by Proposition \ref{prop:sturmstrispe}). Conversely, given a strictly bispecial Sturmian word ,  is a central word (by Proposition \ref{prop:sbscen}), and therefore has two coprime periods  and length equal to . Indeed, it can be proved that  and . The previous properties can be summarized in the following theorem (cf.~\cite{BeDel97}):

\begin{theorem}\label{theor:sbsCP}
  x,y\in \Sigma\}
\end{theorem}

If instead  and  are not coprime, then there exist coprime integers  such that , , for an integer . In this case, we have , that is,  is a power of a primitive Christoffel word. Hence, there exists a central Sturmian word  such that  and . So, we have:

\begin{lemma}\label{lem:npChris}
The word , , is a Christoffel word if and only if , for an integer  and a central word . Moreover,  is a primitive Christoffel word if and only if .
\end{lemma}

Recall from \cite{Del97} that the right (resp.~left) palindromic closure of a word  is the (unique) shortest palindrome  (resp.~) such that  is a prefix of  (resp.~a suffix of ). If  and  is the longest palindromic suffix of  (resp.~ is the longest palindromic prefix of ), then  (resp.~). 

\begin{lemma}\label{lem:rpl}
 Let  be a Christoffel word, . Then   and   are central words.
\end{lemma}

\begin{proof}
 Let  be a Christoffel word, .  By Lemma \ref{lem:npChris}, , for an integer  and a central word . We prove the claim for the right palindromic closure, the claim for the left palindromic closure will follow by symmetry. If , then , so  is a palindrome and then  is a central word. So suppose . We first consider the case when  is the power of a single letter (including the case ). We have that either  or  for some . In the first case, , whereas in the second case . In both cases one has that  is a strictly bispecial Sturmian word, and thus, by Proposition \ref{prop:sbscen}, a central word.

Let now  be not the power of a single letter. Hence, by Proposition \ref{prop:charcentral}, there exist palindromes  such that . Now, observe that 

We claim that the longest palindromic suffix of  is . Indeed, the longest palindromic suffix of  cannot be  itself since  is not a palindrome, so since any  palindromic suffix of  longer than  must start in , in order to prove the claim it is enough to show that the first non-prefix occurrence of  in  is that appearing as a prefix of . Now, since the prefix  of  can be written as , one has by Proposition \ref{prop:charcentral} that  is a central word. It is easy to prove (see, for example, \cite{BuDelFi12}) that the longest palindromic suffix of a central word does not have internal occurrences, that is, appears in the central word only as a prefix and as a suffix. Therefore, since ,  is the longest palindromic suffix of  (by Proposition \ref{prop:charcentral}), and so appears in  only as a prefix and as a suffix. This shows that  is the longest palindromic suffix of .

Thus, we have , and we can write: 

so that  for the palindrome . By Proposition \ref{prop:charcentral},  is a central word.\qed
\end{proof}

We are now ready to state our main result.

\begin{theorem}\label{theor:main}
x,y\in \Sigma\}
\end{theorem}

\begin{proof}
 Let  be a Christoffel word, .  Then, by Lemma \ref{lem:npChris},  is of the form , , for a central word . By Lemma \ref{lem:rpl},  is a prefix of the central word  and a suffix of the central word , and therefore, by Proposition \ref{prop:sbscen} and Lemma \ref{lem:prefsuf},  is a bispecial Sturmian word.
 
 Conversely, let  be a bispecial Sturmian word, that is, suppose that the words , ,  and  are all Sturmian. If  is strictly bispecial, then  is a central word by Proposition \ref{prop:sbscen}, and  is a (primitive) Christoffel word by Theorem \ref{theor:sbsCP}. So suppose . By Lemma \ref{lem:npChris}, it is enough to prove that  is of the form , , for a central word  and letters . Since  is not a strictly bispecial Sturmian word, it is not a palindrome (by Proposition \ref{prop:sturmstrispe}). Let  be the longest palindromic border of  (that is, the longest palindromic prefix of  that is also a suffix of ), so that , , . If ,  and we are done. Otherwise, it must be  for some , since otherwise either the word  would contain  and  as factors (a contradiction with the hypothesis that  is a Sturmian word) or the word  would contain  and  as factors (a contradiction with the hypothesis that  is a Sturmian word). 
 
So . If , then it must be  for some , since otherwise either   would appear as a factor in , and therefore the word  would contain  and  as factors, being not a Sturmian word, or  would appear as a factor in , and therefore the word  would contain  and  as factors, being not a Sturmian word. Hence, if  we are done, and so we suppose . 
 
 By contradiction, suppose that  is not of the form . That is, let , with , , , for different letters  and . If , then either  or . In the first case,  and , for some , and then the word  would contain  and  as factors, being not a Sturmian word. In the second case,  and , for some ; since  is a suffix of , and therefore  for some , we would have that the word  contains both  and  as factors, being not a Sturmian word. Thus, we can suppose . Now, if  and , then the word  would contain the factors  and , being not a Sturmian word; if instead  and , let , so that we can write . The word  would therefore contain the factors  and  (since  is a suffix of ), being not a Sturmian word (see~\figurename~\ref{fig:theorem}). In all the cases we obtain a contradiction and the proof is thus complete.\qed
\end{proof}

\begin{figure}[ht]
\begin{center}
\includegraphics[height=32mm]{theorem.pdf}
\caption{The proof of Theorem \ref{theor:main}.}
\label{fig:theorem}
\end{center}
\end{figure}

So, bispecial Sturmian words are the maximal internal factors of Christoffel words. Every bispecial Sturmian word is therefore of the form , , for different letters  and a central word . The word  is strictly bispecial if and only if . If ,  is a \emph{semicentral word} \cite{BuDelFi12}, that is, a word in which the longest repeated prefix, the longest repeated suffix, the longest left special factor and the longest right special factor all coincide.

\section{Enumeration of bispecial Sturmian words}\label{sec:En}


In this section we give an enumerative formula for bispecial Sturmian words. It is known that the number of Sturmian words of length  is given by

where  is the Euler totient function, i.e.,  is the number of positive integers smaller than or equal to  and coprime with  (cf.~\cite{Mig91,Lip82}).

Let  be a Sturmian word of length . If  is left special, then  and  are Sturmian words of length . If instead  if not left special, then only one between  and  is a Sturmian word of length . Therefore, we have  and hence


Using a symmetric argument, one has that also 

Since \cite{DelMi94} , we have


Therefore, in order to find an enumerative formula for bispecial Sturmian words, we only have to enumerate the non-strictly bispecial Sturmian words. We do this in the next proposition.

\begin{proposition}
For every , one has 
\end{proposition}

\begin{proof}
Let  and  By Theorem \ref{theor:main}, the bispecial Sturmian words of length  are the words in . 

Among the  words in , there are  strictly bispecial Sturmian words, that are precisely the palindromes in . The  words in  that are not palindromes are non-strictly bispecial Sturmian words. The other non-strictly bispecial Sturmian words of length  are the  words in  that are not palindromes. Since the words in  are the reversals of the words in , and since no non-strictly bispecial Sturmian word is a palindrome by Proposition \ref{prop:sturmstrispe}, there are a total of  non-strictly bispecial Sturmian words of length .\qed
\end{proof}

\begin{corollary}\label{cor:formula}
 For every , there are  bispecial Sturmian words of length .
\end{corollary}

\begin{example}
The Christoffel words of length  and their maximal internal factors, the bispecial Sturmian words of length , are reported in Table \ref{tab:example}. 

\begin{table}[h]
\begin{center}
  \begin{tabular}{|c | c | c | }
  
\ pair  \ & \ lower Christoffel word  \ & \ upper Christoffel word  \  \\    \hline 
  &       &         \\
  &       &       \\
   &       &        \\
   &       &        \\
   &       &        \\
   &       &     \\
   &       &        \\
   &       &       \\
   &       &        \\
  &       &        \\
  &       &        \\    
 \hline 
  \end{tabular}\vspace{4mm}
\end{center}\caption{The Christoffel words of length . Their maximal internal factors are the bispecial Sturmian words of length . There are  strictly bispecial Sturmian words, that are the palindromes , ,  and  (underlined), and  non-strictly bispecial Sturmian words: , , , , , , , , , , , ,  and  .\label{tab:example}}
\end{table}
\end{example}

\section{Minimal forbidden words}\label{sec:MF}


Given a factorial language  (that is, a language containing all the factors of its words) over the alphabet , a word  is a minimal forbidden word for  if  does not belong to  but every proper factor of  does (see~\cite{CrMiRe98} for further details). Minimal forbidden words represent a powerful tool to investigate the structure of a factorial language (cf.~\cite{BeMiRe96}). In the next theorem, we give a characterization of the set  of minimal forbidden words for the language . 


\begin{theorem}\label{theor:mf}
 x,y\in \Sigma
\end{theorem}

\begin{proof}
If  is a non-primitive Christoffel word, then by Theorems \ref{theor:sbsCP} and  \ref{theor:main},  is a non-strictly bispecial Sturmian word. This implies that  is not a Sturmian word, since a word  such that  and  are both Sturmian is a central word \cite{DelMi94}, and therefore a strictly bispecial Sturmian word (Proposition \ref{prop:sbscen}). Since  and  are Sturmian words, we have .

Conversely, let . By definition,  is Sturmian, and therefore  must be a  Sturmian word since  is an extendible language. Analogously, since  is Sturmian, the word  must be a Sturmian word. Thus,  is a bispecial Sturmian word, and since ,  is a  non-strictly bispecial Sturmian word. By Theorems \ref{theor:sbsCP} and \ref{theor:main},  is a non-primitive Christoffel word.\qed
\end{proof}

\begin{corollary}\label{cor:formulamf}
 For every , one has 
\end{corollary}

It is known from \cite{Mig91} that , as a consequence of the estimation (see~\cite{HaWr}, p. 268)

From (\ref{eq:HW}) and from the formula of Corollary \ref{cor:formulamf}, we have that 


\bibliographystyle{abbrv}
\bibliography{bisturmian}
\end{document}
