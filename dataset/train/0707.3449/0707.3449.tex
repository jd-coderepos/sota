\documentclass[11pt,a4paper]{article}
\usepackage{amsfonts,amsmath,amsthm,mathbbol,epsf,graphics,verbatim,amssymb,amscd,eucal,bbm}

\oddsidemargin=0pt \evensidemargin=0pt \topmargin=-38pt
\textwidth=453pt \textheight=690pt

\tolerance=1000
\parindent=0pt \parskip=\smallskipamount

\newcommand{\bdl}[2]
{\mbox{}}


\newcommand{\bydef}{\stackrel{\mbox{def}}{=}}
\newtheorem{theorem}{Theorem}[section]
\newtheorem{corollary}[theorem]{Corollary}
\newtheorem{proposition}[theorem]{Proposition}
\newtheorem{lemma}[theorem]{Lemma}
\newtheorem{defiprop}[theorem]{Definition-Proposition}
\newtheorem{definition}[theorem]{Definition}
\newtheorem{conjecture}[theorem]{Conjecture}
\theoremstyle{remark}
\newtheorem{remark}[theorem]{Remark}
\newtheorem{example}[theorem]{Example}
\def\Blackboardfont{\mathbb}


\newcommand{\bU}{ {\bf U} }
\newcommand{\bV}{ {\bf V} }
\newcommand{\bC}{ {\bf C} }
\newcommand{\bF}{ {\bf F} }
\newcommand{\bX}{ {\bf X} }
\newcommand{\bD}{ {\bf D} }
\newcommand{\bI}{ {\bf I} }
\newcommand{\bN}{ {\bf N} }
\newcommand{\bP}{ {\bf P} }
\newcommand{\bZ}{ {\bf Z} }
\newcommand{\bY}{ {\bf Y} }
\newcommand{\bW}{ {\bf W} }
\newcommand{\bM}{ {\bf M} }
\newcommand{\bS}{ {\bf S} }
\newcommand{\bA}{ {\bf A} }
\newcommand{\tA}{ {\tilde{A}} }
\newcommand{\td}{ {\tilde{d}} }
\newcommand{\TA}{ {\bf \tilde{A}} }
\newcommand{\bB}{ {\bf B} }
\newcommand{\rb}{ {\bar{\rho}} }
\newcommand{\indep}{ {\perp\!\!\!\perp} }
\newcommand{\moins}{ {\setminus} }
\newcommand{\set}[2]{\{#1\mid\;#2\}}
\newcommand{\pres}[2]{\langle \: #1 \mid #2 \: \rangle}
\def\prev{\text{Prev}}
\def\next{\text{Next}}
\def\rig{\text{Next}}
\def\Noact{\text{Noact}}
\def\lef{\text{Next}}
\def\leftt{\text{Left}}
\def\rightt{\text{Right}}
\def\edge{ -\!\!\!- }
\def\Rat{\text{Rat}}
\def\cb{\cB}
\def\closcb{\bar{\cB}}

\def\B{{\Blackboardfont B}}
\def\P{{\Blackboardfont P}}
\def\F{{\Blackboardfont F}}
\def\Q{{\Blackboardfont Q}}
\def\Z{{\Blackboardfont Z}}
\def\M{{\Blackboardfont M}}
\def\N{{\Blackboardfont N}}
\def\R{{\Blackboardfont R}}
\def\T{{\Blackboardfont T}}
\def\un{{\mathbb 1}}


\def\cA{{\mathcal A}}
\def\fA{{\mathfrak A}}
\def\cB{{\mathcal B}}
\def\fB{{\mathfrak B}}
\def\cQ{{\mathcal Q}}
\def\cN{{\mathcal N}}
\def\cI{{\mathcal I}}
\def\cT{{\mathcal T}}
\def\fT{{\mathfrak T}}
\def\cJ{{\mathcal J}}
\def\cD{{\mathcal D}}
\def\cF{{\mathcal F}}
\def\cE{{\mathcal E}}
\def\cG{{\mathcal G}}
\def\cR{{\mathcal R}}
\def\cP{{\mathcal P}}
\def\cS{{\mathcal S}}
\def\cH{{\mathcal H}}
\def\cX{{\mathcal X}}
\def\cC{{\mathcal C}}
\def\cM{{\mathcal M}}
\def\morphism{{\mathcal M}}
\def\cL{{\mathcal L}}
\def\cK{{\mathcal K}}
\def\cO{{\mathcal O}}
\def\eps{\varepsilon}
\newcommand{\myline}{\underline{\hspace{\fill}}}
\newcommand{\parag}{\bigskip\noindent}
\newcommand{\cqfd}{\hspace*{\fill}\rule{1.8mm}{1.8mm} \\ }
\def\iff{\Longleftrightarrow}
\def\eref#1{(\ref{#1})}
\def\ov#1{\overline{#1}}
\def\ti#1{\widetilde{#1}}
\def\norm#1{\|#1\|}
\def\weak{\stackrel{w}{\longrightarrow}}
\def\proba{\stackrel{P}{\longrightarrow}}
\def\corresp{\twoheadrightarrow}

\def\pos{\text{Pos}}
\def\neg{\text{Neg}}
\def\Rat{\text{Rat}}
\def\first{\text{First}}
\def\last{\text{Last}}
\def\brack{\text{Brack}}
\def\gars{\text{Gars}}


\newcommand{\one}{\mathbb{1}}
\newcommand{\zero}{\mathbb{0}}

\begin{document}
\sloppy

\title{\bf Zero-Automatic Queues and Product Form}

\author{Thu-Ha {\sc Dao-Thi} and  Jean {\sc Mairesse}
\thanks{LIAFA, CNRS-Universit\'e Paris 7, case
    7014, 2, place Jussieu, 75251 Paris Cedex 05, France. E-mail: {\tt
      (daothi,mairesse)@liafa.jussieu.fr}}}

\maketitle



\begin{abstract}
We introduce and study a new model: {\em 0-automatic
  queues}. Roughly, 0-automatic queues are
characterized by a special buffering mechanism evolving like a
random walk on some infinite group or monoid. The salient result
is that all stable 0-automatic queues have a product form stationary
  distribution and a Poisson output process. When
considering the two simplest and extremal cases of 0-automatic
queues, we recover the simple \emph{M/M/1} queue, and Gelenbe's
\emph{G}-queue with positive and negative customers.
\end{abstract}



\textsl{Keywords:} Queueing theory, M/M/1 queue, G-queue,
quasi-reversibility, product form, Quasi-Birth-and-Death process.

\smallskip

\textsl{AMS classification (2000):} Primary 60K25, 68M20.











\section{Introduction}

Here is an informal description of a special type of 0-automatic
queue (corresponding to a free product of three finite
monoids). Consider a queue with a single server and an infinite
capacity buffer.
Customers are colored either in Red, Blue, or
Green, with a finite set of possible shades within each color:
.
In the buffer, two consecutive customers of the
same color either cancel each other or merge to give a new
customer of the same color. Customers of different colors do not
interact.
This is illustrated in Figure \ref{fi-0aut}.

\begin{figure}[ht]

\caption{A 0-automatic queue.}
\label{fi-0aut}
\end{figure}

The shades get modified in the merging procedure, according
to an internal law: , with  coding for the cancellation.
The only but crucial restriction is that each internal law should
be associative.

\medskip

We now give a more detailed account of the model and
results. Zero-automatic queues may be viewed as the synthesis of a
simple queue and a random walk on a 0-automatic pair. We first
recall these last two models.

The  FIFO queue, or simply  queue, is the
Markovian queue with arrivals and services occurring at constant
rate, say  and , a single server, an infinite
capacity buffer, and a First-In-First-Out discipline. This is
arguably the simplest and also the most studied model in queueing
theory, with at least one book devoted to it~\cite{cohe82}. The
queue-length process is a continuous time jump Markov process and
its infinitesimal generator  is given by: . Under the stability condition
, the queue-length process is ergodic, and its
stationary distribution  is given by:

Besides, and this constitutes the celebrated Burke Theorem, the
departure process in equilibrium has the same law as the arrival
process.


\medskip

Let us introduce the a priori completely unrelated model of random
walk on a plain group studied in \cite{mair04,MaMa}.

Let  be an infinite group or monoid with a finite set of
generators . Let  be a probability measure on 
and let  be a sequence of -valued i.i.d. r.v.'s of law .
Let  be the sequence of -valued r.v.'s defined by:
,
where  is the unit element of  and  is the group or
monoid law. By definition,  is a realization of the
random walk .

We now assume that the pair  is formed by a {\em plain} monoid
with {\em natural} generators. The
definition will be given in Section \ref{se-prel}. For the
moment, it suffices to say that the  elements of  can be set in
bijection with a regular language .
The random walk  is viewed as evolving on .
If , and , then

Now assume that the random walk is transient. Let
 be the probability that the random
walk goes to infinity in the ``direction''  (i.e.
). The following is the main result in
\cite{mair04}:

where .

\medskip

The expressions in \eref{eq-mm1} and \eref{eq-rw} share a common
``multiplicative'' structure.
Guided by this analogy, we want to merge the two models together.
To that purpose, we make the following
elementary observation: if we block the server in an  queue,
the number of waiting customers after  arrivals is . And
 can be viewed as the (not so random) random walk on the
pair  associated with the probability .

Now, replace the trivial random walk  by another, more
complex, random walk  on a plain triple .
Hence, the random walk  constitutes the buffering
mechanism in a queue with a blocked server. A {\em 0-automatic
queue} is the model obtained when unblocking the server. The set
 is the set of possible {\em classes} for customers. Customers
arrive at constant rate . Upon
arrival, a new customer (class ) interacts with the customer
presently at the back-end of the buffer (class ), according to
\eref{eq-buff}. At the front-end of the buffer, customers are
served at constant rate . 

\medskip

Let us comment on the name
{\em zero-automatic}. Plain groups, see \eref{eq-plainm}, are
{\em automatic} in the sense of Epstein et. al~\cite{ECHLPT}. Automatic
groups form an important class of groups extensively studied in
geometric group theory, the adjective ``automatic'' referring to the
existence of automata to recognize and multiply elements of the
group. Now the pairs  formed by a plain group with natural
generators satisfy the {\em 0-fellow traveller property}, see
\cite{ECHLPT}. It was proposed in \cite{mair04} to call 
 a {\em 0-automatic pair}. By extension, a queue built
upon  is {\em 0-automatic}. The name is also supposed
to evoke the local aspect of the interactions between customers in the
buffer, see \eref{eq-buff}. 

\medskip

Let  be the drift or rate of escape to infinity of the random
walk . We prove in Section \ref{se-stab} that the stability
condition for the 0-automatic
queue associated with  is: . Under this
condition, we prove in Section \ref{se-main} that the stationary
distribution  for the queue-content process has a
``multiplicative'' structure:

for some numbers . (These numbers are in general different from
their counterparts in \eref{eq-mm1} and \eref{eq-rw}.)
Furthermore, the departure process from the
queue is a Poisson process of rate . Thus we have an analog
of Burke Theorem for all 0-automatic queues. Using standard
terminology, 0-automatic queues are {\em quasi-reversible}.

To be more precise, given , several variants of 0-automatic
queues can be defined depending on the way customers are incorporated
in an empty queue (boundary condition). There is
precisely {\em one} choice for
which the result in \eref{eq-0autq} holds. The numbers , as well as the right boundary condition,
are obtained implicitly via the unique
solution of a set of algebraic equations, see Theorems
\ref{th-main} and \ref{th-uniq} for a precise statement.

\medskip

Aside from the free monoid, the next simplest example of a
plain monoid is the free group over one generator:
. The 0-automatic queues associated with  
are variations of Gelenbe's
G-queues, or queues with positive and negative customers, which
were quite extensively studied in the 90's, see \cite{gele91,FGSu}
and the bibliography in \cite{GePu}. General 0-automatic queues
can be viewed as a wide generalization of this setting. 
Indeed, in a 0-automatic queue, different types of tasks (customers) can
be modelled. Let us detail four of them which form a
representative sample, without exhausting all the types within the
realm of 0-automaticity.

\medskip

- Classical type. Tasks are processed one by one with no
  simplification occurring in the buffer: . The corresponding
  pair is .

\medskip

- Positive/negative type. Tasks are either positive () or
negative
  () and two consecutive tasks of opposite signs cancel each
  other: . The corresponding pair is
  .  The relevance of this type
  for applications is discussed in \cite{GePu}.

\medskip

- ``One equals many'' type. It takes the same time to process one
or
  several consecutive instances of the same task: . Think for instance of a ticket
  reservation where the number of requests is only reflected by
  an integer value in a menu-bar choice. The corresponding pair is
   where  is the Boolean monoid .

\medskip

- ``Dating agency'' type. Two instances of the same task cancel
each
  other: . Think of a task as being a tennis player looking for
  a partner (to be provided by the server); when two such tasks
  are next to each other in the buffer, they leave to play a
  game instead of waiting in line. The corresponding pair is
  . Instead of tennis players, we may
  consider music trio players, bridge players, etc, the corresponding
  group being , etc.

\medskip

To model a server where several of the above types (and possibly
several copies of the same type) can be processed, one just has to
perform the free product of the corresponding monoids or groups (see
Section \ref{sse-mg} for the definition). 

\medskip



The  queue is the basic primitive for building Jackson
networks, which have the remarkable property of having a
``product-form'' stationary distribution. More generally, networks
made of quasi-reversible nodes tend to have a product form
distribution, see for instance \cite{serf99}. In a subsequent
work~\cite{DaMa06}, we prove that it is indeed the case for Jackson-type
and Kelly-type
networks of 0-automatic queues.

\medskip

A preliminary version without proofs of the present paper has appeared
in the conference proceedings~\cite{DaMa05}. 

\section{Preliminaries}\label{se-prel}

{\em Notations.} We denote respectively by ,  and  the
integers, nonnegative integers and reals. We 
denote by  and  the positive integers and reals.
The symbol  is used for the disjoint union of sets.
Given a set  and , define   by  if  and  otherwise.
Given a set , a
vector , and , set
.

\medskip

Let us recall the needed material on random walks on plain monoids. 
The presentation follows \cite{mair04,MaMa}.

\subsection{Monoids and groups}\label{sse-mg}

Given a set , the free monoid generated by  is
denoted by . The unit element is denoted by  or . As
usual, the elements of  and  are called {\em
letters} and {\em words}, respectively. The subsets of
 are called {\em languages}. The {\em length} (number
of letters) of a word  is denoted by .

Let  be a group or monoid with set of generators
. The unit element of  is denoted  by . When 
is a group, the inverse of  is denoted by . We
always assume that: , and in the group case
that: . The {\em length}
with respect to  of an element  of  is:


The {\em Cayley graph}  of  with respect to
 is the directed graph with nodes  and arcs
 if .

\medskip

Consider a relation , and let
 be the least congruence on  such that  if . Let  be isomorphic to the quotient monoid
. We say that  is a {\em
  monoid presentation} of 
and we write .

Given a set , denote by  the free group generated (as a
group) by . Let  be the set of inverses of the
generators. A monoid presentation of  is


Given two groups or monoids  and , we denote by
 the {\em free product} of  and . Roughly,
the elements of  are the finite alternate sequences
of elements of  and ,
and the law is the concatenation with simplification. More
rigorously, the definition is as follows. Set .
The {\em free  product}  is defined by the monoid
presentation:

If  and  are groups, then  is also a
group. The free product of more than two groups or monoids is
defined analogously.

The Cayley graph of the group  is
represented on Figure \ref{fi-z2z3} (left). 

\subsection{Plain monoids and groups}\label{sse-pmg}

A {\em plain monoid} is a monoid  of the form

where  and  are finite sets and  are finite
monoids. A {\em plain group} is a plain monoid which is also a
group. A plain monoid  defined as in \eref{eq-plainm} is a
plain group iff  and  are groups.

Define

The set  is a finite set of generators of , that we
call {\em natural} generators. Define the language  by:

It is easily seen that the set  is in bijection with the
group elements. Below we often identify  and . 
The following is a consequence of the definition of a plain monoid~: 

To see that \eref{eq-ref1} holds, it is sufficient to check it case by
case. 
It is convenient to introduce the sets: ,

Observe that .

Next property, to be used later on, is another direct consequence of
the definition of a plain monoid~:


\medskip

Consider the directed {\em graph of successors}  where 

Except in the case , observe that the graph  is strongly
connected. 





\subsection{Random walks on monoids and groups}

Let  be a group or monoid with finite set of generators
. Let  be a probability distribution over .
Consider the Markov chain on the state space  with one-step
transition probabilities given by: , . This Markov chain is called the
\textit{(right) random walk} (associated with) .

Let  be a sequence of i.i.d. r.v's distributed
according to . Set


Then  is a realization of the random walk . For
all , we have . Applying Kingman's Subadditive Ergodic
Theorem yields the following (first noticed by Guivarc'h
\cite{guiv80}): there exists  such that

for all . We call  the {\em drift} of the
random walk.

\begin{figure}[ht]

\caption{The random walk .}
\label{fi-z2z3} 
\end{figure}



To illustrate, consider the plain group  and the natural generators
. Let  be a probability measure on
. 
On the left of Figure \ref{fi-z2z3}, we
have represented a finite part of the infinite Cayley graph ,
and the one-step transitions of the random walk  starting
from the state . On the right of the figure, 
we show the same one-step transitions on the group elements viewed as
words of  (written from bottom to top). 


\subsection{Random walks on plain monoids and groups}

It is convenient to introduce the notion of a plain triple.

\begin{definition}\label{de-0aut3}
A triple  is {\em plain} if: (i)  is an
infinite plain monoid not isomorphic to  or ; (ii)  is a set of natural generators; (iii) 
is a probability measure whose support is included in  and
generates .
\end{definition}

\begin{proposition}\label{pr-transient}
If  is a plain triple, then the random walk
 is transient.
\end{proposition}

If  is an infinite plain monoid with the support of 
generating , there are only two cases in
which  is not transient: (1) the triple ; (2) the triples , for any , where  and  are the respective
generators of the two cyclic groups. Since  and  have been excluded from consideration, then the random
walk  is transient, see \cite{mair04} for details.


The case  is specific. Some of the results below remain true
but not all of them. For simplicity, we treat this case separately in \S
\ref{se-examples}.

\medskip

Define:


The Traffic Equations play an essential role in the study of the
random walk .

\begin{definition}\label{de-TE}
The {\em Traffic Equations (TE)} associated with a plain triple
 are the equations of the variables  defined by: ,

An {\em admissible solution} is a solution belonging to  .
\end{definition}

By multiplying both sides of \eref{eq-TE} by ,
we obtain a new set of Equations without denominators. With some
abuse, a solution  in  of this last set of Equations
is still called a solution of the TE.

Next result can be easily deduced from the proof of \cite[Theorem
  4.5]{mair04}.

\begin{proposition}\label{pr-solTE}
Let  be a plain triple. The Traffic Equations have
a unique admissible solution.
\end{proposition}


The interest of Proposition \ref{pr-solTE} is that the harmonic
measure and the drift can be expressed as a function of the
solution to the TE.
Define the set
 by

A word belongs to  iff all its finite prefixes belong to
. The set  should be viewed as the
``boundary'' of .

\medskip

Let  be a
realization of the random walk which is transient by Proposition
\ref{pr-transient}. 
The {\em harmonic measure} of the random walk is the probability
measure  on  with finite-dimensional marginals defined by:

This defines indeed a measure on  because the random walk
is transient, and because  and  differ by at
most their last symbol.
Intuitively, the harmonic measure  gives the direction in which 
goes to infinity.

For a proof of next result, see \cite[Theorem 4.5]{mair04} and also
\cite[Theorem 3.3]{MaMa}. In
the specific case of the free group, the result appears in
\cite{DyMa,SaSt}, see also the survey \cite{ledr00}.

\begin{theorem}\label{th-rw}
Let  be a plain triple. Let
 be the unique
  admissible solution to the Traffic Equations.
Set , for all
. The harmonic measure  of the random
walk  is given by:

The drift of the random walk is given by:

\end{theorem}

\section{The Zero-Automatic Queue}\label{se-0autq}

We first define the 0-automatic queue informally, before doing it
formally in Definition \ref{de-0aut}. 
Let  be a plain monoid,  be a set of natural
generators, and  a probability measure on . 
The associated
0-automatic queue is formed by a simple single server queue with
FIFO discipline and an infinite capacity buffer in which the
buffering occurs according to the random walk . It is a
multi-class queue (classes ) but the class does not
influence the way customers get served, only the way they get
buffered.

\medskip

More precisely, the instants of customer arrivals are given by a
Poisson process of rate , and each customer carries a
mark, or {\em class}, which is an element of . The
sequence of marks is i.i.d. of law . Upon arrival, a new
customer interacts with the customer presently at the back-end of
the buffer, and depending on their respective classes, say  and
, one of three possible events occurs: (i) if ,
then the two customers leave the queue; (ii) if , then the two customers merge to create a customer of type
; (iii) otherwise, customer  takes place at the back-end of
the buffer, behind customer . In the mean time, at the
front-end of the buffer, the customers are served one by one and
at constant rate  by the server. To be complete, one needs to
specify how customers are incorporated when the buffer is empty.
Several variants may be considered, and we view this ``boundary
condition'' as an additional parameter of the model.
The resulting flexibility in the definition of a 0-automatic queue
will turn out to be a crucial point.

\medskip

According to the above description, the queue-content (the
sequence of classes of customers in the buffer) is a continuous
time jump Markov process. The more formal definition of the queue
is given via the infinitesimal generator of this process.

\begin{definition}[Zero-automatic queue]\label{de-0aut}
Consider a plain triple .
Let  be the
set of words defined in \eref{eq-loca}. Consider
, see \eref{eq-cb},  and .
The {\em 0-automatic queue} of type 
is defined as follows. The {\em
  queue-content}  is a continuous time
jump Markov process on the state space  with infinitesimal generator
 defined by: ,

and, for all  such that , and for all
,

and, finally, the boundary condition is,

We denote by  any 0-automatic queue of type
.
\end{definition}

\remark\label{rm-bc} The intuition behind the form of the boundary
condition is as follows: the buffer-content is viewed as the
visible part of an iceberg consisting of an infinite word of
, see \eref{eq-Linfty}. When the buffer is empty, new
customers are incorporated depending on the invisible part of the
iceberg, whose first marginal is assumed to be . This last
point will find an a-posteriori justification in Theorem
\ref{th-main}.

\medskip

The simplest example of 0-automatic queue is the one associated
with the free monoid . The triple ,
where  is a probability measure on , is not plain.
However, it is simple and interesting to generalize Definition \ref{de-0aut}
in order to define a 0-automatic queue associated with the free
group . We now discuss the 0-automatic queues associated with 
 and . 

\paragraph{The simple queue.}

Consider the free monoid  over the single
generator set
.
Hence, for any , there is only one
possible associated queue: , where
. By specializing the infinitesimal generator 
given in Definition \ref{de-0aut}, we get: ,

This is the simple  FIFO queue with arrival rate
 and service rate .

\paragraph{The G-queue.}

Consider the free group  and the
set of generators . Let  be a probability
measure on
 such that . Consider  and . The 0-automatic queue
 has an infinitesimal
generator  given by: ,

This is close to the mechanism of the G-queue, a queue with positive and
negative customers introduced by Gelenbe \cite{gele91,GePu}.
With respect to the G-queue, one originality of the
 queue is that negative and
positive customers play
symmetrical roles. Another one is the treatment of the boundary
condition. 

\medskip

Since the triple  is not plain according to
Def. \ref{de-0aut3}, the above queue is not covered by the results in
Sections \ref{se-stab} and \ref{se-main}. However, part of the results
remain true, and we come back specifically to this model in 
Section \ref{sse-G} and \ref{ssse-fg}. 







\paragraph{Extension.}
It is possible to generalize Definition \ref{de-0aut} in
order to define a 0-automatic queue of type , resp.
.  Roughly, the description would go as follows. The
buffering mechanism is kept unchanged; the sequence of inter-arrival
times and classes of customers is i.i.d. (resp. stationary and ergodic);
the sequence of service times at the server is i.i.d. (resp. stationary
and ergodic) and independent of the arrivals.

\subsection{Comparison with other models in the literature}

Under stability condition, we will see that a 0-automatic queue has
the ``Poisson output'' property. Also, a 0-automatic queue is
``quasi-reversible'', at least in the sense of Chao, Miyazawa, and
Pinedo~\cite[Definition 3.4]{CMPi}.  
There exist many examples of queues with such properties, see for
instance Kelly~\cite{kell79} or \cite{CMPi}. 
However, 0-automatic queues are
quite different from the existing models. 

Let us detail the comparison with the models in \cite{CMPi},
see also \cite{ChMi00}. Their model is a wide generalization of
Gelenbe's G-queue with signals, batch arrivals, and batch departures. 
In a sense, 0-automatic queues can also be viewed as a wide
generalization of G-queues. Other common features between the models 
include: non-linear traffic equations, an output rate different from
the input rate, and subtle boundary conditions to get a product form. 
Despite these similarities, the models are quite orthogonal. 
One big novelty of 0-automatic queues is the possibility for
two customers to merge and create a customer with a new
type. The algebraic foundation of 0-automatic queues is another
originality. 

\medskip

It is also worth comparing the 0-automatic queue with another model for
queues introduced by Yeung and Sengupta~\cite{YeSe}, see also
He~\cite{he} (the {\em YS model} in the following). 

\begin{figure}[ht]

\caption{Effect of an arrival and a departure on the content of the
  buffer in a 0-automatic queue built on the group .}
\label{fi-0autmeca}
\end{figure}

A common feature is the structure of the state space~: a tree for the
YS model (or the cartesian product of a tree and a finite set), and a
more general tree-like graph for the 0-automatic queue. In particular,
both models correspond to multiclass queues, and the buffer content is
coded by a word over the alphabet of classes. Second common
feature, the effect of a new arrival is either to add, to modify
the class of, or to remove, a customer at the back-end of the buffer (in
the YS model, the removal/modification may affect several customers at
the back-end of the buffer). Now, and this is the first central
difference, departures occur at the front-end of the buffer in the
0-automatic queue, and at the back-end in the YS model. Therefore, the
former is a FIFO queue while the latter is a LIFO queue. We have
illustrated the FIFO mecanism of the 0-automatic queue in Figure
\ref{fi-0autmeca}. 
The second important difference concerns the type of results which are
proved. In a stable 0-automatic queue, the buffer content has a
``product form'' stationary distribution, see Theorem \ref{th-main}. In the YS
model, it has only a  ``matrix product form'', see \cite[Section 2]{YeSe} and getting the
stronger ``product form'' requires severe additional assumptions, see 
\cite[Section 6]{YeSe}. 
To conclude the comparison, here again, the original flavor of the
0-automatic queue comes from the underlying group or monoid structure. It is
this algebraic foundation which can be accounted for the ability 
to get the strong product form results. 

\section{Stability Condition for a Zero-Automatic Queue}
\label{se-stab}

Throughout Sections \ref{se-stab} and \ref{se-main}, the model is
as follows. Let  be a plain triple. Fix 
and  in  and  in . Consider the 0-automatic
queue .

\medskip

Let  be the queue-content process, and  the
infinitesimal generator. Next Lemma is a direct consequence of the
strong connectivity of the graph  defined in
\eref{eq-graph}. 

\begin{lemma}
The process  is irreducible.
\end{lemma}

The aim of this Section is to prove Proposition
\ref{pr-stability} which characterizes the stability region of the 0-automatic
queue.

\begin{proposition}\label{pr-stability}
Let
 be the drift of the random walk .
We have:

\end{proposition}

Consider an excursion of  from the instant  at which it is
assumed to leave state , to the instant  which corresponds to
the first return to state . Recall that  is transient iff
, and ergodic iff .

It is convenient to
use the following representation for .
Let  where  are the time
points of a time-stationary 
Poisson process of rate  on . Let  be the corresponding counting process. Let  be the counting process of a time-stationary Poisson
process of rate  on .
Let  be a realization of the
random walk  viewed as evolving on , see
\eref{7}. Assume that  and  are mutually
independent. Let  be the continuous-time jump Markov
process on the state space  defined by:


For all  in the interval , we have:


Here  is the queue-content at time  if no service has
been completed. Observe that the first letter of 
corresponds to the front-end of the buffer (the right-end in Figure
\ref{fi-0aut}), and the last letter to the back-end (the left-end in Figure
\ref{fi-0aut}).

\medskip

The counting process of a Poisson process satisfies a Strong Law
of Large Numbers. We get, a.s.,


We also have, a.s.,

where  is the drift of the random walk
. So we have, a.s.,




We can now prove the following.

\begin{lemma}\label{le-step0}
If  then  is recurrent. If  then  is transient.
\end{lemma}

\begin{proof}
We show the first statement by contraposition. If  is transient
then . Using \eref{eq-repr} and \eref{eq-slln},
we obtain that a.s. on the event , we have:

To avoid a contradiction, we must have  .

Now assume that . Using the
independence of  and , and the regenerative properties of , it is
easily shown that
. In particular
 and  is transient.
\end{proof}

To get the stronger results in Proposition \ref{pr-stability}, the
idea is to approximate the 0-automatic queue by a simple queue with a
Markov additive arrival process, and then to use standard results
from queueing theory.

\medskip

Since  is transient (Proposition \ref{pr-transient}),
there exists an a.s. finite  such that . For notational simplicity, assume that
. Define the random variables: ,

The r.v.'s  are a.s. finite because  is transient, and we
have:
 a.s.

By transience,  is a (random) infinite word on the
alphabet , let us write it as .
By definition, see Section
\ref{se-prel}, the law of  is the harmonic measure
 of the random walk . Observe that:

According to Theorem
\ref{th-rw}, we have: ,


It follows that  is a Markov chain with initial distribution
 and transition matrix  given by:

The matrix  is irreducible as a direct consequence of the
strong connectivity of the graph of successors , see
\eref{eq-graph}. 
Let  be the stationary
distribution of  characterized by . In general, the
Markov chain  is not stationary, i.e.  is
different from . (See \cite[Proposition 3.6]{MaMa} for a
sufficient condition on  ensuring that
.)

\medskip

Consider now the sequence . A consequence of the above
is that  is a Markov chain with transition function
depending only on the first coordinate. According to the classical
terminology, the sequence  is a {\em Markov additive
  process (MAP)}.

\medskip

Consider the simple queue of type  with arrival process
, and a service process driven by .
Let  be the corresponding sequence of service times.
We deduce from \eref{eq-slln}, that a.s. and in :

Let  be the
queue-length process of this queue. Let  be the
first instant of return to 0 for the process .
Applying standard results for  queues, see for
instance \cite[Prop. 4.2, Chapter X]{asmu87}, we get:


Concentrating on the mechanism of the 0-automatic queue, it is not difficult
to see that:

Hence, the queue  is a good approximation of the 0-automatic
queue. In particular, the two implications in
\eref{eq-impli1}-\eref{eq-impli2}
also hold for . In view of Lemma \ref{le-step0}, this completes the proof of Proposition
\ref{pr-stability}.


{\bf Remark.}
The above proof does not rely in an essential way on the Markovian
assumption. For instance, modulo some care,
an analog
of Proposition \ref{pr-stability} can clearly be written for a
0-automatic queue of type .


\section{Stationary Distribution of a Stable Queue}\label{se-main}

\subsection{The Twisted Traffic Equations}

The Traffic Equations, see Definition \ref{de-TE}, play a central role in studying
the random walk.
We now introduce equations which play a similar role
for the queue.


\begin{definition}[Twisted Traffic Equations]\label{de-TTE}
The {\em Twisted Traffic
  Equations } associated with 
are the
equations of the variables

defined by:

\end{definition}

To get a hint of the future role of  the ,
let us examine the case  considered at the
end of Section \ref{se-0autq}. Recall that there is only one possible
variant for the queue  which is equivalent to
the simple  queue.
By simplifying \eref{eq-TTE},
we get:

Compare this with the global balance equations of the  queue:

By substituting  in (\ref{20}), we recognize
(\ref{19}).

\medskip

According to Proposition \ref{pr-solTE}, there is a unique admissible
solution to the Traffic Equations, that we denote by . We denote by  the
drift of the random walk .

Consider . Define

One easily checks that:

This point was the crux of
the argument in proving Proposition \ref{pr-solTE} in
\cite{mair04}. Observe that a simple rewriting of \eref{eq-drift} gives:


\medskip

Let us investigate some properties of the solutions to the
Twisted Traffic Equations.

\medskip

First, if  is a solution to the TTE with ,
then  belongs to . This follows directly from the shape of
the TTE and from the strong connectivity of the graph
, see \eref{eq-graph}. 

Second, if we set  in the Twisted Traffic Equations \eref{eq-TTE}, and perform the obvious
simplifications, we obtain the Traffic Equations \eref{eq-TE}.
It implies that  is a solution to the  for
all  and .

\begin{lemma}\label{le-tte1}
Let  be a solution to the
. We have either , or

\end{lemma}

\begin{proof}
By summing all the Equations of \eref{eq-TTE}, we get:


Replacing  by  in the above, we get:


If , we have seen that the TTE reduces to the TE, which implies by
Proposition \ref{pr-solTE} that . Otherwise, we must have
. This completes the
proof.
\end{proof}

The relevant solutions to the TTE will turn out to be the ones
satisfying \eref{eq-etax}. This leads us to the next
Definition.

\begin{definition}\label{de-admissible}
A solution  to the TTE is called an {\em admissible solution} if
 and if \eref{eq-etax} is satisfied.
\end{definition}

\begin{lemma}\label{le-tte2}
If , then 
is an admissible solution to the TTE.
If  is an admissible solution to the , then
 and .
\end{lemma}

\begin{proof}
Assume that .
We know that  is a solution to the TTE.
We need to check that it is admissible. By definition, it is
admissible if:

We conclude by recalling that: , see \eref{eq-gammaAC}.

Assume now that  is an admissible solution to the .
Since , the TTE reduce to
the TE implying that  . Now replacing  by
 in \eref{eq-etax}, we get:
.
\end{proof}

More generally, admissible solutions always exist:

\begin{lemma}\label{le-goodsol}
There exists an admissible solution to the
.
\end{lemma}

\begin{proof}
Consider the Equations \eref{eq-TTE} and replace  by
. The resulting equations in  can be
viewed as a fixed point equation of the type . The
corresponding application  has the following form. For 
and for ,


Consider . By summing the Equations in
\eref{eq-phi} and using \eref{eq-crux}, we get:

We have proved that
.
The end of the proof follows very closely the proof of Theorem 4.5 in
\cite{mair04}. For the sake of completeness, we recall the argument.

To use a Fixed Point Theorem, we need to define
 on a compact and convex set.
The set , which is the closure of , is a compact
and convex subset of . But the
map  cannot in general be extended continuously on
. More precisely,  can be
defined unambiguously iff  for all .

For , let  be the
set of possible limits of . We have extended  to a correspondence . Clearly this correspondence has a closed graph
and nonempty convex values. Therefore, we are in the domain of
application of the Kakutani-Fan-Glicksberg Theorem, see
\cite[Chapter 16]{AlBo}. The correspondence has at least one fixed
point:  such that . Now using
the shape of the Equations in \eref{eq-phi} and the strong
connectivity of the graph of succesors , we obtain
that  (see \cite[Theorem 4.5]{mair04} for details).

Set . The pair  is
an admissible
solution to the .
\end{proof}

\subsection{The main results}\label{sse-main}

Next Lemma begins
to establish the link between the
Twisted Traffic Equations and the queue .

\begin{lemma}\label{lemma2}
Let  be an admissible solution to the
. Consider the 0-automatic queue of type
. Let  be the infinitesimal generator
of the queue-content process.
Consider the measure  on  defined by:

We have . Conversely, assume there exist
 and  such that the measure 
defined by \eref{eq-px} satisfies . Then
 is an admissible solution to the TTE.
\end{lemma}

\begin{proof}
We have  if and only if: ,

Denote the left and right-hand side of the above equality by 
and , respectively. Define

The left of (\ref{balance}) is:


The right of (\ref{balance}) is given by, for ,

and for ,

Now recall that  .
We obtain:


We see that for , the equality  is
precisely equivalent to the fact that  is a solution to
the . For , the equality  is
precisely equivalent to the fact that  and  satisfy
\eref{eq-etax}.

Therefore, the equality  is precisely equivalent to the fact
that  is an admissible solution to the .
This completes the proof.
\end{proof}


We now have all the
ingredients to prove the central results of the
paper.


\begin{theorem}\label{th-main}
Let  be a plain triple. Fix  and
 in .
Let  be an admissible solution to the TTE.
Consider the 0-automatic queue
  .  Denote by  the
  queue-content process and by  its infinitesimal generator.
We have:

Assume that . The stationary
distribution  of the process  is given by:
,

where  for all .
\end{theorem}


\begin{proof}
Let  be an admissible solution to the Twisted Traffic
Equations. Let  be the measure
defined in \eref{eq-px}.
We have (using that ):

Hence,  iff . Now recall that
, Lemma \ref{lemma2}.
It is standard (see for instance \cite[Chapter 8]{brem99})
that the process  is ergodic iff .
Now, according to Proposition \ref{pr-stability},
 is ergodic iff . By
combining the three equivalences, we get:

The result in \eref{eq-statdist} holds as a direct consequence of Lemma
\ref{lemma2}.

Now let us turn our attention to the null recurrent case. Assume that
.
Using Lemma \ref{le-tte2} and Proposition \ref{pr-stability}, we have:

Let  be an admissible  solution to the TTE. Using the
argumentation in the forthcoming proof of
Theorem \ref{th-uniq}, we deduce that we must have
. Hence we have an equivalence on
the left of \eref{eq-deux}. This completes the proof.
\end{proof}





Assume that .
It follows immediately from Lemma \ref{le-tte2} and Theorem \ref{th-main}
that  is the unique admissible solution to the
TTE. We now prove a more interesting result in the same vein.


\begin{theorem}\label{th-uniq}
Consider the same model as in Theorem \ref{th-main}. Assume that
. Then the TTE have a unique
admissible solution. In particular, there is only one variant of
the 0-automatic queue  with a product form
distribution.
\end{theorem}

\begin{proof}
Let  and 
be two admissible solutions to the TTE.
According to Theorem
\ref{th-main}, we have  and .
Let  and  be the respective stationary distributions
of  and .

We now use a classical result on
ergodic Markov processes, cf for instance \cite[Chapter
  8, Theorem 5.1]{brem99}: the stationary distribution
is proportional to the time spent in each state in an excursion of the
process from  to , for some arbitrary state .

Assume that the queue-content is  at instant 0. Let ,
resp. , be the first instant of jump of ,
resp. .  Let , resp. , be the first
return instant to .
We have, for all  ( stands for
`proportional to'),

It follows that, for all , ,

And, conditioning by the value of ,
resp. , we have, for all , ,


Observe that the generators  and  differ only in
the line indexed by . In other terms, the conditional law of  on the event , is equal to the conditional law of  on the event .
Therefore  and  are obtained as linear combinations
of the same measures  defined by:

For , set . Define  and  accordingly.
Using the above, we have, for all ,

Besides, according to \eref{eq-statdist}, we have  and . In view of
\eref{eq-laststep}, we conclude easily that we must have
.

It remains to prove that .
Let us first show that:

We have: . We can reinterpret this
as: , with  being viewed as a column vector and with 
being the matrix of dimension  defined by:

 Consequently, the 
matrix  is irreducible. 
Now invoking the Perron-Frobenius Theorem, since  has all its
coordinates positive, it implies that  is necessarily the
Perron eigenvector of the matrix, i.e. the unique (up to a
multiplicative constant) eigenvector associated with the spectral
radius. But we also have: , and . By uniqueness of the Perron
eigenvector, we conclude that .

\medskip

Therefore, it remains to prove that: .
Define  by:

Observe that: .

For , set . Define  analogously. Using
\eref{eq-statdist}, we have, for all , for some
constants , ,

Besides,  and , for some constants ,
. Therefore,  and  must grow at the
same exponential speed as a function of . In view of \eref{eq-ref},
we must have: 


For the remaining step, the argument depends on the form of .
Set , with
(), with
 (), with  being
finite monoids for , and with
 for .

For , , . For , . For
, . 

\medskip

We first treat the case .
Then there exists  such that . Set . 
For all , , so we have
. It implies that .
Consider  and , one
has , so, according to \eref{eq-circ},
. Hence, .
So  for all .

\medskip

We now consider the case .
Assume that . Consider . We have: ,
, and . Therefore,  and . Using \eref{eq-circ}, we deduce that:
 and
. We conclude that
 for all .

\medskip

Assume now that . The above argument does not work anymore.
First of all, we want to prove that:

Consider  and . We have:
 and . Hence,  and . Using \eref{eq-circ}, we
have:  and . It implies that:
, which is equivalent to
\eref{eq-prop}. Set
 if .


Now let us sum the TTE corresponding to all the elements of
, and let us perform the simplifications implied by
\eref{eq-prop}. For instance for , we get:

Set  and .
Using that , we get:

But all the terms in the right-hand side of \eref{eq-ouf} are
unchanged when we write the corresponding equation for .
So, . In view of \eref{eq-prop}, we
deduce that  for all .
This completes the proof. 
\end{proof}

When the triple is not plain, the TTE may have several
admissible solutions. This is for instance the case for the triple
 as discussed in Section
\ref{ssse-fg}. 

\remark In the case , if the boundary condition is chosen
according to , where  is not a solution to the TTE,
then the stationary distribution of  exists
(Prop. \ref{pr-stability}). But we do not know 
how to compute it exactly. 
See Remark \ref{rm-bc} for a
justification of the form of the boundary condition. 




\paragraph{Poisson departure processes.}  \\



The celebrated Burke Theorem states that the departure process from a
stable  queue is a Poisson process of the same rate as the
arrival process.
A nice consequence of Theorem \ref{th-main} is that an analog
of Burke Theorem holds for 0-automatic
queues.


\medskip

In a 0-automatic queue, `departures' occur both at the front-end and
at the back-end of the buffer. Here we consider only the front-end
departures, i.e. the ones corresponding to service completions and not
to buffer cancellations.

Let  be the queue-content process of some 0-automatic
queue . A {\em departure} is
an instant of jump of  corresponding to a jump of the type:
 for .
When  (the case \eref{eq-main2} in Definition
\ref{de-0aut}), some special care must be taken. The jumps of type
 which are {\em departures} occur at rate
.
The {\em departure process} is the point process of
departures.

\begin{theorem}\label{th-burke}
The model is the same as in Theorem \ref{th-main}. Assume that
. Let  be an admissible solution
of the . Consider the 0-automatic queue
.
The stationary departure process is a Poisson process of rate .
Furthermore, for all , the queue-content at time  is
independent of the departure process up to
time .
\end{theorem}

\begin{proof}
The simplest proof of Burke Theorem uses reversibility and is due to
Reich, see for instance \cite{kell79} for details. Here, the argument
is similar.

Let  be the stationary queue-content process. Its marginal
distribution at a given instant is  given in
\eref{eq-statdist}. Let  be the corresponding departure
process. By definition, the instantaneous rate of  is
 if  and  otherwise.

Now let us consider the time-reversed point process
. The process  corresponds to the
instants of ``right-increase'' of the time-reversed process
. Therefore, the instantaneous rate
 of  is as follows.
If ,

and if ,

We conclude that  is a Poisson process of rate
. Since Poisson processes are preserved by time-reversal, 
is also a Poisson process of rate .

Also, using the Markov property of Poisson processes,
 is independent of the process 
after time . Under time-reversal, this translates as:  is
independent of the departure process  up to time .
\end{proof}




Here are some additional comments on Theorem \ref{th-burke}.

\medskip

1- The infinitesimal generator of the time-reversed process
 is certainly not the infinitesimal generator of
a 0-automatic queue. This was already the case for the G-queue. But
this is in contrast with the situation for the M/M/1 queue. 

\medskip


2- For , the departure process  of
customers of class  is not a Poisson process.

\medskip

3- Equation \eref{eq-burke} corresponds to the condition defining
``quasi-reversibility'' in Chao, Miyazawa, and
Pinedo~\cite[Definition 3.4]{CMPi}. 

\medskip

4- The saturation principle of Baccelli and Foss~\cite{BaFo93b} holds
for many classical queueing systems. Here is a  rough description of
it.

Consider a queueing system with an infinite capacity
buffer.
Let  be the departure rate in the {\em saturated system} in
which an infinite number of
customers are stacked in the buffer.
Now, if the actual arrival rate in the system is , then the system is
stable, and the departure rate is .
A dual presentation of the same principle is as follows. Let 
be the growth rate of the buffer in the {\em blocked system} where the server has
been shut down.
If the actual service rate in the system is , then the
system is stable, and the departure rate is .

Zero-automatic queues do {\em not} satisfy the saturation principle.
This can be viewed on the following inequalities (to be deduced from
Theorem \ref{th-burke}):

where:  is the actual
 departure rate in equilibrium,  is the departure rate from the
 saturated system, and  is the
 growth rate of the buffer in the blocked system.

\subsection{Quasi-Birth-and-Death processes}

Quasi-Birth-and-Death (QBD) processes appear naturally in the
modelling of several queueing and
communication systems. As such, they have been extensively studied,
see for instance the monographs \cite{LaRa,neut}.
The results in Section \ref{sse-main} can be put in perspective by
considering the relation between 0-automatic queues and QBD
processes.

\medskip

To that purpose, we define an  ``approximated'' and quite
simplified version of the 0-automatic queue. The idea is to keep
track of the queue-content only through the number of customers
and the class of the back-end customer. Clearly a difficulty
arises: if a cancellation occurs at the back-end of the buffer,
there is no way to retrieve the class of the new back-end
customer. This missing information is compensated as follows: the
class is chosen at random according to the relevant conditional
law.

\medskip

Consider a 0-automatic queue. The notations and assumptions are the
ones of Theorem \ref{th-main}. In particular  is an
admissible solution to the TTE. We assume that .  Recall that  is the infinitesimal
generator of the queue-content on the state space
, and that  given by \eref{eq-statdist} is its stationary
distribution.

Consider the application:

Define the infinitesimal generator  on the state space
 by:

For instance, we obtain by using \eref{eq-statdist} and simplifying:


Define a total order on  as
follows:  is the smallest element and  if  or , where  is some total order on
. A couple of lines of computation enable to check the
following. If lines and columns are ranked according to the above
order, the infinitesimal generator  is block
tridiagonal of the form:

where  is of dimension ,  is of dimension ,  is
of dimension , and  are of dimension
. Furthermore the entries of  can be
expressed in function of , and ; and the entries of
 can be expressed in function of
, and .

\medskip

According to the terminology in Neuts \cite{neut}, 
is the infinitesimal generator of a QBD
process with a complex boundary behavior. For any such ergodic
process, the shape of the stationary
distribution is known, see for instance~\cite[Chapter
  1.5]{neut}. So we assume that  is ergodic and we
apply the general results to get the stationary
distribution :

where

In \eref{eq-stqbd2},  is a matrix of dimension  and
is the minimal nonnegative solution to the first Equation. The pair
, where  is a scalar and  is a line
vector of dimension , is the unique positive solution to
the second and third Equations.

\medskip

The stationary distribution  in \eref{eq-stqbd} has a
{\em matrix product form}. 
This matrix product form is said to be a {\em product form} (this
is called {\em Level-Geometric with parameter 0} in \cite{DaQu}) if:
, for some . Clearly, a necessary and sufficient condition for this to hold
is: .  Assume that this last equality holds.
By multiplying the first Equation in \eref{eq-stqbd2} by , and by
simplifying the Equations, we get:

Now let us replace  by  and  by  in
\eref{eq-stqbd3}. The first Equation yields precisely the Twisted
Traffic Equations \eref{eq-TTE} for the pair . The other two Equations
yield exactly: , .

We conclude that the stationary distribution of
 is given by:


This is coherent with the form of  as given in
\eref{eq-statdist}.
So the above provides an a-posteriori and partial justification for the shape of 
It also provides another light on the central role of the TTE.

It is classical in QBD theory~\cite{LaRa,DaQu} to modify the boundary condition to
get a stationary distribution of product form. Here the situation is
more complex since not only the
boundary condition, but also the  matrix, are modified in the
quest for the product form.

Observe also that the result in Theorem \ref{th-main} is
much deeper than the one in \eref{eq-stfinal}. In particular, there
is a-priori no way to retrieve the result on  from the one on the
simplified generator .




\section{Extension and Examples}\label{se-examples}

\subsection{Zero-automatic queues built on 0-automatic pairs}\label{sse-G}

All the results in Sections \ref{se-stab} and \ref{se-main} are
derived for queues built on plain monoids not isomorphic to  
(Definition \ref{de-0aut3}). 
However, the 0-automatic queue built on  is interesting in itself
(it corresponds to Gelenbe's G-queue, see Section \ref{se-0autq}) and
exhibits new phenomena. It is therefore worthwhile to determine the subset of the above
results which remain true for this queue. 

In fact, we define a more general framework, and the notion of
0-automatic triples extending the plain triples of Definition
\ref{de-0aut3}. 

\medskip

Let  be a group or monoid with set of generators
. Denote by  the monoid
homomorphism which associates to a word  of
 the
element  of . A language  of
 is a {\em cross-section} of 
if the restriction of  to  is a bijection. The inverse map  is then called the {\em normal form} map.
Define the language  as in 
\eref{eq-loca}. Define the sets: ,

In the case of a plain monoid with natural generators, we have 
, see \eref{eq-ref1}. 

\begin{definition}\label{de-0autgroup}
Let  be a group with finite set of generators .
We say that the pair  is 0-\textrm{automatic} if
 is a cross-section of G.
\end{definition}

Such pairs were first considered by Stallings~\cite{stal} under
another name. It can be proved using the results from
Stallings that  is necessarily isomorphic to a plain group. 
However the set  may be larger than a natural (see Section \ref{sse-pmg})
set of generators of the plain group. 

\medskip

Now let us extend the notion to monoids. To get good properties it is
necessary to choose a more complex definition proposed in
\cite{mair04}. 

\begin{definition}\label{de-0autmonoid}
Let  be a monoid with finite set of generators .
Assume that  is
a cross-section. Let  be the
corresponding normal form map. Assume that:  s.t. 
, ,



Assume furthermore that :  such that ,

Then we say that the pair  is 0-automatic.
\end{definition}

In the group case , the conditions \eref{eq-monoid1} and
\eref{eq-monoid2} are implied by the fact that the language
 is a cross-section. 

\medskip

The pairs formed by a plain monoid and natural generators are
0-automatic. However, in contrast
with the group case, plain monoids do not exhaust the family
of monoids appearing in 0-automatic pairs. For instance, 
 is a 0-automatic pair, but the monoid 
is not isomorphic to a plain 
monoid. This example is studied in Section \ref{sse-5}. 

\begin{definition}\label{de-0aut4}
A triple  is said to be {\em 0-automatic} if: (i)
 is a 0-automatic pair with  infinite; (ii)  is a
probability measure whose support is included in  and
generates .
\end{definition}

Any plain triple, see Def. \ref{de-0aut3}, is 0-automatic. 
Observe that  may not be transient in a 0-automatic
triple. Also the graph of succesors  defined in
\eref{eq-graph} may not be strongly connected. 

\medskip

Consider a 0-automatic triple ,  , and  , see \eref{eq-cb}. 
The {\em 0-automatic queue} of type  is
defined exactly as in Definition \ref{de-0aut}. 

The definition of the {\em Traffic Equations} gets modified as
follows~:

The definition of the {\em Twisted Traffic Equations} is modified as well~:



We use the convention described after
\eref{eq-cb} to define a solution to the TTE belonging to .

The TE do not necessarily have a unique solution, in
contrast with
Prop. \ref{pr-solTE}.
The TTE as well do not necessarily have a unique solution, and a 
solution  to the TTE does not necessarily belong to
. This is in contrast with Theorem \ref{th-uniq}. 

An analog of Lemma \ref{le-tte1} holds: if  is a
solution to the TTE then either  and  is a solution
to the TE, or


We can now state the following result. Observe in particuler that
there may be several variants (corresponding to different 's) of
the 0-automatic queue with a product form. 

\begin{proposition}\label{pr-bis}
Let  be a 0-automatic triple.
Let  be a solution to the
 satisfying \eref{eq-etax2}. Assume that: . Define
 and
.

Consider the 0-automatic queue of type
. Let  be the infinitesimal generator
of the queue-content process .
Consider the measure  on  defined by:

We have . Besides, we have:

When , the stationary distribution of
 is . The corresponding
stationary departure process is a Poisson process of rate .
\end{proposition}

The proofs of Lemma \ref{lemma2}, Theorem \ref{th-main}, and Theorem
\ref{th-burke} are easily
adapted to get Prop. \ref{pr-bis}.


\subsection{Five illustrating examples}\label{sse-5}

We study five particular 0-automatic queues to illustrate the
above results. We focus on three aspects: (a) the stability
region; (b) the value of the {\em load} ; (c) the existence
of several stationary regimes (for the queues built on 0-automatic
triples instead of plain triples). 

When the model is simple enough,
the TTE can be solved explicitly 
to get closed form formulas as for   below.
In all cases and like any set of algebraic equations, the TTE can
be solved with any prescribed precision.

\medskip

Another goal is to convey the
idea that 0-automatic queues ought to be pertinent in several
modelling contexts, due to the flexibility in their definition.
The five examples below should be interpreted having in mind the 
different ``types'' of tasks detailed in the Introduction (classical,
positive/negative, ``one equals many'', and ``dating agency''). 













\subsubsection{The free product } 

Consider the plain triple , where ,
, .

In \cite[Section 4.2]{MaMa}, the drift is computed, it is given
by:


According to Theorem \ref{th-uniq}, in the stable case, the associated TTE have a
unique admissible solution that we denote by . Solving
the TTE, we get that:


\begin{figure}[ht]

\caption{: The stability region (left) and the
  load  (right).} \label{f-z3-z3}
\end{figure}

In Figure \ref{f-z3-z3} (left), we show the stability region of
the queue. The abscissa is  and the ordinate is .
In Figure \ref{f-z3-z3} (right), we plot the load  as a
function of  and , for  and
. Hence, 
is always smaller or equal to 1, see Theorem \ref{th-main}.

\subsubsection{The free product }

Consider the plain triple , where ,
, . In Figure \ref{f-n-b}, we illustrate
the corresponding buffering mechanism.

\begin{figure}[ht]

\caption{The queue  with  in white and
 in dark gray.} \label{f-n-b}
\end{figure}

The unique solution  of the TE is:
,  The drift of the
random walk is .

According to Theorem \ref{th-uniq}, the associated TTE have a
unique admissible solution that we denote by . Solving
the TTE, we obtain that  is a solution of , where:


The relation between  and  is given by: .

\begin{figure}[ht]

\caption{: The stability region (left) and
the load  (right).} \label{f-n-b-stab}
\end{figure}


In Figure \ref{f-n-b-stab} (left), we show the stability region of
the queue. The abscissa is  and the ordinate is .
In Figure \ref{f-n-b-stab} (right), we plot the load  as a
function of  and , for  and
. Hence,  is
always smaller or equal to 1, see Theorem \ref{th-main}.

\subsubsection{The free product }

Consider the queue associated with the plain triple  where
 and  with
.

\begin{figure}[ht]

\caption{Stability region of the  queue. The axis are , and . }
\label{f-n-z-b}
\end{figure}

The unique solution  of the associated TE is
 Applying
Theorem \ref{th-rw}, the drift of the random walk is given by:
.
From there, we obtain Figure \ref{f-n-z-b}: the stability region
is the region below the surface.

\subsubsection{The free group  and the free product }\label{ssse-fg}

Consider the 0-automatic queue
, where  is a
non-degenerate probability measure on . In
Figure \ref{fi-fg}, we illustrate the corresponding buffering
mechanism. Such a mechanism is similar to the one of Gelenbe's
G-queue. 

\begin{figure}[ht]

\caption{The  queue with  in light gray
  and  in dark gray.}
\label{fi-fg}
\end{figure}

The underlying triple  is not plain
(Def. \ref{de-0aut3}), but it is 0-automatic (Def. \ref{de-0aut4}). 
Here the graph of successors , see
\eref{eq-graph}, is not connected. Also the random walk  is
not transient but null-recurrent when .

The drift of the random walk is easily computed:


Assume first that . Solving the TTE, we get
that  is a solution for all . It means that the queue is stable and has a product form
distribution under any boundary condition. This interesting
behavior can be traced back to the fact that the random walk
 is not transient.

\medskip

Assume now that . There are 2 possible
solutions for the TTE:


The two solutions correspond to extremal values for , it means
that in the buffer, there is only one type of customer with
probability 1: if , there is only  in the buffer;
if , there is only  in the buffer. Here we
recover a model very close to the classical G-queue.

Set  and  and define  and
 accordingly. We have:

The stationary distribution of the 0-automatic queue  is:

where  if , and  if .
When , the 0-automatic queue  also has a product form stationary
distribution of the form \eref{eq-1or2} with  instead of
.

\medskip

Consider now a boundary condition . In particular,
, so the stationary distribution is not of
product form. However, if  , the
stationary distribution  can still be determined explicitly
by solving the global balance equations. It is given by:

where  and  are defined in \eref{eq-TTE-Z-1}.
The expression in \eref{eq-almost} is ``almost'' of product form. So, why do we
prefer an expression like the one in \eref{eq-1or2}~?
The point is that the departure process associated with a
stationary distribution of type \eref{eq-almost} is not Poisson, as opposed to
the one associated with \eref{eq-1or2}. And having a Poisson departure
process is crucial
to build product form networks, see \cite{DaMa06}.

\begin{figure}[ht]

\caption{ and : the loads as a
  function of .}
\label{fi-a-petit c}
\end{figure}

To summarize, when , there are two variants of the 0-automatic
queue with a product form.
We would like to argue that one of the two
makes more ``physical'' sense.

To that purpose,
consider the plain triple  with .
According to
Theorem \ref{th-uniq}, there exists a single variant of the queue with a
product form. Let  be the corresponding load. The question is to
determine which one of the two 
solutions in \eref{eq-TTE-Z-1} is recovered when letting  go to
0.

Since the TTE are difficult to solve explicitly, we content
ourselves with numerical evidence. In Figure \ref{fi-a-petit c},
we plot , , and  as functions
of , for  and . We see
that  tends to the larger solution . The two
vertical lines correspond to the stability regions. They have an
abscissa equal to the inverse of the drift 
 for the random walk on 
and  respectively.


\subsubsection{The monoid }

Consider the {\em bicyclic monoid} . Here we have
a new ``type'' 
of tasks. It is close to the positive/negative type but with no
symmetry between the positive and negative customers. 

Consider the triple , with
, . It is a 0-automatic triple but not a plain
triple. In particular, the graph of successors 
is not strongly connected. 

The drift of the random walk is easily
computed and given by .
Solving the associated TTE, we obtain that there is one solution
if  and two solutions if . More precisely,
these two solutions are:



We have:


In Figure \ref{fi-stable_c}, we show the solutions to the TTE as a function of
 and .

\begin{figure}[ht]

\caption{: The solutions to the TTE.}
\label{fi-stable_c}
\end{figure}


To discriminate between  and , we proceed
as for .
Consider . Set ,
,  with , .  The triple
 is 0-automatic. 

The TE can be solved explicitly. It turns out that there is a unique
solution  which
is determined by:


According to Theorem \ref{th-rw}, the corresponding drift is
.
Now, solving the TTE,
we find a unique admissible
solution , given by:


We have  iff
.

Let us observe what happens when  tends to 0. When
, there is only one solution  for the TTE of
the first case. And, as expected,  tends to
 when . When , there
are two possible solutions  and  for
the TTE of the first case. In this  case,  tends to
 when .


In Figure \ref{ab=1_c_p=2/5} (left), we
plot  and  as functions of  and ,
for . In Figure \ref{ab=1_c_p=2/5} (right), we
plot  and .


\begin{figure}[ht]
 
\caption{ and : the
  loads in function of  and .}
\label{ab=1_c_p=2/5}
\end{figure}




\begin{thebibliography}{10}

\bibitem{AlBo}
C.~Aliprantis and K.~Border.
\newblock {\em Infinite Dimensional Analysis: a Hitchhiker's Guide, 2nd ed.}
\newblock Springer-Verlag, Berlin, 1999.

\bibitem{asmu87}
S.~Asmussen.
\newblock {\em Applied Probability and Queues}.
\newblock John Wiley \& Sons, Chichester, 1987.

\bibitem{BaFo93b}
F.~Baccelli and S.~Foss.
\newblock On the saturation rule for the stability of queues.
\newblock {\em J. Appl. Probab.}, 32(2):494--507, 1995.

\bibitem{brem99}
P.~Br{\'e}maud.
\newblock {\em Markov chains: Gibbs fields, Monte Carlo simulation, and
  queues}, volume~31 of {\em Texts in Applied Mathematics}.
\newblock Springer-Verlag, New York, 1999.

\bibitem{ChMi00}
X.~Chao and M.~Miyazawa.
\newblock Queueing networks with instantaneous movements: a unified approach by
  quasi-reversibility.
\newblock {\em Adv. in Appl. Probab.}, 32(1):284--313, 2000.

\bibitem{CMPi}
X.~Chao, M.~Miyazawa, and M.~Pinedo.
\newblock {\em Queueing networks. Customers, signals, and product form
  solutions}.
\newblock Wiley, 1999.

\bibitem{cohe82}
J.W. Cohen.
\newblock {\em The single server queue}.
\newblock North-Holland, Amsterdam, 1982.
\newblock 2nd edition.

\bibitem{DaMa05}
T.-H. Dao-Thi and J.~Mairesse.
\newblock Zero-automatic queues.
\newblock In {\em Formal Techniques for Computer Systems and Business
  Processes}, volume 3670 of {\em LNCS}, pages 64--78. Springer-Verlag, 2005.

\bibitem{DaMa06}
T.-H. Dao-Thi and J.~Mairesse.
\newblock Zero-automatic networks.
\newblock In {\em Proceedings of Valuetools, Pisa, Italy}. ACM, 2006.

\bibitem{DaQu}
T.~Dayar and F.~Quessette.
\newblock Quasi-birth-and-death processes with level-geometric distribution.
\newblock {\em SIAM J. Matrix Anal. Appl.}, 24(1):281--291, 2002.

\bibitem{DyMa}
E.~Dynkin and M.~Malyutov.
\newblock Random walk on groups with a finite number of generators.
\newblock {\em Sov. Math. Dokl.}, 2:399--402, 1961.

\bibitem{ECHLPT}
D.~Epstein, J.~Cannon, D.~Holt, S.~Levy, M.~Paterson, and W.~Thurston.
\newblock {\em Word processing in groups}.
\newblock Jones and Bartlett, Boston, 1992.

\bibitem{FGSu}
J.-M. Fourneau, E.~Gelenbe, and R.~Suros.
\newblock {}-networks with multiple classes of negative and positive
  customers.
\newblock {\em Theoret. Comput. Sci.}, 155(1):141--156, 1996.

\bibitem{gele91}
E.~Gelenbe.
\newblock Product-form queueing networks with negative and positive customers.
\newblock {\em J. Appl. Probab.}, 28(3), 1991.

\bibitem{GePu}
E.~Gelenbe and G.~Pujolle.
\newblock {\em Introduction to queueing networks. 2nd ed.}
\newblock John Wiley \& Sons, Chichester, 1998.

\bibitem{guiv80}
Y.~Guivarc'h.
\newblock Sur la loi des grands nombres et le rayon spectral d'une marche
  al\'eatoire.
\newblock {\em Ast\'erisque}, 74:47--98, 1980.

\bibitem{kell79}
F.~Kelly.
\newblock {\em Reversibility and Stochastic Networks}.
\newblock Wiley, New-York, 1979.

\bibitem{he}
Q.-M. He.
\newblock The classification of matrix {}-type {M}arkov chains with a
  tree structure and its applications to queueing.
\newblock {\em J. Appl. Probab.}, 40(4):1087--1102, 2003.


\bibitem{LaRa}
G.~Latouche and V.~Ramaswami.
\newblock {\em Introduction to matrix analytic methods in stochastic modeling}.
\newblock ASA-SIAM Series on Statistics and Applied Probability. SIAM,
  Philadelphia, PA, 1999.

\bibitem{ledr00}
F.~Ledrappier.
\newblock Some asymptotic properties of random walks on free groups.
\newblock In J.~Taylor, editor, {\em Topics in probability and Lie groups:
  boundary theory}, number~28 in CRM Proc. Lect. Notes, pages 117--152.
  American Mathematical Society, 2001.

\bibitem{mair04}
J.~Mairesse.
\newblock Random walks on groups and monoids with a {M}arkovian harmonic
  measure.
\newblock {\em Electron. J. Probab.}, Vol. 10, p. 1417-1441, 2005

\bibitem{MaMa}
J.~Mairesse and F.~Math\'eus.
\newblock Random walks on free products of cyclic groups.
\newblock To appear in {\em J. London Math. Soc.}, 2006

\bibitem{neut}
M.~Neuts.
\newblock {\em Matrix-geometric solutions in stochastic models: An algorithmic
  approach}.
\newblock Johns Hopkins University Press, Baltimore, Md., 1981.

\bibitem{SaSt}
S.~Sawyer and T.~Steger.
\newblock The rate of escape for anisotropic random walks in a tree.
\newblock {\em Probab. Theory Related Fields}, 76(2):207--230, 1987.

\bibitem{serf99}
R. Serfozo.
\newblock {\em Introduction to Stochastic Networks}.
\newblock Springer-Verlag, Berlin, 1999.

\bibitem{stal}
J.~Stallings.
\newblock A remark about the description of free products of groups.
\newblock {\em Proc. Cambridge Philos. Soc.}, 62:129--134, 1966.

\bibitem{woes}
W.~Woess.
\newblock {\em Random walks on infinite graphs and groups}.
\newblock Number 138 in Cambridge Tracts in Mathematics. Cambridge University
  Press, 2000.

\bibitem{YeSe}
R.~Yeung and B.~Sengupta.
\newblock Matrix product-form solutions for {M}arkov chains with a tree
  structure.
\newblock {\em Adv. in Appl. Probab.}, 26(4):965--987, 1994.


\end{thebibliography}


\end{document}
