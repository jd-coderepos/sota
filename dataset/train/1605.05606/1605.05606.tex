\documentclass[10pt]{sig-alternate-05-2015} 
\setlength{\paperheight}{11in}
\setlength{\paperwidth}{8.5in}
\usepackage{array,multirow}
\pagenumbering{arabic}
\usepackage{nohyperref}
\usepackage{flushend}
\usepackage{graphicx} 
\usepackage[usenames, dvipsnames]{color} 
\usepackage{url} \usepackage{amsmath}
\usepackage{amssymb}
\usepackage{cite}
\usepackage{booktabs}
\usepackage{amssymb}
\usepackage{xspace}
\usepackage{times}
\usepackage{subfigure}
\usepackage{enumitem}


\newcommand\xref[1]{\S~\ref{#1}}
\newcommand\pxref[1]{(\xref{#1})}

\newcommand{\neta}{Netalyzr\xspace}
\newcommand{\parax}[1]{\vspace{0.2em} \noindent \textbf{#1:}}

\sloppy	\DeclareFixedFont{\afacc}{OT1}{phv}{m}{n}{10}


\newcommand{\note}[1]{{\textcolor{red}{[\textit{#1}]}}}
\newcommand{\todo}[1]{\note{\textcolor{Red}{#1 --TODO}}}

\newcommand{\vp}[1]{\note{\textcolor{RedOrange}{#1 --vp}}}
\newcommand{\fw}[1]{\note{\textcolor{Mulberry}{#1 --fw}}}
\newcommand{\nv}[1]{\note{\textcolor{Red}{#1 --nv}}}
\newcommand{\pr}[1]{\note{\textcolor{Green}{#1 --pr}}}
\newcommand{\ck}[1]{\note{\textcolor{Brown}{#1 --cpk}}}
\newcommand{\ma}[1]{\note{\textcolor{Blue}{#1 --ma}}}
    
\def\blackI{\ding{202}}
\def\blackII{\ding{203}}
\def\blackIII{\ding{204}}
\def\blackIV{\ding{205}}
\def\blackV{\ding{206}}
\def\blackVI{\ding{207}}

\def\first{({\it i})\xspace}
\def\second{({\it ii})\xspace}
\def\third{({\it iii})\xspace}
\def\fourth{({\it iv})\xspace}
\def\fifth{({\it v})\xspace}
\def\sixth{({\it vi})\xspace}

\def\TblSpT{\rule[-1ex]{0pt}{0pt}}
\def\TblSpB{\rule{0pt}{3ex}}

\def\yes{\checkmark}
\def\no{\ding{56}}

\def\UrlNoBreaks{\do:\do\do\{\do\<}\def\UrlFont{\small\ttfamily}
\def\UrlOrds{\do\*\do\~}

\def\na{Netalyzr\xspace}


\providecommand{\vs}{vs.\ }
\providecommand{\ie}{{i.e.}, }
\providecommand{\eg}{{e.g.}, }
\providecommand{\cf}{\emph{cf.,} }
\providecommand{\resp}{\emph{resp.,} }
\providecommand{\etal}{\emph{et al.}\xspace}
\providecommand{\etc}{\emph{etc.}} 


\newcommand\inlinesection[1]{{\bf #1.}}


\newcommand{\privateipport}{\xspace}
\newcommand{\publicipport}{\xspace}
\newcommand{\dstipport}{\xspace}

\newcommand{\privateport}{\xspace}
\newcommand{\publicport}{\xspace}

\newcommand{\privateip}{\xspace}
\newcommand{\publicip}{\xspace}
\newcommand{\dstip}{\xspace}

\newcommand{\btping}{\textit{bt\_ping}\xspace}
\newcommand{\btfindnodes}{\textit{find\_nodes}\xspace}
\newcommand{\btipport}{\textit{IP:port}\xspace}
\newcommand{\btnodeid}{\textit{nodeid}\xspace}
\newcommand{\btnodeids}{\textit{nodeids}\xspace}

 
\begin{document}

\CopyrightYear{2016}
\setcopyright{acmlicensed}
\conferenceinfo{IMC 2016,}{November 14 - 16, 2016, Santa Monica, CA, USA}
\isbn{978-1-4503-4526-2/16/11}\acmPrice{\^1^2^3^3^5^1^{3,6}^{3}^{3,4}^1^2^3^4^5^6iiiABC=ABABBBABBAABABID_{us}ID_{us}ID_{us}1,387ID_{us}IP_{dev}IP_{dev}IP_{cpe}NNN(i)IP_{dev}(ii)IP_{cpe}(iii)IP_{pub}IP_{dev}IP_{cpe}(i)(ii)(iii)IP_{pub}(iv)IP_{pub}\cupIP_{dev}IP_{dev}IP_{dev}IP_{dev}IP_{dev}IP_{cpe}IP_{cpe}IP_{cpe}IP_{cpe}IP_{pub}IP_{cpe}IP_{dev}IP_{cpe}IP_{cpe}N~\ge~100.4~\times~Nxy(i)(ii)(iii)\leq<\leq<\leqt_{idle}ttl_cttl_st_{idle}i \leq ttl_c < jttl_s < n-jt_{exp} < t_{idle}j\sim$60), we can detect and filter results with 
unstable paths.

\begin{table}
\center 
\footnotesize
\resizebox{\columnwidth}{!}{
\begin{tabular}{l c c}
 & \textbf{CGN detected} & \textbf{No CGN detected} \\
\toprule
\textbf{IP address mismatch} & 67.6\%{} & 30.9\%{} \\
\textbf{IP address match} & 0.5\%{} & 0.9\%{} \\
\end{tabular}
}
\caption{Detection rate of TTL-driven NAT enumeration.}
\label{table:ttltestnumbers}
\end{table}

\parax{STUN test} To study the mapping types of CGNs, we implemented a 
STUN~\cite{rfc5389} test in our Netalyzr test suite in October 2015. 
STUN determines the mapping type implemented by on-path NATs. STUN sends probe 
packets to a public STUN server (which answers certain probe packets from a 
different port and/or IP address) and waits for the respective 
replies.\footnote{For more details on the operation of STUN, we refer 
to~\cite{rfc5389}.}

From the TTL-driven NAT enumeration test (deployed in September 2014) we have 
collected more than 38K{} sessions, whereas the STUN
test (deployed in October 2015) produced 23K{} 
sessions. 
To be able to contrast sessions from within CGN-positive networks against 
CGN-negative ones, we augment the results from both tests with the 
results from our CGN detection tests (\xref{sec:cgndetection}).
We further apply filtering rules to the results of both tests to ensure 
that we have collected at least three sessions from a particular network 
(combination of AS number and CGN classification type, e.g. 
``cellular CGN''). After applying the 
filters, this leaves us with 18K{} sessions from the NAT 
enumeration test running via both non-cellular (70\%{})
and cellular networks (30\%{}).
The results cover 608{} ASes, whereof 43\%{} 
(259{} ASes) deploy CGN.
For the STUN test we count 20K{} sessions from non-cellular
(87\%{}) and cellular networks (13\%{}).
The STUN results span 720{} ASes including 170{} 
CGN-positive ASes (24\%{}).





\subsection{Topological Properties of CGNs}
\label{sec:cgnlocation}



\begin{figure}
  \begin{center}
    \includegraphics[width=0.99\linewidth]{figures/ttltest_outermost_nat}
  \caption{Maximum NAT distance from the subscriber.}
    \label{fig:natdistance}
    \vspace{-1em}
  \end{center}
\end{figure}

Figure~\ref{fig:natdistance} shows the distribution of the number of hops 
between the client and the most distant NAT detected, grouped per AS and its 
respective CGN deployment status. We detected NATs as far away as 
18{} hops from the client. As expected, most of the NATs 
in CGN-negative ASes (92\%{}) 
sit just one hop away from the client, i.e. they are typically located right on 
the CPE router. Compared to that, most CGNs are located two to five hops 
away from the client (64\%{} of 
non-cellular and 73\%{} of cellular 
ASes). In non-cellular ASes the CGN distance mostly ranges from two hops up 
to six hops. In the case of cellular ASes, however, the CGN distance ranges 
from one hop to two hops and up to 12 hops away from the client. In fact, we 
find that for 10\%{} of the cellular ASes, 
the CGN is located six or more hops away from the client. A large number of 
hops between client and CGN hints at a centralized CGN infrastructure with 
large aggregation points, which has the potential of affecting the accuracy of 
IP geolocation databases when locating the external IP address of 
clients behind CGN.



\begin{figure}
  \begin{center}
    \includegraphics[width=0.99\linewidth]{figures/ttltest_udp_timeout_boxplot-crop}
  \caption{UDP mapping timeouts of CPEs and CGNs.}
  \vspace{-1em}
    \label{fig:nattimeout}
  \end{center}
\end{figure}

\subsection{Flow-Mapping Properties of CGNs}
\label{sec:flowmappingprops}

The type of NAT mapping (recall \xref{sec:background}) as well as its
state-keeping timeout directly affect the reachability of a host
located behind a NAT, and thus has a profound effect on applications that rely
on peer-to-peer connectivity \cite{fordHolePunching, d2009measurement} or
long-lived sparse flows~\cite{Wang:2011:USM:2043164.2018479}.

\parax{Mapping Timeouts}
Figure~\ref{fig:nattimeout} shows the UDP mapping timeouts for the 
detected CGNs, both in the cellular, as well as in the non-cellular case. Here, 
we aggregate our CGN-positive sessions on a per-AS level. An AS is 
represented by its most frequent timeout value (mode).
We also report timeout values that we detected for CPE devices (shown in the 
right boxplot), where we show a boxplot of all recorded sessions. In NAT444 
scenarios (non-cellular CGN) we need to make sure to report the timeout of CGNs 
rather than the CPE NATs. Therefore, to reason about CGN mapping timeouts, we 
only consider sessions that were detected as CGN (\xref{sec:cgndetection}) and 
where our TTL-driven NAT enumeration detected the NAT at a distance of three or 
more hops away from the client.
We observe that 74\%{} of detected NATs expire idle UDP 
state after 1 minute or less, but we find values ranging from 10s to 
200s.\footnote{Note that our timeout detection mechanism uses a 10 second 
probing interval. Hence, reported values can differ up to 10 seconds from the 
actual NAT timeout.}
CGNs in cellular networks exhibit a larger median mapping timeout 
(65{}s) compared to non-cellular networks 
(35{}s). For CPE NATs we predominantly measured 
a timeout of 65{}s. We find higher variability and 
a lower median of timeout values for non-cellular CGNs when compared to CPE 
NATs. Low CGN timeout values might in turn negatively affect the longevity of 
sparse UDP flows that are also exposed to CPE NATs. While we find lower timeout 
values for CGNs compared to CPEs, we acknowledge that this property does not 
necessarily hold true for CGNs in general, as our test can not detect timeout 
values larger than 200 seconds.

\begin{figure}
  \begin{center}
  \subfigure[Distribution of observed STUN types in CPE NATs.]{
    \includegraphics[width=0.95\linewidth]{figures/stun_cpe-crop}
    \label{fig:stunnocgn}
  }
\subfigure[Most permissive STUN type per AS (only CGN sessions).]{
    \includegraphics[width=0.95\linewidth]{figures/stun_results-crop}
    \label{fig:stuntypes}
  }
\caption{STUN results per AS.}
  \vspace{-1em}
  \label{fig:stun}
  \end{center}
\end{figure}

\parax{Mapping types}
Figure~\ref{fig:stun} shows our STUN results. We order the observed 
mapping types from most restrictive (\textit{symmetric NAT}) to most permissive 
(\textit{full cone NAT}).
In Figure \ref{fig:stunnocgn} we show the NAT mapping type as observed
for CPE routers, while the bars in Figure \ref{fig:stuntypes} indicate
the most permissive type of NAT mapping for our CGN-positive ASes.
Recall that when multiple NAT devices reside on the path, STUN reports the most restrictive behavior of them, which also determines 
eventual NAT traversal. Hence, we argue that the most permissive STUN type 
provides a good approximation for the CGN behavior, 
because there cannot be a STUN result less restrictive than the CGN. 
We observe that, while exhibiting some diversity, less than 2\% 
of the tests showed CPE NATs with very 
restrictive symmetric NATs. In contrast to CPE NATs, we observe 
11\%{} of non-cellular CGN ASes whose most permissive 
mapping type is symmetric. Among these networks we find many 
popular large European ISPs.
For cellular networks we observe a bimodal outcome, with a 
large fraction of both restrictive (40\%{} 
symmetric) and permissive (20\%{} full cone) NAT 
types. We see large operators on both ends of the spectrum, with major cellular 
networks in the US deploying CGNs with symmetric mapping types.

 


Thus, we often measure stricter NAT mapping policies for CGN-positive sessions 
when compared to common home CPE devices. We conclude that a large fraction of 
ISPs deploy CGNs that use symmetric flow mappings, which limits the customers' 
ability to establish direct connections. For this reason, the IETF lists an 
endpoint-independent mapping (which symmetric NATs violate) as their first 
requirement for CGNs~\cite{rfc4787,rfc5382}.

\section{Implications} 



Our analysis shows that ISPs widely deploy CGN. We find that more than 17\% 
of eyeball ASes and more than 90\% of cellular ASes rely on CGNs 
(\xref{sec:networkwideview}), with particularly high deployment rates in Asia 
and Europe---regions in which IPv4 address scarcity cropped up first, as the 
respective registries ran out of readily available IPv4 addresses in 2011 and 
2012. Thus, adopting CGN presents
a viable alternative to buying IPv4 address space from brokers.
CGNs actively extend the lifetime of IPv4 and hence also fuel the 
demand of the growing market for IPv4 address space \cite{RABP14}, which in 
turn affects market prices and possibly hampers the adoption IPv6.

CGNs directly affect ``how much Internet'' a subscriber receives, by \first
limiting available ephemeral port space, \second restricting 
the directionality of connections, and \third limiting connection 
lifetimes due to finite state-keeping budgets.  Studying our identified CGN 
deployments, we find 
a wide spectrum of configurations and degrees of address sharing 
(\xref{sec:drilling}). On the limiting end of the spectrum, we find ISPs 
allocating as little as 512 ephemeral ports per subscriber
(\xref{sec:ipportallocation}), multiplexing up to 128 subscribers per public 
IP address. Comparing NAT flow mapping types and timeouts of CGNs to commonly 
deployed CPE hardware, we find that in many instances CGNs use more 
restrictive flow mapping types when compared to their home counterparts 
(\xref{sec:flowmappingprops}). This rules out peer-to-peer connectivity,
complicating modern protocols such as WebRTC~\cite{rfc7478} that now need to
rely on rendezvous servers.

We argue that the lack of \textit{guidelines} and \textit{regulations}
for CGN deployment compounds the situation. While the IETF publishes
best practices for general NAT behavior \cite{rfc4787,rfc5382,rfc7857}
as well as basic requirements for CGN deployments \cite{rfc6888}
(which, incidentally, many of our identified CGNs violate),
dimensioning NATs at carrier-scale in a way that minimizes collateral
damage remains a black art. Our finding that some large ISPs find the
need to employ publicly routable (indeed, sometimes routed) address
space for internal CGN deployment (\xref{sec:privateaddressspaces})
underlines the graveness of the situation.
While it remains out of scope for this work to precisely measure the
effect of CGNs on end-users' applications, we believe that our
observations can serve as input for establishing such guidelines.
Our findings should also interest regulators, who in some countries
already impose acceptable service requirements on Internet performance
(e.g., the FCC's measurements of advertised vs. achieved throughput
\cite{johnston2013measuring}).  We argue that the presence and service
levels of CGNs should be readily identifiable in ISPs'
offerings. Unfortunately, we find that most ISPs do not cover CGN
deployment in their terms of service.
Lastly, our findings document further erosion of the meaningfulness of
IP address reputation, address-based blacklisting, IP-to-user
attribution, and geolocating end-users (\xref{sec:ipportallocation},
\xref {sec:cgnlocation}), which become all but infeasible in the
presence of CGN.






\section{Conclusion}
This work presents a solid first step towards a better understanding
of the prevalence and characteristics of CGN deployment in today's
Internet. Our methods, based on harvesting the BitTorrent DHT and
extensions to our \neta active measurement framework, prove effective
in uncovering CGN deployments: we detect more than 500 instances in
ISPs around the world. When assessing the properties of these CGNs we
find striking variability in the dimensioning, configuration,
placement, and effect on the subscriber's connectivity. We hope this
study will stimulate a much-needed discussion about best practices,
guidelines, and regulation of CGN deployment. In future work, we plan
to further improve our detection mechanisms and to study the impact of
CGN on application performance.

\section*{Acknowledgments}
\label{sec:acks}

We thank the network operators who participated in our survey for their 
insightful feedback and comments. We thank Daniele Iamartino for support with 
validating the BitTorrent data, and Martin Ott for his support with implementing
the BitTorrent crawler. We also thank the anonymous reviewers for their helpful 
feedback. This work was partially supported by the
US National Science Foundation under grants CNS-1111672 and CNS-1213157,
the Leibniz Prize 
project funds of DFG/German Research Foundation (FKZ FE 570/4-1), and
the BMBF AutoMon project (16KIS0411).  Any
opinions expressed are those of the authors and not necessarily of
the sponsors.

\bibliographystyle{plain}
\bibliography{paper}





\end{document}
