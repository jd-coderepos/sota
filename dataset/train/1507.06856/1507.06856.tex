\documentclass[12pt]{article}

\usepackage{amssymb}
\usepackage{fullpage}
\usepackage{graphicx}

\newtheorem{theorem}{Theorem}
\newtheorem{lemma}{Lemma}
\newtheorem{definition}{Definition}

\newcommand{\IR}{\mathbb{R}}
\newcommand{\IS}{\mathbb{S}}

\newcommand{\skel}{\mathord{\it skel}}
\newcommand{\Ann}{\mathord{\it Ann}}
\newcommand{\Shell}{\mathord{\it Shell}}
\newcommand{\Dil}{\mathord{\it Dil}}

\newcommand{\qed}{\rule{0.5em}{1.5ex}}
\newcommand{\fqed}{{\hfill~\qed}}
\newenvironment{proof}{{\noindent \bf Proof.}}
                      {{\hfill \fqed} \vspace{1em}}

\setlength{\textfloatsep}{5ex}
\newcommand{\topfigrule}{\noindent\rule[-0.5cm]{\textwidth}{0.2mm}\vspace*{-6pt}
\noindent}
\newcommand{\botfigrule}{\noindent\rule[-0.5cm]{\textwidth}{0.2mm}}



\title{On the Stretch Factor of Convex Polyhedra whose Vertices are 
      (Almost) on a Sphere}
\author{
Prosenjit Bose\thanks{School of Computer Science, Carleton University, 
    Ottawa, Canada. These authors were supported by the 
    Natural Sciences and Engineering Research Council of Canada. 
    D.H.\ was supported by an Ontario Graduate Scholarship.}
\and 
Paz Carmi\thanks{Department of Computer Science, Ben-Gurion University 
   of the Negev, Israel.} 
\and 
Mirela Damian\thanks{Department of Computer Science, Villanova 
   University, Villanova, PA 19403, USA. Supported by NSF grant 
   CCF-1218814.}
\and 
Jean-Lou De Carufel\thanks{School of Electrical Engineering and 
    Computer Science, University of Ottawa, Canada.} 
\and  
Darryl Hill\footnotemark[1] 
\and 
Anil Maheshwari\footnotemark[1] 
\and
Yuyang Liu\footnotemark[1] 
\and 
Michiel Smid\footnotemark[1] 
} 
\date{\today}

\begin{document} 

\maketitle 

\begin{abstract} 
Let  be a convex simplicial polyhedron in . The skeleton of 
 is the graph whose vertices and edges are the vertices and edges 
of , respectively. We prove that, if these vertices are on a 
sphere, the skeleton is a -spanner. If the 
vertices are very close to a sphere, then the skeleton is not 
necessarily a spanner. For the case when the boundary of  is between 
two concentric spheres of radii  and , where , and the 
angles in all faces are at least , we prove that the skeleton 
is a -spanner, where  depends only on  and .  
One of the ingredients in the proof is a tight upper bound on 
the geometric dilation of a convex cycle that is contained in an 
annulus. 
\end{abstract} 


\section{Introduction}
Let  be a finite set of points in Euclidean space and let  be a 
graph with vertex set . We denote the Euclidean distance between any 
two points  and  by . Let the length of any edge  in  
be equal to , and define the length of a path in  to be the sum 
of the lengths of the edges on this path. For any two vertices  and 
 in , we denote by  the length of a shortest path in  
between  and . For a real number , we say that  is a 
\emph{-spanner} of , if  for all vertices  
and . The \emph{stretch factor} of  is the smallest value of  
such that  is a Euclidean -spanner of . See~\cite{ns-gsn-07} 
for an overview of results on Euclidean spanners.

It is well-known that the stretch factor of the Delaunay triangulation
in  is bounded from above by a constant. The first proof of this
fact is due to Dobkin \emph{et al.}~\cite{dfs-dgaag-90}, who obtained
an upper bound of . The currently best
known upper bound, due to Xia~\cite{x-sfdtl-13}, is .

Let  be a convex simplicial polyhedron in , i.e., all faces
of  are triangles. The \emph{skeleton} of , denoted by , 
is the graph whose vertex and edge sets are equal to the vertex and edge 
sets of . 

Since there is a close connection between Delaunay triangulations in 
 and convex hulls in , it is natural to ask if the skeleton 
of a convex simplicial polyhedron in  has a bounded stretch 
factor. By taking a long and skinny convex polyhedron, however, this is 
clearly not the case. 

In 1987, Raghavan suggested, in a private communication to
Dobkin \emph{et al.}~\cite{dfs-dgaag-90}, that the skeleton of a convex 
simplicial polyhedron, all of whose vertices are on a sphere, has 
bounded stretch factor. Consider such a polyhedron . By a 
translation and scaling, we may assume that the vertex set  of  
is on the unit-sphere 
 
Observe that this does not change the stretch factor of 's skeleton.  
It is well-known that the convex hull of  (i.e., the polyhedron ) 
has the same combinatorial structure as the spherical Delaunay 
triangulation of ; this was first  observed by 
Brown~\cite{b-gtfga-80}. Based on this, 
Bose \emph{et al.}~\cite{bps-chpss-14} showed that the proof of   
Dobkin \emph{et al.}~\cite{dfs-dgaag-90} can be modified to prove that 
the skeleton of  is a -spanner of its vertex set , where 
.  

In Section~\ref{seconsphere}, we improve the upper bound on the stretch 
factor to . Our proof considers any two 
vertices  and  of  and the plane  through , , 
and the origin. The great arc on  connecting  and  is 
contained in . The path on the convex polygon 
 that is on the same side of  as this 
great arc passes through a sequence of triangular faces of . 
An edge-unfolding of these faces results in a sequence of triangles in 
a plane, whose circumdisks form a \emph{chain of disks}, as defined 
by Xia~\cite{x-sfdtl-13}. The results of Xia then imply the upper 
bound of  on the stretch factor of the skeleton of . 

A natural question is whether a similar result holds for a convex 
simplicial polyhedron whose vertices are ``almost'' on a sphere. 
In Section~\ref{secalmost}, we show that this is not the case: We give 
an example of a set of points that are very close to a sphere, such that 
the skeleton of their convex hull has an unbounded stretch factor. 

In Section~\ref{secACSH}, we consider convex simplicial polyhedra  
whose boundaries are between two concentric spheres of radii  and 
, where , that contain the common center of these spheres, 
and in which the angles in all faces are at least . 
We may assume that the two spheres are centered at the origin. We 
present an improvement of a result by 
Karavelas and Guibas~\cite{kg-skgsa-01}, i.e., we show that for any two 
vertices  and , their shortest-path distance in the skeleton of 
 is at most  times their shortest-path 
distance along the surface of . The latter shortest-path distance is 
at most the shortest-path distance between  and  along the 
boundary of the convex polygon  which is obtained by 
intersecting  with the plane through , , and the origin. This 
convex polygon contains the origin and its boundary is contained 
between the two circles of radii  and  that are centered at the 
origin. Gr{\"u}ne~\cite[Lemma~2.40]{g-gdhd-06} has 
shown that the stretch factor of any such polygon is at most 
 
provided that . In Section~\ref{secannulus}, we improve 
this upper bound to 

which is valid, and tight, for all . As a result, the stretch 
factor of the skeleton of  is at most 
 


\section{Convex Polyhedra whose Vertices are on a Sphere}  
\label{seconsphere} 
In this section, we prove an upper bound on the stretch factor of the 
skeleton of a convex simplicial polyhedron whose vertices are on a 
sphere. As we will see in Section~\ref{subsecSFCP}, our 
upper bound follows from Xia's upper bound in~\cite{x-sfdtl-13} on the 
stretch factor of chains of disks in . We start by reviewing 
such chains. 

\subsection{Chains of Disks} \label{secCoD}
Let  be a sequence of disks in 
, where . For each  with , define

i.e.,  is that part of the boundary of  that is contained 
in . Similarly, for each  with , define 
 
The sequence  of disks is called a \emph{chain of disks}, 
if  
\begin{enumerate} 
\item for each  with , the circles  and 
       intersect in exactly one or two points, and  
\item for each  with , the circular arcs  
      and  have at most one point in common. 
\end{enumerate} 
See Figure~\ref{figCoD} for an example. 

\begin{figure}
\begin{center}
\includegraphics[scale=0.7]{chainofdisks.pdf}
\end{center}
\caption{The top figure shows a chain 
     of disks. The bottom figure 
    shows the graph ; the edges of this graph are 
    black. The edge  has length zero; it consists of just the 
    point  (which is equal to ).}
\label{figCoD}
\end{figure}

Let  and  be two distinct points in the plane such that
\begin{enumerate} 
\item  is on  and not in the interior of , and  
\item  is on  and not in the interior of .  
\end{enumerate} 
For each  with , let  and  be the intersection 
points of the circles  and , where 
 if these two circles are tangent. We label these intersection 
points in such a way that  is on or to the left of the directed 
line from the center of  to the center of , and  is on 
or to the right of this line. Define , , , and 
. For each  with , let  be the circular 
arc on  connecting the points  and  that 
is completely on the same side of  as  and , and 
let  be the circular arc on  connecting the points 
 and  that is on the same side of  as  and 
. 

Consider the graph  with vertex set 
 and edge set 
consisting of 
\begin{itemize}
\item the circular arcs , 
\item the circular arcs , and 
\item the line segments , , \ldots , .
\end{itemize} 
Figure~\ref{figCoD} shows an example. 
 
For each  with , the lengths of the edges  and 
 are equal to the lengths  and  of these arcs,
respectively. For each  with , the length of the edge 
 is equal to . The length of a shortest path in 
 is denoted by . 

\begin{theorem}[Xia~\cite{x-sfdtl-13}]   \label{thmxia} 
Let  be the length of any polygonal path that starts at , ends at 
, and intersects the line segments 
, , \ldots ,  in this order. Then, 
  
\end{theorem} 


\subsection{Bounding the Stretch Factor}  \label{subsecSFCP} 
Let  be a convex simplicial polyhedron in  and assume that 
all vertices of  are on a sphere. By a translation and scaling, we 
may assume that this sphere is the unit-sphere  (without changing 
the stretch factor of 's skeleton). We assume that 
(i) no four vertices of  are co-planar and (ii) the plane through 
any three vertices of  does not contain the origin. 

Fix two distinct vertices  and  of . We will prove that 
, i.e., the length of a shortest path in the skeleton 
 of , is at most . If  is 
an edge of , then this claim obviously holds. We assume from 
now on that  is not an edge of . 

Our proof will use the following notation (refer to 
Figure~\ref{figdefs}): 

\begin{figure}
\begin{center}
\includegraphics[scale=0.6]{defs.pdf}
\end{center}
\caption{Illustrating the notation in Section~\ref{subsecSFCP}.}  
\label{figdefs}
\end{figure}

\begin{itemize} 
\item : the plane through , , and the origin (i.e., the 
      center of ). 
\item : the circle . 
\item : the shorter arc of  connecting  and . 
\item : the convex polygon . 
\item : the path along  from  to  that is on 
      the same side of the line segment  as the arc ; 
      observe that  is a path between  and  along the 
      surface of . 
\item : the sequence of faces of  that the path 
       passes through. Observe that .
\end{itemize} 
Let  be the sequence of triangles obtained from 
an edge-unfolding of the triangles . Thus, 
\begin{itemize}
\item all triangles  are contained in one plane, 
\item for each  with , the triangles  and  
      are congruent, and 
\item for each  with , the triangles  and 
       share an edge, which is the ``same'' edge that is shared 
      by  and , and the interiors of  and  
      are disjoint.  
\end{itemize} 
For each  with , let  be the circumdisk of the 
triangle . Let  and let 
 and  be the vertices of  and  corresponding to  
and , respectively. We will prove the following lemma in 
Section~\ref{seclemCoD}.  

\begin{lemma}  \label{lemCoD}  
The following properties hold: 
\begin{enumerate} 
\item The sequence  is a chain of disks. 
\item  is on  and not in the interior of . 
\item  is on  and not in the interior of .  
\end{enumerate} 
\end{lemma} 

Consider the graph  that is defined by 
 and the two points  and ; see 
Section~\ref{secCoD}. We first observe that  is at most 
the shortest-path distance between  and  in the graph consisting 
of all vertices and edges of the faces . The latter 
shortest-path distance is equal to the shortest-path distance between 
 and  in the graph consisting of all vertices and edges of the 
triangles . Since the latter shortest-path 
distance is at most , it follows that 
 
Let  be the path through  
corresponding to the path . By Lemma~\ref{lemCoD} and 
Theorem~\ref{thmxia}, we have 

Since , it follows that 
 

It remains to bound  in terms of the Euclidean distance 
. Consider again the plane  through , , and the 
origin, the circle , the shorter arc 
 of  connecting  and , and the convex polygon 
. Observe that both  and  are on , 
and both these points are vertices of . Moreover,  is 
contained in the disk with boundary . It follows that 
 

Let  be the angle between the two vectors from the 
origin (which is the center of ) to  and ; see the figure 
below. 

\begin{center}
\includegraphics[scale=0.60]{figalpha.pdf}
\end{center}

Since  has radius , we have  and 
. Therefore, 

The function  is increasing for  
(because its derivative is positive for ), implying 
that

By combining the above inequalities, we obtain 

Thus, assuming Lemma~\ref{lemCoD} holds, we have proved the following 
result. 

\begin{theorem}   \label{thm999}  
Let  be a convex simplicial polyhedron in , all of whose 
vertices are on a sphere. Assume that no four vertices of  are 
co-planar and the plane through any three vertices of  does not 
contain the center of the sphere. Then the skeleton of  is a 
-spanner of the vertex set of , where 
 
\end{theorem} 


\subsection{Proof of Lemma~\ref{lemCoD}}  \label{seclemCoD}  
Lemma~\ref{lemCoD} will follow from Lemma~\ref{lemlocallyD} below.  
The proof of the latter lemma uses an additional result: 

\begin{lemma}   \label{lembelow}
Let  be an integer with . The polyhedron  and the 
origin are in the same closed halfspace that is bounded by the plane 
through the face  of .  
\end{lemma}
\begin{proof} 
Let  be the edge of the convex polygon  that spans the 
face . Since the path  (which contains  as an edge) 
is on the same side of the line segment  as the arc , and 
since the origin is on the other side of this line segment, the polygon 
 and the origin are in the same closed halfplane (in ) 
that is bounded by the line through . This implies the claim. 
\end{proof} 

\begin{lemma}   \label{lemlocallyD} 
Let  be an integer with  and let  be the vertex of 
 that is not a vertex of . Consider the vertex  of the 
unfolded triangle  that corresponds to . Then  is not in 
the circumdisk  of the unfolded triangle .  
\end{lemma} 
\begin{proof}  
Let  and ; thus,  is 
the edge shared by the faces  and  of . Assume without 
loss of generality that  is parallel to the -axis. Let  be 
the cross-section of  that passes through  and is orthogonal 
to . Let , , , and  be the orthogonal projections 
of , , , and  onto the plane supporting , respectively; 
refer to Figure~\ref{fig:lem3}.

\begin{figure}
\centering
\includegraphics[width=0.7\linewidth]{lemma3.pdf}
\caption{Cross-section through  orthogonal to  (a) side view 
of  (b) top view of . }
\label{fig:lem3}
\end{figure}


Let  be the circle with center  and radius  that is 
coplanar with . Since  is obtained by rotating  
about the line through  and , it must be that  lies on 
. Next we show that  is exterior to .

Let  be the intersection point between the line supporting  
and  that is farthest away from . By Lemma~\ref{lembelow}, 
 lies interior to the convex angle  and, therefore, 
 lies exterior to . Moreover, the circular arc of  
with endpoints  and  that extends from  away from  
(marked as a thick curve in Figure~\ref{fig:lem3}b), lies exterior of 
. This circular arc is precisely the locus of . It follows that 
 is exterior to , which implies that  is exterior to 
. Now observe that  is congruent with the disk  
obtained by intersecting  with the plane supporting . 
Since  is coplanar with  and exterior to , we have 
that  is exterior to . This concludes the proof.
\end{proof} 

It is easy to see that Lemma~\ref{lemlocallyD} implies that the sequence 
 is a chain of disks, i.e., this sequence satisfies the 
two properties given in Section~\ref{secCoD}. Moreover, it follows from 
Lemma~\ref{lemlocallyD} that  is on  and not in the 
interior of  and  is on  and not in the interior 
of . Thus, we have completed the proof of Lemma~\ref{lemCoD}.    


\subsection{Convex Polyhedra whose Vertices are Almost on a Sphere}  
\label{secalmost} 

In this section, we give an example of a convex simplicial polyhedron 
whose vertices are ``almost'' on a sphere and whose skeleton has 
unbounded stretch factor. For simplicity of notation, we consider the 
sphere 
 
Let  be a large integer and let  be the subset of  
consisting of the following  points:
\begin{itemize}
\item The  vertices of the cube ,
\item , where ,
\item ,
\item , where  and , and 
\item .
\end{itemize}
The  vertices of the cube and the points  and  are on the
sphere . Since  

the points  and  are in the interior of, but very close to, this 
sphere.

Let  be the convex hull of the point set . Below, we will 
show that, for sufficiently large values of , 
(i)  and  are faces of the polyhedron  and 
(ii)  is not an edge of . Thus, the figure below shows (part of) 
the top view (in the negative -direction) of .

\begin{center}
   \includegraphics[scale=0.5]{topview.pdf}
\end{center}

The shortest path between  and  in the skeleton of  has 
length

which is at least twice the distance between  and  in the
-direction, which is . Since , it follows that
the stretch factor of the skeleton of  is at least

Thus, by letting  go to infinity, the stretch factor of  
is unbounded.

It remains to prove that  and  are faces of  and  
 is not an edge of . The plane through , , and  has 
equation

To prove that  is a face of , it suffices to show that the 
points  and  are below this plane. The point  is below 
this plane if and only if 

which is equivalent to , which obviously holds. The point  
is below this plane if and only if 

which is equivalent to

Using the fact that, for sufficiently large values of , ,
we have

proving the inequality in (\ref{eqtoprove}). Thus,  is a face 
of . By a symmetric argument,  is a face of  as well. 

We finally show that  is not an edge of . For sufficiently 
large values of , both points  and  are above (with respect to 
the -direction) the points  and .
Observe that both  and  are below  and .
It follows that for any plane through  and , (i)  and
 are on opposite sides or (ii)  and  are on opposite sides.
Therefore,  is not an edge of .

Let  be the smallest angle in the face  of the 
polyhedron , i.e., . Then 
 
where  denotes the dot-product. A straightforward calculation 
shows that 

implying that  is proportional to . Thus, as  tends to 
infinity, the smallest angle in any face of  tends to zero.  

Observe that the polyhedron  does not satisfy the assumptions in 
Theorem~\ref{thm999}. By a sufficiently small perturbation of the points 
of , however, we obtain a polyhedron that does satisfy these 
assumptions and whose skeleton has unbounded stretch factor. We 
conclude that Theorem~\ref{thm999} does not hold for all convex 
simplicial polyhedra whose vertices are very close to a sphere. 


\section{Convex Cycles in an Annulus}  \label{secannulus} 
Let  and  be real numbers with . Define  to be 
the \emph{annulus} consisting of all points in  that are on or 
between the two circles of radii  and  that are centered at the 
origin. Thus,  
  
We will refer to the circles of radii  and  that are centered at 
the origin as the \emph{inner circle} and the \emph{outer circle} of the 
annulus, respectively. 

In this section, we consider convex polygons  that contain the origin 
and whose boundary is in . The skeleton  of such 
a polygon is the graph whose vertex and edge sets are the vertex 
and edge sets of , respectively. 

Throughout this section, we will use the function  defined by 

for .  

\begin{lemma}    \label{lemfincr}
The function  is increasing for . 
\end{lemma} 
\begin{proof}
The derivative of  is given by 
 
It is clear that  for . 
\end{proof}  

We will prove the following result:  
 
\begin{theorem}    \label{thmpolygon}  
Let  and  be real numbers with  and let  be a convex 
polygon that contains the origin in its interior and whose boundary is 
contained in the annulus . Then the skeleton of  is an 
-spanner of the vertex set of . 
\end{theorem} 

Theorem~\ref{thmpolygon} refers to the stretch factor of , 
which is the maximum value of  over all
pairs of distinct vertices  and  of . It turns out that the 
proof becomes simpler if we also consider points that are in the interior 
of edges. This gives rise to the notion of geometric dilation, which 
we recall in the following subsection.    


\subsection{Geometric Dilation of Convex Cycles} 
Let  be a convex cycle in . We observe that  is 
rectifiable, i.e., its length, denoted by , is well-defined; 
see, for example, Section~1.5 in Toponogov~\cite{t-dgcs-06}. For any two 
distinct points  and  on , there are two paths along  that 
connect  and . We denote the length of the shorter of these two 
paths by . The \emph{geometric dilation} of  is defined as 
  
Ebbers-Baumann \emph{et al.}~\cite{egk-gdcpc-07} have proved that, for 
a convex cycle ,  is well-defined. That is, the maximum 
in the definition of  exists.   

Let  and  be two points on . We say that these two points form a 
\emph{halving pair} if the two paths along  between  and  have 
the same length. 

\begin{lemma}[Ebbers-Baumann \emph{et al.}~\cite{egk-gdcpc-07}] 
\label{lemEB}  
Let  be a convex cycle in , and let  be the minimum 
Euclidean distance between the points of any halving pair. Then the 
geometric dilation of  is attained by a halving pair with Euclidean 
distance  and   
 
\end{lemma} 

\subsection{Convex Cycles in an Annulus}  \label{secCC} 
In this section, we consider convex cycles  that contain the origin 
in their interior and that are contained in the annulus . 
We will prove that the geometric dilation of such a cycle is at most 
, where  is the function defined in the beginning of 
Section~\ref{secannulus}. Clearly, this result will imply  
Theorem~\ref{thmpolygon}.  

We start by giving an example of a convex cycle whose geometric dilation 
is equal to . Let  be the convex cycle that consists of 
the two vertical tangents at the inner circle of  that have 
their endpoints at the outer circle, and the two arcs on the outer 
circle that connect these tangents; see the figure below. 

\begin{center}
   \includegraphics[scale=0.6]{Cstar.pdf}
\end{center}

A simple calculation shows that the length of  satisfies 
 

\begin{lemma}  \label{lemCstar} 
The geometric dilation of  satisfies 
 
\end{lemma} 
\begin{proof} 
Consider any halving pair  of . Since  is centrally 
symmetric with respect to the origin, we have . The inner circle
of  is between the two lines through  and  that are 
orthogonal to the line segment . Therefore, . 
Thus, by Lemma~\ref{lemEB}, 
 
If we take for  and  the leftmost and rightmost points of the inner 
circle, then . Therefore, we have 
. 
\end{proof}  

In the following lemmas, we consider special types of convex cycles in 
. For each such type, we prove an upper bound of  
on their geometric dilation. In Theorem~\ref{thmcycle}, we will consider 
the general case and reduce the problem of bounding the geometric 
dilation to one of the special types.   

\begin{lemma}   \label{lemoutercircle} 
Let  be a convex cycle in  that contains the origin in 
its interior, and let  and  be two distinct points on  
such that . If both  and  are on the 
outer circle of , then . 
\end{lemma} 
\begin{proof} 
Let  denote the outer circle of . Then   
 
Since, by Lemma~\ref{lemfincr}, , it follows that 
. 
\end{proof} 

\begin{lemma}   \label{lemthree}  
Consider a line segment  that is tangent to the inner circle of 
 and has both endpoints on the outer circle. Let  be the 
convex cycle that consists of  and the longer arc on the outer circle 
that connects the endpoints of . Then 
  
\end{lemma} 
\begin{proof} 
We may assume without loss of generality that  is horizontal and 
touches the lowest point of the inner circle; see the figure below. 

\begin{center}
   \includegraphics[scale=0.6]{figLemL.pdf}
\end{center}

Let  and  form a halving pair of  that attains the geometric 
dilation of . Observe that at least one of  and  is on the 
outer circle of . If both  and  are on the outer 
circle, then  by Lemma~\ref{lemoutercircle}. 
Otherwise, we may assume without loss of generality that 
(i)  is on  and on or to the left of the -axis, and 
(ii)  is on the outer circle, on or to the right of the -axis 
and above the -axis. 

We first prove that . Let  have coordinates  
for some real number  with , and 
let  be the angle between the -axis and the vector from the 
origin to ; see the figure above. Since  and  form a halving 
pair, the clockwise arc from the highest point on the outer circle to 
the point  has length . Therefore,  and, thus, 
the coordinates of the point  are 
.   
If we define the function  by 

for , then . 
The derivative of  satisfies 

which is positive for . Therefore, the 
function  is increasing and 

implying that . 

We conclude that 
  
To complete the proof, it suffices to show that 
  
We observe that the length of  satisfies 
 
Recall that 

Thus, (\ref{eq666}) becomes 

The latter inequality is equivalent to 
 
i.e., 
 
Since the latter inequality follows from Lemma~\ref{lemfincr}, 
we have shown that (\ref{eq666}) holds. 
\end{proof}  

\begin{lemma}   \label{lemfour}  
Let  be a point in  that is on the negative -axis. 
Let  and  be two points on the outer circle of  
such that (i) both  and  have the same -coordinate and are
below the -axis and (ii) both line segments  and  are 
tangent to the inner circle of . 
Let  be the convex cycle that consists of the two line segments 
and , and the longer arc on the outer circle that connects  
and . Then 
  
\end{lemma} 
\begin{proof}  
The figure below illustrates the situation. 

\begin{center}
   \includegraphics[scale=0.6]{figLemfour.pdf}
\end{center}

Let  and  form a halving pair of  that attains the geometric 
dilation of . Observe that at least one of  and  is on the 
outer circle of . If both  and  are on the outer 
circle of , then  by 
Lemma~\ref{lemoutercircle}. Otherwise, we may assume without loss of 
generality that  is on the outer circle of  and  is 
on the line segment , as in the figure below. 

\begin{center}
   \includegraphics[scale=0.6]{figLemfour2.pdf}
\end{center}

Let  be the maximal line segment in  that contains the 
segment . Let  be the convex cycle consisting of  and the 
longer arc on the outer circle connecting the two endpoints of . Then 
 
By Lemma~\ref{lemthree}, we have . 
It follows that . 
\end{proof}  

\begin{lemma}   \label{lemtwo}  
Consider two non-crossing line segments  and  that are tangent 
to the inner circle of  and have their endpoints on the outer 
circle. Let  be the convex cycle that consists of , , 
and the two arcs on the outer circle that connect  and ; 
one of these two arcs may consist of a single point. Then 
  
\end{lemma} 
\begin{proof}  
We may assume without loss of generality that  is symmetric with 
respect to the -axis and  is to the left of , as in the 
figure below. 

\begin{center}
   \includegraphics[scale=0.6]{L1L2.pdf}
\end{center}

If  and  are parallel, then the claim follows from 
Lemma~\ref{lemCstar}. Thus, we assume that  and  are not 
parallel. We may assume without loss of generality that the length of 
the lower arc of  is less than the length of the upper arc, as 
in the figure above. We observe that 
 
i.e., the length of  is equal to the length of the cycle  in 
Lemma~\ref{lemCstar}. Indeed, if we rotate , while keeping it 
tangent to the inner circle, until it becomes parallel to , then 
the length of the cycle does not change. 

Let  and  form a halving pair of  that attains the geometric 
dilation of , i.e., 
  
We consider three cases for the locations of  and  on . 

\vspace{0.5em} 

\noindent 
{\bf Case 1:} Both  and  are on the outer circle of . 

Then we have  by Lemma~\ref{lemoutercircle}. 

\vspace{0.5em} 

\noindent 
{\bf Case 2:}  is on the outer circle of  and  is not 
on the outer circle. 

Since  and  form a halving pair,  must be on the upper arc of 
. We may assume without loss of generality that  is on .  
Let  be the convex cycle consisting of  and the longer arc 
on the outer circle connecting the two endpoints of ; see the 
figure below.  
\begin{center}
   \includegraphics[scale=0.6]{L1L2Case2.pdf}
\end{center}

We have 
 
By Lemma~\ref{lemthree}, we have . 
It follows that . 

\vspace{0.5em} 

\noindent 
{\bf Case 3:} Neither  nor  is on the outer circle of . 

Since  and  form a halving pair, these two points cannot both be 
on the same line segment of . We may assume without loss of 
generality that  is on  and  is on . 

Let  be the maximal line segment in  that is parallel 
and not equal to  and that touches the inner circle.  
Let  be the maximal line segment in  that is parallel 
and not equal to  and that touches the inner circle.  

\begin{center}
   \includegraphics[scale=0.6]{L1L2Case3.pdf}
\end{center}

We claim that  is to the right of  or  is to the left of 
; see the two figures above. Assuming this is true, it follows 
that  and 
  
To prove the claim, assume that  is to the left of  and  is 
to the right of . Let  be the intersection between  and 
, and let  be the intersection between  and ; see 
the figure below. 

\begin{center}
   \includegraphics[scale=0.6]{L1L2Case3b.pdf}
\end{center}

Observe that both  and  are on the -axis, and both  and  
are below the -axis. Therefore, the part of  below the -axis is 
shorter than the part of  above the -axis. Thus, the two paths 
along  between  and  do not have the same lengths. This  
contradicts our assumption that  and  form a halving pair of 
.     
\end{proof}  

We are now ready to consider an arbitrary convex cycle  that contains 
the origin in its interior and that is contained in . 
A \emph{homothet} of  is obtained by scaling  with respect to the 
origin, followed by a translation. Observe that the dilation of a 
homothet of  is equal to the dilation of . 

\begin{theorem}    \label{thmcycle}  
Let  and  be real numbers with  and let  be a convex 
cycle that contains the origin in its interior and that is contained in 
the annulus . Then  
 
where  is the function defined in the beginning of 
Section~\ref{secannulus}. 
\end{theorem} 
\begin{proof} 
Let  and  form a halving pair of  that attains the geometric 
dilation of , i.e., 
  
We first assume that neither  nor  is on the outer circle of 
. 
 
Let  and  be supporting lines of  through  and , 
respectively. Since  and  form a halving pair, .  
Let  be the convex cycle of maximum length in  that is 
between  and . Observe that  contains two line segments 
such that (i) all their four endpoints are on the outer circle (as in 
the left figure below) or (ii) two of their endpoints are on the outer 
circle, whereas the other two endpoints meet in the interior of 
 (as in the right figure below). If (i) holds, we say that 
 is of \emph{type~1}. In the other case, i.e., if (ii) holds, we 
say that  is of \emph{type~2}. We have 
  

\begin{center}
   \includegraphics[scale=0.6]{LpLqC1.pdf}
\end{center}

We claim that there is a homothet  of  that is contained in 
 and that touches the inner circle in two points; see the 
two figures below. 

\begin{center}
   \includegraphics[scale=0.6]{LpLqC23.pdf}
\end{center}

To obtain such a homothet , we do the following. First, we shrink 
, i.e., scale it (with respect to the origin) by a factor of less 
than one, until it touches the inner circle. At this moment, one 
of the lines  and  in the shrunken copy of  touches 
the inner circle. Assume, without loss of generality, that  
touches the inner circle, whereas  does not. Let  denote 
the ``center'' of the scaled copy of , which is the origin. 
We translate  towards  in the direction that is orthogonal to 
. During this translation, we shrink  (with respect to its 
center ) while keeping  on its boundary. We stop translating 
 as soon as  touches the inner circle of .
The resulting translated and shrunken copy of  is the homothet 
. 

Let  and  be the two points on the homothet  that correspond 
to  and , respectively. Then  
  
Let  and  be supporting lines of  through  
and , respectively, and let  be the convex cycle of maximum 
length in  that is between  and ; see the 
two figures above. Observe that  is either of type~1 or of type~2. 
In fact,  may be of type~2, even if  is of type~1. We have 
 
First assume that  is of type~1. Thus, all four endpoints of the 
two line segments of  are on the outer circle of  (as 
in the left figure above). Then  satisfies the conditions of 
Lemma~\ref{lemtwo} and, therefore, 
  

Now assume that  is of type~2. We may assume without loss of 
generality that  is symmetric with respect to the -axis, 
and the intersection point of  and  is on the 
negative -axis. Translate  in the negative -direction 
until it touches the outer circle. Denote the resulting translate 
by . Let  and  be the two points on  that correspond 
to  and , respectively. Then  
 
We consider two cases. 

\vspace{0.5em} 

\noindent 
{\bf Case 1:} The lowest point of  is on the outer circle of 
; see the left figure below. 

\begin{center}
   \includegraphics[scale=0.6]{figThmCase1.pdf}
\end{center}

Let  and  be supporting lines of  through  
and , respectively, and let  be the convex cycle of maximum 
length in  that is between  and ; see 
the right figure above. Observe that 
 
Enlarge the inner circle of  such that it touches the 
two line segments of . Denoting the radius of this enlarged 
circle by , it follows from Lemmas~\ref{lemtwo} and~\ref{lemfincr} 
that 
 

\vspace{0.5em} 

\noindent 
{\bf Case 2:} The leftmost and rightmost points of  are on the 
outer circle of ; see the left figure below. 

\begin{center}
   \includegraphics[scale=0.6]{figThmCase2.pdf}
\end{center}

Let  be the convex cycle consisting of the two line segments 
of  and the upper arc on the outer circle connecting them; see 
the right figure above. Then 
 
Enlarge the inner circle of  such that it touches the 
two line segments of . Let  be the radius of this enlarged 
circle. Since  satisfies the conditions of Lemma~\ref{lemfour}
for , we have  
 
Thus, we have shown that . 

\vspace{0.5em} 

Until now we have assumed that neither  nor  is on the outer 
circle of . Assume now that  or  is on this outer 
circle. Let  be an arbitrary real number and consider 
the annulus . Since neither  nor  is on 
the outer circle of this enlarged annulus, the analysis given above 
implies that 
 
Thus, since this holds for any , we have 
 
Since the function  is continuous, it follows from 
Lemma~\ref{lemfincr} that 
 
This concludes the proof. 
\end{proof} 


\section{Angle-Constrained Convex Polyhedra in a Spherical Shell} 
\label{secACSH} 

Let  and  be real numbers with . Define  
to be the \emph{spherical shell} consisting of all points in  
that are on or between the two spheres of radii  and  that are 
centered at the origin. In other words, 
 

In this section, we consider convex simplicial polyhedra that contain 
the origin in their interiors and whose boundaries are contained in 
. From Section~\ref{secalmost}, the skeletons of such 
polyhedra can have unbounded stretch factors. 

Let  be a real number with . We say that a 
convex polyhedron  is -\emph{angle-constrained}, if the 
angles in all faces of  are at least . 

Let  be a convex simplicial polyhedron that contains the origin in 
its interior, whose boundary is contained in , and that is 
-angle-constrained. In this section, we prove that the stretch 
factor of the skeleton of  is bounded from above by a function of 
 and . Our proof will use an improvement of a result by 
Karavelas and Guibas~\cite{kg-skgsa-01} about chains of triangles in 
; see Lemma~\ref{lemchain}. We start by reviewing such chains.  


\subsection{Chains of Triangles}  \label{secCoT} 
Before we define chains of triangles, we prove a geometric lemma that 
will be used later in this section. 

\begin{lemma}  \label{lemtriangle}  
Let , , and  be three pairwise distinct points in the plane, 
and let . Then 
 
\end{lemma} 
\begin{proof} 
Consider the interior angle bisector of ; see the figure below. 

\begin{center}
\includegraphics[scale=0.7]{triangle1.pdf}
\end{center}

Let  be the distance between  and this bisector, and let 
 be the distance between  and this bisector. Then 

\end{proof} 

Let  and  be two distinct points in , let  be 
an integer, and consider a sequence 
 of triangles in . The 
sequence  is called a 
\emph{chain of triangles with respect to  and }, if 
\begin{enumerate} 
\item  is a vertex of , but not of ,  
\item  is a vertex of , but not of ,  
\item for each  with , the interiors of the triangles 
       and  are disjoint and these triangles share an edge,
      and 
\item for each  with , the line segment  
      intersects the interior of . 
\end{enumerate} 
See Figure~\ref{figchain} for examples. 

\begin{figure}
\begin{center}
\includegraphics[scale=0.7]{chain1.pdf}
\end{center}
\caption{Two examples of chains of triangles with respect to the points 
          and . For clarity, the triangle  in the second 
         example is dashed.} 
\label{figchain}
\end{figure}

Let  be the graph whose vertex and edge sets consist of 
all vertices and edges of the  triangles in , 
respectively. The length of each edge in this graph is equal to the 
Euclidean distance between its vertices. The length of a shortest path 
in  is denoted by . 

\begin{lemma}
\label{lemchain}  
Let  be a real number with , let  and  
be two distinct points in the plane, and let  be a 
chain of triangles with respect to  and . Assume that all angles 
in any of the triangles in  are at least . Then 

\end{lemma} 
\begin{proof}
We assume, without loss of generality, that the line segment  is 
on the -axis and  is to the left of . We start by constructing 
a preliminary path in  from  to  (this is the 
same path as in Karavelas and Guibas~\cite{kg-skgsa-01}): 
\begin{enumerate} 
\item Let  be one of the two edges of the triangle  with 
      endpoint . We initialize the path to be . 
\item Consider the current path and let  be its last point. 
      Assume that . 
      \begin{enumerate} 
      \item If  is below the -axis, then consider all edges in  
             that have  as an endpoint and whose other 
            endpoint is on or above the -axis. Let  be the 
            ``rightmost'' of these edges, i.e., the edge among these 
            whose angle with the positive -axis is minimum. Then we 
            extend the path by the edge , i.e., we add the point 
             at the end of the current path.
      \item If  is above the -axis, then consider all edges in 
             that have  as an endpoint and whose other 
            endpoint is on or below the -axis. Let  be the 
            ``rightmost'' of these edges, i.e., the edge among these 
            whose angle with the positive -axis is maximum. Then we 
            extend the path by the edge , i.e., we add the point 
             at the end of the current path.
      \end{enumerate} 
\end{enumerate} 

\begin{figure}
\begin{center}
\includegraphics[scale=0.7]{chainpath1.pdf}
\end{center}
\caption{The path  in the first chain of 
         triangles in Figure~\ref{figchain}. The segments , 
         , and  belong to group~1, the segments 
          and  belong to group~2, and the segments 
          and  belong to group~3. The path  
         is equal to .}
\label{figchainpath}
\end{figure}

Number the triangles in  as , in the 
order in which they are intersected by the line segment from  to 
. Then the point  is a vertex of a triangle in  
that has a larger index than the index of any triangle that contains 
the vertex . Therefore, if we continue extending the path, it will 
reach the point . Denote the resulting path by 
; see Figure~\ref{figchainpath}. 

As a warming-up, we prove an upper bound on the length of the path .
For each  with , let  be the intersection 
between the line segments  and . Then 
 is at most the length of the path , i.e., 
  
Let . Since 
, it follows from Lemma~\ref{lemtriangle} that 
  
Therefore, 
   

To improve the upper bound on , we divide the 
line segments , , into three groups: A 
segment  belongs to \emph{group~1} if its 
relative interior intersects an edge of some triangle of the chain 
. If the relative interior of  is entirely 
contained in one of the triangles of  and the point 
 is on or above the -axis, then  belongs to 
\emph{group~2}. Otherwise,  belongs to \emph{group~3}.  
Refer to Figure~\ref{figchainpath} for an illustration.  
For , let  denote the total length of all line segments 
 in group~. We may assume without loss of generality 
that . 

Consider again the path . 
For each  such that  belongs to group~2, we replace 
the subpath  in  by the short-cut 
; refer to Figure~\ref{figchainpath}. Let  denote 
the resulting path from  to . 

For each line segment  in group~1, we have 
. If  is in group~2, then 
 
It follows that 
 
Recall that . This inequality is equivalent to 
 
implying that 
 
The function  
is negative for . To prove this, using 
 and a straightforward 
calculation, we observe that  if and only if 
 
The left-hand side in the above inequality has a positive derivative 
for  (this can be verified using 
); thus, 
 

We conclude that, since , the first term on the right-hand 
side in (\ref{pqG}) is non-positive, implying that 
 
This completes the proof.  
\end{proof} 


\subsection{Angle-Constrained Convex Polyhedra}   
Let  be a real number with  and let  be 
a convex simplicial polyhedron that is -angle-constrained. In 
this section, we bound the ratio of the shortest-path distance 
 between  and  in the skeleton of  and the 
shortest-path distance  between  and  along 
the surface of .  

Let  and  be two distinct vertices of  and consider the 
shortest path  along the surface of  from  to . 
Except for  and , this path does not contain any vertex of ; 
see Sharir and Schorr~\cite{ss-spps-86}. Let  be 
the sequence of faces of  that this path passes through. Let 
 be the sequence of triangles 
obtained from an edge-unfolding of the triangles . 
Let  and  be the vertices of  and  corresponding to 
 and , respectively. Sharir and Schorr~\cite{ss-spps-86} 
(see also Agarwal \emph{et al.}~\cite{aaos-supa-97}) have shown that 
\begin{itemize}
\item  is a chain of triangles with respect to  and 
      , as defined in Section~\ref{secCoT}, and 
\item the path  along  unfolds to the line segment 
      , i.e., . 
\end{itemize} 
Consider the graph  that is defined by the chain 
; see Section~\ref{secCoT}. Observe that  
is at most the shortest-path distance between  and  in the graph 
consisting of all vertices and edges of the triangles 
. The latter shortest-path distance is equal to 
. Thus, using Lemma~\ref{lemchain}, we obtain 
 
We have proved the following result: 

\begin{lemma}  \label{lempartial} 
Let  be a real number with  and let  be 
a -angle-constrained convex simplicial polyhedron. 
For any two distinct vertices  and  of , we have 

\end{lemma} 

\subsection{Angle-Constrained Convex Polyhedra in a Spherical Shell}   
We are now ready to prove the main result of Section~\ref{secACSH}: 

\begin{theorem} 
Let , , and  be real numbers with  and 
, and let  be a -angle-constrained 
convex simplicial polyhedron that contains the origin and whose boundary 
is contained in the spherical shell . Then the skeleton of 
 is a -spanner of the vertex set of , where  
 
\end{theorem} 
\begin{proof}  
Let  and  be two distinct vertices of . 
By Lemma~\ref{lempartial}, we have 

Let  be the plane through , , and the origin, and let 
 be the intersection of  and . Then 
  
Since  is a convex polygon satisfying the conditions of 
Theorem~\ref{thmpolygon}, we have 
 
\end{proof} 

\section{Concluding Remarks} 
We have considered the problem of bounding the stretch factor of the 
skeleton of a convex simplicial polyhedron  in . If the 
vertices of  are on a sphere, then this stretch factor is at 
most , which is  times the currently best known 
upper bound on the stretch factor of the Delaunay triangulation in 
. We obtained this result from Xia's upper bound on the stretch 
factor of chains of disks in~\cite{x-sfdtl-13}. Observe that Xia's 
result implies an upper bound on the stretch factor of the Delaunay 
triangulation. The converse, however, may not be true, because the 
chains of disks that arise in the analysis of the Delaunay triangulation 
are much more restricted than general chains of disks; see for example 
Figure~2 in~\cite{x-sfdtl-13}. Thus, an improved upper bound on the 
stretch factor of the Delaunay triangulation may not imply an improved 
upper bound on the stretch factor of the skeleton of . Nevertheless, 
we make the following conjecture: Let  be a real number such that 
the stretch factor of any Delaunay triangulation in  is at most 
. Then the stretch factor of the skeleton of any convex polyhedron 
in , all of whose vertices are on a sphere, is at most 
. 
 
We have shown that the skeleton of a convex simplicial polyhedron  
whose vertices are ``almost'' on a sphere may have an unbounded 
stretch factor. For the case when  contains the origin, its 
boundary is contained in the spherical shell , and the 
angles in all faces are at least , we have shown that the stretch 
factor of 's skeleton is bounded from above by a function that 
depends only on  and . We leave as an open problem to find 
other classes of convex polyhedra whose skeletons have bounded stretch 
factor. 
      

\section*{Acknowledgments} 
Part of this work was done at the 
\emph{Third Annual Workshop on Geometry and Graphs}, held at the 
Bellairs Research Institute in Barbados, March 8--13, 2015.  
We thank the other workshop participants for their helpful comments. 

We thank the anonymous referees for their useful comments. 
We especially thank one of the referees for simplifying the proofs of 
Lemmas~\ref{lemlocallyD} and~\ref{lemtriangle}, and for suggesting 
the use of short-cuts in the proof of Lemma~\ref{lemchain}.  


\begin{thebibliography}{10}

\bibitem{aaos-supa-97}
P.~K. Agarwal, B.~Aronov, J.~O'Rourke, and C.~A. Schevon.
\newblock Star unfolding of a polytope with applications.
\newblock {\em SIAM Journal on Computing}, 26:1689--1713, 1997.

\bibitem{bps-chpss-14}
P.~Bose, S.~Pratt, and M.~Smid.
\newblock The convex hull of points on a sphere is a spanner.
\newblock In {\em Proceedings of the 26th Canadian Conference on Computational
  Geometry}, pages 244--250, 2014.

\bibitem{b-gtfga-80}
K.~Q. Brown.
\newblock {\em Geometric Transforms for Fast Geometric Algorithms}.
\newblock Ph.{D}. thesis, Carnegie-Mellon University, 1979.

\bibitem{dfs-dgaag-90}
D.~P. Dobkin, S.~J. Friedman, and K.~J. Supowit.
\newblock Delaunay graphs are almost as good as complete graphs.
\newblock {\em Discrete \& Computational Geometry}, 5:399--407, 1990.

\bibitem{egk-gdcpc-07}
A.~Ebbers-Baumann, A.~Gr{\"u}ne, and R.~Klein.
\newblock Geometric dilation of closed planar curves: {New} lower bounds.
\newblock {\em Computational Geometry: Theory and Applications}, 37:188--208,
  2007.

\bibitem{g-gdhd-06}
A.~Gr{\"u}ne.
\newblock {\em Geometric Dilation and Halving Distance}.
\newblock Ph.{D}. thesis, Universit{\"a}t Bonn, Germany, 2006.

\bibitem{kg-skgsa-01}
M.~I. Karavelas and L.~J. Guibas.
\newblock Static and kinetic geometric spanners with applications.
\newblock In {\em Proceedings of the 12th ACM-SIAM Symposium on Discrete
  Algorithms}, pages 168--176, 2001.

\bibitem{ns-gsn-07}
G.~Narasimhan and M.~Smid.
\newblock {\em Geometric Spanner Networks}.
\newblock Cambridge University Press, Cambridge, UK, 2007.

\bibitem{ss-spps-86}
M.~Sharir and A.~Schorr.
\newblock On shortest paths in polyhedral spaces.
\newblock {\em SIAM Journal on Computing}, 15:193--215, 1986.

\bibitem{t-dgcs-06}
V.~A. Toponogov.
\newblock {\em Differential Geometry of Curves and Surfaces}.
\newblock Birkh{\"a}user, Boston, 2006.

\bibitem{x-sfdtl-13}
G.~Xia.
\newblock The stretch factor of the {Delaunay} triangulation is less than
  1.998.
\newblock {\em SIAM Journal on Computing}, 42:1620--1659, 2013.

\end{thebibliography}

\end{document} 
