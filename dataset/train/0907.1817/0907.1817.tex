\documentclass{elsart}
\usepackage{amsmath,amssymb,graphicx,caption}
\newtheorem{Thm}{Theorem}

\begin{document}

\begin{frontmatter}






\title{A new intrinsic numerical method for PDE on surfaces \thanksref{NSC}}

\author[Chen]{Sheng-Gwo
Chen\corauthref{corl}}\ead{csg@mail.ncyu.edu.tw},
\author[Chi_Wu]{Mei-Hsiu Chi}\ead{mhchi@math.ccu.edu.tw} and
 \author[Chi_Wu]{Jyh-Yang Wu}\ead{jywu@math.ccu.edu.tw}

\address[Chen]{Department of Applied Mathematics, National Chiayi University,  Chia-Yi 600,
Taiwan.}
\address[Chi_Wu]{Department of Mathematics, National Chung Cheng
University, Chia-Yi 621, Taiwan}

\corauth[corl]{csg@mail.ncyu.edu.tw}
\thanks[NSC]{ Partially supported by NSC, Taiwan}

\begin{abstract}

In this note we shall introduce a simple, effective numerical method
for solving partial differential equations for scalar and
vector-valued data defined on surfaces. Even though we shall follow
the traditional way to approximate the regular surfaces under
consideration by triangular meshes, the key idea of our algorithm is
to develop an intrinsic and unified way to compute directly the
partial derivatives of functions defined on triangular meshes. We
shall present examples in computer graphics and image processing
applications.

\end{abstract}

\begin{keyword}
gradient, laplacian, tangential lifting method, diffusion equations
\end{keyword}
\end{frontmatter}



\section{Introduction}
    Numerical approaches to solve partial differential equations (PDE's)
on surfaces have received growing interest over last decade.
However, they are still not well-understood. Partial differential
equations need to be solved intrinsically and numerically for data
defined on 3D surfaces in many applications. For instance, such
examples exist in texture synthesis (Turk\cite{Turk}, Witkin and
Kass\cite{Witkin}), vector field visualization (Diewald, Preufer and
Rumpf\cite{Clarenz}), weathering (Dorsey and Hanrahan\cite{Dorsey})
and cell-biology (Ayton, McWhirter, McMurty and Voth 2005). Usually,
surfaces are presented by triangular or polygonal forms. Partial
differential equations are then solved on these triangular or
polygonal meshes with data defined on them. The use of triangular or
polygonal meshes is very popular in all areas dealing with 3D
models. However, it has not yet been a widely accepted method to
compute differential characteristics such as principal directions,
curvatures and Laplacians (Chen and Wu\cite{Chen1}, Wu, Chen and Chi
\cite{Chen3,Chen4}, Taubin\cite{Taubin1,Taubin3}). This is because
that there is no unified, simple and effective method to compute
these first and second order differential characteristics of the
triangular or polygonal surface and to solve PDE's for data defined
on triangular or polygonal meshes. In Chen, Chi and Wu\cite{Chen5},
the authors proposed a new intrinsic simple algorithm to handle this
difficulty. In this note, we shall use this new technique to solve
PDE's on surfaces.


    In Bertalmio, Cheng, Osher and Sapiro\cite{Bertalmio} proposed a framework,
the implicit surface algorithm, to solve variational problems and
PDE's for scalar and vector-valued data defined on surfaces.Their
key idea is to use, instead of a triangular or polygonal
representation, an implicit representation. The surface under
consideration is the zero-level set of a higher dimensional
embedding function. Then they smoothly extend the original data on
the surface to the 3D domain, adapt the PDE's accordingly, and
implement all the numerical computations on the fixed Cartesian grid
corresponding to the embedding function. The advantage of their
method is the use of the Cartesian grid instead of a triangular mesh
for the numerical implementation.


\section{Our intrinsic algorithm for solving PDE's on surfaces}
    In ths section we first propose our discrete intrinsic algorithm for
solving PDE's on regular surfaces. We divide our algorithm into two
main steps: First. we approximate the given surface by a suitable
triangular mesh according to the accuracy demand. Second, we use our
new intrinsic differential method developed in Chen, Chi and
Wu\cite{Chen5} to compute the numerical PDE on the fixed triangular
mesh. The first step is now easy to implement since one can find
some good and efficient algorithms in the public domain. The
difficult part lies in the second step. Namely, how can one
effectively compute differential quantities on functions on a
triangular mesh?


  Next, we shall compare our algorithm with the implicit algorithm
proposed by Bertalmio, Cheng, Osher and Sapiro\cite{Bertalmio}. We
list the key steps of these two algorithms about solving PDE's on
surfaces as follows.

\begin{table}[h]

\begin{tabular}{|p{6cm}|p{6cm}|}
\hline  Our intrinsic algorithm & The implicit surface algorithm \\
\hline
 1.
Obtain a triangular mesh approximated to the given surface. & 1.
Obtain an implicit representation of the given fixed surface. \\

\hline
 & 2. Extend smoothly the data on the surface to the 3D
volume \\
 \hline
 & 3. Adapt PDE's accordingly \\
 \hline
 2. Use our new intrinsic differential method to compute the
numerical PDE's on the fixed triangular mesh.  &  4. Perform all the
computations on the fixed Cartesian grid corresponding to the
embedding function. \\
 \hline
\end{tabular}

\end{table}



    One can tell from this comparison that our method is much simpler
and more intrinsic. In many applications, one usually starts with
triangular meshes instead of regular surfaces. In this case, we do
not need Step 1 in our intrinsic algorithm. However, in the implicit
surface algorithm, one will need one extra processing step:
Construct an accurate implicit surface from a triangular mesh. Note
that the triangular mesh may compose of a lot of triangles. This
will cost large computations to obtain the accurate implicit
surface.


    Next, we shall describe a new, simple and effective method
to define the discrete gradient and the discrete LB operator on
functions on a triangular mesh. In order to do so, we first recall
the gradient and the LB operator on functions in a regular surface
 in the 3D Euclidean space .

\subsection{Gradient and LB operators on regular surfaces }
We consider a parameterization  at a point
, where  is an open subset of the 2D Euclidean space
. We can choose, at each point  of , a unit
normal vector . The map  is the local
Gauss map from an open subset of the regular surface  to the
unit sphere in the 3D Euclidean space . The Gauss map
 is differentiable. Denote the tangent space of  at the
point  by .
The tangent space  is a linear space spanned by  where  are coordinates for .


The gradient  of a smooth function  on  can be
computed from




where , , and  are the coefficients of the first
fundamental form and


See do Carmo\cite{Do} for the details. Note that the gradient
 assigns to each point  in  a tangent vector
 such that we have for all ,

where the smooth curve  is in  with 
and .


    The LB operator  acting on the function  is defined by the
integral duality


for all smooth function  on . A direct computation
yields the following local representation for the LB operator on a
smooth function :





To move from regular surfaces to triangular meshes, one need to
avoid the problem of local parametrization  around a vertex .
In other word, one does not have the fist fundamental form , ,
 and their derivatives for the computation of the gradient and
the Laplacian operator of a function on a triangular mesh. To handle
this problem, we give a novel method in Chen, Chi and Wu\cite{Chen5}
to compute these differential quantities. The primary ideas were
developed in Chen, Chi and Wu\cite{Chen4} where we try to estimate
the discrete partial derivatives for 2D scattered data points.


\subsection{ A new discrete algorithm: local tangential lifting(LTL) method}

In this section we shall describe a unified, simple and effective
method to define the discrete gradient and the discrete Laplacian
operator on functions on a triangular mesh. The primary ideas were
developed in Chen, Chi and Wu\cite{Chen3,Chen4} where we try to
estimate the discrete partial derivatives of functions on 2D
scattered data points. Indeed, the method that we shall use to
develop our algorithm is divided into two main steps: first we lift
the 1-neighborhood points to the tangent space and obtain a local
tangential polygon. Second, we use some geometric idea to lift
functions and vectors to the tangent space and then we can compute
their derivatives in the 2D tangent space. This means that the
lifting process allows us to reduce the 2D curved surface problem to
the 2D Euclidean problem and hence the methods in \cite{Chen4}
and\cite{Hiroshi} can be applied.



Consider a triangular mesh , where  is the list vertices and  is the list of triangles. Next, we introduce the notion  of
the local tangential polygon  at the vertex   of  as
follows:
\begin{enumerate}
\item The normal vector  at the vertex  in  is given by

where  is the unit normal to a triangle face  and the
centroid weight is given in \cite{Chen1} by

Here,  is the centroid of the triangle face  determined by

\item The tangent plane  of  at  is now determined by

\item The local tangential polygon  of  in  is formed by the vertices  which is the lifting vertex of  adjacent to  in .
 as in figure
\ref{tangential}.
\end{enumerate}

\begin{figure}
{\center
\includegraphics[height=6.2cm]{tangantial}
\caption{ The tangential polygon .} \label{tangential}
 }
\end{figure}

Let  be a function on . We will lift locally the function 
to a function, denoted by , on the vertices
 in  by simply setting
 And  where  is the origin of
. One can then extend the function  to a
piecewise linear function, still denoted by , on 
as follows.

Consider a face  with vertices ,  and  in . We
obtain a lifting face  with vertices ,
 and   in . Every point   in
 can be written as a linear combination of 
and  . That is,  where
 and . Then we define

Hence, the extended function  is affine on each
triangle  of  and is differentiable on .
The gradient  of  at
the origin  can be obtained by  where the coefficients  and
 satisfy the relations:


As easy computation gives


To obtain the gradient  of  on  at the vertex
, we use again the weighted combination method. Namely, we set
 with

 where  is the centroid of the lifting triangle face
 and is determined by


Next we explain how to obtain a good discrete Laplacian  of a function  on the triangular mesh . From the
discussions above, we obtain the gradient  of  on
 at each vertex . We can use the method of parallel transport
to lift the vector  at  to a vector  in the tangential space . The idea
is to define a orthonormal linear map from  to . To
do so, we choose an orthonormal basis for  by

The corresponding orthonormal basis for  is then given by

Then, the linear map  of the parallel transport is given by

for  in . See figure
\ref{parallel}.


\begin{figure}
{\center
\includegraphics[height=6.2cm]{parallel}
\caption{parallel transport}\label{parallel}
 }
\end{figure}



In this way, we can set the tangential gradient 
at  by

 Hence we obtain a tangential
gradient  of  at each vertex  in
the tangential polygon  and we also set

 See figure \ref{tildeh}.



\begin{figure}
{\center
\includegraphics[height=6.2cm]{tildeh}
\caption{ }\label{tildeh}
 }
\end{figure}



Fix an orthonormal basis  for . The
tangential gradient  can be written as

 The coefficients 
and  can now be viewed as functions on the vertices
 of the tangential polygon . As before, we can
then obtain their gradients  and  at origin. Namely,


with the centroid weights  as in
(\ref{centroidweight}).

Put them in the matrix form to give

 Therefore we have the Laplacian  of  at the
vertex  of  :


Theoretically the definition of the Laplacian  is
independent of the choice of the orthonormal basis .



\section{Linear diffusion}

As a simple example to illustrate our new algorithm, let us consider
a linear diffusion equation on a regular surface :


for  , , where   is the surface Laplacian on 
and  is a smooth
function on . For the numerical implementation of our
intrinsic algorithm, we take the regular surface  to be (1)
the unit sphere  or (2) a torus . In the case of the
sphere , we consider the function




Figure \ref{sphere} and figure \ref{shpere_final} give the solution
of (\ref{heatequation}) and (\ref{function_g}) with initial
functions . Different time steps are shown until the
stationary solution is reached.


Consider  the torus  for  with . we
take ,  and choose the function



Figure \ref{torus} and figure \ref{torus_final} give the solution of
(\ref{heatequation}) and (\ref{function_g2}) with initial functions
. As above, different timesteps are depicted until the
stationary solution is reached.



\section{Reaction-diffusion textures}

The original idea about how reaction-diffusion equations can be used
to create patterns was first introduced in (Turing\cite{Turing}).
The basic idea is to have a number of chemicals that diffuse at
different rates and that react with each others. After the works of
(Turk\cite{Turk}, Witkin and Kass\cite{Witkin}), the use of
reaction-diffusion equations for texture synthesis attracted a lot
of attentions in computer graphics. Turk , Witkin and Kass used
these equations for planar textures and textures on surfaces. Then
the patterns are analyzed by assigning a brightness value to the
concentration of one of the "chemicals".

Consider two chemicals  and  on a surface . In a
simple isotropic model, we have



where  and  are two constants representing the
diffusion rates and  and  are functions that describe the
reaction. For simple isotropic patterns, Turing chose the functions
 and  to be


where  is a constant and  is a random function giving the
irregularities in the chemical concentration.

    By using our intrinsic method described in the previous section, we
can easily generate textures on surfaces without the elaborated
schemes employed in (Turk 1991, Witkin and Kass 1991).

In the case of the sphere , figure \ref{diffusion_sphere} and
figure \ref{diffusion_sphere_final} give the solution of (\ref{Du})
and (\ref{fg}) with initial functions and Different timesteps are
shown until the stationary solution is reached.


On the torus  for  figure \ref{diffusion_torus}
and figure \ref{diffusion_torus_final} give the solution of
(\ref{Du}) and (\ref{fg}) with initial functions  and , , .
Different timesteps are shown until the stationary solution is
reached.



\begin{figure}
{\center
\includegraphics[scale = 1]{sphere}
\caption{sphere}\label{sphere}
 }
\end{figure}


\begin{figure}
{\center
\includegraphics[scale = 1]{sphere_final}
\caption{sphere (stationary solution) }\label{shpere_final}
 }
\end{figure}







\begin{figure}
{\center
\includegraphics[scale = 1]{torus}
\caption{torus}\label{torus}
 }
\end{figure}

\begin{figure}
{\center
\includegraphics[scale = 1]{torus_final}
\caption{torus (stationary solution)}\label{torus_final}
 }
\end{figure}



\begin{figure}
{\center
\includegraphics[scale = 1]{diffusion_sphere}
\caption{sphere}\label{diffusion_sphere}
 }
\end{figure}


\begin{figure}
{\center
\includegraphics[scale = 1]{diffusion_sphere_final}
\caption{sphere (stationary solution) }
\label{diffusion_sphere_final}
 }
\end{figure}

\begin{figure}
{\center
\includegraphics[scale = 1]{diffusion_torus}
\caption{torus}\label{diffusion_torus}
 }
\end{figure}


\begin{figure}
{\center
\includegraphics[scale = 1]{diffusion_torus_final}
\caption{torus (stationary solution)}\label{diffusion_torus_final}
 }
\end{figure}




\begin{thebibliography}{00}

\bibitem{Bertalmio} Bertalmio, M. Sapiro, G. Cheng L.T. and Osher, S. A
framework for solving surface partial differential equations for
computer graphics applications.(2000) CAM Report00-43, UCLA,
Mathematics Department.

\bibitem{Chen1} Chen, S.-G., Wu, J.-Y.. Estimating normal vectors and
curvatures by centroid weights.(2004) Computer Aided Geometric
Design vol 21, pp. 447-458.

\bibitem{Chen3} Chen, S.-G., Chi, M.-H., Wu, J.-Y.. Curvature
estimation and curvature flow for digital curves.(2006) WSEAS
transcations on computers, vol 5, p804-809.

\bibitem{Chen4} Chen, S.-G., Chi, M.-H., Wu, J.-Y.. Boundary and interior
derivatives estimation for 2D scattered data points.(2006) WSEAS
transcations on computers, vol 5, p824-829.

\bibitem{Chen5} Chen, S.-G., Chi, M.-H., Wu, J.-Y.. On a new differential method for triangular
meshes.(2009) preprint.



\bibitem{Dorsey} Dorsey, J. and Hanrahan, P.. Digital materials and virtual
weathering.(2000) Scientific American 282:2, pp46-53


\bibitem{Do} do Carmo, M.. Differential Geometry of curves and
surfaces.(1976) Prentice Hall, Englewood Cliffs, NJ.

\bibitem{Hiroshi} Hiroshi Akima, On estimating partial derivatives for bivariate interpolation of
scattered data, Rocky Mountain Journal 14 (1984), pp. 41-52,
MR0736165.


\bibitem{Taubin1} Taubin, G.. A signal processing approach to fair surface
design.(1995) In SIGGRAPH'95 Proceedings, pp. 351-358.


\bibitem{Taubin3} Taubin, G.. Estimating the tensor of curvatures of a
surface from a polyhedral approximation.(1995) In: proceedings of
the Fifth International Conference on Computer Vision, pp. 902-907.


\bibitem{Turing} Turing, A.. The chemical basis of
morphogenesis.(1952) Philosophical Transactions of the Royal Society
B 237, pp.37-72.

\bibitem{Turk} Turk, G.. Generating textures on arbitrary surfaces using
reaction-diffusion.(1991) Computer Graphics (SIGGRAPH) 25:4,pp.
289-298.

\bibitem{Witkin} Witkin, A. and Kass, M.. Reaction-diffusion
textures.(1991) Computer Graphics (SIGGRAPH) 25:4,pp. 299-308



\end{thebibliography}

\end{document}
