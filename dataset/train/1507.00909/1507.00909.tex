\documentclass[11pt,a4paper]{article}
\pdfoutput=1
\synctex=1



\usepackage[margin=1in]{geometry}
\usepackage{amsmath,amssymb,amsthm,booktabs,color,doi,enumitem,graphicx,url}
\usepackage{thmtools, thm-restate}
\usepackage{stmaryrd}
\usepackage[numbers,sort&compress]{natbib}
\usepackage{microtype}
\usepackage{hyperref}

\urlstyle{same}

\newtheorem{theorem}{Theorem}
\newtheorem{corollary}[theorem]{Corollary}
\newtheorem{lemma}[theorem]{Lemma}
\newtheorem{fact}{Fact}

\theoremstyle{definition}
\newtheorem{definition}{Definition}

\theoremstyle{remark}
\newtheorem{remark}{Remark}

\DeclareMathOperator{\poly}{poly}
\DeclareMathOperator{\polylog}{polylog}
\DeclareMathOperator{\dist}{dist}

\newcommand{\ldo}{\mathsf{LDO}}
\newcommand{\ld}{\mathsf{LD}}

\newcommand{\id}{\text{\sf id}}

\newcommand{\F}{\mathcal{F}}


\hypersetup{
    colorlinks=true,
    linkcolor=black,
    citecolor=black,
    filecolor=black,
    urlcolor=[rgb]{0,0.1,0.5},
    pdfauthor={Pierre Fraigniaud, Juho Hirvonen, Jukka Suomela},
    pdftitle={Node Labels in Local Decision},
}

\newenvironment{myabstract}
               {\list{}{\listparindent 1.5em\itemindent    \listparindent
                        \leftmargin    0pt
                        \rightmargin   0pt
                        \parsep        0pt}\item\relax}
               {\endlist}

\newenvironment{mycover}
               {\list{}{\listparindent 0pt
                        \itemindent    \listparindent
                        \leftmargin    0pt
                        \rightmargin   0pt
                        \parsep        0pt}\raggedright
                \item\relax}
               {\endlist}

\renewcommand*{\thefootnote}{\fnsymbol{footnote}}

\begin{document}

\vspace*{2ex}
\begin{mycover}
{\huge \bfseries Node Labels in Local Decision\par}
\bigskip
\bigskip

\textbf{Pierre Fraigniaud}

\nolinkurl{pierre.fraigniaud@liafa.univ-paris-diderot.fr}
\medskip

{\small Theoretical Computer Science Federation \\ CNRS and University Paris Diderot, France\par}

\bigskip

\textbf{Juho Hirvonen}

\nolinkurl{juho.hirvonen@aalto.fi}
\medskip

{\small Helsinki Institute for Information Technology HIIT, \\ Department of Computer Science, Aalto University, Finland\par}
\bigskip

\textbf{Jukka Suomela}

\nolinkurl{jukka.suomela@aalto.fi}
\medskip

{\small Helsinki Institute for Information Technology HIIT, \\ Department of Computer Science, Aalto University, Finland\par}


\end{mycover}

\bigskip
\begin{myabstract}
\noindent\textbf{Abstract.}
The role of unique node identifiers in network computing is well understood as far as \emph{symmetry breaking} is concerned. However, the unique identifiers also \emph{leak information} about the computing environment---in particular, they provide some nodes with information related to the size of the network. It was recently proved that in the context of \emph{local decision}, there are some decision problems such that (1)~they cannot be solved without unique identifiers, and (2)~unique node identifiers leak a \emph{sufficient} amount of information such that the problem becomes solvable (PODC 2013).

In this work we give study what is the \emph{minimal} amount of information that we need to leak from the environment to the nodes in order to solve local decision problems. Our key results are related to \emph{scalar oracles}  that, for any given , provide a multiset  of  labels; then the adversary assigns the labels to the  nodes in the network. This is a direct generalisation of the usual assumption of unique node identifiers. We give a complete characterisation of the \emph{weakest oracle} that leaks at least as much information as the unique identifiers.

Our main result is the following dichotomy: we classify scalar oracles as \emph{large} and \emph{small}, depending on their asymptotic behaviour, and show that (1)~any large oracle is at least as powerful as the unique identifiers in the context of local decision problems, while (2)~for any small oracle there are local decision problems that still benefit from unique identifiers.
\end{myabstract}

\thispagestyle{empty}
\setcounter{page}{0}
\newpage

\section{Introduction}

This work studies the role of \emph{unique node identifiers} in the context of \emph{local decision problems} in distributed systems. We generalise the concept of node identifiers by introducing \emph{scalar oracles} that choose the labels of the nodes, depending on the size of the network ---in essence, we let the oracle leak some information on  to the nodes---and ask what is the \emph{weakest} scalar oracle that we could use instead of unique identifiers. We prove the following dichotomy: we classify each scalar oracle as \emph{small} or \emph{large}, depending on its asymptotic behaviour, and we show that the large oracles are precisely those oracles that are at least as strong as unique identifiers.


\subsection{Context and background}

The  research trends within the framework of distributed computing are most often pragmatic. Problems closely related to real world applications are tackled under computational assumptions reflecting existing systems, or systems whose future existence is plausible. Unfortunately, small variations in the model settings may lead to huge gaps in terms of computational power. Typically, some problems are unsolvable in one model but may well be efficiently solvable in a slight variant of that model. In the context of \emph{network computing}, this commonly happens depending on whether the model assumes that pairwise distinct identifiers are assigned to the nodes. While the presence of distinct identifiers is inherent to some systems (typically, those composed of artificial devices), the presence of such identifiers is questionable in others (typically, those composed of biological or chemical elements). Even if the identifiers are present, they may not necessarily be directly visible, e.g., for privacy reasons.

The absence of identifiers, or the difficulty of accessing the identifiers, limits the power of computation. Indeed, it is known that the presence of identifiers ensures two crucial properties, which are both used in the design of efficient algorithms. One such property is \textbf{symmetry breaking}. The absence of identifiers makes symmetry breaking far more difficult to achieve, or even impossible if asymmetry cannot be extracted from the inputs of the nodes, from the structure of the network, or from some source of random bits. The role of the identifiers in the framework of network computing, as far as symmetry breaking is concerned, has been investigated in depth, and is now well understood \cite{angluin80local,boldi01effective,chalopin06groupings,czygrinow08fast,diks95anonymous,emek14anonymous,emek14revocable,fraigniaud01labels,goos13local-approximation,hasemann14scheduling,hella15weak-models,lenzen08leveraging,linial92locality,mayer95local,naor95what,norris94classifying-anonymous,yamashita96computing,yamashita99leader,fich03hundreds,kranakis96symmetry,suomela13survey}.

The other crucial property of the identifiers is their ability to \textbf{leak global information} about the framework in which the computation takes place. In particular, the presence of pairwise distinct identifiers guarantees that at least one node has an identifier at least~ in -node networks. This apparently very weak property was proven to actually play an important role when one is interested in checking the correctness of a system configuration  in a decentralised manner. Indeed, it was shown in prior work~\cite{fraigniaud13ld-id} that the ability to check the legality of a system configuration with respect to some given Boolean predicate differs significantly according to the ability of the nodes to use their identifiers. This phenomenon is of a nature different from symmetry breaking, and is far less understood than the latter. 

More precisely, let us define a \emph{distributed language} as a set of system configurations (e.g., the set of properly coloured networks, or the set of networks each with a unique leader). Then let  be the class of distributed languages that are \emph{locally decidable}. That is,  is the set of distributed languages for which there exists a distributed algorithm where every node inspects its neighbourhood at constant distance in the network, and outputs \emph{yes} or \emph{no} according to the following rule:  all nodes output \emph{yes} if and only if the instance is legal. Equivalently, the instance is illegal if and only if at least one node outputs \emph{no}. Let  be defined as  with the restriction the local algorithm is required to be \emph{identifier oblivious}, that is, the output of every node is the same regardless of the identifiers assigned to the nodes. By definition, , but~\cite{fraigniaud13ld-id} proved that this inclusion is strict: there are languages in . This strict inclusion was obtained by constructing a distributed language that can be decided by an algorithm whose outputs depend heavily on the identifiers assigned to the nodes, and in particular on the fact that at least one node has an identifier whose value is at least~.

The gap between  and  has little to do with symmetry breaking. Indeed, decision tasks do not require that some nodes act differently from the others: on legal instances, all nodes must output \emph{yes}, while on illegal instances, it is permitted (but not required) that all nodes output \emph{no}. The gap between  and  is entirely due to the fact that the identifiers leak information about the size  of the network. Moreover, it is known that the gap between  and  is strongly related to computability issues: there is an identifier-oblivious \emph{non-computable} simulation  of every local algorithm  that uses identifiers to decide a distributed language~\cite{fraigniaud13ld-id}. Informally, for every language in , the unique identifiers are precisely as helpful as providing the nodes with the capability of solving undecidable problems.

\subsection{Objective}

One objective of this paper is to measure the \emph{amount of information} provided to a distributed system via the labels given to its nodes. For this purpose, we consider  the classes  and   enhanced with \emph{oracles}, where an oracle  is a function that provides every node with information about its environment.

We focus on the class of \emph{scalar} oracles, which are functions over the positive integers. Given an , a scalar oracle  returns a list  of  labels (bit strings) that are assigned arbitrarily to the nodes of any -node network in a one-to-one manner. The class  (resp., ) is then defined as the class of distributed languages  decidable locally by an algorithm (resp., by an identifier-oblivious algorithm) in networks labelled with oracle~.

If, for every , the  values in the list  are pairwise distinct, then  since the nodes can use the values provided to them by the oracle as identifiers. However, as we shall demonstrate in the paper, this pairwise distinctness condition is not necessary.

Our goal is to identify the interplay between the classes , , , and , with respect to any scalar oracle~, and to characterise  the power of identifiers in distributed systems as far as leaking information about the environment is concerned. 

\subsection{Our results}

Our first result is a characterisation of the weakest oracles providing the same power as unique node identifiers. We say that a scalar oracle  is \emph{large} if, roughly,  ensures that, for any set of  nodes, the largest value provided by  to the nodes in this set grows with  (see Section~\ref{ssec:def-oracle} for the precise definition). We show the following theorem.
\begin{theorem} \label{thm:main}
    For any computable scalar oracle , we have  if and only if  is large.
\end{theorem}

Theorem~\ref{thm:main} is a consequence of the following two lemmas. The first says that small oracles (i.e.\ non-large oracles) do not capture the power of unique identifiers. Note that the following separation result holds for any small oracle, including uncomputable oracles.
\begin{restatable}{lemma}{lemweakseparation} \label{lem:weak-separation}
    For any small oracle , there exists a language .
\end{restatable}
The second is a simulation result, showing that any local decision algorithm using identifiers can be simulated by an identifier-oblivious algorithm with the help of \emph{any} large oracle, as long as the oracle itself is computable. Essentially large oracles capture the power of unique identifiers.
\begin{restatable}{lemma}{lemstrongldofldf} \label{lem:strong-ldof-ldf}
    For any large computable oracle , we have .
\end{restatable}

Theorem~\ref{thm:main} holds despite the fact that small oracles can still produce some large values, and that there exist small oracles guaranteeing that, in any -node network, at least one node has a value at least . Such a small oracle would be sufficient to decide the language  presented in~\cite{fraigniaud13ld-id}. However, it is not sufficient to decide all languages in .

Our second result is a complete description of the hierarchy of the four classes , , , and  of local decision, using identifiers or not,  with or without oracles. The pictures for small and large oracles are radically different. 
\begin{itemize}
\item  For any large oracle , the hierarchy yields a \emph{total order}:

The strict inclusion  follows from~\cite{fraigniaud13ld-id}. The second inclusion  may or may not be strict depending on oracle . 

\item For any small oracle , the hierarchy yields a \emph{partial order}. We have  as a consequence of Lemma~\ref{lem:weak-separation}. However,  and  are incomparable, in the sense that there is a language  for any small oracle , and there is a language  for some small oracles . Hence, the relationships of the four classes can be represented as the following diagram: 
3pt]
 \ldo^f  &&&& \ld \\
  & \nwarrow && \nearrow & \\
  && \ldo &&
\end{array}

    f(n) = (1,2,\dotsc,n),

    f(n) = (0,0,\dotsc,0,1).
(G,x) \in L  \iff \; \mbox{all nodes output \emph{yes}.}

f(n) = (f_1, f_2, \dots, f_n)
\forall c > 0,\, \exists k \geq 1,\, \forall n \ge k, \, f_k^{(n)} \ge c.
    L = \{ G(M, r, N) : r \ge 1,\ N \ge 1, \text{ and Turing machine } M \text{ outputs } 0 \}.
\label{eq:Li}
L_i = \{ M : \text{Turing machine } M \text{ outputs } i \} : i \in \{0,1\}.

    \hat{f}\colon N \mapsto \bigl\{ g(f_1), g(f_2), \dots, g(f_N) \bigr\}

    \ldo \subsetneq \ld \subseteq \ldo^f=\ld^f.

    \ldo \subsetneq \ld = \ldo^f=\ld^f.

    \ldo \subsetneq \ld \subsetneq \ldo^f=\ld^f.

        L = \{ J(M,r,\ell) : r \ge 1,\ \ell \in \{0,1\}, \text{ and Turing machine } M \text{ halts outputs } \ell \}.
    
        f(n) = (0,0,\dotsc,0,b_n),
    
                L_1 = \{ P(M) : \text{Turing machine } M \text{ halts} \}. 
            
                L_2 = \{ H(M,r) : r \ge 1 \text{ and Turing machine } M \text{ outputs 0}\}
            
            that we used in the proof of Lemma~\ref{lem:weak-separation}. It is known that  and  \cite{fraigniaud13ld-id}. Since checking the structure of  is in , it suffices to note that the node that receives the bit vector  of length  can use the \emph{length} of the vector as an upper bound in simulating . Thus .
        \item[:]
            Apply Lemma~\ref{lem:weak-separation}. \qedhere
    \end{itemize}
\end{proof}

We conclude by noting that Theorem~\ref{thm:small-ldof-ld} is also robust to minor variations in the definitions. In particular, the oracle does not need to know the exact value of ; it is sufficient that at least one node receives the bit string , where  is some upper bound on .

\section*{Acknowledgements}

Thanks to Laurent Feuilloley for discussions.

\bibliographystyle{plainnat}
\bibliography{articles}

\end{document}
