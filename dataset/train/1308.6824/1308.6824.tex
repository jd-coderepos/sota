\documentclass{llncs}
\usepackage[ansinew]{inputenc}
\usepackage{graphicx}
\usepackage[english]{babel}
\usepackage{mathtools,amssymb}
\usepackage[left,pagewise] {lineno}
\usepackage {xspace}
\usepackage {url}
\usepackage {plaatjes}
\usepackage {times}
\usepackage {xcolor}
\usepackage[textsize=footnotesize]{todonotes}
\usepackage {microtype}
\usepackage[vlined,ruled]{algorithm2e}
\usepackage {wrapfig}


\newcommand{\Poly}{\ensuremath{\mathcal{P}}}

\newcommand{\bettina}[2][says]{\todo{BS: #2}}
\newcommand{\danny}[2][says]{\todo{DH: #2}}
\newcommand{\david}[2][says]{\todo{DE: #2}}
\newcommand{\kevin}[2][says]{\todo{KV: #2}}
\newcommand{\maarten}[2][says]{\todo{ML: #2}}
\newcommand{\martin}[2][says]{\todo{MN: #2}}



\newcommand{\mkmrm}  [1]{\ensuremath{\mathrm{#1}}\xspace}
\newcommand{\mkmbb}  [1]{\ensuremath{\mathbb{#1}}\xspace}
\newcommand{\mkmcal} [1]{\ensuremath{\mathcal{#1}}\xspace}
\newcommand{\mkmfrak}[1]{\ensuremath{\mathfrak{#1}}\xspace}
\newcommand {\N} {\mkmbb {N}}
\newcommand {\R} {\mkmbb {R}}
\newcommand {\Q} {\mkmbb {Q}}
\newcommand {\Z} {\mkmbb {Z}}

\newcommand {\etal} {\textit {et al.}}
\newcommand {\eps} {\varepsilon}
\setlength{\fboxsep}{.5pt}
\renewcommand\thefootnote{\tiny\protect\framebox{\arabic{footnote}}}
\newcommand{\alp}{\mkmfrak{A}}
\newcommand{\sym}[1]{\mkmfrak{#1}}
\newcommand{\ply}{\Delta}
\newcommand{\shpa}{\mkmrm{s\hspace{-2pt}p}}
\newcommand{\cost}{\mkmrm{cost}}

\let\doendproof\endproof
\renewcommand\endproof{~\hfill\doendproof}

\definecolor {sepia} {hsb} {5,0.75,0.65}


\renewcommand{\paragraph}[1]{\medskip\noindent\textbf{#1.}}

\newcommand{\regions}{\ensuremath{\cal D}}
\newcommand{\arr}{\ensuremath{\cal A(\regions)}}
\newcommand{\res}{\ensuremath{r}}
\newcommand{\thk}{\ensuremath{t}}









\newtheorem{observation}{Observation}
\newenvironment {repeatobservation} [1]
{\noindent {\bf Observation~\ref{#1}.}\ \slshape} {\normalfont}
\newenvironment {repeatclaim} [1]
{\noindent {\bf Claim~\ref{#1}.}\ \slshape} {\normalfont}
\newenvironment {repeatlemma} [1]
{\noindent {\bf Lemma~\ref{#1}.}\ \slshape} {\normalfont}
\newenvironment {repeattheorem} [1]
{\noindent {\bf Theorem~\ref{#1}.}\ \slshape} {\normalfont}
\newenvironment {repeatcorollary} [1]
{\noindent {\bf Corollary~\ref{#1}.}\ \slshape} {\normalfont}



\title{Strict Confluent Drawing}

\author
{
  David Eppstein\inst{1}
  \and Danny Holten\inst{2}
  \and Maarten L\"offler\inst{3},
  \\ Martin N\"ollenburg\inst{4}
  \and Bettina Speckmann\inst{5}
  \and Kevin Verbeek\inst{6}
  }

\institute{Computer Science Department, University of California, Irvine, USA, \email{eppstein@uci.edu}
\and Synerscope BV, Eindhoven, the Netherlands, \email{danny.holten@synerscope.com}
\and Department of Computing and Information Sciences, Utrecht University, the Netherlands, \email{m.loffler@uu.nl}
\and Institute of Theoretical Informatics, Karlsruhe Institute of Technology, Germany, \email{noellenburg@kit.edu}
\and Department of Mathematics and Computer Science, Technical University Eindhoven, the Netherlands, \email{speckman@win.tue.nl}
\and Department of Computer Science, University of California, Santa Barbara, USA, \email{kverbeek@cs.ucsb.edu}
}

\pagestyle{plain} \begin{document}

\maketitle

\begin{abstract}
We define \emph{strict confluent drawing}, a form of confluent drawing in which the existence of an edge is indicated by the presence of a smooth path through a system of arcs and junctions (without crossings), and in which such a path, if it exists, must be unique. We prove that it is NP-complete to determine whether a given graph has a strict confluent drawing but polynomial to determine whether it has an \emph{outerplanar} strict confluent drawing with a fixed vertex ordering (a drawing within a disk, with the vertices placed in a given order on the boundary).
\end{abstract}


\section {Introduction}

Confluent drawing is a style of graph drawing in which edges are not drawn explicitly; instead vertex adjacency is indicated by the existence of a smooth path through a system of arcs and junctions that resemble train tracks. These types of drawings allow even very dense graphs, such as complete graphs and complete bipartite graphs, to be drawn in a planar way~\cite{degm-cd-05}.
Since its introduction, there has been much subsequent work on confluent drawing~\cite{EppGooMen-Alg-07,egm-dcd-06,EppSim-GD-11,hmr-becgcd-06,hss-ttcd-04,qa-cdard-10}, but the complexity of confluent drawing has remained unclear: how difficult is it to determine whether a given graph has a confluent drawing?
Confluent drawings have a certain visual similarity to a graph drawing technique called \emph {edge bundling}~\cite {Cui2008,Dwyer2007,Holten2006,Holten2009,Hurter2012}, in which ``similar'' edges are routed together in ``bundles'',
but we note that these drawings should be interpreted differently. In particular, sets of edges bundled together form visual junctions, however, interpreting them as confluent junctions can create false adjacencies.


Formally, a confluent drawing may be defined as a collection of \emph {vertices}, \emph {junctions} and \emph {arcs} in the plane, such that all arcs are smooth and start and end at either a junction or a vertex, such that arcs intersect only at their endpoints, and such that  all arcs that meet at a junction share the same tangent line there. A confluent drawing  represents a graph  defined as follows: the vertices of  are the vertices of , and there is an edge between two vertices  and  if and only if there exists a smooth path in  from  to  that does not pass any other vertex. (In some variants of confluent drawing an additional restriction is made that the smooth path may not intersect itself~\cite{hss-ttcd-04}; however, this constraint is not relevant for our work.)


\begin{wrapfigure}[9]{r}{0.29\textwidth}
    \vspace{-\baselineskip}
		\vspace{-5ex}
    \centering
    \subfigure[\label{fig:mult-adj}]{
		\includegraphics[scale=1]{figures/mult-adj-new}}
		\hfil
		\subfigure[\label{fig:self-loop}]{
		\includegraphics[scale=1]{figures/self-loop-new}}
	\caption{(a) A drawing with a duplicate path.
		  (b)~A drawing with a self-loop.}
	\label{fig:mult-adj+self-loop}
\end{wrapfigure}
\paragraph {Contribution}
In this paper we introduce a subclass of confluent drawings, which we call \emph {strict} confluent drawings. Strict confluent drawings are confluent drawings with the additional restrictions that between any pair of vertices there can be \emph {at most one} smooth path, and there cannot be any paths from a vertex to itself.
Figure~\ref {fig:mult-adj+self-loop} illustrates the forbidden configurations. To avoid irrelevant components in the drawing, we also require all arcs of the drawing to be part of at least one smooth path representing an edge.
We believe that these restrictions may make strict drawings easier to read, by reducing the ambiguity caused by the existence of multiple paths between vertices. In addition, as we show, the assumption of strictness allows us to completely characterize their complexity, the first such characterization for any form of confluence on arbitrary undirected graphs.




We prove the following:
\begin{itemize}
\item It is NP-complete to determine whether a given graph has a strict confluent drawing.
\item For a given graph, with a given cyclic ordering of its vertices, there is a polynomial time algorithm to find an \emph{outerplanar} strict confluent drawing, if it exists: this is a drawing in a disk, with the vertices in the given order on the boundary of the disk
\item When a graph has an outerplanar strict confluent drawing, an algorithm based on circle packing can construct a layout of the drawing in which every arc is drawn using at most two circular arcs.
\end{itemize}

\begin{figure}[b]
	\centering
		\subfigure[\label{fig:gd2011contest}]{\includegraphics[scale=0.33]{figures/gd2011contest}}
		\hfil
		\subfigure[\label{fig:5wheel-new}]{
		\includegraphics[page=2]{figures/5wheel-new}}
	\caption{(a) Outerplanar strict confluent drawing of the GD2011 contest graph. (b) A graph with no outerplanar strict confluent drawing.}
	\label{fig:sea-creature-5wheel-new}
\end{figure}

See Fig.~\ref{fig:gd2011contest} for an example of an outerplanar strict confluent drawing.  Previous work on  \emph{tree-confluent}~\cite{hss-ttcd-04} and \emph{delta-confluent drawings}~\cite{egm-dcd-06} characterized special cases of outerplanar strict confluent drawings as being the chordal bipartite graphs and distance-hereditary graphs respectively, so these graphs as well as the outerplanar graphs are all outerplanar strict confluent. The six-vertex wheel graph in Fig.~\ref{fig:5wheel-new} provides an example of a graph that does not have an outerplanar strict confluent drawing. (The central vertex  needs to be placed between two of the outer vertices, say,  and . The smooth path from  to the opposite vertex  separates  and , so there must be a junction shared by the -- and -- paths, creating a wrong adjacency with .)




\section {Preliminaries}

Let  be a graph.
We call an edge  in a drawing  \emph {direct} if it consists only of a single arc (that does not pass through junctions).
We call the angle between two consecutive arcs at a junction or vertex \emph{sharp} if the two arcs do not form a smooth path; each junction has exactly two angles that are not sharp, and every angle at a vertex is sharp (so the number of sharp angles equals the degree of the vertex).

\begin {lemma} \label {lem:deg2}
  Let  be a graph, and let  be the edges of  that are incident to at least one vertex of degree .
  If  has a strict confluent drawing , then it also has a strict confluent drawing  in which all edges in  are direct.
\end {lemma}

\begin{proof}
  Let  be a degree-2 vertex in  with two incident edges  and . We consider the representation of  and  in  and modify  so that  and  are single arcs. There are two cases. If  and  leave  on two disjoint paths, then these paths have only merge junctions from 's perspective. We can simply separate these junctions from  and  as shown in Fig.~\ref{sfg:sing-arc-disjoint}. If, on the other hand,  and  share the same path leaving , then their paths split at some point. We need to reroute the merge junctions prior to the split and separate the merge junctions after the split as shown in Fig.~\ref{sfg:sing-arc-joint}. This is always possible since  has no other incident edges. Because  was strict and these changes do not affect strictness,  is still a strict confluent drawing and edges  and  are direct.
\end{proof}

\begin{figure}[htbp]
  \centering
  \subfigure[\label{sfg:sing-arc-disjoint}]{\includegraphics[page=1]{figures/single-arc}}
  \hfill  \subfigure[\label{sfg:sing-arc-joint}]{\includegraphics[page=2]{figures/single-arc}}
  \caption{The two cases of creating single arcs for edges incident to a degree-2 vertex.}
  \label{fig:single-arc}
\end{figure}

\begin {lemma} \label {lem:k22}
  Let  be a graph. If  has no  as a subgraph, whose vertices have degrees  in , then  has a strict confluent drawing if and only if  is planar.
\end {lemma}

\begin{proof}
  Since every planar drawing is also a strict confluent drawing, that implication is obvious.
	So let  be a strict confluent drawing for a graph  without a  subgraph, whose vertices have degrees  in . Since larger junctions, where more than three arcs meet, can easily be transformed into an equivalent sequence of binary junctions, we can assume that every junction in  is binary, i.e., two arcs merge into one (or, from a different perspective, one arc splits into two).
	By Lemma~\ref{lem:deg2} we can further transform  so that all edges incident to degree- vertices are direct.
	Now for any vertex  in  none of its outgoing paths to some neighbor  can visit a merge junction before visiting a split junction as this would imply either a non-strict drawing or a  subgraph with vertex degrees . So the sequence of junctions on any - path consists of a number of split junctions followed by a number of merge junctions. But any such path can be unbundled from its junctions to the left and right and turned into a direct edge without creating arc intersections  as illustrated in Fig.~\ref{fig:no-k22}. This shows that  can be transformed into a standard planar drawing of~.
\end{proof}

\begin{figure}[htbp]
	\centering
		\includegraphics[scale=1]{figures/nok22}
	\caption{Any strict confluent drawing of a graph without a  subgraph can be transformed into a standard planar drawing.}
	\label{fig:no-k22}
\end{figure}

The next lemma characterizes the combinatorial complexity of strict confluent drawings. Its proof is found in Appendix~\ref{app:complexity} and uses Euler's formula and double counting.

\begin{lemma}\label{lem:linear}
	The combinatorial complexity of any strict confluent drawing  of a graph~, i.e., the number of arcs, junctions, and faces in , is linear in the number of vertices of~. 
\end{lemma}

Lemma~\ref{lem:linear} is in contrast to previous methods for confluently drawing interval graphs~\cite{degm-cd-05} and for drawing confluent Hasse diagrams~\cite{EppSim-GD-11}, both of which may produce (non-strict) drawings with quadratically many features.




\section {Computational Complexity}

We will show by a reduction from planar 3-SAT~\cite{l-pftu-82} that it is NP-complete to decide whether a graph  has a strict confluent drawing in which all edges incident to degree- vertices are direct. By Lemma~\ref {lem:deg2}, this is enough to show that it is also NP-complete to decide if  has any strict confluent drawing.



Consider the subdivided grid graph (a grid with one extra vertex on each edge). In this graph, all edges are adjacent to a degree  vertex. Since a grid graph more than one square wide has only one fixed planar embedding (up to choice of the outer face), the subdivided grid graph has only one confluent embedding in which all edges are direct. We will base our construction on a number of such grids.

\tweeplaatjes [scale=.78] {hard-global-3sat} {hard-global-frame}
{ (a) A planar 3-SAT formula.
  (b) The corresponding global frame of the construction: one grid graph per variable, with some vertices identified at each clause. Green boundary edges correspond to positive literals, red edges to negated literals.
      For easier readability the grids in this figure are larger than strictly necessary.
}

Let  be a planar 3-SAT formula. Globally speaking, we will create a grid graph for each variable of , of size depending on the number of clauses that the variable appears in. The external edges of this grid graph are alternatingly colored green and red.
We connect the variable graphs by identifying certain vertices: for each of the three variables that appear in a clause, we select one subdivided edge (that is, three vertices connected by two edges) on the outer face, and identify the endpoints of these edges into a triangle of subdivided edges (that is, a -cycle). We choose a green edge for a positive occurrence of the variable and a red edge for a negated occurrence. This will become clear below.
We call the resulting graph  the \emph {frame} of the construction; all edges of  are adjacent to a degree- vertex and  has only one planar embedding (up to choice of the outer face).
Figure~\ref {fig:hard-global-3sat+hard-global-frame} shows an example.

\eenplaatje {hardness-K4} { and its two strict confluent drawings, without moving the vertices and keeping all arcs inside the convex hull of the vertices.}
\drieplaatjes [scale=.97] {hardness-var-graph} {hardness-var-true} {hardness-var-false}
{ (a) A variable gadget consists of a grid of 's. Green (light) edges of the frame highlight normal literals, red (dark) edges negated ones.
  (b) One of the two possible strict  confluent drawings, corresponding to the value \emph {true}.
  (c) The other strict confluent drawing, corresponding to \emph {false}.
}

\begin{wrapfigure}[18]{r}{0.4\textwidth}
\centering
    \includegraphics[scale=.8]{figures/hardness-clause-attached}
	\caption{Three variables attached to a clause gadget. The top left variable occurs in the clause as a positive literal, the others as negative literals. The clause can be satisfied because the top right variable is set to \emph {false}.}
	\label{fig:hardness-clause-attached}
\end{wrapfigure}
The main idea of the construction is based on the fact that , when drawn with all four vertices on the outer face, has exactly two strict confluent drawings: we need to create a junction that merges the diagonal edges with one pair of opposite edges, and we can choose the pair.
Figure~\ref {fig:hardness-K4} illustrates this.
We will add a copy of  to every cell of the frame graph . Recall that every cell, except for the triangular clause faces, is a subdivided square (that is, an -cycle). We add  on the four grid vertices (not the subdivision vertices). The edges that connect external grid vertices are called \emph{literal edges}.
Figure~\ref {fig:hardness-var-graph} shows this for a small grid.
Since neighboring grid cells share a (subdivided) edge, the 's are not edge-independent. This implies that in a strict confluent drawing, we cannot ``use'' such a common edge in both cells. Therefore, we need to orient the -junctions alternatingly, as illustrated in Figures~\ref {fig:hardness-var-true} and~\ref {fig:hardness-var-false}.
If the grid is sufficiently large (every cell is part of a larger at least size- grid) these choices are completely propagated through the entire grid, so there are two structurally different possible embeddings,  which we use to represent the values \emph {true} and \emph {false} of the corresponding variable. For every green edge of the frame in the \emph{true} state and every red edge in the \emph{false} state there is one remaining literal edge in the outer face, which can still be drawn either inside or outside their grid cells. In the opposite states these literal edges are needed inside the grid cells to create the  junctions.  The availability of at least one literal edge (corresponding to a \emph{true} literal) is important for satisfying the clause gadgets, which we describe next.


\vierplaatjes {hardness-clause} {hardness-clause-1} {hardness-clause-2} {hardness-clause-3}
{ (a) The input graph of the clause.
  (b, c, d) Three different strict confluent drawings.
}



Inside each triangular clause face, we add the graph depicted in Figure~\ref {fig:hardness-clause}. This graph has several strict confluent drawings; however, in every drawing at least one of the three outer edges needs to be drawn inside the subdivided triangle.





\begin {lemma}\label{lem:one-edge-in}
  There is no strict confluent drawing of the clause graph in which all three long edges are drawn outside. Moreover, there is a strict confluent drawing of the clause graph with two of these edges outside, for every pair.
\end {lemma}

\begin {proof}
  Recall that by Lemma~\ref {lem:deg2} the subdivided triangle must be embedded as a -cycle of direct arcs.
To prove the first part of the lemma, assume that the triangle edges are all drawn outside this cycle.
  The remainder of the graph has no -cycles without subdivision vertices (that is, no  with higher-degree vertices), so by Lemma~\ref {lem:k22} it can only have a strict confluent drawing if it is planar. However, it is a subdivided , which is not planar.  To prove the second part of the lemma, we refer to Figures~\ref {fig:hardness-clause-1},~\ref {fig:hardness-clause-2} and~\ref {fig:hardness-clause-3}.
\end {proof}



This describes the reduction from a planar 3-SAT instance to a graph consisting of variable and clause gadgets. Next we show that this graph has a strict confluent drawing if and only if the planar 3-SAT formula is satisfiable. For a given satisfying assignment we choose the corresponding embeddings of all variable gadgets. The assignment has at least one \emph{true} literal per clause, and correspondingly in each clause gadget one of the three literal edges can be drawn inside the clause triangle, allowing a strict confluent drawing by Lemma~\ref{lem:one-edge-in}. Conversely, in any strict confluent drawing, each clause must be drawn with at least one literal edge inside the clause triangle by Lemma~\ref{lem:one-edge-in}, so translating the state of each variable gadget into its truth value yields a satisfying assignment.

To show that testing strict confluence is in NP, recall that by Lemma~\ref{lem:linear} the combinatorial complexity of the drawing is linear in the number of vertices. Thus the existence of a drawing can be verified by guessing its combinatorial structure and verifying that it is planar and a drawing of the correct graph.

\begin {theorem}
  Deciding whether a graph has a strict confluent drawing is NP-complete.
\end {theorem}

\section {Outerplanar Strict Confluent Drawings}


For a graph  with a fixed cyclic ordering of its vertices, we can test in polynomial time whether an outerplanar strict confluent drawing with this vertex ordering exists, and, if so, construct one. This algorithm uses the closely related notion of a canonical diagram of , which is unique and exists if and only if an outerplanar strict confluent drawing exists. From the canonical diagram a confluent drawing can be constructed. We further show that the drawing can be constructed such that every arc consists of at most two circular arcs. 



\subsection{Canonical Diagrams}

We define a \emph{canonical diagram} to be a collection of junctions and arcs connecting the vertices in the given order on the outer face (as in a confluent drawing), but with some of the faces of the diagram \emph{marked}, satisfying additional constraints enumerated below. Figure~\ref{fig:threeviews} shows a canonical diagram and an outerplanar strict confluent drawing of the same graph. In such a diagram, a \emph{trail} is a smooth curve from one vertex to another that follows the arcs (as in a confluent drawing) but is allowed to cross the interior of marked faces from one of its sharp corners to another. The constraints are:
\begin{itemize}
\item Every arc is part of at least one trail.
\item No two trails between the same two vertices can follow different sequences of arcs and faces.
\item Each marked face must have at least four angles, all of which are sharp.
\item Each arc must have either sharp angles or vertices at both of its ends.
\item For each junction  with exactly two arcs in each direction, let  and  be the two faces with sharp angles at . Then it is not allowed for  and  to both be either marked or to be a triangle (a face with three angles, all sharp).
\end{itemize}

\eenplaatje [width=\textwidth] {threeviews} {Three views of the same graph as a node-link diagram (left), canonical diagram (center), and outerplanar strict confluent drawing (right).}

Let  be a junction of a canonical diagram . Then
define the \emph{funnel} of  to be the 4-tuple of vertices  where 
is the vertex reached by a path that leaves  in one direction and
continues as far clockwise as possible,  is the most counterclockwise
vertex reachable in the same direction from ,  is the most clockwise
vertex reachable in the other direction, and  is the most
counterclockwise vertex reachable in the other direction. Note that
none of the paths from  to , , , and  can intersect each other without contradicting the uniqueness of trails. We call the circular intervals of vertices  and  (in the counterclockwise direction) the \emph{funnel intervals} of the respective funnel. We say a circular interval  is \emph{separated} if either  and  are not adjacent in , or there exists a junction in the canonical diagram with funnel intervals  and , where .

A canonical diagram represents a graph  in which the edges in  correspond to trails in the diagram. As we show in Appendix~\ref {app:canonical}, a graph  has a canonical diagram if and only if it has an outerplanar strict confluent drawing, and if a canonical diagram exists then it is unique.

\subsection{Algorithm}

By using the properties of canonical diagrams (see Appendix~\ref{app:canonical}), we may obtain an algorithm that constructs a canonical diagram and strict confluent drawing of a given cyclically-ordered graph , or reports that no drawing exists, in time and space . This bound is optimal in the worst case, as it matches the input size of a graph that may have quadratically many edges.

Steps~1--3 of the algorithm, detailed below, build some simple data structures that speed up the subsequent computations. Step~4 discovers all of the funnels in the input, from which it constructs a list of all of the junctions of the canonical diagram. Step~5 connects these junctions into a planar drawing, a subset of the canonical diagram. Step~6 builds a graph for each face of this drawing that will be used to complete it into the entire canonical diagram, and step~7 uses these graphs to find the remaining arcs of the diagram and to determine which faces of the diagram are marked. Step~8 checks that the diagram constructed by the previous steps correctly represents the input graph, and step~9 splits the marked faces, converting the diagram into a strict confluent drawing.

\begin{enumerate}
\item Number the vertices clockwise around the boundary cycle from  to .
\item\label{algorithm interval adjacency} Build a table, containing for each pair , the number of ordered pairs  with , , and vertices  and  adjacent in . By performing a constant number of lookups in this table we may determine in constant time how many edges exist between any two disjoint intervals of the boundary cycle.
\item Build a table that lists, for each ordered pair  of vertices, the neighbor  of  that is closest in clockwise order to . That is,  is adjacent to , and the interval from  clockwise to  contains no other neighbors of . The table entries for  can be found in linear time by a single counterclockwise scan. Repeat the same construction in the opposite orientation.
\item\label{algorithm find junctions} For each separated interval , let  be the next neighbor of  that is counterclockwise of , and let  be the next neighbor of  that is clockwise of . If (i)  is a neighbor of , (ii)  is a neighbor of , (iii)  is the next neighbor of  that is counterclockwise of , and (iv)  is the next neighbor of  that is clockwise of , then
(if a confluent diagram exists)  must form the funnel of a junction, and all funnels have this form. We check all circular intervals in increasing order of their cardinalities. For each discovered funnel, we mark the intervals that are separated by the corresponding junction. This way we can check in  time whether a circular interval is separated.
If the number of funnels exceeds the linear bound of Lemma~\ref{lem:linear} on the number of junctions in a confluent drawing, abort the algorithm.
\item\label{algorithm funnel junctions} Create a junction for each of the funnels found in step~\ref{algorithm find junctions}. For each vertex , make a set  of the junctions whose funnel includes that vertex; if they are to be drawn as part of a canonical diagram, the junctions of  need to be connected to  by a confluent tree. For any two junctions in , it is possible to determine in constant time whether one is an ancestor of another in this tree, or if not whether one is clockwise of the other, by examining the cyclic ordering of vertices in their funnels. Construct the trees of junctions and their planar embedding in this way. The result of this stage of the algorithm should be a planar embedding of part of the canonical diagram consisting of all vertices and junctions, and the subset of the arcs that are part of a path from a junction to one of its funnel vertices. Check that the embedding is planar by computing its Euler characteristic, and abort the algorithm if it is not.
\item For each face  of the drawing created in step~\ref{algorithm funnel junctions}, and each pair  of junctions belonging to , use the data structure from step~\ref{algorithm interval adjacency} to test whether there is an edge whose trail passes through both  and . This results in a graph  in which the vertices represent the vertices or junctions on the boundary of  and the edges represent pairs of vertices or junctions that must be connected, either by an arc or by shared membership in a marked face. The remaining arcs to be drawn in  will be exactly the edges of  that are not crossed by other edges of ; the marked faces in  will be exactly the faces that contain pairs of crossing edges of .
\item Within each face  of the drawing so far, build a table using the same construction as in step~\ref{algorithm interval adjacency} that can be used to determine the existence of a crossing edge for an edge in  in constant time. Use this data structure to identify the crossed edges, and draw an arc in  for each uncrossed edge. For each face  of the resulting subdivision of , if  has four or more vertices or junctions, find two pairs that would cross and test whether both pairs correspond to edges in ; if so, mark~.
\item Construct a directed graph that has a vertex for each vertex of , two vertices for each junction of the diagram (one in each direction), two directed edges for each arc, and a directed edge for each ordered pair of sharp angles that are non-consecutive in a marked face. By performing a depth-first search in this graph, determine whether there exist multiple smooth paths in the resulting drawing from any vertex of  to any other point in the drawing, and abort the algorithm if any such pair of paths is found. Determine the set of vertices of  reachable from  and verify that it is the same set of vertices that are reachable in the original graph. Additionally, verify that the diagram satisfies the requirements in the definition of a canonical diagram. Abort the algorithm if any inconsistency is found in this step.
\item Convert the canonical diagram into a confluent drawing and return it.
\end{enumerate}

\begin{theorem}
For a given -vertex graph , and a given circular ordering of its vertices, it is possible to determine whether  has an outerplanar strict confluent drawing with the given vertex ordering, and if so to construct one, in time .
\end{theorem}

\subsection{Drawings with low curve complexity}\label{sec:curvecomplexity}

Suppose that we are given a topological description of an outerplanar strict confluent drawing  of a connected graph , describing the tangency pattern and ordering of the arcs at each junction. It still remains to draw  (or possibly an equivalent but combinatorially different outerplanar strict confluent drawing) in the plane using concrete curves for its arcs. If we ignore the tangency requirements at its junctions, the arcs and junctions of  form a planar graph, but applying standard planar graph drawing methods will generate arcs that may not be smooth and that are not tangent to each other at the junctions. So how are we to draw ? In this section we use a circle packing method to draw  with a small number of circular arcs for each arc of~.
Thus, these drawings have low \emph{curve complexity} in the sense of Bekos et al.~\cite{BekKauKob-GD-12}, but with this complexity measured along arcs of the confluent diagram rather than edges of another type of graph drawing.


Given such a drawing , let  be a modified version of  in which every junction is incident to exactly three arcs, formed from  by suppressing two-arc junctions and splitting junctions with more than three arcs.
Assume also (again by adding more junctions if necessary) that each vertex in  has only a single arc incident to it.

Given the topological diagram , we form a planar graph  that has a vertex for each vertex or junction of , and an edge for each arc of .  Additionally, we create an edge in  for each two vertices that are consecutive in the cyclic ordering of the vertices around the disk containing the drawing.


\begin{lemma}
\label{lem:3-regular}
 is planar, 3-regular, and 3-vertex-connected.
\end{lemma}

\begin{proof}
Planarity and 3-regularity follow immediately from the construction of . Every two vertices of  are connected by three vertex-disjoint paths in : at least one (not necessarily a smooth path) through , using the assumption that  is connected, and two more around the boundary of the disk. Therefore, if  were not 3-vertex-connected, only one of its 3-connected components could contain vertices of . The other components would either contain components of  that are not part of any smooth path between vertices of  (forbidden in a strict confluent drawing) or would contain more than one smooth path between the same sets of vertices (also forbidden).
\end{proof}




\begin{theorem}
Let  be an outerplanar strict confluent drawing of a graph , given topologically but not geometrically. Then we can construct an outerplanar strict confluent drawing of  in which each arc of the drawing is represented by a smooth curve that is either a circular arc or the union of two circular arcs.
\end{theorem}

\noindent\textit{Proof.}
By the Koebe--Thurston--Andreev circle packing theorem, there exists a system  of circles representing the faces of , such that two circles are adjacent exactly when the corresponding faces share an edge. We may assume (by performing a M\"obius transformation if necessary) that the outer circle of this circle packing corresponds to the outer face of .  may be found efficiently (although not in strongly polynomial time) by a numerical iteration that quickly converges to the system of radii of the circles, from which their centers can also be computed easily~\cite{ColSte-CGTA-03,Moh-DM-93}.

\begin{wrapfigure}[20]{r}{0.5\textwidth}
    \vspace{-\baselineskip}
    \centering
    \includegraphics[scale=0.4] {figures/confluent-circles}
	\caption{Construction of an outerplanar strict confluent drawing from a circle packing. The vertices of the drawing correspond to triangular gaps adjacent to the outer circle, and the junctions  correspond to the remaining triangular gaps.}
	\label{fig:confluent-circles}
\end{wrapfigure}
Each vertex of  corresponds in  to one of the triangular gaps between the outer circle and two other circles, and may be placed at the point of tangency of the two non-outer circles (one of the vertices of this triangle); see Fig.~\ref{fig:confluent-circles}.
The junctions in  lie at the meeting point of three faces of , and correspond in  to the remaining triangular gaps between three circles. A confluent drawing of  may be formed by removing the outer circle, removing all circular arcs bounding the triangular gaps incident to the outer circle, and in each remaining triangular gap removing the arc that is on the other side of the sharp angle. The resulting drawing contracts some edges of  to form junctions with four incident arcs, but this does not affect the correctness of the drawing.
In the resulting drawing, arcs of the diagram that have merge points or vertices at both of their endpoints are drawn as two circular arcs (possibly both from the same circle); other arcs of the diagram are drawn as a single circular arc.\qed


\section {Conclusions}

We have shown that, in confluent drawing, restricting attention to the strict drawings allows us to completely characterize their complexity, and we have also shown that outerplanar strict confluent drawings with a fixed vertex ordering may be constructed in polynomial time. The most pressing problem left open by this research is to recognize the graphs that have outerplanar strict confluent drawings, without imposing a fixed vertex order. Can we recognize these graphs in polynomial time?

\medskip
\paragraph{Acknowledgements}
This work originated at Dagstuhl seminar 13151, \emph {Drawing Graphs and Maps with Curves}. D.E. was supported in part by the National Science Foundation under grants 0830403 and 1217322, and by the Office of Naval Research under MURI grant N00014-08-1-1015. M.L. was supported by the Netherlands Organisation for Scientific Research (NWO) under grant 639.021.123. M.N. received financial support by the `Concept for the Future' of KIT under grant YIG 10-209.

\small
\bibliographystyle{abbrv}
\bibliography{confluence}
\normalsize
\appendix\clearpage


\section{Combinatorial Complexity} \label {app:complexity}

We first assume that every vertex has degree one and every junction has degree three in the confluent drawing. Note that it is always possible to expand a confluent drawing to satisfy these restrictions. This expansion maintains the number of faces, and can only increase the number of arcs and junctions. Afterwards we will also consider junctions and vertices of higher degrees.
We also use the following lemma to reduce the number of cases we need to consider.

\begin {lemma} \label {lem:noloops}
  Let  be a strict confluent drawing.  cannot contain any smooth closed curves.
\end {lemma}

\begin {proof}
  Such a curve would either be irrelevant or would allow multiple paths (looping around the curve multiple times) for some pair of vertices.
\end {proof}

The following simple lemma is crucial for our bounds.
\begin{lemma}\label{lem:threesharp}
Every face in a strict confluent drawing must have at least three sharp angles.
\end{lemma}
\begin{proof}
Consider a face with less than three sharp angles. If a sharp angle is at a junction, then the arc opposite of the sharp angle must eventually reach a vertex, since by Lemma~\ref {lem:noloops} the drawing cannot have loops. Hence we can assume that there is a vertex at each sharp angle. But then the face must have at least three vertices, for otherwise it contains a self-loop (one vertex) or multiple paths between the same pair of vertices (two vertices).
\end{proof}
In the following,  is the number of vertices,  is the number of junctions,  is the number of arcs,  is the number of faces, and  is the number of sharp angles. We first bound the combinatorial complexity of strict confluent drawings.
\begin{lemma}\label{lem:boundplanar}
Every strict confluent drawing has at most  faces,  junctions, and  arcs.
\end{lemma}
\begin{proof}
By double counting we get that . The total number of sharp angles is , and Lemma~\ref{lem:threesharp} implies that , so that . By combining the above relations with Euler's formula , we directly obtain that , , and .
\end{proof}
We can obtain similar bounds for outerplanar strict confluent drawings. Here we use  to denote the number of internal faces ().
\begin{lemma}\label{lem:boundouterplanar}
Every outerplanar strict confluent drawing has at most  internal faces,  junctions, and  arcs.
\end{lemma}
\begin{proof}
By double counting we get that . The total number of sharp angles in internal faces is  (every vertex is on the outer face), and Lemma~\ref{lem:threesharp} implies that , so that . By combining the above relations with Euler's formula , we directly obtain that , , and .
\end{proof}


\smallskip\noindent
{\bf Complex junctions.} If we allow junctions and vertices to have higher degree, then we can consider \emph{minimal drawings}, that is, confluent drawings with as few arcs and junctions as possible. For strict confluent drawings that are minimal we can obtain stronger upper bounds on the combinatorial complexity. Consider a junction of degree three. We call the arc opposite of the sharp angle the \emph{free arc}. Note that, regardless of what is on the other side of a free arc, we can always contract a free arc without changing the underlying graph represented by the confluent drawing. Furthermore, contracting a free arc will not influence other free arcs. That means that we can contract all free arcs simultaneously. Now consider the unique arc incident to a vertex. If this arc is not free, then the sharp angle of the adjacent junction must be bounded by this arc. In this case we call the respective sharp angle \emph{pointless}. This angle lies in the face that contains the vertex, but does not act as a ``corner'' of that face, and hence does not count towards the at least three sharp angles necessary for a face in a strict confluent drawing, as required by Lemma~\ref{lem:threesharp}. Let  be the total number of pointless sharp angles. Following the arguments of the proof of Lemma~\ref{lem:boundplanar} and instead using , we obtain bounds  and  for junctions and arcs, respectively. Note that at least  vertices must be adjacent to a free arc. Since every arc can be shared by at most two junctions/vertices, there must be at least  free arcs. Hence there can be at most  junctions and at most  arcs (using ) in a minimal strict confluent drawing.

For outerplanar strict confluent drawings, note that all pointless sharp angles lie in the outer face. Using the same reasoning as above with the bounds of Lemma~\ref{lem:boundouterplanar} (using ), we obtain the following lemma.
\begin{lemma}\label{lem:boundminimal}
Every strict confluent drawing that is minimal has at most  junctions and  arcs. Every outerplanar strict confluent drawing that is minimal has at most  junctions and  arcs.
\end{lemma}
Although we cannot prove a tight bound for strict confluent drawings, the above bound for outerplanar strict confluent drawings is in fact tight.
\begin{lemma}
Every outerplanar strict confluent drawing of a clique on  vertices has  internal faces, and at least  junctions and  arcs.
\end{lemma}
\begin{proof}
It is easy to see that every outerplanar strict confluent drawing of a graph with a triangle must contain an internal face  with three junctions. Also, since the underlying graph is a clique, the drawing cannot contain outside connections to an arc of , as the origin of this connection will not be able to reach the junction of  opposite of the arc. Hence, every vertex must reach  through one of the three junctions. This means we can subdivide the drawing into three parts, each a clique including one of the junctions of  of size at least two. This leads to the following general recurrence relation

This recurrence solves to . Note that every arc of the face  must be present in the drawing and cannot be contracted. For the number of internal faces we use  to obtain  internal faces. For the number of arcs we use  to obtain  arcs. Finally, for the number of junctions/vertices we use  to obtain  junctions/vertices, which implies  junctions. For the last two bounds it is important that every subproblem on two elements involves at least one junction, which is true if we start with  vertices.
\end{proof}

\section{Properties of Canonical Diagrams} \label {app:canonical}

\begin{lemma}
\label{lem:consecutive}
In every outerplanar strict confluent drawing or canonical diagram, in which each vertex has at least one incident edge, there must be a pair of  vertices that are consecutive on the outer face of the drawing and adjacent in the corresponding graph. If there are at least three vertices, then there must be at least two such pairs.
\end{lemma}

\begin{proof}
Let  be an adjacent pair of vertices that are as close as possible to each other on the outer face, as measured by the smaller of the two sequences of vertices from  to  and from  to  around the outer face.
Then  must be consecutive. For, if they were not consecutive, then there would be a vertex  between them.  Any trail from  to one of its neighbors would have to cross the trail for , causing  to be adjacent to one of  or . But this would contradict the choice of  as being as close as possible for an adjacent pair.

Next, suppose that there are three or more vertices and let  be an adjacent pair of vertices that are as close as possible to each other by a different distance, the size of the sequence of vertices from  to  or  to  around the outer face, whichever of these two sequences does not contain the consecutive pair . Note that  cannot equal , because  is the pair with the largest distance by this measure. Again, we claim that  must be consecutive, for if they were separated by another vertex  then the trail from  to one of its neighbors would have to cross the trail for , causing  to be adjacent to one of  or  and leading to a contradiction.
\end{proof}

\begin{lemma}
\label{lem:marked-tree}
In every canonical diagram the dual graph of the set of marked faces forms a forest.
\end{lemma}

\begin{proof}
Suppose for a contradiction that this dual graph contains an induced cycle . It is not possible for marked faces to entirely surround a single junction of the diagram, because they all have sharp angles at the junction and every junction has two non-sharp angles. Therefore, the part of the diagram inside  must consist of one or more faces bounded by arcs of the faces in . These surrounded faces may form a single connected region , or they may form multiple connected regions separated by junctions that appear more than once along the boundary of ; in the latter case, these regions and the junctions that connect them form a tree, and we may choose  to be a leaf of the tree. Thus, in either of these two cases there exists a region  consisting of a set of neighboring faces of the diagram, in which all but at most one of the junctions on the boundary of  span an angle of  (the one exceptional junction being the one that connects  to other connected regions within ).

The part of the diagram within  can itself be viewed as a confluent drawing, within which each of the junctions that spans an angle of  is connected to at least one other junction (otherwise the arcs into that junction could not be part of a complete trail). By Lemma~\ref{lem:consecutive}, there exist two consecutive junctions on the boundary of  that are connected by a smooth path within . This path, together with a smooth path connecting the same two junctions within the marked face of  that contains them both, forms a continuous smooth loop in the original diagram, contradicting Lemma~\ref {lem:noloops}. This contradiction shows that  cannot exist.
\end{proof}

\eenplaatje[width=\textwidth] {canonize} {The simplification operations that transform an outerplanar strict confluent drawing into its canonical diagram: contraction of arcs without sharp angles at both ends (left) and merger of marked faces and triangles (right).}

\begin{lemma}
\label{lem:canonize}
A graph  may be represented by a canonical diagram if and only if it may be represented by an outerplanar strict confluent drawing.
\end{lemma}

\begin{proof}
To form a canonical diagram from a confluent drawing of , repeatedly perform the following two simplifications (Figure~\ref{fig:canonize}):
\begin{itemize}
\item Contract any arc that does not have sharp angles or vertices at both of its ends.
\item Let  be any junction with exactly two arcs in each direction, let  and  be the two faces with sharp angles at , and suppose that  and  are both either marked or a triangle. Then merge the two faces  and  by connecting them through , removing the junction from the drawing, and mark the resulting merged face.
\end{itemize}
These operations preserve the properties that each arc is part of a trail, that each edge of  is represented by a combinatorially unique trail, that each non-adjacent pair in  has no trail, and that each marked face have the correct shape. Each step reduces the number of arcs, so this simplification process eventually terminates with a canonical diagram.

Conversely, any canonical diagram can be converted to an equivalent confluent drawing by repeatedly reversing the second of the two simplification operations. The reverse of this operation splits a marked face  by pinching together two nonadjacent sides to form a new junction with four incident arcs; the two faces formed from  are marked if they have more than three angles, or left unmarked if they are triangles. It is not possible to perform this pinching off step if one or both of the sides of the step is the boundary of another marked face. However, by Lemma~\ref{lem:marked-tree} it is always possible to choose a marked face to simplify that forms an isolated vertex or leaf of the dual graph of the set of remaining marked faces; such a face can always safely be simplified.
\end{proof}

In a canonical diagram , define a \emph{pseudotriangle} for a set  of vertices to be a face of  that has exactly three sharp junctions, each of which is part of a trail from the face to a vertex in , and for which all other boundary junctions (if they exist) do not lead to vertices in . A \emph{side} of a pseudotriangle is the path of arcs connecting two of its sharp angles.

\begin{lemma}
\label{lem:candy-clique}
Let  be represented by a canonical diagram  and let  be a clique of two or more vertices in .
Then the arcs and faces traversed by trails connecting vertices of  form a tree of arcs, marked faces, and pseudotriangles for , connected to each other at junctions. In this tree, each junction is incident to exactly two arcs, marked faces, or pseudotriangles, one in each direction.
\end{lemma}

\begin{proof}
We use induction on the size of ; as a base case, the result is clearly true for , for which the single trail forms a path of arcs and marked faces. If  and  is an arbitrary vertex of , then the trails through  form a tree of arcs, marked faces, and pseudotriangles by the induction hypothesis.
In order to reach every vertex of , and avoid making multiple connections to any vertex, the trails from  to  can only connect to this tree in one of three ways:
\begin{enumerate}
\item It may be incident to one of the sharp angles of a marked face of  that already belongs to  the tree.
\item It may be incident to a sharp angle of a marked face of  that is not already in the tree, but that has an arc in the tree, causing the face to be added to the tree.
\item It may be incident to a sharp angle of a pseudotriangle of  that is not already in the tree, but whose opposite side is already in the tree, again causing the face to be added to the tree.
\end{enumerate}
In any of these cases, outside of the tree for , all the trails to  must follow the same sequence of arcs and marked faces, for to do otherwise would violate the uniqueness of trails in canonical diagrams. The tree for , together with the arcs and marked faces traversed by the rest of the trail from this tree to , forms another structure of the same type, and contains trails from all vertices of  to .
\end{proof}

\begin{lemma}
\label{lem:funnel-nonadj}
For every junction  of a canonical diagram with funnel intervals  and , at least one of the intervals must be separated.
\end{lemma}
\begin{proof}
For the sake of contradiction, assume that  and  are not separated. Then the vertices , , , and  form a clique. By Lemma~\ref{lem:candy-clique} the drawing
would have two pseudotriangles or marked faces, one on each side of .
The boundaries of these faces adjacent to  can be the only arcs into , for otherwise one of , , , or  would not be in the funnel (see Fig.~\ref{fig:FunnelSep}). Now assume there is a junction  between  and  (or  and ) on the boundary of the face opposite to  (in this case the face must be a pseudotriangle). Consider the path that leaves  in the direction of  and continues as far clockwise as possible. This path cannot pass through the pseudotriangle, because that would result in multiple paths between  and either  or , which is not allowed. Hence this path must reach , for otherwise  would not be a funnel vertex of . Similarly, the path that leaves  in the direction of  and continues as counterclockwise as possible must reach . The other paths of the funnel must end at a vertex in the circular interval . This would imply that  is separated, which contradicts our assumption. Thus, the faces on either side of  must be marked faces or triangles. But this configuration violates the condition that there be no junction  with four arcs in which the two faces having sharp angles at  are marked or triangles.
\end{proof}

\eenplaatje{FunnelSep} {The configuration of Lemma~\ref{lem:funnel-nonadj}.}

\begin{lemma}
\label{lem:candy-same-junctions}
Let  and  be canonical diagrams for the same graph  with the same vertex ordering.
Then for every junction  of , there must be a junction  in  with the same funnel.
\end{lemma}

\begin{proof}
For a junction  of  with funnel intervals  and , let  be the separated funnel interval. We prove the lemma by induction on the cardinality of . The existence of the junction implies that  and  are edges of . Therefore, the trail in  between  and  must cross the trail between  and  in the canonical diagram. This can happen in three ways (see Fig.~\ref{fig:JunctionCases}): (1) the trails from  and  merge and then split towards  and , (2) the trails from  and  merge and then split towards  and , or (3) the trails cross inside a marked face  from which each vertex  and  is reached by a different junction of .

In case (1) the trails can travel together along a sequence . But because  and  are the funnel intervals for , there can be nothing entering or leaving  between the merge and split points, and  cannot contain a nonzero number of arcs without violating the condition on arcs without sharp angles at both ends. Therefore, the merge point of the trails from  and  is also the split point of the trails to  and , and forms a single junction  in  with the same funnel.

In the latter two cases,  and  are connected by a trail in , therefore they are adjacent in .
Hence there must exist a junction  in  with funnel intervals  and , where . If  contains less than two vertices besides  and , then this is not possible (base case). Otherwise, by induction, there must exist a corresponding junction  in  with funnel intervals  and .
This junction must lie along the trail between  and . By the properties of a funnel, neither  nor  can be connected to  or . This directly makes case (2) impossible, since such a connection is necessary in that configuration. In case (3) this implies that  must lie on the part of the trail between  and  that is along (or through) . This is not possible if the trail passes through , so  must be at the boundary of . But this makes the marked face  invalid, as it may have only sharp angles.
\end{proof}

\eenplaatje{JunctionCases} {The cases of Lemma~\ref{lem:candy-same-junctions}.}

\begin{lemma}
\label{lem:trail-thru-junction}
Let  be a junction of canonical diagram  with funnel , and let  be an
edge of the graph  represented by . Then the trail for edge  passes through  if and only if
exactly one of  and  is in the closed interval  and exactly one
of  and  is in the closed interval .
\end{lemma}

\begin{proof}
If the trail
passed through  but  or  was outside these intervals, it would be
reached from  by a more extreme path than one of , , , or ,
violating the definition of a funnel. If both are inside the same interval, but the trail between them goes through , then the diagram violates the requirement that trails be unique.
And if  and  are both inside
different intervals, then the path must either go through  or it must
cross the funnel twice, again causing a violation of the requirement that trails be unique.
\end{proof}

\begin{lemma}
\label{lem:trail-across-mark}
Let  be a canonical diagram in which two junctions  and  are consecutive on the trail from  to . Then this trail passes across a marked face from  to  if and only if there exist vertices  and , in the cyclic order , , , , such that , , , and  form a clique, and such that  and  are both in the same interval as  in the funnel for  and in the same interval as  in the funnel for .
\end{lemma}

\begin{proof}
Suppose first that the trail crosses a marked face. This face must have sharp angles on both sides of trail ; let  and  be any two vertices reachable from two angles on opposite sides.  Then there also exist trails from  and  to  and  that diverge from the trail from  to  within the marked face. The containment of  and  within the stated intervals follows from Lemma~\ref{lem:trail-thru-junction}.

In the other direction, suppose that the trail passes consecutively through  and  and that there exist  and  satisfying the conditions of the lemma. Since , , , and  form a clique, Lemma~\ref{lem:candy-clique} implies that this clique is represented either by a marked face with at least four sharp angles connecting to all four of these vertices, or by two marked faces or triangles connected by a sequence of junctions, arcs, and additional marked faces. However, because of the assumption in the lemma that  and  belonging to certain intervals, all of these faces must lie between  and ; the additional assumption that  and  are consecutive on the trail means that there is only one possibility, a single marked face with separate sharp angles connecting to the four vertices , , , and . Thus, as the lemma states, the trail crosses a marked face.
\end{proof}

\begin{lemma}
\label{lem:candy-same-arcs}
Let  and  be canonical diagrams for the same graph  with the same vertex ordering.
Then for every arc from junction  to ' of , there must be an arc in  connecting the corresponding junctions.
\end{lemma}

\begin{proof}
Let  be an edge of  whose trail in  uses the arc.
The trails in  and  from  to  both go through the same sets of junctions by Lemma~\ref{lem:trail-thru-junction}.
On the trail in  from  to , the junctions are monotonic in their ordering
by the size of the sets of reachable vertices in each direction, and the same is true in , so the trails go through the same junctions in the same ordering.
Therefore, in , the trail from  to  also goes from  to  with no intervening junctions.
The only way it can avoid using an arc that satisfies the lemma is for it to cross a marked face instead of an arc, but this would violate Lemma~\ref{lem:trail-across-mark} for either  or .
\end{proof}

\begin{theorem}
Every two canonical diagrams for the same graph  with the same vertex ordering are isomorphic.
\end{theorem}

\begin{proof}
This follows from Lemma~\ref{lem:candy-same-junctions} (showing that they have the same set of junctions) and Lemma~\ref{lem:candy-same-arcs} (showing that they have the same set of arcs).
\end{proof}

\end{document}
