\documentclass[article,10pt,twocolumn]{IEEEtran}
\usepackage{blindtext}
\usepackage[table]{xcolor}
\usepackage{tabularx}
\usepackage{comment}
\usepackage{multicol}
\usepackage{booktabs}
\usepackage{balance}
\usepackage{color, colortbl}
\newcommand{\ra}{\rand0.\arabic{rand}}
\usepackage[numbers, square, comma, sort&compress]{natbib}
\usepackage{nomencl}
\makeglossary
\usepackage{pdflscape}
\usepackage{soul}
\let\labelindent\relax
\usepackage{enumitem}
\usepackage{cancel}
\usepackage[english]{babel}
\usepackage{multicol}
\usepackage{multirow}
\usepackage{amsmath}
\usepackage{amsfonts}
\usepackage{amssymb}
\usepackage{float}
\usepackage{stfloats}
\usepackage{subfigure}
\usepackage[hyphens]{url}
\usepackage[hidelinks]{hyperref}
\hypersetup{breaklinks=true}
\usepackage{amsfonts}
\usepackage{pbox}
\usepackage{makecell}
\makenomenclature
\usepackage{epstopdf}
\usepackage[pdftex]{graphicx}
\usepackage[export]{adjustbox}
\usepackage{ccaption}
\definecolor{green}{HTML}{66FF66}
\definecolor{myGreen}{HTML}{009900}
\usepackage{tcolorbox}


\usepackage{tabularx}
\usepackage{array}
\usepackage{colortbl}
\tcbuselibrary{skins}

\newcolumntype{Y}{>{\raggedleft\arraybackslash}X}

\tcbset{tab1/.style={fonttitle=\bfseries\large, fontupper=\normalsize\small, colupper=black!40!black,
colback=white!10!white,colframe=black!75!black,colbacktitle=black!40!black,
coltitle=black,center title,freelance,frame code={
\foreach \n in {north east,north west,south east,south west}
;}}}

\tcbset{tab2/.style={fonttitle=\bfseries, colupper=black!40!black,
colback=white!10!white,colframe=black!50!black,colbacktitle=black!40!black,
coltitle=black,center title}}

\tcbset{tab3/.style={enhanced,fonttitle=\bfseries,fontupper=\normalsize\small,colupper=black!40!black,
colback=white!10!white,colframe=black!50!black,colbacktitle=black!30!black,
coltitle=black,center title}}

\tcbset{tab4/.style={fonttitle=\bfseries\small,fontupper=\normalsize\small,colupper=black!10!black,
colback=white!10!white,colframe=black!10!black,colbacktitle=white!10!white,
coltitle=black,center title}}

\tcbset{tab5/.style={fonttitle=\bfseries\small,fontupper=\normalsize\small,colupper=black!10!black,
colback=white!10!white,colframe=black!10!black,colbacktitle=white!10!white,
coltitle=black,center title}}

\begin{document}
\title{Separation Framework: An Enabler for Cooperative and D2D Communication for Future 5G Networks}

\author{\IEEEauthorblockN{Hafiz A. Mustafa, Muhammad A. Imran, Muhammad Z. Shakir, \\Ali Imran, and Rahim Tafazolli}\\
\IEEEauthorblockA{Institute for Communication Systems (ICS), University of Surrey, Guildford, Surrey,UK\\
Electrical and Computer Engineering Dept., Texas A\&M University, Doha, Qatar\\
School of Electrical and Computer Engineering, University of Oklahoma, Tulsa, USA\\
Email: \{h.mustafa, m.imran, r.tafazolli\}@surrey.ac.uk,\\ muhammad.shakir@qatar.tamu.edu, ali.imran@ou.edu
}
}

\maketitle

\begin{abstract}
Soaring capacity and coverage demands dictate that future cellular networks need to soon migrate towards ultra-dense networks. However, network densification comes with a host of challenges that include compromised energy efficiency, complex interference management, cumbersome mobility management, burdensome signaling overheads and higher backhaul costs. Interestingly, most of the problems, that beleaguer network densification, stem from legacy networks’ one common feature i.e., tight coupling between the control and data planes regardless of their degree of heterogeneity and cell density. Consequently, in wake of 5G, control and data planes separation architecture (SARC) has recently been conceived as a promising paradigm that has potential to address most of aforementioned challenges. In this article, we review various proposals that have been presented in literature so far to enable SARC. More specifically, we analyze how and to what degree various SARC proposals address the four main challenges in network densification namely: energy efficiency, system level capacity maximization, interference management and mobility management. We then focus on two salient features of future cellular networks that have not yet been adapted in legacy networks at wide scale and thus remain a hallmark of 5G, i.e., coordinated multipoint (CoMP), and device-to-device (D2D) communications. After providing necessary background on CoMP and D2D, we analyze how SARC can particularly act as a major enabler for CoMP and D2D in context of 5G. This article thus serves as both a tutorial as well as an up to date survey on SARC, CoMP and D2D. Most importantly, the article provides an extensive outlook of challenges and opportunities that lie at the crossroads of these three mutually entangled emerging technologies.
\end{abstract}

\begin{IEEEkeywords}
Separation Framework, Decoupled Architecture, Cooperative Communication, Energy Efficiency, Coordinated Multipoint, D2D Communication.
\end{IEEEkeywords}

\section{Introduction} \label{sec:introduction}
\IEEEPARstart{T}{raditional} cellular networks are designed with tight coupling of control and data planes. This architecture conforms to the main objective of ubiquitous coverage and spectrally efficient voice-oriented homogeneous services. The recent growth of data traffic overwhelmingly brought a paradigm shift from voice-traffic to data-traffic. Cisco made observations at internet service providers and predicted that the annual global Internet traffic will rise to 1.4 zettabyte by the year 2017 as compared to 528 exabyte (EB) in 2012 \citep{Cisco}. One of the contributors in this massive growth of Internet traffic is the proliferation of mobile devices and machine-to-machine (M2M) communication. Due to this growth, the capacity and coverage requirements exploded in recent years with worldwide mobile traffic forecast of more than 127 EB in the year 2020 \citep{UMTS_2011}. An increase of thousand-fold in wireless traffic is expected in 2020 as compared to 2010 figures \citep{6692781} with expected figure of 50 billion communication devices \citep{Ericsson_2011}. The explosive growth of mobile traffic is being handled by deploying tremendous amount of small cells resulting in heterogeneous network (HetNet)\nomenclature{HetNet}{Heterogeneous Network} \citep{6171992}.

The tight coupling of planes in conventional cellular networks leaves minimum control to consider networks' energy efficiency metric. This metric had a less concern previously due to less number of subscribers, rare data services, sparse deployments, and less awareness of green cellular communication. The green attribute of the cellular communication refers to reduction of unnecessary power consumption and its subsequent impact on the environment in the form of CO emissions \citep{5677351,wu_green_2012,scott_matthews_planning_2010,5978416,6848019,6525595}. The green cellular communication can be realized by bringing energy-awareness in the design, in the devices \citep{4205092} and in the protocols of communication networks. Due to the network scaling and heterogeneity (large number of small cell deployments), this metric became prominent. In this regard, it has been estimated that the energy consumption by the information and communications technology (ICT) results in 2\% of global carbon emissions \citep{6100924}.

Small cell deployment is an agile, cost-effective, and energy efficient solution to meet coverage and capacity requirements. However, large number of deployments (e.g., prediction of 36.8 million small cell shipments by year 2016 according to ABI research \citep{ABI}), the energy efficiency gain due to small cells might be compromised. Moreover, it also poses operational expenditure (OPEX) challenges to the network operators. This heterogeneity has also imbalanced the provision of data services between macro and small cells resulting in severe interference/backhaul-limited communication. In order to overcome the threatening issues of power consumption, the awareness of energy consumption has already been realized and a number of energy conservation techniques/approaches have been investigated in the literature.

Another core issue, rising in future ultra-dense HetNet, is the interference management. The main limiting factor in achieving the optimum capacity is intra/inter-cell interference. Although intra-cell interference, in present cellular networks, has been eliminated by using orthogonal frequency division multiple access (OFDMA)\nomenclature{OFDMA}{Orthogonal Frequency Division Multiple Access} technology and radio resource management (RRM)\nomenclature{RRM}{Radio Resource Management}, provision of underlay co-existing networks (e.g., device-to-device (D2D), M2M), in future ultra-dense environment will again cause intra-cell interference along with existing inter-cell interference. Current interference management techniques mainly comprise mitigation, cancellation, and coordination. The first two techniques are best suited to a single cell environment, whereas for multicell scenarios, coordination techniques comprising inter-cell interference coordination (ICIC)\nomenclature{ICIC}{Inter-cell Interference Coordination}, enhanced ICIC (eICIC)\nomenclature{eICIC}{enhanced Inter-cell Interference Coordination}, coordinated beamforming (CB)\nomenclature{CB}{Coordinated Beamforming}, and coordinated multipoint (CoMP)\nomenclature{CoMP}{Coordinated Multipoint} are more promising to provide homogeneous quality of service with small infrastructural changes over the area \citep{4117538}. The ICIC techniques were introduced to mitigate inter-cell interference for cell-edge users. The main idea is to use either different set of resource blocks (RBs)\nomenclature{RB}{Resource Block} throughout the cell or partition RBs for cell-centre and cell-edge users. In another scheme of ICIC techniques, this RB partitioning can be coupled with different power levels (e.g., power boost for cell-edge users and low power for cell-centre users) to mitigate inter-cell interference. The ICIC techniques have been enhanced to eICIC for HetNet in  generation partnership project (3GPP) \nomenclature{3GPP}{ Generation Partnership Project} Rel-10. These techniques, unlike ICIC, consider both control and traffic channels either in time, frequency or power domains to mitigate inter-cell interference. The main idea of eICIC is based on almost blank subframe (ABS)\nomenclature{ABS}{Almost Blank Subframe}. These blank subframes are reserved for different purposes for macro tier and  small cell tiers. The macro tier mostly uses these subframes for control channels with low power, whereas small cell tiers use them for traffic channels to serve cell-edge users.

The CB and CoMP fall into the category of interference exploitation as compared to interference avoidance schemes (e.g., ICIC, eICIC). In such techniques, joint scheduling, transmission, and processing are carried out to exploit inter-cell interference and enhance cell-edge performance. In CB, the user equipment (UE)\nomenclature{UE}{User Equipment} is served by a single base station (BS)\nomenclature{BS}{Base Station}, however, interference is coordinated between cooperating BSs. To enhance the data rate of individual UE, it can cooperatively be served by a number of BSs in CoMP, however, this approach requires sharing data between cooperating BSs which results in huge backhaul capacity requirements. As compared to interference avoidance, the exploiting techniques (e.g., multicell cooperation (CB and CoMP)) have been identified as a key solution in long term evolution (LTE)\nomenclature{LTE}{Long Term Evolution} and LTE-Advanced (LTE-A)\nomenclature{LTE-A}{Long Term Evolution-Advanced} to improve the cell-edge performance, average data rate, and spectral efficiency by mitigating and exploiting inter-cell interference \citep{3gpp.36.819, marsch_coordinated_2011, 4657145}.

The green aspects of future 5G cellular networks require energy efficient communication which can be realized effectively by completely switching-off under-utilized BSs. However, the switch-off mechanism has severe limitations in current cellular architecture due to coverage holes. In order to avoid coverage holes, one of the candidate solution is the new cellular architecture where control and data planes are separated, i.e., decoupled or separation architecture, to provide ubiquitous coverage and more localized high-rate data services. Another potential advantage in this architecture is the flexible mobility management due to reduced handover signaling. In present architecture, the mobile user is handed over to nearby BS even if there is no active data session. Since, control plane is coupled with data plane, it is mandatory to handover in-active mobile terminals to ensure coverage. This results in handover signaling which is required for coverage but not for data services. On the other hand, the mobile user with no active data session in decoupled architecture can move freely without initiating handover due to ubiquitous coverage. Huge potential savings can be realized in this case, due to reduced handover signaling resulting in energy efficient communication.

In order to realize thousand-fold capacity enhancements in future cellular networks, much higher bandwidth is required. This higher bandwidth is available in millimeter wave (mm-Wave) spectrum. The higher frequency has poor propagation characteristics, however, the corresponding spot-beam coverage is more feasible for low-range high-rate data services. Therefore, coverage at lower frequencies (with good propagation characteristics) and high-rate data services (with limited coverage) requires decoupled architecture. Another aspect that severely limits the system capacity is the ultra-dense cellular environment in future networks (due to more granular tiers in the form of D2D, and M2M overlay/underlay communication). The underlay system offers higher system capacity but causes intra-cell interference and therefore, interference management becomes more complex in this case. For such an environment, cooperation and coordination is the promising solution for interference management in decoupled architecture. 

Keeping in view the above vision, we structure the article in three sections. The first section introduces separation framework and provides survey of existing literature on separation architecture. Since energy efficiency is the key enabler for separation framework, we provide extensive literature review of existing approaches that realize energy efficient communication in current cellular architecture. This is followed by highlighting future requirements of cellular networks from the perspective of system capacity, interference management, and mobility management. We highlight several shortcomings due to coupled planes and provide motivation for separation architecture. The shortcomings in current architecture and potential gains due to decoupling are tabulated at the end of first section. The second section provides a brief tutorial on cooperative communication including underlay D2D cooperation. This section serves as a background to discuss cooperation in separation architecture. The third section presents different scenarios where cooperative communication can be realized in separation framework by highlighting potential advantages and associated complexities. The article is organized as follows. We provide list of acronyms in Table \ref{Table:tabnom}. Section II  provides system performance reviews of traditional and separation architecture. In Section III, the general context of cooperative communication, clustering, and D2D communication for traditional cellular system has been presented. Section IV describes different perspectives to extend cooperative communication to the separation framework. Section V concludes the survey and highlights future research directions in this area.
\begin{table}[t]
\renewcommand{\arraystretch}{1.35}
    \centering
    \caption{List of Acronyms}\label{Table:tabnom}
    \vspace{2mm}
    \begin{tabular}{c||lc}
    	\hline
    	\textbf{Acronym}	&	\textbf{Definition}\\
    	\hline
    	3GPP	&	 Generation Partnership Project\\
    	\hline
     	ABRB	&	Almost Blank Resource Block\\
    	\hline
	ABS	&	Almost Blank Subframe\\
    	\hline
	BBU	&	Base Band Unit\\
    	\hline
	BS	&	Base Station\\
    	\hline
	C-RAN&	Cloud Radio Access Network\\
    	\hline
	CARC	&	Conventional Architecture\\
    	\hline
	CB	&	Coordinated Beamforming\\
    	\hline
	cBS	&	Control BS\\
    	\hline
	CCU	&	CoMP Central Unit\\
    	\hline
	CoMP	&	Coordinated Multipoint\\
    	\hline
	CRC	&	Cyclic Redundancy Check\\
    	\hline
	CRS	&	Common Reference Signal\\
    	\hline
	CSI	&	Channel State Information\\
    	\hline
	CSI-RS&	Channel State Information Reference Signal\\
    	\hline
	CU	&	Central Unit\\
    	\hline
	dBS	&	Data BS\\
    	\hline
	eICIC	&	enhanced Inter-cell Interference Coordination\\
    	\hline
	eLA	&	enhanced Local Area\\
    	\hline
	eNB	&	evolved NodeB\\
    	\hline
	FDD	&	Frequency Division Duplex\\
    	\hline
	HARQ	&	Hybrid Automatic Repeat Request\\
    	\hline
	HetNet&	Heterogeneous Network\\
	\hline
	ICIC	&	Inter-cell Interference Coordination\\
    	\hline
	IMT-Advanced	&	International Mobile Telecommunications-Advanced\\
    	\hline
	ISM	&	Industrial, Scientific, and Medical\\
    	\hline
	JD	&	Joint Detection\\
    	\hline
	JT	&	Joint Transmission\\
    	\hline
	LTE	&	Long Term Evolution\\
    	\hline
	LTE-A	&	Long Term Evolution-Advanced\\
    	\hline
	MIMO	&	Multiple Input Multiple Output\\
    	\hline
	OFDMA&	Orthogonal Frequency Division Multiple Access\\
    	\hline
	PID	&	Physical Cell Identification\\
    	\hline
	PSS	&	Primary Synchronization Signal\\
    	\hline
	RAN	&	Radio Access Network\\
    	\hline
	RB	&	Resource Block\\
    	\hline
	RRH	&	Remote Radio Head\\
    	\hline
	RRM	&	Radio Resource Management\\
    	\hline
	RSRP	&	Reference Signal Received Power\\
    	\hline
	SARC	&	Separation Architecture\\
    	\hline
	SC-RAN&	SARC in C-RAN\\
    	\hline
	SON	&	Self-organizing Network\\
    	\hline
	SSS	&	Secondary Synchronization Signal\\
    	\hline
	TDD	&	Time Division Duplex\\
    	\hline
	UE	&	User Equipment\\
    	\hline
	UMTS & 	Universal Mobile Telecommunications System\\
	\hline
    \end{tabular}
\vspace*{-\baselineskip}
\end{table}
\section{Separation Framework: Performance Measures and Potential Gains}\label{sec:EE}
The current cellular networks comprise tightly coupled control and data planes in the same radio access network (RAN)\nomenclature{RAN}{Radio Access Network}. This architecture meets the main objective of ubiquitous coverage and spectral efficiency for voice services in homogeneous deployments. The massive growth of data traffic overwhelmingly dominated the voice traffic resulting into a paradigm shift from homogeneity to heterogeneity and voice services to data services. The traditional architecture (designed for homogeneous voice services) meets the current requirements of ubiquitous coverage and high spectral efficiency, however, it provides these services by overlooking signaling overheads, backhaul cost, and energy efficiency of the system. In order to enhance the coverage and capacity of current cellular systems, it is common practice to deploy small cells for peak-load scenarios at the cost of reduced energy efficiency, increased overhead signaling (e.g., in terms of frequent handovers) and increased backhaul requirements. In order to mitigate the rising concerns of power consumption, number of solutions, based on dynamic BS switching mechanism, are suggested to exploit the temporal and spatial variations in traffic load. However, the tight coupling of user and control planes restricts the flexibility and leaves less degree of freedom to optimize the system performance (discussed in subsequent discussions). To this end, the idea of control and data planes separation was proposed by the project beyond green cellular generation (BCG2) of GreenTouch consortium in Jan., 2011 \citep{greentouch}. Similar approaches have been suggested in study group of 3GPP on ``New Carrier Type". The Mobile and wireless communications Enablers for Twenty-twenty Information Society (METIS) \citep{METIS} aims to lay the foundation of 5G where control and data plane separation is being considered as a candidate system architecture. The green 5G mobile networks (5grEEn)  is focusing on green aspects of future 5G networks by considering separation of control and data planes. The joint European Union - Japan project Millimeter-Wave Evolution for Backhaul and Access (MiWEBA) is investigating the use of separated control and data planes for mm-Wave based small cells \citep{MiWEBA}.

In order to highlight potential gains due to decoupling of control and data planes, we present conventional architecture (CARC)\nomenclature{CARC}{Conventional Architecture} and futuristic separation architecture (SARC)\nomenclature{SARC}{Separation Architecture} in Fig. \ref{Figure:carc_sarc}.
\begin{figure*}[!htb]
\centering
    \subfigure[Conventional Architecture (CARC)]
    {
        \includegraphics[width = 3.25 in]{CARC.eps}
        \label{Figure:carc}
    }
      \subfigure[Separation Architecture (SARC)]
    {
        \includegraphics[width = 3.25 in]{SARC.eps}
        \label{Figure:sarc}
    }
 \caption{Conventional and control/data plane separation architecture.}
   \vspace{-3mm}
\label{Figure:carc_sarc}
\end{figure*}
As shown in Fig. \ref{Figure:carc}, CARC is a conventional HetNet (comprises macrocell and large number of small cells) where coverage and data services are simultaneously provided at same frequency either by macro or small cell on coupled control and data planes. The advantage of this approach is ubiquitous coverage, however, the serving cell cannot sleep and it has to provide coverage even at low load conditions resulting in under-utilization of resources. The mobile users, irrespective of active or in-active sessions, are always covered by dedicated channels (ubiquitous coverage). However, it results in under-utilization of data plane (since it is coupled with control plane). In Fig. \ref{Figure:sarc}, SARC is a hierarchical HetNet comprising conventional HetNet and an additional tier of D2D/M2M communication, where control and data planes are decoupled. In such an architecture, the ubiquitous coverage and low-rate data services\footnote{The control BS has ubiquitous coverage over a large area as compared to small cell coverage area. Hence, it is more feasible to provide data services to high mobility users by cBS to avoid signaling overhead and frequent handovers in small cells.} are provided by control BS (cBS)\nomenclature{cBS}{Control BS} at lower frequency bands with good channel characteristics. The data services are provided on demand at higher frequency bands by short range high-rate data BSs (dBSs)\nomenclature{dBS}{Data BS}. The advantages of this architecture are ubiquitous coverage (by decoupled control plane for active or in-active users), small cell sleeping possibility without coverage holes, temporal and spatial traffic adaptation, and high-rate data services for active users without compromising the energy efficiency of the system. The reader is referred to \citep{6468982} for feasibility study of detached cells from the perspective of reliability and energy savings.

The control plane is responsible for system configuration and management. It provides system information, synchronization, and reference signals etc. The system information is broadcast and it mainly comprises the information required to join the network. The synchronization information includes frame timings as well as symbol level timings. The reference signals are used to know channel state which is indispensable for scheduling and resource allocation. In contrast to this, data plane is responsible to provide the requested contents along with some acknowledging mechanism (e.g., hybrid automatic repeat request (HARQ)\nomenclature{HARQ}{Hybrid Automatic Repeat Request}). In order to give insights into information exchanges in both the planes, we provide a case study of LTE/LTE-A networks in Table \ref{Table:tab2}.
\begin{table*}[!htb]
    \begin{center}
    \caption{Control data plane information exchange in LTE/LTE-A.}\label{Table:tab2}
    \vspace{2mm}
\begin{tcolorbox}[tab2,tabularx={p{1.2in}||X|c||c}]
\bf{Signals} & \bf{Information Exchange} & \bf{Direction} & \bf{Plane}  \\ \hline

Physical random access channel (PRACH) & Initial synchronization with eNode B (eNB). & \multirow{7}{*}{Uplink}& \multirow{6}{*}{Control}\\ \cline{1-2}

Reference Signals (RS) & Demodulation RS (DRS) - Channel estimation for coherent demodulation, Sounding RS (SRS) - Channel quality estimation over a span of bandwidth. &  &  \\ \cline{1-2}

Physical uplink control channel (PUCCH) & (HARQ ACK/NACK)*, channel quality indicator (CQI), precoding matrix indicator (PMI), rank indicator (RI), scheduling requests. & & \\ \cline{1-2} \cline{4-4}

Physical uplink shared channel (PUSCH) & User uplink data. & & Data \\ \hline

Synchronization & Primary and secondary synchronization (PSS, SSS) for cell identity and frame timing. & \multirow{8}{*}{Downlink} & \multirow{7}{*}{Control} \\ \cline{1-2}

Reference Signals (RS) & Channel state information (CSI-RS), demodulation (DM-RS), cell-specific (CRS), positioning (PRS). &  & \\ \cline{1-2}

Control Indicators & Physical control format indicator channel (PCFICH) to indicate size of PDCCH, physical HARQ indicator channel (PHICH) to ACK/NACK user data on PUSCH. & & \\ \cline{1-2}

Multicast/Broadcast & Physical broadcast channel (PBCH) carrying master information block (MIB), multicast/broadcast single frequency network (MBSFN), multicast channel (PMCH). & & \\ \cline{1-2} \cline{4-4}
 
Physical downlink shared channel (PDSCH) & User multiplexed data. & & Data \\ \hline

\multicolumn{4}{l}{\multirow{1}{*}{* HARQ is sent either as a feedback message on control channel or piggybacking feedback on user's data plane.}} \\ [0.25ex]
\end{tcolorbox}
\vspace{-6mm}
  \end{center}
\end{table*}

The SARC for HetNet offers many potential gains such as energy efficiency, capacity enhancement, reduced overhead signaling, flexible interference and mobility management. Control signaling is provided by cBS, however, certain types of control signaling cannot be fully decoupled. For example, frame/symbol level synchronization and channel state information (CSI)\nomenclature{CSI}{Channel State Information} is required in both planes.

The separation of planes for future cellular networks has been realized very recently. To this end, the control and data plane separation has been suggested in \citep{6152217, 6515050}, where the provision of coverage has been provided by a long range low rate control evolved Node B (eNB)\nomenclature{eNB}{evolved NodeB}. The data services, on the other hand, are provided by dedicated data eNBs. In \citep{6152217}, it is proposed that signaling will provide wider coverage to all UEs regardless of active or in-active data session under data eNB. Such network-wide adaptation provides flexibility to power down certain BSs when no data transmission is needed. In simple strategy of powering down the dBSs, neither control signaling (e.g., synchronization, reference signals, system information etc) nor associated backhaul to the access network is required; no data services are requested by UEs, only coverage is required which is ubiquitously provided by the cBS. The powering down strategy can, therefore, save approximately 80\% of RAN power per BS switch-off \citep{6152217, 6056691,4448824} besides power savings due to backhaul communication links. Therefore, separation of planes promises tremendous increase in energy efficiency, reduced overhead signaling, and relaxed backhaul requirements. In \citep{6152217}, the energy efficiency gain has been emphasized by considering system level approach where under-utilized BSs are realized in sleep mode. In this study, no expected gains in energy efficiency are highlighted. Certain technical challenges including context awareness, resource management, and radio technologies for the signaling network are highlighted without proposing any design guidelines for the separation architecture.

The design of the signaling network in SARC is more challenging as compared to conventional approach. In CARC, the BSs usually do not sleep due to the possibility of coverage holes. Therefore, all BSs are active and no wake-up signaling is required. The handover procedure is usually UE driven based on reference signal received power (RSRP)\nomenclature{RSRP}{Reference Signal Received Power} values. In contrast, data services in SARC, in case of sleeping dBS, can be ensured by (i) optimal dBS selection from sleeping dBSs, and (ii) initiation of wake-up mechanisms. The optimal dBS selection can be quite challenging since cBS has no instantaneous knowledge of channel conditions. This results in more complex signaling procedures as compared to CARC. The new design is required to be robust and energy efficient. Use of low frequencies provides better propagation and obstacle penetration. Moreover, mobility management is flexible in HetNet using SARC architecture. This is because, control plane handover is rarely required since the coverage area of cBS is large as compared to the coverage area of BSs in conventional system. The data plane handover is only required in case of active data requests and in case of in-active users, none of the handovers (control plane or data plane) are required. This has been discussed in more details in Sec. \ref{MM:ho_proc}.

In \citep{6515050}, a two-layer network functionality separation scheme, targeting low control signaling overhead and flexible network reconfiguration for future green networks has been proposed. A frame structure level detail has been proposed in which network functionality including synchronization, system information broadcast, paging, and multicast (synchronization, pilot, frame control, and system/paging/multicast information bearer signals) is incorporated in control network layer (CNL). Whereas, the network functionality of synchronization and unicast (synchronization, pilot, frame control, and unicast information bearer signals) is incorporated in data network layer (DNL). In this study, the main focus is given on advantages of low control signaling overhead. The network area power consumption has been plotted for two architectures showing significant potential gain for separation architecture leading towards future energy efficient green mobile networks. Unlike \citep{6152217}, the authors in \citep{6515050} proposed abstract level network design for control and data planes separation. The categorization of different wireless signals and their mapping relationship with physical channels are presented. However, the challenges highlighted in \citep{6152217} are not discussed in \citep{6515050}. The study also lacks in addressing interference management issues, backhaul requirement, realization of underlay networks (e.g., D2D), mobility management and corresponding handover procedures in separation architecture.

The important focus areas for energy efficient 5G mobile network are highlighted in \citep{6673363}. These areas include system architecture with decoupled control and data planes, ultra-dense HetNet deployment, radio transmission using multiple input multiple output (MIMO)\nomenclature{MIMO}{Multiple Input Multiple Output} configuration and energy efficient backhaul. The transmission planes are categorized into data, control, and management planes. It is emphasized that if these planes are decoupled from each other then independent scaling is possible at most energy efficient locations. Furthermore, the logical separation of control and data planes can provide most efficient discontinuous transmission/reception (DTX/DRX) functionality to save energy in idle modes. Similar to \citep{6152217}, the authors in \citep{6673363} highlighted the requirements and technical challenges to realize future green 5G mobile network. However, the system architecture and radio transmissions design guidelines are not outlined in details as in \citep{6515050}. The solutions to these important areas are considered as deliverables of 5GrEEn.

In \citep{zhao2013software}, hyper-cellular network is introduced as decoupled control and traffic network to realize energy efficient operation of BS. In such a network, data cells are flexible to adapt traffic variations and network dynamics while control cells can flexibly and globally be optimized. The hyper-cellular network is considered as a novel architecture for future mobile communication systems. The approach realizes control and data planes separation using open source radio peripherals and legacy global system for mobile (GSM) network. In this testing, signaling BS provided coverage whereas data BS ensured phone call connectivity. A very promising formulation has been setup by using open base transceiver station (OpenBTS), universal software radio peripheral (USRP) front end, wide bandwidth transceiver (WBX) daughter board, and dell PCs. This formulation provides an insight into real-time practical setup for prototype testing. However, system improvements are not shown in this paper. Moreover, none of the performance metrics (energy efficiency, backhaul relaxation, and throughput) have been analyzed and validated for this simple and basic approach.
\begin{table*}[!htb]
\makeatletter
\newcommand*{\compress}{\@minipagetrue}
\makeatother
\renewcommand{\arraystretch}{1.5}
\caption{Summary of approaches for control and data planes separation.}\label{Table:Appr_CDplane}
\vspace{2mm}
\begin{tcolorbox}[tab1,tabularx={>{\raggedright\arraybackslash}p{1.1in}||>{\raggedright\arraybackslash}p{1in}|X|>{\raggedright\arraybackslash}p{1.45in}}]
\textbf{Project/Paper/Ref.}			&\textbf{Aim}					&\textbf{Working/Highlights} 														& \textbf{Impacts/Conclusion}					\\ \hline\hline
``\textit{Looking beyond green cellular networks}" \citep{6152217}			
					&  {To switch-off BSs flexibly in case of no data transmission}\vspace*{-\baselineskip}
												& \compress \begin{itemize}[leftmargin=1.25em]
													\renewcommand{\labelitemi}{}
													\item Coverage  by long-range low-rate control eNB 
													\item Data Service  by short-range high-rate data eNBs.
													\item Ubiquitous coverage by signaling plane.
													\vspace*{-\baselineskip}
												\end{itemize}																		&\compress\begin{itemize}[leftmargin=0.75em]
																																	\item Selection and activation of BS is not a difficult task compared to optimizing the decision process.
																																	\vspace*{-\baselineskip}
																																\end{itemize} 						\\ \hline
``\textit{On Functionality Separation for Green Mobile Networks: Concept Study over LTE}" \citep{6515050}
					& To reduce control signaling overhead \& realize flexible network reconfiguration
												&\compress\begin{itemize}[leftmargin=1.25em]
													\renewcommand{\labelitemi}{}
													\item Separation scheme based on two-layer network functionality: CNL/DNL
													\item CNL  multicast information bearer signals
													\item DNL  unicast information bearer signals
													\vspace*{-\baselineskip}
												\end{itemize}														
																																& \compress\begin{itemize}[leftmargin=0.75em]
																																	\item Rare Handover in CNL 
																																	\item Call re-establishment is not required 
																																	\item HO signaling is reduced significantly
																																	\vspace*{-\baselineskip}
																																\end{itemize}						\\ \hline 
``\textit{5GrEEn: Towards Green 5G mobile networks}" \citep{6673363}
					& To provide general outlook on system architecture for energy efficient 5G network
												&\compress\begin{itemize}[leftmargin=1.25em]
													\renewcommand{\labelitemi}{}
													\item Ultra-dense HetNet deployment
													\item Radio transmission using MIMO configuration 
													\item Energy efficient backhaul
													\item Transmission planes: data, control, and management
													\vspace*{-\baselineskip}
												\end{itemize}														
																																& \compress\begin{itemize}[leftmargin=0.75em]
																																	\item Separation of control and data plane provides most effective DTX/DRX functionality to save energy in idle modes
																																	\vspace*{-\baselineskip}
																																\end{itemize}						\\ \hline 
``\textit{Software defined radio implementation of signaling splitting in hyper-cellular network}" \citep{zhao2013software}
					& Energy efficient operation of BS
												&\compress\begin{itemize}[leftmargin=1.25em]
													\renewcommand{\labelitemi}{}
													\item Hyper cellular network: Decoupled signaling and data services
													\item Handset is provided coverage by signaling BS 
													\item Phone calls are connected with the help of data BS
													\vspace*{-\baselineskip}
												\end{itemize}														
																																& \compress\begin{itemize}[leftmargin=0.75em]
																																	\item Provides an insight into real-time practical setup for prototype testing
																																	\vspace*{-\baselineskip}
																																\end{itemize}						\\ \hline
``\textit{FP7 project CROWD}" \citep{6702534}
					& Energy optimized connectivity management
												&\compress\begin{itemize}[leftmargin=1.25em]
													\renewcommand{\labelitemi}{}
													\item SDN based MAC control and mobility management
													\vspace*{-\baselineskip}
												\end{itemize}														
																																& \compress\begin{itemize}[leftmargin=0.75em]
																																	\item Complements huge deployments of cellular nodes
																																	\vspace*{-\baselineskip}
																																\end{itemize}						\\ \hline
``\textit{Dual connectivity in LTE HetNets with split control- and user-plane}" \citep{6825019}
					& Dual connectivity and use of CSI-RSs for CSI measurements
												&\compress\begin{itemize}[leftmargin=1.25em]
													\renewcommand{\labelitemi}{}
													\item Different MA small cells are considered as different antennae of MIMO/CoMP array
													\vspace*{-\baselineskip}
												\end{itemize}														
																																& \compress\begin{itemize}[leftmargin=0.75em]
																																	\item CSI-RS is also used to estimate the downlink path loss for uplink power control
																																	\vspace*{-\baselineskip}
																																\end{itemize}						\\ \hline
``\textit{A novel architecture for LTE-B: C-plane/U-plane split and Phantom Cell concept}" \citep{6477646,6554746}
					& To provide high data rate to UE through spatial reuse of spectrum
												&\compress\begin{itemize}[leftmargin=1.25em]
													\renewcommand{\labelitemi}{}
													\item Phantom Cell architecture: high frequency band solution with decoupled control/data plane
													\item Macro cell controls the small cells for connection establishment 
													\item Small cells use high frequency bands to provide high-rate data coverage
													\vspace*{-\baselineskip}
												\end{itemize}														
																																& \compress\begin{itemize}[leftmargin=0.75em]
																																	\item Outperforms conventional small cell architecture in both spectral and energy efficiency metrics
																																	\vspace*{-\baselineskip}
																																\end{itemize}						\\  \hline\hline
\end{tcolorbox}
\vspace{-4mm}
\end{table*}

The control and data planes separation concept has been presented from the perspective of energy optimized connectivity management in seventh framework programme (FP7) CROWD \citep{6702534}. To this end, software defined networking (SDN) based medium access control (MAC) and mobility management has been proposed to complement huge deployments of cellular nodes. Two key challenges, interference and mobility management, are considered for next generation dense wireless mobile networks. The functional architecture has been proposed and several key control applications are identified. More focus is given on mobility management and an SDN-based distributed mobility management (DMM) approach has been suggested. The control applications for interference management range from existing multi-tier scheduling scheme (e.g., eICIC) to LTE access selection schemes. The radio transmission aspects and backhaul limitations have not been outlined in any of the control applications identified in this study.

In \citep{6825019}, the authors measure CSI by using the concept of dual connectivity (using macrocell assisted small cells) and proposing the use of CSI reference signals (CSI-RSs)\nomenclature{CSI-RS}{Channel State Information Reference Signal} instead of common reference signal (CRS) \nomenclature{CRS}{Common Reference Signal}. Since, CSI-RSs are traditionally used by UEs to differentiate between different antennas of a MIMO system, therefore in the proposed network layout, different macrocell assisted small cells are considered as different antennae of MIMO/CoMP array. This strategy results in energy efficient operation (by reducing number of CRS) and provides network-triggered handover (unlike UE-triggered handover in CARC) to realize flexible and enhanced mobility management. Due to the absence of CRS for macrocell assisted small cells, the authors proposed to use CSI-RSs to estimate the downlink path loss for uplink power control. Similar to the previous approaches, the authors in \citep{6825019} focuses only on reducing control signaling to realize energy efficient operation without emphasizing context awareness, radio frame structure, backhaul issues, and interference management.

The 3GPP is presently standardizing enhanced local area (eLA)\nomenclature{eLA}{enhanced Local Area} small cell HetNet (LTE Rel-12) to provide high data rate to UEs through spatial reuse of the spectrum. In \citep{6477646}, a particular eLA architecture called Phantom Cell is proposed by NTT DOCOMO. This architecture is based on control and data planes separation; suggested as a novel architecture for LTE-B. The approach in \citep{6477646} suggests deployment of massive small cells by leveraging high frequency reuse under the coverage of macrocell to achieve high capacity, seamless mobility, and scalability. The two tier configuration is realized as a master-slave configuration where macrocell controls the small cells dynamically for connection establishment and small cells use high frequency bands to provide high-rate data coverage. This high frequency band solution with decoupled control and data planes, where small cells do not transmit cell-specific reference signals, is introduced as Phantom Cell architecture. In order to evaluate the energy efficiency performance of the Phantom Cell architecture, the stochastic geometry is used to compare the results with the conventional frequency division duplex (FDD)\nomenclature{FDD}{Frequency Division Duplex} based LTE picocell deployment in \citep{6554746}. The numerical results indicate that the Phantom Cell architecture outperforms conventional small cell architecture in both spectral and energy efficiency metrics. The authors in \citep{6477646} provide preliminary results for capacity enhancements in separation architecture without considering energy efficiency aspects, whereas \citep{6554746} provides more rigorous analysis for both spectral and energy efficiency of separation architecture. Some interesting conclusions are made about higher spectral efficiency and higher energy efficiency, however, both these studies focused on spectral and energy efficiency metric and did not include other aspects such as context awareness, signaling network, and functional description of the separation architecture. The reader is referred to \citep{7067574, 6848637} for Phantom cell operation at super high and extremely high frequency and related technical issues such as larger path loss in small cell, human body shadowing, massive MIMO architecture, and precoding algorithms to achieve super high data rates. The comparative summary of different approaches for control and data planes separation is presented in Table \ref{Table:Appr_CDplane}.

In the following subsections, we provide motivation for control and data planes separation architecture. In this context, we consider several key performance measures and analyze them in existing architecture. We provide survey of existing approaches, highlight the shortcomings and discuss these measures from the perspective of SARC architecture.
\subsection{Energy Efficiency}\label{ee}
The energy efficiency of RAN mainly depends on power consumption of BS. According to energy aware radio and network technologies (EARTH) project \citep{6056691}, the BS power consumption model comprises power consumed by radio frequency chain (especially power amplifier), signal processing units, and supply units (mains supply, DC-DC, and active cooling) as follows:


In order to ensure energy efficient communication, one simple strategy can be adopted where under-utilized BS, in case of low traffic conditions, should go to sleep mode (hence reducing power consumption  and ). This situation, however, causes coverage holes due to tight coupling of control and data planes unlike futuristic architecture where coverage and data services will be decoupled to provide ubiquitous coverage and on-demand data services.

The power consumption had not been a problem in past due to homogeneous networks and sparse deployments. Therefore, energy efficiency metric had not been considered while designing such cellular networks. Due to technology scaling and proliferation of large number of smart devices, the capacity demands increased tremendously with more energy consumption worldwide. This huge increase in capacity was predicted by wireless world research forum (WWRF) more than a decade ago. The key technological vision from WWRF expected around 7 trillion wireless devices serving 7 billion people by 2017. Moreover, it was predicted that approximately 80-95\% subscribers will be mobile broadband users \citep{wwrf2009,tafazolli2006technologies}. The huge increase in number of subscribers motivated the network operators to deploy small cells in order to quickly meet the customer needs. According to ABI research, by 2016, small cells will cover up to 25\% of all mobile traffic and small cells shipments (both indoor and outdoor) will likely to reach 36.8 million units worth \_2_2_2_2^2\Rightarrow\Rightarrow\Rightarrow\Rightarrow\Rightarrow\Rightarrow\Rightarrow\Rightarrow\Rightarrow\Rightarrow\Rightarrow\Rightarrow\textbf{h}_{k}\textbf{g}_{k,x}x \in \{l,m\}\hat{\textbf{h}}_{k}[n]\hat{\textbf{g}}_{k,x}[n]D_{k}D_{k,x}D_{bh} = D_{k} + D_{k,x}D_{k,x} \geq D_{k}\textbf{e}_{h_{k}}[n]\textbf{e}_{g_{k,x}}[n]\mathcal{CN}(0,1)\eta_{k}\eta_{k,x}b_{0}J_{0}(.)f_{d}T_{s}D_{k,x}D_{k,x}xD_{x}\eta_{k,x} = \eta_{k}D_{k,x} = D_kNd(.)U_1U_2U_3U_4U_5U_6U_7U_7U_1U_6U_2U_5U_7U_1U_6U_2U_4U_5U_1U_6U_2U_4U_5U_3$ has highest influence which make it suitable for content dissemination in clustered mode of D2D communication.
\subsubsection{Prediction based adaptive D2D clustering}
As mentioned in Sec. \ref{sec:comp_clustering}, the clusters can be static or dynamic where the latter offers more gains as compared to former. The dynamic clustering and cooperation framework is suitable for nomadic users \citep{marsch_coordinated_2011}. Since D2D communication is being evolved for proximity services and inter-networking, dynamic clustering and cooperation framework is very feasible for such type of communication. The dynamic clustering can be extended into self-organized adaptive clustering if the user mobility is predicted. For example, by predicting dwell times of potential D2D users at serving dBS, the required signaling for D2D clustering may be performed in a self-organized manner. Another advantage of this approach is that the prediction of dwell times may allow to tackle ping pong effects and reduce handover cost for switching between cellular and D2D modes. The adaptive clusters can further be optimized by considering mobility patterns along with reduced path-loss, common contents and channel condition criterion.
\subsection{D2D CoMP}\label{sec:d2d_comp}
In previous sub-section, we have presented two modes of D2D communication i.e., ad-hoc and clustered (Fig. \ref{Figure:distancebasedd2d}). In both cases, cooperation framework for multicell BS i.e., CB and CoMP can be realized in SARC for D2D communication. This type of cooperation coupled with common information exchange (ad-hoc mode) or content dissemination (clustered mode) is introduced as D2D CoMP. Since cBS has global context of every node in the coverage area, it can discover nodes for either ad-hoc or clustered mode communication e.g., by localizing nodes and applying shortest distance/reduced path-loss criterion. 

In order to get CSI between cooperating and requesting nodes, cBS can send a reference signal and request a CSI feedback. Based on RSRP values, one of the node in cooperation cluster may send CSI directly to the cBS. The cBS can use this CSI to design beamformers and share with nodes in cooperation set for proactively cached common information exchange or content dissemination. D2D CoMP in SARC is shown in Fig. \ref{Figure:d2dCoMP}.
\begin{figure}[!htb]
\centering
\includegraphics[width = 0.96\columnwidth, height = 1.75in]{D2D_CoMP.eps}
\caption{D2D CoMP to manage interference in underlay network.}\label{Figure:d2dCoMP}
\vspace{-4mm}
\end{figure}

In this figure, D2D cooperation regions are shown for ad-hoc and clustered mode D2D CoMP operation. In case of ad-hoc mode, cBS needs to localize and discover an influential partner node\footnote{An influential node can be identified by utilizing the history/context of different nodes and assigning some weight based on the activity of the node e.g., time duration of active sessions, file upload/download frequency etc.} with shortest distance (reduced path-loss) criterion. Once an influential node (containing common information) is identified within proximity of requesting node, cBS can command influential node to send reference signal and subsequently request CSI feedback from the requesting D2D node. For example, in a simple scenario, zero-forcing (ZF) or minimum mean-square-error (MMSE) \citep{6849319} can be used to design precoder to realize CB for ad-hoc mode D2D communication.
 
In case of clustered mode D2D communication, cBS needs to localize a set of influential nodes (known as cooperating nodes in traditional CoMP) that can make cooperation cluster for content dissemination. At this stage, cBS needs to know CSI between requesting and influential nodes. Similar to the ad-hoc mode, cBS can command influential nodes to send reference signal and subsequently request CSI feedback from the requesting node. However, CSI acquisition is more complex as compared to ad-hoc mode due to higher number of distributed influential nodes. Here, we present one strategy to acquire CSI at cBS. In this strategy, cBS will schedule different time slots in a time division multiple access (TDMA) fashion and allocate these slots to the influential nodes. Meanwhile, cBS will command requesting node to acquire time division multiplexed (TDM) reference signals, measure CSI and feedback to the cBS. Once CSI is acquired by the cBS, ZF or MMSE, as mentioned for ad-hoc mode, can be used to design precoders at cBS and shared with influential nodes. The D2D CoMP has potential gains to mitigate interference, however, it comes with the additional cost of higher signaling for CSI acquisition.
\subsection{SARC in Cloud-RAN}\label{sec:cran}
The realization of control and data planes separation has been discussed briefly in \citep{6825019, R1-130566, 6704656} through Carrier Aggregation (CA) and multiple remote radio head (RRH)\nomenclature{RRH}{Remote Radio Head}. Similarly, in \citep{7047300}, the integration of software-defined RAN (SD-RAN) and BCG2 architecture (i.e., decoupled control and data planes) has been suggested to achieve greater benefits and faster realization of both technologies. Motivated by such studies, we present arguments to support SARC in existing cloud RAN (C-RAN)\nomenclature{C-RAN}{Cloud Radio Access Network} architecture. The C-RAN solution comes into two types \citep{CRAN}. The first one is fully centralized where RRH provides radio function and the baseband functions (layer 1, layer 2, etc) are provided by the base band unit (BBU)\nomenclature{BBU}{Base Band Unit}. The second is partially centralized where layer 1 functionality of baseband function is integrated into the RRH. Both C-RAN solutions comprise RRH, the radio function and antennas (located at remote sites as close to the UEs as possible), mobile fronthaul, the fiber link between RRH and BBUs (which can be distributed or centralized at the central office (CO)). In order to realize SARC in C-RAN (SC-RAN)\nomenclature{SC-RAN}{SARC in C-RAN}, some RRHs can be deployed at cBS for ubiquitous coverage and the remaining RRHs for data services. The proposed SC-RAN is shown in Fig. \ref{Figure:scran}.
\begin{figure*}[!htb]
\centering
  \includegraphics[width = 0.85\textwidth, height = 1.75in]{SC-RAN.eps}
  \caption{Split C-RAN Architecture}\label{Figure:scran}
\vspace{-5mm}
\end{figure*}

In this figure, SC-RAN is equivalent to traditional C-RAN with decoupled control and data planes. The BBU stack in CO brings flexibility in C-RAN for joint management of resources and the co-existence of control and data BBUs in SC-RAN can extend this flexibility to share signaling, channel conditions (e.g., CSI), and user data. This results into higher potential to perform joint signal processing e.g., CB and CoMP \citep{CRAN}. The adaptive clustering is more manageable in centralized BBUs in SC-RAN due to global control of the coverage area (cBS BBU). The notion of cell-sleeping can be realized and load balancing, mobility management, and interference management can be accomplished more flexibly with reduced OPEX and higher energy efficiency resulting into future green cellular networks.

The flexibility of realizing SC-RAN comes with the expensive requirement of fronthaul/backhaul links. Since, huge information needs to be exchanged between cooperating dBSs in case of CoMP, high capacity fronthaul/backhaul links are required. In order to address the problem of high capacity backhaul requirements, the distributed caching of contents in femtocells has been proposed in \citep{6195469, 6495773}. These approaches use high storage capacity at femto BS to cache most popular contents and harnessing D2D communication for content delivery. Recently, the backhaul problem in CoMP has been addressed using cache-enabled relays and BSs \citep{6601672, 6665021, 6948325}. All these approaches are based on cache-enabled opportunistic cooperative MIMO (CoMP) framework where a portion of contents are cached at cooperating set of relays or BSs to relax backhaul capacity requirements. Such approaches may be used in SC-RAN, where partially centralized C-RAN (with layer 1 functionality integrated into RRH) can be incorporated so that cache-enabled dBSs can provide high-rate data services in CoMP fashion without requiring huge capacity requirements.
\section{Conclusion}\label{conclusion}
In this article, we outline several performance measures to highlight potential gains and give motivation for evolution of traditional coupled architecture towards control and data planes separation. The different perspectives of energy efficiency, system capacity, interference management and mobility handling are discussed. Since, control and data planes separation approach is in its early stage, little literature exists that addresses some of the performance measures (e.g., \citep{6477646, 6554746} evaluates energy and spectral efficiency). Wherever possible, we provided survey of the approaches proposed for separation architecture; otherwise, we provided our view point for potential advantages and associated complexities in SARC. By considering different scenarios from the perspective of outlined performance measures, it is revealed that there is a huge potential for capacity and energy efficiency enhancements by separating control and data planes. Moreover, the SARC provides flexibility in mobility management at the cost of more complex signaling network. The second part of the article provides background for cooperation framework for interference management in multicell environment. It is emphasized that there are several potential advantages of sending CSI to the cBS and exploiting pro-active caching to realize backhaul relaxed CB and CoMP for interference management in future ultra-dense cellular environment.

Another perspective of cooperation has been presented where cooperation means assisting network for common information exchange or content dissemination between near-by devices in the form of ad-hoc or clustered mode direct communication. D2D CoMP has been introduced where conventional cooperation framework has been suggested to handle intra-cell interference. Due to ubiquitous coverage in SARC, centralized cBS offers more flexibility in CSI acquisition and corresponding beamforming for CB and CoMP operation. The centralized cBS also offers higher degree of freedom to predict nodes for content sharing and it can even be combined with network pro-active caching and adaptive clustering for self-organized D2D communication. 

Motivated by the control and data planes separation framework, in the following, we outline the lessons learned and several potential research directions in this area:
\begin{itemize}
  \item Energy efficiency is the most important aspect of future cellular systems. Among many approaches mentioned in Sec. \ref{ee} (e.g., BS switch-off, smart grid, renewable energy sources), dynamic BS switch-off mechanism can play an important role in realizing green cellular communication. The inherent drawback of coverage holes (due to BS switch-off techniques) and more interference (due to increased transmit power in cell range expansion) does not exist in SARC due to ubiquitous coverage. Some of the research studies (e.g., \citep{6477646, 6554746,7037473}) investigated the potential gains in energy efficiency due to control and data planes separation. In order to investigate full energy efficiency gains, the realistic power consumption models are required. For such models, existing approaches for traditional architecture can be investigated followed by more advanced and sophisticated energy management techniques for SARC.
  \item The higher spectrum and more bandwidth are envisioned to ensure capacity requirements of future cellular networks. In this context, mm-Wave spectrum and carrier aggregation are potential candidates for next generation cellular networks. A lot of research is being conducted to investigate feasibility of mm-Wave spectrum. Designing new channel models for dual connectivity (i.e., mm-Wave for data plane and lower frequency for control plane) has a lot of research potential that can lead towards communication in SARC.
  \item For current HetNet, intra-cell interference does not exist and inter-cell interference management has been standardized. In future ultra-dense networks, intra-cell interference will again be a problem due to underlay systems e.g., D2D communication. In order to overcome this interference, existing techniques of CB and CoMP can be extended at device level (D2D) and the backhaul limitations can be be complemented by exploiting pro-active caching techniques (e.g., \citep{6601672, 6665021, 6948325}). 
  \item In future ultra-dense environment, cells at mm-Wave spectrum will have spot beam coverage. This results in huge capacity enhancements which can further be leveraged by harnessing D2D cooperation for content sharing or content dissemination.
  \item Mobility management is flexible due to higher degree of freedom in SARC. However, this comes at the price of complex signaling network in SARC. The control plane design will be more complex due to more tiers (underlay networks). In this context, lot of research endeavors are required to realize seamless handovers and higher coverage probability while ensuring QoS requirements of each user.
\end{itemize}
\balance
\section*{Acknowledgements}
This work was made possible by NPRP grant No. 5-1047-2437 from Qatar National Research Fund (a member of the Qatar Foundation). The statements made herein are solely the responsibility of the authors. We would also like to acknowledge the support of the University of Surrey 5GIC (http://www.surrey.ac.uk/5gic) members for this work.
\balance
\Urlmuskip=0mu plus 1mu\relax
\bibliographystyle{IEEEtran}
\bibliography{Cooperative_Networks_____1}

\end{document} 