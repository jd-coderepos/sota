\documentclass[a4paper, 11pt]{article}

\usepackage[utf8]{inputenc}
\usepackage[T1]{fontenc}

\usepackage{lmodern}

\usepackage{graphics}
\usepackage[margin=3cm]{geometry}
\usepackage{booktabs}
\usepackage{microtype}
\usepackage{color}
\newcommand{\ra}[1]{\renewcommand{\arraystretch}{#1}}

\begin{document}

\title{\textbf{Memory-only selection of dictionary PINs}}
\author{Martin Stanek \ H_1(X) = - \sum_{i=1}^N p_i \log_2 p_i.  G(X) = \sum_{i=1}^N i\cdot p_i.  \mu_\alpha(X) = \min \{1\leq k \leq N\mid {\textstyle\sum\nolimits_{i=1}^k p_i} \geq \alpha\}.  \lambda_\beta = \sum_{i=1}^\beta p_i. 
\end{itemize}

Since the guesswork and the marginal guesswork are not directly comparable to Shannon entropy, 
the values of  and  are converted into bits using the following formulas \cite{BJM10}:
, .

Since the standard mapping does not cover digits 0 and 1, the ideal security metrics have the following 
values for dictionary PINs (assuming uniform distribution of PINs) :

\begin{itemize}
\renewcommand{\labelitemi}{}
\item PIN length 4:  bits, ,
\item PIN length 5:  bits, .
\end{itemize}


\section{Estimating metrics for dictionary PIN selection}

In order to model frequency distribution of words in a language we use frequency lists based on subtitles.
This is a respected method for analyzing contemporary languages \cite{BN09}. We use two frequency lists for
English -- a carefully prepared SUBTLEXus \cite{SUBTLEXus} containing 60,384 words with a frequency 
higher than 1, and the list compiled from subtitles available from opensubtitles.org \cite{opensub}, 
containing more than 450,000 words (even words with frequency 1). We will denote the results obtained using 
the first/second list with label ``SUBTLEXus''/``opensub'', respectively.

\subsection{Basic statistics}

We compare the metrics for straightforward translation of words to PINs using the standard mapping, see Figure \ref{fig-numpad}, with results obtained in \cite{SS13}. We consider only the words with the 
length equal to the PIN length . The translation starts with stripping the diacritical marks, if they are present. 
Then, the word is mapped to PIN using the standard mapping, e.g. ``love'' and ``hate'' yield 5683 and 4283,
respectively. The frequency of particular word contributes to the probability of resulting PIN.

The results for the PIN lengths 4 and 5 are shown in Table \ref{tab1-comp}. The columns labeled ``uniform''
contain results for PINs derived from a large (spell-checker) English dictionary assuming uniform frequencies 
of words \cite{SS13}. The columns labeled ``RockYou'' contains results for PINs derived from RockYou password 
database where frequencies of words (passwords) were taken into account. It is easy to notice a striking 
difference between scenarios that consider the frequencies of words and those that do not. 

\begin{table*}[h]\centering
\ra{1.2}
\begin{tabular}{@{}lrrrrrrrrrrr@{}}\toprule
 & \multicolumn{2}{c}{SUBTLEXus} &\phantom{a} & \multicolumn{2}{c}{opensub} &\phantom{a}& \multicolumn{2}{c}{uniform \cite{SS13}}
   &\phantom{a}& \multicolumn{2}{c}{RockYou \cite{SS13}}\\
   \cmidrule{2-3} \cmidrule{5-6} \cmidrule{8-9} \cmidrule{11-12}
 &  &  &&  &  &&  &   &&  &  \\ 
\midrule
 (bits) 
  &  &  &&  &  &&  &  &&  &  \\
  (bits) 
  &  &  &&  &  &&  &  &&  &  \\
 (bits) 
  &  &  &&  &  &&  &  &&  &  \\
 (\%) 
  &  &  &&  &  &&  &  &&  &  \\
\bottomrule
\end{tabular}
\caption{Comparison of metrics for straightforward construction of dictionary PINs}\label{tab1-comp}
\end{table*}

A closer look at the most frequent dictionary PINs from SUBTLEXus and opensub reveals the following observations:

\begin{itemize}
\renewcommand{\labelitemi}{}
\item PIN length 4: Top 7 PINs share the same spots in SUBTLEXus and opensub scenarios (the last three spots in 
top 10 are just permuted, i.e. the top 10 contains exactly the same set of PINs in both scenarios). The high marginal 
success rates are caused by the frequencies of the following PINs: 8428 (probability  based on SUBTLEXus, e.g.
produced from the word ``that''), 9428 (, ``what''), 8447 (, ``this''), 4283 (, ``have''), 
9687 (, ``your''), and 5669 (, ``know'').
\item PIN length 5: The sets of top 10 PINs differ in just two PINs, while the first 5 spots are exactly the same
in SUBTLEXus and opensub scenarios. For SUBTLEXus scenario the most frequent PINs are: 84373 (, e.g. produced 
from the word ``there''), 74448 (, ``right''), 22688 (, ``about''), 84465 (, ``think''),
46464 (, ``going''), and 46662 (, ``gonna'').
\end{itemize}

We can conclude that considering uniform frequencies of dictionary words is inadequate for estimating the security of memory-only
selection of dictionary PIN. The results show the deficiency of such PINs clearly -- the marginal success
rates are unacceptable. Moreover, it seems that this strategy is worse (in average) than strategies currently employed by
users. In order to compare memory-only dictionary PINs with ``common'' PIN selection strategies, we present estimates 
of PIN metrics based on RockYou password database and iPhone unlock codes \cite{BPA12} in Table \ref{tab2}.
On the other hand, the lack of digits 0 and 1 in the standard mapping ensures that these digits do not appear in the 
resulting PIN. Therefore the PINs that are often blacklisted, e.g. 0000, 1111 or 1234, or those the users are warned not to 
use, e.g. birthday or anniversary years, cannot be selected this way.

\begin{table*}[h]\centering
\ra{1.2}
\begin{tabular}{@{}lrrr@{}}\toprule
 & RockYou & iPhone \\
 \midrule
 (bits)       &  &  \\
  (bits) &  &  \\
 (bits) &  &  \\
 (\%) &  &  \\
\bottomrule
\end{tabular}
\caption{Estimation of PIN metrics (PIN length 4) \cite{BPA12}}\label{tab2}
\end{table*}

We explore few possibilities for improving memory-only dictionary PINs in the following sections.


\subsection{Blacklisting}

Blacklisting is a common method for improving the security of user-selected PINs. Even if not strictly enforced
(by forbidding the selection of some PINs), at least there are recommendations what PINs a user should not
choose, e.g. see \cite{VISA}:

\begin{quote}
``Select a PIN that cannot be easily guessed (i.e., do not use birth date, 
partial account numbers, sequential numbers like 1234, or repeated 
values such as 1111).''
\end{quote}

There are two possibilities for blacklisting in dictionary PIN scenario: first, blacklisting the most frequent 
words; and second, blacklisting the most frequent PINs. The PIN blacklisting is easier to enforce in practice,
and the values of security metrics are comparable for both approaches. Figure \ref{fig1-black} shows the entropy 
and the marginal success rate () for PIN blacklisting (based on SUBTLELXus) ranging from  to 
blacklisted PINs. The results show only a moderate improvement in security metrics, therefore blacklisting alone
is not a satisfactory method for improving memory-only dictionary PINs.

\begin{figure*}[h]\centering
\begin{center}\includegraphics{blacklist.pdf}\end{center}
\caption{The effect of PIN blacklisting on entropy and marginal success rate}\label{fig1-black}
\end{figure*}


\subsection{Modifications of PIN construction}

In order to improve the security metrics of resulting PIN, some natural modifications to basic translation
of dictionary word to PIN were proposed in \cite{SS13}. We evaluate these modifications when applied 
to our ``frequency-aware'' experiments:

\begin{itemize}
\renewcommand{\labelitemi}{}
\item Stretched mapping -- in order to cover digits  and , we can stretch the standard mapping. 
  Our estimates for this modification use the following mapping: a, b ; c, d  ;
  e, f  ; g, h, i  ; j, k, l  ; m, n  ; o, p, q  ;
  r, s, t ; u, v, w ; x, y, z .
\item Prefix -- instead of taking just words with the desired PIN length, i.e. , we use any word with the
  length greater or equal to  and we use its prefix for translation to PIN.
\item Morphing -- the standard word to PIN translation is enriched by random change of one character.
  Assuming that user can choose a random position in a word/PIN and a random digit, the resulting PIN
  is formed by replacing this position by the chosen digit. For example ``this'' can be translated to
  ``t1is''/8147, ``0his''/0447, ``thi9''/8449, etc. Certainly, this method is more demanding than the
  straightforward use of dictionary words. Our estimates assume uniform distribution of positions and
  digits for this method.
\end{itemize}

The results for all above methods are presented in Table \ref{tab3}. The stretched mapping yields no 
improvement at all -- the most frequent PINs changed their values, but their frequencies remained almost
unchanged. The prefix method offers a moderate improvement in security metrics. Obviously, the morphing 
is the most successful approach by a wide margin.

\begin{table*}[h]\centering
\ra{1.2}
\begin{tabular}{@{}lrrrrrrrrr@{}}\toprule
 & \multicolumn{2}{c}{Stretched map.} &\phantom{a} & \multicolumn{2}{c}{Prefix} &\phantom{a} & \multicolumn{2}{c}{Morphing} \\
   \cmidrule{2-3} \cmidrule{5-6} \cmidrule{8-9}
 &  &  &&  &  &&  &  \\ 
\midrule
 (bits) &  &  &&  &  &&  &  \\
  (bits)  &  &  &&  &  &&  &  \\
 (bits) &  &  &&  &  &&  &  \\
 (\%)  &  &  &&  &  &&  &  \\
\bottomrule
\end{tabular}
\caption{Modifications of PIN constructions -- results based on SUBTLEXus}\label{tab3}
\end{table*}

Comparing these results with the estimates from Table \ref{tab2}, we can notice that the prefix method
for dictionary PINs offers slightly better marginal success rate but worse entropy, guesswork and marginal guesswork. 
An interesting observation is that the morphing offers much better marginal success rate while keeping other
security metrics comparable to real-word estimates from Table \ref{tab2}.

Interestingly, the prefix method and the morphing yield better results than the estimates of PIN entropy 
by NIST \cite{NIST}: 9 and 10 bits for PIN lengths 4 and 5, respectively. On the other hand, plain memory-only
dictionary PINs offer less entropy than these estimates.

\subsection{Blacklisting the prefix and the morphing methods}

We expect that blacklisting of the most frequent PINs can further improve the security metrics 
of promising methods from the previous section (i.e. prefix and morphing methods). Indeed, our 
experiments confirm this expectation. Table \ref{tab4} shows the values of the entropy and the
marginal success rate for various sizes of the blacklist (, , and ). 

\begin{table*}[h]\centering
\ra{1.2}
\begin{tabular}{@{}lrrrrrrrrr@{}}\toprule
 &&& \multicolumn{2}{c}{Prefix} &\phantom{a} & \multicolumn{2}{c}{Morphing} \\
   \cmidrule{4-5} \cmidrule{7-8}
 & blacklist &\phantom{a} &  &  &&  &  \\ 
\midrule
 (bits) 
  & 0 &&  &  &&  &  \\
  &10 &&  &  &&  &  \\
  &20 &&  &  &&  &  \\
\midrule
 (\%) 
  &0  &&  &  &&  &  \\
  &10 &&  &  &&  &  \\
  &20 &&  &  &&  &  \\
\bottomrule
\end{tabular}
\caption{Combination of PIN blacklist and the prefix/morphing method}\label{tab4}
\end{table*}

The blacklisting substantially improves the marginal success rate, but offers only a moderate
improvement of the entropy. A disadvantage of the blacklisting is that it complicates the
implementation of authentication.

\subsection{Two-dictionary PINs}

Many people know more than one language. In such case, it is easy to adopt a strategy where a user
randomly choses a language and then (s)he selects a word for PIN construction. We expect obviously
an improvement in security metrics values. In order to assess the improvement we use English 
and Dutch frequency dictionaries SUBTLEXus and SUBTLEXnl \cite{SUBTLEXnl}. We use words with
frequency above 1 in both dictionaries, and we assume that the user selects the dictionary with
probability . The results for this two-dictionary scenario are shown 
in Table \ref{tab5}, where ``basic'' denotes the construction using words with the length , 
``prefix'' denotes the prefix method, and ``prefix (BL)'' denotes the combination of the prefix 
method with the blacklist of the length .

\begin{table*}[h]\centering
\ra{1.2}
\begin{tabular}{@{}lrrrrrrrrr@{}}\toprule
 & \multicolumn{2}{c}{Basic} &\phantom{a} & \multicolumn{2}{c}{Prefix} &\phantom{a} & \multicolumn{2}{c}{Prefix (BL 10)} \\
   \cmidrule{2-3} \cmidrule{5-6} \cmidrule{8-9}
 &  &  &&  &  &&  &  \\ 
\midrule
 (bits)              &   &   &&  &  &&  &  \\
  (bits)        &   &   &&  &  &&  &  \\
 (bits) &   &   &&  &   &&  &  \\
 (\%)        &  &  &&  &   &&  &  \\
\bottomrule
\end{tabular}
\caption{Security metrics for two-dictionary scenario}\label{tab5}
\end{table*}

As expected, the results are better than results for corresponding single-dictionary scenario methods. 
However, even with the blacklisting the results cannot match the morphing method results for single
dictionary.


\section{Conclusion}

We analyzed the security of memory-only selection of dictionary PINs. The results show that
plain construction of dictionary PINs is unsatisfactory and more involved methods should be used for 
improved security metrics.

\subsection*{Acknowledgment} The author acknowledges support by VEGA 1/0259/13. 


\begin{thebibliography}{99}
\bibitem{AJ13} Aspinall D, Just M. 
  ``Give Me Letters 2, 3 and 6!'': Partial Password Implementations and Attacks. In:Proceedings of 17th 
  International Conference on Financial Cryptography and Data Security (FC 2013). LNCS 7859. 
  Springer; 2013. 
\bibitem{PINstudy} Berry N. PIN analysis. DataGenetics; 2012. \\
  Available at www.datagenetics.com/blog/september32012/index.html [accessed 31 March 2014].
\bibitem{BJM10} Bonneau J, Just M, Matthews G.
  What's in name? Evaluating statistical attacks against personal knowledge questions.
  In: Proceedings of the 14th International Conference on Financial Cryptography and Data Security.
  LNCS 6052. Springer; 2010, pp. 98--113.
\bibitem{BPA12} Bonneau J, Preibusch S, Anderson R.
  A birthday present every eleven wallets? The security of customer-chosen banking PINs.
  In: Proceedings of the 16th International Conference on Financial Cryptography and Data Security.
  LNCS 7397. Springer; 2012, pp. 25--40.
\bibitem{BN09} Brysbaert M, New B.
  Moving beyond Kučera and Francis: A critical evaluation of current word frequency norms and the introduction 
  of a new and improved word frequency measure for American English. Behavior Research Methods, 41(1):977--990,
  Springer-Verlag; 2009.
\bibitem{ISO}  International Organization for Standardization. 
  Financial services -- Personal Identification  Number (PIN) management and security,
  Part 1: Basic principles and requirements for PINs in card-based systems,  ISO 9564-1; 2011.
\bibitem{JL11} Jakobsson M, Liu D. Bootstrapping Mobile PINs Using Passwords.
  W2SP 2011: Web 2.0 Security and Privacy; 2011. \\
  Available at w2spconf.com/2011/papers/mobilePIN.pdf [accessed 31 March 2014].
\bibitem{JL13} Jakobsson M, Liu D. Your Password is Your New PIN.
  SpringerBriefs in Computer Science 2013, pp 25-36, Springer; 2013.
\bibitem{NIST} NIST. Electronic Authentication Guideline. Special Publication 800-63-1; 2011.
\bibitem{opensub} Frequency Word Lists. 2012. \\
  Available at http://invokeit.wordpress.com/frequency-word-lists/ [accessed 31 March 2014].
\bibitem{PCI} PCI Security Standards Council. Payment Card Industry PIN Security Requirements. 
  Version 1.0; 2011. \\
  Available at www.pcisecuritystandards.org/security\_standards [accessed 31 March 2014].
\bibitem{SS13} Staneková L, Stanek M: Analysis of dictionary methods for PIN selection. 
  Computers \& Security, Volume 39, Part B, Elsevier, 2013, pp. 289-298. \\
  http://dx.doi.org/10.1016/j.cose.2013.08.006.
\bibitem{SUBTLEXus} SUBTLEXus, Word Frequency American English, Ghent University, 2009. \\
  Available at http://expsy.ugent.be/subtlexus/ [accessed 31 March 2014]
\bibitem{SUBTLEXnl} SUBTLEXnl, Database of Dutch Word Frequencies, 2010. \\
  Available at http://crr.ugent.be/programs-data/subtitle-frequencies/subtlex-nl [accessed 31 March 2014]
\bibitem{VISA} VISA. Issuer PIN Security Guidelines; 2010.\\
  Available at https://usa.visa.com/download/merchants/visa-issuer-pin-security-guideline.pdf 
  [accessed 31 March 2014].
\bibitem{W10}
  Weir M, Aggarwal S, Collins M, Stern H.
  Testing Metrics for Password Creation Policies by Attacking Large Sets of Revealed Passwords.
  In: Proceedings of the 17th ACM conference on Computer and Communications Security, CCS'10,
  ACM; 2010. pp. 162--175.
\end{thebibliography}

\end{document}
