\documentclass[a4paper, 11pt]{article}
\interfootnotelinepenalty=10000


\topmargin -6.0mm\oddsidemargin 0mm \evensidemargin 0mm \textheight
23cm\textwidth 16cm

\def\baselinestretch{1.5}
\parskip .15cm

\usepackage{amsfonts}
\usepackage{dsfont}
\usepackage{mathrsfs}
\usepackage{amssymb}
\usepackage{bbm}
\usepackage{epsfig}
\usepackage{amsmath}
\usepackage{float}
\usepackage{color}
\usepackage[english]{babel}
\usepackage{cite}
\usepackage{subfigure}
\usepackage{enumerate}

\usepackage{graphicx}

\def \defn{\triangleq}

\newcommand{\T}{^{\mbox{\tiny T}}}
\newtheorem{problem}{Problem}
\newtheorem{theorem}{Theorem}
\newtheorem{assumption}{Assumption}
\newtheorem{proposition}{Proposition}
\newtheorem{lemma}{Lemma}
\newtheorem{corollary}{Corollary}
\newtheorem{definition}{Definition}
\newtheorem{remark}{Remark}
\newtheorem{example}{Example}

\newenvironment{IEEEproof}[1][\bf Proof]{\smallskip\par\noindent\textit{#1: }}{\hspace*{\fill} \rule{6pt}{6pt}\smallskip}
\newenvironment{proof*}[1][Proof]{\smallskip\par\noindent\textbf{#1: }}{\smallskip}
\allowdisplaybreaks

\DeclareMathOperator{\rank}{rank}
\DeclareMathOperator{\re}{Re}
\DeclareMathOperator{\im}{Im}
\def\R{\mathbb{R}}
\def\C{\mathbb{C}}
\def\N{\mathbb{N}}
\def\Z{\mathbb{Z}}
\newcommand{\eps}{\varepsilon}

\newcommand{\tao}[1]{{\color{red}  \textbf{[Tao: #1]}}}

\newenvironment{system}[1]{\setlength{\arraycolsep}{0.5mm}\left\{ \; \begin{array}{#1}}{\end{array} \right.}

\begin{document}
\title{Network Synchronization  with Nonlinear Dynamics  and Switching Interactions\thanks{This work has been supported in part by the Knut and Alice Wallenberg Foundation and the Swedish Research
Council.}}
\date{}

\author{Tao Yang\thanks{T. Yang is with the Pacific Northwest National Laboratory, 902 Battelle Boulevard, Richland, WA 99352 USA (e-mail: Tao.Yang@pnnl.gov).}, Ziyang Meng\thanks{Z. Meng is with the Institute for Information-Oriented Control, Technische Universit\"{a}t M\"{u}nchen, D-80290 Munich, Germany (e-mail: zmeng@lsr.ei.tum.de).}, Guodong Shi\thanks{G. Shi is with the College of Engineering and Computer Science, The Australian National University, Canberra ACT 0200, Australia (e-mail: guodong.shi@anu.edu.au).}, Yiguang Hong\thanks{Y. Hong is with the Key Laboratory of Systems and Control, Institute of Systems Science, Chinese Academy of Science, Beijing 100190, China (e-mail: yghong@iss.ac.cn).} and Karl Henrik Johansson\thanks{K. H. Johansson is with the ACCESS Linnaeus Centre, School of Electrical Engineering, Royal Institute of Technology, Stockholm 10044, Sweden (e-mail: kallej@kth.se).}
}


\maketitle

\begin{abstract}
This paper considers the synchronization problem for networks of coupled nonlinear dynamical systems under switching communication topologies. Two types of nonlinear agent dynamics are considered.
The first one is non-expansive dynamics (stable dynamics with a convex Lyapunov function ) and the second one is dynamics that satisfies a global Lipschitz condition.
For the non-expansive case, we show that various forms of joint connectivity for communication graphs are
sufficient for networks to achieve global asymptotic -synchronization.
We also show that -synchronization leads to state synchronization provided that certain additional conditions are satisfied.
For the globally Lipschitz case, unlike the non-expansive case, joint connectivity alone is not sufficient for achieving synchronization. A sufficient condition for reaching global exponential synchronization is established in terms of the relationship between the global Lipschitz constant and the network parameters. We also extend the results to leader-follower networks. 
\end{abstract}


{\bf Keywords:} Multi-agent systems, nonlinear agents, switching interactions, synchronization.

\section{Introduction} \label{sec-intro}
We consider the synchronization problem for a network of coupled nonlinear agents
with agent set .
Their interactions (communications in the network) are described
by a time-varying directed graph ,
with  as a piecewise constant signal, where  is a finite set of all possible graphs over .
The state of agent  at time  is denoted as  and
evolves according to

where  is piecewise continuous in  and continuous in 
representing the uncoupled inherent agent dynamics,
 is the set of agent 's neighbors at time ,
and  is a piecewise continuous function marking the weight of edge  at time .

Systems of the form \eqref{multi-agent-closed-leaderless} have attracted considerable attention.
Most works focus on the case where the communication graph  is fixed, e.g.,\cite{wu-chua,wu-book,belykh-belykh-hasler,delellis-circuit,yu-chen-cao-tac2011,munz3,liu-cao-wu}. It is shown that
for the case where  satisfies a Lipschitz condition,
synchronization is achieved for a connected graph provided that the coupling strength is sufficiently large.
However, for the case where the communication graph is time-varying, the synchronization problem becomes much more challenging and existing literature mainly focuses on a few special cases when  is linear, e.g., the single-integrator case \cite{jadbabaie-lin-morse,moreau,lin07}, the double-integrator case \cite{ren-beard}, and the neutrally stable case \cite{scardovi-sepulchre,su-huang-tac}. Other studies assume some particular structures for the communication graph \cite{zhao-hill-liu,qin-gao-zheng,wen-duan-chen-yu}.
In particular, in \cite{zhao-hill-liu}, the authors focus on the case where the adjacency matrices associated with all communication graphs are simultaneously triangularizable. The authors of \cite{qin-gao-zheng} consider switching communication graphs that are weakly connected and balanced at all times. A more general case where the switching communication graph frequently has a directed spanning tree has been considered in \cite{wen-duan-chen-yu}. These special structures on the switching communication graph are rather restrictive compared to joint connectivity where the communication can be lost at any time.

This paper aims to investigate whether joint connectivity for switching communication graphs
can render synchronization for the nonlinear dynamics \eqref{multi-agent-closed-leaderless}.
We distinguish two classes depending on whether the nonlinear agent dynamics  is expansive or not.
For the non-expansive case, we focus on the case where the nonlinear agent dynamics is stable with a convex Lyapunov function . We show that various forms of joint connectivity for communication graphs are
sufficient for networks to achieve global asymptotic -synchronization, that is, the function  of the agent state converges to a common value.
We also show that -synchronization implies state synchronization provided that additional conditions are satisfied.
For the expansive case, we focus on when the nonlinear agent dynamics is globally Lipschitz, and
establish a sufficient condition for networks to achieve global exponential synchronization in terms of a relationship between the Lipschitz constant and the network parameters.

The remainder of this paper is organized as follows. Section~\ref{sec-prob} presents the problem definition and main results. Section~\ref{sec-proofs} provides technical proofs. In Section~\ref{sec-lf}, we extend the results to leader-follower networks. Finally, Section~\ref{sec-conclusion} concludes the paper. 

\section{Problem Definition and Main Results}\label{sec-prob}
\subsection{Problem Set-up}
Throughout the paper we make a standard dwell time assumption \cite{liberzon-morse} on  the switching signal : there is a lower bound  between two consecutive switching time instants of . We also assume that
there are constants  such that
 for all .
We denote  and
assume that the initial time is , and the initial state .
A digraph is {\it strongly connected} if it contains a directed path from every node to every other node.
The joint graph of  in the time interval  with  is denoted as .
For the communication graph, we introduce the following definition.

\begin{definition}\label{def-connected-leaderless}
(i).  is {\it uniformly jointly strongly connected} if there exists a constant  such that  is strongly connected for any .

(ii). Assume that  is undirected for all .
 is {\it infinitely jointly connected} if  is connected for any .
\end{definition}

In this paper, we are interested in the following synchronization problems.

\begin{definition}\label{synch-def}
The multi-agent system \eqref{multi-agent-closed-leaderless} achieves global asymptotic -synchronization,
where  is a continuously differentiable function, if
for any initial state , there exists a constant , such that
 for any  and any .
\end{definition}


\begin{definition}\label{syn-asy}
(i) The multi-agent system \eqref{multi-agent-closed-leaderless} achieves global asymptotic synchronization if\\ 
 for any , any  and any .

(ii) Multi-agent system \eqref{multi-agent-closed-leaderless} achieves global exponential synchronization if there exist  and  such that

for any  and any .
\end{definition}


\begin{remark}
-synchronization is a type of output synchronization where the output of agent  is chosen to be . It is related to but different from -synchronization \cite{cortes-auto08,hong-JSSC2010} since
 is a function of an individual agent state while  is a function of all agent states.
\end{remark}


\subsection{Non-expansive Inherent Dynamics}\label{sec-ll}
In this section, we focus on when the nonlinear inherent agent dynamics is non-expansive as indicated by the following assumption.

\begin{assumption}\label{ass-lyap}
 is a continuously differentiable positive definite convex function
satisfying
\begin{itemize}
\item[(i).] ;
\item[(ii).]  for any  and any .
\end{itemize}
\end{assumption}

The following lemma shows how Assumption  \ref{ass-lyap} enforces non-expansive dynamics.
\begin{lemma}\label{lem-invariant-leaderless}
Let Assumption~\ref{ass-lyap} hold. Along the multi-agent dynamics \eqref{multi-agent-closed-leaderless},
 is non-increasing for all .
\end{lemma}

We now state main results for the non-expansive case.
\begin{theorem}\label{thm1-leaderless}
Let Assumption~\ref{ass-lyap} hold.
The multi-agent system \eqref{multi-agent-closed-leaderless} achieves
global asymptotic -synchronization if  is uniformly jointly strongly connected.
\end{theorem}


\begin{theorem}\label{thm2-leaderless}
Let Assumption~\ref{ass-lyap} hold.
Assume that  is undirected for all .
The multi-agent system \eqref{multi-agent-closed-leaderless} achieves global asymptotic -synchronization if  is infinitely jointly connected.
\end{theorem}

\begin{remark}\label{remark-ass-stable}
For the linear time-varying case , if there exists a matrix  such that

then  for  satisfies Assumption~\ref{ass-lyap}.
For the linear time-invariant case , the condition \eqref{LTV-common} is
equivalent to that the matrix  is neutrally stable \cite{su-huang-tac}.
\end{remark}





\subsection{-synchronization vs. State Synchronization}\label{discussion-phi-state}
The following result establishes conditions under which -synchronization may imply state synchronization.

\begin{theorem}\label{prop-guodong}
Let  with  being a fixed, strongly connected digraph under which the multi-agent system \eqref{multi-agent-closed-leaderless} achieves global asymptotic -synchronization
for some positive definite function .
Let Assumption \ref{ass-lyap} hold.
Moreover, assume that
\begin{enumerate}[(i).]
\item  is bounded for any  and any .
\item  for some ; and
\item  is strongly convex.
\end{enumerate}
Then the multi-agent system \eqref{multi-agent-closed-leaderless} achieves global asymptotic synchronization.
\end{theorem}

\subsection{Lipschitz Inherent Dynamics}\label{sec-unstable}
We consider also the case when the nonlinear inherent agent dynamics is possibly expansive.
We focus on when the dynamics satisfies
the following global Lipschitz condition.

\begin{assumption}\label{ass-phi-state-1}
There exists a constant  such that

\end{assumption}

Our  main result for this case is given below.

\begin{theorem}\label{unstable-thm1}
Let Assumption~\ref{ass-phi-state-1} hold.
Assume that  is uniformly jointly strongly connected.
Global exponential synchronization is achieved for the
multi-agent system \eqref{multi-agent-closed-leaderless}
if , where  is a constant depending on the network parameters.
\end{theorem}

\begin{remark}\label{remark-unstable-ass}
Assumption~\ref{ass-phi-state-1} and its variants have been considered in the literature for fixed communication
graphs, e.g.,\cite{wu-chua,wu-book,belykh-belykh-hasler,delellis-circuit,liu-cao-wu,yu-chen-cao-tac2011,munz3}.
Compared with the existing literature, we here study a more challenging case, where the communication graphs are time-varying.
Unlike the fixed case where the global Lipschitz condition is sufficient to guarantee synchronization,
Theorem~\ref{unstable-thm1} established a sufficient synchronization condition related to the Lipschitz constant and the network parameters.
\end{remark}



\section{Proofs of the Main Results}\label{sec-proofs}
In this section, we provide proofs of the main results.
\subsection{Proof of Lemma~\ref{lem-invariant-leaderless}}\label{proof-lem-invariant-leaderless}
Recall that the upper Dini derivative of a function  at  is defined as . The following lemma from \cite{dan,lin07} is useful for the proof.
\begin{lemma}\label{lem1}
Let  be
continuously differentiable and . If 
is the set of indices where the maximum is reached at , then .
\end{lemma}
Denote .
We first note that the convexity property of  implies that \cite[pp.69]{boyd-vandenberghe}

It then follows from Lemma \ref{lem1}, Assumption~\ref{ass-lyap}(ii) and \eqref{convexity-eq} that

where the last inequality follows from . This proves the lemma.


\subsection{Proof of Theorem~\ref{thm1-leaderless}}\label{proof-thm1-leaderless}
It follows from Lemma~\ref{lem-invariant-leaderless} that for any initial state ,
there exists a constant , such that . We shall show that  is the required constant in Definition \ref{synch-def} of -synchronization.

We first note that by Lemma~\ref{lem-invariant-leaderless} that for all , there exist constants ,
such that

Also note that it follows from  that for any , there exists  such that


The proof of Theorem~\ref{thm1-leaderless} is based on a contradiction argument and
relies on the following lemma.
\begin{lemma}\label{lem1-thm1}
Let Assumption~\ref{ass-lyap} hold. Assume that  is uniformly jointly strongly connected.
If there exists an agent  such that
, then there exists  and  such that for all ,
, where

with  given in Definition \ref{def-connected-leaderless}(i) and  is the dwell time.
\end{lemma}
\begin{IEEEproof}
Let us first define .
Then there exists an infinite time sequence 
with  such that  for all .
We then pick up one ,  such that it is greater than or equal to 
and denote it as .

We now prove the lemma by estimating an upper bound of the scalar function  agent by agent.
The proof is based on a generalization of the method proposed in the proof of \cite[Lemma~4.3]{shi-johansson-hong}
but with substantial differences on the agent dynamics and Lyapunov function. Moreover, the convexity of  plays an important role.

\noindent Step 1. Focus on agent .
By using Assumption~\ref{ass-lyap}(ii), \eqref{convexity-eq}, and \eqref{bound-T1}, we obtain that for all ,

It then follows that for all ,

where .

\vspace{2mm}

\noindent Step 2. Consider agent  such that  for
. The existence of such an agent can be shown as follows.
Since  is uniformly jointly strongly connected, it is not hard to see
that there exists an agent  and  such that
 for .

From \eqref{dis-k0}, we obtain for all ,

where .


We next estimate  by considering two different cases.

Case I:  for all .

By using Assumption~\ref{ass-lyap}(ii), \eqref{convexity-eq}, \eqref{bound-T1}, and \eqref{inner-point}, we obtain for all ,


From the preceding relation, we obtain for ,

where .
Therefore, we have

where

By applying the same analysis as we obtained \eqref{dis-k0} to the agent ,
we obtain for all ,

By combining the inequalities \eqref{inner-point}, \eqref{inner-point-2} and \eqref{inner-point-i1}, we obtain for all ,

where .

Case II: There exists a time instant  such that

By applying the similar analysis as we obtained \eqref{leaderless-k1}
to the agent , we obtain for all ,

This leads to

By combining the preceding relation, \eqref{inner-point}, and \eqref{k1-case2}, and using  which follows from , we obtain for all  ,

From the preceding relation and \eqref{inner-key}, it follows that for both cases, we have
for all  ,

From the preceding relation, \eqref{inner-point} and ,
it follows that for all ,




\vspace{2mm}

\noindent Step 3. Consider agent  such that there exists an edge from the set  to the agent  in  for .
The existence of such an agent  and  follows similarly from the argument in Step 2.

Similarly, we can bound  by considering two different cases and obtain that
for all ,

where .

By combining \eqref{inner-key-01} and \eqref{inner-key-2}, and using ,
we obtain that for all ,


\vspace{2mm}

\noindent Step 4. By repeating the above process on
time intervals ,
we eventually obtain that for all ,

where .
The result of the lemma then follows by choosing  and .
\end{IEEEproof}

We are now ready to prove Theorem~\ref{thm1-leaderless} by contradiction.
Suppose that there exists an agent  such that
. It then follows from Lemma~\ref{lem1-thm1} that 
for all , provided that .
This contradicts the fact that .
Thus, there does not exist an agent  such that . Hence,  for all .

\subsection{Proof of Theorem \ref{thm2-leaderless}}\label{proof-thm2-leaderless}
The proof relies on the following lemma.
\begin{lemma}\label{lem1-thm2}
Let Assumption~\ref{ass-lyap} hold. Assume that  is infinitely jointly connected.
If there exists an agent  such that
, then there exist  and  such that

\end{lemma}

\begin{IEEEproof}
The proof of Lemma~\ref{lem1-thm2} is similar to that of Lemma~\ref{lem1-thm1} and based on estimating
an upper bound for the scalar quantity  agent by agent.
However, since  is infinitely jointly connected,
the method that we get the order of the agents based on the intervals induced by the uniform bound  cannot be used here.
We can however apply the strategy in \cite{shi-johansson-hong} for the analysis as shown below: 

\noindent Step 1. In this step, we focus on agent . Since  is infinitely jointly connected,
we can define

and the set

For , agent  has no neighbor, it follows from Assumption~\ref{ass-lyap}(ii) that
.
Thus,  for .
By applying a similar analysis as we obtained \eqref{inner-point}, we have for all ,

where .

\vspace{2mm}

\noindent Step 2. In this step, we focus on all .
We then estimate  for all  by considering two different cases, i.e.,
Case I: If  for all , and
Case II: If there exists a time instant  such that
.
By using the similar argument as the two-case analysis for agent  in the proof of Theorem~\ref{thm1-leaderless},
we eventually obtain for all ,

where  with  given by \eqref{def-mu}.

\vspace{2mm}

\noindent Step 3. We then view the set  as a subsystem. Define 
as the first time when there is an edge between this subsystem and the remaining agents and  accordingly.
By using the similar analysis for agent  in Step 2, we can estimate the upper bound for all the agent in the set
,

\vspace{2mm}

\noindent Step 4. Since  is infinitely jointly connected, we can continue the above process until
 for some .
Eventually, we have

The result of lemma then follows by choosing  and .
\end{IEEEproof}

The remaining proof of Theorem~\ref{thm2-leaderless} follows from a contradiction argument
and Lemma~\ref{lem1-thm2} in the same way as the proof of Theorem~\ref{thm1-leaderless}.


\subsection{Proof of Theorem~\ref{prop-guodong}}\label{proof-prop-guodong}
If the multi-agent system \eqref{multi-agent-closed-leaderless}
reaches  asymptotic -synchronization, i.e.,
  for all , then for any , there exists a  such that


If  the desired conclusion holds trivially, i.e.,  for all 
due to the positive definiteness of .
In the remainder of the proof we assume . We shall prove the result by contradiction. Suppose that state synchronization is not achieved, then there exist two agents  such that .
In other words, there exist an infinite time sequence  with , and a constant  such that  for all , where  is given in condition (ii) of Theorem~\ref{prop-guodong}. We divide the following analysis into three steps.

\vspace{2mm}

\noindent Step 1. In this step, we prove the following crucial claim.

\vspace{1mm}

\noindent{\it Claim.} For any , there are two agents  with  such that .

We establish this claim via a recursive analysis. If either  or   then the result follows trivially. Otherwise we pick up another agent  satisfying that there is an edge between  and . This  always exists since  is strongly connected. Then either  or  must hold. Thus, we have again either established the claim, or we can continue to select another agent different from , , and  and repeat the argument. Since we have a finite number of agents, the desired claim holds.

Furthermore, since there is a finite number of agent pairs, without loss of generality, we assume that the given agent pair  does not vary for different  (otherwise we can always select an infinite subsequence of  for the following discussions).

\vspace{2mm}

\noindent Step 2. In this step, we establish a lower bound of  for a small time interval after a particular  satisfying .
From \eqref{bound-phi} and  given in Assumption~\ref{ass-lyap}(i),
we see that  and  are bounded for all  and for all .
It then follows from condition (i) of Theorem~\ref{prop-guodong} and \eqref{multi-agent-closed-leaderless} that

for all  and some .
Without loss of generality we assume that . Then  will be independent of .

By plugging in the fact that  and using the condition (ii) of Theorem~\ref{prop-guodong}, we obtain that


\vspace{2mm}

\noindent Step 3. We first note that the strong convexity of  implies that \cite[pp.459]{boyd-vandenberghe}
there exists an  such that

By using Assumption~\ref{ass-lyap}(ii) and \eqref{strongly-convexity-eq}, we obtain for ,

By using \eqref{bound-phi}, \eqref{guodong1}, \eqref{guodong2}, and condition (ii) of Theorem~\ref{prop-guodong},
we obtain that for ,

which yields

It is then straightforward to see that

if we take

However, this contradicts  the definition of -synchronization since  is arbitrarily chosen. This completes the proof and the desired conclusion holds.



\subsection{Proof of Theorem~\ref{unstable-thm1}}\label{proof-unstable-thm1}
The proof is based on the convergence analysis of the scalar quantity

where

Unlike the contradiction argument used for proof of Theorem~\ref{thm1-leaderless}, where the convergence rate is unclear, here we explicitly characterize
the convergence rate. The proof relies on the following lemmas.

\begin{lemma}\label{lemma-invariant-unstable}
Let Assumption~\ref{ass-unstable-nonlinear} hold.
Along the multi-agent dynamics \eqref{multi-agent-closed-leaderless},  is non-increasing for all .
\end{lemma}

\begin{IEEEproof}
This lemma establishes a critical non-expansive property along the multi-agent dynamics \eqref{multi-agent-closed-leaderless} for the globally Lipschitz case.
The proof follows from the same techniques as those for proving Lemma \ref{lem-invariant-leaderless} by investigating the Dini derivative of . 

Let  be the set containing all the node pairs that reach the maximum at time , i.e.,
.
It is not hard to obtain that

where the first equality follows from Lemma~\ref{lem1} and \eqref{multi-agent-closed-leaderless},
the first inequality follows from  \eqref{ass-unstable-nonlinear} and  and  for all , and the second inequality follows from \eqref{unstable-lyap-agent-by-agent}.
\end{IEEEproof}



\begin{lemma}\label{lem-thm-unstable}
Let Assumption~\ref{ass-unstable-nonlinear} hold. Assume that  is uniformly jointly strongly connected. Then there exists  such that

where ,  is given by \eqref{defn-T0} and .
\end{lemma}

\begin{IEEEproof}
The proof is based on the convergence analysis of  for all agent pairs 
in several steps, which is similar to the proof of Lemma~\ref{lem1-thm1}. Without loss of generality, we assume that . We also sometimes denote  and  as  and , respectively, for notational simplification.

\noindent Step 1. We begin by considering any agent . Since  is uniformly jointly strongly connected, we know that
 is the root and that there exists a time  and an agent  such that  during .

We first note that it follows from \eqref{unstable-lyap-edge} and Lemma~\ref{lemma-invariant-unstable} that for all ,


Taking the derivative of  along the trajectories of \eqref{multi-agent-closed-leaderless}, we obtain that for all ,

where . The first inequality follows from \eqref{ass-unstable-nonlinear} and \eqref{unstable-lyap-agent-by-agent}, while the second inequality follows from \eqref{max-bound}.

It thus follows that

where .

Similarly we obtain that for all , , where . It then follows from \eqref{Vi1i2bound1} that

where .

\vspace*{2mm}

\noindent Step 2.   Since  is uniformly jointly strongly connected, we know that
that there exists a time instant  and an arc from  to
 during .

We then estimate an upper bound for  by considering two different cases:  and .
We eventually obtain that for all ,

where

It then follows from \eqref{vi1vi2bound-end}, \eqref{V1V3bound-overall}
and  that for all ,

where .

\vspace*{2mm}

\noindent Step 3. By continuing the above process, we obtain that for all ,


\vspace*{2mm}

\noindent Step 4. Since  is uniformly jointly strongly connected, \eqref{eq:bound1} holds for any . By using the same analysis, we eventually obtain that for all ,

Hence the result follows by choosing  with  given by \eqref{beta-star}.
\end{IEEEproof}

We are now ready to prove Theorem~\ref{unstable-thm1}.
By using Lemma~\ref{lem-thm-unstable} and \eqref{unstable-lyap-edge}, we obtain that

where  denotes the largest integer that is not greater than
 and .

It then follows from \eqref{unstable-lyap-edge} and \eqref{unstable-lyap-agent-by-agent} that

Hence, global exponential synchronization is achieved with 
and  provided that .
This concludes the proof of the desired theorem.






\section{Leader-follower Networks}\label{sec-lf}

Our focus so far has been on achieving synchronization for leaderless networks.
In this section we consider the synchronization problem for leader-follower networks.
Suppose that there is an additional agent, labeled as agent ,
which plays as a reference or leader for the agents in the set .
In view of this we also call an agent in  a follower, and denote   as the overall agent set. The overall communication in the network is described by a time-varying directed graph . Here for the sake of simplicity we continue to use  to denote the piecewise constant graph signal.
We also make a standard dwell time assumption \cite{liberzon-morse} on  the switching signal . 

For the leader-follower communication graph, we introduce the following definition.
\begin{definition}\label{def-connected-leader}
(i).  is \emph{leader connected} if
for any follower agent 
there is a directed path from the leader  to follower agent  in 
at time .
Moreover,  is \emph{jointly leader connected} in the time interval  if the union graph  is leader connected.

(ii).  is {\emph{uniformly jointly leader connected}} if there exists a constant  such that the union graph  is leader connected for any .

(iii).  is {\emph{infinitely jointly leader connected}} if the union graph  is leader connected for any .
\end{definition}

For leader-follower  networks, the evolutions of the follower state  and the leader state
 are given by

where the nonlinear function  and  follow from the same definitions as those of the leaderless case in \eqref{multi-agent-closed-leaderless}, and  is a piecewise continuous function marking  the strength of the edge , if any. Assume that there is a constant  such that  for all . We also 
assume that the initial time is  and denote the initial state for the leader as .

For the leader-follower networks, we are interested in the following synchronization problems.


\begin{definition}\label{lf-syn-asy}
(i) The multi-agent system \eqref{multi-agent-closed} achieves global asymptotic synchronization if
 for any , any , any , and .

(ii) Multi-agent system \eqref{multi-agent-closed} achieves global exponential synchronization if there exist  and  such that
there exist  and  such that

for any , any , and .
\end{definition}

\subsection{Non-expansive Inherent Dynamics}
In this section, we extend the results for the case when the agent dynamics is non-expansive to leader-follower networks. We make the following assumption on the agent dynamics.
\begin{assumption}\label{ass-lyap-lf}
 is a continuously differentiable positive definite function
such that the following conditions hold:
\begin{itemize}
\item[(i).] ;
\item[(ii).]  for all  and .
\end{itemize}
\end{assumption}
Assumption~\ref{ass-lyap-lf} is similar to Assumption~\ref{ass-lyap} however with possibly different functions   and Assumption~\ref{ass-lyap-lf}(ii) holds in the relative coordinate.

The convexity property guarantees the non-expansive property along the leader-follower multi-agent dynamics \eqref{multi-agent-closed} as shown in the following lemma whose proof is similar to that of Lemma~\ref{lem-invariant-leaderless} and thus omitted.
\begin{lemma}\label{lem-invariant}
Let Assumption~\ref{ass-lyap-lf} hold. Along the leader-follower multi-agent dynamics \eqref{multi-agent-closed},\\ 
 is non-increasing for all ..
\end{lemma}

We now state our results for the non-expansive case.

\begin{theorem}\label{thm1-leader}
Let Assumption~\ref{ass-lyap-lf} hold.
The multi-agent system \eqref{multi-agent-closed} achieves global asymptotic synchronization if  is uniformly jointly leader connected.
\end{theorem}

\begin{theorem}\label{thm2-leader}
Let Assumption \ref{ass-lyap-lf} hold.
Assume that  is undirected for all .
The multi-agent system \eqref{multi-agent-closed} achieves global asymptotic synchronization is achieved if  is infinitely jointly leader connected.
\end{theorem}

The proofs of Theorems~\ref{thm1-leader} and~\ref{thm2-leader} are given in Appendices~\ref{app-lf1} and~\ref{app-lf2}, respectively, and based on a generalization of the methods proposed in
\cite{moreau,shi-johansson-siam} however the nonlinear agent dynamics results in a different Lyapunov function . The analysis are based on estimating the scalar function  agent by agent and thus yields an estimate of the convergence rate. They are different from the contradiction arguments used in the proofs of Theorems~\ref{thm1-leaderless} and \ref{thm2-leaderless} where the convergence rate is unclear.

\begin{remark}\label{remark-exponential}
If the scalar function  satisfies an additional condition, i.e.,
there exist , such that

then Theorem \ref{thm1-leader} leads to global exponential synchronization.
\end{remark}

\begin{remark}
For the leader-follower case, Theorems~\ref{thm1-leader} and \ref{thm2-leader} show that
global asymptotic synchronization is achieved while for the leaderless case, while global asymptotic -synchronization is achieved as shown in Theorems~\ref{thm1-leaderless} and \ref{thm2-leaderless}. For the leader-follower case,
the uniformly jointly leader connected in Theorem \ref{thm1-leader}
requires the leader to be a center node, while for the leaderless case, the uniformly jointly strongly connected
in Theorem \ref{thm1-leaderless} requires every node to be a center node.
Thus, the connectivity condition of the leaderless case is stronger than that of the leader-follower case.
\end{remark}


\subsection{Lipschitz Inherent Dynamics}

In this section, we extend the result for the case when the agent dynamics is globally Lipschitz to leader-follower networks.

Our main result for this case is given in the following theorem whose proof can be found in Appendix~\ref{app-lf3}.

\begin{theorem}\label{unstable-thm2}
Let Assumption~\ref{ass-phi-state-1} hold. Suppose that  is uniformly jointly leader connected.
The multi-agent system \eqref{multi-agent-closed} achieves global exponential synchronization if , where  is a constant depending on the
network parameters.
\end{theorem}

\section{Conclusions}\label{sec-conclusion}
In this paper, synchronization problems for networks with nonlinear inherent agent dynamics and switching topologies
have been investigated. Two types of nonlinear dynamics were considered: non-expansive and globally Lipschitz.
For the non-expansive case, we found that the convexity of the Lyapunov function plays a crucial rule in the analysis
and showed that the uniformly joint strong connectivity is sufficient for achieving global asymptotic -synchronization.
When communication graphs are undirected, the infinitely joint connectivity is a sufficient synchronization condition.
Moreover, we established conditions under which -synchronization implies state synchronization.
For the globally Lipschitz case, we found that joint connectivity alone is not sufficient to achieve synchronization
but established a sufficient synchronization condition. The proposed condition reveals the relationship between the Lipschitz constant and the network parameters. 
The results were also extended to leader-follower networks.
An interesting future direction is to study the synchronization problem for coupled non-identical nonlinear inherent dynamics under general switching topologies.


\bibliography{IEEEabrv,referenc}
\bibliographystyle{IEEEtran}


\appendix

\section{Proof of Theorem~\ref{thm1-leader}}\label{app-lf1}
The proof of Theorems~\ref{thm1-leader} relies on the following lemma whose proof is similar to that of Lemmas~\ref{lem1-thm1}.
\begin{lemma}\label{lem1-thm3}
Let Assumption~\ref{ass-lyap-lf} hold.  Assume that  is uniformly jointly leader connected.
Then there exists  such that

where , with  given in \eqref{defn-T0}.
\end{lemma}

\begin{IEEEproof}
Without loss of generality, we assume the initial time .
Similar to the proof of Lemma~\ref{lem1-thm1}, we estimate  agent by agent on the subintervals  for  in several steps.

\noindent Step 1. In this step, we focus on an follower agent  such that
 for .
The existence of such a follower agent  and  and follows from the fact
 is uniformly jointly leader connected. 
For convenience, we introduce  for all  as the relative state from the leader agent.
From \eqref{multi-agent-closed}, we obtain the following dynamics.


Similar to \eqref{leaderless-k1}, we obtain that for all ,

We then obtain that 
where

with

For , the edge  may no longer exist.
Nevertheless, we have for ,

It then follows from \eqref{dis-k1} and \eqref{same-worst} that
for all ,

where  and
.

\vspace*{2mm}

\noindent Step 2. In this step, we analysis a follower agent  such that
there either an edge  or an edge  in  for . Again, the existence of  and  due to the uniform joint leader connectivity.
By going through the similar analysis as Step 2 of the proof for Lemma~\ref{lem1-thm1}, we eventually obtain
that for ,

where ,
with

and .

\vspace*{2mm}

\noindent Step 3. By applying the similar analysis on the subintervals  for ,
we obtain that for all  and for all ,

where

and

It follows from \eqref{deltai} and \eqref{hatdeltai} that  for all .
This together with \eqref{allless} leads to
 for all , where

with

The result of lemma then follows by choosing .
\end{IEEEproof}

We are ready to prove Theorem~\ref{thm1-leader}. Without loss of generality, we assume that , it then follows from Lemma~\ref{lem1-thm3} that . Thus, for , we have

It then follows that

This together with the fact that  is positive definite as given in Assumption~\ref{ass-lyap-lf} implies that
the multi-agent system \eqref{multi-agent-closed} achieves global asymptotic synchronization.

\section{Proof of Theorem~\ref{thm2-leader}}\label{app-lf2}
The proof of Theorems~\ref{thm2-leader} relies on the following lemma whose proof is similar to that of Lemmas~\ref{lem1-thm1}.

\begin{lemma}\label{lem1-thm4}
Let Assumption~\ref{ass-lyap-lf} hold. Assume that  is infinitely jointly leader connected.
Then there exist ,  and  such that

\end{lemma}

\begin{IEEEproof}
Since  is infinitely jointly leader connected, there exist a sequence of time instants

such that

for , and  is leader connected for .
Moreover, each edge in 
exists for at least the dwell time  during 
for  and .

We shall estimate  agent by agent on the subintervals , 
for the interval , .

\noindent Step 1. In this step, we focus all the follower agent , where

and

The existence of  and  due to the fact that  is leader connected.
It then follows from the similar analysis as we obtained \eqref{dis-k1} that

where  is given by \eqref{hatdelta1}.

\vspace*{2mm}

\noindent Step 2. In this step, similar to Step 3 of the proof for Lemma~\ref{lem1-thm2},
we view the set  as a subsystem. Define  as the first time when
there is an edge between this subsystem and the remaining follower agents and  accordingly.

By going through the similar analysis as Step 2 of the proof for Lemma~\ref{lem1-thm1},
we eventually obtain that for ,

where

with

and  given by \eqref{hatdelta1}.

\vspace*{2mm}

\noindent Step 3. Since  is infinitely jointly leader connected,
we proceed the above analysis until 
for some  such that

with  defined similarly to  and , and

From the preceding relation and \eqref{tildedelta2},
we obtain for ,

It is then easy to see that  for all .
This together with \eqref{hatdelta1}, \eqref{tildedelta1}, and  implies that for all  ,

where

with  given by \eqref{def-lambda1hat}.
The result then follows by choosing  as defined in \eqref{defnTp1} and \eqref{defnTp2},  and .
\end{IEEEproof}

We are ready to prove Theorem~\ref{thm1-leader}. Without loss of generality, we assume that , it follows then follows from Lemma~\ref{lem-invariant}, Lemma~\ref{lem1-thm4}, and the fact that
 for , which follows from the definition of  given in \eqref{defnTp1} and \eqref{defnTp2}, that for 

Thus, we have for , . This together with the fact that  is positive definite as given in Assumption~\ref{ass-lyap-lf} implies that the multi-agent system \eqref{multi-agent-closed} achieves global asymptotic synchronization.


\section{Proof of Theorem~\ref{unstable-thm2}}\label{app-lf3}
The proof is similar to that of Theorem~\ref{unstable-thm1} however in the relative coordinate  whose evolution is given by \eqref{dynamics-relative}, Again, without loss of generality, we assume the initial time .

The proof is based on the convergence analysis of the nonnegative scalar

where  and



Let us define .
Similar to Lemma~\ref{lemma-invariant-unstable}, we obtain that  for all  along the multi-agent dynamics~\eqref{multi-agent-closed}.
By combining the preceding relation we have , for all  and all .
Following the similar analysis as the proof of Theorem \ref{thm1-leader}, we can show that

where  is given by  \eqref{deltaN} and  is given by \eqref{GES-paras},

with . It then follows that

Hence, global exponential synchronization is achieved with  and  provided that . 

\end{document}
