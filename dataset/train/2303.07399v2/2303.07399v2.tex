

\documentclass[final]{cvpr}
\usepackage{tabulary,xspace}
\usepackage{tabularx}
\usepackage{times}
\usepackage{epsfig}
\usepackage{graphicx}
\usepackage{fixmath,mathtools,nicefrac,mmstyle}
\usepackage[table,xcdraw]{xcolor}
\usepackage{amsmath}
\usepackage{amssymb}
\usepackage{caption}
\usepackage{multirow}
\usepackage{booktabs}
\usepackage[section]{placeins}
\usepackage{float}
\usepackage{wrapfig}


\usepackage[pagebackref,breaklinks,colorlinks]{hyperref}

\newcommand{\tablestyle}[2]{\setlength{\tabcolsep}{#1}\renewcommand{\arraystretch}{#2}\centering\small}
\definecolor{ourscolor}{gray}{.9}
\newcommand{\ours}[1]{\cellcolor{ourscolor}{#1}}
\newcommand{\authorskip}{\hspace{2.5mm}}



\def\cvprPaperID{****} \def\confName{CVPR}
\def\confYear{2021}
\graphicspath{{./figs/}{./figs/result/}}



\begin{document}

\title{RTMPose: Real-Time Multi-Person Pose Estimation based on MMPose}


\author{Tao Jiang \authorskip Peng Lu \authorskip Li Zhang \authorskip Ningsheng Ma \authorskip
Rui Han \authorskip \\ Chengqi Lyu \authorskip Yining Li \authorskip Kai Chen \\label{eq:gau}
\begin{split}
    U&=\phi_u(XW_{u}) \\
    V&=\phi_v(XW_{v}) \\
    O&=(U \odot AV)W_o
\end{split}

A = \frac {1}{n} \mathit{relu} ^2( \frac {Q(X)K(Z)^\top}{ \sqrt {s} }), Z=\phi_z( X W_z)
\label{eq:sord}
y_i=\frac{e^{\phi(r_t, r_i)}}{\sum^K_{k=1}e^{\phi(r_t,r_k)}}

\phi(r_t, r_i)=e^{\frac{-(r_t-r_i)^2}{2\sigma^2}}

y_i=\frac{e^{\phi(r_t, r_i)/\tau}}{\sum^K_{k=1}e^{\phi(r_t,r_l)/\tau}}

\sigma = \sqrt{\frac{W_{S}}{16}}


where  is the bin number in the horizontal and vertical directions respectively. This step improves the accuracy from 72.7\% to 72.8\%.

\paragraph{Larger convolution kernel}
We experiment with different kernel sizes of the last convolutional layer and find that using a larger kernel size gives a performance improvement over using  kernel. Finally, we chose to use a  convolutional layer, which achieves 73.3\% AP. We compare model performances with different kernel sizes in Table~\ref{tab:ablation:kernel_sizes}. Additionally, we also compare the effect of different temperature factors in Table~\ref{tab:ablation:temperature} using the final model architecture.

\begin{table}
    \centering
    \begin{minipage}[t]{0.45\linewidth}
        \centering
        \caption{Comparison of different kernel sizes}\label{tab:ablation:kernel_sizes}
        \vspace{6pt}
        \begin{tabular}{ll}
        \hline
        Kernel Size & mAP  \\ \hline
        1x1         & 72.8 \\
        3x3         & 73.2 \\
        5x5         & 73.2 \\
        \ours{7x7}  & \ours{73.3} \\
        9x9         & 73.0 \\
        \hline
        \end{tabular}
    \end{minipage}
    \hfill
    \begin{minipage}[t]{0.45\linewidth}
        \centering
        \caption{Comparison of different temperature factors.}\label{tab:ablation:temperature}
        \vspace{6pt}
        \begin{tabular}{cc}
        \hline
        1/ & mAP  \\ \hline
        1         & unstable \\
        5         & 73.1 \\
        \ours{10} & \ours{73.3} \\
        15         & 73.0 \\
        \hline
        \end{tabular}
    \end{minipage}
\end{table}

\paragraph{More epochs and multi-dataset training}
Increasing the training epochs brings extra gains to the model performance. Specifically, training 270 and 420 epochs reach 73.5\% AP and 73.7\% AP respectively. To further exploit the model's potential, we enrich the training data by combining COCO~\cite{lin2014coco} and AI Challenger~\cite{JiahongWu2017AIC} datasets together for pre-training and fine-tuning, with a balanced sampling ratio. The performance finally achieves 75.3\% AP.

\subsection{Inference pipeline}
Beyond the pose estimation model, we further optimize the overall top-down inference pipeline for lower latency and better robustness. We use the skip-frame detection mechanism as in BlazePose~\cite{blazepose}, where human detection is performed every K frames, and in the interval frames the bounding boxes are generated from the last pose estimation results. Additionally, to achieve smooth prediction over frames, we use OKS-based pose NMS and OneEuro~\cite{casiez2012oneeuro} filter in the post-processing stage.

%
 

\section{Experiments}\label{sec:Experiments}

\begin{table*}[h]
	\begin{center}
 \caption{Body pose estimation results on COCO validation set. We only report GFLOPs of pose model and ignore the detection model. Flip test is not used.}
 
 \label{tab:compare_coco}
    \resizebox{\linewidth}{!}{
		\begin{tabular}{c|l|c|c|c|c|c|c|c}
		    \toprule
			\multicolumn{2}{c|}{Methods} 
            & Backbone & Detector & Det. Input Size & Pose Input Size & GFLOPs & AP & Extra Data  \\
			\midrule
			\multirow{6}{*}{PaddleDetection~\cite{ppdet2019}} 
            & TinyPose & Wider NLiteHRNet & YOLOv3 &  &  & 0.08 & 52.3  \\ 
			& TinyPose & Wider NLiteHRNet & YOLOv3 &  &  & 0.33 & 60.9  \\ 
            \cmidrule{2-8}
            & TinyPose & Wider NLiteHRNet & Faster-RCNN & N/A &  & 0.08 & 56.1 & AIC(220K) \\ 
			& TinyPose & Wider NLiteHRNet & Faster-RCNN & N/A &  & 0.33 & 65.6 &+Internal(unknown) \\ 
            \cmidrule{2-8}
			& TinyPose & Wider NLiteHRNet & PicoDet-s &  &  & 0.08 & 48.4  \\ 
			& TinyPose & Wider NLiteHRNet & PicoDet-s &  &  & 0.33 & 56.5  \\ 
            \midrule
            \multirow{6}{*}{AlphaPose~\cite{alphapose}} 
			& FastPose & ResNet 50 & YOLOv3 &  &  & 5.91 & 71.2 &\multirow{6}{*}{-} \\ 
			& FastPose(DUC) & ResNet-50 & YOLOv3 &  &  & 9.71 & 71.7 \\ 
			& FastPose(DUC) & ResNet-152 & YOLOv3 &  &  & 15.99 & 72.6 \\ 
            \cmidrule{2-8}
            & FastPose & ResNet 50 & Faster-RCNN & N/A &  & 5.91 & 69.7 \\ 
			& FastPose(DUC) & ResNet-50 & Faster-RCNN & N/A &  & 9.71 & 70.3 \\ 
			& FastPose(DUC) & ResNet-152 & Faster-RCNN & N/A &  & 15.99 & 71.3 \\ 
			\midrule
			\multirow{22}{*}{MMPose~\cite{mmpose2020}}
            & RTMPose-t & CSPNeXt-t & Faster-RCNN  & N/A &  & 0.36 & 65.8 &\multirow{4}{*}{-}\\ 
            & RTMPose-s & CSPNeXt-s & Faster-RCNN  & N/A &  & 0.68 & 69.6 \\ 
            & RTMPose-m & CSPNeXt-m & Faster-RCNN  & N/A &  & 1.93 & 73.6 \\ 
            & RTMPose-l & CSPNeXt-l & Faster-RCNN  & N/A &  & 4.16 & 74.8 \\ 
            \cmidrule{2-9}
            & RTMPose-t & CSPNeXt-t & YOLOv3 &  &  & 0.36 & 66.0&\multirow{18}{*}{AIC(220K)} \\ 
            & RTMPose-s & CSPNeXt-s & YOLOv3 &  &  & 0.68 & 70.3 \\ 
            & RTMPose-m & CSPNeXt-m & YOLOv3 &  &  & 1.93 & 74.7 \\ 
            & RTMPose-l & CSPNeXt-l & YOLOv3 &  &  & 4.16 & 75.7 \\ 
            \cmidrule{2-8}
            & RTMPose-t & CSPNeXt-t & Faster-RCNN & N/A &  & 0.36 & 67.1 \\ 
            & RTMPose-s & CSPNeXt-s & Faster-RCNN & N/A &  & 0.68 & 71.1 \\ 
            & RTMPose-m & CSPNeXt-m & Faster-RCNN & N/A &  & 1.93 & 75.3 \\ 
            & RTMPose-l & CSPNeXt-l & Faster-RCNN & N/A &  & 4.16 & 76.3 \\ 
            \cmidrule{2-8}
            & RTMPose-t & CSPNeXt-t & PicoDet-s &  &  & 0.36 & 64.3 \\ 
            & RTMPose-s & CSPNeXt-s & PicoDet-s &  &  & 0.68 & 68.8 \\ 
            & RTMPose-m & CSPNeXt-m & PicoDet-s &  &  & 1.93 & 73.2 \\ 
            & RTMPose-l & CSPNeXt-l & PicoDet-s &  &  & 4.16 & 74.2 \\ 
            \cmidrule{2-8}
            & RTMPose-t & CSPNeXt-t & RTMDet-nano &  &  & 0.36 & 64.4 \\ 
            & RTMPose-s & CSPNeXt-s & RTMDet-nano &  &  & 0.68 & 68.5 \\ 
            & RTMPose-m & CSPNeXt-m & RTMDet-nano &  &  & 1.93 & 73.2 \\ 
            & RTMPose-l & CSPNeXt-l & RTMDet-nano &  &  & 4.16 & 74.2 \\ 
            & RTMPose-m & CSPNeXt-m & RTMDet-m &  &  & 1.93 & 75.7 \\ 
            & RTMPose-l & CSPNeXt-l & RTMDet-m &  &  & 4.16 & 76.6 \\ 
			\bottomrule
		\end{tabular}
	}
	\end{center}
\end{table*}

\subsection{Settings}

The training settings in our experiments are shown in Tabel.~\ref{tab:train_setting}. As described in Sec.~\ref{sec:training_techniques}, we conduct a heatmap-based pre-training~\cite{Huang_2020_CVPR} which follows the same training strategies used in the fine-tuning except for shorter epochs. All our models are trained on 8 NVIDIA A100 GPUs. And we evaluate the model performance by mean Average Precision (AP).

\begin{table*}[h]
\begin{center}
 \caption{Body pose estimation results on COCO-SinglePerson validation set. We sum up top-down methods' GFLOPs of detection and pose for a fair comparison with bottom-up methods. ``*'' denotes double inference. Flip test is not used.}\label{tab:compare_coco_single}

    \resizebox{\linewidth}{!}{
		\begin{tabular}{c|l|c|c|c|c|c|c|c}
		    \toprule
			\multicolumn{2}{c|}{Methods} 
            & Backbone & Detector & Det. Input Size & Pose Input Size & GFLOPs & AP & Extra Data  \\
			\midrule
            \multirow{2}{*}{MediaPipe~\cite{blazepose}} 
            & BlazePose-Lite & BlazePose & N/A &  & N/A & N/A & 29.3  &\multirow{2}{*}{Internal(85K)}\\ 
			& BlazePose-Full & BlazePose & N/A &  & N/A & N/A & 35.4  \\ 
            \midrule
            \multirow{2}{*}{MoveNet~\cite{RonnyVotel2023NextGenerationPD}} 
            & Lightning & MobileNetv2 & N/A &  & N/A & 0.54 & 53.6*  &\multirow{2}{*}{Internal(23.5K)}\\ 
			& Thunder & MobileNetv2 depth & N/A &  & N/A & 2.44 &  64.8*  \\ 
            \midrule
            \multirow{2}{*}{PaddleDetection~\cite{ppdet2019}} 
& TinyPose & Wider NLiteHRNet & PicoDet-s &  &  & 0.55 & 58.6 & AIC(220K) \\ 
		&  TinyPose & Wider NLiteHRNet & PicoDet-s &  &  & 0.80     & 69.4 &+Internal(unknown)  \\ 
            \midrule
			\multirow{4}{*}{MMPose~\cite{mmpose2020}}
			& RTMPose-t & CSPNeXt-t & RTMDet-nano &  &  & 0.67 & 72.1 &\multirow{4}{*}{AIC(220K)}\\ 
            & RTMPose-s & CSPNeXt-s & RTMDet-nano &  &  & 0.91 & 77.1 \\ 
            & RTMPose-m & CSPNeXt-m & RTMDet-nano &  &  & 2.23 & 82.4 \\ 
            & RTMPose-l & CSPNeXt-l & RTMDet-nano &  &  & 4.47 & 83.5 \\ 
			\bottomrule
		\end{tabular}
	}
\end{center}
\end{table*}

\begin{table*}[h]
\begin{center}
 \caption{Whole-body pose estimation results on COCO-WholeBody~\cite{zoomnet,xu2022zoomnas} V1.0 dataset. We only report the input size and GFLOPs of pose models in top-down approaches and ignore the detection model. ``*'' denotes the model is pre-trained on AIC+COCO. ``\dag'' indicates multi-scale testing. Flip test is used.}\label{tab:compare_cocowhole}
    \vspace{3pt}
    \resizebox{0.9\linewidth}{!}{
		\begin{tabular}{c|l|c|c|cc|cc|cc|cc|cc}
			\toprule
			  & Method & Input Size & GFLOPs & \multicolumn{2}{c|}{whole-body} & \multicolumn{2}{c|}{body}  & \multicolumn{2}{c|}{foot}  & \multicolumn{2}{c|}{face}  & \multicolumn{2}{c}{hand} \\
			\cmidrule{5-14}
			& &   &   &  AP     & AR     & AP   & AR     &  AP  & AR     & AP    & AR   &  AP     & AR  \\
			\midrule
			Whole- & SN\dag~\cite{hidalgo2019single} & N/A & 272.3 & 32.7 & 45.6 & 42.7 & 58.3 & 9.9 & 36.9 & 64.9 & 69.7 & 40.8 & 58.0  \\ 
            body & OpenPose~\cite{openpose} & N/A & 451.1 & 44.2 & 52.3 & 56.3 & 61.2 & 53.2 & 64.5 & 76.5 & 84.0 & 38.6 & 43.3  \\ 
            \midrule
			Bottom- & PAF\dag~\cite{cao2017realtime} & 512512 & 329.1 & 29.5 & 40.5 & 38.1 & 52.6 & 5.3 & 27.8 & 65.6 & 70.1 & 35.9 & 52.8  \\ 
			up & AE~\cite{newell2017associative} & 512512 & 212.4 & 44.0 & 54.5 & 58.0 & 66.1 & 57.7 & 72.5 & 58.8 & 65.4 & 48.1 & 57.4  \\
			\midrule
            & DeepPose~\cite{Toshev_2014_CVPR} & 384288 & 17.3 & 33.5 & 48.4 & 44.4 & 56.8 & 36.8 & 53.7 & 49.3 & 66.3 & 23.5 & 41.0 \\
            & SimpleBaseline~\cite{xiao2018simple} & 384288 & 20.4 & 57.3 & 67.1 & 66.6 & 74.7 & 63.5 & 76.3 & 73.2 & 81.2 & 53.7 & 64.7 \\
			& HRNet~\cite{SunXLW19}  & 384288 & 16.0 & 58.6 & 67.4 & 70.1 & 77.3 & 58.6 & 69.2 & 72.7 & 78.3 & 51.6 & 60.4  \\
            & PVT~\cite{WenhaiWang2021PyramidVT} & 384288 & 19.7 & 58.9 & 68.9 & 67.3 & 76.1 & 66.0 & 79.4 & 74.5 & 82.2 & 54.5 & 65.4 \\
		Top- & FastPose50-dcn-si~\cite{alphapose} &  256192 & 6.1 & 59.2 & 66.5 & 70.6      & 75.6 & 70.2 & 77.5 & 77.5 & 82.5 & 45.7 & 53.9 \\
            down & ZoomNet~\cite{zoomnet} & 384288 & 28.5 & 63.0 & 74.2 & \textbf{74.5} &   \textbf{81.0} & 60.9 & 70.8 & 88.0 & 92.4 & 57.9 & 73.4   \\
            & ZoomNAS~\cite{xu2022zoomnas} & 384288 & 18.0 & \textbf{65.4} & \textbf{74.4} & 74.0 & 80.7 & 61.7 & 71.8 & 88.9 & \textbf{93.0} & \textbf{62.5} & \textbf{74.0} \\
            \cmidrule{2-14}
            & \ours{RTMPose-m*} & \ours{256192} & \ours{2.2} & \ours{58.2} & \ours{67.4} & \ours{67.3} & \ours{75.0} & \ours{61.5} & \ours{75.2} & \ours{81.3} & \ours{87.1} & \ours{47.5} & \ours{58.9}  \\
& \ours{RTMPose-l*} & \ours{256192} & \ours{4.5} & \ours{61.1} & \ours{70.0} & \ours{69.5} & \ours{76.9} & \ours{65.8} & \ours{78.5} & \ours{83.3} & \ours{88.7} & \ours{51.9} & \ours{62.8}  \\
& \ours{RTMPose-l*} & \ours{384288} & \ours{10.1} & \ours{64.8} & \ours{73.0} & \ours{71.2} & \ours{78.1} & \ours{\textbf{69.3}} & \ours{\textbf{81.1}} & \ours{88.2} & \ours{91.9} & \ours{57.9} & \ours{67.7} \\
            & \ours{RTMPose-x} & \ours{384288} & \ours{18.1} & \ours{65.2} & \ours{73.2} & \ours{71.2} & \ours{78.0} & \ours{68.1} & \ours{80.4} & \ours{\textbf{89.0}} & \ours{92.2} & \ours{59.3} & \ours{68.7} \\
            & \ours{RTMPose-x*} & \ours{384288} & \ours{18.1} & \ours{65.3} & \ours{73.3} & \ours{71.4} & \ours{78.4} & \ours{69.2} & \ours{81.0} & \ours{88.9} & \ours{92.3} & \ours{59.0} & \ours{68.5} \\
			\bottomrule
		\end{tabular}}
	\end{center}
\end{table*}

\begin{table}[h]
\small
\centering
\caption{Training settings for RTMPose models.}\label{tab:train_setting}
\vspace{3pt}
\begin{tabular}{l c}
\hline
optimizer & AdamW \\
base learning rate & 0.004 \\
learning rate schedule & Flat-Cosine \\
batch size & 1024 \\
warm-up iterations & 1000 \\
\hline
\multirow{2}{0.15\textwidth}{weight decay} & 0.05 (RTMPose-m/l)\\
& 0 (RTMPose-t/s)\\
\hline
\multirow{2}{0.15\textwidth}{EMA decay} & 0.9998 (RTMPose-s/m/l)\\
& no EMA (RTMPose-t)\\
\hline
\multirow{2}{0.15\textwidth}{training epochs} & 210 (pre-train)\\
& 420 (fine-tune)\\
\hline
\end{tabular}
\vspace{-5pt}
\end{table}

\subsection{Benchmark Results}

\begin{figure*}[htp]
    \centering
    \begin{minipage}[t]{0.48\textwidth}
    \centering
    \includegraphics[width=9cm]{figs/log_gflops.png}
\end{minipage}
    \begin{minipage}[t]{0.48\textwidth}
    \centering
    \includegraphics[width=9cm]{figs/coco-single.png}
\end{minipage}
    \caption{Comparison of GFLOPs and accuracy. Left: Comparison of RTMPose and other open-source pose estimation libraries on full COCO val set. Right: Comparison of RTMPose and other open-source pose estimation libraries on COCO-SinglePerson val set.}
    \label{fig:compare_gflops}
\end{figure*}

\paragraph{COCO}
COCO~\cite{lin2014coco} is the most popular benchmark for 2d body pose estimation. We follow the standard splitting of \texttt{train2017} and \texttt{val2017}, which contains 118K and 5k images for training and validation respectively. We extensively study the pose estimation performance with different off-the-shelf detectors including YOLOv3~\cite{JosephRedmon2018YOLOv3AI}, Faster-RCNN~\cite{ShaoqingRen2015FasterRT}, and RTMDet~\cite{lyu2022rtmdet}. To conduct a fair comparison with AlphaPose~\cite{alphapose} which doesn't use extra training data, we also report the performance of RTMPose only trained on COCO. As shown in Table~\ref{tab:compare_coco}, RTMPose outperforms competitors by a large margin with much lower complexity and shows strong robustness for detection.

\paragraph{COCO-SinglePerson}
Popular pose estimation open-source algorithms like BlazePose~\cite{blazepose}, MoveNet~\cite{RonnyVotel2023NextGenerationPD}, and PaddleDetection~\cite{ppdet2019} are designed primarily for single-person or sparse scenarios, which are practical in mobile applications and human-machine interactions. For a fair comparison, we construct a COCO-SinglePerson dataset that contains 1045 single-person images from the COCO \texttt{val2017} set to evaluate RTMPose as well as other approaches. For MoveNet, we follow the official inference pipeline to apply a cropping algorithm, namely using the coarse pose prediction of the first inference to crop the input image and performing a second inference for better pose estimation results. The evaluation results in Table~\ref{tab:compare_coco_single} show that RTMPose archives superior performance and efficiency even compared to previous solutions tailored for the single-person scenario.

\paragraph{COCO-WholeBody}
We also validate the proposed RTMPose model on the whole-body pose estimation task with COCO-WholeBody~\cite{zoomnet, xu2022zoomnas} V1.0 dataset. As shown in Table~\ref{tab:compare_cocowhole}, RTMPose achieves superior performance and well balances accuracy and complexity. Specifically, our RTMPose-m model outperforms previous open-source libraries~\cite{openpose, alphapose, Yulitehrnet21} with significantly lower GFLOPs. And by increasing the input resolution and training data we obtain competitive accuracy with SOTA approaches~\cite{zoomnet, xu2022zoomnas}.

\paragraph{Other Datasets}
As shown in Table~\ref{tab:compare_others} and Table~\ref{tab:compare_mpii}, we further evaluate RTMPose on AP-10K~\cite{HangYu2021AP10KAB}, CrowdPose~\cite{JiefengLi2018CrowdPoseEC} and MPII~\cite{MykhayloAndriluka20142DHP} datasets. We report the model performance using ImageNet~\cite{JiaDeng2009ImageNetAL} pre-training for a fair comparison with baselines. Besides we also report the performance of our models pre-trained using a combination of COCO~\cite{lin2014coco} and AI Challenger (AIC)~\cite{JiahongWu2017AIC}, which achieves higher accuracy and can be easily reproduced by users with our provided pre-trained weights.

\subsection{Inference Speed}
We perform the export, deployment, inference, and testing of models by MMDeploy~\cite{mmdeploy} to test the inference speed on CPU and GPU respectively. Table~\ref{tab:ncnn_speed} demonstrates the comparison of inference speed on the mobile device. We deploy RTMPose on the Snapdragon 865 chip with ncnn and inference with 4 threads. The TensorRT inference latency is tested in the half-precision floating-point format (FP16) on an NVIDIA GeForce GTX 1660 Ti GPU, and the ONNX latency is tested on an Intel I7-11700 CPU with ONNXRuntime with 1 thread. The inference batch size is 1. All models are tested on the same devices with 50 times warmup and 200 times inference for fair comparison. For TinyPose~\cite{ppdet2019}, we test it with both MMDeploy and FastDeploy, and note that ONNXRuntime speed on MMDeploy is slightly faster  (10.58 ms vs. 12.84 ms). The results are shown in Table~\ref{tab:onnx_trt_speed} and Table~\ref{tab:pipeline_speed}. 

\begin{table}[h]
	\begin{center}
 \caption{Performance on different datasets. ``*'' denotes the model is pre-trained on AIC+COCO and fine-tuned on the corresponding dataset. Flip test is used.}\label{tab:compare_others}
  \vspace{3pt}
    \resizebox{\linewidth}{!}{
		\begin{tabular}{c|l|c|c|c|c}
		    \toprule
			Dataset & Methods & Backbone & Input Size & GFLOPs & AP  \\
            \midrule
			\multirow{4}{*}{AP-10K~\cite{HangYu2021AP10KAB}} 
            & SimpleBaseline~\cite{xiao2018simple} & ResNet-50 &  & 7.28 & 68.0 \\ 
			& HRNet~\cite{SunXLW19} & HRNet-w32 &  & 10.27 & 72.2 \\ 
            & \ours{RTMPose-m} & \ours{CSPNeXt-m} & \ours{} & \ours{\textbf{2.57}} & \ours{\textbf{68.4}}    \\ & \ours{RTMPose-m*} & \ours{CSPNeXt-m} & \ours{} & \ours{\textbf{2.57}} & \ours{\textbf{72.2}}   \\ \midrule
            \multirow{4}{*}{CrowdPose~\cite{JiefengLi2018CrowdPoseEC}} 
            & SimpleBaseline~\cite{xiao2018simple} & ResNet-50 &  & 5.46 & 63.7 \\ 
			& HRNet~\cite{SunXLW19} & HRNet-w32 &  & 7.7 & 67.5 \\ 
& \ours{RTMPose-m} & \ours{CSPNeXt-m} & \ours{} & \ours{\textbf{1.93}} & \ours{\textbf{66.9}}  \\  & \ours{RTMPose-m*} & \ours{CSPNeXt-m} & \ours{} & \ours{\textbf{1.93}} & \ours{\textbf{70.6}} \\  \bottomrule
		\end{tabular}
	}
	\end{center}
\end{table}

\begin{table}[ht!]
	\begin{center}
 \caption{Comparison on MPII~\cite{MykhayloAndriluka20142DHP} validation set. ``*'' denotes the model is pre-trained on AIC+COCO and fine-tuned on MPII. Flip test is used.}\label{tab:compare_mpii}
  \vspace{3pt}
    \resizebox{\linewidth}{!}{
		\begin{tabular}{c|l|c|c|c|c}
		    \toprule
			Dataset & Methods & Backbone & Input Size & GFLOPs & PCKh@0.5  \\
            \midrule
			\multirow{6}{*}{MPII~\cite{MykhayloAndriluka20142DHP}} 
            & SimpleBaseline~\cite{xiao2018simple} & ResNet-50 &  & 7.28 & 88.2\\ 
			& HRNet~\cite{SunXLW19} & HRNet-w32 &  & 10.27 & 90.0 \\ 
            & SimCC~\cite{SimCC} & HRNet-w32 &  & 10.34 & 90.0 \\
            & TokenPose~\cite{li2021tokenpose} & L/D24 &  & 11.0 & 90.2 \\
			& \ours{RTMPose-m} & \ours{CSPNeXt-m} & \ours{} & \ours{\textbf{2.57}} & \ours{\textbf{88.9}}    \\ & \ours{RTMPose-m*} & \ours{CSPNeXt-m} & \ours{} & \ours{\textbf{2.57}} & \ours{\textbf{90.7}}   \\ \bottomrule
		\end{tabular}
	}
	\end{center}
\end{table}


\begin{table}[h]
\caption{Comparison of inference speed on Snapdragon 865. RTMPose models are deployed and tested using ncnn.}\label{tab:ncnn_speed}
 \vspace{-18pt}
	\begin{center}
    \resizebox{\linewidth}{!}{
		\begin{tabular}{c|l|c|c|c|c|c}
		    \toprule
			\multicolumn{2}{c|}{Methods} 
            & Input Size & GFLOPs & AP(GT) & FP32(ms) & FP16(ms)  \\
			\midrule
			\multirow{2}{*}{PaddleDetection~\cite{ppdet2019}} 
            & TinyPose &  & 0.08 & 58.4 & 4.57 & 3.27\\ 
			&  TinyPose &  & 0.33 & 68.3 & 14.07 & 8.33 \\ 
            \midrule
			\multirow{4}{*}{MMPose~\cite{mmpose2020}}
			& RTMPose-t &  & 0.36 & 68.4 & 15.84 & 9.02 \\ 
            & RTMPose-s &  & 0.68 & 72.8 & 25.01 & 13.89 \\ 
            & RTMPose-m &  & 1.93& 77.3 & 49.46 & 26.44\\ 
            & RTMPose-l &  & 4.16 & 78.3 & 85.75 & 45.37 \\ 
			\bottomrule
		\end{tabular}}
	\end{center}
\end{table}

\begin{table}[h]
\caption{Inference speed on CPU and GPU. RTMPose models are deployed and tested using ONNXRuntime and TensorRT respectively. Flip test is not used in this table. }\label{tab:onnx_trt_speed}
 \vspace{-3pt}
	\begin{center}
    \resizebox{\linewidth}{!}{
		\begin{tabular}{c|l|c|c|c|c|cc}
		    \toprule
			\multicolumn{2}{c|}{Results} 
            & Input Size & GFLOPs & AP & CPU(ms) & GPU(ms)  \\
			\midrule
			\multirow{7}{*}{COCO~\cite{lin2014coco}}
            & TinyPose &  & 0.33 & 65.6 & 10.580 & 3.055\\ 
            & LiteHRNet-30 &  & 0.42 & 66.3 & 22.750 & 6.561\\ 
			& \ours{RTMPose-t} & \ours{} & \ours{\textbf{0.36}} & \ours{\textbf{67.1}} & \ours{\textbf{3.204}} & \ours{\textbf{1.064}}\\ 
            & \ours{RTMPose-s} & \ours{} & \ours{\textbf{0.68}} & \ours{\textbf{71.2}} & \ours{\textbf{4.481}} & \ours{\textbf{1.392}}\\ 
            \cmidrule{2-7}
            & HRNet-w32+UDP &  & 7.7 & 75.1 & 37.734 & 5.133\\ 
            & \ours{RTMPose-m} & \ours{} & \ours{\textbf{1.93}}& \ours{\textbf{75.3}} & \ours{\textbf{11.060}} & \ours{\textbf{2.288}}\\ 
            & \ours{RTMPose-l} & \ours{} & \ours{\textbf{4.16}} & \ours{\textbf{76.3}} & \ours{\textbf{18.847}} & \ours{\textbf{3.459}}\\ 
            \midrule
			\multirow{5}{*}{COCO-}
             & HRNet-w32+DARK &  & 7.72& 57.8 & 39.051 & 5.154\\ 
            & \ours{RTMPose-m} & \ours{} & \ours{\textbf{2.22}}& \ours{\textbf{59.1}} & \ours{\textbf{13.496}} & \ours{\textbf{4.000}}\\ 
            & \ours{RTMPose-l} & \ours{} & \ours{\textbf{4.52}} & \ours{\textbf{62.2}} & \ours{\textbf{23.410}} & \ours{\textbf{5.673}}\\ 
            \cmidrule{2-7}
            WholeBody~\cite{zoomnet}& HRNet-w48+DARK &  & 35.52 & 65.3 & 150.765 & 13.974\\ 
            & \ours{RTMPose-l} & \ours{} & \ours{\textbf{10.07}} & \ours{\textbf{66.1}} & \ours{\textbf{44.581}}  & \ours{\textbf{7.678}}\\ 
			\bottomrule
		\end{tabular}}
	\end{center}
\end{table}

\begin{table}[h]
\caption{Pipeline Inference speed on CPU, GPU and Mobile device.}\label{tab:pipeline_speed}
	\begin{center}
    \resizebox{\linewidth}{!}{
		\begin{tabular}{c|l|c|c|c|c|c}
		    \toprule
            Model & Input Size & GFLOPs & Pipeline AP & CPU(ms) & GPU(ms) & Mobile(ms) \\
			\midrule
            RTMDet-nano &  & 0.31 & \multirow{2}{*}{64.4} & \multirow{2}{*}{12.403} &  \multirow{2}{*}{2.467} &  \multirow{2}{*}{18.780}\\ 
            RTMPose-t &  & 0.36 &  &  & \\ 
            \midrule
            RTMDet-nano &  & 0.31 & \multirow{2}{*}{68.5} & \multirow{2}{*}{16.658} &  \multirow{2}{*}{2.730} &  \multirow{2}{*}{21.683}\\ 
            RTMPose-s &  & 0.42 &  &  & \\ 
            \midrule
            RTMDet-nano &  & 0.31 & \multirow{2}{*}{73.2} & \multirow{2}{*}{26.613} &  \multirow{2}{*}{4.312} &  \multirow{2}{*}{32.122}\\ 
            RTMPose-m &  & 1.93 &  &  & \\ 
            \midrule
            RTMDet-nano &  & 0.31 & \multirow{2}{*}{74.2} & \multirow{2}{*}{36.311} &  \multirow{2}{*}{4.644} &  \multirow{2}{*}{47.642}\\ 
            RTMPose-l &  & 4.16 &  &  & \\ 
            \bottomrule
		\end{tabular}}
	\end{center}
\end{table} \section{Conclusion}\label{sec:Conclusion}

This paper empirically explores key factors in pose estimation such as the paradigm, model architecture, training strategy, and deployment. Based on the findings we present a high-performance real-time multi-person pose estimation framework, RTMPose, which achieves excellence in balancing model performance and complexity and can be deployed on various devices (CPU, GPU, and mobile devices) for real-time inference. We hope that the proposed algorithm alone with its open-sourced implementation can meet some of the demand for applicable pose estimation in industry, and benefit future explorations on the human pose estimation task.
 

{\small
\bibliographystyle{ieee_fullname}
\bibliography{egbib}
}

\end{document}
