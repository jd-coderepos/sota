\documentclass[sigconf]{acmart}


\AtBeginDocument{\providecommand\BibTeX{{\normalfont B\kern-0.5em{\scshape i\kern-0.25em b}\kern-0.8em\TeX}}}

\usepackage{algorithm}
\usepackage{algorithmic}
\usepackage{caption}
\usepackage{subcaption}
\usepackage{makecell}

\usepackage{multirow}
\setcopyright{acmcopyright}
\copyrightyear{2018}
\acmYear{2018}
\acmDOI{XXXXXXX.XXXXXXX}

\acmConference[Conference acronym 'XX]{Make sure to enter the correct
  conference title from your rights confirmation emai}{June 03--05,
  2018}{Woodstock, NY}
\acmPrice{15.00}
\acmISBN{978-1-4503-XXXX-X/18/06}








\begin{document}

\title{Select and Trade: Towards Unified Pair Trading with Hierarchical Reinforcement Learning}

\author{Weiguang Han}
\email{han.wei.guang@whu.edu.cn
}
\author{Boyi Zhang}
\email{zhangby@whu.edu.cn
}
\author{Min Peng}
\email{pengm@whu.edu.cn
}
\affiliation{\institution{Wuhan University}
  \city{Wuhan}
  \state{Hubei}
  \country{China}
  \postcode{430072}
}
\author{Qianqian Xie}
\email{qianqian.xie@manchester.ac.uk
}
\affiliation{\institution{University of Manchester}
  \city{Manchester}
  \country{United Kingdom}
}

\author{Yanzhao Lai}
\email{laiyanzhao@swjtu.edu.cn
}
\affiliation{\institution{Southwest Jiaotong University}
  \city{Chengdu}
  \state{Sichuan}
  \country{China}
}
\author{Jimin Huang}
\authornotemark[1]
\email{jimin@chancefocus.com
}
\affiliation{\institution{Chancefocus AMC.}
  \city{Shanghai}
  \country{China}
}

\renewcommand{\shortauthors}{Han, et al.}

\begin{abstract}
    Pair trading is one of the most effective statistical arbitrage strategies which seeks a neutral profit by hedging a pair of selected assets. Existing methods generally decompose the task into two separate steps: pair selection and trading. However, the decoupling of two closely related sub-tasks can block information propagation and lead to limited overall performance. For pair selection, ignoring the trading performance results in the wrong assets being selected with irrelevant price movements, while the agent trained for trading can overfit to the selected assets without any historical information of other assets. To address it, in this paper, we propose a paradigm for automatic pair trading as a unified task rather than a two-step pipeline. We design a hierarchical reinforcement learning framework to jointly learn and optimize two sub-tasks. A high-level policy would select two assets from all possible combinations and a low-level policy would then perform a series of trading actions. Experimental results on real-world stock data demonstrate the effectiveness of our method on pair trading compared with both existing pair selection and trading methods.
\end{abstract}

\begin{CCSXML}
<ccs2012>
   <concept>
       <concept_id>10010405.10010481.10010487</concept_id>
       <concept_desc>Applied computing~Forecasting</concept_desc>
       <concept_significance>500</concept_significance>
       </concept>
   <concept>
       <concept_id>10010147.10010257.10010293.10010317</concept_id>
       <concept_desc>Computing methodologies~Partially-observable Markov decision processes</concept_desc>
       <concept_significance>500</concept_significance>
       </concept>
 </ccs2012>
\end{CCSXML}

\ccsdesc[500]{Applied computing~Forecasting}
\ccsdesc[500]{Computing methodologies~Partially-observable Markov decision processes}
	
\keywords{pair trading, hierarchical reinforcement learning, pair selection, automatic trading}



\received{20 February 2007}
\received[revised]{12 March 2009}
\received[accepted]{5 June 2009}

\maketitle

\section{Introduction}
Since 1987, pair trading, a basic statistical arbitrage approach, has been intensively practised and studied~\cite{Lehoczky2018OverviewAH}.
It is an integral component of the financial market and plays a crucial role in enhancing market efficiency~\cite{Shearer2021StabilityEO}.
On worldwide markets and varied asset types, such as stocks, futures, and cryptocurrency, it has been argued that pair trading is effective in the long term~\cite{Krauss2017}.
In contrast to portfolio selection to find the optimal portfolio with the highest ``risky profit''~\cite{Hunanyan2019PortfolioS}, it seeks ``riskless profit'' by performing arbitrage tradings on the abnormal price movements of two correlated assets~\cite{Lehoczky2018OverviewAH}.
It first picks two correlated assets and monitors the spread between their respective prices. If the spread widens abnormally, it will execute trading operations on two assets and gain a profit when the spread recovers to its usual value. For instance, if Google's price is usually \5 higher, the strategy will short Google (expect its price to fall) and long Facebook (expect its price to increase) and close two tradings when the spread returns to \T_\mathcal{F}\{0,1,\dots,T_\mathcal{F}-1\}T_\mathcal{T}\{0,1,\dots,T_\mathcal{T}-1\}N\mathcal{X} = \{x_1, x_2,\dots,x_{N}\}\{p_0^x, p_1^x,\dots,p_{T_\mathcal{F}-1}^x\}\{p_0^x, p_1^x,\dots,p_{T_\mathcal{T}-1}^x\}x \in \mathcal{X}M = (S^h, O, T^h, R^h, \Psi, Q^h)S^hOT^hR^h\Psis^h \in S^ho \in OQ^h(s^h, o)o_0 \in Os^h_0s^h_0s^h_1T(s^h_1|s^h_1, o_1)v_0 \in \PsiQ(s^h_1, o_0)v_0 \in \Psix \in \mathcal{X}t \in T_fp^o_{x,t}p^c_{x,t}vol_{x,t}o(x_i, x_j)\mathcal{X}v^h_0p^o_{x,t}p^c_{x,t}vol_{x,t}x \in \mathcal{X}t \in T_{\mathcal{F}}h_{t-1}h_{t}h_th_t \in \mathbb{R}^{d_h}{d_h}v^h_{0, t} \in \mathbb{R}^{N \times 3}t \in T_\mathcal{F}score(h_{T_\mathcal{F}}, h_k) = \frac{h_{T_\mathcal{F}} h_k}{\sqrt{d_h}}\hat{h}_{T_\mathcal{F}} \in \mathbb{R}^{N \times d_h}s^h_0 \in \mathbb{R}^{N \times d_h}\mu: \mathcal{S} \rightarrow \mathcal{O}triuo_0T_\mathcal{T}R^h_tM = (S^l, A, T^l, R^l, \Omega, Q^l)S^lAT^lR^l\Omegas^l \in S^la \in AQ^l(s^l, a)a_t \in As^l_ts^l_ts^l_{t+1}T^l(s^l_{t+1}|s^l_t, a_t)v^l_{t+1} \in \OmegaQ(s^l_{t+1}, a_{t})v^l_t \in \Omegav^a_t \in \Omega^aa_{t-1}C_tV_tN_to^p_t \in \Omega^pp^o_{i,t}p^c_{i,t}vol_{i,t}i \in \{X, Y\}A = \{L, C, S\} = \{1, 0, -1\}\{X, Y\}XYXYs^l_tH_t = \{v^l_1, a_1, v^l_2, \cdots, a_{t-1}, v^l_t\}H_th_{t-1}h_{t}h_t \in \mathbb{R}^{d_h}{d_h}v^h_{0, t} \in \mathbb{R}^{2 \times M}Ma_{t-1}v^a_tE_a \in \mathbb{R}^{3 \times d_a}d_ascore(h_t, h_k) = \frac{h_t h_k}{\sqrt{d_h}}\hat{h}_t \in \mathbb{R}^{d_h}s^l_t \in \mathbb{R}^{d_h}\pi: \mathcal{S} \rightarrow \mathcal{A}s^l_to_0TR_ta_tr_{X,t} - r_{Y,t}A(s^h_0; \theta_h^A)=r_1^h + \gamma V(s_1^h; \theta_h^{A-}) - V(s^h_0; \theta_h^A)o_0\mu(o_0|s^h_0; \theta_h^P)A(s^l_t; \theta_l^A)=r_{t+1}^l + \gamma V(s_{t+1}^l; \theta_l^{A-}) - V(s_t^l; \theta_l^A)a_t\pi(a_t|s^l_t;o_0, \theta_l^P)N\mathcal{X}M, N\theta_{h} = \{\theta_h^P, \theta_h^A\}, \theta_{l} = \{\theta_l^P, \theta_l^A\}\theta_{h}, \theta_{l}o_0\mu(o_0|s^h_0)\mathcal{X}a_t\pi(a_t|s^l_t;o_0)\theta_l\theta_h(E(R_t)-R_f)/V(R_t)R_tR_f\Uparrow\Uparrow\Uparrow\Downarrow\Downarrow\Uparrow\Uparrow\Uparrow\Downarrow\Downarrow\Uparrow\Uparrow\Uparrow\Downarrow\Downarrow\Uparrow\Uparrow\Uparrow\Downarrow\Downarrow\Uparrow\Uparrow\Uparrow\Downarrow\Downarrow\Uparrow\Uparrow\Uparrow\Downarrow\Downarrow\Uparrow\Uparrow\Uparrow\Downarrow\Downarrow\Uparrow\Uparrow\Uparrow\Downarrow\Downarrow\Uparrow\Uparrow\Uparrow\Downarrow\Downarrow\Uparrow\Uparrow\Uparrow\Downarrow\Downarrow\Uparrow\Uparrow\Uparrow\Downarrow\Downarrow\Uparrow\Uparrow\Uparrow\Downarrow\Downarrow\Uparrow\Uparrow\Uparrow\Downarrow\Downarrow\Uparrow\Uparrow\Uparrow\Downarrow\Downarrow$ & \textbf{\begin{tabular}[c]{@{}l@{}}0.013\\ (6e-3)\end{tabular}} & \begin{tabular}[c]{@{}l@{}}0.017\\ (8e-3)\end{tabular} & \begin{tabular}[c]{@{}l@{}}0.015\\ (8e-3)\end{tabular} & \begin{tabular}[c]{@{}l@{}}0.017\\ (8e-3)\end{tabular} & \begin{tabular}[c]{@{}l@{}}0.046\\ (0.02)\end{tabular} & \begin{tabular}[c]{@{}l@{}}0.02\\ (8e-3)\end{tabular} \\\bottomrule
\end{tabular}
\caption{Trading performance on CSI 300.}
	\label{overall-result-cn}
\end{table*}

\end{document}
