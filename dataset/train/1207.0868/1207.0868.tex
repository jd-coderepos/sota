\subsection{A vocabulary ${\bf L}$}\label{sec:alpha}

\noindent{\em Symbols of ${\bf L}$}: We fix a vocabulary ${\bf L}$
that includes a set $\bL^{\ca{V}}$ of variable symbols (denoted $v$,
$v_1$ \etc), a set $\bL^{\ca{F}}$ of function symbols (denoted $f$,
$f_1$ \etc), a set $\bL^{\ca{B}}$ of predicate symbols (denoted $B$,
$B_1$ \etc), and a non-empty set $\Type$ of {\em sorts}.  $\Type$
contains the special sort ${\tt bool}$,  along with the special sort
{\tt location}. Each variable $v$ has associated with it a sort in
$\Type$, denoted $sort(v)$. Each function symbol $f$ has an associated
arity and a sort: $sort(f)$ for an $m$-ary function symbol is an $m +
1$-tuple $<\sigma_1,\ldots, \sigma_m, \sigma>$ of sorts in $\Type$,
specifying the sorts of both the domain and range of $f$.  Each
predicate symbol $B$ also has an associated arity and sort: $sort(B)$
for an $m$-ary predicate symbol is an $m$-tuple $<\sigma_1,\ldots,
\sigma_m>$ of sorts in $\Type$.  Constant symbols (denoted $c$, $c_1$
\etc) are identified as the $0$-ary function symbols, with each
constant symbol $c$ associated with a sort, denoted $sort(c)$, in
$\Type$.  The vocabulary ${\bf L}$ also explicitly includes the
distinguished equality predicate symbol $=$, used for comparing
elements of the same sort. \\

\noindent{\em Syntax of ${\bf L}$-terms and ${\bf L}$-atoms}: 
Given any set of variables $V \subseteq \bL^{\ca{V}}$, we inductively
construct the set of $\bL$-terms and $\bL$-atoms over $V$, 
using sorted symbols, as follows:

\begin{itemize}
\item Every variable of sort $\sigma$ is a term of sort $\sigma$.
\item If $f$ is a function symbol of sort
$<\sigma_1,\ldots,\sigma_m,\sigma>$, and $t_j$ is a term of sort 
$\sigma_j$ for $j \in [1,m]$, then $f(t_1,\ldots, t_m)$ is a 
term of sort $\sigma$.
In particular, every constant of sort $\sigma$ is a term of sort 
$\sigma$.
\item If $B$ is a predicate symbol of sort $<\sigma_1, \ldots,
\sigma_m>$, and $t_j$ is a term of sort $\sigma_j$ for $j \in
[1,m]$, then $B(t_1,\ldots, t_m)$ is an atom.
\item If $t_1$, $t_2$ are terms of the same sort, $t_1 = t_2$ 
is an atom.
\end{itemize}

\noindent{\em Semantics of ${\bf L}$-terms and  ${\bf L}$-atoms}: 
Given any set of variables $V \subseteq \bL^{\ca{V}}$, an {\em
interpretation} $I$ of symbols of $\bL$, and $\bL$-terms and $\bL$-atoms
over $V$ is a map satisfying the following: 

\begin{itemize}
\item Every sort $\sigma \in \Type$ is mapped to a nonempty domain
$D_{\sigma}$. In particular, the sort ${\tt bool}$ is mapped to the
Boolean domain $D^{\tt bool}:\{\true, \false\}$, and the sort {\tt location} is
mapped to a domain of {\em control locations} in a program. 
\item Every variable symbol $v$ of sort $\sigma$ is mapped to an
element $v^{I}$ in $D_{\sigma}$.
\item Every function symbol $f$, of sort $<\sigma_1, \ldots, \sigma_m,
\sigma>$ is mapped to a function $f^{I}: D_{\sigma_1} \times \ldots D_{\sigma_m} \to
D_{\sigma}$. In particular, every constant symbol $c$ of sort $\sigma$
is mapped to an element $c^{I} \in D_{\sigma}$. 
\item Every predicate symbol $B$ of sort $<\sigma_1 \ldots \sigma_m>$ 
is mapped to a function $D_{\sigma_1} \times \ldots D_{\sigma_m} \to
D^{\tt bool}$. 
\end{itemize}

Given an interpretation $I$ as defined above, the valuation
$\val^I[t]$ of an $\bL$-term $t$ and the valuation $\val^I[G]$ of an
$\bL$-atom $G$ are defined as follows:

\begin{itemize}
\item For a term $t$ which is a variable $v$, the valuation is $v^I$.
\item For a term $f(t_1, \ldots, t_m)$, the valuation $\val^I[f(t_1, \ldots,
t_m)] = f^I(\val^I[t_1], \ldots, \val^I[t_m])$. 
\item For an atom $G(t_1, \ldots, t_m)$, the valuation $\val^I[G(t_1, \ldots,
t_m)] = \true$ iff $G^I(\val^I[t_1], \ldots, \val^I[t_m]) = \true$. 
\item For an atom $t_1 = t_2$, $\val^I[t_1 = t_2] = \true$ iff 
$\val^I[t_1] = \val^I[t_2]$.  
\end{itemize}

In the rest of the paper, we assume that the interpretation of
constant, function and predicate symbols in $\bL$ is known and fixed.
We further assume that the interpretation of sort symbols to specific
domains is known and fixed. With some abuse of notation, we shall
denote the interpretation of all constant, function and predicate
symbols simply by the symbol name, and identify sorts with their
domains. Examples of some constant, function and predicate symbols
that may be included in $\bL$ are: constant symbols $0, 1, 2$, function
symbols $+,-$, and predicate symbols $<,>$ over the integers, function
symbols $\vee, \neg$ over ${\tt bool}$, the constant symbol $\varphi$
(empty list), function symbol $\bullet$ (appending lists) and
predicate symbol $null$ (emptiness test) over lists, \etc. Finally,
when the interpretation is obvious from the context, we denote the
valuations $\val^I[t]$, $\val^I[G]$ of terms $t$ and atoms $G$ simply as
$\val[t]$, $\val[G]$, respectively.  





\subsection{Concurrent Programs}\label{sec:prog}

In our framework, we consider a (shared-memory) concurrent program to
be an asynchronous composition of a non-empty, finite set of
processes, equipped with a finite set of program variables that range
over finite domains. We assume a simple concurrent programming
language with assignment, condition test, unconditional goto,
sequential and parallel composition, and the synchronization primitive
- conditional critical region (CCR) \cite{Hoare71,Hansen81}. A concurrent
program $P$ is written using the concurrent programming language, in
conjunction with $\bL$-terms and $\bL$-atoms. We assume that the sets
of (data and control) variables, functions and predicates available
for writing $P$ are each finite subsets of $\bL^{\ca{V}}$,
$\bL^{\ca{F}}$ and $\bL^{\ca{B}}$, respectively. 



A concurrent program is given as $P:: [{\tt declaration}] \, [P_1 {\tt
\parallel} \ldots {\tt \parallel} P_k]$, with $k > 0$. The declaration
consists of a finite sequence of declaration statements, specifying
the set of shared data variables $X$, their domains, and possibly
initializing them to specific values. For example, the declaration
statement, $v_1,v_2:\{0,1,2,3\} \; {\tt with} \; v_1 = 0$, declares two
variables $v_1$, $v_2$, each with (a finite integer) domain
$\{0,1,2,3\}$, and initializes the variable $v_1$ to the value $0$.
The initial value of any uninitialized variable is assumed to be a
user/environment input from the domain of the variable. 

A process $P_i$ consists of a declaration of local data variables
$Y_i$ (similar to the declaration of shared data variables in $P$),
and a finite sequence of labeled, {\em atomic} instructions, $l:
inst$.  We denote the unique instruction at location $l$ as $inst(l)$.
The set of data variables $Var_i$ accessible by $P_i$ is given by $X
\cup Y_i$. The set of labels or {\em locations} of $P_i$ is denoted
$L_i = \{l_i^0, \ldots, l_i^{n_i}\}$, with $l_i^0$ being a designated
start location. Unless specified otherwise\footnote{A user may define
an atomic instruction (block) as a sequence of assignment, conditional 
and goto statements}, an atomic instruction $inst$ is an assignment,
condition test, unconditional goto, or CCR. An assignment instruction
$A$, given by $(v_{i_1}, \ldots, v_{i_q}) \, {\bf \assign} \, (t_1,
\ldots, t_q)$, is a parallel assignment of $\bL$-terms $t_1, \ldots,
t_q$, over $Var_i$, to the data variables $v_{i_1}, \ldots, v_{i_q}$
in $Var_i$. Upon completion, an assignment statement at $l_i^r$
transfers control to the next location $l_i^{r+1}$. A condition test,
${\tt if} \; (G) \; l_{if},\, l_{else}$, consists of an $\bL$-atom $G$
over $Var_i$, and a pair of locations $l_{if}, l_{else}$ in $L_i$ to
transfer control to if $G$ evaluates to $\true$, $\false$,
respectively. The instruction ${\tt goto} \; l$ is a transfer of
control to location $l \in L_i$. A CCR is a guarded insruction block,
$G \to inst\_block$, where the enabling guard $G$ is an $\bL$-atom
over $Var_i$ and $inst\_block$ is a sequence of assignment,
conditional and goto statements. The guard $G$ is evaluated atomically
and if found to be $\true$, the corresponding $inst\_block$ is
executed atomically, and control is transferred to the next location.
If $G$ is found to be $\false$, the process {\em waits} at the same
location till $G$ evaluates to $\true$. An unsynchronized process does
not contain CCRs.  





We model the asynchronous composition of concurrent processes by the
nondeterministic interleaving of their atomic instructions. Hence, at
each step of the computation, some process, with an enabled
transition, is nondeterministically selected to be executed next by a
scheduler. The set of program variables is denoted $V = Loc \cup Var$,
where $Loc = \{loc_1, \ldots, loc_k\}$ is the set of control variables
and $Var = Var_1 \cup \ldots \cup Var_k$ is the set of data variables.
The semantics of the concurrent program $P$ is given by a transition
system $(S, S^0, R)$, where $S$ is a set of states, $S_0 \subseteq S$ is
a set of initial states and $R \subseteq S \times S$ is the transition relation. Each state $s
\in S$ is a valuation of the program variables in $V$. We denote
the value of variable $v$ in state $s$ as $\val^s[v]$, and the
corresponding value of a term $t$ and an atom $G$ in state $s$ as
$\val^s[t]$ and $\val^s[G]$, respectively. $\val^s[t]$ and $\val^s[G]$
are defined inductively as in \secref{alpha}. The domain of each
control variable $loc_i \in V$ is the set of locations $L_i$, and the
domain of each data variable is determined from its declaration.
The set of initial states $S_0$ corresponds to all states $s$
with $\val^s[loc_i] = l_i^0$ for all $i \in [1,k]$, and $\val^s[v] =
\vinit$, for every data variable $v$ initialized in its declaration
to some constant $\vinit$.  There exists a transition 
from state $s$ to $s'$ in $R$, with $\val^s[loc_i] =
l_i$, $val^{s'}[loc_i] = l'_i$ and $\val^{s'}[loc_j] = \val^s[loc_j]$
for all $j \neq i$, iff there exists a corresponding {\em local move} in
process $P_i$ involving instruction $inst(l_i)$, such that:
\begin{enumerate}
\item $inst(l_i)$ is the
assignment instruction: $(v_{i_1}, \ldots, v_{i_q}) \, {\bf \assign}
\, (t_1,\ldots, t_q)$, for each variable $v_{i_j}$ with $j \in [1,q]$: 
$\val^{s'}[v_{i_j}] = \val^s[t_j]$, for all other data variables $v$:
$\val^{s'}[v] = \val^s[v]$, and $l'_i$ is the next location in $P_i$
after $l_i$, or, 
\item $inst(l_i)$ is the condition test: ${\tt if} \;
(G) \; l_{if},\, l_{else}$, the valuation of all data variables in
$s'$ is the same as that in $s$, and either $\val^s[G]$ is $\true$ and
$l'_i = l_{if}$, or $\val^s[G]$ is $\false$ and $l'_i = l_{else}$, or,
\item  $inst(l_i)$ is ${\tt goto} \; l$,  the valuation of all data
variables in $s'$ is the same as that in $s$, and $l'_i = l$, or, 
\item $inst(l_i)$ is the CCR $G \to inst\_block$, $\val^s[G]$ is $\true$,
the valuation of all data variables in $s'$ correspond to the atomic
execution of $inst\_block$ from state $s$, and $l'_i$ is the next
location in $P_i$ after $l_i$.  
\end{enumerate}

\noindent We assume that $R$ is total. For terminating processes
$P_i$, we assume that $P_i$ ends with a special instruction,
$halt: {\tt goto} \; halt$.  

\subsection{Specifications}\label{sec:spec}

Our specification language, \vCTL, is an extension of propositonal
CTL, with formulas composed from ${\bf L}$-atoms.  While one can use
propositional CTL for specifying properties of finite-state programs,
\vCTL enables more natural specification of properties of concurrent
programs communicating via typed shared variables.  We describe the
syntax and semantics of this language below.  \\

\noindent{\em Syntax}: 
Given a set of variables $V \subseteq
\bL^{\ca{V}}$, we inductively construct the set of (\vCTL) {\em formulas}
over $V$, using ${\bf L}$-atoms, in conjunction with the
propositional operators $\neg, \vee$ and the temporal operators
$\textsf{A},\tf{E},\tf{X},\tf{U}$,
along with the process-indexed next-time operator $\tf{X}_i$:

\begin{itemize}
\item Every ${\bf L}$-atom over $V$ is a formula. 
\item If $\phi_1$, $\phi_2$ are formulas, then so are $\neg \phi_1$
and $\phi_1 \vee \phi_2$. 
\item If $\phi_1$, $\phi_2$ are formulas, then so are $\tf{E}\tf{X} \, \phi_1$,
$\tf{E}\tf{X}_i \, \phi_1$, $\tf{A}[\phi_1 \, \tf{U} \, \phi_2]$ and
$\tf{E}[\phi_1 \, \tf{U} \, \phi_2]$.
\end{itemize}

We use the following standard abbreviations: $\phi_1 \wedge \phi_2$
for $\neg(\neg \phi_1 \vee \neg \phi_2)$, $\phi_1 \to \phi_2$ for
$\neg \phi_1 \vee \phi_2$, $\phi_1 \lr \phi_2$ for $(\phi_1 \to
\phi_2) \wedge (\phi_2 \to \phi_1)$, $\tf{A}\tf{X} \, \phi$ for $\neg
\tf{E}\tf{X} \, \neg
\phi$, $\tf{A}\tf{X}_i \, \phi$ for $\neg \tf{E}\tf{X}_i \, \neg \phi$,
$\tf{A}\tf{F} \, \phi$ for
$\tf{A}[\true \, \tf{U} \, \phi]$, $\tf{E}\tf{F} \, \phi$ for $\tf{E}[\true
\, \tf{U} \, \phi]$, $\tf{E}\tf{G}
\, \phi$ for $\neg \tf{A}\tf{F} \, \neg \phi$, and  $\tf{A}\tf{G} \, \phi$
for $\neg \tf{E}\tf{F} \,
\neg \phi$.\\

\noindent{\em Semantics}: \vCTL formulas over a set of variables $V$
are interpreted over models of the form $M = (S, R, L)$, where $S$ is
a set of states and $R$ is a a total, multi-process, binary relation
$R = \cup_i R_i$ over $S$, composed of the transitions $R_i$ of 
each process $P_i$. $L$ is a labeling function that assigns to each
state $s \in S$ a valuation of all variables in $V$.  The value of a
term $t$ in a state $s \in S$ of $M$ is denoted as $\val^{(M,s)}[t]$,
and is defined inductively as in \secref{alpha}. A path in $M$ is a
sequence $\pi = (s_0, s_1, \ldots)$ of states such that $(s_j,
s_{j+1}) \in R$, for all $j \geq 0$. We denote the $j^{th}$ state in
$\pi$ as $\pi_j$. 



The satisfiability of a \vCTL formula in a state $s$ of $M$
can be defined as follows:

\begin{itemize}
\item $M,s \models G(t_1, \ldots, t_m)$ iff  $G(\val^{(M,s)}[t_1], \ldots,
\val^{(M,s)}[t_m]) = \true$. 
\item $M,s \models t_1 = t_2$ iff $\val^{(M,s)}[t_1] =
\val^{(M,s)}[t_2]$.
\item $M,s \models \neg \phi$ iff it is not the case that $M,s \models
\phi$.
\item $M,s \models \phi_1 \vee \phi_2$ iff $M,s \models \phi_1$ or
$M,s \models \phi_2$.
\item $M,s \models EX \, \phi$ iff for some $s_1$ such that  $(s,s_1)
\in R$, $M,s_1 \models \phi$. 
\item $M,s \models EX_i \, \phi$ iff for some $s_1$ such that
$(s,s_1) \in R_i$, $M,s_1 \models \phi$. 
\item $M,s \models A[\phi_1 \, U \, \phi_2]$ iff for all paths $\pi$
starting at $s$, $\exists j \, [M,\pi_j \models \phi_2 \text{ and }
\forall k \, (k < j \, \to \, M,\pi_k \models \phi_1)]$.
\item $M,s \models E[\phi_1 \, U \, \phi_2]$ iff there exists a path $\pi$
starting at $s$ such that $\exists j \, [M,\pi_j \models \phi_2 \text{ and }
\forall k \, (k < j \to M,\pi_k \models \phi_1)]$.\\
\end{itemize}



\noindent{\em Programs as Models}:
A program $P = (S,S^0,R)$ can be viewed as a model $M = (S,R,L)$, with
the same set of states and transitions as $P$, and the identity labeling
function $L$ that maps a state to itself. Given an $\vCTL$
specification $\phi$, we say $P \models \phi$ iff for each state $s
\in S^0$, $M,s \models \phi$. 



