\subsection{A vocabulary }\label{sec:alpha}

\noindent{\em Symbols of }: We fix a vocabulary 
that includes a set  of variable symbols (denoted ,
 \etc), a set  of function symbols (denoted ,
 \etc), a set  of predicate symbols (denoted ,
 \etc), and a non-empty set  of {\em sorts}.  
contains the special sort ,  along with the special sort
{\tt location}. Each variable  has associated with it a sort in
, denoted . Each function symbol  has an associated
arity and a sort:  for an -ary function symbol is an -tuple  of sorts in ,
specifying the sorts of both the domain and range of .  Each
predicate symbol  also has an associated arity and sort: 
for an -ary predicate symbol is an -tuple  of sorts in .  Constant symbols (denoted , 
\etc) are identified as the -ary function symbols, with each
constant symbol  associated with a sort, denoted , in
.  The vocabulary  also explicitly includes the
distinguished equality predicate symbol , used for comparing
elements of the same sort. \\

\noindent{\em Syntax of -terms and -atoms}: 
Given any set of variables , we inductively
construct the set of -terms and -atoms over , 
using sorted symbols, as follows:

\begin{itemize}
\item Every variable of sort  is a term of sort .
\item If  is a function symbol of sort
, and  is a term of sort 
 for , then  is a 
term of sort .
In particular, every constant of sort  is a term of sort 
.
\item If  is a predicate symbol of sort , and  is a term of sort  for , then  is an atom.
\item If ,  are terms of the same sort,  
is an atom.
\end{itemize}

\noindent{\em Semantics of -terms and  -atoms}: 
Given any set of variables , an {\em
interpretation}  of symbols of , and -terms and -atoms
over  is a map satisfying the following: 

\begin{itemize}
\item Every sort  is mapped to a nonempty domain
. In particular, the sort  is mapped to the
Boolean domain , and the sort {\tt location} is
mapped to a domain of {\em control locations} in a program. 
\item Every variable symbol  of sort  is mapped to an
element  in .
\item Every function symbol , of sort  is mapped to a function . In particular, every constant symbol  of sort 
is mapped to an element . 
\item Every predicate symbol  of sort  
is mapped to a function . 
\end{itemize}

Given an interpretation  as defined above, the valuation
 of an -term  and the valuation  of an
-atom  are defined as follows:

\begin{itemize}
\item For a term  which is a variable , the valuation is .
\item For a term , the valuation . 
\item For an atom , the valuation  iff . 
\item For an atom ,  iff 
.  
\end{itemize}

In the rest of the paper, we assume that the interpretation of
constant, function and predicate symbols in  is known and fixed.
We further assume that the interpretation of sort symbols to specific
domains is known and fixed. With some abuse of notation, we shall
denote the interpretation of all constant, function and predicate
symbols simply by the symbol name, and identify sorts with their
domains. Examples of some constant, function and predicate symbols
that may be included in  are: constant symbols , function
symbols , and predicate symbols  over the integers, function
symbols  over , the constant symbol 
(empty list), function symbol  (appending lists) and
predicate symbol  (emptiness test) over lists, \etc. Finally,
when the interpretation is obvious from the context, we denote the
valuations ,  of terms  and atoms  simply as
, , respectively.  





\subsection{Concurrent Programs}\label{sec:prog}

In our framework, we consider a (shared-memory) concurrent program to
be an asynchronous composition of a non-empty, finite set of
processes, equipped with a finite set of program variables that range
over finite domains. We assume a simple concurrent programming
language with assignment, condition test, unconditional goto,
sequential and parallel composition, and the synchronization primitive
- conditional critical region (CCR) \cite{Hoare71,Hansen81}. A concurrent
program  is written using the concurrent programming language, in
conjunction with -terms and -atoms. We assume that the sets
of (data and control) variables, functions and predicates available
for writing  are each finite subsets of ,
 and , respectively. 



A concurrent program is given as , with . The declaration
consists of a finite sequence of declaration statements, specifying
the set of shared data variables , their domains, and possibly
initializing them to specific values. For example, the declaration
statement, , declares two
variables , , each with (a finite integer) domain
, and initializes the variable  to the value .
The initial value of any uninitialized variable is assumed to be a
user/environment input from the domain of the variable. 

A process  consists of a declaration of local data variables
 (similar to the declaration of shared data variables in ),
and a finite sequence of labeled, {\em atomic} instructions, .  We denote the unique instruction at location  as .
The set of data variables  accessible by  is given by . The set of labels or {\em locations} of  is denoted
, with  being a designated
start location. Unless specified otherwise\footnote{A user may define
an atomic instruction (block) as a sequence of assignment, conditional 
and goto statements}, an atomic instruction  is an assignment,
condition test, unconditional goto, or CCR. An assignment instruction
, given by , is a parallel assignment of -terms , over , to the data variables 
in . Upon completion, an assignment statement at 
transfers control to the next location . A condition test,
, consists of an -atom 
over , and a pair of locations  in  to
transfer control to if  evaluates to , ,
respectively. The instruction  is a transfer of
control to location . A CCR is a guarded insruction block,
, where the enabling guard  is an -atom
over  and  is a sequence of assignment,
conditional and goto statements. The guard  is evaluated atomically
and if found to be , the corresponding  is
executed atomically, and control is transferred to the next location.
If  is found to be , the process {\em waits} at the same
location till  evaluates to . An unsynchronized process does
not contain CCRs.  





We model the asynchronous composition of concurrent processes by the
nondeterministic interleaving of their atomic instructions. Hence, at
each step of the computation, some process, with an enabled
transition, is nondeterministically selected to be executed next by a
scheduler. The set of program variables is denoted ,
where  is the set of control variables
and  is the set of data variables.
The semantics of the concurrent program  is given by a transition
system , where  is a set of states,  is
a set of initial states and  is the transition relation. Each state  is a valuation of the program variables in . We denote
the value of variable  in state  as , and the
corresponding value of a term  and an atom  in state  as
 and , respectively.  and 
are defined inductively as in \secref{alpha}. The domain of each
control variable  is the set of locations , and the
domain of each data variable is determined from its declaration.
The set of initial states  corresponds to all states 
with  for all , and , for every data variable  initialized in its declaration
to some constant .  There exists a transition 
from state  to  in , with ,  and 
for all , iff there exists a corresponding {\em local move} in
process  involving instruction , such that:
\begin{enumerate}
\item  is the
assignment instruction: , for each variable  with : 
, for all other data variables :
, and  is the next location in 
after , or, 
\item  is the condition test: , the valuation of all data variables in
 is the same as that in , and either  is  and
, or  is  and , or,
\item   is ,  the valuation of all data
variables in  is the same as that in , and , or, 
\item  is the CCR ,  is ,
the valuation of all data variables in  correspond to the atomic
execution of  from state , and  is the next
location in  after .  
\end{enumerate}

\noindent We assume that  is total. For terminating processes
, we assume that  ends with a special instruction,
.  

\subsection{Specifications}\label{sec:spec}

Our specification language, \vCTL, is an extension of propositonal
CTL, with formulas composed from -atoms.  While one can use
propositional CTL for specifying properties of finite-state programs,
\vCTL enables more natural specification of properties of concurrent
programs communicating via typed shared variables.  We describe the
syntax and semantics of this language below.  \\

\noindent{\em Syntax}: 
Given a set of variables , we inductively construct the set of (\vCTL) {\em formulas}
over , using -atoms, in conjunction with the
propositional operators  and the temporal operators
,
along with the process-indexed next-time operator :

\begin{itemize}
\item Every -atom over  is a formula. 
\item If ,  are formulas, then so are 
and . 
\item If ,  are formulas, then so are ,
,  and
.
\end{itemize}

We use the following standard abbreviations: 
for ,  for
,  for ,  for ,  for ,
 for
,  for ,  for , and  
for .\\

\noindent{\em Semantics}: \vCTL formulas over a set of variables 
are interpreted over models of the form , where  is
a set of states and  is a a total, multi-process, binary relation
 over , composed of the transitions  of 
each process .  is a labeling function that assigns to each
state  a valuation of all variables in .  The value of a
term  in a state  of  is denoted as ,
and is defined inductively as in \secref{alpha}. A path in  is a
sequence  of states such that , for all . We denote the  state in
 as . 



The satisfiability of a \vCTL formula in a state  of 
can be defined as follows:

\begin{itemize}
\item  iff  . 
\item  iff .
\item  iff it is not the case that .
\item  iff  or
.
\item  iff for some  such that  , . 
\item  iff for some  such that
, . 
\item  iff for all paths 
starting at , .
\item  iff there exists a path 
starting at  such that .\\
\end{itemize}



\noindent{\em Programs as Models}:
A program  can be viewed as a model , with
the same set of states and transitions as , and the identity labeling
function  that maps a state to itself. Given an 
specification , we say  iff for each state , . 



