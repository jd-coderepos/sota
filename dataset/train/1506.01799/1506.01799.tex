

\subsection{Hitting Set and Radius }

Recall the definition of the HSE-graph from Section~\ref{sec:lb}. 
The reductions in this section will be based on adding gadgets to it.


\paragraph{Undirected Radius.} We are now ready to prove the conditional lower bounds for undirected Radius by simple modifications of the HS-graph.

\begin{reminder}{Theorem~\ref{thm:radius}}
If for some , there is an algorithm that can determine if a given undirected, unweighted graph with  nodes and  edges has radius  or  in  time, then the Hitting Set Conjecture is false.
\end{reminder}

\begin{proof}
Given an instance  of the HSE problem, we construct its HSE-graph  as described above.
We will construct an undirected graph  as follows.
Take  with all its nodes and edges (ignoring the direction of these edges) and add three nodes  to it.
For each node  add edges  and  to .
For each node  add an edge  to .
Finally, add an edge  to .

We now claim that the radius of  is  if  is a ``yes" HSE-instance and the radius is at least  otherwise.

First, assume that  is a ``yes" HSE-instance and therefore there is a node  such that for every  there is a node  such that both edges  and  are in  and therefore the edges  and  are in .
In this case, the distance from  to every other node  in  is at most :
If  then the distance is .
If either  or , then the distance is , via .
If  then the distance is  and if  then the distance is .

Now, assume that  is a ``no" HSE-instance, which implies that for any node , there is a node  be such that there is no  for which the edges  and  are in . 
In this case, there is no path of length  from  to  in : if the path goes through part  and has length  then it must be of the form  for some  which is a contradiction, while if the path goes through  it will have length at least  since the distance from  to any node in  is exactly .
Therefore, if the center of the graph is in , the radius is at least .
On the other hand, if the center of the graph is in  then its distance to  is at least . 
And finally, if the center is  or  then its distance to any node  is at least .
Therefore, the radius of  is at least .

To complete the proof, note that our new graph  has  nodes and  edges, and therefore it can be easily turned into a \emph{sparse} graph on  nodes without changing the radius (add  dummy nodes, connect them to a node  and connect  to every node in ). This implies that a subquadratic algorithm will solve the HSE problem in  time, for some , which refutes the HS conjecture. 

\end{proof}

The following observation about the treewidth (in fact, pathwidth) of the graph in the proof of Theorem~\ref{thm:radius} proves the Radius part of Theorem~\ref{thm:radius}.

\begin{claim}
The Radius instance constructed in the proof of Theorem~\ref{thm:radius} has pathwidth (and therefore treewidth) .
\end{claim}

\begin{proof}
Consider the path decomposition  in which there is a bag  for every node  that contains the nodes , and the bags are ordered arbitrarily in a path.
Every edge appears in a bag and all the bags containing a node of  are connected (they are either a single bag or the whole path).
The sizes of the largest bag is .
\end{proof}

Thus, an algorithm that can compute Radius on treewidth (or pathwidth)  graphs in  can be used to solve the HSE problem where  in  time, refuting the HS conjecture.


\paragraph{Source Radius.} We now present the reduction to Source Radius which allows us to prove Theorem~\ref{thm:SourceRad}.

\begin{reminder}{Theorem~\ref{thm:SourceRad}}
A -approximation algorithm for Source Radius in sparse graphs that runs in subquadratic time refutes the Hitting Set Conjecture.
\end{reminder}

\begin{proof}

We show how an algorithm that distinguishes between radius  and  on a graph with  nodes and  edges allows us to solve an HSE instance on two lists of  sets in .
Given an algorithm for Source Radius as in the statement of the theorem, we can
set  which allows us to solve HSE in subquadratic time since .

Given an instance  of HSE, we construct the corresponding HSE graph  and then use it to construct our Source Radius instance  as follows.
Take  with all its nodes and edges and add the following nodes and paths to it to get .
For each node  add  nodes  and add edges so that there is a path .
Similarly, for each node  add  nodes  and edges so that there is a path .
Add a node  and connect every node  with an  edge to , and connect  to the beginning of 's path by adding the edge .


\begin{claim}
The radius of  is  if  is a ``yes" HSE-instance and is at least  otherwise.
\end{claim}

\begin{proof}
First, assume that  is a ``yes" HSE-instance and therefore there is a node  such that for every  there is a node  such that both edges  and  are in  and therefore these edges are also in .
In this case, the distance from  to every other node  in  is at most :
\begin{enumerate} 
\item If  for some  or , then the distance is  or .
\item If  then the distance is . If  for some  then the distance is  via . If  then the distance is  via .
\item If , then the distance is either  via the edge  or  via a path through  to some  and then to .
\end{enumerate}
On the other hand, assume that  is a ``no" HSE-instance, which implies that for any node , there is a node  be such that there is no  for which the edges  and  are in . 
In this case, there does not exist a center node  that can reach every other node  within less than  distance:
\begin{itemize}
\item  cannot be in  since there is no path of length  from  to  in : the only such paths go through the node  and spend  edges to reach some node  and then take a path of length  from  to .
\item Moreover, for any ,  cannot be the node  since it will have an even larger distance to : it will have to reach  first and then take the path of length  to .
\item  cannot be any node in  or any , since those nodes cannot reach .
\item Finally,  cannot be  since its distance to  is .
\end{itemize}

Therefore, the radius of  is at least .
\end{proof}

Our new graph  has  nodes and  edges, which implies that a subquadratic  algorithm gives a subquadratic algorithm for HSE even when  and refutes the HS conjecture.
\end{proof}

\paragraph{Max Radius.} Now we prove the lower bound for Max Radius. Together with Lemma~\ref{lem:roundtrip}, this proves Theorem~\ref{thm:roundtrip}.

\begin{lemma}
\label{lem:max}
A -approximation algorithm for Max Radius in sparse graphs that runs in subquadratic time refutes the Hitting Set Conjecture.
\end{lemma}

\begin{proof}
The proof proceeds exactly as in the proof of Theorem~\ref{thm:SourceRad}, except that we add the following edges to :
For every node in  or any  node, add an edge to .

The new edges makes sure that there is a path of length  from any node in  to any node in  and now the same claims hold when we replace one-way distance with max-distance. 
We will outline the differences in the arguments.

If  is a ``yes" HSE-instance, then  has max-radius at most . 
As before, we show a node  in  that reaches the other nodes within  distance.
Now, however, we also have that any node will reach  within distance  via , and therefore the max-distance between  and the other nodes is .

If  is a ``no" HSE-instance, we show that no node of  can have distance less than  to the other nodes (this is stronger than having max-eccentricity at least , since we are ignoring the distances from the other nodes).
Assume for contradiction that there is a node with max-eccentricity .
Consider the argument we gave in the proof of Theorem~\ref{thm:SourceRad} and note that it still implies that the center cannot be in  nor  nor any  node.
Here, however, we need a different argument for why the center cannot be in  or any :
any such node will have a node in  that is at distance at least  from it. 
Here we assume that there is no node in  that has edges to every node in  (if such node exists, we check if it has any edges coming from  - if there are we output ``yes" and otherwise we remove the node.)
\end{proof}


\paragraph{Min-Radius.} Finally, we present the lower bounds for Min-Radius. Here, we will have to work harder to make the graph in our construction a DAG.

\begin{lemma}
\label{lem:MinRad}
A -approximation algorithm for Min-Radius on sparse DAGs that runs in subquadratic time refutes the Hitting Set Conjecture.
\end{lemma}


Because we want the input graph to be a DAG, and simultaneously we want the min-distance between any two nodes to be a small constant, we require a special construction.
Given a set  of  nodes , it creates a DAG  with at most  nodes and  edges such that in the topological order of , , and for any two nodes of   where  in the topological order, .

We define  as follows. WLOG  is a power of , otherwise add enough  new nodes after  and grow  until  is a power of .
Using  extra nodes, create a complete balanced binary tree  on top of  where the leaves of  are  in the order  from left to right. 
Let  be the root of . The edges of  are not added to  but we will add equivalent directed edges for them;  and  share the same node set. 

Now, for every node  in , consider the root to  path, and call any node  on the path a  node if the path branches left out of it and a  node otherwise. For every -node  on the - path, add a directed edge  to , and for every -node, add . Note we have only added  edges per node , so the number of edges is .  is a DAG by construction. Moreover, for any two nodes  and , let their LCA in the tree be . If  is not  and not , then it is a -node for one of them, say , and a -node for the other, . 
Hence, there is a path of length  in  between any pair of nodes where neither is a descendent of the other in . If  is a descendent of , then the min-distance between then is .

We can generalize the construction above slightly for any integer , in a construction  so that for any two nodes  and  with  in the topological 
order, where  is not a descendent of  in , their min-distance is . To do this, take  and replace every non-leaf node  by a directed path , and replace every in-edge  by  and every out-edge by  by . We say that  is a copy of  in .
For competeness, call the leaves (the nodes of ) copies of themselves and for any leaf , let .
For any directed edge  of , if  goes towards the root in , then connect all copies  of  to , and if  goes down in , then connect  to all copies of . The number of nodes of  is  and the number of edges is .



Now, for any nodes of ,  and , not descendents of one another, their min-distance is  and is attained by taking their LCA,  in , taking the edge  followed by the path to  and then the edge . If  is a descendent of , then the min-distance is at most .

Here is another way to describe  using binary numbers:

Let us identify each node in  with a number in .
We add nodes  for each  and  in the form of a binary tree.
Each such node will be connected with a path of length : .
These new nodes will be connected to and from  as follows:
for each node , if , we add an edge , and if , we add an edge .
Note that we only added  edges, and now there is a path from  to  for all  in  of length , via some .
Then, we also connect the tree nodes with themselves: Let  and . If , then add edges ,  and  for all  , and if , we add ,  and  for all  .
This completes the construction of this tree structure. 



\begin{proof}



Given an instance  of HSE, we construct the corresponding HSE graph  and then use it to construct our Min Radius instance  as follows.

Take  with all its nodes and edges and add the following nodes and paths to it to get .

For each node  add  nodes  and add edges so that there is a path .

Create {\bf two} copies of construction  (sharing ). Having two copies rather than one serves to enforce that the center of the graph must be in , and not some extra node of  we added.






Finally, add  nodes , connect every node  with an edge , then connect every node in  with an edge , then add a path .
Also, add a node  and edges  for every node .

\begin{claim}
The min-radius of  is  if  is a ``yes" HSE-instance and is at least  otherwise.
\end{claim}

\begin{proof}
First, assume that  is a ``yes" HSE-instance and therefore there is a node  such that for every  there is a node  such that both edges  and  are in  and therefore these edges are also in .
In this case, the min-distance from  to every other node  in  is at most :
\begin{enumerate} 
\item If  for some  or , then the distance is  or .
\item If  for some  then the distance is . If  the distance is .
\item If , then the distance is at most  via the path .
\item If , then let  be the most significant bit in which the integers  differ, and note that there must be some  such that  and , or vice versa. Either way, there is a path of length  via  between the two nodes.

\item Finally, if  for some  and , then let  be the largest integer so that the first  most significant bits of  and  are the same. As before, this implies that for some  there is a path , or a path . Thus, the min-distance is at most .

\end{enumerate}

On the other hand, assume that  is a ``no" HSE-instance, which implies that for any node , there is a node  be such that there is no  for which the edges  and  are in . 
In this case, there does not exist a center node  that can reach or be reached from every other node  within less than  distance:
\begin{itemize}
\item  cannot be in  since there is no path of length  from  to some node  in  (the one that  cannot reach in ): the only such paths either go through the -path and spend  edges to reach some node  for which , or go through the tree structure, incurring an extra cost of  edges, to reach some node , and then take a path of length  from  or  to .
\item  cannot be a node : let  be such that  is not an edge, then the min-distance between  and  is infinite. Similarly,  cannot be a node  or  for a node , since it will have infinite min-distance to the node .
\item  cannot be on the  path, since it will have infinite min-distance to the node . Thus,  cannot be  as well.
\item  The final case is when  is a node in the tree structure  for some  in the right range.
This is the more tricky case.
We claim that the min-distance to the copy of this node, in the isomorphic copy of the tree that we added, is infinite. 
To see this, first note that by construction, a path between the two trees must pass through .
Then, note that there is certain threshold , such that for any node  that  can reach, , while for any node  that can reach , . 
Since this is also true for the copy of , we conclude that there is no path between these two nodes, and the min-distance is infinite. (In other words, any path from  to  goes through the two copies of , and since these are copies of the same DAG, it can't be that in one DAG there is a path from  to some , and the other, a path from  to .)
\end{itemize}

Therefore, the radius of  is at least .
\end{proof}

Our new graph  has  nodes and  edges.
And note that, by construction,  is a DAG since we do not create any cycles.
It can be turned into a \emph{sparse} graph on  nodes, without changing the min-radius, by adding dummy nodes, all connected to a new node  and connecting  to every node in .
We will choose  large enough such that .
Thus, a subquadratic  time -approximation algorithm for Min-Radius gives a subquadratic algorithm for HSE even when  and refutes the HS conjecture.
\end{proof}


\subsection{Orthogonal Vectors and Diameter}
Similarly to the HSE graph, we define an OV graph  from an OV instance  where the vectors in  and  have dimension .  has partitions named  where  and  (abusing notation slightly) correspond exactly to the sets of vectors  and  of the OV instance, and  is . For every , create an edge  for all  for which  and for each  and  for which , add . The OV problem is now to find some  such that  is not reachable from .


\paragraph{Min-Diameter.}
\begin{lemma}
\label{lem:MinDiamDAG}
If there is a -approximation algorithm for Min-Diameter on a sparse DAG that runs in subquadratic time, then the OV conjecture is false.
\end{lemma}
\begin{proof}

We show how an algorithm that distinguishes between diameter  and  on a DAG allows us to solve the OV problem.
Let  be the OV graph.
We construct a directed graph  as follows.

Take  (with all its nodes and edges) and add two nodes  to it.
Order the nodes in  and add construction  from the Min-Radius section above.
Similarly, order the nodes in  and add  and order  and add .
Finally, add edge , edges  for all nodes , and for every , add .

\begin{claim}
The min-diameter of  is  if  is a NO instance of the OV problem and is at least  otherwise.
\end{claim}

\begin{proof}
First, assume that  is a NO instance and therefore for every pair  there is a node  such that both edges  are in  and therefore are also in .
In this case, for all pairs of nodes , the min-distance in  is no more than :
\begin{enumerate} 
\item If  then the distance is .
\item If either , or , or , then the distance is at most , by the  construction.
\item If  and  (or vice versa) then the distance is at most  via the path through .
\item If  and  (or vice versa) then the distance is at most  via the path through .
\item If  then it has distance  to every node in  and distance  to every node in . The  case is symmetric, except that the distance to nodes of  is also .
\item If ,  (or vice versa), then the min-distance is  through .
\end{enumerate}
On the other hand, assume that  is a YES instance and let  be such that there is no  for which the edges  and  are in . 
In this case, there is no path of length  from  to  in : if the path goes through part  and has length  then it must be of the form  for some  which is a contradiction. If the path goes through  it will have distance at least  since the distance from  to any node in  is exactly .
Finally, if the path goes through , through , then it also has length at least .
Therefore, the min-diameter of  is at least .
\end{proof}
Our new graph  has  nodes and  edges, which implies that a subquadratic algorithm refutes the OV conjecture. 
 is a DAG by construction:  are DAGs and the rest of the comparisons in the topological order are .
\end{proof}
\begin{lemma}
\label{lem:MinDiam}
If there is a -approximation algorithm for Min-Diameter on a sparse weighted graph that runs in subquadratic time, then the OV Conjecture is false.
\end{lemma}

\begin{proof}

We show how an algorithm that distinguishes between diameter  and  on a sparse graph allows us to solve OV.
Let  be the OV graph. We construct a directed graph  as follows.
Take  (with all its nodes and edges) and add three nodes  to it.
The edges that were present in  will have weight  in .
We connect  to and from  with edges  for every node , with weights , and then we add edges  for every node , with weight .
We connect  to and from  with edges  for every node , with weights ,  and then we add edges  for every node , with weight .
Finally, we connect  to and from  with edges  for every node , with weights , and we add an edge  with weight .

\begin{claim}
The min-diameter of  is  if  is a NO instance of OV and is at least  otherwise.
\end{claim}

\begin{proof}
First, assume that  is a NO instance and therefore for every pair  there is a node  such that both edges  are in  and therefore are also in .
In this case, for all pairs of nodes , the min-distance in  is no more than :
\begin{enumerate} 
\item If  then the distance from  to  is .
\item If either , or , or , then the distance is , via .
\item If  and  (or vice versa) then the distance from  to  (or vice versa) is at most  via the path through .
\item If  and  (or vice versa) then the distance from  to  (or vice versa) is at most  via the path through .
\item If  then it has min-distance  to every node in , distance  from every node  via the path , and distance  from  via the path  for any . The  case is symmetric.
\item If  then it has min-distance  to every node in  and min-distance  to every node in .
\end{enumerate}
On the other hand, assume that  is a YES instance and let  be such that there is no  for which the edges  and  are in . 
In this case, there is no path of length less than  from  to  in : if the path does not any of the nodes  it must be of the form  which is a contradiction, while if it uses  it must be of the form  which will have length at least , the case in which the path goes from  to  and uses  is symmetric, and finally, if we use  we also incur an addition weight of  resulting in length at least .
Therefore, the min-diameter of  is at least .
\end{proof}

\end{proof}

\paragraph{Roundtrip Diameter.}

\begin{lemma}
\label{lem:RoundDiam}
If there is a -approximation algorithm for Roundtrip-Diameter on a sparse graph that runs in subquadratic time, then the OV conjecture is false.
\end{lemma}

\begin{proof}
Any algorithm distinguishing between roundtrip diameter  and  can distinguish between undirected diameter  and  - take the undirected graph, bi-direct the edges and run the algorithm for roundtrip-diameter.
The latter task cannot be done in subquadratic time \cite{RV13}.
\end{proof}







\subsection{Lower Bound for Estimating All eccenricities}


The following construction shows that the  factor that we have for Radius and Diameter (in undirected graphs) is unlikely to be achievable if we want to estimate all the eccentricities in the graph. 
Only a  approximation is known for this problem in subquadratic time \cite{diametersoda14}.

\begin{reminder}{Theorem~\ref{thm:allecc}}
 A  approximation algorithm for the eccentricities of all nodes in undirected sparse graphs that runs in subquadratic time refutes the Orthogonal Vectors Conjecture.
\end{reminder}

\begin{proof}
Let  be an instance of OV, we will use it to construct an undirected sparse graph  as follows.
Abusing the notation, construct a set of nodes  that contains a node  for every vector in , and similarly construct a set of nodes  from the vectors .
Add a set of nodes  corresponding to the coordinates  and for every vector  we add an edge  iff .
Note that now, there is a -path from a node  to a node  iff the vectors  are not orthogonal.
Then, we also add a set  that contains a copy  of every node , such that  is only connected with one edge to .
Finally, add two nodes , connect  to every node in , connect  to every node in , and add the edge .

We now claim that for every node , the eccentricity is  if the vector  is orthogonal to some vector , and it is  otherwise.

To prove our claim we first show that any node  will have distance  to all the nodes in :
there is a -path via  to every other node in  and there is a -path to every node in  via .
Now, on the one hand, if  is not orthogonal to any , then in our graph, there will be a -path via  from  to every node in  and therefore there will be a -path to every node in .
Thus, in this case, the eccentricity is .
On the other hand, if  is orthogonal to some , then the only paths from  to  have length , and therefore the distance from  to  is , and the eccentricity of  is at least .

This claim shows that an algorithm estimating all the eccentricities within  allows us to solve OV by this reduction.
It is important to note that the radius and diameter will not be determined by nodes in  in our graph, and therefore a better than  approximation for those parameter (which is achievable in subquadratic time \cite{RV13}) is not enough to solve OV.
Indeed, the node  will have eccentricity  regardless of the OV instance.

The number of nodes in the graph we constructed is  and the number of edges is .
It can be easily turned into a sparse graph on  nodes.
Thus, a subquadratic  approximation allows us to solve OV in time , for some , which is enough to refute the OV conjecture.

\end{proof}


Our lower bounds for Diameter are summarized in Table~\ref{tab:res-diam}.

\begin{table*}\centering\small
\begin{tabular}{|c | c | c | c|}
\hline 
\multicolumn{4}{|c|}{Diameter Variants} \\
\hline
Problem & Definition & Upper Bound & OV Conjecture \\
\hline
  \UndirectedDiameter{} &
   &
   in  [\cite{RV13}] &
    [\cite{RV13}]\\
\hline
  \MaxDiameter{} &
   &
   in  [\cite{RV13}] &
   [\cite{RV13}] \\
\hline
  \MinDiameter{} &
   &
   in  [Lem~\ref{lem:min-diameter-upper}] &
   on weighted  [Lem~\ref{lem:MinDiam}] \\
\hline
  \MinDiameter{} on DAGs &
   &
   in  [Thm~\ref{thm:min-diameter-dag-upper}] &
    [Lem~\ref{lem:MinDiamDAG}] \\
\hline
  \RoundtripDiameter{} &
   &
   in  [metric]&
   [Lem~\ref{lem:RoundDiam}] \\
\hline
\end{tabular}
\caption{Our Bounds for Various Diameter Problems}
\label{tab:res-diam}
\end{table*}
