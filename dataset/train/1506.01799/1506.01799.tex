

\subsection{Hitting Set and Radius }

Recall the definition of the HSE-graph from Section~\ref{sec:lb}. 
The reductions in this section will be based on adding gadgets to it.


\paragraph{Undirected Radius.} We are now ready to prove the conditional lower bounds for undirected Radius by simple modifications of the HS-graph.

\begin{reminder}{Theorem~\ref{thm:radius}}
If for some $\eps>0$, there is an algorithm that can determine if a given undirected, unweighted graph with $n$ nodes and $m=O(n)$ edges has radius $2$ or $3$ in $O(n^{2-\eps})$ time, then the Hitting Set Conjecture is false.
\end{reminder}

\begin{proof}
Given an instance $A,B,U$ of the HSE problem, we construct its HSE-graph $G$ as described above.
We will construct an undirected graph $G'$ as follows.
Take $G$ with all its nodes and edges (ignoring the direction of these edges) and add three nodes $x,y,z$ to it.
For each node $a \in A$ add edges $\{a,x\}$ and $\{a,y\}$ to $G'$.
For each node $u \in U$ add an edge $\{u,x\}$ to $G'$.
Finally, add an edge $\{y,z\}$ to $G'$.

We now claim that the radius of $G'$ is $2$ if $G$ is a ``yes" HSE-instance and the radius is at least $3$ otherwise.

First, assume that $G$ is a ``yes" HSE-instance and therefore there is a node $a^*\in A$ such that for every $b \in B$ there is a node $u \in U$ such that both edges $(a^*,u)$ and $(u,b)$ are in $E(G)$ and therefore the edges $\{a^*,u\}$ and $\{u,a^*\}$ are in $E(G')$.
In this case, the distance from $a^*$ to every other node $v$ in $G'$ is at most $2$:
If $v \in B$ then the distance is $2$.
If either $v \in A$ or $v\in U$, then the distance is $2$, via $x$.
If $v=y$ then the distance is $1$ and if $v=z$ then the distance is $2$.

Now, assume that $G$ is a ``no" HSE-instance, which implies that for any node $a\in A$, there is a node $b \in B$ be such that there is no $u \in U$ for which the edges $(a,u)$ and $(u,b)$ are in $E(G)$. 
In this case, there is no path of length $2$ from $a$ to $b$ in $G'$: if the path goes through part $U$ and has length $2$ then it must be of the form $\{a,u\}, \{u,b\}$ for some $u \in U$ which is a contradiction, while if the path goes through $x$ it will have length at least $3$ since the distance from $x$ to any node in $B$ is exactly $2$.
Therefore, if the center of the graph is in $A$, the radius is at least $3$.
On the other hand, if the center of the graph is in $B \cup U \cup \{x\}$ then its distance to $z$ is at least $3$. 
And finally, if the center is $y$ or $z$ then its distance to any node $b \in B$ is at least $3$.
Therefore, the radius of $G'$ is at least $3$.

To complete the proof, note that our new graph $G'$ has $O(n)$ nodes and $O(n |U|)$ edges, and therefore it can be easily turned into a \emph{sparse} graph on $O(n|U|)$ nodes without changing the radius (add $n|U|$ dummy nodes, connect them to a node $d$ and connect $d$ to every node in $A$). This implies that a subquadratic algorithm will solve the HSE problem in $O(n^{2-\eps} \cdot |U|^{2-\eps})$ time, for some $\eps>0$, which refutes the HS conjecture. 

\end{proof}

The following observation about the treewidth (in fact, pathwidth) of the graph in the proof of Theorem~\ref{thm:radius} proves the Radius part of Theorem~\ref{thm:radius}.

\begin{claim}
The Radius instance constructed in the proof of Theorem~\ref{thm:radius} has pathwidth (and therefore treewidth) $O(|U|)$.
\end{claim}

\begin{proof}
Consider the path decomposition $P$ in which there is a bag $B_v$ for every node $v \in A\cup B$ that contains the nodes $B_v = U \cup \{x,y,z\} \cup \{v\}$, and the bags are ordered arbitrarily in a path.
Every edge appears in a bag and all the bags containing a node of $G'$ are connected (they are either a single bag or the whole path).
The sizes of the largest bag is $|U|+4$.
\end{proof}

Thus, an algorithm that can compute Radius on treewidth (or pathwidth) $k$ graphs in $2^{o(k)}\cdot n^{2-\eps}$ can be used to solve the HSE problem where $|U|=\omega(\log{n})$ in $O(n^{2-\eps})$ time, refuting the HS conjecture.


\paragraph{Source Radius.} We now present the reduction to Source Radius which allows us to prove Theorem~\ref{thm:SourceRad}.

\begin{reminder}{Theorem~\ref{thm:SourceRad}}
A $(2-\delta)$-approximation algorithm for Source Radius in sparse graphs that runs in subquadratic time refutes the Hitting Set Conjecture.
\end{reminder}

\begin{proof}

We show how an algorithm that distinguishes between radius $t+1$ and $2t$ on a graph with $O(tn)$ nodes and $O(n |U| + tn)$ edges allows us to solve an HSE instance on two lists of $n$ sets in $U$.
Given an algorithm for Source Radius as in the statement of the theorem, we can
set $t=|U|=\omega(\log{n})$ which allows us to solve HSE in subquadratic time since $(2-\delta)(t+1) <2t$.

Given an instance $A,B,U$ of HSE, we construct the corresponding HSE graph $G$ and then use it to construct our Source Radius instance $G'$ as follows.
Take $G$ with all its nodes and edges and add the following nodes and paths to it to get $G'$.
For each node $b \in B$ add $t-1$ nodes $b_1,\ldots,b_{t-1}$ and add edges so that there is a path $b \to b_1 \to b_2 \to \cdots \to b_{t-1}$.
Similarly, for each node $a \in A$ add $t-2$ nodes $a_1,\ldots,a_{t-2}$ and edges so that there is a path $a_1 \to a_2 \to \cdots \to a_{t-2} \to a$.
Add a node $x$ and connect every node $a \in A$ with an $(a,x)$ edge to $x$, and connect $x$ to the beginning of $a$'s path by adding the edge $(x,a_1)$.


\begin{claim}
The radius of $G'$ is $t+1$ if $G$ is a ``yes" HSE-instance and is at least $2t$ otherwise.
\end{claim}

\begin{proof}
First, assume that $G$ is a ``yes" HSE-instance and therefore there is a node $a^*\in A$ such that for every $b \in B$ there is a node $u \in U$ such that both edges $(a^*,u)$ and $(u,b)$ are in $E(G)$ and therefore these edges are also in $E(G')$.
In this case, the distance from $a^*$ to every other node $v$ in $G'$ is at most $t+1$:
\begin{enumerate} 
\item If $v = b_i$ for some $i \leq t-1$ or $v=b$, then the distance is $2+i \leq t+1$ or $2$.
\item If $v = x$ then the distance is $1$. If $v= a_i$ for some $i \leq t-2$ then the distance is $i+1 \leq t-1$ via $x$. If $v = a \neq a^*$ then the distance is $t$ via $x$.
\item If $v \in U$, then the distance is either $1$ via the edge $(a^*,u)$ or $t+1$ via a path through $x$ to some $a \neq a^*$ and then to $v$.
\end{enumerate}
On the other hand, assume that $G$ is a ``no" HSE-instance, which implies that for any node $a\in A$, there is a node $b \in B$ be such that there is no $u \in U$ for which the edges $(a,u)$ and $(u,b)$ are in $E(G')$. 
In this case, there does not exist a center node $w$ that can reach every other node $v$ within less than $2t$ distance:
\begin{itemize}
\item $w$ cannot be in $A$ since there is no path of length $2t-1$ from $a$ to $b_{t-1}$ in $G'$: the only such paths go through the node $x$ and spend $t$ edges to reach some node $a' \neq a$ and then take a path of length $t+1$ from $a'$ to $b_{t-1}$.
\item Moreover, for any $i \leq t-2$, $w$ cannot be the node $a_i$ since it will have an even larger distance to $b_{t-1}$: it will have to reach $a$ first and then take the path of length $2t+1$ to $b_{t-1}$.
\item $w$ cannot be any node in $U \cup B$ or any $b_i$, since those nodes cannot reach $x$.
\item Finally, $w$ cannot be $x$ since its distance to $b_{t-1}$ is $t-1+t+1=2t$.
\end{itemize}

Therefore, the radius of $G'$ is at least $2t$.
\end{proof}

Our new graph $G'$ has $N=O(tn)$ nodes and $O(n|U| + tn)=O(N)$ edges, which implies that a subquadratic $O(N^{2-\delta})$ algorithm gives a subquadratic algorithm for HSE even when $|U|=\omega(\log{n})$ and refutes the HS conjecture.
\end{proof}

\paragraph{Max Radius.} Now we prove the lower bound for Max Radius. Together with Lemma~\ref{lem:roundtrip}, this proves Theorem~\ref{thm:roundtrip}.

\begin{lemma}
\label{lem:max}
A $(2-\delta)$-approximation algorithm for Max Radius in sparse graphs that runs in subquadratic time refutes the Hitting Set Conjecture.
\end{lemma}

\begin{proof}
The proof proceeds exactly as in the proof of Theorem~\ref{thm:SourceRad}, except that we add the following edges to $G'$:
For every node in $U \cup B$ or any $b_i$ node, add an edge to $x$.

The new edges makes sure that there is a path of length $t+1$ from any node in $G'$ to any node in $A$ and now the same claims hold when we replace one-way distance with max-distance. 
We will outline the differences in the arguments.

If $G$ is a ``yes" HSE-instance, then $G'$ has max-radius at most $t+1$. 
As before, we show a node $a^*$ in $A$ that reaches the other nodes within $t+1$ distance.
Now, however, we also have that any node will reach $a^*$ within distance $t+1$ via $x$, and therefore the max-distance between $a^*$ and the other nodes is $t+1$.

If $G$ is a ``no" HSE-instance, we show that no node of $G'$ can have distance less than $2t$ to the other nodes (this is stronger than having max-eccentricity at least $2t$, since we are ignoring the distances from the other nodes).
Assume for contradiction that there is a node with max-eccentricity $<2t$.
Consider the argument we gave in the proof of Theorem~\ref{thm:SourceRad} and note that it still implies that the center cannot be in $A$ nor $x$ nor any $a_i$ node.
Here, however, we need a different argument for why the center cannot be in $U \cup B$ or any $b_i$:
any such node will have a node in $B$ that is at distance at least $2t$ from it. 
Here we assume that there is no node in $U$ that has edges to every node in $B$ (if such node exists, we check if it has any edges coming from $A$ - if there are we output ``yes" and otherwise we remove the node.)
\end{proof}


\paragraph{Min-Radius.} Finally, we present the lower bounds for Min-Radius. Here, we will have to work harder to make the graph in our construction a DAG.

\begin{lemma}
\label{lem:MinRad}
A $(2-\delta)$-approximation algorithm for Min-Radius on sparse DAGs that runs in subquadratic time refutes the Hitting Set Conjecture.
\end{lemma}


Because we want the input graph to be a DAG, and simultaneously we want the min-distance between any two nodes to be a small constant, we require a special construction.
Given a set $X$ of $n$ nodes $v_1,\ldots,v_n$, it creates a DAG $DG(X)$ with at most $O(n)$ nodes and $O(n\log n)$ edges such that in the topological order of $DG(X)$, $v_i<v_{i+1}$, and for any two nodes of $DG(X)$ $x,y$ where $x<y$ in the topological order, $d(x,y)\leq 2$.

We define $DG(X)$ as follows. WLOG $n$ is a power of $2$, otherwise add enough $(<n)$ new nodes after $v_n$ and grow $n$ until $n$ is a power of $2$.
Using $n-1$ extra nodes, create a complete balanced binary tree $T$ on top of $X$ where the leaves of $T$ are $X$ in the order $v_1,\ldots, v_n$ from left to right. 
Let $r$ be the root of $T$. The edges of $T$ are not added to $DG(X)$ but we will add equivalent directed edges for them; $T$ and $DG(X)$ share the same node set. 

Now, for every node $x$ in $T$, consider the root to $x$ path, and call any node $p$ on the path a $0$ node if the path branches left out of it and a $1$ node otherwise. For every $0$-node $p$ on the $r$-$x$ path, add a directed edge $(x,p)$ to $DG(X)$, and for every $1$-node, add $(p,x)$. Note we have only added $\log n$ edges per node $x$, so the number of edges is $O(n\log n)$. $DG(X)$ is a DAG by construction. Moreover, for any two nodes $x$ and $y$, let their LCA in the tree be $u$. If $u$ is not $x$ and not $y$, then it is a $0$-node for one of them, say $x$, and a $1$-node for the other, $y$. 
Hence, there is a path of length $2$ in $DG(X)$ between any pair of nodes where neither is a descendent of the other in $T$. If $y$ is a descendent of $x$, then the min-distance between then is $1$.

We can generalize the construction above slightly for any integer $t\geq 1$, in a construction $DG_t(X)$ so that for any two nodes $x$ and $y$ with $x<y$ in the topological 
order, where $x$ is not a descendent of $y$ in $T$, their min-distance is $t+1$. To do this, take $DG(X)$ and replace every non-leaf node $\ell$ by a directed path $\ell_1,\ell_2,\ldots,\ell_t$, and replace every in-edge $(x,\ell)$ by $(x,\ell_1)$ and every out-edge by $(\ell,y)$ by $(\ell_t,y)$. We say that $\ell_j$ is a copy of $\ell$ in $T$.
For competeness, call the leaves (the nodes of $X$) copies of themselves and for any leaf $x$, let $x_1=x_t=x$.
For any directed edge $(x,y)$ of $DG(X)$, if $(x,y)$ goes towards the root in $T$, then connect all copies $x_i$ of $x$ to $y_1$, and if $(x,y)$ goes down in $T$, then connect $x_t$ to all copies of $y$. The number of nodes of $DG_t(X)$ is $O(tn)$ and the number of edges is $O(tn\log n)$.



Now, for any nodes of $DG_t(X)$, $x$ and $y$, not descendents of one another, their min-distance is $t+1$ and is attained by taking their LCA, $\ell$ in $T$, taking the edge $(x,\ell_1)$ followed by the path to $\ell_t$ and then the edge $(\ell_t,y)$. If $x$ is a descendent of $y$, then the min-distance is at most $t$.

Here is another way to describe $DG_t(X)$ using binary numbers:

Let us identify each node in $X$ with a number in $[n]$.
We add nodes $v_{i,j}$ for each $i \in [\log{n}]$ and $j \in [2^{i-1}]$ in the form of a binary tree.
Each such node will be connected with a path of length $t-1$: $v_{i,j} \to v_{i,j}^{(1)} \to \cdots \to v_{i,j}^{(t-2)} \to v'_{i,j}$.
These new nodes will be connected to and from $X$ as follows:
for each node $a \in X$, if $a \in [n/2^i (2j-2)+1,\ldots,n/2^i (2j-1)]$, we add an edge $a \to v_{i,j}$, and if $a \in [n/2^i (2j-1)+1,\ldots,n/2^i (2j)]$, we add an edge $v'_{i,j} \to a$.
Note that we only added $O(n \log{n})$ edges, and now there is a path from $a$ to $a'$ for all $a<a'$ in $A$ of length $t+1$, via some $v_{i,j}$.
Then, we also connect the tree nodes with themselves: Let $i,i' \in [\log{n}]$ and $j \in [2^{i-1}], j' \in [2^{i'-1}]$. If $j' \in [2^{i'-i} (2j-2)+1, \ldots,  2^{i'-i} (2j-1)]$, then add edges $v_{i',j'}\to v_{i,j}$, $v'_{i',j'} \to v_{i,j}$ and $v^{(h)}_{i',j'} \to v_{i,j}$ for all  $h \in [t-1]$, and if $j' \in [2^{i'-i} (2j-1)+1, \ldots,  2^{i'-i} (2j)]$, we add $v'_{i,j}\to v_{i',j'}$, $v'_{i,j} \to v'_{i',j'}$ and $v'_{i,j}\to v^{(h)}_{i',j'}$ for all  $h \in [t-1]$.
This completes the construction of this tree structure. 



\begin{proof}



Given an instance $A,B,U$ of HSE, we construct the corresponding HSE graph $G$ and then use it to construct our Min Radius instance $G'$ as follows.

Take $G$ with all its nodes and edges and add the following nodes and paths to it to get $G'$.

For each node $b \in B$ add $t-1$ nodes $b_1,\ldots,b_{t-1}$ and add edges so that there is a path $b \to b_1 \to b_2 \to \cdots \to b_{t-1}$.

Create {\bf two} copies of construction $DG_t(A)$ (sharing $A$). Having two copies rather than one serves to enforce that the center of the graph must be in $A$, and not some extra node of $DG_t(A)$ we added.






Finally, add $t$ nodes $x_1,\ldots,x_t$, connect every node $a \in A$ with an edge $a \to x_1$, then connect every node in $U$ with an edge $x_t \to u$, then add a path $x_1\to \cdots \to x_t$.
Also, add a node $y$ and edges $a \to y$ for every node $a \in A$.

\begin{claim}
The min-radius of $G'$ is $t+1$ if $G$ is a ``yes" HSE-instance and is at least $2t$ otherwise.
\end{claim}

\begin{proof}
First, assume that $G$ is a ``yes" HSE-instance and therefore there is a node $a^*\in A$ such that for every $b \in B$ there is a node $u \in U$ such that both edges $(a^*,u)$ and $(u,b)$ are in $E(G)$ and therefore these edges are also in $E(G')$.
In this case, the min-distance from $a^*$ to every other node $v$ in $G'$ is at most $t+1$:
\begin{enumerate} 
\item If $v = b_i$ for some $i \leq t-1$ or $v=b$, then the distance is $2+i \leq t+1$ or $2$.
\item If $v = x_i$ for some $i \leq t$ then the distance is $i$. If $v=y$ the distance is $1$.
\item If $v \in U$, then the distance is at most $t+1$ via the path $x_1 \to \cdots \to x_{t} \to v$.
\item If $v=a \in A$, then let $i$ be the most significant bit in which the integers $a,a^*$ differ, and note that there must be some $j \in [2^{i-1}]$ such that $a \in [n/2^i (2j-1)+1,\ldots,n/2^i (2j)]$ and $a^* \in [n/2^i (2j-2)+1,\ldots,n/2^i (2j-1)]$, or vice versa. Either way, there is a path of length $t+1$ via $v_{i,j}\to \cdots \to v'_{i,j}$ between the two nodes.

\item Finally, if $v= v^{(h)}_{i,j}$ for some $i \in [\log{n}], j \in[2^{i-1}]$ and $h \in [t-1]$, then let $i^*$ be the largest integer so that the first $i^*$ most significant bits of $a$ and $i$ are the same. As before, this implies that for some $j^*$ there is a path $a \to v_{i^*,j^*} \to \cdots \to v^{(h)}_{i^*,j^*} \to v^{(h)}_{i,j}$, or a path $v^{(h)}_{i,j} \to v^{(h)}_{i^*,j^*} \to \cdots \to v'_{i^*,j^*} \to a$. Thus, the min-distance is at most $t+1$.

\end{enumerate}

On the other hand, assume that $G$ is a ``no" HSE-instance, which implies that for any node $a\in A$, there is a node $b \in B$ be such that there is no $u \in U$ for which the edges $(a,u)$ and $(u,b)$ are in $E(G')$. 
In this case, there does not exist a center node $w$ that can reach or be reached from every other node $v$ within less than $2t$ distance:
\begin{itemize}
\item $w$ cannot be in $A$ since there is no path of length $2t-1$ from $w=a$ to some node $b_{t-1}$ in $G'$ (the one that $a$ cannot reach in $G$): the only such paths either go through the $x_i$-path and spend $t$ edges to reach some node $u \in U$ for which $(a,u)\notin E(G)$, or go through the tree structure, incurring an extra cost of $t$ edges, to reach some node $a' \neq a$, and then take a path of length $t+1$ from $u$ or $a'$ to $b_{t-1}$.
\item $w$ cannot be a node $u \in U$: let $b \in B$ be such that $(u,b)$ is not an edge, then the min-distance between $u$ and $b$ is infinite. Similarly, $w$ cannot be a node $b$ or $b_i$ for a node $b \in B$, since it will have infinite min-distance to the node $b'\neq b$.
\item $w$ cannot be on the $x_i$ path, since it will have infinite min-distance to the node $y$. Thus, $w$ cannot be $y$ as well.
\item  The final case is when $w$ is a node in the tree structure $v^{(h)}_{i,j}$ for some $i,j,h$ in the right range.
This is the more tricky case.
We claim that the min-distance to the copy of this node, in the isomorphic copy of the tree that we added, is infinite. 
To see this, first note that by construction, a path between the two trees must pass through $A$.
Then, note that there is certain threshold $T=n/2^i (2j-1)$, such that for any node $a \in A$ that $v^{(h)}_{i,j}$ can reach, $a> T$, while for any node $a' \in A$ that can reach $v^{(h)}_{i,j}$, $a' \leq T$. 
Since this is also true for the copy of $v^{(h)}_{i,j}$, we conclude that there is no path between these two nodes, and the min-distance is infinite. (In other words, any path from $A$ to $A$ goes through the two copies of $DG(A)$, and since these are copies of the same DAG, it can't be that in one DAG there is a path from $x$ to some $a$, and the other, a path from $a$ to $x$.)
\end{itemize}

Therefore, the radius of $G'$ is at least $2t$.
\end{proof}

Our new graph $G'$ has $O(tn+n)$ nodes and $O(n|U| + tn\log{n})$ edges.
And note that, by construction, $G'$ is a DAG since we do not create any cycles.
It can be turned into a \emph{sparse} graph on $N=O(nt|U|)$ nodes, without changing the min-radius, by adding dummy nodes, all connected to a new node $w$ and connecting $w$ to every node in $A$.
We will choose $t$ large enough such that $(2-\delta)(t+1)<2t$.
Thus, a subquadratic $O(N^{2-\delta})$ time $(2-\delta)$-approximation algorithm for Min-Radius gives a subquadratic algorithm for HSE even when $|U|=\omega(\log{n})$ and refutes the HS conjecture.
\end{proof}


\subsection{Orthogonal Vectors and Diameter}
Similarly to the HSE graph, we define an OV graph $G$ from an OV instance $(A,B)$ where the vectors in $A$ and $B$ have dimension $d=O(\log n)$. $G$ has partitions named $A,B,C$ where $A$ and $B$ (abusing notation slightly) correspond exactly to the sets of vectors $A$ and $B$ of the OV instance, and $C$ is $[d]$. For every $a\in A$, create an edge $(a,c)$ for all $c\in C$ for which $a[c]=1$ and for each $b\in B$ and $c\in C$ for which $b[c]=1$, add $(c,b)$. The OV problem is now to find some $a\in A, b\in B$ such that $b$ is not reachable from $a$.


\paragraph{Min-Diameter.}
\begin{lemma}
\label{lem:MinDiamDAG}
If there is a $(3/2-\eps)$-approximation algorithm for Min-Diameter on a sparse DAG that runs in subquadratic time, then the OV conjecture is false.
\end{lemma}
\begin{proof}

We show how an algorithm that distinguishes between diameter $2$ and $3$ on a DAG allows us to solve the OV problem.
Let $G$ be the OV graph.
We construct a directed graph $G'$ as follows.

Take $G$ (with all its nodes and edges) and add two nodes $x,y$ to it.
Order the nodes in $A$ and add construction $DG(A)$ from the Min-Radius section above.
Similarly, order the nodes in $B$ and add $DG(B)$ and order $C$ and add $DG(C)$.
Finally, add edge $(x,y)$, edges $(a,x),(x,c),(c,y),(y,b)$ for all nodes $a \in DG(A), b \in DG(B), c \in DG(C)$, and for every $a\in DG(A)\setminus A$, add $(a,y)$.

\begin{claim}
The min-diameter of $G'$ is $2$ if $G$ is a NO instance of the OV problem and is at least $3$ otherwise.
\end{claim}

\begin{proof}
First, assume that $G$ is a NO instance and therefore for every pair $a\in A,b \in B$ there is a node $c \in C$ such that both edges $(a,c),(c,b)$ are in $E(G)$ and therefore are also in $E(G')$.
In this case, for all pairs of nodes $u,v \in V(G')$, the min-distance in $G'$ is no more than $2$:
\begin{enumerate} 
\item If $u \in A, v \in B$ then the distance is $2$.
\item If either $u,v \in DG(A)$, or $u,v\in DG(B)$, or $u,v\in DG(C)$, then the distance is at most $2$, by the $DG(\cdot)$ construction.
\item If $u \in DG(A)$ and $v \in DG(C)$ (or vice versa) then the distance is at most $2$ via the path through $x$.
\item If $u \in DG(B)$ and $v \in DG(C)$ (or vice versa) then the distance is at most $2$ via the path through $y$.
\item If $u=x$ then it has distance $1$ to every node in $DG(A) \cup DG(C) \cup \{y\}$ and distance $2$ to every node in $DG(B)$. The $u=y$ case is symmetric, except that the distance to nodes of $DG(A)\setminus A$ is also $1$.
\item If $u\in DG(A)\setminus A$, $v\in DG(B)$ (or vice versa), then the min-distance is $2$ through $y$.
\end{enumerate}
On the other hand, assume that $G$ is a YES instance and let $a\in A, b \in B$ be such that there is no $c \in C$ for which the edges $(a,c)$ and $(c,b)$ are in $E(G)$. 
In this case, there is no path of length $2$ from $a$ to $b$ in $G'$: if the path goes through part $C$ and has length $2$ then it must be of the form $a \to c \to b$ for some $c \in C$ which is a contradiction. If the path goes through $x$ it will have distance at least $3$ since the distance from $x$ to any node in $B$ is exactly $2$.
Finally, if the path goes through $DG(A)\setminus A$, through $y$, then it also has length at least $3$.
Therefore, the min-diameter of $G'$ is at least $3$.
\end{proof}
Our new graph $G'$ has $O(n)$ nodes and $O(n\log{n})$ edges, which implies that a subquadratic algorithm refutes the OV conjecture. 
$G'$ is a DAG by construction: $DG(A), DG(B),DG(C)$ are DAGs and the rest of the comparisons in the topological order are $DG(A)<x<DG(C)<y<DG(B)$.
\end{proof}
\begin{lemma}
\label{lem:MinDiam}
If there is a $(2-\eps)$-approximation algorithm for Min-Diameter on a sparse weighted graph that runs in subquadratic time, then the OV Conjecture is false.
\end{lemma}

\begin{proof}

We show how an algorithm that distinguishes between diameter $t+1$ and $2t$ on a sparse graph allows us to solve OV.
Let $G$ be the OV graph. We construct a directed graph $G'$ as follows.
Take $G$ (with all its nodes and edges) and add three nodes $x,y,z$ to it.
The edges that were present in $G$ will have weight $t/2$ in $G'$.
We connect $x$ to and from $A$ with edges $(a,x),(x,a)$ for every node $a \in A$, with weights $w(a,x)=1,w(x,a)=t$, and then we add edges $(c,x)$ for every node $c \in C$, with weight $w(c,x)=1$.
We connect $y$ to and from $B$ with edges $(b,y),(y,b)$ for every node $b \in B$, with weights $w(b,y)=t, w(y,b)=1$,  and then we add edges $(y,c)$ for every node $c \in C$, with weight $w(y,c)=1$.
Finally, we connect $z$ to and from $C$ with edges $(c,z),(z,c)$ for every node $c \in C$, with weights $w(c,z)=w(z,c)=t/2$, and we add an edge $(y,x)$ with weight $w(y,x)=1$.

\begin{claim}
The min-diameter of $G'$ is $t+1$ if $G$ is a NO instance of OV and is at least $2t$ otherwise.
\end{claim}

\begin{proof}
First, assume that $G$ is a NO instance and therefore for every pair $a\in A,b \in B$ there is a node $c \in C$ such that both edges $(a,c),(c,b)$ are in $E(G)$ and therefore are also in $E(G')$.
In this case, for all pairs of nodes $u,v \in V(G')$, the min-distance in $G'$ is no more than $2$:
\begin{enumerate} 
\item If $u \in A, v \in B$ then the distance from $u$ to $v$ is $2 t/2 = t$.
\item If either $u,v \in A$, or $u,v\in B$, or $u,v\in C$, then the distance is $t+1$, via $x,y, or z$.
\item If $u \in A$ and $v \in C$ (or vice versa) then the distance from $v$ to $u$ (or vice versa) is at most $t+1$ via the path through $x$.
\item If $u \in B$ and $v \in C$ (or vice versa) then the distance from $u$ to $v$ (or vice versa) is at most $t+1$ via the path through $y$.
\item If $u=x$ then it has min-distance $1$ to every node in $A \cup C \cup \{y\}$, distance $t+1$ from every node $b\in B$ via the path $b \to y \to x$, and distance $t/2+1$ from $z$ via the path $z \to c \to x$ for any $c \in C$. The $u=y$ case is symmetric.
\item If $u=z$ then it has min-distance $t/2$ to every node in $C$ and min-distance $t/2+t/2$ to every node in $A \cup B$.
\end{enumerate}
On the other hand, assume that $G$ is a YES instance and let $a\in A, b \in B$ be such that there is no $c \in C$ for which the edges $(a,c)$ and $(c,b)$ are in $E(G)$. 
In this case, there is no path of length less than $2t$ from $a$ to $b$ in $G'$: if the path does not any of the nodes $x,y,z$ it must be of the form $a \to c \to b$ which is a contradiction, while if it uses $x$ it must be of the form $a \to \cdots \to x \to a' \to c \to \cdots \to \cdots b$ which will have length at least $2t$, the case in which the path goes from $b$ to $a$ and uses $y$ is symmetric, and finally, if we use $z$ we also incur an addition weight of $t$ resulting in length at least $2t$.
Therefore, the min-diameter of $G'$ is at least $2t$.
\end{proof}

\end{proof}

\paragraph{Roundtrip Diameter.}

\begin{lemma}
\label{lem:RoundDiam}
If there is a $(3/2-\eps)$-approximation algorithm for Roundtrip-Diameter on a sparse graph that runs in subquadratic time, then the OV conjecture is false.
\end{lemma}

\begin{proof}
Any algorithm distinguishing between roundtrip diameter $4$ and $6$ can distinguish between undirected diameter $2$ and $3$ - take the undirected graph, bi-direct the edges and run the algorithm for roundtrip-diameter.
The latter task cannot be done in subquadratic time \cite{RV13}.
\end{proof}







\subsection{Lower Bound for Estimating All eccenricities}


The following construction shows that the $3/2$ factor that we have for Radius and Diameter (in undirected graphs) is unlikely to be achievable if we want to estimate all the eccentricities in the graph. 
Only a $5/3$ approximation is known for this problem in subquadratic time \cite{diametersoda14}.

\begin{reminder}{Theorem~\ref{thm:allecc}}
 A $(5/3-\delta)$ approximation algorithm for the eccentricities of all nodes in undirected sparse graphs that runs in subquadratic time refutes the Orthogonal Vectors Conjecture.
\end{reminder}

\begin{proof}
Let $A,B \subseteq \{0,1\}^d, |A|=|B|=n$ be an instance of OV, we will use it to construct an undirected sparse graph $G$ as follows.
Abusing the notation, construct a set of nodes $A$ that contains a node $a$ for every vector in $A$, and similarly construct a set of nodes $B$ from the vectors $B$.
Add a set of nodes $C$ corresponding to the coordinates $j \in [d]$ and for every vector $v \in A \cup B$ we add an edge $\{v,j\}$ iff $v[j]=0$.
Note that now, there is a $2$-path from a node $a \in A$ to a node $b\in B$ iff the vectors $a,b$ are not orthogonal.
Then, we also add a set $B'$ that contains a copy $b'$ of every node $b \in B$, such that $b'$ is only connected with one edge to $b$.
Finally, add two nodes $x,y$, connect $x$ to every node in $A$, connect $y$ to every node in $C$, and add the edge $\{x,y\}$.

We now claim that for every node $a\in A$, the eccentricity is $5$ if the vector $a$ is orthogonal to some vector $b \in B$, and it is $3$ otherwise.

To prove our claim we first show that any node $a \in A$ will have distance $\leq 3$ to all the nodes in $A \cup C \{x,y\}$:
there is a $2$-path via $x$ to every other node in $C$ and there is a $3$-path to every node in $C$ via $x,y$.
Now, on the one hand, if $a \in A$ is not orthogonal to any $b\in B$, then in our graph, there will be a $2$-path via $C$ from $a$ to every node in $B$ and therefore there will be a $3$-path to every node in $B'$.
Thus, in this case, the eccentricity is $3$.
On the other hand, if $a \in A$ is orthogonal to some $b \in B$, then the only paths from $a$ to $b$ have length $4$, and therefore the distance from $a$ to $b'$ is $5$, and the eccentricity of $a$ is at least $5$.

This claim shows that an algorithm estimating all the eccentricities within $(5/3-\delta)$ allows us to solve OV by this reduction.
It is important to note that the radius and diameter will not be determined by nodes in $A$ in our graph, and therefore a better than $5/3$ approximation for those parameter (which is achievable in subquadratic time \cite{RV13}) is not enough to solve OV.
Indeed, the node $x$ will have eccentricity $4$ regardless of the OV instance.

The number of nodes in the graph we constructed is $O(n)$ and the number of edges is $O(nd)$.
It can be easily turned into a sparse graph on $O(nd)$ nodes.
Thus, a subquadratic $(5/3-\delta)$ approximation allows us to solve OV in time $(n^{2-\eps} \cdot d^{2-\eps})$, for some $\eps>0$, which is enough to refute the OV conjecture.

\end{proof}


Our lower bounds for Diameter are summarized in Table~\ref{tab:res-diam}.

\begin{table*}\centering\small
\begin{tabular}{|c | c | c | c|}
\hline 
\multicolumn{4}{|c|}{Diameter Variants} \\
\hline
Problem & Definition & Upper Bound & OV Conjecture \\
\hline
  \UndirectedDiameter{} &
  $\max\limits_{u,v} d(u,v)$ &
  $3/2$ in $\tO(m\sqrt{n})$ [\cite{RV13}] &
  $3/2$  [\cite{RV13}]\\
\hline
  \MaxDiameter{} &
  $\max\limits_{u,v} d(u \to v)$ &
  $3/2$ in $\tO(m\sqrt{n})$ [\cite{RV13}] &
  $3/2$ [\cite{RV13}] \\
\hline
  \MinDiameter{} &
  $\max\limits_{u,v} \min\{d(u \to v), d(v \to u)\}$ &
  $n^\epsilon$ in $\tO(mn^{1-\epsilon})$ [Lem~\ref{lem:min-diameter-upper}] &
  $2$ on weighted  [Lem~\ref{lem:MinDiam}] \\
\hline
  \MinDiameter{} on DAGs &
  $\max\limits_{u<v} d(u \to v)$ &
  $2$ in $\tO(m)$ [Thm~\ref{thm:min-diameter-dag-upper}] &
  $3/2$  [Lem~\ref{lem:MinDiamDAG}] \\
\hline
  \RoundtripDiameter{} &
  $\max\limits_{u,v} \{d(u \to v) + d(v \to u)\}$ &
  $2$ in $\tO(m)$ [metric]&
  $3/2$ [Lem~\ref{lem:RoundDiam}] \\
\hline
\end{tabular}
\caption{Our Bounds for Various Diameter Problems}
\label{tab:res-diam}
\end{table*}
