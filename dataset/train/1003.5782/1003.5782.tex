\documentclass[fleqn,12pt]{article}

\usepackage{graphicx}
\usepackage[figuresright]{rotating}
\usepackage{setspace}




\newtheorem{theorem}{Theorem}
\newtheorem{lemma}{Lemma}
\newtheorem{corollary}{Corollary}
\newtheorem{claim}{Claim}
\newtheorem{conjecture}{Conjecture}
\newtheorem{fig}{Figure}
\newtheorem{proposition}{Proposition}



\usepackage{espcrc1}







\newtheorem{definition}{Definition}
\newtheorem{example}{Example}
\newtheorem{problem}{Problem}
\newtheorem{remark}{Remark}




\newenvironment{proof}[1][Proof.]{\begin{trivlist}
\item[\hskip \labelsep {\bfseries #1}]}{\end{trivlist}}







\usepackage{amsmath}
\usepackage{amsfonts}



\begin{document}

\title{Bricks and conjectures of Berge, Fulkerson and Seymour}

\author{Vahan V. Mkrtchyan\address[MCSD]{Paderborn Institute for Advanced Studies in Computer Science and Engineering,
Paderborn University, Warburger Str. 100, 33098 Paderborn, Germany}\thanks{The author is supported by a fellowship from Heinrich Hertz-Stiftung}
\thanks{email: vahanmkrtchyan2002@\{ysu.am, ipia.sci.am,
yahoo.com\}}
                        and
        Eckhard Steffen \addressmark[MCSD]\thanks{email: es@upb.de}}



\runtitle{Bricks and conjectures of Berge, Fulkerson and Seymour}
\runauthor{Vahan Mkrtchyan, Eckhard Steffen}
\maketitle

\begin{abstract}
An -graph is an -regular graph where every odd set of vertices is connected by at least  edges to the rest of the graph. Seymour conjectured that any -graph is -edge-colorable, 
and also that any -graph contains  perfect matchings such that each edge belongs to two of them. We show that the minimum counter-example to either of these conjectures is a brick.
Furthermore we disprove a variant of a conjecture of Fan, Raspaud. 
\end{abstract}


\section{Introduction and definitions}

We consider finite graphs  with vertex set  and edge set . The graphs might have multiple 
edges but no loops, and throughout this paper we assume that the graphs under consideration are connected (otherwise we will mention this explicitely). 
Terms and concepts that we do not define can be found in Chapter 3 of \cite{Handbook}.

A perfect matching of a graph  is a matching covering all vertices of , and  denotes the number of odd components of .  
Tutte characterized the graphs with perfect matching.

\begin{theorem}\label{Tutte}(Tutte) A graph  has a perfect matching if and only if , for each .
\end{theorem}

A class of graphs possessing a perfect matching is the class of -graphs \cite{Seymour_1979}, which are -regular graphs with   for every odd , 
where  is the set of edges of  with precisely one end in . The following conjectures are due to Seymour \cite{Seymour_1979}.


\begin{conjecture}\label{SeymourColoring} Any -graph is -edge-colorable.
\end{conjecture}

\begin{conjecture}\label{SeymourPerfects} Any -graph contains  perfect matchings such that each edge belongs to precisely two of them.
\end{conjecture}

Conjecture \ref{SeymourPerfects} was first formulated by Berge, Fulkerson for  (Berge Fulkerson Conjecture).  
The Berge Fulkerson Conjecture implies the following conjecture made by Fan, Raspaud \cite{FanRaspaud}.

\begin{conjecture}\label{FanRaspaud} Any -graph contains three perfect matchings  such that .
\end{conjecture}

Conjecture \ref{SeymourPerfects} implies that every -graph has  perfect matchings such that any three of them have an empty intersection. In the context of the conjectures
of Seymour, Berge, Fulkerson and of Fan, Raspaud, the following statement seems to be a natural generalization of Conjecture \ref{FanRaspaud}.

\begin{conjecture}\label{FanRaspaud_r} For , any -graph contains  perfect matchings such that the intersection of any three of them is empty.
\end{conjecture}

As the Berge, Fulkerson Conjecture implies Conjecture \ref{FanRaspaud}, Conjecture \ref{SeymourPerfects} implies Conjecture \ref{FanRaspaud_r}.
This paper studies the structure of minimum counter example to Conjectures \ref{SeymourColoring} and \ref{SeymourPerfects}, 
and a variation of \ref{FanRaspaud_r}.

A graph  is matching covered if every edge belongs to a perfect matching. It is factor-critical if  has a perfect matching for every vertex , and it 
is bi-critical if  has a perfect matching for every pair of vertices  and .  A barrier in a matching covered graph is a set 
such that . Note that a single vertex in a matching covered graph is a barrier. A non-bipartite, bi-critical and -vertex-connected graph is a brick. 
This paper proves that a minimal counter example to Conjectures \ref{SeymourColoring} and \ref{SeymourPerfects} is a brick. 

There are almost no results on these conjectures, and one might think about some variations of the original conjectures. 
It is a natural question whether a perfect matching can be fixed in Conjectures \ref{SeymourPerfects} or \ref{FanRaspaud}, say: 
Let  be an -graph () and  a perfect matching of , then  has  perfect matchings  such that 
for any  the intersection  is empty. We will show that this this is not true for any odd . 


\section{The main results}

Let  be an -graph. An odd set  with  is non-trivial, if .
We start with the following theorem. 

\begin{theorem}\label{decomposition}Any -graph  satisfies at least one of the following conditions:
\begin{enumerate}
	\item [(1)]  is bipartite;
	\item [(2)] there is non-trivial odd , such that ;
	\item [(3)]  is bi-critical.
\end{enumerate}
\end{theorem}

\begin{proof} Clearly, an -graph is matching covered (see Exercise 3.4.4 in \cite{Lov} p. 113), and let
 be any maximal barrier with . Then,  has  componets , which are factor-critical (see Lemma 4.6. in \cite{Handbook} p. 198). 
Since  are odd, it follows that , for . On the other hand, the set  can receive at most  edges, 
thus  is an independent set and , for . 

Now, consider the following two cases:

Case 1: There is a maximal barrier  with . 
If the components of  are isolated vertices, then  is bipartite and hence (1). If there is a component  of , 
that is not an isolated vertex, then  and  satisfies the condition (2) of the theorem.

Case 2: for any maximal barrier , we have .
Let .  is factor-critical, which means that for any vertex , the graph  has a perfect matching, thus  satisfies the condition (3).

\end{proof}



\begin{theorem}\label{BrickTheorem} Let  be a bi-critical non-bipartite -graph, that contains no non-trivial odd set  with . Then  is a brick.
\end{theorem}

\begin{proof} Since  is bi-critical and non-bipartite, and -graphs are -vertex-connected, it suffices to show that  does not have a -vertex-cut. 

Suppose that there is one and let  be the vertices of the -cut. Since  is bi-critical, the graph  has a perfect matching, thus all connected 
components  of  are even. Let  denote the number of edges of  that connect the vertices  and . For  let  and  
be the number of edges that connects  and  to , respectively. Clearly,


The sets  and  are odd. Thus  and . It follows that

and

and hence
 and , which is a contradiction.

\end{proof}

\begin{theorem} A minimum counter-example to either of conjectures \ref{SeymourColoring} and \ref{SeymourPerfects} is a brick.
\end{theorem}

\begin{proof} Let  be a minimum counter-example to conjecture \ref{SeymourColoring}. Since bipartite -regular graphs are -edge-colorable,  is not bipartite. Next, we show that there is no non-trivial odd , such that . 

Suppose that there is one. Then consider the two graphs  and  that are obtained from  by contracting  and  to a vertex, respectively. Clearly  and  are -graphs; moreover since they are smaller than , they are -edge-colorable. Now, it is not hard to see that an -edge-coloring of  can be obtained from those of  and , which would contradict the choice of .

Thus,  contains no non-trivial odd , with . Theorem \ref{decomposition} implies that  is bi-critical, and hence  is a brick by 
Theorem \ref{BrickTheorem}.

The proof for conjecture \ref{SeymourPerfects} follows the same lines.

\end{proof}

An -graph  is unslicable if for any perfect matching , the graph  is not an -graph. Rizzi \cite{Rizzi} constructed unslicable -graphs for every . 

\begin{theorem} For every , there is a -graph  with perfect matching , such that for any  perfect matchings  there are 
 such that .
\end{theorem}

\begin{proof} Let   be an odd number and  an unslicable -graph. Let  be a (multi-) cycle of length , where every edge has multiplicity , 
i.e.  is -regular. Replace every vertex  of  by a copy  of  to obtain a -regular graph . Note that the {\em old} edges of  in  form a 
perfect matching  of . 

We first show that  is an -graph. Clearly  is even. Let  be an odd set and assume that . If  contains an edge of 
a cycle , then it contains at least  of them, i.e. . This implies that  is a cut vertex in , contradicting the fact that  is 2-vertex-connected. Thus 
 contains at least one old edge of  and hence , contradicting our assumption. 

Thus we may assume that  contains only old edges of . Let  be the subset of  from which  is obtained in the transfornmation from  to . 
 is an odd set in  and hence 
 is an odd set in . But then , contradicting the fact that  is an -graph. Thus  is a -graph. 

Now assume that  has  perfect matchings  such that , for any . 
Consider . Since it is an odd cut, it follows
that  , for . Furthermore  by the choice of . If there is a perfect matching, say  such that 
, then, since we assume that  (), the remaining  perfect matching  share  edges.
Thus there are , such that the intersection of , , and  is not empty. 
Thus every perfect matching of  contains precisely one edge of , and they are pairwise disjoint on . 
But this implies that they induce  pairwise disjoint perfect matchings  on . Thus  contains an -graph, contradicting the fact that
 is unslicable. 
\end{proof}


\begin{thebibliography}{99}

\bibitem{FanRaspaud} G. Fan and A. Raspaud, Fulkerson's Conjecture and circuit covers, J. Combin. Theory Ser. B 61 (1994) 133-138.

\bibitem{Handbook} R.L. Graham, M. Gr\"otschel, L. Lov\'asz, Handbook of Combinatorics, Vol. 1, 2, Elsevier, Amsterdam, 1995

\bibitem{Lov} L. Lov\'asz, M.D. Plummer, Matching theory, Ann. Discrete Math. 29 (1986)

\bibitem{Rizzi} R. Rizzi, Indecomposable -graphs and other counterexamples, J. Graph Theory 32 (1999) 1-15

\bibitem{Seymour_1979} P. D. Seymour, On multi-colourings of cubic graphs, and conjectures of Fulkerson and Tutte, Proc. London Math. Soc. 38 (1979) 423-460



\end{thebibliography}
\end{document}
