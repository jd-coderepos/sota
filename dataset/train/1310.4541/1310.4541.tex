

\documentclass[a4paper,twoside]{article}


\usepackage{epsfig}
\usepackage{subfigure}
\usepackage{calc}
\usepackage{amssymb}
\usepackage{amstext}
\usepackage{amsmath}
\usepackage{amsthm}
\usepackage{multicol}
\usepackage{pslatex}
\usepackage{SciTePress}
\usepackage[small]{caption}

\usepackage{multirow} \usepackage{booktabs} \usepackage{lineno}

\newtheorem{fact}{Fact}
\newtheorem{definition}{Definition}

\subfigtopskip=0pt
\subfigcapskip=0pt
\subfigbottomskip=0pt


\begin{document}

\title{A Dynamic Programming Solution to the\\Monotonic Path of Minimal Cost in a 3-Rows Matrix}

\author{\authorname{Marcelo Cicconet and Davi Geiger}
\affiliation{New York University}
\email{cicconet@gmail.com, geiger@cims.nyu.edu}
}

\keywords{dynamic programming, optimal path}

\abstract{We consider the problem of finding the path of minimal
cost going from left to right in a 3-rows matrix, starting at the third row, and not going downwards, where there's an additional cost related to not changing rows,
such that the higher the change in intensity
within the row, the higher the cost of not moving upwards.}

\onecolumn \maketitle \normalsize \vfill


\paragraph{Problem Statement}

We consider a 3-rows matrix, similar to the one shown in Figure~\ref{fig:dp}~(a),
in which the values range from 0 to 1.
We are looking for the path of minimal
cost going from left to right, starting at the third row, and not going downwards.
That is, once the path is in row , it can either remain
on that row or go to row  (when  exists).
Furthermore, there's an additional cost related to not changing rows,
such that the higher the change in intensity
within the row, the higher the cost of not moving upwards.

\begin{figure}[t!]
\begin{center}
	\includegraphics[width=\linewidth]{Fig3EntPlusRad.png}\\
	(a)\\
	\vspace{0.3cm}
	\includegraphics[width=\linewidth]{FigDP_Control_Ex2_D.png}\\
	(b)
\end{center}
\caption{(a) Example of input. (b) Optimal path.}
\label{fig:dp}
\end{figure}

\paragraph{Solution by Dynamic Programming}
Dynamic programming involves building two matrices:
, of accumulated path costs, and , of predecessors (which allows recovering
the path of minimal cost).

Let  be the matrix of costs (e.g., Figure~\ref{fig:dp}~(a), with white color
representing cost zero).

We start by defining a ``windowed'' derivative, .
Let  be a window-size parameter; and ,  the number of rows, columns in ,
respectively. For , and ,
set


and the remaining columns of  by replicating the closest filled column.

Let  be a ``decay'' parameter. We define a ``derivative strength,'' ,
as


 ranges from  to  when  goes from  to .

The first column of  is set as the first column of .
For , we set

\vspace{-0.5cm}


\noindent
where  assumes values  or  and  controls the
weight of the penalty .
For the example shown in Figure~\ref{fig:dp} we used , ,
and . These parameters are set empirically.

To recover the optimal path , we compute
, and for
, we set .

Figure~\ref{fig:dp} shows an example of pair input/solution.

\paragraph{Reference}
Bellman, R. Dynamic Programming. Dover Publications, Incorporated. 2003.


\end{document}
