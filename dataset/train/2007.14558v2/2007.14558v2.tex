

\documentclass[letterpaper, 10 pt, conference]{ieeeconf}  



\IEEEoverridecommandlockouts                              

\overrideIEEEmargins                                      

\clearpage{}\usepackage{epsfig}
\usepackage{graphicx}
\usepackage{amsmath}
\usepackage{amssymb}
\usepackage{color, soul}

\usepackage[labelformat=simple]{subcaption}
\renewcommand\thesubfigure{(\alph{subfigure})}
\usepackage{multirow}
\usepackage[breaklinks=true,bookmarks=false]{hyperref}
\usepackage{cleveref}

\usepackage{booktabs}
\usepackage{comment}
\usepackage{makecell}


\hypersetup{
    colorlinks=true,
    linkcolor=blue,
    filecolor=magenta,      
    urlcolor=magenta,
}

\clearpage{}
\clearpage{}\newcommand{\xhdr}[1]{\vspace{5pt} \noindent {\textbf{#1} }}
\newcommand{\etal}{\textit{et al}.}
\newcommand{\etc}{\textit{etc}.}
\newcommand{\norm}[1]{\left\lVert#1\right\rVert}


\DeclareMathOperator*{\argmax}{arg\,max} \DeclareMathOperator*{\argmin}{arg\,min}\clearpage{}




\title{\Large \bf
Supplementary File: \\
\large BiTraP: Bi-directional Pedestrian Trajectory Prediction with Multi-modal Goal Estimation}

\author{Yu Yao, Ella Atkins, Matthew Johnson-Roberson, Ram Vasudevan, and Xiaoxiao Du
\thanks{\tt\scriptsize *\{brianyao, ematkins, mattjr, ramv, xiaodu\}@umich.edu}
}

\begin{document}
\maketitle
\section{CVAE Preliminaries}
A Conditional Variation Autoencoder (CVAE) is a conditional generative model designed to output target data $Y$ based on latent variable $Z$ and observation $X$~\cite{sohn2015learning}. A CVAE consists of three modules: a \textbf{conditional prior network} $p_\theta(Z|X)$ to model latent variable $Z$ conditioned on observation $X$, a \textbf{recognition network} $q_\phi(Z|X, Y)$ to capture dependencies between $Z$ and target $Y$, and a \textbf{generation network} $p_\psi(Y|X, Z)$ to generate the target $Y$, where $\phi$, $\theta$, and $\psi$ represent network parameters. Stochastic latent variable $Z\in\mathbb{R}^d$ is sampled from a pre-defined distribution format such as a Gaussian distribution. The CVAE samples $Z$ and generates target $Y$ conditioned on observation $X$. The objective of a typical CVAE model is to maximize its variational lower bound
\begin{align}
\begin{split}
\label{eq:vlb}
    \max_{\theta,\phi,\psi}  &\mathbb{E}_{
    q_\phi(Z|X,Y)}
    \Big[
        \log{p_\psi(Y|X,Z)}
    \Big]\\
    &-KL\Big(q_\phi(Z|X,Y)||p_\theta(Z|X)\Big),
    \end{split}
\end{align}
where the first term maximizes the expectation of the log-likelihood 
of the target in the predicted distribution; 
the $K$-$L$ (Kullback–Leibler) divergence term minimizes the difference 
between the recognition network and the conditional prior network. In this paper, we designed a modified CVAE with two generation networks and optimize both networks end-to-end.

\section{Robot Navigation Simulation Experiment Using BiTraP}

To quantitatively analyze application of the BiTraP-GMM and BiTraP-NP models to robot navigation tasks, we designed a simulated robot navigation
experiment based on the ETH-UCY bird's-eye view dataset. In this experiment, given  predicted pedestrian trajectory distributions in a scene using our BiTraP models and pre-planned paths for a robot, we show that we are able to compute the collision likelihood for each path, and thus are able to predict collision rate and select the safest path for the robot. Assuming a mobile robot navigates among pedestrians, we present results on two tasks: 1) Select the safest path for the robot and 2) Predict whether a path will collide with any other pedestrians in the scene. In this section, we first introduce our experiment setup. Then, we present evaluation results of our BiTraP models on path selection and collision prediction tasks.


\begin{figure}[htbp]
    \centering
    \includegraphics[width=0.48\textwidth]{SuppFigures/simulation_exp.png}
    \caption{Generation of Monte Carlo (MC) robot trajectories for collision detection experiments using Bezier curves. We illustrate five MC trajectory samples including start (robot icon) and end (red star) waypoints. Predicted trajectory distributions of neighbor pedestrians are plotted as a heat map; their walking directions are indicated by black arrows.}
    \label{fig:simulation_exp}
\end{figure}

\xhdr{Experimental Setup.} We selected all samples with more than one pedestrian in the test split~\cite{gupta2018social} from ETH-UCY. Each sample has a node pedestrian (the pedestrian used for testing in previous work) and several neighbor pedestrians (the pedestrians used for social modeling in previous work) as
in~\cite{gupta2018social,salzmann2020trajectron++}. 
We regard the node pedestrian as a "robot" navigating among other neighbor pedestrians. The starting and goal points of the "robot" are the same as the current position and goal point of the node pedestrian. A sample scene with one ``robot'' navigating among four other pedestrians is shown in Fig.~\ref{fig:simulation_exp}. For the robot, $100$ Monte Carlo (MC) paths were generated from start state to end point following quadratic and cubic Bezier curves~\cite{gallier2000curves}. Other more complex path planners could be used to generate additional experimental datasets. We assume the robot must reach the designated goal in 12 time steps, matching the prediction horizon for the pedestrian node in each scene. We uniformly generate waypoints along the path and randomly shift each by up to $\pm50\%$ of the step length, resulting in a trajectory sequence containing 12 random waypoints. Other pedestrians follow their original (ground truth) trajectories in the scene. For each neighbor pedestrian, we run BiTraP-NP and BiTraP-GMM separately. Each method samples 2000 future trajectories to fit one Gaussian Kernel Density Estimation (KDE) model for each pedestrian as the predicted future distribution. 
Then, we compute the maximum KDE log-likelihood of all the waypoints on all robot MC paths and treat this log-likelihood value as a collision score. The higher the collision score, the more likely a collision will happen along this path. Given these collision scores, we compute the safest path collision rate (SPCR) as reported in \textit{Task 1} below.  Receiver operating characteristic (ROC) and precision-recall (P-R) curve results are reported in \textit{Task 2}. 

\xhdr{Task 1: Predict the Safest Path.} We mark the robot MC path in each scene with minimum collision score as the ``safest" (lowest collision likelihood) path. Then, we compute Euclidean distances between each safest path waypoint and other pedestrians' ground truth future trajectories. A collision is tallied if the minimum distance between a path and any pedestrians in the scene is less than $0.2$ meters. Collision rate is computed as the number of paths with collision divided by the total number of safest paths. Due to the randomness in MC path generation, we conducted the simulation experiment five times with BiTraP-NP and BiTraP-GMM predictors separately and report collision rate mean ($\mu$) and standard deviation ($\sigma$) values in Table~\ref{tab:spcr_auc_ap}. As a comparison, we also present the collision rate of a randomly selected path among the 100 MC paths. The randomly selected paths do not have very high collision rates since the paths are planned based on pedestrian ground truth start and goal positions which are less likely to be involved in a collision. Compare to randomly selected paths, paths selected by our methods reduce the SPCR by a large margin. This shows that our predictors are effective for safest path selection. Both of our BiTraP methods achieve collision rate lower than $1\%$ on \textit{ETH}, \textit{Hotel} and \textit{Zara1} datasets. The \textit{Univ} dataset is more difficult due to its high pedestrian density, and \textit{Zara2} is most difficult because many pedestrian trajectories are quite close to each other. 
BiTraP-GMM shows lower SPCRs than BiTraP-NP on four datasets, indicating that it predicted more accurate (compared to ground truth) distributions. On \textit{Zara1}, BiTraP-NP outperforms BiTraP-GMM by a small margin. BiTraP-NP ANLL and FNLL metric values as reported in the main paper are still higher than BiTraP-GMM values. A possible explanation is that BiTraP-NP predicts more diverse distributions thus detects some collisions not identified by BiTraP-GMM.

\begin{table}[h]
    \centering
        \caption{SPCR($\mu\pm\sigma$), AUC and AP results of our methods on ETH-UCY data group.}
        \resizebox{.49 \textwidth}{!} {
    \begin{tabular}{c|c|c|c}
        \toprule
         & Random from 100 & BiTraP-NP & BiTraP-GMM \\
         & (SPCR) & (SPCR/AUC/AP) & (SPCR/AUC/AP) \\
        \midrule
        ETH & $0.6\pm0.4\%$ & $0.3\pm0.1\%/\,92.3/\,24.2$& $\mathbf{0.1\pm0.1\%}/\,\mathbf{95.5}/\,\mathbf{26.0}$ \\
        HOTEL & $0.4\pm0.3\%$ & $0.1\pm0.1\%/\,86.4/\,22.4$ & $\mathbf{0.0\pm0.0\%}/\,\mathbf{91.6}/\,\mathbf{29.1}$ \\
        Univ & $8.5\pm1.4\%$ & $5.8\pm0.5\%/\,81.0/\,33.4$ & $\mathbf{3.6\pm0.2\%}/\,\mathbf{87.6}/\,\mathbf{43.4}$ \\
        Zara1 & $2.4\pm0.5\%$ & $\mathbf{0.6\pm0.2\%}/\,88.9/\,38.6$ & $0.8\pm0.3\%/\,\mathbf{90.4}/\,\mathbf{41.6}$ \\
        Zara2 & $6.1\pm0.6\%$ & $3.2\pm0.1\%/\,81.0/\,44.0$ & $\mathbf{2.5\pm0.3\%}/\,\mathbf{87.5}/\,\mathbf{52.6}$ \\
        \bottomrule
    \end{tabular}}
    \label{tab:spcr_auc_ap}
\end{table}



\begin{figure}[h!]
    \begin{subfigure}[h]{0.48\textwidth}
        \includegraphics[width=0.95\textwidth]{SuppFigures/eth_roc_pr.png}
        \caption{ETH}
        \label{fig:eth_roc_pr}
    \end{subfigure}
    \begin{subfigure}[h]{0.48\textwidth}
        \includegraphics[width=0.95\textwidth]{SuppFigures/hotel_roc_pr.png}
        \caption{Hotel}
        \label{fig:hotel_roc_pr}
    \end{subfigure}
    
    \begin{subfigure}[h]{0.48\textwidth}
        \includegraphics[width=0.95\textwidth]{SuppFigures/univ_roc_pr.png}
        \caption{Univ}
        \label{fig:univ_roc_pr}
    \end{subfigure}
    \begin{subfigure}[h]{0.48\textwidth}
        \includegraphics[width=0.95\textwidth]{SuppFigures/zara1_roc_pr.png}
        \caption{Zara1}
        \label{fig:zara1_roc_pr}
    \end{subfigure}
    
    \begin{subfigure}[h]{0.48\textwidth}
        \includegraphics[width=0.95\textwidth]{SuppFigures/zara2_roc_pr.png}
        \caption{Zara2}
        \label{fig:zara2_roc_pr}
    \end{subfigure}
    \caption{ROC (left) and P-R (right) curves of BiTraP-NP and BiTraP-GMM on ETH dataset.}
    \label{fig:roc_pr}
\end{figure}


\begin{table*}[ht!]
    \centering
    \begin{tabular}{l|cccc|cccc}
        \toprule
        \multirow{2}{*}{Method} & \multicolumn{4}{c|}{KDE NLL} & \multicolumn{4}{c}{FDE ML} \\
        \cmidrule{2-9} 
        & @1s & @2s & @3s & @4s & @1s & @2s & @3s & @4s \\
        \midrule
        Trajectron++ base~\cite{salzmann2020trajectron++} & -2.69 & -2.46 & -1.76 & -1.09 & 0.03 & 0.17 & 0.37 & 0.60 \\
        Trajectron++ $\int$, map~\cite{salzmann2020trajectron++} & -5.58 & -3.96 & -2.77 & -1.89 & \textbf{0.01} & 0.17 & 0.37 & 0.62 \\
        BiTraP-GMM (ours) & \textbf{-6.08} & \textbf{-4.21} & \textbf{-2.98} & \textbf{-2.05} & 0.02 & \textbf{0.15} & \textbf{0.35} & \textbf{0.58} \\
        \bottomrule
    \end{tabular}
    \caption{Pedestrian-only trajectory prediction results on nuScenes dataset.}
    \label{tab:nuscenes_ped}
\end{table*}
\xhdr{Task 2: Predict Collision for Any Path.} The collision rate metric above only evaluates the safest path as selected by a trajectory predictor thus neglects all other paths. In the real-world, a trajectory predictor must be sufficiently accurate for the robot to accurately predict future collisions with high precision with a low missing rate (high true positive rate, TPR) and a low false alarm rate (low false positive rate, FPR). To show the performance of BiTraP-NP and BiTraP-GMM predictors in terms of these metrics, we plotted the collision prediction ROC curve and P-R curve as follows. First, we collected all MC paths for the robot and tallied their collision scores. By setting a threshold $\gamma$, we can classify a path as collided (positive) or not collided (negative) and compute the TPR (i.e., recall), FPR and precision values. The ground truth label of each path is computed in the same way as before. By decreasing $\gamma$ from a maximum value to minimum value (6 and -10 in this work), we plot the ROC and P-R curves shown in Fig.~\ref{fig:roc_pr}. The corresponding area under curve (AUC) and average precision (AP) are presented in Table~\ref{tab:spcr_auc_ap}. In this work, AP is computed by equally spaced recall levels \{1/40, 2/40,...,1\} following~\cite{simonelli2019monodis}.




As shown in Fig.~\ref{fig:roc_pr} and Table~\ref{tab:spcr_auc_ap}, both BiTraP-NP and BiTraP-GMM methods achieve high AUCs (e.g., $>90$ on \textit{ETH}). Generally, BiTraP-GMM outperforms BiTraP-NP by a small margin in terms of both AUC and AP (e.g., 95.5 vs 92.3 AUC, and 26.0 vs 24.2 AP on \textit{ETH}). Note that in real-world mobile robot applications missed collision detection (false negative) is unacceptable due to safety. That is to say, a high TPR (recall) is required. As can be observed in the higher TPR regions (x-axis) of the P-R curves, BiTraP-GMM outperforms BiTraP-NP on \textit{ETH} (Fig.~\ref{fig:eth_roc_pr}) and \textit{Hotel} (Fig.~\ref{fig:hotel_roc_pr}), and both methods perform similarly on \textit{Zara1} (Fig.~\ref{fig:zara1_roc_pr}). On \textit{Univ} (Fig.~\ref{fig:univ_roc_pr}) and \textit{Zara2} (Fig.~\ref{fig:zara2_roc_pr}), when the TPR is greater than a relatively high value (say 0.8), the FPR are higher  ($>0.2$) than in the other datasets, indicating increased chance of false alarms on these two datasets. 

Compared to the ROC curve, the P-R curve is more suitable for imbalanced datasets  due to the fact that it evaluates the fraction of true positives among positive predictions. This fits our case where the ratio of with-collision to no-collision paths is around 1:140, a large imbalance. On \textit{Univ} and \textit{Zara2} (Fig.~\ref{fig:univ_roc_pr} and~\ref{fig:zara2_roc_pr}), BiTraP-GMM has higher precision than BiTraP-NP across almost all recall values. On the other hand, on \textit{ETH}, \textit{Hotel} and \textit{Zara1} (Fig.~\ref{fig:eth_roc_pr}~\ref{fig:hotel_roc_pr} and~\ref{fig:zara1_roc_pr}), the two methods achieve similar precision at higher recall regions (e.g., when recall$>0.6$). This is because when the threshold $\gamma$ is too low, many paths are predicted as collided by both methods. 

The ROC and P-R curves also verified our observation regarding the diversity of the predicted trajectory distribution as described in the main paper. At a fixed TPR on the ROC curves, we observe that BiTraP-NP always has a greater FPR than BiTraP-GMM, consistent with our hypothesis that BiTraP-NP predicts more diverse distributions,  thus predicts more false alarms. Similarly, with fixed recall in P-R curves, BiTraP-NP has lower precision due the greater number of false alarms.

In summary, this simulated robot collision experiment demonstrated our proposed BiTraP trajectory predictor can be used in future robotic applications, such as predicting collisions and selecting safest paths in robot navigation tasks. Results from this supplementary experiment are consistent with our main paper's observations and further verify our hypothesis regarding the diversity/compactness of predicted trajectory distributions, i.e., BiTraP-NP predicts more diverse distributions while BiTraP-GMM predicts more compact distributions. The SPCR, ROC (AUC) and P-R (AP) metrics used in this experiment act as a supplement to the currently reported and widely used ADE/FDE and KDE-NLL metrics in the main paper. We believe these additional metrics and experiments offer an intuitive and complementary performance evaluation of the two proposed BiTraP models (NP and GMM) and their applications for tasks such as collision prediction and path selection. 

\section{Experiment and Result on nuScenes Dataset}
Among the datasets we have evaluated on, JAAD and PIE are first-person view only while ETH and UCY are focusing on campus or sidewalks only. To further present the performance of BiTraP in bird's eye view autonomous driving scenarios, we evaluate on the nuScenes dataset~\cite{nuscenes2019}. The nuScenes dataset contains trajectories collected from 850 scenes, 700 for training and 150 for testing~\cite{nuscenes2019}. We followed~\cite{salzmann2020trajectron++} to extract training and testing trajectories and trained our model using the same configurations as in ETH-UCY experiment. Note that we treat the pedestrian position at 4 seconds in the future as the target of our goal or end-point during training. 

\xhdr{Evaluation metrics.} To be comparable with~\cite{salzmann2020trajectron++}, the most-likely (ML) prediction is used to compute the final displacement error (FDE). We also use the kernel density estimation negative log-likelihood (KDE NLL) as in our other experiments.


\begin{table}[htbp]
    \centering
    \begin{tabular}{l|cccc}
        \toprule
        \multirow{2}{*}{Method} & \multicolumn{4}{c}{FDE ML} \\
        \cmidrule{2-5} 
         & @1s & @2s & @3s & @4s \\
        \midrule
        Trajectron++ base~\cite{salzmann2020trajectron++} & 0.18 & 0.57 & 2.25 & 2.24 \\
        Trajectron++ $\int$, map~\cite{salzmann2020trajectron++} & \textbf{0.07} & 0.45 & 1.14 & 2.20 \\
        BiTraP-GMM (ours) & 0.08 & \textbf{0.43} & \textbf{1.06} & \textbf{1.99} \\
        \bottomrule
    \end{tabular}
    \caption{Vehicle-only trajectory prediction results on nuScenes dataset.}
    \label{tab:nuscenes_veh}
\end{table}

\xhdr{Results.} As can be seen in Table~\ref{tab:nuscenes_ped}, adding dynamic integration and map encoding to the base Trajectron++ improved the distribution accuracy by a large margin but does not affect the FDE ML, indicating similar modes but smaller variances of the predicted distributions. Trajectron++ based methods used interactions and/or encoded map as inputs while our BiTraP-GMM only takes target pedestrians past trajectory. As in Table~\ref{tab:nuscenes_ped}, BiTraP-GMM improves the KDE-NLL at all evaluated time steps and also improves FDE after 2 seconds, showing how does the bi-directional strategy improves prediction accuracy. Note that the Trajectron++ benchmark lacks a ablation with integration but not map encoding (e.g. Trajectron++ $\int$) to show the necessity of map. However, our experiment shows that map may not be a very important information when predicting pedestrian trajectories on nuScenes dataset since BiTraP-GMM outperforms ``Trajectron++ $\int$, map".


\bibliographystyle{IEEEtran} 
\bibliography{Reference_ieee}


\end{document}