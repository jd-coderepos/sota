\documentclass{article}
\usepackage[dvips]{epsfig}


\begin{document}
\title{Cryptanalysis of the RSA-CEGD protocol}
\author{Juan M. E. Tapiador, Almudena Alcaide,\\
Julio C. Hernandez-Castro, and Arturo Ribagorda\
<(x_b, xx_b, y_b), s_b, C_{bt}>

m_1 = <C_{bt}, y_b, s_b, y_a, r_a>

<(x_b', xx_b', y_b'), s_b', C_{bt}>

m_2 = <C_{bt}, y_b', s_b', y_a', r_a'>


\noindent However,  chooses  as the message to send. As
this is a valid proof, the STTP will recover numbers  and
, which will be sent to  and , respectively. The key
point is that both numbers are not related to the current protocol
execution, but with the previous one. This way,  can use 
to obtain the receipt  contained in . Even though 
also receives , this number is useless for him to recover the
key required to access . In fact, this  might provide
 with access to the former e-goods he tried to buy. However, in
all likelihood he is not aware of this.

As a result,  has a valid receipt of  having received
e-goods , though  does not actually own it. Therefore, the
protocol does not provide fairness for .


\subsection{Indistinguishability of evidences of origin}\label{Sec:EOO}
The protocol establishes the item  as proof of
non-repudiation of origin. However, parties' identities are not
included in such a token, nor any other information related to the
current protocol execution. Even using authenticated channels,
evidences obtained do not link together the sender, the originator,
the receiver, the current protocol execution, etc. This fact yields
to a weakness related to the indistinguishability of evidences
exchanged during the protocol, in particular, evidence of origin
(EOO).

Suppose  and  perform a protocol execution, so finally
 obtains  and an , where
. This evidence does not assure itself that  is
the intended receiver. In other words, if the exchange would have
been carried out between parties  and , then the EOO
received by  would have been identical (assuming that the same
symmetric key,  is used). This way, once  owns  and
EOO, he might provide another party, , with both items by using
a traditional channel. As a result,  possesses the e-goods
coupled with a valid EOO for her. Party , on the other hand,
does not own a receipt issued by . Consequently, the protocol
neither provides fairness for .

\subsection{On the security of a modified RSA-CEGD}\label{Sec:OtherVersion}
In \cite{NZCG05}, Nenadi\'c \emph{et al.} presented a different
version of the RSA-CEGD protocol with slight modifications. The
structure of this new proposal remains unaltered with respect to the
original version. In particular, items sent by  during step E2
are the same that appears in the protocol here studied, i.e. the
VRES, the authorization token, and 's certificate. Clearly, the
attacks described above are still applicable for this version.


\section{Conclusions}\label{Sec:Conclusions}
In this paper, we have demonstrated how the RSA-CEGD protocol
suffers from severe vulnerabilities. Our attacks show up that this
scheme can lead to an unfair situation for any of the two parties
involved in the exchange. To the best of our knowledge, the
aforementioned weaknesses have not been pointed out before.


\begin{thebibliography}{99}


\bibitem{BP01}
G. Bella and L. Paulson. ``Mechanical Proofs about a Non-repudiation
Protocol''. \textit{Proc. 14th Intl. Conf. Theorem Proving in Higher
Order Logic}. LNCS, pp.91--104. Springer-Verlag, 2001.

\bibitem{GR02}
S. G\"urgens and C. Rudolph. ``Security Analysis of (Un-) Fair
Non-repudiation Protocols'', \textit{FASec 2002}, LNCS 2629, pp.
97--114. Springer-Verlag, 2002.

\bibitem{KMZ02}
S. Kremer, O. Markowitch, and J. Zhou. ``An intensive survey of fair
non-repudiation protocols''. \textit{Computer Comunications},
25(17):1606--1621. Elsevier, 2002.

\bibitem{NZCG04}
A. Nenadi\'c, N. Zhang, S. Barton. ``A Security Protocol for
Certified E-goods Delivery'', \textit{Proc. IEEE Int. Conf.
Information Technology, Coding, and Computing (ITCC'04)}, Las Vegas,
NV, USA, IEEE Computer Society, 2004, pp. 22--28.

\bibitem{NZCG05}
A. Nenadi\'c, N. Zhang, B. Cheetham, and C. Goble. ``RSA-based
Certified Delivery of E-Goods Using Verifiable and Recoverable
Signature Encryption'', \textit{Journal of Universal Computer
Science}, 11(1):175--192. Springer-Verlag, 2005.

\bibitem{Sch98}
S. Schneider. ``Formal Analysis of a Non-repudiation Protocol''.
\textit{IEEE Computer Security Foundations Workshop}. IEEE Computer
Society Press, 1998.

\bibitem{RR00}
I. Ray and I. Ray. ``An Optimistic Fair Exchange E-commerce Protocol
with Automated Dispute Resolution''. \textit{Proc. Int. Conf.
E-Commerce and Web Technologies, EC-Web 2000}. LNCS 1875, pp.
84--93. Springer-Verlag, 2000.

\bibitem{ZG96}
J. Zhou and D. Gollman. ``A fair non-repudiation protocol''.
\textit{Proc. 1996 Symp. on Research in Security and Privacy}, pp.
55--61. Oakland, CA, USA. IEEE Computer Society Press, 1996.

\bibitem{ZG98}
J. Zhou and D. Gollman. ``Towards verification of non-repudiation
protocols''. \textit{Proc. 1998 Intl. Refinement Workshop and Formal
Methods Pacific}, pp. 370--380. 1998.

\end{thebibliography}


\end{document}
