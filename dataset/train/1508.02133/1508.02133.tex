\RequirePackage{amsmath}
\documentclass[runningheads]{llncs}
\usepackage{amssymb}
\usepackage{verbatim}
\usepackage{epic,gastex}
\usepackage{array}
\usepackage{algorithmicx,algorithm,algpseudocode}
\usepackage{xcolor}
\usepackage[lowtilde]{url}
\usepackage{graphicx}

\DeclareMathOperator{\dt}{.}
\DeclareMathOperator{\rt}{rt}

\DeclareSymbolFont{rsfscript}{OMS}{rsfs}{m}{n}
\DeclareSymbolFontAlphabet{\mathrsfs}{rsfscript}
\newcommand{\LineIf}[2]{\State \algorithmicif\ {#1}\ \algorithmicthen\ {#2} \algorithmicend\ \algorithmicif}



\begin{document}

\title{On the Number of\\Synchronizing Colorings of Digraphs}

\author{Vladimir V. Gusev\inst{1}\thanks{Supported by the Communaut{\'e} fran\c{c}aise de Belgique
- Actions de Recherche Concert{\'e}es, by the Belgian Programme on
Interuniversity Attraction Poles, by the Russian foundation for basic research(grant 13-01-00852), Ministry of Education and Science of the Russian Federation (project no.\ 1.1999.2014/K), Presidential Program for Young Researchers(grant MK-3160.2014.1) and the Competitiveness Program of Ural Federal University.}
\and
Marek Szyku{\l}a\inst{2}\thanks{Supported in part by Polish NCN grant DEC-2013/09/N/ST6/01194.}}

\authorrunning{V. V. Gusev, M. Szyku{\l}a}

\tocauthor{V. V. Gusev,  M. Szyku{\l}a}

\institute{ICTEAM Institute, Universit{\'e} catholique de Louvain, Belgium,\\
Ural Federal University, Russia\\
\email{vl.gusev@gmail.com}
\and Institute of Computer Science,\\
University of Wroc{\l}aw, Wroc{\l}aw, Poland\\
\email{msz@cs.uni.wroc.pl}}

\maketitle

\begin{abstract}
We deal with -out-regular directed multigraphs with loops (called simply \emph{digraphs}).
The edges of such a digraph can be colored by elements of some fixed -element set in such a way that outgoing edges of every vertex have different colors. Such a coloring corresponds naturally to an automaton.
The road coloring theorem states that every primitive digraph has a synchronizing coloring.

In the present paper we study how many synchronizing colorings can exist for a digraph with  vertices.
We performed an extensive experimental investigation of digraphs with small number of vertices. This was done by using our dedicated algorithm exhaustively enumerating all small digraphs.
We also present a series of digraphs whose fraction of synchronizing colorings is equal to , for every  and the number of vertices large enough.

On the basis of our results we state several conjectures and open problems. In particular, we conjecture that  is the smallest possible fraction of synchronizing colorings, except for a single exceptional example on 6 vertices for .
\end{abstract}



\section{Introduction}

Throughout the paper we deal with directed multigraphs  of a fixed out-degree  with loops, where  is a finite set of  \emph{vertices} and  is a finite multiset of \emph{edges}. For each  there are exactly  outgoing edges . These are simply called \emph{digraphs} throughout the paper.
For every vertex  of a digraph , the  outgoing edges can be colored differently by one of the  colors from a finite set , giving raise to a \emph{deterministic finite (semi)automaton}  with the set of states , the alphabet , and the transition function , where  whenever an edge  was colored by . Every such automaton  is called a \emph{coloring} of the digraph .
Thus we identify automata with colorings of their underlying digraphs.
We extend the transition function  to  on subsets and words in the natural way.
For , where  and , we also write shortly .

Automaton  is called \emph{synchronizing}
if there exist a word  and a state  such that for every state  we have . Such a word  is called \emph{reset} (or \emph{synchronizing}) word for .
The length of the shortest synchronizing word of  is called \emph{reset threshold} and is denoted by .
Recent surveys of the theory of synchronizing automata may be found in~\cite{KariVolkov2013Handbook,Volkov2008Survey}.

Call a digraph \emph{primitive} (or \emph{aperiodic}) if it is strongly connected and the  of all its cycles is equal to 1.
It is easy to show that an underlying digraph of a synchronizing automaton is primitive. In 1977 Adler, Goodwyn and Weiss conjectured~\cite{AGW1977} that every primitive digraph has a synchronizing coloring. This conjecture became widely known as the \emph{road coloring} problem. It was arguably one of the most important conjectures in automata theory until it was finally proved by Trahtman in 2007~\cite{Tr2009RoadColoring}. One of the goals of the present paper is to find the right quantitative formulation of the road coloring theorem. 

Another part of our motivation comes from the algorithmic issues related to the road coloring problem. How to find a synchronizing coloring of a given digraph?
A non-trivial algorithm working in time  is known for this task~\cite{BealPerrin2014AQuadraticAlgorithmForRoadColoring}. On the other hand, M.-P. B{\'e}al suggested during her talk at CANT 2012 that a random sampling of colorings in a search for a synchronizing one may lead to a simple and practically effective algorithm for the problem. Since one can check whether a coloring is synchronizing in  time, it remains to show that a random coloring is synchronizing with high probability. In our research we were partially motivated by this observation.

There are other computational problems related to the synchronizing colorings of digraphs, such as deciding existence of a synchronizing coloring for a fixed reset word~\cite{VorelRoman2015ComplexityOfRoadColoringWithPrescribedResetWords}, or for a fixed reset threshold~\cite{Roman2012P-NPThresholdForSynchronizingRoadColoring}.
Also, several open problems concerning synchronizing automata and the road coloring problem have been stated by M.V. Volkov~\cite{Volkov2008OpenProblemsOnSynchronizingAutomata}.





For a given -out-regular digraph  with  vertices,
the  is the number of synchronizing colorings to the number  of all possible colorings.
Note that we distinguish edges of , i.e. two colorings are the same if all edges have the same color. Therefore, there is always exactly  different colorings of .
A digraph  is  if its synchronizing ratio of  is equal to 1.

In this paper we perform an experimental and theoretical study on the synchronizing ratio of digraphs. Our main contributions are as follows:
\begin{enumerate}
\item We developed an efficient algorithm for enumerating and checking synchronizing ratios of nonisomorphic digraphs.
\item Using the algorithm, we performed extensive experiments revealing various phenomena concerning the synchronizing ratio.
These provide evidence to state several conjectures and form a basis for further investigation.
\item We found out that for small  and  there are no primitive strongly connected digraphs with synchronizing ratio less than , except for a single particular example for  and .
\item We constructed digraphs with synchronizing ratio , for every  and . This shows that there are many examples with different synchronizing ratio in the range .
\end{enumerate}


\section{General Statements}

A strongly connected component  of a digraph  is called a \emph{sink component} if there are no edges going from  to .
It is \emph{reachable} if for any vertex  there is a directed path from  to a vertex in .

\begin{proposition}\label{pro:nonsc}
If a digraph  has a synchronizing coloring then it has a unique reachable sink component .
Furthermore, the synchronizing ratio of  is equal to the synchronizing ratio of the digraph induced by .
\end{proposition}
\begin{proof}
The proof of the first statement belongs to folklore.
It is not hard to see that an arbitrary coloring  of digraph  is synchronizing if and only if the subautomaton  induced by the sink component  is synchronizing.
Therefore, the set of all colorings of  can be divided into groups of equal size, each group containing the colorings with the same induced subcoloring of .
Since colorings from each group are altogether synchronizing or non-synchronizing, we obtain that the
synchronizing ratio of  is equal to the synchronizing ratio of the digraph induced by .
\qed
\end{proof}

Since a one-vertex digraph is totally synchronizing we have the following corollary:
\begin{corollary}
A digraph with a sink state is either totally synchronizing or none of its colorings is synchronizing.
\end{corollary}

Due to Proposition~\ref{pro:nonsc} the study of synchronizing ratios and totally synchronizing digraphs can be reduced to the case of strongly connected digraphs.

Surprisingly, the underlying digraphs of several automata presented in the literature appear to be totally synchronizing. One important example of such a digraph is well known to the community, see for example~\cite{Volkov2008OpenProblemsOnSynchronizingAutomata}. Recall that the \v{C}ern\'{y} automaton  (\cite{Cerny1964}) can be defined as , where  for , , , and  for . The proof of the following folklore result has not yet appeared in the literature.

\begin{proposition}
The underlying digraph of  is totally synchronizing.
\end{proposition}
\begin{proof}
Let  be an arbitrary coloring of the underlying digraph of .
It is well known that an automaton is synchronizing if and only if every pair of states  is synchronizing, i.e. there is a word  such that  (see~\cite{Cerny1964}, or \cite[Proposition~2.1]{Volkov2008Survey}).

We will show that any pair of states  of  satisfy this condition.
Let  be the length of the shortest path from  to .
We will proceed by induction on .
Consider a pair ; without loss of generality we may assume that .

If  then let  be the letter on the edge from  to .
We apply  so , and .

Consider the case .
Let  be the letter on the loop on the state .
If  then let  be the letter on the edge from  to ; otherwise let .
If  then .
Otherwise we apply the letter , and in the same manner consider the pair .
Note that .
Following in this way, after at most  steps, we will reach a pair .
For  we choose , we obtain .
\qed
\end{proof}

Underlying digraphs of many other automata that appeared in literature are also totally synchronizing.
In a similar fashion one can show that the underlying digraphs of the series of slowly synchronizing automata (see~\cite{AGV2013,KS2014SynchronizingAutomataWithLargeResetLengths}) are totally synchronizing. Also, almost all presented examples of automata with two cycle lengths have this property~\cite{GusevPribavkina2014ResetThresholdsOfAutomataWithTwoCycleLengths}.

For the sake of completeness we mention the following notions from related topics.
A word  is called \emph{totally synchronizing} if  is a reset word for any coloring of totally synchronizing digraph . See~\cite{Cardoso2014PhD} for an analysis of totally synchronizing digraphs and words in some special classes of digraphs. A word over an alphabet  is called \emph{-synchronizing} if it is a reset word for all synchronizing automata with  states over the alphabet . See~\cite{Cherubini2007SynchronizingAndCollapsingWords} for the introduction to the topic.


\section{Experimental Investigation of Digraphs}

We performed a series of experiments to reveal some properties of the synchronizing ratio of digraphs. These include both exhaustive enumeration of small digraphs and larger random digraphs.
We are interested mostly in primitive strongly connected digraphs (cf.~Proposition~\ref{pro:nonsc}).
In the case of exhaustive enumeration we checked the synchronizing ratio of all nonisomorphic -out-regular digraphs with a given  vertices.

\subsection{Algorithms}

To check as many cases as possible and obtain a large data set, we needed to design and implement our algorithms carefully.
This is especially important during the exhaustive search, since the number of digraphs grows very fast with  and .
Here we briefly describe our algorithms, skipping numerous technical improvements and tricks in the implementation.
Some of our ideas are based on~\cite{KS2013GeneratingSmallAutomata}, where the \v{C}ern\'{y} conjecture was verified by an exhaustive enumeration for all binary automata up to  states.

To determine the synchronizing ratio of a digraph, we can just enumerate all its colorings and count the synchronizing ones.
Checking whether a coloring (automaton) is synchronizing can be easily done in  time \cite{Cerny1964,Ep1990}.
Note that in many cases, some colorings give rise to the same particular automaton (e.g.~if there are are two or more parallel edges  then we can permute its colors obtaining the same automaton).
Also, every coloring has  equivalent colorings obtained only by permuting the colors. Using these facts we could greatly reduce the total number of really checked colorings for synchronization.

Checking whether a digraph is strongly connected and the  of its cycles is 1 can be effectively done in  time basing on the algorithms from~\cite{Tarjan1972} and \cite{JS1996GraphTheoreticAnalysisOfMarkovChains}, respectively.

Now, computing the synchronizing ratios of a set of random digraphs follows easily, and we can proceed this in parallel on a grid.
However, in an exhaustive enumeration, the number of digraphs grows very fast in terms of  and  (see~Table~\ref{tab:exhaustive_results_2}), and the main problem was to deal with it.

\begin{algorithm}
\caption{Exhaustive checking of digraphs.}
\label{alg:exhaustive}
\begin{algorithmic}[1]
\Require{ -- the number of vertices (states)}
\Require{ -- the out-degree (size of the alphabet)}
\State  the set of all simple graphs with  vertices.
\ForAll{simple graphs }\Comment{In parallel}
  \State  \Call{EmptySet}{}
  \ForAll{digraphs  with underlying graph }\Comment{ Orient and multiply the edges of , and add loops}
    \If{ is primitive}
      \State  the canonical representation of .
      \If{}
        \State .\Call{insert}{}
        \State Count synchronizing colorings of .
      \EndIf
    \EndIf
  \EndFor
\EndFor
\end{algorithmic}
\end{algorithm}

Our algorithm for exhaustive checking of digraphs is summarized in~Algorithm~\ref{alg:exhaustive}.
First, in line~1, we generate all nonisomorphic simple graphs with  vertices. A \emph{simple graph} is a graph with undirected edges joining two distinct vertices. This can be done effectively by the algorithm from \cite{McKayPiperno2014PracticalGraphIsomorphismII}, implemented in package \texttt{nauty}.
Now, we can process each such a simple graph in parallel. In line~4, for every simple graph  we orient and multiply its edges so that there are at most  outgoing edges for each vertex. Then we interpret the missing edges as loops.
Clearly, an isomorphic copy of every digraph can be obtained in this way from its underlying simple graph, and the digraphs obtained from two nonisomorphic simple graphs are also nonisomorphic. We can, however, obtain isomorphic digraphs from the same simple graph.
In line~5 we skip non-strongly connected and non-primitive digraphs.
In line~6 we compute the \emph{canonical representation} of a generated digraph ; this is the lexicographically minimal representation among all digraphs isomorphic to  (cf.~\cite{McKayPiperno2014PracticalGraphIsomorphismII}).
To skip isomorphic copies obtained from the same simple graph, in line~3 we introduce the set  of canonical representations of generated digraphs. Then in line~7, we check if an isomorphic copy of the digraph  was already generated; if not, in line~8 we insert it to the set. The set  can be effectively implemented as a radix trie, allowing to perform both membership test and insertion in linear time, and providing some compression (which is also important in view of the number of generated digraphs).
Finally, we can count synchronizing colorings of the generated digraph (line~9).

\subsection{Experimental Results from Exhaustive Enumeration}

\begin{table}
\caption{The minimum, average, and standard deviation of the number of synchronizing colorings of all strongly connected aperiodic -out-regular digraphs with  vertices}
\label{tab:exhaustive_results_1}
\begin{center}
\begin{tabular}{|c|c|r|r|r|r|r|}
  \hline
  &&\multicolumn{1}{c|}{\ Min\ }&\multicolumn{1}{c|}{\ Min ratio\ }&\multicolumn{1}{c|}{\ Avg\ }& \multicolumn{1}{c|}{\ Avg ratio\ } &\multicolumn{1}{c|}{\ Std dev\ } \\ \hline\hline
  2 & 2  &  2         & 0.5       & 3               & 0.750           & 1.000  \\ \hline  
  2 & 3  &  4         & 0.5       & 6.833           & 0.854           & 1.280  \\ \hline
  2 & 4  &  8         & 0.5       & 14.640          & 0.915           & 2.243  \\ \hline  
  2 & 5  &  16        & 0.5       & 30.987          & 0.968           & 2.146  \\ \hline  
  2 & 6  &  30        & 0.469     & 63.139          & 0.986           & 2.381  \\ \hline
  2 & 7  &  64        & 0.5       & 127.365         & 0.995           & 2.033  \\ \hline  
  2 & 8  &  128       & 0.5       & 255.483         & 0.998           & 1.866  \\ \hline  
  2 & 9  &  256       & 0.5       & 511.563         & 0.999           & 1.617  \\ \hline  
  2 & 10 &  512       & 0.5       & 1,023.607       &  & 1.468  \\ \hline
  3 & 2  &   24       & 0.667     & 31.2            & 0.867           & 5.879  \\ \hline 
  3 & 3  &  144       & 0.667     & 208.800         & 0.967           & 14.163  \\ \hline 
  3 & 4  &  864       & 0.667     & 1,284.987       & 0.991           & 36.346  \\ \hline 
  3 & 5  & 5,184      & 0.667     & 7,765.775       & 0.999           & 50.091  \\ \hline 
  3 & 6  & 31,104     & 0.667     & 46,643.953      &  & 78.679  \\ \hline 
  3 & 7  & 186,624    & 0.667     & 279,921.191     &  & 108.167  \\ \hline 
  4 & 2  & 432        & 0.75      & 533.333         & 0.926           & 61.738  \\ \hline  
  4 & 3  & 10,368     & 0.75      & 13,704.874      & 0.991           & 367.767  \\ \hline
  4 & 4  & 248,832    & 0.75      & 331,421.072     & 0.999           & 2,233.171  \\ \hline
  4 & 5  & 5,971,968  & 0.75      & 7,961,941.49    &  & 7,104.373  \\ \hline
  5 & 2  & 11,520     & 0.75      & 13,782.857      & 0.957           & 1,048.941  \\ \hline  
  5 & 3  & 1,382,400  & 0.75      & 1,723,468.312   & 0.997           & 720,951.433  \\ \hline
  5 & 4  & 165,888,000& 0.75      & 207,324,196.845 &  & 412,162.118  \\ \hline
\end{tabular}
\end{center}
\end{table}

The algorithms in C++ and compiled with GCC 4.8.1.
The computations were performed in parallel on a small grid consisted of computers with 8 processors Quad-Core AMD Opteron(tm) 8350 (2 GHz) and 64GB of RAM.

We were able to check all 2-out-regular digraphs with up to 10 vertices, 3-out-regular up to 7 states, 4-out-regular up to 5 states, and 5-out-regular up to 4 states.
In the case of -out-regular digraphs with  states, the total processor time was more than 60 days (about 1 day of parallelized computation).
The case of  and  took even more, about 72 days; the total number of colorings was , but, thanks to optimization, we required to check only  automata.

The results concerning synchronizing ratios are summarized in Table~\ref{tab:exhaustive_results_1}.
In Table~\ref{tab:exhaustive_results_2} we present the exact number of strongly connected aperiodic digraphs, and totally synchronizing digraphs.
We observe that the fraction of totally synchronizing digraphs within the class of strongly connected aperiodic digraphs is growing.


\begin{table}
\caption{The number of nonisomorphic strongly connected aperiodic digraphs, the number of totally synchronizing digraphs, and their fraction}
\label{tab:exhaustive_results_2}
\begin{center}
\begin{tabular}{|c|c|r|r|r|}
  \hline                       
   &  &\multicolumn{1}{c|}{\ S.c. aperiodic\ }&\multicolumn{1}{c|}{\ Totally synchronizing\ }&\multicolumn{1}{c|}{\ Fraction\ }\\ \hline\hline
  2 & 2 & 2 & 1 & 0.500 \\ \hline  
  2 & 3 & 12 & 6 & 0.500 \\ \hline  
  2 & 4 & 100 & 66 & 0.660 \\ \hline  
  2 & 5 & 1220 & 890 & 0.729 \\ \hline  
  2 & 6 & 19,064 & 14,973 & 0.785\\ \hline  
  2 & 7 & 361,157 & 296,303 & 0.82\\ \hline  
  2 & 8 & 8,001,589 & 6,754,895 & 0.844\\ \hline  
  2 & 9 & 202,635,930 & 174,246,295 & 0.860\\ \hline  
  2 & 10 & 5,765,318,112 & 5,026,305,042 & 0.872\\ \hline  
  3 & 2 & 5 & 3 & 0.600 \\ \hline     
  3 & 3 & 85 & 63 & 0.741 \\ \hline  
  3 & 4 & 3,148 & 2,672 & 0.849 \\ \hline  
  3 & 5 & 199,489 & 182,326 & 0.914 \\ \hline  
  3 & 6 & 19,059,581 & 18,006,297 & 0.945 \\ \hline  
  3 & 7 & 2,537,475,117 & 2,443,850,969 & 0.963 \\ \hline    
  4 & 2 & 9 & 6 & 0.666 \\ \hline
  4 & 3 & 357 & 302 & 0.846 \\ \hline
  4 & 4 & 39,680 & 36,762 & 0.926 \\ \hline
  4 & 5 & 9,089,413 & 8,779,342 & 0.966 \\ \hline
  5 & 2 & 14 & 10 & 0.714 \\ \hline
  5 & 3 & 1,102 & 990 & 0.898 \\ \hline
  5 & 4 & 304,082 & 291,530 & 0.959 \\ \hline
\end{tabular}
\end{center}
\end{table}









In Table~\ref{tab:gaps}, for  and , we present the number of nonisomorphic digraphs with particular numbers of synchronizing colorings.
Interestingly, there are several graphs in the distribution, and the gaps grow for smaller number of synchronizing colorings.
This picture is similar for the other values of  and  that we checked. The number of gaps seems to grow with  and .

\begin{table}
\caption{The number of nonisomorphic digraphs with the given number of synchronizing colorings for  and }
\label{tab:gaps}
\begin{center}
\begin{tabular}{|l||c|c|c|c|c|c|c|c|c|c|c|c|c|c|c|c|c|} \hline
\# synch. col. & 128 & 130 & \ldots & 158 & 160 & 162 & \ldots & 174 & 176 & 178 & 180 & 182 & 184 & 186 & 188 & 190  & 192 \\ \hline
\# digraphs    & 72  & 0   & \ldots & 0   & 24  & 0   & \ldots & 0   & 1   & 0   & 0   & 0   & 5   & 0   & 0   & 1    & 813 \\ \hline\hline
\# synch. col. & 194 & 196 & 198    & 200 & 202 & 204 & 206    & 208 & 210 & 212 & 214 & 216 & 218 & 220 & 222 & 224  & 226 \\ \hline
\# digraphs    & 0   & 1   & 1      & 12  & 1   & 1   & 6      & 202 & 0   & 2   & 1   & 134 & 4   & 22  & 14  & 4,022& 60  \\ \hline
\end{tabular}
\end{center}
\begin{center}
\begin{tabular}{|l||c|c|c|c|c|c|c|c|} \hline
\# synch. col. & 228   & 230  & 232   & 234    & 236    & 238    & 240      & 242 \\ \hline
\# digraphs    & 73    & 170  & 852   & 179    & 1,226  & 610    & 21,933   & 699 \\ \hline \hline
\# synch. col. & 244   & 246  & 248   & 250    & 252    & 254    & \multicolumn{2}{c|}{256} \\ \hline
\# digraphs    & 4,523 & 3,171& 44,230& 27,438 & 310,400& 825,791& \multicolumn{2}{c|}{6,754,895} \\ \hline
\end{tabular}
\end{center}
\end{table}

\subsection{Experiments on Random Digraphs}

To deal also with larger digraphs, we performed additional experiments with random digraphs.
We used the uniform model of a random digraph, that is, for every outgoing edge from a vertex  we choose the destination vertex uniformly at random and independently from the other choices.

For  we checked  digraphs for every , and  for .
Since the number of possible colorings grow very fast with , for  we tested , and for  we tested only .
We additionally checked the same numbers of random of digraphs in the class of strongly connected aperiodic digraphs, within the same range of  and .

The results from random tests for larger  show the same patterns as observed in those from exhaustive search.
Figure~\ref{fig:totally} shows the fraction of totally synchronizing digraphs in random samples of strongly connected aperiodic digraphs.
The picture is very similar in the class of all digraphs.


\begin{comment}
\begin{figure}
\centering
\includegraphics[width=1.2\textwidth]{ratio.eps}
\caption{Average synchronizing ratio of random digraphs.\MSZ{Add k=4.}}
\label{fig:ratio}
\end{figure}
\end{comment}

\begin{figure}
\centering
\includegraphics[width=\textwidth]{totallySA3BW.eps}
\caption{The fraction of totally synchronizing digraphs in the class of strongly connected and aperiodic digraphs}
\label{fig:totally}
\end{figure}


\section{Digraphs with Specific Synchronizing Ratios}

In this section we present different examples of digraphs with particular values of the synchronizing ratio.
According to our computational experiments the smallest possible value of the synchronizing ratio among all digraphs is equal to . This value is achieved by the digraph  (Figure~\ref{fig:30}).
By direct computation one can verify that only 30 colorings of  are synchronizing.

\begin{proposition}
There is a 2-out-regular digraph with 6 states and the synchronizing ratio .
\end{proposition}

\begin{figure}[ht]
\begin{center}
\unitlength .5mm
\begin{picture}(80,76)(0,-76)
\gasset{Nw=16,Nh=16,Nmr=8} 
\node(n1)(60.0,-16.0){}
\node(n2)(76.0,-44.0){} 
\node(n3)(60.0,-72.0){}
\node(n4)(20.0,-72.0){}
\node(n5)(4.0,-44.0){}
\node(n6)(20.0,-16.0){}
\drawedge[ELdist=1.7](n1,n3){}
\drawedge[curvedepth=3](n1,n6){}
\drawedge[ELdist=1.7](n2,n6){}
\drawedge[curvedepth=3](n2,n3){}
\drawedge[curvedepth=3](n3,n2){}
\drawedge[ELdist=1.7](n3,n5){}
\drawedge[ELdist=1.7](n4,n2){}
\drawedge[curvedepth=3](n4,n5){}
\drawedge[curvedepth=3](n5,n4){}
\drawedge[ELdist=1.7](n5,n1){}
\drawedge[ELdist=1.7](n6,n4){}
\drawedge[curvedepth=3](n6,n1){}
\end{picture}
\end{center}
\caption{Digraph .}
\label{fig:30}
\end{figure}
The exceptional example  seems to be unique.
We did not find any other digraph with this particular value of the synchronizing ratio.
Furthermore, according to our computational experiments the smallest value of the synchronizing ratio among all other -out-regular digraphs seems to be equal to . There are many examples that reach this bound, and
in the following theorem we construct a series of digraphs with this property.

\begin{theorem}
For every  there is a -out-regular digraph with  vertices and the synchronizing ratio .
\end{theorem}
\begin{proof}
We will define the digraph  as follows; see Figure~\ref{fig:half}.
The set of vertices  is .
The edges  and  are of multiplicity 1, the edges  and  are of multiplicity , and the edges  are of multiplicity .

Consider now an arbitrary coloring of the digraph .
Let  be the letter on the edge  and  be the letter on the edge . If  then every letter acts as a permutation, and so the automaton is not synchronizing.
If  then  is a reset word for the given coloring.
Hence, a coloring is synchronizing if and only if . Therefore, the synchronizing ratio of  is equal to .
\qed
\end{proof}


\begin{figure}[th]
\begin{center}
\unitlength .5mm
\begin{picture}(80,90)(0,-90)
\gasset{Nw=16,Nh=16,Nmr=8,loopdiam=10}
\node(n1)(40.0,-16.0){1}
\node(n)(4.0,-40.0){}
\node(n2)(76.0,-40.0){2}
\node(n3)(72.0,-72.0){3}
\node(n-1)(8.0,-72.0){}
\node[Nframe=n](ndots)(40,-88){}
\drawedge(n,n1){}
\drawedge(n,n2){}
\drawedge[curvedepth=3](n2,n3){}
\drawedge[curvedepth=-3](n2,n3){}
\drawedge[ELdist=1.7](n1,n2){}
\drawedge[curvedepth=3](n-1,n){}
\drawedge[curvedepth=-3](n-1,n){}
\drawedge[curvedepth=3](n3,ndots){}
\drawedge[curvedepth=-3](n3,ndots){}
\drawedge[curvedepth=3](ndots,n-1){}
\drawedge[curvedepth=-3](ndots,n-1){}
\drawloop[ELdist=1.5,loopangle=-90](n1){}
\end{picture}
\begin{picture}(80,78)(-20,-90)
\gasset{Nw=16,Nh=16,Nmr=8,loopdiam=10}
\node(n1)(40.0,-16.0){1}
\node(n)(4.0,-40.0){}
\node(n2)(76.0,-40.0){2}
\node(n3)(72.0,-72.0){3}
\node(n-1)(8.0,-72.0){}
\node[Nframe=n](ndots)(40,-88){}
\drawedge[ELdist=2.0](n,n1){}
\drawedge[ELdist=2.0, curvedepth=3](n,n2){}
\drawedge[ELdist=2.0, curvedepth=-3](n,n2){}
\drawedge[curvedepth=3](n2,n3){}
\drawedge(n2,n3){}
\drawedge[curvedepth=-3](n2,n3){}
\drawedge[ELdist=1.7](n1,n2){}
\drawedge[curvedepth=3](n-1,n){}
\drawedge(n-1,n){}
\drawedge[curvedepth=-3](n-1,n){}
\drawedge[curvedepth=3](n3,ndots){}
\drawedge(n3,ndots){}
\drawedge[curvedepth=-3](n3,ndots){}
\drawedge[curvedepth=3](ndots,n-1){}
\drawedge(ndots,n-1){}
\drawedge[curvedepth=-3](ndots,n-1){}
\drawloop[ELdist=1.5,loopangle=-90](n1){}
\drawloop[ELdist=1.5,loopangle=-90, loopdiam=7](n1){}
\end{picture}
\end{center}
\caption{The digraphs  and .}
\label{fig:half}
\end{figure}

We generalize the above result to obtain digraphs with different values of the synchronizing ratio.

\begin{theorem}
For every integers  and  there is a -out-regular digraph with  vertices and the synchronizing ratio .
\end{theorem}
\begin{proof}
We will define a digraph  as follows; see Figure~\ref{fig:synch_frac}. The set of vertices  is . There are edges  of multiplicity  for every . There are edges  of multiplicity  for every .
For every  the vertex  has a loop.
The remaining edges of multiplicity 1 are of the form  for every .

Consider now an arbitrary coloring  of the digraph .
Let  be the letter of the edge  and  be the letter of the loop , for .
We will show that the automaton  is synchronizing if and only if  for every .
It will immediately imply that the synchronizing ratio of  is equal to .

If  for every  then every letter acts as a permutation; thus the automaton is not synchronizing.
Assume now that there is  such , and let  be the smallest integer with this property.
In order to show that  is synchronizing it remains to prove that any pair can be synchronized. Let  and  be a pair of states. Since the automaton is strongly connected, there is a word  mapping  to . Let .
Let  be a shortest word mapping  to a state in . Note that .
Now we have , and also , because .
By the fact that  is the smallest integer with the property  we obtain , which concludes the proof.
\qed
\end{proof}

\begin{figure}[th]
\begin{center}
\unitlength .5mm
\begin{picture}(200,90)(0,-90)
\gasset{Nw=16,Nh=16,Nmr=8,loopdiam=7}
\node(n1)(25,-16.0){1}
\node(n)(0,-40.0){}
\node(n2)(50.0,-40.0){2}
\node(n31)(50.0,-82.0){}
\node(n3)(75,-16){3}
\node(n4)(100,-40.0){4}
\node(n6)(150,-40.0){}
\node(n7)(175,-16.0){}
\node(n8)(200,-40.0){}
\node(n9)(192,-72){}
\node(n10)(150,-82.0){}
\node(n-1)(8.0,-72.0){}
\node[Nframe=n](ndots)(100,-88){}
\node[Nframe=n](ndots2)(125,-40){}
\drawedge(n,n2){}
\drawedge[curvedepth=3](n,n1){}
\drawedge[curvedepth=-3](n,n1){}
\drawedge[curvedepth=3](n1,n2){}
\drawedge[curvedepth=-3](n1,n2){}
\drawedge[curvedepth=3](n-1,n){}
\drawedge(n-1,n){}
\drawedge[curvedepth=-3](n-1,n){}
\drawedge[curvedepth=3](ndots,n31){}
\drawedge(ndots,n31){}
\drawedge[curvedepth=-3](ndots,n31){}
\drawedge[curvedepth=3](n31,n-1){}
\drawedge(n31,n-1){}
\drawedge[curvedepth=-3](n31,n-1){}
\drawloop[ELdist=1.5,loopangle=-90](n1){}
\drawloop[ELdist=1.5,loopangle=-90](n3){}
\drawloop[ELdist=1.5,loopangle=-90](n7){}
\drawedge[curvedepth=3](n2,n3){}
\drawedge[curvedepth=-3](n2,n3){}
\drawedge[curvedepth=3](n3,n4){}
\drawedge[curvedepth=-3](n3,n4){}
\drawedge[curvedepth=3](n6,n7){}
\drawedge[curvedepth=-3](n6,n7){}
\drawedge[curvedepth=3](n7,n8){}
\drawedge[curvedepth=-3](n7,n8){}
\drawedge[curvedepth=3](n8,n9){}
\drawedge(n8,n9){}
\drawedge[curvedepth=-3](n8,n9){}
\drawedge[curvedepth=3](n9,n10){}
\drawedge(n9,n10){}
\drawedge[curvedepth=-3](n9,n10){}
\drawedge[curvedepth=3](n10,ndots){}
\drawedge(n10,ndots){}
\drawedge[curvedepth=-3](n10,ndots){}
\drawedge(n2,n4){}
\drawedge(n6,n8){}
\end{picture}
\end{center}
\caption{The digraph .}
\label{fig:synch_frac}
\end{figure}

\begin{remark}
There exist many other digraphs with synchronizing ratio .
Note that we can replace the -path  of  with any acyclic multigraph such that:
any vertex is reachable from the vertex ; from any vertex we can reach the vertex ; and every path from  to  is of the same length .
\end{remark}
















\section{Conclusions and open problems}

In this section we summarize all the conjectures and open problems.
All of the conjectures are supported by experimental evidence.

\begin{conjecture}
The minimum value of the synchronizing ratio among all -out-regular digraphs with  vertices is equal to , except for the case  and  when it is equal to .
\end{conjecture}

The conjecture was verified for small values of  and  (see Table~\ref{tab:exhaustive_results_1}).
It implies that a uniformly random coloring of a primitive strongly connected digraph is synchronizing with probability at least , and hence it would justify the algorithm finding synchronizing coloring randomly.

To state the next conjecture, let say that a  in the distribution of the number of synchronizing colorings is a maximal interval of integers divisible by , such that there are no digraphs with the number of synchronizing colorings in this interval.
Thus the conjecture above states that for every  and  large enough there is the gap .

\begin{conjecture}
For every  and , there is an  large enough such that there are at least  gaps in the distribution of the number of synchronizing colorings of -out-regular digraphs with  vertices.
\end{conjecture}

The following conjecture can be stated either in the class of strongly connected and aperiodic digraphs, or in the class of all digraphs.

\begin{conjecture}
For every , the fraction of totally synchronizing digraphs among all -out-regular digraphs with  vertices tends to  as  goes to infinity.
\end{conjecture}
A recent non-trivial theorem states that a random automaton is synchronizing with high probability~\cite{Berlinkov2013OnTheProbabilityToBeSynchronizable,Nicaud2014FastSynchronizationOfRandomAutomata}. Conjecture 3 can be seen as a further development of this statement.

We conclude with the following problem related to computing the number of synchronizing colorings.

\begin{problem}
Given a -out-regular digraph  with  vertices, what is the computational complexity of checking whether  is totally synchronizing.
\end{problem}

\subsubsection{Acknowledgment}
The authors want to thank Mikhail Volkov for his significant contributions to the theory of synchronizing automata on the occasion of his  birthday.


\bibliographystyle{plain}
\begin{thebibliography}{10}

\bibitem{AGW1977}
R.~L. Adler, L.~W. Goodwyn, and B.~Weiss.
\newblock {Equivalence of topological Markov shifts}.
\newblock {\em Israel Journal of Mathematics}, 27(1):49--63, 1977.

\bibitem{AGV2013}
D.~S. Ananichev, M.~V. Volkov, and V.~V. Gusev.
\newblock {Primitive digraphs with large exponents and slowly synchronizing
  automata}.
\newblock {\em Journal of Mathematical Sciences}, 192(3):263--278, 2013.

\bibitem{BealPerrin2014AQuadraticAlgorithmForRoadColoring}
M.-P. B{\'e}al and D.~Perrin.
\newblock {A quadratic algorithm for road coloring}.
\newblock {\em Discrete Applied Mathematics}, 169(0):15--29, 2014.

\bibitem{Berlinkov2013OnTheProbabilityToBeSynchronizable}
M.~V. Berlinkov.
\newblock {On the probability of being synchronizable}.
\newblock \url{http://arxiv.org/abs/1304.5774}, 2013.

\bibitem{Cardoso2014PhD}
{\^A}.~Cardoso.
\newblock {\em {The \v{C}ern\'{y} Conjecture and Other Synchronization
  Problems}}.
\newblock PhD thesis, University of Porto, Portugal, 2014.
\newblock \url{http://hdl.handle.net/10216/73496}.

\bibitem{Cerny1964}
J.~{\v{C}ern\'{y}}.
\newblock {Pozn\'{a}mka k homog\'{e}nnym eksperimentom s kone\v{c}n\'{y}mi
  automatami}.
\newblock {\em {Matematicko-fyzik\'alny \v{C}asopis Slovenskej Akad\'emie
  Vied}}, 14(3):208--216, 1964.
\newblock In Slovak.

\bibitem{Cherubini2007SynchronizingAndCollapsingWords}
A.~Cherubini.
\newblock {Synchronizing and Collapsing Words}.
\newblock {\em Milan Journal of Mathematics}, 75(1):305--321, 2007.

\bibitem{Ep1990}
D.~Eppstein.
\newblock {Reset sequences for monotonic automata}.
\newblock {\em SIAM Journal on Computing}, 19:500--510, 1990.

\bibitem{GusevPribavkina2014ResetThresholdsOfAutomataWithTwoCycleLengths}
V.~Gusev and E.~Pribavkina.
\newblock {Reset Thresholds of Automata with Two Cycle Lengths}.
\newblock In {\em Implementation and Application of Automata}, volume 8587 of
  {\em LNCS}, pages 200--210. Springer, 2014.

\bibitem{JS1996GraphTheoreticAnalysisOfMarkovChains}
J.~P. Jarvis and D.~R. Shier.
\newblock {Graph-theoretic analysis of finite Markov chains}.
\newblock In {\em Applied Mathematical Modeling: A Multidisciplinary Approach}.
  CRC Press, 1996.

\bibitem{KariVolkov2013Handbook}
J.~Kari and M.~V. Volkov.
\newblock {\v{C}ern\'{y}'s conjecture and the road coloring problem}.
\newblock In {\em Handbook of Automata}. European Science Foundation, to
  appear.

\bibitem{KS2013GeneratingSmallAutomata}
A.~Kisielewicz and M.~Szyku{\l}a.
\newblock {Generating Small Automata and the \v{C}ern\'{y} Conjecture}.
\newblock In {\em Implementation and Application of Automata}, volume 7982 of
  {\em LNCS}, pages 340--348. Springer, 2013.

\bibitem{KS2014SynchronizingAutomataWithLargeResetLengths}
A.~Kisielewicz and M.~Szyku{\l}a.
\newblock {Synchronizing Automata with Large Reset Lengths}.
\newblock \url{http://arxiv.org/abs/1404.3311}, 2014.

\bibitem{McKayPiperno2014PracticalGraphIsomorphismII}
B.~D. McKay and A.~Piperno.
\newblock Practical graph isomorphism, {II}.
\newblock {\em Journal of Symbolic Computation}, 60:94--112, 2014.

\bibitem{Nicaud2014FastSynchronizationOfRandomAutomata}
C.~Nicaud.
\newblock Fast synchronization of random automata.
\newblock \url{http://arxiv.org/abs/1404.6962}, 2014.

\bibitem{Roman2012P-NPThresholdForSynchronizingRoadColoring}
A.~Roman.
\newblock {P--NP Threshold for Synchronizing Road Coloring}.
\newblock In {\em Language and Automata Theory and Applications}, volume 7183
  of {\em LNCS}, pages 480--489. Springer Berlin Heidelberg, 2012.

\bibitem{Tarjan1972}
R.~Tarjan.
\newblock {Depth-first search and linear graph algorithms}.
\newblock {\em SIAM Journal on Computing}, 1(2):146--160, 1972.

\bibitem{Tr2009RoadColoring}
A.~N. Trahtman.
\newblock {The Road Coloring Problem}.
\newblock {\em Israel Journal of Mathematics}, 172(1):51--60, 2009.

\bibitem{Volkov2008OpenProblemsOnSynchronizingAutomata}
M.~V. Volkov.
\newblock Open problems on synchronizing automata.
\newblock
  \small\url{http://csseminar.kadm.usu.ru/SLIDES/WroclawABCD2008/volkov_abcd_problems.pdf},
  Workshop ``Around the {\v{C}ern\'{y}} conjecture'', Wroc{\l}aw, 2008.

\bibitem{Volkov2008Survey}
M.~V. Volkov.
\newblock Synchronizing automata and the \u{C}ern\'{y} conjecture.
\newblock In {\em Language and Automata Theory and Applications}, volume 5196
  of {\em LNCS}, pages 11--27. Springer, 2008.

\bibitem{VorelRoman2015ComplexityOfRoadColoringWithPrescribedResetWords}
V.~Vorel and A.~Roman.
\newblock {Complexity of Road Coloring with Prescribed Reset Words}.
\newblock In {\em Language and Automata Theory and Applications}, volume 8977
  of {\em LNCS}, pages 161--172. Springer, 2015.

\end{thebibliography}



\end{document}