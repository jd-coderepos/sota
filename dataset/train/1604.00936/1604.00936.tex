In the present section, building on \cite{AbramskyVaananen08, Roelofsen_algebraic_13}, and using standard facts pertaining to discrete Stone and Birkhoff dualities, we give an alternative algebraic presentation of the team semantics. This presentation shows how two natural types emerge from the team semantics, together with natural maps connecting them. These maps will  support the interpretation of additional {\em multi-type} connectives which will be used to define a new, multi-type language into which we will translate the original language and axioms of inquisitive logic. Finally, in Section \ref{sec:formal} we will introduce a structural  multi-type sequent calculus for the translated axiomatization.

\subsection{Order-theoretic analysis}
\label{ssec:semanticanalysis}





In what follows, we let  abbreviate the initial set  of proposition variables; we let  denote the set of Tarski assignments. Elements of  are denoted by the variables  and , possibly sub- and super-scripted. Let  denote the (complete and atomic) Boolean algebra . Elements of  are information states (teams), and are denoted by the variables  and , possibly sub- and super-scripted. Consider the relational structure 
By discrete Birkhoff-type duality, a perfect Heyting algebra\footnote{A Heyting algebra is {\em perfect} if it is complete, completely distributive and completely join-generated by its completely join-prime elements. Equivalently, any perfect algebra can be characterized up to isomorphism as the complex algebra of some partially ordered set.} arises as the complex algebra of . Indeed,
let . Elements of  are downward closed collections of teams, and are denoted by the variables  and , possibly sub- and super-scripted. The operation  is defined as follows: for any  and ,









Three natural maps can be defined between the perfect Boolean algebra  and the perfect HAO . Indeed, any team  can be associated with the downward-closed collection of teams . Conversely, any (downward-closed) collection of teams  can be associated with the team   for some  Thirdly, for any team , the collection of teams   is downward closed. These assignments respectively define the following maps:


The maps ,  and  turn out to be adjoints to one another as follows:\footnote{In order-theoretic notation we write  ).}

\begin{lemma}
\label{lemma: bh left adjoint of hb}
For all  and ,

\end{lemma}
By general order-theoretic facts, from these adjunctions it follows that ,  and  are all order-preserving (monotone), and moreover,  preserves all meets of  (including the empty one, i.e.\ ), that is,  commutes with arbitrary intersections,  preserves all joins and all meets of , that is,  commutes with arbitrary unions and intersections, and  preserves all joins  of , that is,  commutes with arbitrary unions.
Notice also that  for all  and ,


The following lemma  will be needed to prove the soundness of the rule KP of the calculus introduced in section \ref{sec:formal}.

\begin{lemma}
\label{lemma: soundness of rules}
For all , ,
\begin{center} \label{lem:sem:item2}
;
\end{center}
\end{lemma}
\begin{proof}
Assume that  and . Then  and  for some . Hence . To show that , let . Then by assumption, either   or . However,  implies that , and hence , as required.
\end{proof}










The following lemma collects  relevant  properties of :
\begin{lemma}
\label{lemma:properties of bh}
For all ,
\begin{itemize}
\item[(a)]   and ;
\item[(b)] ;
\item[(c)] .
\end{itemize}
\end{lemma}


 \begin{proof}
(a) Immediate.\\

\noindent 
\begin{tabular}{r c l}
(b)  &&\\
&&\\
&&\\
&&\\
\end{tabular}



\noindent 
\begin{tabular}{r c l}
(c)  &&\\
&&\\
&&\\
&&
\end{tabular}
\end{proof}

\subsection{Multi-type inquisitive logic}
\label{ssec:multi-type Inql}
The existence of the maps ,  and  motivates the introduction of the following language, the formulas of which are given in two types,  and , defined by the following simultaneous recursion:
\begin{center}

\end{center}

Let  and  abbreviate  and   respectively.
Notice that a canonical assignment exists , defined by . This assignment can be extended to -formulas as usual via the homomorphic extension .
The homomorphic extension  can be composed with  so as to yield a second homomorphic extension .
The maps  and
 are defined as below:

\begin{center}
\begin{tabular}{r c l c r c l}
 && 
& \quad\quad\quad &
 && \\
 && 
&&
 && \\
 && 
&&
 && \\
 && 
&&
 && .\\
 && 
&&
\\
\end{tabular}
\end{center}
The following lemma is an immediate consequence of the definitions of  and  , and of Lemma \ref{lemma:properties of bh}:
\begin{lemma}
For all -formulas  and ,
\begin{center}
\begin{tabular}{r c l c r c l}
 && 
& \quad\quad\quad\quad\quad &
 && \\
 && 
&&
 && .\\
\end{tabular}
\end{center}
\end{lemma}

\begin{comment}






Let us define the following translation :

\begin{center}
\begin{tabular}{r c l}
 && \\
 && \\
 && \\
 && .\\
\end{tabular}
\end{center}

\begin{lemma}
For every ,  .
\end{lemma}
\begin{proof}
By induction on the shape of .  If  for some , then . \marginnote{this proof uses that  commutes with ; this holds if we are in  rather than in .}

If , then . 

If , then  

If , then  

\end{proof}
Moreover, we define a translation map  as follows:
\begin{center}
\begin{tabular}{r c l c r c l}
 &  &  &&&\\
 &  &  & && \\
 &  &  & && \\
\end{tabular}
\end{center}
\marginnote{there is something wrong here.  is given twice. Maybe you wanted to 
???}
\end{comment}
Let us define the multi-type counterpart of flat formulas of  inquisitive logic: 
\begin{definition}
A formula  is {\em flat} if for every team , 
\end{definition}  
\begin{lemma}
\label{lemma:semantic flatness}
The following are equivalent for any :\\
1.  is flat;\\
2. .
\end{lemma}
\begin{proof}
By definition,  is flat iff  .
Moreover, the following chain of identities holds:
\begin{center}
\begin{tabular}{r c l l}
& & \\
& &  & (Lemma \ref{lemma: bh left adjoint of hb})\\
& & ,\\
\end{tabular}
\end{center}
which completes the proof.
\end{proof}

We are now in a position to define the following translation of \Inql-formulas into formulas of the multi-type language introduced above:
\CPC-formulas  and  will be translated into -formulas via , and \Inql-formulas  and  into -formulas via  as follows:

\begin{center}
\begin{tabular}{r c l c r c l}
 &&  &&  && \\
 &&  &&  && \\
 &&  &&  &&  \\
 &&  &&  && . \\
\end{tabular}
\end{center}


The translation above justifies the introduction of the following Hilbert-style presentation of the logic which is the natural multi-type counterpart of \Inql:
\begin{itemize}
\item Axioms \subitem (A1) \CPC axiom schemata for -formulas;
\subitem (A2) \IPC axiom schemata for -formulas;
\subitem (A3)  \subitem (A4) .
\end{itemize}
plus Modus Ponens rules for both -formulas and -formulas.

In the following section, we are going to introduce the calculus for this logic.