







Process calculi, such as \textsc{ccs} and the $\pi$-calculus, 
have a so-called \emph{communication rule} that allows to synchronize sub-processes to
perform silent actions.
The involved process terms have complementary actions that allow to
interact 
by a ``hand-shake''.
However, it is an open question how such a communication rule can be
obtained for general graph transformations systems
via the Borrowed Context technique.
Roughly, 
the label of a transition does not contain information about which
reaction rule was used to derive it;
in fact, 
the same label might be derived using different rules. 
Intuitively, 
we do not know how to identify the  two hands that have met to shake hands.


To elaborate on this using the  metaphor of handshakes, 
assume that we have an agent that needs a hand to perform a
handshake 
or to deliver an object. 
If we observe this agent reaching out for another hand, 
we cannot conclude from it which of the two possible actions will follow. 
In general, 
even after the action is performed,
it still is not possible to know the decision of the agent --
without extra information, which might however not be observable. 
\marginpar{???! 
\\Intuitively, 
we first show that if the agent was able to shake a hand, and its
partner too, 
we could have done it, even if it is not what ``we did''.}
However, 
with suitable assumptions about the ``allowed actions'',
all necessary information might be available. 







First, we recall from~\cite{BEK06} 
that  \textsc{dpobc}-diagrams (as defined in Definition~\ref{def:dpobc}) can be composed under certain circumstances. 
\begin{fact}
  Let \[\theaction \quad\text{ and }\quad \action{G'}{J'}{F'}{K'}{H'} \] 
  be two transitions obtained from two \textsc{dpobc}-diagrams
  with the same rule $\rho = L \leftarrow I \rightarrow R$.
Then, it is possible to build a \textsc{dpobc}-diagram with the same rule for the composition of $J \rightarrow G$ and $J' \rightarrow G'$ along some common interface $J \gets \minint{D}{L} \to J'$. 
\end{fact}

Take the following example as illustration of this fact. 

\begin{example}[Composition of transitions] 
  Let $J \to G$ be a state of $\S_{ex}$ 
  that contains an edge $\alpha$ with its second connection in the interface
  as shown in Figure~\ref{subfig:st}.
  Further, let $J' \to G'$ be a state 
  that contains  an edge $\beta$ with its second connection in the interface
  as shown in Figure~\ref{subfig:nd}.
  Both graphs can trigger a reaction from rule $\alpha/\beta/\gamma$. 
  Such a composition is shown in Figure~\ref{subfig:rd}.
\end{example}

 \begin{figure}[htb]
 \centering
  \subfigure[A first transition]{
    \label{subfig:st}
      \scalebox{.45}{\begin{picture}(0,0)\includegraphics{transition1.pdf}\end{picture}\setlength{\unitlength}{4144sp}\begingroup\makeatletter\ifx\SetFigFont\undefined \gdef\SetFigFont#1#2#3#4#5{\reset@font\fontsize{#1}{#2pt}\fontfamily{#3}\fontseries{#4}\fontshape{#5}\selectfont}\fi\endgroup \begin{picture}(8189,2744)(24,-1928)
\put(3151,-241){\makebox(0,0)[lb]{\smash{{\SetFigFont{20}{24.0}{\familydefault}{\mddefault}{\updefault}{\color[rgb]{0,0,0}$\beta$}}}}}
\put(3826,-106){\makebox(0,0)[lb]{\smash{{\SetFigFont{20}{24.0}{\familydefault}{\mddefault}{\updefault}{\color[rgb]{0,0,0}$\gamma$}}}}}
\put(496,-376){\makebox(0,0)[lb]{\smash{{\SetFigFont{20}{24.0}{\familydefault}{\mddefault}{\updefault}{\color[rgb]{0,0,0}$\alpha$}}}}}
\put(6976,-1096){\makebox(0,0)[lb]{\smash{{\SetFigFont{20}{24.0}{\familydefault}{\mddefault}{\updefault}{\color[rgb]{0,0,0}$R_4$}}}}}
\put(6886,299){\makebox(0,0)[lb]{\smash{{\SetFigFont{20}{24.0}{\familydefault}{\mddefault}{\updefault}{\color[rgb]{0,0,0}$G$}}}}}
\put(496,164){\makebox(0,0)[lb]{\smash{{\SetFigFont{20}{24.0}{\familydefault}{\mddefault}{\updefault}{\color[rgb]{0,0,0}$G$}}}}}
\end{picture} }
  }
  
  \subfigure[A second transition]{
        \label{subfig:nd}
      \scalebox{.45}{\begin{picture}(0,0)\includegraphics{transition2.pdf}\end{picture}\setlength{\unitlength}{4144sp}\begingroup\makeatletter\ifx\SetFigFont\undefined \gdef\SetFigFont#1#2#3#4#5{\reset@font\fontsize{#1}{#2pt}\fontfamily{#3}\fontseries{#4}\fontshape{#5}\selectfont}\fi\endgroup \begin{picture}(8189,2654)(24,-1838)
\put(3286,-241){\makebox(0,0)[lb]{\smash{{\SetFigFont{20}{24.0}{\familydefault}{\mddefault}{\updefault}{\color[rgb]{0,0,0}$\alpha$}}}}}
\put(3961,-106){\makebox(0,0)[lb]{\smash{{\SetFigFont{20}{24.0}{\familydefault}{\mddefault}{\updefault}{\color[rgb]{0,0,0}$\gamma$}}}}}
\put(586,-376){\makebox(0,0)[lb]{\smash{{\SetFigFont{20}{24.0}{\familydefault}{\mddefault}{\updefault}{\color[rgb]{0,0,0}$\beta$}}}}}
\put(496,164){\makebox(0,0)[lb]{\smash{{\SetFigFont{20}{24.0}{\familydefault}{\mddefault}{\updefault}{\color[rgb]{0,0,0}$G'$}}}}}
\put(6976,254){\rotatebox{360.0}{\makebox(0,0)[lb]{\smash{{\SetFigFont{20}{24.0}{\familydefault}{\mddefault}{\updefault}{\color[rgb]{0,0,0}$G'$}}}}}}
\put(6976,-1186){\rotatebox{360.0}{\makebox(0,0)[lb]{\smash{{\SetFigFont{20}{24.0}{\familydefault}{\mddefault}{\updefault}{\color[rgb]{0,0,0}$R_4$}}}}}}
\end{picture} }
  }
  
  \subfigure[The composition of the transitions]{
    \label{subfig:rd}
      \scalebox{.45}{\begin{picture}(0,0)\includegraphics{transition3.pdf}\end{picture}\setlength{\unitlength}{4144sp}\begingroup\makeatletter\ifx\SetFigFont\undefined \gdef\SetFigFont#1#2#3#4#5{\reset@font\fontsize{#1}{#2pt}\fontfamily{#3}\fontseries{#4}\fontshape{#5}\selectfont}\fi\endgroup \begin{picture}(8154,3374)(59,-2558)
\put(3479,-230){\makebox(0,0)[lb]{\smash{{\SetFigFont{20}{24.0}{\familydefault}{\mddefault}{\updefault}{\color[rgb]{0,0,0}$\gamma$}}}}}
\put(6886,299){\makebox(0,0)[lb]{\smash{{\SetFigFont{20}{24.0}{\familydefault}{\mddefault}{\updefault}{\color[rgb]{0,0,0}$G$}}}}}
\put(537,284){\makebox(0,0)[lb]{\smash{{\SetFigFont{20}{24.0}{\familydefault}{\mddefault}{\updefault}{\color[rgb]{0,0,0}$G$}}}}}
\put(537,-256){\makebox(0,0)[lb]{\smash{{\SetFigFont{20}{24.0}{\familydefault}{\mddefault}{\updefault}{\color[rgb]{0,0,0}$\alpha$}}}}}
\put(568,-1539){\makebox(0,0)[lb]{\smash{{\SetFigFont{20}{24.0}{\familydefault}{\mddefault}{\updefault}{\color[rgb]{0,0,0}$G'$}}}}}
\put(658,-999){\makebox(0,0)[lb]{\smash{{\SetFigFont{20}{24.0}{\familydefault}{\mddefault}{\updefault}{\color[rgb]{0,0,0}$\beta$}}}}}
\put(7036,-316){\makebox(0,0)[lb]{\smash{{\SetFigFont{20}{24.0}{\familydefault}{\mddefault}{\updefault}{\color[rgb]{0,0,0}$R_4$}}}}}
\put(6931,-1726){\makebox(0,0)[lb]{\smash{{\SetFigFont{20}{24.0}{\familydefault}{\mddefault}{\updefault}{\color[rgb]{0,0,0}$G'$}}}}}
\end{picture} }
  }
  \caption{An example of composition.}
  \label{fig:compo}
 \end{figure}
Hence, 
we see that is in general possible to combine  transitions to obtain new transitions. 
However, 
we emphasize at this point, 
that derivability of a counterpart of the communication rule of \textsc{ccs}
is not the same question as the composition 
of pairs of transitions that come equipped with \emph{complete}~\textsc{bc}-diagrams. 
To clarify the problem, 
consider the following example where we cannot  infer the used rule from the 
transition label. 


\begin{example}
  Let $G$ be a graph composed of two edges $\alpha$ and $\beta$ and consider a transition label where an edge $\gamma$ is ``added''. 
  Then it is justified by both rules $\alpha/\gamma$ and $\beta/\gamma$ (see Figure \ref{fig:twoPossibilities}). 
\end{example}

 \begin{figure}[htbp]
 \centering
  \subfigure[A transition from rule $\alpha/\gamma$]{
      \scalebox{.45}{\begin{picture}(0,0)\includegraphics{transition4.pdf}\end{picture}\setlength{\unitlength}{4144sp}\begingroup\makeatletter\ifx\SetFigFont\undefined \gdef\SetFigFont#1#2#3#4#5{\reset@font\fontsize{#1}{#2pt}\fontfamily{#3}\fontseries{#4}\fontshape{#5}\selectfont}\fi\endgroup \begin{picture}(7904,1889)(-151,-1136)
\put(718, 69){\makebox(0,0)[lb]{\smash{{\SetFigFont{20}{24.0}{\familydefault}{\mddefault}{\updefault}{\color[rgb]{0,0,0}$\beta$}}}}}
\put(3493,-136){\makebox(0,0)[lb]{\smash{{\SetFigFont{20}{24.0}{\familydefault}{\mddefault}{\updefault}{\color[rgb]{0,0,0}$\gamma$}}}}}
\put(7106,-49){\makebox(0,0)[lb]{\smash{{\SetFigFont{20}{24.0}{\familydefault}{\mddefault}{\updefault}{\color[rgb]{0,0,0}$\beta$}}}}}
\put( 73, 69){\makebox(0,0)[lb]{\smash{{\SetFigFont{20}{24.0}{\familydefault}{\mddefault}{\updefault}{\color[rgb]{0,0,0}$\alpha$}}}}}
\put(6336,171){\makebox(0,0)[lb]{\smash{{\SetFigFont{20}{24.0}{\familydefault}{\mddefault}{\updefault}{\color[rgb]{0,0,0}$R_2$}}}}}
\end{picture} }
  }
  
  \subfigure[A transition from rule $\beta/\gamma$]{
      \scalebox{.45}{\begin{picture}(0,0)\includegraphics{transition5.pdf}\end{picture}\setlength{\unitlength}{4144sp}\begingroup\makeatletter\ifx\SetFigFont\undefined \gdef\SetFigFont#1#2#3#4#5{\reset@font\fontsize{#1}{#2pt}\fontfamily{#3}\fontseries{#4}\fontshape{#5}\selectfont}\fi\endgroup \begin{picture}(7934,1884)(-151,-1081)
\put(718, 69){\makebox(0,0)[lb]{\smash{{\SetFigFont{20}{24.0}{\familydefault}{\mddefault}{\updefault}{\color[rgb]{0,0,0}$\beta$}}}}}
\put(3493,-136){\makebox(0,0)[lb]{\smash{{\SetFigFont{20}{24.0}{\familydefault}{\mddefault}{\updefault}{\color[rgb]{0,0,0}$\gamma$}}}}}
\put(6286,-79){\makebox(0,0)[lb]{\smash{{\SetFigFont{20}{24.0}{\familydefault}{\mddefault}{\updefault}{\color[rgb]{0,0,0}$\alpha$}}}}}
\put( 73, 69){\makebox(0,0)[lb]{\smash{{\SetFigFont{20}{24.0}{\familydefault}{\mddefault}{\updefault}{\color[rgb]{0,0,0}$\alpha$}}}}}
\put(7086,181){\makebox(0,0)[lb]{\smash{{\SetFigFont{20}{24.0}{\familydefault}{\mddefault}{\updefault}{\color[rgb]{0,0,0}$R_3$}}}}}
\end{picture} }
  }
  \caption{Same transition label for different rules.}
  \label{fig:twoPossibilities}
 \end{figure}
We shall avoid this problem by restricting to suitable classes of graph transformation systems. 
Moreover, 
for simplicities sake, 
we shall focus on the derivation of ``silent'' transitions
in the spirit of the communication rule of \textsc{ccs}. 
\begin{definition}[Silent label]
  A label $J \to F \gets K$ is \emph{silent} or $\tau$
  if $J = F = K$;
  a \emph{silent transition} is a transition with a silent label. 
\end{definition}
  Intuitively, 
  a silent transition is one that does not induce any ``material''
  change that is visible to an external observer
  that only has access to  the interface of the states. 
  Hence, 
  in particular, 
  a silent transition does not involve additions of the environment
  during the transition.
  Moreover, 
  the interface remains unchanged. 
  This latter requirement does not have any counterpart 
  in process calculi, 
  as the interface is given implicitly by the set of all free names.
  (In graphical encodings of process terms~\cite{bonchi2009labelled}
  it is possible to have free names in the interface
  even though there is no corresponding input or output prefix in
  the term.)



  Now, with the focus on silent transitions,
  for a given rule
  $\redrule{L}{I}{R}$
  we can illustrate the idea of complementary actions as follows.
  If a graph $G$ contains a
  subgraph $D$ of $L$  and moreover a graph $G'$ has the complementary
  subgraph of $D$ in $L$ in it,
  then $G$ and $G'$ can be combined to obtain a
  big graph $\bar G$ -- the ``parallel composition'' of $G$ and $G'$ --
  that has the whole left hand side $L$ as a subgraph
  and thus $\bar G$ can perform the reaction.
  {A natural example for this 
    are Lafont's interaction nets
    where the left hand side consist exactly of two hyper-edges, 
    which in this case are called cells.}
  \marginpar{ so we will only consider complementary subgraphs with
    common interface a single ``wire'' (a single element of dimension
    $0$). But this can be generalized with some technicalities to any
    kind of ``minimal interface''.}
  The intuitive idea of complementary (basic) actions is 
  captured by the notion of \emph{active pairs}. 


\begin{definition}[Active pairs] 
  For any inclusion $D \to L$, where $D \neq L$ and for all nodes $v$ of $D$, $\deg(v) > 0$, 
  let the following square be its initial pushout  
  \[ \begin{tikzpicture}[scale=1,baseline={(current bounding box.west)},semithick]

  \mnode[J]{\minint{D}{L}}{0,1}
  \mnode{D}{0,0}
  \mnode[D']{\compl{D}{L}}{1,1}
  \mnode{L}{1,0}

  \foreach \u/\v in {J/D,J/D',D/L,D'/L}
  {\ardrawd{\u}{\v}};

  \POC{L}{135}
\end{tikzpicture}, 
\]
i.e.\ $\compl{D}{L}$ is the smallest subgraph of $L$  that allows for completion to a pushout. 
We call $\compl{D}{L}$ the \emph{complement of $D$ in $L$} and $\minint{D}{L}$ the \emph{minimal interface of $D$ in $L$}
and we write $\{ D,D' \} \equiv L$ if $D' = \compl{D}{L}$. 
 The set of \emph{active pairs} is 
 \begin{displaymath}
   \mathbb D = \big\{\ \{D,\compl{D}{L}\}\ \mid  L \gets I \to R \in \R,D \to L,~ D \neq L,~ \forall  v \in D\ldotp \deg(v) > 0 
 \big\}.
 \end{displaymath}
  Abusing notation, 
  we also denote by $\mathbb D$ the union of $\mathbb D$.  
\label{def:activepair}
\end{definition}
It is easy to verify that the complement of $\compl{D}{L}$ in $L$ is $D$ itself and that its minimal interface is also $\minint{D}{L}$.
It is the set of ``acceptable'' partial matches in the sense that they do not yield a $\tau$-reaction on their own.
Indeed, if $D$ is equal to $L$, 
then the resulting transition of this partial match is a $\tau$-transition.
And if it is just composed of vertices, its complement is $L$
and thus not acceptable.

\begin{example}[Active pairs]
  In our running example, 
  the set $\mathbb D$ of our example is in obvious bijection to
    \[\big\{ \{\alpha,\beta\},\{\alpha,\gamma\},\{\beta,\gamma\},\{\alpha,\beta+\gamma\},\{\alpha+\beta,\gamma\},\{\alpha+\gamma,\beta\}  \big\}.\] 
  The minimal interface of any pair is a single vertex.
\end{example}
This completes the introduction of preliminary concepts to tackle the issues that 
have to be resolved to obtain ``proper'' compositionality of transitions. 

\subsection{Towards a partial solution}
Let us address the problem of identifying the rule that is ``responsible'' for a given interaction. 
We start by considering the left inclusions of labels, 
which intuitively describe possible borrowing actions from the environment.
Relative to this,  we define the \emph{admissible rules} as those rules that
can be used to let states evolve while borrowing the specified ``extra material'' from the environment.

\begin{definition}[Admissible rule]
  \label{def:admissibility}
  Let $J\to G$ be a state
  and let $J\to F$ be an inclusion (which represents a possible contribution of the context). 
  A rule $\rho$ is \emph{admissible  (for $J\to F$)}  
  if $L \not \to G$ and it is possible to find $D \in \mathbb D$ and $L$ the left-hand side of $\rho$, such that the following diagram commutes 
\begin{displaymath}
  \begin{tikzpicture}[scale=1.0,baseline={(current bounding box.west)},semithick]

  \mnode[star]{J^{^{_ L}}_{^D}}{0,0}
  \mnode{G}{0,2}
  \mnode{J}{0,1}
  \mnode{F}{1,1}
\mnode[Gc]{G^c}{1,2}
  \mnode{D}{1,0}
  \mnode{L}{1,3}

  \foreach \u/\v in {star/J,star/D,J/G,J/F,D/F,L/Gc,G/Gc,F/Gc}
  {\ardrawd{\u}{\v}};

  \draw [->] (1.2,0) arc (-90:0:.8) -- (2,2.2) arc (0:90:.8) ;
  \draw [->] (L) -- (G) node[midway] {$\backslash$} ;
  
  \POC{Gc}{225}


\end{tikzpicture}
\end{displaymath}
where $J^{^{_ L}}_{^D} \to D$ is the minimal interface of $D$ in $L$. 
We call $D$ the \emph{rule addition}.

\end{definition}
This just means that $G$ \emph{can} evolve using the rule $\rho$ if  $D$ is added at the proper location. 



\begin{proposition}[Precompositionality]
   Let $\theactionLight$ and $\actionLight{G'}{J'}{F'}{K'}{H'}$ be two transitions such that a single rule $\rho$ is admissible  for both, and let $D$ and $D'$ be their respective rule additions.
  If $\{D,D'\} \in \mathbb D$, 
  it is possible to compose $G$ and $G'$ into a graph $\bar G$ in a way to be able to derive a
  $\tau$-transition using rule $\rho$.
\label{prop:compositionality}
\end{proposition}

\begin{proof}
  We first show that in such a case, $D' \rightarrow G$ and the pushout of $\redrule{G}{D'}{L}$ is exactly $G^c$. Similarly, $D \rightarrow G'$ and the pushout of $\redrule{G'}{D}{L}$ is exactly $G'^c$.
  Then, it is easy to see that it is possible to build the
  \textsc{dpobc}-diagram $\mathtt D_1$  using rule $\rho$  on $G$
  (respectively $G'$) yelding the transition
  $\action{G}{J}{F}{K_1}{H_1}$ for some $K_1,H_1$ (respectively the
  \textsc{dpobc}-diagram $\mathtt D_2$ yelding the transition
  $\action{G}{J}{F}{K_2}{H_2}$ for some $K_2,H_2$), and then compose
  $\mathtt D_1$ and $\mathtt D_2$.









This follows from  $\{D,D'\} \in \mathbb D$ and $\bar G \equiv \bar {G^c}$.
Indeed, $\bar E = L$ so the top left morphism of the composed \textsc{dpobc}-diagram is an isomorphism
and so are the ones under it, using basic pushout properties.
\end{proof}

This first result  motivates the following definition. 
\begin{definition}[$\tau$-compatible]
  In the situation of Proposition~\ref{prop:compositionality}, 
  we say the two transitions  are $\tau$-\emph{compatible}. 
\end{definition}

\begin{remark}
  In general, in Proposition~\ref{prop:compositionality}, 
  the result of the $\tau$-transition cannot be constructed from $H$ and $H'$;
  thus we do not yet speak of compositionality.
\end{remark}

\begin{example}
  Let $G$ be a graph composed of two edges $\alpha$ and $\gamma$ and $G'$ of two edges $\beta$ and $\gamma$  (see Figure~\ref{fig:tauComp}).
  Then the rule $\alpha/\beta$ is admissible for both transitions
  and moreover they are $\tau$-compatible. 
  The rule $\alpha/\beta$ yields the respective rule additions.
  ``Glueing'' $G$ and $G'$ by their interface results in a graph with edges $\alpha, \beta$ and two $\gamma$s;
  the latter graph  can perform a $\tau$-reaction  from rule $\alpha/\beta$,
  which however does not give the desired result
  since the target state is not the ``expected  composition'' of $H$ and $H'$. 
  In other words, 
  although we have been able to  construct a $\tau$-transition,
  it is not the composition of the original transitions.
\end{example}

 \begin{figure}[htbp]
 \centering
  \subfigure[A transition from rule $\beta/\gamma$]{
      \scalebox{.45}{\begin{picture}(0,0)\includegraphics{transition6.pdf}\end{picture}\setlength{\unitlength}{4144sp}\begingroup\makeatletter\ifx\SetFigFont\undefined \gdef\SetFigFont#1#2#3#4#5{\reset@font\fontsize{#1}{#2pt}\fontfamily{#3}\fontseries{#4}\fontshape{#5}\selectfont}\fi\endgroup \begin{picture}(7599,1889)(39,-1136)
\put(3504,-84){\makebox(0,0)[lb]{\smash{{\SetFigFont{20}{24.0}{\familydefault}{\mddefault}{\updefault}{\color[rgb]{0,0,0}$\beta$}}}}}
\put(936, 76){\makebox(0,0)[lb]{\smash{{\SetFigFont{20}{24.0}{\familydefault}{\mddefault}{\updefault}{\color[rgb]{0,0,0}$\gamma$}}}}}
\put(6421,156){\makebox(0,0)[lb]{\smash{{\SetFigFont{20}{24.0}{\familydefault}{\mddefault}{\updefault}{\color[rgb]{0,0,0}$R_3$}}}}}
\put(263, 64){\makebox(0,0)[lb]{\smash{{\SetFigFont{20}{24.0}{\familydefault}{\mddefault}{\updefault}{\color[rgb]{0,0,0}$\alpha$}}}}}
\put(6996, 51){\makebox(0,0)[lb]{\smash{{\SetFigFont{20}{24.0}{\familydefault}{\mddefault}{\updefault}{\color[rgb]{0,0,0}$\alpha$}}}}}
\end{picture} }
  }
  
  \subfigure[A transition from rule $\alpha/\gamma$]{
      \scalebox{.45}{\begin{picture}(0,0)\includegraphics{transition7.pdf}\end{picture}\setlength{\unitlength}{4144sp}\begingroup\makeatletter\ifx\SetFigFont\undefined \gdef\SetFigFont#1#2#3#4#5{\reset@font\fontsize{#1}{#2pt}\fontfamily{#3}\fontseries{#4}\fontshape{#5}\selectfont}\fi\endgroup \begin{picture}(7709,1889)(39,-1136)
\put(956, 76){\makebox(0,0)[lb]{\smash{{\SetFigFont{20}{24.0}{\familydefault}{\mddefault}{\updefault}{\color[rgb]{0,0,0}$\gamma$}}}}}
\put(354,171){\rotatebox{360.0}{\makebox(0,0)[lb]{\smash{{\SetFigFont{20}{24.0}{\familydefault}{\mddefault}{\updefault}{\color[rgb]{0,0,0}$\beta$}}}}}}
\put(7089, 51){\rotatebox{360.0}{\makebox(0,0)[lb]{\smash{{\SetFigFont{20}{24.0}{\familydefault}{\mddefault}{\updefault}{\color[rgb]{0,0,0}$\beta$}}}}}}
\put(3446,-99){\rotatebox{360.0}{\makebox(0,0)[lb]{\smash{{\SetFigFont{20}{24.0}{\familydefault}{\mddefault}{\updefault}{\color[rgb]{0,0,0}$\alpha$}}}}}}
\put(6419, 91){\makebox(0,0)[lb]{\smash{{\SetFigFont{20}{24.0}{\familydefault}{\mddefault}{\updefault}{\color[rgb]{0,0,0}$R_2$}}}}}
\end{picture} }
  }
  \caption{$\tau$-compatible, but not composable: different rules.}
  \label{fig:tauComp}
 \end{figure}

We can see from the examples here that the difficulty of defining a composition of transitions comes mainly from three facts. 
The first is that a partial match can have several subgraphs triggering a reaction. 
This is delt with by the construction of the set of active pairs.
The second one is the possibility to connect multiple edges together, not knowing which one exactly is consumed in the reaction. 
Finally,  a given edge can have multiples ways of triggering a reaction.

\subsection{Sufficient conditions}
We now give two frameworks in which neither of the two last problems do occur. 
Avoiding each of them separately is enough to define compositionality properly.
Both cases  are inspired by the study of interaction net systems~\cite{Laf95,EhrReg06,MazzaPhd06},
which can be represented in the 
{}obvious{}    manner as graph transformation systems. 
In these systems, 
the \textsc{dpobc}-diagram built from an admissible rule of a transition is necessarily the one that has to be used to derive the transition. 
In one case, 
it works for essentially the same reasons as in \textsc{ccs}: 
every active element can only interact with a unique other element, 
such as $a$ vs. $\bar a$, $b$ vs. $\bar b$.
In the other one, 
the label itself is not enough, 
but since we also know where it ``connects'' to the graph, 
it is possible to ``find'' the partner that was involved in the transition.

We introduce interaction graph systems, 
which are caracterized among other rewriting systems by the form of the left-hand sides of the reaction rules, 
composed of exactly two hyperedges connected by a single node.
We fix a labeling alphabet $\Lambda$.
\begin{definition}
  An \emph{activated pair} is a hypergraph $L$ on $\Lambda$ composed of two hyperedges $e$ and $f$ and a node  $v$ such that $v$ appears exactly once in $\cnct(e)$ and once in $\cnct(f)$.
  If $v$ is the $i$-th incident vertex of $e$ labelled $\alpha$ and the $j$-th incident vertex of $f$ labelled $\beta$, we denote the activated pair by $\pair{e_i}{f_j}$ and label it by $\pair{\alpha_i}{\beta_j}$.

  An \emph{interaction graph system} $(\Lambda,\R)$ is given by a set of reaction rules $\R$ over hypergraphs on $\Lambda$ 
  where all left-hand side of rules are activated pairs, 
  and  nodes are never deleted, 
  i.e.\ for any rule $\rho = \redrule{L}{I}{R}$, 
  \begin{itemize}
    \item $L$ is an activated pair;
    \item for any node $v$, $v \in L \Rightarrow v \in I$.
  \end{itemize}
  \label{def:interSys}
\end{definition}
Note that for any interaction graph system, 
the set $\mathbb D$ is composed of pairs $\{D,D'\}$ where each of them is composed of an edge and its connected vertices.
Also the minimal interface of any active pair $\{D,D'\}$ is a single node.
It is also the case that it is enough for interfaces to be composed of vertices only.




\begin{example}
\centerparagraph{Simply wired hypergraphs}
  Lafont interaction nets are historically the first interaction nets. 
  They appear as an abstraction of linear logic proof-nets~\cite{Laf95}.
  Originally, 
  Lafont nets have several particular features, but the one we are interested in is the condition on connectivity.
\begin{definition}
  Let $N = \hg$ be a hypergraph on $\Lambda$.


  The graph $N$ is \emph{simply wired} if $\forall v \in V$, $\deg(v) \leq 2$.
  When $\deg(v) = 1$, 
  we say that $v$ is \emph{free}.
\end{definition}

In other words, vertices are only incident to  at most two edges of a graph.
Note that in this special case 
no issues arise if we restrict to the sub-category of simply wired hypergraphs.
For this, we argue that the purpose of the interface is the possible addition of extra context; 
thus, in simply wired hypergraphs, 
it is meaningless for a vertex that is already connected to two edges to be in the interface.

\begin{definition}[Lafont interaction graph system]
  A \emph{Lafont interaction graph} is a simply connected graph such that its interface consists  of free vertices only.
  A \emph{Lafont system} $\mathbb L = (\Lambda,\R)$ is given by  reaction rules over Lafont interaction graphs;
  it is \emph{partitioned} if two left-hand sides only overlap trivially, 
  i.e.\ for two rules $\rho_j = L_j \leftarrow I_j \rightarrow R_j \in \R$ ($j=1,2$), 
  either $L_1 = L_2$ or $L_1 \cap L_2$ is the empty graph (without any nodes and any hyperedges). 
\end{definition}

\begin{lemma}
  Let $\mathbb L$ be a partitioned Lafont system, 
  let $J \rightarrow G$ be a state, let $\theaction$ be a non-$\tau$ transition.
  Then there is exactly one  admissible rule for this transition.
\label{lem:therule1}
\end{lemma}
\end{example}



\begin{example}
  \centerparagraph{Hypergraphs with unique partners}
  By generalizing Lafont interaction nets, we obtain so called \emph{multiwired} interaction nets. 
  But then we lose the unicity of the rule for a given transition label.
  It can be recovered by another condition.


  \begin{definition}[Unique partners]
    Let $\mathbb I = (\Lambda,\R)$ be an interaction graph system.
    We say it is \emph{with unique partners} if for any $\alpha \in \Lambda$ and for all $i \leq \ar(\alpha)$, 
    there exists a unique $\beta \in \Lambda$ and a unique $j \leq \ar(\beta)$ such that $\pair{\alpha_i}{\beta_j}$ is the label of a left-hand side of a rule in $\R$.
  \end{definition}


  \begin{lemma}
    Let $J \rightarrow G$ a state of\/ $\mathbb I$ and $\theaction$ a non-$\tau$ reaction label.
    Then there is exactly one  admissible rule $\rho$ for this transition.
    \label{lem:therule2}
  \end{lemma}
\end{example}



Finally, we conclude our investigation with the following  positive  result. 
\begin{theorem}[Compositionality]
  Let $(\Lambda,\R)$ be a Lafont interaction graph system, 
  or an interaction graph system with unique partners.
  Let $\mathbb D$ be its set of active pairs.
  
  Let $t_1 = \theaction$ and $t_2 = \action{G'}{J'}{F'}{K'}{H'}$ be two non-$\tau$ transitions and $D$ and $D'$  their respective rule additions.

  
  If $\{ D,D' \} \equiv L \in \mathbb D$,  let $\bar G$ and $\bar H$ are described by  the following  diagrams
\begin{displaymath}
\begin{tikzpicture}[scale=.8,baseline={(current bounding box.west)},semithick]
  
  \mnode[star]{\minint{D}{L}}{0,1}
  \mnode[j]{J}{1,0}
  \mnode[jj]{J'}{1,2}
  \mnode[g]{G}{3,0}
  \mnode[gg]{G'}{3,2}
  \mnode[barg]{\bar G}{4,1}
  \mnode[barj]{\bar J}{2,1}
\POC{barg}{180}
  \POC{barj}{180}
\foreach \u/\v in {star/j,star/jj,j/barj,jj/barj,j/g,jj/gg,star/g,star/gg,g/barg,gg/barg} 
  {\ardrawd{\u}{\v}};
\mnode[r]{R}{6,1}
  \mnode[h]{H}{7,0}
  \mnode[hh]{H'}{7,2}
  \mnode[barh]{\bar H}{8,1}
\POC{barh}{180}
\foreach \u/\v in {r/h,r/hh,h/barh,hh/barh} 
  {\ardrawd{\u}{\v}};
\end{tikzpicture}
\end{displaymath}
where $\minint{D}{L} \to J$ and $\minint{D}{L} \to J'$ are the inclusions from the admissibility  of $\rho$ for states $J \to G$ and $J' \to G'$ (Definition~\ref{def:admissibility}).

  Then \[\action{\bar G}{\bar J}{\bar J}{\bar J}{\bar H}.\]
\end{theorem}
\begin{proof}[Sketch of proof]
  By Lemma~\ref{lem:therule1} or~\ref{lem:therule2}, 
  there exists exactly one rule $\rho \in \R$ with $L$ as a left-hand side that allows to derive transitions $t_1$ and $t_2$ -- 
  it is indeed the same rule for both. 
  Let $\mathtt D$ be the composition diagram of the
  \textsc{dpobc}-diagrams justifying the transitions.


  It is first shown that $\bar G \equiv \bar G_c$. 
  Since the upper and lower left squares of $\mathtt D$ are pushouts
  we can infer that  $\bar D \equiv L$ and $\bar J \equiv \bar F$.
  Finally, since no vertex is deleted (see Definition~\ref{def:interSys}),
  we have  $\bar J \to \bar C$ and thus $\bar K \equiv \bar J$.

  So $\mathtt D$ is a \textsc{bc}-diagram of a $\tau$-reaction from $\bar J \to \bar G$ to $\bar J \to \bar H$.

\end{proof}



