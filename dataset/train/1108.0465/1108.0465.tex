







Process calculi, such as \textsc{ccs} and the -calculus, 
have a so-called \emph{communication rule} that allows to synchronize sub-processes to
perform silent actions.
The involved process terms have complementary actions that allow to
interact 
by a ``hand-shake''.
However, it is an open question how such a communication rule can be
obtained for general graph transformations systems
via the Borrowed Context technique.
Roughly, 
the label of a transition does not contain information about which
reaction rule was used to derive it;
in fact, 
the same label might be derived using different rules. 
Intuitively, 
we do not know how to identify the  two hands that have met to shake hands.


To elaborate on this using the  metaphor of handshakes, 
assume that we have an agent that needs a hand to perform a
handshake 
or to deliver an object. 
If we observe this agent reaching out for another hand, 
we cannot conclude from it which of the two possible actions will follow. 
In general, 
even after the action is performed,
it still is not possible to know the decision of the agent --
without extra information, which might however not be observable. 
\marginpar{???! 
\\Intuitively, 
we first show that if the agent was able to shake a hand, and its
partner too, 
we could have done it, even if it is not what ``we did''.}
However, 
with suitable assumptions about the ``allowed actions'',
all necessary information might be available. 







First, we recall from~\cite{BEK06} 
that  \textsc{dpobc}-diagrams (as defined in Definition~\ref{def:dpobc}) can be composed under certain circumstances. 
\begin{fact}
  Let  
  be two transitions obtained from two \textsc{dpobc}-diagrams
  with the same rule .
Then, it is possible to build a \textsc{dpobc}-diagram with the same rule for the composition of  and  along some common interface . 
\end{fact}

Take the following example as illustration of this fact. 

\begin{example}[Composition of transitions] 
  Let  be a state of  
  that contains an edge  with its second connection in the interface
  as shown in Figure~\ref{subfig:st}.
  Further, let  be a state 
  that contains  an edge  with its second connection in the interface
  as shown in Figure~\ref{subfig:nd}.
  Both graphs can trigger a reaction from rule . 
  Such a composition is shown in Figure~\ref{subfig:rd}.
\end{example}

 \begin{figure}[htb]
 \centering
  \subfigure[A first transition]{
    \label{subfig:st}
      \scalebox{.45}{\begin{picture}(0,0)\includegraphics{transition1.pdf}\end{picture}\setlength{\unitlength}{4144sp}\begingroup\makeatletter\ifx\SetFigFont\undefined \gdef\SetFigFont#1#2#3#4#5{\reset@font\fontsize{#1}{#2pt}\fontfamily{#3}\fontseries{#4}\fontshape{#5}\selectfont}\fi\endgroup \begin{picture}(8189,2744)(24,-1928)
\put(3151,-241){\makebox(0,0)[lb]{\smash{{\SetFigFont{20}{24.0}{\familydefault}{\mddefault}{\updefault}{\color[rgb]{0,0,0}}}}}}
\put(3826,-106){\makebox(0,0)[lb]{\smash{{\SetFigFont{20}{24.0}{\familydefault}{\mddefault}{\updefault}{\color[rgb]{0,0,0}}}}}}
\put(496,-376){\makebox(0,0)[lb]{\smash{{\SetFigFont{20}{24.0}{\familydefault}{\mddefault}{\updefault}{\color[rgb]{0,0,0}}}}}}
\put(6976,-1096){\makebox(0,0)[lb]{\smash{{\SetFigFont{20}{24.0}{\familydefault}{\mddefault}{\updefault}{\color[rgb]{0,0,0}}}}}}
\put(6886,299){\makebox(0,0)[lb]{\smash{{\SetFigFont{20}{24.0}{\familydefault}{\mddefault}{\updefault}{\color[rgb]{0,0,0}}}}}}
\put(496,164){\makebox(0,0)[lb]{\smash{{\SetFigFont{20}{24.0}{\familydefault}{\mddefault}{\updefault}{\color[rgb]{0,0,0}}}}}}
\end{picture} }
  }
  
  \subfigure[A second transition]{
        \label{subfig:nd}
      \scalebox{.45}{\begin{picture}(0,0)\includegraphics{transition2.pdf}\end{picture}\setlength{\unitlength}{4144sp}\begingroup\makeatletter\ifx\SetFigFont\undefined \gdef\SetFigFont#1#2#3#4#5{\reset@font\fontsize{#1}{#2pt}\fontfamily{#3}\fontseries{#4}\fontshape{#5}\selectfont}\fi\endgroup \begin{picture}(8189,2654)(24,-1838)
\put(3286,-241){\makebox(0,0)[lb]{\smash{{\SetFigFont{20}{24.0}{\familydefault}{\mddefault}{\updefault}{\color[rgb]{0,0,0}}}}}}
\put(3961,-106){\makebox(0,0)[lb]{\smash{{\SetFigFont{20}{24.0}{\familydefault}{\mddefault}{\updefault}{\color[rgb]{0,0,0}}}}}}
\put(586,-376){\makebox(0,0)[lb]{\smash{{\SetFigFont{20}{24.0}{\familydefault}{\mddefault}{\updefault}{\color[rgb]{0,0,0}}}}}}
\put(496,164){\makebox(0,0)[lb]{\smash{{\SetFigFont{20}{24.0}{\familydefault}{\mddefault}{\updefault}{\color[rgb]{0,0,0}}}}}}
\put(6976,254){\rotatebox{360.0}{\makebox(0,0)[lb]{\smash{{\SetFigFont{20}{24.0}{\familydefault}{\mddefault}{\updefault}{\color[rgb]{0,0,0}}}}}}}
\put(6976,-1186){\rotatebox{360.0}{\makebox(0,0)[lb]{\smash{{\SetFigFont{20}{24.0}{\familydefault}{\mddefault}{\updefault}{\color[rgb]{0,0,0}}}}}}}
\end{picture} }
  }
  
  \subfigure[The composition of the transitions]{
    \label{subfig:rd}
      \scalebox{.45}{\begin{picture}(0,0)\includegraphics{transition3.pdf}\end{picture}\setlength{\unitlength}{4144sp}\begingroup\makeatletter\ifx\SetFigFont\undefined \gdef\SetFigFont#1#2#3#4#5{\reset@font\fontsize{#1}{#2pt}\fontfamily{#3}\fontseries{#4}\fontshape{#5}\selectfont}\fi\endgroup \begin{picture}(8154,3374)(59,-2558)
\put(3479,-230){\makebox(0,0)[lb]{\smash{{\SetFigFont{20}{24.0}{\familydefault}{\mddefault}{\updefault}{\color[rgb]{0,0,0}}}}}}
\put(6886,299){\makebox(0,0)[lb]{\smash{{\SetFigFont{20}{24.0}{\familydefault}{\mddefault}{\updefault}{\color[rgb]{0,0,0}}}}}}
\put(537,284){\makebox(0,0)[lb]{\smash{{\SetFigFont{20}{24.0}{\familydefault}{\mddefault}{\updefault}{\color[rgb]{0,0,0}}}}}}
\put(537,-256){\makebox(0,0)[lb]{\smash{{\SetFigFont{20}{24.0}{\familydefault}{\mddefault}{\updefault}{\color[rgb]{0,0,0}}}}}}
\put(568,-1539){\makebox(0,0)[lb]{\smash{{\SetFigFont{20}{24.0}{\familydefault}{\mddefault}{\updefault}{\color[rgb]{0,0,0}}}}}}
\put(658,-999){\makebox(0,0)[lb]{\smash{{\SetFigFont{20}{24.0}{\familydefault}{\mddefault}{\updefault}{\color[rgb]{0,0,0}}}}}}
\put(7036,-316){\makebox(0,0)[lb]{\smash{{\SetFigFont{20}{24.0}{\familydefault}{\mddefault}{\updefault}{\color[rgb]{0,0,0}}}}}}
\put(6931,-1726){\makebox(0,0)[lb]{\smash{{\SetFigFont{20}{24.0}{\familydefault}{\mddefault}{\updefault}{\color[rgb]{0,0,0}}}}}}
\end{picture} }
  }
  \caption{An example of composition.}
  \label{fig:compo}
 \end{figure}
Hence, 
we see that is in general possible to combine  transitions to obtain new transitions. 
However, 
we emphasize at this point, 
that derivability of a counterpart of the communication rule of \textsc{ccs}
is not the same question as the composition 
of pairs of transitions that come equipped with \emph{complete}~\textsc{bc}-diagrams. 
To clarify the problem, 
consider the following example where we cannot  infer the used rule from the 
transition label. 


\begin{example}
  Let  be a graph composed of two edges  and  and consider a transition label where an edge  is ``added''. 
  Then it is justified by both rules  and  (see Figure \ref{fig:twoPossibilities}). 
\end{example}

 \begin{figure}[htbp]
 \centering
  \subfigure[A transition from rule ]{
      \scalebox{.45}{\begin{picture}(0,0)\includegraphics{transition4.pdf}\end{picture}\setlength{\unitlength}{4144sp}\begingroup\makeatletter\ifx\SetFigFont\undefined \gdef\SetFigFont#1#2#3#4#5{\reset@font\fontsize{#1}{#2pt}\fontfamily{#3}\fontseries{#4}\fontshape{#5}\selectfont}\fi\endgroup \begin{picture}(7904,1889)(-151,-1136)
\put(718, 69){\makebox(0,0)[lb]{\smash{{\SetFigFont{20}{24.0}{\familydefault}{\mddefault}{\updefault}{\color[rgb]{0,0,0}}}}}}
\put(3493,-136){\makebox(0,0)[lb]{\smash{{\SetFigFont{20}{24.0}{\familydefault}{\mddefault}{\updefault}{\color[rgb]{0,0,0}}}}}}
\put(7106,-49){\makebox(0,0)[lb]{\smash{{\SetFigFont{20}{24.0}{\familydefault}{\mddefault}{\updefault}{\color[rgb]{0,0,0}}}}}}
\put( 73, 69){\makebox(0,0)[lb]{\smash{{\SetFigFont{20}{24.0}{\familydefault}{\mddefault}{\updefault}{\color[rgb]{0,0,0}}}}}}
\put(6336,171){\makebox(0,0)[lb]{\smash{{\SetFigFont{20}{24.0}{\familydefault}{\mddefault}{\updefault}{\color[rgb]{0,0,0}}}}}}
\end{picture} }
  }
  
  \subfigure[A transition from rule ]{
      \scalebox{.45}{\begin{picture}(0,0)\includegraphics{transition5.pdf}\end{picture}\setlength{\unitlength}{4144sp}\begingroup\makeatletter\ifx\SetFigFont\undefined \gdef\SetFigFont#1#2#3#4#5{\reset@font\fontsize{#1}{#2pt}\fontfamily{#3}\fontseries{#4}\fontshape{#5}\selectfont}\fi\endgroup \begin{picture}(7934,1884)(-151,-1081)
\put(718, 69){\makebox(0,0)[lb]{\smash{{\SetFigFont{20}{24.0}{\familydefault}{\mddefault}{\updefault}{\color[rgb]{0,0,0}}}}}}
\put(3493,-136){\makebox(0,0)[lb]{\smash{{\SetFigFont{20}{24.0}{\familydefault}{\mddefault}{\updefault}{\color[rgb]{0,0,0}}}}}}
\put(6286,-79){\makebox(0,0)[lb]{\smash{{\SetFigFont{20}{24.0}{\familydefault}{\mddefault}{\updefault}{\color[rgb]{0,0,0}}}}}}
\put( 73, 69){\makebox(0,0)[lb]{\smash{{\SetFigFont{20}{24.0}{\familydefault}{\mddefault}{\updefault}{\color[rgb]{0,0,0}}}}}}
\put(7086,181){\makebox(0,0)[lb]{\smash{{\SetFigFont{20}{24.0}{\familydefault}{\mddefault}{\updefault}{\color[rgb]{0,0,0}}}}}}
\end{picture} }
  }
  \caption{Same transition label for different rules.}
  \label{fig:twoPossibilities}
 \end{figure}
We shall avoid this problem by restricting to suitable classes of graph transformation systems. 
Moreover, 
for simplicities sake, 
we shall focus on the derivation of ``silent'' transitions
in the spirit of the communication rule of \textsc{ccs}. 
\begin{definition}[Silent label]
  A label  is \emph{silent} or 
  if ;
  a \emph{silent transition} is a transition with a silent label. 
\end{definition}
  Intuitively, 
  a silent transition is one that does not induce any ``material''
  change that is visible to an external observer
  that only has access to  the interface of the states. 
  Hence, 
  in particular, 
  a silent transition does not involve additions of the environment
  during the transition.
  Moreover, 
  the interface remains unchanged. 
  This latter requirement does not have any counterpart 
  in process calculi, 
  as the interface is given implicitly by the set of all free names.
  (In graphical encodings of process terms~\cite{bonchi2009labelled}
  it is possible to have free names in the interface
  even though there is no corresponding input or output prefix in
  the term.)



  Now, with the focus on silent transitions,
  for a given rule
  
  we can illustrate the idea of complementary actions as follows.
  If a graph  contains a
  subgraph  of   and moreover a graph  has the complementary
  subgraph of  in  in it,
  then  and  can be combined to obtain a
  big graph  -- the ``parallel composition'' of  and  --
  that has the whole left hand side  as a subgraph
  and thus  can perform the reaction.
  {A natural example for this 
    are Lafont's interaction nets
    where the left hand side consist exactly of two hyper-edges, 
    which in this case are called cells.}
  \marginpar{ so we will only consider complementary subgraphs with
    common interface a single ``wire'' (a single element of dimension
    ). But this can be generalized with some technicalities to any
    kind of ``minimal interface''.}
  The intuitive idea of complementary (basic) actions is 
  captured by the notion of \emph{active pairs}. 


\begin{definition}[Active pairs] 
  For any inclusion , where  and for all nodes  of , , 
  let the following square be its initial pushout  
  
i.e.\  is the smallest subgraph of   that allows for completion to a pushout. 
We call  the \emph{complement of  in } and  the \emph{minimal interface of  in }
and we write  if . 
 The set of \emph{active pairs} is 
 
  Abusing notation, 
  we also denote by  the union of .  
\label{def:activepair}
\end{definition}
It is easy to verify that the complement of  in  is  itself and that its minimal interface is also .
It is the set of ``acceptable'' partial matches in the sense that they do not yield a -reaction on their own.
Indeed, if  is equal to , 
then the resulting transition of this partial match is a -transition.
And if it is just composed of vertices, its complement is 
and thus not acceptable.

\begin{example}[Active pairs]
  In our running example, 
  the set  of our example is in obvious bijection to
     
  The minimal interface of any pair is a single vertex.
\end{example}
This completes the introduction of preliminary concepts to tackle the issues that 
have to be resolved to obtain ``proper'' compositionality of transitions. 

\subsection{Towards a partial solution}
Let us address the problem of identifying the rule that is ``responsible'' for a given interaction. 
We start by considering the left inclusions of labels, 
which intuitively describe possible borrowing actions from the environment.
Relative to this,  we define the \emph{admissible rules} as those rules that
can be used to let states evolve while borrowing the specified ``extra material'' from the environment.

\begin{definition}[Admissible rule]
  \label{def:admissibility}
  Let  be a state
  and let  be an inclusion (which represents a possible contribution of the context). 
  A rule  is \emph{admissible  (for )}  
  if  and it is possible to find  and  the left-hand side of , such that the following diagram commutes 

where  is the minimal interface of  in . 
We call  the \emph{rule addition}.

\end{definition}
This just means that  \emph{can} evolve using the rule  if   is added at the proper location. 



\begin{proposition}[Precompositionality]
   Let  and  be two transitions such that a single rule  is admissible  for both, and let  and  be their respective rule additions.
  If , 
  it is possible to compose  and  into a graph  in a way to be able to derive a
  -transition using rule .
\label{prop:compositionality}
\end{proposition}

\begin{proof}
  We first show that in such a case,  and the pushout of  is exactly . Similarly,  and the pushout of  is exactly .
  Then, it is easy to see that it is possible to build the
  \textsc{dpobc}-diagram   using rule   on 
  (respectively ) yelding the transition
   for some  (respectively the
  \textsc{dpobc}-diagram  yelding the transition
   for some ), and then compose
   and .









This follows from   and .
Indeed,  so the top left morphism of the composed \textsc{dpobc}-diagram is an isomorphism
and so are the ones under it, using basic pushout properties.
\end{proof}

This first result  motivates the following definition. 
\begin{definition}[-compatible]
  In the situation of Proposition~\ref{prop:compositionality}, 
  we say the two transitions  are -\emph{compatible}. 
\end{definition}

\begin{remark}
  In general, in Proposition~\ref{prop:compositionality}, 
  the result of the -transition cannot be constructed from  and ;
  thus we do not yet speak of compositionality.
\end{remark}

\begin{example}
  Let  be a graph composed of two edges  and  and  of two edges  and   (see Figure~\ref{fig:tauComp}).
  Then the rule  is admissible for both transitions
  and moreover they are -compatible. 
  The rule  yields the respective rule additions.
  ``Glueing''  and  by their interface results in a graph with edges  and two s;
  the latter graph  can perform a -reaction  from rule ,
  which however does not give the desired result
  since the target state is not the ``expected  composition'' of  and . 
  In other words, 
  although we have been able to  construct a -transition,
  it is not the composition of the original transitions.
\end{example}

 \begin{figure}[htbp]
 \centering
  \subfigure[A transition from rule ]{
      \scalebox{.45}{\begin{picture}(0,0)\includegraphics{transition6.pdf}\end{picture}\setlength{\unitlength}{4144sp}\begingroup\makeatletter\ifx\SetFigFont\undefined \gdef\SetFigFont#1#2#3#4#5{\reset@font\fontsize{#1}{#2pt}\fontfamily{#3}\fontseries{#4}\fontshape{#5}\selectfont}\fi\endgroup \begin{picture}(7599,1889)(39,-1136)
\put(3504,-84){\makebox(0,0)[lb]{\smash{{\SetFigFont{20}{24.0}{\familydefault}{\mddefault}{\updefault}{\color[rgb]{0,0,0}}}}}}
\put(936, 76){\makebox(0,0)[lb]{\smash{{\SetFigFont{20}{24.0}{\familydefault}{\mddefault}{\updefault}{\color[rgb]{0,0,0}}}}}}
\put(6421,156){\makebox(0,0)[lb]{\smash{{\SetFigFont{20}{24.0}{\familydefault}{\mddefault}{\updefault}{\color[rgb]{0,0,0}}}}}}
\put(263, 64){\makebox(0,0)[lb]{\smash{{\SetFigFont{20}{24.0}{\familydefault}{\mddefault}{\updefault}{\color[rgb]{0,0,0}}}}}}
\put(6996, 51){\makebox(0,0)[lb]{\smash{{\SetFigFont{20}{24.0}{\familydefault}{\mddefault}{\updefault}{\color[rgb]{0,0,0}}}}}}
\end{picture} }
  }
  
  \subfigure[A transition from rule ]{
      \scalebox{.45}{\begin{picture}(0,0)\includegraphics{transition7.pdf}\end{picture}\setlength{\unitlength}{4144sp}\begingroup\makeatletter\ifx\SetFigFont\undefined \gdef\SetFigFont#1#2#3#4#5{\reset@font\fontsize{#1}{#2pt}\fontfamily{#3}\fontseries{#4}\fontshape{#5}\selectfont}\fi\endgroup \begin{picture}(7709,1889)(39,-1136)
\put(956, 76){\makebox(0,0)[lb]{\smash{{\SetFigFont{20}{24.0}{\familydefault}{\mddefault}{\updefault}{\color[rgb]{0,0,0}}}}}}
\put(354,171){\rotatebox{360.0}{\makebox(0,0)[lb]{\smash{{\SetFigFont{20}{24.0}{\familydefault}{\mddefault}{\updefault}{\color[rgb]{0,0,0}}}}}}}
\put(7089, 51){\rotatebox{360.0}{\makebox(0,0)[lb]{\smash{{\SetFigFont{20}{24.0}{\familydefault}{\mddefault}{\updefault}{\color[rgb]{0,0,0}}}}}}}
\put(3446,-99){\rotatebox{360.0}{\makebox(0,0)[lb]{\smash{{\SetFigFont{20}{24.0}{\familydefault}{\mddefault}{\updefault}{\color[rgb]{0,0,0}}}}}}}
\put(6419, 91){\makebox(0,0)[lb]{\smash{{\SetFigFont{20}{24.0}{\familydefault}{\mddefault}{\updefault}{\color[rgb]{0,0,0}}}}}}
\end{picture} }
  }
  \caption{-compatible, but not composable: different rules.}
  \label{fig:tauComp}
 \end{figure}

We can see from the examples here that the difficulty of defining a composition of transitions comes mainly from three facts. 
The first is that a partial match can have several subgraphs triggering a reaction. 
This is delt with by the construction of the set of active pairs.
The second one is the possibility to connect multiple edges together, not knowing which one exactly is consumed in the reaction. 
Finally,  a given edge can have multiples ways of triggering a reaction.

\subsection{Sufficient conditions}
We now give two frameworks in which neither of the two last problems do occur. 
Avoiding each of them separately is enough to define compositionality properly.
Both cases  are inspired by the study of interaction net systems~\cite{Laf95,EhrReg06,MazzaPhd06},
which can be represented in the 
{}obvious{}    manner as graph transformation systems. 
In these systems, 
the \textsc{dpobc}-diagram built from an admissible rule of a transition is necessarily the one that has to be used to derive the transition. 
In one case, 
it works for essentially the same reasons as in \textsc{ccs}: 
every active element can only interact with a unique other element, 
such as  vs. ,  vs. .
In the other one, 
the label itself is not enough, 
but since we also know where it ``connects'' to the graph, 
it is possible to ``find'' the partner that was involved in the transition.

We introduce interaction graph systems, 
which are caracterized among other rewriting systems by the form of the left-hand sides of the reaction rules, 
composed of exactly two hyperedges connected by a single node.
We fix a labeling alphabet .
\begin{definition}
  An \emph{activated pair} is a hypergraph  on  composed of two hyperedges  and  and a node   such that  appears exactly once in  and once in .
  If  is the -th incident vertex of  labelled  and the -th incident vertex of  labelled , we denote the activated pair by  and label it by .

  An \emph{interaction graph system}  is given by a set of reaction rules  over hypergraphs on  
  where all left-hand side of rules are activated pairs, 
  and  nodes are never deleted, 
  i.e.\ for any rule , 
  \begin{itemize}
    \item  is an activated pair;
    \item for any node , .
  \end{itemize}
  \label{def:interSys}
\end{definition}
Note that for any interaction graph system, 
the set  is composed of pairs  where each of them is composed of an edge and its connected vertices.
Also the minimal interface of any active pair  is a single node.
It is also the case that it is enough for interfaces to be composed of vertices only.




\begin{example}
\centerparagraph{Simply wired hypergraphs}
  Lafont interaction nets are historically the first interaction nets. 
  They appear as an abstraction of linear logic proof-nets~\cite{Laf95}.
  Originally, 
  Lafont nets have several particular features, but the one we are interested in is the condition on connectivity.
\begin{definition}
  Let  be a hypergraph on .


  The graph  is \emph{simply wired} if , .
  When , 
  we say that  is \emph{free}.
\end{definition}

In other words, vertices are only incident to  at most two edges of a graph.
Note that in this special case 
no issues arise if we restrict to the sub-category of simply wired hypergraphs.
For this, we argue that the purpose of the interface is the possible addition of extra context; 
thus, in simply wired hypergraphs, 
it is meaningless for a vertex that is already connected to two edges to be in the interface.

\begin{definition}[Lafont interaction graph system]
  A \emph{Lafont interaction graph} is a simply connected graph such that its interface consists  of free vertices only.
  A \emph{Lafont system}  is given by  reaction rules over Lafont interaction graphs;
  it is \emph{partitioned} if two left-hand sides only overlap trivially, 
  i.e.\ for two rules  (), 
  either  or  is the empty graph (without any nodes and any hyperedges). 
\end{definition}

\begin{lemma}
  Let  be a partitioned Lafont system, 
  let  be a state, let  be a non- transition.
  Then there is exactly one  admissible rule for this transition.
\label{lem:therule1}
\end{lemma}
\end{example}



\begin{example}
  \centerparagraph{Hypergraphs with unique partners}
  By generalizing Lafont interaction nets, we obtain so called \emph{multiwired} interaction nets. 
  But then we lose the unicity of the rule for a given transition label.
  It can be recovered by another condition.


  \begin{definition}[Unique partners]
    Let  be an interaction graph system.
    We say it is \emph{with unique partners} if for any  and for all , 
    there exists a unique  and a unique  such that  is the label of a left-hand side of a rule in .
  \end{definition}


  \begin{lemma}
    Let  a state of\/  and  a non- reaction label.
    Then there is exactly one  admissible rule  for this transition.
    \label{lem:therule2}
  \end{lemma}
\end{example}



Finally, we conclude our investigation with the following  positive  result. 
\begin{theorem}[Compositionality]
  Let  be a Lafont interaction graph system, 
  or an interaction graph system with unique partners.
  Let  be its set of active pairs.
  
  Let  and  be two non- transitions and  and   their respective rule additions.

  
  If ,  let  and  are described by  the following  diagrams

where  and  are the inclusions from the admissibility  of  for states  and  (Definition~\ref{def:admissibility}).

  Then 
\end{theorem}
\begin{proof}[Sketch of proof]
  By Lemma~\ref{lem:therule1} or~\ref{lem:therule2}, 
  there exists exactly one rule  with  as a left-hand side that allows to derive transitions  and  -- 
  it is indeed the same rule for both. 
  Let  be the composition diagram of the
  \textsc{dpobc}-diagrams justifying the transitions.


  It is first shown that . 
  Since the upper and lower left squares of  are pushouts
  we can infer that   and .
  Finally, since no vertex is deleted (see Definition~\ref{def:interSys}),
  we have   and thus .

  So  is a \textsc{bc}-diagram of a -reaction from  to .

\end{proof}



