

\documentclass[journal,12pt,draftcls,onecolumn]{IEEEtran}




\usepackage{graphicx}
\usepackage{caption}
\usepackage{subcaption}
\usepackage{amssymb}
\usepackage{mathrsfs}
\usepackage{cite}
\usepackage{amsmath}
\usepackage[table]{xcolor}
\usepackage{tabularx}
\usepackage{amsthm}
\usepackage{url}
\usepackage{soul}


\newcommand{\tran}{\mathrm{T}}
\newcommand{\bsm}{\begin{smallmatrix}}
\newcommand{\esm}{\end{smallmatrix}}
\newcommand{\bbm}{\begin{bmatrix}}
\newcommand{\ebm}{\end{bmatrix}}
\newcommand{\bpm}{\begin{pmatrix}}
\newcommand{\epm}{\end{pmatrix}}
\newcommand{\bom}{\begin{matrix}}
\newcommand{\eom}{\end{matrix}}
\newcommand{\bs}{\begin{small}}
\newcommand{\es}{\end{small}}
\newcommand{\rank}{\mathrm{rank}}
\newcommand{\diag}{\mathrm{diag}}
\newcommand{\CA}[1]{(,)}

\newcommand{\ssp}[1]{\mathscr{#1}}      \newcommand{\setssp}[1]{\mathfrak{#1}}   \newcommand{\fld}[1]{\mathbb{#1}}       \newcommand{\op}[1]{\mathcal{#1}}       \newcommand{\sumbanach}[1]{\sum{#1}}
\newcommand{\spanset}[1]{\ \mathrm{span}\{#1\}\ } \newcommand{\idv}{{\bf x}_{-\infty}^\infty}				\newcommand{\fdv}{{\bf x}}
\newcommand{\pardiff}[2]{\frac{\partial {#1}}{\partial {#2}}}
\newcommand{\twodvecP}[1]{\bbm {#1}_r(i+1,j)\\{#1}_s(i,j+1)\ebm}
\newcommand{\twodvec}[1]{\bbm {#1}_r(i,j)\\{#1}_s(i,j)\ebm}
\DeclareMathOperator{\ima}{Im}
\newcommand{\infd}{Inf-D }
\def\QEDclosed{\hfill\IEEEQEDclosed}
\renewcommand{\qed}{\QEDclosed}
\renewenvironment{proof}[1][\proofname]{\noindent\nobreakspace{\bfseries #1}:\;}{\qed\par}

\newcommand{\myhl}[1]{{#1}}
\newcommand{\hilight}[1]{\hl{#1}}
\newtheorem{theorem}{Theorem}
\newtheorem{lemma}{Lemma}
\newtheorem{corollary}{Corollary}
\newtheorem{remark}{Remark}
\newtheorem{definition}{Definition}
\newcounter{MYtempeqncnt}

\hyphenation{op-tical net-works semi-conduc-tor}


\begin{document}

\title{A Geometric Approach to Fault Detection and Isolation of Two-Dimensional (2D) Systems}


\author{Amir~Baniamerian, Nader~Meskin and Khashayar Khorasani\thanks{*This publication was made possible by NPRP grant No. 4-195-2-065 from the
Qatar National Research Fund (a member of Qatar Foundation). The
statements made herein are solely the responsibility of the authors}\thanks{A. Baniamerian and K. Khorasani are with the Department of Electrical and Computer Engineering,
        Concordia University, Quebec, Canada
        {\tt\small am\_bani@encs.concordia.ca} and {\tt\small kash@ece.concordia.ca}}\thanks{N. Meskin is with the Department of Electrical Engineering, Qatar University,
        Doha, Qatar
        {\tt\small nader.meskin@qu.edu.qa}}}


\maketitle

\begin{abstract}
In this work, we develop a novel fault detection and isolation (FDI) scheme for discrete-time two-dimensional (2D) systems that are represented by the Fornasini-Marchesini model II (FMII). This is accomplished by generalizing the  basic invariant subspaces including unobservable, conditioned invariant and unobservability subspaces of 1D systems to 2D models. These extensions have been achieved and facilitated by representing a 2D model as an infinite dimensional (Inf-D) system  on a Banach vector space, and by particularly constructing algorithms that compute these subspaces in a \emph{finite and known} number of steps. By utilizing the introduced subspaces the FDI problem is formulated and necessary and sufficient conditions for its solvability are provided. Sufficient conditions for solvability of the FDI problem for 2D systems using both deadbeat and LMI filters are also developed. Moreover, the capabilities and advantages of our proposed approach are demonstrated  by performing  an analytical comparison with the currently available 2D geometric methods in the literature. Finally, numerical simulations corresponding to an approximation of a hyperbolic partial differential equation (PDE)  system of a heat exchanger, that is mathematically represented as a 2D model, have also been provided. \end{abstract}

\begin{IEEEkeywords}
2D systems, Fornasini-Marchesini model, infinite dimensional systems, fault detection and isolation, geometric approach, invariant subspaces, LMI-based observer design, deadbeat observer.
\end{IEEEkeywords}


\IEEEpeerreviewmaketitle

\section{Introduction}
\IEEEPARstart{O}{ver} the past few decades, the problem of fault detection and isolation (FDI) of dynamical systems has increasingly received larger interest and attention from the control community \cite{IsermannBook2006} (c.f. to the references therein). The increasing complexity of human-made machines and devices, such as gas turbine engines and chemical processes, has necessitated the need for development of more complex and sophisticated FDI algorithms. Not with standing this requirement,  development of  FDI algorithms for systems that are governed by partial differential equations (PDE) has not been fully addressed and received extensive attention in the literature. One approach to investigate the FDI problem of PDEs relies on obtaining an approximate model of the system. First, the PDE system is approximated by a simple model (such as an ordinary differential equation (ODE) model), and then sufficient conditions for solvability of the FDI problem are derived based on this approximate model.

It is well-known that {\it parabolic PDE} systems can be approximated by  ODE representations. These systems can be approximated through application of finite element methods  where sufficient conditions can then be derived by using singular perturbation theory \cite{Christo_Book}. The FDI problem of parabolic PDEs has been addressed by using the corresponding approximate models in \cite{ACC2012, Davis_Journal}.
On the other hand, by discretizing through  spatial coordinates, one can approximate  {\it hyperbolic PDE} systems by  ODE models. However, unlike parabolic PDE systems, one cannot apply model decomposition, order reduction and singular perturbation theory to hyperbolic PDE systems \cite{Christofer_Hyperbolic}. Moreover, the order of the resulting approximate ODE systems can be dramatically high. Therefore, researchers have investigated hyperbolic PDE systems by using  other formal methods such as the theory of semigroups \cite{Curtain_Book} and  backstepping methods \cite{Krstic_Hyperbolic}.

As shown in \cite{HyperPDE_2D}, a {\it single} hyperbolic PDE system can be approximated by using two-dimensional (2D) systems. In \cite{ACC2013}, we have shown that this approximation is also applicable to a {\it system} of hyperbolic PDEs. Moreover, as shown in \cite{Parabolic3D}, parabolic PDE systems can be approximated by three-dimensional (3D) Fornasini and Marchesini model II (FMII)  representations \cite{Kaczorek_Book, 2DRef, Stability2D2013, FMinBook}. We have shown in the Appendix that it is straightforward to extend the results of our work to 3D systems as well (these results are not included here due to space limitations). Consequently, our proposed methodology in this work can also be applied to parabolic PDE systems. There are only a few results on FDI of 2D systems in the literature, such as those by using dead-beat observers \cite{BisiaccoMultiDim} and parity equations \cite{FMResidual, BisiaccoLetter}. On the other hand, the FDI of hyperbolic PDE systems (by using \underline{even an approximate model}) has \underline{not} yet been adequately addressed in the literature.

It should be noted that 2D system theory has other applications in the control field. For example, a class of discrete-time linear repetitive processes can be modeled by  2D systems. These  processes play important roles in tracking control and robotics, where the controlled system is required to perform a periodic task with  high precision (refer to \cite{RepitativeProcess_Book} for more details on repetitive systems). One of the main approaches to control linear repetitive processes is the iterative learning control (ILC)\cite{RepitativeProcess_Book}. Since the ILC problem can be formulated as a control problem in the 2D system theory \cite{ILC_2D_Tac}, 2D systems have been increasingly applied to spatio-temporal and repetitive process control problems in the literature.


The FDI of 1D linear, time-invariant systems has been extensively investigated during the past few decades \cite{IsermannBook2006} (and the references therein). The geometric approach \cite{Mass_Thesis, Massoumnia1986,  Massoumnia1989} has provided a valuable tool for investigating the FDI problem of a large class of dynamical systems such as linear time-invariant \cite{Massoumnia1989}, Markovian jump systems \cite{DrMeskin_TacFullPaper}, time-delay systems \cite{DrMeskin_Delay}, linear impulsive systems \cite{DrMeskin_Impulsive}, and parabolic PDE  systems \cite{ACC2012}. Moreover, the geometric approach is also extended to affine nonlinear systems in \cite{IsidoriFDI}.  Furthermore, hybrid geometric FDI approaches for linear and nonlinear systems have also been provided in \cite{MKR_SysTech2010} and \cite{MKR_Nonlin2010}, where a set of residual generators are equipped with a discrete event-based system fault diagnoser to solve the FDI problem.

Motivated by the above discussion, in this paper we investigate the FDI problem of 2D systems and apply the results to a 2D approximate model of a hyperbolic PDE system. As stated earlier, a hyperbolic PDE system can be approximated by  1D systems, where the order of the approximate 1D system will significantly increase by decreasing the gridding size \cite{Christofer_Hyperbolic}. We also provide an example where faults in the 1D approximate model are not isolable, whereas one can detect and isolate  faults by applying the 2D approximate model and by using our proposed methodology.


Recently, the geometric theory of 2D  systems has attracted much interest, where basic concepts such as conditioned invariant and controllable subspaces are studied in detail for the Fornasini and Marchesini model I (FMI) \cite{ ntogramatzidis2012GeometryArticle, ntogramatzidis2012Siam}. The hybrid 2D systems have also been investigated from the geometric point of view in \cite{Valcher2013}. The geometric FDI approach for 2D systems, for the first time, was addressed in \cite{ACC2013}, where invariant subspaces of the Roesser model are defined and the FDI problem is formulated based on these subspaces.

Two-dimensional (2D) systems have  been extensively investigated from a system theory point of view \cite{Kaczorek_Book, 2DRef, Stability2D2013, FMinBook}. Particularly, system theory concepts such as stability \cite{2DLyapunov, Stability2D2013}, controllability \cite{Controlability2D}, observability\cite{Valcher2013}, and state reconstruction \cite{2DKalman} have been investigated in the literature. However, due to complexity of 2D systems, unlike one-dimensional (1D) systems,  there are various definitions that are introduced for controllability and observability properties. Not surprisingly, the duality between the observability and controllability does not hold in 2D systems.
In this paper, we investigate observability of 2D systems from a \underline{{\it new}} geometric point of view  which has its roots in system theory of infinite dimensional (Inf-D) systems.

Compared to the results reported in \cite{ACC2013} and \cite{ACC2014}, we have specific generalization and novel contributions in this work. We \underline{first} investigate the Fornasini and Marchesini model II (FMII) as an Inf-D system that allows us to deal with Inf-D subspaces (albeit with a finite dimensional (Fin-D) representation). The invariant subspaces and the corresponding algorithms  that are introduced in \cite{ACC2013} for the Roesser model are then \underline{generalized} to the 2D FMII systems. It is worth noting that although the introduced subspaces in this work are Inf-D, the corresponding algorithms for constructing the subspaces converge in a finite and known number of steps.

In addition, in \cite{ACC2013} only sufficient conditions for solvability of the FDI problem of the Roesser model were provided. As shown in \cite{ACC2014}, the invariance property  of an unobservable subspace is a generic property of 2D systems. We first derive a \underline{single} necessary and sufficient condition for detectability and isolability of faults (that is formally introduced in the next section). It is shown that this condition also necessary for solvability of the FDI. Furthermore, in \cite{ACC2013} by utilizing the existence of an LMI-based observer  only sufficient conditions for solvability of the Roesser model FDI problem were provided. In other words, the procedure to design the observer gains is not provided in \cite{ACC2013,ACC2014}. However, in our paper, we derive both necessary and sufficient conditions, where sufficient conditions are based on  (a) an ordinary, (b) a delayed deadbeat, and (c) an LMI-based 2D Luenberger filters. Moreover,  we develop a procedure to design LMI-based filter gains. 

It must be noted that recently related work has appeared in \cite{Malek_3DFDI} and \cite{Malek_3DFDIConf}. These two papers investigate the FDI problem of three-dimensional (3D) FMII models. Although, a geometric FDI methodology is also developed in \cite{Malek_3DFDI}, our work is distinct and unique from \cite{Malek_3DFDI} in \underline{{\it three}} main perspectives:
\begin{enumerate}
	\item The approach proposed  in \cite{Malek_3DFDI} is based on the results of \cite{Massoumnia1986}, whereas our approach is based on the generalization of the results of \cite{Massoumnia1986} as reported in \cite{Massoumnia1989} (the results in \cite{Massoumnia1989} are more general than \cite{Massoumnia1986}) for 2D models (for more information refer to Remark \ref{Rem:GeneralFilter}).
	\item In \cite{Malek_3DFDI}, necessary and sufficient conditions for solvability of the FDI problem were derived for a \underline{subclass of detection filters} where it was assumed that the output map of the detection filters and that of the system are identical. However,  in our work here we consider a \underline{general class of detection filters} for the residual generation and relax this condition.
	\item As shown in Section \ref{Sec:FDIComparison}, the observability property of the 2D model is a fundamental requirement and assumption in \cite{Malek_3DFDI} (although it is stated in \cite{Malek_3DFDI} that this assumption was made  for simplicity of their presentation). However, our proposed solution \underline{does not} require this condition and assumption, and consequently our approach leads to a less restrictive solution.
\end{enumerate}

Another approach that was developed in the literature \cite{BisiaccoMultiDim,BisiaccoLetter} has its roots in 2D deadbeat observers \cite{Bisiacco_Obs}. In \cite{BisiaccoMultiDim}, the FDI  problem is investigated by using polynomial matrices and unknown input deadbeat observers, where the right zero primeness of the 2D Popov-Belevitch-Hautus (PBH) matrix (this is reviewed in Subsection \ref{Sec:DeabeatObs})  is a necessary condition for solvability of the FDI problem. In \cite{BisiaccoLetter}, this condition was relaxed and  necessary and sufficient conditions that are based on an extended parity equation approach were obtained.

To provide a fair and comprehensive comparison with the currently available result in \cite{Malek_3DFDI}, in this work it is shown that on one hand the solvability of the FDI problem by using the method in \cite{Malek_3DFDI} is also sufficient to accomplish the FDI task by using our proposed approach. On the other hand, there are certain 2D systems that are \underline{not solvable} by the approach in \cite{Malek_3DFDI}, however our proposed approach \underline{can both detect and isolate} the faults. Also, by comparing our proposed results with the algebraic-based methods in \cite{BisiaccoMultiDim,BisiaccoLetter}, where they can solve and derive necessary and sufficient conditions for solvability of the FDI problem, we will highlight and emphasize  two important considerations as follows:
\begin{enumerate}
	\item The algebraic methods, in contrast to our geometric approach,  need a closed-form and analytical solution to certain polynomial matrix equations (in two variables). However, our proposed approach is derived and solved by using commonly available and relatively straightforward numerical methods.
	\item As shown subsequently in Section \ref{Sec:FDIComparison}, there are certain examples, where the necessary conditions in \cite{BisiaccoMultiDim,BisiaccoLetter} are not satisfied. However, our proposed approach can \underline{both detect and isolate} these fault scenarios.
\end{enumerate}
To summarize, the \underline{main contributions} of this paper can be highlighted as follows:
\begin{enumerate}
	\item By reformulating  2D models as  Inf-D systems, the invariance property of the unobservable subspace is investigated (this is provided in Section \ref{Sec:InvSpace}), where an Inf-D unobservable subspace is also introduced. This result enables one to formally address the  solvability of the FDI problem without restriction on the initial conditions (unlike in \cite{ACC2013, ACC2014} where a restrictive assumption that the unobservable subspace is Fin-D  is imposed). Since 2D systems are Inf-D dynamical system, our proposed Inf-D representation and framework enables one to address the FDI problem in its most general scenario than that in \cite{ACC2013} and \cite{ACC2014}.
	\item Two important Inf-D invariant subspaces (namely, the conditioned invariant and the unobservability) are introduced for the FMII models. Although, these subspaces are Inf-D, we provide explicit algorithms that can be invoked to compute these subspaces in a finite and known number of steps.
	\item The FDI problem of 2D systems is formulated in terms  of the above introduced invariant subspaces, and necessary and sufficient conditions for its solvability are derived and formally analyzed.
	\item A novel procedure is developed for designing an observer (also known as a detection filter)  by utilizing the linear matrix inequalities (LMI) technique.
	\item Three sets of sufficient conditions for solvability of the FDI problem by utilizing the ordinary, the delayed deadbeat, as well as our proposed LMI-based observers (detection filters) are also provided.
	\item Analytical comparisons between our proposed approach and the one in \cite{Malek_3DFDI} are presented. We show that the sufficient conditions in\cite{Malek_3DFDI} are also sufficient for solvability of the FDI problem by using our  approach. However, an example is provided that shows our method can both detect and isolate faults, whereas the approach in  \cite{Malek_3DFDI}  \underline{cannot} be used. In other words, it is shown that if the method in the above literature can accomplish the FDI task, our proposed approach can also accomplish this task. However, if our scheme \underline{cannot} achieve the fault detection and isolation goals for a given system, then it is guaranteed that \underline{the schemes} in \cite{Malek_3DFDI} cannot also achieve these goals. Moreover, there are 2D systems  where our approach can achieve the FDI objectives whereas the results in the  literature \underline{cannot} solve the FDI problem.
\item Our proposed methodology and strategy is applied to an important application area of a heat exchanger (a hyperbolic PDE system), where it is shown that one can \underline{simultaneously} detect and isolate two different faults namely, the leakage and the fouling faults.
\end{enumerate}

The remainder of the paper is organized as follows. The {\it preliminary} results including the Inf-D representation, the FDI problem formulation, the 2D deadbeat observer and the 2D Luenberger observers (detection filters) are presented in Section \ref{Sec:Backgground}. The unobservable subspaces of the FMII 2D model are introduced in Section \ref{Sec:InvSpace}. The geometric property of these subspaces and the invariant concept of the FMII model are also presented in Section \ref{Sec:InvSpace}. In Section \ref{Sec:FDI}, necessary and sufficient conditions for solvability of the FDI problem are derived and developed. Analytical comparisons between our proposed approach and the available \emph{geometric methods} in the literature, namely  \cite{Malek_3DFDI} and \cite{Malek_3DFDIConf} are provided in this section. Furthermore,  numerical comparisons with both \textit{geometric and algebraic methods} in \cite{Malek_3DFDI,Malek_3DFDIConf,BisiaccoMultiDim,BisiaccoLetter} are presented in this section. Simulation results for the FDI problem of a heat exchanger that is expressed as a PDE system are conducted in Section \ref{Sec:Simulation}. Finally, Section \ref{Sec:Conc} concludes the paper and provides suggestions for future work.

{\bf Notation:}
In this work,  are used to denote subspaces. For a given vector , the subspace  is denoted by . The inverse image of the subspace  with respect to the operator  is denoted by . The block diagonal matrix  is denoted by . The real, complex, integer and positive integer numbers are denoted by , ,  and , respectively.  denotes the set . In this paper, we deal with infinite dimensional (Inf-D) subspaces and vectors. An Inf-D vector is designated by the bold letters . The Inf-D subspace  is denoted by , where . Let  and  , where . The vector space  is defined as . It can be shown that  is a Banach (but not necessarily Hilbert) space.  Let  and . The Inf-D vector  is expressed as , where  for all , and associated with  we simply  use . The other notations are provided within the text of the paper as appropriate.
\section{Preliminary Results}\label{Sec:Backgground}
In this section, we first review 2D systems and their various representational models. Subsequently,  a 2D system is expressed as an infinite dimensional (Inf-D) system that allows one to geometrically analyze the unobservable subspaces (this is to be defined and specified in the next section). The FDI problem is also formulated in this section. Moreover, we review the 2D Popov-Belevitch-Hautus (PBH) matrix and 2D deadbeat observers in this section. Finally, an LMI-based approach is introduced to design a 2D Luenberger observer (also known as a detection filter) for 2D systems.
\subsection{Discrete-Time 2D Systems} \label{Sec:Dis_2D}    2D models can be used for representing a large class of problems such as approximating hyperbolic PDE systems \cite{ACC2013, HyperPDE_2D}, image processing  and digital filtering \cite{Roess}. System theory concepts such as observability, controllability and feedback stabilization have also been investigated in the literature  for 2D systems \cite{Kaczorek_Book, FMinBook, ACC2013, ntogramatzidis2012Siam, Valcher2013}.
There are various models that are adopted in the literature for 2D systems including the Rosser model \cite{Roess}, the Fornasini-Marichesini model I (FMI) and model II (FMII) \cite{FMinBook,Kaczorek_Book}. The FMI can be formulated as a Roesser model and the Roesser model is a special case of the FMII model \cite{Kaczorek_Book}. In this work, we consider and concentrate on the FMII model, and consequently our results are also derived for this general class of 2D systems. 

Consider the following FMII model \cite{FMinBook}, 
where , , and  denote the state, input and output vectors, respectively. The fault signals and the corresponding fault signatures are designated by,  and , respectively. Also,  denotes the number of faults in the system. Since in this work all the introduced invariant subspaces are based on the operators ,  and , we designate the system \eqref{Eq:FMII} by the triple (,,). \begin{remark}\label{Rem:onFMII-General}
	Note that system \eqref{Eq:FMII} represents and captures the presence of \underline{both} actuator and component faults. To represent sensor faults, one can augment the sensor dynamics and model the sensor faults as actuator faults in the augmented system (for a complete discussion on this issue refer to \cite{Mass_Thesis} - Chapters 3 and 4). Also, it should be pointed out that the fault signal  affects the system through two different fault signatures  and . An alternative fault model could have been expressed according to the following representation,
	
	Model \eqref{Eq:FMII} is more general than the one given by equation  \eqref{Eq:FMII_SingFault}. This is due to the fact that by denoting   for all , one can represent the model \eqref{Eq:FMII_SingFault} as in the model \eqref{Eq:FMII}. \qed
\end{remark}

Let us now consider the Roesser model \cite{Roess} which is expressed as

and where  represents the state, and the variables , ,  and  are defined as in equation \eqref{Eq:FMII}. By defining

one can formulate the Roesser model \eqref{Eq:RoesserModel} as in equation \eqref{Eq:FMII}. In this paper, we assume that  and  in model \eqref{Eq:FMII} are not necessarily commutative (i.e. ), and hence, the results that are subsequently developed  can also be applied to the Roesser model \eqref{Eq:RoesserModel}. It should be noted that the commutativity of   and  is a strong condition that renders the results in\cite{Valcher2013} (where it is assumed that  and  are commutative) not applicable to the system \eqref{Eq:RoesserModel}.

In this work, we will investigate and develop FDI strategies for the model \eqref{Eq:FMII}. It is assumed that  and  in model \eqref{Eq:FMII} are not necessarily commutative (i.e. ), and hence, the results that are subsequently developed  can also be applied to the Roesser model. It should be emphasized that the commutativity of   and  is a strong condition that renders the results in\cite{Valcher2013} (where   and  are assumed to commutate) not applicable to Roesser systems.

\subsection{Infinite Dimensional (\infd) Representation} \label{Sec:IDRep}
In this subsection, we reformulate the 2D model \eqref{Eq:FMII} as an Inf-D system that will be used to derive the invariance property of  unobservable subspaces (for details refer to Section \ref{Sec:UnObservable}).

Consider the fault free system \eqref{Eq:FMII}, that is with , and with zero input (we are mainly interested in the unobservable subspaces and do not need to be concerned with the control inputs in the FDI problem). By considering  , it can be shown that under the above conditions the system \eqref{Eq:FMII} can be represented as,

where , , and  is an Inf-D matrix with  and  as diagonal and upper diagonal blocks, respectively, with the remaining elements set to zero, and . In other words, we have,



Note that since we invoke an Inf-D representation to investigate an unobservable subspace, and where this subspace is defined by only  and , therefore for sake of  presentation simplicity,  an Inf-D system is used that has no fault and zero input. 


There are various formulations for the initial conditions  of the FMII model \eqref{Eq:FMII}. These are based on the separation set that is introduced in \cite{FM_SeparationSet}. There are two separation sets that are commonly used in the literature.  In the first formulation
the initial conditions are denoted by  \cite{FMinBook} (this is compatible with the model \eqref{Eq:IDRep}). The second formulation is expressed as  and , where  and  \cite{Kaczorek_Book}.
The second formulation is more compatible with applications (particularly, in case that the system \eqref{Eq:FMII} is an approximate model of a PDE system - refer to Section \ref{Sec:Simulation}). It will be shown subsequently that  since we derive the conditions based on invariant unobservable subspace (this is formally defined in the next section), our proposed methodology is applicable to \underline{both} initial condition formulations. In other words, we use the Inf-D  representation  to only show the results and evaluate the developed algorithms. However, to apply our results there is no need to deal with Inf-D systems and subspaces, and therefore, one can apply our proposed methods to 2D systems corresponding to both initial condition formulations. 

\myhl{We start with the first formulation of the initial} conditions subject to the boundedness assumption (this is, ). However, as shown in Section \ref{Sec:A12Inv}, our proposed results also hold for  the second initial condition formulation.

As stated in the Notation section, it can be  shown that  defined for equation \eqref{Eq:IDRep} is an Inf-D Banach space. The system theory corresponding to Inf-D systems is more significantly challenging than Fin-D system theory (1D systems) (refer to \cite{Zwart_Book}). However, as shown subsequently, the operator  is bounded and consequently, one can readily extend the result of 1D systems to the system \eqref{Eq:IDRep} \cite{Curtain_Book,Zwart_Book}. Let us first define the notion of bounded operators.
\begin{definition}\label{{Def:BoundedOp}}\cite{Curtain_Book}
	Consider the operator , where  and  are Banach vector spaces  with the norms  and , respectively. The operator  is bounded if there exists a real number  such that  for all .
\end{definition}
\begin{lemma}\label{Lem:Boundedness}
	The operator  as defined in the Inf-D system \eqref{Eq:IDRep} is bounded.
\end{lemma}
\begin{proof}
	Let , where  denotes the norm of  and . It follows readily that .
	Therefore, . This completes the proof of the lemma.
\end{proof}
The above lemma enables one to now formulate the unobservable subspace of the 2D system \eqref{Eq:FMII} in a geometric framework (for details refer to Section \ref{Sec:InvSpace}) based on the operator  (and consequently, in terms of  and ).


\begin{remark}
	 Although, in \cite{conte1988GeometryArticle} all the results such as the controlled invariant subspaces are presented on , the developed approach in \cite{conte1988GeometryArticle} has its roots in the theory of systems over rings. In this paper, we propose an alternative approach that is based on Inf-D systems that are  defined on a Banach vector space. Similar to \cite{conte1988GeometryArticle}, our proposed methodology including the algorithms and the conditions for solvability of the FDI problem can also be addressed in a Fin-D scheme. However, as shown in the literature the duality property does not hold for 2D systems \cite{FMMinimalRealization,FMinBook}. Therefore, by simply invoking duality the results of this paper cannot be derived from those in \cite{conte1988GeometryArticle}.
\end{remark}

\subsection{The FDI  Problem of 2D FMII Model}\label{Sec:FDIProb}
In this subsection,  we formulate the FDI problem for the 2D system \eqref{Eq:FMII}. In this paper, without loss of any generality, it is assumed that the system \eqref{Eq:FMII} is subject to two faults, and therefore we construct two residuals such that each one is sensitive to only one fault and is decoupled from the other.

More precisely, consider the faulty FMII model \eqref{Eq:FMII}. The solution to the FDI problem of the 2D FMII system can be stated as that of generating two residuals  such that,

		&\forall u , f_2 \  \mathrm{and} \ f_1=0 \;\;\; \mathrm{then} \  r_1\rightarrow 0 \nonumber, \\ &\mathrm{and}\;\mathrm{if} \ \ f_1\neq 0 \;\;\; \mathrm{then} \ r_1\neq  0  \label{FDI_InObs1},\\
		&\forall u,  f_1  \  \mathrm{and} \ f_2=0 \;\;\; \mathrm{then} \  r_2\rightarrow 0 \nonumber, \\ &\mathrm{and}\;\mathrm{if} \ \ f_2\neq 0 \;\;\; \mathrm{then} \ r_2\neq  0 \label{FDI_InObs2}.
	


The above residuals are to be constructed by employing fault detection filters. For the 2D system
\eqref{Eq:FMII}, we consider the following FMII-based {\it fault detection filter},
\bs

\es
where  denotes the state of the filter and is used to define the residual signal . The solution to the FDI problem is now reduced to that of selecting the filter gains , , , , , ,  and  corresponding to the filter \eqref{Eq:Filter}.
\begin{remark}\label{Rem:GeneralFilter}
	The detection filter \eqref{Eq:Filter} can be selected as a full-order () or as a partial-order () 2D Luenberger observer.
As shown subsequently in Section \ref{Sec:FDI}, this level of generality allows one to analytically compare our proposed methods with those results reported in \cite{Malek_3DFDI}.\qed
\end{remark}

\begin{remark}\label{Rem:ProbName}
	In this paper, we investigate the FDI problem by employing two main steps, namely (i) decoupling the faults, and (ii) designing  filter gains for each fault. The first step for decoupling   addresses  the existence of three maps ,  and , such that the fault  signatures  and   are members of the unobservable subspace (defined in the next section) of the system (, , ). The same terminology is used to decouple . Moreover, the second step is mainly concerned with existence of the filter \eqref{Eq:Filter} such that stability of the error dynamics is guaranteed. In this paper, if the first step is solvable for the fault  we say that  is detectable and isolable.  Finally, it is stated that there is a solution to the FDI problem if for all the fault signals   \underline{both} steps above are solvable.
\end{remark}

\subsection{Deadbeat Observers}\label{Sec:DeabeatObs}
In Section \ref{Sec:FDI},  necessary and sufficient conditions for solvability of the FDI problem are derived. We provide  sufficient conditions for accomplishing the FDI task by using a delayed deadbeat detection filter and an ordinary (i.e., without a delay) deadbeat observer (refer to the subsequent Corollaries \ref{Col:FDI_DelayedDeadBeat_Suff} and \ref{Col:FDI_DeadBeat_Suff}). Towards these end, in this section we formally define a (delayed) deadbeat filter. For a comprehensive discussion on 2D deadbeat observers refer to \cite{Bisiacco_Obs,BisiaccoLetter}.

Consider the system \eqref{Eq:FMII} under the fault free situation. A (delayed) deadbeat observer is constructed according to,
\bs

\es
where  and  are the input and output signals as defined in the system \eqref{Eq:FMII}. Note that since  the output is assumed to not be directly affected by the input signal,  is only a linear combination of  and .  If there exists a number  such that  for all , the filter \eqref{Eq:DeadbeatObs} is designated as an \underline{ordinary (without a delay) deadbeat observer}. On the other hand, if there exist \underline{non-negative} integers  and  such that  and , the observer \eqref{Eq:DeadbeatObs} is  designated as a \underline{{\it delayed} deadbeat observer}. The necessary and sufficient conditions for existence of a (delayed) deadbeat observer are specified in the following theorem \cite{Bisiacco_Obs, BisiaccoLetter}.
\begin{theorem}\label{Thm:DeadbeatObsExistance}
	Consider the 2D system \eqref{Eq:FMII} under a fault free situation and the following 2D Popov-Belevitch-Hautus (PBH)  matrix,
	
	where ,  and  are defined as in equation \eqref{Eq:FMII}, and . Then,
	\begin{enumerate}
		\renewcommand{\labelenumi}{(\roman{enumi})}
		\item there exists a delayed  deadbeat observer if and only if  for all  (that is,  is right monomic) \cite{BisiaccoLetter} (Section 3) and
		\item there exists an ordinary deadbeat observer if and only if  for all  (that is,  is right zero prime) \cite{Bisiacco_Obs}.\qed
	\end{enumerate}
\end{theorem}
We use the above theorem in Section \ref{Sec:FDI} to show the existence of deadbeat filters to derive sufficient conditions for solvability of the FDI problem (the subsequent Corollaries \ref{Col:FDI_DelayedDeadBeat_Suff} and \ref{Col:FDI_DeadBeat_Suff}).
\subsection{LMI-based Observer (Detection Filter) Design}\label{Sec:Observer}
As shown in \cite{Bisiacco_Obs}, design of a deadbeat observer \eqref{Eq:DeadbeatObs} requires that one works with polynomial matrices (this is not always a straightforward process). In this subsection, we address the design process for the FMII system observer, or the detection filter gains, by using linear matrix inequalities (LMI). These results will be used to explicitly design a 2D Luenberger detection filter (that can also be formulated as in equation \eqref{Eq:Filter}) subsequently in Section \ref{Sec:FDI} for the  purpose of accomplishing the solution to the FDI problem. 

In order to show the asymptotic stability of the state estimation error dynamics, one needs to apply the following stability lemmas.

\begin{lemma}\label{Lem:2DLyapunov}\cite{2DLyapunov}
	The 2D FMII system \eqref{Eq:FMII} (under the fault free situation) is asymptotically stable if there exist two symmetric positive definite matrices  such that,
	
	where  and .\qed
\end{lemma}
\begin{lemma}\label{Lem:2DLMIBasic}\cite{LMI1D}
	Consider the LMI condition ,
	where ,  and . There exists a matrix  satisfying the previous LMI condition if and only if
	 and , where the columns of  and  are bases of the  and , respectively.\qed
\end{lemma}


Now consider the 2D system \eqref{Eq:FMII} under the fault free situation and the corresponding  state estimation observer as given by,
\bs

\es
It follows readily that the state estimation error dynamics, as defined by , is governed  by,

The following theorem and corollary provide an LMI-based condition for existence of the state estimation observer gains  and  such that the error dynamics \eqref{Eq:ObserverError} is asymptotically stable.
\begin{theorem}\label{Thm:ObserverGain2LMIs}
	Consider the 2D system \eqref{Eq:FMII} under the fault free situation.  There exist two maps  and  two symmetric positive definite matrices  and  such that the LMI \eqref{Eq:2DLyapunov} is satisfied for  and  if and only if  and  satisfy the LMI condition , where  and the columns of  are the basis of .
\end{theorem}
\begin{proof}
	Note that  without loss of any generality, it is assumed that  is full row rank, and  that is equivalent to partial state measurement. Let . By using the Schur complement lemma, we have  if and only if,
	
	It follows that
	 if and only if  (or  and ). By defining  and , where , and using Lemma \ref{Lem:2DLMIBasic} the LMI condition \eqref{Eq:ConLMI2DTemp} is satisfied if and only if there exits a matrix  such that,
	\bs
	
	\es
	where . Again, by using the Schur complement lemma, we have
	. This completes the proof of the theorem.
\end{proof}
An important corollary to the above theorem and Lemma \ref{Lem:2DLyapunov} can be stated as follows.
\begin{corollary}\label{Col:ObserverDesignLMI}
	Consider the 2D system \eqref{Eq:FMII} under the fault free situation and the state estimation observer \eqref{Eq:GeneralObserver}. If there are two symmetric positive definite matrices  and  satisfying the LMI condition , then there exists two maps  and  such that the error dynamics \eqref{Eq:ObserverError} is asymptotically stable.
\end{corollary}
\begin{proof}
	Follow directly from Theorem \ref{Thm:ObserverGain2LMIs} and Lemma \ref{Lem:2DLyapunov}, and the details are omitted for sake of brevity.
\end{proof}
\begin{remark}\label{Rem:ObserverLMIConstructive}
	Note that by solving the LMI condition , one can obtain symmetric  positive definite matrices  and . Hence, the state estimation observer gains  and  are computed by solving the equation \eqref{Eq:ObserverDesignLMI_Gain} (which is an LMI condition in terms of the gains  and ). Therefore, Corollary \ref{Col:ObserverDesignLMI} not only provides sufficient conditions for existence of a state estimation observer, but also provides an approach for computing the observer gains  and . \qed
\end{remark}
The results of this section will now be used subsequently in Sections \ref{Sec:UnObservable} and \ref{Sec:FDI} to address the unobservable subspace of the system \eqref{Eq:FMII} as well as to provide sufficient conditions for  solvability of the FDI problem, respectively.


\section{Invariant Subspaces for 2D FMII Models}\label{Sec:InvSpace}
As described earlier, 2D systems can be represented as \infd systems (i.e. the initial condition is a vector of an Inf-D subspace). In this section, we first use the Inf-D representation \eqref{Eq:IDRep} to formally define and construct an unobservable subspace. Next, we define a subspace of  the unobservable subspace  (this we called as an invariant unobservable subspace) of the 2D system \eqref{Eq:FMII} that can be represented as an infinite sum of the same finite dimensional subspaces. Therefore, one can compute the invariant unobservable subspace (that is, the \infd subspace) in a \underline{finite} number of steps. Also, it is shown that the invariant unobservable subspace enjoys an important geometric property that is crucial for solving the FDI problem.



\subsection{Unobservable Subspace}\label{Sec:UnObservable}




As described in the previous section, 2D systems can be represented as Inf-D systems. In this subsection, the Inf-D representation \eqref{Eq:IDRep} is utilized to formally define and construct an unobservable subspace.




The unobservable subspace of the system \eqref{Eq:IDRep} (and consequently of the system \eqref{Eq:FMII}) is defined as,

where  and  are defined as in equation \eqref{Eq:IDRep}. Note that we define the above unobservable subspace by following along the steps in \cite{Curtain_1986}, the results in \cite{Zwart_Book} (Chapter I), and the fact that the operator  in equation \eqref{Eq:IDRep} is bounded (refer to Lemma \ref{Lem:Boundedness}).

One of the main difficulties in geometric analysis of Inf-D systems is the convergence of any developed algorithm that involves computation of certain set of subspaces in a \underline{finite} number of steps. For example, consider the unobservable subspace \eqref{Eq:UnobserSpaceGeneral}. In Fin-D systems, the algorithm for computing the unobservable subspace converges in a finite number of steps \cite{Wonham_Book}. Moreover, one is generally interested in investigating the FMII models in a Fin-D representation \eqref{Eq:FMII}.  Motivated by the above, below two important subspaces of  that are denoted by  and  are introduced. The subspaces  and   can be computed in a \underline{finite} number of steps and also allows one to derive necessary and sufficient conditions for solvability of the FDI problem.

Consider the initial condition  and , where . One can show that the  state solution of the model \eqref{Eq:FMII} under the fault free situation is given by \cite{FMinBook},
\vspace{-1mm}

where the matrices 's and 's are defined by the following recursive  expressions,
\vspace{-2mm}

Based on the solution that is given by equation  \eqref{Eq:FMIISol}, and considering that , a finite observability matrix (given that its null space is a finite dimensional subspace) can be defined as follows,

Let . Since , we designate  as the \emph{finite} unobservable subspace of the system \eqref{Eq:FMII}.  Also, recall from the 2D Cayley-Hamilton theorem \cite{FMinBook} that for all , one sets , where 's are real numbers. Therefore, for all , , and consequently  can be computed in a \underline{finite} number of steps as,


Now, we consider the following subspace,

It follows that if , then  for all , and given the zero input assumption one gets  for all  (in equation \eqref{Eq:IDRep}). By considering , where  is defined as in equation \eqref{Eq:IDRep} and , it can be shown that . Also, note that although  is an \infd subspace, it can be computed in a \underline{finite} number of steps (one only needs to compute ). However, as explained in \cite{ACC2013,ACC2014} the invariance property (this is addressed in the next subsection) of  is not lucid. Therefore, in the following a subspace of  is introduced such that it enjoys this geometric property. To define the subspace  one needs the following notation.

Let us express  to denote the sequence of multiplications of  and ,  where  is a multi-index parameter that specifies the sequence of the multiplication.  For example, consider , where we have .
The notation  denotes the number of all  and  that are involved in the corresponding multiplication (for the above example, we have ). Now, consider the following subspace (for more details on  refer to \cite{ACC2013}),


The following lemma shows that the subspace that is used in \cite{ntogramatzidis2012Siam, Malek_3DFDI, Malek_3DFDIConf} as the unobservable (non-observable) subspace is indeed .


\begin{lemma}\label{Lm:NsisSub}
	The subspace  can be computed in a finite number of steps according to the following algorithm,
	
\end{lemma}
\begin{proof}
	First, note that  and . In other words, . Note that for every pair of operators  and , one can show that . Therefore, it follows that . This completes the proof of the lemma.
\end{proof}

Now, we set . Note that although , one can compute it in a finite number of step (by computing ).

\subsection{-Invariant Subspaces}\label{Sec:A12Inv}
As stated in the Subsection \ref{Sec:IDRep}, the 2D system \eqref{Eq:FMII} can be represented as an \infd system \eqref{Eq:IDRep}. In order to formulate the corresponding Inf-D invariant subspaces one needs the next two definitions.
\begin{definition}\cite{Zwart_Book}\label{Def:AID_Inv}
	Consider the \infd system \eqref{Eq:IDRep}, where the operator  is bounded (according to Lemma \ref{Lem:Boundedness}). The closed subspace  is called -invariant if .\qed
\end{definition}

\begin{definition}\cite{conte1988GeometryConf}\label{Def:A12_Inv}
	The subspace   is said to be an {\it -invariant subspace} for the 2D system \eqref{Eq:FMII} if
,
where  and  are defined as in equation \eqref{Eq:FMII}.\qed
\end{definition}
Note that  is -invariant if and only if it is invariant with respect to  \underline{and}  (i.e.  and ).
The following theorem provides the connection between the Definitions \ref{Def:AID_Inv} and \ref{Def:A12_Inv}.
\begin{theorem}\label{Thm:AID_A12Inv}
	Consider the 2D system \eqref{Eq:FMII} and the \infd system \eqref{Eq:IDRep}. Let , where . The subspace  is -invariant if and only if  is -invariant.
\end{theorem}
\begin{proof}
	First, note that every  can be expressed as , where  and . Therefore, one only needs to show the result for .\\
	({\bf If part}): Assume  is -invariant. Consider the Inf-D vector . It follows that   . Since  is -invariant, it follows that . \\
	({\bf Only if part}): Let  and . Consequently, . Since  , it follows that  and , and consequently  is -invariant. This completes the proof of the theorem.
\end{proof}



Consider the subspaces  and . If  is the largest -invariant subspace that is contained in , we denote . We have shown in \cite{ACC2013} that , and it is the largest -invariant subspace that is contained in . Therefore,  one can write . Since  is -invariant, by Theorem \ref{Thm:AID_A12Inv},  is -invariant. Therefore, if  (that is,  for all ) and zero input,  for all  and  for all  (in equation \eqref{Eq:IDRep}). We designate   as the invariant unobservable subspace.
\begin{remark}\label{Rem:InitCond}
	As stated in Subsection \ref{Sec:IDRep}, there are two different types of initial condition formulations. In this paper, we use the first formulation that is compatible with the Inf-D system \eqref{Eq:IDRep}. Recall that the second formulation is expressed as  and , where .  Now, let  and . The -invariance property of  verifies that . In other words,  is also the invariant unobservable subspace of system \eqref{Eq:FMII} with the second formulation of the initial conditions. Therefore, without loss of any generality, one can apply our proposed approach to \underline{both} initial condition formulations as provided in Section \ref{Sec:Dis_2D}. Moreover,  is the largest -invariant in the form  that is contained in , where  is defined in \eqref{Eq:IDRep}.
\end{remark}


\subsection{Conditioned Invariant Subspaces}

Another important subspace in the geometric FDI toolbox is the conditioned invariant (i.e., the -invariant) subspace that is defined next. This definition is an extension of the one that has appeared and presented in \cite{conte1988GeometryConf} and \cite{conte1988GeometryArticle}.
\begin{definition}\label{Def:CondInv}
	The subspace  (where ) is said to be the conditioned invariant subspace for the 2D system \eqref{Eq:FMII} if there exist two output injection maps  such that  and
	. In other words,  is  -invariant (i.e., invariant with respect to  and ). We designate  as the finite conditioned invariant subspace (since ) of the 2D system \eqref{Eq:FMII}.\qed
\end{definition}

Similar to 1D systems, one can now state the following result.
\begin{lemma}\label{Lem:CA_Inv}
	The following statements are equivalents.
	\begin{enumerate}
		\renewcommand{\labelenumi}{(\roman{enumi})}
		\item The subspace  is conditioned invariant.
		\item  .
		\item .
	\end{enumerate}
	where .
\end{lemma}
\begin{proof}
	 and : By definition, there exists two maps  and  such that  is -invariant. By utilizing Theorem \ref{Thm:AID_A12Inv}  is -invariant, where
	
	By following along the same lines as in Lemma \ref{Lem:Boundedness}, one can show that  is bounded. Consequently, the result of 1D system is also valid for the Inf-D system \eqref{Eq:IDRep}. Hence, we have  (that shows ). By considering the structure of  and  it follows that .\\
	: Since  is bounded, the domain of  is equal to , and therefore, the result of 1D system is also valid for the Inf-D system \eqref{Eq:IDRep}. Therefore, there exists a bounded operator  such that  is -invariant. By considering the structure of  and , it is easy to show that one solution for  is given by
	
	Hence, by using Theorem \ref{Thm:AID_A12Inv}, the subspace  is -invariant and consequently  is a conditioned invariant subspace. This completes the proof of the lemma.
\end{proof}

In the geometric FDI approach, one is interested in conditioned invariant subspaces that are containing a given subspace \cite{Massoumnia1989}. Let us define all the conditioned invariant subspaces containing a subspace  () as  .  It can be shown that for a given subspace  (or ), the set  is closed under intersection, and hence the set  has a minimal member as . The minimal conditioned invariant subspace containing a given subspace  (that is, ) is obtained by invoking the following non-decreasing algorithm that is provided below,

and , where . 

Note that the above algorithm converges in a \underline{finite} number of steps. Also, let  be a finite conditioned invariant subspace.  The set of all maps  such that  is -invariant is designated by .
\subsection{Unobservability Subspace}
The unobservability subspace \cite{Mass_Thesis, DrMeskin_Delay} is the cornerstone of geometric FDI approach in 1D systems. The following definition generalizes and extends this concept to the FMII 2D models. The Fin-D notion of this definition, for the first time in the literature, was introduced and utilized in \cite{ACC2013} for the FDI problem of 2D Roesser systems. \begin{definition}\label{Def:Unobser}
	The subspace  is said to be an unobservability subspace for the 2D system \eqref{Eq:FMII} if there are three maps  such that
 and . We designate  as the finite unobservability subspace of the 2D system \eqref{Eq:FMII}.
\qed
\end{definition}
Note that  is also conditioned invariant subspace and a generic unobservable subspace of the system (, , ).
For accomplishing the gaol of the FDI task, one first computes an unobservability subspace and then obtains the matrix  \cite{Mass_Thesis}. Therefore, it is necessary to compute the unobservability subspace without having any knowledge of . By following along the same lines as those used in \cite{ACC2013} (Section IV.D), one can show that the limit of the following algorithm is the smallest unobservability subspace  (and consequently ) that contains a given subspace , that is,

where  is the minimal finite conditioned invariant subspace containing ,  and .

Finally, we set , where . Finally, it should be noted that since  (this follows by applying  in the algorithm \eqref{Eq:Unob_MainAlg}), one obtains .

In this section, we first defined the invariance property of the 2D system \eqref{Eq:FMII} that are Inf-D subspace of the \infd system \eqref{Eq:IDRep}. Next, an invariance unobservable subspace (that is generically equivalent to an unobservable subspace)  was introduced. Moreover, the conditioned and unobservability subspaces that are crucial in determining the solution to the FDI problem have been introduced. By utilizing the above results necessary and sufficient conditions for solvability of the FDI problem are subsequently derived and provided.

\section{Necessary and Sufficient Conditions for Solvability of the FDI problem}\label{Sec:FDI}
In this section, we first present  sufficient conditions for detectability and isolability of faults. Next, by employing three different filters (namely, ordinary, delayed deadbeat, and LMI-based filters) sufficient conditions for solvability of the FDI problem are presented.

Consider the faulty FMII model \eqref{Eq:FMII} (i.e., the system is subjected to two faults  and ) and the detection filter \eqref{Eq:Filter} designed to detect and isolate the fault . By augmenting the detection filter dynamics \eqref{Eq:Filter} with the faulty 2D model \eqref{Eq:FMII}, one obtains,
\bs

\es
where  ( refers to the dimension of ),  and,
\bs

\es
In this section, by assuming that the unobservable subspace of the above augmented system is -invariant, it is shown that the sufficient condition is also necessary for solvability of the FDI problem. Moreover, an analytical  comparison between our proposed approach and the method developed in \cite{Malek_3DFDI} is also provided to highlight the strength and capabilities of our proposed methodology when compared to the literature.
\subsection{Main Results}\label{Sec:FDIMainResult}
The following lemma provides an important property for the invariant unobservable subspace  (and ) that is associated with the system \eqref{Eq:AugSys}.
\begin{lemma}\label{Lem:UnobsTot2Small}
	Consider the 2D system \eqref{Eq:AugSys} and its invariant unobservable subspace . Then  is an unobservability subspace of the 2D system \eqref{Eq:FMII}, with  and  representing the embedding operator into  (i.e.,  and ).
\end{lemma}
\begin{proof}
	First, recall that .
	Note that, , and assume that . According to the fact that  is -invariant \cite{ACC2013}, we have , and if  then . It follows that . By following along the same lines as those above one can show that . Therefore,  is a conditioned invariant subspace.
	Moreover, given that  is contained in , we have . Therefore, the subspace  is a finite conditioned invariant subspace contained in . Since  is the largest -invariant subspace in , it follows that  is the largest  invariant subspace in  (i.e.,  is a finite unobservability subspace of the 2D system \eqref{Eq:FMII}). In other words,  is a finite unobservability subspace of the system \eqref{Eq:FMII}. This completes the proof of the lemma.
\end{proof}
The following theorem provides a \underline{single} necessary and  sufficient condition for detectability and isolability of faults (i.e., the existence  of a subsystem such that it is decoupled from all faults but one - refer to  Subsection \ref{Sec:FDIProb} for more details).
\begin{theorem}\label{Thm:NecSuffFDI}
	Consider the 2D system \eqref{Eq:FMII} that is subject to two faults  and . The fault   is detectable and isolable (in the sense of Remark \ref{Rem:ProbName}) if and only if the following condition is satisfied,
	
	where   is the smallest finite uobservability subspace (refer to Definition \ref{Def:Unobser}) of the 2D system \eqref{Eq:FMII} containing  (this represents the limit of the algorithm \eqref{Eq:Unob_MainAlg}, where one sets  in the algorithm \eqref{Eq:CIAlg}).\qed
\end{theorem}
\begin{proof}
	({\bf If part}): Let ,  and  be the corresponding operators to  (i.e. ). Also, consider the following detection filter, governed according to 
		
		where  is the canonical projection of  on , ,  where  and ,  are filter gains  and  is the unique solution to .
		Now, by defining , one obtains
		
		Note that  and by the condition \eqref{Eq:NecSufCon}, at least for  and/or , we have . Therefore, the residual signal  is decoupled from  and is sensitive to . In other words,  is detectable and isolable.\\
		{(\bf Only if part)}: We show the necessary part by contradiction. Let  be detectable and isolable, and . Hence, the residual signal  is decoupled from  and sensitive to . However, from Lemma \ref{Lem:UnobsTot2Small} one obtains . Consequently, the fault  is not detectable by using the residual  in \eqref{Eq:Res}, which is in contradiction with the solvability of the FDI problem. 
\end{proof}


	As stated in Remark \ref{Rem:ProbName}, the FDI problem has two main steps. Theorem \ref{Thm:NecSuffFDI} provides the necessary and sufficient condition for the first step (that is, detectability and isolability of ). Therefore,  condition \eqref{Eq:NecSufCon} is also necessary for solvability of the FDI problem. For the second step, we need to determine , and consequently ,  in the filter \eqref{Eq:Filter_F1} such that the residual dynamics \eqref{Eq:Res} is asymptotically stable. 
	In what follows, we provide the conditions that are based on three different observers (namely, the ordinary, the delayed deadbeat, and the LMI-based filters). We first need to present the following theorem. However, one first needs the following notation. Consider the PBH matrix \eqref{Eq:PBHMatrix}. The matrix   is the minimal left annihilator of the PBH matrix if ,  is full row-rank, and any other left annihilator  can be expressed as , where  is a polynomial matrix of  and .
\begin{theorem}\cite{BisiaccoLetter}\label{Thm:BisiacoLetter}
	Consider the 2D system \eqref{Eq:FMII}. The FDI problem is solvable (by using the approach that is proposed in \cite{BisiaccoLetter}) if and only if,
	
	where  is the number of faults,  denotes the minimal left annihilator of the PBH matrix  \eqref{Eq:PBHMatrix}, and  for .
\end{theorem}

The following corollary provides the sufficient conditions for solvability of the FDI problem by utilizing  our proposed method as well as a (delayed) deadbeat filter.
\begin{corollary}\label{Col:FDI_DelayedDeadBeat_Suff}
	Consider the 2D system \eqref{Eq:FMII} that is subject to two faults  and . Provided condition \eqref{Eq:NecSufCon} is satisfied and  condition of Theorem \ref{Thm:BisiacoLetter} is satisfied for the quotient subsystem (,,), where  and  are defined in \eqref{Eq:Filter_F1}, then  the FDI problem is solvable. \end{corollary}
\begin{proof}
	 By invoking Theorem \ref{Thm:NecSuffFDI},  the residual signal  in \eqref{Eq:Filter_F1} is decoupled from all faults but . Now, given that the condition \eqref{Eq:BisiaccoLetterCon} is satisfied for the subsystem (,,), by determining the observer gains  and , one can  design a delayed deadbeat observer such that  (refer to \cite[Theorem 1]{BisiaccoLetter}). Hence, the FDI problem is solvable. This complete the proof of the corollary.
\end{proof}

The following corollary provides a sufficient condition for solvability of the FDI problem by utilizing  an {\it ordinary (without delay) deadbeat} detection observer.

\begin{corollary}\label{Col:FDI_DeadBeat_Suff}
	Consider the 2D system \eqref{Eq:FMII} that is subject to two faults  and . Let the condition \eqref{Eq:NecSufCon} be satisfied and the PBH matrix of the quotient subsystem (,,) be right zero prime, where  and  are defined in \eqref{Eq:Filter_F1}. Consequently, the FDI problem is solvable by using a deadbeat (without a delay) observer. \end{corollary}
\begin{proof}
	 By invoking Theorem \ref{Thm:NecSuffFDI},  the residual signal in \eqref{Eq:Res} is only affected by . Therefore, the detection and isolation of  is reduced to that of designing  and  gains.  Since the PBH matrix of the subsystem (,,) is right zero prime, by invoking Theorem \ref{Thm:DeadbeatObsExistance} one can determine the gains such that the error dynamics \eqref{Eq:Res} is asymptotically stable and the FDI problem is solvable. This complete the proof of the corollary.
\end{proof}

As shown in Theorem \ref{Thm:NecSuffFDI}, under certain conditions one can obtain a residual signal that is decoupled from all faults but one. Therefore, design of a residual generator to detect and isolate the fault  in the 2D system \eqref{Eq:FMII} is reduced to that of detecting this fault in the system \eqref{Eq:Filter_F1} by using an observer. The Corollaries \ref{Col:FDI_DelayedDeadBeat_Suff} and \ref{Col:FDI_DeadBeat_Suff} provide sufficient conditions for solvability of the FDI problem by using \emph{delayed} and {\it ordinary deadbeat filters}, respectively. However, as pointed out in \cite{Bisiacco_Obs},  design of an observer for FMII models is based on polynomial matrices. This method is unfortunately not always numerically or analytically straightforward to develop and therefore, in this work we also develop another set of sufficient conditions for solvability of the FDI problem by using a 2D Luenberger observer.

The next corollary provides sufficient conditions for solvability of the FDI problem by using a 2D Luenberger observer.
\begin{corollary}\label{Col:FDI_LMI_Suff}
	Consider  the 2D model \eqref{Eq:FMII}, where the condition \eqref{Eq:NecSufCon} is satisfied. The FDI problem is solvable if there exist two symmetric  positive definite matrices  and  such that,
	A_2^p)^\tran\ebm (R_1+R_2)[A_1^p,A_2^p])W_m<0,
	
	\begin{split}
		x(i+1,j+1) &= \bbm 0 &1\\0 &0\ebm x(i,j+1) + \bbm 0 &0\\0 &1\ebm x(i+1,j)+ \bbm 1\\0\ebm f_1+ \bbm 0\\1\ebm f_2,\\
		y(i,j) &= \bbm 0 &1\ebm x(i,j).
	\end{split}
\label{Eq:ConterExm}
	\begin{split}
		A_1 &= \bbm \bsm 0 &0\\ 0 &0.5\esm &0.5I\\ 0 &0_{2\times2}\ebm,\; A_2 = 0.5\bbm 0_{2\times 2} &0\\ I &I\ebm,\;L_1^2=L_2^2=0,\; B_1=B_2=0\\
		L_1^1 &= [0,0,0,1]^\tran,\;L_2^1 = [0,0,-1,1]^\tran,\; C=\bbm 1 &0 &0 &0\\ 0 &0 &0 &1\ebm
	\end{split}

	\rank(\bbm I-z_1A_1-z_2A_2 &z_1[L^1_1,L^1_2]\\
	C	&0\ebm) = n\;,\;\;\forall z_1,z_2\in\fld{C}

	\left\{\begin{split}
		&a-0.5z_2c+e =0\\
		&-0.5z_1a+(1-0.5z_2)c = 0
	\end{split}\right.\;\;,\;\; \left\{\begin{split}
	&0.5z_1b+(1-0.5z_2)d + f =0\\
	&(1-0.5z_1)b-0.5z_2d = 0
\end{split}\right..

	\begin{split}
		\omega(i+1,j+1)&=A_1^p\omega(i,j+1)+ A_2^p\omega(i+1,j),\\
		r_1(i,j)&=M_1\omega(i,j)-H_1y(i,j).
	\end{split}
\label{Eq:HEfaulty}
\begin{split}
\frac{\partial T_f}{\partial t} &= -\alpha_f \frac{\partial T_f}{\partial z} - \beta(T_f - T_g) -  f_2(z,t),\\
\frac{\partial T_{g}}{\partial t} &= -\alpha_{g} \frac{\partial T_{g}}{\partial z} - \beta(T_{g} - T_f) +  f_1(z,t) +f_2(z,t),
\end{split}
\label{Eq:GeneralPDE}
\pardiff{\tilde{x}}{t} = \tilde{A}_1\pardiff{\tilde{x}}{z} + \tilde{A}_2\tilde{x} + \tilde{B} {u} + \sum_{k=1}^p \tilde{L}_k\tilde{f}_k,
\label{Eq:GeneralPDEApr}
\begin{split}
\frac{\tilde{x}(i\Delta z, (j+1)\Delta t)-\tilde{x}(i\Delta z, j\Delta t)}{\Delta t} = &\tilde{A}_1 \frac{\tilde{x}(i\Delta z, j\Delta t)-\tilde{x}((i-1)\Delta z, j\Delta t)}{\Delta z} + \tilde{A}_2 \tilde{x}(i\Delta z, j\Delta t) \\&+ \tilde{B} \tilde{u}(i\Delta z, j\Delta t)+\sum_{k=1}^p \tilde{L}_k\tilde{f}_k(i\Delta z, j\Delta t).
\end{split}
\label{Eq:GeneralFMIIApr}
\begin{split}
x(i+1,&j+1) =  A_1x(i,j+1) + A_2x(i+1,j) + B_1 u(i+1,j) +\sum_{k=1}^p {L}_{k}{f}_k(i+1,j),
\end{split}

\begin{split}
A_2= \bbm 0 &0\\ -\frac{\Delta t}{\Delta z} \tilde{A}_1  &(I+\frac{\Delta t}{\Delta z} \tilde{A}_1 + \Delta t \tilde{A}_2)\ebm,\;
B_1 = \bbm 0\\ \tilde{B}\ebm, \; L_k=\bbm 0\\ \tilde{L}_k\ebm, \; f(i,j)=\tilde{f}(i+1,j).
\end{split}
\label{Eq:FMIISimComplete}
\begin{split}
x(i+1,j+1) = &\bbm 0_{2\times2}  &I_{2\times2}\\ 0_{2\times2} &0_{2\times2}\ebm x(i,j+1) +
\bbm 0_{2\times2} &0_{2\times2}\\
I_{2\times2} &\bsm -0.1 &0.1\\0.1 &-0.1\esm
\ebm x(i+1,j)+ L_{1}^1 f_1(i+1,j)\\&+ B_2^1 u(i+1,j) + L_{2}^1 f_2(i+1,j),\\
y(i,j) = &Cx(i,j),
\end{split}
\label{Eq:SimQuSubSys}
\begin{split}
\omega(i+1,j+1) &= \left[ \bsm 0 &0 &-1.42\\ 0 &0 &0\\0 &0 &0\esm\right] \omega(i,j+1) + \left[ \bsm 0 &0 &0\\0 &0 &0\\ -0.7 &-0.7 &0\esm\right] \omega(i+1,j) + \bbm \bsm 0 &0\\0 &0\\0.7 &0.7\esm \ebm u(i+1,j) + \bbm\bsm 0\\1\\0\esm\ebm f_1(i+1,j),\\
y_p(i,j) &= [1,0,0]\omega(i,j),
\end{split}
\label{Eq:SimQuSubSys2}
\begin{split}
\omega(i+1,j+1) = &\left[ \bsm 0 &0 &-1.42\\ 0 &0 &0\\0 &0 &0\esm\right] \omega(i,j+1) + \left[ \bsm 0 &0 &0\\0 &0 &0\\ -0.7 &-0.7 &0\esm\right] \omega(i+1,j) + \bbm \bsm 0 &0\\0 &0\\0.7 &0.7\esm \ebm u(i+1,j) \\ &+ D_{o2}P_2y(i,j+1),\\
r_1(i,j) &= [1,0,0]\omega(i,j)-Hy(i,j).
\end{split}

th_k = \begin{aligned}
& \underset{\ell<N_0}{\text{Max}}
& & |r_k^\ell(i,j)|
\end{aligned}, \;\; \mathrm{for}\; i,j\leq N_1
\label{Eq:FDILogic}
\begin{split}
\mathrm{if}\;r_1 >th_1 \; \Rightarrow \; \mathrm{the\;fault}\; f_1\;\mathrm{has\;occurred}.\\
\mathrm{if}\;r_2 >th_2 \; \Rightarrow \; \mathrm{the\;fault}\; f_2\;\mathrm{has\;occurred}.\\
\end{split}
\label{Eq:1DApproximation}
\begin{split}
\dot{x} &= Ax+Bu+\sum_{k=1}^N (L_1^k f_1^k+L_2^kf_2^k),\\
y &= Cx,
\end{split}
\label{Eq:1DApproximation_par}
\begin{split}
A = \bbm A_1 &0 &0  &\cdots\\
A_2 &A_1 &0  &\cdots\\
0 &A_2 &A_1  &\cdots\\
0 &0 &\ddots &\ddots &\ddots\ebm, C = \bbm 1 &0 &0 & 0&\cdots\\
0 &0 &1 &0 &\cdots\\
0  &0 &0 &0 &\ddots\ebm
\end{split}
\label{Eq:FM3D}
\begin{split}
&x(i+1,j+1,k+1)= A_1x(i,j+1,k+1)+ A_2x(i+1,j,k+1)\\&+A_3x(i+1,j+1,k)+ B_1u(i,j+1,k+1)\\&+ B_2u(i+1,j,k+1)+B_3u(i+1,j+1,k),\\
&y(i,j,k) = Cx(i,j,k).
\end{split}
\label{Eq:IDRep3D}
\begin{split}
{\bf x}(k+1) &= \op{A}{\bf x}(k),\;\;\;\; k\in\underline{\fld{N}} \\
{\bf y}(k) &=\op{C}{\bf x}(k),
\end{split}

\op{A} = \bbm \ &\ddots &\ddots &\cdots & & &\cdots\\
\cdots &0 &A_1 &A_2 &A_3 &0 &\cdots\\
\cdots &0 &0 &A_1 &A_2 &A_3 &\cdots\\
\cdots & &\cdots & & &\ddots &\ddots \ebm , 
\op{C} = \bbm \ddots & &\cdots & &\cdots\\
\cdots &0 &C &0 &\cdots\\
\cdots &0 &0 &C &\cdots \\
\cdots & &\cdots & &\ddots\ebm												

\begin{split}
A^{(i,j,k)} &= A_1 A^{(i-1,j,k)}+A_2 A^{(i,j-1,k)}+ A_3A^{(i,j,k-1)},\\
A^{(0,0,0)} &= I\;\;\; A^{(i,j,k)}=0\;\mathrm{if} \; i \; \mathrm{or} \; j\; \mathrm{or}\; k < 0.
\end{split}

In view of the above, one can now extend the invariant unobservable subspace  to
,
where,  is a multi-index parameter, and  and  are defined according to the same manner that were defined for 2D systems (refer to the notation in the Introduction section). For instance, if , then  and . Also, we have .

By following along the same lines as in Theorem \ref{Thm:AID_A12Inv} and Lemma \ref{Lem:CA_Inv}, we have the following corollaries.
\begin{corollary}\label{Col:AID_A123}
	Consider the 3D system \eqref{Eq:FM3D} and the Inf-D system \eqref{Eq:IDRep3D}. Let , where . The subspace  is -invariant if and only if  is -invariant.
\end{corollary}
\begin{corollary}
	Consider the 3D system \eqref{Eq:FM3D} and the Inf-D system \eqref{Eq:IDRep3D}. The following statements are equivalents.
	\begin{enumerate}
		\renewcommand{\labelenumi}{(\roman{enumi})}
		\item The subspace  is conditioned invariant.
		\item  .
		\item .
	\end{enumerate} 
	where .
\end{corollary}

By following along the same steps as above, the invariant subspace and the results provided in Sections \ref{Sec:InvSpace} and \ref{Sec:FDI} can be extended to  nD systems. Therefore, one can generalize the results of this paper to nD systems without major barriers and challenges.

\bibliographystyle{IEEEtr}
\bibliography{2DJournalRef}

\end{document}
