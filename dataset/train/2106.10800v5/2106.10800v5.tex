\documentclass[final]{article}

\usepackage[nonatbib,final]{neurips_2021}

\usepackage[utf8]{inputenc} \usepackage[T1]{fontenc}    \usepackage{hyperref}       \usepackage{url}            \usepackage{booktabs}       \usepackage{amsfonts}       \usepackage{nicefrac}       \usepackage{microtype}      \usepackage{xcolor}         

\usepackage{preamble}


\title{Lossy Compression for Lossless Prediction }

\author{Yann Dubois\\
  Vector Institute \\
  \texttt{yanndubois@cs.toronto.edu} \\
  \And  
  Benjamin Bloem-Reddy\\
  The University of British Columbia \\
  \texttt{benbr@stat.ubc.ca} \\
  \AND
  Karen Ullrich \\
  Facebook AI Research \\
  \texttt{karenu@fb.com} \\
  \And
  Chris J.~Maddison \\
  University of Toronto, Vector Institute \\
  \texttt{cmaddis@cs.toronto.edu} \\
}

\begin{document}
\doparttoc \faketableofcontents 

\maketitle

\begin{abstract}
Most data is automatically collected and only ever ``seen'' by algorithms.
Yet, data compressors preserve perceptual fidelity rather than just the information needed by algorithms performing downstream tasks.
In this paper, we characterize the bit-rate required to ensure high performance on all predictive tasks that are invariant under a set of transformations, such as data augmentations.
Based on our theory, we design unsupervised objectives for training neural compressors.
Using these objectives, we train a generic image compressor that achieves substantial rate savings (more than  on ImageNet) compared to JPEG on 8 datasets, without decreasing downstream classification performance.
\end{abstract}

 
\section{Introduction}
\label{sec:introduction}







Progress in important areas requires processing huge amounts of data. 
For climate prediction, models are still data-limited \cite{rolnick_tackling_2019}, despite the Natl. Center for Computational Sciences storing 32 million gigabytes (GB) of climate data \cite{zgurovsky_big_2020}. For autonomous driving, capturing a realistic range of rare events with current methods requires around 3 trillion GB of data.\footnote{\citet{kalra_driving_2016} estimated that autonomous vehicles would have to drive hundreds of billions of miles to demonstrate reliability in rare events. At 1 TB / hour \citep{yeong_sensor_2021} and 30 miles / hour, this is  GB.}
At these scales, data are only processed by task-specific algorithms, and storing data in human-readable formats can be prohibitive.
We need compressors that retain only the information needed for algorithmic execution of downstream tasks.


\begin{wrapfigure}{r}{0.45\textwidth}
\vspace{-\baselineskip}
\centering
\captionsetup[subfigure]{labelformat=empty}

\begin{minipage}{0.45\textwidth}
\centering
\begin{subfigure}[h]{0.32\columnwidth}
 \centering
 \includegraphics[width=\textwidth]{figures/augmnist_plus/source.png}
\caption{Source}
\end{subfigure}
\begin{subfigure}[h]{0.32\columnwidth}
 \centering
 \includegraphics[width=\textwidth]{figures/augmnist_plus/vae.png}
\caption{Standard rec.}
\end{subfigure}
\begin{subfigure}[h]{0.32\columnwidth}
 \centering
 \includegraphics[width=\textwidth]{figures/augmnist_plus/ivae.png}
\caption{Our rec.}
\end{subfigure}
 \vspace*{-0.5em}
\addtocounter{figure}{-1} \captionof{figure}{
Our unsupervised coder improves compression by only keeping information necessary for typical tasks.
(left) source augmented MNIST digit; (center) a neural perceptual compressor achieves 130 bit-rate; (right)
our invariant compressor achieves 48 bit-rate.
 \label{fig:mnist_intro}
}
\end{minipage}\vspace{-2\baselineskip}

\end{wrapfigure}
 
Existing lossy compressors are not up to the challenge, because they aim to reconstruct the data for human perception \cite{johnston_transform_1988,heeger_model_1995,lee_perceptual_2012,blau_rethinking_2019,pan_digital_1993,mentzer_high-fidelity_2020}. 
However, much of perceptual information is not needed to perform the tasks that we care about.
Consider classifying images, which can require about 1 MB to store. 
Classification is typically invariant under small image transformations, such as rescalings or rotations, and could instead be performed using a representation that discards such information (see \cref{fig:mnist_intro}). 
The amount of unnecessary perceptual information is likely substantial, as illustrated by the fact that typical image classification can be performed using a detailed caption, which requires only about 1 kB to store ( fewer bits).





Our goal is to quantify the bit-rate needed to ensure high performance on a collection of prediction tasks.
In the simple case of a \emph{single} supervised task, the minimum bit-rate is achieved by compressing predicted labels, and essentially corresponds to the Information Bottleneck (IB; \cite{tishby_information_2000}).
Our challenge, instead, is to ensure good performance on \emph{any} future tasks of interest, which will rarely be completely known at compression time, or might be too large to enumerate.


We overcome this challenge by focusing on sets of tasks that are \textit{invariant} under user-defined transformations (\eg, translation, brightness, cropping), as is the case for many tasks of interest to humans \cite{heaton_ian_2018,shorten_survey_2019}.
This structure allows us to characterize a worst-case invariant task, which bounds the relative predictive performance on all invariant tasks.
As a result, the bit-rate required to perform well on \textit{all} invariant tasks is exactly the rate to compress the worst-case labels.
At a high level, the worst-case task is to recognize which examples are transformed versions of one another, and rate savings come from discarding information from those transformations. 

We also provide two unsupervised neural compressors to target the optimal rates.
One is similar to a variational autoencoder \cite{kingma_auto-encoding_2014} that reconstructs canonical examples (\cref{fig:mnist_intro}).
Our second is a simple modification of contrastive self-supervised learning (SSL; \cite{oord_representation_2019}), which allows us to convert pre-trained SSL models into powerful, generic compressors.
Our contributions are:
\begin{itemize}[noitemsep,leftmargin=*]
\item We formalize the notion of compression for downstream predictive tasks.
\item We characterize the bits needed for high performance on any task invariant to augmentations.
\item We provide unsupervised objectives to train compressors that approximate the optimal rates.
\item We show that our compressor outperforms JPEG by orders of magnitude on 8 datasets on which it was never trained (\ie, zero-shot).
E.g., on ImageNet \cite{deng_imagenet_2009}, it decreases the bit-rate by .
\end{itemize}

\cmnote{we need to hammer home the idea that all human goals are specified in language -> every thing we could possibly be itnerested in predicting can likely be compiled to a detailed caption -> massive compression gains}
\bbnote{For classification this might be true, but not for more sophisticated tasks. Climate prediction?}
 
\section{Rate-distortion theory background}
\label{sec:background}

The goal of lossy compression theory is to find the number of bits (\emph{bit-rate}) required to store outcomes  of a random variable (r.v.) , so that it can be reconstructed within a certain tolerance.
This is accomplished in \citepos{shannon_coding_1959} rate-distortion (RD) theory by mapping  into a r.v.\  with low mutual information .
Specifically, given a distortion measure , the RD theory characterizes the minimal achievable bit-rate for a distortion threshold  by

In lossy compression,  is usually a reconstruction of , \ie,  it aims to faithfully approximate .
As a result, typical distortions, \eg, the mean squared error (MSE), assume that the sample spaces  of both r.v.s are the same.
This assumption is not required.
Indeed, any distortion  of the form , where there exists a  such that  is finite, is a valid choice \citep{berger_rate_1968}.
This shows that RD theory can be used outside of reconstructions.
In the following we refer to  as a compressed \textit{representation} of  to distinguish it from a reconstruction. 
\section{Minimal bit-rate for high predictive performance}
\label{sec:theory}


In this section, we characterize the bit-rate needed to represent  to ensure high performance on downstream tasks.
Our argument has three high-level steps:
\begin{inlinelist}
    \item define a distortion that controls downstream performance when predicting from  instead of ;
    \item simplify and validate this distortion when desired tasks satisfy an invariance condition;
    \item apply RD theory with the valid distortion.
\end{inlinelist}
For simplicity, our presentation is relatively informal; formal proofs are in \cref{appx:preliminaries,appx:proofs}.



\subsection{A distortion for worst-case predictive performance}


Suppose  is an image. Potential downstream tasks might include , whether the image displays a dog; or , whether the image is hand-drawn. Formally, these and other downstream tasks are expressed as , a set of random variables that are jointly distributed with . 
Let  denote the Bayes (best possible) risk when predicting  from .
For ease of presentation in the main paper, we consider only classification tasks  and Bayes risk of the standard log loss .
We deal with MSE and regression in \cref{appx:theorem_mse}.

In this setting, a meaningful distortion  quantifies the difference between predicting any  from the compressed , as opposed to using . This is the worst-case excess risk,

If , it is possible to achieve \textit{lossless prediction}: performing as well using  as using .
More generally, bounding \disttext{} by  ensures that  for all tasks in . 
However, there are two issues that need to be addressed before \cref{eqn:distortion_def} can be used. 
First, it is not clear whether \disttext{} is a valid distortion for RD theory.
Second, the worst excess-risk \disttext{} assumes access to all downstream tasks of interest  during compression, which is unrealistic in general. 


\subsection{Invariant tasks}


The tasks that we care about are not arbitrary, and often share structure. 
One such structure is invariance to certain pre-specified transformations of input data. For example, computer vision tasks are often invariant to mild transformations such as brightness changes.
Such invariance structure is common in realistic tasks, as seen by the wide-spread use  of data augmentations \cite{shorten_survey_2019} in machine learning (ML), which encourage predictions to be the same for an unaugmented  and an augmented .
Motivated by this we focus on sets of invariant tasks  .
\begin{figure}[t]
\centering
 \begin{subfigure}{0.30\linewidth}
\includegraphics[width=\linewidth]{figures/max_inv/max_invariants-rotation.pdf}
\vspace*{-1.5em}
  \caption{Rotation}
 \label{fig:Mx_rot}
 \end{subfigure}
\quad
\begin{subfigure}{0.30\linewidth}
\includegraphics[width=\linewidth]{figures/max_inv/max_invariants-scaling.pdf}
\vspace*{-1.5em}
  \caption{Scaling}
  \label{fig:Mx_scaling}
 \end{subfigure}
\quad
\begin{subfigure}{0.30\linewidth}
\includegraphics[width=\linewidth]{figures/max_inv/max_invariants-function.pdf}
\vspace*{-1.5em}
  \caption{Any transformation }
  \label{fig:Mx_f}
 \end{subfigure}
\hfill
\begin{subfigure}{0.30\linewidth}
\includegraphics[width=\linewidth]{figures/max_inv/max_invariants-permutation.pdf}
\vspace*{-1.5em}
  \caption{Permutation}
 \label{fig:Mx_perm}
 \end{subfigure}
\quad
\begin{subfigure}{0.30\linewidth}
\includegraphics[width=\linewidth]{figures/max_inv/max_invariants-canonization.pdf}
\vspace*{-1.5em}
  \caption{Graph isomorphism }
  \label{fig:Mx_graph}
  \end{subfigure}
\quad
\begin{subfigure}{0.30\linewidth}
\includegraphics[width=\linewidth]{figures/max_inv/max_invariants-augmentations.pdf}
\vspace*{-1.5em}
  \caption{Any data augmentation}
  \label{fig:Mx_aug}
  \end{subfigure}
\caption{Maximal invariants  are representatives of equivalence classes.
Example s include the: (a) Euclidean norm for rotations; (b)  unit vector for scaling; (c)  when equivalence classes are pre-images by ; (d) empirical measure for permutations; (e) canonical graph for graph isomorphisms; (f) unaugmented input for data augmentations.}
  \label{fig:max_inv}
\end{figure}
 
We consider a general notion of invariance, namely invariance specified by an equivalence relation  on .\footnote{As a reminder,  is an equivalence relation iff for all : (reflexivity) , (symmetry) , and (transitivity)  and .
}
 The equivalence induces a partition of  into disjoint \textit{equivalence classes}, and we are interested in  tasks whose conditional distributions are constant within these classes.
\begin{definition} \label{def:invariant_tasks_interest:main}
The set of \textit{invariant tasks of interest} with respect to an  equivalence , denoted , is all random variables  such that  for any . 
\end{definition}



\subsection{Rate-distortion theory for invariant task prediction} 

The key to simplifying \disttextinv{} is the existence of a (non-unique) worst-case invariant task, denoted .
Such task contains all and only information to which tasks  are not invariant; we call them \textit{maximal invariants}.
A maximal invariant  with respect to  is any function satisfying\footnote{
This extends the definition of maximal invariants \cite{eaton_group_1989} beyond invariances to group actions.
}

A maximal invariant removes all information that tasks are invariant to, as it maps equivalent
inputs to the same output, \ie, .
Yet, it retains the minimal information needed to perform invariant tasks, by mapping non-equivalent inputs  to different outputs .
In other words,  indexes the equivalence classes. 
For example, the Euclidean norm is a maximal invariant for rotation invariance, as all vectors that are rotated versions of one another can be characterized by their radial coordinate. 
For data augmentations, the canonical (unaugmented) version of the input is a maximal invariant. Other examples are shown in  \cref{fig:max_inv}.

We prove in \cref{appx:invariant_distortion} that under weak regularity conditions, maximal invariant tasks exist in , and that they achieve the supremum in \cref{eqn:distortion_def}.
This allows us to show that \disttextinv{} reduces to the Bayes risk of predicting  from  and that it is a valid distortion measure. Crucially, this allows us to quantify downstream performance without enumerating invariant tasks.
\begin{restatable}{proposition}{distrisk}
\label{prop:nicer_dist}
Let  be an equivalence relation and  a maximal invariant that takes at most countably many values, with .
Then \disttextinv{} \eqref{eqn:distortion_def} with log loss is a valid distortion and

\end{restatable}

Here we used , as  is a deterministic function.  
Also, note that the countable requirement holds when tasks are invariant to some rounding of the input, as is typically the case due to floating-point storage.
We accommodate the uncountable case for squared-error loss in \cref{appx:theorem_mse}.

With a valid distortion in hand, we invoke the RD theorem with \disttextinv{} to obtain our ``Rate-Invariance'' (RI) theorem.
The RI theorem characterizes the bit-rate needed to store  while ensuring small log-loss on invariant tasks.
We obtain analogous results for squared-error loss.
\begin{restatable}[Rate-Invariance]{theorem}{rateinvdist}\label{thm:rate_invariance_distortion}
Assume the conditions of \cref{prop:nicer_dist}. 
Let , and  denote the minimum achievable bit-rate for transmitting  such that for any  we have .
Then  if  and otherwise it is finite and

\end{restatable}
\begin{wrapfigure}{r}{0.4\textwidth}
\vspace{-1\baselineskip}
 \centering
 \includegraphics[width=0.98\linewidth]{figures/theoretical_RD/schematic_rd.pdf}
 \vspace{-1\baselineskip}
 \caption{Rate-Invariance function.}
\label{fig:schema_RD}
\vspace{-1\baselineskip}
\end{wrapfigure}
 To ensure lossless prediction, i.e., , our theorem states that we require a bit-rate of .
Intuitively, this is because  contains the minimal information needed to predict losslessly any .\footnote{
We prove in \cref{appx:theorem_lossless} that  for any losses used in ML.
}
Furthermore, the theorem relates compression and prediction by showing that allowing a  decrease in log-loss performance on all tasks can save \textit{exactly}  bits. 
Intuitively, this is a linear relationship, because expected log-loss is measured in bits.
On the right of \cref{eq:rate_inv} we further decompose  into two terms to provide another interpretation:
(i) , which, for discrete , is the bit-rate required to losslessly compress , and (ii) , which quantifies the information removed due to the invariance of desired tasks.
Importantly, removing this information does not impact the best possible predictive performance.
See \cref{fig:schema_RD}.


The bit-rate gains can be substantial, depending on the invariances.
Consider compressing a sequence of  \iid fair coin flips.
 Suppose one is only interested in predicting permutation invariant labels.
 Then instead of compressing the entire sequence in  bits, one could compress the number of heads, which is a maximal invariant for permutation invariance, in  bits.\footnote{The number of heads  follow a binomial distribution so .
 Here  is also a minimal sufficient statistic for .
 More generally, if  is invariant to  and  is the coarsest such relation, then minimal sufficiency coincides with maximal invariance.
In practice, however,  will rarely be  invariant.
 }
As more interesting examples, we recover in \cref{appx:recovering} results from \begin{inlinelist}
\item unlabeled graph compression \cite{rashevsky_life_1955};
\item multiset compression \cite{varshney_benefiting_2007};
\item single task compression (IB; \cite{tishby_information_2000}).
\end{inlinelist}
The equivalence  can be induced by any transformations, such as transforming an image to its caption. 
We use this idea in \cref{sec:clip_experiments} to obtain > compression on ImageNet without sacrificing predictive performance.  
\section{Unsupervised training of invariant neural compressors}
\label{sec:learning}
In this section, we design practical, invariant neural compressors that bound optimal rates. 
Derivations are in \cref{appx:objectives}.
In particular, we are interested in the  encoders  of the RD function (\cref{eq:rate_distortion}) under the invariance distortion \disttextinv{}.
To accomplish this, we can optimize the following equivalent (\ie, it induces the same RI function) Lagrangian, where  takes the role of ,\footnote{
As the RI function is not strictly convex (\cref{fig:schema_RD}), it should be beneficial to use  to ensure that sweeping over  is equivalent to sweeping over  \cite{kolchinsky_caveats_2019}.
We did not see any difference in practice.
}

In ML, the maximal invariant  is often not available.
Instead, invariances are implicitly specified by sampling a random augmentation from , applying it to a datapoint , and asking that the model's prediction be invariant between  and . 
For example, invariance to cropping can be enforced by randomly cropping images while retaining the original label. 
We show in \cref{appx:objectives}, that in such case, we can treat the augmented  as the new source,  as the representation of , and the unaugmented  as the maximal invariant task .
Indeed,  is equal to  up to a constant, so we can rewrite \cref{eq:unconstrained} as the following equivalent objective,
\ydnote{here the equivalence is for any beta (i.e. for a given beta it induces the same segment of RI function.
But really we only care about the fact that it's same RI function, so I use "equivalence" which was already defined above.
}

Such reformulation is possible if random augmentations retain the invariance structure  but ``erase'' enough information about equivalent inputs, specifically, if .
We discuss the second requirement in \cref{appx:objectives}  but note that it will likely not be a practical issue if the dataset is small compared to the support .
With this, we have an objective whose r.v.s. are easy to sample from. 
However, both terms in \cref{eq:unconstrained_no_Mx} are still challenging to estimate. 

In the following, we develop two practical variational bounds to \cref{eq:unconstrained_no_Mx}, which can be optimized by stochastic gradient descent \cite{bottou_large-scale_2010} over the encoder's parameters. Both approximations use the standard lossy neural compression bound
 where  is called an entropy model (or a prior) \cite{balle_end--end_2017,theis_lossy_2017}. This has the advantage that the learned  can be used for entropy coding  \cite{rissanen_generalized_1976,duda_asymmetric_2009}. See \citet{balle_variational_2018} for possible entropy models. Our two approximations differ in how they upper bound .
The first uses a reconstruction loss, which attempts to reconstruct the unaugmented input  from .
The second uses a discrimination loss, which attempts to recognize which examples are augmented versions of the input.












\begin{figure}
\centering
\includegraphics[width=\linewidth]{figures/objectives/objectives-arrows_horizontal.pdf}
\vspace*{-1.5em}
\caption{
Our unsupervised objectives for invariant image compression under data augmentation use the same encoder, but differ in their approximation to the invariance distortion.
Both models encode the augmented data, pass the representation through an entropy bottleneck which ensures that they are compressed, and use a distortion to retain the information about the identity of the original data.
The models differ in how they retain that information: 
(VIC) by reconstructing unaugmented inputs;
(BINCE) by recognizing which inputs come from the same original data. 
}\label{fig:objectives}
\vspace{-1\baselineskip}
\end{figure} 
\subsection{Variational Invariant Compressor (VIC)}



Our first model is a modified neural compressor in which inputs are augmented but target reconstructions are not.
We refer to it as a \textit{variational invariant compressor} (VIC).
See \cref{fig:objectives} for an illustration.
VIC has an encoder , an entropy model , and a decoder .
Given a data sample , we apply a random augmentation , and encode it to get a representation .
The decoder then attempts to reconstruct the unaugmented  from . This leads to the objective,

The term  is an entropy bottleneck, which bounds the rate  and ensures that unnecessary information is removed.
The term  bounds the distortion  and ensures that VIC preserves the information needed for invariant tasks.




\subsection{Bottleneck InfoNCE (BINCE)}
\label{sec:BINCE}



Our second compressor retains all predictive information without reconstructing the data. It has two components: an entropy bottleneck and an InfoNCE \cite{oord_representation_2019} objective, which is the standard in contrastive SSL. We refer to this as the \textit{bottleneck InfoNCE} (BINCE), see \cref{fig:objectives}. BINCE has an advantage over VIC in that it avoids the problem of reconstructing possibly  high dimensional data.



\begin{wrapfigure}{r}{0.45\textwidth}
\vspace{-2.1\baselineskip} \begin{minipage}{0.45\textwidth}
 \centering
\newcommand{\vspacing}{} \begin{algorithm}[H]
\small
\caption{BINCE's forward pass for }
\label{alg:BINCE}
    \begin{algorithmic}[1]
    \Require \vspacing ,  , , , , , ,  
\State \vspacing  \Comment{Augment}
    \State \vspacing  
    \Comment{Encode}
    \State \vspacing  
    \State \vspacing   times  
    \State \vspacing  
    \State \vspacing  
    \State \vspacing  
    \State \vspacing 
    \State \vspacing   \\
    \Return \vspacing 
\end{algorithmic}
\end{algorithm}
\end{minipage}
\vspace{-1.7\baselineskip}
\end{wrapfigure} 
\Cref{alg:BINCE} shows how to train BINCE, where each call to  returns an independent augmentation of its input.
As with VIC, for every datapoint , we obtain a representation  by applying an augmentation  and passing it through the encoder .
We then sample a ``positive'' example  by encoding a different augmented version of the same underlying datapoint .
Finally, we sample  ``negative'' examples  by encoding augmentations  of datapoints  that are different from .
This results in a sequence .
For conciseness we will denote the above sampling procedure as .
The final loss uses a discriminator  that is optimized to score the equivalence of two representation,

BINCE retains the necessary information by classifying (as seen by the softmax) which  is associated with an equivalent example .
Both VIC and BINCE give rise to efficient compressors by passing  through  and entropy coding using .
In theory they can both recover the optimal rate for lossless predictions, \ie, , in the limit of infinite samples (,) and unconstrained variational families.
In practice, BINCE  has the advantage over VIC of 
\begin{inlinelist}
\item not requiring a high dimensional decoder; and
\item  giving (for suitable ) representations that are approximately linearly separable \cite{saunshi_theoretical_2019,tosh_contrastive_2021,lee_predicting_2020} and thus easy to predict from \cite{chen_simple_2020,oord_representation_2019}.
\end{inlinelist}
The disadvantages of BINCE are that it
\begin{inlinelist}
\item does not provide to reconstructions diminishes interpretability; and
\item has a high bias, unless the number of negative samples  is large \cite{poole_variational_2019,song_multi-label_2020}, which is computationally intensive.
\end{inlinelist}
















 
\section{Experiments}
\label{sec:experiments}
We evaluated our framework focusing on two questions:
\begin{inlinelist}
\item What compression rates can our framework achieve at what cost?
\item Can we train a general purpose predictive image compressor?
\end{inlinelist}
For all experiments, we train the compressors, freeze them, train the downstream predictors, and finally evaluate both on a test set.
For classical compressors, standard neural compressors (VC) and our VIC, we used either reconstructions  as inputs to the predictors or representations .
As BINCE does not provide reconstructions, we predicted from the compressed  using a multi-layer perceptron (MLP).
We used ResNet18 \cite{he_deep_2016} for encoders and image predictors.
For entropy models we used \citepos{balle_variational_2018} hyperprior, which uses uniform quantization.
We optimized hyper-parameters on validation using random search.
For classification tasks, we report classification error instead of log-loss.
The former is more standard and gave similar results (see \cref{appx:mnist}).
For experimental details see \cref{appx:reproducability}. 
For additional results see \cref{appx:results}.
Code is at \codeurl{}.

\subsection{Building intuition with toy experiments}
\label{sec:toy_experiments}


\begin{figure}
     \centering
     \begin{minipage}{.36\linewidth}
     \begin{subfigure}[h]{\linewidth}
         \centering
         \includegraphics[width=\linewidth]{figures/banana/RD_curve_workshop.png}
         \vspace{-2em}
         \caption{Rate-Invariance curves}
         \label{fig:bananas_RI}
     \end{subfigure}
     \end{minipage}
     \qquad
     \begin{minipage}{.48\linewidth}
\begin{subfigure}{0.32\columnwidth}
         \centering
         \includegraphics[width=\textwidth]{figures/banana/quantization_vae_100.png}
         \vspace{-1.3em}
         \caption{VC high rate}
         \label{fig:bananas_sweepvae_100}
     \end{subfigure}
     \begin{subfigure}{0.32\columnwidth}
         \centering
         \includegraphics[width=\textwidth]{figures/banana/quantization_vae_10.png}
         \vspace{-1.3em}
         \caption{VC}
         \label{fig:bananas_sweepvae_10}
     \end{subfigure}
     \begin{subfigure}{0.32\columnwidth}
         \centering
         \includegraphics[width=\textwidth]{figures/banana/quantization_vae_1.png}
         \vspace{-1.3em}
         \caption{VC low rate}
         \label{fig:bananas_sweepvae_1}
     \end{subfigure}
     
     
     \begin{subfigure}{0.32\columnwidth}
         \centering
         \includegraphics[width=\textwidth]{figures/banana/quantization_ivae_100.png}
         \vspace{-1.3em}
         \caption{VIC high rate}
         \label{fig:bananas_sweepivae_100}
     \end{subfigure}
     \begin{subfigure}{0.32\columnwidth}
         \centering
         \includegraphics[width=\textwidth]{figures/banana/quantization_ivae_10.png}
         \vspace{-1.3em}
         \caption{VIC }
         \label{fig:bananas_sweepivae_10}
     \end{subfigure}
     \begin{subfigure}{0.32\columnwidth}
         \centering
         \includegraphics[width=\textwidth]{figures/banana/quantization_ivae_1.png}
         \vspace{-1.3em}
         \caption{VIC low rate }
         \label{fig:bananas_sweepivae_1}
     \end{subfigure}
     \end{minipage}
\caption{
Compression rates of a Banana source \cite{balle_nonlinear_2020} can be decreased when downstream tasks are rotation invariant. 
(Left) Our invariant compressor (VIC, blue) outperforms neural compressors (VC, orange). 5 runs with standard errors in gray.
(Right) VIC quantizes the space using disks to remove unnecessary angular information.
Pink lines are quantization boundaries, dots are code vectors with size proportional to learned probabilities.
Low rates correspond to low  in \cref{eq:unconstrained_no_Mx}.
}
\label{fig:bananas_sweeps}
\vspace{-1\baselineskip}
\end{figure}
\ydnote{TODO: make gray line in RD curve more visible} 
To build an visual intuition, we compressed samples from a 2D banana source distribution \cite{balle_nonlinear_2020}, assuming rotation invariant tasks, \eg, classifying whether points are in the unit circle.
We also compressed MNIST digits as in \cref{fig:mnist_intro}.
Digits are augmented (rotations, translations, shearing, scaling) both at train and test time to ensure that our invariance assumption still holds.

\paragraphQ{Where do our rate gains come from}
For rotation invariant tasks, our method (VIC) discards unnecessary angular information by learning disk-shaped quantizations (\cref{fig:bananas_sweeps}, bottom right).
Specifically, VIC retains only radial information 
by mapping all randomly rotated points (disks) back to maximal invariants (pink dots).
In contrast, standard neural compressors (VC) attempt to reconstruct all information, which requires a finer partition (\cref{fig:bananas_sweeps}, top right).
As a result (\Cref{fig:bananas_RI}), VIC needs a smaller bit-rate (-axis) for the same desired performance (\disttextinv{}, -axis).
The area under the RD curve (AURD) for VIC is  against  for VC, \ie, expected bit-rate gains are around .
Similar gains are achieved for augmented MNIST in  \cref{fig:augmnist++} by reconstructing canonical digits.

\begin{figure}[h]
     \centering
     \begin{subfigure}[h]{0.31\columnwidth}
         \centering
         \includegraphics[width=\textwidth]{figures/augmnist_plus/RD_curve_workshop_error.png}
         \vspace{-1.7em}
         \caption{Rate-Error curve}
         \label{fig:augmnist++_RD}
     \end{subfigure}
     \begin{subfigure}[h]{0.68\columnwidth}
         \centering
         \includegraphics[width=\textwidth]{figures/augmnist_plus/rec_imgs_allin1_multirow.png}
         \vspace{-1.2em}
         \caption{Reconstructions that allow  downstream accuracy}
         \label{fig:augmnist++_RD_rec}
     \end{subfigure}
\caption{
(Left) By reconstructing prototypical digits our VIC (blue) achieves higher compression of augmented MNIST digits than standard neural compressors (VC, orange) without hindering downstream  classification. 5 runs.
(Right) The source examples (first row) as well as reconstructions for the non-invariant (second row) and invariant compressor (last row).
}
\label{fig:augmnist++}
\vspace{-0.5em}
\end{figure} 
\paragraphQ{Can we recover the optimal bit-rate}
We investigated whether our losses can achieve the optimal bit-rate for lossy prediction by using supervised augmentations, \ie,  randomly samples a train example  that has the same label.
For MNIST the single-task optimal bit-rate is  bits. 
 VIC and BINCE respectively achieve  and  bits, which shows that our losses are relatively good despite practical approximations. Details in \cref{appx:mnist}.


\paragraphQ{What is the impact of the choice of augmentations}
The choice of augmentation  implicitly defines the desired task-set , \ie,  is the set of all tasks for which  does not remove information.
As a result \cref{thm:rate_invariance_distortion} can be rewritten as , so the rate decreases when  removes more information from .
To illustrate this we trained our VIC using three augmentation sets on MNIST, all of which keep the true label invariant but progressively discard more  information.
VIC respectively achieves a rate of , , and  bits, which shows the importance of using augmentations that remove  information.
Details and BINCE results are in \cref{appx:mnist}.




\subsection{Evaluating our methods with controlled experiments}
To investigate our methods, we compressed the STL10 dataset \cite{coates_analysis_2011}.
We augment (flipping, color jittering, cropping) the train \textit{and} test set, to ensure that the task invariance assumptions are satisfied.
We focus on more realistic settings in the next section.
In each experiment, we sampled  combinations of hyper-parameters to ensure equal computational budget across models and baselines.


\input{tables/STL10_distortions_short}



\paragraphQ{How do our BINCE and  VIC compare to standard compressors}
In \cref{table:distortion_variation} we compare compressors at the lowest downstream error that they achieved.
As benchmark, we use PNG's lossless compression.
Predicting from PNG corresponds to standard image classification, and obtains a rate of  bits per image for  accuracy.
Classical lossy methods (JPEG, WebP) achieved up to  bit-rate gains with little drop in performance.
In comparison, our BINCE method achieved   compression gains with no impact on predictions.
Both our invariant (VIC) and standard (VC) neural compressors significantly decreased classification accuracy, which we believe can be explained by the encoders architecture (ResNet18) that we use for consistency (see \cref{appx:stl10}).


\paragraphQ{Should we predict from representations  or reconstructions }
In \cref{table:distortion_variation} we analyzed the impact of predicting from  instead of  for VIC and see that this increases accuracy by . 
In contrast, predicting from  for VC decreases performance by  (see \cref{appx:stl10}).
This suggests that invariant reconstructions  might not be easy to predict from with standard image predictors.


\paragraphQ{Are we learning invariant compressors}
Invariant compressors should provide RD curves that are robust to test distribution shift in the desired augmentations.
We thus trained our VIC by applying the augmentations  of the time but varying that probability  at test time.
In \cref{appx:stl10} we show that this distribution shift have negligible influence on RD curves.





\subsection{A zero-shot compressor using pre-trained self-supervised models}
\label{sec:clip_experiments}

BINCE includes a standard contrastive SSL loss. So, we investigated whether existing pre-trained SSL models \cite{chen_simple_2020,radford_learning_2021} can be used to build generic compressors. 
In particular, we investigated whether CLIP \cite{radford_learning_2021} could be quickly turned into a powerful task-centric compressor for computer vision.
In the introduction, we motivated large compression gains by noting that typical image classification tasks can be predicted from detailed captions instead of images (around  more bits). 
CLIP is a vision transformer \cite{dosovitskiy_image_2020} pre-trained on 400M pairs of images and text  using a contrastive loss.
The ``augmentation''  is then a function that maps  to its associated  and vis-versa. This will partition the images and texts into sets, each of which are associated directly or by transitivity in CLIP's dataset.
This suggests that CLIP is retaining the image information that corresponds to a detailed caption, and may be turned into a generic compressor for image classification.




CLIP can essentially be seen as a BINCE model with an image-to-text augmentation, but without an entropy bottleneck.
(For details about the CLIP-BINCE relation see \cref{appx:clip_bince}.)
We thus constructed an approximation of our desired image-to-text BINCE compressor by two simple steps.
First, we downloaded and froze CLIP's parameters. Second, we trained, on the small MSCOCO dataset \cite{lin_microsoft_2015}, an entropy bottleneck to compress CLIP's representation.
The latter step can be done by training any lossy compressor on CLIP's representations, we did so using \citepos{balle_variational_2018} hyperprior entropy model with a learned rounded precision.
We then evaluated our resulting compressor on 8 datasets (various classification tasks and image shapes) that were never seen during training (zero-shot), by training an MLP for downstream predictions on each dataset.
One can see this as a multi-task setting (each dataset is a distinct task). We investigate the case of multiple labels per images in \cref{appx:galaxy}.

\begin{table}[h]
\vspace{-1\baselineskip}
\caption{
Converting a pretrained SSL model into a zero-shot compressor achieves substantial bit-rate gains while allowing test accuracies similar to supervised models predicting from raw images. 
}
\small
\center
\begin{tabular}{lrrrrrrrrr}
\toprule
& ImageNet  & STL & PCam & Cars & CIFAR10 & Food       & Pets & Caltech  \\ \midrule 
Rate gains vs JPEG &    &  &    &   &  &  &  &   \\ \midrule 
Our Acc.  &    &   &  &   &  &   &  &   \\
Supervised Acc.  &    &   &    &  &   &   &  &  \\ 
\bottomrule
\end{tabular}
\label{table:clip}
\end{table} 
\paragraphQ{Can we use pretrained SSL to obtain a generic compressor}
\cref{table:clip} shows that we can exploit existing state-of-the-art (SOTA) SSL models to get a powerful image compressor, which achieves  bit-rate gains on ImageNet compared to JPEG (at the quality level used for storing ImageNet).
The bit-rate gains (1\textsuperscript{st} row) are significant across all zero-shot datasets, even for biological tissues (PCam; \cite{veeling_rotation_2018}).
Importantly, these gains come at little cost in test performance.
Indeed, the test accuracies of MLPs from our representations (2\textsuperscript{nd} row) is similar to a near SOTA model trained on the uncompressed images (3\textsuperscript{rd} row is from \citet{radford_learning_2021}).
These results are not surprising as JPEG is optimized to retain perceptual rather than classification information.
Note that the large variance in rate gains come from JPEG rates due to different images shapes (see \cref{table:clip_vs_EB}).

\paragraph{Our CLIP compressor retains all the information needed to get 0 error for those tasks}
\Cref{table:clip} provides the test performance for MLPs, while our theory discusses Bayes risk, which is independent of specific predictors and generalization. 
We estimated the excess Bayes risk for our datasets by counting the images (in train and test) that get compressed to the same  but have different labels. 
We found that we are in the lossless prediction regime for those datasets.


\begin{table}[h]
\vspace{-1\baselineskip}
\caption{
Our entropy bottleneck (EB) on CLIP improves compression of representations up to  with little impact on predictions.
The same compressor is used across datasets. Rates are per image.
}
\small{}
\center
\begin{tabular}{llrrrrrrrr}
\toprule
 & &  ImageNet  & STL & PCam & Cars & CIFAR10  & Food      & Pets & Caltech  \\ 
\midrule 
\multirow{5}{*}{\rotatebox[origin=c]{90}{\centering ~Bit-Rate  }} 
& JPEG & 1.49e6  & 4.71e4 & 9.60e4 & 1.92e5 & 1.05e4  & 1.54e5     & 1.81e5 & 1.69e5  \\ 
& CLIP & 1.52e4  & 1.52e4 & 1.52e4 & 1.52e4 & 1.52e4    & 1.52e4     & 1.52e4 & 1.52e4  \\ 
 & \ \  \textbf{+EB high}  & 2.47e3  & 2.46e3 & 2.61e3 & 2.59e3 & 2.53e3  & 2.39e3      & 2.33e3 & 2.46e3  \\ 
 & \ \  \textbf{+EB}    & 1.35e3  & 1.34e3 & 1.49e3 & 1.47e3 & 1.41e3 & 1.27e3      & 1.21e3 & 1.34e3  \\ 
 & \ \  \textbf{+EB low}  & 9.63e2  & 9.52e2 & 1.09e3 & 1.07e3 & 1.02e3 & 8.89e2      & 8.35e2 & 9.53e2  \\ \midrule 
\multirow{4}{*}{\rotatebox[origin=c]{90}{\centering ~Test Acc. }}  
& CLIP  & 76.5  & 98.6 & 84.5 & 80.8 & 95.3  & 88.5      & 89.7 &  93.2  \\  
 & \ \  \textbf{+EB high}  & 76.6  & 98.7 & 82.7 & 80.4 & 95.3  & 88.5      & 89.6 & 93.5  \\  
 & \ \  \textbf{+EB}  & 76.3  & 98.7 & 80.9 & 79.6 & 95.2  & 88.3    & 89.5 & 93.4  \\
 & \ \  \textbf{+EB low}  & 76.0  & 98.7 & 80.1 & 78.9 & 94.8 &  87.6      & 88.6 & 92.9  \\ \bottomrule
\end{tabular}
\label{table:clip_vs_EB}
\end{table}

 
\paragraphQ{What is the effect of the entropy bottleneck}
In \cref{table:clip_vs_EB} we compare the pretrained CLIP, to our CLIP compressor with an entropy bottleneck (EB) trained at different values for .
When trained with a high , our EB improves bit-rates by an average of  without impacting predictions.
For our compressor from \cref{table:clip} (CLIP+EB ) the gains increase to  with little predictive impact.
The sacrifice in predictions is more clear for  bit-rate gains (low ).
This shows that CLIP's raw representations retain unnecessary information as it not explicitly trained  to discard information.

\paragraphQ{How would end-to-end BINCE compare to staggered training}
Compression gains can likely be larger by end-to-end training of BINCE, which would require access to CLIP's original dataset.\footnote{We investigated finetuning CLIP on MSCOCO but it suffered from catastrophic forgetting.}
To get an idea of potential gains we compared end-to-end and staggered BINCE on augmented MNIST in \cref{appx:mnist}. We found significant rate improvements ( to  bits) for similar test accuracy. 

\paragraph{Our CLIP compressor is simple to use}
In \cref{appx:code_clip}, we provide a minimal script (150 lines) to train a generic compressor in less than five minutes on a single GPU.
The script contains an efficient entropy coder for our model (200 images/second), which shows its practicality. As usual in SSL, the compressed representations are also more computationally efficient to work with than standard compressors.
In our minimal script we achieve the desired performance ( on STL) using a linear model that is trained in one second, which is  faster than the baseline in \cref{table:clip}. This shows that our pipeline can improve computational efficiency in addition to storage efficiency.


\begin{table}[h]
\vspace{-1\baselineskip}
\caption{
Text-image invariance is better than invariance to standard augmentations for image classification.
CLIP and SimCLR are both ResNet50 pretrained with InfoNCE but different augmentations.
}
\small
\center
\begin{tabular}{llrrrrrrrrr}
\toprule
&& ImageNet  & STL & PCam & Cars & CIFAR10 & Food       & Pets & Caltech  \\ 
\midrule 
\multirow{2}{*}{\rotatebox[origin=c]{90}{\centering \scriptsize ~Rate  }} 
& CLIP+EB   &    &  &    &   &  &  &  &   \\ 
& SimCLR+EB   &    &  &    &   &  &  &  &   \\  
\midrule 
\multirow{2}{*}{\rotatebox[origin=c]{90}{\centering \scriptsize ~Acc.  }} 
& CLIP+EB   &    &   &  &   &  &   &  &   \\
& SimCLR+EB   &    &   &    &  &   &   &  &  \\ 
\bottomrule
\end{tabular}
\label{table:ssl}
\end{table} 

\paragraphQ{What augmentations to use for SSL compression}
\Cref{table:ssl} compares two ResNet50 pretrained with contrastive learning using invariance to text-image (CLIP) or standard image augmentations (SimCLR \cite{chen_simple_2020}) such as cropping or flipping. 
We see that CLIP's augmentation usually give better compression and downstream performance, which shows the importance of the choice of augmentations.
This also supports our motivation of using text-image augmentations, which are likely label-preserving for a vast amount of tasks but discard large amounts of unnecessary information. 

 
\section{Related work}
\label{sec:related}
In \cref{appx:related} we discuss more related work, including invariances in compression and the link to SSL.


\paragraph{Task-centric compression}
To our knowledge, our paper is the first to formalize compression only for predictions.
IB \cite{tishby_information_2000} uses a task-centric distortion, but is not used for compression as it requires supervised training, so there are no advantages compared to compressing predicted labels.
Some authors used heuristics to bypass the supervised issue, \eg, focusing on low frequencies for classification \cite{liu_deepn-jpeg_2018} or high frequencies for segmentation \cite{liu_machine_2019}.
Other authors have incorporated predictive errors to perceptual distortions \cite{liu_recognizable_2016,liu_classification-distortion-perception_2019}, but cannot compress without the perceptual distortion for the same reason as IB. 
One exception is \citepos{weber_observer_2020} compressor, which (when removing their perceptual distortion) minimizes MSE in the hidden layers of a pretrained classifier.
Even more related is \citepos{singh_end--end_2020} work on compressing pretrained features for transfer learning, which is practice is similar to our compression of SSL features.
Their work do not provide theoretical justifications, and are constrained to tasks that are similar to those used for pretraining.




 
\section{Discussion and Outlook}
\label{sec:conclusion}

Given the ever increasing amount of data that is processed by task-specific algorithms, it is necessary to rethink the current task-agnostic compression paradigm.
We formalized the first compression framework for retaining only the information necessary for high performance on desired tasks. 
Using our theory, we provide two unsupervised objectives for training neural compressors.
Experimentally, we show that these compressors can achieve  bit-rates that are orders of magnitude ( on ImageNet) smaller than standard image compressors without losing predictive performance.

There are a number of caveats that should be addressed. 
First, to achieve better rates, our theory requires an irrecoverable loss of information. This can be an issue if the set of desired tasks changes.
For example, if one uses text-image invariances then it may be impossible to perform image segmentation from the compressed representations.
One solution would be to keep an original copy and use invariant compression for duplicated data, \eg, for the thousands copies of ImageNet.
A second issue is the interpretability of the compressed representations.
This can be partially addressed by reconstructing prototypical data as in \cref{fig:mnist_intro} (post-hoc decoders could be trained for BINCE).
A third caveat is that the compressed representations may be harder to learn from, \eg, neural networks may struggle to predict from representations even if the information is retained.
Although our experiments actually showed the opposite, this should be addressed theoretically, \eg, using decodable information \cite{xu_theory_2020,dubois_learning_2020}.
Finally, successful use of our framework requires access to label-preserving augmentations  that discard significant information about . 
Finding such an  may be challenging for some tasks.
Given that augmentations are ubiquitous in ML, the community will hopefully continue developing task-specific augmentations which we could take advantage of.

Nevertheless, we achieved orders of magnitude improvements in compression for predictions, and we believe that our improvements are just the beginning.
For example, many tasks can be answered by referencing a detailed natural language description of the data. In these cases, the improvements can be very large, potentially 1M for videos.\footnote{A movie takes around 10GB to store, but the information relevant to humans (\eg, ``what happened to the house?'', ``how did the movie end?'') can likely be stored in a detailed movie script and would require 100KB.}
In the long-term, we hope that abandoning perceptual reconstructions will enable individuals to process data at scales that are currently only possible at large institutions, and our society to take advantage of large data sources in a more sustainable way.  
\clearpage
\newpage

\begin{ack}
We would like to thank Alex Alemi, David Duvenaud, Andriy Mnih, Emile Mathieu, Jonah Philion, Yangjun Ruan, and Ilya Sutskever for their helpful feedback and encouragements.
Resources used in preparing this research were provided, in part, by the Province of Ontario, the Government of Canada through CIFAR, and companies sponsoring the Vector Institute.
BBR acknowledges the support of the Natural Sciences and Engineering Research Council of Canada (NSERC): RGPIN-2020-04995, RGPAS-2020-00095, DGECR-2020-00343.
\end{ack}
 
\bibliographystyle{IEEEtranN}
\bibliography{bibliography}

\clearpage
\newpage



\appendix
\addcontentsline{toc}{section}{Appendix} \part{Appendix} 



\parttoc 



\clearpage
\newpage


\section{Preliminaries}
\label{appx:preliminaries}


\subsection{Notation}
\label{appx:notation}



\kword{Probability} 
We assume a background standard probability space  that is rich enough to support all random variables used. 
Letters that are upper-case  represent a random variable, while realizations are denoted with the associated lower case . 
The sample space of a random variable will be written using a calligraphic , and we will say that  takes value in (t.v.i) .
We denote the probability distribution of  as  and the probability density function, if it exists, as .
 denotes that  has a certain distribution (here, Gaussian).
Expectations are written as: , or  when the density exists.
Independence between two random variables  and  is denoted with .
To denote conditional independence between two random variables  and  given  we either use  or say that  forms a Markov Chain. 
 denotes a composition of two functions  and , but in the case of random variable we also use the shorthand .

\kword{Information theory} 
For notational convenience (see \cref{assumption:density}  below), when dealing with log loss we will always assume the existence of probability densities, in which case the KL divergence between two probability distributions on ,  and , is .
The mutual information between random variables  and  is .
The (differential or discrete) entropy of a random variable is , while the conditional (differential) entropy is .




\kword{Equivalence} 
 denotes that  and  are equivalent with respect to (w.r.t.) an equivalence relation on  (the exact relation being implicit).
The equivalence class of  under  consist of all elements that are equivalent to , \ie .
The set of all equivalence classes (the quotient set) will be denoted as , while the canonical projection is denoted as .

\kword{Risk minimization} 
We will often use variational optimization. When the variational family is not made explicit it means that the optimization is over all functions with the correct domain and codomain, \eg   means that that the optimization is done over the collection of all conditional probability densities  on  given the random variable . 

For a fixed ``action'' or ``decision'' space , a loss function is defined as . The (expected) risk of a predictor  is . 
The Bayes (best achievable) risk when predicting  from  using some (unspecified) loss is denoted as . 
When the loss  is specified, we denote the Bayes risk as .
For the case of log loss (always assumed in the main text) we have  .
For MSE loss we have .
Letters , , and  refer to the input, representation and target of a predictive task, respectively.





\subsection{Assumptions}
\label{appx:assumptions}


In this section, we discuss the assumptions that we make throughout our paper. 
Specifically, we discuss why we make those assumption and why such assumptions should hold in practice.
\textbf{All our assumptions should hold in most practical scenarios.}
The following assumptions will be implicit in the rest of our work.



\ydnote{We should remove any notion of M(X) in this assumption and have at as a lemma that comes from the fact that X satisfies this assumption}
\begin{assumption}[Finite risk]
\label{assumption:variance}
We restrict ourselves to tasks , such that\   for any finite constant  in the domain of the predictor .
Similarly we restrict ourselves to  with , and to equivalences relations on  such if there exists a maximal invariant then there exists \textit{some} maximal invariant  with  for any finite constant .
\end{assumption}

\Cref{assumption:variance} ensures that we can take differences of Bayes risks as in \cref{def:excess_distortion}.
For the case of log loss our assumption is equivalent to requiring finite (differential or discrete) entropy of ,  and .
For MSE loss, this is equivalent to finite variance for ,  and . Specifically, we will restrict ourselves to random variables  and  that are bounded in . (See \cref{assumption:l2:bounded} in \cref{appx:theorem_mse}.)
Note that  comes directly from  for the two main losses that we consider.
Indeed, for log loss this comes directly from the data processing inequality.
For MSE such  can easily be constructed by mapping any  to a value in  that is smaller than the expected value over the equivalence class .
\ydnote{Should probably be a lemma. The problem is that I don't know how to prove it for \textit{any} loss which would be needed for \cref{appx:theorem_lossless}.}

\begin{assumption}[Existence of regular conditional probabilities] 
\label{assumption:regular} 
We restrict ourselves to standard Borel measurable spaces, so that the existence of regular conditional probability distributions is ensured. 
This is necessary to ensure the existence of probability kernel in \cref{lemma:desintegration}. 
This technical assumption essentially holds for all practical purposes. Unless stated otherwise, we denote  the Borel -algebra of a set .
\end{assumption}

\begin{assumption}[Measurability of functions] \label{assumption:measurability} 
We assume that all functions introduced in the following sections are measurable with respect to the ``natural'' measurable spaces of the functions' domain and codomain. (A few special functions will be shown to be measurable.) 
In particular, we require 
\begin{inlinelist}
\item the measurability of  which implies that  is a random variable; and
\item the measurability of the projection , which implies that there always exists a maximal invariant in the form of the projection . 
\end{inlinelist}
This technical assumption holds for essentially all practical purposes.
\end{assumption}



\Cref{assumption:variance,assumption:measurability,assumption:regular} are used throughout our work.
Two further assumptions are needed for log loss, which we remove in \cref{appx:theorem_mse} when we obtain results for MSE.

\begin{assumption}[Countably many equivalence classes] 
\label{assumption:discrete} 
For the log loss risk (\cref{appx:theorem_logloss}) we restrict our discussion to equivalences  such that the quotient set  is countable.
This ensures that  is a discrete random variable thereby ensuring that our invariance distortion  is independent of the choice of maximal invariant  as the conditional entropy is  invariant to bijections.
\end{assumption}

Note that \cref{assumption:discrete}  holds when  is countable which always happens in practice due to floating point arithmetic, \ie every real number has to be rounded to the closest 64 bits number.
Another perspective is to say that  is actually uncountable, but that all tasks we care about are always invariant to rounding to the nearest 64 bits number due to floating point arithmetic.
As a result, the maximal invariant is the usual maximal invariant rounded to the closest floating point. 
For example, if  is a 2D Gaussian we cannot work directly with translations on the y-axis (which gives uncountably many , one for each real number on the x-axis), but can work with y-axis invariance combined with invariance to rounding on the x-axis (e.g. closest 64 bits number).

\begin{assumption}[Convenience assumption: Existence of densities]\label{assumption:density} 
In sections \cref{appx:invariant_distortion,appx:theorem_logloss}, where we work with log loss, we restrict ourselves to cases where the (conditional) probability mass/density function exist, \ie, to probability distributions that are absolutely continuous w.r.t.\ to some (shared) underlying measure.
This assumption is not needed but it simplifies the notation, and ensures that the differential entropy of random variables is well defined.
Such assumption could be removed by using the general definition of mutual information as a supremum over partitions and by defining continuous entropy as  \cite{kolmogorov_shannon_1956,pinsker_information_1964}, also known as \citepos{jaynes_information_1957} limiting density of discrete points.
\end{assumption}

\subsection{Definitions}
\label{appx:definitions}

In the main paper we were relatively informal in our definitions, here we restate our main definitions more formally.



\begin{definition}[Maximal invariant]\label{def:maximal_invariant}
Let  denote an equivalence relation on .
We say that a measurable function  is a \textit{maximal invariant} w.r.t.\   if

\end{definition}

Note that our notion of maximal invariants generalizes the notion of maximal invariants in probabilistic group theory \cite{eaton_group_1989}. 
We refer the reader to \citet{lehmann_testing_2005} for many examples in the group case. As in the group case, a maximal invariant typically is not unique. 


The invariance structure that we want our tasks to have is based on their conditional distributions given , defined as follows.

\begin{definition}[Conditional invariance]\label{def:cond_invariance}
We say that  is conditionally invariant w.r.t.\  , if the regular conditional distribution  is invariant w.r.t.\ , \ie  we have 

\end{definition}
\ydnote{use notation from lemma 4.}


\begin{definition}[Invariant tasks of interest]\label{def:invariant_tasks_interest}
The set of all invariant tasks of interest  w.r.t.\ to a loss and an  equivalence  is the set of all random variables  that are conditionally invariant w.r.t.\  and that satisfy \cref{assumption:variance} (finite risk).
\end{definition}



First we require the notion of a valid distortion \cite{berger_rate_1968}, which ensures that we can apply the rate distortion theorem.


\begin{definition}[Valid distortion]\label{def:valid_distortion}
Let  and  be two random variables that take values in  and , respectively.
Then an (expected) distortion  is \textit{valid} w.r.t.\  if there exists a point-wise distortion  such that  for some  and 

\end{definition}

In the context of the current work, a representation  which arises by encoding  using  should not depend on any particular task . 

\begin{definition}[Representation for a task set]\label{def:representations}
Let  be two random variables and  be a set of random variables.
 is a \textit{representation} of  for  if for all  such that  and  are not almost surely equal, we have the pairwise conditional independence .
\end{definition}

Note that if  then it is not almost surely equal to any . The condition allows for the possibility that , but it must be conditionally independent, given , of all other random variables in . 


We now recall the excess risk distortion \disttext{}.

\begin{definition}[Excess risk distortion]\label{def:excess_distortion}
Let  and  be two random variables.
Let  be a set of random variables such that under a loss , the Bayes risks in \eqref{eq:excess:risk:dist} below are well defined for each .
The \textit{excess risk distortion} \disttext{} is defined as:

\end{definition}





\clearpage
\newpage 
\section{Proofs: optimal bit-rate}
\label{appx:proofs}


In this section we prove all results from \cref{sec:theory}. 

\subsection{Basic properties of equivalence relations and maximal invariants}
\label{appx:maximal:invariants}

To begin, we collect some basic properties of equivalence relations and maximal invariants.
As these a general result that might be of interest beyond our work (especially \cref{lemma:desintegration}) we will prove them without assuming the existence of densities, \ie, without \cref{assumption:density}.
Recall that  is the projection from  onto its quotient by , denoted .


\begin{lemma}[\citet{mac_lane_algebra_1999}, Theorem 19]
\label{lemma:projection:theorem}
Given an equivalence relation  on , let  be any function such that . Then there is exactly one function  for which . If  is a surjection and , then  is a bijection.
\end{lemma}

\begin{lemma}\label{lemma:maxinv_bijection}
Let  and  be two maximal invariants w.r.t.\ . Then there exists a bijective function  such that .
\end{lemma}
\begin{proof}
From \cref{lemma:projection:theorem},  is a maximal invariant if and only if there is a bijective function  such that\ the maximal invariant is the composition of  and the projection onto equivalence classes, \ie .
Let  be the corresponding bijection for .
Then we have  with  which is indeed bijective: .
\end{proof}

\begin{lemma}\label{lemma:maxinv_invariance}
Let  be any maximal invariant w.r.t. .
Then a measurable function  is invariant with respect to  if and only if there exists a measurable function  such that  for all , in which case  is measurable with respect to the -algebra generated by .
\end{lemma}
\begin{proof}
Clearly  is -invariant because  is, and measurability of  follows from measurability of  and .

From \cref{lemma:projection:theorem}, if  is -invariant then there is a function  such that . Since  and  are measurable, so too is . 
Again by \cref{lemma:projection:theorem}, there exists a bijective mapping  such that .
We thus conclude that , for .
The measurability of  follows from the measurability of  and of . 
\end{proof}

Finally, we establish a key conditional independence relationship, which shows that for invariant tasks,   forms a Markov Chain. 
This is a generalization of an probabilistic group theoretical results (Theorem 4.4 in \citet{eaton_group_1989}, Theorem 7 in \citet{bloem-reddy_probabilistic_2020}), to any equivalences (rather than only group orbits) and without making the assumption of (marginal) invariance of  to .


\begin{lemma}\label{lemma:desintegration}
Let  and  be two random variables, and  be a maximal invariant w.r.t.  as in \cref{def:maximal_invariant}.
Then  is conditionally invariant w.r.t.  as in \cref{def:cond_invariance} if and only if .
\end{lemma}
\begin{proof}
Let  be a -valued random variable that is conditionally invariant w.r.t. . Recall that  is the Borel -algebra of . 
By \cref{assumption:regular}, there exists a probability kernel  from  into , such that for each set  ,  is a measurable function mapping .

Conditional invariance means that  for each . 
That is, as a function of ,  is invariant w.r.t.\ . 
By \cref{lemma:maxinv_invariance}, , where  is a probability kernel from  into . Therefore, for any ,

\ydnote{detail why k' is a probability kernel for camera ready}
which can be extended to arbitrary measurable functions on  by a standard (monotone class) argument. This in turn implies that  is a version of , i.e., they are equal almost surely , and therefore .
\ydnote{Is there a reason to use  instead of  ? }
\end{proof}

Finally, we prove that in realistic settings, there exists at least one .
\begin{lemma}\label{lemma:one_Mx_is_task}
Let  be the invariant tasks of interest w.r.t.\   and any loss function.  
Then there exists at least one maximal invariant that belongs to .
\end{lemma}
\begin{proof}
First, we have to prove by construction that a maximal invariant always exists. 
By definition equivalent elements have the same equivalence class and so . The projection map is measurable by assumption (\cref{assumption:measurability}), so it is a maximal invariant.

\ydnote{This should be changed to a lemma and \cref{assumption:variance} on  }
Second, due to the existence of at least one maximal invariant  we have by \cref{assumption:variance} that that there exists at least one  s.t. .
This  is therefore in .

\end{proof}

We close this section by establishing some properties of Bayes risk in this context. The following lemma, a data-processing inequality, appears in \citet{xu_minimum_2020}; we include it here for completeness, and provide a slightly more detailed proof.

\begin{lemma}[Data-processing inequality for Bayes risk] \label{lemma:dpi:bayes}
	Let  be a Markov chain of random variables. For any loss function ,
	
\end{lemma}
\begin{proof}
	Recall that one characterization of conditional independence is that  if and only if  almost surely for some measurable function  and  with  \citep[][Prop.\ 6.13]{kallenberg_foundations_2002}.


	Let  be a Bayes decision rule for predicting  from , and likewise for . By definition,
	
	For any ,  is a valid decision rule for predicting  from  with risk at least as great as . Therefore, .

\end{proof}

\begin{corollary}\label{lemma:Mx_is_X}
	Let  be the invariant tasks of interest with respect to   and any loss function, and  any maximal invariant. For any ,
	
\end{corollary}
\begin{proof}
	The result follows from applying \cref{lemma:dpi:bayes} to the trivial conditional independence  and the non-trivial conditional independence from \cref{lemma:desintegration} .
\end{proof}


\subsection{\texorpdfstring{\Cref{prop:nicer_dist}}{Prop. 1}: simplifying and validating \texorpdfstring{\disttext{}}{invariant distortion} for log loss}
\label{appx:invariant_distortion}


In this section we show that \cref{def:excess_distortion} is a valid distortion for log loss, and we prove the equivalence between \cref{def:excess_distortion} and . That equivalence is the key to prove \cref{thm:rate_invariance_distortion}.


The main steps in the proof are the following:
\begin{enumerate}[noitemsep]
\item Using the strict properness of the log loss, we relate the Bayes risk to the entropy:

\item We show that the supremum is achieved by :

\item Since  is a deterministic function and  is discrete, we have . Therefore,

\item We conclude, as desired, that

\end{enumerate}









The first step consists of relating the log loss Bayes risk and conditional entropy. 
This is a simple lemma that directly comes from the fact that the conditional distribution   is the Bayes predictor.

\begin{lemma}\label{lemma:risk_entropy}
Let  be random variables then the log loss Bayes risk is equal to the conditional (discrete or differential) entropy:

\end{lemma}
\begin{proof}

Where \cref{eq:risk_MI:strict_proper} uses the strict properness of the logarithmic scoring function rule \citep{gneiting_strictly_2007}.
\end{proof}

In the rest of the section, we will often be working with  and  . 
Importantly, we would like our results to be independent of the choice of maximal invariant .
We now prove that this will indeed be the case as all these (conditional) entropy terms are independent of the choice of .
We only prove it for the marginal entropy  but the same proof holds for conditional entropies.

\begin{lemma}\label{lemma:same_entropy}
Let  denote an equivalence relation on  satisfying \cref{assumption:discrete}.
Let  and  be two different maximal invariants w.r.t.\  . Then .
\end{lemma}
\begin{proof}
Due to \cref{assumption:discrete},  is a discrete random variable and so  is the discrete entropy, which is invariant to bijective functions \cite{kraskov_estimating_2004}.
From \cref{lemma:maxinv_bijection} we know that there exists a bijection between  and  from which we conclude that  as desired.
\end{proof}



One of the requirements on  to be set of downstream tasks  is the finiteness of . Thus, as a consequence of \cref{lemma:one_Mx_is_task,lemma:same_entropy}, in the case of log loss, all  are always in the set of downstream tasks .


\begin{lemma}\label{lemma:all_Mx_is_task}
Let  be the invariant tasks of interest w.r.t.\   and the log loss.
Then all maximal invariants are in .
\end{lemma}
\begin{proof}
Any  is conditionally invariant due to the \cref{def:maximal_invariant}.
From \cref{assumption:variance} we know that there exists at least one  with finite entropy, by \cref{lemma:same_entropy} they must all have finite entropy.
We conclude that all  (\cref{def:invariant_tasks_interest}).
\end{proof}



We are now ready to prove the desired proposition.



\begin{manualprop}{\ref{prop:nicer_dist}}[Invariant Distortion for log loss]\label{appx:prop:invariant_distortion}
Let  be the invariant tasks of interest w.r.t.\   and the log loss. Let  be any maximal invariant, and  be a representation of  for .
Then the excess distortion w.r.t.\ log loss, \disttextinv{}, is a valid distortion and

\end{manualprop}
\begin{proof}
We first prove that , from which it is straightforward to show that \disttextinv{} is a valid distortion. Starting from the definition of \disttextinv{}, we have


\Cref{eq:Y_MZ_X} uses the fact that  (\cref{lemma:desintegration}), that  by \cref{def:representations}, and that   because  (again using \cref{def:representations}).
\Cref{eq:Y_X_MZ} uses \cref{lemma:desintegration}.
To go from \cref{eq:Y_X_MZ} to \cref{eq:post_chainrule} we use the symmetry of conditional mutual information.
\Cref{appx:eq:posent} uses the discreteness of  due to \cref{assumption:discrete}, so  with equality when  which is possible due to \cref{lemma:one_Mx_is_task,lemma:all_Mx_is_task}.

From \cref{appx:eq:posent} it is easy to see that \disttextinv{} is valid as  with  which due to the discreteness of  (\cref{assumption:discrete}) is a function whose codomain is  as desired.
Due to \cref{assumption:variance} we know that for all constant  we have , so \disttextinv{} is valid.
\end{proof}

Note that  is very simple to work with if we have access to some .
Unfortunately, in practice  might not be known, but often we will have access to some other random variable  which has all the information necessary about .
See for example \cref{appx:vic_bince}.
We now prove that in such case we can optimize  instead of .

\begin{proposition}[Invariant Distortion without ]\label{prop:dist_no_Mx}
Let  be as in \cref{prop:nicer_dist}.
Let  be a random variable\ such that  and  almost surely. Then

where  depends only on  and not on .
\end{proposition}
\begin{proof}
From \cref{def:representations} and the fact that  (\cref{lemma:all_Mx_is_task}) we know that  forms a Markov Chain (MC). 
Due to our assumption 
 a.s. we also have that  a.s. so   forms a MC.
Putting all together we obtain the MC , which allows us to derive the following.

where the last line uses the Markov Chain  to provide a more interpretable value for the constant.
\end{proof}



\subsection{\texorpdfstring{\Cref{thm:rate_invariance_distortion}}{Theorem 2}: optimal bit-rate under log loss}
\label{appx:theorem_logloss}

Our main theoretical result is to characterize the minimal achievable rate to bound the Bayes risk of any invariant task.
Here we provide the proof for the case of log loss risk.
The result follows from \citepos{shannon_coding_1959} rate distortion theorem, and the validity of \disttextinv{}  (\cref{appx:prop:invariant_distortion}).

For convenience, we restate the well known rate distortion theorem. 

\begin{lemma}(\citet{shannon_coding_1959}; Theorem 7.2.4 and 7.2.5 from \citet{berger_rate_1971})\label{lemma:rate_distortion}
Let  be a valid distortion. 
The minimum achievable
bit-rate for transmitting an \iid source  with expected distortion less than  is given by the rate-distortion function:

\end{lemma}

We can now state our rate-invariance theorem.

\begin{manualthm}{\ref{thm:rate_invariance_distortion}}[Rate-invariance for log loss]
Let .
Let  be an equivalence relation on  that partitions  into countably many equivalence classes (\cref{assumption:discrete}).
Let  be the invariant tasks of interest w.r.t.\  (,) and the log loss,  be any maximal invariant, and  be a representation of  for .
Let  denote the minimum achievable bit-rate for transmitting an \iid source of  such that\ for any  we have .
Then   is finite and given by

\end{manualthm}

\begin{proof}
We first prove that .
We then prove that the rate  is achievable and so \cref{eq:appx:rate_invariance_distortion:HMX} holds.
Finally, we prove that  so \cref{eq:appx:rate_invariance_distortion:HX} holds which concludes the proof.

We want to transmit  such that  we have , in other words we would like .
As \disttextinv{} is valid (\cref{appx:prop:invariant_distortion}) we can directly apply the rate distortion theorem (\cref{lemma:rate_distortion}):


Where \cref{eq:rate_invariance_distortion:ineq_positivity} uses the data processing inequality (DPI). 
As the rate is always non-negative we have  .

We now prove that  is attainable and so .
Specifically we need to find a representation  of  such that 

The first case is trivial: set  to be independent of  and , \eg a constant.
Then,  and .

For the second case we need  to be an equality when . 
This happens iff  inequalities \cref{eq:rate_invariance_distortion:ineq_positivity} and \cref{eq:rate_invariance_distortion:ineq_delta} are equalities, \ie iff 
 (for equality of the DPI \cite{cover_elements_2006}) and
.
We do so by starting from  (such that ) and ``erasing'' a fraction  of bits, similarly to binary erasure channels, until .
Let  for some  and let  be a random variable that t.v.i. in  and have the following conditional density parametrized by :

A simple computation then gives , where the first equality uses \cref{lemma:risk_entropy} and the last equality uses  due to the discreteness of  (\cref{assumption:discrete}).
We can thus achieve  by setting .
Note that we will never divide by zero as  would be in the first case of \cref{eq:two_cases_theorem}. 
Importantly this  still satisfies  as it was constructed solely using  and independent noise.

We thus proved that  is obtainable and that .
From which we conclude that the best achievable bit-rate is .
\Cref{eq:appx:rate_invariance_distortion:HX}, follows from , which is a valid decomposition as both (differential conditional) entropy term are finite due to \cref{assumption:variance}.
The finiteness of  comes from the  fact that  due to \cref{assumption:variance}.
\ydnote{To replace with lemma sayign that  due to \cref{assumption:variance}}
\end{proof}


By setting  we directly get the best achievable rate for the lossless prediction but lossy compression setting. 

\begin{corollary}[Invariant source coding for log loss]\label{corr:invariant_source_coding}
Let ,, , ,  be as in \cref{thm:rate_invariance_distortion}.
Let  denote the minimum achievable bit-rate for transmitting an \iid source of  such that for any  we have .
Then   is finite and given by

\end{corollary}

\subsection{Recovering previous results in the literature}
\label{appx:recovering}


\Cref{corr:invariant_source_coding} recovers many previous results in the literature:
\begin{description}
\item[Unlabeled Graphs] 
Let us consider the task of compressing unlabeled graphs, here we consider tasks that are invariant to graph isomorphisms.
A possible maximal invariant is the graph canonization and  becomes the well known \textit{structural entropy} (also called topological information content) \cite{rashevsky_life_1955,yongwook_choi_compression_2009}.
If all isomorphic graphs are permissible and equiprobable, \citet{yongwook_choi_compression_2009} show that the structural entropy is  .
This is \cref{corr:gains}, where the second term corresponds to  with a uniform distribution on isomorphic graphs.
\item[Multisets] 
Let us derive the best achievable bit-rate for compressing multisets.
Let  be any sequence and  be invariant to permutations of that sequence.
One possible maximal invariant in that case is the empirical measure (also called type), \ie, the counts  of each of the  elements that are present in the sequence .
Lossless compression of multisets thus requires , as discussed by \citet{varshney_benefiting_2007}.
Using \cref{corr:gains} we can also characterize the bits gains that you obtain by considering the invariance, namely, .
This recovers theorem 1 of \citet{varshney_benefiting_2007}, where  is called the ``order entropy''.
Note that similarly to our example in the main text about \iid coin flips, the amount of bits needed to losslessly compress the multiset grows as   \cite{varshney_benefiting_2007}. 
\item[Information Bottleneck (IB)] 
Suppose you are interested in predicting a single task , where  is a (deterministic) ``target function''.
The task is invariant to any transformations between examples in the preimage of the labeling.
So the maximal invariant is  and the distortion becomes .
Then the rate-distortion function (\cref{appx:eq:rate_distortion}) becomes the information bottleneck (IB) \cite{tishby_information_2000}.
Using \cref{corr:invariant_source_coding} we see that for lossless predictions the optimal rate is  as  shown in \cite{wu_learnability_2019,fischer_conditional_2020}.
From a compression stand point this is nevertheless not very useful as , so IB for deterministic labels tells you to entropy code the labels . 
\item[Lossless]
Let  be discrete.
Every task will always be invariant to the equality ```` equivalent relation.
In this case the maximal invariant is the identity function, and we recover Shannon's source coding theorem .
\end{description}


\subsection{Generalizing \texorpdfstring{\Cref{thm:rate_invariance_distortion}}{Theorem 2}: optimal bit rate for lossless prediction and any loss}
\label{appx:theorem_lossless}

\bbnote{Skipping this section for now, will return.}


\Cref{corr:invariant_source_coding} characterizes the minimal achievable bit rate for the lossless prediciton regime w.r.t. log loss.
Here we show that the same result generalizes to essentially all loss function of practical interest.





The invariant source coding theorem does not hold for \textit{any} loss, for example if a loss is a constant function then the Bayes risk  will not depend on the input , and so the best achievable bit rate will trivially be 0 which is different than .
But it essentially holds for all losses that are minimized only by the ``correct'' predictor.
Specifically it holds for all losses that we dub information preserving.

\begin{definition}[Information preserving losses]\label{def:meaningful}
Let  be any loss function such that the Bayes risk  is well defined for all random variable .
We say that  is an \textit{information preserving loss} iff the optimal risk of deterministic targets is achieved only using inputs that have all the information about the output, \ie, iff for any function  and and r.v.s  we have

\end{definition}



In particular we have that if  is a discrete r.v. then .


Essentially all losses used in practice satisfy \cref{def:meaningful}.
For example it holds for the following very general families of losses:
\begin{description}
\item[Strictly proper scoring rules] Let  be a scoring rule that essentially quantifies with  the price/loss incurred by probabilistic prediction  when  is observed (lower is better).
 is \textit{strictly proper} \cite{gneiting_strictly_2007} (w.r.t. ) iff:

with equality if and only if .
Common examples are the log loss \cite{good_rational_1952}, 
Brier score \cite{brier_verification_1950}, spherical score \cite{good_comment_1971}, or the maximum mean discrepancy with characteristic bounded kernels \cite{sriperumbudur_injective_2008,huszar_scoring_2013}.
\item[Point-wise loss functions] Let  be a loss function that essentially quantifies with  the price/loss incurred by the point prediction  when  is observed (lower is better).
As is standard \cite{gneiting_making_2010} we assume that :

This holds for most pointwise losses of interest: mean squared error, mean absolute error, 0-1 loss (accuracy), Huber loss, \dots 
\end{description}

\begin{lemma}
Strictly proper scoring rules and point-wise loss functions are information preserving (\cref{def:meaningful}).
\end{lemma}
\begin{proof}
First let us consider point-wise loss functions.
Suppose that  is such that .
As  is a point-wise loss function (\cref{eq:loss_function}) and  is a deterministic function, we have that .
As a result we have  which using again \cref{eq:loss_function} is equivalent to the existence of  s.t.    as desired.


Now let us consider the case of proper scoring rules.
Suppose that  is such that .
As  is a strictly proper scoring rule we have that . 
The latter is a delta function, so the former must also. We thus have the existence of  s.t.    as desired.
\end{proof}


We now have all the tools to prove the general invariant source coding theorem.

\begin{theorem}[General invariant source coding]\label{thm:invariant_source_coding}
Let  be an invariance relation on  that satisfies \cref{assumption:discrete}.
Let  be the invariant tasks of interest w.r.t (,) and any loss function  as in \cref{def:invariant_tasks_interest},  be any (,) maximal invariant as in \cref{def:maximal_invariant}, and  be a representation of  for  as in \cref{def:representations}.
Let  denote the minimum achievable bit-rate for transmitting an \iid source of  such that for any  we have .
Then   is finite and given by

\end{theorem}
\begin{proof}
By \cref{lemma:Mx_is_X}  we know that for all  we have   for any loss function.
As a result, the lossless prediction bit rate is at most  because by \citepos{shannon_mathematical_1948} source coding theorem   can be transmitted using  bits as it is discrete (\cref{assumption:discrete}) and its entropy is finite (\cref{assumption:variance}).
\ydnote{TO replace with lemma finite risk of M(X)}
 
Let us now show that it is not possible to achieve a lower rate.
By \cref{lemma:one_Mx_is_task} there exists at least one maximal invariant such that .
We now prove that there is no  such that  and can be transmitted with less than  bits.
Suppose that ,  then because  is a meaningful loss function we have there exists  function  s.t. .
Using the discreteness of   (\cref{assumption:discrete}), we thus have .
Using the RD theorem (\cref{lemma:rate_distortion}) we know that the minimum bit rate for transmitting  under the constraint  is .
We thus find that transmitting a  which ensures lossless predictions cannot require less  than  bits which concludes the proof that .
To get \cref{eq:appx:invariant_source_coding:HX} we use the same decomposition as in \cref{thm:rate_invariance_distortion}.
\end{proof}


\subsection{Generalizing \texorpdfstring{\Cref{thm:rate_invariance_distortion}}{Theorem 2}: optimal bit rate under MSE loss}
\label{appx:theorem_mse}


In \cref{appx:theorem_logloss} we proved the rate-invariance theorem for the case of log loss.
Log loss is the standard loss function for classification in ML, but in the case of regresssion it is more common to use the MSE loss function.
In \cref{thm:invariant_source_coding} we have seen that our results for lossless prediction regime also holds for MSE (and other losses).
Due to the importance of MSE in ML, we also provide a full rate-invariance theorem for MSE. 

We assume that  for all , with some . Tasks taking values in fewer dimensions can always be padded with zeros. In this section  denotes the Euclidean norm, the -norm of a random variable  is , and  is the Hilbert space of all -valued random variables with finite -norm (random variables that are almost surely equal are identified as the same element of ). Since  and  remain unchanged but we may consider different -algebras, we use, e.g.,  for short.

Importantly, we do not require \cref{assumption:discrete} (countable ). 
Instead, we require the following common (known as a finite power constraint in compression \cite{cover_elements_2006}) regularity condition on  to ensure that we can attain a relevant supremum.

\ydnote{should change this assumption and paragraph to  and  and then have lemma for 
}
\begin{assumption}[-boundedness]\label{assumption:l2:bounded}
    We assume that  is bounded in . That is, there is some  such that  for all .
\end{assumption}

Note that this is a slightly more stringent version of \cref{assumption:variance}, as it essentially requires bounded variance of  rather than only finite variance.

Let  denote the -algebra generated by a maximal invariant (all maximal invariants generate the same -algebra because they are one-to-one functions of each other \cref{lemma:maxinv_bijection}), and let  denote all -valued functions that are -measurable and bounded in . Then , and by \cref{lemma:one_Mx_is_task},  is non-empty. 

The main step in the proof of a rate-invariance theorem for MSE is the following proposition, which shows that the excess risk distortion for MSE is a valid distortion that can be expressed in terms of a maximum over .










\begin{proposition}[Invariant Distortion for MSE]\label{appx:prop:invariant_distortion_mse}
Let  be the invariant tasks of interest w.r.t.\    and w.r.t.\ the MSE.
Fix any maximal invariant  that is also in , and let  be a representation of  for . 
Then the excess distortion  w.r.t.\ MSE, \disttextinv{}, is a valid distortion and

\end{proposition}
\ydnote{If we wanted to be picky we should use the same expectation notation everywhere (I don't feel strongly about either). Also not important for main deadline.}
\begin{proof}
	For compactness of notation, we use, for example,  to denote expectation with respect to , and  to denote an iterated expectation. Our proof makes use of conditional expectation in  being defined as projection in a Hilbert space. See \citep[e.g.,][Ch.\ 22-23]{jacod_probability_2004}.

	Firstly, fix some . It is well known that
	
	Taking the infimum over all measurable , we have
	
	when  -almost everywhere. Now by the conditional invariance,  (\cref{lemma:desintegration}), which implies  for any measurable function . Therefore, 
	
	when  -almost everywhere.

	Similarly, for fixed  t.v.i  with ,
	

	Observe that for any , \eqref{eq:phi:decomp} and the first term of \eqref{eq:psi:decomp} will cancel in the excess risk distortion. Therefore, 
	
	
	When taking the supremum over ,  can only affect \disttextinv{} through its conditional expectation given , . That conditional expectation is a -measurable function, so  for all . Therefore,
	
	and we can take the supremum over functions  instead of , which yields
	
	Expanding each quadratic and canceling terms involving , we find
	
	
	Now, since conditional expectation given  is just projection onto the (Hilbert) subspace , we have
	
	where  is the subspace orthogonal to  in . Since  and  are both closed (sub-)Hilbert spaces, it is straightforward to show that so too is their intersection . The bounded (by ) elements of  are just the closed ball of radius , so 
	
	Now, since neither  nor  is empty, their intersection is empty if and only if , i.e., : all maximal invariants can be written as functions of . In that case, . Alternatively, if  is not empty, then choose some  from it such that .
	


	Defining  yields a valid distortion \disttextinv{}.
\end{proof}

Note that the last last part of the proof makes it clear that for MSE the invariance distortion is either  or .
Intuitively this happens because MSE risk is not invariant to bijections so it possible to make any predictive mistake arbitrarily bad by setting  to be arbitrarily large at this mistaken prediction. 
This suggests that for the MSE risk (and other loss functions that are not invariant to bijections) the expected excess risk might be better suited than the worst case excess risk that we considered.


As the invariant distortion under MSE is valid, we can now simply incorporate it into the rate distortion theorem to get the desired theorem.

\begin{theorem}[Rate-invariance for MSE]\label{thm:rate_invariance_distortion_mse}
Let .
Let  be the invariant tasks of interest w.r.t.\ (,) and the MSE,  be any maximal invariant in , and  be a representation of  for .
Let  denote the minimum achievable bit-rate for transmitting an \iid source of  such that\ for any  we have .
Then   is given by

where .
\end{theorem}
\begin{proof}
The result \eqref{eq:rate_invariance_distortion_mse} follows from the fact that \disttextinv{} is a valid distortion (\cref{appx:prop:invariant_distortion_mse}) and the rate-distortion theorem (\ref{lemma:rate_distortion}). 
\end{proof}
\ydnote{For final version we'll have to deal with the fact that the distortion seems to be either 0 or B so there are only 2 rates that can be reached.
Is it stillf usefule ?
THe derivations seems very useful, but the result less so.
Really for MSE we should a about median / mean instead of sup excess risk ...
}
\ydnote{For final version (not a priority) we might want to provide also an upper bound, which is achieved if  is gaussian with variance , specifically, . This shows that for  we do have finite rate.} 
As a corollary, we obtain the following lower bound for the rate, which may be useful in practice.
\begin{corollary}
	Let  be any -valued maximal invariant with a probability density with respect to Lebesgue measure. Let  be any homeomorphism of  (including the identity map), and  any maximum distortion achieving function. Then the following lower bounds hold:
	 
\end{corollary}
\begin{proof}
	First, by the DPI, for any homeomorphism  of ,
	
	The first inequality is an equality if and only if ; the second if and only if  is a homeomorphism of  (and therefore is itself a maximal invariant). Second, using the translation-invariance of differential entropy and the fact that conditioning reduces differential entropy, 
	
	Now, 
	
	The maximum entropy distribution subject to this second-moment constraint is the -dimensional Gaussian distribution , where  is a diagonal covariance matrix with entries . The differential entropy of that Gaussian distribution is , and by Jensen's inequality,
	

	Putting this together with the first inequality in \eqref{eq:information:chain},
	
	The same argument holds for either of  or , yielding the stated lower bounds.
\end{proof}




\clearpage
\newpage 
\section{Variational objectives}
\label{appx:objectives}


In this section we will derive the variational bounds for estimating the rate and the distortion.
In contrast to the proofs of main theoretical results (previous section) derivations  will be less formal.
Throughout this section we focus on the log loss and implicitly make all assumptions described in \cref{appx:assumptions}.

Recall that the optimal bit-rate is simply the Rate Distortion function using our invariance distortion (Rate-Invariance function; \cref{eq:rate_invariance_constrained} ), so any optimal encoder (for a given ) can be obtained by using the following arg minimum:

As optimization in machine learning is typically unconstrained, we prefer using the following Lagrangian formulation.

Both of these formulations are equivalent in that the set of encoders that minimize \cref{appx:eq:unconstrained_RD} for some  is equal to the set of encoders that minimize \cref{appx:unconstrained_RD} for some  \cite{everett_generalized_1963,berger_rate-distortion_2003}.

Note that due to the piece-wise linearity of our RI function (\cref{fig:schema_RD}), \citet{kolchinsky_caveats_2019} tells us that sweeping over  using \cref{appx:eq:unconstrained_RD} will only enable us to learn the vertices of the RI function, namely,  for  and  for ,
while any point on the RI curve is simultaneously optimal for .
In other words, although the solutions of \cref{appx:eq:unconstrained_RD} span the entire RI curve, it is, in theory, not possible to decide which points on the RI curve to obtain by sweeping over beta.
\citet{kolchinsky_caveats_2019} shows that this can be easily solved by considering the squared distortion , in which case sweeping over  would be equivalent to sweeping over delta  in \cref{appx:unconstrained_RD}.
We did not see any difference in practice so preferred using the more understandable  .

Both terms  and  are hard to estimate from samples, so the rest of the section is devoted to deriving variational upper bounds on them.


\subsection{Variational upper bound for the rate term \texorpdfstring{}{I[Z;X]}}
\label{appx:variational_rate}


Let us discuss how to approximate the rate term .
The mutual information is well known to be hard to estimate from samples \cite{paninski_estimation_2003,mcallester_formal_2020}, but fortunately many variational bounds have previously proposed, see \citet{poole_variational_2019} for examples.
In the following we denote a family of variational distributions over  (priors or entropy models) as .


\subsubsection{Mutual information bottleneck}
\label{appx:mi_bottleneck}

 The first bound that we consider is the standard upper bound on , \eg, in VAE \cite{kingma_auto-encoding_2014} or VIB \cite{alemi_deep_2017}.
Specifically:

The approximation gap is then .
The bound has the advantage that if  then the bound is tight.
The major issue with the mutual information bottleneck, is that no efficient compressors can in general achieve the rate given by it \citep{agustsson_universally_2020}.\footnote{
See \citet{flamich_compressing_2020} or \citet{schulman_sending_2020} for an  algorithm.
}
For example, if we decided to entropy code  using the entropy model  then we would achieve  bits which is  more than what is given by our bound.\footnote{
Bits-back coding \citep{wallace_classification_1990} can efficiently reach the desired bit-rate only because it is in the lossless setting.
}

\subsubsection{Entropy bottleneck}
\label{appx:entropy_bottleneck}

One specific case of the mutual information bottleneck which enables efficient compression, is when  is discrete and arises from a deterministic transformation of . 
Indeed, in this case  so entropy coding (\eg \cite{rissanen_generalized_1976,duda_asymmetric_2009}) can reach the rate given by our bound.
Using the same derivation as for the mutual information bottleneck, we get,

This is the standard bound used in neural compressors \cite{balle_end--end_2017,theis_lossy_2017}.
The entropy bottleneck bound has the following downsides compared to mutual information bottleneck:
\begin{itemize}
\item It is generally not true that for any  the optimal rate can be achieved by a discrete and deterministic .
For the specific case of  and with \cref{assumption:discrete} it is the case, as we can simply set .
\item It is not suitable for gradient based optimization w.r.t. to the encoder (due to the discreteness of ) so we typically have to add noise during training \cite{balle_end--end_2017} which can cause a mismatch between training and testing \cite{agustsson_universally_2020}.
\end{itemize}

Despite these issues we will mostly use the entropy bottleneck bound in experiments as we want our method to give rise to practical compressors.

\subsection{Variational upper bound for the distortion term 
\texorpdfstring{}{R[M(X)|Z]}}
\label{appx:variational_distortion}


Let us now consider variational upper-bounds on the distortion .
For conciseness we will consider the same setting as in the main paper, \ie, log loss risk and countably many equivalence classes (\cref{assumption:discrete}).
But it is easy to see that the direct distortion bound generalizes to any loss without \cref{assumption:discrete}. \footnote{ which comes from the fact that we are taking an  over a subset  of all possible predictors.}

\subsubsection{Direct distortion}
\label{appx:direct_dist}

The obvious variational bound on the Bayes risk is the Bayes risk constrained to some functional family. 
 denotes a variational family of regular conditional distributions (decoders), then, 

which comes from the fact that we are taking an  over a subset  of all possible distribution.
A simple derivation shows that the approximation gap is , so the bound is tight if .
This direct distortion is simple, but typical variational families will require predicting (``reconstructing'') an expected prediction  which is challenging when   is in high dimension (\eg unaugmented images).


\subsubsection{Contrastive distortion}
\label{appx:contrastive_dist}

We now consider a bound that does not require explicitely predicting , by considering a noise contrastive estimator \cite{gutmann_noise-contrastive_2010}.
Suppose that for any  we can sample from a sequence , where  and .
Let  be a variational family of discriminators which is used scores how likely  are to be sampled from the joint  rather than the product of the marginal , then,

\Cref{eq:infonce_bound} uses InfoNCE \cite{oord_representation_2019}, which is a lower bound on mutual information \cite{poole_variational_2019,song_multi-label_2020}.
The last equation removes constants w.r.t.  and , as these terms do not have to be optimized over.
We see that we are only left with a log softmax term
\footnote{Taking exponentials is not necessary, any function  would work as a discriminator, we use  to ensure positivity as this has a nice softmax interpretation and is standard in practice.
Our derivation is still general as we can set .
}
that essentially aims to classify which of all the  comes from the .
The bound is tight if the variational family  contains the optimal predictor 
and as the number of negatives tends to infinity.
For a detailed discussion about noise contrastive estimation under the log loss, refer to \cite{gutmann_noise-contrastive_2010,ma_noise_2018,rhodes_variational_2019}.

Note that the contrastive distortion has the advantage of not having to reconstruct high dimensional data (\eg for images), but it suffers from bias in the case where the number of negatives  is small \cite{poole_variational_2019}.

One additional derivation which we will need in the following section, is that an upper bound can also be obtained by replacing  by any other r.v.  s.t.  forms a Markov Chain. 
Indeed starting from \cref{eq:MIMXz}, we have,

The bound can still be tight if in addition we have the following Markov Chain , which implies that .


\subsection{Case study: VIC and BINCE under data augmentations}
\label{appx:vic_bince}

The derivations in the previous 2 subsections are relatively general and abstract.
As a case study, we now discuss the two objectives that we propose in the main paper for the case where we have access to the desired data augmentation  and where we use neural functional families.
Namely, the variational families for the entropy model , the encoder , the decoder , and the discriminator  are all parametric neural families.
Throughout this subsection we will consider that we only have access to  through a dataset  of samples which were independently sampled from .

Let us formalize what we mean by having access to the correct data augmentations.
Let us denote as  the r.v. over a set of augmentations , \ie, a stochastic process.
Let  be the augmented source.
Note that by  we mean the r.v. which arises by sampling an augmentation  from the stochastic process , and then applying it to some samples  from .
\begin{assumption}[Augmentations]\label{assumption:augmentations}
We assume knowledge of a random augmentation generator  that satisfies the following two key properties
\begin{itemize}
\item \textbf{Retain invariance}. We require  to retain the invariance structure to , specifically,  almost surely.
\item \textbf{Remove information}. We require  to remove as much information as possible about the input. 
Specifically,  almost surely.
\end{itemize}
\end{assumption}
The first requirement is simple but clearly not sufficient.
For example, the identity function does satisfy such requirement for any equivalence relation ( by definition), yet it does not correspond to what we think as an augmentation because it does not remove any information about the input.
The second requirement formalizes exactly what is required, namely that the augmentation must remove all information about the input besides the knowledge about its equivalence class (which is needed for the first requirement).


The first requirement will typically hold. 
The second in more stringent.
Note that it holds if for all equivalent examples  in  we have .
Indeed  and similarly , using  we have  for all equivalent  which implies that  as desired.
In practice this only needs to hold for examples in our datasets, \ie, the second requirement holds if for all equivalent  in a dataset we have .
This is likely to hold in practice as the number of examples that are equivalent in a dataset will be small if  as is typically the case.
In particular, if a dataset does not contain any equivalent examples, \ie, for any  we have  then the requirement trivially holds.





\subsection{Issue: dealing with unknown \texorpdfstring{}{M(X)}}
\label{appx:unkown_M}

One issue which arises in practice is that we generally do not have access to .
We will overcome this issue by taking advantage of the fact that we have access to data augmentations   that induce our equivalence relation.
Intuitively, we will treat the augmented r.v.  as if it were the source, and use the actual source  instead of .
Note that by  we mean the r.v. which arises by sampling an augmentation  from the stochastic process , and then applying it to some samples  from .
From now on, let us denote as  the representation that arises from the augmented source.
\ydnote{Would be slightly more precise to introduce a new letter  but I feel it will confuse people}
Under suitable conditions on  we can replace the previous objective \cref{appx:eq:unconstrained_RD} with the following equivalent objective, which we denote as , 

By equivalence of those objectives we mean that for any  the set of RD tuples  that are achieved by solutions of \cref{appx:eq:unconstrained_RD}  is equal to the set of RD tuples  that are achieved by solutions of \cref{appx:eq:unconstrained_RD_no_Mx}.
In other words, they generate the same segment of the RI function.

First, let us show why we can replace  by , \ie, show that for any   we have that  is equivalent to .
This can be seen from the fact that the optimal bit rate in \cref{thm:rate_invariance_distortion} only depends on  and .
In particular,  does not depend on the distribution of the source inside the equivalence classes, .
Indeed, an optimal representation will compress all that information.\footnote{Note that the bit rate gains  clearly depend on , but not the actual bit-rate .}
As a result, we can attain the same optimal bit rate by considering any source  that is a transformed version of  as long as the transformation does not change the distribution of the maximal invariant, \ie, .
This is clearly the case for  as our augmentation retains invariance (\cref{assumption:augmentations}).

Now let us consider why and when replacing  by  makes sense.
Using \cref{prop:dist_no_Mx} we know that if  is s.t.  and  forms a Markov Chain then we can replace (up to constants which are not important for ) the distortion term
 by . 
These are exactly our requirements on augmentations (\cref{assumption:augmentations}).

For the rest of this section we will thus be working with  (\cref{appx:eq:unconstrained_RD_no_Mx}) instead of \cref{appx:eq:unconstrained_RD}.
Note that this means that, in theory, we should always use the augmented  from now on, \ie, not only at train time but also at test time.


\subsubsection{Variational Invariant Compressor (VIC)}
\label{appx:vic}

As seen in the main text the VIC loss is essentially a neural compressor where inputs are augmented but not the target reconstructions.
We derive it by combining our entropy bottleneck bound (\cref{appx:entropy_bottleneck}) and our direct distortion (\cref{appx:contrastive_dist}), which gives the following upper bound on ,



Using a Monte Carlo estimate for the expectation over , we get our desired objective,

In practice, we approximate the expectation over  and  using a single sample for computational efficiency.
A full algorithm is provided in \cref{alg:vic} and illustrated in \cref{fig:objectives} of the main text.

\begin{algorithm}[H]
\caption{Variational Invariant Compressor (VIC). Single sample forward pass.}
\label{alg:vic}
    \begin{algorithmic}[1]
    \Require Encoder ,  Entropy Model ,  Decoder 
    \Require Dataset , random augmentation generator , Lagrange multiplier 
    \State   \Comment{sample}
    \State  \Comment{random augment}
    \State  \Comment{encode}
    \State  \Comment{Entropy Bottleneck}
    \State  \Comment{Direct Distortion} \\
    \Return 
\end{algorithmic}
\end{algorithm} 
Note that  tends to  when using unconstrained variational families and as the dataset grows to infinity.
This essentially shows that VIC objective (with infinite samples and unconstrained families) will learn the optimal deterministic and discrete  (as discussed in \cref{appx:entropy_bottleneck}), in particular, when  it will learn an encoder which is optimal for the lossless prediction regime.

\subsubsection{Bottleneck InfoNCE (BINCE)}

VIC for images and data augmentation suffers from the issue that it needs a predictor which reconstructs a high dimensional image (as discussed in \cref{appx:direct_dist}).
To solve this issue we discuss our BINCE objective, which as seen in the main text, is essentially a standard contrastive self-supervised (SSL) objective with an additional entropy bottleneck.
We derive it by combining our entropy bottleneck bound (\cref{appx:entropy_bottleneck}) and our contrastive distortion (\cref{appx:contrastive_dist}).

\ydnote{Sorry this paragraph is not easy to explain, also from what I know we are the first ones to derive that properly and show that all of these deep learning tricks are actually valid.
I'm hesitating to use the infamous "proof left to the reader'' because no time / space to typeset everything and I know it's true (proof in my notes)
}
Note that in \cref{appx:contrastive_dist} for each  we needed a sequence  of outcomes of  that are sampled either from the conditional  or the marginal  .
As we will replace  by  ( see \cref{appx:unkown_M}) we now need a sequence of r.v.   s.t.  is  sampled from the conditional  while each  are independently sampled from the marginal .
Furthermore, as is standard in self-supervised learning (\eg \cite{chen_simple_2020,oord_representation_2019}) we will actually be using a sequence  of positive and negative representations instead of .
We do so by independently augmenting and encoding each r.v. in .
Using our requirement on the augmentations (\cref{assumption:augmentations}) we thus have the following Markov Chain .
As a result, we can use  instead of  in InfoNCE (see \cref{eq:contrastive_distortion_bound_markov}).
For conciseness we will denote the above sampling procedure as .
We then have the following upper bound on ,

Using a Monte Carlo estimate for the expectation over , we get our desired objective,

In practice we approximate the expectation over , and  using a single sample for computational efficiency.
Just as with VIC we have that  tends to  when using unconstrained variational families and as the dataset and number of negatives  grows to infinity.

\ydnote{Should batch forward pass really be in this section?}
In the main paper we provided a simple algorithm (\cref{alg:BINCE}) to compute BINCE for a single example .
This is computationally intensive as it requires sampling one sequence of r.v. for each example.
In practice, this is nevertheless easily amenable to batch computation.
Indeed, negative representations   are positive representations  for a different example.
As a result, we can first sample a batch   from .
Then augment it to two different sequences .
And finally represent each sequences to obtain .
Then for any  we have that  is a positive example while all other  are negatives.
We thus only need to sample a single representation per example in the dataset.
A full algorithm for batch computations is provided in \cref{alg:BINCE_batch} (using only one of the 2 augmented batches for notational convenience).


\begin{algorithm}[t]
\caption{Batch forward pass for BINCE}
\label{alg:BINCE_batch}
    \begin{algorithmic}[1]
    \Require encoder ,  entropy model , discriminator 
    \Require augmentations , data , Lagrangian coefficient , batch size  
    \State  times   \Comment{sample}
    \State  \Comment{Random augment 1}
    \State  \Comment{Random augment 2}
    \State  \Comment{Encode}
    \State 
    \State  average  over  \Comment{Entropy Bottleneck}
    \State 
    \For{}
    \State  \Comment{Select positive} 
    \State  \Comment{Softmax} 
    \State  \Comment{Contrastive Distortion} 
    \EndFor{}
    \Return 
\end{algorithmic}
\end{algorithm} 

\subsection{CLIP as BINCE's distortion}
\label{appx:clip_bince}

One of our main experiment (\cref{sec:clip_experiments}), consists in using a pretrained CLIP to make a powerful image compressor.
We are able to do so by realizing that CLIP essentially corresponds to BINCE's distortion (second term in \cref{eq:bottlenecked_simclr} with the following choices: 
\begin{itemize}
\item \textbf{Augmentation}: text to image transformation. CLIP’s dataset contains pairs of associated images and detailed text  . The ``augmentation'' is then a function that maps  to its associated   and vis versa.
This will partition the joint image-text space of  into sets, each of which are associated (directly or by transitivity) with a common text description or image.
\item \textbf{Discriminator}: a dot product, i.e, .
\item \textbf{Encoder}: a deterministic function defined by cases. Specifically, sampling from  gives the output of the visual transformer (image encoder)  if  is an image and the output of the text transformer (text encoder)  if  is a sentence.
\end{itemize}

The only minor difference is that CLIP performs contrastive learning between text-image and image-text but never text-text and image-image. BINCE would instead make no distinction between modalities as the equivalence class is on the joint image and text space. Both are nevertheless valid approximations to .

Although CLIP's augmentation will always give rise to a valid equivalence relation, 
it would in theory recover the degenerate solution of  for all  if the dataset was ``infinite''.
Indeed, any image could possible just be described by the text ``an image'', which would recover the aforementioned degenerate solution. 
There are different ways of collecting the datasets that could avoid this issue, \eg, ensuring that the description is more precise than that.
In practice, this is unlikely to be an issue as the dataset is finite.

Another possible theoretical issue of CLIP's augmentation/equivalence structure, is that it is likely that very few images have a common associated text in CLIP's dataset (or vis-versa). 
In theory, this would thus recover the degenerate solution where no points are equivalent to one another, \ie,  for all .
In practice, this issue is probably avoided due to the fact that images will get clustered as long as the the text description is similar enough for the text encoder to provide (essentially) the same text encoding (due to computational/architectural constraints). 
I.e., the images will actually get partitioned based on the value of the \textit{representation} of their associated text rather than the text itself.  

\clearpage
\newpage 
\section{Extended related work}
\label{appx:related}
\paragraph{Invariances and symmetries}
 Invariances are ubiquitous in ML, as seen by the use of data augmentations \cite{shorten_survey_2019} and invariant models \cite{shawe-taylor_building_1989,wood_representation_1996,bruna_invariant_2013,cohen_group_2016,zaheer_deep_2017,kondor_generalization_2018,bloem-reddy_probabilistic_2020}.
These force models not to rely on nuisances to improve generalization \cite{dao_kernel_2019,chen_group-theoretic_2020,lyle_benefits_2020}.
We directly discard such nuisances from the data to improve compression.
Others have used symmetries in  for lossless compression of multisets \cite{varshney_benefiting_2007}, graphs \cite{choi_compression_2012,dehmer_history_2011,kontoyiannis_compression_2020}, or structured images \cite{sanchez_symmetry-based_2009,amraee_compression_2011,mitra_symmetry_2013,gnutti_representation_2015,bairagi_symmetry-based_2015}.
We, instead, use invariance of the tasks  for lossless prediction.

\paragraph{Neural lossy compression}
Most research in neural compression is either focused on estimation and optimization of the rate term \cite{chen_variational_2017,minnen_joint_2018,johnston_computationally_2019,yang_improving_2020,yang_variational_2020,minnen_channel-wise_2020,lee_context-adaptive_2019,agustsson_universally_2020} or on developing perceptually meaningful distortions \cite{blau_rethinking_2019,chen_perceptually_2020,agustsson_generative_2019,mentzer_high-fidelity_2020}.
Our paper also develops a new distortion, but does not optimize for perception.
Improvements in the rate objectives are orthogonal to our work and can also help our method.



\paragraph{Self-supervised learning}
Our objective (\cref{eq:unconstrained_no_Mx} in main text)  can be seen as contrastive SSL \cite{oord_representation_2019,chen_simple_2020} with an information bottleneck,
a version of \cite{zbontar_barlow_2021,bardes_vicreg_2021} with an information instead of a variance bottleneck,
a SSL VIB \cite{alemi_deep_2017}, or
an invariant VAE \cite{kingma_auto-encoding_2014}.
At a higher level our work differs on two key aspects.
First, minimizing the information  arises from our desire to perform compression rather than to (optimally \cite{dubois_learning_2020}) help generalization \cite{shamir_learning_2010,vera_role_2018}.
Second, we provide the first formalism of a minimal pretext task  that retains all information about any invariant task.
This is related to the multi-view literature, where one only needs to retain information which is invariant across views \cite{sridharan_information_2008,tosh_contrastive_2021,lee_predicting_2020,tsai_self-supervised_2021}.
The main difference is that we prove the existence of a single pretext task.

The most similar setting to ours is the recent work of \citet{mitrovic_representation_2021}, which (in Appx. D) analyses contrastive learning using equivalence relations. 
Specifically, they also consider tasks  whose conditional distribution are invariant to an equivalence relation.
Their Theorem 1 is then similar to our \cref{lemma:desintegration}, but only considers the restricted case of deterministic labeling and finite sample space . Furthermore, they only talk about invariant representations (sufficiency), while we characterize all invariant representations using the maximal invariant (necessity and sufficiency).




\paragraph{Information theory and predictions}
\cref{thm:rate_invariance_distortion} relates exactly predictive loss and compression rate.
Although such results is to our knowledge (surprisingly) new, it fits in a long line of work that relates Bayes predictions and generalized information theory \cite{degroot_uncertainty_1962,grunwald_game_2004,gneiting_strictly_2007,duchi_multiclass_2018,farnia_minimax_2016,dubois_learning_2020,xu_minimum_2020}. 


\paragraph{Maximal invariants and minimal sufficient statistics}
As seen by our coin toss example, if the marginal  is invariant to the equivalence, \ie, , then maximal invariants coincide with minimal sufficient statistics  \cite{halmos_application_1949,bahadur_sufficiency_1954}.
In our work we are interested in predicting a target  rather than reconstructing the source . 
A sufficient statistic w.r.t. to another r.v.  is referred to as adequate statistics.\footnote{The standard definition of adequacy from \citet{skibinsky_adequate_1967} also requires the statistic to be sufficient, here we use ``adequacy in the wide sense'' as defined by \citet{takeuchi_characterizations_1975}}
Maximal invariants can thus be seen as minimal adequate statistics for the set of all invariant tasks of interest .
Using minimal adequate statistics as good representations for performing a task has been well investigated in ML to improving generalization \cite{shamir_learning_2010,jiang_learning_2017,cvitkovic_minimal_2019,achille_information_2018,soatto_visual_2016,dubois_learning_2020}.
The main difference with our work is that 
\begin{inlinelist}
\item we consider adequacy for a collection of tasks instead of a single task;
\item minimality arises from a compression perspective rather than for generalization.
\end{inlinelist}
Although we are not aware of any use of minimal adequacy for compression (even single task), minimal sufficiency is often used for compressing distributions \cite{hayashi_minimum_2018,iri_fine_2019}.











\clearpage
\newpage 
\section{Reproducibility}
\label{appx:reproducability}
In this section we provide further details of the hyperparameters chosen for the various experiments in the main text.
The code to reproduce all experiments can be found at \codeurl{}.
We checkpoint and use the model which achieves the smallest \textit{validation} loss for evaluation.
Unless stated otherwise, all the models are trained for 100 epochs, using Adam \cite{kingma_adam_2015} as the optimizer, and a batch-size of 128. 
The learning rate starts at  that decreases exponentially until reaching  at the end of training.
For all convolutional layers we use Kaiming normal initialization\cite{he_delving_2015}, for all linear layers we use Kaiming uniform initialization\cite{he_delving_2015},  while all biases are always initialized at 0.
Activation functions are ReLUs while other unspecified parameters are PyTorch \cite{paszke_pytorch_2019} defaults.
For our invariant models, instead of optimizing  we optimize , which is a more standard formulation for VIB, VAE, and neural compressors.
In the following sections we will sometimes refer to  as .

\subsection{Banana}
\label{appx:reproducability_banana}

For the Banana dataset most of the arguments were selected so as to replicate Fig.1.B. from \cite{balle_nonlinear_2020}. 
\footnote{Their code can be found at \url{https://github.com/tensorflow/compression/blob/master/models/toy_sources/toy_sources.ipynb}} 

\paragraph{Data}
The data distribution is obtained by starting from a bivariate Gaussian .
It is then transformed to a banana distribution using the following transformation: .
We then rotate it and shift it: .
For every epoch we resample  new points, \ie, examples are never seen twice during training).

\paragraph{Hyperparameters}
For all Banana experiments we use a 2 dimensional representation , a learning rate of  that decreases exponentially until reaching  at the end of training, and a batch size of .
The encoder (and decoder if there is one) is always a 2-hidden layer MLP with 1024 hidden neurons, and softplus activation.
In all cases we an entropy bottleneck with a factorized prior from \cite{balle_variational_2018}.

\paragraph{Experiment: \cref{fig:bananas_sweeps}}
We train both a standard variational compressor (VC) and our variational invariant compressor (VIC).
The downstream performance loss is the MSE when predicting the maximal invariant.
In both cases we use , which was chosen so that the downstream performance is similar for both.
For VIC the data is first augmented using rotations, passed through an encoder, then the decoder predicts the maximal invariant , \ie, the point with the same radius but positioned at 225 degrees.
Note that we use this maximal invariant (instead of the more natural ) to ensure that the reconstructions (codebooks) can be plotted in in a nice way in the original space .
The choice of maximal invariant does not impact the learned partition of the space.

Each plot (\cref{fig:bananas_sweeps} right) is generated by first taking a meshed grid of  source points in .
Then we quantize every point in the mesh by passing it through our encoder.
The partition of the space (delimited with pink contours) corresponds to all points in the mesh that got mapped to the same quantized representation.
To obtain the codebook (pink dots), we pass the quantized representations through our learned decoders.
Finally, we plot the distribution of our learned entropy model by rescaling the codes so that their area is proportional to the rate assigned by the entropy model, \ie, .

To obtain rate-invariance curves (\cref{fig:bananas_sweeps} left), we  sweep over .
For each point in \cref{fig:bananas_RI} we plotted the average over 5 seeds and plotted in gray the standard errors (both in the rate and distortion direction).
To compute the area under the curve we used the trapezoidal rule on each of the RI curves obtained by a single seed, we then aggregated to area under the curve for the 5 seeds to obtain the mean and standard error.


\paragraph{Experiment: \cref{fig:bananas_xtrnslt}}
Here augmentations are translations on the -axis.
The BINCE model was trained using \cref{alg:BINCE}, \ie , without assuming knowledge of the maximal invariant.
For VIC we used  as the maximal invariant.

\paragraph{Experiment: \cref{fig:bananas_ytrnslt}}
Here augmentations are translations on the -axis.
For VIC we used  as the maximal invariant.
To plot of the induced distribution in  (here the -axis), we sample  new points, pass them through our encoder and decoder to obtains the codes, and then plot a histogram of the obtained codes (shown in salmon).
In blue we also plot the (approximate) distribution of the source when marginalized our invariances (\ie only consider the  component).

\subsection{General Image framework}
\label{appx:reproducability_general}

Here we discuss the framework which we used for most of our image experiments. Unless stated otherwise, for all (ours or standard) neural image compressors we use the same general framework and architectures.

Specifically, an image  first passes through a ResNet18 \cite{he_deep_2016} to obtain a 128 dimensional representation  that t.v.i .
We then pass  through an entropy bottleneck with a scaled hyperprior entropy model from \citet{balle_variational_2018} which gives us the quantized .
For our entropy bottleneck we used \citepos{begaint_compressai_2020} implementation which is a Pytorch re-implementation of \cite{balle_variational_2018}.
Note that the choice of entropy model and quantizer is orthogonal to our work, and any choice that works neural compression would work for us.

In the case where we have to decode an image (VIC and VC models), we pass the quantized  through a linear layer to reshape it to a latent image in .
The latent image then passes through a 4-layer transposed CNN decoder where after each layer the number of channels gets divided by two and the width and height of the image doubles. 
The last layer outputs an image with the correct number of channels (1 for MNIST, 3 for other datasets), which is treated as the reconstruction  of the augmented input (for standard compressors) or the non-augmented input (for our VIC).

To simulate how well you could perform on downstream tasks of interest (that are not known when learning the compressors), we evaluate how well a model can classify the labels from the dataset.
Specifically, once the models are trained we freeze them, apply them to the dataset and train neural network to classify the inputs using either the quantized representation  or the reconstruction .
In the former case we a  MLP with preactivation batch normalization \cite{ioffe_batch_2015}.
In the latter case we use a ResNet18 for predictions.

Finally, we obtain the desired bit-rate by considering the expected log loss of the trained entropy model on the test distribution (\ie theoretical bit-rate).
The desired distortion is obtained by evaluating the predictor on the compressed test dataset.



\subsection{MNIST}
\label{appx:reproducability_mnist}

For our MNIST \cite{lecun_gradient-based_1998} experiments we compare again our VIC (as described in \cref{alg:vic}) against a standard neural compressor.

\paragraph{Data}
In order to evaluate our framework in a relatively well understood setting we use the well known  MNIST \cite{lecun_gradient-based_1998} dataset, which we rescale to  pixels.
For this toy setting we want to understand how our model performs when trained with augmentations that induce the equivalent relation w.r.t. which we are invariant, \ie, we assume that we know the ``correct'' augmentation.
To do so we augment both the training and the test set in the same way.
Specifically, we apply random rotations sampled from  degrees, random translations between  percentage of pixels, random shearing between  degrees, and random scaling by a factor in . 

\paragraph{Experiment: \cref{fig:mnist_intro} and \cref{fig:augmnist++_RD_rec}}
For a fair comparison we took trained a standard compressor and a VIC so that the downstream accuracy on augmented MNIST is the closest possible to  accuracy (note that augmented MNIST is slightly harder than standard MNIST).
We then randomly sampled reconstructions for the source, standard reconstructions, and VIC reconstructions which we plot.
The quantitative results are average over 5 runs and standard errors are provided in \cref{fig:augmnist++_RD_rec}.

\paragraph{Experiment: \cref{fig:augmnist++_RD}}
For the rate-error curve we swept over  and plotted the curves and computed the area under the curve in the same way as previously discussed for the Banana rate-invariance curve.




\subsection{STL10}
\label{appx:reproducability_stl10}

\paragraph{Data}
We use the STL10 dataset \cite{coates_analysis_2011} which is similar to CIFAR but with fewer labeled training examples. There are 10 classes of 96x96 pixeled, colored images. There are 500 labeled and 100000 unlabeled examples for training, 800 labels for test. Note that the unlabeled images come from a broader distribution of images.
For augmentations we use horizontal flips, resizing and cropping, the color transform and we randomly transform the image to gray scale with a likelihood of .
As for MNIST we augmented both the train and the test distribution.
The compressors were trained on the unlabeled data, while the predictors were trained on the train distribution.

\paragraph{Hyperparameters}
We used an entropy bottleneck with a scaled hyperprior entropy model from \cite{balle_variational_2018}.  
When training with BINCE, VIC or VC, the encoder is a ResNet18 architecture. For hyper-parameter tuning we randomly sampled 100 hyperparameters from the following search space:  latent dimension size , rate-distortion trade-off  , the optimizer's (ADAM) learning rate , the learning rate schedule(exponential decay or cosine decay), and the batch size .
For prediction on the learned features we trained an MLP with  hidden units, one or two layers, and dropout probability between . We optimized again the learning rate of the ADAM optimizer as before.
For predictions from the reconstructions  , we trained a ResNet18, with the same optimizer parameters as above.

\paragraph{Experiment: \cref{table:distortion_variation}}
In this experiment we compare the compression performance of PNG \cite{graphics_png_isoiec_2003}, WebP \cite{webp_google_2018}, JPEG \cite{group_jpeg_itu-t_1992}, VC, VIC and BINCE. 
Since VC and VIC allow to predict either on features or on reconstructions we test both. We sweep uniformly over the log-scale of  for the neural compressors and sweep the classical compressors over an equivalent quality range. 
The extensive results from which \cref{table:distortion_variation} is derived are in \cref{table:distortion_variation_long}. The rate-distortion curves belonging to this experiment are \cref{fig:STL10_dist}.
The rate-distrotion curves correspond to the pareto optimal curves of the encoders and predictors from the 1000 models.



\subsection{Galaxy Zoo}
\label{appx:reproducability_galaxy}
 \paragraph{Data}
Celestial objects and events emit radio frequencies. These frequencies are recorded through large antennas. 
Modern radio astronomy relies on the aggregation of radio signals in time and space. This means that one antenna records over long stretches of time. Due to the rotation of the earth, this translates to many spatial measurements. Further, the inclusion of many antennas in various locations can provide a dense net of observations. The entire system is refereed to as aperture synthesis telescope (AST). 
Images of the sky are generated by combining the sequences of observations stemming from different antennas. 
ASTs generate enormous amounts of data, much of which is redundant and further will never be observed by humans.
In fact commonly applied techniques to the observations, such as weighting (e.g. Briggs weighting) and blurring of signals removes information from the original observations. 
Our approach is thus a natural extension to the techniques already present in the radio astronomy community.
However, the process of image reconstruction from radio frequency observation series is too complex for the scope of this paper. 
We thus work on the Galaxy Zoo 2 (GZ2) dataset, that contains of already inferred images of celestial objects. 
We believe that good rates on this dataset should hint at even better possible rates when working directly with the raw data.
GZ2 contains 37 classification tasks, such as answering queries about shapes and counts information of galaxies.
Although the tasks are classification tasks, we use the standard GZ2 evaluation that consists in regressing (RMSE evaluation) the expected (over different labellers) label probability.
Our data is hosted on the kaggle platform.
This means we have no access to test labels but only for total test loss. 
This is why we compute summary statistics on the validation data set.
We choose to reduce the original dataset by center cropping to 256 pixels per dimension.
We applied random rotations, horizontal and vertical flips, scaling () and color transforms to this data. 
We used CNN encoders \cite{balle_end--end_2017,minnen_joint_2018} and an entropy bottleneck with a scaled hyperprior entropy model from \cite{balle_variational_2018}.
 
\paragraph{Hyperparameters}
For all experiments we used ResNet50 when predicting from images (\ie encoders and predictors from reconstructions).
As with the STL10 experiments, we trained each model and baseline by selecting a set of 100 hyperparameters randomly selected from a large search space.
When training with BINCE, we sampled the latent dimension size , rate-distortion trade-off  , the optimizer's (ADAM) learning rate , the learning rate schedule(exponential decay or cosine decay) and the batch size . For prediction on the learned features we would train an MLP with 2048 hidden unit, two layers and dropout probability . We optimized the again the learning rate of the ADAM optimizer as before.
For the classical compressors we trained a ResNet50 on their reconstructions, with the same optimizer parameters as above.

\subsection{Pretrained CLIP}
\label{appx:reproducability_clip}

\ydnote{add lambda and beta}

\paragraph{Data}
In addition to the pretrained CLIP, we trained the entropy bottleneck.
As we do not have access to the dataset from CLIP, we could not train the entropy bottleneck on the initial data.
Instead we had to use a different dataset.
We used MSCOCO \cite{lin_microsoft_2015} for image captioning, as we initially thought that we would need access to pairs of images and sentences to finetune CLIP.\footnote{
At the end we did not use the sentences as freezing CLIP worked very well.}
Note that in our experiments the choice of dataset for training the entropy bottleneck (\eg CIFAR10 \cite{krizhevsky_learning_2009}) had very little impact on the quality of the final compressor.

To evaluate our compressor in the most realistic setting possible, we selected 10 different datasets.
The datasets were chosen so that
\begin{inlinelist}
\item the source images are of very different shapes and content;
\item they are easily be accessible online;
\item images are already compressed by JPEG;
\item neither the entropy bottleneck nor CLIP should have been trained on the selected datasets;
\item the task are very different classification tasks.
\end{inlinelist}
To ensure that CLIP was (nearly) not pretrained on the selected datasets we selected a subset of the datasets that CLIP was evaluated on and which did not show significant data overlap (see \citepos{radford_learning_2021} section 5 for a discussion about data overlap).
\Cref{table:clip_data} shows the details about the 10 datasets that we use for evaluating our model.
When there is no prespecified validation split, we randomly sampled  of the training data for validation.

\begin{table}[h]
\caption{
Datasets used to evaluate our zero-shot compressor. -1 for the shape means variable. 
}
\begin{center}
\begin{tabular}{lrrrrrr}
\toprule
Dataset & Classes  & Train size & Valid size & Test size & Metric & Shape   \\ 
\midrule 
ImageNet \cite{deng_imagenet_2009} & 1000 &  1,281,167 &    & 50,000 & accuracy & (-1,-1,3) \\
CIFAR10 \cite{krizhevsky_learning_2009} & 10 & 50,000 & & 10,000 & accuracy & (32,32,3) \\
CIFAR100 \cite{krizhevsky_learning_2009} & 100 & 50,000 & & 10,000 & accuracy & (32,32,3) \\
Cars196 \cite{krause_3d_2013} & 196 & 8,144 &  & 8,041 & accuracy & (-1,-1,3) \\
Pets37 \cite{parkhi_cats_2012} & 37 & 3,680 &  &  3,669 & balanced acc. & (-1,-1,3) \\
Caltech101 \cite{li_learning_2007} & 102 & 3,060 & & 6,085 & balanced acc. & (-1,-1,3) \\
Food101 \cite{nilsback_automated_2008} & 101 & 75,750 &   & 25,250 & accuracy. & (-1,-1,3) \\
STL10 \cite{coates_analysis_2011} & 10 &  5000 & &  8000 & accuracy & (96,96,3) \\
PCam \cite{veeling_rotation_2018,ehteshami_bejnordi_diagnostic_2017} & 2 &  262,144 & 32,768 & 32,768 & accuracy & (96,96,3)  \\
\bottomrule
\end{tabular}
\end{center}
\label{table:clip_data}
\end{table}
 
\paragraph{Reproducing the results}
For clarity and reproducebility we also provide a self contained script to train a very similar version to our compressor in \cref{code:array_compressor} and \cref{code:clip_code}.
The main changes being that we change the training data (using CIFAR10), the entropy bottleneck (to the simpler factorized prior from \cite{balle_variational_2018}), and use a simpler evaluation pipeline (only use STL10 with a simplified MLP).
The entire script (including evaluation and actual compression of a dataset) takes less than ten minutes to run on a single GPU and provides a general zero-shot compressor.
The full code that we used is accessible at \codeurl{}.

\paragraph{Training the zero-shot compressor}
To train our compressor we first download the official pretrained CLIP model\footnote{https://github.com/openai/CLIP}.
Specifically the vision transformer \cite{dosovitskiy_image_2020} that they refer to as \texttt{"ViT-B/32"}.
We then freeze it, and add an entropy bottleneck with \citepos{balle_variational_2018} hyperprior.
We then train the entropy bottleneck on the MSCOCO dataset.

To train the entropy bottleneck we need a distortion measure.
In theory, to get our BINCE objective, we should use the distortion that CLIP was trained with, \ie, we should compress the representation in such a way that CLIP can still distinguish examples from the same equivalence class. Minimizing such distortion can lead to catastrophic forgetting as the representations only ensure that CLIP can distinguish equivalent example from our very small dataset. \footnote{We tried many ways of finetuning CLIP with very small learning rates and frozen components, but although the rate gains were large (around  to ) this lead to significant decrease in performance, most probably due to catastrophic forgetting.}
We instead use a very simple MSE distortion in the representation space.
Specifically, we trained the entropy bottleneck to minimize , where  denotes the reconstructed (quantized) representation.
This can be seen in line 22 of \cref{code:array_compressor}.

One important point to notice is that in standard neural compressors the quantization is  a component wise rounding to the closest integer.
This typically does not constrain the compressor, as the compressor is trained in an end-to-end fashion so that the encoder can increase or decrease some components of  to effectively increase or decrease the quantization.
As our encoder is frozen, it cannot learn to adaptively change the scale of  so we needed to learn the size of the quantization interval instead, \ie, the rounding precision.
A simple (and equivalent) way of doing that consists in passing the representation through a (learned) component wise linear transformation (\ie 2 parameters per component) then through the entropy bottleneck (quantization) and finally we reverse the linear transformation.
This can be seen in line 12 and 14 of \cref{code:array_compressor}.

Generally we found that training the entropy model was very robust to hyperparameter changes. 
We used the following:  epochs, a 512 dimensional  (given by CLIP), a batch size of , a learning rate of  with 3, a  scheduler that decreases the learning rate by  every 12 epochs (\ie uniformly 3 times during training), Adam with decoupled weight decay (AdamW; \cite{loshchilov_decoupled_2019}) as an optimizer, weight decay of  and a  dimensional side information for the hyperprior.
For the our main CLIP compressor (CLIP+EB) we use an RD hyperparameter  which is linearly annealed from  to  in the first  epochs of training (although annealing did not seem important).
For our CLIP+EB\textsuperscript{} and CLIP+EB\textsuperscript{} (see \cref{table:clip_all}) we respectively use  and .
For data augmentations we used similar ones as used by CLIP namely, normalizing the image by the mean and std form their training dataset (\texttt{mean=[0.48145466, 0.4578275, 0.40821073]} and \texttt{mean=[0.26862954, 0.26130258, 0.27577711]}), resizing the smallest side of the image to  pixels with bicubic interpolation, then taking a random  crop.
During the evaluation the random cropping is replaced by a center cropping.


\paragraph{Evaluating the zero-shot compressor}
For evaluating the rate obtained by our compressor, we provide the negative log likelihood of our entropy model for the each test dataset.
For evaluating the downstream predictive performance for each dataset we train a 2 hidden layer MLP of dimensions  with batch normalization and ReLU activation.
For each dataset we provide the best model from 25 randomly sampled models, that arise by randomly sampling the following hyperparameters:
\begin{itemize}
\item batch size:  with logarithmic sampling.
\item optimizer: Adam, SGD, AdamW.
\item weight decay:  with logarithmic sampling.
\item learning rate:  with logarithmic sampling.
\item scheduler: exponential decay (with total decay by 100 or 1000), decreasing learning rate on validation loss plateau, cosine scheduler, decaying learning rate at fixed intervals.
\item dropout \cite{srivastava_dropout_2014}: .
\end{itemize}

We then provide the result of the best model. In \cref{table:clip} we compare those results to the same vision transformer that we use for CLIP, but trained directly on the raw images.
These results were obtained from \citepos{radford_learning_2021} table 10.
For a better comparison to standard SSL models, in \cref{table:clip_all} we also provide the test accuracy of a linear layer (a support vector machine) from the representations. The regularization parameter of the SVM were all selected using 10 values and three fold cross validation.
For the rates we compare to the average JPEG size of images in each datasets (all the selected datasets are compressed by default in JPEG).
For the rates of the raw CLIP model we losslessly compress the representation using numpy's \texttt{savez} function (zip) \cite{harris_array_2020}.

\subsection{Minimal code to train the CLIP compressor in < 5 min.}
\label{appx:code_clip}

In this section we provide minimal code to train our zero-shot compressor and to use our compressor to entropy code an entire dataset.
Note that the model is simplified (\eg using factorized prior instead of a hyperprior, and training on CIFAR10) so the bit-rates is slightly increased but it still achieves orders of magnitude gains compared to JPEG. 
We use CIFAR10 for training the entropy coder and STL10 for downstream evaluation (as both are downloadable through torchvision).
To evaluate the model, we use a linear support vector machine from our representation .

For this minimal code,
training takes around  minutes on a single GPU.
The theoretical bit-rate that we achieve is around , while the practical bit-rate achieved by entropy coding is around .
In comparison the bit-rate of JPEG (with 95 quality) is 4.71e4.
The entropy coder compresses around  images per seconds, and decompresses around around  images per seconds. Decompression is slow as we do not perform it in batch (for simplicity of the code), while encoding is batched processed.
Downstream classification accuracy on STL10 is  which is better than the uncompressed representations from CLIP, from which linear probe achieves  accuracy.

To run the code you need first need to install the following libraries:
\begin{minted}{bash}
pip install git+https://github.com/openai/CLIP.git
pip install scikit-learn==0.24.2 lightning-bolts==0.3.3 compressai==1.1.4
\end{minted}


The minimal boilerplate code (\cref{code:clip_code}) downloads the training data and the pretrained CLIP (from line 42), trains the compressor (from line 46), entropy code the evaluation data (from line 57), and finally evaluates downstream performance (from line 67).
The actual compressor is defined in \cref{code:array_compressor}.


\clearpage
\begin{listing}
\vspace{-2\baselineskip}
\inputminted[
fontsize=\scriptsize,
frame=lines,
linenos
]{python}{figures/clip/clip.py}
\caption{Minimal boilerplate code for training a zero-shot compressor in less than 5 minutes.
For the actual compressor (\texttt{ArrayCompressor}) see \cref{code:array_compressor}.
}
\label{code:clip_code}
\end{listing} 
\clearpage
\begin{listing}
\inputminted[
fontsize=\scriptsize,
frame=lines,
linenos
]{python}{figures/clip/array_compressor.py}
\caption{Minimal code for training an entropy bottleneck to convert a pretrained SSL model into a powerful zero-shot compressor.
For the training and evaluation code see \cref{code:clip_code}.
}
\label{code:array_compressor}
\end{listing} 


\clearpage
\newpage 
\section{Additional experimental results}
\label{appx:results}
\subsection{Banana}
\label{appx:banana}

In \cref{sec:toy_experiments} we compared a classical compressor to our VIC in the case of rotation invariant tasks. 


Here show results for invariances to different equivalences and provide more intuition as to what BINCE and VIC achieve.


\begin{figure}[h]
     \centering
     \begin{subfigure}[h]{0.25\columnwidth}
         \centering
         \includegraphics[width=\textwidth]{figures/banana/quantization_vae.png}
         \caption{Standard Compression}
         \label{fig:bananas_xtrnslt_vae}
     \end{subfigure}
\begin{subfigure}[h]{0.25\columnwidth}
         \centering
         \includegraphics[width=\textwidth]{figures/banana/quantization_ivae_xtrnslt.png}
         \caption{VIC}
         \label{fig:bananas_xtrnslt_ivae}
     \end{subfigure}
\begin{subfigure}[h]{0.25\columnwidth}
         \centering
         \includegraphics[width=\textwidth]{figures/banana/quantization_nce_xtrnslt.png}
         \caption{BINCE}
         \label{fig:bananas_xtrnslt_bince}
     \end{subfigure}
\caption{
VIC and BINCE improves compression of Banana distribution when downstream tasks are invariant to translation on the -axis by quantizing the space into horizontal stripes.
(a) standard compression with a rate of  bits and an invariant distortion of  ; (b) our VIC with a rate of  bits and an invariant distortion of .
(c) our BINCE with a rate of  bits and an invariant distortion of .
}
\label{fig:bananas_xtrnslt}
\vspace{-0.5em}
\end{figure} 
\paragraph{-translation and BINCE}
\Cref{fig:bananas_xtrnslt} considers the case where downstream tasks are invariant to -translations.
We used  as the maximal during training.
We see that our model can essentially perform as well on all downstream tasks for only  of the bit-rate.
Unsurprisingly we see that the codebook is in shape of horizontal stripes as these can cover the entire distribution with a few codes (small bit rate) while incurring a small invariance distortion (which only depends on the  value).

We visualized a BINCE model (\cref{fig:bananas_xtrnslt_bince}) in addition to VIC.
Although the exact partition for both models is quite different (BINCE does not seem to learn equal sized partitions), both models clearly learn to partition the space into horizontal stripes.
Once important difference, is that VIC also provides a codebook (shown with pink dots), as it can reconstruct a quantized version of the input, while BINCE only learns a latent representation and does not provide any reconstructions.

\begin{figure}[h]
     \centering
     \begin{subfigure}[h]{0.24\columnwidth}
         \centering
         \includegraphics[width=\textwidth]{figures/banana/quantization_vae.png}
         \caption{Standard}
         \label{fig:bananas_ytrnslt_vae}
     \end{subfigure}
     \hfill
     \begin{subfigure}[h]{0.24\columnwidth}
         \centering
         \includegraphics[width=\textwidth]{figures/banana/Mx_vae_ytrnslt.png}
         \caption{Standard }
         \label{fig:bananas_ytrnslt_vae_Mx}
     \end{subfigure}
     \hfill
     \begin{subfigure}[h]{0.24\columnwidth}
         \centering
         \includegraphics[width=\textwidth]{figures/banana/quantization_ivae_ytrnslt.png}
         \caption{-translation VIC}
         \label{fig:bananas_ytrnslt_ivae}
     \end{subfigure}
     \hfill
     \begin{subfigure}[h]{0.24\columnwidth}
         \centering
         \includegraphics[width=\textwidth]{figures/banana/Mx_ivae_ytrnslt.png}
         \caption{VIC }
         \label{fig:bananas_ytrnslt_ivae_Mx}
     \end{subfigure}
\caption{
VIC improves compression of Banana distribution when downstream tasks are invariant to translation on the -axis by (implicitly) estimating the density .
(a) standard compression with a rate of  bits and an invariant distortion of  ; 
(b) the induced marginal distribution  of the -value of the reconstructions from the standard neural compressor;
(c) our compression with a rate of  bits and an invariant distortion of .
(b) the induced marginal distribution  of the -value of the reconstructions from the VIC;
}
\label{fig:bananas_ytrnslt}
\vspace{-0.5em}
\end{figure} 
\paragraph{-translation and induced distribution}
\Cref{fig:bananas_ytrnslt} considers the case where downstream tasks are invariant to -translations.
In this case the maximal invariant used during training is chosen to be .
Similarly to the case of -translation and rotations, we see that our model can perform as well as a standard compressor for a fraction of the rate.

To provide a better intuition as to why this is the case we also plot the distribution of the reconstructions when marginalized over the -axis.
In other words we plot the distribution of  when applied to the reconstructions, \ie, the  component of the reconstructions.
We see that although the partition of the source space is very different for a standard compressor (\cref{fig:bananas_ytrnslt_vae}) and for our VIC (\cref{fig:bananas_ytrnslt_ivae}), the induced distribution (and partition) in the marginalized space are actually very similar (\cref{fig:bananas_ytrnslt_vae_Mx} and \cref{fig:bananas_ytrnslt_ivae_Mx}).
This shows where our bit-rate gains come from. 
Indeed from \cref{prop:nicer_dist} we know that in the case of invariant tasks one only needs to model the distribution of  (\eg the distribution of the  component here), and we see that both the standard compressor and VIC does that similarly well.
The main difference being that VIC does so in an optimal way while the standard compressor needs to partition the input space in a finer way to achieve a similar induced partition in the  space.




















\paragraphQ{What is the relation between rate and predictions}
\Cref{thm:rate_invariance_distortion} shows that, for log loss, the minimum rate is linearly related to the loss  in downstream performance.
Our theory (\cref{appx:theorem_mse}) suggests a logarithmic relationship for MSE.
This is seen for VIC and VC in \cref{fig:bananas_sweeps} of the main text (log scale -axis).

\paragraph{On lossy compression and equivalences}
Efficient lossy compression is about learning a partition (\eg Voronoi diagrams, or  \cref{fig:bananas_xtrnslt} ) of the input space to map many inputs to the same code.
We use the fact that any partition can be constructed from an equivalence relation \cite{schechter_handbook_1996} to learn compressors that are invariant to desired transformations.
The shape of the partitions are then induced by the transformations, which perturb points in their quantization bins (equivalence classes), \eg, rotations in \cref{fig:bananas_sweeps} of the main text.
The size of the partition, \eg, disks width in \cref{fig:bananas_sweeps} of the main text, depend on the desired performance .
The pink dots are representatives of the partition, \ie, maximal invariants.
The key is that using our objectives we can learn arbitrary quantization using only desired transformations, which ML practictioners already use for data augmentations.


\subsection{MNIST}
\label{appx:mnist}





\begin{figure}[ht]
     \centering
     \begin{subfigure}[h]{0.4\columnwidth}
         \centering
         \includegraphics[width=\textwidth]{figures/augmnist_plus/RD_curve_workshop_error.png}
         \caption{Rate-Error}
         \label{fig:augmnist++_err}
     \end{subfigure}
     \hfill{}
     \begin{subfigure}[h]{0.4\columnwidth}
         \centering
         \includegraphics[width=\textwidth]{figures/augmnist_plus/RD_curve_workshop_loss.png}
         \caption{Rate-Log loss}
         \label{fig:augmnist++_log}
     \end{subfigure}
\caption{
Augmented MNIST RD curves for downstream predictions are very similar when using classification error (left) instead of the log loss (right) from our theory.
}
\label{fig:augmnist++_err_vs_log}
\end{figure} 
\paragraphQ{How do RD curves change if we use classification error instead of log loss}
Our theory \cref{thm:rate_invariance_distortion} only ensures good downstream log loss risk.
Nevertheless, we used here the classification error () throughout the main test as it is more commonly used for evaluating classification performance.
\Cref{fig:augmnist++_err_vs_log} shows that RD curves are very similar for when using classification error instead of accuracy.
This is not very surprising as log loss is the standard (differential) proxy of classification error in ML.

\begin{table}[h]
\caption{
Using label-preserving augmentations that remove more information about  decreases the rate without hindering classification performance.
Single run.
}
\small
\center
\begin{tabular}{llrr}
\toprule
&Augmentations& Rate  & Test Acc.   \\ 
\midrule 
\multirow{2}{*}{\centering  ~VIC } 
 & Small set  &    &    \\ 
  & Large set  &    &    \\ 
   & Supervised (largest set)  &    &    \\ 
\midrule 
\multirow{2}{*}{\centering  ~BINCE  }
 & Small set  &    &    \\ 
  & Large set  &    &    \\ 
   & Supervised (largest set)  &    &    \\ 
\bottomrule
\end{tabular}
\label{table:mnist_augmentations}
\end{table} 

\paragraphQ{What is the impact of the choice of augmentations}
The rate decreases when  removes more information from .
Indeed, we have , where the second equality comes from \cref{assumption:augmentations} and the last equality from the determinism of .
As a result we can rewrite \cref{thm:rate_invariance_distortion} as .
To illustrate this we trained our VIC and BINCE using three augmentation sets on MNIST, all of which keep the true label invariant but progressively discard more  information:
\begin{inlinelist}
\item standard image augmentations such as random translations, shears, and rescalings;
\item those same standard image augmentations, but drawn from larger ranges of possible translations, shears, scales, etc;
\item supervised ``augmentations'' line in \cite{khosla_supervised_2020} that remove everything except label information, \ie, for every image   let  be a random image with the same label.
\end{inlinelist}
\Cref{table:mnist_augmentations} shows that using label-preserving augmentations that remove more information about  greatly decreases the rate without hindering classification performance.
The fact that the supervised augmentations achieve a much better rate, shows that typical SSL compression is still very far from single-task label compression.
SSL compression retains information for at least  disjoint labels.


\begin{table}[h]
\caption{
End-to-end compression of augmented MNIST works much better than staggered compression for both VIC and BINCE. Single run.
}
\small
\center
\begin{tabular}{llrr}
\toprule
&& Rate  & Test Acc.   \\ 
\midrule 
\multirow{2}{*}{\centering  ~VIC } 
 & Staggered  &    &    \\ 
  & End-to-end &    &   \\  
\midrule 
\multirow{2}{*}{\centering  ~BINCE  }
 & Staggered &    &   \\
   & End-to-end &    &    \\ 
\bottomrule
\end{tabular}
\label{table:end2end}
\end{table} 

\paragraphQ{How much does end-to-end improve compared to staggered training}
We evaluated end-to-end training for both our losses against a staggered version that consists in first optimizing the distortion and then adding an entropy bottleneck to performing lossy compression of the learned representations (as in \cref{sec:clip_experiments}).
\Cref{table:end2end} shows that end-to-end training can give large gains compared to the staggered method and that our compression gains with CLIP could be even further improved.


\subsection{STL10}
\label{appx:stl10}
In the main text we used the STL10 data set to answer some principled questions about our method in controlled experiments.
We provide additional results here.


\begin{figure}[h]
     \centering
     \begin{subfigure}[h]{0.45\columnwidth}
         \centering
         \includegraphics[width=\textwidth]{figures/STL10/rd_dists_feat.pdf}
         \caption{From features }
         \label{fig:STL10_dist_Z}
     \end{subfigure}
     \hfill{}
     \begin{subfigure}[h]{0.49\columnwidth}
         \centering
         \includegraphics[width=\textwidth]{figures/STL10/rd_distsrec.pdf}
         \caption{From reconstruction }
         \label{fig:STL10_dist_X}
     \end{subfigure}
\caption{
BINCE achieves the best RI curves followed by VIC and then VC for STL10 data.
Rate-error curves when predicting downstream tasks from: (left) compressed representations , (right) reconstructions .
}
\label{fig:STL10_dist}
\end{figure} 
\ydnote{the fact that the font in \cref{fig:STL10_dist} is different is very ugly. Need to solve for camera ready.}
\paragraphQ{How does the choice of distortion measures or bounds thereof affect RI curves}
Supplementing the results in the main text, we show more extensive results comparing the effect of the distortion measures (invariant or not) or bounds thereof (BINCE and VIC are different bounds on ) on RI curves.
When predicting from compressed representations   (\cref{fig:STL10_dist} left), BINCE achieves the best RI curves followed by VIC and VC.
When predicting from reconstructions  (\cref{fig:STL10_dist} right), VIC still performs a little better than VC although the gap shrinks.
\Cref{table:distortion_variation} shows all quantitative results for best achieved downstream performance (as in \cref{table:distortion_variation} from the main text).

\begin{table}[ht]
\caption{
We compare classical compression formats (PNG, JPEG, webP) to neural (VC) and invariant (BINCE, VIC) ones. 
BINCE achieves the same error rate but compresses  better.
}
\center
\small
\begin{tabular}{llrrr}
\toprule
Distortion  & Predict from & Best error   & Rate [Mb/img] & Compression factor\\ 
\midrule 
PNG \cite{graphics_png_isoiec_2003}    &  reconstructions   &  19.2 & 14.20 &  \\
JPEG  \cite{group_jpeg_itu-t_1992} &  reconstructions   &  19.9 &  4.60 &  \\
WebP   \cite{webp_google_2018}&  reconstructions   &  20.3 &   1.12 &  \\
VC     &  reconstructions   &  40.2 &   0.23 &  \\
VC     &  features          &  52.6 &   1.08 &  \\
\midrule 
VIC (ours)   &  reconstructions  &  44.3 &   0.05 &  \\
VIC  (ours)  &  features 	     &  35.3 &   0.08 &  \\
BINCE (ours) &  features         &  19.2 &   0.12 &  \\	
\bottomrule
\label{table:distortion_variation_long}
\end{tabular}
\end{table}

 
Note that VIC and VC achieve much worst downstream performance than BINCE.
Based on preliminary results, we believe that this comes from the fact that, for consistency, in all experiments we used ResNet18 encoders.
Indeed, ResNet18 have an global averaging pooling layer that averages the ``latent image'' over spatial dimensions (width and height).
As a result, the representations  does not retain any spatial information, which is often useful for improving reconstructions.
Preliminary results showed that removing this pooling layer improves downstream predictions significantly.
Importantly, this impacts both VIC and VC so although the absolute performance improved by removing this layer the relative error did not seem to.

 For all our models the we estimated (using a naive sample estimate) the mutual information  and found that .  This shows that all information about labels is retained, i.e., no images get compressed to the same Z but have different labels. The difference in test accuracy (which is also similar for training) must thus come from the fact that some information is easier to use/decode from \cite{dubois_learning_2020,xu_theory_2020}.
Indeed, our framework only concerns information retention rather than information usability. 
This is further supported by the fact that BINCE gets very good downstream accuracy, because it has been shown in theory \cite{saunshi_theoretical_2019,tosh_contrastive_2021,lee_predicting_2020} and in practice \cite{chen_simple_2020,oord_representation_2019} that contrastive representations are approximately linearly decodable.


\begin{figure}[!htb]
    \centering
    \begin{minipage}{.44\textwidth}
        \centering
        \includegraphics[width=\textwidth]{figures/STL10/rd_rates.pdf}
        \caption{The choice of variational bounds on the rate term  has little effect on RI curves for STL10 data.
        ``MI unitgaussian'' is the upper bound on mutual information used in VIB and VAE;
        ``H factorized'' is \citepos{balle_variational_2018} upper bound on  with a factorized entropy model;
        ``H hyper'' is \citepos{balle_variational_2018} upper bound on  with a hyperprior entropy model.}
        \label{fig:rate_rd}
    \end{minipage}\hspace{0.1\textwidth}
    \begin{minipage}{0.44\textwidth}
        \centering 
        \includegraphics[width=\textwidth]{figures/STL10/rd_action_distribution.pdf}
        \caption{VIC is robust to distribution shifts in the augmentations as it is invariant to the augmentations.
        Specifically, test time shifts in augmentation probability seem to have little effect on the rate-distortion curve for the case of STL10 data.}
        \label{fig:action_dist_rd}
    \end{minipage}
\end{figure}



\begin{table}[h]
\caption{Hierarchical hyperprior works worst (higher rates) for low distortion, when compared to a factorized prior and a mutual information bottleneck on STL10 data.
}
\begin{center}
\small
\begin{tabular}{llrr}
\toprule
Bottleneck & Entropy model & Lossless Rate [bits/img] & Lossless loss\\ 
\midrule 
Entropy       &Factorized prior \cite{balle_variational_2018} 		               & 598.5 	& 0.3117 \\
Entropy        &Hierarchical prior \cite{balle_variational_2018}  & 934.7  & 0.3100 \\ 
Mutual Information       & Unit Gaussian 	                                                   & 592.5  & 0.3074 \\
\bottomrule
\end{tabular}
\end{center}
\label{table:rate_variation_aurd}
\end{table} 
\paragraphQ{How does the choice of bounds on the rate term  impact RI curves}
For the main paper we always used the standard \cite{balle_end--end_2017} neural compressor's upper bound on , namely, the entropy bound .
To understand the effect of using other bounds on  and different entropy models .
Specifically, in \cref{fig:rate_rd} we compare three different bounds on mutual information:
(MI unitgaussian) the mutual information bound from VAE and VIB ;
(H Factorized) the entropy bound  where  is \citepos{balle_variational_2018} factorized entropy model;
(H Hyper) the entropy bound where  is \citepos{balle_variational_2018} hyperprior entropy model.
\ydnote{\karen{} we need to change the legend in \cref{fig:rate_rd} to be more understandable / presentable. I would use latex and the following ``'', `` factorized'', `` hyper'' }
In our experiments, however, we find that neither of these choices influence the RD curves at typical distortion levels as seen \cref{fig:rate_rd}. 
In our experiments we use ``H Hyper'', which does seem to enable very low rates (high distortions) but seems to perform worst at very high rates (see \cref{table:rate_variation_aurd})


\kunote{Todo: correctly implement and run this experiment again}

\paragraphQ{How important is the distribution over augmentations}
As discussed in the main text, RD curves of our VIC show negligible difference when the distribution of augmentation shifts from training to test time.
We provide this evidence in \cref{fig:action_dist_rd}. Here we trained a VIC compressor on data with various augmentations, if these were (jointly) applied or not would be decided by a fair coin flip. At test time, we changed the coin to be biased with . 


\ydimp{markpage}

\subsection{Pretrained CLIP}
\label{appx:clip}

\begin{table}[h]
\caption{
Converting a pretrained SSL model into a zero-shot compressor achieves substantial bit-rate gains while allowing test accuracies similar to  predicting from raw images.
CLIP refers to the original CLIP with lossless compression of the representations.
CLIP+EB refers to our CLIP compressor.
CLIP+EB\textsuperscript{} and CLIP+EB\textsuperscript{} are our CLIP compressors trained respectively for a larger and smaller bit-rate.
We provide downstream evaluation using an MLP and a linear (SVM) predictor.
Baselines: JPEG and compression of features from a ImageNet pretrained classifier (Transfer + EB).
}
\scriptsize
\center
\begin{tabular}{lllrrrrrrrrr}
\toprule
 & & & ImageNet  & STL & PCam & Cars & CIFAR10 & CIFAR100 & Food      & Pets & Caltech  \\ 
\midrule 
\multirow{5}{*}{\rotatebox[origin=c]{90}{\centering ~Rate [Bits/img]  }} 
&& JPEG & 1.49e6  & 4.71e4 & 9.60e4 & 1.92e5 & 1.05e4 & 1.05e4 & 1.54e5     & 1.81e5 & 1.69e5  \\ 
 & & Transfer + EB & 3.95e3  & 3.33e3 &3.99e3  & 3.18e3 &  3.92e3  &  & 3.26e3     & 3.70e3 & 3.40e3  \\ 
 && CLIP & 1.52e4  & 1.52e4 & 1.52e4 & 1.52e4 & 1.52e4  & 1.52e4  & 1.52e4     & 1.52e4 & 1.52e4  \\ 
& & \textbf{CLIP+EB}\textsuperscript{} & 2.47e3  & 2.46e3 & 2.61e3 & 2.59e3 & 2.53e3 & 2.54e3 & 2.39e3      & 2.33e3 & 2.46e3  \\ 
 && \textbf{CLIP+EB} & 1.35e3  & 1.34e3 & 1.49e3 & 1.47e3 & 1.41e3 & 1.42e3 & 1.27e3      & 1.21e3 & 1.34e3  \\ 
& & \textbf{CLIP+EB}\textsuperscript{} & 9.63e2  & 9.52e2 & 1.49e3 & 1.52e2 & 1.02e2 & 1.09e3 & 8.89e2      & 8.35e2 & 9.53e2  \\ \midrule 
\multirow{10}{*}{\rotatebox[origin=c]{90}{\centering ~Test Accuracies  }}
 &\multirow{2}{*}{\rotatebox[origin=c]{90}{\centering ~\cite{radford_learning_2021}  }}
 & JPEG &  76.6   & 99.0 & 82.6 & 49.1 & 96.7 & 86.3 & 81.8       & 90.4 & 94.5   \\ 
 && CLIP \cite{radford_learning_2021} & 76.1  & 98.3 & 83.9 & 81.8 & 95.1 & 80.5 & 88.8      & 90.0 & 93.0  \\ 
  \cmidrule{2-12}
 &\multirow{4}{*}{\rotatebox[origin=c]{90}{\centering ~MLP }}
 & Transfer + EB  & 72.7 & 96.1 & 79.4 & 42.0 &  87.0 & & 66.8    & 91.3 &  89.9  \\ 
 && CLIP  & 76.5  & 98.6 & 84.5 & 80.8 & 95.3 & 80.9 & 88.5      & 89.7 &  93.2  \\  
 && \textbf{CLIP+EB}\textsuperscript{} & 76.6  & 98.7 & 82.7 & 80.4 & 95.3 & 80.9 & 88.5      & 89.6 & 93.5  \\  
 && \textbf{CLIP+EB }& 76.3  & 98.7 & 80.9 & 79.6 & 95.2 & 80.1 & 88.3    & 89.5 & 93.4  \\
 && \textbf{CLIP+EB}\textsuperscript{} & 76.0  & 98.7 & 80.1 & 78.9 & 94.8 & 78.6 & 87.6      & 88.6 & 92.9  \\ \cmidrule{2-12}
  &\multirow{4}{*}{\rotatebox[origin=c]{90}{\centering Linear  }}
 & CLIP  &   & 98.6 & 83.8 & 80.8 & 95.0 & 79.8 & 85.0      & 89.3 & 93.8  \\ 
  && \textbf{CLIP+EB}\textsuperscript{} &     & 98.7 & 83.2 & 80.8 & 95.0 & 79.7 & 85.0      & 89.2 & 93.6  \\  
 && \textbf{CLIP+EB} &   & 98.7 & 81.1 & 79.9 & 94.8 & 79.0 & 83.6      & 88.3 & 93.7  \\
 && \textbf{CLIP+EB}\textsuperscript{} &   & 98.6 & 80.5 & 78.9 & 94.4 & 80.5 & 82.5      & 87.8 & 93.5  \\
\bottomrule
\end{tabular}
\label{table:clip_all}
\end{table}
 
In \cref{table:clip_all} we provide all the quantitative results for our zero-shot CLIP experiments from which we derived the tables in the main text.

We note that zero-shot compression can still be analysed using our framework. 
Indeed CLIP was trained on  sampled from a r.v.  over images on the internet.
As these datasets are on internet, they are samples from the joint  for a specific task .
One can see this as a multi-task setting (each dataset is a distinct task).


\paragraphQ{What is the effect of using a more powerful predictor from the representation}
In our framework we only discuss about information but never whether this information can easily be decoded by the predictors of interest. 
We investigated the effect of using more powerful predictors from our representation to understand how easy it is to decode the information in our representation.
In particular, we evaluated all our CLIP compressors (\ie at different ), by considering predictions from our compressed representation using a two layer MLP and using a linear classifier (SVM).
\cref{table:clip_all} shows that the advantage of using an MLP compared to a linear model is small, which suggests that our CLIP compressed representation store information in a way that is easily decodable.
This is typical from contrastive self-supervised models \cite{oord_representation_2019,chen_simple_2020}.


\paragraphQ{How does compressing SSL compare to compressing features form transfer learning}
In previous work, \citet{singh_end--end_2020} had considered the case of compressing features from single-task transfer learning instead of self-supervised method.
In \cref{table:clip_all} we compare compression of both type of features. Specifically ``Transfer + EB'' shows compression of a pretrained ResNet50 on ImageNet.
We see that It generally performs worst than our CLIP compressor both in terms of test accuracies and bit-rate.
One issue with this comparison is that the architectures of both models are not the same is likely that ``transfer+EB'' does not even perform better than CLIP on ImageNet.


\subsection{Galaxy Zoo}
\label{appx:galaxy}

\begin{table}[ht]
\small
\caption{Comparisons between pretrained CLIP BINCE, a BINCE trained end-to-end, and SOTA perceptual compressors on Galaxyzoo data. 
CLIP BINCE achieves the smallest bit-rate.
}
\begin{tabular}{llllllll}
\toprule
\textbf{Compressor}           & \textbf{rate} & \textbf{test loss} & \textbf{val. loss mean} & \textbf{median}  &  \textbf{max} & \textbf{min}                   &  \textbf{std }                                                 \\
   &      &  &  &   &  &                    &                                                              \\
\midrule
PNG              & 53.73 & 0.007   & 0.008       & 0.86        & 0.07                     & 1.04                     & 1.62   \\      
 JPEG              & 1.68 & 0.012   & 0.013        & 1.25         & 0.11                     & 1.23                     & 2.61    \\ 
 WebP              & 0.48  & 0.010  & 0.011        & 1.20         & 0.10                   & 1.16                     & 2.29       \\
\midrule
BINCE (CLIP) & 0.33  & 0.011    & 0.011        & 1.13         & 0.10                     & 7.59                     & 2.29   \\
BINCE (end to end)        & 1.77 & 0.012   & 0.012        & 1.43         & 0.11                     & 1.31                     & 2.50 \\
\bottomrule
\end{tabular}
\label{table:galaxy}
\end{table}

 
Humanity observes earth and sky at high temporal and spatial resolution, this can easily fill entire data centers. What is more, multiple copies of these series often exist over the world. 
At recording time it is usually not clear what kind of queries need to be answered about the recordings in the future; What was the weather like 10 years ago? Did a glacier resolve here? ect.  
To investigate our method in such real world scenario we compressed the GalaxyZoo telescope dataset (GZ2) and its 37 classification tasks.
In \cref{table:galaxy}, we compare a classical lossless and lossy method, to our BINCE at same distortions.


\paragraphQ{How well does CLIP pretraining work compared to in domain training}
In the main paper we have seen that CLIP pretraining gives rise to a compressor that generalizes very well across different datasets.
We evaluated how well the CLIP compressor generalizes compare to training BINCE directly end-to-end on GZ2's training set.
\Cref{table:galaxy} shows that the CLIP compressor works much better ( rate gains) than the end-to-end BINCE.
This suggests that pretraining can really be beneficial for training invariant compressors, and that our CLIP compressor can generalize very well across datasets.
\ydnote{We should compare BINCE end to end resnet50 with CLIP resnet 50 instead of the visual transformer CLIP. i.e. use the same encoder. }


\paragraphQ{How does our CLIP compressor generalize to very different images compared to SOTA compressors}
In the main text we have seen that our CLIP compressor generalizes very well across different datasets compared to high quality JPEG.
To better understand the limits of the generalization capacity of our CLIP compressor, we compared it to a SOTA classical compressor (WebP) on images that are completely different than the ones CLIP was trained on, namely Galaxy images (not typical images on internet).
We see in \cref{table:galaxy} that in this challenging setting our CLIP compressor only achieves relatively small gains ().


\kunote{Make this great! in da rebuttal :P}


  
\end{document}