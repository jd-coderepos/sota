\onecolumn
\lstset{language=[LaTeX]TeX,
  backgroundcolor=\color{gray!25},
  basicstyle=\ttfamily,
  breaklines=true,
  breakindent=0pt,
  columns=fullflexible
}


\begin{lstlisting}[caption={Demonstration used for math aqua}, label=math aqua,,captionpos=b]
[Example 1]
[Instruction and Question] 
A person is traveling at 20 km/hr and reached his destiny in 2.5 hr then find the distance?

[Options]
(A) 53 km
(B) 55 km
(C) 52 km
(D) 60 km
(E) 50 km

[Answer] 
(E)

[Rationale]
The distance that the person traveled would have been 20 km/hr * 2.5 hrs = 50 km. So the answer is (E). 



[Example 2]
[Instruction and Question] 
John found that the average of 15 numbers is 40. If 10 is added to each number then the mean of the numbers is?

[Options]
(A) 50
(B) 45
(C) 65
(D) 78
(E) 64

[Answer] 
(A)

[Rationale]
If 10 is added to each number, then the mean of the numbers also increases by 10. So the new mean would be 50.  So the answer is (A).



[Example 3]
[Instruction and Question] 
If a / b = 3/4 and 8a + 5b = 22,then find the value of a.

[Options]
(A) 1/2
(B) 3/2
(C) 5/2
(D) 4/2
(E) 7/2

[Answer] 
(B)

[Rationale]
If a / b = 3/4, then b = 4a / 3. So 8a + 5(4a / 3) = 22. This simplifies to 8a + 20a / 3 = 22, which means 44a / 3 = 22. So a is equal to 3/2.  So the answer is (B).



[Example 4]
[Instruction and Question] 
How many keystrokes are needed to type the numbers from 1 to 500?

[Options]
(A) 1156
(B) 1392
(C) 1480
(D) 1562
(E) 1788

[Answer] 
(B).

[Rationale]
There are 9 one-digit numbers from 1 to 9. There are 90 two-digit numbers from 10 to 99. There are 401 three-digit numbers from 100 to 500. 9 + 90(2) + 401(3) = 1392.  So the answer is (B).



[Example 5]
[Instruction and Question]

    \end{lstlisting}
    




\begin{lstlisting}[caption={Demonstration used for math prompt}, label=math prompt,,captionpos=b]
[Example 1]
[Instruction and Question] 
There are 15 trees in the grove. Grove workers will plant trees in the grove today. After they are done, there will be 21 trees. How many trees did the grove workers plant today?

[Answer]
6

[Rationale]
We start with 15 trees. Later we have 21 trees. The difference must be the number of trees they planted. So, they must have planted 21 - 15 = 6 trees.



[Example 2]
[Instruction and Question] 
If there are 3 cars in the parking lot and 2 more cars arrive, how many cars are in the parking lot?

[Answer] 
5

[Rationale]
There are 3 cars in the parking lot already. 2 more arrive. Now there are 3 + 2 = 5 cars. 



[Example 3]
[Instruction and Question] 
Leah had 32 chocolates and her sister had 42. If they ate 35, how many pieces do they have left in total?

[Answer] 
39

[Rationale]
Leah had 32 chocolates and Leah's sister had 42. That means there were originally 32 + 42 = 74 chocolates. 35 have been eaten. So in total they still have 74 - 35 = 39 chocolates. 



[Example 4]
[Instruction and Question] 
Jason had 20 lollipops. He gave Denny some lollipops. Now Jason has 12 lollipops. How many lollipops did Jason give to Denny?

[Answer] 
8

[Rationale]
Jason had 20 lollipops. Since he only has 12 now, he must have given the rest to Denny. The number of lollipops he has given to Denny must have been 20 - 12 = 8 lollipops. 



[Example 5]
[Instruction and Question] 
Shawn has five toys. For Christmas, he got two toys each from his mom and dad. How many toys does he have now?

[Answer] 
9

[Rationale]
He has 5 toys. He got 2 from mom, so after that he has 5 + 2 = 7 toys. Then he got 2 more from dad, so in total he has 7 + 2 = 9 toys. 



[Example 6]
[Instruction and Question] 
There were nine computers in the server room. Five more computers were installed each day, from monday to thursday. How many computers are now in the server room?

[Answer] 
29

[Rationale]
There are 4 days from monday to thursday. 5 computers were added each day. That means in total 4 * 5 = 20 computers were added. There were 9 computers in the beginning, so now there are 9 + 20 = 29 computers. 



[Example 7]
[Instruction and Question] 
Michael had 58 golf balls. On tuesday, he lost 23 golf balls. On wednesday, he lost 2 more. How many golf balls did he have at the end of wednesday?

[Answer] 
33

[Rationale]
Michael initially had 58 balls. He lost 23 on Tuesday, so after that he has 58 - 23 = 35 balls. On Wednesday he lost 2 more so now he has 35 - 2 = 33 balls. 



[Example 8]
[Instruction and Question] 
Olivia has 3 each. How much money does she have left?


[Answer] 
8

[Rationale]
She bought 5 bagels for 3 = 23 in beginning, so now she has 15 =  56.8 mln ) for the first nine months of 2007 from 37.4 mln euro ( 6.5 on a rabbit toy, 12.51. He found a dollar bill on the ground. What was the total cost of Keith's purchases?

[Answer]
24.8

[Rationale]
Keith spent 5.79 on pet food, and a cagecost him  4.25 million from its orange exports, how much money does it generate from its yearly total of exports ?\nOptions: a.  25.5 m, c.  127.5 m, e.  9. Then he spent  17. \n\n Question: How much did Edward have before he spent his money?

[Answer]
34

[Rationale]
To solve the problem, we need to use subtraction, not addition.\nSince Edward spent 8 more, his total spending was 8 = 17, so to find out how much money he had before he spent it all, we need to add 17.




[Example 8]
[Instruction and Question]
In this task you will be given an arithmetic operation in Italian and you have to find its answer. The operations 'addition' and 'subtraction' have been replaced with their italian translations i.e you need to perform addition when you see 'aggiunta' and subtraction in case of 'sottrazione'.\n\n1195 sottrazione 570 aggiunta 7850 aggiunta 8195 sottrazione 2484

[Answer]
14186

[Rationale]
1195 - 570 + 7850 + 8195 - 2484 = 625 + 7850 + 8195 - 2484 = 8475 + 8195 - 2484 = 16670 - 2484 = 14186.



[Example 9]
[Instruction and Question]
A ploynomial equation is a sum of terms. Here each term is either a constant number, or consists of the variable x raised to a certain power and multiplied by a number. These numbers are called weights. For example, in the polynomial: 2x^2+3x+4, the weights are: 2,3,4. You can present a polynomial with the list of its weights, for example, equation weights = [6, 4] represent the equation 6x + 4 and equation weights = [1, 3, 4] represent the equation 1x^2 + 3x + 4. In this task, you need to compute the result of a polynomial expression by substituing a given value of x in the given polynomial equation. Equation weights are given as a list.\n\nx = 3, equation weights = [6, 3, 2]

[Answer]
65

[Rationale]
The given polynomial equation is: 6x^2 + 3x + 2\nSubstituting x = 3, we get: 6(3)^2 + 3(3) + 2 = 54 + 9 + 2 = 65



[Example 10]
[Instruction and Question]

    \end{lstlisting}
    




\begin{lstlisting}[caption={Demonstration used for sni exqa}, label=sni exqa,,captionpos=b]
[Example 1]
[Instruction and Question]
Given a paragraph about movies, and a set of conversational question answers about the paragraph, generate a relevant question to the topic of the paragraph. The paragraph has the prefix 'CONTEXT:'. Each conversation question has a prefix `Q:` followed by the answer prefix `A:`.\n\nCONTEXT: They didn't change the whole climax scene. In one version Rudra's father saying him that the girl he loved is actually his step sister, so he can not marry her. On the other hand, the scene is same but there is no dialogue as it is muted and audience needs to understand from their reaction and background score. This change was made because the film got poor audience response due to the dialogue delivery of the conversation.  <sep>  Q: What are the differences between two climax of Solo? A: In one version Rudra's father saying him that the girl he loved is actually his step sister, so he can not marry her

[Answer]
What happens in the other version?

[Rationale]
The given conversation question provides information about one version of the climax scene in the movie, but does not provide any information about the other version. Therefore, asking about what happens in the other version can provide more information about the differences between the two versions of the climax scene.



[Example 2]
[Instruction and Question]
In this task, you're given a question, along with a context passage which has extra information available on certain terms mentioned in it, i.e., the proper nouns in the passage. Your job is to determine whether information from more than one term is needed to answer the question. Indicate your choice as `a` for Yes or `b` for No.\n\nQuestion: Who was the designer of the Swordfish? Passage:As designed in 1931, the He 46 was a two-seat sesquiplane of mixed construction. The upper wing was swept back 10\u00b0. This is a common trick when a change in an aircraft's design moves the center of gravity to the rear; rather than moving the whole wing back to counter for the shift, it is easier to simply angle the wings back slightly, which requires very little change to the basic design, or even the overall structure of the wing. (The similarly angled wings on the Fairey Swordfish are due to just such a mid-program change to the weight distribution of the aircraft.) The tailplane was mounted high and braced by struts. The undercarriage was fixed, and the tail was fitted with a skid rather than a wheel. The He 46 prototype first flew in late 1931; its flight characteristics were good, but design improvements were incorporated. The small lower wing was removed, while the mainplane was increased in area by 22

[Answer]
b

[Rationale]
The question only asks for the designer of the Swordfish, which is mentioned directly in the passage as "The similarly angled wings on the Fairey Swordfish." No other information or proper nouns are needed to answer the question.



[Example 3]
[Instruction and Question]
Indicate a text span in the passage about which an inquisitive question is asked. (note, don't answer the question).\n\nSentence: In trading on the Stock Exchange of Hong Kong , the shares closed Wednesday at HK $ 4 . 80 each .<sep>Question: Is there any fee associated with owning foreign shares?

[Answer]
Stock Exchange of Hong Kong

[Rationale]
The text span should indicate a specific location where the shares were traded and the question pertains to the fee associated with owning shares in foreign markets. The relevant text span is "Stock Exchange of Hong Kong".



[Example 4]
[Instruction and Question]
Given a passage and a query based on the passage, generate an unambiguous, concise and simple answer to the query from information in the passage. Note that the answer may not be present in exact form.\n\nBuy chicken before this date for storage in your home refrigerator. Always cook or freeze the chicken within two days of bringing it home, regardless of what the package date tells you. Never store raw chicken in the refrigerator at home for longer than two days. Once you've cleared up what belongs where, the next step is knowing how long the food will last. Storing food is a delicate matter -- in the fridge, food might get moist and moldy, and in the freezer it could lose all its moisture or get freezer burn. If you leave food in the fridge or freezer too long, it could not only lose its integrity but it may also become unsafe to eat. FoodSafety.gov has two useful charts listing how long certain foods will stay good in both the refrigerator and the freezer. The first chart lists all kinds of commonly refrigerated foods and the second is specifically for eggs and food made with eggs. Chrissy Metz, star of <strong>This is Us</em>, appeared on <strong>Jimmy Kimmel Liv... 1  Chrissy Metz, star of This is Us, appeared on Jimmy Kimmel Live on Tuesday wearing a stylishly on-point outfit that people are dying to copy\u2014and it\u2019s possible to do so without breaking the bank. References. 1  Basics for handling food safely. U.S. Department of Agriculture. 2  Foodborne illness: What consumers need to know. U.S. Department of Agriculture. 3  Food safety counts! U.S. Department of Agriculture. 4  Dietary Guidelines for Americans, 2010. U.S. Department of Health and Human Services. 5  Slow cookers and food safety. Keep your refrigerator below 40 degrees Fahrenheit, and store raw chicken in the coldest part of the fridge, which is usually the back of the bottom shelf or in the meat drawer. Make sure to decrease the storage time if you can't guarantee that the temperature will stay under 40 degrees F. (Also) I feel like I\u2019m limited to what I can do with the chicken once its cooked. Thanks for your help, SS. Dear SS: According to Foodsafety.gov, whole, uncooked poultry (chicken or turkey) is safe in the refrigerator for 1-2 days and safe in the freezer for up to 12 months. Poultry pieces are safe in the refrigerator for 1-2 days and safe in the freezer for up to nine months. Cooked poultry is safe refrigerated for 3-4 days and in the freezer for 2-6 months. You can also freeze your cooked breaded chicken cutlets (Foodsafety.gov specifically lists nuggets and patties) for 1-3 months. How long can you keep chicken in the fridge? Raw chicken: You'll need to either cook or freeze it fast. According to recommendations from the U.S. Food and Drug Administration, raw chicken will only keep in the refrigerator for about 1-2 days. (Same goes for turkey and other poultry.) If you store it in the freezer in a sealed, freezer-safe container or vacuum-sealed package, it should stay safe to thaw and eat for up to 9 months, in pieces, or up to 12 months, for a whole chicken. Cooked chicken: You have a little more time, but not much. I have to say that I have kept chicken in the fridge for longer than 2 days and not died. Some even have best before dates 3-4 days in. The quality definitely suffers, and I'm sure there's some risk, but it's probably low. How long does raw chicken last after being frozen and thawed? Chicken that has been defrosted in the fridge can be kept for an additional 1 to 2 days in the refrigerator before cooking; chicken that was thawed in the microwave or in cold water should be cooked immediately. I didn't even realize it until the next day when I found the container with the fresh chicken and realized that there were two containers of it in the fridge because I'd forgotten to throw away the old chicken from the previous week. Query: how long is chicken good in the refrigerator

[Answer]
According to Foodsafety.gov, whole, uncooked poultry (chicken or turkey) is safe in the refrigerator for 1-2 days

[Rationale]
Here's an direct answer: According to Foodsafety.gov, whole, uncooked poultry (chicken or turkey) is safe in the refrigerator for 1-2 days.



[Example 5]
[Instruction and Question]
You are given a sentence and a question, construct 2 answer options in a specific format i.e. ['option1','option2']. The answer options have to be text spans from the given sentence. Only one among the options should be the correct answer to the given question. Note that, the correct answer to the given question will require understanding of coreference resolution. Coreference resolution is the task of clustering mentions in text that refer to the same underlying real world entities.\n\nSentence: In July, Kamtchatka declared war on Yakutsk. Since Yakutsk's army was much better equipped and ten times larger, they were victorious within weeks. Question: Who was victorious

[Answer]
['Yakutsk', 'Kamchatka']

[Rationale]
The correct answer to the given question is 'Yakutsk', as they were the ones who won the war. However, to understand this, coreference resolution is required, as 'they' refers to Yakutsk in the sentence. So, the answer options should be: ['Yakutsk', 'Kamtchatka'] where 'Yakutsk' is the correct answer and 'Kamtchatka' is the incorrect option.



[Example 6]
[Instruction and Question]
Given a text paragraph and a fact, generate 'True' if the fact can be inferred/concluded from the paragraph, otherwise generate 'False'. The input passage contains as set of logical statements so the fact can be marked either 'True' or 'False' if the relavant info regarding the fact is present in the given passage. If the Fact can't be marked True or False from the given paragraph, mark it as False. Don't use knowledge outside the given paragraph to check the fact.\n\nParagraph: If something is rough then it is big. Gary is kind. Erin is rough. Erin is red. Gary is green. Gary is blue. Erin is blue. If Gary is rough then Gary is green. All kind things are blue. All quiet things are big. Bob is blue. Bob is rough. If something is green and blue then it is quiet. Bob is green. Red, quiet things are blue. Bob is big.     Question: Erin is not red.

[Answer]
False

[Rationale]
There is a statement in the paragraph that indicates Erin is red.  Therefore, the fact that "Erin is not red" can't be inferred from the paragraph.



[Example 7]
[Instruction and Question]

    \end{lstlisting}
    




\begin{lstlisting}[caption={Demonstration used for sni ethics}, label=sni ethics,,captionpos=b]
[Example 1]
[Instruction and Question]
In this task, you are given two simple actions (associated with \"A\", \"B\"). You must identify which action is considered less ethical. Do not generate anything else apart from one of the following characters: 'A', 'B'.\n\n(A) telling my friend to stop venting about her breakup with me\n(B) telling my bf that in case I get pregnant, I get to make the last call on whether we are keeping it or not

[Answer]
A

[Rationale]
For action A, telling a friend to stop venting about their breakup may be seen as unsupportive and insensitive, but it does not involve making a significant decision that affects another person's life. It is a more minor offense.\nFor action B, telling a partner that you will make the final decision about whether to keep a potential child or not, without their input or agreement, is a significant violation of their autonomy and a potentially manipulative or coercive action.\nTherefore, action B has a greater ethical implication and is considered less ethical than action A.



[Example 2]
[Instruction and Question]
In this task you are given an anecdote, which recount a complex ethical situation. The question you are given is, has the author of the anecdote acted ethically? answer this question with \"RIGHT\" or \"WRONG\".\n\nTLDR at the bottom.\n\n&amp;#x200B;\n\nThis did happen a few years ago but to this day it still haunts me. I met a girl at work (lets call her T), and she was perfect. We went out a few times and everything was good. Unfortunately T had to move away for school, not super far but far enough that we couldn't hang out to much. \n\nNow T and her sister (we can call the sister K) both worked with me. I was not the biggest fan of K but she was the sister of T, and i was determined to be in good graces with her, as Those two were very close.  So, maybe a week after T went back to school, I was working a shift with K and another friend (we can call him Q) the first half of the shift was fine, people pouring in and we barely had time to talk. Things took a turn when we all went on break together. I was craving a burrito and there was a Starbucks near by where I was going, so I offered K a ride since she was going there. We get back and Q is sitting there in the break room. \n\nWe sit down and Q looks at me and asks if I want to see this YouTube video of \"The Ultimate Vape\". I'm bored and have nothing better to do so I agree. We get to the point where its time to choose the flavor for the vape, and its churro flavor. This is what sets off the argument, because the man in the video didn't roll the R in churro (this is important to K because her and T are both half Black half Mexican women).  this is the conversation that followed.\n\nQ: Some people just can't roll their R's.\n\nK: That's why i hate White people,\n\nQ: ... that's kind of racist...\n\nThat's all i can fully remember was said but the events that followed were K saying you can't be racist towards White people and that's what THEIR argument was about. I considered both K and Q to be my friends but they were not necessarily friends with each other. So I sit aside let them have at it because if I choose a side I'm definitely loosing one of them as a friend.\n\nAt some point K turns to me, and she says,\n\nK: hey LexeL... I forgot my phone in my car, can you use yours to look up the definition of Racism?\n\nMe: I'm sorry K but this is between you two and id rather not get in the middle.\n\nAnd then she goes back to Q. i should mention that arguing over the Dictionary definition of racism had gone on for 3 minutes prior to this conversation. and Q had correctly said, Word for Word, what the definition of Racism was but K would not accept it.\n\nWe can skip to the end of break. I'm getting ready to go downstairs and K stops me.\n\nK: Have you talked to T lately?\n\nMe: Well I messaged her on Friday, but I know shes busy so I don't expect to talk to her until we Skype tomorrow. Why?\n\nK: Oh... well i just would expect to hear from her at all.\n\nThis interaction led me to believe that K had talked to T way before her argument with Q, and that she knew something I didn't, I spent the rest of my shift, nerves on end hoping I'd get to the bottom of this. \n\nK got off work before I did, terrible start to the rest of my night TBH. So as soon as I get off work this is the following conversation that I know by heart.\n\nText messages -\n\nT: what are you playing at??\n\nMe: Listen, idk what you've heard, who you've talked to, or what you're thinking, but I think we should talk.\n\nT: I agree.\n\nMe: Are you able to Skype tonight?\n\nT: We should just call on the phone...\n\nMe: okay.\n\nI have quite the drive home and I call her as soon as its over, we have a long talk about the situation that happened and she says that my story lines up with K's story. But... sadly I was in the wrong. Because I took no side in the situation, I actually took Q's side. \"a bystander is just taking the oppressors side without vocalizing it\" T said that she didn't think we should continue talking. We said goodbye for the last time and I never really got over it. so Am i the Asshole for not choosing her sisters side when her sister was wrong?\n\nTLDR; The sister of the girl I was seeing and another friend got into an argument about the definition of racism during break and i didn't choose a side, thus making me the bad guy in the eyes of the girl i was seeing.\n\nI wanna thank anyone for reading, even the TLDR, and I appreciate any comments. I'm just looking to learn from a past experience.\n\nThanks again!

[Answer]
RIGHT

[Rationale]
The author chose to remain neutral and not take a side in the argument, which led to the girlfriend's sister feeling unsupported and angry with the author. While it is understandable that the girlfriend's sister felt hurt and unsupported, the author acted ethically by not taking sides in the argument. Choosing a side in an argument between two friends can be tricky and can lead to further conflicts and hurt feelings. By choosing not to take a side, the author avoided causing more harm to either friend, and instead encouraged them to work out their differences and come to a resolution on their own.\nThe author's decision to remain neutral was a responsible and ethical choice that prioritized the well-being of both friends. Therefore, the answer to the question ""Has the author acted ethically?"" is ""RIGHT"".



[Example 3]
[Instruction and Question]
In this task you are given an anecdote. You must find out whether the anecdote is something the author has done or is considering doing. Label the instances as \"HISTORICAL\" when the author has already done something and label them as \"HYPOTHETICAL\" when the author is considering doing something in the future.\n\nSo I dated a girl for a few months, it was shaping up to be a longterm thing, but things went bad. We basically went a few months with no contact, until she contacted me, saying she was a few months pregnant and needed my support. I was floored. We'd barely spoken since the breakup, and I was clear as day about the fact that I NEVER wanted kids. \n\nThis was one of the worst things that could conceivably happen to me. She suggested getting back together, which I shut down right out of the gate. I explained that we were through months ago, and that I don't want to be a father, and hung up. She is dead set on keeping it, there have been tests, and the kid is mine, though I dont know how the fuck it happened, since we were fairly strict with contraceptives.\n\nI have a new GF, and we've just moved to a great new apartment. This phonecall literally shook me to my core, and I am freaking out like crazy. I've been searching for every possible way out, and I heard of a financial abortion, or whatever its called. Basically I want to resign any rights and responsibilities for the kid, since I never wanted him/her to begin with. I refuse to pay out the ass for a kid I don't want, and I obviously won't and can't force the mother to get rid of the baby, or give it up for adoption.\n\nHer friends have called me, saying that I really hurt her by refusing to acknowledge this kid, and calling me selfish and whatnot, but I can't handle all that stuff right now, or ever probably. Am I the asshole for wanting a \"financial abortion\", that is, no contact, no child support, no custody, and is it morally and legally okay???

[Answer]
HISTORICAL

[Rationale]
The author describes a past event in which they dated someone, broke up, and then received news that their ex-girlfriend was pregnant. The author expresses reluctance to be a father and considers ways to resign from any rights and responsibilities for the child, which suggests that the event has already occurred. Therefore, the label for this anecdote is "HISTORICAL".



[Example 4]
[Instruction and Question]
Given a post that is a real-life anecdote of a complex ethical situation and an associated claim about its type, verify if the claim is true or not. The claim asks if the posts are historical or hypothetical. The posts are \"HISTORICAL\" when the author has already done something and they are \"HYPOTHETICAL\" when the author is considering doing something in the future. Say 'yes' if the claim is true otherwise say 'no'\n\nCLAIM : The following post is HISTORICAL. POST : Please excuse spelling I'm on mobile. So first off I would like to let everyone know that I am in a theater  program and it was two days before the show starts (it starts on the 29th) and we had beed rehearsing since after lunch until 5:30. Some kids were whispering during the rehearsal and some others were on their phone backstage. I wasn't on my phone or talking unless I was telling someone else to be quiet.\n     Our theater teacher (who we will call R) was angry at people for not paying attention. R might have just been stressed, but so were we and being yelled at after finishing a two hour show rehearsal was not what we needed. My friend, who we will call J, was there too and after he had finished yelling at everyone I asked her, and I quote, \"does he always get like this before a show?\", and she called me an asshole. Please don't say anything mean about J because she is still my friend, and R was just stressed. So Reddit AITA?\n\nTLDR: My theater teacher yelled at us two days before a show and I asked my friend if he always gets like this before a show, and my friend called me an asshole.

[Answer]
no

[Rationale]
The author of the post is questioning if it was appropriate to ask their friend if their theater teacher always behaves in such a manner when under stress.\nIn the post, the author uses the past tense to describe the events, but they do not explicitly state when the events occurred. However, they mention that the show starts on the 29th, which suggests that the events occurred in the past, before that date.\nBased on the above analysis, it seems that the post is not hypothetical, as the author describes events that have already occurred. Therefore, the appropriate category is ""HISTORICAL."" In that case, the answer would be ""no"".



[Example 5]
[Instruction and Question]
Given a real-life anecdote of a complex ethical situation, judge whether or not a specific person is wrong here. Answer 'not answerable' if more information is required and 'answerable' if all the information is sufficient to answer who is wrong in this situation.\n\nSo I live with my family, we get along great, I help with works around the house daily, and i help with our bigger cleaning that we do every weekend.\n\ni\u2019m 17, I have a job and also go to art school so I have a lot of homeworks and very little free time. I have a boyfriend and a group of friends.\n\nI usually spend my fridays at the bf and then go home.\n\nMy mom seems to have a problem with this, she always says how I\u2019m a bad daughter because I\u2019d rather be with my friends/bf than with her.\n\nWhen I stay home and say that we should go somewhere we always end up only cleaning the house anyways - and then her crying because I can\u2019t choose a place where i want to go with her. \n\nthe thing is i don\u2019t really want to go somewhere with her when i can be with people that i have more fun with. \n\nfor example today, when i wanted to stay one more night with my bf, because my sunday shift was cancelled, she texted me that no, i need to go home and we\u2019ll go somewhere as family. \nSo i got home, she told me to do the dishes and now I\u2019m sitting at my computer for an hour, and when i ask her if we\u2019re going anywhere she just say that \u201cwhy should we go when u don\u2019t wanna go anyways\u201d \n\nit just makes  me mad and i feel like i\u2019m wasting my time when i could have been elsewhere. 

[Answer]
answerable

[Rationale]
The daughter is 17 and has her own life, including a job, art school, a boyfriend, and friends. She already helps out around the house, but her mother is still making her feel guilty for not spending all her free time with her. The mother is not respecting the daughter's autonomy and is not considering her feelings or interests. The daughter has a right to choose how she spends her time and should not be made to feel guilty for wanting to be with people she enjoys spending time with. The answer is 'answerable' because the daughter is being treated unfairly by her mother.



[Example 6]
[Instruction and Question]

    \end{lstlisting}
    




\begin{lstlisting}[caption={Demonstration used for sni dialogue}, label=sni dialogue,,captionpos=b]
[Example 1]
[Instruction and Question]
The input is taken from a negotiation between two participants who take the role of campsite neighbors and negotiate for Food, Water, and Firewood packages, based on their individual preferences and requirements. Given an utterance and recent dialogue context containing past 3 utterances (wherever available), output Yes if the utterance contains the self-need strategy, otherwise output No. self-need is a selfish negotiation strategy. It is used to create a personal need for an item in the negotiation, such as by pointing out that the participant sweats a lot to show preference towards water packages.\n\nContext: '2 packages of both of what?  Water and food?' 'I mean water and food' 'Is food something you need the most of or is it water?'\nUtterance: 'I need 2 packages of both food and water on each'

[Answer]
No

[Rationale]
The speaker simply states their need for two packages of food and two packages of water without indicating any particular preference or personal need for either item.\nFor example, if the speaker had said "I need two packages of water because I sweat a lot and need to stay hydrated," that would be an example of the self-need strategy.\nSince the utterance in this case does not use the self-need strategy, the answer is No.



[Example 2]
[Instruction and Question]
You are given a request sent by a user in a human and computer dialogue. Your task is to generate a prediction of what the requester is actually trying to do. The input is a sentence written by the human user, trying to achieve certain objective through this human computer interaction, and the output should be a sentence or two succinctly describing the human user's objective mentioned previously.\n\nTell me about Korean language.

[Answer]
Find information and resources on the Korean language

[Rationale]
The objective of the human user is to obtain information about the Korean language. So, I would find information and resources on the Korean language.



[Example 3]
[Instruction and Question]
Read an exchange between SENDER A and SENDER B, and answer whether SENDER A is a 'Bot' or a 'Human'. There are only a maximum of two senders (A and B) per conversation, and there are no conversations where SENDER A and SENDER B are both bots, or both humans; only one sender per conversation can be a bot, and only one can be a human. Knowing this, need only to classify whether SENDER A is 'Human' or 'Bot'; you do not need to write anything about SENDER B. Typically, bots will have a more disjointed manner of speaking, and will make statements that don't relate to each other, don't make coherent sense, or otherwise appear unnatural. Humans will make statements in a more or less coherent and logical way, potentially including asking the bot why the bot is not making sense or contradicting itself.\n\nSENDER A: In Las Vegas, you're an afterthought, you're an old so while we could go gambling, we could go hookering, we could get drunk, we could go to the convention...\nSENDER B: hello\nSENDER A: Hey, Mr. San Francisco, I'd like you to meet...\nSENDER B: i love britney spears and you?\nSENDER A: The New York City Ballet Gala, duh.\nSENDER B: i'm on a diet now, have you ever followed any diet?\nSENDER A: Thanks, Zeke! No way. My sparkle jelly stays on my wrist until I die, and then probably even after that.\nSENDER B: lol, sounds funny\nSENDER A: Big Buddha Cheese for muscle spasms.\nSENDER B: do you like spending your (or not so) money?\nSENDER A: Look, II have got... 25 bucks for food.\nSENDER B: could you give them to me?\nSENDER A: This Friday.  How much did you sell 'em?\nSENDER B: a lot, lol\nSENDER A: II'm not gonna sell them.\nSENDER B: well, why?..\nSENDER A: But I'll pay you like a regular customer.\nSENDER B: oh, thanks\nSENDER A: I... I don't mean selling programmes or finding prizes for the tombola.\nSENDER B: l should go, sorry\nSENDER A: I... I'd rather stay here, if you don't mind.\nSENDER B: no, i don't, stay here, buddy\nSENDER A: I'll go on alone.\nSENDER B: bye\nSENDER A: Hi, Les, I'll be right with you.\n

[Answer]
Bot

[Rationale]
SENDER A talks about Las Vegas, then the New York City Ballet Gala, and then Big Buddha Cheese for muscle spasms. These statements don't seem to follow a logical train of thought or relate to each other. Additionally, SENDER A's responses don't seem to reflect an understanding of what SENDER B is saying or engage in a meaningful conversation. Therefore, the output is 'Bot'.



[Example 4]
[Instruction and Question]
You have to predict the type of conversation sentence given as input. Your predictions can be one among: 'information' if the input statement states a piece of information, 'question' if the input statement asks for some information, 'directive' if the input sentence involves the management or guidance of operations, and 'commissive' when it constitutes a statement that commits the speaker to some future action.\n\n Basically , you just can't drive . 

[Answer]
information

[Rationale]
In this case, the sentence "Basically, you just can't drive" is not asking for any information or guidance, nor is it committing the speaker to any future action. Instead, it is simply conveying a piece of information about the listener's driving ability. Therefore, it can be classified as an "information" type of sentence."



[Example 5]
[Instruction and Question]
The input is a conversation between an automated system and a user looking for suggestions for pubs, restaurants and coffee shops in Cambridge. In the dialogue, the user may provide some criteria for the type of place they want such as price range, cuisine, etc. Given such a dialogue, output the price range the user if looking for which can take one of four values: Cheap, Moderate, Expensive and Don't Care. Output cannot be any other value than the four values. Note that if the user asks for the prices of items provided at a pub, restaurant or coffee shop but doesn't mention the price range they are looking for then the correct classification will be Don't Care. If the user does not mention a particular price, the correct classification will be Don't Care.\n\nSystem: 'Thank you for calling the Cambridge Information system. Your call will be recorded for research purposes. You may ask for information about a place to eat, such as a restaurant, a pub, or a cafe. How may I help you?'\nUser: 'sil'\nSystem: 'What kind of venue are you looking for? For example, a pub, or a restaurant.'\nUser: 'yes i uh im trying to find a mexican restaurant in the expensive price range'\nSystem: 'Nanna Mexico, which serves mexican and tex mex food, is a great restaurant. It's in the expensive price range.'\nUser: 'can i get the phone number and area of the venue'\nSystem: 'The phone number of Nanna Mexico is 01223902150. The price for main courses ranges from 5 pounds 35 to 15 pounds 50 and the price for a cappuccino is 2 pounds 95.'\nUser: 'the area of the restaurant'\nSystem: 'Nanna Mexico, which serves mexican and tex mex food, is an outstanding restaurant. It's in the expensive price range.'\nUser: 'the address of the restaurant'\nSystem: 'Nanna Mexico is located on 33 Regent Street.'\nUser: 'uh thank you goodbye'\n

[Answer]
Expensive

[Rationale]
"the user explicitly states that they are looking for a Mexican restaurant in the "expensive price range", and the system recommends Nanna Mexico which it confirms is indeed in the "expensive price range".So, the correct classification for the user's price range preference is "expensive"."



[Example 6]
[Instruction and Question]
In this task you will be given two dialogues. You need to determine if both dialogues have the same underlying emotion. The possible emotions are happy, sad, angry, or other. If they do output 'yes', if not output 'no'.\n\nDialogue 1: 'i am not understand what you are saying  really that's what i was saying pls clearly convey the msg don't irritate me'. Dialogue 2: 'so rude why u didnt ans my question'.

[Answer]
yes

[Rationale]
Both dialogues express a negative emotion, with Dialogue 1 expressing frustration and Dialogue 2 expressing anger. Therefore, both dialogues have the same underlying emotion, which is not a positive emotion like happiness, and not a neutral emotion.



[Example 7]
[Instruction and Question]
You are shown a random response (from a computer or a human). Generate a language query such that it leads to this reply.\n\nTotal travel time is 50 minutes. Would you like me to book it?

[Answer]
Not right now, thanks. As I'll be in the center part of town I'd like to find an expensive restaurant there. Do you have any suggestions?

[Rationale]
The user may have asked for restaurant recommendations in the central part of town, and after receiving the recommendations, the system could have informed the user about the travel time to their desired destination. The system may then offer to book the recommended restaurant for the user.




[Example 8]
[Instruction and Question]
In this task you are given a small conversation between two persons and 4 options on how the conversation should continue. Your job is to choose the most reasonable option. The conversation and the options are separated by a newline character. Each dialogue in the conversation are separated by a comma. F and M indicate female and male, respectively.\n\nM: Today we have Jane Couch with us. Good evening and welcome, Jane. ,F: Good evening. ,M: Now this is a question that everyone would like to ask you. How did you get interested in boxing? ,F: I saw a little television program about women boxing about 6 years ago. And at that moment, I said to myself, I'm going to do that. And it just changed my life. But my parents disagreed at first. ,M: And were you interested in sport at school? ,F: No, nothing. I wasn't ever fit or anything. I just don't look after myself at all. ,M: What do you think is the most important for a top boxer? ,F: Believe it or not, anyone can fight. But to make it to the top, you have to know your skills. And you have to have a quick brain and the fitness. The fitness side of it is the most important. ,M: And just lastly, when is your next big fight? ,F: The next one is going to be the end of February, and I will have another 2 at the beginning of April, and at the end of May. ,M: Well, that's great. We're looking forward to seeing your performance. Thank you very much Jane.\n(A) F: Yeah, I also look forward to my next fight in April. (B) F: Come and see my next fight in early February. (C) F: I\u2019m sorry, I don\u2019t understand. Could you say it again? (D) F: I have several months before my next fight in May.

[Answer]
C

[Rationale]
Option (A) and (B) are incorrect as they are not related to the conversation at hand and are not a natural continuation of the dialogue. Option (D) is incorrect as it provides incorrect information regarding the timing of the next fight. Option (C) is the most reasonable option as it is a natural response when one doesn't understand something and wants it to be repeated.



[Example 9]
[Instruction and Question]
You will be given a person's personality, and a history of a conversation this person has had. You will be given four candidate sentences to complete the conversation, based on the context. Choose one and answer with the text.\n\nPersonality: I've long hair for a boy.\nI draw during classes instead of taking notes.\nI wish I could live in a fantasy world.\nI like lizards.\nChat history: -Hello! How are you? What kind of things do you like?\n -My age is 25. I like to read. What do you like?\n -I love lizards, I used to have a firey orange bearded dragon!\n -Cool! How old are you?\n -I am 28. are you in school?\n -Nope. I work at the library. You?\n -I tend to draw in class, when I should take notes, but get good grades.\n -What is your major? I hated school, but do love my job.\n -Art. Do you have any cool dreams? Things you wish you could do?\n -Nope. Just like sitting in the library, all alone, nobody to bother me.\n -Sometimes I want to live in a magical world with unicorns and dragons.\n -Are you male or female? I hate all this competition between females.\nCandidates 1) Female. What color is your hair? Is it long or short? 2) Blood doesn't bother me very much haha. Do you like sports? 3) I teach 1st grade, I've a lot of sixth graders that are wonderful. 4) Not good. I want to drive fast but I can not because can be dangerous.

[Answer]
Female. What color is your hair? Is it long or short?

[Rationale]
The given personality traits and chat history do not provide any relevant context to the candidate options except for the fact that the person likes lizards. In the chat history, the person mentions drawing in class instead of taking notes, and their major is art. Therefore, the most reasonable response among the candidate options is to ask about their hair since it is a relevant question and is not out of context. Hence, the answer is "Female. What color is your hair? Is it long or short?"



[Example 10]
[Instruction and Question]

    \end{lstlisting}
    




\begin{lstlisting}[caption={Demonstration used for sni data2text}, label=sni data2text,,captionpos=b]
[Example 1]
[Instruction and Question]
This task is to find the number of 'For' loops present in the given cpp program.\n\nint numcount=0;\nvoid divide(int num,int x)\n{\n\tint i;\n\tif(num==1)\n\t\tnumcount++;\n\tfor(i=x;i<=num;i++)\n\t{\n\t\tif(num

[Answer]
3

[Rationale]
The first for loop in the code is inside the main function and is used to read the input values from the user.\nThe divide function contains the second for loop,\nThe third and final for loop in the code is used to print the output values to the console.\nSo overall, there are three for loops in this code: one in main to read input values, one in divide to perform calculations, and one in main to print output values.



[Example 2]
[Instruction and Question]
In this task, you are given a natural language interpretation of commands (consist of logical operations) to select relevant rows from the given table. Your job is to generate command (in terms of logical operations) from given natural language interpretation. Define body (contains a collection of statements that define what the this logical operator does) of each logical operator between '{}' parenthesis. Here are the definitions of logical operators that you can use while generating command: \n 1. count: returns the number of rows in the view. \n 2. only: returns whether there is exactly one row in the view. \n 3. hop: returns the value under the header column of the row. \n 4. and: returns the boolean operation result of two arguments. \n 5. max/min/avg/sum: returns the max/min/average/sum of the values under the header column. \n 6. nth_max/nth_min: returns the n-th max/n-th min of the values under the header column. \n 7. argmax/argmin: returns the row with the max/min value in header column. \n 8. nth_argmax/nth_argmin: returns the row with the n-th max/min value in header column. \n 9. eq/not_eq: returns if the two arguments are equal. \n 10. round_eq: returns if the two arguments are roughly equal under certain tolerance. \n 11. greater/less: returns if the first argument is greater/less than the second argument. \n 12. diff: returns the difference between two arguments. \n 13. filter_eq/ filter_not_eq: returns the subview whose values under the header column is equal/not equal to the third argument. \n 14. filter_greater/filter_less: returns the subview whose values under the header column is greater/less than the third argument. \n 15. filter_greater_eq /filter_less_eq: returns the subview whose values under the header column is greater/less or equal than the third argument. \n 16. filter_all: returns the view itself for the case of describing the whole table \n 17. all_eq/not_eq: returns whether all the values under the header column are equal/not equal to the third argument. \n 18. all_greater/less: returns whether all the values under the header column are greater/less than the third argument. \n 19. all_greater_eq/less_eq: returns whether all the values under the header column are greater/less or equal to the third argument. \n 20. most_eq/not_eq: returns whether most of the values under the header column are equal/not equal to the third argument. \n 21. most_greater/less: returns whether most of the values under the header column are greater/less than the third argument. \n 22. most_greater_eq/less_eq: returns whether most of the values under the header column are greater/less or equal to the third argument.\n\nselect the rows whose surface record fuzzily matches to clay . among these rows , select the rows whose week record fuzzily matches to may . the number of such rows is 2 .

[Answer]
eq { count { filter_eq { filter_eq { all_rows ; surface ; clay } ; week ; may } } ; 2 }

[Rationale]
The given natural language interpretation consists of two conditions that need to be satisfied to select the relevant rows:\nThe first condition requires selecting the rows whose "surface" record fuzzily matches with "clay".\nThe second condition requires selecting the rows among the previous selection whose "week" record fuzzily matches with "may".\nTo represent the above conditions using logical operators, we can use the "filter_eq" operator to select the rows whose "surface" record is equal to "clay" and "week" record is equal to "may". We can then count the number of rows that satisfy this condition using the "count" operator.\nSo, the first part of the command will be: "eq { count { filter_eq { filter_eq { all_rows ; surface ; clay } ; week ; may } }".\nFinally, we need to check if the number of rows that satisfy the above condition is equal to 2. To represent this condition, we can use the "eq" operator again, and pass the previous command as the first argument and 2 as the second argument.\nSo, the final command will be: "eq { count { filter_eq { filter_eq { all_rows ; surface ; clay } ; week ; may } } ; 2 }".



[Example 3]
[Instruction and Question]
Given a command in a limited form of natural language, provide the correct sequence of actions that executes the command to thus navigate an agent in its environment. A command can be broken down into many different actions. Actions are uppercase and are individual steps that serve as the building blocks for a command. For commands, 'left' and 'right' are used to denote the direction of an action. The word 'opposite' turns the agent backward in the specified direction. The word 'around' makes the agent execute an action while turning around in the specified direction. The word 'and' means to execute the next scope of the command following the previous scope of the command. The word 'after' signifies to execute the previous scope of the command following the next scope of the command. The words 'twice' and 'thrice' trigger repetition of a command that they scope over two times or three times, respectively. There are only six actions: 'I_LOOK', 'I_WALK', 'I_RUN', 'I_JUMP', 'I_TURN_LEFT', and 'I_TURN_RIGHT'. These actions respectively align with the commands 'look', 'walk', 'run', 'jump', 'turn left', and 'turn right'. Actions and commands do not have quotations in the input and output.\n\nlook around left and run left

[Answer]
I_TURN_LEFT I_LOOK I_TURN_LEFT I_LOOK I_TURN_LEFT I_LOOK I_TURN_LEFT I_LOOK I_TURN_LEFT I_RUN

[Rationale]
1. The command is "look around left and run left".\n2. The first action is "I_TURN_LEFT", since the command starts with "look around left".\n3. The next action is "I_LOOK", since the command includes "around", which means to execute an action while turning around in the specified direction, and "left", which specifies the direction.\n4. Steps 2 and 3 are repeated four times, since there are five left turns in the command. This gives us a total of 5 "I_TURN_LEFT" actions and 5 "I_LOOK" actions.\n5. Finally, the command includes "and run left", which means to execute the next scope of the command following the previous scope of the command. So, after the previous steps, the last action is "I_RUN".



[Example 4]
[Instruction and Question]
In this task you are expected to provide an SQL statement from an english description of what that SQL statement does. The description may include multiple steps but you should only ouput one SQL statement that accomplishes every step. An SQL query works by selecting data from a table where certain conditions apply. A table contains columns where every row in that table must have a value for each column. Every table has a primary key that uniquely identifies each row, usually an id. To choose which columns are returned you specify that after the \"SELECT\" statement. Next, you use a \"FROM\" statement to specify what tables you want to select the data from. When you specify a table you can rename it with the \"AS\" statement. You can reference that table by whatever name follows the \"AS\" statement. If you want to select data from multiple tables you need to use the \"JOIN\" statement. This will join the tables together by pairing a row in one table with every row in the other table (Cartesian Product). To limit the number of rows returned you should use the \"ON\" statement. This will only return rows where the condition specified after the statement is true, this is usually an equals operator with primary keys. You can also use the \"WHERE\" statement to specify that only rows with column values statisfying a certain condition, should be returned. The \"GROUP BY\" statement will group rows together that have equal column values for whatever columns follows the statement. The \"HAVING\" statement will return groups that statisfy whatever condition follows the statement. Any column(s) being returned from grouped rows must either be an aggregate function, (AVG, MAX, COUNT, SUM, ...) of a column, or the column(s) that the data was grouped by. To sort the returned data you can use the \"ORDER BY\" command which will order the data by whatever aggregate function or column follows the statement. The \"DESC\" statement will sort in descending order and the \"ASC\" statement will sort in ascending order. Finally, you can use the \"LIMIT\" statement to return a certain number of rows. When \"*\" is used in an SQL statement every column is returned. For example, SELECT * FROM table WHERE attribute = 1, will select every column from rows with the attribute column equal to 1.\n\nStep 1: For each row in artist table, find the corresponding rows in song table.\nStep 2: find artist's artist_name of the results of step 1 whose song's country equals UK.\nStep 3: find the song's artist_name of song table for which song's country equals english.\nStep 4: show the rows that are in both the results of step 2 and the results of step 3

[Answer]
SELECT T1.artist_name FROM artist AS T1 JOIN song AS T2 ON T1.artist_name = T2.artist_name WHERE T2.country = \"UK\" INTERSECT SELECT T2.artist_name FROM song AS T2 WHERE T2.country = \"english\"

[Rationale]
Step 1: For each row in artist table, find the corresponding rows in song table.\nStep 2: Find artist's artist_name of the results of step 1 whose song's country equals UK.\nStep 3: Find the song's artist_name of song table for which song's country equals english.\nStep 4: Show the rows that are in both the results of step 2 and the results of step 3.\nPutting it all together, we get the following SQL statement:\nSELECT T1.artist_name FROM artist AS T1 JOIN song AS T2 ON T1.artist_name = T2.artist_name WHERE T2.country = \"UK\" INTERSECT SELECT T2.artist_name FROM song AS T2 WHERE T2.country = \"english\"



[Example 5]
[Instruction and Question]

    \end{lstlisting}
    




\begin{lstlisting}[caption={Demonstration used for sni cbqa}, label=sni cbqa,,captionpos=b]
[Example 1]
[Instruction and Question]
In this task, you're given an open-domain question. Your task is to provide an answer to the given question. There is only one unique answer for each question. Your answer should be short, and refer to an entity, date, number, etc. Do not put your answer in the context of a sentence.\n\nWho played in the super bowl in 2017?

[Answer]
New England Patriots, Atlanta Falcons

[Rationale]
The New England Patriots and the Atlanta Falcons played in the Super Bowl in 2017. Since the instruction asked not put your answer in the context of a sentence, the final answer is: New England Patriots, Atlanta Falcons



[Example 2]
[Instruction and Question]
In this task, you're given a pair of sentences, sentence 1 and sentence 2, that agree with each other. Your job is to alter sentence 2 so that the pair neither agree nor contradict each other. Generated sentences must be short, with less than 15 words. New information can be introduced. Avoid using pronouns to confuse the subject of the sentence.\n\nSentence 1: Two people engage in conversation over an interesting topic. Sentence 2: Two people are having an interesting conversation

[Answer]
Two people are debating

[Rationale]
Change in wording should avoid direct agreement or contradiction with sentence 1 while maintaining the same general topic.The new sentence introduces a different word (debating) that suggests a slightly different nuance from the original sentence (engaging in conversation).



[Example 3]
[Instruction and Question]
Given a question, generate a paraphrase of that question wihout changing the meaning of it. Your answer should reword the given sentence, but not add information to it or remove information from it. The answer to your question should be the same as the answer to the original question.\n\nQuestion: what are some of hans christian andersen books?

[Answer]
what books did hans chrisian andersen make?

[Rationale]
Here's a paraphrase of the question: "what books did hans chrisian andersen make?"



[Example 4]
[Instruction and Question]
Classify the input text into either disfluent questions or proper questions. A disfluent question is a question that has some interruptions in it while framing. A proper question is the correct form of the question without any disfluency.\n\nWhat factors positively impacted Jacksonville following the war or rather negatively impacted it after the war?

[Answer]
disfluent

[Rationale]
The given text contains hesitation and interruption while framing the question. The phrase "or rather" interrupts the flow of the sentence and creates a disfluency.\nWe could infer that it is disfluent.



[Example 5]
[Instruction and Question]
Given a statement, generate a question such that the answer is contained in that statement.\n\nBinoculars are optical tools.

[Answer]
What are optical tools?

[Rationale]
The question should induce binoculars as an example of an optical tool. That would be "What are optical tools?".



[Example 6]
[Instruction and Question]
Given a trivia question, classify broad topical category from this list: 'theater', 'geology', 'book', 'tv', 'astronomy', 'aviation', 'military', 'government', 'boxing', 'projects', 'metropolitan_transit', 'law', 'venture_capital', 'broadcast', 'biology', 'people', 'influence', 'baseball', 'spaceflight', 'media_common', 'cvg', 'opera', 'olympics', 'chemistry', 'visual_art', 'conferences', 'sports', 'language', 'travel', 'location', 'award', 'dining', 'martial_arts', 'comic_strips', 'computer', 'user', 'tennis', 'music', 'organization', 'food', 'event', 'transportation', 'fictional_universe', 'measurement_unit', 'meteorology', 'distilled_spirits', 'symbols', 'architecture', 'freebase', 'internet', 'fashion', 'boats', 'cricket', 'film', 'medicine', 'finance', 'comic_books', 'celebrities', 'soccer', 'games', 'time', 'geography', 'interests', 'common', 'base', 'business', 'periodicals', 'royalty', 'education', 'type', 'religion', 'automotive', 'exhibitions'.\n\nAccording to Elvis Presley, what do you find at the end of Lonely Street?

[Answer]
music

[Rationale]
Elvis presely is a famous singer. Based on this fact, the question can be broadly classified as 'music'.



[Example 7]
[Instruction and Question]
Given a question as input, give its short_answer as the output \n\n1 acre equals how many hectares?

[Answer]
0.404686

[Rationale]
One acre is equal to 0.404686 hectares (approx).




[Example 8]
[Instruction and Question]
You will be given a trivia clue, and the category it belongs to. You should answer with the best answer that belongs in the category and is described by the clue. For simplicity, answers should be in all lower cased letters.\n\nCategory: AROUND THE WORLD \nClue: According to legend, this country's city of Cuzco was built by Nanco Capac, the first ruler of the Incas

[Answer]
peru

[Rationale]
The answer would be the country that has Cuzco as a city. Cuzco is a city in Peru.



[Example 9]
[Instruction and Question]
In this task, you have to generate the title of the recipe given its required ingredients and directions.\n\ningredients:  '4 whole Eggs', '1-2/3 cup Brown Sugar', '1-1/3 cup White Sugar', '1 teaspoon Vanilla', '2-2/3 teaspoons Baking Soda', '2/3 cups Butter', '2 cups Smooth Peanut Butter', '6 cups Rolled Oats', '1 cup Chocolate Chips', '1 cup Butterscotch Chips', '1 cup Smarties',<sep> directions: 'Mix the first 8 ingredients well. Stir in rolled oats, chocolate and butterscotch chips, and Smarties. Mix well.', 'Drop by tablespoonfuls onto a parchment paper-lined cookie sheet. Bake at 350 degrees Fahrenheit for 15 minutes.'

[Answer]
Monster Cookies

[Rationale]
The wide variety of ingredients, such as peanut butter, chocolate chips, butterscotch chips, and Smarties indicates it is a cookie recipe. The large size of the cookies also adds to the idea of them being "monster" cookies. The directions mention dropping the dough by tablespoonfuls onto a cookie sheet, which reinforces the idea of large, monster-sized cookies.



[Example 10]
[Instruction and Question]

    \end{lstlisting}
    






\begin{lstlisting}[caption={Demonstration used for dialogue prompt}, label=dialogue prompt,,captionpos=b]
[Example 1]
[Instruction and Question]
The input is taken from a negotiation between two participants who take the role of campsite neighbors and negotiate for Food, Water, and Firewood packages, based on their individual preferences and requirements. Given an utterance and recent dialogue context containing past 3 utterances (wherever available), output Yes if the utterance contains the self-need strategy, otherwise output No. self-need is a selfish negotiation strategy. It is used to create a personal need for an item in the negotiation, such as by pointing out that the participant sweats a lot to show preference towards water packages.\n\nContext: '2 packages of both of what?  Water and food?' 'I mean water and food' 'Is food something you need the most of or is it water?'\nUtterance: 'I need 2 packages of both food and water on each'

[Answer]
No

[Rationale]
The speaker simply states their need for two packages of food and two packages of water without indicating any particular preference or personal need for either item.\nFor example, if the speaker had said "I need two packages of water because I sweat a lot and need to stay hydrated," that would be an example of the self-need strategy.\nSince the utterance in this case does not use the self-need strategy, the answer is No.



[Example 2]
[Instruction and Question]
You are given a request sent by a user in a human and computer dialogue. Your task is to generate a prediction of what the requester is actually trying to do. The input is a sentence written by the human user, trying to achieve certain objective through this human computer interaction, and the output should be a sentence or two succinctly describing the human user's objective mentioned previously.\n\nTell me about Korean language.

[Answer]
Find information and resources on the Korean language

[Rationale]
The objective of the human user is to obtain information about the Korean language. So, I would find information and resources on the Korean language.



[Example 3]
[Instruction and Question]
Read an exchange between SENDER A and SENDER B, and answer whether SENDER A is a 'Bot' or a 'Human'. There are only a maximum of two senders (A and B) per conversation, and there are no conversations where SENDER A and SENDER B are both bots, or both humans; only one sender per conversation can be a bot, and only one can be a human. Knowing this, need only to classify whether SENDER A is 'Human' or 'Bot'; you do not need to write anything about SENDER B. Typically, bots will have a more disjointed manner of speaking, and will make statements that don't relate to each other, don't make coherent sense, or otherwise appear unnatural. Humans will make statements in a more or less coherent and logical way, potentially including asking the bot why the bot is not making sense or contradicting itself.\n\nSENDER A: In Las Vegas, you're an afterthought, you're an old so while we could go gambling, we could go hookering, we could get drunk, we could go to the convention...\nSENDER B: hello\nSENDER A: Hey, Mr. San Francisco, I'd like you to meet...\nSENDER B: i love britney spears and you?\nSENDER A: The New York City Ballet Gala, duh.\nSENDER B: i'm on a diet now, have you ever followed any diet?\nSENDER A: Thanks, Zeke! No way. My sparkle jelly stays on my wrist until I die, and then probably even after that.\nSENDER B: lol, sounds funny\nSENDER A: Big Buddha Cheese for muscle spasms.\nSENDER B: do you like spending your (or not so) money?\nSENDER A: Look, II have got... 25 bucks for food.\nSENDER B: could you give them to me?\nSENDER A: This Friday.  How much did you sell 'em?\nSENDER B: a lot, lol\nSENDER A: II'm not gonna sell them.\nSENDER B: well, why?..\nSENDER A: But I'll pay you like a regular customer.\nSENDER B: oh, thanks\nSENDER A: I... I don't mean selling programmes or finding prizes for the tombola.\nSENDER B: l should go, sorry\nSENDER A: I... I'd rather stay here, if you don't mind.\nSENDER B: no, i don't, stay here, buddy\nSENDER A: I'll go on alone.\nSENDER B: bye\nSENDER A: Hi, Les, I'll be right with you.\n

[Answer]
Bot

[Rationale]
SENDER A talks about Las Vegas, then the New York City Ballet Gala, and then Big Buddha Cheese for muscle spasms. These statements don't seem to follow a logical train of thought or relate to each other. Additionally, SENDER A's responses don't seem to reflect an understanding of what SENDER B is saying or engage in a meaningful conversation. Therefore, the output is 'Bot'.



[Example 4]
[Instruction and Question]
You have to predict the type of conversation sentence given as input. Your predictions can be one among: 'information' if the input statement states a piece of information, 'question' if the input statement asks for some information, 'directive' if the input sentence involves the management or guidance of operations, and 'commissive' when it constitutes a statement that commits the speaker to some future action.\n\n Basically , you just can't drive . 

[Answer]
information

[Rationale]
In this case, the sentence "Basically, you just can't drive" is not asking for any information or guidance, nor is it committing the speaker to any future action. Instead, it is simply conveying a piece of information about the listener's driving ability. Therefore, it can be classified as an "information" type of sentence."



[Example 5]
[Instruction and Question]
The input is a conversation between an automated system and a user looking for suggestions for pubs, restaurants and coffee shops in Cambridge. In the dialogue, the user may provide some criteria for the type of place they want such as price range, cuisine, etc. Given such a dialogue, output the price range the user if looking for which can take one of four values: Cheap, Moderate, Expensive and Don't Care. Output cannot be any other value than the four values. Note that if the user asks for the prices of items provided at a pub, restaurant or coffee shop but doesn't mention the price range they are looking for then the correct classification will be Don't Care. If the user does not mention a particular price, the correct classification will be Don't Care.\n\nSystem: 'Thank you for calling the Cambridge Information system. Your call will be recorded for research purposes. You may ask for information about a place to eat, such as a restaurant, a pub, or a cafe. How may I help you?'\nUser: 'sil'\nSystem: 'What kind of venue are you looking for? For example, a pub, or a restaurant.'\nUser: 'yes i uh im trying to find a mexican restaurant in the expensive price range'\nSystem: 'Nanna Mexico, which serves mexican and tex mex food, is a great restaurant. It's in the expensive price range.'\nUser: 'can i get the phone number and area of the venue'\nSystem: 'The phone number of Nanna Mexico is 01223902150. The price for main courses ranges from 5 pounds 35 to 15 pounds 50 and the price for a cappuccino is 2 pounds 95.'\nUser: 'the area of the restaurant'\nSystem: 'Nanna Mexico, which serves mexican and tex mex food, is an outstanding restaurant. It's in the expensive price range.'\nUser: 'the address of the restaurant'\nSystem: 'Nanna Mexico is located on 33 Regent Street.'\nUser: 'uh thank you goodbye'\n

[Answer]
Expensive

[Rationale]
"the user explicitly states that they are looking for a Mexican restaurant in the "expensive price range", and the system recommends Nanna Mexico which it confirms is indeed in the "expensive price range".So, the correct classification for the user's price range preference is "expensive"."



[Example 6]
[Instruction and Question]
In this task you will be given two dialogues. You need to determine if both dialogues have the same underlying emotion. The possible emotions are happy, sad, angry, or other. If they do output 'yes', if not output 'no'.\n\nDialogue 1: 'i am not understand what you are saying  really that's what i was saying pls clearly convey the msg don't irritate me'. Dialogue 2: 'so rude why u didnt ans my question'.

[Answer]
yes

[Rationale]
Both dialogues express a negative emotion, with Dialogue 1 expressing frustration and Dialogue 2 expressing anger. Therefore, both dialogues have the same underlying emotion, which is not a positive emotion like happiness, and not a neutral emotion.



[Example 7]
[Instruction and Question]
You are shown a random response (from a computer or a human). Generate a language query such that it leads to this reply.\n\nTotal travel time is 50 minutes. Would you like me to book it?

[Answer]
Not right now, thanks. As I'll be in the center part of town I'd like to find an expensive restaurant there. Do you have any suggestions?

[Rationale]
The user may have asked for restaurant recommendations in the central part of town, and after receiving the recommendations, the system could have informed the user about the travel time to their desired destination. The system may then offer to book the recommended restaurant for the user.




[Example 8]
[Instruction and Question]
In this task you are given a small conversation between two persons and 4 options on how the conversation should continue. Your job is to choose the most reasonable option. The conversation and the options are separated by a newline character. Each dialogue in the conversation are separated by a comma. F and M indicate female and male, respectively.\n\nM: Today we have Jane Couch with us. Good evening and welcome, Jane. ,F: Good evening. ,M: Now this is a question that everyone would like to ask you. How did you get interested in boxing? ,F: I saw a little television program about women boxing about 6 years ago. And at that moment, I said to myself, I'm going to do that. And it just changed my life. But my parents disagreed at first. ,M: And were you interested in sport at school? ,F: No, nothing. I wasn't ever fit or anything. I just don't look after myself at all. ,M: What do you think is the most important for a top boxer? ,F: Believe it or not, anyone can fight. But to make it to the top, you have to know your skills. And you have to have a quick brain and the fitness. The fitness side of it is the most important. ,M: And just lastly, when is your next big fight? ,F: The next one is going to be the end of February, and I will have another 2 at the beginning of April, and at the end of May. ,M: Well, that's great. We're looking forward to seeing your performance. Thank you very much Jane.\n(A) F: Yeah, I also look forward to my next fight in April. (B) F: Come and see my next fight in early February. (C) F: I\u2019m sorry, I don\u2019t understand. Could you say it again? (D) F: I have several months before my next fight in May.

[Answer]
C

[Rationale]
Option (A) and (B) are incorrect as they are not related to the conversation at hand and are not a natural continuation of the dialogue. Option (D) is incorrect as it provides incorrect information regarding the timing of the next fight. Option (C) is the most reasonable option as it is a natural response when one doesn't understand something and wants it to be repeated.



[Example 9]
[Instruction and Question]
You will be given a person's personality, and a history of a conversation this person has had. You will be given four candidate sentences to complete the conversation, based on the context. Choose one and answer with the text.\n\nPersonality: I've long hair for a boy.\nI draw during classes instead of taking notes.\nI wish I could live in a fantasy world.\nI like lizards.\nChat history: -Hello! How are you? What kind of things do you like?\n -My age is 25. I like to read. What do you like?\n -I love lizards, I used to have a firey orange bearded dragon!\n -Cool! How old are you?\n -I am 28. are you in school?\n -Nope. I work at the library. You?\n -I tend to draw in class, when I should take notes, but get good grades.\n -What is your major? I hated school, but do love my job.\n -Art. Do you have any cool dreams? Things you wish you could do?\n -Nope. Just like sitting in the library, all alone, nobody to bother me.\n -Sometimes I want to live in a magical world with unicorns and dragons.\n -Are you male or female? I hate all this competition between females.\nCandidates 1) Female. What color is your hair? Is it long or short? 2) Blood doesn't bother me very much haha. Do you like sports? 3) I teach 1st grade, I've a lot of sixth graders that are wonderful. 4) Not good. I want to drive fast but I can not because can be dangerous.

[Answer]
Female. What color is your hair? Is it long or short?

[Rationale]
The given personality traits and chat history do not provide any relevant context to the candidate options except for the fact that the person likes lizards. In the chat history, the person mentions drawing in class instead of taking notes, and their major is art. Therefore, the most reasonable response among the candidate options is to ask about their hair since it is a relevant question and is not out of context. Hence, the answer is "Female. What color is your hair? Is it long or short?"



[Example 10]
[Instruction and Question]

    \end{lstlisting}
    





