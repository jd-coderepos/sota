In this section, we present our experimental results. We
implemented Java code to compute the \pbgdistance\ defined above. 
Besides using real street-maps from Berlin and Athens we used a set 
of perturbed graphs to analyze our distance measure; see \subsecref{controlled}.
Our code is available on {\tt mapconstruction.org}.

\subsection{Datasets and Runtimes}
We test our algorithm using map data from Berlin and Athens.  For Berlin, we
have maps from two sources:  TeleAtlas (TA) from 2007 and
OpenStreetMap (OSM) from April 2013.  We have both small (16 km) and large (2500
km) maps of Berlin. Similarly, for Athens, we have TA maps from 2007 and OSM maps
from 2010.  To compute the \pbgdistance, we selected several rectangular
regions with the same
coordinates from each map; \tableref{datasets} contains the UTM coordinates of
the
southwest-most and northeast-most corners of the rectangular
regions, and Tables \ref{table-OSMstat} and \ref{table-TAstat} contains some
statistics about the datasets.  From these tables, we see that the OSM Berlin
maps contain more vertices and edges than the TA maps; however, the OSM
Athens-small map contains slightly fewer vertices and edges than the TA map.
\begin{table}[tp]
\centering
\tbl{Data Set Regions.\label{table-datasets}}
{
\begin{tabular}{|l|r|r|r|r|r|}
\hline
Data set & xLow & xHigh & yLow & yHigh& Area\\
\hline
\asmall\ & m& m& m& m& km 
 km\\
\bsmall\ & m& m& m& m& km 
 km\\
\blarge\ & m& m& m& m& km 
 km\\
\hline
\end{tabular}}
\begin{tabnote}
  The data set regions are defined by the extreme southwest  and
the extreme northeast  UTM coordinates.
\end{tabnote}
\end{table}

\begin{table}[t]
\centering
\tbl{Statistics for OSM maps.\label{table-OSMstat}}{\begin{tabular}{|l|r r  |r |r|}
\hline
&\multicolumn{2}{c|}{\# vertices}&\# edges& total length\\
&all&&&\\
\hline
\asmall\ &2,318&1,026&3,758&323 km\\
\bsmall\ &2,166&841&3,051&267 km\\
\blarge\ &87,395&31,777&123,863&17,552 km\\
\hline
\end{tabular}}
\end{table}

\begin{table}[t]
\centering
\tbl{Statistics for TA maps.\label{table-TAstat}}{\begin{tabular}{|l|r r  |r |r|}
\hline
&\multicolumn{2}{c|}{\# vertices}&\# edges& total length\\
&all&&&\\
\hline
\asmall\ &2,770&1,210&4,343&339 km\\
\bsmall\ &1,507&634&2,303&227 km\\
\blarge\ &49,605&19,897&72,650&10,426 km\\
\hline
\end{tabular}}
\end{table}

\begin{table}[t!]
\tbl{Runtimes.\label{table-runtime}}
{
\begin{tabular}{|l|r|r r|r|r|}
\hline
Dataset&Link & Teleatlas & OSM to &  Execution & Machine\\
&Length & to OSM & Teleatlas &  Mode & Specification\\
\hline
& LinkOne &4.9 min&6.1 min& &\\
\asmall\ & LinkTwo &22.0 min&27.8 min& &Intel(R) Xeon(R)\\
& LinkThree &74.9 min&98.4 min& sequential &CPU E3-1270 v2\\
\cline{1-4}
 & LinkOne & 5.8 min & 5.6 min & &  Ram\\
\bsmall\ & LinkTwo &24.4 min & 22.0 min &  & \\
& LinkThree &78.4 min & 69.0 min &  & \\
\hline
 &LinkOne& 4.0 h& 4.5 h& &\url{http://www.cbi.utsa.edu/}\\
 \blarge\ &LinkTwo& 15.0 h & 20.0 h &parallel &\url{hardware/cluster}\\
 &LinkThree& 49.0 h& 53.0 h& &\\
\hline
\end{tabular}
}
\end{table}

The path-based algorithm for computing the distance between two maps is scalable
and
has reasonable runtime; for the \bsmall\ dataset, it took 
minutes to
map all \length\ two paths of OSM to the TeleAtlas map.
The runtime and machine specification for corresponding experiments are
summarized in \tableref{runtime}.
Although this algorithm does not need to run in real-time, the computation can
be sped up by incorporating an efficient data structure for spatial search. As
the runtime is not our main focus in our current implementation, we perform
an exhaustive
search to find all points on the street-map which are close
to the start vertex of a path.

The algorithm can be trivially parallelized by decomposing the set of paths in
the first graph into
multiple sets and finding their (Fr\'echet-)closest paths in the
second graph independently. We implemented this parallel version of the
path-based distance computation.
For the \blarge\ dataset, there were almost  paths to consider.  When
we executed the parallel computation, we used  threads, resulting in
runtimes listed in \tableref{runtime}.
This simple
parallelization allowed for the path-based distance to be computed on the
\blarge\ map in only two days time, as opposed to taking weeks to compute.












\subsection{Computing the Path-Based Distance}
\label{subsec-goodVertices}
The results presented in \secref{distance} require that the graphs are
-separated and that no vertex is of degree three. 
In the datasets that
we use in this section,  of the vertices have degree three; see 
Tables~\ref{table-OSMstat} and~\ref{table-TAstat}.  If we
relax the degree three condition, then we can compare paths in one graph to
the (Fr\'echet-)closest path in the other graph.   In our experiments, we 
allow degree three vertices, at the cost of having a slightly less informative 
distance measure.
As discussed in \subsecref{length3}, even though there are contrived examples for which the approximation guarantees of our
theorems do not apply, we can still use the path-based distance to
capture differences between the graphs. 


\begin{figure}[htb]
    \centering
    \subfloat[Spatial.]
    {\includegraphics[scale=0.6]{goodVertex/spatial}}
    \subfloat[Topological.]
    {\includegraphics[scale=0.6]{goodVertex/topological}}
    \subfloat[Geometric.]
    {\includegraphics[scale=0.6]{goodVertex/structural}}
    \caption{Three typical examples where vertices that are not
    well-separated may arise: when one graph is a translation of the
    other, when one graph has more topological structure than the other, 
    and when
    seemingly corresponding edges (according to the topology of the graphs) have
    very different geometric embeddings.}
    \label{fig-badVertex}
\end{figure}
We look to Figures~\ref{fig-badVertex} and~\ref{fig-goodvsbad} to understand the
settings for which vertices are not well-separated.
\figref{badVertex} shows three
generic examples: poor separation due to spatial, topological, or geometric
differences in the graphs.
In practice, we see regions of well-separated vertices
and regions of vertices that are not well-separated; see \figref{goodvsbad}.
  \addtocounter{footnote}{-1}
  \begin{figure}[t]
  \centering \subfloat[Mostly well-separated region.]{\includegraphics[height=1.5in]
  {goodvsbad/athens_osm_linkthree_good}}
  \hspace{.2in}
  \subfloat[Mostly not well-separated region.]{\includegraphics[height=1.5in]
  {goodvsbad/athens_teleatlas_linkthree_bad} }
  \caption{We plot portions of the OSM and TA maps for
  the \asmall\ dataset\footnotemark.  The -separated
    vertices are displayed as blue circles and the other vertices as red stars.
  In (a), we see OSM
    (green) overlayed on TA (gray) where the road networks are very similar.  In
  (b), we see TA (green) overlayed on OSM
    (gray) where the TA map has more streets than the OSM one, resulting in a
  large number of vertices that are not -separated.
  }
  \label{fig-goodvsbad}
  \end{figure}
  \footnotetext{The 
  and -coordinates are the offset (in meters)
  from an arbitrary location, given in UTM
  coordinates.  That location is UTM Zone 34S,  meters east,
   meters
  north in (a) and UTM Zone 34S,  meters east,  meters
  north in (b).}

When the vertices are not -separated for sufficiently small , then
discerning the topological structure of the individual maps -- let alone the
differences between them -- becomes difficult.
For our three datasets, we counted the number of -separated vertices for
 and ; see
\tableref{goodbad}.  If we look at -separated vertices for the
\bsmall\ dataset, we see that  ( out
of ) vertices are well-separated.  Similarly, we found that  of the
TeleAtlas vertices in \bsmall\ are
-separated.

\begin{table}[t]
 \centering
  \tbl{Statistics about well-separated vertices.\label{table-goodbad}}
  {
  \begin{tabular}{|l|c|r r|r r|}
  \hline
  & &\multicolumn{2}{c|}{\# vertices}&\multicolumn{2}{c|}{\#-separated
vertices}\\
  Dataset&&OSM&Teleatlas&OSM&Teleatlas\\
    \hline
  && & &2,076 &2,020 \\
  \asmall\ && 2,318 & 2,770
&1,803 &1,616 \\
  &&&&1,490 &1,215 \\
  \hline
           && & &1,159 &1,149
\\
  \bsmall\ && 2,166 & 1,507
&811 &897 \\
           &&  &&510 &665 \\
  \hline
   &&&&38,402 &34,723 \\
  \blarge\ &&87,395&49,605&27,445 &32,344
\\
   &&&&18,898 &24,963 \\
   \hline
\end{tabular}
  }
\end{table}

\addtocounter{footnote}{-1}
\begin{figure}[t]
\subfloat[]{\includegraphics[width=0.3\textwidth]
{goodvsbad/growingbadregion_first}}
\hspace{0.1in}
\subfloat[]{\includegraphics[width=0.3\textwidth]
{goodvsbad/growingbadregion_second}}
\hspace{0.1in}
\subfloat[]{\includegraphics[width=0.3\textwidth]
{goodvsbad/growingbadregion_third}}
\caption{We demonstrate how a region of vertices that are not
-separated grows as  increases.
Here, we see TA (in green) overlayed on OSM (in gray) for a section of the
\asmall\ dataset\footnotemark.  In blue circles, we mark the
-separated vertices, and in red stars, we mark the vertices that are
not -separated.}
\label{fig-growingbad}
\end{figure}
\footnotetext{The 
and -coordinates are the offset (in meters)
from an arbitrary location, given in UTM
coordinates.  That location is UTM Zone 34S,  meters east,
 meters north.}

Since, in our theorems,
 depends on the directed
path-based distance between two sets of paths (one from each graph), that
implies that determining if a vertex is -separated depends not only on the
graph containing the vertex, but also on the second graph to which we compare
it.
Recall from \defref{goodVertex} that a vertex  is -separated if
 is finite. As  increases, the probability that  is infinite also
increases.   As  is non-decreasing with respect to , the number of vertices that
are not well-separated also grows with , as demonstrated in
\figref{growingbad} for a section of the \asmall\ dataset.


In Table~\ref{tab-distance}, we show the values of the path-based distance for our datasets.
Since the path-based distance is defined as the maximum of map-matching 
distances, see Definition~\ref{def-pbDistance}, it is dominated by the most 
dissimilar sections in the maps. We therefore also record the  
percentile and the mean value of the set of map-matching distances, weighted by the  
overall length of the paths. These quantities may be more suitable in practice 
as these are less sensitive to outliers than the maximum. 
The distribution of map-matching distances can provide insights into differences
between the graphs; however, unless the
assumptions of the theorems of \secref{distance} are satisfied, it will not
guarantee vertex, edge, and intersection
correspondences for every vertex, edge, and intersection.

We see in Table~\ref{tab-distance} that for \asmall, the distance from OSM to TA is consistently smaller than the distance from TA to OSM, which suggests that the TA map contains more detail than the OSM map. This behavior is reversed for \bsmall\ and \blarge\, where the distance from TA to OSM is consistently smaller, which suggests that the OSM map contains more detail. 
When considering the mean values, the \asmall\ maps in particular appear to 
generally be in good correspondence, for one link, two link, and three link 
paths. For all data sets the effect of the different link-lengths is clearly 
noticeable: For the mean values highlighted in gray in Table~\ref{tab-distance}, 
when increasing the link-length by one, the distances increase by  meters 
for the small data sets, and by  meters for the large data set.


\begin{table}[t]
\centering
\tbl{Path-Based Distance.\label{tab-distance}}{
\begin{tabular}{|l| c| cc|c c| c c|}
\hline
&  & \multicolumn{2}{c|}{Path-based distance} 
&\multicolumn{2}{c|}{-percentile} & \multicolumn{2}{c|}{Mean}\\
&& OSM& TA to &  OSM & TA to & OSM & TA to\\
Datasets& Path set   & to TA & OSM  & to TA & OSM & to TA & OSM\\
\hline
 &&150m&188m&11m&43m&\cellcolor{gray!25}8m&15m\\
 \asmall\ &&157m&251m&42m&71m&\cellcolor{gray!25}17m&27m\\
 &&166m&251m&71m&96m&\cellcolor{gray!25}30m&45m\\
 \hline
 &&203m&127m&70m&29m&25m&\cellcolor{gray!25}13m\\
\bsmall\ &&203m&130m&98m&61m&40m&\cellcolor{gray!25}24m\\
 &&210m&157m&121m&82m&63m&\cellcolor{gray!25}40m\\
\hline
 &&m&1,150m&652m&57m&m&\cellcolor{gray!25}26m\\
\blarge\ 
&&m&1,450m&896m&116m&m&\cellcolor{gray!25}50m\\
 &&m& 1,450m&1095m&174m&m&\cellcolor{gray!25}82m\\
\hline
\end{tabular}
}
\end{table}


When one map has more streets than the other, as is the case with the
Berlin-large map, the
path-based distance between them can be very large (as should be expected).
Since the Berlin-large map contains part of the city of Berlin
as well as parts of its southern suburbs (see Figures \ref{fig-blarget} and
\ref{fig-blargeo}), the local path-based distances observed by the edges and
the vertices are not randomly distributed; there are regions in which
the graphs are very similar, as well as regions where the graphs are dissimilar.
For this reason,
the value of the distance alone is difficult to interpret; therefore, we turn
to the local signatures to provide more insight.





\subsection{Using the Local Signature}
\label{subsec-experimentsSignature}

In this subsection, we study the local signature
 for a fixed  and
for a given integer . 
We first show how heat-maps can be used as a visualization tool for the local
similarity captured by the signature. This
can help identify similar regions in the map. In a different approach, we then
define a cumulative distribution function
in order to capture a summary of the local signature in terms of percentage of
the overall graph length.



\paragraph{Heat-Maps}
We compute heat-maps by coloring an edge lighter shades of yellow to darker
shades of red based on the value of that
signature (smaller to larger).  An edge  which is drawn in lighter
shade implies correspondences in  between 
and a subgraph induced by its neighborhood (depending on ) which is very
similar spatially, structurally and topologically.
On the other hand, a darker red edge implies one or more possible
dissimilarities: missing edge, missing vertex,
spatial/structural differences, etc.


First, we demonstrate the special features \length\ one, \length\ two and 
\length\
three signatures can capture.  As was shown in
\subsecref{lengthone}, the value  can bound distances between
edges and their corresponding paths, but fails to identify when a
vertex (i.e.,  connection between two edges) is missing.  In 
Figures~\ref{fig-lengthone} and
\ref{fig-linkbsmall}, the heat-map of the 
TA map is overlayed on the OSM map (in gray) of \bsmall.  In
\figref{lengthone}, we can see that edges which are in the TA map but do not 
have corresponding edges in the OSM map have a large directed distance 
(indicated by the red color), when signatures
are computed using \length\ one paths.  
At the same time, the \length\ one signature fails to identify 
the  difference between two edges that
become close and two edges that intersect.  This difference, however, can be
identified
using \length\ two paths; see the regions inside the green boxes in
\figref{linkbsmall}.


\addtocounter{footnote}{-1}
\begin{figure}[t!p!]
\centering \subfloat{\includegraphics[width=.45\textwidth]
{link1vslink3/linkOne_teleatlasme1}}\ \ \
\subfloat{\includegraphics[width=.45\textwidth]
{link1vslink3/linkOne_teleatlasme2}}
\caption{We plot the -signature of the TeleAtlas map, overlayed on
the OpenStreetMap (in
gray) for the \bsmall\ dataset\footnotemark.  
TA edges with a close OSM path have a low signature and
are shown in yellow.
TA edges without any corresponding path in OSM have a high signature and are 
shown in red.
}
\label{fig-lengthone}
\end{figure}
\footnotetext{The 
and -coordinates are the offset (in meters)
from an arbitrary location, given in UTM
coordinates.  That location is UTM Zone 33U,  meters east,
 meters north and UTM Zone 33U,  meters east,
 meters north respectively.}



Although \length\ two helps to find missing vertices, in some cases,
it fails to capture differences in more global notions of connectivity.
As one can see in the hypothetical example in
\figref{testlinkthree},  can be arbitrarily small with 
arbitrarily large.  \thmref{final} ensures that for , the
value of  cannot become arbitrarily larger than .







\paragraph{Cumulative Distribution of Local Signature Values}
For a given edge  and a given integer , let
 be the
path-based signature at . Similar to the distribution of 
discussed above, the distribution of  provides insight
into the similarity between  and .  Therefore, we investigate this
distribution in more detail.
Given a distance threshold  and a fixed link-length , we define the
weighted cumulative distribution of
 as follows:

where .  In \figref{percentilePB}, we plot
 and  for
three different data sets.  We interpret these plots as follows: assume
, then  of the points in
 observe a path-based distance of at most  meters.






\addtocounter{footnote}{-1}
\begin{figure}[ptbh]
\centering \subfloat[-signature]
{\label{linkonebsmall1}\includegraphics[width=.45\textwidth]
{link1vslink3/linkOne_teleatlas}}\ \ \
\subfloat[-signature]
{\label{linktwobsmall1}\includegraphics[width=.45\textwidth]
{link1vslink3/linkTwo_teleatlas}}

\subfloat[-signature]
{\label{linkonebsmall2}\includegraphics[width=.45\textwidth]
{link1vslink3/linkOne_teleatlasmv2}}\ \ \
\subfloat[-signature]
{\label{fig-linktwobsmall2}\includegraphics[width=.45\textwidth]
{link1vslink3/linkTwo_teleatlasmv2}}
\caption{
We plot the path-based signature of the TeleAtlas map, overlayed on the
OpenStreetMap (in gray) for the \bsmall\ dataset\footnotemark. Note that
   fails to capture missing vertices, but, as shown in the green boxes,
  shows a large 
distance when the gray TA map has an intersection (in Figures (a) and (c)) which is missing
in the colored OSM map (in Figures (b) and (d)). }
\label{fig-linkbsmall}
\end{figure}
\footnotetext{
The 
and -coordinates are the offset (in meters)
from an arbitrary location, given in UTM
coordinates.  
That location is UTM Zone 33U  meters east,
 meters north in (a)-(b), and UTM Zone 33U  meters east,
 meters north in (c)-(d).}

\begin{figure}[tbph]
\centering \subfloat[-signature]
{\label{linktwotest}\includegraphics[width=.34\textwidth]
{link1vslink3/linktwotest}}
\subfloat[-signature]
{\label{linkthreetest}\includegraphics[width=.34\textwidth]
{link1vslink3/linkthreetest}}
\caption{ can be arbitrarily small with 
arbitrarily large.}
\label{fig-testlinkthree}
\end{figure}


\begin{figure}[tbph]
\centering \subfloat[Athens-small, TA mapped to OSM.]{\includegraphics[width=.45\textwidth]
{percentilePB/athens_small_osm_teleatlas}\label{fig-percentilePBathens_teleatlas
}}
\subfloat[Athens-small, OSM mapped to TA.]{\includegraphics[width=.45\textwidth]
{percentilePB/athens_small_teleatlas_osm}\label{fig-percentilePBathens_osm}}

\subfloat[Berlin-small, TA mapped to 
OSM.]{\label{fig-percentilePBberlin_teleatlas}\includegraphics[
width=.45\textwidth]
{percentilePB/berlin_small_osm_teleatlas}
}
\subfloat[Berlin-small, OSM mapped to TA.]{\includegraphics[width=.45\textwidth]
{percentilePB/berlin_small_teleatlas_osm}\label{fig-percentilePBberlin_osm}}

\subfloat[Berlin-large, TA mapped to OSM.]{\includegraphics[width=.45\textwidth]
{percentilePB/berlin_large_osm_teleatlas}\label{
fig-percentilePBberlinl_teleatlas}}
\subfloat[Berlin-large, OSM mapped to TA.]{\includegraphics[width=.45\textwidth]
{percentilePB/berlin_large_teleatlas_osm}\label{fig-percentilePBberlinl_osm}}

\caption{For each point  in a graph , we assign to it the value
, where  is the edge containing
  .  Above, we plot the cumulative distribution of these assigned values.
  In particular, we take note of when the path-based distance is at most 
  meters (indicated by the vertical gray lines),
  since the path-based distance exceeding  meters indicates that the graphs
  are not similar near the edge .}
\label{fig-percentilePB}
\end{figure}





\figref{percentilePBberlin_osm} shows results for computing the distance from
the 2013 OSM map to the 2007 TA map for \bsmall.  Here, we can see  of the
streets (more precisely,  km out of
 km) have very close correspondence (less than or equal
 to  meters) to
the map of TA based on \length\ one.
That means that for  of  locations chosen in the OSM map, there exists a
path in  that has Fr\'echet distance 
at most  meters to the edge containing the chosen location.  Due to the 
large
Fr\'echet distance 
to any path in the TA
dataset, we can conclude that about  of the streets in this area of Berlin
were either omitted in the TA map or
may have been new constructions between  and .
This result is
consistent with the observation in Section~\ref{subsec-goodVertices} that
 
of the vertices are -separated.
On the other
hand,    of the streets that existed in the 2007 map have a
corresponding street in the 2012 map.  New
roads being built is a common 
occurrence, but removing roads is not.  Since these
street maps were taken from different sources, one explanation is that
different types of roads can be ignored by OSM
but would be recorded by TA.



Contrasting the \bsmall\ dataset is the \asmall\ dataset.  In \asmall, we see
that  of the streets from the 2007
TA map have corresponding link-length one paths in the 2010 OSM map, and 
of the OSM map
have a corresponding path in the TA map. Perhaps some of this discrepancy
can be explained by time.  The Berlin
road network had six years to change; whereas, the Athens dataset had only three
years to change.

Looking at  for  provides further insights.  The cumulative
distribution  gave
information akin to Hausdorff distance.  The distribution of 
describes how well short distances (link-length two) are preserved between the 
graphs.  The larger  gets, the longer
the paths are that need to be mapped from
 to~.  In the \blarge\ dataset, only ,  and  of the TA 
can find paths close to all
link-length one, two and three paths respectively. That also means  of streets in TA
are missing (or have dissimilar correspondence) in OSM,  streets in TA are dissimilar 
to OSM  because an adjacent turn and/or street is missing (similar to \figref{linktwobsmall2}).


\subsection{Experiments with Perturbed Data}
\label{subsec-controlled}


\begin{figure}[hb]
\centering
\subfloat[]
{\label{fig-pGraph-p10}\includegraphics[width=.25\textwidth]
{controlled/ex-p10}}
\hspace{1cm}
\subfloat[]{\label{fig-pGraph-p50}\includegraphics[width=.25\textwidth]
{controlled/ex-p50}}
\hspace{1cm}
\subfloat[]{\label{fig-pGraph-p90}\includegraphics[width=.25\textwidth]
{controlled/ex-p90}}
\caption{The original  is the grey graph, a
regular grid over . As the perturbation
index  increases, the perturbation of vertices in a graph increases.
The red graphs represent perturbed graphs for ,  and , 
from left to right.}
\label{fig-pGraphs}
\end{figure}


In order to assess the ability of our distance measure to quantify dissimilarity 
between 
maps, we created a map  and nine sets of estimations of that map with
increasing allowable deviations from .  The map  is a regular grid over
 using the coordinates with even integers as the
vertices; see the grey graph in \figref{pGraphs}.
The perturbation parameter is , which is allowed to be between zero and one.  
We perturbed
the vertex at
 in  by choosing two numbers  and  uniformly at random in the
interval  and moving the vertex at  to .
Thus, as  increases, the distance of the perturbed graph~ to the 
original graph  should increase as well, in expectation.  For each
value~, 
we generate  perturbed graphs: . For example,
see 
the red graphs in \figref{pGraphs} illustrating~, and .


\begin{table}[bht]
\tbl{Three sample maps.\label{table-theta}}
{
\begin{tabular}{|r|l|c|r|}
\hline
\multicolumn{1}{|c|}{} & \multicolumn{1}{c|}{} & 
\multicolumn{1}{c|}{} &
\multicolumn{1}{c|}{}
\\\hline\hline
 &  &  &  \\\hline
 &  &  &  \\\hline
 &  &  &  \\
\hline
\end{tabular}
}
\end{table}
We first take a look at the three graphs in \figref{pGraphs}.
The values for  and the distance 
 are listed in \tableref{theta}. 
We observe that when increasing the perturbation  parameter , the angle 
 decreases.  Hence, the upper bound for
 from \corref{runtime} becomes
quite large (for  it is  and for  it is ).
However, we notice that due to the perturbation scheme,
 since each coordinate can be
perturbed by at most . And, in fact, the link-three based distance is quite
close to that bound.


\begin{figure}[ht]
\centering
\includegraphics[width=.5\textwidth]{controlled/pmap-gtLinkThree}
\caption{For fixed , we show the boxplot for the
distribution of observed distances. We can see as  increases the path based
distances increase as well. The path based distance captures dissimilarities at
different levels.
\label{fig-controlled_result}}
\end{figure}
We compute  for each  and  summarize
our results in \figref{controlled_result}, using boxplots to illustrate the
distribution of path-based distances for each . We observe that
the path-based 
distance   increases as the perturbation parameter  increases, indicating 
that our distance measure can capture dissimilarities of varying
levels. Graphs that are more similar (lower values of ) have the smallest distances, and graphs that are more dissimilar (higher values of ) have the largest distances. 










\subsection{Comparison with {Biagioni and Eriksson 2012}}
\label{subsec-comparejames}
In this section, we compare our distance measure with the sampling-based
distance measure presented in
\subsecref{comparingStreetMaps}.  We used the code provided
by \citeN{Biagioni:2012:MIF:2424321.2424333}, making
modifications to allow for a different input format as well as to make the
output comparable to the path-based signature presented in this
paper.  In particular, we take vertices (including degree
two) from a graph as the seed locations
instead of the random sampling used in \cite{Biagioni:2012:MIF:2424321.2424333}.
The resulting marbles and holes distance (-score) has three
parameters:
\begin{enumerate}
\item{ Sampling density: how densely the map should be sampled (marbles for
generated map and holes for ground-truth
    map); we use one sample every five meters.}
\item{ Matched distance: the maximum distance between a matched marble-hole
pair; we vary from  to  meters.}
\item{ Maximum path length: from seed, the maximum distance from start location
one will explore; we use  meters.}
\end{enumerate}

Before commenting on the differences between the signatures, we first compare
and contrast the -score to the path-based
distance.
We compute the -scores for the Berlin-small dataset by varying the {\em
matched distance} from  meters to  meters, summarizing the results in
\figref{fscore}.
Here, we can see the -score is quite low and this finding is
consistent with the observation we can draw from
\figref{percentilePBberlin_osm} that  of the roads in OSM Berlin-small
map are new construction.
Although the two Berlin-small maps do not look very dissimilar,
the addition of more roads means that the topology of the maps has
changed.  Even if these changes are localized to a small area, the addition of
topological features punishes the whole graph for being dissimilar in a
tiny portion.  Choosing the matched distance to be  meters,
the -score is only  .
This computation took 
minutes, which is on the same order of magnitude as the
computation of our distance measure.



\begin{figure}[tbph]
\centering
\includegraphics[width=0.45\textwidth]{comparejames/fscore}
\caption{By increasing the matched distance threshold, the -score also
increases as it is easier for marbles and holes to be matched.  This
threshold is just one of the three parameters to the approach of
\citeN{Biagioni:2012:MIF:2424321.2424333}.}
\label{fig-fscore}
\end{figure}


\begin{figure}[tbph]
\centering
\subfloat[-score-signature]
{\label{fig-james_signature}\includegraphics[width=.45\textwidth]
{comparejames/james-berlin-signature}}\ \ \
\subfloat[-signature]
{\label{fig-signature_linktwo}\includegraphics[width=.45\textwidth]
{comparejames/linktwo-signature-berlin-small}}
\caption{OSM map overlayed on the TA map (in gray) for the \bsmall\ dataset\footnotemark.
The matched distance is  meters.
Note that the gray TA map has an intersection, while the colored OSM map does not. The  signature captures this as indicated by the darker orange color.
}

\label{fig-comparejames}
\end{figure}
\footnotetext{The 
and -coordinates are the offset (in meters)
from an arbitrary location, given in UTM
coordinates.  That location is UTM Zone 33U  meters east,
 meters north.}


In order to compare two distance measures in a finer level, we compute
-scores
for each start location individually
and compute edge signatures by averaging the -scores at the two endpoint
vertices.   We compute this -score signature and plot its heat-map in an
analogous fashion as we do for the path-based
local signatures: we color an edge
yellow if it observes similar behavior in both graphs (i.e., high -score) and
red if the distance indicates that the
graphs are dissimilar (i.e., low -score).


Balancing the tuning parameters above is difficult. \figref{comparejames} shows
a case where the graph sampling based
distance measure fails to capture the difference between the two road networks
due to the fact that the maximum path
length was set too high.  In gray, we plot the TA map, and we overlay
that with the OSM map colored
according to the adapted -score in \figref{james_signature} and our local
signature in \figref{signature_linktwo}.
In the green box, we notice that there is an intersection in the OSM map that is
missing in the TA map, since one of the
streets ends before it meets the other street.  The -score measure fails to
identify the missing intersection since
there is a detour available to reach the other road within  meters of the
seed (the intersection in the green box).

    \begin{figure}[tbph]
    \centering
    \includegraphics[height=1in]{comparejames/toyexample}
    \caption{We consider evaluating the two distance measures on these two
    paths (taken to be graphs).  In this
      example, one graph is straight and the second graph oscillates frequently,
  but
    the deviation from the straight line
      path is relatively small.  Hence, the path-based distance is small
  indicating
    that the graphs are very close, but the
      -score will be close to zero indicating that the graphs are very
    dissimilar.}
    \label{fig-james-toyexample}
    \end{figure}

Again, as this distance measure is based on one-to-one correspondence, it picks
up streets on the OSM map very
clearly (in red), which are missing in the TA map;  whereas our
measure picks them based on their proximity to the nearest street (darker yellow to
orange); see lower right corner of \figref{comparejames}.
There are examples for which our distance measure would find the graphs to be
similar, but the -score would be
very low indicating that the graphs are not similar.  For example, consider
\figref{james-toyexample}. The first graph (in orange)
consists of exactly one straight line path.
The second graph (in cyan) consists of one path close to the orange one, but
oscilates frequently.  Depending on the circumstances, one distance measure
would be preferred over the other.  If the exact distance of the path traveled
matters (e.g., in computing the cost of transporting goods), then
using the -score may be preferred; whereas, if the topology of the map is
important (e.g., when deciding if roads have been closed or new roads
added), then the path-based distance would be preferred.

The above illustrates one of the strengths of our distance measure: no tuning
parameters must be adjusted in order to
get a meaningful distance measure. Moreover, we have theoretical guarantees that
capture the difference in navigation
patterns between the two~graphs.

