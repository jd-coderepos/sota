

\section{Preliminary Methodology}
\label{sec:methoresults}


Our measurement study proceeds in four steps: (1) selection of websites; (2) mapping domain names to IP addresses; (3) mapping these IP addresses to IP prefixes routed in the Internet and their origin ASes; and (4) validating if these ASes support RPKI. The purpose is to explore how widely deployed RPKI is among web providers. We now describe our methodology, which is meant to be simple and widely reproducible. All data will be made available.

\iffalse
\begin{figure}
  \includegraphics[width=\columnwidth]{toolchain}
  \caption{Toolchain overview.}
  \label{fig:toolchain}
\end{figure}
\fi


\paragraph*{(1) Selecting Domain Names} 


To explore RPKI deployment among websites, it is necessary to first select a sample set. To achieve this, we extract the top 1 million websites from the Alexa list~\cite{alexa-ranking}. This is the default approach taken by most web measurements (\eg \cite{amsu-wcc-11,rkw-ddatt-12,lgs-bspdj-13,ctbo-qmclm-14}), and allows us to explore how RPKI deployment varies across popularity~ranks.





\paragraph*{(2) Mapping Domains to IP Addresses}



To map domain names taken from the Alexa into IP addresses, we utilise several public resolvers (to improve reproducibility).
Using Google DNS, we collect all \texttt{A}, \texttt{AAAA}, and \texttt{CNAME} records for all the Alexa domain names, including the names appended
with the prefix \url{www}. We refer to the former as a ``w/o www domain''. The measurements were performed from Berlin repeatedly over several weeks in 2014 and 2015.
We also used Open DNS and \url{us01} of the DNS Looking Glass~\cite{dns-lg} to verify that the results returned from Google DNS were not miscellaneous.
We obtain similar results from all three datasets.
We exclude all invalid DNS answers, \ie all
special-purpose IPv4 and IPv6 addresses reserved by the IANA.

Note that, although the distributed nature of DNS may lead to  answers that vary between locations, we delay exploration of CDN redirection strategies to future work.
This is because we are primarily interested in checking whole ASes, rather than where individual clients are redirected to.
In Section~\ref{sec:results}, we show that our main results remain independent of the DNS server selection because CDNs are reluctant to create ROAs at all.
Analyzing the effects of many vantage points will be part of our future work.















\begin{figure}
  \includegraphics[width=\columnwidth]{prefixmatch-www-small}
  \caption{Comparison of IP deployment for www and w/o www domain names.}
  \label{fig:prefixmatch-www}
\end{figure}


To briefly analyze the overlap between www and w/o www domains, we quantify
the amount of equal prefixes per domain. Figure~\ref{fig:prefixmatch-www}
shows that for the first 100k domains more than 76\% of the IP prefixes are
equal for both names. For the remaining domains, more than 94\% of the names
refer to the same prefix. As a side observation, in future work it should be explored how this fact can help accelerate continuous DNS measurements.



\paragraph*{(3) Mapping IP Addresses to Prefixes and ASNs}
To map addresses to ASNs, we take dumps of the active tables of the RIPE RIS route servers. For each IP address of a domain
name, we extract all covering prefixes and derive the origin AS from the AS
path (i.e., the right most ASN in the AS path). Note that entries with an
AS\_SET are excluded from our study as this leads to an ambiguity of the attribute, which is why the 
function is deprecated with the deployment of RPKI~\cite{RFC-6472}.


\paragraph*{(4) RPKI Validation}
For the validation of the BGP data, we follow the necessary steps to
perform origin validation at BGP routers. ROA data of all trust anchors (APNIC, AfriNIC, ARIN, LACNIC, and
RIPE) are collected and validated. Only cryptographically correct ROAs are
further used to check the IP prefixes obtained 
from the BGP table dumps.


The outcome of this process is a comprehensive list of all Alexa websites that \one can be resolved from our DNS vantage point and \two mapped to an IP prefix AS pair.
This list is \three annotated with RPKI origin validation outcome.






\section{Results}
\label{sec:results}


We have used the above methodology to collect data on RPKI deployment for all 1 million Alexa domains.
After excluding 0.07\% incorrect DNS answers, we gathered 1,167,086 IP addresses
for the www domains and 1,154,170 IP addresses for the w/o www domains.
These addresses map to 1,369,030 and 1,334,957 different prefix-AS
pairs respectively. 0.01\% of the IP addresses are not reachable from our
BGP vantage~points.

In the following subsections, we present the core RPKI
validation outcomes, and explore reasons for the observed deployment
state. For better visibility, we do not present results per domain but
apply a binning of 10k domains in all graphs, after experimenting with different bin sizes. As a domain name may refer to multiple IP
addresses, which may belong to different IP prefixes and ASes, several RPKI
states may exist per domain. To represent heterogeneous RPKI deployment, we assign corresponding probabilities to domain names (\eg 3/5 or 60\% RPKI coverage of \url{foo.bar}).




\subsection{Less Popular Content is More Secured}

We first explore how many websites are secured via RPKI, shown in Figure~\ref{fig:rpki-validation}.
On average, only 6\% of the web server prefixes are covered by RPKI (either
correctly or incorrectly announced in the BGP). This is somewhat worrying considering the international prominence
of the websites under study.
However, it also effectively illustrates that web operation and network operation are decoupled.
Roughly 0.09\% of the
prefixes are invalid according to the RPKI prefix origin validation, spread evenly across all Alexa ranks. This
observation is in qualitative agreement with the general RPKI deployment. Note
that the current invalid BGP announcements do not necessarily indicate
hijacking, but rather potential misconfiguration \cite{wms-tdbrh-12}. The amount of
invalids is evenly distributed among all web domains.


\begin{figure}
\includegraphics[width=0.49\textwidth]{rpki-validation-details}
  \caption{RPKI validation outcome for the 1 million Alexa domains (\emph{valid} =
  the origin AS is allowed to announce the prefix, \emph{invalid} =  the
  origin AS is not allowed to announce the prefix, and \emph{not found} =
  the announced prefix is not covered by the RPKI).}
  \label{fig:rpki-validation}
\end{figure}




Perhaps more interesting is the distribution of RPKI support across the different popularity rankings.
By inspecting Figure~\ref{fig:rpki-validation}, distinct trends
can be seen. 
Perversely, domains with a high rank (\ie
popular sites in the top 100k) are less likely to be secured than less popular sites.
Among the first 100k domains (\eg \url{google.com}, \url{blogspot.com}), only 4.0\% of web server prefixes are
secured via RPKI.
In contrast, for the last 100k domains, 5.5\% are secured. The absolute numbers are small, but a clear trend is visible and may
reflect the deployment strategy of different stakeholders.


\iffalse
\begin{table}
\begin{tabular}{rlc|rlc}
\toprule
\multicolumn{2}{l}{\textbf{Rank \& Domain}} & \textbf{RPKI} &
\multicolumn{2}{l}{\textbf{Rank \& Domain}} & \textbf{RPKI}
\\
1 & google.com & \xmark
&
6 & wikipedia.org & \xmark
\\
2 & facebook.com & \cmark
&
7 & qq.com & \xmark
\\
3 & youtube.com & \xmark
&
8 & linkedin.com & \xmark
\\
4 & yahoo.com & \xmark
&
9 & live.com & \xmark
\\
5 & baidu.com & \xmark
&
10 & twitter.com & \xmark
\\
\bottomrule

\end{tabular}
\caption{Top10 Alexa domains and their RPKI coverage.}
\label{tbl:top10sites}
\end{table}
\fi

Table~\ref{tbl:top10rpkisites} shows the first top 10 domains that have RPKI support, \ie either all or at least one prefix which is associated with the domain name is part of the RPKI.
It is clearly visible that \one almost all of the very popular sites are unsecured; and \two sometimes there is differing RPKI support for a website's www and w/o www domains; and \three most of the content is only partially secured.
We now analyze the deployment by CDNs in more detail.


\begin{table}
\center
\small
\begin{tabular}{rlcc}
\toprule
& & \multicolumn{2}{c}{\textbf{RPKI Coverage}}
\\
\cmidrule{3-4}
\multicolumn{2}{l}{\textbf{Alexa Rank \& Domain Name}} & \textbf{www} & \textbf{w/o www}
\\
2 & facebook.com & \cmark\xspace(3/3) & \cmark\xspace (2/2)
\\
70 & cdncache1-a.akamaihd.net & n/a & \pmark\xspace (1/3)
\\
73 & huffingtonpost.com & \pmark\xspace (1/3) & \xmark\xspace (0/3)
\\
92 & cnet.com & \pmark\xspace (1/3) & \xmark\xspace (0/2)
\\
95 & dailymail.co.uk & \pmark\xspace (1/3) & \xmark\xspace (0/1)
\\
117 & indiatimes.com & \pmark\xspace (1/3) & \xmark\xspace (0/1)
\\
120 & kickass.to & \ \pmark\xspace (1/10) & \ \pmark\xspace (1/10)
\\
130 & booking.com & \cmark\xspace (4/4) & \cmark\xspace (2/2)
\\
\bottomrule

\end{tabular}
\caption{Top~10 Alexa domains that have partial (\pmark) or
full (\cmark) RPKI coverage, including number of prefixes.}
\label{tbl:top10rpkisites}
\end{table}








\subsection{CDN Content Benefits from 3\textsuperscript{rd} Party ISPs}


CDNs are a critical part of modern web delivery. Many websites rely on them to deliver their content.
Next, we focus on the RPKI deployment of CDNs. To study this, we search the RPKI repository for attestation
objects that belong to the ASes of well-known CDNs. Specifically, we inspect the infrastructures of Akamai, Amazon,
Cdnetworks, Chinacache, Chinanet, Cloudflare, Cotendo, Edgecast, Highwinds,
Instart, Internap, Limelight, Mirrorimage, Netdna, Simplecdn, and Yottaa.
It is worth noting that the results of this approach do \emph{not} depend
on DNS measurements and thus do not include a bias that might
result from any DNS measurement point.









To derive the AS numbers of these CDNs, we apply keyword spotting on common AS assignment lists.
This leads to a lower bound for the current state of deployment.
We discover 199 ASes operated by these CDNs.
From these, we find only four entries in the RPKI.
These four prefixes are owned by Internap and are tied to three origin ASes. One might mistakenly think that Internap has therefore engaged widely with RPKI. However, Internap operates at least 41 ASes, the bulk of which are not secured via RPKI. No other CDN has made any deployment. Thus, these CDNs do not actively participate in the creation of RPKI attestation objects. This is in contrast to web hosters or common ISPs that, as shown previously, have far higher levels of penetration ().




Another interesting trend has been for CDNs to place caches in third party networks (\eg eyeball ISPs).
This allows the CDN to ``inherit'' RPKI support from the third party network.
Based on our previous observation, \ie these CDNs do not create ROAs, we know that every RPKI-enabled CDN-content is served by a third party network.





\subsection{Are the CDNs to blame?}

To quantify the impact that CDNs have on global website support for RPKI, we conduct a basic classification of CDN domains.
Generally, CDNs use CNAME chains (Canonical Names) to redirect DNS requests to
their caches. We exploit this fact to identify how many of our Alexa domains use CDNs.
We say a domain is served by a CDN, if the IP address
of its domain name is indirectly accessed via two or more CNAMEs (\eg
\url{www.huffingtonpost.com} 
\url{www.huffingtonpost.com.\\edgesuite.net} 
\url{a495.g.akamai.net}  \url{212.201.100.136}).
We confirm this rough heuristic by comparing its results with an independent
classification provided by HTTP\-Archive. HTTP\-Archive classifies the first
300k Alexa domains based on DNS pattern matching of CNAMEs, which is distinct from our test of DNS indirections. Furthermore, the HTTPArchive monitoring agent is located in Redwood City, CA, USA, and thus a geographically separated vantage point.

\begin{figure}
  \includegraphics[width=\columnwidth]{cnameprob-httparchive-google-www-small}
  \caption{Popularity of CDNs---comparison of CDN detection heuristics for 1M Alexa domains.}
  \label{fig:cnameprob-httparchive-google-www}
\end{figure}

Figure~\ref{fig:cnameprob-httparchive-google-www} compares the distributions of 
CDN-hosted web domains as determined by our classification approach and  
HTTPArchive (in 10k bins). The two almost identically shaped curves clearly
indicate that popular websites are more likely to be served by CDNs. 
Quantitatively, our approach indicates fewer CDNs than HTTPArchive. This is not surprising, since there are CDN deployments without CNAME chains. However, a conservative (un\-der)-estimate of CDN domains sharpens our view on the RPKI-protection of CDN domains: (Over-)en\-larg\-ing the set of CDN domains will mix deployment cases and diffuse the overall picture.

Integrating the above results, we now explore why highly ranked websites do not support RPKI. Specifically, we focus on the relation between RPKI-enabled and CDN-served websites.
Figure~\ref{fig:rpki-cdn-prob} depicts the distribution of RPKI-enabled websites across the Alexa rankings. The figure separates websites into those that utilise a CDN and those that do not.
In contrast to the Alexa domains at large, RPKI deployment is fairly independent of the rank for CDNs. Results fluctuate around an average of . This is almost an order of magnitude lower than the overall RPKI deployment rate, which is plotted for comparison.


Combining the lines of argument, we have shown that \one~CDN deployment is strongly enhanced for popular domains, but \two~RPKI deployment for CDNs is low---independent of content popularity. As a result, a high density of CDN sites reduces the RPKI-enabled portion of domains.
This holds for those ranks of the Alexa list where CDNs are more common: at the top ranks.
In summary, the observable degradation of routing security for popular websites is caused by the resistance of CDN operators to adopt RPKI in their ASes.




\begin{figure}
  \includegraphics[width=\columnwidth]{cdn-rpki-prob}
  \caption{RPKI deployment statistics on CDNs and for the unconditioned Web.}
  \label{fig:rpki-cdn-prob}
\end{figure}

