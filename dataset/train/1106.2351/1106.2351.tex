\documentclass[11pt,letter]{article}
\usepackage{amsmath}
\usepackage{amsfonts}
\usepackage{amssymb}
\usepackage{graphicx}
\usepackage{enumerate}
\usepackage[ruled]{algorithm2e}
\usepackage{tabularx,booktabs}

 \textwidth 16cm \textheight 23cm \oddsidemargin 0cm \evensidemargin 0cm
\voffset -2cm


\newtheorem{thm}{Theorem}[section]
\newtheorem{prop}[thm]{Proposition}
\newtheorem{cor}[thm]{Corollary}
\newtheorem{lemma}[thm]{Lemma}
\newtheorem{de}[thm]{Definition}

\newenvironment{proof} {\par \noindent \textbf{Proof: }}{\QED \par \bigskip \par}
\newcommand{\QED}{\hfill}
\newcommand{\rz}{\vspace{0.1cm}}

\renewcommand{\thefootnote}{\fnsymbol{footnote}}

\title {
    \bf {On vertex covers and matching number \\ of trapezoid graphs}
}


\author
{
{\large \sc Aleksandar Ili\' c \footnotemark[3]} \\
{\em \normalsize Faculty of Sciences and Mathematics, Vi\v segradska 33, 18000 Ni\v s, Serbia} \\
{\normalsize e-mail: { \tt aleksandari@gmail.com }}
\and
{\large \sc Andreja Ili\' c} \\
{\em \normalsize Faculty of Sciences and Mathematics, Vi\v segradska 33, 18000 Ni\v s, Serbia} \\
{\normalsize e-mail: { \tt andrejko.ilic@gmail.com }} }

\begin{document}

\maketitle


\begin{abstract}
The intersection graph of a collection of trapezoids with corner points lying on two parallel lines
is called a trapezoid graph. Using binary indexed tree data structure, we improve algorithms for
calculating the size and the number of minimum vertex covers (or independent sets), as well as the
total number of vertex covers, and reduce the time complexity from  to ,
where  is the number of trapezoids. Furthermore, we present the family of counterexamples for
recently proposed algorithm with time complexity  for calculating the maximum cardinality
matching in trapezoid graphs.
\end{abstract}

{\bf {Keywords:}} trapezoid graphs; vertex cover; matching; algorithms; data structures; binary
indexed tree. \vspace{0.2cm}

{{\bf AMS Classifications:} 05C85, 68R10.} \vspace{0.2cm}

\footnotetext[3] { Corresponding author. }


\section{Introduction}


A trapezoid diagram consists of two horizontal lines and a set of trapezoids with corner points
lying on these two lines. A graph  is a trapezoid graph when a trapezoid diagram exists
with trapezoid set , such that each vertex  corresponds to a trapezoid  and an
edge exists  if and only if trapezoids  and  intersect within the
trapezoid diagram. A trapezoid  between these lines has four corner points , ,
 and  -- which represent the upper left, upper right, lower left and lower right corner
points of trapezoid~, respectively. No two trapezoids share a common endpoint.

Ma and Spinrad \cite{MaSp94} showed that trapezoid graphs can be recognized in  time, while
Mertzios and Corneil \cite{MeCo09} designed structural trapezoid recognition algorithm based on the
vertex splitting method in  time, which is easier for implementation. Here 
stands for the number of vertices and  for the number of edges. Trapezoid graphs are perfect,
subclass of cocomparability graphs and properly contain both interval graphs and permutation
graphs~\cite{Li07}.

Trapezoid graphs were first investigated by Corneil and Kamula \cite{CoKa87}. These graphs and
their generalizations were applied in various fields, including modeling channel routing problems
in VLSI design \cite{DaGoPi88} and identifying the optimal chain of non-overlapping fragments in
bioinformatics \cite{AbOh05}. Many common graph problems, such as minimum connected dominating sets
\cite{TsLiHs07}, all-pair shortest paths \cite{MoPaPa02}, maximum weighted cliques \cite{BePaPa02},
all cut vertices \cite{HoPaPa04}, chromatic number and clique cover \cite{FeMuWe97}, all hinge
vertices \cite{BePaPa03}, minimum vertex covers \cite{LiCh09} in trapezoid graphs, can be solved in
polynomial time. For other related problems see \cite{ChCo96,CrGa10,Li94,Li95}.

Let  be a simple undirected graph with  and . A subset  is called a vertex cover (VC for short) of  if and only if every edge in  has at least one
endpoint in . A vertex cover  is called a minimal vertex cover if and only if no proper
subset of C is also a vertex cover. The size of a vertex cover is the number of vertices that it
contains. A vertex cover with minimum size is called a minimum vertex cover. Notably, a minimum VC
is always a minimal VC, but a minimal VC may not be a minimum VC.

Given a graph  and a fixed parameter , determining whether  has a vertex cover of at most
 vertices is one of the best-known NP complete problems \cite{Ka72}. Many recent studies have
focused on developing faster algorithms for the VC problem in special classes of graphs. The number
of vertex covers in a graph is important, particularly because this characteristic typically arises
in problems related to network reliability \cite{PrBa83}. Okamoto et al. \cite{OkUnUe05} proposed
 time algorithms for counting the numbers of independent sets and maximum independent
sets in a chordal graph. Lin et al. \cite{Li07,LiCh08} considered interval graphs and obtained
efficient linear  algorithms for counting the number of VCs, minimal VCs, minimum VCs, and
maximum minimal VCs in an interval graph. Recently, Lin and Chen \cite{LiCh09} presented 
algorithms for the same vertex cover properties in a trapezoid graph.

In graph theory, the notions of vertex cover and independent set are dual to each other. A subset
 is called an independent set of  if and only if every edge in  is incident on
no more than one vertex in .

\begin{thm}[\cite{CoLeRiSt01}]
\label{thm-ind} A set  is a minimal (minimum) vertex cover of  if and only if  is an
maximal (maximum) independent set of .
\end{thm}

A matching in a graph is a set of edges in which no two edges are adjacent. A single edge in a
graph is obviously a matching. A maximum matching is a matching with the maximum cardinality and
the cardinality of the maximum matching is called a matching number. Currently, the best known
algorithm for the constructing maximum matching in general graphs is Edmonds' algorithm
\cite{Ga80,MiVa80}, which is based on the alternating paths, blossom and shrinking and runs in time
. For special classes of graphs, such as interval, chordal and permutation
graphs, more efficient algorithms were designed (see \cite{AnAtChLe00,Ch96,ChPaCh97} and references
therein). Ghosh and Pal \cite{GhPa05} presented an efficient algorithm to find the maximum matching
in trapezoid graphs, which turns out to be not correct.

The rest of the paper is organized as follows. In Section 2 we introduce the binary indexed tree
data structure. In Section 3 we design  time dynamic programming algorithm for
calculating the size of the minimum vertex cover, the number of vertex covers and the number of
minimum vertex covers, improving the algorithms from \cite{LiCh09}. In Section 4 we present family
of counterexamples for  algorithm from \cite{GhPa05} for finding the maximum matching in
trapezoid graphs.



\section{Binary indexed data structure}


The binary indexed tree (BIT) is an efficient data structure introduced by Fenwick \cite{Fe94} for
maintaining the cumulative frequencies. The BIT was first used to support dynamic arithmetic data
compression and algorithm coding.

Let  be an array of  elements. The binary indexed tree supports the following basic
operations:
\begin{enumerate}[()]
\item for given value  and index , add  to the element , ;
\item for given interval , find the sum of values , .
\end{enumerate}



Naive implementation of these operations have complexities  and , respectively. We can
achieve better complexity, if we speed up the second operation which will also affect the first
operation. Another approach is to maintain all partial sums from  to  for . This way the operations have complexities  and , respectively.

The main idea of binary indexed tree structure is that sum of elements from the segment 
can be represented as sum of appropriate set of subsegments. The BIT structure is based on
decomposition of the cumulative sums into segments and the operations to access this data structure
are based on the binary representation of the index. This way the time complexity for both
operations will be the same . We want to construct these subsegments such that each
element is contained in at most  subsegments, which means that we have to change the
partial sums of at most  subsegments for the first procedure. On the other hand, we want to
construct these subsegments such that for every  the subsegment  is
divided in at most  subsegments. It follows that these subsegments can be , where  is the position of the last digit  (from left to right) in the binary
representation of the index . We store the sums of subsegments in the array  (see
Algorithm 1 and Algorithm 2), and the element  represents the sum of elements from index
 to . The structure is space-efficient in the sense that it needs the same
amount of storage as just a simple array of  elements. Instead of sum, one can use any
distributive function (such as maximum, minimum, product). We have the following

\begin{prop}
Calculating the sum of the elements from  to , , and updating the
element  in the binary indexed tree is performed in  time.
\end{prop}

The fundamental operation involves calculating a new index by stripping the least significant 1
from the old index, and repeating this operation until the index is zero. For example, to read the
cumulative sum , we form a sum

An illustrative example is presented in Table 1.

\begin{table}[ht]
\centering \begin{tabular} {l|llllllllllllllll}
\toprule

 & 1 &  2 &  3 &  4  & 5 &  6 &  7 &  8 &  9 &  10 & 11 & 12 & 13 & 14 & 15  & 16 \\
\midrule
 &  1 &  0 &  2 &  1 &  1 &  3 &  0 &  4 &  2 & 5 &  2 &  2 &  3 &  1 &  0 &  2 \\
 &  1 & 1 &  2 & 4 &  1 &  4 & 0 & 12 & 2 & 7 & 2 &  11 & 3 &  4 & 0 & 29\\
Cumulative sums &  1 &  1 &  3 &  4  & 5 &  8 &  8 &  12 & 14 & 19 & 21 & 23 & 26 & 27 & 27 & 29 \\
\bottomrule
\end{tabular}

\caption{An example with array  of length , binary indexed tree  and cumulative
sums.}
\end{table}




Using the fast bitwise AND operator, we can calculate the largest power of 2 dividing the integer
 simply as .


\begin{algorithm}
    \KwIn{The value  and element index .}

    \While {}
    {
        \;
        \;
    }
    \caption{ Updating the binary indexed tree. }
\end{algorithm}


\begin{algorithm}
    \KwIn{The index .}
    \KwOut{The sum of elements .}

    \;
    \While{}
    {
        \;
        \;
    }
    \Return \;

    \caption{ Calculating the cumulative sum. }
\end{algorithm}



\section{The algorithm for minimum vertex covers}



Let  denote the set of trapezoids in the trapezoid graph . For
simplicity, the trapezoid in  that corresponds to vertex  in  is called trapezoid .
Without loss of generality, the points on each horizontal line of the trapezoid diagram are labeled
with distinct integers between  and .

For simplicity and easier implementation, we will consider the maximum independent sets (according
to Theorem \ref{thm-ind}). At the beginning, we add two dummy trapezoids  and  to ,
where  for the vertex 0 and 
for the vertex . Trapezoid  lies entirely to the left of trapezoid , denoted by , if  and . It follows that  is a partial order over the
trapezoid set  and  is a strictly partially ordered set.

In order to get  algorithms, we will use binary indexed tree data structure for sum
and maximum functions.


\subsection{The size of maximum independent set and the total number of independent sets}


Arrange all trapezoids by the upper left corner . Let  denote the size of the
maximum independent set of  that contain trapezoid . For
the starting value, we set . The following recurrent relation holds

This produces a simple  dynamic programming algorithm (similar to \cite{LiCh09}).



To speed up the above algorithm, we will store the values from  in the binary indexed
tree  for maintaining the partial maximums. First initialize each element of 
with . We need an additional array , such that  contains the index of the
trapezoid with left or right coordinate equal to . We are going to traverse the coordinates from
 to , and let  be the current coordinate. In order to detect which already processed
trapezoids are to the left of the current trapezoid , we calculate the value 
when the current coordinate is the left upper coordinate of  and insert it in the binary
indexed tree when it is the right upper coordinate . Then we have two cases:

\begin{enumerate}[()]
\item coordinate  is the upper left coordinate of the trapezoid .
We only need to consider the trapezoids that have lower right coordinate smaller than . As
explained, these trapezoids have their corresponding values stored in the segment  in
the binary indexed tree and therefore we can calculate  based on the maximum values
among .

\item coordinate  is the upper right coordinate of the trapezoid . In this case,
the value  is already calculated and we put this value in the binary indexed tree at
the position .
\end{enumerate}

The final solution is . The pseudo-code is presented in Algorithm 3.

\begin{algorithm}
    \KwIn{The trapezoids  and the array .}
    \KwOut{The number of maximum independent sets stored in the array .}

    Initialize binary indexed tree for maximum \;
    \For{ \KwTo }
    {
        \;
        \If{}
        {
            \;
        }
        \Else(\tcp*[h]{  })
        {
            ;
        }
    }
    \Return \;

    \caption{ The size of the maximum independent set. }
\end{algorithm}


\medskip

The set of all VCs and the number of VCs of  are denoted by  and , respectively.
Let  be the set of vertices adjacent to  in the graph . In \cite{LiCh09} the authors
proved the following result.

\begin{lemma}
\label{le-1} For a graph  and arbitrary vertex , it holds

\end{lemma}


For counting the total number of independent sets, we can use the same method as above. The only
difference is that we will have sum instead of maximum function in the binary indexed tree. Let
 denote the number of independent sets of trapezoids  that contain trapezoid . According to Lemma \ref{le-1} the following recurrent relation
holds

The total number of independent sets is simply  and time complexity is the
same .


\subsection{The number of maximum independent sets}



For counting the number of maximum independent sets, we will need one additional array
, such that  represents the number of maximum independent sets
among trapezoids  such that  belongs to the
independent set. After running the algorithm for calculating the array , for each  we need to sum the number of independent sets of all indices , such that  and  in order to get the number of independent sets with
maximum cardinality. The pseudo-code of the algorithm for counting the number of maximum
independent sets from \cite{LiCh09} is presented in Algorithm 4.

\begin{algorithm}
    \KwIn{The trapezoids  and the array .}
    \KwOut{The number of maximum independent sets stored in the array .}

    \;
    \For{ \KwTo }
    {
        \;
        \For{ \KwTo }
        {
            \If{ \textbf{and}  \textbf{and} }
            {
                \;
            }
        }
    }
    \Return \;

    \caption{ The number of maximum independent sets. }
\end{algorithm}


We will use again binary indexed trees for designing  algorithm. The main problem is
how to manipulate the partial sums. Namely, for each trapezoid , we need to sum the values
 of those trapezoids  which are entirely on the left of  with
additional condition .

This can be done by considering all pairs  of two neighboring values . Using two additional arrays for the coordinates and trapezoid indices, we can
traverse trapezoids  from left to right, such that  or . First, for each  one needs to carefully preprocess all
trapezoids and store the indices and coordinates of all trapezoids  with 
in order to keep the memory limit . Therefore, all trapezoids will be stored in the dynamic
array of arrays , such that  contains all trapezoids  with .
Then we fill the binary indexed tree with the values for trapezoids with , and
calculate the value  for trapezoids with , as before.
Instead of resetting the binary indexed tree for each , we can again traverse the trapezoids
with  and update the values of BIT to .

Since every trapezoid will be added and removed exactly once from the binary indexed tree data
structure, the total time complexity is . This reusing of the data structure is novel
approach to the best of our knowledge and makes this problem very interesting.
\medskip

We conclude this section by summing the results in the following theorem.

\begin{thm}
The proposed algorithms calculate the size and the number of maximum independent sets, as well as
the total number of independent sets in time  of trapezoid graph  with 
vertices.
\end{thm}

One can easily extend this algorithm for efficient construction the independence polynomial of a
trapezoid graph

where  is the number of independent sets of cardinality  and  is the
independence number of .


\section{The algorithm for the maximum matching}

Ghosh and Pal \cite{GhPa05} presented an efficient algorithm for finding the maximum matching in
trapezoid graphs. The proposed algorithm takes  time and  space. They defined the
right spread  of a trapezoid  as the maximum . Let  be the
set of edges which form a matching of the graph . Deletion of a vertex from the graph means
deletion of that vertex and its adjacent edges.

The algorithm is designed as follows. Calculate the right spread  and arrange all trapezoids
by  in ascending order for  in linear time . Select the trapezoid 
with the minimum value of the right spread . After that, select the second minimum value . If trapezoids  and  are adjacent, then put the edge  in the maximum
matching , remove the vertices  and  from the graph  and mark the corresponding
elements of the array . If  then continue selecting the next minimum
elements  from the array  until the trapezoids  and  are adjacent. In that case
remove the vertices  and , put the edge  in the maximum matching  and mark the
corresponding elements of ; otherwise remove the trapezoid  from the graph  since it
cannot be matched. Continue this procedure as long as . Since for each trapezoid we
potentially traverse all trapezoids with larger values , the time complexity of the proposed
algorithm is .


\begin{figure}[ht]
  \center
  \includegraphics [width = 13cm]{counterexample1.eps}
  \caption { The minimal counterexample for the matching algorithm. }
\end{figure}


Obviously, the algorithm works for any set of  trapezoids . In the configuration on Figure
2, there are four trapezoids and only trapezoids  and  are not adjacent. The
trapezoids are ordered such that . By proposed algorithm in
\cite{GhPa05}, we will match the trapezoids  and  and get the maximal matching of cardinality
1. But obviously, the size of the maximum matching is 2 by pairing the trapezoids  and . Therefore, this algorithm is not correct and gives only an approximation for the maximum
matching. One can easily construct a family of counterexamples based on this minimal example by
adding trapezoids to the left and right.

We leave as an open problem to design more efficient algorithm for finding maximum matching in
trapezoid graphs.


\vspace{0.4cm} {\bf Acknowledgement. }  This work was supported by Research Grants 174010 and
174033 of Serbian Ministry of Education and Science.


\begin{thebibliography}{99}

\bibitem{AbOh05}
    M. I. Abouelhoda, E. Ohlebusch,
    \emph{Chaining algorithms for multiple genome comparison},
    J. Discrete Algorithms 3 (2005) 321--341.

\bibitem{AnAtChLe00}
    M. G. Andrews, M. J. Atallah, D. Z. Chen, D. T. Lee,
    \emph{Parallel Algorithms for Maximum Matching in Complements of Interval Graphs and Related
    Problems}, Algorithmica 26 (2000) 263--289.

\bibitem{BePaPa02}
    D. Bera, M. Pal, T. K. Pal,
    \emph{An efficient algorithm to generate all maximal cliques on trapezoid graphs},
    Int. J. Comput. Math. 79 (2002) 1057--1065.

\bibitem{BePaPa03}
    D. Bera, M. Pal, T. K. Pal,
    \emph{An Efficient Algorithm for Finding All Hinge Vertices on Trapezoid Graphs},
    Theory Comput. Systems 36 (2003) 17--27.

\bibitem{Ch96}
    M. S. Chang,
    \emph{Algorithms for maximum matching and minimum fill-in on chordal bipartite graphs},
    Lecture Notes in Computer Science 1178, Springer, Berlin, 1996, 146--155.

\bibitem{ChCo96}
    F. Cheah, D. G. Corneil,
    \emph{On the structure of trapezoid graphs},
    Discrete Appl. Math. 66 (1996) 109--133.

\bibitem{ChPaCh97}
    Y. Chung, K. Park, Y, Cho,
    \emph{Parallel Maximum Matching Algorithms in Interval Graphs},
    International Conference on Parallel and Distributed Systems, 1997, 602--609.

\bibitem{CoLeRiSt01}
    T. H. Cormen, C. E. Leiserson, R. L. Rivest, C. Stein,
    \emph{Introduction to Algorithms},
    Second Edition, MIT Press, 2001.

\bibitem{CoKa87}
    D. G. Corneil, P. A. Kamula,
    \emph{Extensions of permutation and interval graphs},
    In: Proceedings of 18th Southeastern Conference on Combinatorics, Graph Theory and Computing, 1987, 267--275.

\bibitem{CrGa10}
    C. Crespelle, P. Gambette,
    \emph{Unrestricted and complete Breadth First Search of trapezoid graphs in  time},
    Inform. Process. Lett. 110 (2010) 497--502.

\bibitem{DaGoPi88}
    I. Dagan, M. C. Golumbic, R. Y. Pinter,
    \emph{Trapezoid graphs and their coloring},
    Discrete Appl. Math. 21 (1988) 35--46.

\bibitem{FeMuWe97}
    S. Felsner, R. M\" uller, L. Wernisch,
    \emph{Trapezoid graphs and generalizations, geometry and algorithms},
    Discrete Appl. Math. 74 (1997) 13--32.

\bibitem{Fe94}
    P. M. Fenwick,
    \emph{A new data structure for cumulative frequency tables},
    Software - Practice and Experience 24 (1994), 327--336.

\bibitem{Ga80}
    H. N. Gabow,
    \emph{Data structures for weighted matching and nearest common ancestors with linking},
    in: Proceedings of the First Annual ACM SIAM Symposium on Discrete Algorithms, 1990, 434--443.

\bibitem{GhPa05}
    P. K. Ghosh, M. Pal,
    \emph{An efficient algorithm to find the maximum matching on trapezoid graphs},
    J. Korean Soc. Ind. Appl. Math. IT Series 9 (2) (2005) 13--20.

\bibitem{HoPaPa04}
    M. Hota, M. Pal, T. K. Pal,
    \emph{Optimal sequential and parallel algorithms to compute all cut vertices on trapezoid graphs},
    Comput. Optim. Appl. 27 (1) (2004) 95--113.

\bibitem{Ka72}
    R. M. Karp,
    \emph{Reducibility among combinatorial problems},
    In: R. E. Miller, J. W. Thatcher (Eds.), Complexity of Computer Computation, Plenum Press, New York,
    1972, 85--103.

\bibitem{Li94}
    Y. D. Liang,
    \emph{Domination in trapezoid graphs},
    Inform. Process. Lett. 52 (1994) 309--315.

\bibitem{Li95}
    Y. D. Liang,
    \emph{Steiner set and connected domination in trapezoid graphs},
    Inform. Process. Lett. 56 (1995) 101--108.

\bibitem{Li07}
    M. S. Lin,
    \emph{Fast and simple algorithms to count the number of vertex covers in an interval graph},
    Inform. Process. Lett. 102 (2007) 143--146.

\bibitem{LiCh08}
    M. S. Lin, Y. J. Chen,
    \emph{Linear time algorithms for counting the number of minimal vertex covers with minimum/maximum size in an interval graph},
    Inform. Process. Lett. 107 (2008) 257--264.

\bibitem{LiCh09}
    M. S. Lin, Y. J. Chen,
    \emph{Counting the number of vertex covers in trapezoid graph},
    Inform. Process. Lett. 109 (2009) 1187--1192.

\bibitem{MaSp94}
    T. H. Ma, J. P. Spinrad,
    \emph{On the 2-chain subgraph cover and related problems},
    J. Algorithms 17 (1994) 251--268.

\bibitem{MeCo09}
    G. B. Mertzios, D. G. Corneil, \emph{Vertex splitting and the recognition of trapezoid graphs},
    Technical Report AIB-2009-16, RWTH Aachen University, 2009.

\bibitem{MiVa80}
    S. Micali, V. V. Vazirani,
    \emph{An  algorithm for finding maximum matching in general graphs},
    Proceedings of the 21st IEEE Symposiums on Foundations of Computer Science, 1980, 17--27.

\bibitem{MoPaPa02}
     S. Mondal, M. Pal, T. K. Pal,
     \emph{An optimal algorithm for solving all-pairs shortest paths on trapezoid graphs},
     Int. J. Comput. Eng. Sci. (IJCES) 3 (2002) 103--116.

\bibitem{OkUnUe05}
    Y. Okamoto, T. Uno, R. Uehara,
    \emph{Counting the number of independent sets in chordal graphs},
    J. Discrete Algorithms 6 (2) (2005) 229--242.

\bibitem{PrBa83}
    J. S. Provan, M. O. Ball,
    \emph{The complexity of counting cuts and of computing the probability that a graph is connected},
    SIAM J. Computing 12 (4) (1983) 777--788.

\bibitem{TsLiHs07}
    Y. T. Tsai, Y. L. Lin, F. R. Hsu,
    \emph{Efficient algorithms for the minimum connected domination on trapezoid graphs},
    Inform. Sci. 177 (2007) 2405--2417.


\end{thebibliography}



\end{document}
