\documentclass[a4paper]{llncs}



\usepackage{graphicx}
\usepackage{amssymb}
\usepackage{amsmath}
\usepackage{enumerate}
\usepackage{scalefnt}
\usepackage{setspace} 

\usepackage{url}
\newcommand{\keywords}[1]{\par\addvspace\baselineskip
\noindent\keywordname\enspace\ignorespaces#1}
\newcommand{\bm}[1]{\mbox{\boldmath{}}}
\newcommand{\bms}[1]{\mbox{\boldmath{{\footnotesize}}}}
\newcommand{\bmss}[1]{\mbox{\boldmath{{\tiny}}}}
\newcommand{\argmax}{\operatornamewithlimits{argmax}}
\newcommand{\argmin}{\operatornamewithlimits{argmin}}

\newtheorem{defi}{Definition}
\newtheorem{thm}{Theorem}
\newtheorem{lem}{Lemma}
\newtheorem{clm}{Claim}
\newtheorem{coro}{Corollary}
\newtheorem{rmk}{Remark}
\newtheorem{fct}{Fact}
\newtheorem{cjt}{Conjecture}
\newtheorem{pro}{Proposition}

\begin{document}



\mainmatter

\title{Improved Algorithms for Multiple Sink Location Problems in Dynamic Path Networks}

\author{
Yuya Higashikawa \inst{1}
\and Mordecai J.~Golin \inst{2}
\and Naoki Katoh \inst{1} \thanks{Supported by JSPS Grant-in-Aid for Scientific Research(A)(25240004)}
}

\institute{
Department of Architecture and Architectural Engineering, Kyoto University, Japan, 
\{as.higashikawa, naoki\}@archi.kyoto-u.ac.jp \and 
Department of Computer Science and Engineering, The Hong Kong University of Science and Technology, Hong Kong,
golin@cs.ust.hk }


\maketitle


\begin{abstract}
This paper considers the -sink location problem in dynamic path networks.
In our model, a dynamic path network consists of an undirected path with positive edge lengths, uniform edge capacity, and positive vertex supplies.
Here, each vertex supply corresponds to a set of evacuees.
Then, the problem requires to find the optimal location of  sinks in a given path so that each evacuee is sent to one of  sinks.
Let  denote a -sink location.
Under the optimal evacuation for a given , 
there exists a -dimensional vector , called -divider, 
such that each component represents the boundary dividing all evacuees between adjacent two sinks into two groups, 
i.e., all supplies in one group evacuate to the left sink and all supplies in the other group evacuate to the right sink.
Therefore, the goal is to find  and  which minimize the maximum cost or the total cost,
which are denoted by the minimax problem and the minisum problem, respectively.
We study the -sink location problem in dynamic path networks with continuous model,
and prove that the minimax problem can be solved in  time and the minisum problem can be solved in  time,
where  is the number of vertices in the given network.
Note that these improve the previous results by \cite{hgk14_2}.
\keywords{sink location, dynamic network, evacuation planning}
\end{abstract}



\section{Introduction}
The Tohoku-Pacific Ocean Earthquake happened in Japan on March 11, 2011, 
and many people failed to evacuate and lost their lives due to severe attack by tsunamis. 
From the viewpoint of disaster prevention from city planning and evacuation planning,  
it has now become extremely important to establish effective evacuation planning systems against large scale disasters. In particular, 
arrangements of tsunami evacuation buildings in large Japanese cities near the coast has become an urgent issue. 
To determine appropriate tsunami evacuation buildings, we need to consider where evacuation buildings are assigned 
and how to partition a large area into small regions so that one evacuation building is designated in each region. 
This produces several theoretical issues to be considered. 
Among them, this paper focuses on the location problem of multiple evacuation buildings 
assuming that we fix the region such that all evacuees in the region are planned to evacuate to one of these buildings. 
In this paper, we consider the simplest case for which the region consists of a single road. 


In order to represent the evacuation, we consider the {\it dynamic} setting in graph networks, which was first introduced by Ford et al.~\cite{ff58}.
In a graph network under the dynamic setting, each vertex is given supply and each edge is given length and capacity which limits the rate of the flow into the edge per unit time.
We call such networks under the dynamic setting {\it dynamic networks}.
Dynamic networks can be considered in discrete and continuous models.
In discrete model, each input value is given as an integer.
Then each supply can be regarded as a set of evacuees, and edge capacity is defined as the maximum number of evacuees who can enter an edge per unit time.
On the other hand, in continuous model, each input value is given as a real number.
Then each supply can be regarded as fluid, and edge capacity is defined as the maximum amount of supply which can enter an edge per unit time.
In either model, we assume that all supply at a vertex is sent to the same sink.
{\it The -sink location problem in dynamic networks} is defined as the problem which requires to find the optimal location of  sinks in a given network 
so that all supply of each vertex is sent to one of  sinks in the shortest time.

For the 1-sink location problem in dynamic networks, the following two criteria can be naturally considered: {\it maximum cost criterion} and {\it total cost criterion}
(in static networks, these criteria correspond to the center problem and the median problem in facility location, respectively).
If a sink location  is given in a dynamic network with discrete model, 
the cost of  for an evacuee is defined as the minimum time required to send him/her to 
(by taking into account the congestion).
Then two criteria are defined as the maximum of cost of  for all evacuees and the sum of cost of  for all evacuees, respectively.
Now let us turn to continuous model. 
In continuous model, we define the {\it unit} as the infinitesimally small portion of supply,
then the cost is defined on each unit.
If a sink location  is given in a dynamic network with continuous model, 
the cost of  for a unit is defined as the minimum time required to send the unit to .
Also two criteria are defined as the maximum of cost of  for all units and the sum of cost of  for all units, respectively.
Definitions for -sink location problem are given later.
Then, {\it the minimax} (resp. {\it minisum}) {\it -sink location problem in dynamic networks} requires to find a -sink location in a given dynamic network which minimizes the maximum (resp. total) cost.
Mamada et al.~\cite{mumf06} studied the minimax 1-sink location problem in dynamic tree networks with discrete model assuming that the sink must be located at a vertex,
and proposed an  time algorithm. 
Higashikawa et al.~\cite{hgk14} also studied the same problem as \cite{mumf06} assuming that edge capacity is uniform and the sink can be located at any point in the network,
and proposed an  time algorithm. 
Recently, Higashikawa et al.~\cite{hgk14_2} studied the -sink location problems in a dynamic path network with continuous model assuming that edge capacity is uniform and the sink can be located at any point in the network, and proved that the minimax problem can be solved in  time and the minisum problem can be solved in  time.

In this paper, we study the same problems as \cite{hgk14_2}, and improve the previous time bounds: 
 to  for the minimax problem and  to  for the minisum problem.





\section{Minimax -sink location problem}
\label{sec:minimax}

\subsection{Preliminaries}
\label{sec:mmp}

\subsubsection{Model definition:}
Let  be an undirected path where   and ,  
such that  and  are endpoints of  for .
Let  be a dynamic network 
with the underlying graph being a path ,
 is a function that associates each edge  with positive length , 
 is also a function that associates each vertex  with positive weight  representing the amount of supply at ,
 is a positive constant representing the amount of supply which can enter an edge per unit time,
and  is also a constant representing the time required by flow for traversing the unit distance.
We call such networks with path structures {\it dynamic path networks}.
In the following, we use the notation  to denote the set of all points .
Also, for a vertex  with , we abuse the notation  to denote the distance from  to ,
and for a point , we abuse the notation  to denote the distance from  to .
Then, we can regard  as embedded on a real line such that .
For two points  with , 
let  (resp. ,  and ) denote the part of  which consists of all points  
such that  (resp. ,  and ).

\subsubsection{-sink location and -divider:}
Suppose that  sinks are located at points  such that , respectively.
Note that each sink can be located at any point in .
In this paper, we assume that if we place a sink at a vertex, all supply of the vertex can finish the evacuation in no time.
So, without loss of generality, we assume 
(otherwise, at least one sink can be located at each vertex). 
Let  which is a -dimensional vector, called {\it -sink location}.
Let us consider the optimal evacuation for a given .
In this paper, we assume that all units of a vertex are sent to the same sink.
We call a directed path along which all units of a vertex are sent to a sink {\it evacuation path}.
Then, any two evacuation paths never cross each other in an optimal evacuation
(otherwise, we can realize the better or equivalent evacuation by exchanging the two destinations of crossing evacuation paths).
Suppose that there exists only one vertex  in  and all units of the vertex are sent to ,
then  can be moved to  without increasing the cost of any unit.
Therefore, if we optimally locate  sinks with , there exist at least two vertices in  for any  with , i.e.,
there exist two vertices  and  with  in 
such that all supplies on  are sent to  and
all supplies on  are sent to .
We call such a vertex  {\it dividing vertex}.
For an integer  with  with , let  be an index of the dividing vertex in .
By the above discussion,  holds for  where  and .
Let  which is a -dimensional vector, called {\it -divider}.
For a given , we need only consider  such that  is given on  for , where  and .

\subsubsection{Problem definition:}
For given  and ,
and also for an integer  with , let  denote the minimum time required to send all supplies on  to , where  and .
Letting , the minimax -sink location problem is defined as follows:







\subsection{Recursive formulation}
We now consider a subproblem of the above mentioned problem:
for some integers  and  with  and , the -sink location problem in .
For , let  denote the optimal -sink location and  denote the optimal -divider.
Note that  is a -dimensional vector and  is also a -dimensional vector,
so  is not defined for .
Also, let  denote the optimal cost of -sink location in , 
i.e., the minimum time required to send all supplies on  divided by  to .
Note that if  holds, the optimal sink location is trivial, i.e., .

Next, we show the recursive formula of .
For integers  and  with  and ,
let us consider the optimal -sink location and -divider for , i.e.,  and .
Since any two evacuation paths never cross each other in an optimal evacuation,
there exists an integer  with  such that
all supplies on  are sent to the rightmost sink
and all supplies on  are sent to the other  sinks.
Thus, we have the following recursion: 

Here, let  be an integer which minimizes the maximum of  and  on :

Then,  and  can be represented by using  as follows:






\subsection{Known properties of 1-sink location problem}
Here, we introduce the properties of 1-sink location problem, which were explicitly shown in \cite{hgk14_2} (based on \cite{chknsx13,hacgknsx14}). 
For fixed integers  and  with , let us consider how to compute the optimal 1-sink location in .
Suppose that a sink is located at a point  in .
Let  denote the minimum time required to send all supplies on  to .
Here, let  (resp. ) denote the minimum time required to send all supplies on  (resp. ) to  where  and .
Then,  is the maximum of  and ,
i.e., 

For discrete model, Kamiyama et al. \cite{kkt06} showed that
 and  are expressed as follows:

From these, we can immediately develop the formulae for continuous model as follows:

Note that  (resp. ) is a piecewise linear strictly increasing (resp. decreasing) function of .
Therefore, a function  is unimodal in , and there exists the unique point which minimizes , they is, .
Then, as \cite{chknsx13,hacgknsx14,hgk14_2} showed, we immediately have the following claim.
\begin{clm}
For any integers  and  with  and a point , \\
{\rm (i)} if  holds,  holds, and \\
{\rm (ii)} if  holds,  holds.
\label{clm2}
\end{clm}
In the following, when  is at a vertex  with ,
we use the notation  (resp. ) to denote the value  (resp. ).
Then, we have the following claim (which was also shown in \cite{chknsx13,hacgknsx14,hgk14_2}).
\begin{clm}
For given integers  and  with , 
suppose that for the interval  with ,  and  hold,
and let  denote the solution to an equation for : .
Then, \\
{\rm (i)} if  holds,  is a point dividing the interval  with the ratio of  to  
and  holds, \\
{\rm (ii)} if  holds,  and  hold, and \\
{\rm (iii)} if  holds,  and  hold. \\
\label{clm2.1}
\end{clm}













\subsection{Key properties of -sink location problem}
\label{sec:mmkp}
In this section, we show several key properties of the -sink location problem.
Here, 
for integers  and  with  and , 
let  denote a function defined on :

Note that for fixed  and ,  is monotonically increasing in  and  is monotonically decreasing in .
Thus, we have the following claim.
\begin{clm}
For any integers  and  with  and ,
function  is unimodal in  on .
\label{clm1}
\end{clm}
Let  be an integer which minimizes  for :

By Claim \ref{clm1}, there uniquely exists .
By (\ref{eq0.2}) and (\ref{eq0.3}), we have

Then, we prove the following two lemmas.
\begin{lem}
For any integers  and  with  and ,
 holds.
\label{lem1}
\end{lem}

\begin{lem}
For any integers  and  with , ,  and ,
 holds.
\label{lem2}
\end{lem}

\subsubsection{Proof of Lemma \ref{lem1}:}
In order to prove Lemma \ref{lem1}, we first confirm a fundamental property.
\begin{clm}
For any integers  with , and  and  with ,
 holds.
\label{clm:mm1.1}
\end{clm}
We prove Lemma \ref{lem1} by contradiction: 
there exist integers  and  with  and  such that
 holds.
For ease of notation in the proof, we use the notations  and  as follows:

From the assumption of  and Claim \ref{clm:mm1.1}, we can derive the following inequalities:

Since  minimizes  (refer to (\ref{eq1}) and (\ref{eq1.1})),
we have the following inequality:

Also, without loss of generality, we assume that  is maximized unless the cost increases. 
By this assumption, we have the following inequality:

Then, we consider three cases:
[Case 1] ; [Case 2] ; [Case 3]  and . \\
\noindent
[Case 1]: By (\ref{eq:mm1.5}), (\ref{eq:mm1.7}) and the condition of , we have , which contradicts (\ref{eq:mm1.3}).\\
\noindent
[Case 2]: By (\ref{eq:mm1.5}), (\ref{eq:mm1.6}) and the condition of , we have .
By this and (\ref{eq:mm1.2}), we have .
Also, by (\ref{eq:mm1.5}), (\ref{eq:mm1.7}) and the condition of , we have .
By this and (\ref{eq:mm1.3}), we have , which contradicts .\\
\noindent
[Case 3]: By (\ref{eq:mm1.2}) and the condition of , we have 

Also, by (\ref{eq:mm1.3}) and the condition of , we have

If  holds, we have  by (\ref{eq:mm1.8}) and (\ref{eq:mm1.9}), which contradicts the condition of  or (\ref{eq:mm1.6}).
If  holds, we have  by (\ref{eq:mm1.9}), which contradicts (\ref{eq:mm1.7}).
\qed


\subsubsection{Proof of Lemma \ref{lem2}:}
In order to prove Lemma \ref{lem2}, we first confirm the following claim (refer to the definitions of (\ref{eq3.1}) and (\ref{eq3.2})).
\begin{clm}
{\rm (i)} For any integers  and  with  and any points  and  with ,
 and  hold.\\
{\rm (ii)} For any integers  and  with  and any points  and  with ,
 and  hold.
\label{clm:mm2.1}
\end{clm}
We prove Lemma \ref{lem2} by contradiction: 
there exist integers  and  with , ,  and  such that
 holds.
By this assumption, we have the following inequality:

For ease of notation in the proof, we use the notations  and  as follows:

From (\ref{eq:mm2.4}) and Claim \ref{clm:mm2.1}, we can derive the following inequalities:

Since  and  are the unique points which minimize   and , respectively
(refer to (\ref{eq4})), we have the following inequalities:

Then, we consider three cases:
[Case 1] ; [Case 2] ; [Case 3]  and . \\
\noindent
[Case 1]: By (\ref{eq:mm2.6}), (\ref{eq:mm2.8}) and the condition of , we have , which contradicts (\ref{eq:mm2.3}).\\
\noindent
[Case 2]: By (\ref{eq:mm2.5}), (\ref{eq:mm2.7}) and the condition of , we have , which contradicts (\ref{eq:mm2.2}).\\
\noindent
[Case 3]: By (\ref{eq:mm2.2}) and the condition of , we have 

Also, by (\ref{eq:mm2.3}) and the condition of , we have

If  holds, we have  by (\ref{eq:mm2.9}), which contradicts (\ref{eq:mm2.5}).
Also, if  holds, we have  by (\ref{eq:mm2.10}), which contradicts (\ref{eq:mm2.8}).
If  and  hold, we have  by (\ref{eq:mm2.7}), (\ref{eq:mm2.9}) and (\ref{eq:mm2.10}), 
that is,  holds, which contradicts (\ref{eq:mm2.6}).
\qed










\subsection{Algorithm based on dynamic programming}
\label{sec:mma}
The algorithm basically computes , , , , , , , , ,  in this order.
For some integers  and  with  and ,
let us consider how to obtain .
Actually, in order to obtain , the algorithm needs  for  and , which are supposed to have been obtained.
By (\ref{eq0.0}), (\ref{eq1}) and (\ref{eq1.1}), we have

Here, we assumed that  has already been obtained.
Thus, in order to obtain , we only need to compute .
Recall that  is the unique point which minimizes function  (refer to (\ref{eq1}) and (\ref{eq1.1})).
Now, the algorithm knows where  exists, and by Lemma \ref{lem1},  holds.
So the algorithm starts to compute  for ,
and continues to compute in ascending order of , as will be shown below.
Note that function  is unimodal in  by Claim \ref{clm1}, 
which implies that  is strictly decreasing until .
Thus, if the algorithm reaches the first integer  such that ,
it outputs  as .
Then, the algorithm also outputs  as .


\subsubsection{Computation of  for :}
As above mentioned, the algorithm first computes  with  which is defined as follows:

Since the algorithm has already obtained ,
we only need to compute .
To do this, we actually need to find .
On the other hand, the algorithm has already obtained  as follows:

which implies that  has been obtained.
By Lemma \ref{lem2},  holds.
Let  and  be the indices of vertices  
such that  with 
and  with , respectively (see Figure \ref{fig1}).
\begin{figure}[h]
\centering
\includegraphics[width=70mm,clip]{fig1c.eps} 
\caption{Illustrations of  and }
\label{fig1}
\end{figure}
By Claim \ref{clm2}, for any interval  with , there exists  in 
if  and  hold.
Therefore, if we maintain the data structure so that we can compute these values,
the algorithm can test if there exists  or not
(what the data structure is or how we can maintain and use it will be explained in the next subsection).
Then, the algorithm starts to test for ,
and continues to test in ascending order of .
If an interval where  exists, that is,  is found,
then  and  can be computed in  time by Claim \ref{clm2.1}.


\subsubsection{Computation of  for :}
Now, suppose that for an integer  with ,
the algorithm has already obtained , that is,  and .
For an integer  with , let  be the index of a vertex with 
such that .
Note that  has also been obtained (see Figure \ref{fig2}).
Then, the computation of  comes down to finding  which is greater than or equal to ,
and so, it can be treated in the similar manner as the computation of .
\begin{figure}[h]
\centering
\includegraphics[width=70mm,clip]{fig2.eps} 
\caption{Illustrations of  and }
\label{fig2}
\end{figure}




\subsection{How to compute  and }
\label{s2}
As mentioned in Section \ref{sec:mma}, in order to obtain  for fixed  and all 
(note that   for ), 
the algorithm computes ,

.
In this computation,
the algorithm actually computes 
where  is the index of vertex with  for any ,
and also, 
where  is the index of vertex with  for any .
In order to compute  and  for any integers  and  with , 
the algorithm maintains the specific data structures  and , respectively.
Depending on the situation, the algorithm updates  to  or  and  to  or .
We below show definitions of the two data structures and how to maintain these.

\subsubsection{Definition of  and how to maintain :} 
In this discussion, we assume that  holds ( when ).
Let us consider the evacuation of all supplies on  to .
We define the vertex indices  as 

Note that  holds. 
For every integer  with , we also define the value of  as  where .
Here, we notice that for every integer  with , the first unit of  never be induced to stop at any vertex  with .
The data structure  consists of the two sequences  and .
Note that we define the size of  as .
Recall that in continuous model, the cost is defined on each infinitesimal unit of supply,
i.e., the cost of  for a unit is defined as the minimum time required to send the unit to .
Here, we notice that for any integer  with , the first unit of  never be induced to stop at .
Then,  can be computed as 




In order to update  to , the algorithm tests if  holds or not. 
If it holds, the algorithm sets  so that

Otherwise, the algorithm sets  so that


On the other hand, in order to update  to , the algorithm first sets  and .
Then, the algorithm repeatedly tests if  holds or not in descending order of  from .
If it holds, the algorithm sets  so that

until  holds for  with some integer .
Let  denote the number of such tests required to update  to , which can be represent as


Recall that in the computation to obtain  for fixed  and all 
for a given integer  with , the algorithm updates  to  where  and .
Let  denote the total number of such tests required to update  to ,
and  denote the sum of  for .
By (\ref{eq15}) and (\ref{eq16}), we have , so the upper bound of  can be obtained as

which implies that  is amortized .




\subsubsection{Definition of  and how to maintain :} 
In this discussion, we assume that  holds ( when ).
Let us consider the evacuation of all supplies on  to .
We define the vertex indices  as

Note that  holds. 
For every integer  with , we also define the value of  as .
Here, we notice that  is the rightmost vertex of which the first unit never be induced to stop at any vertex  with  where .
In addition, let .
The data structure  consists of the offset value  and the two sequences  and .
Note that we define the size of  as .
Then,  can be computed as 


In order to update  to , the algorithm tests if  holds or not. 
If it holds, the algorithm sets  so that

Otherwise, nothing changes, that is, the algorithm sets .

On the other hand, in order to update  to , the algorithm first sets  and 
compute  and .
Then, the algorithm repeatedly tests if 
 holds or not in descending order of  from .
If it holds, the algorithm sets  so that

until  holds for  with some integer .
Let  denote the number of such tests required to update  to , which can be represent as

which is amortized  by the same discussion as that for  defined at (\ref{eq18}).

\begin{clm}
For any integers  and  with ,
 and  can be computed in  time once  and  have been obtained.
\label{clm:ds1}
\end{clm}
\begin{clm}
{\rm (i)} For any integers  and  with ,
 and  can be updated to  and  in amortized  time, respectively. \\
{\rm (ii)} For any integers  and  with ,
 and  can be updated to  and  in amortized  time, respectively.
\label{clm:ds2}
\end{clm}

\subsection{Time complexity}


As mentioned in Section \ref{sec:mma} and at the beginning of Section \ref{s2}, in order to obtain  for fixed  and all , 
 intervals are tested in total as follows:
in order to test if there exists  in an interval  or not, 
the algorithm needs to confirm that  and  hold by Claim \ref{clm2}, 
which takes  time once  and  have been obtained by Claim \ref{clm:ds1}.
Thus, such computations take  time in total.

On the other hand, let us consider the total time required to update the data structures.
For fixed  and , when  is obtained, the algorithm maintains  and ,
where  exists in .
When  is obtained after repeatedly updating these four vertex sets, 
the algorithm maintains  and ,
where  exists in .
Recall that  and  hold by Lemmas \ref{lem1} and \ref{lem2}.
Thus, in order to obtain , the algorithm updates the four vertex sets  times,
and so, for fixed  and all , the algorithm updates these sets  times in total,
which takes  time by Claim \ref{clm:ds2}.


Therefore,  for all  and  can be obtained in  time.
\begin{thm}
The minimax -sink location problem in a dynamic path network with uniform capacity can be solved in  time.
\label{thm:mm1}
\end{thm}














\section{Minisum -sink location problem}
\label{sec:minisum}
In this section, an input graph of this problem is a dynamic path network defined in Section \ref{sec:minimax}.
As a preliminary step, let us consider the minisum 1-sink location problem.



\subsection{Properties of the minisum 1-sink location problem}
\label{sec:minisum1}
Suppose that a sink is located at a point  where  is the input path with  vertices.
In continuous model, the cost is defined on each infinitesimal unit of supply,
i.e., the cost of  for a unit is defined as the minimum time required to send the unit to .
Let  denote the total cost of , i.e., the sum of cost of  for all units on .
Here, let  (resp. ) denote the sum of cost of  for all units on  (resp. ).
Then,  is the maximum of  and , i.e., 

Without loss of generality, we assume  and .
Now, suppose that  is located in an open interval  with ,
then let us explain how function  is determined.

\subsubsection{Case 1:}
For every integer  with ,  holds.
In this case, the first unit of each vertex on  can reach  after leaving the original vertex 
without being blocked due to the existence of other units at an intermediate vertex. 
For an integer  with , let  denote the sum of cost of  for all units of .
Here, suppose that there are  units at  with sufficiently large , i.e., the size of each unit is equal to ,
and these units continuously reach .
Then by (\ref{eq3.1}), the -th unit finishes reaching  at time .
Therefore, by taking  to the infinity,  can be represented as follows:

and also  is represented as follows:


\subsubsection{Case 2:}
We define the vertex indices  as

Note that  holds. 
For every integer  with , we also define the value of  as  where .
Here, we notice that for every integer  with , the first unit of  never be induced to stop at any vertex  with .
Then, as with (\ref{eq:ms2}),  is represented as follows:

Note that  holds.



\vspace{4mm}

We can compute  in the similar manner as .
Thus, for an open interval  with , function  is linear in  with slope .
Now let us consider an open interval  with  such that  holds.
Then, we can see that for any two points  with ,  holds.
We will show that for sufficiently small ,  holds.
We confirm

From (\ref{eq:ms4}), (\ref{eq:ms5}) and the assumption of , 
we can derive .
In general, we have the following claim.
\begin{clm}
{\rm (i)} For an open interval  with  such that ,
 holds where . \\
{\rm (ii)} For an open interval  with  such that ,
 holds where . 
\label{clm:ms1}
\end{clm}
Let  denote the optimal sink location which minimizes .
Then, Claim \ref{clm:ms1} implies that  is located at some vertex.
\begin{clm}
There exists  at a vertex.
\label{clm:ms2}
\end{clm}


\subsection{Algorithm and time complexity for the minisum 1-sink location problem}
We propose the algorithm which can solve the minisum 1-sink location problem in a dynamic path network.
Basically, the algorithm first computes  for  in ascending order of , and next  for  in descending order of .
After computing all these values,  can be computed and evaluated for  in  time.
Then, by Claim \ref{clm:ms2}, the optimal sink location  is at a vertex which minimizes  for .
Below, we show how to compute  (computation of  can be treated in the similar manner).

First, the algorithm sets , .
By (\ref{eq:ms2}),  is computed in  time as follows:

Now, suppose that for some integer  with ,  has been set as a non-negative integer,
 and  have been obtained for all  with  in the same manner as mentioned in Case 2, Section \ref{sec:minisum1},
and  has been already computed as follows:

Let  and suppose that  has also been computed.
We then show how to compute .
The algorithm newly sets 

Next, the algorithm tests if  for  in descending order.
If so, it updates  and  as follows:

and deletes .
If the maximum integer  such that  is found or  is obtained, the algorithm stops testing.
In the former case, after the algorithm tests  times,  remain.
Then, after computing  as ,
by (\ref{eq:ms3}),  can be computed as

Also, for the next recursive step, the algorithm eventually sets 



Since the algorithm tests  times to compute ,
it needs to test  times to compute  for .
Here, by , we have


\begin{lem}
The minisum 1-sink location problem in a dynamic path network with uniform capacity can be solved in  time.
\label{lem:ms1}
\end{lem}


\subsection{Extension to the minisum -sink location problem}
Let  representing a -sink location given on  and   representing a -divider given on 
(which are defined in the same manner as mentioned in Section \ref{sec:mmp}).
For a given , we need only consider  such that  is given on  for , where  and .
For given  and ,
and for an integer  with , let  denote the sum of cost of  for all supplies on .
Letting , the minisum -sink location problem is defined as follows:


We below show that this problem can be transformed to an equivalent problem, which requires to find the minimum -link path in a weighted, complete, directed acyclic graph (DAG) \cite{s98}.
First, for integers  and  with , let  denote the optimal cost for the minisum -sink location problem in .
Let us consider a DAG  such that  and
for every vertex pair  with , 
there exists an edge which is directed from  to  and associated with the weight of .
Then,  is equivalent to a problem requiring to find a path in  from  to  which contains exactly  edges such that the sum of weights is minimized.
Schieber \cite{s98} showed that this problem can be solved by querying edge weights  times
if the input DAG satisfies the {\it concave Monge property}, that is,  holds for any integers  and  with .
Since each weight query takes  time by Lemma \ref{lem:ms1}, if the concave Monge property is proved,  can be solved in  time.
Therefore, we prove the following lemma.
\begin{lem}
For any integers  and  with ,
 holds.
\label{lem:ms2}
\end{lem}
\begin{proof}
For integers  and  with , and a 1-sink location ,
let  denote the sum of cost of  for all supplies on ,
and let  (resp. ) denote the sum of cost of  for all supplies on  (resp. ).
Also, let .
By the definitions, we have

Then, we consider two cases: [Case 1]  and [Case 2] .
Here, let us prove only Case 1 (Case 2 can be symmetrically proved).
We first show that 

By (\ref{eq:ms2.1}), the left side of (\ref{eq:ms2.3}) is equal to  and the right side of (\ref{eq:ms2.3}) is equal to .
Let  (clearly ), that is,

Then, we have 

By (\ref{eq:ms2.4}) and (\ref{eq:ms2.5}), we obtain

which is equivalent to (\ref{eq:ms2.3}) as mentioned above.
On the other hand, by the optimality of  and , we have

Then, by (\ref{eq:ms2.3}), (\ref{eq:ms2.7}), (\ref{eq:ms2.8}) and the definitions of  and , we obtain

which implies that the lemma holds in Case 1.
\qed
\end{proof}

\begin{thm}
The minisum -sink location problem in a dynamic path network with uniform capacity can be solved in  time.
\label{thm:ms1}
\end{thm}

\section{Conclusion}
In this paper, we study the -sink location problem in dynamic path networks with continuous model assuming that edge capacity is uniform and sinks can be located at any point in the network, and prove that the minimax problem can be solved in  time and the minisum problem can be solved in  time.

On the other hand, we leave as an open problem to reduce the time bound to  for the minisum problem,
and extend the solvable networks into dynamic path networks with general capacities or more general networks (e.g., trees).







\clearpage

\begin{thebibliography}{99}
	\bibitem{cc10}
		D.~Chen and R.~Chen, 
		\newblock ``A Relaxation-Based Algorithm for Solving the Conditional -Center Problem'', 
		\newblock {\em Operations Research Letters},
		\newblock 38(3), pp.~215-217, 2010.

	\bibitem{chknsx13}
		S.W.~Cheng, Y.~Higashikawa, N.~Katoh, G.~Ni, B.~Su and Y.~Xu,
		\newblock ``Minimax Regret 1-Sink Location Problems in Dynamic Path Networks",
		\newblock {\em Proc. The 10th Annual Conference on Theory and Applications of Models of Computation} (TAMC 2013),
		\newblock LNCS 7876, pp.~121-132, 2013.

	\bibitem{ff58}
		L.~R.~Ford~Jr. and D.~R.~Fulkerson,
		\newblock ``Constructing Maximal Dynamic Flows from Static Flows",
		\newblock {\em Operations Research},
		\newblock 6, pp.~419-433, 1958.	
			
	\bibitem{hacgknsx14}
		Y.~Higashikawa, J.~Augustine, S.W.~Cheng, M.J.~Golin, N.~Katoh, G.~Ni, B.~Su and Y.~Xu,
		\newblock ``Minimax Regret 1-Sink Location Problem in Dynamic Path Networks",
		\newblock {\em Theoretical Computer Science},
		\newblock DOI: 10.1016/j.tcs.2014.02.010, 2014.		
	
	\bibitem{hgk14}
		Y.~Higashikawa, M.~J.~Golin, N.~Katoh,
		\newblock ``Minimax Regret Sink Location Problem in Dynamic Tree Networks with Uniform Capacity",
		\newblock {\em Proc. The 8th International Workshop on Algorithms and Computation} (WALCOM 2014),
		\newblock LNCS 8344, pp.~125-137, 2014.
	
	\bibitem{hgk14_2}
		Y.~Higashikawa, M.~J.~Golin, N.~Katoh,
		\newblock ``Multiple Sink Location Problems in Dynamic Path Networks",
		\newblock {\em Proc. The 10th International Conference on Algorithmic Aspects of Information and Management} (AAIM 2014),
		\newblock LNCS 8546 (to appear).	
					
	\bibitem{kkt06}
		N.~Kamiyama, N.~Katoh and A.~Takizawa, 
		\newblock ``An Efficient Algorithm for Evacuation Problem in Dynamic Network Flows with Uniform Arc Capacity",
		\newblock {\em IEICE Transactions},
		\newblock 89-D(8), pp.~2372-2379, 2006. 
	
	\bibitem{mumf06}
		S.~Mamada, T.~Uno, K.~Makino and S.~Fujishige,
		\newblock ``An  Algorithm for the Optimal Sink Location Problem in Dynamic Tree Networks",
		\newblock {\em Discrete Applied Mathematics},
		\newblock 154(16), pp.~2387-2401, 2006. 
		
	\bibitem{s98}
		B.~Schieber,
		\newblock ``Computing a Minimum Weight -Link Path in Graphs with the Concave Monge Property",
		\newblock {\em Journal of Algorithms},
		\newblock 29(2), pp.~204-222, 1998. 		
		 
\end{thebibliography}




\end{document}