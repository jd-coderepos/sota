\documentclass{llncs}
\usepackage{amsmath}
\usepackage{amssymb}

\newcommand{\md}{\ensuremath \textrm{md}}
\newcommand{\dd}{\ensuremath \textrm{dd}}
\newcommand{\palto}{type}
\newcommand{\M}{\ensuremath{\mathcal M}}

\begin{document}

\title{Parameterized Modal Satisfiability}

\author{Antonis Achilleos \and Michael Lampis \and Valia Mitsou}

\institute{Computer Science Department,  \newline Graduate Center, City University of New York, \newline 365 5th Ave New York, NY 10016 USA
\email{antach@corelab.ntua.gr,mlampis@gc.cuny.edu,vmitsou@cs.gc.cuny.edu} }


\maketitle

\begin{abstract}
We investigate the parameterized computational complexity of the
satisfiability problem for modal logic and attempt to pinpoint
relevant structural parameters which cause the problem's
combinatorial explosion, beyond the number of propositional
variables . To this end we study the modality depth, a natural
measure which has appeared in the literature, and show that, even
though modal satisfiability parameterized by  and the modality
depth is FPT, the running time's dependence on the parameters is a
tower of exponentials (unless P=NP). To overcome this limitation we
propose several possible alternative parameters, namely diamond
dimension, box dimension and modal width. We show fixed-parameter
tractability results using these measures where the exponential
dependence on the parameters is much milder than in the case of
modality depth thus leading to FPT algorithms for modal
satisfiability with much more reasonable running times.

\end{abstract}


\section{Introduction}

In this paper we consider the computational complexity of deciding
two fundamental logic problems, namely formula satisfiability and
formula validity, for modal logics, focusing on the standard modal
logic K. We attempt to present a new point of view on this important
topic by making use of the parameterized complexity framework, which
was pioneered by Downey and Fellows. Although the complexity of
satisfiability for modal logic has been studied extensively in the
past, to the best of our knowledge this is the first time this has
been done from an explicitly parameterized perspective. Moreover,
the parameterized complexity of logic problems has been a fruitful
field of research and we hope to extend this success to modal logic
(some examples are the celebrated theorem of Courcelle
\cite{DBLP:journals/iandc/Courcelle90} or the results of
\cite{DBLP:journals/apal/FrickG04}; for an excellent survey on the interplay
between logic, graph problems and parameterized complexity see
\cite{DBLP:journals/eccc/Grohe07}).

Modal logic is a family of systems of formal logic where the truth value of a sentence  can be qualified by
modality operators, usually denoted by  and . Depending on the specific modal logic and the
application one considers,  and  can be informally read to mean, for example, ``it is
necessary that '', or ``it is known that '' for  and ``it is possible that '' for
. The fundamental normal modal logic system is known as K, while other common variations of this logic
system include T,D,S4,S5. Modal logic systems provide a diverse universe of logics able to fit many modern
applications in computer science (for example in AI or in game theory), making modal logic a widespread topic of
research. The interested reader in the recent state of modal logic and its applications is directed to
\cite{handbook-modal}.

As in propositional logic, the satisfiability problem for modal logic is one of the most important and fundamental
problems considered and many results are known about its (traditional) computational complexity. Ladner in
\cite{lad} showed that satisfiability for K, T and S4 is PSPACE-complete, while for S5 the problem is NP-complete.
Furthermore,
in \cite{DBLP:conf/aiml/ChagrovR02}
it is shown that satisfiability for K and K4 is PSPACE-complete even
for formulae without any variables.
It should be noted that the satisfiability of propositional logic is a subcase of satisfiability for any normal
modal logic, thus for any normal modal logic the problem is NP-hard.  Other results are known for multimodal
logics; all of the above are PSPACE-complete in the multimodal case.  In this paper we will focus on normal
monomodal logics and mainly on K. For an introduction to modal logic and its complexity see
\cite{130913,Fagin1995ReasoningAboutKnowledge}.

Traditional computational complexity theory attempts to characterize the complexity of a problem as a function of
the input size . The notion of parameterized complexity introduces to every hard problem a structural parameter
, which attempts to capture the aspect of the problem which causes its intractability. The central notion of
tractability in this theory is called fixed-parameter tractability (FPT): an algorithm is called FPT if it runs in
time , where  is any recursive function and  a constant. For an introduction to the vast
area of parameterized complexity see \cite{downey1999pc,flum2006pct}.

Because the definition of FPT allows for any recursive function , fixed-parameter tractable problems can
have complexities which depend on  in very different ways, ranging from sub-exponential to non-elementary.
Thus, it is one of the main goals of parameterized complexity research to find the best possible  for every
problem and this will be one of the main concerns of our work.



\textit{\textbf{Our contribution}}
In this paper we study the complexity of modal satisfiability and
validity from a parameterized, or multi-variate, point of view. Just
as parameterized complexity attempts to refine traditional
complexity theory by more specifically identifying the aspects of an
intractable problem which cause the problem's unavoidable
combinatorial explosion, we attempt to identify some structural aspects of
modal formulae which can have an impact on the solvability of
satisfiability.


One natural parameter for the satisfiability problem (in any logic)
is the number of propositional letters in the formula, which we
denote by . In propositional logic, when  is taken as a
parameter, the propositional satisfiability problem trivially
becomes fixed-parameter tractable.  As was already mentioned, this
does not generally hold in the case of satisfiability for modal
logics where the problem is hard even for constant number of
variables.

On the other hand since the satisfiability problem for modal logics
is a generalization of the same problem for propositional logics,
considering the modal satisfiability problem without bounding the
number of variables or imposing some other propositional restriction
on the formulae will result in an intractable problem. Although it
would be interesting to investigate modal satisfiability when
certain structural propositional restrictions are placed (for
example, we could say we are interested in formulae such that
removing all modality symbols leaves a 2-CNF or a Horn formula,
which are tractable cases of propositional satisfiability) this goes
beyond the scope of this work\footnote{However, see
\cite{nguyen2005cfm} for related (non-parameterized) complexity
results}. In this paper we will focus on strictly modal structural
formula restrictions and therefore we will assume that the best way
to make propositional satisfiability tractable is to restrict the
number of variables. For our purposes the conclusion is that for
modal satisfiability to become tractable, bounding  is necessary
but not sufficient.

Motivated by the above we take the approach of a double
parameterization: we investigate the complexity of satisfiability
and validity when  is considered a parameter and at the same time
some other aspect contributing to the problem's complexity is
identified and bounded.

We first study a natural notion of formula complexity called
modality depth or modal depth. This complexity measure was already
known in \cite{DBLP:journals/ai/Halpern95} where in fact a
fixed-parameter tractability result was shown when the problem is
parameterized by the sum of  and the modality depth of the
formula. However, since parameterized complexity was not well-known
at the time, in \cite{DBLP:journals/ai/Halpern95} it is only pointed
out that the problem is solvable in linear time for fixed values of
the parameters, without mentioning how different values of  and
the depth affect the running time.  We address this by upper
bounding the running time by an exponential tower of height equal to
the modality depth of the formula. More importantly, we show a lower
bound argument which proves that even though the problem is FPT, this
exponential tower in the running time cannot be avoided unless P=NP
(Theorem~\ref{thm:lower}).  Our hardness proof follows an approach of encoding
a propositional formula into a modal formula with very small modality depth.
This draws a nice connection with previously known lower bound results of this
form which also use a similar idea to prove the hardness of some (non-modal)
model checking  problems for first and second-order logic
(\cite{DBLP:journals/apal/FrickG04} and the relevant chapter in
\cite{flum2006pct}).


This result indicates that modal depth is unlikely to be a very useful
parameter because even for formulae where the depth is very moderate the
satisfiability problem is still very hard. This begs the natural question of
whether there is a way to work around the lower bound of
Theorem~\ref{thm:lower} by using another formula complexity measure in the
place of modal depth.  It is worth noting that this is a major difference
between Theorem~\ref{thm:lower} and the results of
\cite{DBLP:journals/apal/FrickG04} on FO and MSO model checking on trees,
because in that case the lower bound applies to the problem parameterized by
the formula \emph{size}, not its quantifier depth.  Since a natural formula
complexity measure would likely be bounded by some simple function of the
formula size, a search for good formula parameters is very unlikely to bear
fruit in that case. However, we show that the modal satisfiability case is
quite different: we define and study several new notions of modal formula
complexity and show that unlike modality depth, these notions can be used to
obtain not only fixed-parameter tractability results but also much more
reasonable running times.

Specifically, we define the notions of diamond dimension and box
dimension of a modal formula and show that satisfiability is FPT
when parameterized by  and the diamond dimension and validity is
FPT when parameterized by  and box dimension and that in both
cases the running times are doubly exponential in the parameters.
Then we define a measure called modal width and show that both
satisfiability and validity are FPT when parameterized by  and
the modal width and the dependence on the parameters is singly
exponential. Thus, our work points out that trying to solve satisfiability for
formulae where our proposed measures has a moderate value can be done much more
efficiently than by using the already known modality depth. All our work
focuses on K, but many of our results easily carry over to other modal logics
without much modification.



\textit{\textbf{Notation}}
The modal language of logic  contains exactly the formulae that can be
constructed using the standard propositional operators () and
the unary modality operators ().  Standard Kripke semantics are
considered here: a Kripke structure is a set of states , an accessibility
relation  between states and a valuation of propositional letters in each
state. A modal formula's truth value in a state is defined in the usual way, as
in propositional logic, with the addition of  being true in  iff
 is true in every accessible state.   is usually
considered short for .  We implicitly assume that our
language includes the constants  and , for false and true, but
these too can also be considered shorthand for  and  respectively. When a formula  is true (satisfied) in a state  of a Kripke
structure  we write . A formula  is said to be satisfiable if there exists a Kripke structure  and a state  of that structure that satisfy the formula.
A formula  is said to be valid if any Kripke structure  and state  of that structure satisfy the formula.





\section{Modal Depth} \label{sec:depth}

In this Section we give the definition of modality depth. As we will see, a
fixed-parameter tractability result can be obtained when satisfiability is
parameterized by  and the modality depth of the input formula.  This was
first observed in \cite{DBLP:journals/ai/Halpern95}, but in this section we
more precisely bound the running time (in \cite{DBLP:journals/ai/Halpern95} it
was simply noted that the running time is linear for constant depth and
constant  with a hidden constant which ``may be huge''). More importantly we
show that the ``huge constant'' cannot be significantly improved by giving a
hardness proof which shows that, if the running time of an algorithm for modal
satisfiability is significantly less than an exponential tower of height equal
to the modality depth, then P=NP.

\begin{definition}
The modality depth of a modal formula  is defined inductively
as follows:
\begin{itemize}
\item , if  is a propositional letter,
\item ,
\item ,
\item 
\end{itemize}
\end{definition}

\begin{theorem} (\cite{DBLP:journals/ai/Halpern95}) \label{thm:depth_upper}
Modal satisfiability and modal validity for the logic  are FPT when parameterized by  and .
\end{theorem}

\begin{proof}
We define the -\palto\ of a state  in a Kripke structure  to
be the set .
We will prove by induction on  that if we restrict ourselves to formulae
with at most  variables then for any  there are at most 
-\palto s, where  is the function recursively defined: ,
\hbox{}.

\begin{description}
\item[For ] If , then the formula is propositional, thus
the -\palto\ of any state is directly defined by the set of propositional
letters assigned true in the state. The number of all such possible sets of
variables is .
\item[For the case of ] The -\palto\ of
a state  depends on the assignment of the propositional letters in  and
on the truth values of formulae of the forms  and ,
where . Notice that these truth values depend only on the set
of -\palto s of the accessible states from . Thus the number of different
-\palto s on a state  is .
\end{description}

Now, suppose that  is a satisfiable formula of modality depth .
We will show how to construct a Kripke structure of about  states to
satisfy . To achieve this, for all  and for all
-types we will construct a state of that -type, thus in total we will
construct  states. To construct the
 states that give all the different -types we just construct
 states, each with a different valuation of the propositional variables.
For the subsequent levels, to construct all the states for all the different
-types we pick for each state a set of successor states out of the
states that give us the different -types and a valuation of the
propositional variables. If  is satisfiable, it must be satisfiable in
this model by adding a new state , selecting a subset of the states that
give us the different -types to be its successors and a valuation of the
propositional variables in . The number of combinations of all possible
subsets of successors and all variable valuations is , so the problem
is solvable in , because the model has
 states and thus size  and model checking can be
performed in bilinear time.

Since modal validity is the dual problem of modal satisfiability and negating
the formula doesn't change its modality depth, the same results hold for this
problem too.

\end{proof}


Let us now proceed to the main result of this Section, which is that even
though modal satisfiability is fixed-parameter tractable, the exponential tower
in the running time cannot be avoided. Specifically, we will show that solving
modal satisfiability parameterized by modality depth, even for constant , 
requires a running time which is a tower of exponentials with height depending
on the modality depth. We will prove this under the assumption that PNP,
by reducing the problem of propositional satisfiability to our problem.

Suppose that we are given a propositional CNF formula  with variables
 and we need to check whether there exists a satisfying
assignment for it.  We will encode  into a modal formula with small
depth and a constant number of variables. In order to do so we inductively
define a sequence of modal formulae.

\newcommand{\LL}{\mathcal{L}} \newcommand{\CC}{\mathcal{C}}
\newcommand{\FF}{\mathcal{F}} \newcommand{\SSS}{\mathcal{S}}
\newcommand{\CCC}{\mathcal{CA}}


\begin{itemize}


\item In order to encode the variables of  we need some formulae to
encode numbers(the indices of the variables). The modal formula
 is defined inductively as follows:  and
 where by  we
denote the -th bit of  when  is written in binary and the
least significant bit is numbered 0. So, for example , ,  and so on.  Observe that
 can only be true in a state with no successor states.  Also,
what is important is that these formulae allow us to encode very large numbers
using only a very small modality depth and no variables (or just one variable
if  is considered short for ).

\item Next, we need to encode the literals of . The modal formula
 is defined as . The
formula  is defined as .

\item Now, to encode clauses we set .

\item Finally, to encode the whole formula we use 

\end{itemize}

So far we have described how to construct a modal formula  from
.  encodes the structure of . Now we need to add
two more ingredients: we must describe with a modal formula that  is
satisfied by an assignment and that the assignment is consistent among clauses.
We give two more formulae:

\begin{itemize}

\item , where
we have introduced a single variable .

\item 

\end{itemize}

Our full construction is, given a propositional CNF formula  with 
variables named , we create the modal formula
. 


\begin{lemma}

 is satisfiable if and only if  is satisfiable in K.

\end{lemma}

\begin{proof}

Suppose that  is true in a state  of some Kripke structure.  Then
 is true in  therefore for each  we have either
 is true in  or
 is true in . From this we
create a satisfying assignment: for those  for which the first holds we set
 and for the rest . We will show that this assignment
satisfies .

Suppose that it does not, therefore there is some clause  which is not
satisfied. However, since  is true in  there exists a state 
with  such that  is true in . In every successor state of 
we have that  is true for some literal  of  and there
exists such a state for every literal of . Also, in  we have that
 is true, therefore in  we have . Therefore, in some  such that  we have 
and we also have that  is true for some literal  of .
Suppose that  is a negated literal, that is . Then
.
Therefore, since  is true in  this means that  is
true. Because  and  are both true in  there
exists an  such that  and  is true in . But then
 is true in  which implies that
our assignment gives the value false to . Since  contains 
it must be satisfied by our assignment, a contradiction. Similarly, if
 then . Clearly,
 and  cannot be true in the same state for  therefore in  we
have  which implies . Therefore in some  with
 we have  which implies that our assignment sets  to true
and since  has the literal  it must be satisfied.

The other direction is easier. First, we must construct for every  a
Kripke structure to satisfy it. For  this is a structure with just one
state with no successors. For  we take the union of the structures for
every  such that . In this union for all  such that 
there is a state for which  is true, call it . We add a state 
and set  for all  such that . Clearly,  is true in
.

Now the construction of a Kripke structure for  is straightforward. We
take the union of the structures for , with , thus we have a
state where  is true for every .  For every  with  we
create two more states: the first has as its only successor the state where
 is true.  The other has two successors: the state where  is true and
the state where  is true. Thus, for each  we have a state where
 is true and a state where  is true. For every clause
we create a state and for each literal  in the clause we add a transition
to the state where  is true. Therefore, for each clause  we have
a state where  is true. Finally, we add a state and transitions to
all the states where some  is true. Clearly,  is true in
that state, which we call the root state. It is not hard to see that 
will also be satisfied independent of where  is true, because for every 
we have made a unique state  where  is true and  is at distance
exactly 3 from the root.

Take a satisfying assignment; for every  which is true set the variable
 to true in the states of the Kripke structure where  is true. Set 
to false in every other state. Now, we must show that  is true in the
root state. This is not hard to verify because for every clause in the original
formula there is a true literal, call it . If that literal is not negated
then in the state where  is true we have  (because
the literal is not negated) and  (because the literal is true, so its
variable is true thus we must have set  to true in the variable's
corresponding state). Therefore  is true in the literal's corresponding state and
 is true in the clause's corresponding state. Similar arguments
can be made for a negated literal. Since we start with a satisfying assignment
the same can be said for every clause, thus  is also true in the root
state. 

\end{proof}



\begin{lemma} \label{lem:smdepth}

Suppose that  is a propositional CNF formula with  variables. Then,
if  the formula 
has modality depth at most , where  is the inductively defined
function  and .

\end{lemma}

\begin{proof}

First observe that the modality depth of  is at most . Therefore, we just have to bound the modality depth of .

We will use induction on  to show that .  For  we have , therefore 
and the proposition holds.

Suppose that the proposition holds for .

Observe that  because writing  in binary takes at
most  bits. If we have  then .
From the inductive hypothesis  for .  Therefore,
 and the proposition holds. 

\end{proof}





\begin{theorem} \label{thm:lower}

There is no algorithm which can solve modal satisfiability in K for formulae
with a single variable and modality depth  in time 
with , unless P=NP.

\end{theorem}

\begin{proof}

Suppose that there exists an algorithm A which in time 
can decide if a modal formula  with modality depth  and just one
variable is satisfiable. We will use this algorithm to solve propositional
satisfiability in polynomial time.

Given a propositional CNF formula  we construct  as described,
and if  has  variables let . Then
 and of course  can be constructed in time
polynomial in . Now we can use the hypothetical algorithm to see if
 is satisfiable.

We have that . Therefore, running this algorithm will take
time .  But by
the definition of  we have , therefore this bound is
polynomial in  and therefore, also in , which means that we
can solve an NP-complete problem in polynomial time. 

\end{proof}
 
\renewcommand{\dd}{\ensuremath \mathrm{d}_\Diamond}
\newcommand{\bd}{\ensuremath \mathrm{d}_\Box}

\section{Diamond Dimension} \label{sec:dimension}

In this Section we attempt to find some structural characteristics
of modal formulae which will allow us to beat the prohibitive
running time of modality depth. We define two measures, diamond
dimension and box dimension and show how they can be used to solve
satisfiability and validity respectively with a much lower running
time than modality depth.

\begin{definition}

Let  be a modal formula in negation normal form, that is, with the 
symbol appearing only directly before propositional variables. Then its diamond
dimension, denoted by  is defined inductively as follows:

\begin{itemize}

\item , if  is a propositional letter

\item 

\item 

\item 

\item 

\end{itemize}

\end{definition}



For some intuition, observe that satisfiability becomes easy if we can somehow
place a small upper bound on the number of states needed in a satisfying model.
Our goal with this measure is to prove that if  is small then
's satisfiability can be checked in models with few states. This is why
the two properties of  which can increase  are 
(which requires the creation of a new state) and  (which requires the
creation of states for both parts of the conjunction).


\begin{theorem}

If a modal formula  is satisfiable and  then there exists
a Kripke structure with  states which satisfies .

\end{theorem}

\begin{proof}

Suppose that there exists a Kripke structure which satisfies , that is
there exists some state  in that structure where  holds. We will
construct a working set of modal formulae  which will satisfy the following
properties:

\begin{itemize}

\item All formulae in  hold in .

\item ( is a valid formula.

\item .

\end{itemize}

We begin with  which obviously satisfies the above properties. We
will apply a series of transformations to  while retaining these properties
until eventually we reach a point where every formula in  is simple (in a
sense we will make precise later) and then we will construct a model with the
promised number of states for .

While possible we apply the following rules to :

\begin{enumerate}

\item If there exists a formula  such that  then remove  from  and add  and 
to .

\item If there exists a formula  such that  then remove  from . If  is true
in state  add  to , otherwise add  to .

\item If there are two formulae  and 
in  then remove them and insert the formula .

\end{enumerate}

It should be clear that rule one does maintain the properties of . Rule two
also maintains the properties: property one is maintained because we assumed
that  is true in state  therefore if  is not true we add
 which must be true. The other properties are also straightforward.
Finally, rule three follows from the fact that  is a valid formula.

It should be clear that applying all the rules until none applies will take
polynomial time.  When we can no longer apply the rules we have that , where the
 are propositional literals; in other words, we have (at most) one formula
that starts with a .

Now we will use induction on the diamond dimension to prove our theorem. Let
 be a function which upper bounds the number of states in the smallest
model which are needed to satisfy formulae of depth  (we are going to
calculate  recursively and prove that it is finite).  First, we can say
that , because a formula with diamond dimension 0 has no diamonds.
Therefore,  contains one formula that starts with a  and some
literals, for which there exists an assignment to make them all true (because
of the first property of ). Clearly, a model with just one state where we
pick this assignment will also make the formula that starts with 
trivially true, and by the second property of  will satisfy .

For the inductive step, suppose that all the satisfiable formulae of dimension
at most  need at most  states to be satisfied, where  is
the formula's dimension.  Let's consider the diamond dimension of all the
formulae in .   There are three cases: either  does not have a formula
that starts with a , or it doesn't have any formulae that start with
, or it has both.

Suppose that all the formulae in  are literals or start with . In
this case, we have for all  that . Using the
inductive hypothesis we get that the number of states to satisfy each formula
 is at most . Clearly, we can create a model which is
the union of the models for all the  plus one state where we give an
appropriate assignment to the literals and appropriate transitions so that
 is true for all . This model has at most  states.

If we have no formulae starting with diamonds we can easily see that the same
model as in the base case suffices, since  is trivially true in a
state without successors. So in this case we have just one state.

Finally, if we have both types of formulae in  we construct the
following model: consider all the formulae , for
all . Clearly, they are satisfiable, because  is true in . We know from the third property of
 that .
Therefore, . Now, we
take the union of the models for each , and each
model has at most  states. We add one state and transitions
to the appropriate states where  are true, which
together with an appropriate assignment makes all formulae of 
true in that state. The number of states is at most .

Using the simple fact that  we get from the above that 
is upper bounded by  which gives that .


\end{proof}

\begin{corollary}

Given a modal formula  with  variables and diamond
dimension  we can solve the satisfiability problem for
 in time .

\end{corollary}

\begin{proof}

It follows from the proof of the previous theorem that if  is
satisfiable there exists a model of a specific type which can
satisfy it; specifically it can be satisfied in a model where the
states are connected in a tree where the root has  children, each
of which has  children, each of which has  children and so
on. This tree has  states and exhausting all possible truth
assignments to the variables in all the states and using the fact
that model checking can be performed in linear time we get the stated running
time.

\end{proof}



Let us now tackle the validity problem. The most straightforward way
to check the validity of a formula is to check whether its negation
is satisfiable. Diamond dimension is not likely to help us directly
in this case because if a formula has low diamond dimension this
does not imply that its negation also has low dimension. Therefore,
we define a dual measure called box dimension.

\begin{definition}

Let  be a modal formula in negation normal form. Then its box dimension,
denoted by  is defined inductively as follows:

\begin{itemize}

\item , if  is a propositional letter

\item ,


\item ,  

\end{itemize}

\end{definition}

\begin{theorem}

For any formula  we have .

\end{theorem}

\begin{proof}

We use induction on the length of the formula. For formulae which are just
propositional letters or literals it is obviously true. Now take a formula
. If  then . Also,  and . Thus,  by the inductive hypothesis. The proof is similar in the other
cases.

\end{proof}

\begin{corollary}

Given a modal formula  with  variables and box dimension 
we can solve the validity problem for  in time .


\end{corollary}
 
\newcommand{\mw}{\textrm{mw}}

\section{Modal Width} \label{sec:width}

In this section we give another structural parameter for modal
formulae called modal width in an attempt to solve modal
satisfiability even more efficiently. We will show that
satisfiability and validity can be solved in time only singly
exponential in the modal width and .

First we define inductively the function  which given a modal formula
returns a set of modal formulae. Intuitively, whether  holds in a given
state  of a Kripke structure depends on two things: the values of the
propositional variables in  and the truth values of some formulae 
in the successor states of . These formulae are informally the subformulae
of  which appear at modal depth 1.  gives us exactly this set of
formulae.

\begin{itemize}

\item  if  is a propositional letter

\item , 

\item 

\end{itemize}

Now we inductively define the set , which intuitively corresponds to
the set of subformulae of  at depth .

\begin{itemize}

\item 

\item 

\end{itemize}

Finally, we can now define the modal width of a formula  at depth  as
 and the modal width of a formula as
.

Before we go on, let us prove a basic observation regarding  and
.

\begin{lemma} 
For all  we have .
\end{lemma}

\begin{proof}

Observe that for all formulae  such that  we have
. Using this fact the proof
follows easily by induction on .

\end{proof}



\begin{theorem}

There exists an algorithm which decides the satisfiability of a
modal formula  with  variables,  and
 in time .

\end{theorem}

\begin{proof}



We will need to use a function  which, given a modal formula
, returns a propositional formula which corresponds to  with all
modal subformulae replaced by new propositional variables.  can be
inductively defined as follows:

\begin{itemize}

\item  if  is a propositional letter.

\item ,
  ,
  

\item , where  is
a new propositional letter.

\end{itemize}

Notice that once again we consider  as shorthand
for .

Let  be the set of propositional
variables appearing in . For all , for
all  and for all  we define
the formula . Clearly there are at most  formulae
 defined and for each one of these we will compute
whether it is satisfiable or not using dynamic programming. We will
use a boolean matrix  of size  to store the results.

First, we have . It is not hard to see that all
formulae  are indeed satisfiable, so we initialize the
corresponding entries in  to True. Suppose now that for some  we have
filled out completely all entries . We will show how to fill out
any position in row , say position . The crucial part now is that
if we consider the formula , it will have some new variables
 which correspond to modal subformulae which all appear in .

The formula  has at most  variables. It is not hard to see
that if  is satisfiable, then  is also satisfiable,
so our first step is to check this. The truth assignments for the  variables
are easy to infer, therefore we only need to go through the  possible assignments
for the new variables. For each satisfying assignment we find we then need to
check if a model that satisfies  can be built from it.

So, suppose that  is the set of new variables, and we have found an assignment
which sets the variables of  to true and the rest to false and
satisfies . Each variable  of  corresponds to a formula
 with . If  we must make sure that
 is true in all successors of the state  where  will hold, in
the model we are building. Let  be the set of formulae 
which we conclude that must hold in all successors of  in this way.

If  we have that  must
hold in , thus  must have a successor where  is true, or equivalently
 is false. Let  be the set of formulae 
for which we conclude that they must be false in some successor of  in this way.

To decide if it is possible to build appropriate successors to 
so that all these conditions are satisfied, we look at row  of . Specifically
we consider the set of entries  such that  and
. Informally, these correspond to formulae which are satisfiable
(because the corresponding entry is set to true) and which also can serve as successors
to  without violating the conditions of , that is, in any state where they hold
all formulae which we need to be true in all successors of  are indeed true. Now, we
simply check if for each  there exists an entry in the set we have
selected so far with . If this is the case we can conclude that 
is satisfiable and set the corresponding entry of  to True, otherwise we conclude that
no satisfying model can be built from the assignment we get from , even though
 is satisfied. This whole process of computing  and 
and checking through row  of  can be performed in time .

To decide if the initial formula  is satisfiable, we compute  and
perform the same process: for every satisfying assignment of  we look at
corresponding entries of row  of  to see if a model for  can be built. The
total time for this algorithm is , because for each of the at most
 entries of  we need to check through at most  assignments and for each
we spend at most .



\end{proof}
 
\section{Conclusions and Open Problems}

In this paper we defined and studied several modal formula
complexity measures and investigated how each can be used to attack
cases of modal satisfiability. Our results show that proving
fixed-parameter tractability is only a first step in such problems,
because the dependence on the parameters can vary significantly and
some parameters offer much better algorithmic footholds than others.

It is worthy of remark that the measures of formula complexity we have
discussed are not directly comparable; for example it is possible to construct
a formula with small modality depth and very high modal width, or vice-versa.
In this sense it is not possible to infer solely from our results which formula
complexity measure is the ``best'', since each corresponds to a different
family of modal formulae. However, our results can be seen as a first attempt
at drawing a complexity ``map'' for different modal formula parameters, looking
for areas where satisfiability becomes more or less tractable. This perspective
creates a nice connection between this work and for example the research area
of graph widths, where the complexity of model checking problems on graphs is
explored in different graph families depending on a graph complexity measure.
This is a well-developed area whose insights may be applicable and helpful in
the study of the problems of this paper. (For a summary of the current
complexity ``map'' for graph width parameters see Figure 8.1\ in \cite{DBLP:journals/eccc/Grohe07})


Possible future directions are the investigation of yet more natural formula
complexity measures and attempting to improve the running times or to show good
lower bounds for the already known measures. Finally, extending our results to
other modal logics, such as modal logics where Kripke structures are required
to be reflexive or transitive (e.g. S4) would be an interesting next step.




\bibliographystyle{abbrv} \bibliography{paper}





\end{document}
