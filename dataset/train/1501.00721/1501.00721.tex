

\documentclass[12pt]{article}

\usepackage[cm]{fullpage}

\usepackage[usenames,dvipsnames,svgnames,table]{xcolor}
\usepackage{bm}\usepackage{paralist}\usepackage{picins}\usepackage{mleftright}\usepackage{tabto}\usepackage{xifthen}\usepackage{stmaryrd}




\usepackage{xspace}\usepackage{ifpdf}\usepackage{graphicx}\usepackage[normalem]{ulem} \usepackage{caption}\usepackage{amsmath}\usepackage{amssymb}\usepackage[amsmath,thmmarks]{ntheorem}\theoremseparator{.}\usepackage{wrapfig}

\usepackage{euscript}
\usepackage{mathrsfs}

\usepackage{verbatim}

\usepackage{bbding}




\usepackage{bbm}\usepackage{bbding}

\usepackage{ifluatex}
   \usepackage{ifxetex}

   \ifluatex \usepackage{fontspec}
      \usepackage[utf8]{luainputenc}
   \else
       \ifxetex \usepackage{fontspec}
\else
          \usepackage[T1]{fontenc}
          \usepackage[utf8]{inputenc}
       \fi
   \fi





\providecommand{\BibLatexMode}[1]{}
\providecommand{\BibTexMode}[1]{#1}

\ifx\UseBibLatex\undefined \renewcommand{\BibLatexMode}[1]{}
  \renewcommand{\BibTexMode}[1]{#1}
\else
  \renewcommand{\BibLatexMode}[1]{#1}
  \renewcommand{\BibTexMode}[1]{}
\fi


\BibLatexMode{\usepackage[bibencoding=ascii,style=alphabetic,backend=biber]{biblatex}\usepackage{sariel_biblatex}}










\newcommand{\imarginpar}[1]{\tabto*{-0.5cm}\smash{\llap{\parbox[t]{1in}{#1}}}\tabto{\TabPrevPos}}

\newcommand{\kentnew}[2]{\imarginpar{\textcolor{red}{Kent}}\textcolor{blue}{[DELETE] {\small #1}}{\color{red} [REPLACE] #2 }\xspace } \newcommand{\kentdelete}[1]{}


\newcommand{\IfPrinterVer}[2]{#2}\newcommand{\IfColorMath}[2]{#2}\providecommand{\Mh}[1]{{#1}}

\IfFileExists{.latex_printer_friendly}{\def\GenPrinterVer{1}}{}

\ifx\GenPrinterVer\undefined
   \IfFileExists{.latex_color}{\def\GenColorMath{1}}{}
\else
   \renewcommand{\IfPrinterVer}[2]{#1}\fi

\ifx\GenColorMath\undefined \else \renewcommand{\IfColorMath}[2]{#1}\renewcommand{\Mh}[1]{{\textcolor{red}{#1}}}\fi


\usepackage{secdot}\sectiondot{subsection}\sectiondot{subsubsection}


\IfPrinterVer{\usepackage{hyperref}}{\usepackage{hyperref}\hypersetup{breaklinks,ocgcolorlinks, colorlinks=true,urlcolor=[rgb]{0.25,0.0,0.0},linkcolor=[rgb]{0.5,0.0,0.0},citecolor=[rgb]{0,0.2,0.445},filecolor=[rgb]{0,0,0.4},
      anchorcolor=[rgb]={0.0,0.1,0.2}}
}



\newcommand{\etal}{\textit{et~al.}\xspace}
\newcommand{\Hastad}{H\r{a}stad\xspace}



\newlength{\savedparindent}
\newcommand{\SaveIndent}{\setlength{\savedparindent}{\parindent}}
\newcommand{\RestoreIndent}{\setlength{\parindent}{\savedparindent}}

\newcommand{\Term}[1]{\textsf{#1}}
\newcommand{\emphind}[1]{\emph{#1}\index{#1}}
\definecolor{blue25}{rgb}{0, 0, 11}
\newcommand{\emphic}[2]{\textcolor{blue25}{\textbf{\emph{#1}}}\index{#2}}

\ifx\PDFBlackWhite\undefined
\else
\renewcommand{\emphic}[2]{\textbf{\emph{#1}}}
\fi

\newcommand{\emphi}[1]{\emphic{#1}{#1}}
\newcommand{\emphOnly}[1]{\emph{\textcolor{blue25}{\textbf{#1}}}}



\newcommand{\cardin}[1]{\left| {#1} \right|}\newcommand{\ceil}[1]{\left\lceil {#1} \right\rceil}
\newcommand{\logup}[1]{\ceil{\log #1}}
\newcommand{\pth}[1]{\mleft({#1}\mright)}

\newcommand{\sepw}[1]{\left|\, {#1} \right.}
\newcommand{\sep}[1]{\,\left|\, {#1} \Bigl. \right.}
\newcommand{\prac}[2]{\pth{\frac{#1}{#2}}}
\newcommand{\aftermathA}{\par\vspace{-\baselineskip}}
\newcommand{\brc}[1]{\left\{ {#1} \right\}}
\newcommand{\setof}[1]{\left\{ {#1} \right\}}
\newcommand{\Set}[2]{\left\{ #1 \;\middle\vert\; #2 \right\}}
\newcommand{\pbrc}[1]{\mleft[ {#1} \mright]}
\newcommand{\SetDiff}{\triangle}

\newcommand\lhs{\hspace{2em}&\hspace{-2em}}

\newcommand{\Ex}[1]{\mathop{\mathbf{E}}\pbrc{#1}}
\newcommand{\EX}[1]{\mathop{\mathbf{E}}\Big[ #1 \Big]}
\newcommand{\Prob}[1]{\mathop{\mathbf{Pr}}\pbrc{#1}}

\newcommand{\remove}[1]{}

\newtheorem{theorem}{Theorem}\newtheorem{lemma}[theorem]{Lemma}\newtheorem*{restate*}[theorem]{Restatement of }\newtheorem{corollary}[theorem]{Corollary}

\newtheorem{conjecture}[theorem]{Conjecture}

\newtheorem{observation}[theorem]{Observation}

\newcommand{\myqedsymbol}{\rule{2mm}{2mm}}
\renewcommand{\qedsymbol}{\ensuremath{\blacksquare}}


\theoremstyle{remark}\theoremheaderfont{\sf}\theorembodyfont{\upshape}

\newtheorem{defn}[theorem]{Definition}
\newtheorem*{defn:unnumbered}{Definition}

\newtheorem*{remark:unnumbered}[theorem]{Remark}\newtheorem{remark}[theorem]{Remark}\newtheorem{example}[theorem]{Example}


\theoremheaderfont{\em}\theorembodyfont{\upshape}\theoremstyle{nonumberplain}\theoremseparator{}\theoremsymbol{\myqedsymbol}\newtheorem{proof}{Proof:}\newtheorem{proofNoColon}{Proof}

\newenvironment{proofB}{
  \noindent\emph{Proof:}
}{
  \hfill\myqedsymbol
}

\numberwithin{figure}{section}\numberwithin{table}{section}\numberwithin{equation}{section}


\newcommand{\HLinkSuffix}[3]{\hyperref[#2]{#1\ref*{#2}{#3}}}
\newcommand{\HLinkShort}[2]{\hyperref[#2]{#1\ref*{#2}}}
\newcommand{\HLink}[2]{\hyperref[#2]{#1~\ref*{#2}}}
\newcommand{\HLinkPage}[2]{\hyperref[#2]{#1~\ref*{#2}}}



\newcommand{\eqlab}[1]{\label{equation:#1}}\newcommand{\eqrefpar}[1]{\hyperref[equation:#1]{(\ref*{equation:#1})}} 


\newcommand{\figlab}[1]{\label{fig:#1}}
\newcommand{\figref}[1]{\HLink{Figure}{fig:#1}}
\newcommand{\figrefpage}[1]{\HLinkPage{Figure}{fig:#1}}

\newcommand{\seclab}[1]{\label{sec:#1}} \newcommand{\secref}[1]{\HLink{Section}{sec:#1}} \newcommand{\secrefpage}[1]{\HLinkPage{Section}{sec:#1}} 

\newcommand{\corlab}[1]{\label{cor:#1}}
\newcommand{\corref}[1]{\HLink{Corollary}{cor:#1}}\newcommand{\correfshort}[1]{\HLinkShort{C}{cor:#1}}\newcommand{\correfpage}[1]{\HLinkPage{Corollary}{cor:#1}}

\providecommand{\deflab}[1]{\label{def:#1}}
\newcommand{\defref}[1]{\HLink{Definition}{def:#1}}
\newcommand{\defrefpage}[1]{\HLinkPage{Definition}{def:#1}}

\newcommand{\apndlab}[1]{\label{apnd:#1}}
\newcommand{\apndref}[1]{\HLink{Appendix}{apnd:#1}}
\newcommand{\apndrefpage}[1]{\HLinkPage{Appendix}{apnd:#1}}

\newcommand{\exmlab}[1]{\label{example:#1}}
\newcommand{\exmref}[1]{\HLink{Example}{example:#1}}

\newcommand{\lemlab}[1]{\label{lemma:#1}}
\newcommand{\lemref}[1]{\HLink{Lemma}{lemma:#1}}
\newcommand{\lemrefpage}[1]{\HLinkPage{Lemma}{lemma:#1}}
\newcommand{\lemrefshort}[1]{\HLinkShort{L}{lemma:#1}}

\newcommand{\itemlab}[1]{\label{item:#1}}
\newcommand{\itemref}[1]{\HLinkSuffix{(}{item:#1}{)}}
\newcommand{\itemrefpage}[1]{\HLinkPageSuffix{(}{item:#1}{)}}

\newcommand{\remlab}[1]{\label{rem:#1}}
\newcommand{\remref}[1]{\HLink{Remark}{rem:#1}}
\newcommand{\remrefpage}[1]{\HLinkPage{Remark}{rem:#1}}
\newcommand{\remrefshort}[1]{\HLinkShort{R}{rem:#1}}

\newcommand{\obslab}[1]{\label{observation:#1}}
\newcommand{\obsref}[1]{\HLink{Observation}{observation:#1}}
\newcommand{\obsrefshort}[1]{\HLinkShort{O}{observation:#1}}
\newcommand{\obsrefpage}[1]{\HLinkPage{Observation}{observation:#1}}

\newcommand{\thmlab}[1]{{\label{theo:#1}}}
\newcommand{\thmref}[1]{\HLink{Theorem}{theo:#1}}
\newcommand{\thmrefpage}[1]{\HLinkPage{Theorem}{theo:#1}}
\newcommand{\thmrefshort}[1]{\HLinkShort{T}{theo:#1}}


\newcommand{\ETH}{\Term{ETH}\xspace}
\newcommand{\SETH}{\Term{SETH}\xspace}
\newcommand{\TrSAT}{\ProblemC{SAT}\xspace}
\newcommand{\kSAT}{\ProblemC{SAT}\xspace}


\newcommand{\opt}{\mathrm{Opt}}




\providecommand{\Holder}{H\"older\xspace}


\providecommand{\Mh}[1]{{#1}}

\newcommand{\Left}{\ell}\newcommand{\Right}{r}
\newcommand{\Int}{I}

\newcommand{\IntRange}[1]{\left\llbracket #1 \right\rrbracket}


\newcommand{\minksum}{\oplus}





\renewcommand{\th}{th\xspace}

\newcommand{\obj}{\Mh{f}}\newcommand{\objA}{\Mh{g}}\newcommand{\objL}{\Mh{g}}\newcommand{\objB}{\Mh{h}}\newcommand{\objC}{\Mh{e}}\newcommand{\objH}{\Mh{s}}\newcommand{\ds}{\displaystyle}

\newcommand{\ObjSet}{{\Mh{\mathcal{U}}}}\newcommand{\ObjSetA}{\Mh{\mathcal{V}}}\newcommand{\ObjSetB}{\Mh{\mathcal{H}}}

\newcommand{\Cover}{\Mh{\mathcal{C}}}



\newcommand{\bNotation}[1]{\Mh{#1}}\newcommand{\ball}{\bNotation{b}}\newcommand{\ballA}{\bNotation{b'}}\newcommand{\ballB}{\bNotation{b''}}\newcommand{\ballY}[2]{\Mh{\mathbbm{b}}\pth{#1, #2}}\newcommand{\BallSet}{\Mh{\mathcal{B}}}

\newcommand{\SetA}{\Mh{X}}\newcommand{\SetB}{\Mh{Y}}\newcommand{\SetC}{\Mh{U}}

\newcommand{\TriSet}{\Mh{\EuScript{T}}}
\newcommand{\cen}{\Mh{c}}



\newcommand{\optFSet}{\Mh{\mathcal{O}}} \newcommand{\locFSet}{\Mh{\mathcal{L}}} \newcommand{\locSet}{\Mh{L}}\newcommand{\optSet}{\Mh{O}}

\newcommand{\flower}{\Mh{F}}

\newcommand{\flowerX}[1]{\flower_{#1}}

\newcommand{\optFl}[1]{\flowerX{#1}}\newcommand{\locFl}[1]{\flowerX{#1}'}

\newcommand{\bdDiv}{\Mh{\mathcal{B}}} 


\newcommand{\lpnt}{\Mh{l}}\newcommand{\opnt}{\Mh{o}}\newcommand{\genusC}{\Mh{g}}
\newcommand{\dblC}{\Mh{c_{\mathrm{dbl}}}}
\newcommand{\dblCd}{\Mh{{c}_d}}\newcommand{\rad}{\Mh{\ell}}\newcommand{\Weight}{\Mh{W}}\newcommand{\weightOp}{\operatorname{\Mh{w}}}
\newcommand{\weightX}[1]{\weightOp\pth{#1}}
\newcommand{\excessOp}{\operatorname{\Mh{excess}}}
\newcommand{\excessOpX}[1]{\excessOp\pth{#1}}
\newcommand{\excess}{\Mh{\mathcal{E}}}


\newcommand{\volX}[1]{\operatorname{vol}\pth{#1}}
\newcommand{\diam}{\operatorname{\Mh{diam}}}\newcommand{\diamX}[1]{\operatorname{\Mh{diam}}\pth{#1}}


\newcommand{\nDiv}{m} \newcommand{\Division}{\mathcal{W}}
\newcommand{\Family}{\Mh{\EuScript{F}}}\newcommand{\FamilyA}{\Mh{\EuScript{G}}}

\newcommand{\Lines}{\mathcal{L}} \newcommand{\linesegment}{\ell}  \newcommand{\lineA}{\ell_1}      \newcommand{\lineB}{\ell_2}      

\newcommand{\SegSet}{\mathcal{S}}
\newcommand{\seg}{s}\newcommand{\segA}{t}


\newcommand{\VSet}{W} \newcommand{\VSetA}{\mathcal{X}}
\newcommand{\VSetB}{\mathcal{Y}} \newcommand{\VSetC}{\mathcal{Z}}

\newcommand{\sphereC}{{\mathbb{{S}}}}\newcommand{\sphereCBig}{\mathbb{S}}\newcommand{\sphereX}[2]{\sphereCBig\pth{#1, #2}}

\renewcommand{\Re}{{\mathbb{R}}}
\newcommand{\reals}{\mathbb{R}} 

\newcommand{\naturalnumbers}{\mathbb{N}} \newcommand{\integers}{\mathbb{Z}} \newcommand{\nnintegers}{\integers_{\geq 0}}



\newcommand{\NaturalNumbers}{\mathbb{N}} \newcommand{\Red}{\Re^d}

\newcommand{\distCharX}[1]{\mathsf{d}_{#1}}
\newcommand{\distSetY}[2]{\mathsf{d}\pth{#1,#2}}
\newcommand{\repX}[1]{\mathrm{rep}\pth{#1}}

\newcommand{\gradC}{\mathbbm{d}} 

\newcommand{\gradY}[2]{\gradC_{#1}\pth{#2}}




\providecommand{\lexprod}{\bullet} \newcommand{\edgedensityof}[1]{\frac{\cardin{\EdgesX{#1}}}{\cardin{\verticesof{#1}}}}
\newcommand{\mathedgedensityof}[1]{\cardin{\EdgesX{#1}} / \cardin{\verticesof{#1}}}






\newcommand{\Otilde}{\widetilde{O}}


\newcommand{\centerX}[1]{\mathrm{center}\pth{#1}}
\newcommand{\FDecomp}[2]{\raisebox{-2pt}{\text{\small \SixFlowerPetalDotted}}\pth{#1, #2}}

\newcommand{\TreeX}[1]{T_{#1}}
\newcommand{\nnK}[3]{\Mh{\mathrm{NN}}_{#1}\pth{#2, #3}}
\newcommand{\ServeX}[1]{\Mh{S}\pth{#1}}

\newcommand{\MaxDemand}{{\widehat{\delta}}}

\newcommand{\reachX}[1]{\Mh{\tau}\pth{#1}}\newcommand{\MaxReach}{{\widehat{\tau}}}

\newcommand{\clusters}{\Mh{\mathcal{W}}} \newcommand{\cluster}{\Mh{C}} 

\newcommand{\clusterA}{\Mh{\mathcal{C}}} 

\newcommand{\clusterX}[1]{\cluster\pth{#1}}


\providecommand{\excessCmd}{\operatorname{excess}} \newcommand{\excessof}[1]{\excessCmd\pth{#1}} 



\newcommand{\distSetG}[2]{\distCharX{\graph}\pth{ #1, #2}}
\newcommand{\distG}[2]{\distCharX{\graph}\pth{ #1, #2}}

\newcommand{\trA}{\Mh{\pi}}
\newcommand{\trB}{\Mh{\sigma}}

\newcommand{\distY}[2]{\mleft\| #1 - #2 \mright\|}
\newcommand{\normX}[1]{\left\| #1 \right\|}


\newcommand{\minor}[1][]{\prec_{#1}}\newcommand{\ShallowMinors}[2]{\triangledown_{\!#2}\pth{#1}}

\newcommand{\ShallowMinorsGraph}[1]{\triangledown_{\!#1}\pth{\graph}}
\newcommand{\Clique}{K}
\newcommand{\Injections}[2]{(#1 \hookrightarrow #2)}
\newcommand{\Prop}{\Mh{\Pi}}\newcommand{\Propv}{\Mh{\Pi}_{\Vertices}}\newcommand{\Propo}{\Mh{\Pi}_{\ObjSet}}

\providecommand{\CNFX}[1]{ {\em{\textrm{(#1)}}}}\providecommand{\CNFSoCG}{\CNFX{SoCG}}\providecommand{\CNFCCCG}{\CNFX{CCCG}}\providecommand{\CNFFOCS}{\CNFX{FOCS}}\providecommand{\CNFSTOC}{\CNFX{STOC}}\providecommand{\CNFSODA}{\CNFX{SODA}}

\newcommand{\cDensity}{\Mh{\rho}} \newcommand{\densityOp}{\Mh{\mathop{\mathrm{density}}}}\newcommand{\densityX}[1]{\densityOp\pth{#1}}\newcommand{\cDensityA}{\Mh{\sigma}} \newcommand{\cBoundary}{\Mh{\nu}} \newcommand{\volume}{\Mh{\operatorname{vol}}} \newcommand{\volumeof}[1]{\volume\of{#1}} 

\newcommand{\NbrX}[1]{\Mh{N}\pth{#1}}

\newcommand{\PntSet}{\ensuremath{\Mh{P}}\xspace}\newcommand{\PntSetA}{\ensuremath{\Mh{Q}}\xspace}

\newcommand{\PointDec}[1]{\Mh{#1}}

\newcommand{\pnt}{\PointDec{p}}\newcommand{\pntA}{\PointDec{q}}\newcommand{\pntB}{\PointDec{u}} \newcommand{\pntC}{\PointDec{v}}

\newcommand{\IntSet}{\mathcal{I}} \newcommand{\interval}{I}       \newcommand{\intervalA}{I}      \newcommand{\intervalB}{J}      



\newcommand{\dvolumeof}[1]{\diametricvolumeof{#1}}



\newcommand{\eps}{\Mh{\varepsilon}}

\newcommand{\SepSet}{\Mh{Z}}


\newcommand{\DomSet}{\Mh{D}}
\newcommand{\DomSetA}{\widehat{\Mh{D}}}

\newcommand{\CovSet}{\Mh{R}} \newcommand{\CovSetA}{\Mh{\widehat{{R}}}} 

\newcommand{\UndomSet}{\mathcal{C}} \newcommand{\DiskOrg}{\mathsf{disk}}
\newcommand{\CH}{\mathcal{CH}} \newcommand{\CHX}[1]{{\CH}\pth{#1}}

\newcommand{\feasibles}{\mathcal{F}}

\newcommand{\Metric}{\mathcal{X}}


\newcommand{\Vertices}{\Mh{V}}\newcommand{\VerticesA}{\Mh{U}}

\newcommand{\SetX}{\Mh{X}}
\newcommand{\SetY}{\Mh{Y}}
\newcommand{\SetZ}{\Mh{Z}}

\newcommand{\SetL}{\Mh{L}}\newcommand{\SetR}{\Mh{R}}

\newcommand{\VerticesX}[1]{\Mh{V}\pth{#1}}\newcommand{\verticesof}[1]{\Mh{V}\pth{#1}}

\newcommand{\Edges}{\Mh{E}}
\newcommand{\EdgesX}[1]{\Edges\pth{#1}}
\newcommand{\edgesof}[1]{\EdgesX{#1}}

\newcommand{\optset}{\Mh{O}} \newcommand{\locset}{\Mh{L}} 

\newcommand{\Independents}{\mathcal{I}}
\newcommand{\edgeY}[2]{{#1#2}}

\newcommand{\Edge}{\Mh{e}}


\newcommand{\Opt}{\Mh{{O}}}\newcommand{\locSol}{\Mh{{L}}}

\newcommand{\locSolA}{\Mh{\widehat{{L}}}}\newcommand{\OptA}{\Mh{{\widehat{{O}}}}}



\newcommand{\BVertices}{\Mh{B}}

\newcommand{\Sol}{\mathcal{S}}

\newcommand{\OptCls}{\mathcal{O}}
\newcommand{\locSolClusters}{\mathcal{L}}
\newcommand{\bdCls}{{\mathcal{B}}}

\newcommand{\bSize}{\Mh{{b}}} \newcommand{\oSize}{\Mh{{o}}} \newcommand{\lSize}{\Mh{{l}}} 


\newcommand{\cDemand}{\beta}



\newcommand{\iCov}{\Mh{\omega}}\newcommand{\ICovGraph}[2]{#1\pbrc{#2}}

\newcommand{\icgA}{\ICovGraph{\graph}{\Family}} \newcommand{\IObjSet}[2]{#1\pbrc{#2}}

\newcommand{\ProblemC}[1]{\textsf{#1}}

\DefineNamedColor{named}{ColorComplexityClass}{cmyk}{0.64,0.0,0.95,0.80}
\providecommand{\ComplexityClass}[1]{{{\textcolor[named]{ColorComplexityClass}{\textsc{#1}}}}}

\newcommand{\Interval}{J}




\newcommand{\ZPP}{\ComplexityClass{ZPP}\xspace}
\newcommand{\POLYT}{\ComplexityClass{P}\xspace}

\newcommand{\poly}{\operatorname{poly}}\newcommand{\polylog}{\operatorname{polylog}}
\newcommand{\MaxSNPHard}{\ComplexityClass{MaxSNP-Hard}\xspace}

\newcommand{\PTAS}{\Term{PTAS}\xspace}
\newcommand{\QPTAS}{\Term{QPTAS}\xspace}

\newcommand{\NP}{\ComplexityClass{NP}\xspace}
\providecommand{\NPComplete}{{\ComplexityClass{NP-Complete}}\index{NP!Complete}\xspace }
\providecommand{\NPHard}{{\ComplexityClass{NP-Hard}}\index{NP!hard}\xspace}
\newcommand{\APXHard}{\ComplexityClass{APX-Hard}\xspace}

\newcommand{\ssspace}{\vspace{1em}} \renewcommand{\ssspace}{}

\newcommand{\cSD}{\sigma}

\newcommand{\XSays}[2]{{{\fbox{\tt #1:}
    } #2 \marginpar{\textcolor{red}{#1}}
    {\fbox{\tt
        end}}}}
\newcommand{\sariel}[1]{{\XSays{Sariel}{#1}}}
\newcommand{\Sariel}[1]{{\XSays{Sariel}{#1}}}
\newcommand{\Kent}[1]{\XSays{Kent}{#1}} \newcommand{\kent}{\Kent}

\newcommand{\Nesetril}{N{e{\v s}et{\v r}il}\xspace}
\newcommand{\si}[1]{#1}

\newcommand{\IGraph}[1]{\graph_{#1}}
\newcommand{\GInduced}[1]{\graph_{|{#1}}}


\newcommand{\Partition}{\Mh{\mathcal{P}}}\newcommand{\PartitionA}{\Mh{\mathcal{P}'}}

\newcommand{\atgen}{\symbol{'100}}


\newcommand{\KentThanks}{\thanks{Department of Computer Science; University of Illinois; 201 N. Goodwin Avenue; Urbana, IL, 61801, USA; {\tt quanrud2\atgen{}illinois.edu}; {\tt\href {http://illinois.edu/\string~quanrud2/}{http://illinois.edu/\string~quanrud2/}}. }}\newcommand{\SarielThanks}[1][]{\thanks{Department of Computer Science; University of Illinois; 201 N. Goodwin Avenue; Urbana, IL, 61801, USA; {\tt sariel\atgen{}illinois.edu}; {\tt \url{http://sarielhp.org/}}. #1}} 

\newcommand{\EPTAS}{\Term{EPTAS}\xspace} \newcommand{\nfrac}[2]{#1/#2}
\newcommand{\ts}{\hspace{0.6pt}} 


\newcommand{\demandOp}{\Mh{\delta}}\newcommand{\demandX}[1]{\demandOp\pth{#1}}

\newcommand{\ldense}{crammed\xspace}
\newcommand{\Ldense}{Crammed\xspace}
\newcommand{\cLD}{\sigma}

\newcommand{\distSet}[2]{d\pth{#1,#2}}



\newenvironment{results}{
  \begin{compactenum}[(a)]
  }{
  \end{compactenum}
}

\newenvironment{properties}{
  \begin{compactenum}[(i)]
  }{
  \end{compactenum}
}



\DeclareMathAlphabet{\mathantt}{OT1}{antt}{li}{it}
\DeclareMathAlphabet{\mathpzc}{OT1}{pzc}{m}{it}

\DeclareMathAlphabet{\mathcalligra}{T1}{calligra}{m}{n}









\newcommand{\lenX}[1]{\normX{#1}}

\newcommand{\headsX}[1]{\Mh{\mathrm{heads}}\pth{#1}}


\newcommand{\pleftX}[1]{\Left\pth{#1}}
\newcommand{\prightX}[1]{\Right\pth{#1}}

\newcommand{\xLof}[1]{x_\Left\pth{#1}}\newcommand{\hL}[1]{y_\Left\pth{#1}}

\newcommand{\xR}[1]{x_\Right\pth{#1}}\newcommand{\hR}[1]{y_\Right\pth{#1}}

\newcommand{\kSD}{k}

\newcommand{\exSize}{\Mh{\lambda}}

\newcommand{\IncludeGraphics}[2][]{\typeout{}\typeout{Graphics: #2}\typeout{\ includegraphics[#1]{#2}}\includegraphics[#1]{#2}
  \typeout{}}


\newcommand{\IncGraphPage}[4][]{\IncludeGraphics[page=#4,#1]{{#2/#3}}
}

\newcommand{\Instance}{I}

\newcommand{\defGraph}{\graph = (\Vertices,\Edges)}

\newcommand{\class}{\Mh{\mathcal{C}}}
\newcommand{\classDY}[2]{\class_{\!#2}^{#1}}\newcommand{\minorsDY}[2]{{\Mh{\nabla}_{\!#1}} \pth{#2} } 

\newcommand{\GraphNotation}[1]{\Mh{#1}}


\newcommand{\clusterZ}[1]{\Mh{C}_{#1}}
\newcommand{\cvX}[1]{\Mh{c}_{#1}} \newcommand{\prmtX}[1]{\Mh{\pi}\pth{#1}}\newcommand{\clusteredges}{\EdgesX{\icgA}} \newcommand{\clusteredge}[2]{\edgeY{\clusterZ{#1}}{\clusterZ{#2}}}
\newcommand{\scoopedge}[2]{\edgeY{\scoop{#1}}{\scoop{#2}}}
\newcommand{\rcvX}[1]{\cvX{\prmtX{#1}}}\newcommand{\randomcluster}[1]{\clusterZ{\prmtX{#1}}}\newcommand{\scoop}[1]{\clusterZ{#1}'}
\newcommand{\rsX}[1]{\Mh{C}_{\!\prmtX{#1}}'}\newcommand{\subclusters}{\Family'} \newcommand{\scoops}{\Family'} \newcommand{\leafedges}{\Edges'_1} \newcommand{\smalledges}{\Edges_1} \newcommand{\bigedges}{\Edges_2} 


\newcommand{\graph}{\GraphNotation{G}}\newcommand{\graphA}{\GraphNotation{H}}\newcommand{\graphB}{\GraphNotation{K}}\newcommand{\graphC}{\GraphNotation{F}}\newcommand{\graphD}{\GraphNotation{L}}

\newcommand{\restatementLemma}[2]{\noindent \textbf{Restatement of #1.} {\emph{#2{}}}}




\newcommand{\TheoremDefExt}[3]{\expandafter\newcommand\csname bodyThm#1\endcsname{#2}
  \ifthenelse{\isempty{#3}}{\begin{theorem}\thmlab{#1}#2
    \end{theorem}}{\begin{theorem}[#3]\thmlab{#1}#2
    \end{theorem}}}

\newcommand{\TheoremDef}[3]{\TheoremDefExt{#1}{#2}{}}

\newcommand{\TheoremBody}[1]{\csname bodyThm#1\endcsname }

\newcommand{\TheoremRestate}[1]{\noindent \textbf{Restatement of \thmref{#1}.} {\emph{\TheoremBody{#1}}}}



\newcommand{\LemmaDefExt}[3]{\expandafter\newcommand\csname bodyLemBody#1\endcsname{#2}
  \ifthenelse{\isempty{#3}}{\begin{lemma}\lemlab{#1}#2
    \end{lemma}}{\begin{lemma}[#3]\lemlab{#1}#2
    \end{lemma}}}

\newcommand{\LemmaDef}[3]{\LemmaDefExt{#1}{#2}{}}

\newcommand{\LemmaBody}[1]{\csname bodyLemBody#1\endcsname }

\newcommand{\LemmaRestate}[1]{\noindent \textbf{Restatement of \lemref{#1}.} {\emph{\LemmaBody{#1}}}}

\newcommand{\mytfrac}[2]{\frac{#1\rule[-3\lineskip]{-0.001cm}{0\lineskip}}{#2 \rule{-0.001cm}{7\lineskip}}}


\definecolor{sarielChangeColor}{rgb}{0.3,0.451,0.055}\newcommand{\SC}[1]{{\textcolor{sarielChangeColor}{#1}}\marginpar{\textcolor{sarielChangeColor}{Sariel}}}\newcommand{\SCStart}{\color{sarielChangeColor}\marginpar{CH }\xspace}\newcommand{\SCEnd}{\color{black}\marginpar{CH }\xspace}



\providecommand{\institutex}{Department of Computer Science, University of Illinois\\ 201 N. Goodwin Avenue, Urbana, IL, 61801, USA.\\ \email{\{sariel,quanrud2\}@illinois.edu}}
\newcommand{\SDEL}[1]{{\textcolor{blue}{\sout{#1}}}}

\newcommand{\ProblemE}[3]{\begin{minipage}{0.31\linewidth}
    \smallskip #1\smallskip \end{minipage}
  &
  \begin{minipage}{0.26\linewidth}
    \smallskip #2\smallskip \end{minipage}
  &
  \begin{minipage}{0.3\linewidth}
    \smallskip #3\smallskip \end{minipage}\\}



\IfFileExists{sariel_computer.sty}{\def\sarielComp{1}}{}
\ifx\sarielComp\undefined \newcommand{\SarielComp}[1]{}
\newcommand{\NotSarielComp}[1]{#1}\else
\newcommand{\SarielComp}[1]{#1}\newcommand{\NotSarielComp}[1]{}\fi

\SarielComp{\ifx\colorMath\undefined \else
  \DefineNamedColor{named}{ColorMath}{rgb}{0.5,0.2,0}
  \renewcommand{\Mh}[1]{{\textcolor{ColorMath}{#1}}}
  \fi
}



\IfPrinterVer{\definecolor{blue25}{rgb}{0,0,0}
  \DefineNamedColor{named}{RedViolet} {rgb}{0,0,0}\DefineNamedColor{named}{ChangeColor}{rgb}{0, 0, 0}
}{\definecolor{blue25}{rgb}{0,0,0.7}
  \DefineNamedColor{named}{RedViolet} {cmyk}{0.07,0.90,0,0.34}
  \DefineNamedColor{named}{ChangeColor}{rgb}{0, 0.2, 0.0}
}



\newcommand{\SH}{


   \bigskip \noindent
   \begin{minipage}{0.99\linewidth}
       \hrule
       \hrule

       \smallskip

       \centerline{\huge \textcolor{red}{SH: Stopped here
          }}

       \smallskip

       \hrule
       \hrule
       \hrule
       \hrule

       \bigskip \end{minipage}

   \xspace }







\BibLatexMode{}

\begin{document}


\title{Approximation Algorithms for Polynomial-Expansion and
  Low-Density Graphs\thanks{Work on this paper was partially supported by a NSF AF
    awards CCF-1421231, and CCF-1217462. A preliminary version of this paper appeared in E{S}A 2015
    \cite{hq-aapel-15}. A talk by the first author in SODA 2016 was
    based to some extent on the work in this paper.}}

\author{Sariel Har-Peled\SarielThanks{}\and Kent Quanrud\KentThanks{}}


\maketitle





\begin{abstract}
  We investigate the family of intersection graphs of low density
  objects in low dimensional Euclidean space.  This family is quite
  general, includes planar graphs, and in particular is a subset of
  the family of graphs that have polynomial expansion.

  We present efficient -approximation algorithms for
  polynomial expansion graphs, for \ProblemC{Independent Set},
  \ProblemC{Set Cover}, and \ProblemC{Dominating Set} problems, among
  others, and these results seem to be new. Naturally, \PTAS{}'s for
  these problems are known for sub{}classes of this graph family.
  These results have immediate applications in the geometric
  domain. For example, the new algorithms yield the only \PTAS known
  for covering points by fat triangles (that are shallow).

  We also prove corresponding hardness of approximation for some of
  these optimization problems, characterizing their intractability
  with respect to density. For example, we show that there is no \PTAS
  for covering points by fat triangles if they are not shallow, thus
  matching our \PTAS for this problem with respect to depth.
\end{abstract}

\section{Introduction}

\paragraph{Motivation.}

Geometric set cover, as the name suggests, is a geometric
instantiation of the classical set cover problem. For example, given a
set of points and a set of triangles (with fixed locations) in the
plane, we want to select a minimum number of triangles that cover all
of the given points. Similar geometric variants can be defined for
independent set, hitting set, dominating set, and the like.

For relatively simple shapes, such geometric instances should be
computationally easier than the general problem. By now there is a
large yet incomplete collection of results on such problems, listed
below. For example, one can get -approximation to the
geometric set cover problem when the regions are disks, but we do not
have such an approximation algorithm if the regions are fat triangles
of similar size.  This discrepancy seems arbitrary and somewhat
baffling, and the purpose of this work is to better understand these
subtle gaps.


\paragraph{Plan of attack.}
We explore the type of graphs that arises out of these geometric
instances, and in the process introduce the class of low-density
graphs. We explore the properties of this graph class, and the
optimization problems that can be approximated efficiently on the
broader class of graphs that have small separators. Separabilitity
turns out to be the key ingredient needed for efficient approximation.
We also study lower bounds for such instances, characterizing when
they are computationally hard.


\begin{figure}[t]
\begin{center}
    \begin{tabular}{|l|l|l|}
      \hline
Objects & \si{Approx.} \si{Alg.} & Hardness\\
      \hline \ProblemE{Disks/pseudo-disks}{ \PTAS \cite{mrr-sahsg-14}}{ Exact version \NPHard\\ \cite{fg-oafac-88} }
\hline \ProblemE{Fat triangles of same size}{  \cite{cv-iaags-07}}{ \APXHard: \lemref{no:PTAS:fat:tr:set:cover}\\
      I.e., no \PTAS possible.
      }
\hline \ProblemE{Fat objects in }{  \cite{abes-ibulf-14}}{\APXHard:  \lemrefshort{no:PTAS:fat:tr:set:cover}}
\hline \ProblemE{Objects , 
      density\\
      E.g. fat objects,  depth.}{\PTAS: \thmref{g:hitting:set:cover}}{Exact version \NPHard\\\cite{fg-oafac-88}}
      \hline \ProblemE{Objects with  density}{\QPTAS: \thmref{g:hitting:set:cover}}
      {No \PTAS under \ETH\\
      \lemref{e:t:h:g:polylog:d}}\hline \ProblemE{Objects with  density  in }{\PTAS:
      \thmref{g:hitting:set:cover}
      \\
      RT:
      .}{No -approx

      with RT
      

      \si{assuming} \ETH: \lemrefshort{e:t:h:g:polylog:d}}
\hline
\end{tabular}\hfill
  \end{center}

  \vspace{-0.3cm}
  \caption{Known results about the complexity of geometric
    set-cover. The input consists of a set of points and a set of
    objects, and the task is to find the smallest subset of objects
    that covers the points.  To see that the hardness proof of Feder
    and Greene \cite{fg-oafac-88} indeed implies the above, one just
    needs to verify that the input instance their proof generates has
    bounded depth.  A \QPTAS is an algorithm with running time
    .  }
  \figlab{set:cover:summary}
\end{figure}

\subsection{Background}

\subsubsection{Optimization problems}

\paragraph{Independent set.}

Given an undirected graph , an \emph{independent set} is a
set of vertices  such that no two vertices
in  are connected by an edge. It is \NPComplete to decide if a
graph contains an independent set of size  \cite{k-racp-72}, and
one cannot approximate the size of the maximum independent set to
within a factor of , for any fixed , unless
 \cite{h-chaw-99}.


\paragraph{Dominating set.}

Given an undirected graph , a \emph{dominating set} is a
set of vertices  such that every vertex
in  is either in  or adjacent to a vertex in
. It is \NPComplete to decide if a graph contains a
dominating set of size  (by a simple reduction from set cover,
which is \NPComplete \cite{k-racp-72}), and one cannot obtain a
 approximation (for some constant ) unless 
\cite{rs-sbepl-97}.

\subsubsection{Graph classes}
\paragraph{Density.}

Informally, a set of objects in  is \emph{low-density} if no
object can intersect too many objects that are larger than it. This
notion was introduced by van \si{der} Stappen
\etal~\cite{sobv-mpelo-98}, although weaker notions involving a single
resolution were studied earlier (e.g. in the work by Schwartz and
Sharir \cite{ss-empae-85}). A closely related geometric property to
density is \emph{fatness}. Informally, an object is fat if it contains
a ball, and is contained inside another ball, that up to constant
scaling are of the same size.  Fat objects have low union complexity
\cite{aps-sugo-08}, and in particular, shallow fat objects have low
density \cite{f-mpafo-92}. Here, a set of shapes is \emph{shallow} if
no point is covered by too many of them.

\paragraph{Intersection graphs.}

A set  of objects in  induces an \emph{intersection
  graph}  having  as its set of vertices,
and two objects  are connected by an edge if
and only if .  Without any
restrictions, intersection graphs can represent any graph. Motivated
by the notion of density, a graph is a \emph{low-density graph} if it
can be realized as the intersection graph of a low-density collection
of objects in low dimensions.

There is much work on intersection graphs, from interval graphs, to
unit disk graphs, and more. The circle packing theorem
\cite{k-kdka-36,a-ocpls-70,pa-cg-95} implies that every planar graph
can be realized as a coin graph, where the vertices are interior
disjoint disks, and there is an edge connecting two vertices if their
corresponding disks are touching. This implies that planar graphs are
low density.  Miller \etal \cite{mttv-sspnng-97} studied the
intersection graphs of balls (or fat convex object) of bounded depth
(i.e., every point is covered by a constant number of balls), and
these intersection graphs are readily low density. Some results
related to our work include: (i) planar graphs are the intersection
graph of segments \cite{cg-epgig-09}, and (ii) string graphs (i.e.,
intersection graph of curves in the plane) have small separators
\cite{m-nossg-14}.

\paragraph{Polynomial expansion.}

The class of low-density graphs is contained in the class of graphs
with polynomial expansion. This class was defined by \Nesetril and
Ossona \si{de} Mendez as part of a greater investigation on the
sparsity of graphs (see the book \cite{no-s-12}). A motivating
observation to their theory is that the sparsity of a graph (the ratio
of edges to vertices) is not necessarily sufficient for tractability.
For example, a clique (which is a graph with maximum density) can be
disguised as a sparse graph by splitting every edge by a middle
vertex. Furthermore, constant degree expanders are also sparse.  For
both graphs, many optimization problems are intractable (intuitively,
because they do not have a small separator).

Graphs with bounded expansion are \emph{nowhere dense graphs}
\cite[Section 5.4]{no-s-12}. Grohe \etal \cite{gks-dfopndg-14}
recently showed that first-order properties are fixed-parameter
tractable for nowhere dense graphs. In this paper, we study graphs of
polynomial expansion \cite[Section 5.5]{no-s-12}, which intuitively
requires a graph to not only be sparse, but have shallow minors that
are sparse as well.

It is known that graphs with polynomial expansion have sublinear
separators \cite{no-gcbe1-08}. The converse, that any graph that has
hereditary sublinear separators has polynomial expansion, was recently
shown by Dvo{\v{r}}{\'{a}}k and Norin \cite{dn-ssspe-15}. As such, our
work looks beyond the geometric setting to consider the general role
of separators in approximation.


\subsubsection{Further related work}

There is a long history of optimization in structured graph
classes. Lipton and Tarjan first obtained a \PTAS for independent set
in planar graphs by using separators
\cite{lt-stpg-79,lt-apst-80}. Baker \cite{b-aancp-94} developed
techniques for covering problems (e.g.\ dominated set) on planar
graphs. Baker's approach was extended by Eppstein \cite{e-dtmcg-00} to
graphs with bounded local treewidth, and by Grohe \cite{g-ltwem-03} to
graphs excluding minors. Separators have also played a key role in
geometric optimization algorithms, including:
\begin{inparaenum}[(i)]
\item \PTAS for independent set and (continuous) piercing set for fat
  objects \cite{c-ptasp-03, mr-irghs-10},
\item \QPTAS for maximum weighted independent sets of polygons
  \cite{aw-asmwi-13,aw-qmwis-14,h-qssp-14}, and
\item \QPTAS for \ProblemC{Set Cover} by pseudodisks
  \cite{mrr-qgscp-14}, among others.
\end{inparaenum}
Lastly, Cabello and Gajser \cite{cg-spfgem-14} develop \PTAS{}'s for
some of the problems we study in the specific setting of minor-free
graphs.


\begin{figure}[t]
\begin{center}
    \begin{tabular}{|l|l|l|}
      \hline
      Objects & \si{Approx.} \si{Alg.} & Hardness\\
      \hline \ProblemE{Disks/pseudo-disks}{\PTAS \cite{mr-irghs-10}}{\smallskip Exact version \NPHard\\via point-disk duality \cite{fg-oafac-88}}
\hline
      \ProblemE{Fat triangles of similar size.}
      { \cite{aes-ssena-10}}{\smallskip \APXHard: \lemref{no:PTAS:fat:hit:set}}
\hline
      \ProblemE{Objects with  density.}
      {\PTAS:     \thmref{g:hitting:set:cover}}{Exact \si{ver.} \NPHard \cite{fg-oafac-88}}
\hline
      \ProblemE{Objects  density.}
      {\QPTAS:     \thmref{g:hitting:set:cover}}{\smallskip No \PTAS under \ETH\\
      \lemref{e:t:h:g:polylog:d} /
      \lemrefshort{no:PTAS:fat:hit:set}}
\hline \ProblemE{Objects with  density  in }{\PTAS:
      \thmref{g:hitting:set:cover}\\ run time
      
}{No -approx

      with RT
      

      assuming \ETH: \lemrefshort{e:t:h:g:polylog:d}}
      \hline
\end{tabular}
  \end{center}
  \vspace{-0.3cm}
  \caption{Known results about the complexity of \emph{discrete}
    geometric hitting set. The input is a set of points, and a set of
    objects, and the task is to find the smallest subset of points
    such that any object is hit by one of these points.  }
\figlab{hitting:set:summary}
\end{figure}

\subsection{Our results}

We systematically study the class of graphs that have low density,
first proving that they have polynomial expansion.  We then develop
approximation algorithms for this broader class of graphs, as follows:

\smallskip \SaveIndent \begin{compactenum}[(A)]\RestoreIndent \setlength{\itemsep}{3pt}
\item \textbf{\PTAS for independent set.} For graphs that have sublinear hereditary separators we show \PTAS
  for independent set, see \secref{approx:v:separators}. This covers
  graphs with low density and polynomial expansion. These results are
  not surprising in light of known results \cite{ch-aamis-12}, but
  provide a starting point and contrast for subsequent results.

\item \textbf{\PTAS for packing problems.}  The above \PTAS also hold
  for packing problems, such as finding maximal induced planar
  subgraph, and similar problems, see \exmref{geometric:packing} and
  \lemref{indep:easy}.


\item \textbf{\PTAS for independent/packing when the output is
    sparse.} More surprisingly, one get a \PTAS even if the subgraph induced on
  the union of two solutions has polynomial expansion. Thus, while the
  input may not be sparse, as long as the output is sparse, one can
  get an efficient approximation algorithms, see \thmref{independent}.

  In particular, this holds if the output is required to have low
  density, because the union of two sets of objects with low density
  is still low density.  The resulting algorithms in the geometric
  setting are faster than those for polynomial expansion graphs, by
  using the underlying geometry of low-density graphs.


\item \textbf{\PTAS for dominating set.}~Low density graphs remain low density even if one merges locally
  objects that are close together, see
  \lemref{density:shallow:minors}. More generally, if one considers a
  collection of -shallow subgraphs (i.e., subgraphs with radius 
  in the edge distance) of a polynomial expansion graph, then their
  intersection graph also has polynomial expansion, as long as
  \emph{no} vertex in the original graph participates in more than
  constant number of subgraphs.

  This surprising property implies that local search algorithms
  provides a \PTAS for problems like \ProblemC{Dominating Set} for
  graphs with polynomial expansion, see \secref{dom:set}.

\item \textbf{\PTAS for multi-cover dominating set with reach
    constraints.}~These results can be extended to multi-cover variants of dominating
  set for such graphs, where every vertex can be asked to be dominated
  a certain number of times, and require that the these dominated
  vertices are within a certain distance.  See
  \lemref{ptas:subset:dom:2}.

\item \textbf{Connected dominating set.}~The above algorithms also extend to a \PTAS for connected dominating
  set, see \secref{connected:dominating:set}.

\item \textbf{\PTAS for vertex cover for graphs with polynomial
    expansion.} See \obsref{v:c:poly:expansion}.


\item \textbf{\PTAS for geometric hitting set and set cover.}~The new algorithms for dominating sets readily provides \PTAS's for
  discrete geometric set cover and hitting set for low density inputs,
  see \secref{geometric:applications}.

\item \textbf{Hardness of approximation.}~The low-density algorithms
  are complimented by matching hardness results that suggest the new
  approximation algorithms are nearly optimal with respect to depth
  (under \SETH: the assumption that \ProblemC{SAT} over  variables
  can not be solved in better than  time).
\end{compactenum}
\smallskip The context of our results, for geometric settings, is summarized in
\figref{set:cover:summary} and \figref{hitting:set:summary}.

\paragraph{Sparsity is not enough.}

It is natural to hope that the above algorithms would work for sparse
graphs (i.e., graphs that have linear number of edges). Unfortunately,
as mentioned earlier, constant degree expanders, which play a
prominent rule in theoretical computer science, are sparse, and the
above algorithms fail for them as they do not have separators.

\paragraph{Low level technical contributions.}

We show that graphs with polynomial expansion retain polynomial
expansion even if one is allowed to locally connect a vertex to other
nearby vertices in a controlled way. To this end, we extend the notion
of -shallow minors to shallow packings (see
\defref{shallow:packing}). We then use a probabilistic argument to
show that the associated intersection graph still has polynomial
expansion, see \lemref{edge-density-shallow-cover}. The proof of this
lemma is elegant and might be of independent interest.



\bigskip
\noindent \textbf{Paper organization.} We describe low-density graphs in \secref{low:density} and prove some
basic properties.  Bounded expansion graphs are surveyed in
\secref{poly-expansion}.  \secref{approx:algorithms} present the new
approximation algorithms.  \secref{hardness} present the hardness
results. Conclusions are provided in \secref{conclusions}.


\section{Preliminaries} 


\subsection{Low-density graphs} \seclab{low:density}

\begin{defn}
  For a graph , and any subset
  , let  denote the
  \emphi{induced subgraph} of  over . Formally, we have
  
\end{defn}

\begin{defn}
  Consider a set of objects .  The \emphi{intersection graph}
  of , denoted by , is the graph having
   as its set of vertices, and an edge between two objects
   if they intersect.  Formally,
  
\end{defn}




One of the two main thrusts of this work is investigating the
following family of graphs.

\begin{defn}\deflab{low:density}A set of objects  in  (not necessarily convex or
  connected) has \emphi{density } if any object  (not
  necessarily in ) intersects at most  objects in
   with diameter greater than or equal to the diameter of
  . The minimum such quantity is denoted by
  .  If  is a constant, then 
  has \emphi{low density}.

  A graph that can be realized as the intersection graph of a set of
  objects  in  with density  is
  \emphi{-dense}.
\end{defn}


\begin{defn}\deflab{degenerate}A graph  is \emphi{-degenerate} if any subgraph of
   has a vertex of degree at most .
\end{defn}


\begin{observation}
  \obslab{low:density:g:sparse}A -dense graph  is
  -degenerate. Indeed, consider the set of objects
   that induces . Let  be the object with
  smallest diameter  in . By choice of , any other
  object intersecting  has diameter at least . Since at
  most  objects, in , with diameter at least 
  intersect  (including  itself), the degree of  in
   is . Clearly, this argument applies to any
  subgraph of .
\end{observation}


\subsubsection{Fatness and density}
For , an object  is
\emphi{-fat} if for any ball  with a center inside
, that does not contain  in its interior, we have

\cite{bksv-rimga-02}\footnote{There are several different, but roughly equivalent,
  definitions of fatness in the literature, see \si{de} Berg
  \cite{b-ibucf-08} and the followup work by Aronov \etal
  \cite{abes-ibulf-14} for some recent results. In particular, our
  definition here is what \si{de} Berg refers to as being
  \emph{locally fat}.}. A set  of objects is \emphi{fat} if
all its members are -fat for some constant .  A
collection of objects  has \emphi{depth}  if any point in
the underlying space lies in at most  objects of . The
depth index of a set of objects is a lower bound on the density of the
set, as a point can be viewed as a ball of radius zero. The following
is well known, and we include a proof for the sake of completeness.


\begin{lemma}
  A set  of -fat objects in  with depth 
  has density . In particular, if 
  and  are bounded constants, then  has low density.
\end{lemma}

\begin{wrapfigure}{r}{.16\textwidth}
  \vspace{-2em}
  \hfill\includegraphics[width=.16\textwidth]{figs/depth}
  \vspace{-2em}
\end{wrapfigure}
\noindent\textit{Proof:}
Let  be any ball in .  Let 
be the set of all objects in  that have diameter larger than
 and intersect it, and consider any object
.  A ball  centered at a point
 does not contain  in its interior
because .  By the
definition of -fatness, we have

Furthermore,  is contained in the ball
, and

By assumption, each point in  can be covered by at most 
objects of , hence

Thus, , bounding the number of
``large'' objects of  intersecting .\hfill\myqedsymbol


\begin{defn}
  A metric space  is a \emph{doubling space} if there is a
  universal constant  such that any ball  of radius
   can be covered by  balls of half the radius. Here 
  is the \emphi{doubling constant}, and its logarithm is the
  \emph{doubling dimension}.
\end{defn}

\begin{observation}
  \obslab{d:c:d}In  the doubling constant is , and the
  doubling dimension is  \cite{v-cbseb-05}, making the doubling
  dimension a natural abstraction of the notion of dimension in the
  Euclidean case.
\end{observation}

\begin{lemma}
  \lemlab{density:smaller:objects}Let  be a set of objects in  with density
  . Then, for any , a ball
   can intersect at most
   objects of  with
  diameter , where  is the  function in
  base two, and  is the doubling constant of .
\end{lemma}
\begin{proof}
  By the definition of the doubling constant, one can cover  by
   balls of radius .  As such, one can cover
   with  balls of radius
  . Each of these balls, by definition of density, can
  intersect at most  objects of  of diameter at
  least .
\end{proof}

The density definition can be made to be somewhat more flexible, as
follows.

\begin{lemma}
  Let  be a parameter, and let  be a collection of
  objects in  such that, for any , any ball with radius 
  intersects at most  objects with diameter
  . Then  has density
  .
\end{lemma}
\begin{proof}
  Let  be a ball with radius . We can cover  with
   balls with radius . Each
  -radius ball can intersect at most  objects
  with diameter larger than , so 
  intersects at most  objects
  with diameter at least .
\end{proof}


\subsubsection{Minors of objects}

\begin{defn}\deflab{t:shallow}A graph  is \emphi{-shallow} if there is a vertex
  , such that for any vertex
   there is a path  that connects 
  to , and  has at most  edges. The vertex  is a
  \emphi{center} of , denoted by .  The
  minimum integer  such that  is -shallow is the
  \emphi{radius} of .
\end{defn}

Let  and  be two sets of objects in . The
set  is a \emphi{minor} of  if it can be obtained
by deleting objects and replacing pairs of overlapping objects 
and  (i.e., ) with their union
.  Consider a sequence of unions and deletions
operations transforming  into .  Every object
 corresponds to a set of objects of
, such that
.  The set
 is a \emphi{cluster} of objects of .

Surprisingly, even for a set  of fat and convex shapes in the
plane with constant density, their intersection graph
 can have arbitrarily large cliques as minors (see
\figref{clique:minor}). Note that the clusters in
\figref{clique:minor} induce intersection graphs with large graph
radius.

\begin{defn}
  For sets of objects  and , if \smallskip \begin{compactenum}[\quad(i)]
  \item  is a minor of , and
  \item the intersection graph of each cluster of  (that
    corresponds to an object in ) is -shallow,
  \end{compactenum}\smallskip then  is a \emphi{-shallow minor} of .
\end{defn}

The following lemma shows that there is a simple relationship between
the depth of a shallow minor of objects and its density.

\begin{figure}[t]
  \centerline{\begin{tabular}{ccccc}
      \IncludeGraphics[page=1,
      width=.17\textwidth]{figs/clique_minor}&
        \IncludeGraphics[page=2,
        width=.17\textwidth]{figs/clique_minor}&
        \IncludeGraphics[page=3,
        width=.17\textwidth]{figs/clique_minor}&
        \IncludeGraphics[page=4,
        width=.17\textwidth]{figs/clique_minor}&
        \IncludeGraphics[page=5,
        width=.17\textwidth]{figs/clique_minor}\\
(A)&(B)&(C)&(D)&(E)
    \end{tabular}}

  \caption{\small (A) and (B) are two low-density collections of  disjoint
    horizontal slabs, whose intersection graph (C) contains  rows
    as minors. (D) is the intersection graph of a low-density
    collection of vertical slabs that contain  columns as
    minors. In (E), the intersection graph of all the slabs contain
    the  rows and  columns as minors that form a 
    bipartite graph, which in turn contains the clique  as a
    minor.  }
  \figlab{clique:minor}\vspace{-1em}
\end{figure}

\begin{lemma}\lemlab{density:shallow:minors}Let  be a collection of objects with density  in
  , and let  be a -shallow minor of
  . Then  has density at most
  .
\end{lemma}
\begin{proof}
  Every object  has an associated cluster
   such that
  , and these
  clusters are disjoint.  Let
   be the
  induced partition of  into clusters (which may be a
  partition of a subset of ). Consider any ball
  , and suppose that 
  intersects  and has diameter at least . Let
   be its cluster, and let
   be its associated intersection
  graph.  By assumption,  has (graph) radius at most , and
  diameter at most .

  Let  be any object in  that intersect
  , let  denote the shortest path metric
  of  (under the number of edges), and let  be the
  object in  closest to  (under
  ), such that . If
  there is no such object then the diameter of
  , which is a contradiction.

  Consider the shortest path 
  between  and  in , where
  .  By the choice of , we have ,
  for , and the distance between  and 
  is bounded by
  
  The object  is the \emph{representative} of , denoted
  by .

  Now, let
  
  be the representatives of the large objects in 
  intersecting .  The representatives in  are all
  distinct, have diameters , intersect ,
  and belong to  - a set with density .  Setting
  , \lemref{density:smaller:objects} implies that
  .  Since
   \cite{v-cbseb-05}, it follows that
  , implying the claim.
\end{proof}


\subsection{Graphs with polynomial expansion}
\seclab{poly-expansion} 

\subsubsection{Basic properties}

\begin{defn}
  \deflab{shallow:minor}Let  be an undirected graph. A \emphi{minor} of  is
  a graph  that can be obtained by contracting edges,
  deleting edges, and deleting vertices from . If  is
  a minor of , then each vertex  of  corresponds
  to a \emphi{cluster} -- a connected set  of vertices
  in  (i.e., these are the vertices of a tree in the forest
  formed by the contracted edges).  The graph  is a
  \emphi{-shallow minor} (or a \emphi{minor of depth }) of
  , where  is an integer, if for each vertex
  , the induced subgraph
   of the corresponding cluster
   is -shallow (see \defref{t:shallow}).
  Let  denote the set of all graphs that are
  minors\footnote{I.e., these graphs can not legally drink alcohol.}
  of  of depth .
\end{defn}

\begin{defn}[{\cite{no-gcbe1-08}}]
  \deflab{r:shallow:density}The \emph{greatest reduced average density of rank }, or just the
  \emphi{-shallow density}, of  is the quantity
   \end{defn}


\begin{defn}
  \deflab{expansion}A graph \emphi{class} is a (potentially infinite) set of graphs
  (e.g., the class of planar graphs).  The \emphi{expansion} of a
  graph class  is the function
  
  defined
  by
  
  The class  has \emphi{bounded expansion} if  is finite
  for all . Specifically, a class  with bounded expansion
  has \emphi{polynomial expansion} (resp., \emph{subexponential
    expansion} or \emph{constant expansion}) if  is bounded by a
  polynomial (resp., subexponential function or constant).  The
  polynomial expansion is of \emphi{order } if .
  Naturally, a graph  has \emph{polynomial expansion} of order
   if it belongs to a class of graphs with {polynomial expansion}
  of order .
\end{defn}

\begin{observation}
  \obslab{bounded:expansion:degenerate}If a graph  has bounded expansion, then  has average
  degree at most
  
  by taking the graph  as its own -shallow minor (with
  every vertex is its own cluster). In particular, the vertex 
  with minimum degree has degree at most . Similarly, any
  subgraph of  has a vertex  with degree at most ,
  so the graph , by virtue of its bounded expansion, is
  -degenerate (see \defref{degenerate}).
\end{observation}

As an example of a class of graphs with constant expansion, observe
that planar graphs have constant expansion because a minor of a planar
graph is planar, and by Euler's formula, every planar graph is
sparse. More surprisingly, \lemref{density:shallow:minors} together
with \obsref{low:density:g:sparse} implies that low-density graphs
have polynomial expansion.

\begin{lemma}
  Let  be fixed.  The class of -dense graphs
  in  has polynomial expansion bounded by
  .
\end{lemma}

\subsubsection{Separators}

\begin{defn}\deflab{separator}Let  be an undirected graph.  Two sets
   are \emphi{separate} in 
  if
  \begin{inparaenum}[(i)]
  \item  and  are disjoint, and
  \item there is no edge between the vertices of  and 
    in .
  \end{inparaenum}
  A set  is a \emphi{separator} for a set
  , if
  , and
   can be partitioned into two
  \emph{separate} sets  and , with
  
  and
  
  (Here the choice of  is arbitrary, and any constant smaller
  than  is sufficient.)
\end{defn}

\Nesetril and Ossona \si{de} Mendez showed that graphs with
subexponential expansion have subexponential-sized separators. For the
simpler case of polynomial expansion, we have the following\footnote{A
  proof is also provided in \cite{hq-naape-16-arxiv}.}.

\begin{theorem}[\expandafter{\cite[Theorem 8.3]{no-gcbe2-08}}]\thmlab{p:e:separator}Let  be a class of graphs with polynomial expansion of order
  . For any graph  with  vertices and 
  edges, one can compute, in 
  time, a separator of size
  
  where .\end{theorem}


\thmref{p:e:separator} yields a sublinear separator for low-density
graphs of size

Geometric arguments give a somewhat stronger separator. For the sake
of completeness, we provide next a proof of the following result, but
we emphasize that it is essentially already known
\cite{mttv-sspnng-97,sw-gsta-98,c-ptasp-03}. This proof is arguably
simpler and more elegant than previous proofs.

\begin{figure}
  \noindent \begin{minipage}[b]{0.29\linewidth}
    (A) \!\!\!\!\hspace{-0.3cm}\IncludeGraphics[page=1]{figs/separator}\hspace*{-3.11cm}\end{minipage}\quad \begin{minipage}[b]{0.33\linewidth}
    (B)\!\!\!\!\IncludeGraphics[page=2]{figs/separator}
    \hspace*{-3.11cm}\end{minipage}\qquad\begin{minipage}[b]{0.3\linewidth}
    \captionof{figure}{Illustration of the proof of
      \lemref{separator:low:density}.\0.2cm](B) All the objects intersecting  are in the
      separating set.}
  \end{minipage}
\end{figure}
\begin{lemma}
  \lemlab{separator:low:density}Let  be a set of  objects in  with density
   (see \defref{low:density}), and let  be
  some prespecified number.  Then, one can compute, in expected 
  time, a sphere  that intersects
   objects of
  . Furthermore, the number of objects of  strictly
  inside  is at least , and at most .  For
   this results in a balanced (global) separator. (Note that the  notation hides constants that depend on .)
\end{lemma}

\begin{proof}
  For every object , choose an arbitrary
  representative point . Let  be the
  resulting set of points. Let  be the smallest ball
  containing  points of . As in \cite{h-speps-13},
  randomly pick  uniformly in the range . We claim that
  the sphere  bounding the ball
   is the desired separator.

  To this end, consider the distance , where
   is some real number to be specified shortly.  We count
  the number of objects intersecting  as follows:

  \SaveIndent \medskip \begin{compactenum}[(A)]\RestoreIndent \item \textbf{Large objects (diameter ).} The sphere  can intersect only
     objects with diameter
    . Indeed, cover the sphere  with
     balls of radius , and let  be
    this set of balls. Next, charge each object of diameter larger
    than  intersecting  to the ball of  that
    intersects it.  Each ball of  get charged at most
     times.
\smallskip \item \textbf{Small objects (diameter ).} Let  be the set of objects of  with
    diameter  fully contained in
    , which contains all the small objects
    of  that intersects . The ball  can be
    covered by  balls of radius , where  is the
    doubling constant of  (see \obsref{d:c:d}), and each ball
    of radius  contains at most  representative points. Thus,
     contains at most  points of ,
    implying that .


    For an object , consider the closest
    point  and the furthest point  in  from
    .  The object  is in the separating set (and
    ``intersects'' ) if  separates  from
     ( may be disconnected).  As  is chosen uniformly
    at random from , we have that
    
    The expected number of objects of  intersecting
     is
    
  \end{compactenum}
  \smallskip We conclude that the separator size, in expectation, is
  
  Solving for  yields
  , and the resulting separator is (in
  expectation) of size
  .

  As for the running time, it is sufficient to find a two
  approximation to the smallest ball that contains  points of
  , and this can be done in linear time
  \cite{hr-nplta-13}. Using such an approximation slightly
  deteriorates the constants in the bounds. By Markov's inequality,
   intersects at most  objects of  with
  probability . If this is not true, we rerun the
  algorithm. In expectation, the algorithm succeeds in finding a
  sphere that intersects at most  objects of  within a
  constant number of iterations.
\end{proof}

\begin{remark}
  \obslab{improved:bound}\si{Mark de Berg} (personal communication) pointed out the current
  simplified proof of \lemref{separator:low:density}. The authors
  thank him for pointing out the simpler proof.
\end{remark}

It was recently shown that any graph with strongly sublinear
hereditary separators has polynomial expansion \cite{dn-ssspe-15}. In
conjunction with the preceding separator (for low-density objects),
this yields a second proof that the intersection graphs of low-density
objects have polynomial expansion, however with weaker bounds.

A weighted version of the above separator follows by a similar
argument.

\begin{lemma}
  \lemlab{w:s:low:density}Let  be a set of  objects in  with density
  , and weights . Let
   be the total
  weight of all objects in .  Then one can compute, in
  expected linear time, a sphere  that intersects
   objects of
  . Furthermore, the total weight of objects of 
  strictly inside/outside  is at most , where 
  is a constant that depends only on .
\end{lemma}

\begin{proof}
  The argument follows the one used in
  \lemref{separator:low:density}. We pick a representative point from
  each object, and assign it the weigh of the object. Next, we compute
  the smallest ball containing  of the total weight of
  the points, and the rest of the proof follows readily, observing
  that in the worst case,  objects might be involved in the
  calculations.
\end{proof}


\subsubsection{Divisions}


Consider a set . A \emphi{cover} of  is a set
 such
that . A set  is a
\emphi{cluster}. A cover of a graph  is a cover of its
vertices. Given a cover , the \emph{excess} of a vertex
 that appears in  clusters is . The
\emphi{total excess} of the cover  is the sum of excesses
over all vertices in .

\begin{defn}
  A cover  of  is a \emphi{-division} if
  \begin{inparaenum}[(i)]
  \item for any two clusters , the sets
     and  are separated in 
    (i.e., there is no edge between these sets of vertices in
    ), and
  \item for all clusters , we have
    .
  \end{inparaenum}

  A vertex  is an \emphi{interior vertex} of a cover
   if it appears in exactly one cluster of  (and
  its excess is zero), and a \emphi{boundary vertex} otherwise. By
  property (i), the entire neighborhood of an interior vertex of a
  division lies in the same cluster.
\end{defn}

\begin{remark:unnumbered}
  A division is not only a cover of the vertices, but also a cover of
  the edges. Consider a {-division}  of a graph
  , and an edge . We claim that there
  must be a cluster  in , such that both  and  are in
  . Indeed, if not, then there are two clusters  and ,
  such that  and , but then
   and  are not
  separated in , contradicting the definition.
\end{remark:unnumbered}


\begin{remark:unnumbered}
  The property of having -divisions is slightly stronger than
  being weakly hyperfinite.  Specifically, a graph is \emph{weakly
    hyperfinite} if there is a small subset of vertices whose removal
  leaves small connected components \cite[Section
  16.2]{no-s-12}. Clearly, -divisions also provide such a set
  (i.e., the boundary vertices). The connected components induced by
  removing the boundary vertices are not only small, but the
  neighborhoods of these components are small as well.
\end{remark:unnumbered}

As noted by Henzinger \etal \cite{hkrs-fspap-97}, strongly sublinear
separators obtain -divisions with total excess  for
.  Such divisions were first used by
Frederickson in planar graphs \cite{f-faspp-87}. A proof of the
following known result is provided in \cite{hq-naape-16-arxiv}.
\begin{lemma}
  \lemlab{divisions:small:excess}Let  be a graph with  vertices, such that any induced
  subgraph with  vertices has a separator with
   vertices, for some  and
  .  Then, for , the graph  has
  -divisions with total excess , where
  
  Furthermore, the -division can be computed in polynomial
  time.
\end{lemma}

\begin{remark}[Divisions for weaker separators]
  \remlab{w:div:small:excess}One can still obtain divisions for graph classes with weaker
  separators of size . Rather than a
  -division with excess , we get a
  -division with excess  for some function .
  Consequently, the \PTAS throughout this paper extend to a slightly
  broader class of graphs than polynomial expansion, see
  \remref{s:exp:expansion}.
\end{remark}

\begin{corollary}
  \corlab{dev;p:e:l:dense}\begin{inparaenum}[(A)]
  \item \itemlab{p:e:divisions} Let  be a graph with polynomial expansion of degree 
    and  vertices, and let  be fixed. Then  has
    -divisions with total
    excess .

    \smallskip \item \itemlab{l:d:divisions} Let  be a -dense graph with  vertices
    arising out of a given set of objects in . Then 
    has -divisions, with 
    and total excess at most . This division can be computed
    in  time.
  \end{inparaenum}
\end{corollary}

\begin{proof}
  (A) By \thmref{p:e:separator},  has separator with
  parameters  and .  Plugging
  this into \lemref{divisions:small:excess} implies
  -divisions where
  

  (B) By \lemref{separator:low:density}, any subgraph of  with
   vertices has a separator of size
  , for some
  constant . One can break up  in the natural recursive
  fashion using separators (see the proof of
  \lemref{divisions:small:excess} in \cite{hq-naape-16-arxiv} for
  details), until each portion has size , and
  
  where  is some absolute constant.  As can be easily verified,
  this holds for . Setting
   implies that the resulting -divisions with
  excess .

  As for the running time, computing the separator for a graph with
   vertices takes expected  time (assuming basic operation
  like deciding if an object intersects a sphere can be done in
  constant time), using the algorithm of
  \lemref{separator:low:density}, and the recursion depth is
  .
\end{proof}


\subsection{Hereditary and mergeable properties}
\seclab{h:m:prop}

Let  be a property defined over subsets
of vertices of a graph  (e.g.,  is
the set of all independent sets of vertices in ). The property
 is \emphi{hereditary} if for any
, if  satisfies
, then  satisfies .  The property  is
\emphi{mergeable} if for any  that
are separate in , if  and  each satisfy ,
then  satisfies . We assume that whether or
not  can be checked in polynomial time.

Given a set  and a property ,
the \emphi{packing problem} associated with , asks to find the
largest subset of  satisfying .

\begin{example}
  \exmlab{geometric:packing}Some geometric flavors of packing problems that corresponds to
  hereditary and mergeable properties include:
  \begin{compactenum}[\quad(A)]
  \item Given a collection of objects , find a maximum
    independent subset of .

  \item Given a collection of objects , find a maximum subset
    of  with density at most , where 
    is prespecified.

  \item Find a maximum subset of  whose intersection graph is
    planar or otherwise excludes a graph minor.

  \item Given a point set , a constant , and a collection
    of objects , find the maximum subset of  such
    that each point in  is contained in at most  objects
    in .
  \end{compactenum}
\end{example}




\section{Approximation algorithms}
\seclab{approx:algorithms}

\subsection{Approximation algorithms using separators}
\seclab{approx:v:separators}

Graphs whose induced subgraphs have sublinear and efficiently
computable separators are already strong enough to yield \PTAS for
mergeable and hereditary properties (see \secref{h:m:prop} for
relevant definitions). Such algorithms are relatively easy to derive,
and we describe them as a contrast to subsequent results, where such
an approach no longer works. As the following testifies, one can
-approximate, in polynomial time, the independent set in a
low-density or polynomial-expansion graphs (as independent set is a
mergeable and hereditary property).

\begin{lemma}
  \lemlab{ptas:h:prop}Let  be a graph with  vertices, with the following
  properties: \smallskip \begin{compactenum}[\quad(A)]
  \item Any induced subgraph of  on  vertices has a
    separator with  vertices, for some
    constants  and , and this separator can
    be computed in polynomial time. (I.e., low density and polynomial
    expansion graphs have such separators.)

  \item There is a hereditary and mergeable property  defined
    over subsets of vertices of .

  \item The largest set , is of size at least ,
    where  is some absolute constant.
  \end{compactenum}\smallskip Then, for any , one can compute, in
   time, a set
   such that
  , where
  
\end{lemma}

\begin{proof}
  Set . By \lemref{divisions:small:excess}, one can
  compute a -division for  in polynomial time, such
  that its total excess is
  
  where  is as stated above.  Throw away all the boundary
  vertices of this division, which discards at most
   vertices. The remaining clusters
  are separated from one another, and have size at most . For
  each cluster, we can find its largest subset with property 
  by brute force enumeration in 
  time per cluster.  Then we merge the sets computed for each cluster
  to get the overall solution. Clearly, the size of the merged set is
  at least .  The
  overall running time of the algorithm is
  .
\end{proof}

\begin{example}[Largest induced planar subgraph]
  Consider a graph  with  vertices and with polynomial
  expansion of order . Assume, that the task is to find the largest
  subset , such that the induced subgraph
   is, say, a planar graph.  Clearly, this property
  is hereditary and mergeable, and checking if a specific induced
  subgraph is planar can be done in linear time \cite{ht-ept-74}.

  By \obsref{bounded:expansion:degenerate}, the graph  is
  -degenerate, for some , since  has a polynomial
  expansion.  Consequently,  contains an independent set of
  size . This independent set is a valid
  induced planar subgraph of size .  Thus, the algorithm
  of \lemref{ptas:h:prop} applies, resulting in an
  -approximation to the largest induced planar subgraph. The
  running time of the resulting algorithm is
  , for some function .\end{example}

\begin{lemma}
  \lemlab{indep:easy}Let  be a parameter, and  be a given set of 
  objects in  that are -dense. Then one compute a
  -approximation to the largest independent set in
  . The running time of the algorithm is
  , where
  .

  More generally, one can compute, with the same running time, an
  -approximate solution for all the problems described in
  \exmref{geometric:packing}.
\end{lemma}
\begin{proof}
  Consider the intersection graph , and
  observe that by the low-density property, it always have a vertex of
  degree  (i.e., take the object in  with the
  smallest diameter). As such, removing this object and its neighbors
  from the graph, adding it to the independent set and repeating this
  process, results in an independent set in  of size
  . Thus implying that the largest independent set has
  size .  Now, apply the algorithm of \lemref{ptas:h:prop}
  to  using the improved -divisions of
  \corref{dev;p:e:l:dense} \itemref{l:d:divisions}.  Here, we need the
  total excess to be bounded by , which implies
  that
  

  For the second part, observe that all the problems mentioned in
  \exmref{geometric:packing} have solution bigger than the independent
  set of , and the same algorithm applies with minor
  modifications.
\end{proof}

\begin{remark:unnumbered}
  For computing the largest independent set, one does not need to
  assume the low density on the input -- a more elaborate algorithm
  works, see \lemref{indep:low:density:output} below.

  It is tempting to try and solve problems like dominating set on
  polynomial-expansion graphs using the algorithm of
  \lemref{indep:easy}.  However, note that a dominating set in such a
  graph (or even in a star graph) might be arbitrarily smaller than
  the size of the graph. Thus, having small divisions is not enough
  for such problems, and one needs some additional structure.
\end{remark:unnumbered}


\subsection{Local search for independent set and packing problems}

Chan and Har-Peled \cite{ch-aamis-12} gave a \PTAS for independent set
with planar graphs, and the algorithm and its underlying argument
extends to hereditary graph classes with strongly sublinear separators
(see also the work by Mustafa and Ray \cite{mr-irghs-10}).


\subsubsection{Definitions}

Let  be a hereditary and mergeable property, and let 
be a fixed integer.  For two sets,  and , their
\emph{symmetric difference} is

Two vertex sets  and  are \emphi{-close} if
; that is, if one can
transform  into  by adding and removing at most
 vertices from . A vertex set  is
\emphi{-locally optimal} in  if there is no
 that is -close to  and ``improves''
upon . In a maximization problem  \emphi{improves}
  . In a minimization
problem, an improvement decreases the cardinality.


\subsubsection{The local search algorithm in detail}
\seclab{l:s:alg}


The \emphi{-local search algorithm} starts with an
arbitrary (and potentially empty) solution  and, by
examining all -close sets, repeatedly makes -close
improvements until terminating at a -locally optimal
solution. Each improvement in a maximization (resp., minimization)
problem increases (resp., decreases) the cardinality of the set, so
there are at most  rounds of improvements, where  is the size of
the ground set of . Within a round we can exhaustively try all
exchanges in time , bounding the total running time by
.

\subsubsection{Analysis of the algorithm}

\begin{theorem}
  \thmlab{independent}Let  be a given graph with  vertices, and let 
  be a hereditary and mergeable property defined over the vertices of
   that can be tested in polynomial time, Furthermore, let
   and  be parameters, and assume that for any two
  sets , such that
  , we have that 
  has a -division with total excess
  .  Then, the -local search
  algorithm computes, in  time, a
  -approximation for the maximum size set
   satisfying .
\end{theorem}

\begin{proof}
  Let  be an optimal maximum set satisfying
  , and  be a -locally maximal set satisfying
  . Consider the induced subgraph
  , and observe that, by
  assumption, there exists a -division
   of , with
  boundary vertices  and
  
  For , let

  \centerline{\begin{tabular}{lll}
      ,
      &&
         ,\\
,&&
         ,\\
      , &
        and
      &
        .
    \end{tabular}}\medskip Fix , and consider the set
  . Since
  , and  is
  -close to .  Since  is hereditary,
  , and since  and
   are separated, their union 
  is in . Thus, the local search algorithm considers the valid
  exchange from  to . As the set  is
  -locally optimal, the exchange replacing  by
   can not increase the overall cardinality.  Since
  
  this implies that . Summing over all , we
  have
  
  as desired.
\end{proof}


\begin{remark}
  It is illuminating to consider the requirements to make the argument
  of \thmref{independent} go through. We need to be able to break up
  the conflict graph between the local and optimal solutions into
  small clusters, such that the total number of boundary vertices
  (counted with repetition) is small. Surprisingly, even if all (or
  most of) the vertices of a single cluster are boundary vertices, the
  argument still goes through.
\end{remark}

\begin{lemma}
  \lemlab{indep:low:density:output}Let  and  be parameters, and let  be a
  given collection of objects in  such that any independent set
  in  has density . Then the local search
  algorithm computes a -approximation for the maximum size
  independent subset of  in time
  .
\end{lemma}

\begin{proof}
  Observing that the union of two -dense sets results in a
  -dense set, and using the algorithm of
  \thmref{independent}, together with the divisions of
  \corref{dev;p:e:l:dense} \itemref{l:d:divisions}, implies the
  result.
\end{proof}

\begin{remark:unnumbered}
  (A) We emphasize that \lemref{indep:low:density:output} requires
  only that independent sets of the input objects  have low
  density -- the overall set  might have arbitrarily large
  density.

  (B) All the problems of \exmref{geometric:packing} have a \PTAS
  using the \lemref{indep:low:density:output} as long as the output
  has low density.
\end{remark:unnumbered}



\subsection{Dominating Set} \seclab{dom:set}

We are interested in approximation algorithm for the following
generalization of the dominating set problem.

\begin{defn}
  Let  be an undirected graph, and let  and
   be two subsets of .  The set 
  \emph{dominates}  if every vertex in  either is in
   or is adjacent to some vertex in . In the
  \emphi{dominating subset problem}, one is given an undirected graph
   and two subsets of vertices  and ,
  such that  dominates . The task is to compute the
  smallest subset of  that still dominates .
\end{defn}

One can approximate the dominated set, as the reader might expect, via
a local search algorithm.  Before analyzing the algorithm, we need to
develop some tools to be able to argue about the interaction between
the local and optimal solution.

\subsubsection{Shallow packings}

\begin{defn}
  \deflab{shallow:packing}Given a graph , a collection of sets
   is a
  \emphi{-shallow packing} of , or just a
  \emphi{-packing}, if for all , the induced graph
   is -shallow (see \defref{t:shallow}), and
  every vertex of  appears in at most  sets of
  \footnote{We allow a set  to appear in  more
    than once; that is,  is a multiset.}.

  The \emphi{induced packing graph}  has
   as the set of vertices, and two clusters
   are connected by an edge if they
  share a vertex (i.e., ), or
  there are vertices  and , such that
  .
\end{defn}

For example, the induced packing graph of a -packing is the
-shallow minor induced by the clusters of the packing (see
\defref{shallow:minor}).




\begin{lemma}
  \lemlab{edge-density-shallow-cover}Let  be an undirected graph, and  an
  -packing of . Then the induced packing graph
   has edge density
  
  where  is the -shallow density of ,
  see \defref{r:shallow:density}.
\end{lemma}
\begin{proof}
  Let the clusters of  be
  . For each cluster
  , designate a center vertex
   that can reach any other vertex in
   by a path contained in  of length 
  or less.  Let  be a random
  permutation of the cluster indices, chosen uniformly at random, and
  initialize , where
  .  For , in order,
  check if  has been ``scooped''; that is, if
  
  and if so, ignore it. Otherwise, let  be the set of
  vertices of the connected component of  in the induced
  subgraph of  over
  
  and add  to .  Intuitively, the set  is a
  -packing of  resulting from randomly shrinking the
  clusters of .


  \smallskip We bound the number of edges in
   by a function of the
  expected number of edges in the random graph
  .  Let
  
  be the set of edges between pairs of clusters where the center of
  one cluster is also in the other cluster. Since a center 
  can be covered at most  times by , we have
  .  Next, consider
  the set of remaining edges,
  
  between adjacent clusters where neither center lies in the opposing
  cluster. For an edge , consider
  the probability that .

  Since  and  are adjacent in ,
  there is a path  in  from  to  of
  length at most  that is contained in
  , and a sufficient condition for
   is that  is contained in
  . This holds if the permutation 
  ranks  and  ahead of any other index  such that
   intersects the vertices of . There are at most
   vertices on , where each vertex can appear in at most
   clusters of , and overall there are at most
  
  clusters that compete for control over the vertices of  in
  . The probability that, among these relevant clusters, the
  random permutation  ranks  and  before all others is
  
  Therefore, for , we have
  
  By linearity of expectation, and since
   is a -shallow minor of
  , we have
  
  We conclude that
  
  as desired.
\end{proof}

\begin{remark:unnumbered}
  Results similar to \lemref{edge-density-shallow-cover} are already
  known \cite{no-s-12}. However, our result has a polynomial
  dependency on  and , while the known results seems to
  imply an exponential dependency.
\end{remark:unnumbered}



\begin{lemma}
  \lemlab{expansion:shallow:cover} Consider a graph , and an -packing  of
  . Then, for any integer , we have
  
  In particular, if  and  are constants, and  has
  polynomial of order , then  has polynomial expansion of
  order .
\end{lemma}

\begin{proof}
  For , consider a -shallow minor  of
  . Every cluster in this cover corresponds to an expanded
  cluster in the original graph  with radius ,
  and a vertex might participate in  such clusters. That is,
  the resulting set  of clusters is an
  -packing of .  By
  \lemref{edge-density-shallow-cover}, we have
  
  By \defref{r:shallow:density}, we have
  
\end{proof}


\subsubsection{Lexical product and shallow density}

An interesting consequence of the above is an improvement over known
bounds for the shallow density under lexical product (this result is
not required for the rest of the paper).  Given two graphs 
and , the \emphi{lexical product} 
is the graph obtained by blowing up each vertex in  with a
copy of . More formally,  has vertex
set  and an edge
between two vertices  and  if either (a)
, or (b)  and
.
\begin{corollary}
  \corlab{grad-lexical-product}For any graph , clique , and
  , we have
  
  In particular, if  is constant and  has polynomial
  expansion of order , then  has
  polynomial expansion of order .
\end{corollary}
\begin{proof}
  A -shallow minor of  is the induced
  packing graph of the -packing formed by its
  clusters. Thus, the claimed inequality follows from
  \lemref{edge-density-shallow-cover}.
\end{proof}

\corref{grad-lexical-product} is an exponential improvement over the
best previously known bounds, on the order of

by \Nesetril and Ossona \si{de} Mendez \cite{no-gcbe1-08} (see also
the comments following the proof of Proposition 4.6 in
\cite{no-s-12}).



\subsubsection{Low density objects and -packing{}s}
\seclab{l:d:o:packing}

\begin{defn}
  For a set of objects , a collection of subsets
  
  forms a \emphi{-shallow packing} of  if, for all
  , the intersection graph  is -shallow
  (see \defref{t:shallow}), and every object of  appears in
  at most  sets of .  The \emphi{induced object set}
   is the collection of objects
  
  formed by taking the union of each cluster in .
\end{defn}

\begin{lemma}\lemlab{shallow:cover:objects}Let  be a collection of objects with density  in
  , and let  be an -shallow packing. Then
  the induced object set  has density
  .
\end{lemma}

\begin{proof}
  Consider the multiset of objects
   where each
  object  is repeated according to its multiplicity in
  . Since each object in  appears in  at
  most  times,  has density .  Every
  cluster  can be interpreted as a new cluster
   of objects of , where the resulting set of
  clusters  are
  now disjoint.

  As such,  is a -shallow minor of
  . By \lemref{density:shallow:minors}, the graph
   has density
  .
\end{proof}

\subsubsection{The result}

Shallow packings arise in the analysis of the approximation algorithm
for dominating set, where vertices are clustered together around the
the vertex that dominates them. In this setting, we prefer the
following simple and convenient terminology.

\begin{defn}
  \deflab{flower:head}Given a dominating set  of
  vertices in a graph , and a set of vertices
   being dominated by , we
  generate a sequence of clusters
   that
  specifies for every element of , which elements it covers.

  Initially, we set  and .
  In the \th iteration, for , let
  
  where  is the set of vertices adjacent to  in
  .  Conceptually,  induces a star-like graph
   over , where every vertex of  is
  connected to . The cluster  (and implicitly to
  ) is a \emphi{flower}, where  is its \emphi{head}.
  The collection of clusters
  
  is the \emphi{flower decomposition} of the given instance.  Note
  that a flower is a -shallow graph, and a flower decomposition is
  a -shallow packing.
\end{defn}

\begin{theorem} \thmlab{ptas:subset:dom} Let  be a graph with  vertices and with polynomial
  expansion of order , let  be
  two sets of vertices such that  dominates , and
  let  be fixed. Then, for
  
  the -local search algorithm computes, in
   time, a -approximation for the
  smallest cardinality subset of  that dominates .
\end{theorem}
\begin{proof}
  The algorithm starts with the whole collection  as the
  local solution, and performs legal local exchanges
  of size  that decrease the size of the local solution by at
  least one until no such exchange is available (see
  \secref{l:s:alg}).

  Let  and  be the
  optimal and locally minimal sets dominating , respectively.
  Let  and
   be the corresponding flower
  decompositions. In the following, for vertices  and
  , we use  and  to
  denote their flower in these decompositions, respectively.

  Let  be the
  induced packing graph of .  The
  set  is a -shallow cover of , and
  \lemref{expansion:shallow:cover} implies that  has
  polynomial expansion of order .  By \corref{dev;p:e:l:dense}
  \itemref{p:e:divisions},  has
  -division
  
  with a set of boundary vertices , and total excess
  
  For , let \smallskip
  \begin{compactenum}[\qquad(i)]
  \item
    

  \item
    , and 

  \item .
  \end{compactenum}

  \smallskip Fix , and consider the set
  .  If a vertex
   is not dominated by ,
  then 
  for some , and
   for
  some  with  adjacent to
   in . The cluster  is an
  interior vertex of , so  must be in the
  cluster , and . Therefore, the
  alternative solution  dominates , and overall,
   dominates .

  Since  is -locally minimal, and the exchange size
  is
  
  the new solution  is at least as large as
  . Expanding
  
  we have . Summed over all
  the clusters , we have,
  
  Solving for , we conclude that
  
  as desired.
\end{proof}


\subsubsection{Extensions -- multi-cover and reach}


One can naturally extend dominating set in the following ways:
\smallskip \begin{compactenum}[(A)]
\item \textbf{Demands}: For every , there is an integer
  , which is the \emphi{demand} of ; that is,
   has to be adjacent to at least  vertices in the
  dominating set. In the context of set cover, this is known as the
  multi-cover problem, see \cite{cch-smcpg-12}. Let
   be the
  \emphi{demand} of the given instance.

\item \textbf{Reach}: Instead of the dominating set being adjacent to
  the vertices that are being covered, for every vertex
   one can associate a distance  --
  which is the maximum number of hops the dominating vertex can be
  away from  in the given graph. The \emphi{reach} of the given
  instance is
  
\end{compactenum}
\smallskip Thus, a vertex  with demand  and reach ,
requires that any dominating set would have  vertices in
edge distance at most  from it.

\begin{lemma} \lemlab{ptas:subset:dom:2} Let  be a graph with  vertices and with polynomial
  expansion of order , sets  and
  , such that  dominates
  , and let  be fixed. Furthermore, assume that for
  each vertex , there are associated demand and reach,
  where the reach  and demand  of the given
  instances are bounded by a constant.

  Then, for , the
  -local search algorithm computes, in 
  time, a -approximation for the smallest cardinality
  subset of  that dominates  under the reach and
  demand constraints.
\end{lemma}

\begin{proof}
  Let  be an arbitrary ordering on the vertices of .
  For a set of vertices  and a vertex
  , let  be the  closest
  vertices to  in , with respect to the length of the
  shortest path in , and resolving ties by . The
  ordering  ensures that  is uniquely
  defined for any vertex in the graph.

  In the following argument, fix a set  that
  dominates  and complies with the given constraints, and
  assign every vertex of  to each of the vertices of
  .  For a vertex , let
   be the set of vertices assigned to it.  For each vertex
  , let  be the minimal subtree of the {BFS}
  tree rooted at  that includes all the vertices of
  .  The \emph{flower}
   is -shallow in
  .  Let
  
  be the resulting \emph{flower decomposition} of .

  We claim that a vertex  of  is covered at most
   times by the flowers of . More precisely, we
  prove that  is covered by a flower  only if
  . For the sake of contradiction,
  suppose  and that
  . Then  is not assigned to
  , so there must be a vertex  assigned to  and an associated
  shortest path
  
  from  to  through , where  is the subpath from
   to  and  is the subpath from  to . Since
  , and both sets
   and  have
  the same cardinality, there exists another vertex
  .  Let  be the shortest-path
  from  to . By construction of ,
  either
  
  or
  
  and . This implies that either
  
  or  and
  
  where  denotes concatenation of paths.  In any case, if ties are
  broken by , then  is closer to  than  is, a
  contradiction to the premise that 
  and . Thus, if  is in a
  flower , then .

  Now, consider the local solution  and the optimal solution
  .  Let  and
   be the flower decompositions
  of the local and optimal solutions, respectively. Each flower
  decomposition includes an element at most  times, so the
  combined collection  is a
  -shallow packing. By
  \lemref{expansion:shallow:cover}, the induced packing graph
   has polynomial expansion of
  order .  We now follow the argument used in the proof of
  \thmref{ptas:subset:dom}, providing the details for the sake of
  completeness.

  Let
  
  There is a -division of  into clusters
  , with
   boundary vertices and total excess
  .  For
  , let \smallskip
  \begin{compactenum}[\qquad(i)]
  \item ,

  \item
    ,
    and 

  \item .
  \end{compactenum}Fix , and consider the cover
  .
  Consider a vertex  such that there is a flower in
   that covers it (i.e., the vertex
  ``lost'' coverage in this potential exchange).  This implies that
   must be covered by a flower ; that is,
  by a flower that corresponds to a vertex of  that is
  internal to .  Any flower  that
  covers  is adjacent to  in , by the definition
  of  and as  and  share a vertex.  As
   is internal to , all the flowers of 
  that cover  are in , and in particular, all the
  flowers covering  in the optimal solution belong to
  . Thus, the coverage provided by  meets the
  demand and reach requirements of . The rest of the argument now
  follows the proof of \thmref{ptas:subset:dom}.
\end{proof}


\subsubsection{Extension: Connected dominating set}
\seclab{connected:dominating:set}

The algorithms of \thmref{ptas:subset:dom} and
\lemref{ptas:subset:dom:2} can be extended to handle the additional
constraint that the computed dominating set is also connected.  In
this setting, the local search algorithm only considers beneficial
exchanges that result in a connected dominating set.



\begin{lemma}
  \lemlab{ptas:subset:dom:3}Let  be a graph with  vertices and polynomial
  expansion of order , and let  be a
  connected dominating set. For each vertex , let
   be its associated demands, and let
   be bounded by a
  constant (here, the dominating set has to dominate all the vertices
  in the graph). Then, for
  , the
  -local search algorithm computes, in 
  time, a -approximation for the smallest cardinality subset
  of  that is \emph{connected} and dominates 
  under the demand constraints.
\end{lemma}


\begin{proof}
  We extend the notations and argument used in
  \thmref{ptas:subset:dom}. To recap, let  and
   be the optimal and locally minimum sets
  dominating , respectively.  Let
   and
   be the corresponding flower
  decompositions (see \defref{flower:head}). In the following, for
  vertices  and , we use
   and  to denote their flower in these
  decompositions, respectively.

  Let  be the
  induced packing graph of . As
  before, we can apply \corref{dev;p:e:l:dense}
  \itemref{p:e:divisions} to  to generate a
  -division
  
  with a set of boundary vertices , and total excess
  
  For , let \smallskip
  \begin{compactenum}[\qquad(i)]
  \item
    

  \item
    , and 

  \item .
  \end{compactenum}Fix , and consider the set
  . By the exact
  same argument as \thmref{ptas:subset:dom},  is a
  dominating set. However,  may not necessarily be
  connected.

  Let  be the set of head vertices
  of the boundary flowers of the \th cluster.  Because the removed
  patch  is only connected to the rest of  via the
  boundary vertices , each component of 
  contains at least one boundary vertex in
  . Similarly, each component of
   contains at least one boundary vertex in
  . Together, every component of  contains at
  least one vertex in , so  has at most
   components.

  Consider the shortest path  within  between any
  two vertices  that are in separate components of
  . By minimality of , the interior vertices of
   are not in .  If  has more than 4
  vertices, then there exists an intermediate vertex 
  that is adjacent to neither  nor .  Write
  , where  is the subpath
  from  to  and  is the subpath from  to . Both
  subpaths  and  contain at least two
  edges. Since ,  is adjacent to some vertex
  . Since  and  lie in different in components,
   lies in a different component from either  or .  If 
  and  lie in different components, then the path consisting of
   followed by the edge from  to  is a shorter path
  than , a contradiction. A similar contradiction arises if
   and  lies in different components. It follows, by
  contradiction, that  has at most 4 vertices, all of which
  lie in .  By adding the entire path  to
  , we can connect these two components by adding at most
   vertices from .


  By repeatedly connecting the closest pair of components of
   like that, we can augment  to a connected
  dominating set  while adding at most
   vertices. If we expand our
  search size to , then  is a
  connected dominating set with
  , and the local
  optimality of  implies that
  
  As in the previous proofs, summing this inequality over all 
  implies the claim.
\end{proof}

\lemref{ptas:subset:dom:3} extends to constantly bounded reach with an
added assumption.
\begin{lemma}
  \lemlab{ptas:subset:dom:4} Let  be a graph with  vertices and with polynomial
  expansion of order , and let  be a
  given set. Assume that
  \begin{compactenum}[\quad(i)]
  \item for each vertex , there are associated demand
     and reach  constraints,
\item  and
    ,
\item the set  is a valid dominating set complying with the
    demand and reach constraints,
\item for any two vertices , the shortest path (in
    the number of edges) in  between  and  is contained
    in .
  \end{compactenum}
  Then, for , the
  -local search algorithm computes, in 
  time, a -approximation for the smallest cardinality
  subset of  that is \emph{connected} and dominates
   under the reach and demand constraints.
\end{lemma}

\begin{proof}
  The same proof as that of \lemref{ptas:subset:dom:3} goes through,
  except now the shortest paths between distinct components can be
  shown to have length at most  vertices. Condition
  (iv) is necessary to keep these paths lying in . The search size
  is increased by a factor of  instead of 2, which is
  only a constant factor difference.
\end{proof}


\subsubsection{Discussion}


\begin{observation}[\PTAS for vertex cover for polynomial expansion
  graphs]
  \obslab{v:c:poly:expansion}The algorithm of \thmref{g:hitting:set:cover} can be used to get a
  \PTAS for vertex cover. Indeed, let  be an undirected
  graph with polynomial expansion. We introduce a new vertex in the
  middle of every edge of , and let  be the resulting
  graph, with  be the set of new vertices. Clearly, replacing
  an edge by a path of length two changes the expansion of a graph
  only slightly, see \defref{expansion}, and in particular, 
  has polynomial expansion. Now, solving the dominating subset for
   as the set required covering, and  as the
  initial dominating set, in the graph  solves the original
  vertex cover problem in the original graph. The desired \PTAS now
  follows from \thmref{g:hitting:set:cover}.
\end{observation}

\begin{remark}[\PTAS for graphs with subexponential expansion] \remlab{s:exp:expansion}As noted in \remref{w:div:small:excess}, one can still obtain
  -divisions for some (larger) function  in graphs with
  hereditary separators size . To this end,
  one can verify (by the same proof as \thmref{p:e:separator}, see
  \cite{no-gcbe2-08,hq-naape-16-arxiv}) that for a small constant ,
  if a graph class  has expansion
  , for  and 
  sufficiently small constants, then  has separators of the
  desired size . Thus, the above approximation
  algorithms yield a \PTAS (with much worse dependence on )
  for any graph class  with subexponential expansion
  , where  and
   are some constants. We are not aware of any natural graphs in
  this class that do not have polynomial expansion.
\end{remark}





\subsection{Geometric applications}
\seclab{geometric:applications}

The above implies \PTAS's for dominating set type problems on
low-density graphs.  Let  be a collection of objects in
 and  a collection of points.  Two natural geometric
optimization problems in this setting are:
\begin{compactenum}[\;\;(A)]
\item \emphi{Discrete hitting set}: Compute the minimum cardinality
  set  such that for every
  , we have .
  That is, every object of  is stabbed by some point of
  .

  If we consider the natural intersection graph
   and the sets
   and , then this is an
  instance of dominating subset problem. The algorithm of
  \thmref{ptas:subset:dom} applies because  is low density and
  therefore has polynomial expansion.


\item \emphi{Discrete set cover}: Compute the smallest cardinality set
   such that for every point
  , we have
  
  That is, all the points of  are covered by objects in
  . Setting  and 
  (i.e., flipping the sets in the hitting set case), and arguing as
  above, implies a \PTAS.
\end{compactenum}
\smallskip For these geometric optimization problems, we can improve the running
time of \thmref{ptas:subset:dom} by applying the stronger separator
theorem for low-density graphs.
\begin{theorem}
  \thmlab{g:hitting:set:cover}Let  be a collection of  objects in  with density
  ,  be a set of  points in , and let
   be a parameter. Then, for
  , the local search algorithm,
  with exchanges of size  implies the following:
  \begin{compactenum}[\quad(A)]
  \item An approximation algorithm that, in 
    time, computes a set  that is an
    -approximation for the smallest cardinality set that
    hits .

  \item An approximation algorithm that, in 
    time, computes a set  that is an
    -approximation for the smallest cardinality set that
    covers .
  \end{compactenum}
\end{theorem}

\begin{proof}
  Since points have zero diameter, the union 
  also has density . This reduces geometric hitting set
  and discrete geometric set cover to dominating subset problem on the
  intersection graph of .

  The approximation algorithm is described in \thmref{ptas:subset:dom}
  (applied to ). Here we can do slightly better, using smaller
  exchange size, as the graph  has low density. To this end,
  observe that the analysis of \thmref{ptas:subset:dom} argues about
  the induced packing graph of  for some -shallow
  packing . By \lemref{shallow:cover:objects}, the graph
   has density
  .  Thus, by
  \corref{dev;p:e:l:dense} (B),  has a -division
  with excess , where
  .  The algorithm of
  \thmref{ptas:subset:dom} modified to use these improved divisions
  implies the result.
\end{proof}

\begin{remark}
  To our knowledge, the algorithms of \thmref{g:hitting:set:cover} are
  the first \PTAS's for discrete hitting set and discrete set cover
  with shallow fat triangles and similar fat objects. Previously, such
  algorithms were known only for disks and points in the plane.
\end{remark}






\section{Hardness of approximation}
\seclab{hardness}

Some of the results of this section appeared in an unpublished
manuscript \cite{h-bffne-09}.  Chan and Grant \cite{cg-eaahr-14} also
prove some related hardness results, which were (to some extent) a
followup work to the aforementioned manuscript.

\subsection{A review of complexity terms} \seclab{complexity}


The \emphi{exponential time hypothesis} (\emphi{\ETH})
\cite{ip-ocks-01, ipz-wphse-01} is that \TrSAT can not be solved in
time better than , where  is the number of
variables. The \emphi{strong exponential time hypothesis}
(\emphi{\SETH}), is that the time to solve \kSAT is at least
, where  converges to  as  increases.

A problem that is \APXHard does not have a \PTAS unless
.  For example, it is known that \ProblemC{Vertex Cover}
is \APXHard even for a graph with maximum degree 
\cite{acgkm-ca-99}.  Thus, showing that a problem is \APXHard implies
that one can not do better than a constant
approximation. Specifically, if one can get a -approximation
for such a problem, for any constant , then one can
-approximate \TrSAT (for the max version of \TrSAT, the
purpose is to maximize the number of clauses satisfied). By the
\Term{PCP} Theorem, this would imply an exact algorithm for \TrSAT.


\begin{observation}\obslab{eth:no:a}Consider an instance of \TrSAT of size , for some
  constant  sufficiently large. \ETH implies that we cannot
  solve this instance in polynomial time, since the running time
  required to solve this instance is , which is super
  polynomial. (This argument works for any function  such that
  .)

  This innocuous observation has a surprising implication -- we cannot
  even  approximate a solution for such an instance by the
  \Term{PCP} result. Namely, \ETH implies that even polylogarithmic
  sized instances cannot be solved in polynomial time.
\end{observation}

\begin{observation}
  Showing that a problem  is \APXHard implies that:
  \begin{compactenum}[(A)]
  \item The problem  does not have a \PTAS (unless ).

  \item Under \ETH, the problem  does not have a \QPTAS, where a
    \QPTAS is an -approximation algorithm with running time
    .

  \item Furthermore, under \ETH, polylogarithmic sized instances of
     cannot be approximated to within a -multiplicative
    factor in polynomial time.
  \end{compactenum}
\end{observation}


\subsection{Discrete hitting set for fat triangles}

In the \emphi{fat-triangles discrete hitting set problem}, we are
given a set of points in the plane  and a set of fat
triangles , and want to find the smallest subset of 
such that each triangle in  contains at least one point in
the set.
\begin{figure}
  \centerline{\begin{tabular}{c|c|c|c}
      \IncludeGraphics[page=1,width=.21\textwidth]{figs/triangles}&\quad\IncludeGraphics[page=4,width=.21\textwidth]{figs/triangles}&
        \quad\IncludeGraphics[page=2,width=.21\textwidth]{figs/triangles}&\quad \IncludeGraphics[page=3,width=.21\textwidth]{figs/triangles}\\
      (A) & (B) & (C) & (D)
    \end{tabular}}\caption{Illustration of the proof of \lemref{no:PTAS:fat:hit:set}: (A) A -regular graph with its  coloring. (B) Placing the vertices on a circle. (C) Three edges and their associated triangles. (D) All the triangles.
}\figlab{illustration}
\end{figure}
\begin{lemma}
  \lemlab{no:PTAS:fat:hit:set}There is no \PTAS for the fat-triangle discrete hitting set problem,
  unless .
One can prespecify an arbitrary constant , and the claim
  would hold true even if the following conditions hold on the given
  instance :
\begin{compactenum}[\quad(A)]
  \item Every angle of every triangle in  is between
     and  degrees.

  \item No point of  is covered by more than three triangles
    of .

  \item The points of  are in convex position.

  \item All the triangles of  are of similar
    size. Specifically, each triangle has side length in the range
    (say) .

  \item The points of  are a subset of the vertices of the
    triangles of .
  \end{compactenum}
\end{lemma}


\begin{proof}
  Let  be a connected instance of
  \ProblemC{Vertex Cover} which has maximum degree three, and it is
  not an odd cycle. We remind the reader that \ProblemC{Vertex Cover}
  is \APXHard for such instances \cite{acgkm-ca-99}.

  By Brook's theorem \cite{cr-btb-15}\footnote{Brook's theorem states
    that any connected undirected graph  with maximum degree
    , the chromatic number of  is at most 
    unless  is a complete graph or an odd cycle, in which case
    the chromatic number is .}, this graph is three
  colorable, and let  be the
  partition of  by their colors. Let
   be three points on the unit circle that
  form a regular triangle. For , place a circular interval
   centered at  of length . Now, for
  , we place the vertices of  as distinct points
  in .

  Let  and .
  For , let  be the \th edge of
  . Assume, for the sake of simplicity of exposition, that
   and .  Pick an arbitrary
  point  in , and
  create the triangle . Set
  , and continue to
  the next edge.

  At the end of this process, we have  triangles
   that are arbitrarily close to
  being regular triangles, and all their edges are arbitrarily close
  to being of the same length, see \figref{illustration}. It is easy
  to verify that a minimum cardinality set of points
   that hits all the triangles in  is
  a minimum vertex cover of .
\end{proof}


\subsection{Friendly geometric set cover}

Let  be a set of  points in the plane, and  be a
set of  regions in the plane, such that
\begin{compactenum}[\qquad(I)]
\item the shapes of  are convex, fat, and of similar size,
\item the boundaries of any pair of shapes of  intersect in
  at most  points,
\item the union complexity of any  shapes of  is ,
  and
\item any point of  is covered by a constant number of shapes
  of .  \end{compactenum}\smallskip We are interested in finding the minimum number of shapes of 
that covers all the points of .  This variant is the
\emphi{friendly geometric set cover} problem.

\begin{lemma}
  \lemlab{no:PTAS:friendly:s:c}There is no \PTAS for the friendly geometric set cover problem,
  unless .
\end{lemma}

\begin{proof}
  Let  be a set of  elements, and  a set of subsets of
   each containing at most  elements of . In the
  \emphi{minimum -set cover} problem, we want to find the smallest
  subcollection  that covers . The
  problem is \MaxSNPHard for , meaning there is no \PTAS
  unless  \cite{acgkm-ca-99}.

  We will reduce an instance  of the minimum -set
  cover problem (for ) into an instance of the friendly geometric
  set cover problem.

  \begin{figure}
    \centerline{\begin{tabular}{ccc}
        \begin{minipage}[b]{0.3\linewidth}
          \centerline{\IncGraphPage [width=0.99\linewidth]{figs}{gear}{1}}
        \end{minipage}&\begin{minipage}[b]{0.3\linewidth}
            \centerline{\IncGraphPage [width=0.99\linewidth]{figs}{gear}{2}}
          \end{minipage}&\begin{minipage}[b]{0.3\linewidth}
            \centerline{\IncGraphPage[width=0.99\linewidth]{figs}{gear}{3}}
          \end{minipage}
        \\
        \smallskip (i) & (ii) & (iii)
      \end{tabular}}

    \caption{(i) A region  constructed for the set
      . Observe that in the
      construction, the inner disk is even bigger. As such, no
      two points are connected by an edge of the convex-hull when
      we add in the inner disk to the convex-hull. As such, each
      point ``contribution'' to the region  is separated
      from the contribution of other points. (ii) How  two such
      regions together.
(iii) Their intersection.  }
\figlab{intersection:of:regions}
  \end{figure}

  Let  be a set of  elements, and
   a collection of  subsets of
  . We place  points equally spaced on the unit radius circle
  centered at the origin, and let
   be the resulting set of
  points. For each point , let .  For each
  set  (of size at most 3), we define the region
  
  where  is the convex hull,
  , and  denotes
  the disk of radius  centered at the origin. Visually, 
  is a disk with three (since ) teeth coming out of it, see
  \figref{intersection:of:regions}. Note that the boundary of two such
  shapes intersects in at most  points.

  It is now easy to verify that the resulting instance of geometric
  set cover  is
  friendly, and clearly any cover of  by these shapes can be
  interpreted as a cover of  by the corresponding sets of
  . Thus, a \PTAS for the friendly geometric set cover
  problem implies a \PTAS for the minimum -set cover, which is
  impossible unless .
\end{proof}


\subsection{Set cover by fat triangles}
\seclab{hardness-set-cover-fat-triangles}



In the \emphi{fat-triangle set cover problem}, specified by a set of
points in the plane  and a set of fat triangles ,
one wants to find the minimum subset of  such that its union
covers all the points of .

\begin{lemma}
  \lemlab{no:PTAS:fat:tr:set:cover}There is no \PTAS for the fat-triangle set cover problem, unless
  .  Furthermore, one can prespecify an arbitrary constant
  , and the claim would hold true even if the following
  conditions hold on the given instance :
  \begin{compactenum}[\quad(A)]
  \item The minimum angle of all the triangles of  is larger
    than  degrees.

  \item No point of  is covered by more than two triangles of
    .

  \item The points of  are in convex position.

  \item All the triangles of  are of similar
    size. Specifically, each triangle has diameter in the range (say)
    .

  \item Each triangle of  has two angles in the range
    , and one angle in the range
    .

  \item The vertices of the triangles of  are the points of
    .
  \end{compactenum}
\end{lemma}

\begin{proof}
  Consider a graph  with maximum degree three, and observe
  that a \ProblemC{Vertex Cover} problem in such a graph can be
  reduced to \ProblemC{Set Cover} where every set is of size at most
  . Indeed, the ground set  is the edges of , and every
  vertex  gives a rise to the set
  ,
  which is of size at most . Clearly, any cover  of size  for
  the set system
  
  has a corresponding vertex cover of  of the same size. Thus,
  \ProblemC{Set Cover} with every set of size (at most) three is
  \APXHard (this is of course well known). Note that in this set cover
  instance, every element participates in exactly two sets (i.e., the
  two vertices adjacent to the original edge).

  The graph  has maximum degree three, and by Vizing's theorem
  \cite{bm-gta-76}, it is  edge-colorable\footnote{Vizing's theorem
    states that a graph with maximum degree  can be edge
    colored by  colors. In this specific case, one can reach
    the same conclusion directly from Brook's theorem. Indeed, in our
    case, the adjacency graph of the edges has degree at most , and
    it does not contain a clique of size . As such, this graph is
    -colorable, implying the original graph edges are
    -colorable.}.  With regards to the set problem, the ground set
  of the set system  can be colored by  colors such
  that no set in this set system has a color appearing more than once.

  We are given an instance of the \ProblemC{Vertex Cover} problem for
  a graph with maximum degree , and we transform it into a set
  cover instance as mentioned above, denoted by
  . Let
  , and color  (as described above) by  colors
  such that no set of  has the same color repeated twice,
  let  be the partition of  by the color of the
  points.

  \parpic[r]{\IncludeGraphics{figs/8_way}}
  Let  denote the circle of radius one centered at the
  origin. We place four relatively short arcs on , placed on the
  four intersection points of  with the  and  axes, see
  figure on the right.  Let  denote these four
  circular intervals.  We equally space the elements of  (as
  points) on the interval , for . Let 
  be the resulting set of points.


  For every set , take the convex hull of
  the points corresponding to its elements as its representing
  triangle . Note, that since the vertices of  lie on three
  intervals out of , it follows that it must be
  fat, for all . As such, the resulting
  set of triangles
   is fat, and
  clearly there is a cover of  by  triangles of 
  if and only if the original set cover problem has a cover of size
  .


  Any triangle having its three vertices on three different intervals
  of  is close to being an isosceles triangle with
  the middle angle being  degrees. As such, by choosing these
  intervals to be sufficiently short, any triangle of  would
  have a minimum degree larger than, say,  degrees, and
  with diameter in the range between  and .

  This is clearly an instance of the fat-triangle set cover
  problem. Solving it is equivalent to solving the original
  \ProblemC{Vertex Cover} problem, but since it is \APXHard, it
  follows that the fat-triangle set cover problem is \APXHard.
\end{proof}

\begin{remark}
  For fat triangles of similar size a constant factor approximation
  algorithm is known \cite{cv-iaags-07}.
  \lemref{no:PTAS:fat:tr:set:cover} implies that one can do no
  better. Naturally, it might be possible to slightly improve the
  constant of approximation provided by the algorithm of Clarkson and
  Varadarajan \cite{cv-iaags-07}.
However, for fat triangles of different sizes, only a 
  approximation is known \cite{abes-ibulf-14}. It is natural to ask if
  this can be improved.
\end{remark}

\subsubsection{Extensions}

\begin{lemma}
  \lemlab{no:P:T:A:S:circles}Given a set of points  in the plane and a set of circles
  , finding the minimum number of circles of  that
  covers  is \APXHard; that is, there is no \PTAS for this
  problem.
\end{lemma}
\begin{proof}
  Slightly perturb the point set used in the proof of
  \lemref{no:PTAS:fat:tr:set:cover}, so that no four points of it are
  co-circular. Let  denote the resulting set of points. For
  every set , we now take the circle
  passing through the three corresponding points. Clearly, this
  results in a set of circles (that are almost identical, but yet all
  different), such that finding the minimum number of circles covering
  the set  is equivalent to solving the original problem.
\end{proof}


\begin{lemma}
  \lemlab{no:PTAS:cover:planes}Given a set of points  in  and a set of planes
  , finding the minimum number of planes of  that
  covers  is \APXHard; that is, there is no \PTAS for this
  problem.
\end{lemma}
\begin{proof}
  Let  be the point set and  be the set of circles
  constructed in the proof of \lemref{no:P:T:A:S:circles}, and map
  every point in it to three dimensions using the mapping
  .  This is a standard lifting
  map used in computing planar Delaunay triangulations via convex-hull
  in three dimensions, see \cite{bcko-cgaa-08}.  Let
   be the resulting point set.

  It is easy to verify that a circle of  is mapped by
   into a curve that lies on a plane. We will abuse notations
  slightly, and use  to denote this plane.  Let
  . Furthermore, for a circle ,
  we have that .  Namely,
  solving the set cover problem  is equivalent
  to solving the original set cover instance .
\end{proof}


The recent work of Mustafa \etal \cite{mr-irghs-10} gave a \QPTAS for
set cover of points by disks (i.e., circles with their interior), and
for set cover of points by half-spaces in three dimensions. Thus,
somewhat surprisingly, the ``shelled'' version of these problems are
harder than the filled-in version.  


\subsection{Independent set of triangles in 3D}


Given a set  of  objects in  (say, triangles in
3d), we are interested in computing a maximum number of objects that
are \emphi{independent}; that is, no pair of objects in this set
(i.e., independent set) intersects. This is the geometric realization
of the \emphi{independent set} problem for the intersection graph
induced by these objects.

\begin{lemma}
  \lemlab{no:PTAS:3:d:i:s}There is no \PTAS for the maximum independent set of triangles in
  , unless .
\end{lemma}
\begin{proof}
  The problem \ProblemC{Independent Set} is \APXHard even for graphs
  with maximum degree  \cite{acgkm-ca-99}. Let
   be a given graph with maximum degree
  , where .  We will create a
  set of triangles, such that their intersection graph is .

  If one spreads  points  on the positive
  branch of the moment curve in  \cite{s-eubnf-91,
    ek-alnfc-03}, their Voronoi diagram is \emphi{neighborly}; that
  is, every Voronoi cell is a convex polytope that shares a non-empty
  two dimensional boundary face with each of the other cells of the
  diagram. Let  denote the cell of the point  in this
  Voronoi diagram, for .

  Now, for every vertex , we form a set 
  of (at most) three points, as follows. If , then
  we place a point  on the common boundary of  and ,
  and we add this point to both  and .  After
  processing all the edges in , each point set  has
  at most three points, as the maximum degree in  is three.

  For , let  be the triangle formed by the
  convex-hull of  (if  has fewer than three
  points then the triangle is degenerate).

  Let .  Observe that the triangles
  of  are disjoint except maybe in their common vertices, as
  their interior is contained inside the interior of , and the
  cells  are interior disjoint. Clearly
   if and only if . Thus,
  finding an independent set in  is equivalent to finding an
  independent set of triangles of the same size in . We
  conclude that the problem of finding maximum independent set of
  triangles is \APXHard, and as such does not have a \PTAS unless
  .
\end{proof}


Implicit in the above proof is that any graph can be realized as the
intersection graph of convex bodies in  (we were a bit more
elaborate for the sake of completeness and since we needed slightly
more structure). This is well known and can be traced to a result of
Tietze from 1905 \cite{t-upnr-05}.


\subsection{Hardness of approximation with respect to depth}




Using \obsref{eth:no:a}, we can derive hardness of approximation
results even for ``small'' instances.  Here, we reconsider the
\emph{geometric set cover problem}: Given a set of objects 
in , and a set of points , we would like to find
minimum cardinality subset of the objects in  that covers the
points of .

\begin{lemma}
  \lemlab{e:t:h:g:polylog:d}Assuming the exponential time hypothesis () (see
  \secref{complexity}), consider a given set of fat triangles
   of density , and a set of points ,
  such that .  We have the
  following:
  \begin{compactenum}[\quad(A)]
  \item If  then one cannot
    -approximate the geometric set cover (or geometric
    hitting set) instance  in polynomial time,
    where  is a sufficiently large constant.

  \item There is an absolute constant , such that no
    -approximation algorithm for the geometric set cover (or
    hitting set) instance  has running time
    .
  \end{compactenum}
\end{lemma}
\begin{proof}
  (A) Suppose we had such a \PTAS, and consider an instance
   of \TrSAT of size at least , where  is
  a sufficiently large constant.  \ETH implies that any algorithm
  solving such an instance must have running time at least
  .  On the other hand, the instance 
  can be converted to a set cover instance of fat triangles with
   triangles/points and  density, by
  \lemref{no:PTAS:fat:tr:set:cover}. As such, a \PTAS in this case,
  would contradict \ETH.

  (B) Consider a constant  sufficient small.  By part (A), an
  instance of \TrSAT with  variables, can be converted
  into an instance of geometric set cover (of fat triangles) with
  depth  (where  is some constant). But then, if
  , this implies that this instance be solved in
  polynomial time, contradicting \ETH.

  The same conclusions holds for geometric hitting set, by using
  \lemref{no:PTAS:fat:hit:set}.
\end{proof}


\section{Conclusions}
\seclab{conclusions}

In this paper, we studied the class of graphs arising out of low
density objects in , and showed that they are subclass of
graphs with polynomial expansion. We provided \PTAS's for independent
set and dominating set problems (and some variants) for such
graphs. This gives rise to \PTAS for some generic variants of these
problems. Coupled with hardness results, we characterize the
complexity of geometric variants of set cover and hitting set as a
function of depth (for example, for fat triangles).

At this point in time, it seems interesting to better understand low
density graphs. In particular, how exactly do they relate to graphs of
low genus, and whether one can develop efficient approximation
algorithms and hardness of approximations to other problems for this
family of graphs. (In particular, there is strong evidence that low
genus graphs are low density graphs.) For example, as a concrete
problem, can one get a \PTAS for TSP for low-density graphs or
polynomial expansion graphs?

\paragraph{Acknowledgments.}

The authors thank Mark \si{de} Berg for useful discussions related to
the problems studied in this paper. In particular, he pointed out the
improved bound for \lemref{separator:low:density}. We also thank the
anonymous referees. We are particularly grateful to the anonymous
referee of a previous version of this paper who pointed out the
connection of our work to graphs with bounded expansion.

\hypersetup{allcolors=black}

\BibTexMode{\newcommand{\etalchar}[1]{}
 \providecommand{\CNFX}[1]{ {\em{\textrm{(#1)}}}}
  \providecommand{\tildegen}{{\protect\raisebox{-0.1cm}{\symbol{'176}\hspace{-0.03cm}}}}
  \providecommand{\SarielWWWPapersAddr}{http://sarielhp.org/p/}
  \providecommand{\SarielWWWPapers}{http://sarielhp.org/p/}
  \providecommand{\urlSarielPaper}[1]{\href{\SarielWWWPapersAddr/#1}{\SarielWWWPapers{}/#1}}
  \providecommand{\Badoiu}{B\u{a}doiu}
  \providecommand{\Barany}{B{\'a}r{\'a}ny}
  \providecommand{\Bronimman}{Br{\"o}nnimann}  \providecommand{\Erdos}{Erd{\H
  o}s}  \providecommand{\Gartner}{G{\"a}rtner}
  \providecommand{\Matousek}{Matou{\v s}ek}
  \providecommand{\Merigot}{M{\'{}e}rigot}
  \providecommand{\Hastad}{H\r{a}stad\xspace}
  \providecommand{\CNFCCCG}{\CNFX{CCCG}}
  \providecommand{\CNFBROADNETS}{\CNFX{BROADNETS}}
  \providecommand{\CNFESA}{\CNFX{ESA}}
  \providecommand{\CNFFSTTCS}{\CNFX{FSTTCS}}
  \providecommand{\CNFIJCAI}{\CNFX{IJCAI}}
  \providecommand{\CNFINFOCOM}{\CNFX{INFOCOM}}
  \providecommand{\CNFIPCO}{\CNFX{IPCO}}
  \providecommand{\CNFISAAC}{\CNFX{ISAAC}}
  \providecommand{\CNFLICS}{\CNFX{LICS}}
  \providecommand{\CNFPODS}{\CNFX{PODS}}
  \providecommand{\CNFSWAT}{\CNFX{SWAT}}
  \providecommand{\CNFWADS}{\CNFX{WADS}}
\begin{thebibliography}{vdSOdBV98}

\bibitem[ACG{\etalchar{+}}99]{acgkm-ca-99}
G.~Ausiello, P.~Crescenzi, G.~Gambosi, V.~Kann, A.~Marchetti-Spaccamela, and
  M.~Protasi.
\newblock  {\em Complexity and approximation}.
\newblock Springer-Verlag, Berlin, 1999.

\bibitem[AdBES14]{abes-ibulf-14}
\href{http://cis.poly.edu/~aronov/}{B.~{Aronov}}, \href{http://www.win.tue.nl/~mdberg/}{M.~de~{Berg}}, E.~Ezra, and \href{http://www.math.tau.ac.il/~michas}{M.~{Sharir}}.
\newblock \href{http://dx.doi.org/10.1137/120891241}{Improved bounds for the
  union of locally fat objects in the plane}.
\newblock {\em SIAM J. Comput.}, 43(2):543--572, 2014.

\bibitem[AES10]{aes-ssena-10}
\href{http://cis.poly.edu/~aronov/}{B.~{Aronov}}, E.~Ezra, and \href{http://www.math.tau.ac.il/~michas}{M.~{Sharir}}.
\newblock \href{http://dx.doi.org/10.1137/090762968}{Small-size
  {}-nets for axis-parallel rectangles and boxes}.
\newblock {\em SIAM J. Comput.}, 39(7):3248--3282, 2010.

\bibitem[And70]{a-ocpls-70}
E.M. Andreev.
\newblock  On convex polyhedra in lobachevsky spaces.
\newblock {\em Sbornik: Mathematics}, 10:413--440, April 1970.

\bibitem[APS08]{aps-sugo-08}
\href{http://www.cs.duke.edu/~pankaj}{P.~K.~{Agarwal}}, \href{http://www.math.nyu.edu/~pach}{J.~{Pach}}, and \href{http://www.math.tau.ac.il/~michas}{M.~{Sharir}}.
\newblock
  \href{https://users.cs.duke.edu/~pankaj/publications/surveys/union.pdf}{State
  of the union--of geometric objects}.
\newblock In J.~E. Goodman, \href{http://www.math.nyu.edu/~pach}{J.~{Pach}}, and R.~Pollack, editors, {\em Surveys in
  Discrete and Computational Geometry Twenty Years Later}, volume 453 of {\em
  Contemporary Mathematics}, pages 9--48. Amer. Math. Soc., 2008.

\bibitem[AW13]{aw-asmwi-13}
A.~Adamaszek and A.~Wiese.
\newblock  Approximation schemes for maximum weight independent set of
  rectangles.
\newblock In {\em Proc. 54th Annu. IEEE Sympos. Found. Comput. Sci.
  {\em(FOCS)}}, pages 400--409, 2013.

\bibitem[AW14]{aw-qmwis-14}
A.~Adamaszek and A.~Wiese.
\newblock  A {QPTAS} for maximum weight independent set of polygons with
  polylogarithmic many vertices.
\newblock In {\em Proc. 25th ACM-SIAM Sympos. Discrete Algs. {\em(SODA)}},
  pages 400--409, 2014.

\bibitem[Bak94]{b-aancp-94}
B.~S. Baker.
\newblock  Approximation algorithms for {NP}-complete problems on planar
  graphs.
\newblock {\em \href{http://www.acm.org/jacm/}{J. Assoc. Comput. {Mach.}}}, 41:153--180, 1994.

\bibitem[BM76]{bm-gta-76}
J.~A. Bondy and U.~S.~R. Murty.
\newblock \href{http://www.ecp6.jussieu.fr/pageperso/bondy/books/gtwa/gtwa\
  .html}{{\em Graph Theory with Applications}}.
\newblock North-Holland, 1976.

\bibitem[CCH12]{cch-smcpg-12}
C.~Chekuri, K.~Clarkson, and \href{http://sarielhp.org}{S.~{{Har-Peled}}}.
\newblock \href{http://sarielhp.org/papers/08/multi_cover}{On the set
  multi-cover problem in geometric settings}.
\newblock {\em ACM Trans. Algo.}, 9(1):9, 2012.

\bibitem[CG09]{cg-epgig-09}
J.~Chalopin and D.~Gon{\c{c}}alves.
\newblock  Every planar graph is the intersection graph of segments in the
  plane: extended abstract.
\newblock In {\em Proc. 41st Annu. ACM Sympos. Theory Comput. {\em(STOC)}},
  pages 631--638, 2009.

\bibitem[CG14a]{cg-spfgem-14}
S.~Cabello and D.~Gajser.
\newblock \href{http://arxiv.org/abs/1410.5778}{Simple ptas's for families of
  graphs excluding a minor}.
\newblock {\em CoRR}, abs/1410.5778, 2014.

\bibitem[CG14b]{cg-eaahr-14}
Timothy~M. Chan and Elyot Grant.
\newblock  Exact algorithms and {APX}-hardness results for geometric packing
  and covering problems.
\newblock {\em Comput. Geom. Theory Appl.}, 47(2):112--124, 2014.

\bibitem[CH12]{ch-aamis-12}
\href{http://www.math.uwaterloo.ca/~tmchan/}{T.~M.~{Chan}} and \href{http://sarielhp.org}{S.~{{Har-Peled}}}.
\newblock  Approximation algorithms for maximum independent set of
  pseudo-disks.
\newblock {\em \href{http://link.springer.com/journal/454}{Discrete Comput. {}Geom.}}, 48:373--392, 2012.

\bibitem[Cha03]{c-ptasp-03}
Timothy~M. Chan.
\newblock  Polynomial-time approximation schemes for packing and piercing fat
  objects.
\newblock {\em J. Algorithms}, 46(2):178--189, 2003.

\bibitem[CR15]{cr-btb-15}
Daniel~W. Cranston and Landon Rabern.
\newblock  Brooks' theorem and beyond.
\newblock {\em J. Graph Theo.}, 80(3):199--225, 2015.

\bibitem[CV07]{cv-iaags-07}
\href{http://cm.bell-labs.com/who/clarkson/}{K.~L. {Clarkson}} and \href{http://www.cs.uiowa.edu/~kvaradar/}{K.~R. {Varadarajan}}.
\newblock  Improved approximation algorithms for geometric set cover.
\newblock {\em \href{http://link.springer.com/journal/454}{Discrete Comput. {}Geom.}}, 37(1):43--58, 2007.

\bibitem[dB08]{b-ibucf-08}
\href{http://www.win.tue.nl/~mdberg/}{M.~de~{Berg}}.
\newblock  Improved bounds on the union complexity of fat objects.
\newblock {\em \href{http://link.springer.com/journal/454}{Discrete Comput. {}Geom.}}, 40(1):127--140, 2008.

\bibitem[dBCKO08]{bcko-cgaa-08}
\href{http://www.win.tue.nl/~mdberg/}{M.~de~{Berg}}, \href{http://www.win.tue.nl/~ocheong}{O.~{Cheong}}, {M. van} Kreveld, and \href{http://www.cs.uu.nl/people/markov/}{M.~H. {Overmars}}.
\newblock \href{http://www.cs.uu.nl/geobook/}{{\em Computational Geometry:
  Algorithms and Applications}}.
\newblock Springer-Verlag, Santa Clara, CA, USA, 3rd edition, 2008.

\bibitem[dBKSV02]{bksv-rimga-02}
\href{http://www.win.tue.nl/~mdberg/}{M.~de~{Berg}}, M.~J. Katz, {A. F.}~{van~der} Stappen, and J.~Vleugels.
\newblock  Realistic input models for geometric algorithms.
\newblock {\em Algorithmica}, 34(1):81--97, 2002.

\bibitem[DN15]{dn-ssspe-15}
Z.~{Dvo{\v{r}}{\'{a}}k} and S.~{Norin}.
\newblock \href{http://arxiv.org/abs/1504.04821}{{Strongly sublinear separators
  and polynomial expansion}}.
\newblock {\em ArXiv e-prints}, April 2015.

\bibitem[EK03]{ek-alnfc-03}
\href{http://compgeom.cs.uiuc.edu/~jeffe/}{J.~{Erickson}} and S.~Kim.
\newblock  Arbitrarily large neighborly families of congruent symmetric convex
  3-polytopes.
\newblock In A.~Bezdek, editor, {\em Discrete Geometry:~In Honor of W.
  Kuperberg's 60th Birthday}, Lecture Notes Pure Appl. Math., pages 267--278.
  Marcel-Dekker, 2003.

\bibitem[Epp00]{e-dtmcg-00}
\href{http://www.ics.uci.edu/~eppstein/}{D.~{Eppstein}}.
\newblock  Diameter and treewidth in minor-closed graph families.
\newblock {\em Algorithmica}, 27:275--291, 2000.

\bibitem[FG88]{fg-oafac-88}
T.~Feder and D.~H. Greene.
\newblock  Optimal algorithms for approximate clustering.
\newblock In {\em Proc. 20th Annu. ACM Sympos. Theory Comput. {\em(STOC)}},
  pages 434--444, 1988.

\bibitem[Fre87]{f-faspp-87}
G.~N. Frederickson.
\newblock  Fast algorithms for shortest paths in planar graphs, with
  applications.
\newblock {\em SIAM J. Comput.}, 16(6):1004--1022, 1987.

\bibitem[GKS14]{gks-dfopndg-14}
M.~Grohe, S.~Kreutzer, and S.~Siebertz.
\newblock  Deciding first-order properties of nowhere dense graphs.
\newblock In {\em Proc. 46th Annu. ACM Sympos. Theory Comput. {\em(STOC)}},
  pages 89--98, 2014.

\bibitem[Gro03]{g-ltwem-03}
M.~Grohe.
\newblock  Local tree-width, excluded minors, and approximation algorithms.
\newblock {\em Combinatorica}, 23(4):613--632, 2003.

\bibitem[{Har}09]{h-bffne-09}
\href{http://sarielhp.org}{S.~{{Har-Peled}}}.
\newblock  Being fat and friendly is not enough.
\newblock {\em CoRR}, 2009.

\bibitem[{Har}13]{h-speps-13}
\href{http://sarielhp.org}{S.~{{Har-Peled}}}.
\newblock  A simple proof of the existence of a planar separator.
\newblock {\em ArXiv e-prints}, April 2013.

\bibitem[{Har}14]{h-qssp-14}
\href{http://sarielhp.org}{S.~{{Har-Peled}}}.
\newblock  Quasi-polynomial time approximation scheme for sparse subsets of
  polygons.
\newblock In {\em Proc. 30th Annu. Sympos. Comput. Geom. {\em(SoCG)}}, pages
  120--129, 2014.

\bibitem[H{\aa}s99]{h-chaw-99}
Johan H{\aa}stad.
\newblock \href{http://dx.doi.org/10.1007/BF02392825}{Clique is hard to
  approximate withinn {}}.
\newblock {\em Acta Mathematica}, 182(1):105--142, 1999.

\bibitem[HKRS97]{hkrs-fspap-97}
M.~R. Henzinger, P.~Klein, S.~Rao, and S.~Subramanian.
\newblock  Faster shortest-path algorithms for planar graphs.
\newblock {\em J. Comput. Sys. Sci.}, 55:3--23, August 1997.

\bibitem[HQ15]{hq-aapel-15}
\href{http://sarielhp.org}{S.~{{Har-Peled}}} and K.~Quanrud.
\newblock  Approximation algorithms for polynomial-expansion and low-density
  graphs.
\newblock In {\em Proc. 23nd Annu. Euro. Sympos. Alg.\CNFESA}, volume 9294 of
  {\em Lect. Notes in Comp. Sci.}, pages 717--728, 2015.

\bibitem[HQ16]{hq-naape-16-arxiv}
\href{http://sarielhp.org}{S.~{{Har-Peled}}} and K.~{Quanrud}.
\newblock \href{http://arxiv.org/abs/1603.03098}{Notes on approximation
  algorithms for polynomial-expansion and low-density graphs}.
\newblock {\em ArXiv e-prints}, March 2016.

\bibitem[HR13]{hr-nplta-13}
\href{http://sarielhp.org}{S.~{{Har-Peled}}} and B.~Raichel.
\newblock \href{http://cs.uiuc.edu/~sariel/papers/12/aggregate/}{Net and prune:
  A linear time algorithm for {Euclidean} distance problems}.
\newblock In {\em Proc. 45th Annu. ACM Sympos. Theory Comput. {\em(STOC)}},
  pages 605--614, New York, NY, USA, 2013. ACM.

\bibitem[HT74]{ht-ept-74}
J.~E. Hopcroft and R.~E. Tarjan.
\newblock  Efficient planarity testing.
\newblock {\em \href{http://www.acm.org/jacm/}{J. Assoc. Comput. {Mach.}}}, 21(4):549--568, 1974.

\bibitem[IP01]{ip-ocks-01}
R.~Impagliazzo and R.~Paturi.
\newblock  On the complexity of {}-{SAT}.
\newblock {\em J. Comput. Sys. Sci.}, 62(2):367--375, 2001.

\bibitem[IPZ01]{ipz-wphse-01}
R.~Impagliazzo, R.~Paturi, and F.~Zane.
\newblock  Which problems have strongly exponential complexity?
\newblock {\em J. Comput. Sys. Sci.}, 63(4):512--530, 2001.

\bibitem[Kar72]{k-racp-72}
R.~M. Karp.
\newblock
  \href{http://www.cs.berkeley.edu/luca/cs172/karp.pdf}{Reducibility
  among combinatorial problems}.
\newblock In {\em Complexity of Computer Computations}, pages 85--103, 1972.

\bibitem[Koe36]{k-kdka-36}
P.~Koebe.
\newblock  Kontaktprobleme der konformen {Abbildung}.
\newblock {\em Ber. Verh. S{\"a}chs. Akademie der Wissenschaften Leipzig,
  Math.-Phys. Klasse}, 88:141--164, 1936.

\bibitem[LT79]{lt-stpg-79}
R.~J. Lipton and R.~E. Tarjan.
\newblock  A separator theorem for planar graphs.
\newblock {\em SIAM J. Appl. Math.}, 36:177--189, 1979.

\bibitem[LT80]{lt-apst-80}
R.~J. Lipton and R.~E. Tarjan.
\newblock  Applications of a planar separator theorem.
\newblock {\em SIAM J. Comput.}, 9(3):615--627, 1980.

\bibitem[Mat14]{m-nossg-14}
\href{http://kam.mff.cuni.cz/~matousek}{J. Matou{\v s}ek}.
\newblock  Near-optimal separators in string graphs.
\newblock {\em Combin., Prob. {\&} Comput.}, 23(1):135--139, 2014.

\bibitem[MR10]{mr-irghs-10}
Nabil~H. Mustafa and Saurabh Ray.
\newblock  Improved results on geometric hitting set problems.
\newblock {\em \href{http://link.springer.com/journal/454}{Discrete Comput. {}Geom.}}, 44(4):883--895, 2010.

\bibitem[MRR14a]{mrr-qgscp-14}
N.~H. {Mustafa}, R.~{Raman}, and S.~{Ray}.
\newblock \href{http://arxiv.org/abs/1403.0835}{{QPTAS} for geometric set-cover
  problems via optimal separators}.
\newblock {\em ArXiv e-prints}, 2014.

\bibitem[MRR14b]{mrr-sahsg-14}
N.~H. Mustafa, R.~Raman, and S.~Ray.
\newblock  Settling the {APX}-hardness status for geometric set cover.
\newblock In {\em Proc. 55th Annu. IEEE Sympos. Found. Comput. Sci.
  {\em(FOCS)}}, pages 541--550, 2014.

\bibitem[MTTV97]{mttv-sspnng-97}
G.~L. Miller, S.~H. Teng, W.~P. Thurston, and S.~A. Vavasis.
\newblock  Separators for sphere-packings and nearest neighbor graphs.
\newblock {\em \href{http://www.acm.org/jacm/}{J. Assoc. Comput. {Mach.}}}, 44(1):1--29, 1997.

\bibitem[NO08a]{no-gcbe1-08}
J.~{Ne{\v s}et{\v r}il} and P.~{Ossona de Mendez}.
\newblock  Grad and classes with bounded expansion {I}. decompositions.
\newblock {\em Eur. J. Comb.}, 29(3):760--776, 2008.

\bibitem[NO08b]{no-gcbe2-08}
J.~{Ne{\v s}et{\v r}il} and P.~{Ossona de Mendez}.
\newblock  Grad and classes with bounded expansion {II}. algorithmic aspects.
\newblock {\em Eur. J. Comb.}, 29(3):777--791, 2008.

\bibitem[NO12]{no-s-12}
J.~Ne{\v s}et{\v r}il and P.~{Ossona de Mendez}.
\newblock  {\em Sparsity -- Graphs, Structures, and Algorithms}, volume~28 of
  {\em Alg. Combin.}
\newblock Springer, 2012.

\bibitem[PA95]{pa-cg-95}
\href{http://www.math.nyu.edu/~pach}{J.~{Pach}} and \href{http://www.cs.duke.edu/~pankaj}{P.~K.~{Agarwal}}.
\newblock \href{http://www.addall.com/Browse/Detail/0471588903.html}{{\em
  Combinatorial Geometry}}.
\newblock John Wiley \& Sons, 1995.

\bibitem[RS97]{rs-sbepl-97}
R.~Raz and S.~Safra.
\newblock  A sub-constant error-probability low-degree test, and a sub-constant
  error-probability {PCP} characterization of {NP}.
\newblock In {\em Proc. 29th Annu. ACM Sympos. Theory Comput. {\em(STOC)}},
  pages 475--484, 1997.

\bibitem[Sei91]{s-eubnf-91}
\href{http://www-tcs.cs.uni-sb.de/seidel/}{R.~{Seidel}}.
\newblock \href{http://dimacs.rutgers.edu/Volumes/Vol04.html}{Exact upper
  bounds for the number of faces in {}-dimensional {V}oronoi diagrams}.
\newblock In P.~Gritzman and B.~Sturmfels, editors, {\em Applied Geometry and
  Discrete Mathematics: The Victor Klee Festschrift}, volume~4 of {\em DIMACS
  Series in Discrete Mathematics and Theoretical Computer Science}, pages
  517--530. Amer. Math. Soc., 1991.

\bibitem[SS85]{ss-empae-85}
J.~T. Schwartz and \href{http://www.math.tau.ac.il/~michas}{M.~{Sharir}}.
\newblock  Efficient motion planning algorithms in environments of bounded
  local complexity.
\newblock Report 164, Dept. Comput. Sci., Courant Inst. Math. Sci., New York
  Univ., New York, NY, 1985.

\bibitem[{Sta}92]{f-mpafo-92}
{A. F. van~der} {Stappen}.
\newblock
  \href{http://www.staff.science.uu.nl/~stapp101/PhDThesis_AFvanderStappen.pdf}{{\em
  Motion Planning Amidst Fat Obstacles}}.
\newblock PhD thesis, Utrecht University, Netherlands, 1992.

\bibitem[SW98]{sw-gsta-98}
W.~D. Smith and N.~C. Wormald.
\newblock  Geometric separator theorems and applications.
\newblock In {\em Proc. 39th Annu. IEEE Sympos. Found. Comput. Sci.
  {\em(FOCS)}}, pages 232--243, 1998.

\bibitem[Tie05]{t-upnr-05}
H.~Tietze.
\newblock  Uber das problem der nachbargeibiete im raum.
\newblock {\em Monatshefte Math.}, 15:211--216, 1905.

\bibitem[vdSOdBV98]{sobv-mpelo-98}
A.~F. van~der Stappen, \href{http://www.cs.uu.nl/people/markov/}{M.~H. {Overmars}}, \href{http://www.win.tue.nl/~mdberg/}{M.~de~{Berg}}, and J.~Vleugels.
\newblock  Motion planning in environments with low obstacle density.
\newblock {\em \href{http://link.springer.com/journal/454}{Discrete Comput. {}Geom.}}, 20(4):561--587, 1998.

\bibitem[{Ver}05]{v-cbseb-05}
J.-L. {Verger-Gaugry}.
\newblock  Covering a ball with smaller equal balls in {}.
\newblock {\em \href{http://link.springer.com/journal/454}{Discrete Comput. {}Geom.}}, 33(1):143--155, 2005.

\end{thebibliography}

}

\BibLatexMode{\printbibliography}











\end{document}
