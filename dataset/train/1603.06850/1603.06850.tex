 We assume familiarity with the standard syntax and terminology of many-sorted first-order logics. We use vector notation: $\vec{v} = (v_1,\ldots,v_n)$ denotes an ordered sequence of terms. For two vectors $\vec{u}$ and $\vec{v}$, we write their concatenation as $\vec{u}\vec{v}$.







Within this paper we consider the domains of arrays, array
indices, and array elements to be $\Ab = \Zb^*$,
$\Nb = \{\,0,1,\ldots \,\}$, and $\Zb = \{\,\ldots,-1,0,1,\ldots \,\}$
respectively.

Presburger arithmetic has the signature
$\Sigma_{\mathbb{Z}} = \{\,0,1,+,<\,\}$; we use it for array indices
and elements, as well as other arithmetic assertions, possibly with
embedded array terms.
We write $\true$ and $\false$ to denote a valid and an unsatisfiable Presburger formula, respectively.

The theory of integer-indexed arrays extends Presburger arithmetic
with functions \var{read}, and \var{write}, and has the signature
$\Sigma_{A} = \Sigma_{\mathbb{Z}} \cup \{\,\cdot[\cdot],\cdot\{\cdot
\leftarrow \cdot\}\, \}$.
The \var{read} function $a[i]$ returns the $i$-th element of array
$a$, and the \var{write} function $a\{i \leftarrow x\}$ returns array $a$ where the $i$-th element is replaced by $x$. These functions should satisfy the \emph{read}-over-\emph{write} axioms as described by McCarthy \cite{McCarthy62}; the decision procedure for the quantifier-free array theory is presented in \cite{QuantifierFreeArrays}.


\subsection{Syntax}
\label{afl-syntax}

\emph{Array Folds Logic (AFL)} extends the quantifier-free theory of integer arrays with the ability to perform counting. The extension works by incorporating \Fold\ terms into arithmetic expressions; such a term folds some function over the array by applying it to each array element consecutively.

AFL contains the following sorts: array sort $\asort$, integer sort $\isort$, Boolean sort $\bsort$, and two enumerable sets of sorts for integer vectors $\vsort{m}$ and functional constants $\fsort{m} = \vsort{m} \times \isort \rightarrow \vsort{m}$, for each $m \in \Nb$, $m>0$. The syntax of the AFL terms is shown in Table~\ref{tab:syntax}; $a$ and $b$ denote array variables, $x$ denotes an integer variable, $n$ and $m$ denote integer constants.

\emph{Array terms}~$A$ of sort $\asort$ are represented either by an array variable $a$, or by the \var{write} term $a\{T \leftarrow T\}$.

\emph{Integer terms}~$T$ of sort $\isort$ can be integer constants $n \in \Zb$, integer variables $x$, integer addition, \var{read} term $a[T]$ for the index represented as an integer term, or the term $|a|$, which represents the length of array $a$. 

\emph{Boolean terms} $B$  of sort $\bsort$ are formed by array equality, usual Presburger and Boolean operators, and equality between vectors of sort $\vsort{m}$.

\emph{Vector terms} $V^m$ of sort $\vsort{m}$ are either a list of $m$ integer terms, or a \Fold\ term. The former is written as a vertical list in parentheses; they can be omitted when $m=1$. The latter, written as $\Fold_{a}\:{v}\:{f}$, represents the result of the transformation of an input vector $v$ of sort $\vsort{m}$ by folding a functional constant $f$ of sort $\fsort{m}$ over an array $a$.  The first element of $v$ specifies an initial value of the array index; the remaining elements give initial values for the counters that can be used inside $f$. The resulting vector after the transformation gives the final values for the array index and the counters.

\emph{Functional constants} (when no confusion can arise, we call them  \emph{functions}) $F^m$ of sort $\fsort{m}$ can only be  a parenthesized list of branches (guarded commands); the length of the list is unrelated to $m$. A function $f$ of sort $\fsort{m}$ can refer to the following implicitly declared variables: $\elm$ for the currently inspected array element; $\idx$ for the current array index; $\cnt_1,\ldots,\cnt_{m-1}$ for the counters; $\st$ for the state (control flow) variable. All other variables that occur inside $f$ are considered as free variables of sort $\isort$. 

\emph{Guards} are conjunctions of \emph{atomic guards}, which can compare array elements, indices, and counters to integer terms; the state variable can only be compared  to integer constants. 
\emph{Updates} are lists of \emph{atomic updates}; they can increment or decrease counters by a constant, assign a constant to the state variable, \Skip, i.e. perform no updates, or execute a \Break\ statement, which terminates the \Fold\ at the current position.  Counter or state updates define a function $\Zb \ra \Zb$.
Guards and updates translate into logical formulas that either constraint the current variable values, or relate the current and the next-state (primed) variable values in the obvious way; we denote this translation by $\Phi$. E.g.,
the update $\update \equiv (\cnt_1 \pen)$ defines the formula $\Phi(\update) \equiv (\cnt_1' = \cnt_1+n)$. 

\begin{table}[t]
\begin{align*}
  A~~~::=~& a ~~|~~ a\{T \leftarrow T\} \\
T~~~::=~& n ~~|~~ x ~~|~~ T+T ~~|~~ a[T] ~~|~~ |a| \\
B~~~::=~& a = b  ~~|~~ T = T ~~|~~ T < T ~~|~~ \neg B ~~|~~ B \land B ~~|~~    
  V^m = V^m \\
  V^m~::=~& \Vect{T;\cdots;T} ~~|~~ \Fold_{a}\: V^m\: F^m 
  ~~~~~~~~~~~~~~~~~~~~~~~~~~~~ \\
F^m~::=~& \Vect{\guard \!\Then\! \update;\cdots;\guard \!\Then\! \update} \\
\guard ~::=~&  \elm \approx T ~|~ \idx \approx T ~|~ \cnt_m \approx T  ~|~  \st \approx  n ~|~ \guard \land \guard \hspace{15mm}\text{($\approx \;\in\: \{>,<,=, \ne\}$)}\\
  \update ~::=~&  \cnt_m \pen ~|~ \st \leftarrow n ~|~ \Skip ~|~ \Break ~|~  \update \:; \update
\end{align*}
\vspace*{2mm}
\caption{Syntax of AFL.}
\label{tab:syntax}
\end{table}


We require that guards of all branches are mutually exclusive. There is an implicit ``catch-all'' branch with the \Break\ statement, whose guard evaluates to $\true$ exactly when guards of all other  branches evaluate to $\false$. We also require that each branch contains at most one update for each implicit variable.

We restrict the control flow in functions, which is defined by state variable $\st$. Notice that $\st$ is syntactically finite state. Thus, given a set of function branches $\var{Br}$, we define an edge-labeled control flow graph $G = \tuple{S,E, \gamma}$, where:
\begin{itemize}
\item states $S = \bigmset{0} \cup \bigmathset{n}{\st \!\leftarrow\! n \in \var{Br}}$;
\item edges $E = \bigcup_{\guard \!\!\Then\!\! \update \:\in\: \var{Br}} \bigmathset{(s_1,s_2)}{s_1 \!\models\! \guard \land s_2 \!=\! \var{ite}(\st \!\leftarrow\! n \in \update,n,s_1)}$;
\item $\gamma$ is the labeling of edges with the set of formulas $\Phi(\guard)$ and $\Phi(\update)$ for each guard or update which occurs in the same branch.
\end{itemize}
We require that edges in the strongly-connected components of $G$ are labeled with counter updates that are, for each counter, all non-decreasing, or all non-increasing. Thus, $G$ is a DAG of SCCs, where counters within each SCC behave in a monotonic way. We use this restriction to derive from $f$ a reversal-bounded counter machine (see Definition~\ref{def:fscm}).

The presented syntax is minimal and can be extended with convenience functions and predicates such as $\{-, n \cdot, , \le, \ge, \for, \texttt{++},\texttt{--},\texttt{-=}n \}$ in the usual way. We allow to use $*$ to denote the absence of constraints: this is useful for vector notation. We replace each $*$ in the formula with a unique unconstrained variable.




\subsection{Semantics}


\begin{table}[t]
\begin{align*}
   1.~& \ev{\Vect{t^1_1;\cdots;t^1_m}\! =\! \Vect{t^2_1;\cdots;t^2_m}}\!\!\!\!\!\!\!\!\!\!\! &~\equiv~  &\left(\ev{t^1_1} = \ev{t^2_1} \right) \land \ldots \land \left(\ev{t^1_m} = \ev{t^2_m} \right) \\
2.~& \ev{\Fold_{a}\: v\: f} &~\equiv~  &\evsk{\Fold_{a}\: v\: f} \text{, where } \kappa(\FV) = \Vect{v;0}\\   
3.~&\evsk{\Fold_{a}\: v\: f} &~\equiv~ &\text{if  } (\evk{\idx}<0) \text{ or } (\evk{\idx}\ge |a|) \text{ or } (\var{false} \in \evsk{f}) \text{ then } v\\  
& & &\text{else} \evskp{\Fold_{a}\: v'\: f}\!\!\!\text{, where } \kappa'(\FV) = \Vect{v';\st'} = \evsk{f}\big(\kappa(\FV)\big)\\  
4.~&\evsk{f}\Vect{v_1;\ldots;v_m} &~\equiv~  & \Vect{v'_1;\ldots;v'_m}, \text{ where } v'_j \equiv \text{ if } \var{upd}(v_j) \in \evsk{f} \text{ then } \var{upd}(v_j) \text{, else } v_j \\
5.~&\evsk{\Vect{\guard_1 \Rightarrow \update_1;\cdots;\guard_m \Rightarrow \update_m}} &~\equiv~  & \{\idx'\!=\!\idx\!+\!1\} \cup \evsk{\guard_1 \Rightarrow \update_1} \cup \ldots \cup  \evsk{\guard_m \Rightarrow \update_m}\\  
6.~&\evsk{\guard \Rightarrow \update} &~\equiv~ &\text{if } \evsk{\guard}=\var{true} \text{  then  } \evsk{\update} \text{  else  } \emptyset\\
7.~&\evsk{ \elm \approx t} &~\equiv~  &\evk{\elm} \approx \ev{t} \hspace{20mm}\text{(similarly for } \idx \approx T ,\; \cnt_m \approx T,\;    \st \approx  n \text{)}\\  
8.~&\evsk{\guard_1 \land \guard_2} &~\equiv~  &\evsk{\guard_1} \land \evsk{\guard_2} \\  
9.~&\evsk{\update_1 ; \update_2} &~\equiv~  &\evsk{\update_1} \cup \evsk{\update_2} \\  
10.~&\evsk{\cnt_m \pen } &~\equiv~ & \{\cnt_m' \!=\! \cnt_m \!+\! n \}\\
11.~&\evsk{\st \leftarrow n } &~\equiv~  & \{\st' = n \}\\  
12.~&\evsk{\Skip} &~\equiv~ & \emptyset\\
13.~&\evsk{\Break} &~\equiv~ & \mset{\var{false}} 
\end{align*}
\vspace*{2mm}
\caption{Semantics of AFL}
\label{tab:semantics}
\end{table}


For a given AFL formula $\phi$, we denote the sets of free variables of $\phi$ of sort $\asort$ and $\isort$ by $\VA$ and $\VI$, respectively. All free variables are implicitly existentially quantified. For functions of sort $\fsort{m}$, we denote by $\FV$ the set of their implicit variables $\{\idx,\cnt_1,\ldots,\cnt_{m-1},\st \}$.


The treatment of array writes and reads in the quantifier-free array theory is standard \cite{QuantifierFreeArrays}, and we do not elaborate on it here.
Array equalities partition the set of array variables into equivalence classes; all other constraints are then translated into constraints over a representative of the corresponding equivalence class.

An \emph{interpretation} for AFL is a tuple $\sigma = \tuple{\lambda,\mu}$, where
$\lambda: \VI \ra \Zb$ assigns each integer variable an integer, and
$\mu: \VA \ra \Zb^*$ assigns each array variable a finite sequence of integers.



The semantics of an AFL term $t$ under the given interpretation $\sigma$ is defined by the \emph{evaluation} $\left[ t \right]^\sigma$. 
Terms that constitute functions are evaluated in the additional \emph{context} $\kappa$. For a function $f$ of sort $\fsort{m}$, $\kappa: \FV \ra \Zb^{m+1}$ maps internal variables of $f$ to integers. 
The evaluation of Presburger, Boolean, and array terms is standard; the remaining ones are shown in Table~\ref{tab:semantics}. We give some explanations here (the remaining semantic rules are self-explanatory):
\begin{enumerate}
\item Vector equality resolves to a conjunction of equalities between components.
\item A \Fold\ term evaluates in the initial context that is defined by the given initial vector of counters $v$, and assigns 0 to state variable $\st$.
\item A contextual \Fold\ term checks whether the array index is out of bounds, or a $\Break$ statement is executed in the current context (this is the only way for $\evsk{f}$ to contain $\false$). If yes, \Fold\ terminates, and returns the current vector $v$. Otherwise \Fold\ continues with the updated vector and context.
\item If an update $\var{upd}(v_j)$ for some variable $v_j$ is present in the function evaluation, then it is applied. Otherwise, the old variable value is preserved.
\item An evaluation of a function, represented by a list of branches, is a union of updates from its branch evaluations. Index $\idx$ is always incremented by 1.
\item A guarded command evaluates to its update if its guard evaluates to \var{true}.
\item A comparison over an internal variable evaluates it in the context $\kappa$, and the comparison term is evaluated in the interpretation $\sigma$.
\end{enumerate}






















