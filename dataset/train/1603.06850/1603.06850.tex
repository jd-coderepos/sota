 We assume familiarity with the standard syntax and terminology of many-sorted first-order logics. We use vector notation:  denotes an ordered sequence of terms. For two vectors  and , we write their concatenation as .







Within this paper we consider the domains of arrays, array
indices, and array elements to be ,
, and 
respectively.

Presburger arithmetic has the signature
; we use it for array indices
and elements, as well as other arithmetic assertions, possibly with
embedded array terms.
We write  and  to denote a valid and an unsatisfiable Presburger formula, respectively.

The theory of integer-indexed arrays extends Presburger arithmetic
with functions \var{read}, and \var{write}, and has the signature
.
The \var{read} function  returns the -th element of array
, and the \var{write} function  returns array  where the -th element is replaced by . These functions should satisfy the \emph{read}-over-\emph{write} axioms as described by McCarthy \cite{McCarthy62}; the decision procedure for the quantifier-free array theory is presented in \cite{QuantifierFreeArrays}.


\subsection{Syntax}
\label{afl-syntax}

\emph{Array Folds Logic (AFL)} extends the quantifier-free theory of integer arrays with the ability to perform counting. The extension works by incorporating \Fold\ terms into arithmetic expressions; such a term folds some function over the array by applying it to each array element consecutively.

AFL contains the following sorts: array sort , integer sort , Boolean sort , and two enumerable sets of sorts for integer vectors  and functional constants , for each , . The syntax of the AFL terms is shown in Table~\ref{tab:syntax};  and  denote array variables,  denotes an integer variable,  and  denote integer constants.

\emph{Array terms}~ of sort  are represented either by an array variable , or by the \var{write} term .

\emph{Integer terms}~ of sort  can be integer constants , integer variables , integer addition, \var{read} term  for the index represented as an integer term, or the term , which represents the length of array . 

\emph{Boolean terms}   of sort  are formed by array equality, usual Presburger and Boolean operators, and equality between vectors of sort .

\emph{Vector terms}  of sort  are either a list of  integer terms, or a \Fold\ term. The former is written as a vertical list in parentheses; they can be omitted when . The latter, written as , represents the result of the transformation of an input vector  of sort  by folding a functional constant  of sort  over an array .  The first element of  specifies an initial value of the array index; the remaining elements give initial values for the counters that can be used inside . The resulting vector after the transformation gives the final values for the array index and the counters.

\emph{Functional constants} (when no confusion can arise, we call them  \emph{functions})  of sort  can only be  a parenthesized list of branches (guarded commands); the length of the list is unrelated to . A function  of sort  can refer to the following implicitly declared variables:  for the currently inspected array element;  for the current array index;  for the counters;  for the state (control flow) variable. All other variables that occur inside  are considered as free variables of sort . 

\emph{Guards} are conjunctions of \emph{atomic guards}, which can compare array elements, indices, and counters to integer terms; the state variable can only be compared  to integer constants. 
\emph{Updates} are lists of \emph{atomic updates}; they can increment or decrease counters by a constant, assign a constant to the state variable, \Skip, i.e. perform no updates, or execute a \Break\ statement, which terminates the \Fold\ at the current position.  Counter or state updates define a function .
Guards and updates translate into logical formulas that either constraint the current variable values, or relate the current and the next-state (primed) variable values in the obvious way; we denote this translation by . E.g.,
the update  defines the formula . 

\begin{table}[t]

\vspace*{2mm}
\caption{Syntax of AFL.}
\label{tab:syntax}
\end{table}


We require that guards of all branches are mutually exclusive. There is an implicit ``catch-all'' branch with the \Break\ statement, whose guard evaluates to  exactly when guards of all other  branches evaluate to . We also require that each branch contains at most one update for each implicit variable.

We restrict the control flow in functions, which is defined by state variable . Notice that  is syntactically finite state. Thus, given a set of function branches , we define an edge-labeled control flow graph , where:
\begin{itemize}
\item states ;
\item edges ;
\item  is the labeling of edges with the set of formulas  and  for each guard or update which occurs in the same branch.
\end{itemize}
We require that edges in the strongly-connected components of  are labeled with counter updates that are, for each counter, all non-decreasing, or all non-increasing. Thus,  is a DAG of SCCs, where counters within each SCC behave in a monotonic way. We use this restriction to derive from  a reversal-bounded counter machine (see Definition~\ref{def:fscm}).

The presented syntax is minimal and can be extended with convenience functions and predicates such as  in the usual way. We allow to use  to denote the absence of constraints: this is useful for vector notation. We replace each  in the formula with a unique unconstrained variable.




\subsection{Semantics}


\begin{table}[t]

\vspace*{2mm}
\caption{Semantics of AFL}
\label{tab:semantics}
\end{table}


For a given AFL formula , we denote the sets of free variables of  of sort  and  by  and , respectively. All free variables are implicitly existentially quantified. For functions of sort , we denote by  the set of their implicit variables .


The treatment of array writes and reads in the quantifier-free array theory is standard \cite{QuantifierFreeArrays}, and we do not elaborate on it here.
Array equalities partition the set of array variables into equivalence classes; all other constraints are then translated into constraints over a representative of the corresponding equivalence class.

An \emph{interpretation} for AFL is a tuple , where
 assigns each integer variable an integer, and
 assigns each array variable a finite sequence of integers.



The semantics of an AFL term  under the given interpretation  is defined by the \emph{evaluation} . 
Terms that constitute functions are evaluated in the additional \emph{context} . For a function  of sort ,  maps internal variables of  to integers. 
The evaluation of Presburger, Boolean, and array terms is standard; the remaining ones are shown in Table~\ref{tab:semantics}. We give some explanations here (the remaining semantic rules are self-explanatory):
\begin{enumerate}
\item Vector equality resolves to a conjunction of equalities between components.
\item A \Fold\ term evaluates in the initial context that is defined by the given initial vector of counters , and assigns 0 to state variable .
\item A contextual \Fold\ term checks whether the array index is out of bounds, or a  statement is executed in the current context (this is the only way for  to contain ). If yes, \Fold\ terminates, and returns the current vector . Otherwise \Fold\ continues with the updated vector and context.
\item If an update  for some variable  is present in the function evaluation, then it is applied. Otherwise, the old variable value is preserved.
\item An evaluation of a function, represented by a list of branches, is a union of updates from its branch evaluations. Index  is always incremented by 1.
\item A guarded command evaluates to its update if its guard evaluates to \var{true}.
\item A comparison over an internal variable evaluates it in the context , and the comparison term is evaluated in the interpretation .
\end{enumerate}






















