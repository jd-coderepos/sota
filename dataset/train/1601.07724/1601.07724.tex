\documentclass{CSML}

\def\dOi{12(2:6)2016}
\lmcsheading {\dOi}
{1--28}
{}
{}
{Jul.~\phantom03, 2014}
{Jun.~13, 2016}
{}

\ACMCCS{[{\bf Theory of computation}]: Logic---Automated reasoning;
  Formalisms---Rewrite systems}
\subjclass{F.4.1. Mathematical Logic (Mechanical theorem proving); F.4.2. Grammars and Other Rewriting Systems (Parsing) }

\pdfoutput=1

\usepackage{amsmath}
\usepackage{url}
\urlstyle{rm}
\usepackage[utf8]{inputenc}
\usepackage{latexsym}
\usepackage{stmaryrd}
\usepackage{bussproofs}
\EnableBpAbbreviations
\usepackage{tabularx}
\usepackage[sectionbib]{natbib}
\usepackage{multirow}
\usepackage{hyperref}



\newcommand{\comment}[1]{}




\newtheorem{theorem}{Theorem}
\numberwithin{theorem}{section}
\newtheorem{lemma}[theorem]{Lemma}
\newtheorem{corollary}[theorem]{Corollary}
\newtheorem{axiom}[theorem]{Axiom}
\newtheorem{definition}[theorem]{Definition}
\newtheorem{remark}[theorem]{Remark}
\newtheorem{example}[theorem]{Example}


\newenvironment{pcode}
{\par\noindent\advance\leftskip\mathindent\par\noindent}

\setlength{\marginparwidth}{2.5cm}
\newcommand{\todo}[2][?]{\marginpar{\raggedright \tiny \textbf{TODO (by #1):} #2}}
\newcommand{\done}[2][?]{}

\DeclareUnicodeCharacter{00A0}{~} \DeclareUnicodeCharacter{00A7}{\S} \DeclareUnicodeCharacter{00AC}{\ensuremath{\neg}} \DeclareUnicodeCharacter{00B0}{^{\circ}} \DeclareUnicodeCharacter{00B1}{^1}
\DeclareUnicodeCharacter{00B2}{^2} \DeclareUnicodeCharacter{00B7}{\ensuremath{\cdot}} \DeclareUnicodeCharacter{00B9}{\textsuperscript{l}} \DeclareUnicodeCharacter{00D7}{\ensuremath{\times}} \DeclareUnicodeCharacter{00F7}{\ensuremath{\div}} \DeclareUnicodeCharacter{02B3}{\ensuremath{{^r}}} \DeclareUnicodeCharacter{02E1}{\ensuremath{{^l}}} \DeclareUnicodeCharacter{0393}{\ensuremath{\Gamma}} \DeclareUnicodeCharacter{0394}{\ensuremath{\Delta}} \DeclareUnicodeCharacter{0397}{\ensuremath{\textrm{H}}} \DeclareUnicodeCharacter{0398}{\ensuremath{\Theta}} \DeclareUnicodeCharacter{039B}{\ensuremath{\Lambda}} \DeclareUnicodeCharacter{039E}{\ensuremath{\Xi}} \DeclareUnicodeCharacter{03A3}{\ensuremath{\Sigma}} \DeclareUnicodeCharacter{03A6}{\ensuremath{\Phi}} \DeclareUnicodeCharacter{03A8}{\ensuremath{\Psi}} \DeclareUnicodeCharacter{03A9}{\ensuremath{\Omega}} \DeclareUnicodeCharacter{03B1}{\ensuremath{\mathnormal{\alpha}}} \DeclareUnicodeCharacter{03B2}{\ensuremath{\beta}} \DeclareUnicodeCharacter{03B3}{\ensuremath{\mathnormal{\gamma}}} \DeclareUnicodeCharacter{03B4}{\ensuremath{\mathnormal{\delta}}} \DeclareUnicodeCharacter{03B5}{\ensuremath{\mathnormal{\varepsilon}}} \DeclareUnicodeCharacter{03B6}{\ensuremath{\mathnormal{\zeta}}} \DeclareUnicodeCharacter{03B7}{\ensuremath{\mathnormal{\eta}}} \DeclareUnicodeCharacter{03B8}{\ensuremath{\mathnormal{\theta}}} \DeclareUnicodeCharacter{03B9}{\ensuremath{\mathnormal{\iota}}} \DeclareUnicodeCharacter{03BA}{\ensuremath{\mathnormal{\kappa}}} \DeclareUnicodeCharacter{03BB}{\ensuremath{\mathnormal{\lambda}}} \DeclareUnicodeCharacter{03BC}{\ensuremath{\mathnormal{\mu}}} \DeclareUnicodeCharacter{03BD}{\ensuremath{\mathnormal{\mu}}} \DeclareUnicodeCharacter{03BE}{\ensuremath{\mathnormal{\xi}}} \DeclareUnicodeCharacter{03C0}{\ensuremath{\mathnormal{\pi}}} \DeclareUnicodeCharacter{03C1}{\ensuremath{\mathnormal{\rho}}} \DeclareUnicodeCharacter{03C3}{\ensuremath{\mathnormal{\sigma}}} \DeclareUnicodeCharacter{03C4}{\ensuremath{\mathnormal{\tau}}} \DeclareUnicodeCharacter{03C6}{\ensuremath{\mathnormal{\varphi}}} \DeclareUnicodeCharacter{03C7}{\ensuremath{\mathnormal{\chi}}} \DeclareUnicodeCharacter{03C8}{\ensuremath{\mathnormal{\psi}}} \DeclareUnicodeCharacter{03C9}{\ensuremath{\mathnormal{\omega}}} \DeclareUnicodeCharacter{03D5}{\ensuremath{\mathnormal{\phi}}} \DeclareUnicodeCharacter{03F5}{\ensuremath{\mathnormal{\epsilon}}} \DeclareUnicodeCharacter{10627}{\ensuremath{\lbana}}
\DeclareUnicodeCharacter{10628}{\ensuremath{\rbana}}
\DeclareUnicodeCharacter{1D62}{_i} \DeclareUnicodeCharacter{2020}{\ensuremath{\dagger}} \DeclareUnicodeCharacter{2022}{\ensuremath{\text{\textbullet}}} \DeclareUnicodeCharacter{2026}{\ensuremath{\ldots}}
\DeclareUnicodeCharacter{202F}{{\,}}
\DeclareUnicodeCharacter{2032}{\ensuremath{^\prime}}  \DeclareUnicodeCharacter{2033}{\ensuremath{^\second}} \DeclareUnicodeCharacter{2034}{\ensuremath{^\third}}  \DeclareUnicodeCharacter{2070}{\ensuremath{^0}} \DeclareUnicodeCharacter{2071}{\ensuremath{^1}}
\DeclareUnicodeCharacter{2072}{\ensuremath{^2}}
\DeclareUnicodeCharacter{2073}{\ensuremath{^3}}
\DeclareUnicodeCharacter{2074}{\ensuremath{^4}}
\DeclareUnicodeCharacter{2075}{\ensuremath{^5}}
\DeclareUnicodeCharacter{2076}{\ensuremath{^6}}
\DeclareUnicodeCharacter{2077}{\ensuremath{^7}}
\DeclareUnicodeCharacter{2078}{\ensuremath{^8}}
\DeclareUnicodeCharacter{2079}{\ensuremath{^9}}
\DeclareUnicodeCharacter{207A}{\ensuremath{^+}} \DeclareUnicodeCharacter{207B}{\ensuremath{^-}} \DeclareUnicodeCharacter{2080}{\ensuremath{{}_0}} \DeclareUnicodeCharacter{2081}{\ensuremath{{}_1}}
\DeclareUnicodeCharacter{2082}{\ensuremath{{}_2}}
\DeclareUnicodeCharacter{2083}{\ensuremath{{}_3}}
\DeclareUnicodeCharacter{2084}{\ensuremath{{}_4}}
\DeclareUnicodeCharacter{2085}{\ensuremath{{}_5}}
\DeclareUnicodeCharacter{2086}{\ensuremath{{}_6}}
\DeclareUnicodeCharacter{2087}{\ensuremath{{}_7}}
\DeclareUnicodeCharacter{2088}{\ensuremath{{}_8}}
\DeclareUnicodeCharacter{2089}{\ensuremath{{}_9}}
\DeclareUnicodeCharacter{2096}{\ensuremath{{}_k}} \DeclareUnicodeCharacter{208A}{\ensuremath{_+}} \DeclareUnicodeCharacter{208B}{\ensuremath{_-}} \DeclareUnicodeCharacter{2099}{\ensuremath{_n}} \DeclareUnicodeCharacter{2102}{\ensuremath{\mathbb{C}}} \DeclareUnicodeCharacter{2115}{\ensuremath{\mathbb{N}}} \DeclareUnicodeCharacter{211D}{\ensuremath{\mathbb{R}}} \DeclareUnicodeCharacter{2124}{\ensuremath{\mathbb{Z}}} \DeclareUnicodeCharacter{214B}{\ensuremath{\parr}} \DeclareUnicodeCharacter{2190}{\ensuremath{\leftarrow}} \DeclareUnicodeCharacter{2191}{\ensuremath{\uparrow}} \DeclareUnicodeCharacter{2192}{\ensuremath{\rightarrow}} \DeclareUnicodeCharacter{2193}{\ensuremath{\downarrow}} \DeclareUnicodeCharacter{2194}{\ensuremath{\leftrightarrow}} \DeclareUnicodeCharacter{2196}{\nwarrow} \DeclareUnicodeCharacter{2197}{\nearrow} \DeclareUnicodeCharacter{219D}{\ensuremath{\leadsto}} \DeclareUnicodeCharacter{21A6}{\ensuremath{\mapsto}} \DeclareUnicodeCharacter{21C6}{\ensuremath{\leftrightarrows}} \DeclareUnicodeCharacter{21D0}{\ensuremath{\Leftarrow}} \DeclareUnicodeCharacter{21D2}{\ensuremath{\Rightarrow}} \DeclareUnicodeCharacter{21D4}{\ensuremath{\Leftrightarrow}} \DeclareUnicodeCharacter{2200}{\ensuremath{\forall}} \DeclareUnicodeCharacter{2203}{\ensuremath{\exists}} \DeclareUnicodeCharacter{2205}{\ensuremath{\varnothing}} \DeclareUnicodeCharacter{2208}{\ensuremath{\in}} \DeclareUnicodeCharacter{2209}{\ensuremath{\not\in}} \DeclareUnicodeCharacter{220B}{\ensuremath{\ni}}
\DeclareUnicodeCharacter{220E}{\ensuremath{\qed}} \DeclareUnicodeCharacter{2211}{\ensuremath{\sum}}\DeclareUnicodeCharacter{2215}{\mathbb{N}} \DeclareUnicodeCharacter{2217}{\ensuremath{\ast}} \DeclareUnicodeCharacter{2218}{\ensuremath{\circ}} \DeclareUnicodeCharacter{2219}{\ensuremath{\bullet}} \DeclareUnicodeCharacter{221E}{\ensuremath{\infty}} \DeclareUnicodeCharacter{2223}{\ensuremath{\mid}} \DeclareUnicodeCharacter{2227}{\wedge}\DeclareUnicodeCharacter{2228}{\vee}\DeclareUnicodeCharacter{2229}{\ensuremath{\cap}} \DeclareUnicodeCharacter{222A}{\ensuremath{\cup}} \DeclareUnicodeCharacter{2237}{::} \DeclareUnicodeCharacter{223C}{\ensuremath{\sim}} \DeclareUnicodeCharacter{2243}{\ensuremath{\simeq}} \DeclareUnicodeCharacter{2245}{\ensuremath{\cong}} \DeclareUnicodeCharacter{2248}{\ensuremath{\approx}} \DeclareUnicodeCharacter{225C}{\ensuremath{\stackrel{\scriptscriptstyle {\triangle}}{=}}} \DeclareUnicodeCharacter{225D}{\ensuremath{\stackrel{\scriptscriptstyle {def}}{=}}} \DeclareUnicodeCharacter{225F}{\ensuremath{\stackrel{\scriptscriptstyle ?}{=}}} \DeclareUnicodeCharacter{2260}{\ensuremath{\neq}}\DeclareUnicodeCharacter{2261}{\ensuremath{\equiv}}\DeclareUnicodeCharacter{2264}{\ensuremath{\le}} \DeclareUnicodeCharacter{2265}{\ensuremath{\ge}} \DeclareUnicodeCharacter{226B}{\ensuremath{\gg}} \DeclareUnicodeCharacter{227A}{\ensuremath{\prec}} \DeclareUnicodeCharacter{2282}{\ensuremath{\subset}} \DeclareUnicodeCharacter{2283}{\ensuremath{\supset}} \DeclareUnicodeCharacter{2286}{\ensuremath{\subseteq}} \DeclareUnicodeCharacter{2287}{\ensuremath{\supseteq}} \DeclareUnicodeCharacter{2293}{\ensuremath{\sqcup}} \DeclareUnicodeCharacter{2293}{\sqcap} \DeclareUnicodeCharacter{2294}{\sqcup} \DeclareUnicodeCharacter{2295}{\ensuremath{\oplus}} \DeclareUnicodeCharacter{2297}{\ensuremath{\otimes}} \DeclareUnicodeCharacter{22A1}{\ensuremath{\boxdot}} \DeclareUnicodeCharacter{22A2}{\ensuremath{\vdash}}
\DeclareUnicodeCharacter{22A4}{\ensuremath{\top}} \DeclareUnicodeCharacter{22A5}{\ensuremath{\bot}} \DeclareUnicodeCharacter{22A7}{\models} \DeclareUnicodeCharacter{22A8}{\models} \DeclareUnicodeCharacter{22A9}{\Vdash} \DeclareUnicodeCharacter{22B8}{\ensuremath{\multimap}} \DeclareUnicodeCharacter{22C3}{\ensuremath{\bigcup}} \DeclareUnicodeCharacter{22C4}{\ensuremath{\diamond}} \DeclareUnicodeCharacter{22C6}{\ensuremath{\star}}
\DeclareUnicodeCharacter{22EE}{\ensuremath{\vdots}} \DeclareUnicodeCharacter{22EF}{\ensuremath{\cdots}} \DeclareUnicodeCharacter{2308}{\ensuremath{\lceil}}
\DeclareUnicodeCharacter{2309}{\ensuremath{\rceil}}
\DeclareUnicodeCharacter{230A}{\ensuremath{\lfloor}}
\DeclareUnicodeCharacter{230B}{\ensuremath{\rfloor}}
\DeclareUnicodeCharacter{25A1}{\ensuremath{\square}} \DeclareUnicodeCharacter{25AF}{\mathop{\talloblong}} \DeclareUnicodeCharacter{25B7}{\ensuremath{\triangleright}} \DeclareUnicodeCharacter{25B9}{\ensuremath{\rhd}} \DeclareUnicodeCharacter{25C7}{\ensuremath{\diamond}} \DeclareUnicodeCharacter{2605}{\ensuremath{\star}}   \DeclareUnicodeCharacter{2713}{\ensuremath{\checkmark}} \DeclareUnicodeCharacter{27C2}{\ensuremath{^\bot}} \DeclareUnicodeCharacter{27E6}{\ensuremath{\llbracket}} \DeclareUnicodeCharacter{27E7}{\ensuremath{\rrbracket}} \DeclareUnicodeCharacter{27E8}{\ensuremath{\langle}} \DeclareUnicodeCharacter{27E9}{\ensuremath{\rangle}} \DeclareUnicodeCharacter{27F6}{{\longrightarrow}} \DeclareUnicodeCharacter{27F7}{{\longleftrightarrow}} \DeclareUnicodeCharacter{27F9}{\ensuremath{\Longrightarrow}} \DeclareUnicodeCharacter{29F8}{\ensuremath{\textfractionsolidus}} \DeclareUnicodeCharacter{2A02}{\ensuremath{\bigotimes}} \DeclareUnicodeCharacter{2A04}{\mathop{\dot{\cup}}} \DeclareUnicodeCharacter{2AEB}{\ensuremath{\perp\!\!\!\!\perp}} \DeclareUnicodeCharacter{2AFE}{\mathop{\talloblong}} \DeclareUnicodeCharacter{1D7D9}{\ensuremath{\mathds{1}}} \DeclareUnicodeCharacter{1D538}{\ensuremath{\mathds{A}}} \DeclareUnicodeCharacter{1D539}{\ensuremath{\mathds{B}}} \DeclareUnicodeCharacter{1D49F}{\ensuremath{\mathcal{D}}} \DeclareUnicodeCharacter{1D4A6}{\ensuremath{\mathcal{K}}} 


\DeclareUnicodeCharacter{220E}{\ensuremath{\blacksquare}} 

\DeclareMathAlphabet{\mathbfsf}{\encodingdefault}{\sfdefault}{bx}{n}

\newcommand{\kw}[1]{\ensuremath{\mathbf{#1}}}
\makeatletter
\@ifundefined{lhs2tex.lhs2tex.sty.read}{\@namedef{lhs2tex.lhs2tex.sty.read}{}\newcommand\SkipToFmtEnd{}\newcommand\EndFmtInput{}\long\def\SkipToFmtEnd#1\EndFmtInput{}}\SkipToFmtEnd

\newcommand\ReadOnlyOnce[1]{\@ifundefined{#1}{\@namedef{#1}{}}\SkipToFmtEnd}
\usepackage{amstext}
\usepackage{amssymb}
\usepackage{stmaryrd}
\DeclareFontFamily{OT1}{cmtex}{}
\DeclareFontShape{OT1}{cmtex}{m}{n}
  {<5><6><7><8>cmtex8
   <9>cmtex9
   <10><10.95><12><14.4><17.28><20.74><24.88>cmtex10}{}
\DeclareFontShape{OT1}{cmtex}{m}{it}
  {<-> ssub * cmtt/m/it}{}
\newcommand{\texfamily}{\fontfamily{cmtex}\selectfont}
\DeclareFontShape{OT1}{cmtt}{bx}{n}
  {<5><6><7><8>cmtt8
   <9>cmbtt9
   <10><10.95><12><14.4><17.28><20.74><24.88>cmbtt10}{}
\DeclareFontShape{OT1}{cmtex}{bx}{n}
  {<-> ssub * cmtt/bx/n}{}
\newcommand{\tex}[1]{\text{\texfamily#1}}	

\newcommand{\Sp}{\hskip.33334em\relax}


\newcommand{\Conid}[1]{\mathit{#1}}
\newcommand{\Varid}[1]{\mathit{#1}}
\newcommand{\anonymous}{\kern0.06em \vbox{\hrule\@width.5em}}
\newcommand{\plus}{\mathbin{+\!\!\!+}}
\newcommand{\bind}{\mathbin{>\!\!\!>\mkern-6.7mu=}}
\newcommand{\rbind}{\mathbin{=\mkern-6.7mu<\!\!\!<}}\newcommand{\sequ}{\mathbin{>\!\!\!>}}
\renewcommand{\leq}{\leqslant}
\renewcommand{\geq}{\geqslant}
\usepackage{polytable}

\@ifundefined{mathindent}{\newdimen\mathindent\mathindent\leftmargini}{}

\def\resethooks{\global\let\SaveRestoreHook\empty
  \global\let\ColumnHook\empty}
\newcommand*{\savecolumns}[1][default]{\g@addto@macro\SaveRestoreHook{\savecolumns[#1]}}
\newcommand*{\restorecolumns}[1][default]{\g@addto@macro\SaveRestoreHook{\restorecolumns[#1]}}
\newcommand*{\aligncolumn}[2]{\g@addto@macro\ColumnHook{\column{#1}{#2}}}

\resethooks

\newcommand{\onelinecommentchars}{\quad-{}- }
\newcommand{\commentbeginchars}{\enskip\{-}
\newcommand{\commentendchars}{-\}\enskip}

\newcommand{\visiblecomments}{\let\onelinecomment=\onelinecommentchars
  \let\commentbegin=\commentbeginchars
  \let\commentend=\commentendchars}

\newcommand{\invisiblecomments}{\let\onelinecomment=\empty
  \let\commentbegin=\empty
  \let\commentend=\empty}

\visiblecomments

\newlength{\blanklineskip}
\setlength{\blanklineskip}{0.66084ex}

\newcommand{\hsindent}[1]{\quad}\let\hspre\empty
\let\hspost\empty
\newcommand{\NB}{\textbf{NB}}
\newcommand{\Todo}[1]{\textbf{To do:}~#1}

\EndFmtInput
\makeatother
\ReadOnlyOnce{polycode.fmt}\makeatletter

\newcommand{\hsnewpar}[1]{{\parskip=0pt\parindent=0pt\par\vskip #1\noindent}}

\newcommand{\hscodestyle}{}



\newcommand{\sethscode}[1]{\expandafter\let\expandafter\hscode\csname #1\endcsname
   \expandafter\let\expandafter\endhscode\csname end#1\endcsname}



\newenvironment{compathscode}{\par\noindent
   \advance\leftskip\mathindent
   \hscodestyle
   \let\\=\@normalcr
   \let\hspre\pboxed}{\endpboxed\let\hspost\pboxed}{\endpboxed\parray}{\endparray\parray}{\endparray\pboxed}{\endpboxed\def\column##1##2{}\let\>\undefined\let\<\undefined\let\\\undefined
   \newcommand\>[1][]{}\newcommand\<[1][]{}\newcommand\ }

\newcommand{\inlinehs}{\sethscode{inlinehscode}}



\newenvironment{joincode}{\let\orighscode=\hscode
   \let\origendhscode=\endhscode
   \def\endhscode{\def\hscode{\endgroup\def\@currenvir{hscode}\\}\begingroup}
\orighscode\def\hscode{\endgroup\def\@currenvir{hscode}}}{\origendhscode
   \global\let\hscode=\orighscode
   \global\let\endhscode=\origendhscode}

\makeatother
\EndFmtInput
\ReadOnlyOnce{jfpcompat.fmt}\makeatletter
\def\@authortable{\leavevmode \hbox \bgroup O(n^3)W\mathcal G = (N, \Sigma, P, S)\mathcal G'N\SigmaPSANS ::= \epsilonw0,+,\cdot\sing{}0,+,\cdot\powerset N(\cdot)\sing{}CA ∈ C_{ij}A \generates[*] w[i..j]\sing{}\initial wj ≠ i+1W = \initial w(i,i+1)W(0,n)WW^+CCW(·)(·)CABijklXYZ(r,c)rcrcwA \generatestrans w[i..j]B \generatestrans w[k..l]Z ::= X YI(w)w(\cdot)0+0(\cdot)(\cdot)(+)(+)sC C = W + C · C C0,+,·\_^+WC = \closure WWC = W = 0WCCABWA'B'A' = \closure AB' = \closure BX'A_{11}A_{22}A_{12}B_{12}B_{11}B_{22}#1_{21}#2_{11}#3_{22}Y_{12}VABYX = V(A,Y,B)VAYBX = V(A,Y,B)X = Y + A·X + X·BVABYABXX(+)VXVX_{21}X_{11}X_{22}X_{12}O(\log^2 n)\{ a^n b^n c^n \mid n ∈ ℕ\}x → ax + bx = a{∗~} b$ is a fixpoint''.
It appears that our linear equation \ensuremath{\Varid{x}\;\mathrel{=}\;\Varid{y}\;\Varid{+}\;\Varid{a}\;\Varid{x}\;\Varid{+}\;\Varid{x}\;\Varid{b}} in
\fref{sec:completion} is the natural generalisation of the affine map
fixed point to the case of non-commutative algebra and that (the
corner case \ensuremath{\Conid{V}} of) Valiant's algorithm computes this fixed point for
upper triangular matrices.
Future work includes exploring this relation in more detail and
perhaps generalise out development to arbitrary (non-triangular)
matrices.

\subsection{Sparse matrices}
The efficiency of Valiant's algorithm in the average case relies on
using sparse matrices \citep{bernardy_efficient_2015}.
The above proof does not deal with sparseness.
Yet, it is straightforward to support sparseness as outlined by
\citet{bernardy_efficient_2015}:
one needs to change the \ensuremath{\Conid{U}} type to be a disjunction between the
2-by-2 case and the empty matrix case.



\newcommand{\HREF}[2]{\href{#2}{#1}} \bibliographystyle{abbrvnat}
\begin{thebibliography}{21}
\providecommand{\natexlab}[1]{#1}
\providecommand{\url}[1]{\texttt{#1}}
\expandafter\ifx\csname urlstyle\endcsname\relax
  \providecommand{\doi}[1]{doi: #1}\else
  \providecommand{\doi}{doi: \begingroup \urlstyle{rm}\Url}\fi

\bibitem[Bernardy and Claessen(2013)]{bernardy_efficient_2013}
J.-P. Bernardy and K.~Claessen.
\newblock Efficient divide-and-conquer parsing of practical context-free
  languages.
\newblock In \emph{Proc. of ICFP 2013}, pages 111--122, 2013.

\bibitem[Bernardy and Claessen(2015)]{bernardy_efficient_2015}
J.-P. Bernardy and K.~Claessen.
\newblock Efficient parallel and incremental parsing of practical context-free
  languages.
\newblock \emph{J. of Funct. Prog.}, 25, 2015.
\newblock ISSN 1469-7653.
\newblock \doi{10.1017/S0956796815000131}.

\bibitem[Bird and {de Moor}(1997)]{birddemoor96}
R.~Bird and O.~{de Moor}.
\newblock \emph{Algebra of Programming}, volume 100 of \emph{International
  Series in Computer Science}.
\newblock Prentice-Hall International, 1997.

\bibitem[{Bååth Sjöblom}(2013)]{bth_sjblom_agda_2013}
T.~{Bååth Sjöblom}.
\newblock \emph{An {Agda} proof of the correctness of {Valiant}'s algorithm for
  context free parsing}.
\newblock {MSc} thesis, Chalmers University of Tech., 2013.

\bibitem[Chomsky(1957)]{chomsky_syntactic_1957}
N.~Chomsky.
\newblock \emph{Syntactic Structures}.
\newblock Mouton de Gruyter, 1957.

\bibitem[Chomsky(1959)]{chomsky_certain_1959}
N.~Chomsky.
\newblock On certain formal properties of grammars.
\newblock \emph{Information and control}, 2\penalty0 (2):\penalty0 137--167,
  1959.

\bibitem[Coquand and Siles(2011)]{coquand_decision_2011}
T.~Coquand and V.~Siles.
\newblock A decision procedure for regular expression equivalence in type
  theory.
\newblock In \emph{Certified Programs and Proofs}, pages 119--134. Springer,
  2011.

\bibitem[Danielsson(2010)]{danielsson_total_2010}
N.~A. Danielsson.
\newblock Total parser combinators.
\newblock In \emph{Proc. of ICFP 2010}, ICFP '10, pages 285--296. ACM, 2010.

\bibitem[Danielsson and {The Agda Team}(2013)]{danielsson_agda_2013}
N.~A. Danielsson and {The Agda Team}.
\newblock The {Agda} standard library, version 0.7, 2013.

\bibitem[Dolan(2013)]{dolan_fun_2013}
S.~Dolan.
\newblock Fun with semirings: A funct. pearl on the abuse of linear algebra.
\newblock In \emph{Proc. of the 18th {ACM} {SIGPLAN} International Conf. on
  Funct. Prog.}, ICFP '13, pages 101--110. ACM, 2013.

\bibitem[Firsov and Uustalu(2013)]{firsov_certified_2013}
D.~Firsov and T.~Uustalu.
\newblock Certified parsing of regular languages.
\newblock In \emph{Certified Programs and Proofs}, pages 98--113. Springer,
  2013.

\bibitem[Firsov and Uustalu(2014)]{firsov_certified_2014}
D.~Firsov and T.~Uustalu.
\newblock Certified {CYK} parsing of context-free languages.
\newblock \emph{J. Log. Algebr. Meth. Program.}, 83\penalty0 (5-6):\penalty0
  459--468, 2014.

\bibitem[Jourdan et~al.(2012)Jourdan, Pottier, and
  Leroy]{jourdan_validating_2012}
J.~Jourdan, F.~Pottier, and X.~Leroy.
\newblock Validating {LR(1)} parsers.
\newblock In \emph{{ESOP} 2012}, pages 397--416, 2012.

\bibitem[Lange and Lei{\ss}(2009)]{lange_cnf_2009}
M.~Lange and H.~Lei{\ss}.
\newblock To {CNF} or not to {CNF}? an efficient yet presentable version of the
  {CYK} algorithm.
\newblock \emph{Informatica Didactica (8)(2008--2010)}, 2009.

\bibitem[Lee(2002)]{lee_fast_2002}
L.~Lee.
\newblock Fast context-free grammar parsing requires fast boolean matrix
  multiplication.
\newblock \emph{J. of the ACM (JACM)}, 49\penalty0 (1):\penalty0 1--15, 2002.

\bibitem[Mu et~al.(2008)Mu, Ko, and Jansson]{mukojansson08:mpc:dcc}
S.-C. Mu, H.-S. Ko, and P.~Jansson.
\newblock Algebra of programming using dependent types.
\newblock In \emph{Mathematics of Program Construction}, volume 5133/2008 of
  \emph{LNCS}, pages 268--283. Springer, 2008.
\newblock \doi{10.1007/978-3-540-70594-9_15}.

\bibitem[Mu et~al.(2009)Mu, Ko, and Jansson]{MuKoJansson2009AoPA}
S.-C. Mu, H.-S. Ko, and P.~Jansson.
\newblock Algebra of programming in {Agda}: dependent types for relational
  program derivation.
\newblock \emph{J. Funct. Program.}, 19:\penalty0 545--579, 2009.
\newblock \doi{10.1017/S0956796809007345}.

\bibitem[Norell(2007)]{norell_practical_2007}
U.~Norell.
\newblock \emph{Towards a practical programming language based on dependent
  type theory}.
\newblock {PhD} thesis, Chalmers Tekniska Högskola, 2007.

\bibitem[Okhotin(2014)]{okhotin_parsing_2014}
A.~Okhotin.
\newblock Parsing by matrix multiplication generalized to boolean grammars.
\newblock \emph{Theor. Comp. Sci.}, 516\penalty0 (0):\penalty0 101 -- 120,
  2014.

\bibitem[Ridge(2014)]{ridge_simple_2014}
T.~Ridge.
\newblock Simple, efficient, sound and complete combinator parsing for all
  context-free grammars, using an oracle.
\newblock In \emph{Soft. Language Engineering - 7th International Conf., {SLE}
  2014, V{\"{a}}ster{\aa}s, Sweden, September 15-16, 2014. Proc.}, pages
  261--281, 2014.

\bibitem[Valiant(1975)]{valiant_general_1975}
L.~Valiant.
\newblock General context-free recognition in less than cubic time.
\newblock \emph{J. of computer and system sciences}, 10\penalty0 (2):\penalty0
  308--314, 1975.

\end{thebibliography}


\end{document}
