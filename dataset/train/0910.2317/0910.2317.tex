\documentclass[12pt]{article}

\usepackage{amsmath}
\usepackage{amsfonts}
\usepackage{amssymb}
\usepackage{latexsym}
\usepackage{ifthen}
\usepackage{theorem}
\usepackage{graphicx}

\theorembodyfont{\rmfamily}
\newtheorem{prop}{Proposition}[section]
\newtheorem{lemma}[prop]{Lemma}
\newtheorem{theorem}[prop]{Theorem}
\newtheorem{cor}[prop]{Corollary}
\newtheorem{definition}[prop]{Definition}
\newtheorem{rem}[prop]{Remark}
\newtheorem{observ}[prop]{Observation}
\newtheorem{claim}[prop]{Claim}
\newtheorem{fact}[prop]{Fact}

\newcommand{\R}{{\mathbb R}}
\newcommand{\ra}{\rightarrow}
\newcommand{\mt}{\mapsto}

\newcommand{\X}{{\mathcal X}}

\DeclareMathOperator*{\supp}{supp}

\begin{document}

\title{An algorithm for computing cutpoints in finite metric spaces}

{\tiny
\author{\bf{Andreas Dress}\\
\small{CAS-MPG Partner Institute and Key Lab for Computational Biology (PICB)}\\
\small{Shanghai Institutes for Biological Sciences (SIBS)}\\
\small{Chinese Academy of Sciences (CAS)}\\
\small{320 Yue Yang Road, Shanghai 200031, P.\,R.\,China}\\
\small{email: andreas@picb.ac.cn}\\
\bf{Katharina T. Huber}\\
\small{University of East Anglia, School of Computing Sciences}\\
\small{Norwich, NR4 7TJ, UK}\\
\small{e-mail: Katharina.Huber@cmp.uea.ac.uk}\\ 
\bf{Jacobus Koolen}\\
\small{Department of Mathematics and Pohang Mathematics Institute}\\
\small{POSTECH}\\
\small{Pohang, South Korea}\\
\small{email: koolen@postech.ac.kr}\\ 
\bf{Vincent Moulton}\\
\small{University of East Anglia, School of Computing Sciences}\\
\small{Norwich, NR4 7TJ, UK}\\
\small{e-mail: vincent.moulton@cmp.uea.ac.uk}\\ 
\bf{Andreas Spillner}\\
\small{University of Greifswald, Department of Mathematics and Computer Science}\\
\small{17489 Greifswald, Germany}\\
\small{e-mail: andreas.spillner@uni-greifswald.de}\\
}}

\date{}

\maketitle

\newpage 

\begin{abstract}
The theory of the {\em tight span}, a  
cell complex 
that can be associated to every metric , 
offers a unifying view on existing approaches
for analyzing distance data, in particular 
for decomposing a
metric  
into a sum of simpler metrics
as well as for representing it
by certain specific edge-weighted graphs,
often referred to as 
{\em realizations} of .
Many of these approaches involve the explicit or implicit
computation of the so-called cutpoints of 
(the tight span of) ,
such as the algorithm for computing the
``building blocks'' of
optimal realizations of  recently presented by A.Hertz and S.Varone.
The main result of this paper is an algorithm for computing
the set of these cutpoints for a metric  on a finite
set with  elements in  time.
As a direct consequence, this improves the run time of the aforementioned
-algorithm by Hertz and Varone by ``three orders of magnitude''.
\end{abstract}

\noindent
\textbf{Keywords:} metric, cutpoint, realization, tight span, decomposition, block

\section{Introduction}
\label{section:introduction}


The decomposition of a given metric into simpler metrics 
(see e.g. \cite{deza:laurent:cuts:1997})
is a fundamental problem in classification featuring  applications in, for example, 
clustering (e.g. \cite{bryant:berry:clustering:2001}), 
``networking'' (e.g. \cite{chu-gar-gra-01a}), and 
phylogenetics (e.g. \cite{huson:bryant:networks2005}). 
The theory of the \emph{tight span} 
of a metric  defined on a set  \cite{isbell:six:theorems:1964,dress:tight:extensions:1984}
offers a unifying view on various existing approaches developed for 
this task. 
In this paper, we focus on decompositions of metrics  
defined on a finite set  that are induced by \emph{cutpoints} of , that is, maps  
such that  is disconnected. These decompositions
are closely related to  certain \emph{graph realizations} of ,
that is, connected edge-weighted graphs  with  
for which  holds for all  (where  denotes the shortest-path metric induced by  on .

To describe these graph realizations,
recall (see e.g.\,\cite{wes-96a})
that a vertex  in a  
graph  is called a 
\emph{cut vertex} (of ) if there exist vertices  with   such that every path from  to  in  passes through . 
Moreover, a maximal subset  
with the property that the induced graph 
has no cut vertex is called a \emph{block} of .
A graph realization  of  is called a \emph{block realization} of  if  is a {\em block graph}, i.e., every 
block of  is a clique, and every vertex in  has degree at least 3 and
is a cut vertex of . 
An example of a block realization is presented in Figure~\ref{figure:block:realization}(b).

\begin{figure}
\centering
\includegraphics[scale=1.0]{block_realization_example.eps}
\caption{(a) An example of a metric  on .
         (b) A block realization of : The vertices in the shaded
             region form a block and edge  is a bridge.
         (c) A block realization of the restriction  of  to
             the subset .}
\label{figure:block:realization}
\end{figure}

In a recent series of papers 
\cite{dre-hub-koo-08a,dress:huber:compatible:decompositions:2008,dress:huber:koolen:moulton:cut:points:2007},
it has been observed 
that a map  is a  cutpoint if and only
if the graph
 defined, for every , by 
 and   is disconnected (where, as usual, 
 denotes the \emph{support} of , that a map  in 

for which the graph  is disconnected 
must be contained in --- and, hence, a cutpoint of --- ,
and that a cutpoint  has a neighborhood that is homeomorphic to the open interval 
 if and only if   is the disjoint union of two cliques. 
As such maps are of little interest for constructing block realizations, 
we will focus our attention in particular to the set of those cutpoints, denoted by , 
for which either  holds or
 is not the disjoint union of two cliques.


In this paper, we present an algorithm with run time  to compute ,
where , improving the run time of the algorithm presented
in \cite{dress:virtual:cutpoints:2007}. Once the set  is available, 
it is straight-forward to compute a corresponding canonical 
block realization  of  in  time.
And, from that block realization it is then easy
to derive, for every block  of , a 
corresponding metric  on  that assigns, to any two elements
, the value  defined as the
total weight of those edges on any shortest path
from  to  in  that have both
end points in . For example in Figure~\ref{figure:block:realization}(b) 
the distance  between
 and  is 5 where  is the block in the
shaded region.

This yields a decomposition of  into 
a sum of metrics 
of the form 
 where  runs through the
blocks of  that can
be computed in  time. As a consequence, our algorithm improves 
the run time of the algorithm presented in 
\cite{hertz:varone:cutpoint:partition} 
that follows a 2-step approach: It computes
first those metrics 
 that correspond to blocks  with only 2 vertices, the so-called
\emph{bridges}, (see \cite{hertz:varone:bridge:partition} for
details) and then the remaining metrics  for the blocks  that
are not bridges.

Our paper is structured as follows. In the next section, we
introduce the concept of block splits and show how they can help in the computation
of . In Section~\ref{section:key:properties}, we establish the key properties
of block splits and cutpoints in  that we use in our new algorithm
for computing , and
we present this algorithm in Section~\ref{section:algorith:computing:cut:d}.


\section{Block splits}
\label{section:block:splits}


In this section, we present a key concept used
in our algorithm for computing the set ,
where  is the given metric on a finite set :
Recall that a \emph{split}  of  is a bipartition of 
into two non-empty subsets  and , also denoted by  or .
A split  of  is called a \emph{block split}
of  (relative to  if there exists a map 
 with 
such that  is the disjoint union of two cliques whose
vertex sets are  and , respectively. 
Note that the condition used in the definition of a block
split above is slightly stronger than the condition
given in \cite[p. 10]{imrich:optimal:realizations:1984}. Also note
that block splits, although not given a specific name,
play an important role in the algorithm for
computing bridges presented in \cite{hertz:varone:bridge:partition}.
The set of block splits of  induced by  is  denoted by . 
In the following, we will also often simply write 
 for , .

Our first goal is to establish a property of block splits
that allows to efficiently check whether a given split
of  is a block split. To this end, we first recall the 
following well-known fact.

\begin{lemma}
\label{lemma:bivariate:map}
Given two sets  and  and a bi-variate map  from the Cartesian product
 into the real numbers (or any Abelian group), there exist maps  and  with  for all  and  if and only if 
 holds for all  and  if and only if 
 holds for some fixed elements  and  and all  and .
\end{lemma}

\noindent\textsl{Proof}:
If  there exist maps  and  with , one clearly has  \ for all  and  while, conversely, if 
 holds for some fixed elements  and  and all  and , one has 
  for, e.g., the two maps  and 
. 
\hfill\\

Next, we define, for any map  and any subset  of , the \emph{virtual distance}  from  to  (relative to ) by
 

We will also write  rather that  in case  coincides with the so-called \emph{Kuratowski map}  
\cite{kuratowski:non:separable:metric:spaces:1935} associated with an element  defined by  for all .
Note that  holds for all  and  as above. Note also that, given a split  of  
with  for all  and , and any two elements  and , one has    

and that  has been dubbed the
\emph{isolation index} of  \cite{bandelt:dress:canonical:1992}. 

To illustrate the above definitions, 
note that, 
for the metric given in Figure~\ref{figure:block:realization}(a),
the split  is a block split with 
, , , , and  and, therefore, 
, for all  and , 
the weight of the edge  separating  from  in the corresponding block realization depicted
in Figure~\ref{figure:block:realization}(b).


More generally, we have

\begin{lemma}
\label{lemma:characterization:block:splits}
A split  of  
is a block split of  if and only 
if the isolation index  of  is positive and, choosing
arbitrary elements  and ,
 holds for all  and .
\end{lemma}

\noindent\textsl{Proof}:
Assume first that  is a block split. By the definition of a block split, 
there exists a map  for which
 is the disjoint union of two cliques whose
vertex sets are  and  and, therefore, we clearly have 
. Moreover, for the restrictions  and 
 of  to  and , respectively, we have 
 for all  and , and,
therefore,  must hold for all  and
 in view of Lemma~\ref{lemma:bivariate:map} applied to the bivariate
map .
In consequence, by Equation~(\ref{definition:isolation:index}),
we have  for all ,  and,
so, choosing any   and , we also have


Conversely, choosing arbitrary elements  and ,
if  holds for all  and  and the isolation index of  is positive,
then, in view of Lemma~\ref{lemma:bivariate:map},  
we may choose any two non-negative real numbers  with  and 
define the map 
 
by  for all  and 
 for all 
. In view of Equation~(\ref{definition:isolation:index}), we clearly have
 for all  
 and .
Moreover, we have  for all  
  as
 holds for all  
  and every  in view of the definition of  \big(indeed,  is one of the terms whose minimum, over all , coincides with \big), and we have   for all  
  if and only 
 
 holds as   implies  for all  with . By symmetry, we have also  for all  
  and  if and only if
 
 holds. Thus,  is a subset of , and it coincides with this set if and only if  holds. 
So,  must indeed be a block split, as required.
\hfill\\
 
It is also worth noting that, for every block split , every 
 with  (or, 
equivalently, with  for all  and  ) actually is of the 
form  for some 
 with : Indeed, 
in view of Equation~(\ref{definition:isolation:index}),
we have  for 
all   and  in this case, implying in particular that neither side 
changes once we replace  by any other element in  nor  by any other element 
in . So, choosing any fixed  and , we may put 
 and  in which case we have
 , 
 and  for all  and  .
Moreover, we have  in view of 
,
and, similarly, . 

In other words, given any block split , the set 

forms an straight line segment in  parametrized by the straight line segment
 in , and the two end points 
 (closer to ) and 
 (closer to ) must each be either cut points of  that do not have a
neighborhood that is homeomorphic to the open interval 
 or elements of the set  consisting of all Kuratowski maps 
for . Hence we have the following.

\begin{cor}
\label{corollary:endpoints:bridges}
For every block split  the maps 
 and  must be contained in the set 
. 
\end{cor}

\section{Key properties of  and }
\label{section:key:properties}


As we have seen in the previous section, it is sometimes helpful to consider
the bigger set  rather than . Since we can easily identify
those Kuratowski maps that are not in , 
we will now focus mainly on . The following lemma establishes the 
key properties of  and  that we will use in our algorithm 
to compute these sets recursively.

\begin{lemma}
\label{lemma:recursive:approach}
Let  be an arbitrary element of . 
Define 
and let  denote the restriction of  to .
Then the following assertions hold.
\begin{itemize}
\item[(i)]
If  is a block split of , then either  or 
the restriction  of  to  is a block split of .
\item[(ii)]
If  has the property that there is no
block split  of  with ,
then the restriction  of  to  is an element
of  and  holds.
\end{itemize}
\end{lemma}

\noindent\textsl{Proof}:
(i) Clearly, if  is a block split of  with , then  
is a split of , and there exists 
a map  such that  is the disjoint union of two cliques with
vertex sets  and  implying that the restriction  of   to  is in  and that 
is the disjoint union of two cliques with vertex sets  and , respectively.
This establishes (i).

To see that (ii) holds, suppose  and that there is no
block split  of  with . 
Clearly, the restriction  of  to  is in .
So, it remains to show that  is disconnected, but 
not the disjoint union of two cliques, which implies in particular
that .

Assume for a contradiction that  is connected or
the disjoint union of two cliques. We first note that this implies that
 has at least one connected component that is a clique.
To see this, observe that if  is connected,
then  has precisely two connected components, one of whom
consists of the single vertex , thus trivially forming a clique.
Similarly, if  is the disjoint union of two cliques,
then one of these cliques is also a connected component of .

Let  denote the vertex set of a connected component of 
that forms a clique. Note that this implies that .
Put  and . 
We want to show that  is a block split with , yielding
the required contradiction. To this end, choose arbitrary elements  
and . Since ,  cannot be the vertex set of a
clique in , and so there must exist two distinct elements
 with the property that , implying that
 holds. Since

clearly holds for all  and ,
we have, in  view of Equation~(\ref{definition:isolation:index})
and the definition of , 

and, therefore,  is indeed a block split.

It remains to show that . More specifically, we will show
that  holds. By the definition
of  and in view of the fact that  and
 holds, we have indeed
,
for every , and
,
for every , as claimed.
\hfill\\

We close this section with establishing bounds on the size of the sets 
 and  that we will use in the analysis
of the run time of our algorithm in Section~\ref{section:algorith:computing:cut:d}. 

\begin{lemma}
\label{lemma:number:cut:points}
Let  be a metric on a finite set  with  elements.
Then   and  holds.
\end{lemma}


\noindent\textsl{Proof}:
To establish the first claim, it suffices to
note that any two splits  
are \emph{compatible}, that is, at least one 
of the four intersections , , 
 and  is empty, since it
is well known that every set of pairwise compatible splits of
 contains at most  splits (see e.g. Proposition~2.1.3 and 
Theorem~3.1.4 in \cite{sem-ste-03a}). 
So, assume for a contradiction that there exist two splits 
and  in  that are not compatible. Then we can
choose arbitrary elements , ,
 and . By the definition
of a block split, there exist maps , , 
for which the graph  is the 
disjoint union of two cliques with vertex sets  and .
But then, by the definition of  and ,

holds, a contradiction. 

Next we show . Since, clearly, , it
suffices to show that . In \cite{dre-hub-koo-08a},
it is shown that there exists a block realization  of  such that
the cut vertices in  are in one-to-one correspondence with the elements
in . Moreover, the number of cut vertices in any graph
is well known to be less than the number of blocks of this graph (see e.g. \cite{har-pri-66a}).
Hence, it suffices to show that the number of blocks in  is at most . 
Yet, it has been shown in \cite{dress:huber:compatible:decompositions:2008} that
there is a canonical bijection from the set of blocks of  to a set  of
\emph{\!strongly compatible} partitions of , that is, of 
partitions such that there exist, for any two distinct partitions 
 and , two necessarily unique subsets  and  of  with 
 (generalizing the concept of compatibility
for splits to arbitrary partitions of . 
Therefore, it suffices to show that, for all , every set of pairwise compatible partitions of 
contains at most  partitions which we will establish 
by induction on the size of . Clearly, if  then there is only one partition of .

Now assume . If every partition in  is
a split of , then  must hold.
Otherwise, there exists a partition  that
contains at least three subsets of . For every , fix
an arbitrary element , define
 to be the set of the restrictions  of those partitions  
 with the property that there exists some  with , and note that any such partition  can be recovered from its restriction  as it must consist of all subsets  in that restriction that do not contain  and the complement of their union. Thus, it is not hard to see that, for every , any
two partitions of  in  are compatible,
that  holds, and that  holds for every . Hence, by induction,
 
as required.
\hfill


\section{The algorithm for computing }
\label{section:algorith:computing:cut:d}


In this section, we present our new algorithm for computing   
called \textsc{ComputeCutPoints}( which follows the recursive
approach suggested by Lemma~\ref{lemma:recursive:approach}.
This algorithm can be regarded as a speed-up of the algorithm for computing
cutpoints presented in \cite{dress:virtual:cutpoints:2007}, 
which, as mentioned in the introduction, also improves upon the
run time of the algorithm presented in \cite{hertz:varone:cutpoint:partition}.
In Figure~\ref{algorithm:cut:points}, we present a pseudocode
for this algorithm. Besides  the algorithm returns
the set  and the auxiliary set ,
which, for every split , contains
the 4-tuple , where
 and  are fixed elements that are
arbitrarily chosen during the course of the algorithm.

To illustrate how our algorithm 
computes , consider the metric  presented in
Figure~\ref{figure:block:realization}(a). Suppose in Line~3
of the pseudocode in Figure~\ref{algorithm:cut:points}, 
we select the element . Consider
the restriction  of  to the subset .
A block realization of  is presented in 
Figure~\ref{figure:block:realization}(c). It is easy to check
that the set of block splits of  is

Note that the splits in  are in one-to-one correspondence
with the edges of the block realization in Figure~\ref{figure:block:realization}(c).
The set  consists of the Kuratowski maps in  and
one additional map  with
, ,  and .
Note that this map corresponds to the cut vertex 
in Figure~\ref{figure:block:realization}(c) as 
equals the length of a shortest path from  to
 in the block realization for every .

Given  and , the algorithm first computes
the set  of block splits of  and
the auxiliary set  (Lines~6-21).
In our example it is easy to check that each of the splits
in  gives rise to precisely one split in 
, that is,

Next the set  is computed (Lines~22-27) by first
adding the maps  and  for every .
For the metric  in Figure~\ref{figure:block:realization}(a), this
yields, in addition to the Kuratowski maps ,  and ,
the 3 cutpoints
,  and ,
where  represents the
map  with .
Note that these cutpoints correspond to the 3 cut vertices in the block realization
of  in Figure~\ref{figure:block:realization}(b).
For our example, the computation of  is completed
by adding the Kuratowski maps  and  (Line~27).

\begin{figure}
\begin{tabbing}
XXX\= XX\= XX\=  XX\= XX\= XX\= XX\=  XXXXX\=  XXXXXXXXXXX\= \kill \\
{\large \textsc{ComputeCutPoints}()}\\
\rule{13cm}{0.5pt}\\
Input: \> \> \>a metric  on \\
Output: \> \> \>, , \\
\rule{13cm}{0.5pt}\\
1. \> \textbf{if} , \textbf{then} \textbf{return} ,
       and .\\
2. \> Initialize ,  and .\\
3. \> Select  arbitrarily.\\
4. \> Put , and let  denote the restriction of  to .\\
5. \> Compute recursively ,  and .\\
6. \> \textbf{for each}  \textbf{do}\\
7. \> \> Put  and .\\
8. \> \> Put ,  and extend  to .\\
9. \> \> Compute .\\
10. \> \> Compute .\\
11. \> \> \textbf{if}  is a block split of , \textbf{then}\\
12. \> \> \> Insert  into  and  into .\\
13. \> \> Put ,  and extend  to .\\
14. \> \> Compute .\\
15. \> \> Compute .\\
16. \> \> \textbf{if}  is a block split of , \textbf{then}\\ 
17. \> \> \> Insert  into  and  into .\\
18. \> Put ,  and select  arbitrarily.\\
19. \> Compute  and .\\
20. \> \textbf{if}  is a block split of , \textbf{then}\\
21. \> \> Insert  into  and  into .\\ 
22. \> \textbf{for each}  \textbf{do}\\
23. \> \> Insert  and  into .\\
24. \> \textbf{for each}  \textbf{do}\\
25. \> \> Extend  to  putting .\\
26. \> \> \textbf{if}  is a cutpoint of , 
         \textbf{then} insert  into .\\
27. \> \textbf{for each}  \textbf{do} insert  into .\\
28. \> \textbf{return} ,  and .
\end{tabbing}
\caption{Pseudocode for our algorithm for computing .}
\label{algorithm:cut:points}
\end{figure}


\begin{theorem}
\label{theorem:algorithm:cut:points}
Given a metric  on a set  with  elements, the
algorithm \textsc{ComputeCutPoints}( computes 
in  time.
\end{theorem}

\noindent\textsl{Proof}:
We first show that our algorithm is correct. To do this we use induction
on the size  of . Our induction hypothesis is that our
algorithm computes  and the set  of block splits of  correctly.
If , there is nothing to prove.
Now suppose that  holds. 
Let  be the element in  selected by our algorithm (Line~3), 
put , and let  denote
the restriction of  to  (Line~4).
By Lemma \ref{lemma:recursive:approach}(i), the set  of block splits of 
can be computed from the set  of block splits of .
By induction, the recursive call (Line~5) will correctly compute  and,
therefore, our algorithm will correctly compute  (Lines~6-21).
Similarly, by Corollary~\ref{corollary:endpoints:bridges}
and Lemma~\ref{lemma:recursive:approach}(ii), the set  can be computed
from  and . We have argued already that the computation
of  is correct and, again by induction, the recursive call (Line~5) 
will correctly compute . Hence, our algorithm will correctly 
compute  (Lines~22-27). 

We next show that our algorithm has run time .
We claim that an upper bound  on the run time
will satisfy the recurrence .
Using standard techniques for solving recurrences 
(see e.g. \cite{cormen:intro:algorithms:2001}),
this yields . So, it remains to show
that all operations except those performed in the recursive call (Line~5)
can be done in  time. 

We first focus on the computation of  from  (Lines~6-21).
Let  be an arbitrary split in . We can assume
that  and  are available 
from the 4-tuple .
We want to check whether the split 
 is a block split of  (Line~11).
By Lemma \ref{lemma:characterization:block:splits} it suffices to check
whether 
and  holds for all , ,
using  and .
Note that, since  is a block split of , it suffices to
check whether  holds for all , which can
be done in  time. 
Moreover, since 
 and

hold (Lines~9-10), we can also compute
 in  time. 

To summarize, whether  is a block split of  or not
can be checked in  time. Using completely similar arguments, it can
also be shown that checking whether  is a block split of  (Line~16)
can be done in  time. Note that, by Lemma~\ref{lemma:number:cut:points}, 
there are  block splits of .
Thus, our algorithm will perform  iterations of the loop in Line 6 
and each iteration is completed in  time, yielding  in total
for Lines~6-17.

To finish the computation of , we need to check whether
the split  is a block split of  (Lines~18-21). 
To do this, we fix  and choose an arbitrary .
Then, we compute  and ,
which can be done in  time, and check whether 
 holds.
We also check whether  holds for all ,
which can be done in  time. 
This finishes the analysis of the time needed to compute .

Next, we focus on the computation of  (Lines~22-27).
We use a data structure \textsc{Dic}
to store the elements in  computed so far. 
Since, by Lemma \ref{lemma:number:cut:points},
, the data structure \textsc{Dic} can be implemented in such a way
that inserting a single element of  into \textsc{Dic} and, later on, 
checking whether an element of  has
already been stored in \textsc{Dic} both takes  time, see e.g. \cite{gon-00a}.
Moreover, we assume that, for every ,
the connected components of the graph  have been computed
and the cliques among them have been marked. 

So, first consider an arbitrary block split .
If we have  and , then
we compute  along with the connected components of ,
marking the cliques among them, in  time for all . 
Next we consider the case that there exists some 
such that  and  (the following argument is completely 
analogous if  and .
Let  and  be the elements that we fixed
for  in the course of the algorithm and
let  and  be the maps in 
associated with the split .
Then we have 
 
for all  and

for all , since  clearly holds.
Hence, computing , the connected components of  
and marking the cliques among them can be done in  time, based on 
 and the connected components of .
Similarly, if  holds,
, the connected components of 
and the cliques among them can be computed in  time.
Otherwise, that is, if  holds,
the graph induced by 
on  is the disjoint union of two cliques
with vertex sets  and , respectively. 
To see this, note that
,
 
and 
holds for all  and .
But then, also in this case, the connected components of 
and the cliques among them can easily be computed in  time.

It remains to consider an arbitrary  (Lines 24-26).
Extending  to  (Line 25), that is, computing
 can be done in  time.
Recall that we assume that the connected components of the 
graph  and the cliques among them have been computed.
From this information, we can compute in  time
the connected components of  and determine which
of them are cliques. Hence the loop in Line~24 will take  time,
as required. Similarly, the loop in Line~27 will also take  time.
This finishes the analysis of the run time of our algorithm and thus the proof of
the theorem.
\hfill\\


\subsubsection*{Acknowledgments}
Authors Moulton and Spillner were supported by the 
Engineering and Physical Sciences Research Council 
[grant number EP/D068800/1].  
A.\,Dress thanks the Chinese Academy of Sciences, 
the Max-Planck-Gesellschaft, and the German BMBF for
their support, as well as the Warwick Institute for Advanced Study
where, during two wonderful weeks, the basic outline of this paper was
conceived. Huber and Koolen thank the Royal Society for their
support in the context of a International Joint Project
grant. Koolen was also partially supported by the Korea 
Research Foundation of the Korean Government under grant 
number KRF-2007-412-J02302. 
We would also like to thank the anonymous referees
for their helpful comments on earlier versions of
this paper.



\begin{thebibliography}{10}

\bibitem{bandelt:dress:canonical:1992}
H.~J. Bandelt and A.~Dress.
\newblock A canonical decomposition theory for metrics on a finite set.
\newblock {\em Advances in Mathematics}, 92:47--105, 1992.

\bibitem{bryant:berry:clustering:2001}
D.~Bryant and V.~Berry.
\newblock A structered family of clustering and tree construction methods.
\newblock {\em Advances in Applied Mathematics}, 27:705--732, 2001.

\bibitem{chu-gar-gra-01a}
F.~Chung, M.~Garrett, R.~Graham, and D.~Shallcross.
\newblock Distance realization problems with applications to internet
  tomography.
\newblock {\em Journal of Computer and System Sciences}, 63:432--448, 2001.

\bibitem{cormen:intro:algorithms:2001}
T.~H. Cormen, C.~E. Leiserson, R.~L. Rivest, and C.~Stein.
\newblock {\em Introduction to algorithms}.
\newblock MIT Press, 2001.

\bibitem{deza:laurent:cuts:1997}
M.~Deza and M.~Laurent.
\newblock {\em Geometry of Cuts and Metrics}.
\newblock Springer, 1997.

\bibitem{dress:tight:extensions:1984}
A.~Dress.
\newblock Trees, tight extensions of metric spaces, and the cohomological
  dimension of certain groups: A note on combinatorial properties of metric
  spaces.
\newblock {\em Advances in Mathematics}, 53:321--402, 1984.

\bibitem{dress:virtual:cutpoints:2007}
A.~Dress, K.~T. Huber, J.~Koolen, and V.~Moulton.
\newblock An algorithm for computing virtual cut points in finite metric
  spaces.
\newblock In {\em International Conference on Combinatorial Optimization and
  Applications (COCOA)}, volume 4616 of {\em LNCS}, pages 4--10. Springer,
  2007.

\bibitem{dre-hub-koo-08a}
A.~Dress, K.~T. Huber, J.~Koolen, and V.~Moulton.
\newblock Block realizations of finite metrics and the tight-span construction
  {I}: the embedding theorem.
\newblock {\em Applied Mathematics Letters}, 21:1306--1309, 2008.

\bibitem{dress:huber:compatible:decompositions:2008}
A.~Dress, K.~T. Huber, J.~Koolen, and V.~Moulton.
\newblock Compatible decompositions and block realizations of finite metrics.
\newblock {\em European Journal of Combinatorics}, 29:1617--1633, 2008.

\bibitem{dress:huber:koolen:moulton:cut:points:2007}
A.~Dress, K.~T. Huber, J.~Koolen, and V.~Moulton.
\newblock Cut points in metric spaces.
\newblock {\em Applied Mathematics Letters}, 21:545--548, 2008.

\bibitem{gon-00a}
T.~Gonzalez.
\newblock Simple algorithms for the on-line multidimensional dictionary and
  related problems.
\newblock {\em Algorithmica}, 28:255--267, 2000.

\bibitem{har-pri-66a}
F.~Harary and G.~Prins.
\newblock The block-cutpoint-tree of a graph.
\newblock {\em Publicationes Mathematicae Debrecen}, 13:103--107, 1966.

\bibitem{hertz:varone:bridge:partition}
A.~Hertz and S.~Varone.
\newblock The metric bridge partition problem.
\newblock {\em Journal of Classification}, 24:235--249, 2007.

\bibitem{hertz:varone:cutpoint:partition}
A.~Hertz and S.~Varone.
\newblock The metric cutpoint partition problem.
\newblock {\em Journal of Classification}, 25:159--175, 2008.

\bibitem{huson:bryant:networks2005}
D.~Huson and D.~Bryant.
\newblock Application of phylogenetic networks in evolutionary studies.
\newblock {\em Molecular Biology and Evolution}, 23:254--267, 2005.

\bibitem{imrich:optimal:realizations:1984}
W.~Imrich, J.~Simoes-Pereira, and C.~Zamfirescu.
\newblock On optimal embeddings of metrics in graphs.
\newblock {\em Journal of Combinatorial Theory, Series B}, 36:1--15, 1984.

\bibitem{isbell:six:theorems:1964}
J.~Isbell.
\newblock Six theorems about metric spaces.
\newblock {\em Commentarii Mathematici Helvetici}, 39:65--74, 1964.

\bibitem{kuratowski:non:separable:metric:spaces:1935}
C.~Kuratowski.
\newblock Quelques probl\`emes concernant les espaces m\'etriques
  non-s\'eperables.
\newblock {\em Fundamenta Mathematicae}, 25:534--545, 1935.

\bibitem{sem-ste-03a}
C.~Semple and M.~Steel.
\newblock {\em Phylogenetics}.
\newblock Oxford University Press, 2003.

\bibitem{wes-96a}
D.~West.
\newblock {\em Introduction to graph theory}.
\newblock Prentice Hall, 1996.

\end{thebibliography}


\end{document}
