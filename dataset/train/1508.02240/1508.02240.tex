\title{Bootstrapping Real-world Deployment of\\
Future Internet Architectures}

\author{Taeho Lee Pawel Szalachowski David Barrera Adrian Perrig Heejo Lee David Watrin\\
\begin{tabular}{cp{0.1cm}cp{0.1cm}c}
ETH Zurich && Korea University && Swisscom AG\\
\parbox{5cm}{\{kthlee, psz, david.barrera\\ aperrig\}@inf.ethz.ch}
    && heejo@korea.ac.kr && David.Watrin@swisscom.com \\
\end{tabular}
}
\maketitle

\section*{Abstract}
The past decade has seen many proposals for future Internet architectures.
Most of these proposals require substantial changes to the current networking
infrastructure and end-user devices, resulting in a failure to move from theory
to real-world deployment. This paper describes one possible strategy for
bootstrapping the initial deployment of future Internet architectures by
focusing on providing high availability as an incentive for early adopters.
Through large-scale simulation and real-world implementation, we show that with
only a small number of adopting ISPs, customers can obtain high availability
guarantees. We discuss design, implementation, and evaluation of an
availability device that allows customers to bridge into the future Internet
architecture without modifications to their existing infrastructure.

\section{Introduction}
\label{sec:introduction}
To be successful in their deployment, designers of new technologies must not
only solve technical issues (\eg compatibility, efficiency, etc.), but must
also offer convincing incentives for users to invest time and money to adopt
the technology. Without proper incentives, achieving substantial real-world
deployment is likely a futile effort.

The deployment of new Internet architectures is especially challenging given
the millions of hours of effort that have gone into building the networks these
new architectures are designed to replace.  Over the past two decades, a number
of new ideas that provide strong technical advantages over existing networks
and protocols (\eg mobile IP~\cite{rfc5944} and IP multicast~\cite{rfc1112})
have been proposed.  However, they have failed to gain mainstream adoption. The
resistance to change may be justified, since sizeable financial investments and
millions of hours spent on training make embracing potentially disruptive
changes a less than ideal proposition for incumbents. 

Future Internet Architectures (FIAs) aim to solve many problems with
current-generation networks (\eg security, availability, and scalability). To
solve these issues, FIAs suggest using radically different paradigms.  For
example, NDN~\cite{Jacobson:2009:NNC:1658939.1658941} provides efficient data
delivery by treating data (rather than the end-points) as a first-class citizen
on the network; MobilityFirst~\cite{Raychaudhuri:2012:MRT:2412096.2412098}
designs a mobility-centric network; XIA~\cite{Han2012} delivers a flexible
means of evolving the Internet's core; and \scion~\cite{scion2011} and
NIRA~\cite{nira2003} focus on making the Internet more resilient to failures.
Despite the appealing benefits offered by these architectures, they have yet to
gain mainstream adoption.

In this paper we present a case study on how to bootstrap deployment of FIAs by
focusing on how to convince early adopters. We demonstrate that deployment of a
new Internet architecture can provide tangible benefits to early adopters while
requiring minimal changes to existing infrastructure during the initial stage
of deployment.  Our goal is to present a feasible adoption plan to gradually
(and without friction) gain traction. Thus, instead of proposing a generic
one-size-fits-all plan (\ie attempting to convince manufacturers, ISPs,
developers, and end users that they should immediately deploy a particular
FIA), we focus on providing a single tangible benefit to early adopters:
increased network availability.  For this goal, we propose using FIAs as
fallback (backup) networks when specific quality of service metrics are not met
by current Internet paths. We note, however, that we focus exclusively on
initial deployment. Later steps will need other incentives since network
availability alone may be insufficient to motivate wide-scale deployment. 

Based on simulations, real-world implementation, and evaluation, we show that
with only a small number of deploying nodes, new Internet architectures, such
as NIRA and \scion, can transparently improve network availability while
introducing negligible performance overhead once deployed.  Our results provide
evidence that the seemingly impossible task of deploying a new Internet
architecture may, in fact, be possible.

Our contributions are as follows:

\begin{itemize}
\setlength{\itemsep}{-3pt}
	
	\item We describe the value proposition for both end-users and ISPs showing
that our approach benefits from natural scalability and wide-scale customer
base reachability (Section~\ref{sec:background}).

    \item We design and implement \name (Device for ENhancing Availability), a
bump-in-the-wire device that can automatically detect and establish fail-over
data paths over FIAs, and transparently fail-over to those paths when network
quality is low. \name requires no user configuration and is designed to be
minimally intrusive (Sections~\ref{sec:overview} and~\ref{sec:design}).

    \item We demonstrate, through a full-scale BGP simulation over the CAIDA
Inferred AS Relationship dataset (45,942 ASes), that a small number (less than
five) of ISPs deploying a FIA can significantly reduce attacks on availability
(Section~\ref{ssec:simulation}).

\end{itemize}

\section{Related Work}
\label{sec:related}
We review related work in two overarching themes: increasing network
availability and incremental deployment of new architectures. 

Several proposals aim to increase Internet availability. One avenue towards
this objective is through the use of indirection. Andersen \etal\cite{RON2001}
propose Resilient Overlay Network (RON) to shorten recovery time incurred from
BGP outages by using an alternate network path through an overlay. While the
work demonstrates the effectiveness of an overlay against BGP outages, it does
not describe Internet-scale deployment issues where different ISPs can own
different parts of the overlay. Our paper discusses deployment incentives for
possibly competing ISPs. 

Another closely related work is Advertised Reliable Routing Over Waypoints
(ARROW). In their paper, Peter~\etal\cite{DBLP:conf/sigcomm/PeterJZWAK14} show
that by redirecting traffic through tunnels, Internet reliability can be
enhanced. The authors validate their claim through BGP simulations considering
IP prefix hijacking attacks. The work herein corroborates the results of Peter
\etal finding that redirecting traffic to a victim AS using a tunnel provides
higher resilience against prefix hijacking attacks when an adversary targets
the victim's IP prefixes. Our simulation (see Section~6.4) encompasses a more
comprehensive adversary strategy; unlike in ARROW, we analyze the case where
adversaries are capable of attacking the tunnels as well as the victim
AS.\footnote{We have contacted the authors and confirmed the limitations of
adversary model described in Section~5.8 of their paper (\ie the adversary only
announces the prefixes of the victim).}  Additionally, our paper studies
incremental deployment aspects of FIAs, while the ARROW paper considers a
minimal change that can provide the greatest increase in availability. For
instance, the authors do not discuss how and who will manage the
\textit{Internet Atlas}, which is crucial for incremental deployment since such
an atlas likely requires global coordination between the deploying ASes.

There have been proposals to study incremental deployability aspects of an
architecture. In LISP (The Locator Identifier Separation Protocol), Saucez
\etal\cite{LISPDeployment2012} postulate ``there must be a clear deployment
path that benefits early adopters''.  Ratnasamy \etal\cite{Evolvable2005}
distilled the properties that facilitate the deployment of a new architecture.
They propose the use of IP anycast as the means to access the new architecture.
We build upon the insights that Saucez \etal and Ratnasamy \etal have provided
in terms  incremental deployability. However, we avoid using IP anycast as the
medium to access the new architecture because the use of anycast may preclude
ISPs from engaging into a direct business relationship with customers. Instead,
in our proposal an interested customer purchases a value-added service from a
deploying ISP, and the ISP would distribute a gateway device to access its
service.

\section{Background and Motivation}
\label{sec:background}
In this section, we motivate the need for network availability for ISPs and
customers focusing on limitations of the Border Gateway Protocol (BGP). 

\subsection{Demand for Network Availability}

The ever-increasing dependence on the Internet in our society has made
availability a critical requirement for network infrastructure. We define
availability to be the resilience against any type of attack or other network
fault (\eg misconfiguration) that could prevent or decrease the quality of a
network connection.

Due to their core business model (\ie offering connectivity to customers), ISPs
require high network availability. We can infer the existence of this demand by
analyzing the number of multi-homed endpoint ASes on the Internet. Using a
snapshot\footnote{The CAIDA dataset was collected on August 1st, 2014.} of the
CAIDA Inferred AS Relationships dataset~\cite{caida}, we observe that 22,386
out of 38,772 (approximately 57\%\footnote{Note that this ratio could be an
over-estimate as there are single-homed stub ASes without AS numbers.})
endpoint ASes have two or more provider ASes. Of these multi-homed ASes, 819
have five or more provider ASes. While there are reasons other than
availability for ASes to be multi-homed (\eg load-balancing, cost-balancing),
multi-homed ASes are more likely to be accessible even if one of their BGP
paths is unavailable.

Customers span the spectrum from residential end-users to large-scale
enterprises to governments.  A recent study found that residential users are
becoming sensitive to quality of their home Internet
connections~\cite{survey2013}. Businesses,  especially those which are
geographically diverse, make use of Virtual Private Network (VPN) technologies
to interconnect sites. For these businesses, the availability of intranet
resources is tied to the availability of their Internet connectivity.
Governments may want to ensure availability of their critical infrastructure
to control smart grids or to ensure the success of online elections~\cite{hk_poll}. 

\subsection{Degradation of Network Availability}
\label{sec:bgp_resilience}

Several factors may impact network availability on the Internet. These factors
range from transient link failures to route changes to security vulnerabilities
in routing protocols. Attacks on BGP are one of the major contributing factors
to availability issues on the Internet today~\cite{BGPVulnerabilities}.  One of
the most common threats in this context is known as \textit{IP Prefix
Hijacking}. Briefly, a BGP route is defined as an IP prefix and an AS-level
path which leads to that prefix. Thus, IP prefix hijacking denotes the
advertisement (malicious or accidental) of a prefix which is not authorized for
use by an advertising entity and which diverts the BGP path for that prefix to
the hijacker.  Due to the lack of authentication in BGP messages, IP prefix
hijacking is relatively easy to perform as the adversary can announce any
prefix of another AS (the standard threat model for this type of attack). As
a consequence, false injection of update messages can be used to launch
blackhole attacks~\cite{SPV2004} which hijack and drop traffic to compromise
availability.  Although the following does not fundamentally protect BGP
against hijacking attacks, we describe two practices that help in building a
more secure path in BGP.

\begin{itemize}
\setlength{\itemsep}{-3pt}

    \item{\textbf{/24 Prefix Announcements}}: Due to BGP's default route
selection policy, longer prefixes are harder to compromise than shorter
prefixes~\cite{conf/sp/HuM07,conf/sigcomm/BallaniFZ07}.  However, ISPs are
hesitant to accept /24 prefix announcements pervasively because doing so would
dramatically increase the size of routing tables on their border routers.

    \item{\textbf{Shorter BGP Paths}}: Shorter paths are usually more resilient
than longer paths because border routers typically prefer routes with shorter
AS-Path length. As the number of AS hops decreases, there are fewer network
locations (routers) from which an attack could be launched. Thus, if a route
between two end-hosts could be divided into multiple shorter path segments, it
would be more secure than the single longer path between the two end-hosts. We
confirm this effect through BGP simulation in Section~\ref{ssec:simulation}. 

\end{itemize}

\subsection{Value Proposition for Early Adopters}
\label{ssec:approach}
\paragraph{Value for ISPs.} As early adopters, ISPs that deploy an FIA obtain
several benefits.  First, deploying ISPs benefit from higher availability by
increasing their resilience to hijacking attacks (see
Figure~\ref{fig:sim-hijack}); and, the benefit exist even with a small number
(see Figure~\ref{fig:sim-reach}) of deploying ASes.

The benefit is accrued without sacrificing the scalability of BGP routing
tables. If every domain were to announce /24 prefixes for higher availability
(\ie stronger resilience against IP prefix hijacking attacks), the BGP routing
table does not scale.  Instead, in our approach, since only a few initially
deploying ASes announce /24 prefixes to protect their tunnels, the overhead on
BGP routing tables is small. Furthermore, beyond initial deployment, our
approach benefits from natural scalability; as more ISPs deploy the FIA, fewer
/24 announcements are needed since two contiguous ISPs that deploy FIA can
communicate directly without a /24 announcement.  Consequently, we expect the
number of announced /24 tunnel prefixes to increase during the early stage of
deployment and decrease as more ISPs deploy.

Second, there may exist a business incentive for deploying ISPs to offer high
availability to customers of other ISPs (\eg via IP tunnels as described in
Section~\ref{ssec:deployment_model}). Lastly, ISPs can create a niche market
for a set of customers who demand higher availability than BGP can provide but
cannot afford dedicated leased lines.

\paragraph{Value for Customers.} Customers may obtain high-availability (even
if the customer's local ISP does not provide such functionality) by subscribing
to a remote provider via an access tunnel.  More importantly, customers can be
bridged into high-availability architectures without changes to software on
their local devices or modifications to their Internet service. We show one
method to achieve this zero-configuration setup in Section~\ref{sec:overview}. 

\section{Overview and Requirements for a High-availability Device}
\label{sec:overview}
Guaranteeing benefits under partial deployment alone is not sufficient for a
successful initial deployment. Customers must be able to subscribe to an FIA
without carrying out any complicated tasks, such as configuring their network
devices or updating them to a new network stack---updating the network stack on
light bulbs, cameras, televisions, refrigerators, and other Internet of things
devices may be infeasible. To this end, we develop a \textit{bump-in-the-wire}
interface device which we refer to as \name (Device for ENhancing Availability)
to be placed between the customer's network and their Internet provider. An
alternative approach could be to deploy such interface devices at the ISPs
themselves (this approach is used by Peter
\etal~\cite{DBLP:conf/sigcomm/PeterJZWAK14}), completely removing end-user
involvement. However, this strategy would preclude users from subscribing to
the FIA if their immediate ISP does not deploy it.

The primary function of \name is to increase Internet availability to its
subscriber by leveraging a given FIA as a fail-over network. That is, if the IP
path between two customers (deploying end-points) becomes unavailable, \name
would detect the unavailability and automatically fail-over to a path through
the FIA by encapsulating IP packets.

To provide high-availability guarantees, \name devices need to implement four
functionalities. For a given communication flow between a subscriber and a
peer, the \name needs to \textit{1)}~identify the presence of a peer;
\textit{2)}~if present, establish FIA path(s) to be used as fail-over;
\textit{3)}~continuously measure the packet loss rate of the path(s); and
\textit{4)}~fail-over to a path offered by the FIA if necessary. 

Although our design and implementation of \name remains generic for any FIA
that is capable of providing high-availability guarantees (\eg NIRA and
\scion), throughout the following sections, we draw on \scion as the chosen
example FIA. We find it useful to focus on one particular FIA to keep the
discussion concrete and to drive the narrative. In addition, our access to the
global \scion testbed enables for a thorough evaluation of our implementation.

For space reasons, we do not provide details of the \scion architecture in this
paper; however, the discussions in the following sections should be clear
without any prior knowledge of \scion (Zhang \etal~\cite{scion2011}). 

\subsection{ISP Deployment Model}
\label{ssec:deployment_model}

To minimize impact on their current infrastructure, ISPs are likely to deploy a
high-availability infrastructure as an overlay to their current
networks~\cite{peterson2003blueprint}. Customers, as well as other ISPs will
use IP tunnels to bridge into these overlays. 

Hence, constructing reliable tunnels is essential to provide high-availability
guarantees. To this end, /24 prefix blocks are used to reduce the risk of
traffic hijacking attacks against the tunnels. 

For our deployment strategy, we define the following tunnel types (shown in
Figure~\ref{fig:tunnels}), and explain how tunnels are protected against
hijacking by using the practices described in Section~\ref{sec:bgp_resilience}.

\paragraph{Access Tunnel.} A tunnel with which a deploying ISP can offer
high-availability services to customers whose ISPs do not support these
features. As shown in Figure~\ref{fig:tunnels}a, to protect the tunnel between
A and itself from hijacking, the deploying AS (D) announces a /24 prefix
that  contains the IP address used for that tunnel. This announcement reduces
the probability of an adversary successfully hijacking this tunnel if they are
multiple hops away from A and D.

\paragraph{Inter-site Tunnel.} A tunnel that connects two non-contiguous
deploying ASes over the Internet (see Figure~\ref{fig:tunnels}). The ASes need
to protect the tunnels that link deploying sites from prefix hijacking attacks.
Similar to the access tunnels above, each of the two ASes announces a /24
prefix block that contains the IP address used as its tunnel end-point address. 

\paragraph{End-to-end Tunnel.} A series of tunnels that consist of access
tunnels and inter-site tunnels to connect customers A and B over the FIA
deployment. For instance, in Figure~\ref{fig:tunnels}, the end-to-end tunnel
consists of one access tunnel and one inter-site tunnel.

\begin{figure}
	\centering
	\includegraphics[width=0.9\linewidth]{tunnels.eps}
	
	\caption{Three tunnels that are used in FIA deployment. In each subfigure,
		circles marked D represent ASes that have deployed FIA, and circles 
		marked
		ND represent non-deploying ASes.}
	
	\label{fig:tunnels}
\end{figure}

\subsection{End-to-End Communication Scenario}
\label{sec:e2e_scenario}

\begin{figure*}
\centering
\includegraphics[width=0.75\textwidth]{comm_scn.eps}
\caption{An example of end-to-end communication. The destination identifier is
implemented as the 5-tuple information. See Section~\ref{sec:design}.}
\label{fig:e2e_comm}
\end{figure*}

Before describing the design and implementation of \name, we sketch a
high-level picture of how an end-to-end communication flow would behave in a
typical Internet-wide scenario shown in Figure \ref{fig:e2e_comm}.

In this scenario, there are two \scion providers (Prov, Prov) that have
deployed gateways (GW, GW) to serve their \scion subscribers. There are
three end-hosts (Host, Host, Host), where Host has a public IP
address, and Host and Host are behind a NAT. To enhance their Internet
availability, Host has subscribed to the \scion service from Prov, and
Host and Host from Prov. All hosts have placed \name devices as
bump-in-the-wire on their network connection.

Consider a scenario where Host initiates communication to Host. Upon
receiving a packet from Host, \namens checks if it knows the peer \name
associated with the destination address of the packets. If it does not know the
associated peer, \namens initiates a Discovery Process
(Section~\ref{subsection:Discovery}) and exchanges bootstrapping information
that is necessary for constructing \scion overlay paths to the remote peer. In
addition, the bootstrapping information contains information necessary for NAT
traversal, which we discuss in Section~\ref{sec:discussion}.

A \name monitors the packet loss  rates of the paths (both IP and \scion paths)
to all of the discovered peer \names
(Section~\ref{subsection:path_management}). Whenever the quality of the IP path
to a peer \name degrades below a given threshold, and the quality of a path
provided by \scion is acceptable, all of the flows associated with the peer
\name are encapsulated and redirected to the \scion path to ensure continued
availability.


Redirection over \scion is accomplished via a series of tunnels using the
packet structure shown in Figure \ref{fig:pkt_structure}. The first and last
tunnels are access tunnels (see Figure~\ref{fig:tunnels}a) between the \names
and the \scion gateways.  The intermediate tunnel is a series of inter-site
tunnels (see Figure~\ref{fig:tunnels}b) that links the two \scion ASes to which
the two end-hosts are subscribed. 

In our communication example between Host and Host, the packets from
Host to Host traverse from \namens to \namens via tunnels
across \namens to GW, Prov to Prov, and GW to \namens.
\namens removes the tunnel header and the SCION header, and forwards the
regular IP packets to Host.

\begin{figure}[h]
\renewcommand{\arraystretch}{2.5}
\linespread{0.5}\selectfont\centering
\begin{tabular} {|c|c|c|c|}
\hline
\parbox[c][][t]{1.3cm}{\centering{\small IP Tunnel Header}} &
\parbox[c][][t]{1.3cm}{\centering{\small \scion Header}} &
\parbox[c][][t]{1.7cm}{\centering{\small Data Header (\eg IP+TCP)}} &
\parbox[c][][t]{1.7cm}{\centering{\small Data Payload}}\\ie ),

    \item[:] length of the tunneled path betweeen  and 
        \\ie ).

\end{description}

We assume that the first node of an \ete tunnel path () is the source
while the last () is the destination.  Then,  expresses the
length of the BGP path between source and destination.  We also assume that
traffic from source to destination over the \ete tunnel traverses overlay nodes
 in that order. The tunnel's path on an AS-level
is a concatenation of BGP paths: .

For our simulation, we consider two adversary strategies designed to hijack
traffic from source to destination: \textit{1)} weak adversary which announces
only the destination's prefix, and \textit{2)} strong adversary which announces
all prefixes of .  In both cases an adversary
launches attacks from a randomly compromised AS, however he cannot compromise
ASes on the path between source and destination. Note that
ARROW~\cite{DBLP:conf/sigcomm/PeterJZWAK14} only considers the weak adversary
model and does not consider the strong adversary model.

In this experiment, we simulate 8 scenarios by varying , , and
 to analyze the resilience that tunnels can provide against prefix
hijacking attacks. For each scenario, while incrementing the number of
adversarial ASes from 1 to 7, we construct and simulate 1000 random and unique
tunnel deployments, where  and  are randomly chosen from multi-homed
stub ASes and the other tunnel nodes are chosen from all other ASes.\footnote{
To build more confidence of our results, we have run four independent sets of
1000 deployments for all 8 scenarios and confirmed that the results were
similar across the four sets.}

Figure~\ref{fig:sim-hijack} summarizes the results of our simulation. In each
graph in the figure, the -axis represents the number of adversaries (varied
from 1 to 7) and the -axis represents probability values that an attack on
the tunneled path will be successful. The two figures on the left show the
hijack probability of the tunneled path for source and destination AS pairs
that are four BGP hops apart (=4); and, on the right five BGP hops
apart (=5). Moreover, the upper two figures show the results against
weak adversaries while the lower two figures show the results against strong
adversaries. Lastly, each plot also shows hijack probability for the BGP paths
(green line with plus markers).

Our results show that the hijack probability increases as the number of
adversaries increases and that the probability is higher for the strong
adversary model than the weak adversary model. Furthermore, against the weak
adversary model, the probabilities of hijacking the tunneled paths are similar
for the two cases that have the same tunnel segment length (\ie ) but
different total length (\ie )---on the upper two figures, the purple lines
with diamond makers and the red line with inverted triangle markers almost
overlap with each other. This is because the weak adversary model can only
attack the last tunnel segment (\ie () as the weak adversaries
only announce the prefix of the destination (\ie ).

The results show that the tunneled paths have lower hijack probability than the
BGP paths even if the length of the tunneled path (\ie ) is longer than
that of the BGP paths (\ie ). But, the result also shows that if the
length of the tunneled paths becomes too long (\eg twice the length of the BGP
paths), the tunneled paths become more susceptible to hijacking attacks.
However, in practice, it is highly unlikely that the tunneled paths would be
twice the length of the BGP paths.

Lastly, the results show that the composition of the tunneled path affects the
resilience against prefix hijacking attacks: a tunneled path that is composed
of shorter individual segments is more resilient than a path that is composed
of longer individual segments. For the two cases where , the hijack
probability is significantly lower when the length of the individual tunnel is
kept shorter. In Figure~\ref{fig:sim-hijack}, this can be seen by comparing the
blue line with 'x' markers and the purple line with diamond markers.  Moreover,
the result shows that the tunneled paths that have longer total length but
shorter individual tunnel segments (\ie , red lines with inverted
triangle markers) is more resilient than the tunneled paths that have shorter
total length but longer individual segments (\ie , blue lines with
'x' markers).

\subsection{Potential Customer Base}
\label{sssec:coverage}

In this section, we investigate the proportion of multi-homed stub ASes that
are \textit{N}-hops (\ie ) away from any of the deploying ASes as we
vary the number of ASes which deploy the FIA. This experiment should assist in
quantifying the potential customer base (measured in number of ASes) as the FIA
deployment increases.

\begin{figure}[h!]
	\center
	\includegraphics[width=1.\columnwidth]{coverage.eps}
	\caption{Proportion of ASes that can communicate over FIA.}
	\label{fig:sim-reach}                                                       
	 
\end{figure}

In this experiment, we randomly choose ASes to deploy the FIA such that the
distance between deploying ASes would be at most \textit{N} BGP hops
(\ie).  Then, for all multi-homed stub ASes, we evaluate the number of
ASes that are within \textit{N} hops from any of the FIA deploying ASes. The
two steps are repeated 5,000 times, and we report the average in Figure
~\ref{fig:sim-reach}, which shows the proportion of the ASes that could be
reached within N hops from the FIA deployment as number of deploying AS changes
from 1 to 20.

Our results show that for higher values of , a larger portion of the
multi-homed stub ASes were within the reach of the deploying ASes. Combining
the results from Section~\ref{sssec:tun_resiliency}, we can conclude that for
large values of , deploying ASes obtain larger customer-base at the
expense of availability guarantees against prefix hijacking attacks.
Conversely, when offering higher availability guarantees, the customer-base
becomes much smaller (\ie ).

As expected, the number of deploying ASes increases with the size of customer
base. In fact, when there are 20 deploying ASes, the customer-base nearly spans
across the entire Internet for the cases with . In addition, even
with only one deploying AS, there are multi-homed stub ASes that could benefit
from the FIA deployment, and for the cases with , more than half of
the multi-homed stub ASes could access the FIA deployment within  hops.

Lastly, Figure~\ref{fig:sim-reach} offers another interesting
observation---reduced benefit (in terms of increase in customer base) as more
ASes participate in FIA deployment. This can be inferred from the gradient of
the plots, which tends to zero as \textit{N} increases. Hence, the result
suggests that beyond the initial phase of the deployment, a different incentive
model may be necessary to further drive the FIA deployment.

\section{Discussion}
\label{sec:discussion}
\subsection{Practical Implications on Deployment}

\paragraph{NAT Implications.} As discussed in
Section~\ref{subsection:requirements}, a \name should work in the presence of
middle-boxes such as NATs. When a \name is placed in-between a NAT and an
end-host (see \namens in Figure \ref{fig:e2e_comm}), communication over the
\scion path may not work. For instance, when the communication between Host
and Host in Figure \ref{fig:e2e_comm} is carried out through the \scion
path, \namens cannot deliver Host's packets to Host without being
given Host's internal address.

To address this problem, when \namens and \namens exchange bootstrapping
information during the \name Discovery Process, \namens includes the private
address of Host.

Another problem is the communication between GW and \namens. Because of
NAT, GW cannot send a packet to \namens before \namens starts
forwarding packets via \scion. To avoid this problem, \namens periodically
exchanges keep-alive messages to its assigned gateway, performing NAT
hole-punching. Since the \name itself does not have an IP address, it borrows
one of the IP addresses it has learned from the end-hosts to perform the
hole-punching.

\paragraph{MTU Implications.} The measurement and correct configuration of MTU
size is critical for any scheme that requires encapsulation, such as the
tunnels used in our system. When end-hosts perform the path MTU discovery
protocol (PMTUD~\cite{rfc1191}), \name can respond with an MTU size small
enough to provide enough space for encapsulation. In the event that an end-host
does not respond to PMTUD messages (fewer than 10\% of devices do not
~\cite{Luckie2010}), packet fragmentation would be required, increasing the
complexity.

\subsection{Device Configuration and Distribution}

Although \name does not require any configuration by end-users, it requires
configuration by ISPs prior to distribution to customers.  Configuration would
include information necessary for bootstrapping the \name into the FIA
deployment (\ie in case of \scion, this includes \scion ISD ID, AS
AID\footnote{In SCION, each AS is identified by the ID of the ISolation Domain
(ISD) that the AS is in as well as the identifier of the AS (AID).}, and IP
address of the gateway that a \name would contact to access the \scion
network). We note that such configurations are common in today's Internet when
ISPs distribute cable or DSL modems to their subscribers. In addition, for
customers that already have access devices (\ie cable or DSL modems), the
firmwares of these devices could be updated to support \name functionality.

\subsection{Applicability to Other FIAs}

The \name is a generic device that creates tunnels to interconnect two FIA
deploying islands. Hence, it is possible to use \name for other FIAs, as long
as tunneling can be used to interconnect two FIA deploying islands. The
important part in bootstrapping a FIA is to identify the proper incentives for
the early adopters. For instance, for more efficient distribution and access to
the content to early adopters, NDN could be deployed. In NDN deployment, \name
would translate user data requests into an interest packet and send it to the
NDN deployment. When the requested data arrives at the \name, it delivers the
data to the user.

\section{Conclusion}
\label{sec:conclusion}

We have motivated, designed, and evaluated an incremental deployment strategy
for Future Internet Architectures. Through both simulation and real-world
implementation and testing, we demonstrated tangible availability improvements
for deploying ISPs with comparatively little deployment cost. Our evaluation
used \scion to assess the feasibility of our proposal, but the proposed
incremental deployment strategy remains compatible with other FIAs. 

While availability alone will likely be insufficient to motivate full
wide-scale deployment, this paper focuses on convincing early adopters to use a
new architecture. Once a base set of ISPs and customers have deployed a FIA, a
new strategy may be necessary to encourage a majority of ISPs to deploy. For
example, the next set of deploying users could be interested in mobility or
content-centric networking, but such analysis is left open to future research.

\section{Acknowledgements}

We thank Qi Li and Yao Shen for their assistance during the early stage of the
project. We also thank Stephanos Matsumoto for interesting discussions and
feedback on the project. The research leading to these results has received
funding from Swisscom, and from the European Research Council under the
European Union's Seventh Framework Programme (FP7/2007-2013) / ERC grant
agreement 617605. We gratefully acknowledge their support.

\bibliographystyle{abbrv}
\bibliography{0-string,paper}
