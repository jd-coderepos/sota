



\documentclass[titlepage,oneside,fleqn,12pt]{article}

\pdfoutput=1



\oddsidemargin 0.0in
\topmargin -0.5in
\headheight 0.3in
\headsep 0.2in
\textwidth 6.5in
\textheight 9.0in
\setlength{\parindent}{0in}

\oddsidemargin 0.0in
\usepackage[tiny,rm]{titlesec}
\newpagestyle{trbstyle}{
	\sethead{\small Roncoli, Bekiaris-Liberis, Papageorgiou}{}{\thepage}
}
\pagestyle{trbstyle}

\renewcommand*{\refname}{\uppercase{References}}
\titleformat{\section}{\bfseries}{}{0pt}{\uppercase}
\titlespacing*{\section}{0pt}{12pt}{*0}
\titleformat{\subsection}{\bfseries}{}{0pt}{}
\titlespacing*{\subsection}{0pt}{12pt}{*0}
\titleformat{\subsubsection}{\itshape}{}{0pt}{}
\titlespacing*{\subsubsection}{0pt}{12pt}{*0}

\usepackage{enumitem}
\setlist[1]{labelindent=0.5in,leftmargin=*}
\setlist[2]{labelindent=0in,leftmargin=*}

\usepackage{ccaption}
\usepackage{amsmath}
\makeatletter
\renewcommand{\fnum@figure}{\textbf{FIGURE~\thefigure} }
\renewcommand{\fnum@table}{\textbf{TABLE~\thetable} }
\makeatother
\captiontitlefont{\bfseries \boldmath}
\captiondelim{\;}
\setlength{\mathindent}{0pt}




\usepackage[sort,numbers]{natbib}
	\newcommand{\trbcite}[1]{\citeauthor{#1} ({\it \citenum{#1}})}
	\newcommand{\trbnum}[1]{{\it \citenum{#1}}}
\setcitestyle{round}

\makeatletter
	\renewcommand\@biblabel[1]{#1.\hspace*{15pt}}
\makeatother

\setlength{\bibsep}{0pt plus 0.3ex}





















\usepackage{mathptmx}


\usepackage[T1]{fontenc}
\usepackage{textcomp}




\usepackage{graphicx}
\usepackage[hidelinks]{hyperref}
\usepackage{color}
\usepackage{amsfonts}
\usepackage{multirow}



\begin{document}

\thispagestyle{empty}

\begin{titlepage}
\begin{flushleft}

{\bfseries HIGHWAY TRAFFIC STATE ESTIMATION USING SPEED MEASUREMENTS: CASE STUDIES ON NGSIM DATA AND HIGHWAY A20 IN THE NETHERLANDS}\0.5cm]

\textbf{Nikolaos Bekiaris-Liberis} \\
Dynamic Systems and Simulation Laboratory \\
Technical University of Crete \\
Chania, 73100, Greece \\
Email: \href{mailto:nikos.bekiaris@gmail.com}{\textcolor{blue}{\underline{\smash{nikos.bekiaris@gmail.com}}}}\1cm]







\vspace{2.5cm}

\today
\end{flushleft}
\end{titlepage}


\newpage

\setcounter{page}{2}

\section{Abstract}
This paper presents two case studies where a macroscopic model-based approach for traffic state estimation, which we have recently developed, is employed and tested. The estimation methodology is developed for a ``mixed'' traffic scenario, where traffic is composed of both ordinary and connected vehicles. Only average speed measurements, which may be obtained from connected vehicles reports, and a minimum number (sufficient to guarantee observability) of spot sensor-based total flow measurements are utilised.
In the first case study, we use NGSIM microscopic data in order to test the capability of estimating the traffic state on the basis of aggregated information retrieved from moving vehicles and considering various penetration rates of connected vehicles. In the second case study, a longer highway stretch with internal congestion is utilised, in order to test the capability of the proposed estimation scheme to produce appropriate estimates for varying traffic conditions on long stretches. In both cases the performances are satisfactory, and the obtained results demonstrate the effectiveness of the methodology, both in qualitative and quantitative terms.
\\label{eqrho}
\rho_i(k+1) = \rho_i(k)+\frac{T}{\Delta_i}\big(q_{i-1}(k)-q_i(k)+r_i(k)-s_i(k)\big),
\label{flowtotal}
q_i(k) = \rho_i(k) v_i(k),
\label{eqrh1}
\rho_i(k+1) = \frac{T}{\Delta_i}v_{i-1}(k)\rho_{i-1}(k)+\left(1-\frac{T}{\Delta_i}v_i(k)\right)\rho_i(k)+\frac{T}{\Delta_i}\left(r_i(k)-s_i(k)\right).
\label{eqStab}
\max_{i,k} \frac{\Delta_i}{T} v_{i}(k) < 1.

\theta_i(k+1)=\theta_i(k)+\xi_i^{\theta}(k),\label{thetai}

 \theta_i=\left\{\begin{array}{ll}\frac{T}{\Delta_i}r_{n_i},&\mbox{if }\\\frac{T}{\Delta_i}s_{n_i},&\mbox{if }\end{array}\right\},

{x}=\left(\rho_1,\ldots,\rho_N,\theta_1,\ldots,\theta_{l_r+l_s}\right)^T, \label{stnew}

{x}(k+1)={A}(k){x}(k)+{B}{u}(k),\label{barx}

{A}(k)&=&\left\{\begin{array}{lll}{a}_{ij}=\frac{T}{\Delta_i}v_{i-1}(k),&\mbox{if  and }\\{a}_{ij}=1-\frac{T}{\Delta_i}v_i(k),&\mbox{if }\\{a}_{n_ij}=1,&\mbox{if  and }\\{a}_{n_ij}=-1,&\mbox{if  and }\\{a}_{ij}=1,&\mbox{if  and }\\{a}_{ij}=0,&\mbox{otherwise}\end{array}\right\}\label{adef1}\\
{B}\!&=&\!\left[\begin{array}{ll}{b}_{ij}=\frac{T}{\Delta_i},&\mbox{if  and }\\{b}_{m_ij}=\frac{T}{\Delta_{m_i}},&\mbox{if , , , and }\\{b}_{ij}=0,&\mbox{otherwise}\end{array}\!\!\right]\label{defbu}\\
{u}(k)&=&\left[\begin{array}{ll}{u}_i=q_0(k),&\mbox{if }\\{u}_{i+1}=r_{m_i}-s_{m_{i}},&\mbox{if }\end{array}\right],\label{newu}

{y}(k)={C}{x}(k),\label{newy}

{C}&=&{\left[\begin{array}{ll}{c}_{ij}=1,&\mbox{for all  and some }\\{c}_{ij}=1,&\mbox{if  and }\\{c}_{ij}=0,&\mbox{otherwise}\end{array}\right]},\label{16rhonew}

\hat{{x}}=\left(\hat{\rho}_1,\ldots,\hat{\rho}_N,\hat{\theta}_1,\ldots,\hat{\theta}_{l_r+l_s}\right)^T,

\hat{{x}}(k+1)&=&{A}(k)\hat{{x}}(k)+{B}{u}(k)+{{A}}(k){K}(k)\left({{z}}(k)-{C}\hat{{x}}(k)\right)\label{123}\\
{K}(k)&=&{P}(k){C}^T\left({C}{P}(k){C}^T+{R}\right)^{-1}\\
{P}(k+1)&=&{A}(k)\left(I-{K}(k){C}\right){P}(k){A}(k)^T+{Q},\label{kal1}

\hat{{x}}(k_0)&=&\mu\\
{P}(k_0)&=&H,\label{1234}
 \label{eq:perfIndexDens}
CV_\rho = \frac{\sqrt{ \frac{1}{KN} \sum_{k=1}^{K} \sum_{i=1}^{N} \left[ \hat{\rho}_i(k) - \rho_i(k) \right]^2}}{\frac{1}{KN} \sum_{k=1}^{K} \sum_{i=1}^{N} \rho_i(k)}.
 \label{covar}
w = \frac{1}{KN} \sum_{i=1}^N \sum_{k=1}^K \frac{T^2}{\Delta_i^2} \; E \left[ \rho_i^2(k) \left( \hat{v}_i(k) - \bar{v}_i(k) \right)^2 \right] .

The resulting value for a penetration rate of connected vehicles equal to 20\% is , while for a penetration rate of 5\% the resulting mean covariance is . These values suggest that better performance may be achieved by utilising higher values of  for lower penetration rates. Nevertheless, experimental tests not reported here) indicate that, for low penetration rates, the performance of the proposed estimation scheme does not vary significantly for different values of .

Unfortunately, NGSIM data were collected for a short highway stretch within a peak period characterised by congested traffic conditions only; therefore the capability of the proposed estimation scheme in capturing the flow breakdown occurrence cannot be tested with these data. In order to extend the evaluation range of the proposed estimation scheme, another case study is performed and it is presented in the next section.

\subsection{Case Study 2: Highway A20, the Netherlands}

\subsubsection{Network and Data Description}

The second case study is based on real data obtained from detectors of a stretch of the highway A20 from Rotterdam to Gouda in the Netherlands, taken from (\trbnum{Schakel2014}). The topological characteristics of this network, which incorporates a non-trivial combination of lane-drops, on-ramps, and off-ramps, the congestion pattern, and the relatively closely-spaced detectors make it a challenging test-bed for the estimation scheme proposed in the previous section.

The considered stretch is about 11\,km in length, includes 4 on-ramps and 3 off-ramps (however, the on- and off-ramp ``Gas station'' are ignored since they are characterised by extremely low flows). The stretch includes a lane-drop located at 4477\,m (see Figure \ref{fig:dutchStretch}). There are 32 detectors placed rather homogeneously at a distance of 300\,m on average, as illustrated in Figure \ref{fig:dutchStretch}. In addition, a flow detector is located at off-ramp ``Moordrecht''.

Data from the morning rush hour of Monday, June 8, 2009 are employed, where a strong congestion is created around 6:20 AM because of the increased flow entering from on-ramp ``Nieuwerkerk a/d IJssel''; consequently, the congestion spills back and worsens at the lane-drop area, reaching up to the ``Gas station'' area; then it disappears after 7:40 AM because of the reduced demand. A 3d-plot illustrating the traffic conditions from 5 AM to 9 AM is shown in Figure \ref{fig:3d-dens_real}. This congestion pattern allows to test and evaluate the proposed estimator under varying traffic conditions, which include the formation and dissipation of a stretch-internal congestion which is not visible at the stretch boundaries.

\begin{figure}
\begin{center}
	\includegraphics[width=\textwidth]{Dutchstretch}
	\caption{A graphical representation of the considered stretch of highway A20 from Rotterdam to Gouda, the Netherlands. Detector positions are indicated as the distance (in m) from the network entrance. The detectors used by the estimator for obtaining flow measurements within the case study are underlined.} 
	\label{fig:dutchStretch}
\end{center}
\end{figure}

\begin{figure}
\begin{center}
	\includegraphics[width=0.9\textwidth]{plot_3d_dens_real}
	\caption{The space/time evolution of real densities for Case Study 2. The presence of a congestion originated at the Nieuwerkerk a/d IJssel merge area that spills-back and further deteriorate at the lane-drop, marked as a dotted line, is clearly visible.} 
	\label{fig:3d-dens_real}
\end{center}
\end{figure}

The network is space-discretised with  segments, where each segment is delimited by a pair of detectors. According to this space-discretisaton, on-ramps are placed within segments 3, 17, and 25; whereas off-ramps are placed within segments 14 and 21 (see also Figure \ref{fig:dutchStretch}).

The proposed estimation algorithm is fed with the following information: flow measurements retrieved from a limited number of detectors according to the configuration shown in Figure \ref{fig:dutchStretch}, that is sufficient to guarantee observability as explained in the previous section (for details, see \trbcite{Bekiaris2015}); and speed measurements retrieved from all detectors, on the basis of the assumption that similar information may be obtained from connected vehicles reports. Specifically, the detector located at the highway entrance (0\,m) is used to obtain the input of the system ; all ramp flows are assumed unknown, therefore four additional detectors are utilised (one between every pair of unmeasured ramps), choosing the ones located at 4477\,m (segment ), 5125\,m (segment ), 6566\,m (segment ), and 8162\,m (segment ); and, finally, one measurement is taken at the network exit (10828\,m), employed as  (see also Figure \ref{fig:dutchStretch}).

The ground truth is represented by the densities computed using the measurements from each detector as , where  and  are the measured flow and speed respectively; and by the only ramp flow measurement available, namely the flow at off-ramp Moordrecht.

Using the same measurements both for feeding the estimator and for constructing ground truth implies that the estimator is not subject to any measurement error. However, in order to assess the performance of the estimation scheme in more realistic scenarios, two additional cases are considered. In the first, a zero-mean Gaussian white additive noise, characterised by a standard deviation of 300\,veh/h, is added to the flow measurements that are utilised by the estimator; in the second, a zero-mean Gaussian white noise, with a standard deviation of 5\,km/h, is added to the speed measurements that are employed by the filter, in addition to the flow measurement noise.

\subsubsection{Performance Evaluation}

The parameters used for the Kalman filter are the following: ;  ; ; .
The performance index in Equation \ref{eq:perfIndexDens} is used for numerical evaluation of the estimation methodology. In all the tested cases the estimates are very accurate. In the first case, where noise is not added to the measurements utilised by the estimator, a performance index value  is obtained. When the flow measurements used by the filter are subject to noise, we obtain , which is only a slight degradation with respect to the noise-free case, since the proposed estimation scheme is capable of efficiently handling the effect of additive noise. On the other hand, when noise is also added to the speed measurements, a more significant deterioration of the performance index () and noisier estimates are obtained; this is due to the fact that, in this case, noise in not purely additive, but it also appears as multiplicative within the dynamic equations of the filter. By visual inspection of the 3d-plot shown in Figure \ref{fig:3d-dens_est}, which illustrates the estimated densities when noise is added to both flow and speed measurements, we can see that the estimator reliably reconstructs the congestion pattern within the network, both in time and space. Furthermore, Figure \ref{fig:trajDutch} depicts the measured and the estimated density trajectories around the location where traffic congestion starts: we can see that the estimation scheme captures the onset of congestion with accurate timing and at the correct location. Finally, Figure \ref{fig:trajRampDutch} illustrates the results of the flow estimation at off-ramp Moordrecht (the only one for which flow measurements are actually available), from which it is clear that the algorithm estimates the off-ramp flow with good accuracy.

\begin{figure}
\begin{center}
	\includegraphics[width=0.9\textwidth]{plot_3d_dens_est}
	\caption{The space/time evolution of estimated densities for Case Study 2, when additive noise affects both flow and speed measurements. By comparison with Figure \ref{fig:3d-dens_real} it is clear that the estimator is capable of reproducing the real congestion pattern (both in time and space).}
	\label{fig:3d-dens_est}
\end{center}
\end{figure}

\begin{figure}
\begin{center}
	\includegraphics[width=\textwidth]{plot_trajDutch}
	\caption{The real and estimated density trajectories at the location where congestion starts, for Case Study 2, when additive noise affects both flow and speed measurements. The proposed scheme is capable of estimating with high accuracy the onset of the congestion.} 
	\label{fig:trajDutch}
\end{center}
\end{figure}

\begin{figure}
\begin{center}
	\includegraphics[width=0.5\textwidth]{plot_trajRampDutch}
	\caption{The measured and estimated flows at off-ramp Moordrecht, for Case Study 2,  when additive noise affects both flow and speed measurements.} 
	\label{fig:trajRampDutch}
\end{center}
\end{figure}


\section{Conclusions} \label{sec:concl}

The macroscopic model-based traffic estimation scheme proposed in (\trbnum{Bekiaris2015}) has been tested on two case-studies using data stemming from different real experiments. In both cases, the results demonstrated the effectiveness of the methodology, both in qualitative and quantitative terms. The first case, based on the NGSIM microscopic data, demonstrated the capability of properly estimating the traffic state using aggregated information retrieved from moving vehicles; a set of experiments, using different penetration rates of connected vehicles, demonstrated the robustness of the proposed methodology with respect to potentially inaccurate speed measurements. The second case, where a longer highway stretch and time horizon are employed, has permitted us to verify the capability of properly estimating the traffic state in case a congestion is created within the network, making this estimation scheme particularly useful for traffic control applications. 

Ongoing work involves testing more complex scenarios, using microscopic simulations, to assess the effectiveness of the approach in case connected vehicles are characterised by a significantly different behaviour; e.g., when they are automated.


\section*{Acknowledgement}
The research leading to these results has received funding from the European Research Council under the European Union's Seventh Framework Programme (FP/2007-2013) / ERC Grant Agreement n. 321132, project TRAMAN21.

The authors would like to thank Prof. Vincenzo Punzo and Dr. Marcello Montanino from University of Naples, Italy, for their support in providing the reconstructed  NGSIM data used in the first case study; and Prof. Bart van Arem and Dr. Wouter Schakel from TU Delft, the Netherlands, for data and information related to the network used in the second case study.

\clearpage

\bibliographystyle{trb}
\bibliography{collection}

\end{document}
