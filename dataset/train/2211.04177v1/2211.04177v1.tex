\documentclass[runningheads]{llncs}
\usepackage{graphicx}
\usepackage[utf8]{inputenc}
\usepackage[english]{babel}
\usepackage{amsmath}
\usepackage{xcolor}
\usepackage{subcaption}
\captionsetup[subfigure]{justification=justified,singlelinecheck=false}



\graphicspath{{./images/}}
\DeclareGraphicsExtensions{.pdf,.jpeg, .png, jpg}


\newcommand*{\thead}[1]{\multicolumn{1}{c}{\bfseries #1}} \newcommand{\hide}[1]{}
\newcommand{\hidefoot}[1]{\footnote{{\bf #1}}}
\newcommand{\todo}[1]{{\bf TO DO: {#1}}}
\newcommand{\dquote}[1]{``#1''}
\newcommand{\etal}{\textit{et al.}~}
\newcommand{\ie}{\textit{i.e.}~}
\newcommand{\eg}{\textit{e.g.}~}
\newcommand{\alert}[1]{\textcolor{red}{#1}}
\newcommand{\answer}[1]{\textcolor{blue}{#1}}
\renewcommand*\thesubfigure{\arabic{subfigure}} 


\begin{document}
\title{Learning advisor networks for noisy image classification}




	\author{Simone Ricci \orcidID{0000-0001-9838-6076} \and
		Tiberio Uricchio \orcidID{0000-0003-1025-4541} \and
		Alberto Del Bimbo \orcidID{0000-0002-1052-8322} 
	}
\authorrunning{S. Ricci et al.}
\institute{
		University of Florence, Italy \\
		\email{\{name.surname\}@unifi.it}
	}
\maketitle              \begin{abstract}
		
		In this paper, we introduced the novel concept of advisor network to address the problem of noisy labels in image classification. Deep neural networks (DNN) are prone to performance reduction and overfitting problems on training data with noisy annotations. Weighting loss methods aim to mitigate the influence of noisy labels during the training, completely removing their contribution. This discarding process prevents DNNs from learning wrong associations between images and their correct labels but reduces the amount of data used, especially when most of the samples have noisy labels. Differently, our method weighs the feature extracted directly from the classifier without altering the loss value of each data. The advisor helps to focus only on some part of the information present in mislabeled examples, allowing the classifier to leverage that data as well. We trained it with a meta-learning strategy so that it can adapt throughout the training of the main model. We tested our method on CIFAR10 and CIFAR100 with synthetic noise, and on Clothing1M which contains real-world noise, reporting state-of-the-art results.
		
		\keywords{Meta learning  \and Advisor network \and Noisy labels.}
	\end{abstract}
\section{Introduction}
	Modern image classification systems are based on using deep neural network models that are trained on a huge number of labeled images \cite{krizhevsky2012imagenet}. Due to the extreme cost of labeling such an amount of images and difficulty in covering many concepts, researchers recently have looked into methods that generate labels automatically. One significant line of research exploits available labeled images from non-experts (e.g. from social networks, online stores) that can be easily retrieved in large quantities but may have been mislabeled \cite{algan2021image}.
	
	Deep neural networks typically consist of a large number of parameters that are highly shared among feature dimensions and states, enabling flexibility in learning different tasks and classes. This flexibility has the advantage to lead to strong discriminative models unless data annotations are corrupted by noise, leading to performance reduction and overfitting problems \cite{jiang2020beyond}. Recent methods tried to address the problem by using curriculum learning \cite{bengio2009curriculum}, directly estimating the labels noise in the set \cite{hendrycks2018using}, or measuring the confidence of the network during training \cite{kumar2010self}, also using another co-trained network \cite{han2018co}. The idea was usually to understand mislabeled samples out of distribution and reduce their influence on the learning by dampening their loss or decreasing their impact directly from the training set.
	
	In this paper, we proposed a meta-learning approach to address the problem of noisy labels in image classification based on an advisor network, developed to help the classifier. While a standard image classification model is trained, the advisor network observes the main network activations and adjusts features at training time when noisy label images are identified as input. This allows the classifier model to get information even from mislabeled samples where some noise structure is present. We only retained the main model as the final classifier, while the advisor was discarded. Unlike the teacher-student paradigm, the advisor network was not trained to solve the image classification task, but only to help the learning process of the classifier model by its altering activations.   
	
	In summary, our contribution is:
	\begin{itemize}
		\item We propose the use of an advisor network, i.e. the use of an additional network at training time, learned by meta-learning, that can adjust activations and gradient of the main network that is being trained.
		\item We develop such concept for the task of image classification, allowing the training of an image classification network in presence of artificial label noise.
		\item We test our approach in presence of artificial label noise and on a popular noisy dataset, obtaining state-of-the-art performance.  
	\end{itemize}
	
	\section{Related works}
	
	\subsection{Noisy training labels}
	
	Numerous works deal with the problem of noisy labels in training data. It has been shown that the performance of machine learning systems degrades in the presence of label noise \cite{nettleton2010study}, \cite{pechenizkiy2006class}. A first solution involves a loss correction to mitigate the effect of mislabeled samples on the classifier network. For example GLC \cite{hendrycks2018using}, Reed \cite{reed2014training}, M-correction \cite{arazo2019unsupervised}, F-correction \cite{goldberger2016training} and S-adaptation \cite{patrini2017making} estimated the matrix of corruption probabilities used to change the wrong labels to the correct ones. Instead, \cite{tanaka2018joint}, \cite{ma2018dimensionality}, \cite{yi2019probabilistic} modeled the annotations noise distribution linearly combining the output of the network and the noisy label to estimate true labels. Another different approach was assigning a weight to each sample. A lower weight value avoids the contribution of that sample to the training of the network. In this way, it is possible to assign low values to mislabeled examples and high values to correct ones. MentorNet \cite{jiang2018mentornet} and MentorMix \cite{jiang2020beyond} found the latent weights with data-driven curriculum learning and the student-teacher paradigm. Other contributions include data augmentation strategies like Mixup \cite{zhang2018mixup}, Advaug \cite{cheng2020advaug} and DevideMix \cite{li2019dividemix}. Differently from these methods, we modified the network activation using an advisor instead of the loss value.
	
	\subsection{Meta learning}
	
	There are methods \cite{azadi2015auxiliary}, \cite{vahdat2017toward}, \cite{veit2017learning}, \cite{li2017learning} that needs supplemental clean label to handle the noise. This assumption of clean data is also true for a solution that exploits the Meta-learning paradigm. It consists of the use of machine learning algorithms to assist the training and optimization of other machine learning models. Meta-learning \cite{ren2018learning}, \cite{shu2019meta}, \cite{li2019learning}, \cite{wang2020training} had used to address the noisy labels problem. With small clean validation data, the meta-model learns how to correct the biased training labels. For example, L2R \cite{ren2018learning} weighed each example giving less importance to the mislabeled samples. MLNT \cite{li2019learning} simulated regular training with synthetic noisy labels. MW-Net \cite{shu2019meta} learned an explicit weighting function that can be easily adapted to different types of annotations noise. MLC \cite{wang2020training} estimated the label noise transition matrix. Contrary to all aforementioned meta-learning solutions, our method does not act by directly modifying the loss of the neural network. We applied a meta-attention layer inside a neural network. The weights of the attention are learned by the advisor network. In this way, the mislabeled data can be leveraged to improve the overall performance of the main model.
	
	\section{Method}
	
	\subsection{Task}
	
	In this paper, we developed a method that can handle images with noisy labels when training networks for image classification. We started from the idea that also a mislabeled example contains information that can contribute to a greater generalization of the network. The model should concentrate only on some convenient parts of these data. Our idea was to exploit the attention mechanism to enhance the useful parts of the information and lower the rest. We made use of an auxiliary advisor network that learns automatically a function that weighs the features extracted from a DNN during its training. This advisor network should be aware of the state of the main model and the meta-learning training solves this constraint. Our method Meta Feature Re-Weighting (MFRW) acts like a meta-attention layer. Different from weighting loss methods that tend to completely remove the influence of mislabeled examples during the training our MFRW can take advantage of them.
	
	We first introduce meta-learning basics and formulation typical of methods that learn robust deep neural networks from noisy labels. Then in Section \ref{mfrw}, we explain our method showing the architecture of the whole process.
	
	\subsection{Meta learning for Noisy Image Classification}\label{back}
	
	In general meta-learning (ML) is referred to the process of improving a learning algorithm over multiple learning episodes, also called commonly learning to learn. Usually, ML is divided into two learning algorithms: an inner (or base) algorithm that solves a task, such as images classification, defined by a training dataset and objective function; an outer (or upper/meta) algorithm that updates the inner one, such that the main model it learns improves an outer objective function. ML was applied to solve the problem of noisy labels in training data \cite{ren2018learning}, \cite{shu2019meta}. We introduce the symbols useful for understanding ML in this particular setting and the three basic steps into which the entire learning process is divided.
	
	Let  be the noisy annotated training set, where N is the total number of examples, composed of an image  and the correspondent one-hot label  over  classes. In general if we have a deep neural network (DNN) model , where  are its parameters and  is its prediction on the input image , we can obtain the optimal parameters  by minimizing the softmax cross-entropy loss  on the training set . ML, applied to the Noisy Image Classification task, requires the presence of an additional verified dataset. This validation set  is much smaller than the training set, .
	
	In \cite{shu2019meta} a meta-model was used to implement the ML process. A multilayer perceptron network with only one hidden layer learns how to weigh each training example. Let , parameterized by , be the meta-model that maps a loss value to a scalar weight. In this case, the optimal parameters  are derived using the following weighted loss:
	
	
	with  as the weight predicted by the meta model for the -th training example. Instead the meta model is trained minimizing the validation loss:
	
	
	where  is the loss for the -th validation example.
	
	Equations Eq. (\ref{MW1}) and Eq. (\ref{MW2}) can be minimized alternating optimization via gradient descent. One solution that ensures the efficiency of the algorithm and its convergence \cite{shu2019meta} adopts an online strategy to update  and  through a single optimization loop, which is divided into three main steps.
	
	The first step is called Virtual-Train because the original DNN will not be updated and the optimization is carried out on a virtual model that is the copy of the original one. Consider the -th iteration and associated mini batches  and , where  and  are the size of mini-batch respectively. The virtual update can be derived by:
	
	where  is the learning rate for the DNN and  is its parameter at the current iteration. Then there is the Meta-Train step, where given the optimized virtual model the meta model is updated by:
	
	with  the learning rate for the meta model. In last step, Actual-Train, the base DNN model is optimized taking into account the previously updated meta model.
	
	 becomes the  in Eq. (\ref{MW3}) for the -th iteration.
	
	
	\subsection{Meta Feature Re-Weighting (MFRW)}\label{mfrw}
	
	Attention for a DNN is a mechanism that tries to mimic the cognitive attention of the human brain. It intensifies the important parts of the input and reduces the rest. In Meta Feature Re-Weighting (MFRW) the attention is applied with a Hadamard product between the feature extracted from a DNN and a vector of weights automatically learned from a meta-model. In order to get this, we separated the main model  in two-part: the backbone , that has an image  as input and gives out a feature vector , and the classifier , that has  as input and a probability score vector  as output. In this way, it was possible to manipulate the feature  directly with our meta-model .
	
	The meta-model takes two different inputs  and gives back a vector of weights . The first input  is the feature extracted from the backbone  relative to the example . This is important for the meta-model because it makes the  strictly connected to the feature that needs to be modified. The other input is the loss  of the example  calculated from the prediction obtained by the main full model . This gives the meta-model information about how much  is a ``hard" or an ``easy" example for the main model. The two inputs together let the meta-model differentiate a feature belonging to a noisy  from the one related to a correct . In dot-product attention the multiplication is done element-wise, so the  has to be of the same size as , and its values must be in the range .
	
	MFRW is divided into 4 main phases for each iteration. Figure \ref{fig:schema1} shows the overall process of our method divided by step. We put our method at the -th iteration and we will describe each step to reach the -th.
	
	\begin{figure}
		\centering
		\includegraphics[width=0.85\linewidth]{images/schema_all_6}
		\caption{Illustration of an iteration of the proposed Meta Feature Re-Weighting (MFRW) method. Each iteration is divided into four steps. First, a Loss Pre-Calculation is performed to calculate in advance the loss  value of the training batch . The second step is the Virtual-Train, where a clone of the main model is virtually updated. Here the meta-model modifies the feature of the main model by multiplying it with a vector of weights. The purple color indicates the weighted features. The third step shows the Meta-Train process. With a meta batch of clean example  the meta-model is updated minimizing the loss  given by the previous virtually updated network. In the last phase Actual-Train, the real main model is trained with the meta-model optimized (yellow color)}
		\label{fig:schema1}
	\end{figure}
	
	Our method needs an additional initial phase Loss Pre-Calculation respect to \cite{shu2019meta} and what is described in \ref{back}. We must calculate in advance the value of loss  related to the training batch . This is done at the beginning to obtain a loss value dependent on the original feature and not on the weighted one. It is not an expensive step because it is a direct loss inference, without gradient calculation.
	
	The second step is the Virtual-Train. Here  and  are temporary clone of the original ones. The train batch  is passed in  to obtain the features . Then  goes inside  with the relative loss values  to get the vector of weights . We multiplied element-wise  with  to get a new feature vector with attention . This is given to  to obtain the score  and then the correspondent loss . We now virtually update  and  parameters, but not the one of .
	
	Like \cite{shu2019meta} we have a clean and balanced meta dataset that will be used to train the meta-model  in the third step Meta-Train. Here we have  and  virtually updated from the step before. Now we pass a meta batch  through them in order to get a validation loss . Then  is updated minimizing . In this way, the meta-model is optimized to help the main model minimize its error on clean data. Here the optimization takes into consideration also the previous Virtual-Train, thus this is the most expensive part of the method.
	
	The last phase is the Actual-Train where the original  and  are optimized taking into account the updated meta-model .
	
	The meta-model is used only during the training time of the main network. It will be discarded at test time when only the main network is retained as the final model.
	
	\subsection{Meta model architecture}
	
	Our meta-model  has a really simple architecture. The inputs of the network are a feature  and a loss value . Each input is projected in a fixed size embedding space through a separate fully connected layer. Then the embeddings are concatenated and passed to another fully connected layer that projects them into a larger common space. Its size is the sum of the dimension of each previous embedding. Finally, a linear layer is used to pass the data from the common space to a vector with a size equal to the one of the feature , that is given as input. Because the output must be an attention weight in the range  we put a sigmoid activation after the last layer.
	
	
	\section{Experiments}


To demonstrate the effectiveness of our method, we conducted experiments on synthetically generated datasets with controlled noise structure and level. Then we tested its ability to generalize with experiments on a real-world dataset.
	


	
	\subsection{Datasets}
	
	Following previous work \cite{shu2019meta}, \cite{ren2018learning}, \cite{jiang2018mentornet}, we used CIFAR-10 and CIFAR-100 which are the typical choice to generate synthetic datasets containing different types of noise structures. They are composed of 50K training images and 10K test images of size 32×32. Of the training set, 1000 images with clean labels are randomly selected to create the validation set for meta-training.
	
	In addition to synthetic datasets, there is a collection of data containing real-world noise. Clothing1M \cite{xiao2015learning} is a dataset that is composed of 1 million images of clothing taken from online shopping websites. There are  categories like T-shirt, Shirt, Knitwear, etc. The labels are obtained from the text of the images provided by the sellers and not from expert annotators, that's why there are errors. The validation set of  clean data is used as the meta dataset. This dataset allowed our strategy to be evaluated as a concrete application for fine-grained classification with noisy training annotations.
	
	\subsection{Implementation details}
	
	We used the same settings for the experiments on CIFAR-10 and CIFAR-100. The backbone was a Resnet-32 trained through SGD with a momentum of , weight decay of 5e-4, batch size of , and a starting learning rate of 0.1. The learning rate decreased to its  at the 50 epoch and 70 epoch, stopped at the 100 epochs.
	
	With Clothing1M we used as backbone a ResNet-50 pre-trained on ImageNet. It was trained through SGD with a momentum of , weight decay of 1e-3, and a starting learning rate of 0.01. The batch size was  and it was preprocessed resizing the image to 256 × 256, cropping the center 224 × 224, and performing normalization. The learning rate was divided by  after 5 epochs and stopped at 10 epochs.
	
	In every experiment, the meta-model was optimized with Adam and a learning rate of 1e-4. The embedding space size was set always to .
	
	\subsection{Results}\label{noise}
	
		Flip (or asymmetric) is a noise that is designed to mimic the structure where labels are only replaced by similar classes, e.g. dogcat. We choose to test our method on that type of noise because it usually appends that the label error could depend on the ambiguity between classes and similar visual patterns \cite{xiao2015learning}. We created a synthetic version of CIFAR-10 and CIFAR-100. The noise ratio was controlled by a parameter , which represents the probability that a clean example is contaminated by noise. In this way we could test our method on different level of noise, from  (no noise), to  (heavy noise).
	
	\begin{table}[!ht]
		\centering
		\caption{\label{flip} Top-1 accuracy on CIFAR10 and CIFAR100 dataset with Flip noise. The backbone used was a ResNet-32.  denotes the different levels of noise. The results for the cited method are reported directly from their original papers.  indicates the results obtained by our implementation. The first and the second best results are respectively marked in bold and underline.}
		\resizebox{\columnwidth}{!}{
			\begin{tabular}{c|c|c|c|c|c||c|c|c|c|c}
				\hline
				Dataset     & \multicolumn{5}{c||}{Flip CIFAR-10}                                        & \multicolumn{5}{c}{Flip CIFAR-100}                                       \\ \hline
				Noise        & 0.0   & 0.2            & 0.4            & 0.6            & 0.8            & 0.0   & 0.2            & 0.4            & 0.6            & 0.8            \\ \hline
				CrossEntropy \cite{shu2019meta}    & 92.89 & 76.83          & 70.77          & -              & -              & 70.50 & 50.86          & 43.01          & -              & -              \\
				Reed-Hard \cite{reed2014training}  & 92.31 & 88.28          & 81.06          & -              & -              & 69.02 & 60.27          & 50.40          & -              & -              \\ 
				S-Model \cite{goldberger2016training}    & 83.61 & 79.25          & 75.73          & -              & -              & 51.46 & 45.45          & 43.8           & -              & -              \\ 
				Self-paced \cite{kumar2010self} & 88.52 & 87.03          & 81.63          & -              & -              & 67.55 & 63.63          & 53.51          & -              & -              \\ 
				Focal Loss \cite{lin2017focal} & \underline{93.03} & 86.45          & 80.45          & -              & -              & 70.02 & 61.87          & 54.13          & -              & -              \\ 
				Co-teaching \cite{han2018co} & 89.87 & 82.83          & 75.41          & -              & -              & 63.31 & 54.13          & 44.85          & -              & -              \\ 
				D2L \cite{ma2018dimensionality}        & 92.02 & 87.66          & 83.89          & -              & -              & 68.11 & 63.48          & 51.83          & -              & -              \\ 
				Fine-tuning \cite{shu2019meta} & \textbf{93.23} & 82.47          & 74.07          & -              & -              & \textbf{70.72} & 56.98          & 46.37          & -              & -              \\ 
				MentorNet \cite{jiang2018mentornet}   & 92.13 & 86.3           & 81.76          & -              & -              & 70.24 & 61.97          & 52.66          & -              & -              \\ 
				L2RW \cite{ren2018learning}        & 89.25 & 87.86          & 85.66          & -              & -              & 64.11 & 57.47          & 50.98          & -              & -              \\ 
				GLC \cite{hendrycks2018using}        & 91.02 & 89.68          & \underline{88.92}          & -              & -              & 65.42 & 63.07          & \textbf{62.22} & -              & -              \\ 
				MW-net \cite{shu2019meta}     & 92.04 & 90.33          & 87.54          & -              & -              & 70.11 & \underline{64.22}          & 58.64          & -              & -              \\ 
				CrossEntropy     &   92.33    & 90.56          & 86.25          & 26.67          & 13.58          & 70.18       & \textbf{65.02} & 50.25          & 18.67          & 4.32           \\ 
				MW-net \cite{shu2019meta}     &  92.19     & \underline{90.74}          & 87.63          & \underline{42.41}          & \underline{27.19}          & \underline{70.57}      & 64.13          & 51.23          & \underline{19.89}          & \underline{7.42}           \\  \hline
				\textbf{Ours}        &  91.87     & \textbf{91.09} & \textbf{90.26} & \textbf{89.34} & \textbf{82.47} & 68.93      & 63.54          & \underline{59.07}          & \textbf{56.13} & \textbf{20.29} \\ \hline
			\end{tabular}
		}
	\end{table}
	
	Table \ref{flip} shows the accuracy results on the test set of CIFAR-10 and CIFAR-100 datasets with flip label noises. The compared methods are directly cited with the result in their paper. For MW-Net \cite{shu2019meta} and the direct training (CrossEntropy) we report also our reproduced results. The accuracy gained over the other methods was significant. We can see that at a higher noise rate our result outperforms MW-Net and CrossEntropy by a large margin, indicating the effectiveness of our method on the synthetic Flip noise.  From the results of Table \ref{flip} is possible to notice a limitation of our strategy that occurs when there is no noise () in the training annotations. We obtained worse accuracy values than the training with the classic softmax cross-entropy loss on both CIFAR-10 and CIFAR-100. The advisor network introduces a bias from the distribution of the meta set to the training data. Because the training annotations are completely correct the introduction of this meta bias makes the accuracy a little worse than without.
	
	We introduced also two new noise settings, namely Flip2 and Flip3. The difference from Flip is that the noise is equally distributed over multiple similar classes, two and three respectively. Table \ref{flip2}, \ref{flip3} show respectively the result for noise of type Flip2 and Flip3. We can see how our method performs better than the others, especially in very noisy situations.
	
	\begin{table}[!ht]
		\caption{\label{flip2} Accuracy result on CIFAR10 and CIFAR100 dataset with Flip2 noise.  denotes the different level of noise.  indicates the results obtained by our implementation. The first and the second best results are respectively marked with bold and underline.}
		\centering
		\begin{tabular}{c|c|c|c|c||c|c|c|c}
			\hline
			Dataset  & \multicolumn{4}{c||}{Flip2 CIFAR-10} & \multicolumn{4}{c}{Flip2 CIFAR-100} \\ \hline
			Noise     & 0.2        & 0.4       & 0.6       & 0.8       & 0.2        & 0.4        & 0.6       & 0.8        \\ \hline
			CrossEntropy  & \underline{90.71}      & 87.83      & 75.83 & 11.86     & \underline{64.91}      & 57.7       & 36.55     & 7 \\
			MW-net \cite{shu2019meta}   & \textbf{90.93}      & \underline{88.83}      & \underline{86.85} & \underline{27.49}     & \textbf{65.37}      & \textbf{59}         & \underline{36.97}    & \underline{7.99}  \\ \hline
			\textbf{Ours}     & 90.66      & \textbf{89.72}      & \textbf{87.75}    & \textbf{73.83}  & 63.07      & \underline{57.96}      & \textbf{45.35}     & \textbf{22.41} \\ \hline
		\end{tabular}
	\end{table}
	
	\begin{table}[!t]
		\caption{\label{flip3} Result for Flip3 noise on CIFAR10 and CIFAR100 dataset.  denotes the different level of noise.  indicates the results obtained by our implementation. The first and the second best results are respectively marked with bold and underline.}
		\centering
		\begin{tabular}{c|c|c|c|c||c|c|c|c}
			\hline
			Dataset  & \multicolumn{4}{c||}{Flip3 CIFAR-10} & \multicolumn{4}{c}{Flip3 CIFAR-100} \\ \hline
			Noise     & 0.2        & 0.4       & 0.6       & 0.8       & 0.2        & 0.4        & 0.6       & 0.8       \\ \hline
			CrossEntropy  & 90.13      & 88.44     & 82.31    & 20.34 & \underline{65.29}      & \underline{59.35}      & 44      & \underline{11.07}  \\
			MW-net \cite{shu2019meta}   & \textbf{90.56}      & \underline{88.49}     & \underline{85.65}  &\underline{22.69}   & \textbf{65.33}      & \textbf{62.74}      & \underline{45.77}  &10.33   \\ \hline
			\textbf{Ours}     & \underline{90.31}      & \textbf{88.96}     & \textbf{87.73}    & \textbf{75.53} & 62.98      & 59.08      & \textbf{52.28}   &\textbf{25.72}  \\ \hline
		\end{tabular}
	\end{table}
	
	
	Table \ref{clothing} shows the results on Clothing1M. As we can see our method outperforms the current state-of-the-art result.
	
	
	\begin{table}[!t]
		\centering
		\caption{\label{clothing} Comparison with state-of-the-art methods in test accuracy  on Clothing1M dataset. Results for baselines are copied from original papers.}
		\begin{tabular}{l|c}
			\hline
			Method       & Accuracy (\%) \\ \hline
			CrossEntropy \cite{shu2019meta}          & 68.94         \\
			F-correction \cite{patrini2017making} & 69.84         \\
			JoCoR \cite{wei2020combating}       & 70.30         \\
			S-Model \cite{goldberger2016training} & 70.36         \\
			M-correction \cite{arazo2019unsupervised} & 71.00         \\
			MLC \cite{wang2020training}         & 71.06         \\
			Joint-Optim \cite{tanaka2018joint}  & 72.16         \\
			MLNT \cite{li2019learning}        & 73.47         \\
			P-correction \cite{yi2019probabilistic} & 73.49         \\
			MW-Net \cite{shu2019meta}      & 73.72         \\
			MentorMix \cite{jiang2020beyond}   & 74.30         \\
			FaMUS \cite{xu2021faster} & 74.43 \\
			DivideMix \cite{li2019dividemix} & 74.76 \\
			AugDesc \cite{nishi2021augmentation} & 75.11 \\ \hline
			\textbf{Ours}         & \textbf{75.35}         \\ \hline
		\end{tabular}
	\end{table}
	
	\section{Conclusions}
	
	In this paper, we introduced Meta Feature Re-Weighting (MFRW), which makes use of a novel concept of advisor network to mitigate the problem of training DNNs on corrupted labels. We empirically show the effectiveness of our method on a synthetic and real-world noisy dataset for the classification task. The experimental results demonstrate that the advisor strategy can leverage information present in noisy data helping the main network to achieve a better generalization performance. Our method yields state-of-the-art performance on the Clothing1M dataset. Future research in this area may include adapting the advisor network to different problems than noise, like class imbalance.
	
\bibliographystyle{splncs04}
	\bibliography{acmart}
\end{document}
