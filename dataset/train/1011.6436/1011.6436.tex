\newcommand{\eraseGk}[1]{\ensuremath{\left\llbracket #1 \right\rrbracket}}
\newcommand{\eraseUk}[1]{\ensuremath{\left\llbracket #1 \right\rrbracket}}
\newcommand{\eraseTk}[1]{\ensuremath{\left\llbracket #1 \right\rrbracket}}
\newcommand{\erasePk}[1]{\ensuremath{\mathcal{E}\left\llbracket #1 \right\rrbracket}}
\newcommand{\eraseCk}[1]{\ensuremath{\mathcal{C}\left\llbracket #1 \right\rrbracket}}
\newcommand{\eraseGammak}[1]{\ensuremath{\left\llbracket #1 \right\rrbracket}}
\newcommand{\eraseDeltak}[1]{\ensuremath{\left\llbracket #1 \right\rrbracket}}
\begin{figure}[t] \begin{center}
\subfigure[Choice without ]{\includegraphics[width=3.4cm]{gfx/nosync}\label{fig:choice:nosync}\hspace*{8mm}}
\subfigure[Choice using ]{\hspace*{4mm}\includegraphics[width=3.4cm]{gfx/sync}\label{fig:choice:sync}\hspace*{4mm}}
\subfigure[Choice after erasure]{\hspace*{8mm}\includegraphics[width=3.4cm]{gfx/erasure}\label{fig:choice:erasure}}
\end{center}
\vspace{-5mm}
\caption{Synchronisation message flows}
\label{fig:choice}
\end{figure} This section studies an erasure of symmetric synchronisation, which 
translates away symmetric sums using existing session primitives, which we
hereafter simply call {\em the erasure}. The erasure removes all occurrences of
the  constructor while preserving static and dynamic semantics, i.e.
typability and reduction. It uses a conductor process for each
session. The messages and protocol used to implement the
synchronisation are illustrated in Fig.~\ref{fig:choice} where the numbers
indicate the sequence of the messages.
Fig.~\ref{fig:choice}(a) shows the communication between the processes
without using  in Fig.~\ref{fig:example_nosync}.
Fig.~\ref{fig:choice}(b) shows the communication between the processes
using  in Fig.~\ref{fig:example_sync}, where no messages are sent, because
the synchronisation ensures the same branch is chosen.
Fig.~\ref{fig:choice}(c) shows the conduction messages in the
processes where the synchronisation has been erased in
Fig.~\ref{fig:example_erasure}. First the patient, the doctor and the
nurse send the cases they can accept to the conductor, who chooses a common
case and sends the selected case to the patient, the doctor and the nurse.

\begin{figure}[t] \scriptsize
\vspace{1mm}

\2mm]

\
  \eraseCk{G}_{\tilde s,n,a} &=& a[\pp[n+1]](\tilde s, \texttt{in}_{\tilde s,1}, \texttt{out}_{\tilde s,1}, \ldots, \texttt{in}_{\tilde s, n}, \texttt{out}_{\tilde s, n}).\eraseCk{G}^\star_{\tilde s,n} \\
  \eraseCk{\{l: G_l\}_{l \in L;M}}^\star_{\tilde s,n} 
    &=& \texttt{out}_{\tilde s \pp[1]} \triangleright \{\text{cases}_{L_1 \cup M}: \ldots : \texttt{out}_{\tilde s \pp[n]} \triangleright \{\text{cases}_{L_n \cup M}: \\
    & & \RAND\{\texttt{in}_{\tilde s\pp[1]} \triangleleft l;\ldots;\texttt{in}_{\tilde s \pp[n]} \triangleleft l; \eraseCk{G_l}^\star_{n,\tilde s} \}_{l \in \bigcap_{i=1}^n L_i \cup M}
     \}_{L_n \subseteq L} \ldots \}_{L_1 \subseteq L}
\begin{array}{rcl}
  \eraseGk{G} &=& \eraseGk{G}^\star_{\keyword{max}(\keyword{pid}(G)), \keyword{max}(\keyword{sid}(G))} \\
  \eraseGk{\{l: G_l\}_{l \in L;M}}^\star_{n,m}
  &=& \pp[1] \to \pp[n+1]: (m+2) \{cases_{L_1 \cup M}: \\
  & & \pp[2] \to \pp[n+1]: (m+4) \{cases_{L_2 \cup M}: \ldots \\
  & & \pp[n] \to \pp[n+1]: (m+2\cdot n) \{cases_{L_n \cup M}: \\
  & & \pp[n+1] \to \pp[1]: (m+1) \{l: \pp[n+1] \to \pp[2]: (m+3) \{l: \ldots \\
  & & \pp[n+1] \to \pp[n]: (m+2\cdot n-1) \{l: \eraseGk{G_l}^\star_{n,m} \} \ldots \} 
\}_{l \in \bigcap_{i=0}^n L_i \cup M} \}_{L_n \subseteq L} \ldots \}_{L_1 \subseteq L}
\end{array}1mm] Next we prove that process congruence ( is preserved by the erasure.
\begin{THM}[Congruence Preservation] \label{thm:fullabstraction:soundness}
\label{thm:fullabstraction:completeness}
\label{thm:ext:5221}
\BREAK
If  then
for all  we have that  if and only if there is a derivation

such that .
\end{THM} \newcommand{\cin}[1]{\texttt{in}_{#1}}
\newcommand{\cout}[1]{\texttt{out}_{#1}}
\newcommand{\cina}{\texttt{i}_{\pp[1]}}
\newcommand{\couta}{\texttt{o}_{\pp[1]}}
\newcommand{\cinb}{\texttt{i}_{\pp[2]}}
\newcommand{\coutb}{\texttt{o}_{\pp[2]}}
\newcommand{\PC}{\texttt{PC}}
Congruence preservation suggests the erasure preserves semantic properties.
We start by stating the soundness theorem. To do this we define conductors for
partially completed sessions:  as the set of possible partial conductor
processes generated from .
By using the partial conductors from the session environment it is now possible
to state the soundness theorem.
\begin{THM}[Soundness] \label{thm:erasure:semantics}
If , ,
 coherent and 
 for some  then there is
a derivation  and 
 \\
such that  and .
\end{THM} {\sc Proof:} By induction on the derivation of .
\1mm] We can extend the above theorem to multiple
steps by induction on the number of steps.
Also the found evaluation of  performs
exactly the same communication on all non-conductor channels as the original
evaluation .
\newcommand{\tores}{\nobreak{\ensuremath{\rightharpoondown}} }
\newcommand{\toplus}{\nobreak{\ensuremath{\rightharpoonup}} }

We will now define \emph{conduction steps}, since they play an important role
in formulating the completeness theorem.  This is because all steps performed
by the result of the erasure can be mimicked by the original process up to
\emph{conduction steps}.
A step from  to  is a conduction step, written  if the step performs
\emph{label selection} or \emph{label branching} on a conductor channel or
unfolding of a \emph{conductor process}; otherwise we write .
We observe all the extra steps introduced by the erasure are of the form
, while the other steps are of the form . Therefore there is a
one-to-one correspondence between the  steps of the erased process,
and the steps in the original process.

\begin{THM}[Semantic Completeness] \label{thm:erasure:completeness}
If  
then there exists a derivation 
 and  such that  and  and .
\end{THM} {\sc Proof:} By induction on the number of non-conduction steps in ,
using confluence and single-step completeness results.


\paragraph{Healthcare Cooperation (3): Synchronisation Erasure} \begin{figure}[t] \scriptsize
\vspace*{-4mm}
\begin{tabular}{p{9cm}p{7cm}}
\begin{lstlisting}
P_C' = // Conductor
a[4](d,s,r,in_p,out_p,
     in_d,out_d,in_n, out_n).
out_p :>
{cases_DN: out_d :>
 {cases_DN: out_n :>
  {cases_DN:
    rand {
     in_p <: CaseD;
     in_d <: CaseD;
     in_n <: CaseD;
     out_p :>
     {cases_DN: out_d :>
      {cases_DN: out_n :>
       {cases_DN:
         rand
         {in_p <: CaseDD;
          in_d <: CaseDD;
          in_n <: CaseDD;
          end,
          in_p <: CaseDN;
          in_d <: CaseDN;
          in_n <: CaseDN;
          end },
        cases_D: ... },
       cases_D: ... },
      cases_D: ... } },
   cases_N: ... },
  cases_N: ... },
 cases_N: ... }
\end{lstlisting}
&
\begin{lstlisting}
P_P' = // Patient
a[2..4](d,s,r, in_p, out_p,
  in_d, out_d, in_n, out_n).
out_p<:cases_DN;in_p:>
{CaseD: d<<<eData>;out_p<:cases_DN;in_p:>
 {CaseDD: s>>(schedule);r>>(result);0,
  CaseDN: s>>(schedule);r>>(result);0},
 CaseN: d<<<eData>;out_p<:cases_DN;in_p:>
 {CaseND: s>>(schedule);r>>(result);0,
  CaseNN: s>>(schedule);r>>(result);0} }
P_D' = // Doctor
a[2..4](d,s,r, in_p, out_p,
  in_d, out_d, in_n, out_n).
out_d<:cases_DN;in_d:>
{CaseD: d>><data>;out_d<:cases_DN;in_d:>
 {CaseDD: s<<(eSchedule);r<<(eResult);0,
  CaseDN: r<<(eResult);0},
 CaseN: out_d<:cases_DN;in_d:>
 {CaseND: s<<(eSchedule);r>>(eResult);0,
  CaseNN: r>>(eResult);0} }
P_N' = // Nurse
a[2..4](d,s,r, in_p, out_p,
  in_d, out_d, in_n, out_n).
out_n<:cases_DN;in_n:>
{CaseD: out_n<:cases_DN;in_n:>
 {CaseDD: 0,
  CaseDN: s<<(eSchedule);0},
 CaseN: d>><data>;out_n<:cases_DN;in_n:>
 {CaseND: 0,
  CaseNN: s<<(eSchedule);0} }
\end{lstlisting}
\end{tabular}
\vspace*{-5mm}
\caption{Example Processes after Erasure}
\label{fig:example_erasure}
\end{figure} The result of the erasure on the healthcare example from
Section~\ref{sec:typesystem} is shown in Fig.~\ref{fig:example_erasure}.
Since we have shown that the processes from the synchronisation example in
Fig.~\ref{fig:example_sync} are well-typed in
Proposition~\ref{prop:example_sync:typed}, we can apply
Theorem~\ref{thm:erasure:typable} to 
provide . 

As this example illustrates, the result of the erasure does not
capture the nature of the situation in the same way, because it introduces
a conductor process, which is not a natural part of the situation.  It
is not compact either, as the conductor process has  cases. Further we
lose an accurate type abstraction of the dynamics of symmetric
synchronisation, because it is not clear from the encoded type structure
whether it is just a sequence of asymmetric branching actions or
the (intended) atomic multiparty synchronisation, since some of the
key operational structures of the encoding (e.g.~random selection) is
lost in the encoded type.
