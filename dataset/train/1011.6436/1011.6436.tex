\newcommand{\eraseGk}[1]{\ensuremath{\left\llbracket #1 \right\rrbracket}}
\newcommand{\eraseUk}[1]{\ensuremath{\left\llbracket #1 \right\rrbracket}}
\newcommand{\eraseTk}[1]{\ensuremath{\left\llbracket #1 \right\rrbracket}}
\newcommand{\erasePk}[1]{\ensuremath{\mathcal{E}\left\llbracket #1 \right\rrbracket}}
\newcommand{\eraseCk}[1]{\ensuremath{\mathcal{C}\left\llbracket #1 \right\rrbracket}}
\newcommand{\eraseGammak}[1]{\ensuremath{\left\llbracket #1 \right\rrbracket}}
\newcommand{\eraseDeltak}[1]{\ensuremath{\left\llbracket #1 \right\rrbracket}}
\begin{figure}[t] \begin{center}
\subfigure[Choice without $\SYNC$]{\includegraphics[width=3.4cm]{gfx/nosync}\label{fig:choice:nosync}\hspace*{8mm}}
\subfigure[Choice using $\SYNC$]{\hspace*{4mm}\includegraphics[width=3.4cm]{gfx/sync}\label{fig:choice:sync}\hspace*{4mm}}
\subfigure[Choice after erasure]{\hspace*{8mm}\includegraphics[width=3.4cm]{gfx/erasure}\label{fig:choice:erasure}}
\end{center}
\vspace{-5mm}
\caption{Synchronisation message flows}
\label{fig:choice}
\end{figure} This section studies an erasure of symmetric synchronisation, which 
translates away symmetric sums using existing session primitives, which we
hereafter simply call {\em the erasure}. The erasure removes all occurrences of
the $\SYNC$ constructor while preserving static and dynamic semantics, i.e.
typability and reduction. It uses a conductor process for each
session. The messages and protocol used to implement the
synchronisation are illustrated in Fig.~\ref{fig:choice} where the numbers
indicate the sequence of the messages.
Fig.~\ref{fig:choice}(a) shows the communication between the processes
without using $\SYNC$ in Fig.~\ref{fig:example_nosync}.
Fig.~\ref{fig:choice}(b) shows the communication between the processes
using $\SYNC$ in Fig.~\ref{fig:example_sync}, where no messages are sent, because
the synchronisation ensures the same branch is chosen.
Fig.~\ref{fig:choice}(c) shows the conduction messages in the
processes where the synchronisation has been erased in
Fig.~\ref{fig:example_erasure}. First the patient, the doctor and the
nurse send the cases they can accept to the conductor, who chooses a common
case and sends the selected case to the patient, the doctor and the nurse.

\begin{figure}[t] \scriptsize
\vspace{1mm}
$\erasePk{
\MainInfer[Mcast]
     {\Gamma \vdash \overline a[\pp[2..n]](\tilde s).P \rhd \Delta}
     {\Gamma \vdash a: \langle G \rangle
\quad \DD_1::\Gamma \vdash P \rhd \Delta,\tilde s:(G \hup \pp[1])@(\pp[1],n)
\quad \begin{array}[b]{c}|\tilde s|=\keyword{max}(\keyword{sid}(G)) 
\\ n=\keyword{max}(\keyword{pid}(G)) \end{array}
     }
  }  =
\begin{array}{l}
  \eraseCk{G}_{\tilde s,n,a} \mid \\
  \overline a[\pp[2..n,n+1]](\tilde s, \texttt{in}_{\tilde s 1},
                                       \texttt{out}_{\tilde s 1}, \ldots,
                                       \texttt{in}_{\tilde s n},
                                       \texttt{out}_{\tilde s n}).\\
  \erasePk{\DD_1}
\end{array}$
\\[2mm]
$\erasePk{
\MainInfer[Macc]
     {\Gamma \vdash a[\pp[p]](\tilde s).P \rhd \Delta}
     {\Gamma \vdash a: \langle G \rangle
\quad \DD_1::\Gamma \vdash P \rhd \Delta,\tilde s:(G \hup \pp)@(\pp,n)
\quad \begin{array}[b]{c}|\tilde s|=\keyword{max}(\keyword{sid}(G)) 
\\ n=\keyword{max}(\keyword{pid}(G)) \end{array}
     }
  }  = a[\pp](\tilde s, \texttt{in}_{\tilde s1}, \texttt{out}_{\tilde s1}, \ldots, \texttt{in}_{\tilde s n}, \texttt{out}_{\tilde sn},). \erasePk{\DD_1}$
\\[2mm]
$\erasePk{
\MainInfer[Sync]
     {\Gamma \vdash \SYNC[\tilde s,n]\{l: P_l\}_{l \in L'} \rhd \Delta, \tilde s: \{l: T_l\}_{l \in L;M,n}@(\pp,n)}
     {\forall l\in L': \DD_l::\Gamma \vdash P_l \rhd \Delta, \tilde s: T_l@(\pp,n)
\quad L' \subseteq L \cup M
\quad M \subseteq L'
     }
  } =
\texttt{out}_{\tilde s \pp} \triangleleft \text{cases}_{L'}; \texttt{in}_{\tilde s \pp} \triangleright \{l: \erasePk{\DD_l}\}_{l \in L'}
$
\\[2mm]
$\erasePk{
\MainInfer[Label]
     {\Gamma \vdash s_k \triangleleft h; P \rhd \Delta, \tilde s: k \oplus \{l: T_l\}_{l \in L}@(\pp,n)}
     {\DD_1::\Gamma \vdash P \rhd \Delta, \tilde s: T_h@(\pp,n) \quad h \in L}
  }  = s_k \triangleleft h; \texttt{out}_{\tilde s\pp} \triangleleft h;\erasePk{\DD_1}$
\\
\normalsize
\center
The other cases are defined monomorphic
\caption{Erasure of Synchronisation from Typing-Derivation}
\label{fig:erasure:process}
\end{figure} \begin{figure} \small
\vspace{-3mm}
\begin{center}
\begin{eqnarray*}
  \eraseCk{G}_{\tilde s,n,a} &=& a[\pp[n+1]](\tilde s, \texttt{in}_{\tilde s,1}, \texttt{out}_{\tilde s,1}, \ldots, \texttt{in}_{\tilde s, n}, \texttt{out}_{\tilde s, n}).\eraseCk{G}^\star_{\tilde s,n} \\
  \eraseCk{\{l: G_l\}_{l \in L;M}}^\star_{\tilde s,n} 
    &=& \texttt{out}_{\tilde s \pp[1]} \triangleright \{\text{cases}_{L_1 \cup M}: \ldots : \texttt{out}_{\tilde s \pp[n]} \triangleright \{\text{cases}_{L_n \cup M}: \\
    & & \RAND\{\texttt{in}_{\tilde s\pp[1]} \triangleleft l;\ldots;\texttt{in}_{\tilde s \pp[n]} \triangleleft l; \eraseCk{G_l}^\star_{n,\tilde s} \}_{l \in \bigcap_{i=1}^n L_i \cup M}
     \}_{L_n \subseteq L} \ldots \}_{L_1 \subseteq L}
\end{eqnarray*}
\end{center}
\vspace{-3.5mm}
\caption{Conductor Process Generation from a Global Type}
\label{fig:erasure:conductor}
\end{figure} \subsection{Erasure Definitions} Based on this idea, we translate the synchronisation and symmetric sum types
into the original system \cite{CHY08}, step by step as follows.

\paragraph{Step 1: Process Erasure} Only well-typed processes are eligible for erasure, because conductor
processes are generated from the global types. Therefore the erasure
$\erasePk{\cdot}$ is defined on the type derivation in Fig.~\ref{fig:erasure:process} and the
result is the erased process.
We use the notation $\DD::\Gamma \vdash P \rhd \Delta$ to denote a derivation $\DD$
with the conclusion $\Gamma \vdash P \rhd \Delta$.

The case for session request increments the number of participants by one, to make room for the conductor process,
and adds two session channels per user ($\texttt{in}_{\tilde s,\pp}$ and
$\texttt{out}_{\tilde s,\pp}$), for communicating with the conductor.
The conductor process $\eraseCk{G}_{\tilde s,n,a}$ (defined in Step 2) is inserted in
parallel with the resulting session requesting process to ensure it is
available.

The case for synchronisation sends the accepted labels to the conductor, waits
to receive one of the accepted labels and proceeds with the selected branch.




\paragraph{Step 2: Conductor Generation} The conductor process $\eraseCk{G}_{\tilde s,n,a}$ was inserted in
parallel with the session requests by the process erasure in Step 1.
The main cases of the conductor generation $\eraseCk{\cdot}$ are in
Fig.~\ref{fig:erasure:conductor}.
Notice that $\eraseCk{G}_{\tilde s,n,a}$ is only a wrapper for
$\eraseCk{G}^\star_{\tilde s,n}$ which prefixes the session
acceptance on channel $a$.  In $\eraseCk{G}_{\tilde s,n,a}$,
$\tilde s$ is the original session channels, $n$ is the
number of original participants, $G$ is the original session type, and $a$ is
the channel the session is created over.

The conductor process generated from a synchronisation receives the accepted
labels from each participant, selects a common label using $\RAND$ and sends
the selected label back to each participant before conducting the chosen
branch.


\paragraph{Step 3: Type Translations} \begin{figure} \footnotesize
\vspace*{-2mm}
\begin{center}
 \[\begin{array}{rcl}
  \eraseGk{G} &=& \eraseGk{G}^\star_{\keyword{max}(\keyword{pid}(G)), \keyword{max}(\keyword{sid}(G))} \\
  \eraseGk{\{l: G_l\}_{l \in L;M}}^\star_{n,m}
  &=& \pp[1] \to \pp[n+1]: (m+2) \{cases_{L_1 \cup M}: \\
  & & \pp[2] \to \pp[n+1]: (m+4) \{cases_{L_2 \cup M}: \ldots \\
  & & \pp[n] \to \pp[n+1]: (m+2\cdot n) \{cases_{L_n \cup M}: \\
  & & \pp[n+1] \to \pp[1]: (m+1) \{l: \pp[n+1] \to \pp[2]: (m+3) \{l: \ldots \\
  & & \pp[n+1] \to \pp[n]: (m+2\cdot n-1) \{l: \eraseGk{G_l}^\star_{n,m} \} \ldots \} 
\}_{l \in \bigcap_{i=0}^n L_i \cup M} \}_{L_n \subseteq L} \ldots \}_{L_1 \subseteq L}
\end{array}\]
\end{center}
\vspace*{-3mm}
\caption{Erasure Mapping for Global Types}
\label{fig:erasure:global}
\end{figure} To prove that typability is preserved by the erasure, we define translations of
global types, local types, message types, global type environments and local
type environments to find the types for the result of the erasure.  The main cases for global types are defined in Fig.~\ref{fig:erasure:global}.
The translation $\eraseGk{G}$ of global types is just a wrapper for
$\eraseGk{G}^\star_{n,m}$ where $n$ is the number of participants,
and $m$ is the number of session channels in the original type.

As previously suggested, the symmetric sum is translated to nested branching,
where each participant sends the accepted labels to the conductor, receives
the selected label and continues with the selected branch.


\subsection{Correctness} We now prove the correctness of the erasure mapping. 
We start by proving that the typing is preserved, and
the types of the result process is given by the defined type translations.
\begin{THM}[Type Preservation] \label{thm:erasure:typable}
If $ \DD::\Gamma \vdash P \rhd \Delta$
then $\eraseGammak{\Gamma} \vdash \erasePk{\DD} \rhd \eraseDeltak{\Delta}$
\end{THM} {\sc Proof:} By induction on the type derivation $\DD$.
The proof uses a lemma stating that the generated conductor processes are
well-typed.
\\[1mm] Next we prove that process congruence ($P \equiv Q)$ is preserved by the erasure.
\begin{THM}[Congruence Preservation] \label{thm:fullabstraction:soundness}
\label{thm:fullabstraction:completeness}
\label{thm:ext:5221}
\BREAK
If $\DD_1::\Gamma \vdash P \rhd_{\tilde t} \Delta$ then
for all $Q$ we have that $P \equiv Q$ if and only if there is a derivation
$\DD_2::\Gamma \vdash Q \rhd_{\tilde t} \Delta$
such that $\erasePk{\DD_1} \equiv \erasePk{\DD_2}$.
\end{THM} \newcommand{\cin}[1]{\texttt{in}_{#1}}
\newcommand{\cout}[1]{\texttt{out}_{#1}}
\newcommand{\cina}{\texttt{i}_{\pp[1]}}
\newcommand{\couta}{\texttt{o}_{\pp[1]}}
\newcommand{\cinb}{\texttt{i}_{\pp[2]}}
\newcommand{\coutb}{\texttt{o}_{\pp[2]}}
\newcommand{\PC}{\texttt{PC}}
Congruence preservation suggests the erasure preserves semantic properties.
We start by stating the soundness theorem. To do this we define conductors for
partially completed sessions: $\PC(\Delta)$ as the set of possible partial conductor
processes generated from $\Delta$.
By using the partial conductors from the session environment it is now possible
to state the soundness theorem.
\begin{THM}[Soundness] \label{thm:erasure:semantics}
$ $If $\DD::\Gamma \vdash P \rhd_{\tilde t} \Delta$, $P \to P'$,
$\Delta$ coherent and 
$P_C \in \PC(\Delta \circ \Delta'')$ for some $\Delta''$ then there is
a derivation $\DD'::\Gamma \vdash P' \rhd_{\tilde t} \Delta'$ and 
$P_C' \in \PC(\Delta' \circ \Delta'')$ \\
such that $\Delta \to^{0/1} \Delta'$ and $\erasePk{\DD} | P_C \to^\star \erasePk{\DD'} | P_C'$.
\end{THM} {\sc Proof:} By induction on the derivation of $P \to P'$.
\\[1mm] We can extend the above theorem to multiple
steps by induction on the number of steps.
Also the found evaluation of $\erasePk{\DD} \to^\star \erasePk{\DD'}$ performs
exactly the same communication on all non-conductor channels as the original
evaluation $P \to^\star P'$.
\newcommand{\tores}{\nobreak{\ensuremath{\rightharpoondown}} }
\newcommand{\toplus}{\nobreak{\ensuremath{\rightharpoonup}} }

We will now define \emph{conduction steps}, since they play an important role
in formulating the completeness theorem.  This is because all steps performed
by the result of the erasure can be mimicked by the original process up to
\emph{conduction steps}.
A step from $P_1$ to $P_2$ is a conduction step, written $P_1 \tores P_2$ if the step performs
\emph{label selection} or \emph{label branching} on a conductor channel or
unfolding of a \emph{conductor process}; otherwise we write $P_1 \toplus P_2$.
We observe all the extra steps introduced by the erasure are of the form
$\tores$, while the other steps are of the form $\toplus$. Therefore there is a
one-to-one correspondence between the $\toplus$ steps of the erased process,
and the steps in the original process.

\begin{THM}[Semantic Completeness] \label{thm:erasure:completeness}
If $\erasePk{\DD_1::\Gamma \vdash P_1 \rhd \emptyset}
\to^\star Q'$ 
then there exists a derivation 
$\DD_2::\Gamma \vdash P_2 \rhd \emptyset$ and $Q$ such that $P_1 \to^\star P_2$ and $\erasePk{\DD_2} \tores^\star Q$ and $Q' \tores^\star Q$.
\end{THM} {\sc Proof:} By induction on the number of non-conduction steps in $\erasePk{\DD_1} \to^\star Q'$,
using confluence and single-step completeness results.


\paragraph{Healthcare Cooperation (3): Synchronisation Erasure} \begin{figure}[t] \scriptsize
\vspace*{-4mm}
\begin{tabular}{p{9cm}p{7cm}}
\begin{lstlisting}
P_C' = // Conductor
a[4](d,s,r,in_p,out_p,
     in_d,out_d,in_n, out_n).
out_p :>
{cases_DN: out_d :>
 {cases_DN: out_n :>
  {cases_DN:
    rand {
     in_p <: CaseD;
     in_d <: CaseD;
     in_n <: CaseD;
     out_p :>
     {cases_DN: out_d :>
      {cases_DN: out_n :>
       {cases_DN:
         rand
         {in_p <: CaseDD;
          in_d <: CaseDD;
          in_n <: CaseDD;
          end,
          in_p <: CaseDN;
          in_d <: CaseDN;
          in_n <: CaseDN;
          end },
        cases_D: ... },
       cases_D: ... },
      cases_D: ... } },
   cases_N: ... },
  cases_N: ... },
 cases_N: ... }
\end{lstlisting}
&
\begin{lstlisting}
P_P' = // Patient
a[2..4](d,s,r, in_p, out_p,
  in_d, out_d, in_n, out_n).
out_p<:cases_DN;in_p:>
{CaseD: d<<<eData>;out_p<:cases_DN;in_p:>
 {CaseDD: s>>(schedule);r>>(result);0,
  CaseDN: s>>(schedule);r>>(result);0},
 CaseN: d<<<eData>;out_p<:cases_DN;in_p:>
 {CaseND: s>>(schedule);r>>(result);0,
  CaseNN: s>>(schedule);r>>(result);0} }
P_D' = // Doctor
a[2..4](d,s,r, in_p, out_p,
  in_d, out_d, in_n, out_n).
out_d<:cases_DN;in_d:>
{CaseD: d>><data>;out_d<:cases_DN;in_d:>
 {CaseDD: s<<(eSchedule);r<<(eResult);0,
  CaseDN: r<<(eResult);0},
 CaseN: out_d<:cases_DN;in_d:>
 {CaseND: s<<(eSchedule);r>>(eResult);0,
  CaseNN: r>>(eResult);0} }
P_N' = // Nurse
a[2..4](d,s,r, in_p, out_p,
  in_d, out_d, in_n, out_n).
out_n<:cases_DN;in_n:>
{CaseD: out_n<:cases_DN;in_n:>
 {CaseDD: 0,
  CaseDN: s<<(eSchedule);0},
 CaseN: d>><data>;out_n<:cases_DN;in_n:>
 {CaseND: 0,
  CaseNN: s<<(eSchedule);0} }
\end{lstlisting}
\end{tabular}
\vspace*{-5mm}
\caption{Example Processes after Erasure}
\label{fig:example_erasure}
\end{figure} The result of the erasure on the healthcare example from
Section~\ref{sec:typesystem} is shown in Fig.~\ref{fig:example_erasure}.
Since we have shown that the processes from the synchronisation example in
Fig.~\ref{fig:example_sync} are well-typed in
Proposition~\ref{prop:example_sync:typed}, we can apply
Theorem~\ref{thm:erasure:typable} to 
provide $\nobreak{a:\langle \eraseGk{G} \rangle \vdash P'_C \mid P'_P
\mid P'_D \mid P'_N \rhd \emptyset}$. 

As this example illustrates, the result of the erasure does not
capture the nature of the situation in the same way, because it introduces
a conductor process, which is not a natural part of the situation.  It
is not compact either, as the conductor process has $64$ cases. Further we
lose an accurate type abstraction of the dynamics of symmetric
synchronisation, because it is not clear from the encoded type structure
whether it is just a sequence of asymmetric branching actions or
the (intended) atomic multiparty synchronisation, since some of the
key operational structures of the encoding (e.g.~random selection) is
lost in the encoded type.
