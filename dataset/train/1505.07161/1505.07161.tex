\documentclass[submission,copyright,creativecommons]{eptcs}
\usepackage[utf8]{inputenc}
\usepackage[english]{babel}
\usepackage{amsmath,amsfonts,amsthm,amssymb}
\usepackage{stmaryrd}
\usepackage[all]{xy}
\xymatrixrowsep{2ex}
\xymatrixcolsep{2ex}
\hypersetup{
  pdftitle={Presenting Finite Posets},
  pdfauthor={Samuel Mimram},
  unicode=true,
  colorlinks=true,
  linkcolor=black,
  citecolor=black,
  urlcolor=black
}
\let\P\undef
\let\C\undef
\usepackage{macros}
\usepackage{graphicx}

\relpenalty=10000
\binoppenalty=10000

\renewcommand{\labelitemi}{--}

\newtheorem{theorem}{Theorem}
\newtheorem{proposition}[theorem]{Proposition}
\newtheorem{lemma}[theorem]{Lemma}
\newtheorem{corollary}[theorem]{Corollary}
\theoremstyle{definition}
\newtheorem{definition}[theorem]{Definition}
\theoremstyle{remark}
\newtheorem{example}[theorem]{Example}
\newtheorem{remark}[theorem]{Remark}
\newtheorem{notation}[theorem]{Notation}

\renewcommand{\leq}{\leqslant}
\renewcommand{\geq}{\geqslant}

\providecommand{\event}{TERMGRAPH 2014}
\newcommand{\titlerunning}{Presenting Finite Posets}
\newcommand{\authorrunning}{Samuel Mimram}
\title{\titlerunning}
\author{\authorrunning
\institute{CEA, LIST / École Polytechnique\thanks{This work was partially funded by the french ANR project CATHRE ANR-13-BS02-0005-02.}}
\email{\href{mailto:samuel.mimram@lix.polytechnique.fr}{samuel.mimram@lix.polytechnique.fr}}
}

\begin{document}
\maketitle

\begin{abstract}
  We introduce a monoidal category whose morphisms are finite partial orders,
  with chosen minimal and maximal elements as source and target
  respectively. After recalling the notion of presentation of a monoidal
  category by the means of generators and relations, we construct a presentation
  of our category, which corresponds to a variant of the notion of bialgebra.
\end{abstract}

String rewriting systems have been originally introduced by
Thue~\cite{thue1914probleme} in order to study word problems in monoids. A
string rewriting system  consists of a set , called the
\emph{alphabet}, and a set  of \emph{rules}.
The monoid , obtained by quotienting the free monoid
 over  by the smallest congruence (\wrt concatenation)
containing , is called the monoid \emph{presented} by the rewriting system.
The rewriting system can thus be thought of as a small description of the
monoid, and the word problem consists in deciding whenever two words
 represent the same word, \ie are such that . Now,
when the rewriting system is convergent, \ie both terminating and confluent,
normal forms provide canonical representatives of the equivalence classes: two
words  are equivalent by the congruence  if and only
if they have the same normal form, and the word problem can be thus be decided
in this case.

\begin{example}
  \label{ex:pres-monoid}
  Consider the rewriting system  with  and
  , where  denotes the empty
  word. This rewriting system is easily shown to be terminating, and the two
  critical pairs can be joined:
  
  the system is thus convergent. Normal forms are words of the form  or
  , with , and from this it is easy to deduce that the rewriting
  system presents the additive monoid : there is an obvious
  bijection between the set of normal forms and the elements of the monoid, and
  it remains to be checked that this bijection is compatible with product and
  units.
\end{example}

Starting from this point of view, it is natural to wonder whether the notion of
rewriting system can be extended to more general settings, in particular to
provide presentations for -categories (a monoid can be seen as the particular
case of a -category with only one object). A satisfactory answer to this
question was provided by Street and Power's
\emph{computads}~\cite{street1976limits,power1991n}, which were rediscovered as
Burroni's \emph{polygraphs}~\cite{burroni1993higher}, and really deserve the
name of \emph{higher-dimensional rewriting systems} at the light of the
preceding comments~\cite{mimram:trt}. Those were the subject of my introductory
invited presentation at the TERMGRAPH workshop in 2014, along with the way usual
rewriting techniques generalize to this framework.

The aim of this article is to describe a particular example of a presentation of
a monoidal category (\ie a 2-category with one 0-cell) by a 2-dimensional
rewriting system, whose morphisms are finite partial orders. This example was
discovered during the author's PhD thesis~\cite{mimram:phd} (Section~4.3.1)
while studying asynchronous game semantics. We will focus here on this example
only and introduce only the material which is necessary to handle it: we will
only define presentations for monoidal categories (see~\cite{burroni1993higher}
for the general definition of higher-dimensional rewriting system) and we will
not mention the links to game semantics
(see~\cite{mimram:phd,mellies2007asynchronous,mimram2011structure} for a more
detailed account of the relationships between game semantics, partial orders,
and presentations of categories). The proof has been split in a series of simple
enough lemmas, which is why we have left out most proofs.

\paragraph{Related works.}
The general methodology used here is strongly inspired from the one developed by
Lafont who studied many examples of 2-dimensional
presentations~\cite{lafont2003towards}. A variant of this example was studied by
Fiore and Devesas Campos~\cite{fiore2013algebra} using similar
techniques. Finally, a very interesting tool, based on distributive laws between
monads, has been introduced by Lack in order to build presentations of monoidal
categories by combining presentations of smaller (and simpler) monoidal
subcategories~\cite{lack2004composing}: a particularity of the present example
is that this technique does not apply here.



\paragraph{Plan.}
We begin by formally introducing the category~ of finite posets we will be
interested in giving a presentation in Section~\ref{sec:finpos}, detail what we
mean by a presentation of a monoidal category in Section~\ref{sec:presenting},
and give a few examples of presentations along with the one for  in
Section~\ref{sec:poalgebras}. The rest of the article is devoted to proving that
we actually have a presentation: we define a canonical factorization of
morphisms of  in Section~\ref{sec:cf}, which is used to finish the proof in
Section~\ref{sec:pres}. We conclude and hint at possible generalizations of this
result in Section~\ref{sec:concl}.


\section{A category of finite posets}
\label{sec:finpos}
A \emph{poset} is a pair  consisting of a set
, whose elements are called \emph{events}, and a relation
 which is reflexive, antisymmetric
and transitive. Two events  are \emph{independent} when
neither  nor .

\begin{definition}
  We write  for the category whose objects are posets and morphisms are
  increasing functions.
\end{definition}

\noindent
This category can be shown to be cocomplete~\cite{adamek1994locally}, and in
particular pushouts always exist. There is a full and faithful inclusion functor
, such that the image of a set  is the poset , and we
will allow ourselves to implicitly see a set as a poset in this way: a poset in
the image of this functor is called \emph{discrete}. Given an integer~,
we write  for the set  (or the associated
discrete poset), and  for the totally ordered poset
. Given a poset  and a set , we
write  for the poset such that  and  is the restriction of  to this
set.

The main category of interest in this article is defined as follows:

\begin{definition}
  We write  for the category whose objects are natural numbers , and
  morphisms  are equivalence classes of triples
  consisting of
  \begin{itemize}
  \item a finite poset ,
  \item a monomorphism , called \emph{source}, whose images are
    minimal events in the poset,
  \item a monomorphism , called \emph{target}, whose images are
    maximal events in the poset,
  \end{itemize}
  quotiented by the equivalence relation identifying two such triples 
  and  whenever there exists an isomorphism  such that
   and :
  
  Given two composable morphisms  and , we
  consider the poset  defined by the following pushout in 
  
  and define their composite as
  
  The identity on an object  is . We leave to the
  reader as an easy exercise to check that composition is well-defined and that
  the axioms of categories are satisfied (and more generally, we will leave to
  the reader the check that constructions subsequently performed on the category
  are compatible with the equivalence relation). An event  in a
  morphism  is called \emph{external} if it is in the image of
  either  or , \ie , and \emph{internal}
  otherwise.
\end{definition}

\begin{remark}
  The above definition is a variant of the well-known construction of the
  bicategory of spans~\cite{benabou1967introduction} inside a category with
  pullbacks such as~.
\end{remark}

\begin{example}
  \label{ex:P-composition}
  The morphism  with  with ,
   and , , , , ,  will be
  drawn by its Hasse diagram, \ie the graph of its immediate successor relation:
  
  The bullets represent events of the poset, the filled ones denoting internal
  events and empty ones denoting external events, and events are increasing from
  bottom to top. Notice that the names of the bullets do not really matter,
  because of the quotient used when defining morphisms, and we will not figure
  them in the following. The image of the source (\resp target) is figured by
  the empty bullets at the bottom (\resp top), where the  (\resp )
  are always represented with  increasing from left to right. Composition is
  performed by ``gluing'' diagrams along the interface (by pushout) and erasing
  the external events of this interface:
  
\end{example}

Notice that this category is self-dual, in the sense that there is an
isomorphism , which is the identity on objects and respects the
monoidal structure introduced in next section
(Definition~\ref{def:P-monoidal}). Also, it contains the category of relations
as a subcategory:

\begin{definition}
  \label{def:rel}
  We write  for the category with integers as objects, such that a
  morphism  is a relation . The
  composite of two morphisms  and  is the relation  such that  if and only if there exists
   such that  and . The identity
  relation  is such that  if and only if .
\end{definition}

\noindent
The following lemma will allow us to implicitly see a relation as a morphism in
the category :

\begin{lemma}
  \label{lemma:rel-poset}
  The following functor  is faithful. The functor is the identity on
  objects, and the image of a relation  is the morphism , with , such that for any two
  elements  and  of  we have  if and only if , and the maps  and
   are the injections of the coproduct.
\end{lemma}

\begin{example}
  \label{ex:rel-P}
  The relation  with  can be seen as
  the following morphism of~:
  
\end{example}

\section{Presenting monoidal categories}
\label{sec:presenting}
We recall that a \emph{strict monoidal category}  consists of a
category  together with a functor  called
\emph{tensor product} and an object  called \emph{unit} such that the
tensor is associative and admits  as unit: for every objects ,  and
,  and . In this article, we only consider monoidal categories which are strict. A
\emph{symmetry} in such a category consists of a natural transformation of
components  such that for every objects
, ,  we have

A functor between monoidal categories is \emph{monoidal} when it respects the
tensor product and the unit, and we moreover suppose that monoidal functors
between symmetric monoidal categories preserve the symmetry. We write 
for the category of monoidal categories and monoidal functors.

\begin{definition}
  \label{def:P-monoidal}
  The category  can be made into a monoidal category with  as unit, and
  tensor being defined by addition on objects () and by disjoint
  union on morphisms. Moreover, a symmetry can easily be defined.
\end{definition}

\noindent
A monoidal category, such as , with integers as objects and tensor product
being given on objects by addition is often called a \emph{PRO}, or a
\emph{PROP} when it is additionally equipped with a
symmetry~\cite{maclane1965categorical}.

\begin{example}
  Using the diagrammatic notations of Example~\ref{ex:P-composition}, we have
  
\end{example}

In order to define presentations of monoidal categories, we first need to define
an appropriate notion of signature for them. Since those contain objects and
morphisms, a signature will consist of generators for both of them. For the sake
of brevity, we follow here the formalization specific to monoidal categories
(also sometimes called \emph{tensor schemes}~\cite{joyal1991geometry}),
see~\cite{burroni1993higher} for the general case.


\begin{definition}
  A \emph{(monoidal) signature}  consists of
  \begin{itemize}
  \item a set  of \emph{object generators},
  \item a set  of \emph{morphism generators},
  \item two functions , where 
    denotes the free monoid over , assigning to a morphism generator
    its \emph{source} and \emph{target} respectively.
  \end{itemize}
  The category  has signatures as objects and a morphism
   is a pair
   of functions  and
   such that  and
  , where  is
  the extension of  as a morphism of monoids.
\end{definition}

A monoidal signature generates a free monoidal category in the following
sense. Given a category~, we write  for the class of morphisms
of~.

\begin{proposition}
  Consider the forgetful functor  sending a monoidal
  category  to the signature
  , with  being the set of objects
  of ,  consisting of triples  such that
  , and
  functions  and  are given by first and third projection
  respectively. This functor admits a left adjoint . Given
  a signature , the monoidal category  is often denoted
   and called the \emph{free monoidal category} on the signature.
More explicitly,  is the smallest category whose class of objects is
  , for every morphism generator  there is a morphism
  , for every object there is a formal identity on
  this object, morphisms are closed under formal composites and tensor products,
  and are quotiented by the axioms of monoidal categories.
\end{proposition}

\begin{remark}
  \label{rem:free-moncat}
  Suppose fixed a signature  and a monoidal
  category . Suppose moreover that we are given an object 
  for every object generator , and a morphism
   for
  every morphism generator  with 
  and . Then, by the freeness property of ,
  the operation  extends uniquely as a monoidal functor .
\end{remark}



\begin{remark}
  \label{rem:1-object}
  When the set  is reduced to one element,  is
  isomorphic to , and we denote by integers its elements. In this case, the
  monoidal category  is a PRO.
\end{remark}

\noindent
The following lemma is often useful in order to prove properties of morphisms of
 by induction over the number of generators they consist in:

\begin{lemma}
  \label{lemma:free-moncat-slice}
  Every morphism  of  can be written as a finite composite
   where each morphism~ is of the form
   for some objects  and
   and morphism generator~.
\end{lemma}

The notion of rewriting system, adapted to monoidal categories, can finally be
defined as follows. We say that two morphisms in a category are \emph{parallel}
when they have the same source, and the same target.

\begin{definition}
  A \emph{(monoidal) rewriting system} is a pair  consisting of a
  signature  and a set  of
  pairs of morphisms which are parallel. The monoidal category \emph{presented}
  by such a rewriting system is the monoidal category 
  obtained by quotienting the morphisms of~ by the smallest
  congruence~, \wrt composition and tensor product, containing~.
\end{definition}

\begin{remark}
  \label{rem:free-moncat-rel}
  Since  is the congruence generated by~, a functor
   such that  for every  induces a
  quotient functor .
\end{remark}

\noindent
We often write  for a rule . We recall
below the most well-known example of presentation of a monoidal category: the
simplicial category. It is fundamental since it is at the heart of simplicial
algebraic topology.

\begin{example}
  \label{ex:pres-moncat}
  Consider the signature  with
  , , 
  (where~ denotes the empty word), , and
  . As explained in Remark~\ref{rem:1-object}, since the
  set~ contains only one generator, the monoid  is
  isomorphic to . The morphism generators can thus be denoted
  
  along with their source and target. We will use this notation for defining
  morphism generators in the following. We consider the set of rules defined by
  
  If we use the string diagrammatic notation for morphisms, and draw  and
   respectively as on the left below, the rules can be figured as on the
  right:
  
  (notice that diagrams have to be read from bottom up, in order to be
  consistent with the notation for posets of Section~\ref{sec:finpos}). The
  rewriting system can be shown to be terminating and the five critical pairs
  can be joined, it is thus convergent. Normal forms are tensor products of
  ``right combs'', \eg
  
  Now, consider the \emph{simplicial category}~, with integers as
  objects, and such that a morphism  is an increasing function
  . It is monoidal when equipped with the expected tensor product
  given on objects by addition, \ie it is a PRO. With the above graphical
  representation in mind, it is easy to see that normal forms for the rewriting
  system are in bijection with morphisms of~, and that this bijection is
  compatible with composition and tensor product. For instance, the normal
  form~\eqref{eq:ex-rcomb} corresponds to the function  such that
   if , and  if ,
  see~\cite{lafont2003towards, mimram:trt} for details.
\end{example}

The rewriting system described in previous example corresponds to the theory of
monoids: from the above presentation, we immediately deduce that, given a
monoidal category~, monoidal functors  are in bijection with
monoids in~, see Definition~\ref{def:monoid} and~\cite{maclane:cwm} (and
moreover monoidal natural transformations correspond to morphisms of
monoids). We will thus call the rewriting system the \emph{theory} of monoids
(the ``the'' is slightly abusive since there are multiple choices for
orientations of rules and even of generators, but those will give the same
presented category).

The proof that the theory of monoids is a presentation of the simplicial
category, that we sketched in Example~\ref{ex:pres-moncat}, is a direct
adaptation of the classical rewriting argument in the case of monoids presented
in Example~\ref{ex:pres-monoid}. There are some theories for which no
orientation of the rules gives rise to a convergent rewriting system. However,
what matters here is only the fact that we have canonical representatives for
equivalence classes of morphisms in  modulo the congruence generated
by the rules: we can hope to define them ``by hand'' (in which case we call them
\emph{canonical forms}), instead of obtaining them as normal forms for a
rewriting system. In order to show that a rewriting system~ is a
presentation for a monoidal category~, one can thus apply the following
recipe discovered by Lafont~\cite{lafont2003towards}:
\begin{enumerate}
\item define a functor  which is bijective on objects by
  interpreting generators of  into , see
  Remark~\ref{rem:free-moncat},
\item show that it induces a functor  by verifying
  that every pair of morphisms in  have the same image, see
  Remark~\ref{rem:free-moncat-rel},
\item define (\eg inductively) a subset of morphisms in , whose
  elements are called canonical forms,
\item show that every morphism in  is equivalent by  to a
  canonical form by induction, see Lemma~\ref{lemma:free-moncat-slice},
\item show that the functor~ restricted to canonical forms is a bijection.
\end{enumerate}
The first two steps allow us to define a functor ,
interpreting the presented monoidal category into~ and step 3 and 4 amount
to choose at least one representative in each equivalence class of morphisms
modulo . Finally, step 5 allows us to conclude. Namely, the
functor~ is full since every morphism of  is the image of the equivalence
class of the corresponding canonical form. The functor is also faithful: given
two morphisms  and  such that , by step~4 there is a canonical
form~ associated to  and a canonical form  associated to~
and those are necessarily equal by step~5, and therefore . Notice that we
can conclude \emph{a posteriori} that each equivalence class contains exactly
one canonical form.
In the following, we use a variant of this methodology in order to build a
presentation for the monoidal category~. We will define the presentation and
interpret it in Section~\ref{sec:pres}.

\section{The theory of poalgebras}
\label{sec:poalgebras}
In this section, we first introduce some classical algebraic structures and
provide the monoidal category presented by the associated theory (\ie rewriting
system), and then define the algebraic structure corresponding to the
category~, which is a variant of those. We suppose fixed a monoidal
category~ equipped with a symmetry~.

\begin{definition}
  \label{def:monoid}
  A \emph{monoid}  in  consists of an object  together with
  two morphisms  and  such that
   and
  . Such a monoid
  is \emph{commutative} when . A \emph{comonoid}
   is a monoid in .
\end{definition}

\begin{proposition}[\cite{maclane:cwm, lafont2003towards}]
  \label{prop:free-monoid}
  The theory for monoids presents the simplicial category~, see
  Example~\ref{ex:pres-moncat}. The theory for commutative monoids presents the
  PROP~ such that a morphism  is a function
  .
\end{proposition}

\begin{definition}
  \label{def:bialg}
  A \emph{bialgebra}  consists of a monoid
   and a comonoid  such that
  
  and . Such a bialgebra is \emph{bicommutative}
  when both the monoid and the comonoid structure are commutative, and
  \emph{qualitative} when moreover .
\end{definition}

\begin{proposition}[\cite{maclane1965categorical, hyland2000symmetric, pirashvili2001prop, lafont2003towards, lack2004composing, mimram2011structure}]
  \label{prop:free-bialgebra}
  The theory for bicommutative bialgebras presents the PROP~
  such that a morphism  is an -matrix with coefficients
  in  together with usual composition. The theory for qualitative
  bicommutative bialgebras presents the PROP~, see
  Definition~\ref{def:rel}.
\end{proposition}

\begin{example}
  Consider the rewriting system  corresponding to the theory for
  qualitative bicommutative bialgebras. The relation of Example~\ref{ex:rel-P}
  corresponds to the following morphism of :
  
  which can be represented graphically as
  .
\end{example}

In the above propositions, we have been considering presentations of PROPs and
not PROs as earlier. In order to present them we could have used a variant of
rewriting systems in order to freely generate symmetries. However, thanks to the
following proposition~\cite{burroni1993higher}, there is a way to explicitly
incorporate an explicit symmetry into the presentations without changing the
notion of presentation. This is the one we have been implicitly using.

\begin{proposition}
  Suppose that a PRO  is presented by a rewriting system~. Then
  the rewriting system  presents the free PROP on the PRO~
  where  is obtained by adding a new morphism generator 
  to , and  is obtained from  by adding the rule
   together with, for every morphism generator
  , two rules
  
  where the morphism  is defined by induction on
   by
  
  Graphically, a representation of the rules~\eqref{eq:gamma-nat-rules} is
  
\end{proposition}

\begin{example}
  The PROP corresponding to monoids is presented by the rewriting system
   with , and , and  consisting of rules~\eqref{eq:monoid-rules}
  together with
  
  A presentation for the PROP~ corresponding to commutative
  monoids is obtained by further adding the rule , see
  Proposition~\ref{prop:free-monoid}.
\end{example}

The aim of this article is to show that the category~, defined in
Section~\ref{sec:finpos}, admits the following theory as presentation.

\begin{definition}
  A \emph{poalgebra}  in a symmetric
  monoidal category  consists of a qualitative bicommutative bialgebra
   in  together with a morphism
   which is satisfying the following axiom, called
  \emph{transitivity}:
  
\end{definition}

\begin{remark}
  The additional axiom seems to break the horizontal symmetry of the structure,
  but this is not the case since the the dual axiom is implied:
  
\end{remark}

To sum up the axioms introduced in previous section, the theory of poalgebras is
the rewriting system  with  as object generators,
and the morphism generators in  are, together with their diagrammatic
notation,

and the rules in  are

    \mu\circ(\eta\otimes\id_1)&\To\id_1
    &
    \mu\circ(\id_1\otimes\eta)&\To\id_1
    &
    \mu\circ(\mu\otimes\id_1)&\To\mu\circ(\id_1\otimes\mu)
    &
    \mu\circ\gamma&\To\mu
    \\
    (\varepsilon\otimes\id_1)\circ\delta&\To\id_1
    &
    (\id_1\otimes\varepsilon)\circ\delta&\To\id_1
    &
    (\delta\otimes\id_1)\circ\delta&\To(\id_1\otimes\delta)\circ\delta
    &
    \gamma\circ\delta&\To\delta
  
  \delta\circ\mu&\To(\mu\otimes\mu)\circ(\id_1\otimes\gamma\otimes\id_1)\circ(\delta\otimes\delta)
  &
  \varepsilon\circ\eta&\To\id_0
  &
  \mu\circ\delta&\To\id_1
  &
  \mu\circ(\id_1\otimes\sigma)\circ\delta&\To\sigma
  
    \gamma\circ\gamma&\To\id_2
    &
    (\gamma\otimes\id_1)\circ(\id_1\otimes\gamma)\circ(\gamma\otimes\id_1)&\To(\id_1\otimes\gamma)\circ(\gamma\otimes\id_1)\circ(\id_1\otimes\gamma)
  
    \gamma\circ(\eta\otimes\id_1)&\To\id_1\otimes\eta
    &
    \eta\otimes\id_1&\To\gamma\circ(\id_1\otimes\eta)
    &
    \varepsilon\otimes\id_1&\To(\id_1\otimes\varepsilon)\circ\gamma
    &
    (\varepsilon\otimes\id_1)\circ\gamma&\To\id_1\otimes\varepsilon
  
    \gamma\circ(\mu\otimes\id_1)&\To (\id_1\otimes\mu)\circ(\gamma\otimes\id_1)\circ(\id_1\otimes\gamma)
    &
    (\mu\otimes\id_1)\circ(\id_1\otimes\gamma)\circ(\gamma\otimes\id_1)&\To\gamma\circ(\id_1\otimes\mu)
    \\
    (\delta\otimes\id_1)\circ\gamma&\To(\id_1\otimes\gamma)\circ(\gamma\otimes\id_1)\circ(\id_1\otimes\delta)
    &
    (\gamma\otimes\id_1)\circ(\id_1\otimes\gamma)\circ(\delta\otimes\id_1)&\To(\id_1\otimes\delta)\circ\gamma
  
    \gamma\circ(\sigma\otimes\id_1)&\To(\id_1\otimes\sigma)\circ\gamma
    &
    (\sigma\otimes\id_1)\circ\gamma&\To\gamma\circ(\id_1\otimes\sigma)
  
or graphically,
\renewcommand{\ss}[1]{\strid[scale=0.7]{#1}}
{\allowdisplaybreaks

}
This rewriting system is not confluent and whether the rules can be oriented and
completed in order to obtain a convergent rewriting system is still an open
question, although some progress have been made in this
direction~\cite{mimram:phd} (Section~5.5).
The main result of this article is the following theorem, and the rest of this
article is devoted to its proof.

\begin{theorem}
  The category~ is the PROP presented by the theory of poalgebras.
\end{theorem}

\section{Canonical factorizations}
\label{sec:cf}
In this section, we further study the category~, and define a canonical way
of writing a morphism in~ as a composite from a simple family of
morphisms. This will be used in Section~\ref{sec:pres} in order to show that
canonical forms of the presentation are in bijection with morphisms of . We
have omitted the proofs which can generally be done by easy (but not necessarily
short) induction.

The factorization of a morphism that we are going to introduce will not actually
depend on the morphism only, but on a linearization of it, in the following
sense: given a poset , a \emph{linearization}~ of  is a
morphism  in , for some , which is both epi
and mono.
While this definition is nice from a categorical point of view, it is sometimes
cumbersome to work with. It is easy to see that a morphism
 is a linearization of  if and only if the underlying
function  is a bijection (and thus  is the
cardinal of ). A linearization of  thus amounts to an enumeration
 of the events of  (writing
 as ), in way compatible with the partial order, and it turns
out to be simpler to axiomatize the inverse function. We will thus use the
following definition in the rest of the article:

\begin{definition}
  \label{def:linearization}
  A \emph{linearization} of a poset  is a bijective function
   such that, for every  with
  , we have .
\end{definition}

\noindent
The following lemma can be shown by induction on the cardinal of finite
posets. It is actually still true for non-finite posets if we assume the axiom
of choice~\cite{szpilrajn1930extension}, but we will not need this here.

\begin{lemma}
  Every finite poset admits a linearization.
\end{lemma}

\noindent
In order to characterize when two morphisms are linearizations of a given
poset~, we define a relation~ on linearizations of~, as
follows. Given two linearizations  of , we have
 when there exists  such that ,
 and  for  and , \ie  and 
only differ by swapping two consecutive independent events; otherwise said,
writing  for the transposition exchanging 
and  in the set , we have .

\begin{lemma}
  \label{lemma:lin-swap}
  Any two linearizations of a poset~ are equivalent by the equivalence
  relation generated by~.
\end{lemma}

\noindent
In the following, by a \emph{linearization of the internal events} of a morphism
 in , we mean a linearization of the poset
, or equivalently an
injective function  whose image is the set of internal
events of~ and which satisfies the condition of
Definition~\ref{def:linearization}.

We now introduce a notion of ``canonical factorization'' for morphisms of ,
which will depend on a linearization of the internal events of the morphism.
Given  and , we write  for the morphism
with one internal node~, so that
, with the canonical injections as
source~ and target~, and whose partial order
is such that



\begin{example}
  \label{ex:X302}
  For instance, the morphism  is
  .
\end{example}

\noindent
These morphisms can be used as basic ``building blocks'' for morphisms as follows.

\begin{lemma}
  \label{lemma:poset-cf-comp}
  Suppose given
  
  The composite
  
  is the morphism  with  internal events, that we denote by
   with , which is the smallest partial order such that
   when ,  when ,  when
  , and  when . Moreover, the function
   thus defined is a linearization of the internal
  events of~.
\end{lemma}

\noindent
The data \eqref{eq:fact} is called a \emph{factorization} of the
morphism~\eqref{eq:fact-comp}, the morphism~\eqref{eq:fact-comp} is called the
\emph{composite} of the factorization \eqref{eq:fact}, and the above
linearization of the internal events is said to be \emph{induced} by the
factorization. From the preceding lemma, it is easy to deduce that any morphism
of~ can be put into this form:

\begin{lemma}
  \label{lemma:poset-cf-fact}
  Suppose given a morphism  of~ with  internal events,
  together with a linearization  of the internal
  events of~. We have
  
  with
  
  and  is the relation defined by
  
\end{lemma}

\begin{example}
  \label{ex:poset-cf}
  The poset  of the left of
  
  admits the factorization
  
  as shown in the middle. Another possible factorization of the same morphism is
  depicted on the right.
\end{example}

\noindent
Because partial orders are transitive, by using Lemma~\ref{lemma:poset-cf-comp}
the factorizations~\eqref{eq:poset-fact} of morphisms of~ provided by
Lemma~\ref{lemma:poset-cf-fact} can easily be shown to be transitive, in the
following sense:

\begin{definition}
  \label{def:cf-trans}
  A factorization of a morphism of  of the form \eqref{eq:poset-fact} is
  \emph{transitive} when
  \begin{itemize}
  \item given  and  such that  and
    , we have ,
  \item given ,  and  such that  and
     we have .
  \end{itemize}
A transitive factorization of a morphism is called a \emph{canonical
    factorization}.
\end{definition}

\begin{lemma}
  The factorizations~\eqref{eq:poset-fact} of a morphism of~ provided by
  Lemma~\ref{lemma:poset-cf-fact} are transitive. Moreover, the factorization
  associated to the composite of a transitive factorization, with the induced
  linearization of the internal events, is itself.
\end{lemma}

\noindent
To sum up, we have just shown that the data~\eqref{eq:fact} of a
factorization~\eqref{eq:poset-fact} is a way to faithfully encode a morphism
of~ together with a linearization of its internal events:

\begin{proposition}
  \label{prop:poset-lin-cf}
  Pairs , constituted of a morphism~ of~ together with a
  linearization  of its internal events, are in bijection with factorizations
  of the form~\eqref{eq:poset-fact} which are transitive.
\end{proposition}

Because a morphism might admit multiple linearizations of its internal events, a
morphism generally admits multiple
factorizations~\eqref{eq:poset-fact}. However, we know by
Lemma~\ref{lemma:lin-swap} that any two of them are related by , and the
following lemma characterizes how replacing a linearization by another one
related by  affects the factorization.

\begin{lemma}
  \label{lemma:P-switch}
  Suppose that  is a linearization of  and  is
  such that  is another linearization of~. Suppose that
  
  are the factorizations associated by Proposition~\ref{prop:poset-lin-cf} to
   and  respectively. Then we have  for ,
   for , and
  .
\end{lemma}

\begin{example}
  The two factorizations of the morphism given in Example~\ref{ex:poset-cf} only
  differ by a swapping of~ and~. It can easily be checked that the
  relation in the second factorization can be deduced from the one of the first
  by precomposing by the transposition~.
\end{example}

\section{A presentation of the monoidal category of finite posets}
\label{sec:pres}
As announced earlier, we use the general methodology explained at the end
of Section~\ref{sec:presenting} in order to show this. We first define an
interpretation  by defining the interpretation of the
generators, as per Remark~\ref{rem:free-moncat}. The object generator
 is interpreted as the object  of , and the interpretations
of the morphism generators are the following morphisms of :
5ex]
  \intp{\eta}&\intp{\mu}&\intp{\varepsilon}&\intp{\delta}&\intp{\sigma}&\intp{\gamma}
\end{array}

I^n=\strid[height=1cm]{I}
\qquad
H^n=\strid[height=1cm]{H}
\qquad
S^n=\strid[height=1cm]{S}
\qquad
G^n=\strid[height=2cm]{G}
\qquad
W^n_i=\strid[height=2cm]{W}

W^n_I\qeq W^n_{i_{k-1}}\circ\ldots\circ W^n_{i_1}\circ W^n_{i_0}\qcolon n+1 \qto n+1

  X^3_{0,2}
  \qeq
  \strid[height=2cm]{X302}
  
    \label{eq:tp-nf}
    R\circ X^{m+k-1}_{I_{k-1}}\circ\ldots\circ X^{m+1}_{I_1}\circ X^m_{I_0}
  
    X^{n+1}_JX^n_I
    &\qeq S^{n+1}W^{n+1}_JH^{n+1}S^nW^n_IH^n&&\qeq \ol S^n\ol W^n_IS^{n+1}W^{n+1}_J\ol H^nH^n\\
    &\qeq S^{n+1}W^{n+1}_J\ol S^nH^{n+1}W^n_IH^n&&\qeq \ol S^n\ol W^n_IS^{n+1}\ol H^nW^n_JH^n\\
    &\qeq S^{n+1}W^{n+1}_J\ol S^n\ol W^n_IH^{n+1}H^n&&\qeq \ol S^n\ol W^n_IS^{n+1}\ol H^nW^n_JH^n\\
    &\qeq S^{n+1}W^{n+1}_J\ol S^n\ol W^n_I\ol H^nH^n&&\qeq \ol S^n\ol W^n_I\ol H^nS^nW^n_JH^n\\
    &\qeq S^{n+1}\ol S^nW^{n+1}_J\ol W^n_I\ol H^nH^n&&\qeq \ol S^n\ol W^n_IG^nH^{n+1}S^nW^n_JH^n\\
    &\qeq \ol S^nS^{n+1}W^{n+1}_J\ol W^n_I\ol H^nH^n&&\qeq \ol S^nG^nW^{n+1}_IH^{n+1}S^nW^n_JH^n\\
    &\qeq \ol S^nS^{n+1}\ol W^n_IW^{n+1}_J\ol H^nH^n&&\qeq G^nX^{n+1}_IX^n_J
  
Each of the equalities involved are direct to show by using the definition of
  the morphisms and the axioms symmetric monoidal categories.
For instance, one can show that  using the exchange
  law of monoidal categories: .  Other steps are similar.
\end{proof}

\begin{lemma}
  We write  (so that ). Given
  ,  and , we have
  .
\end{lemma}

\noindent
As a direct corollary of the two previous lemmas, we have

\begin{lemma}
  \label{lemma:TP-switch}
  Given , sets  indexed by , a relation
  , and  such that , we have  with  for ,  for , and
  .
\end{lemma}

\noindent
Finally, we are now able to show that the axioms of poalgebras are a
presentation of the category~.

\begin{theorem}
  \label{thm:p-qtbbe}
  The categories~ and~ are isomorphic as symmetric monoidal categories:
  the category  is presented by the theory of poalgebras.
\end{theorem}
\begin{proof}
  The canonical functor  is bijective on objects and was
  shown to be full in Lemma~\ref{lemma:tp-p-full}. In order to conclude, we have
  to show that it is faithful. Suppose given two morphisms  of
  , such that . By Lemma~\ref{lemma:TP-trans-fact}, we can
  suppose that they are written as a canonical factorization, and their images
  by  are thus canonical factorizations in~, in the sense of
  Definition~\ref{def:cf-trans}. They therefore correspond to two linearizations
  of the same poset. By Lemma~\ref{lemma:lin-swap}, we can transform one
  permutation into the other by a series of commutations of consecutive
  elements, each one corresponding to transforming the canonical form in~ in
  the way described in Lemma~\ref{lemma:P-switch}. We write  for this sequence of canonical forms. By
  Lemma~\ref{lemma:TP-switch}, we can mimic these transformations on ,
  obtaining a sequence of canonical factorizations 
  in~ such that for every  we have~. Therefore  and
  the functor~ is faithful.
\end{proof}

\section{Conclusion}
\label{sec:concl}
We have shown that the monoidal category  of partial orders, is the theory
of poalgebras. This monoidal category contains as subcategories various
well-known categories such as the simplicial category , the category
 of functions, the category  of relations, etc. and the
presentations of those categories (see Section~\ref{sec:poalgebras}) can thus be
recovered as particular cases of our result.

Many interesting variants of this result can be obtained along the same
lines. For instance, by adding the axioms 
and  to the theory of poalgebra, we obtain a presentation
of the subcategory of  whose morphisms are of the form  with 
and  surjective. Similarly, by removing the transitivity
axiom~\eqref{eq:transitivity} we can present a variant of the category where
morphisms are partial orders which are not supposed to be transitive, which can
be seen as directed acyclic graph, and thus recover the result of Fiore and
Devesas Campos~\cite{fiore2013algebra}. A variant of this theory, where the
morphism~ has been replaced by a family of generators with various
number of inputs and outputs, has also been considered in~\cite{beauxis2011non}
in order to provide an axiomatic definition of the notion of ``network''.

This work should be considered as a first step and can be generalized in various
promising ways. This presentation can be extended from posets to \emph{event
  structures}~\cite{winskel1995models}, which are partial orders equipped with a
notion of incompatibility, thus allowing us to hope for algebraic methods in
order to study concurrent processes, which those structures naturally
model. Another very interesting direction consists in exploring
higher-dimensional structures: for instance, we would like to extend this work
to construct a presentation of the monoidal bicategory of integers, partial
orders and increasing functions. In particular the full monoidal subbicategory
whose 1-cells are forests (\ie partial orders such that two incomparable
elements do not have an upper bound) is expected to be presented by (a variant
of) the theory of commutative strong monads, and to be closely related to
Moerdijk's dendroidal sets~\cite{moerdijk2010dendroidal}.

\bibliographystyle{eptcs}
\bibliography{papers}
\end{document}
