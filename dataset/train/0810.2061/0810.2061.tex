\documentclass{llncs}

\usepackage{mathpartir,amsmath,amssymb,bm,color}
\usepackage{kro}
\usepackage{enumerate}
\usepackage{fullpage}

\newif\iflong \longfalse

\title{On characterising strong bisimilarity\\
  in a fragment of CCS with replication\\
  -- note --}
\date{}
\author{Daniel Hirschkoff\inst{1} and Damien Pous\inst{2}}
\institute{ENS Lyon, Universit\'e de Lyon, CNRS, INRIA \and SARDES,
  LIG, Grenoble, CNRS, INRIA} 

\begin{document}

\maketitle

\begin{abstract}
  We provide a characterisation of strong bisimilarity in a fragment
  of CCS that contains only prefix, parallel composition,
  synchronisation and a limited form of replication. The
  characterisation is not an axiomatisation, but is instead presented
  as a rewriting system.

  We discuss how our method allows us to derive a new congruence
  result in the -calculus: congruence holds in the sub-calculus
  that does not include restriction nor sum, and features a limited
  form of replication. We have not formalised the latter result in all
  details.
\end{abstract}

\section{Introduction}

We study algebraic properties of strong bisimilarity in a sub-calculus
of CCS. Like in previous work~\cite{hirschkoff:pous:lmcs:08}, of which
the present study is a continuation, an important aspect of the
setting we analyse is the absence of the sum construct, and, more
generally, of any operator that would allow us to decompose parallel
composition.


We present a rewriting system that allows us to characterise 
strong bisimilarity () in a very basic calculus that only
features prefixes, parallel composition, and replicated prefixes, with
the additional constraint that these can occur only at top-level. The
restriction and choice (or sum) operators are not included.  Handling
replication is the novel aspect w.r.t.\
\cite{hirschkoff:pous:lmcs:08}, and raises several difficulties when
trying to analyse the algebraic properties of .

Let us focus on the properties of replication w.r.t.\ strong
bisimilarity. In our setting, the most important bisimilarity law for
replication is written
\begin{mathpar}
  \!a.P\,|\,a.P ~=~ \!a.P
  \enspace,
\end{mathpar}
\noindent and expresses that a replicated process acts as an
unbounded number of copies of that process in parallel.

It appears that we can generalise the above equality, by allowing a
replicated process to erase one of its copies (we are reading the
equality from left to right here) not only at top-level, but
arbitrarily deep in a term. In other words, if  is a context (a
process with a hole), the law
\begin{mathpar}
  \!a.P\,|\,C[a.P] ~=~ \!a.P\,|\,C[\nil]
\end{mathpar}
\noindent should hold for strong bisimilarity (the previous equality is
obtained by taking ).

This equality, together with the law , are the
basic ingredients we need in order to characterise strong bisimilarity
between replicated terms. However, these equations are not enough, as
the following example shows: process 
is bisimilar to . It seems reasonable to consider
 as the normal form of . Intuitively,  can be obtained
from  by inserting a copy of  ``inside'' , and,
symmetrically, a copy of  inside .  A related difficulty
appears with equalities like , where the copy is
inserted in the replicated component itself.

Describing this phenomenon of ``mutual replication'' in all its
generality would lead to complicated equational schemata, and we have
not been able to come up with a simple, readable, presentation of
strong bisimilarity based on equational laws. Instead, we introduce a
rewriting relation on processes that allows us to compute normal forms
w.r.t.\ strong bisimilarity (in particular, we are able to rewrite
 into ). This has the advantage of exposing the basic laws
that are at work when normalising a process. We show that our
characterisation of strong bisimilarity still holds when we enrich the
calculus with synchronisation. In turn, the method we describe can be
applied to derive a new congruence result on a subset of the
-calculus (we must say we have not checked all details of this
result yet).


\medskip\noindent\textsl{Outline.}  We describe the subset of CCS we
work with in Sect.~\ref{sec:set}; in Sect.~\ref{sec:tec}, we introduce
a notion of normal forms and prove useful some technical results. The
rewriting system is defined in Sect.~\ref{sec:rwt}, where we show that
it allows us to reach normal forms. Section~\ref{sec:comm} is devoted
to the extension of our results to a calculus with synchronisations,
closer to the standard CCS. In Sect.~\ref{sec:piccl}, we give
concluding remarks, discussing in particular how these results lead to
a new congruence property in the -calculus.


\section{The Setting}
\label{sec:set}


We work in the subset of CCS defined by the following grammar, where
we rely on a countable set of \emph{actions} :

Our calculus features no communication, no restriction, no sum, and
allows replication only on prefixes, at top-level. We use  to
range over processes. A \emph{finite} process  is a process which
does not contain an occurrence of the replication operator.
For , we shall sometime write  as
, and denote by  or
 the process
.
Note that  will always denote a process having replicated
components only.

We use  to range over single-hole \emph{contexts} mapping finite
processes to processes. Accordingly, we use  to range over
(single-hole) \emph{finite contexts}, mapping finite processes to
finite processes. Note that the hole cannot occur directly under a
replication in .


The labelled transition system associated to this process calculus is
standard (Fig.~\ref{fig:lts} -- note that there is no synchronisation
rule, this will be addressed in Sect.~\ref{sec:comm}), and
yields a notion of \emph{strong bisimilarity}, written , which
is a congruence.
\begin{figure}[t]
  \centering
  \begin{mathpar}
    \inferrule{ }{\alpha.F \xr \alpha F} \and
    \inferrule{F_1 \xr\alpha F'_1}{F_1\,|\,F_2 \xr \alpha F'_1\,|\,F_2} \and
    \inferrule{F_2 \xr\alpha F'_2}{F_1\,|\,F_2 \xr \alpha F_1\,|\,F'_2} \\
    \inferrule{ }{\!\alpha.F \xr \alpha \!\alpha.F\,|\, F} \and
    \inferrule{P_1 \xr\alpha P'_1}{P_1\,|\,P_2 \xr \alpha P'_1\,|\,P_2} \and
    \inferrule{P_2 \xr\alpha P'_2}{P_1\,|\,P_2 \xr \alpha P_1\,|\,P'_2} 
  \end{mathpar}
  \caption{Labelled Transition System for our Subset of CCS}
  \label{fig:lts}
\end{figure}



We shall rely on the following characterisation of strong bismilarity
for finite processes, which is established
in~\cite{hirschkoff:pous:lmcs:08}:

\begin{defi}[Distribution law]
  \label{def:distrlaw}
  Let \eqD{} be the smallest congruence generated by the laws of an
  abelian monoid for parallel composition (the neutral element being
  \nil), and the following equation schema, called \emph{distribution
    law}, where there are as many occurrences of  on both sides of
  the equation.
  \begin{mathpar}
    \alpha.(F|\alpha.F|\dots|\alpha.F) ~=~ \alpha.F|\alpha.F|\dots|\alpha.F
    \enspace,
  \end{mathpar}
\end{defi}
It is easy to show that this congruence is decidable, and we have
\begin{thm}
  \label{thm:distrlaw}
  \eqD{} coincides with strong bisimilarity  on finite processes.
\end{thm}


\section{Preliminary Technical Results}
\label{sec:tec}

We present some technical results about strong bisimilarity. Most of
these help us isolating the replicated part from the finite part in
processes being compared. Indeed, when characterising strong
bisimilarity, we shall prove that  implies that the
replicated parts of  and  are bisimilar, and we also need
somehow to reason about the finite parts of  and .

The following property is necessary to derive correction of the
rewrite system we define below.
\begin{prop}\label{prop:simplelaw}
  If , then .
\end{prop}
\begin{proof}
  We show that  is a
  strong bisimulation up to transitivity and parallel composition
  (cf.~\cite{SW01,phd:pous}).

  There are three cases to consider in the bisimulation game:
  \begin{itemize}
  \item the hole occurs at top-level in the context () and the
    right-hand side process does the following transition:
    . By hypothesis,  so
    that we find  such that  and . By injecting the latter equality, we obtain
     so that  closes the diagram.
  \item the hole occurs under a replicated prefix in the context
    () and this prefix is fired: we have
     and . This is where we need the up-to
    technique: these processes are not related by  (recall that we
    work with single-hole contexts). However, we can deduce , by considering the context
    , and checking that
    . We finally check that 
    and  are related by the closure of  under parallel
    contexts (by removing the  component).
  \item in the last case, either the hole occurs under a
    non-replicated prefix in the contexts (), or the
    contexts triggers a transition that does not involve or duplicate
    the hole; this case is treated by a simple reasoning -- just play
    the bisimulation game.  \qed
  \end{itemize}
\end{proof}

As a consequence, we obtain the validity of the following laws:
\begin{mathpar}
  \!\alpha.F~|~C[\alpha.F] ~\sim~ \!\alpha.F~|~C[0] \quad(A)\and
  \!\alpha.D[\alpha.D[\nil]] ~\sim~ \!\alpha.D[\nil] \quad(A')
\end{mathpar}

\begin{lem}\label{lem:cancel:finite:part}
  If , then .
\end{lem}
\begin{proof}
  Purely algebraically: replicate everything and add  in
  parallel, this yields ,
  from which we deduce .
(Note that, when writing , we actually refer to the process
  obtained by adding replication at top-level on the finite components
  of ; we easily show that this operation preserves bisimilarity.)
\qed
\end{proof}

\begin{lem}\label{lem:pouslemma}
  If  and , then there
  exists  s.t.\  and .
\end{lem}
\begin{proof}
  By firing  on the right-hand side, we find  such that
   and , from which we
  deduce . Then we show that the singleton
  relation  is a
  bisimulation up to bisimilarity and parallel contexts.
  \begin{itemize}
  \item when a transition on  is triggered, with ,
    we reason up to parallel composition in order to remove the 
    component on both sides;
  \item when a transition on  (or ) is triggered, we
    have to relate processes  and
     ; we reason up to
    bisimilarity in order to rewrite  into  and then
    up to parallel context in order to remove the  component. \qed
  \end{itemize}
\end{proof}

\medskip

Now we define our notion of normal forms (\emph{seeds}).
\begin{defi}[Size, seed] The size of , noted \size{P}, is the number of prefixes in .

  A \emph{seed} of , noted \seed{P} is a process of minimal size
  such that .
\end{defi}

The seed of a process is defined modulo bisimilarity. We establish in
this section that all seeds of a process are actually equated by
\eqD{} (Prop.~\ref{prop:compare:seeds}).  Note that, because  is
a congruence in our calculus, if  is a seed, then  is a
seed. Indeed, if  and , then
, which contradicts the fact that  is a
seed.

\medskip
\noindent \textbf{Notations.} 

We shall use  to range over seeds having only replicated
components.
We write  whenever there exist 
and  such that .
Note that  for some  if and only if
 for some finite context .
For , we write  to denote the
fact that , i.e., that
 does not contain a sub-term of the form . On the
contrary, we write  when there exists  such that
, that is, when  is a parallel composition of
sub-terms of the s. In the sequel, we shall use  to range over
finite processes satisfying the latter property.

We can remark that if  (resp.\ ) and , then  (resp.\ ).  

\begin{lem}\label{lem:disprops}
  (i)~If  is a seed, then ; 
  (ii)~if , then .
\end{lem}
\begin{proof}
  \begin{enumerate}[(i)]
  \item By contradiction, if , then
    . By law~,  which
    contradicts the minimality hypothesis about .
  \item Again, by contradiction, suppose that .
    Since, , there exist  such that
    , from which we deduce
     by 
    (we necessarily have ). This is contradictory with the
    fact that  is a seed. \qed
  \end{enumerate}
\end{proof}

\begin{lem}\label{lem:Rnil}
   and  entail .
\end{lem}
\begin{proof}
  Suppose by contradiction . By
  Lemma~\ref{lem:cancel:finite:part}, we have  and
   by replicating all processes. By
  Lemma~\ref{lem:pouslemma} there exists  such that . Now, since ,
  there exist some  such that ; if ,
  we have obtained a smaller seed; otherwise, we use  to show
  that 
  is a smaller seed.  \qed
\end{proof}

\begin{lem}\label{lem:cancel:finite}
   entails .
\end{lem}
\begin{proof}
  Write  for the seed of , . We have
  . By emptying  on the
  left\footnote{In the present case, `emptying ' means playing
    all prefixes in  in the bisimulation game between 
    and  -- we shall reuse this terminology in some proofs
    below.}, we obtain  for some .  Now, by emptying on the right, we get .  Injecting the latter equivalence in the one we have
  previously obtained gives
  \begin{mathpar}
    S_1\sim S_1|R_1|F"_2|R_2
    \enspace.
  \end{mathpar}
  If , we can apply
  Lemma~\ref{lem:cancel:finite:part} to deduce .
  But this gives a contradiction by Lemma~\ref{lem:Rnil}. Hence
  , which gives us, since we have established
  , that . Finally,
  .  \qed
\end{proof}

\begin{lem}\label{lem:cancel:FR}
  If , , and , then .
\end{lem}
\begin{proof}
  We proceed by induction on the size of . If , we have
   by Lemma~\ref{lem:Rnil}; otherwise, we first prove that 
  and  have the same size:
  \begin{itemize}
  \item if , by emptying  on the left-hand side,
    we find  such that ,  and ; this is contradictory with
    Lemma~\ref{lem:Rnil};
  \item if , by emptying  on the right-hand
    side, we find  with  such that
    , ,  and
    . Then we write  and deduce
     by Lemma~\ref{lem:cancel:finite:part}; then,
    by firing the  prefix, we find  such that
     and . We check that  so that we can apply the induction hypothesis and
    deduce that , whence ,
    and  by Thm.~\ref{thm:distrlaw}. This
    is contradictory with  ( is a sub-term of
    ). 
  \end{itemize}
  This concludes the proof that  and  have the same size. We
  then show that the relation  is a bisimulation:
  \begin{itemize}
  \item when , we find  such that  and ; by induction, , and we
    deduce that  is a derivative of , since otherwise, we would
    have  which is impossible.
  \item when , either we find  such that
     and , which allows us to close the
    diagram, by induction; or we find  such that . In this case, we empty  on the right-hand side,
    yielding  and  such that ; by
    Lemma~\ref{lem:cancel:finite:part}, , and 
    by induction, which is contradictory. \qed
  \end{itemize}
\end{proof}



\begin{lem}\label{lem:FF}
  If  and  (), then .
\end{lem}
\begin{proof}
  First observe that if , then we can empty
   by playing challenges on the left hand side, and we obtain
   with , which is impossible by
  Lemma~\ref{lem:cancel:FR}. Hence .
  
  We then show that  is a
  bisimulation. If , then ,
  which by hypothesis entails that  can answer this challenge.
  By the remark above,  necessarily answers by firing ,
  since otherwise we would get equivalent processes with finite parts
  having different sizes. This allows us to show that  can
  answer the challenge, and that \R{} is a bisimulation.
  \qed
\end{proof}

\begin{lem}\label{lem:RR}
   and  () entail .
\end{lem}
\begin{proof}
  By Lemma~\ref{lem:FF}, we have  ( by
  Lemma~\ref{lem:disprops}(ii)).  We conclude with
  Thm.~\ref{thm:distrlaw}:  and  are finite processes.  \qed
\end{proof}

\begin{lem}\label{lem:simseed}
  If , then .
\end{lem}
\begin{proof}
  Write  and , play each prefix on the left-hand side and apply
  Lemma~\ref{lem:RR} to show that there exists a map , such that
  . This map is
  bijective: otherwise we could construct a smaller seed.
\qed
\end{proof}

\begin{prop}[Uniqueness of seeds]\label{prop:compare:seeds}
  Suppose , where  and  are seeds. Then .
\end{prop}
\begin{proof}
  Write  and . As remarked above, 
  and  are necessarily seeds because  and  are (hence the
  notation). By Lemma~\ref{lem:cancel:finite}, , whence
   by Lemma~\ref{lem:simseed}.
Necessarily,  and , which allows us to deduce,
  using Lemma~\ref{lem:FF}, that . Finally, , by
  Thm.~\ref{thm:distrlaw}.
\qed
\end{proof}












\section{Rewriting Processes to Normal Forms}
\label{sec:rwt}



\begin{defi}[Rewriting, convertibility]\label{def:rewr:conv}
  Any process  induces a relation between processes, written
  \rewr{P}, defined by the following axioms, modulo :
\begin{mathpar}
  C[\alpha.F] \rewr{\!\alpha.F|F'} C[\nil]
  \quad\text{(B1)}
  \and
  \!\alpha.F |\!\alpha.F | P \rewr{Q} ~\!\alpha.F |  P
  \quad\text{(B2)}
\end{mathpar}
The reflexive transitive closure of \rewr{P} is written \wrewr{P};
we say that  and  are \emph{convertible}, written ,
whenever there exists a process  such that  and
.
\end{defi}

Example: we can check that process  is
normalised into  via the sequence

using axiom (B1). This is the way our rewriting relation proceeds to
compute normal forms. In this case, an equational reasoning would be
possible, as follows:  (we use the law  for the first step).

\begin{lem}\label{lem:rewr:TT}
  If , then .
\end{lem}
\begin{proof}
  By induction over the number of rewrite steps. If this number is
  zero, then this is obvious; suppose now . The
  induction hypothesis gives . We reason by cases over the
  axiom that is used to rewrite  into :
\begin{itemize}
  \item (B1): this means that , 
    and . From , we deduce
     by Prop.~\ref{prop:simplelaw},
    hence .
  \item (B2): we easily have , hence .
    \qed
  \end{itemize}
\end{proof}

\begin{lem}\label{lem:rewr:sn}
  Given , the relation \wrewr{P} terminates.
\end{lem}
\begin{proof}
  The size of processes strictly decreases along reductions.  \qed
\end{proof}

\begin{lem}\label{lem:eqD:rewr}
  For all , either , or  for some
   s.t.\ .
\end{lem}
\begin{proof}
  Write
  \begin{mathpar}
    P = (\prod_i \!\alpha_i.F_i)~|~ F^P\and \text{and}\and \seed{P} =
    (\prod_j \!\alpha_j.S_j)~|~ F^S \enspace,
\end{mathpar}
\noindent and set .
By definition, , which gives, by
Lemma~\ref{lem:cancel:finite}, 

A transition by the left hand side process is answered by the right
hand side process, yielding process
,
which 
gives, by injecting equivalence~(\ref{eq:bang:parts}), 
.

By Lemma~\ref{lem:FF}, this gives: either  , which
means by Theorem~\ref{thm:distrlaw} , or 
 (indeed,  is impossible, since this
would allow us to compute a seed having a smaller size than
\seed{P}). In the latter case, , this means that  can be
rewritten using axiom (B1), and the resulting process is bisimilar to
.

Suppose now that we are in case  for all possible transitions
from the s, that is, for all , there exists  such that
. We observe that the converse (associating a
 to all s) also holds, and that the number of parallel
components in  is necessarily greater than the
number of components in . In the case where this number is strictly
greater, this means that \rewr{\seed{P}} can be used to rewrite the
left hand side process in~(\ref{eq:bang:parts}), using axiom (B2). In
this case, the resulting process is bisimilar to .

We are left with the case where the two processes have the same number
of components, which entails that they are equated by .

To sum up, we have shown that either  can be
rewritten, or . In the latter case, we can
inject equivalence~(\ref{eq:bang:parts}) in
, which gives
. We can apply Lemma~\ref{lem:FF} again,
which gives two possibilities. The first possibility is that
, in which case , and finally
. The second possibility is that  (as above,
 is not possible since this would allow us to compute
a seed of smaller size). In that case, we can 
rewrite  using (B1), and getting a process bisimilar to .

Finally, either , or  can be rewritten using
\rewr{\seed{P}}.
\qed
\end{proof}

\begin{prop}\label{prop:seed}
  For all , .
\end{prop}
\begin{proof}
  Follows by Lemmas~\ref{lem:eqD:rewr} and~\ref{lem:rewr:sn}.
\qed
\end{proof}

\begin{thm}[Characterisation]\label{thm:charac}
   iff .
\end{thm}
\begin{proof}
  Suppose . By definition, this gives the existence of
   s.t.\  and . We deduce  by
  applying Lemma~\ref{lem:rewr:TT} twice and transitivity of .
  Hence .
  
  To establish the converse, suppose . Write, using
  Proposition~\ref{prop:seed},  and
  . By definition,  and
  , which entails .
This gives by Proposition~\ref{prop:compare:seeds}
  , which finally gives .
\qed
\end{proof}

This result gives a way to decide whether , via
\convert. Indeed, although Definition~\ref{def:rewr:conv} does not
tell how to find process , that allows one to derive ,
Thm.~\ref{thm:charac} allows us to reduce this problem to checking
whether . For this, it suffices to look for
\seed{P} among all processes of size smaller than \size{P}.


\section{Adding Synchronisations}
\label{sec:comm}

We can now move to a calculus closer to standard CCS, called
\miniccs{}, by instantiating actions with the following grammar, where
 range over a countable set of \emph{names}: actions are either
input or output prefixes.
\begin{mathpar}
  \alpha ~::=~ a\OR \out a
\end{mathpar}
The LTS we obtain with this definition is not that of CCS: we need to
add the following rules for synchronisations, where  is the
label for internal moves.
\begin{mathpar}
  \inferrule{P\xr a P' \and Q\xr{\out a} Q'}{P|Q \xr\tau P'|Q'}
\and
  \inferrule{P\xr{\out a} P' \and Q\xr{ a} Q'}{P|Q \xr\tau P'|Q'}
\end{mathpar}

In doing so, we change the notion of strong bisimilarity: the standard
CCS bisimilarity, that we shall denote using , tests internal
moves while our notion of bisimilarity () plays visible
challenges only. Therefore, we have .

The following result says that \convert{} is actually enough to
capture strong bisimilarity on \miniccs. As a consequence, we do not
need to test  transitions to obtain the discriminating power of
.

\begin{prop}\label{prop:equivccs}
  Let  and  be two processes. Then  if and only if
  .
\end{prop}
\begin{proof}
  By Thm.~\ref{thm:charac} and the above remark, it suffices to show
  that . This amounts to check that the
  distribution law and Prop.~\ref{prop:simplelaw} are valid for
  : we just need to check that silent challenges can be
  answered in the corresponding bisimulation candidates.  \qed
\end{proof}

Note that the  prefix is not included in this presentation of
\ccs; indeed, adding  to the syntax of actions  would
\emph{a priori} break the inclusion : tests
performed by  on -transitions would be too restrictive,
since the only way to answer would be to use a  prefix --
synchronisations would not be allowed. In the light of
Prop.~\ref{prop:equivccs}, we actually believe that  prefixes
could be added, that is, that they are played in one-to-one
correspondence in bisimilarity games.


\medskip

We conclude this section by proving that bisimilarity is closed under
substitutions in \miniccs.  We use  to range over
\emph{substitutions}, that are functions mapping names to names; we
write  for the process we obtain by applying  on all
names of .

\begin{prop}[ is closed under substitutions in \miniccs]
  \label{prop:subst:clos:ccs}
  If , then for any , .
\end{prop}
\begin{proof}
  We show the property for \convert. Suppose , which
  gives the existence of  such that, in particular, .
  By inspecting the shape of axioms (B1) and (B2), and reasoning by
  induction over the number of rewrite steps, we can deduce that
  . Similarly,
  . Hence
  .
\qed
\end{proof}


\section{Concluding Remarks}\label{sec:piccl}

\subsection{Extending our Characterisation}
\label{sec:ccl}

In absence of restriction in the calculus, it is easy to see that
applying replication to prefixed processes only is of no harm in terms
of expressiveness, because of the following rather standard structural
congruence laws for  (which are of course valid strong
bisimilarity laws):
\begin{mathpar}
  \!(P|Q) \equiv  \!P |\!Q
  \and
  \!!P \equiv \!P
  \and
  \!\nil \equiv \nil
\end{mathpar}




We have started investigating the question of characterising  in
the case where replication is not at top-level (but where nested
replications -- that is, replications that occur under replications --
are forbidden). The law
\begin{mathpar}
  \alpha.C[\!\alpha.C[\nil]] ~=~ \!\alpha.C[\nil]
\end{mathpar}
\noindent seems important to capture  in this setting. We do not
know at the moment whether it is sufficient to characterise .

Handling nested replications seems even more challenging.

\subsection{Congruence of Strong Bisimilarity in the -calculus}
\label{sec:pi}


Because of the input prefix, congruence of strong bisimilarity
requires closure of this relation under substitutions. 
In presence of sum, this property fails; as~\cite{SW01} shows, this is
also the case as soon as replication and restriction are present in
the calculus (in absence of sum).

\cite{hirschkoff:pous:lmcs:08} shows that congruence holds when we
renounce to replication, that is, in the sub-calculus that features
input and output prefixes, parallel composition and restriction.

Our investigations have convinced us that the same holds if instead we
renounce to restriction: we believe that the reasoning seen above can
be ported to the following subset of the -calculus:
\begin{mathpar}
  F ~::=~ \nil \OR F|F\OR a(x).F\OR \outm{a}{b}.F
  \and
  P ~::=~ F\OR \!a(x).F\OR P|P
\end{mathpar}
The analogue of Prop.~\ref{prop:subst:clos:ccs} gives us closure under
substitutions of strong bisimilarity, which in turn yields congruence.
Note that when bisimilarity is closed under substitutions, the ground,
early and late versions of the equivalence coincide. To adapt our
method from \miniccs{} to the -calculus, we work with ground
bisimilarity.

\iflong
\paragraph{Some technical results.}

\comment{ca, ca vient d'avant, il faut comprendre si c'est a
  incorporer ou a cacher pudiquement}



\begin{mathpar}
  a(x).P\xr{a(x)}P
  \and
  \outm{a}{b}.P\xr{\outm{a}{b}}P
  \and
  \!a(x).P\xr{a(x)}  \!a(x').(P[x'/x])~|~P
  \quad \boxed{?\!?}
\end{mathpar}

\begin{defi}[Visible ground bisimilarity]
  A symmetric relation \R{} is a visible ground bisimulation whenever
   implies
  \begin{itemize}
  \item if , then there exist  s.t.\ 
    , and there exists , fresh for  and , s.t.\ 
    ;
  \item if , then there exists  s.t.\
     and .
  \end{itemize}
  Visible ground bisimilarity, \visgr, is the greatest visible ground
  bisimulation.
\end{defi}

\begin{lem}\label{ground:subst}
   implies , for all injective
  substitution .
\end{lem}

\begin{lem}
  If  and , then there exists  such
  that , and, \emph{for all}  fresh for ,
  .
\end{lem}

Remark: if , then ; this holds
because we do not consider restriction.


\paragraph{Questions related to HOpi.}

\begin{itemize}
\item LICS08 studies HOpi without sum and without restriction. It says
  in the intro that it is remarkable to have a calculus where
  termination is undecidable but (strong) bisimilarity is decidable.
  Is the case for us? (showing undecidability of termination in LICS08
  means encoding Minsky machines: steak!)
\item A possibility would be to encode their HOpi into our pi: if the
  encoding preserves (non-)termination, we can rely on their encoding
  (we lack restriction to write the usual encoding, but maybe this is
  not a problem). But I'm afraid this does'nt work: encoding nested
  outputs (messages in messages) brings nested replications, and we
  are out of our calculus\dots
\end{itemize}


\paragraph{Other questions.}
\begin{itemize}
\item What about weak bisimilarity in our CCS calculus?
\end{itemize}

\fi



\bibliography{refs.bib}
\bibliographystyle{plain}

\end{document}
