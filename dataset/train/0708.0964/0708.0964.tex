We follow the usual definitions of graphs, including paths, simple paths,
cycles, simple cycles, and connectivity:
\cite{kant} is a useful source on the subject.
The accepted definition of graph does not allow self-loops nor multiple edges
nor infinite sets of vertices,
so it is a finite simple graph in Tutte's language \cite{tutte},
and a graph  can be specified as a pair  giving
its vertices and edges.  is a set of
unordered pairs of distinct vertices in .  Two vertices
 are {\em adjacent} or {\em neighbours} if
.

Given ,
when  is considered to be a  vertex,  means ,
and when  is considered to be an edge,  means .

\numpara
\label{subgraphs etcetera} {\bf Subgraphs, etcetera.}
Given  and ,  is a {\em subgraph}
of  if  and 

Given  and given , the {\em subgraph of 
spanned by } is the graph  where


The {\em degree} (in )  of a vertex  is the number of edges
incident to it, or the number of neighbours it has.
The word `node' is reserved in \cite{tutte} to denote
vertices whose degree  .

A {\em path} in  is a sequence  of vertices
where  and for , .
It is {\em simple} if all the vertices  are distinct.
The {\em inner vertices} in a simple path are
.

A {\em cycle} is a path  (that is, its
first and last vertices are the same).
It is a {\em simple cycle} if  or the path 
is a simple path.

If we write, say,  for a cycle, it is implied
that  is the second-last vertex rather than a recurrence
of the first, so properly the cycle is .

If  are
two graphs then we define 
If  and  then
 where
 We extend
this notation loosely but with little risk
of confusion: if  is a vertex then , and if  is a subgraph, or a path, or a cycle,
then  is the same as  where  is the set
of vertices in .


 is {\em connected} if every two vertices are connected
by a path in .
 is {\em biconnected} if it is
connected and for every ,  is connected.
 is {\em triconnected} if it is biconnected and
for any ,  is connected.
(Here  is a pair of vertices, not necessarily
an edge.)

A {\em path (graph)} is either a trivial graph or
a connected graph in which two vertices have degree  and all
others have degree .
A {\em simple cycle (graph)} is a connected nonempty graph all of
whose vertices have degree .

This paper is concerned with {\em nodal 3-connectivity}
(defined in \ref{nodal 3-connectivity}),
which requires biconnectivity but is weaker
than triconnectivity.

\begin{definition}
\label{planar graph defined} Let  be a graph.
\begin{itemize}
\item
The unit interval 
is denoted .
Given distinct points  and  in ,
a {\em simple curve-segment} joining  to  is
continuous, injective map 
such that  and 
\item
Let  be a map taking
each vertex  to a point  in the plane ,
and each edge  to a simple curve-segment
 joining  to .

The {\em relative interior} of , which depends on ,
is the open curve-segment


\item
The map  is a {\em plane embedding} of  if the points
 are distinct and the relative interiors of any two
edges are disjoint.

\item
A plane embedding  is {\em straight-edge} if  is
a line-segment for every edge .
\item
 is {\em planar} if a plane embedding exists.
\end{itemize}
\end{definition}

One often speaks of a planar graph  with a specific
plane embedding of  in mind, so it really means a
plane embedded graph.  A very significant difference is
that a plane embedded graph has a definite external face
(Definition \ref{faces of graph}),
whereas there is no notion of external face, nor
perhaps even of face, in a planar graph without a prescribed
embedding.
Figure \ref{nontutte.fig} shows a planar graph with two
quite different embeddings.

\begin{figure}
\centerline{
\includegraphics[height=1in]{figures/nontutte}}
\caption{a graph with different plane embeddings. Also, the
barycentric map is not an embedding.}
\label{nontutte.fig}
\end{figure}

Plane embeddings could somehow be pathological
and they should be discussed in terms of
the Jordan Curve Theorem mentioned below.
However, the following proposition could be used
to simplify the arguments.

\begin{proposition}
\label{straight-edge embeddings} Every planar graph
admits a straight-edge
embedding {\rm \cite{dpp,od94,read}}.\qed
\end{proposition}

\numpara
\label{topology in two dimensions} {\bf Topology in two dimensions.} See \cite{moise,stillwell}.
We assume the basic notions
of open and closed sets, connectedness, and path-connectedness.
If  and  then the
-neighbourhood of  is

If  is any subset of  then
its {\em closure}, written , is
 and
its {\em boundary}  is
 If 
is open then . We
are not concerned with connectedness, but with the
rather stronger notion of path-connectedness:
a set 
is {\em path-connected} if for any 
there exists a path from  to , a continuous map 
such that  and .

\numpara
\label{Jordan curves} {\bf Jordan curves.} A {\em Jordan curve} is a subset of
 homeomorphic to the unit circle .
That is,  is a Jordan
curve iff there exists a continuous injective map 
whose range is the set .


\begin{proposition}
\label{path components in graph and plane} Let  and  be two vertices in a plane embedding  of
a graph .  Then they are in the same component
of  as a graph if and only if they are in
the same path-component of  as a topological
subspace of . Also if  is a simple cycle then
its image under
 is a Jordan curve.  (Proof easy.)\qed
\end{proposition}



Part (i) of Proposition \ref{jordan curve theorem} below
states the Jordan Curve Theorem, which is a difficult result.
Proofs usually involve
algebraic topology \cite{greenberg}, but less advanced
methods can be used
\cite{moise,stillwell}.
Actually for our purposes we need only consider polygonal
Jordan curves, which makes the proofs much easier.
Part (ii) is elementary.

\begin{proposition}
\label{jordan curve theorem} {\rm (i)} (Jordan Curve Theorem {\rm \cite{greenberg,moise,stillwell}}).
If  is a Jordan curve
then  is the union of two
open, path-connected components, 
 and ,
, the {\em inside}, is bounded, and , the {\em outside}
or {\em exterior},
is unbounded, and 

{\rm (ii)} If  is any path-connected open set such that
, then  or .\qed
\end{proposition}


\numpara
\label{edges inside and outside Jordan curves} {\bf Edges inside and outside Jordan curves.}
If  is a Jordan curve and  an edge of a
graph, and  an embedding such that  doesn't meet 
except perhaps at  or , then
the relative interior of  (Definition
\ref{planar graph defined}) satisfies


In this case we say  is inside or outside  as appropriate.
In Section \ref{ambient isotopy section}
we shall need a certain refinement of the Jordan curve theorem:

\begin{proposition}
\label{schoenflies theorem} {\bf (Jordan-Sch\"onflies Theorem).}
Let  be the unit disc in  and 
the unit circle.  Then if  is a Jordan curve (a homeomorphic
image of ), the homeomorphism of 
extends to a homeomorphism between  and 

More generally, if  and  are two Jordan curves then
the homeomorphism between  and  extends to a homeomorphism
between  and itself taking  to 
and  to 
(See {\rm \cite{moise}}.)\qed
\end{proposition}


\begin{definition}
\label{faces of graph} Given a plane embedding  of a graph , by abuse of notation
let  also denote the union of points and curve-segments
constituting its image in the plane.  This is a closed and
bounded set of points in the plane.

A {\em face} of  is a path-connected component
of .

All faces except one are bounded.  The unbounded face
is called the {\em external face} or {\em outer face}.  Vertices on the
external face  are called {\em external}; the others are
{\em internal}.

The plane embedding is {\em triangulated} if every bounded
face is incident to exactly three edges, and {\em fully triangulated}
if every face, bounded and unbounded, is incident to three edges.
\end{definition}

Faces are open sets in .

\begin{definition}
\label{triangulation} Let  be a plane embedding of a graph
. A {\em triangulation} of the
graph is a triangulated plane embedding  of
a graph  where  and
, where  for all
 and  for all .
\end{definition}

\begin{proposition}
\label{all embeddings triangulable} Every plane embedded graph can be triangulated
{\rm \cite{kant}}.\qed
\end{proposition}

\begin{proposition}
\label{partial F is a subgraph} {\rm (i)} If  is a face of a plane embedded graph ,
then  is a subgraph of , and
{\rm (ii)} . (Proof omitted.)\qed
\end{proposition}


\numpara
\label{convexity} {\bf Convex sets in the plane.}
We note the basic definitions and results (see \cite{brondstred}).
A set  is {\em convex} if for any two points ,
the line-segment  is entirely contained in .  Suppose  is
a finite set of points in the plane.
The {\em convex hull} 
is the smallest convex set containing ,
that is, the intersection of all convex sets containing .
It is also the intersection of all closed half-planes containing
.  Either  is empty, or a point, or a
line-segment, or it is bounded by a convex polygon
whose corners are in .  In the latter case  is
the intersection of those closed half-planes containing
 whose boundaries contain sides of .

\begin{proposition}
\label{closure convex} If  is convex then its closure
 is convex. (Proof easy.)\qed
\end{proposition}



\begin{definition}
\label{convex combination map} {\bf (convex combination maps)} {\rm \cite{floater03}}.
A {\em convex embedding} of a planar graph  is a straight-edge
embedding
in which all bounded faces are convex, and the outer boundary
is a simple polygon.

\begin{figure}
\centerline{
\includegraphics[width=2in]{figures/delaun20}\hspace*{.5in}
\includegraphics[width=2in]{figures/bary20}\hspace*{.5in}\ \ 
}
\caption{Delaunay triangulation of 20 points and barycentric
embedding of the same graph with the same bounding
polygon.}
\label{delaun20.fig}
\end{figure}


Let  be a plane embedded graph whose external boundary is
a simple cycle .
Another map  from its vertices to
points in the plane is {\em a convex combination map} if
{\rm (a)} there exist coefficients  ( vertices) such that
\begin{itemize}
\item
,
and
.
\item
If  is an external vertex then 
\item
If  and  are adjacent and  is internal
then .
\item
Otherwise .
\end{itemize}

{\rm (b)}
the external
vertices are mapped (in cyclic order) to the corners
of a convex polygon, and {\rm (c)} for
every internal vertex , that is, for every vertex ,


The map is a {\em barycentric map} if
for each internal vertex  and neighbour  of
, .
If a barycentric map determines a straight-edge embedding
of  then it is called a {\em barycentric embedding}.
\end{definition}

For example, Figure \ref{delaun20.fig} shows a Delaunay triangulation
with 20 vertices, and a barycentric embedding of the same graph.

The definition of convex embedding does not exclude the
possibility that several edges on a face boundary be collinear.
Tutte's definition of convex embedding
\cite{tutte60} requires that the external boundary
be a convex polygon, which would rule out most triangulated
graphs.  Hence we require that it be a simple
polygon, though not necessarily convex.

In a barycentric map, every internal vertex is the average, centroid,
or barycentre, of its neighbours.  In a convex combination map
every internal vertex is a proper weighted average of its
neighbours.

The following simple lemma is very useful.

\begin{lemma}
\label{lem: one-sided}
Let  be a convex combination map,  a closed convex
set, and  an internal vertex such that
for all neighbours  of , .
If, for some neighbour  of ,  (the topological interior
of ), then .
\end{lemma}

{\bf Proof.} Fix a neighbour  such that ,
and fix  so for all points  in the
plane, if , then .

Since  is internal,
  and .
The sum can be written
as   where  is
a proper weighted average of the other neighbours of  --- or
 if .

Since  is convex,
 This is the open disc around  of
radius , so . {\bf Q.E.D.}\medskip

\begin{lemma}
\label{convex combination map inside P} If  is a convex combination map taking the external boundary
of a connected plane embedded graph  to a convex polygon ,
then all vertices and edges are mapped by  into .
\end{lemma}

{\bf Proof.}
Let .
Since  is convex, it is enough to show that for every
vertex ,   External vertices are mapped
to corners of  hence into 

Suppose there is an internal vertex  such that
   is the intersection of finitely many
closed half-planes, and one of them
does not contain   By changing coordinates if
necessary, it can be arranged that  is bounded above
by the -axis and there exist vertices 
such that  is above the -axis.
Choose  so  has maximal -coordinate, , say, and
let  be the close half-plane .

Since  is connected, there is a path
 where  is an external
vertex.  Since ,  is in the interior 
of ,
so without loss of generality,  and
by Lemma \ref{lem: one-sided}, , a contradiction.
{\bf Q.E.D.}\medskip

\begin{lemma}
\label{lem: conv comb emb conv faces}
If a convex combination map is an embedding, then its
embedded faces are convex.
\end{lemma}

{\bf Proof.} Let  be a bounded face.
Since  is a straight-edge embedding, 
is a simple polygon, and we need only show it has
no concave corners.  However, if  is a concave
corner then  is an inner vertex and
there is a convex wedge  such that 
for all neighbours  of . Let  be a closed
half-plane such that  and . By Lemma \ref{lem: one-sided}, , a contradiction.
{\bf Q.E.D.}\medskip


\numpara
\label{matrix defining a convex combination map} {\bf Matrix defining a convex combination map.} Given
a plane embedded graph  whose external boundary
is a simple cycle , convex combination maps are easily specified
using a matrix .  Suppose that  has
 vertices , the first  of them
belonging to , the last  being
internal vertices, and the coordinates of their
images are .  Any map from
vertices to points, including any straight-edge
embedding, is equivalent to a column vector of height .

Let  be the  matrix whose first
 rows are identical with those of the identity
matrix, and whose
last  rows express the barycentric mapping
equations (\ref{weighted average of neighbours}). Equivalently,
for , let
 Equation
\ref{weighted average of neighbours} can be written in
the form
 For
any convex combination map  (with
 given), let  be the column vector of
height 
whose first  entries give the -coordinates
of the corners of 
and whose other entries are zero; similarly
let  specify the -coordinates. Then  is
equivalent to column vectors  and  satisfying



\begin{lemma}
\label{unique convex combination map} {\rm (i)} If  is connected then the above matrix  is
invertible.

{\rm (ii)} If  is a connected plane embedded graph whose external
boundary is a simple cycle, and whose external
vertices are mapped in cyclic order to the corners of
a convex polygon, and weights  are given,
then this map extends to a unique
convex combination map of .
\end{lemma}

{\bf Sketch of proof.}
(See \cite{tutte,bsst,floater97,white}.)
Tutte's proof of (i) \cite{tutte,bsst} says that
the determinant of  (scaled up) is the number of spanning trees of
a certain connected graph related to .
There is a much more transparent proof given
in \cite{floater97} and also in \cite{white} saying
that if  has nonzero kernel then one can follow
a path from an external vertex to an internal vertex
where the internal vertex cannot satisfy Equation
\ref{weighted average of neighbours}.
Part (ii) follows trivially.\qed





\begin{definition}
\label{nodal 3-connectivity} A graph  is {\em nodally 3-connected} if it
is biconnected and for every two subgraphs
 and  of , if  and
 consists of just two vertices (and
no edges), then  or  is a simple path.
\end{definition}

\begin{proposition}
\label{nodally 3-connected and no deg 2} Every triconnected graph is nodally 3-connected,
and every nodally 3-connected graph with no vertices
of degree 2 is triconnected. (Proof omitted.)\qed
\end{proposition}

\begin{definition}
\label{peripheral polygon} A {\em peripheral polygon} in a connected graph 
is a simple cycle  such that  is
connected.
\end{definition}

The following result of Tutte's is fundamental.

\begin{proposition}
\label{tutte's theorem} {\em (Tutte \cite{tutte})}. If  is a nodally 3-connected
planar graph\footnote{with a few
exceptions: see Figure \ref{nontutte2.fig}. The result is phrased
differently in \cite{tutte}.}  and  is a peripheral polygon, and
the vertices of  are mapped (in cyclic order) onto the corners of a convex
polygon , then that map extends to a unique barycentric map
which is a convex, straight-edge embedding of .\qed
\end{proposition}

It is easy to give a counterexample when  is not
nodally 3-connected.  For example, in Figure \ref{nontutte.fig},
any barycentric map must map
the inner square face to a line-segment.  The figure
illustrates different plane embeddings of the same
graph, which is not nodally 3-connected.

We shall rely more heavily on the following

\begin{proposition}
\label{floaters theorem} {\em (Floater \cite{floater03}).}
If  is a triangulated (plane embedded) graph, then every convex
combination map of  is an embedding.\qed
\end{proposition}

Theorem \ref{planar nodally 3-connected} below shows that,
except regarding the external face, a {\em planar} graph
is nodally 3-connected if and only if barycentric maps
are plane embeddings.



Lemmas
\ref{plane disconnected face disconnected} and
\ref{planar biconnected} below are fairly obvious
and well-known, but still worth mentioning.

\begin{lemma}
\label{plane disconnected face disconnected} A plane embedded graph  is
connected if and only if for every face ,
the boundary  is (path-)connected.\qed
\end{lemma}




\begin{proposition}
\label{eulers formula} {\bf (Euler's Formula.)}
If  is a plane (straight-edge) embedded graph
then 
 where , and  are the
numbers of vertices, edges, faces, and components
of . (Proof omitted.)\qed
\end{proposition}

\begin{lemma}
\label{link of u} Let  be a straight-edge embedded plane graph in which all face boundaries
are simple cycles, and let  be any vertex of .

Let  be a list of neighbours of
 consecutive in anticlockwise order; possibly 
but otherwise they are distinct.  For 
let  be the face occurring between the edges (line-segments)
 and  in the anticlockwise sense.  (The
faces  are not necessarily distinct.)

Let  be the subgraph formed by the edges and vertices
in .

Then any two vertices in the list  are joined by a path in
. See Figure \ref{B.fig}.
\end{lemma}

\begin{figure}
\centerline{\includegraphics[height=1.5in]{figures/B}}
\caption{neighbours of  connected by paths avoiding .}
\label{B.fig}
\end{figure}

{\bf Proof.}
 is also the subgraph consisting
of all vertices and edges in .
Since each face is a simple cycle, 
is a path joining  to .  Thus
 contains paths joining all these vertices
. {\bf Q.E.D.}\medskip

\begin{lemma}
\label{planar biconnected} A plane straight-edge embedded graph  is biconnected if and only if
the graph consists of a single vertex or a single edge, or
the boundary of every face is a simple cycle.
\end{lemma}

{\bf Sketch proof.}
(i): If. A single vertex or edge is biconnected, so we
assume that the boundary of every face is a simple cycle.
 is connected (Lemma
\ref{plane disconnected face disconnected}).

For any vertex  and all neighbours  of  there
exist paths connecting these neighbours which avoid 
(Lemma \ref{link of u}).  Therefore all these neighbours are
in the same component of , and it follows that
 is connected.  Hence  is biconnected.

(ii): Only if. Suppose that  is connected, not a single vertex
or edge, and there exists a face  whose
boundary is not a simple cycle (graph):
 is connected but
contains a node  whose degree (in , not in )
differs from .  If  contained a vertex  of
degree  then (since  is nontrivial)  would be disconnected.
If it contained a vertex of degree , then  would
be disconnected or not biconnected.  Hence we can assume
that all vertices on  have degree  in .

Let  be a vertex of degree  in .  Let
 be the vertices adjacent to  in anticlockwise
order.  For ,  ( forms a clockwise
part of the boundary of a face incident to .  Since  has
degree  in , at least two of these
paths are incident to  and there are fewer than 
distinct faces incident to 

Let . All faces incident to  in 
merge into a single face of  and the other faces of 
are preserved.
The Euler formula gives
 for , since  is connected.  Correspondingly
for ,
 Now  and .
Since in  fewer than  faces are merged into a single face,
. Therefore
 so ,
 is disconnected, and  is not biconnected. {\bf Q.E.D.}\medskip

\numpara
\label{witnesses} {\bf Witnesses for a non-nodally 3-connected graph.}
Suppose  is not nodally 3-connected.  We
say that  are {\em witnesses} if ,
 contains just two vertices  and no edge,
neither  nor  are path graphs, and neither  nor 
equals .

\begin{lemma}
\label{edges incident from H and K} {\rm (i)}
Given witnesses , if  is a path in 
connecting  to ,
then  contains three consecutive vertices  where
, and , ,
, and , so  or .

{\rm (ii)} Any path (respectively, cycle) which avoids  and  except
perhaps at its endpoints
(respectively, perhaps once), is entirely in 
or in .
\end{lemma}

{\bf Proof.} (i) The first vertex in  is in ,
so the first edge is in .  Similarly the last edge is in .
Therefore there exist three consecutive vertices 
on the path where  and .
Then , so  or  and  is
incident to edges from  and from .

(ii)
Now let  be a path which avoids  and 
except perhaps at its endpoints. This includes the
possibility of a cycle, viewed as a path which
begins and ends at the same vertex : we allow
, but no other vertex on the cycle, to equal
 or .

If the path is not entirely
in  nor in , then it contains a triple
 where  or , a contradiction. {\bf Q.E.D.}\medskip



The proof of
Theorem \ref{planar nodally 3-connected} is long.  To lighten
it somewhat, we prove

\begin{lemma}
\label{2 or 3 faces} Let  be a plane embedded graph in which all face
boundaries are simple cycles.  Then
{\rm (i)} either  is a simple cycle with two faces, or\hfil\break
{\rm (ii)} for no two faces  is  a simple
cycle, and
if there are 3 faces  such that
 are all nonempty and connected, therefore simple paths, and they all
join the same two vertices  and ,
then there are exactly three faces, and
 consists of two nodes connected by three paths.
\end{lemma}

{\bf Proof.} Since all face boundaries are simple cycles,
 is biconnected, hence connected.

(i) Suppose
  that
is  is a Jordan curve .
By Theorem \ref{jordan curve theorem} (ii),
 is the inside of  and  the outside
or vice-versa, so  is a simple cycle with two faces.

(ii)  W.l.o.g.\   and  are bounded.  Their
intersection  is a simple path, which means that
 is simply connected,
and .

The only faces meeting the relative interior of  (respectively, ) are
 and  (respectively,  and ), so .
These are different paths joining  to  on ,
so . Again, ,
Thus .

 is either the inside or outside of 
(Theorem \ref{jordan curve theorem}), but 
are inside, so it is the outside, and  is the
unbounded face.  Thus there are three faces and
 is the union of three paths 
with two nodes in common.
{\bf Q.E.D.}\medskip

\begin{theorem}
\label{planar nodally 3-connected} A plane (straight-edge)
embedded graph is nodally 3-connected iff it is biconnected and
the intersection of any two face boundaries is connected.
\end{theorem}

{\bf Proof.} We can assume  is biconnected,
since that is required for nodal 3-connectivity.
Since  is biconnected either it is empty or
trivial, or a single edge, or
every face is bounded by a simple cycle.
In the first three cases the graph is obviously
nodally 3-connected and
biconnected with one face, so we need
only consider the fourth case and can assume that every face is bounded by
a simple cycle.

We can assume that  is straight-edge embedded. Therefore
the boundary of every face is a simple polygon.

{\bf Only if:} Suppose  and  are different faces and
 is disconnected. R.T.P.
 is not nodally 3-connected.

Let
 and  be vertices in different components of
.  For 
there are two paths  and  joining
 to  in .  These paths
are polygonal.

One can also construct a path  within , loosely speaking
by displacing  slightly into , and connecting its endpoints
to  and .  The resulting path is in
 except at its endpoints.  Similarly one can construct
a path  in  except at its endpoints.
These paths together form a (polygonal) Jordan curve  which 
meets  only at  and .  By construction, 
is inside  and  is outside .

Let  (respectively, ) be the subgraph consisting of all
vertices and edges of 
which lie inside or on  (respectively, outside or on ).
The only vertices in  are  and , and 
contains no edge.  contains
 and therefore is not a path graph, since otherwise
 and  and  would be in the same component of
.
Similarly  is not a path graph.  Therefore
 is not nodally 3-connected.\hfil\break

{\bf If:}  Suppose  is biconnected but not nodally 3-connected,
and  are witnesses.   has more than one face,
so all face boundaries are simple cycles.

{\em Claim 1.} The subgraphs
 and  are nonempty.
If every vertex in  were also in ,
then the vertices in  are in ,
that is,  and .  Either  has no
edges, in which case , or
it has the edge  and is a path graph.
Neither is possible.  Therefore 
and similarly  are nonempty.

{\em Claim 2.} Neither  nor  are isolated
vertices in  nor in .

Otherwise suppose  is isolated in .
Let  be any path joining  to .
By Lemma \ref{edges incident from H and K}, every
path connecting  to 
contains a vertex,  or , incident to
edges from  and from .  By hypothesis, 
is not; so every such path contains .  By
Claim 1, at least one such path exists, so
 is not connected, and 
is not biconnected.

{\em Claim 3.}
Both  and  have neighbours both in  and
in . Suppose all neighbours of  are in .
Since  is not isolated in , there is an edge
 in  incident to .  But  is a neighbour
of , therefore , so .  The only
edge in  incident to  is .

Consider a path in 
joining  to . Let
 be the first vertex where the path meets ,
and let  be the vertex before  on the path. Since
 and , :
 or .  However, if , then, since ,
 and .  Therefore .  This
implies that every path from  to 
contains . Again by Claim 1, such paths exist, so
 is not biconnected.

This contradiction shows
that not all neighbours of  are in ; neither are
they in , and the same goes for .

{\em Claim 4.} The vertices  and  share a face in common.
Otherwise let  be the neighbours
of .  We know (Lemma \ref{link of u}) that
they are all connected by paths in ,
where  is the union of boundaries of bounded
faces incident to .  Assuming  is incident to none
of these faces, these
paths would also avoid . This implies that all neighbours
of  are in  or in , contradicting Claim 3.


{\em Claim 5.} The vertices  and  have at least two
faces in common.  Let
 be the faces incident
to  in anticlockwise order around .  At least
one of these faces, w.l.o.g.\ , is incident to  and to .
Suppose no other face is.

There are two cases.
If  or , w.l.o.g.\ , is an internal vertex, then
all faces incident to  are bounded, and by Lemma
\ref{link of u}, the subgraph
 would
be connected and contain neither  nor .  Then
all vertices in this subgraph would belong to 
or to .  Since it includes all neighbours of 
in , it would contradict Claim 3.

If both  and  are external vertices, then 
is the external face,
and all bounded faces incident to  avoid .
This time we consider the subgraph
.
Again this is
a connected subgraph containing all neighbours of 
in , and again it omits both  and , so
again all vertices in it are in  or in , and
again Claim 3 is contradicted.

Therefore  and  have at least two faces  and  in common.

{\em Claim 6.}
If  and  are incident to three faces
, , and , then the boundaries of at least two
of these faces have disconnected intersection.
Otherwise,
by Lemma \ref{2 or 3 faces},  consists of two
nodes  connected by three paths. If 
where  then  or  is
a path graph:  is nodally 3-connected.

This contradiction shows that the one of the pairs
 is disconnected, as
claimed.

{\em Claim 7.}
If there are exactly two faces  and  incident
to  and to ,
then  is disconnected.

Otherwise 
is a path  joining a vertex  to another
vertex  and containing a
subpath  joining  to .  Not all of
 need be distinct, but it is assumed
that they occur in that order in .

By
Lemma \ref{edges incident from H and K},
all vertices in 
belong to  or to : w.l.o.g.\ 
to . The boundary cycles   and
 include two other paths,  and ,
respectively, joining 
to . Let , a Jordan curve.

If  then  meets  at  alone,
or not at all, and by Lemma
\ref{edges incident from H and K}, all vertices on
, plus those in , belong to  or to .

If all vertices on  belong to , then all
vertices outside  also belong to , because for
any vertex  outside , one can choose a shortest
path joining  to a vertex in .
Neither  nor  occur as internal vertices on this
path, so all vertices on the path are in  or  (Lemma
\ref{edges incident from H and K}),
i.e., , since the last vertex is in .

We have counted all vertices in : those outside ,
those on , and those on , and all are in ,
so , which is false.

On the other hand,
if all vertices on , and in , belong to ,
then all vertices outside  belong to ,
and  is a path graph, which is false.
This proves Claim 7 in the case ,
and by symmetry in the case .

If  and  then : let  and
 be the other subpaths joining  to 
in  and  respectively.
By Lemma \ref{edges incident from H and K},
each subpath  is contained in  or
in .  Again we have a Jordan curve
.

If  and  are not both external vertices,
w.l.o.g.\
 is an internal vertex, then
 and  are bounded faces incident to ,
and since ,
they are consecutive in cyclic order.
Let  (respectively, ) be the second
vertex (following ) in  (respectively, ).
The only faces incident to  and to  are 
 and  so  and 
differ from  and
 and  are connected
by a path which avoids  and 
(Lemma \ref{link of u}).
Therefore, by Lemma 
\ref{edges incident from H and K},  and  are both
in  or in , and so are all vertices on .
The same goes
for all vertices outside , so either  or
 is a path graph, a contradiction.


This leaves the case where  and  are external
vertices with exactly
two faces in common,  and , whose
boundaries have connected intersection.
Since  and  are external vertices,
one of these faces,
 say, is the external face.
Since  is not nodally 3-connected, it is not
a simple cycle, and  is a simple path
joining  to  (Lemma \ref{2 or 3 faces}).
Let  and  be the other paths
joining  to  on  (respectively,
).  is the
external cycle, a Jordan curve, and  separates
its interior into two regions of which  is one.
Let . It is a Jordan curve surrounding
the other region.

Let  be the second vertices
on . Again
there is a path joining  to  which
avoids  and , and
all vertices on  are in  or , and
the same holds for all vertices inside .
If they are all in  then , and if they
are all in  then , a simple path.
This contradiction finishes the proof of Claim 7.

Claims 6 and 7 taken together amount to the desired result.
{\bf Q.E.D.}\medskip

\numpara
\label{chord-free triangulated graphs} {\bf Chord-free triangulated graphs.}
A triangulated plane embedded graph is one in which every
bounded face is bounded by three edges.  In a triangulated
biconnected graph the external boundary is also a simple cycle.
It can only fail to be nodally 3-connected if a bounded
face meets the external boundary in a disconnected set.
Equivalently, one of its edges is a chord joining
two vertices on the external boundary,
and the other two edges are not both on the external boundary
\cite{white}.

\begin{figure}
\centerline{\includegraphics[height=1in]{figures/quadril}}
\caption{a nodally 3-connected but not triconnected triangulated
planar graph}
\label{quadril.fig}
\end{figure}

The graph in Figure \ref{quadril.fig} is nodally 3-connected but
not triconnected.


A fully triangulated planar graph is a triangulated planar
graph in which there are three external edges. In other words,
the external face also is bounded by a 3-cycle.  Therefore the
external cycle has no chords, so every fully triangulated
planar graph is nodally 3-connected.

Also let  be a fully triangulated planar graph containing
a vertex  of degree 2. Let  and 
be the neighbours of . There are only two faces
incident to  and they are both incident to
 and .  One of them must be
the external face. Thus  and  are the
three external vertices.  They also bound
the only bounded face.   is a 3-cycle,
and therefore triconnected.

On the other hand,
if  is fully triangulated
then it is nodally 3-connected,
so if it contains no vertex of degree 
then it is triconnected
(Proposition
\ref{nodally 3-connected and no deg 2}).  Therefore

\begin{corollary}
Every fully triangulated planar graph is triconnected.\qed
\end{corollary}

