We follow the usual definitions of graphs, including paths, simple paths,
cycles, simple cycles, and connectivity:
\cite{kant} is a useful source on the subject.
The accepted definition of graph does not allow self-loops nor multiple edges
nor infinite sets of vertices,
so it is a finite simple graph in Tutte's language \cite{tutte},
and a graph $G$ can be specified as a pair $(V,E)$ giving
its vertices and edges. $E$ is a set of
unordered pairs of distinct vertices in $V$.  Two vertices
$u,v$ are {\em adjacent} or {\em neighbours} if
$\{u,v\} \in E$.

Given $G = (V,E)$,
when $u$ is considered to be a  vertex, $u\in G$ means $u\in V$,
and when $e$ is considered to be an edge, $e\in G$ means $e\in E$.

\numpara
\label{subgraphs etcetera} {\bf Subgraphs, etcetera.}
Given $G=(V,E)$ and $G' = (V',E')$, $G'$ is a {\em subgraph}
of $G$ if $V'\subseteq V$ and $E'\subseteq E.$

Given $G$ and given $S \subseteq V$, the {\em subgraph of $G$
spanned by $S$} is the graph $(S,E')$ where
$$ E' = \{ \{u,v\}\in E:~~ u,v \in S\}.$$

The {\em degree} (in $G$) $\deg(v)$ of a vertex $v$ is the number of edges
incident to it, or the number of neighbours it has.
The word `node' is reserved in \cite{tutte} to denote
vertices whose degree $\not=$ $2$.

A {\em path} in $G$ is a sequence $u_0, \ldots, u_k$ of vertices
where $k\geq 0$ and for $0 \leq j \leq k-1$, $\{u_j, u_{j+1} \} \in E$.
It is {\em simple} if all the vertices $u_j$ are distinct.
The {\em inner vertices} in a simple path are
$\{u_1,\ldots,u_{k-1}\}$.

A {\em cycle} is a path $u_0,\ldots, u_k,u_0$ (that is, its
first and last vertices are the same).
It is a {\em simple cycle} if $k=0$ or the path $u_0,\ldots,u_k$
is a simple path.

If we write, say, $v_1,\ldots, v_n$ for a cycle, it is implied
that $v_n$ is the second-last vertex rather than a recurrence
of the first, so properly the cycle is $v_1,\ldots, v_n,v_1$.

If $G_i = (V_i,E_i)$ are
two graphs then we define 
$$G_1 \cap G_2 = (V_1\cap V_2, E_1\cap E_2)\quad\text{and}\quad
G_1 \cup G_2 = (V_1\cup V_2, E_1\cup E_2).$$If $G = (V,E)$ and $S \subseteq V$ then
$ G\backslash S = (V',E')$ where
$$ V' = V \backslash S\quad\text{and}\quad
E' = \{ \{u,v\} \in E:~ u\notin S ~\text{and}~ v\notin S\}.$$ We extend
this notation loosely but with little risk
of confusion: if $x$ is a vertex then $G\backslash x = 
G\backslash \{x\}$, and if $H$ is a subgraph, or a path, or a cycle,
then $G\backslash H$ is the same as $G\backslash S$ where $S$ is the set
of vertices in $H$.


$G$ is {\em connected} if every two vertices are connected
by a path in $G$.
$G$ is {\em biconnected} if it is
connected and for every $u\in G$, $G\backslash u$ is connected.
$G$ is {\em triconnected} if it is biconnected and
for any $u,v\in G$, $G\backslash \{u,v\}$ is connected.
(Here $\{u,v\}$ is a pair of vertices, not necessarily
an edge.)

A {\em path (graph)} is either a trivial graph or
a connected graph in which two vertices have degree $1$ and all
others have degree $2$.
A {\em simple cycle (graph)} is a connected nonempty graph all of
whose vertices have degree $2$.

This paper is concerned with {\em nodal 3-connectivity}
(defined in \ref{nodal 3-connectivity}),
which requires biconnectivity but is weaker
than triconnectivity.

\begin{definition}
\label{planar graph defined} Let $G = (V,E)$ be a graph.
\begin{itemize}
\item
The unit interval $\{t \in \IR:~ 0 \leq t \leq 1\}$
is denoted $[0,1]$.
Given distinct points $x$ and $y$ in $\IR^2$,
a {\em simple curve-segment} joining $x$ to $y$ is
continuous, injective map $\pi: [0,1] \to \IR^2$
such that $\pi(0) = x$ and $\pi(1) = y.$
\item
Let $f$ be a map taking
each vertex $u$ to a point $f(u)$ in the plane $\IR^2$,
and each edge $e=\{u,v\}$ to a simple curve-segment
$f(e)$ joining $f(u)$ to $f(v)$.

The {\em relative interior} of $e$, which depends on $f$,
is the open curve-segment
$$ \int(e) = f(e) \backslash \{f(u)\} \backslash \{f(v)\}.$$

\item
The map $f$ is a {\em plane embedding} of $G$ if the points
$f(u)$ are distinct and the relative interiors of any two
edges are disjoint.

\item
A plane embedding $f$ is {\em straight-edge} if $f(e)$ is
a line-segment for every edge $e$.
\item
$G$ is {\em planar} if a plane embedding exists.
\end{itemize}
\end{definition}

One often speaks of a planar graph $G$ with a specific
plane embedding of $G$ in mind, so it really means a
plane embedded graph.  A very significant difference is
that a plane embedded graph has a definite external face
(Definition \ref{faces of graph}),
whereas there is no notion of external face, nor
perhaps even of face, in a planar graph without a prescribed
embedding.
Figure \ref{nontutte.fig} shows a planar graph with two
quite different embeddings.

\begin{figure}
\centerline{
\includegraphics[height=1in]{figures/nontutte}}
\caption{a graph with different plane embeddings. Also, the
barycentric map is not an embedding.}
\label{nontutte.fig}
\end{figure}

Plane embeddings could somehow be pathological
and they should be discussed in terms of
the Jordan Curve Theorem mentioned below.
However, the following proposition could be used
to simplify the arguments.

\begin{proposition}
\label{straight-edge embeddings} Every planar graph
admits a straight-edge
embedding {\rm \cite{dpp,od94,read}}.\qed
\end{proposition}

\numpara
\label{topology in two dimensions} {\bf Topology in two dimensions.} See \cite{moise,stillwell}.
We assume the basic notions
of open and closed sets, connectedness, and path-connectedness.
If $x\in \IR^2$ and $\varepsilon > 0$ then the
$\varepsilon$-neighbourhood of $x$ is
$$ B(x,\varepsilon) = \{ y \in \IR^2:~~ |y-x| < \varepsilon\}.$$
If $S$ is any subset of $\IR^2$ then
its {\em closure}, written $\overline{S}$, is
$$ \overline{S} =
\{ x\in\IR^2:~~ (\forall \varepsilon>0)
B(x,\varepsilon)\cap S \not=\emptyset\},$$ and
its {\em boundary} $\partial S$ is
$$\partial S = \overline{S}\cap\overline{\IR^2\backslash S}.$$ If $S$
is open then $S\cap \partial S = \emptyset$. We
are not concerned with connectedness, but with the
rather stronger notion of path-connectedness:
a set $S$
is {\em path-connected} if for any $x,y\in S$
there exists a path from $x$ to $y$, a continuous map $\pi:[0,1]\to S$
such that $\pi(0) = x$ and $\pi(1) = y$.

\numpara
\label{Jordan curves} {\bf Jordan curves.} A {\em Jordan curve} is a subset of
$\IR^2$ homeomorphic to the unit circle $S^1$.
That is, $J$ is a Jordan
curve iff there exists a continuous injective map $h: S^1 \to \IR^2$
whose range is the set $J$.


\begin{proposition}
\label{path components in graph and plane} Let $x$ and $y$ be two vertices in a plane embedding $f$ of
a graph $G$.  Then they are in the same component
of $G$ as a graph if and only if they are in
the same path-component of $G$ as a topological
subspace of $\IR^2$. Also if $C$ is a simple cycle then
its image under
$f$ is a Jordan curve.  (Proof easy.)\qed
\end{proposition}



Part (i) of Proposition \ref{jordan curve theorem} below
states the Jordan Curve Theorem, which is a difficult result.
Proofs usually involve
algebraic topology \cite{greenberg}, but less advanced
methods can be used
\cite{moise,stillwell}.
Actually for our purposes we need only consider polygonal
Jordan curves, which makes the proofs much easier.
Part (ii) is elementary.

\begin{proposition}
\label{jordan curve theorem} {\rm (i)} (Jordan Curve Theorem {\rm \cite{greenberg,moise,stillwell}}).
If $J$ is a Jordan curve
then $\IR^2 \backslash J$ is the union of two
open, path-connected components, 
$\int(J)$ and $\ext(J)$,
$\int(J)$, the {\em inside}, is bounded, and $\ext(J)$, the {\em outside}
or {\em exterior},
is unbounded, and $\partial (\int(J)) = \partial (\ext(J)) = J.$

{\rm (ii)} If $S$ is any path-connected open set such that
$\partial S = J$, then $S=\int(J)$ or $S=\ext(J)$.\qed
\end{proposition}


\numpara
\label{edges inside and outside Jordan curves} {\bf Edges inside and outside Jordan curves.}
If $J$ is a Jordan curve and $e=\{u,v\}$ an edge of a
graph, and $f$ an embedding such that $f(e)$ doesn't meet $J$
except perhaps at $f(u)$ or $f(v)$, then
the relative interior of $e$ (Definition
\ref{planar graph defined}) satisfies
$$ \int(e) \subseteq \int(C)
\quad\text{or}\quad
\int(e) \subseteq \ext(C).$$

In this case we say $e$ is inside or outside $J$ as appropriate.
In Section \ref{ambient isotopy section}
we shall need a certain refinement of the Jordan curve theorem:

\begin{proposition}
\label{schoenflies theorem} {\bf (Jordan-Sch\"onflies Theorem).}
Let $D^1$ be the unit disc in $\IR^2$ and $S^1 = \partial D^1,$
the unit circle.  Then if $J$ is a Jordan curve (a homeomorphic
image of $\partial D^1$), the homeomorphism of $\partial D^1$
extends to a homeomorphism between $D^1$ and $\overline{\int(J)}.$

More generally, if $J$ and $J'$ are two Jordan curves then
the homeomorphism between $J$ and $J'$ extends to a homeomorphism
between $\IR^2$ and itself taking $\int(J)$ to $\int(J')$
and $\ext(J)$ to $\ext(J').$
(See {\rm \cite{moise}}.)\qed
\end{proposition}


\begin{definition}
\label{faces of graph} Given a plane embedding $f$ of a graph $G$, by abuse of notation
let $G$ also denote the union of points and curve-segments
constituting its image in the plane.  This is a closed and
bounded set of points in the plane.

A {\em face} of $G$ is a path-connected component
of $\IR^2\backslash G$.

All faces except one are bounded.  The unbounded face
is called the {\em external face} or {\em outer face}.  Vertices on the
external face  are called {\em external}; the others are
{\em internal}.

The plane embedding is {\em triangulated} if every bounded
face is incident to exactly three edges, and {\em fully triangulated}
if every face, bounded and unbounded, is incident to three edges.
\end{definition}

Faces are open sets in $\IR^2$.

\begin{definition}
\label{triangulation} Let $f$ be a plane embedding of a graph
$G = (V,E)$. A {\em triangulation} of the
graph is a triangulated plane embedding $f'$ of
a graph $G' = (V',E')$ where $V' = V$ and
$E' \supseteq E$, where $f'(u) = f(u)$ for all
$u\in V$ and $f'(e) = f(e)$ for all $e \in E$.
\end{definition}

\begin{proposition}
\label{all embeddings triangulable} Every plane embedded graph can be triangulated
{\rm \cite{kant}}.\qed
\end{proposition}

\begin{proposition}
\label{partial F is a subgraph} {\rm (i)} If $F$ is a face of a plane embedded graph $G$,
then $\partial F$ is a subgraph of $G$, and
{\rm (ii)} $G=\bigcup_F \partial F$. (Proof omitted.)\qed
\end{proposition}


\numpara
\label{convexity} {\bf Convex sets in the plane.}
We note the basic definitions and results (see \cite{brondstred}).
A set $A$ is {\em convex} if for any two points $a,b \in A$,
the line-segment $ab$ is entirely contained in $A$.  Suppose $S$ is
a finite set of points in the plane.
The {\em convex hull} $\hull(S)$
is the smallest convex set containing $S$,
that is, the intersection of all convex sets containing $S$.
It is also the intersection of all closed half-planes containing
$S$.  Either $\hull(S)$ is empty, or a point, or a
line-segment, or it is bounded by a convex polygon
whose corners are in $S$.  In the latter case $\hull(S)$ is
the intersection of those closed half-planes containing
$S$ whose boundaries contain sides of $S$.

\begin{proposition}
\label{closure convex} If $A$ is convex then its closure
$\overline{A}$ is convex. (Proof easy.)\qed
\end{proposition}



\begin{definition}
\label{convex combination map} {\bf (convex combination maps)} {\rm \cite{floater03}}.
A {\em convex embedding} of a planar graph $G$ is a straight-edge
embedding
in which all bounded faces are convex, and the outer boundary
is a simple polygon.

\begin{figure}
\centerline{
\includegraphics[width=2in]{figures/delaun20}\hspace*{.5in}
\includegraphics[width=2in]{figures/bary20}\hspace*{.5in}\ \ 
}
\caption{Delaunay triangulation of 20 points and barycentric
embedding of the same graph with the same bounding
polygon.}
\label{delaun20.fig}
\end{figure}


Let $G$ be a plane embedded graph whose external boundary is
a simple cycle $C$.
Another map $f$ from its vertices to
points in the plane is {\em a convex combination map} if
{\rm (a)} there exist coefficients $\lambda_{uv}$ ($u,v$ vertices) such that
\begin{itemize}
\item
$\lambda_{uv} \geq 0$,
and
$\sum_v \lambda_{uv} = 1$.
\item
If $v$ is an external vertex then $\lambda_{vv} = 1.$
\item
If $u$ and $v$ are adjacent and $u$ is internal
then $\lambda_{uv} > 0$.
\item
Otherwise $\lambda_{uv} = 0$.
\end{itemize}

{\rm (b)}
the external
vertices are mapped (in cyclic order) to the corners
of a convex polygon, and {\rm (c)} for
every internal vertex $u$, that is, for every vertex $u\notin C$,
\begin{equation}
\label{weighted average of neighbours} f(u) =  \sum_v \lambda_{uv} f(v)
\end{equation}

The map is a {\em barycentric map} if
for each internal vertex $u$ and neighbour $v$ of
$u$, $\lambda_{uv} = 1/\deg(u)$.
If a barycentric map determines a straight-edge embedding
of $G$ then it is called a {\em barycentric embedding}.
\end{definition}

For example, Figure \ref{delaun20.fig} shows a Delaunay triangulation
with 20 vertices, and a barycentric embedding of the same graph.

The definition of convex embedding does not exclude the
possibility that several edges on a face boundary be collinear.
Tutte's definition of convex embedding
\cite{tutte60} requires that the external boundary
be a convex polygon, which would rule out most triangulated
graphs.  Hence we require that it be a simple
polygon, though not necessarily convex.

In a barycentric map, every internal vertex is the average, centroid,
or barycentre, of its neighbours.  In a convex combination map
every internal vertex is a proper weighted average of its
neighbours.

The following simple lemma is very useful.

\begin{lemma}
\label{lem: one-sided}
Let $f$ be a convex combination map, $H$ a closed convex
set, and $v$ an internal vertex such that
for all neighbours $u$ of $v$, $f(u)\in H$.
If, for some neighbour $u$ of $v$, $f(u)\in H^o$ (the topological interior
of $H$), then $v \in H^o$.
\end{lemma}

{\bf Proof.} Fix a neighbour $u$ such that $f(u)\in H^o$,
and fix $\varepsilon > 0$ so for all points $x$ in the
plane, if $|x|<\varepsilon$, then $x+f(u)\in H$.

Since $v$ is internal,
$$ f(v) = \sum_w \lambda_{vw} f(w),$$  and $f(v)\in H$.
The sum can be written
as $\lambda_{vu} f(u) + (1-\lambda_{vu})y$  where $y$ is
a proper weighted average of the other neighbours of $v$ --- or
$O$ if $\lambda_{vu}=1$.

Since $H$ is convex,
$$ \{ \lambda_{vu}(x+f(u)) + (1-\lambda_{vu})y:~ |x| < \varepsilon \}
\subseteq H.$$ This is the open disc around $f(v)$ of
radius $\lambda_{vu}\varepsilon$, so $f(v)\in H^o$. {\bf Q.E.D.}\medskip

\begin{lemma}
\label{convex combination map inside P} If $f$ is a convex combination map taking the external boundary
of a connected plane embedded graph $G$ to a convex polygon $P$,
then all vertices and edges are mapped by $f$ into $\hull(P)$.
\end{lemma}

{\bf Proof.}
Let $D=\hull(P)$.
Since $D$ is convex, it is enough to show that for every
vertex $u$, $f(u)\in D.$  External vertices are mapped
to corners of $P,$ hence into $D.$

Suppose there is an internal vertex $w$ such that
$f(w)\notin D.$  $D$ is the intersection of finitely many
closed half-planes, and one of them
does not contain $f(w).$  By changing coordinates if
necessary, it can be arranged that $D$ is bounded above
by the $x$-axis and there exist vertices $u$
such that $f(u)$ is above the $x$-axis.
Choose $u$ so $f(u)$ has maximal $y$-coordinate, $h$, say, and
let $H$ be the close half-plane $y \leq h$.

Since $G$ is connected, there is a path
$$ u_0, \ldots , u_k = u $$ where $u_0$ is an external
vertex.  Since $f(u_0) \in D$, $f(u_0)$ is in the interior $H^o$
of $H$,
so without loss of generality, $f(u_{k-1})\in H^o$ and
by Lemma \ref{lem: one-sided}, $f(u)\in H^o$, a contradiction.
{\bf Q.E.D.}\medskip

\begin{lemma}
\label{lem: conv comb emb conv faces}
If a convex combination map is an embedding, then its
embedded faces are convex.
\end{lemma}

{\bf Proof.} Let $F$ be a bounded face.
Since $f$ is a straight-edge embedding, $f(\partial F)$
is a simple polygon, and we need only show it has
no concave corners.  However, if $f(v)$ is a concave
corner then $v$ is an inner vertex and
there is a convex wedge $V$ such that $f(u)\in V$
for all neighbours $u$ of $v$. Let $H$ be a closed
half-plane such that $V\subseteq H$ and $V\backslash H^o
=\{f(v)\}$. By Lemma \ref{lem: one-sided}, $f(v)\in H^o$, a contradiction.
{\bf Q.E.D.}\medskip


\numpara
\label{matrix defining a convex combination map} {\bf Matrix defining a convex combination map.} Given
a plane embedded graph $G$ whose external boundary
is a simple cycle $C$, convex combination maps are easily specified
using a matrix $A$.  Suppose that $G$ has
$m$ vertices $v_1,\ldots,v_m$, the first $n$ of them
belonging to $C$, the last $m-n$ being
internal vertices, and the coordinates of their
images are $x_i, y_i, 1 \leq i \leq m$.  Any map from
vertices to points, including any straight-edge
embedding, is equivalent to a column vector of height $2m$.

Let $A$ be the $m\times m$ matrix whose first
$n$ rows are identical with those of the identity
matrix, and whose
last $m-n$ rows express the barycentric mapping
equations (\ref{weighted average of neighbours}). Equivalently,
for $1 \leq i,j \leq m$, let
$$ a_{ij} = \begin{cases}
1\quad\text{if}~i=j,\\
0\quad\text{if}~ i \not= j ~\text{and}~ j \leq n,~\text{and}\\
-\lambda_{v_iv_j} \text{if} ~ i \not= j.
\end{cases}
$$ Equation
\ref{weighted average of neighbours} can be written in
the form
$$ \sum a_{ij} x_j = 0\quad\text{and}\quad \sum a_{ij}y_j = 0,
\quad ( n < i \leq m).$$ For
any convex combination map $f$ (with
$\lambda_{uv}$ given), let $B_x$ be the column vector of
height $m$
whose first $n$ entries give the $x$-coordinates
of the corners of $P$
and whose other entries are zero; similarly
let $B_y$ specify the $y$-coordinates. Then $f$ is
equivalent to column vectors $X$ and $Y$ satisfying
$$ AX = B_x;\quad AY = B_y.$$


\begin{lemma}
\label{unique convex combination map} {\rm (i)} If $G$ is connected then the above matrix $A$ is
invertible.

{\rm (ii)} If $G$ is a connected plane embedded graph whose external
boundary is a simple cycle, and whose external
vertices are mapped in cyclic order to the corners of
a convex polygon, and weights $\lambda_{uv}$ are given,
then this map extends to a unique
convex combination map of $G$.
\end{lemma}

{\bf Sketch of proof.}
(See \cite{tutte,bsst,floater97,white}.)
Tutte's proof of (i) \cite{tutte,bsst} says that
the determinant of $A$ (scaled up) is the number of spanning trees of
a certain connected graph related to $G$.
There is a much more transparent proof given
in \cite{floater97} and also in \cite{white} saying
that if $A$ has nonzero kernel then one can follow
a path from an external vertex to an internal vertex
where the internal vertex cannot satisfy Equation
\ref{weighted average of neighbours}.
Part (ii) follows trivially.\qed





\begin{definition}
\label{nodal 3-connectivity} A graph $G$ is {\em nodally 3-connected} if it
is biconnected and for every two subgraphs
$H$ and $K$ of $G$, if $G = H\cup K$ and
$H\cap K$ consists of just two vertices (and
no edges), then $H$ or $K$ is a simple path.
\end{definition}

\begin{proposition}
\label{nodally 3-connected and no deg 2} Every triconnected graph is nodally 3-connected,
and every nodally 3-connected graph with no vertices
of degree 2 is triconnected. (Proof omitted.)\qed
\end{proposition}

\begin{definition}
\label{peripheral polygon} A {\em peripheral polygon} in a connected graph $G$
is a simple cycle $C$ such that $G\backslash C$ is
connected.
\end{definition}

The following result of Tutte's is fundamental.

\begin{proposition}
\label{tutte's theorem} {\em (Tutte \cite{tutte})}. If $G$ is a nodally 3-connected
planar graph\footnote{with a few
exceptions: see Figure \ref{nontutte2.fig}. The result is phrased
differently in \cite{tutte}.}  and $C$ is a peripheral polygon, and
the vertices of $C$ are mapped (in cyclic order) onto the corners of a convex
polygon $P$, then that map extends to a unique barycentric map
which is a convex, straight-edge embedding of $G$.\qed
\end{proposition}

It is easy to give a counterexample when $G$ is not
nodally 3-connected.  For example, in Figure \ref{nontutte.fig},
any barycentric map must map
the inner square face to a line-segment.  The figure
illustrates different plane embeddings of the same
graph, which is not nodally 3-connected.

We shall rely more heavily on the following

\begin{proposition}
\label{floaters theorem} {\em (Floater \cite{floater03}).}
If $G$ is a triangulated (plane embedded) graph, then every convex
combination map of $G$ is an embedding.\qed
\end{proposition}

Theorem \ref{planar nodally 3-connected} below shows that,
except regarding the external face, a {\em planar} graph
is nodally 3-connected if and only if barycentric maps
are plane embeddings.



Lemmas
\ref{plane disconnected face disconnected} and
\ref{planar biconnected} below are fairly obvious
and well-known, but still worth mentioning.

\begin{lemma}
\label{plane disconnected face disconnected} A plane embedded graph $G$ is
connected if and only if for every face $F$,
the boundary $\partial F$ is (path-)connected.\qed
\end{lemma}




\begin{proposition}
\label{eulers formula} {\bf (Euler's Formula.)}
If $G$ is a plane (straight-edge) embedded graph
then 
$$v-e+f = c+1,$$ where $v,e,f$, and $c$ are the
numbers of vertices, edges, faces, and components
of $G$. (Proof omitted.)\qed
\end{proposition}

\begin{lemma}
\label{link of u} Let $G$ be a straight-edge embedded plane graph in which all face boundaries
are simple cycles, and let $u$ be any vertex of $G$.

Let $x_0, \ldots , x_k$ be a list of neighbours of
$u$ consecutive in anticlockwise order; possibly $x_0 = x_k$
but otherwise they are distinct.  For $1 \leq j \leq k$
let $F_j$ be the face occurring between the edges (line-segments)
$u x_{j-1}$ and $u x_j$ in the anticlockwise sense.  (The
faces $F_j$ are not necessarily distinct.)

Let $B$ be the subgraph formed by the edges and vertices
in $\bigcup_j \partial F_j$.

Then any two vertices in the list $x_j$ are joined by a path in
$B \backslash u$. See Figure \ref{B.fig}.
\end{lemma}

\begin{figure}
\centerline{\includegraphics[height=1.5in]{figures/B}}
\caption{neighbours of $u$ connected by paths avoiding $u$.}
\label{B.fig}
\end{figure}

{\bf Proof.}
$B \backslash u$ is also the subgraph consisting
of all vertices and edges in $\bigcup_j (\partial F_j \backslash u)$.
Since each face is a simple cycle, $\partial F_j \backslash u$
is a path joining $x_{j-1}$ to $x_j$.  Thus
$B \backslash u$ contains paths joining all these vertices
$x_j$. {\bf Q.E.D.}\medskip

\begin{lemma}
\label{planar biconnected} A plane straight-edge embedded graph $G$ is biconnected if and only if
the graph consists of a single vertex or a single edge, or
the boundary of every face is a simple cycle.
\end{lemma}

{\bf Sketch proof.}
(i): If. A single vertex or edge is biconnected, so we
assume that the boundary of every face is a simple cycle.
$G$ is connected (Lemma
\ref{plane disconnected face disconnected}).

For any vertex $x$ and all neighbours $x_j$ of $x$ there
exist paths connecting these neighbours which avoid $x$
(Lemma \ref{link of u}).  Therefore all these neighbours are
in the same component of $G\backslash x$, and it follows that
$G\backslash x$ is connected.  Hence $G$ is biconnected.

(ii): Only if. Suppose that $G$ is connected, not a single vertex
or edge, and there exists a face $F$ whose
boundary is not a simple cycle (graph):
$\partial F$ is connected but
contains a node $x$ whose degree (in $\partial F$, not in $G$)
differs from $2$.  If $\partial F$ contained a vertex  of
degree $0$ then (since $G$ is nontrivial) $G$ would be disconnected.
If it contained a vertex of degree $1$, then $G$ would
be disconnected or not biconnected.  Hence we can assume
that all vertices on $\partial F$ have degree $\geq 2$ in $\partial F$.

Let $u\in \partial F$ be a vertex of degree $\geq 3$ in $\partial F$.  Let
$x_1,\ldots , x_k$ be the vertices adjacent to $u$ in anticlockwise
order.  For $1 \leq j\leq k$, $x_j u x_{j+1}$ ($x_{k+1}=x_1)$ forms a clockwise
part of the boundary of a face incident to $u$.  Since $u$ has
degree $\geq 3$ in $\partial F$, at least two of these
paths are incident to $F$ and there are fewer than $k$
distinct faces incident to $u.$

Let $G' = G\backslash \{u\}$. All faces incident to $u$ in $G$
merge into a single face of $G',$ and the other faces of $G$
are preserved.
The Euler formula gives
$$ v - e + f = 2$$ for $G$, since $G$ is connected.  Correspondingly
for $G'$,
$$ v' - e' + f' = 1 + c'.$$ Now $v' = v-1,$ and $e' = e - k$.
Since in $G'$ fewer than $k$ faces are merged into a single face,
$f' > f+1-k$. Therefore
$$ v' - e' + f' > v-1 - e + k + f + 1 - k = 2,$$ so $c' > 1$,
$G'$ is disconnected, and $G$ is not biconnected. {\bf Q.E.D.}\medskip

\numpara
\label{witnesses} {\bf Witnesses for a non-nodally 3-connected graph.}
Suppose $G$ is not nodally 3-connected.  We
say that $H,K,u,v$ are {\em witnesses} if $G=H\cup K$,
$H\cap K$ contains just two vertices $u,v$ and no edge,
neither $H$ nor $K$ are path graphs, and neither $H$ nor $K$
equals $G$.

\begin{lemma}
\label{edges incident from H and K} {\rm (i)}
Given witnesses $H,K,u,v$, if $L$ is a path in $G$
connecting $H\backslash K$ to $K\backslash H$,
then $L$ contains three consecutive vertices $r,s,t$ where
$\{r,s\}\in H$, and $\{s,t\}\in K$, $r\in H\backslash K$,
$t\in K\backslash H$, and $s \in H\cap K$, so $s=u$ or $s=v$.

{\rm (ii)} Any path (respectively, cycle) which avoids $u$ and $v$ except
perhaps at its endpoints
(respectively, perhaps once), is entirely in $H$
or in $K$.
\end{lemma}

{\bf Proof.} (i) The first vertex in $L$ is in $H\backslash K$,
so the first edge is in $H$.  Similarly the last edge is in $K$.
Therefore there exist three consecutive vertices $r,s,t$
on the path where $\{r,s\} \in H$ and $\{s,t\}\in K$.
Then $s\in H\cap K$, so $s=u$ or $s=v$ and $s$ is
incident to edges from $H$ and from $K$.

(ii)
Now let $P$ be a path which avoids $u$ and $v$
except perhaps at its endpoints. This includes the
possibility of a cycle, viewed as a path which
begins and ends at the same vertex $w$: we allow
$w$, but no other vertex on the cycle, to equal
$u$ or $v$.

If the path is not entirely
in $H$ nor in $K$, then it contains a triple
$r,s,t$ where $s = u$ or $s=v$, a contradiction. {\bf Q.E.D.}\medskip



The proof of
Theorem \ref{planar nodally 3-connected} is long.  To lighten
it somewhat, we prove

\begin{lemma}
\label{2 or 3 faces} Let $G$ be a plane embedded graph in which all face
boundaries are simple cycles.  Then
{\rm (i)} either $G$ is a simple cycle with two faces, or\hfil\break
{\rm (ii)} for no two faces $F,F'$ is $\partial F \cap \partial F'$ a simple
cycle, and
if there are 3 faces $F_1,F_2,F_3$ such that
$$
Q_1 = \partial F_1 \cap \partial F_2,
Q_2 = \partial F_2 \cap \partial F_3, \quad\text{and}\quad
Q_3 = \partial F_3 \cap \partial F_1
$$ are all nonempty and connected, therefore simple paths, and they all
join the same two vertices $u$ and $v$,
then there are exactly three faces, and
$G$ consists of two nodes connected by three paths.
\end{lemma}

{\bf Proof.} Since all face boundaries are simple cycles,
$G$ is biconnected, hence connected.

(i) Suppose
$\partial F \cap \partial F' = \partial F,$  that
is $\partial F \cap \partial F'$ is a Jordan curve $J$.
By Theorem \ref{jordan curve theorem} (ii),
$F$ is the inside of $J$ and $F'$ the outside
or vice-versa, so $G$ is a simple cycle with two faces.

(ii)  W.l.o.g.\  $F_1$ and $F_2$ are bounded.  Their
intersection $Q_1$ is a simple path, which means that
$X=\overline{F_1}\cup \overline{F_2}$ is simply connected,
and $\partial X = \partial F_1 \cup \partial F_2 \backslash
\int( Q_1)$.

The only faces meeting the relative interior of $Q_1$ (respectively, $Q_3$) are
$F_1$ and $F_2$ (respectively, $F_3$ and $F_1$), so $Q_1 \not= Q_3$.
These are different paths joining $u$ to $v$ on $\partial F_1$,
so $\partial F_1 = Q_1\cup Q_3$. Again, $\partial F_2 = Q_1 \cup Q_2$,
Thus $\partial X = Q_2 \cup Q_3 = \partial F_3$.

$F_3$ is either the inside or outside of $\partial F_3$
(Theorem \ref{jordan curve theorem}), but $F_1\cup F_2$
are inside, so it is the outside, and $F_3$ is the
unbounded face.  Thus there are three faces and
$G$ is the union of three paths $Q_1\cup Q_2\cup Q_3$
with two nodes in common.
{\bf Q.E.D.}\medskip

\begin{theorem}
\label{planar nodally 3-connected} A plane (straight-edge)
embedded graph is nodally 3-connected iff it is biconnected and
the intersection of any two face boundaries is connected.
\end{theorem}

{\bf Proof.} We can assume $G$ is biconnected,
since that is required for nodal 3-connectivity.
Since $G$ is biconnected either it is empty or
trivial, or a single edge, or
every face is bounded by a simple cycle.
In the first three cases the graph is obviously
nodally 3-connected and
biconnected with one face, so we need
only consider the fourth case and can assume that every face is bounded by
a simple cycle.

We can assume that $G$ is straight-edge embedded. Therefore
the boundary of every face is a simple polygon.

{\bf Only if:} Suppose $F_1$ and $F_2$ are different faces and
$\partial F_1\cap \partial F_2$ is disconnected. R.T.P.
$G$ is not nodally 3-connected.

Let
$u$ and $v$ be vertices in different components of
$\partial F_1\cap \partial F_2$.  For $i=1,2$
there are two paths $P_i$ and $Q_i$ joining
$u$ to $v$ in $\partial F_i$.  These paths
are polygonal.

One can also construct a path $P_1'$ within $F_1$, loosely speaking
by displacing $P_1$ slightly into $F_1$, and connecting its endpoints
to $u$ and $v$.  The resulting path is in
$F_1$ except at its endpoints.  Similarly one can construct
a path $P_2'$ in $F_2$ except at its endpoints.
These paths together form a (polygonal) Jordan curve $J$ which 
meets $G$ only at $u$ and $v$.  By construction, $P_1 \cup P_2$
is inside $J$ and $Q_1 \cup Q_2$ is outside $J$.

Let $H$ (respectively, $K$) be the subgraph consisting of all
vertices and edges of $G$
which lie inside or on $J$ (respectively, outside or on $J$).
The only vertices in $H\cap K$ are $u$ and $v$, and $H\cap K$
contains no edge. $H$ contains
$P_1 \cup P_2$ and therefore is not a path graph, since otherwise
$P_1 = P_2$ and $u$ and $v$ would be in the same component of
$\partial F_1 \cap \partial F_2$.
Similarly $K$ is not a path graph.  Therefore
$G$ is not nodally 3-connected.\hfil\break

{\bf If:}  Suppose $G$ is biconnected but not nodally 3-connected,
and $H,K,u,v$ are witnesses.  $G$ has more than one face,
so all face boundaries are simple cycles.

{\em Claim 1.} The subgraphs
$H\backslash K$ and $K\backslash H$ are nonempty.
If every vertex in $K$ were also in $H$,
then the vertices in $K$ are in $H\cap K$,
that is, $u$ and $v$.  Either $K$ has no
edges, in which case $H=G$, or
it has the edge $\{u,v\}$ and is a path graph.
Neither is possible.  Therefore $H\backslash K$
and similarly $K\backslash H$ are nonempty.

{\em Claim 2.} Neither $u$ nor $v$ are isolated
vertices in $H$ nor in $K$.

Otherwise suppose $u$ is isolated in $K$.
Let $L$ be any path joining $H\backslash K$ to $K\backslash H$.
By Lemma \ref{edges incident from H and K}, every
path connecting $H\backslash K$ to $K\backslash H$
contains a vertex, $u$ or $v$, incident to
edges from $H$ and from $K$.  By hypothesis, $u$
is not; so every such path contains $v$.  By
Claim 1, at least one such path exists, so
$G\backslash v$ is not connected, and $G$
is not biconnected.

{\em Claim 3.}
Both $u$ and $v$ have neighbours both in $H\backslash K$ and
in $K\backslash H$. Suppose all neighbours of $u$ are in $H$.
Since $u$ is not isolated in $K$, there is an edge
$\{u,t\}$ in $K$ incident to $u$.  But $t$ is a neighbour
of $u$, therefore $t\in H\cap K$, so $t=v$.  The only
edge in $K$ incident to $u$ is $\{u,v\}$.

Consider a path in $G$
joining $H\backslash K$ to $K\backslash H$. Let
$t$ be the first vertex where the path meets $K\backslash H$,
and let $s$ be the vertex before $t$ on the path. Since
$\{s,t\}\in K$ and $s\notin K\backslash H$, $s\in H\cap K$:
$s=u$ or $s=v$.  However, if $s=u$, then, since $t\in K$,
$t=v$ and $t\notin K\backslash H$.  Therefore $s=v$.  This
implies that every path from $H\backslash K$ to $K\backslash H$
contains $v$. Again by Claim 1, such paths exist, so
$G$ is not biconnected.

This contradiction shows
that not all neighbours of $u$ are in $H$; neither are
they in $K$, and the same goes for $v$.

{\em Claim 4.} The vertices $u$ and $v$ share a face in common.
Otherwise let $x_1,\ldots,x_k$ be the neighbours
of $u$.  We know (Lemma \ref{link of u}) that
they are all connected by paths in $B\backslash u$,
where $B$ is the union of boundaries of bounded
faces incident to $u$.  Assuming $v$ is incident to none
of these faces, these
paths would also avoid $v$. This implies that all neighbours
of $u$ are in $H$ or in $K$, contradicting Claim 3.


{\em Claim 5.} The vertices $u$ and $v$ have at least two
faces in common.  Let
$F_1,\ldots$ be the faces incident
to $u$ in anticlockwise order around $u$.  At least
one of these faces, w.l.o.g.\ $F_1$, is incident to $u$ and to $v$.
Suppose no other face is.

There are two cases.
If $u$ or $v$, w.l.o.g.\ $u$, is an internal vertex, then
all faces incident to $u$ are bounded, and by Lemma
\ref{link of u}, the subgraph
$\bigcup_{i\geq 2} (\partial F_i \backslash u)$ would
be connected and contain neither $u$ nor $v$.  Then
all vertices in this subgraph would belong to $H$
or to $K$.  Since it includes all neighbours of $u$
in $G$, it would contradict Claim 3.

If both $u$ and $v$ are external vertices, then $F_1$
is the external face,
and all bounded faces incident to $u$ avoid $v$.
This time we consider the subgraph
$\bigcup_{i\geq 2} (\partial F_i \backslash u)$.
Again this is
a connected subgraph containing all neighbours of $u$
in $G$, and again it omits both $u$ and $v$, so
again all vertices in it are in $H$ or in $K$, and
again Claim 3 is contradicted.

Therefore $u$ and $v$ have at least two faces $F$ and $F'$ in common.

{\em Claim 6.}
If $u$ and $v$ are incident to three faces
$F_1$, $F_2$, and $F_3$, then the boundaries of at least two
of these faces have disconnected intersection.
Otherwise,
by Lemma \ref{2 or 3 faces}, $G$ consists of two
nodes $u,v$ connected by three paths. If $G= H\cup K$
where $H\cap K = \{u,v\}$ then $H$ or $K$ is
a path graph: $G$ is nodally 3-connected.

This contradiction shows that the one of the pairs
$\partial F_i \cap \partial F_j$ is disconnected, as
claimed.

{\em Claim 7.}
If there are exactly two faces $F$ and $F'$ incident
to $u$ and to $v$,
then $\partial F\cap \partial F'$ is disconnected.

Otherwise $\partial F\cap \partial F'$
is a path $Q'$ joining a vertex $u'$ to another
vertex $v'$ and containing a
subpath $Q$ joining $u$ to $v$.  Not all of
$u',u,v,v'$ need be distinct, but it is assumed
that they occur in that order in $Q'$.

By
Lemma \ref{edges incident from H and K},
all vertices in $Q$
belong to $H$ or to $K$: w.l.o.g.\ 
to $H$. The boundary cycles  $\partial F$ and
$\partial F'$ include two other paths, $Q_1$ and $Q_2$,
respectively, joining $u'$
to $v'$. Let $J=Q_1\cup Q_2$, a Jordan curve.

If $u'\not= u$ then $J$ meets $H\cap K$ at $v$ alone,
or not at all, and by Lemma
\ref{edges incident from H and K}, all vertices on
$J$, plus those in $Q'\backslash Q$, belong to $H$ or to $K$.

If all vertices on $J$ belong to $H$, then all
vertices outside $J$ also belong to $H$, because for
any vertex $y$ outside $J$, one can choose a shortest
path joining $y$ to a vertex in $J$.
Neither $u$ nor $v$ occur as internal vertices on this
path, so all vertices on the path are in $H$ or $K$ (Lemma
\ref{edges incident from H and K}),
i.e., $H$, since the last vertex is in $H$.

We have counted all vertices in $G$: those outside $J$,
those on $J$, and those on $Q'$, and all are in $H$,
so $H=G$, which is false.

On the other hand,
if all vertices on $J$, and in $Q'\backslash Q$, belong to $K$,
then all vertices outside $J$ belong to $K$,
and $H=Q$ is a path graph, which is false.
This proves Claim 7 in the case $u\not= u'$,
and by symmetry in the case $v\not= v'$.

If $u=u'$ and $v=v'$ then $Q=Q'$: let $Q_1$ and
$Q_2$ be the other subpaths joining $u$ to $v$
in $\partial F$ and $\partial F'$ respectively.
By Lemma \ref{edges incident from H and K},
each subpath $Q_i$ is contained in $H$ or
in $K$.  Again we have a Jordan curve
$J = Q_1 \cup Q_2$.

If $u$ and $v$ are not both external vertices,
w.l.o.g.\
$u$ is an internal vertex, then
$F$ and $F'$ are bounded faces incident to $u$,
and since $\partial F \cap \partial F' = Q$,
they are consecutive in cyclic order.
Let $u_1$ (respectively, $u_2$) be the second
vertex (following $u$) in $Q_1$ (respectively, $Q_2$).
The only faces incident to $u$ and to $v$ are 
$F$ and $F',$ so $u_1$ and $u_2$
differ from $v$ and
$u_1$ and $u_2$ are connected
by a path which avoids $u$ and $v$
(Lemma \ref{link of u}).
Therefore, by Lemma 
\ref{edges incident from H and K}, $u_1$ and $u_2$ are both
in $H$ or in $K$, and so are all vertices on $J$.
The same goes
for all vertices outside $J$, so either $H=G$ or
$H=Q$ is a path graph, a contradiction.


This leaves the case where $u$ and $v$ are external
vertices with exactly
two faces in common, $F$ and $F'$, whose
boundaries have connected intersection.
Since $u$ and $v$ are external vertices,
one of these faces,
$F',$ say, is the external face.
Since $G$ is not nodally 3-connected, it is not
a simple cycle, and $Q = \partial F\cap \partial F'$ is a simple path
joining $u$ to $v$ (Lemma \ref{2 or 3 faces}).
Let $Q_1$ and $Q_2$ be the other paths
joining $u$ to $v$ on $\partial F$ (respectively,
$\partial F'$). $\partial F' = Q \cup Q_2$ is the
external cycle, a Jordan curve, and $Q_1$ separates
its interior into two regions of which $F$ is one.
Let $J=Q_1\cup Q_2$. It is a Jordan curve surrounding
the other region.

Let $u_i,~ i=1,2,$ be the second vertices
on $Q_i$. Again
there is a path joining $u_1$ to $u_2$ which
avoids $u$ and $v$, and
all vertices on $J$ are in $H$ or $K$, and
the same holds for all vertices inside $J$.
If they are all in $H$ then $H=G$, and if they
are all in $K$ then $H=Q$, a simple path.
This contradiction finishes the proof of Claim 7.

Claims 6 and 7 taken together amount to the desired result.
{\bf Q.E.D.}\medskip

\numpara
\label{chord-free triangulated graphs} {\bf Chord-free triangulated graphs.}
A triangulated plane embedded graph is one in which every
bounded face is bounded by three edges.  In a triangulated
biconnected graph the external boundary is also a simple cycle.
It can only fail to be nodally 3-connected if a bounded
face meets the external boundary in a disconnected set.
Equivalently, one of its edges is a chord joining
two vertices on the external boundary,
and the other two edges are not both on the external boundary
\cite{white}.

\begin{figure}
\centerline{\includegraphics[height=1in]{figures/quadril}}
\caption{a nodally 3-connected but not triconnected triangulated
planar graph}
\label{quadril.fig}
\end{figure}

The graph in Figure \ref{quadril.fig} is nodally 3-connected but
not triconnected.


A fully triangulated planar graph is a triangulated planar
graph in which there are three external edges. In other words,
the external face also is bounded by a 3-cycle.  Therefore the
external cycle has no chords, so every fully triangulated
planar graph is nodally 3-connected.

Also let $G$ be a fully triangulated planar graph containing
a vertex $v$ of degree 2. Let $u$ and $w$
be the neighbours of $v$. There are only two faces
incident to $v$ and they are both incident to
$u,v,$ and $w$.  One of them must be
the external face. Thus $u,v,$ and $w$ are the
three external vertices.  They also bound
the only bounded face.  $G$ is a 3-cycle,
and therefore triconnected.

On the other hand,
if $G$ is fully triangulated
then it is nodally 3-connected,
so if it contains no vertex of degree $2$
then it is triconnected
(Proposition
\ref{nodally 3-connected and no deg 2}).  Therefore

\begin{corollary}
Every fully triangulated planar graph is triconnected.\qed
\end{corollary}

