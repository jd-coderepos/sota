In this section, we evaluate the effectiveness of \system in preventing the six types of attack scenarios for audio channels discussed in Section \ref{attacks}, examine whether 17 widely-used apps and services can still be run effectively under \system, and measure the performance impact of \system.  
\vspace{-0.08in}


\subsection{Attack Prevention}

Table~\ref{table:attack} compares the effectiveness of \system in preventing the six types of attack scenarios on audio channels outlined in Section~\ref{attacks} to Android, other related work described in Section~\ref{related}, and Simple Isolation\footnote{No simultaneous access to the microphone and speaker by two different processes.}.  \system is capable of preventing all six types of attack scenarios.  Other defenses prevent no more than two types of these attack scenarios because they lack awareness of the impact of external parties.
Further, other defenses are of limited applicability or may often cause false positives, as described in the next section.

\begin{comment}
We start the analysis by evaluating the effect of enforcing {\em Simple Isolation} between apps and services by preventing simultaneous access to the microphone and speaker by two different processes (\texttt{Simple Isolation} in Table \ref{table:attack}). Simple isolation only considers audio channels created between processes, therefore does not consider audio channels involving external parties, i.e. user or an in-proximity device. 

We then evaluate \system while enforcing MLS policies (\system in Table \ref{table:attack}) and while taking into account also external parties as possible ends of an audio channel. 

We also report that the current Android OS (\texttt{Base Android} in Table \ref{table:attack}) is unable to prevent the attacks analyzed. Finally, we report results relative to defence mechanisms proposed in related work discussed in Section \ref{related} (rows 4-7 in Table \ref{table:attack}). For evaluation purposes, we authenticate the device owner by using the {\em device screen status}. In particular, \system assumes that the external party is the device owner, whenever the screen is unlocked. Otherwise, \system assumes the external party is a user different from the device owner or an in proximity device. The use of the screen status as mechanism to identify an external party is motivated by the fact that the screen pass-code is supposed to be known only by the device owner or the screen is unlocked using fingerprints, e.g. Touch ID used on Apple devices. Other mechanisms could be adopted to identify the an external parties if the mobile device supports touchless voice authentication, such as speaker recognition or a spoken pass-phrase, which would allows the user to send voice commands and activate apps and services without touching the device.

\end{comment}


\begin{table}[t]
\centering
\caption{Attack Prevention Analysis} 

\tiny


\label{table:attack}
\tabcolsep=0.07cm
\vspace*{+2pt}

\begin{tabular}{l|c|c|c|c|c|c|} 
\cline{2-7} 
\multicolumn{1}{c|}{\multirow{9}{*}{\begin{tabular}[c]{@{}c@{}}\textit{Legend:} \\\\ \cmark Attack Prevented \\ \xmark Attack Succeeded \\   Attack might \\ be Prevented \\  \end{tabular}}}& 

\multicolumn{1}{c|}{\multirow{9}{*}{ \begin{turn}{90} \hspace{-2mm} \begin{tabular}{@{}c@{}}\emph{Touchless Control} \\ Type 1\\Scenario 1\end{tabular} \end{turn} }} & 
\multicolumn{1}{c|}{\multirow{9}{*}{\begin{turn}{90}  \begin{tabular}{@{}c@{}}\hspace{-3mm}\emph{Keylogger} \\ Type 1\\Scenario 2 \end{tabular}\end{turn} }}& 
\multicolumn{1}{c|}{\multirow{9}{*}{\begin{turn}{90} \hspace{-2mm} \begin{tabular}{@{}c@{}}\emph{Device Control} \\ Type 2 \\ Scenario 3 \end{tabular} \end{turn} }} &
\multicolumn{1}{c|}{\multirow{9}{*}{\begin{turn}{90} \hspace{-1mm}\begin{tabular}{@{}c@{}}\emph{Speak Out} \\ Type 2 \\Scenario 4 \end{tabular} \end{turn} }}& 
\multicolumn{1}{c|}{\multirow{9}{*}{\begin{turn}{90}\hspace{-2mm} \begin{tabular}{@{}c@{}}\emph{Voice Commands} \\ Type 3 \\Scenario 5 \end{tabular} \end{turn}}}& 
\multicolumn{1}{c|}{\multirow{9}{*}{\begin{turn}{90}\hspace{-2mm} \begin{tabular}{@{}c@{}}\emph{Stealthy Recording  } \\ Type 3 \\Scenario 6 \end{tabular} \end{turn}}}\\  

&& & & & & \\  
&& & & & & \\  
&& & & & & \\  
&& & & & & \\ 
&& & & & & \\  

&& & & & & \\ 
&& & & & & \\ 
&& & & & & \\ \hline  

\rowcolor{Gray}\multicolumn{1}{|c|}{ \texttt{Base Android}}  &  \xmark & \xmark  & \xmark &\xmark&\xmark&\xmark \\ \hline   \hline
\rowcolor{Gray1}\multicolumn{1}{|c|}{\texttt{Simple Isolation}}&  \cmark & \cmark  & \xmark &\xmark & \xmark & \xmark \\ \hline

\rowcolor{Gray2}\multicolumn{1}{|c|}{\texttt{\system}}& \cmark & \cmark & \cmark  & \cmark & \cmark & \cmark \\ \hline  \hline
\multicolumn{1}{|c|}{\texttt{\cellcolor[gray]{0.8}\begin{tabular}{@{}c@{}}Google Voice Search bug fix \end{tabular} }}& \cmark& \xmark  & \xmark &\xmark&\xmark&\xmark\\ \hline
\multicolumn{1}{|c|}{\texttt{\cellcolor[gray]{0.8}\begin{tabular}{@{}c@{}}Control Access to Speaker\end{tabular}} }& \cmark & \xmark & \xmark &\xmark & \xmark & \xmark  \\ \hline
\multicolumn{1}{|c|}{\texttt{\cellcolor[gray]{0.8}\begin{tabular}{@{}c@{}}System Services Permission\end{tabular}} } &   &   & \xmark &\xmark&\xmark&\xmark \\ \hline
\multicolumn{1}{|c|}{\cellcolor[gray]{0.8}\texttt{\begin{tabular}{@{}c@{}}Voiceprint Recognition\end{tabular}}}  & \cmark & \xmark & \xmark & \xmark & \cmark & \xmark \\ \hline \end{tabular}
\vspace*{-\baselineskip}
\end{table}



\textbf{Touchless Control} (Table \ref{table:attack} column 1) refers to the attack where a malicious app makes use of an audio channel of Type 1 to exploit a system service receiving voice commands to perform security-sensitive operations \cite{Jang,Diao}. This type of attack is prevented by \texttt{Simple Isolation}, because two processes cannot access - at the same time - the microphone and speaker. Similarly, \system detects an unsafe flow from a low-integrity process (malicious app) to a high-integrity process (system service). Rows 4-7 show that other defense mechanisms can prevent this attack. In particular, the  symbol (in row 6) highlights that, a permission mechanism used to authorize the use of System Services could prevent the attack, if the user does not grant the permissions to use the {\em Media Server} for a malicious app.

\textbf{Keylogger} (Table \ref{table:attack} column 2) refers to an attack where a malicious app uses an audio channel of Type 1 to eavesdrop the password typed by the device owner and spoken out by the {\em TalkBack} accessibility service \cite{Jang}. \texttt{Simple Isolation} prevents the attack because two processes cannot access the microphone an speaker at the same time. \system prevents the attack because a unsafe flow from a high-secrecy process (accessibility service) to a low-secrecy process (malicious app) is detected. Furthermore, the  symbol (in row 6) highlights the fact that a permission mechanism, used to authorize the use of System Services, could prevent the attack if the user does not grant the malicious app permission to use the {\em Media Server}. 

\textbf{Device Control} (Table \ref{table:attack} column 3) is an attack performed by using a malicious app running on a device as a source of malicious voice commands (such as those reported in Table \ref{table:commands}) to attack another nearby device. The attack is performed by using an audio channel of Type 2. \texttt{Simple Isolation} does not prevent the attack because it does not consider audio channels involving external parties. On the other hand, \system detects the unsafe flow from a low-integrity process to a high-integrity external party. 

\textbf{Speak Out} (Table \ref{table:attack} column 4) refers to a malicious app eavesdropping voice and sound through the device's microphone to collect security-sensitive information, such as private conversations, successively leaked to an adversary through the device's speaker as soon as the device owner, victim of the attack, is away from the device. The adversary makes use of an audio channel of Type 2 to bypass any lock screen protection mechanism. \texttt{Simple Isolation} does not prevent the attack because it does not consider audio channels involving external parties. On the other hand, \system detects the unsafe flow from a low-integrity process to a high-integrity external party. Furthermore, \system prevents a malicious app from eavesdropping on the device owner because the user approval is required before a low-secrecy process can access the microphone.

\textbf{Voice Commands} (Table \ref{table:attack} column 5) is an attack performed by an adversary directly interacting with the target device via malicious voice commands. In this attack, the adversary uses an audio channel of Type 3. \texttt{Simple Isolation} does not prevent the attack because it does not consider audio channels involving external parties. On the other hand, \system prevents a malicious user, different from the device owner, from delivering voice commands to a system service, by identifying the external party (user) as low-integrity and the system service as high-integrity. As shown in row 7, \texttt{Voiceprint Recognition} can prevent the attack, unless the adversary replays recorded device owner voice commands.

\textbf{Stealthy Recording} (Table \ref{table:attack} column 6) refers to an attack where a malicious app uses an audio channel of Type 3 to stealthily record audio through the device's microphone in order to eavesdrop the device owner and the surrounding environment. \system prevents a malicious app from eavesdropping the device owner voice because the user approval is required before a low-secrecy process can access the device's microphone.

\vspace{-0.08in}


\subsection{System Functionality}



\begin{table*}

\centering
\caption{System Functionality Analysis}


\tiny
\label{table:functionality}
\tabcolsep=0.08cm
\vspace*{+2pt}

\begin{tabular} {l|c|c|c|c|c|c|c|c|c|c|c|c|c|c|c|c|c|}  
\cline{2-18}
\multicolumn{1}{c|}{\multirow{6}{*}{\begin{tabular}[c]{@{}c@{}}  \textit{Legend:} \\  \\\cmark App Runs \\ \texttt{IV} Integrity Violation \\ \texttt{SV} Secrecy Violation \\ \texttt{SIV} Secrecy and Integrity \\ Violation \end{tabular}}}
&  \multicolumn{7}{c|}{\cellcolor[gray]{0.9}System Apps}& \multicolumn{10}{c|}{  \cellcolor[gray]{0.6} Market Apps }\\ 
\cline{2-18} 
& \multirow{5}{*}{ \begin{turn}{90}Voice Dialer \hspace{2.5mm} \end{turn} } 
& \multirow{5}{*}{ \begin{turn}{90}Music \hspace{9mm} \end{turn} }
& \multirow{5}{*}{ \begin{turn}{90}Voice Search \hspace{2mm} \end{turn} }
& \multirow{5}{*}{ \begin{turn}{90}Phone \hspace{9mm} \end{turn} }
& \multirow{5}{*}{ \begin{turn}{90}Hangouts \hspace{5.5mm} \end{turn} }
& \multirow{5}{*}{ \begin{turn}{90}Browser \hspace{7mm} \end{turn}  } 
& \multirow{5}{*}{ \begin{turn}{90}Maps \hspace{9.5mm} \end{turn}  }
& \multirow{5}{*}{ \begin{turn}{90}Pandora \hspace{6.5mm} \end{turn}  }
& \multirow{5}{*}{ \begin{turn}{90}Spotify \hspace{7.5mm} \end{turn} }
& \multirow{5}{*}{ \begin{turn}{90}Viber \hspace{9.5mm} \end{turn}  }
& \multirow{5}{*}{ \begin{turn}{90}WhatsApp \hspace{4mm} \end{turn}  } 
& \multirow{5}{*}{ \begin{turn}{90}Snapchat \hspace{5.5mm} \end{turn}  }
& \multirow{5}{*}{ \begin{turn}{90}Facebook \hspace{5.5mm} \end{turn} }
& \multirow{5}{*}{ \begin{turn}{90}Skype \hspace{9mm} \end{turn}  } 
& \multirow{5}{*}{ \begin{turn}{90}Voice Memos \hspace{1.5mm} \end{turn}  }
& \multirow{5}{*}{ \begin{turn}{90}Voice Recorder \hspace{-0.5mm} \end{turn}  }
& \multirow{5}{*}{ \begin{turn}{90}Call Recorder \hspace{1mm} \end{turn}  } \\  
& & & & & & & & & & & & & & & & & \\  
& & & & & & & & & & & & & & & & & \\  
& & & & & & & & & & & & & & & & & \\   
& & & & & & & & & & & & & & & & & \\  
& & & & & & & & & & & & & & & & & \\   
& & & & & & & & & & & & & & & & & \\   
& & & & & & & & & & & & & & & & & \\   \hline  

\multicolumn{1}{|l|}{\cellcolor[gray]{0.8}\texttt{Simple Isolation}} & \cmark & \cmark  & \cmark & \cmark & \cmark & \cmark & \cmark & \cmark & \cmark & \cmark & \cmark & \cmark  & \cmark & \cmark & \cmark & \cmark & \cmark \\ \hline  

\multicolumn{1}{|l|}{\cellcolor[gray]{0.8}\system \texttt{MLS}}  & \cmark & \cmark  & \cmark & \cellcolor[gray]{0.6}\texttt{SV} & \cellcolor[gray]{0.6}\texttt{SV} & \cmark & \cmark & \cellcolor[gray]{0.8}\texttt{IV} & \cellcolor[gray]{0.8}\texttt{IV} & \cellcolor[gray]{0.4}\texttt{SIV}  &\cellcolor[gray]{0.4}\texttt{SIV} & \cellcolor[gray]{0.4}\texttt{SIV}  &\cellcolor[gray]{0.4}\texttt{SIV} &\cellcolor[gray]{0.4}\texttt{SIV} &\cellcolor[gray]{0.4}\texttt{SIV} & \cellcolor[gray]{0.4}\texttt{SIV} & \cellcolor[gray]{0.4}\texttt{SIV}  \\ \hline  

 \multicolumn{1}{|l|}{\cellcolor[gray]{0.8}\system \texttt{User Approval}}& \cmark  & \cmark  &  \cmark  & \cellcolor[gray]{0.6}\texttt{SV} & \cellcolor[gray]{0.6}\texttt{SV} & \cmark  & \cmark  &  \cellcolor[gray]{0.8}\texttt{IV} &  \cellcolor[gray]{0.8}\texttt{IV} &  \cellcolor[gray]{0.8}\texttt{IV}  &   \cellcolor[gray]{0.8}\texttt{IV} &  \cellcolor[gray]{0.8}\texttt{IV} &  \cellcolor[gray]{0.8} \texttt{IV} &  \cellcolor[gray]{0.8}\texttt{IV} &  \cellcolor[gray]{0.8} \texttt{IV} & \cellcolor[gray]{0.8} \texttt{IV} &  \cellcolor[gray]{0.8}\texttt{IV} \\ \hline 

\multicolumn{1}{|l|}{\cellcolor[gray]{0.8}\system \texttt{Resolver 1}} &\cmark  & \cmark  &  \cmark  & \cmark & \cmark & \cmark & \cmark & \cellcolor[gray]{0.8}\texttt{IV} & \cellcolor[gray]{0.8}\texttt{IV} & \cellcolor[gray]{0.4}\texttt{SIV}  &\cellcolor[gray]{0.4}\texttt{SIV} & \cellcolor[gray]{0.4}\texttt{SIV}  &\cellcolor[gray]{0.4}\texttt{SIV} &\cellcolor[gray]{0.4}\texttt{SIV} &\cellcolor[gray]{0.4}\texttt{SIV} & \cellcolor[gray]{0.4}\texttt{SIV} & \cellcolor[gray]{0.4}\texttt{SIV}\\ \hline 

\multicolumn{1}{|l|}{\cellcolor[gray]{0.8}\system \texttt{Resolver 2}} &\cmark  & \cmark  &  \cmark  & \cellcolor[gray]{0.6}\texttt{SV} & \cellcolor[gray]{0.6}\texttt{SV} & \cmark & \cmark & \cmark & \cmark & \cellcolor[gray]{0.6}\texttt{SV} & \cellcolor[gray]{0.6}\texttt{SV}  & \cellcolor[gray]{0.6}\texttt{SV} & \cellcolor[gray]{0.6}\texttt{SV} & \cellcolor[gray]{0.6}\texttt{SV} & \cellcolor[gray]{0.6}\texttt{SV} & \cellcolor[gray]{0.6}\texttt{SV} & \cellcolor[gray]{0.6}\texttt{SV}\\ \hline

\multicolumn{1}{|l|}{\cellcolor[gray]{0.8}\system} & \cmark & \cmark  & \cmark & \cmark & \cmark & \cmark & \cmark & \cmark & \cmark & \cmark & \cmark & \cmark  & \cmark & \cmark & \cmark & \cmark & \cmark \\ \hline  

\hline

\multicolumn{1}{|l|}{  \cellcolor[gray]{0.6}Requested User Approval}& \cellcolor[gray]{0.8}no & \cellcolor[gray]{0.8}no & \cellcolor[gray]{0.8}no  & \cellcolor[gray]{0.8}no & \cellcolor[gray]{0.8}no & \cellcolor[gray]{0.8}no & \cellcolor[gray]{0.8}no & \cellcolor[gray]{0.8}no & \cellcolor[gray]{0.8}no & yes  & yes  & yes &  yes & yes &  yes & yes & yes \\ \hline 

\multicolumn{1}{|l|}{  \cellcolor[gray]{0.6}User Notified}& yes & \cellcolor[gray]{0.8}no & yes  & yes & yes & yes & yes & \cellcolor[gray]{0.8}no & \cellcolor[gray]{0.8}no & yes  & yes  & yes &  yes & yes &  yes & yes & yes \\ \hline

\multicolumn{1}{|l|}{  \cellcolor[gray]{0.6}App uses Microphone}& yes & \cellcolor[gray]{0.8}no & yes  & yes & yes & yes & yes & \cellcolor[gray]{0.8}no & \cellcolor[gray]{0.8}no & yes  & yes  & yes &  yes & yes &  yes & yes & yes \\ \hline
 
\multicolumn{1}{|l|}{  \cellcolor[gray]{0.6}App uses Speaker}& yes & yes & yes  & yes & yes & yes & yes & yes & yes & yes  & yes  & yes &  yes & yes &  yes & yes & yes \\ \hline

\end{tabular}
\quad
\begin{tabular}{l}
\tiny 1  Snapchat (Take Video)\\
\tiny 2  Facebook (Send Voice Message)\\
\tiny 3  Whatsapp (Send Voice Message)\\
\tiny 4  Voice Recorder (Send Voice Message)\\
\tiny 5  Viber (Send Voice Message) \\
\tiny 6  Google Voice Search (Voice Search)\\
\tiny 7  Browser (Watch Video)\\
\tiny 8  Skype (Video Call)\\
\tiny 9  Call Recorder (Record Phone Call)\\
\tiny 10 Pandora (Listen to a Song)\\
\tiny 11 Spotify (Listen to a Song)\\
\end{tabular}

\vspace*{-\baselineskip}
\end{table*}


We next evaluate the impact of \system on the ability of apps and services to operate normally. 
\begin{comment}
We start by considering the the effect of enforcing  {\em Simple Isolation} among processes accessing the microphone and speaker. We proceed by considering the impact on using \system to enforce MLS. We then analyze the effectiveness of user approval on accesses to the microphone. Finally, we evaluate the utility of two different resolvers to address audio flow errors that would prevent some of the market and system apps from working correctly.
\end{comment}
The results of our analysis are reported in Table \ref{table:functionality}.  We evaluate \system for 10 market apps and 7 system apps distributed with the Android OS from Google. We select market apps that are among the most popular apps available on Google Play.  We also choose market and system apps that use either the speaker or microphone or both, as indicated by the last four rows in Table~\ref{table:functionality}.  


\begin{comment}
In the last four columns, we report if the app needs an approval from the user before accessing the microphone, if the user is notified by the microphone icon and notification light, and whether the app makes use of the microphone and speaker. This last information is obtained systematically by analyzing the app code and checking for the \texttt{RECORD\_AUDIO} permission in the \texttt{AndroidManifest.xml} file. By default, all apps access the speaker.
\end{comment}

Row 1 in Table \ref{table:functionality} shows that, by enforcing {\em Simple Isolation}, all the system and market apps would work fine although interaction among apps is not allowed. For example, the user cannot use the Voice Recorder app to tape the music produced by the Music app. Therefore, although the system functionality is preserved, there is an indirect impact on how apps can interact. 

We then analyze the impact of using \system when MLS is applied to enforce Biba and Bell-LaPadula. From row 2 in Table \ref{table:functionality}, we observe that two security violations are reported for the Phone and Hangouts system apps, which are due to the fact that these apps produce a sound on incoming calls or message receptions, even when the external party is identified as low-secrecy (i.e. device screen locked). This is seen by \system as a flow from a high-secrecy party (system app) to a low-secrecy party (user different from the device owner). Furthermore, from row 2, we observe two integrity violations in correspondence of Pandora and Spotify. This is due to the fact that these apps access the speaker to produce music and sounds, therefore \system sees a flow from a low-integrity party to a external high-integrity party. Finally, row 2 reports secrecy and integrity violations for the remaining market apps from Viber to Call Recorder. The integrity violations are due to the same reason explained for Pandora and Spotify, whereas the secrecy violations are due to the fact that \system sees a flow from an external high-secrecy party (i.e. device owner) to a low-secrecy party (market app). 

Row 3 shows how \system resolves the secrecy violations relative to the market apps by using the user approval mechanism. In particular, whenever a market app uses the microphone, the device owner is notified and can approve or deny the access.

Row 4 shows the effect of using a resolver (\texttt{Resolver 1}) that allows system apps, like Phone and Hangouts, to play approved ring tones and notification sounds even when the external party is identified as low-secrecy. 

Row 5 shows the effect of using a second resolver (\texttt{Resolver 2}) to allow market apps, like Pandora and Viber, to play approved audio files that do not contain malice, such as ring tones, notification sounds and sound tracks, even when the external party is identified as high-integrity.

By combining the user approval and the resolvers, \system runs all the tested apps correctly without impact on the system functionality, as show in row 6.

\subsection{Performance Overhead}

Existing benchmarks for Android, such as AM-Bench \cite{am-bench} and Android Workload Suite \cite{aws}, do not target apps making extensive use of the microphone/speaker and are not publicly available. Instead, we measured the performance overhead  introduced by \system through the following 3 experiments on a Nexus 5 running Android aosp-5.0.1\_r1. 

We first measured the overhead introduced by \system while handling each access request for the microphone and speaker, by measuring the time interval from the time the request is made by the process running the app to the time the access is granted/denied by the \emph{Media Server}. We found that, on average over 10,000 requests, the original Android system required 20.351.90s to handle an access request for the speaker and 25.362.01s for the microphone. When \system is activated to mediate accesses to the microphone and speaker, each access request for the speaker is handled in 24.471.86s and 30.111.99s for the microphone. The main reason for the overhead introduced by \system was the time necessary to recreate the context (i.e., PIDs of involved processes) used to make the access control decision. Note that enabling the \system notification mechanism (icon microphone and notification light) increased the access time for the microphone to 38.432.11s.

Second, we measured the overhead of running a sequence of 11 well-known apps (listed on the left side of Table \ref{table:functionality}) that create audio sessions, closing each app after running a 30s session before opening the subsequent one. We found that both, the original Android system and \system, took 59121.93s to complete on average.  Unfortunately, such a measurement required a human in the loop, but the audio session length reduced the variability caused by human actions.  The main overhead for \system incurred on opening and closing speaker and microphone connections, which are infrequent and low overhead operations.

Third, we used the Voice Recorder app to examine the overhead if we tried to create audio sessions as quickly as humanly possible on the device.  To do this, the user continuously tapped on the microphone button, as quickly as possible, to keep recording new voice messages. The experiment was repeated 5 times over a duration of approximately one minute. We registered a maximum of 21 access requests/minute for the microphone and a maximum of 53 access requests/minute for the speaker. We found that, with the highest possible number of audio sessions in a minute, both, the original Android and \system completed the task in approximately 59.54s, indicating overhead not detectable.




 



