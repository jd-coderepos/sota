\documentclass[conference]{IEEEtran} 


\usepackage{amsmath}
\usepackage{latexsym} 
\usepackage{epsfig}

\newtheorem{problem}{Problem}
\newtheorem{statement}{Statement}
\newtheorem{proposition}{Proposition}
\newtheorem{definition}{Definition} 
\newtheorem{corollary}{Corollary}
\newtheorem{theorem}{Theorem} 
\newtheorem{lemma}{Lemma}
\newtheorem{claim}{Claim}

\newcommand{\sAppendix}[1]{
	{\large\bfseries\appendixname\par
	\nohyphens\centering#1\par}
	\vspace{\baselineskip}
}

\def\Proof{\par\noindent{\bf Proof:}\indent}
\def\EOP{\hfill\par\vskip1em}
\def\FUNC#1{\mbox{\tt #1}}
\def\PROG#1{}
\def\MYBOX{\medspace]\negthickspace\negmedspace[\medspace}
\def\TT#1{\texttt{#1}}


\begin{document}

\title{Void Traversal for Guaranteed Delivery in Geometric Routing}
\date{}

\author{Mikhail Nesterenko\thanks{This research was
supported in part by DARPA contract OSU-RF\#F33615-01-C-1901
and by NSF CAREER Award 0347485}
and Adnan Vora \\
Computer Science Department \\
Kent State University \\
Kent, OH, 44242 \\
mikhail@cs.kent.edu, avora@cs.kent.edu \\
}

\date{}

\maketitle

\begin{abstract}
Geometric routing algorithms like GFG (GPSR) are lightweight, scalable
algorithms that can be used to route in resource-constrained ad hoc
wireless networks.  However, such algorithms run on planar graphs
only.  To efficiently construct a planar graph, they require a
unit-disk graph. To make the topology unit-disk, the maximum link
length in the network has to be selected conservatively. In practical
setting this leads to the designs where the node density is rather
high. Moreover, the network diameter of a planar subgraph is greater
than the original graph, which leads to longer routes. To remedy this
problem, we propose a void traversal algorithm that works on arbitrary
geometric graphs. We describe how to use this algorithm for geometric
routing with guaranteed delivery and compare its performance with GFG.
\end{abstract}


\section{Introduction} \label{SecIntro}
Geometric routing is a promising approach for message transmission in
ad hoc wireless networks.  Unlike traditional ad hoc routing,
geometric routing algorithms have no control messages or routing
tables, the nodes maintain no information about data messages between
transmissions and each data message is of constant size. Hence,
geometric routing is quite scalable. Geometric routing is particularly
appropriate for wireless sensor networks. Networked sensors usually
have limited resources for routing information, yet many applications
for sensor networks require ad hoc configurations of large scale.

In geometric routing, each node knows it own geographic coordinates.
Each node is also aware of the coordinates of other nodes and links in
a part of the network around it. The coordinates are either obtained
from GPS receivers or a localization algorithm
\cite{Bulusu:2004:SCL}. The source node of a data message has the
coordinates of the target but the source does not know the complete
route to it. The source selects one of its neighbors and sends the
message to it.  After getting the message the receiver forwards the
message further.  The objective of the algorithm is to deliver the
message to the target.

The simplest version of a geometric routing algorithm is \emph{greedy}
routing. In greedy routing the source, as well as each intermediate
node, examines its neighboring nodes and forwards the message to the
one that is the closest to the target. Unfortunately, the greedy
routing algorithm fails if the message recipient is \emph{local
minimum}: all its neighbors are further away from the target than the
node itself. In \emph{compass} routing \cite{CCCG99*51}, the next hop
is selected such that the angle between the direction to the next hop
node and to the destination is minimal. However, compass routing
livelocks even on planar graphs.

GFG \cite{Bose01} (also known as GPSR \cite{MOBICOM00*243}) is the
first algorithm that guarantees delivery of the message. The algorithm
is designed for planar graphs. It uses greedy routing. To get out of a
local minimum GFG sequentially traverses the faces of the planar graph
that intersect the line between the source and target. GFG assumes
that the original communications graph is a unit-disk graph. For this
graph, the authors construct its Gabriel subgraph. This subgraph
preserves the connectivity of the original graph and it can be
constructed locally by each node. Datta et al \cite{DSW02} propose
further improvements to GFG.  Kuhn et al \cite{DialM`02*24} present a
similar algorithm with asymptotically optimal worst-case performance.

\ \\ \textbf{Our contribution}. \ \ The guaranteed delivery geometric
routing algorithms, of which we are aware, have the following
shortcoming.  Their local minimum avoidance part runs over a planar
graph. The efficient construction of a planar graph requires that the
original graph is unit-disk. As recent work \cite{G+02b} demonstrates
this assumption is not realistic for connection topologies formed by
common networked sensors such as Berkeley motes
\cite{Hill:2002:MWP}. Motes' radio propagation patters prove to be
quite intricate.

In this paper we demonstrate how to carry out geometric routing with
guaranteed delivery on arbitrary non-planar graphs. The foundation of
our approach is a void traversal algorithm. We describe algorithms
VOID-1 and VOID-2 that incorporate void traversal for routing across
the whole network. We also present GVG --- an algorithm that uses
greedy routing and void traversal to get out of local minima.  The
void traversal is based on each node storing the network topology of
its neighborhood. The neighborhood size needs to meet a condition we
call \emph{intersection semi-closure}.  Most practical geometric
graphs meet this condition such that the storage requirements for each
node are independent of the size of the network and rather small.

We compared the performance of VOID-2 with FACE-2 which is the
foundation of GFG. We used randomly generated graphs as a base for
comparison. A large number of graphs did not have unit-disk subgraphs.
Hence FACE-2 cannot be run. On graphs where FACE-2 and VOID-2 are run,
depending on the graph generation parameters, the paths selected by
VOID-2 were 35-75\% shorter. The memory storage requirements for
VOID-2 were independent of the network size and only modestly
increased compared to those of FACE-2.


\ \\ \textbf{Paper organization.} \ \ This paper is organized as
follows. We give definitions and describe our notation in Section
\ref{SecPrelim}. We present our void traversal algorithm in Section
\ref{SecTraversal}. In Section \ref{SecRouting} we describe how this
algorithm can be used for geometric routing. In Section \ref{SecEval}
we describe the results of the performance comparison of the void
traversal versus face traversal algorithms.  Section
\ref{SecConclusion} concludes the paper.


\section{Preliminaries} 
\label{SecPrelim}

\noindent
\textbf{Notation.}\ \ We model a sensor network as a geometric graph.
A geometric graph  is a set of \emph{nodes}
(\emph{vertices})  on a Euclidean plane connected by \emph{edges}
. The number of nodes is . Such graph is \emph{planar} if
its edges intersect only at vertices. \emph{Void} is a region on a
plane such that any two points of this region can be connected by a
curve that does not intersect any of the edges of the graph. A
boundary of a void contains the segments of the edges adjacent to the
void. In every finite graph there is one unbounded \emph{external}
void.  A void in a planar graph is \emph{face}. Observe that the
boundary of a face forms a simple cycle. Moreover, each edge of a
planar graph borders at most two faces. An edge of a non-planar graph
may contain the segments of the borders of arbitrarily many faces.

\emph{Neighborhood}  of a node  is a subgraph of . A
\emph{neighborhood relation} \PROG{N} over graph  associates
each vertex of  with a subgraph of .  Denote  an
edge between nodes  and . Denote  a path from  to
. Denote also  the Euclidean distance between  and .




\ \\ \noindent
\textbf{Intersection semi-closure.}

\begin{definition} \label{DefSemiclosure}
A neighborhood relation \PROG{N} over a geometric graph  is
\emph{-incident edge intersection semi-closed} (or just
\emph{-intersection semi-closed}) if, for every two intersecting
edges  and , either:
\begin{itemize}
\item there exist , , either one no more than
 hops and ,  and  belong to ; or
\item there exist , , either one no more than
 hops and , , and  belong to .
\end{itemize}

\end{definition}

The attractive feature of the intersection semi-closed neighborhood
relation is that for any edge , the information about every
edge that intersects  as well as how to reach such edge is
contained in the union of  and . Observe that for a
particular graph, depending on the value of , such a relation
may not exist. However, for any graph there always exists
a -incident edge semi-closed relation.

To illustrate Definition \ref{DefSemiclosure}, we apply it to
unit-disk graphs.  \emph{Unit-disk graph} is a geometric graph where
two nodes  and  are connected by an edge if an only if .

\begin{figure} 
\center
\epsfig{figure=udisk.eps,width=6cm,clip=} 
\caption{Edge  belongs to  in a unit-disk graph}
\label{FigUdisk} 
\end{figure}

\begin{lemma}\label{UDiskNeighborhood}
A neighborhood relation over a unit-disk graph is 2-intersection
semi-closed if for every node  and every edge  such that
 and  it follows that .
\end{lemma}

\Proof By definition \PROG{N} is 2-intersection semi-closed if
for every edge  and every edge  intersecting ,
and the route to  and  are either in  or . These
routes are no longer than 2 hops.

Let  and  be according to the conditions of the lemma.
Since  is unit-disk, if  and  then
 and . According to the conditions of the
theorem, both of these edges as well as  itself belong to
. If this is the case, then both the route to the endpoints and
the edge itself belong to . Moreover, these routes are at most
1-hop long. Hence, the definition of intersection semi-closure is
satisfied. Similar argument applies to .

Suppose that  and . Refer to Figure
\ref{FigUdisk} for illustration. Since  intersects , 
could not be further away from  than . Hence, 
and . Since  is a unit-disk,  and . By the conditions of the lemma  and . Suppose that  is closer to  than to . Then, the the
distance between  and  is at most .  By the
conditions of the lemma . Which means that 
contains the edge  and, since  also contains ,
 contains the routes to both  and . Moreover, the
 is just 2-hops long: . The argument for 
is similar. The lemma follows.  \EOP


\ \\
\noindent
\textbf{Geometric routing.}\ \ 
Consider a connected geometric graph , a neighborhood
relation \PROG{N} over it and a pair of nodes .
Source  has a message to transmit to target . The source knows
the coordinates of the target. The message may be transmitted via
intermediate nodes. Each node may potentially add data to the
transmitted message. In the sequel we ignore the payload of the
message and assume that it contains only the routing information.

A routing algorithm specifies a procedure for intermediate node
selection.  A \emph{geometric routing algorithm with guaranteed
delivery} ensures the eventual message delivery under the following
two constraints: each node  receiving the message, selects the next
node only on the basis of  and the contents of the message 
received; the message size is independent of network size.  Observe
that the nodes are not allowed to keep any information about the
transmitted message after it has been forwarded.


\section{Void Traversal Algorithm}
\label{SecTraversal}

\begin{figure} 
\center \epsfig{figure=alg.eps,width=8cm,clip=}
\caption{Traversal of void .}
\label{FigTraversal} 
\end{figure}

\noindent
\textbf{Overview.}\ \ The algorithm traverses an internal void
clockwise following the segments of the edges that comprise the border
of the void. The external void is traversed counter-clockwise.  Refer
to Figure \ref{FigTraversal} for the illustration. Given an edge, for
example , that contains the segment of the void's border, the
algorithm has to select the next edge. For this the algorithm needs to
determine the ends of the segment of  that borders the
void. One of the ends ---  is the intersection of the previous edge
 and . The other end ---  is the point of the
intersection of  and another edge  such that  is
closest to  in the direction of the traversal of the void.

Recall that in the intersection semi-closed neighborhood relation, the
union of the neighborhoods of  and  contains every edge
intersecting  as well as a route to this edge. Hence, to
accomplish its objective the algorithm has to examine the
neighborhoods of  and .

\ \\ \noindent
\textbf{Details.}\ \
The algorithm uses two types of messages: \TT{edge\_change}, and
\TT{edge\_selection}.  \TT{edge\_change} contains: previous edge,
current edge and the direction of the traversal of the current
edge. \TT{edge\_selection} message contains: previous edge, and
suggested next edge.

When a node  receives \TT{edge\_change} message from node , it
determines the intersection point  between the previous edge
 and the current edge . Notice that the intersection
point may be  itself.  Then  examines  to find the edge
whose intersection point is the closest to  in the direction of
traversal. This edge is the suggested next edge.  Note that the graph
is intersection semi-closed and  may not contain some of some
edges that intersect . For example: .

Hence,  selects  as the suggested next edge. Edge 
intersects  at point .  If  does not find any edge that
intersects , it keeps the suggested next edge field empty.
Node  sends \TT{edge\_selection} to the other node  incident to
the current edge.

When node  receives \TT{edge\_selection} from ,  compares the
suggested next edge  that  receives from  with the edges
in .  If  finds an edge  intersecting 
whose intersection point  is closer to  than , then  makes
 the suggested next edge. Otherwise, the suggested next edge
remains the same. If neither  nor  find the suggested next edge
in this manner,  considers the edges incident to itself. Node 
selects the edge nearest to  counter-clockwise.

Node  forms an \TT{edge\_change} message. This message contains
 as the previous edge and  as the current edge.  From
the information contained in \TT{edge\_selection} that  receives
from ,  is able to determine the direction of the traversal of
. The traversal direction is included in
\TT{edge\_change}. 

After composing \TT{edge\_change},  sends this message to either
 or .  Since the graph is intersection semi-closed,  may
not contain the route to either nodes. In this case  returns the
message to  and  forwards it to the appropriate node.

The above discussion is summarized in the following theorem.

\begin{theorem}\label{TraversalOK}
The Void Traversal Algorithm correctly traverses an arbitrary void in
a geometric graph with intersection semi-closed neighborhood relation.
\end{theorem}

\section{Using Void Traversal \\ to Guarantee Delivery}
\label{SecRouting}

\textbf{VOID-1 and VOID-2.}\ \ The geometric routing algorithms we
discuss are VOID-1 and VOID-2.  They are rather straightforward
extensions of algorithms FACE-1 and FACE-2 respectively. The latter
two are planar graph geometric routing algorithms presented by Bose et
al \cite{Bose01}.  FACE-1 has a better worst-case performance. However
FACE-2 is more efficient in practice. Hence, we describe VOID-2 in
detail and mention briefly how VOID-1 is designed. Our algorithms are
based on the following observation.

\begin{proposition}
Let  and  be two arbitrary points on the edges or
vertices of a geometric graph. Let  be the void such that 
lies on its border and the line segment  intersects . If
there is a point  where  intersects the border of 
then .
\end{proposition}

\begin{figure}
\center \epsfig{figure=void2.eps,width=8cm,clip=}
\caption{Example route of VOID-2.}
\label{FigVoid2Pic} 
\end{figure}



\begin{figure}[h]
\begin{tabbing}
1234\=1234\=1234\=1234\=12345\=12345\=12345\=12345\=12345\=12345\=\kill
\textbf{Algorithm}: VOID-2 \\
\> \\
\> \\
\>\textbf{repeat} \\
\>\>/* let  be the void with  on  \\
\>\> \ \  its boundary that intersects  */\\
\>\>traverse  until reaching      \\
\>\>an edge containing point   \\
\>\>intersecting  \\
\>\> \\
\>\textbf{until}  /* target reached */ \\
\end{tabbing}
\caption{Pseudocode for VOID-2. VOID-2 uses void 
         traversal to guarantee delivery.}
\label{FigVoid2Alg}
\end{figure}


Both VOID-1 and VOID-2 sequentially traverse the voids that intersect
the line . Refer to Figure \ref{FigVoid2Pic} for illustration.
While traversing a void, VOID-1 and VOID-2 look for an intersection
point with line . Observe that there may be multiple such
points. For example, void  in Figure \ref{FigVoid2Pic} has four
such points. VOID-1 and VOID-2 differ in their actions when an
intersection point is encountered. VOID-1 traverses the whole void,
selects the intersection point that is closest to , moves the
message to this point and switches to the traversal of the next
void. For example, VOID-1 would traverse  completely and switch
to . VOID-2, on the other hand, switches to the next void as soon
as the first intersection point is found. The pseudocode for VOID-2 is
given in Figure \ref{FigVoid2Alg} and an example route that VOID-2
selects is shown in Figure \ref{FigVoid2Pic}.

On the basis of Theorem \ref{TraversalOK} and the discussion above
we state the following.

\begin{theorem}\label{VoidsOK}
Given an arbitrary geometric graph, an intersection semi-closed
neighborhood relation and an arbitrary pair of nodes in this graph,
both VOID-1 and VOID-2 eventually deliver a message from the first
node to the second.
\end{theorem}


\ \\ \textbf{GVG.} As in GFG, void traversal is only needed for a
message to leave a local minimum. Otherwise greedy routing can be
used. Care must be taken to avoid a livelock when message returns to
the same local minimum.  The complete algorithm --
\emph{greedy-void-greedy} (GVG) works as follows. The message is at
first forwarded according to greedy routing. When a node discovers
that it is a local minimum, it notes the distance to target and the
routing switches to VOID-1 or VOID-2.  The routing switches back to
greedy if the distance to target is less than that of the local
minimum.  The process repeats if necessary.  Observe that, according
to Theorem \ref{VoidsOK}, the delivery of a message by either VOID-1
or VOID-2 is guaranteed. Hence, either GVG eventually switches to
greedy routing or delivers the message to the target. Hence the
following theorem.

\begin{theorem}
Given an arbitrary geometric graph, an intersection semi-closed
neighborhood relation and an arbitrary pair of nodes in this graph,
GVG eventually delivers a message from the first node to the second
one in that pair.
\end{theorem}










\section{Performance Evaluation}
\label{SecEval}

Note that the greedy phase of GFG and GVG is the same. Hence we only
compare the performance of local minimum avoidance parts: void and
face traversal. We use FACE-2 and VOID-2 for comparison as they are
more efficient in practice then FACE-1 and VOID-1.

As a primary metric we compare the length, in the number of hops, of
the routes selected by face and void traversal algorithms. Note that a
void traversal algorithm makes routing decisions on the basis of
larger neighborhood information. Therefore, we compared the individual
node memory requirements for face and void traversal algorithms as
well.


\ \\ \textbf{Graph Generation.}  We used randomly generated sets of
graphs with 50, 100 and 200 nodes in a fixed area of 2 by 2 units. The
nodes were uniformly distributed over the area. We used a rather
simple model for fading effect of radio reception.  For each set of
graphs we selected the connectivity unit  to be ,  and
 respectively. We also selected a fading factor  to be either
,  and .  The connectivity between nodes was determined as
follows. We deterministically added an edge for every pair of nodes
that were no more than  away from each other.  See Figure
\ref{figProbability}.  If the distance between two nodes was between
 and , the edge between them is added
probabilistically. The probability linearly decreased from  to
. Notice that when the fading factor  is equal to one, the
random graphs that we generated were unit-disk.


\begin{figure} 
\center 
\epsfig{figure=probability.eps,width=6cm,clip=}
\caption{Edge existence probability with respect to the distance
         between nodes.}
\label{figProbability}
\end{figure}


\begin{figure*}[ht]
\begin{minipage}[b]{.47\linewidth}
\epsfig{figure=plot2.eps,width=\linewidth}
\end{minipage}\hfill
\begin{minipage}[b]{.47\linewidth}
\epsfig{figure=plot1.eps,width=\linewidth}
\end{minipage}
\caption{Example routes selected by VOID-2 and FACE-2 between nodes 27
and 11 in a 50-node graph and its unit-disk based subgraph respectively.
The fading factor  is 2.}
\label{figRoutes}
\end{figure*}


In order to compare FACE-2 and VOID-2 we needed to compute unit-disk
subgraphs of the the generated graphs. However, frequently the
subgraphs were disconnected: out of 350 of randomly generated 50-node
graphs with  either  or , only a single graph had a connected
unit-disk subgraph. These subgraphs had to be discarded as unsuitable
for FACE-2. To make the performance comparison, we adopted a different
graph selection strategy. We generated random, connected unit-disk
graphs and added extra links according to the connectivity rules
above. Notice that this strategy favors FACE-2 as it discounts the
graphs that are not usable by FACE-2.

We generated 5 graphs for each number of nodes and fading factor.
This yielded the total of 45 graphs and their planar subgraphs.

\ \\ \textbf{Route Length Comparison.} We implemented the FACE-2 and
VOID-2 algorithms in Java and Matlab. We generated 10 random pairs of
nodes for each set of graphs and used VOID-2 and FACE-2 to compute the
routes between the nodes in each pair. VOID-2 used the original graph
and FACE-2 --- its unit-disk based planar subgraph. The
example routes are shown in Figure \ref{figRoutes}.

We used paired comparison to estimate the improvement from FACE-2 to
VOID-2 in the length of the selected routes as follows. For each pair
of source and destination nodes we calculated the difference in the
hop-count between routes and normalized it to the hop count of the
route selected by FACE-2. That is the comparison is based on the
formula: . Figure
\ref{figCompare} summarizes the results. Observe that even on
unit-disk graphs (), VOID-2 generates routes that are 35-45\%
shorter than the routes selected by FACE-2. The relative performance
of VOID-2 further improves as the fading factor and density of the
graphs increases.


\ \\ \textbf{Memory Usage Comparison.} In both VOID-2 and FACE-2, each
node  stores the information about its neighborhood nodes .
The size of the neighborhood differs in the two algorithms. In
evaluating the memory usage, we estimated the average size of the
neighborhood. For the FACE-2, the average size of the neighborhood is
the average degree  of the planar subgraph. In our experiments, the
average size of neighborhood in VOID-2, was found to be 
where  is the fading factor.


\section{Conclusion}
\label{SecConclusion}

The elegance and efficiency of geometric routing algorithms is
remarkable. Recent advent of large-scale wireless sensor networks
increased the relevance of such algorithms. However, the algorithms'
demand for unit-disk based graph planarity hampered the attractiveness
of geometric routing.  With this paper, we lift this restriction and
help the adoption of geometric routing algorithms as the routing
solution of choice in sensor networks.


\section*{Acknowledgments} We would like to thank Ivan Stojmenovic
of the University of Ottawa and M. Kazim Khan of Kent State University
for their helpful comments.

\suppressfloats[t]

\begin{figure}[t]
\center \epsfig{figure=plot3.eps,width=8cm,clip=}
\caption{Route length improvement from FACE-2 to VOID-2.}
\label{figCompare}
\end{figure}



\bibliographystyle{IEEEtran}
\bibliography{geometric,routing}




\end{document}
