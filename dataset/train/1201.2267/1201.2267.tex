\documentclass{paper}

\usepackage{subfigure}

\usepackage{graphicx}
\usepackage{amssymb}
\usepackage{amsmath}
\usepackage{amsthm}
\usepackage{cite}
\usepackage{fullpage}

\newcommand\R{{\mathbb R}}
\newcommand\N{{\mathbb N}}
\newcommand\eps{{\varepsilon}}
\newcommand\eqdef{:=}
\newcommand\matousek{{Matou{\v{s}}ek}}
\DeclareMathOperator{\lev}{lev}

\newtheorem {theorem} {Theorem}[section]
\newtheorem {prop}[theorem] {Proposition}
\newtheorem {lemma}[theorem] {Lemma}
\newtheorem {observation}[theorem] {Observation}
\newtheorem {cor}[theorem] {Corollary}
\newtheorem {claim}[theorem] {Claim}
\newtheorem {invariant}[theorem] {Invariant}

\title{A Lower Bound for Shallow Partitions}

\author{Wolfgang Mulzer
         \and
        Daniel Werner\thanks{DW was funded by 
	 Deutsche Forschungsgemeinschaft within the Research Training
         Group (Graduiertenkolleg) ``Methods for Discrete Structures''.}
	}
\institution{\texttt{\{mulzer,werner\}@inf.fu-berlin.de}\\\\
        Institut f\"ur Informatik,\\
        Freie Universit\"at Berlin,\\ 14195 Berlin, Germany}

\begin{document}
\maketitle

\begin{abstract}
Let  be a planar -point set. A \emph{-partition} of
 is a subdivision of  into  parts of
roughly equal size and a sequence of triangles such that
each part is contained in a triangle. A line is \emph{-shallow} 
if it has at most  points of  below it.
The crossing number of a -partition is the maximum number
of triangles in the partition that any -shallow line intersects.
We give a lower bound of  
for this crossing number, answering a 20-year old question
of \matousek.
\end{abstract}

\section{Introduction}

\emph{Range searching} is a fundamental problem in computational 
geometry that has long driven innovation in the field~\cite{AgarwalEr99}:
given a set of  points in  dimensions, find a data
structure such that all points inside a given query range
can be found efficiently. Depending on the precise
nature of the query range and on the dimension,
many different versions of the problem can be studied. Consequently, 
a wide variety of techniques have been developed to address them. Among these
tools we can find such classics as range trees and
d-trees~\cite[Chapter~5]{deBergChvKrOv08}, 
-nets and cuttings~\cite{Chazelle00}, spanning trees with small 
crossing number~\cite{Welzl92}, 
geometric partitions~\cite{Matousek92b}, and many more. For several problems,
almost matching lower bounds are known (in certain models of 
computation)~\cite{Chazelle00}.

\emph{Geometric partitions} provide the most effective means for solving the
\emph{simplex range searching} problem, where the query range is
given by a -dimensional simplex~\cite{Chan10,Matousek92b}.
They provide a way to subdivide a point set into 
parts of roughly equal size, such that (i) each part is contained in a simplex;
and (ii) any given hyperplane intersects only few of these simplices. 
This makes it possible to construct a tree-like data structure 
in which each node corresponds to a simplex in an appropriate geometric partition.
With a careful implementation, one can achieve query time
 with linear space~\cite{Chan10} (here  is
the output size, i.e., the number of reported points).

If the query simplex degenerates to a half-space, we can
do better~\cite{Matousek92}. For this, we need a more specialized version of
geometric partitions, called \emph{shallow partitions}. Again, these
partitions provide a way for subdividing a -dimensional point set into
parts of roughly equal size, such that each part is contained in a simplex and
such that a hyperplane intersects only few of these simplices. 
This time, however, we restrict ourselves to \emph{shallow} hyperplanes. 
Such hyperplanes have only few points to one side. 
Thus, we only have the guarantee
that any shallow hyperplane will intersect few simplices of the partition
(see below for details).
This makes it possible to decrease the number of simplices that are intersected 
and to achieve better bounds for halfspace range searching. Namely, one
can obtain for  a linear-space data structure that answers a query in time
, where  is the output 
size~\cite{Chan10} (for , one can achieve query time 
and linear space~\cite{AfshaniCh09}).

Shallow partitions (as well as their cousins---shallow cuttings) have proved
invaluable tools in computational geometry and have found numerous further
applications. Nonetheless, there still remain some open questions. 
As mentioned above, we would like every shallow hyperplane 
to intersect as few simplices of the shallow partition as possible.
But what exactly is possible? For dimension , the original bound
by \matousek~\cite{Matousek92b} is known to be asymptotically tight. For
lower dimensions, however, \matousek\ asked whether his result could be 
improved.
It took almost 20 years until Afshani and Chan~\cite{AfshaniCh09}
provided the first lower bound in three dimensions, almost
matching the upper bound. For the plane, however, so far no nontrivial
lower bounds appear in the literature. 

Here, we will give a construction that
provides such a lower bound for shallow partitions in two dimensions.
Our result almost matches the upper bound and also gives an alternative
proof for the bound of Afshani and Chan~\cite{AfshaniCh09}.
A similar construction has been discovered independently by
Afshani~\cite{Afshani10}.


\section{Shallow partitions}

We begin by providing the details of \matousek's shallow partition theorem
in two dimensions.
Let  be a planar -point set in general position.
Let  be a parameter. 
A \emph{-partition}  for  consists of two parts:
(i) a sequence , , ,  of
pairwise disjoint subsets of  such that 
and  for ; and
(ii) a sequence , , , 
of triangles such that  for all .

Now let    be a line that does not contain any point in , and let
 denote the open halfplane below . 
We say that  is \emph{-shallow} if 
. Given a -partition 
of , the \emph{crossing number} of  is the 
maximum number of triangles in  that are intersected
by any -shallow line. For any given , the goal is to find 
a -partition of  whose crossing number is as small as possible.
\matousek~\cite[Theorem~3.1]{Matousek92} 
proved the following theorem.
\begin{theorem} 
\label{thm:shallowpart}
Let  be a planar -point set in general position and let 
. Then there exists a -partition of  with crossing number
. \hfill
\end{theorem}

\matousek's original proof uses cuttings and a variant of the 
iterative reweighting technique (also known as the multiplicative weights 
update method~\cite{AroraHaKa10}), and it readily generalizes to higher 
dimensions. 
More recently, Har-Peled and Sharir~\cite[Lemma~3.3]{HarPeledSh11} give
an approach for proving Theorem~\ref{thm:shallowpart} with elementary 
means, but it is not clear whether their technique can be applied to 
higher dimensions. As mentioned in the introduction, 
\matousek~\cite{Matousek92} 
asked whether
the crossing number in Theorem~\ref{thm:shallowpart} can be
improved to . He conjectured that the answer is no. 
Afshani and Chan~\cite{AfshaniCh09} proved that for any  there are
arbitrarily large point sets
in  such that the crossing number
of any -partition for them is 
. 
However, their construction
does not apply for two dimensions. 
Hence, we will describe here a different---and arguably simpler---construction
that yields the same lower bound for the plane.
Independently, Afshani~\cite{Afshani10} used very similar ideas to
obtain the same lower bound.

\section{The Lower Bound}

Let  be the minimum crossing number that
a -partition can achieve for any planar -point set
in general position. 
For the lower bound, we shall consider the dual setting. 
We use the standard duality 
transform along the unit paraboloid that maps the point 
 to the line  and vice
versa~\cite{Mulmuley94}.

A point set  dualizes to a set  of planar lines. We now
define the \emph{-level} of , ~\cite{SharirAg95}. 
It is the closure of the set 
of all points that lie on a line of  and 
that have exactly  lines of  beneath them. 
We observe that  is an -monotone polygonal curve 
whose edges and vertices come from the arrangement
of . 
Let  be the upper convex hull of .
For each vertex  of , we let  
denote the set of lines
beneath it. We call  the \emph{conflict set} of . 
We have ,\footnote{Note 
that  may also contain vertices with
only  lines of  beneath them, but these vertices
cannot appear on , since they correspond to
a concave bend in .} 
hence  is dual to
a -shallow line  in the primal plane.


Now we can interpret shallow partitions in the dual plane:
\begin{prop}\label{prop:dualpartition}
Let  be an -monotone downward convex chain, and let
 be a set of  lines such that for each
vertex  of  the conflict set  has cardinality .
Then there exists a coloring of  
such that (i) each color class 
has size at most ; and (ii) each conflict set  contains
at most  different colors.
\end{prop}
\begin{proof}
Consider the primal plane, where  corresponds to a point set .
By assumption, there exists a -partition  of  
with crossing number . Each vertex  of  
corresponds 
to a -shallow line , and at most one triangle of  
can be wholly contained in . Thus, the claim follows from the 
properties of the duality transform.
\end{proof}

We are now ready to describe the construction. 
Let  be a power of 2 and 
let  be an -monotone convex chain with  vertices.
We denote these vertices by
, , , from left to right. 
Now, for , let
 be a set of  lines such that
the first line in  lies exactly below the vertices  to ,
the second line lies below  to ,
the third line lies below  to ,
etc. We set . See Fig. \ref{fig:Ls}.
\begin{center}
\begin{figure}[htb] 
  \centering\subfigure{ 
    \includegraphics[scale=.7]{figures/L1}

   } 
  \centering\subfigure{ 
    \includegraphics[scale=.7]{figures/L2}

   } 
     \centering\subfigure{ 
    \includegraphics[scale=.7]{figures/Llogm}
   } 
     \caption{Sets of lines .} 
       
     \label{fig:Ls} 
\end{figure}
\end{center}













Assume for now that  is a multiple of ,
and let  consist of  copies of . We perturb the
lines in  such that they are all distinct while
their relationship with the vertices of  remains unchanged.
It follows that  has exactly  lines, with 
exactly  lines in each conflict set  (recall that by definition
).

By Proposition~\ref{prop:dualpartition}, there is a coloring of 
 such that each color class 
has size at most  and such that each conflict set contains at most
 colors. The structure of  lets us interpret this
coloring as follows:
let  be a complete binary tree with 
 nodes and height . We label the leaves of  with 
the vertices , , , from left to right. 
Thus, every node  of  corresponds to an interval of
consecutive vertices of , namely the leaves
of the subtree rooted in .
By assigning to  the lines that lie exactly below the
vertices in this interval,
we  obtain a partition of  into sets of
size . This leads to an 
interpretation of shallow partitions as multi-colorings
of trees, see Fig.~\ref{fig:tree}.
\begin{figure}[htb] 
  \begin{center}
   \includegraphics[scale=1]{figures/tree}
  \end{center}
  \caption{A tree with height . The leaves correspond to
  the vertices , and the level of height  corresponds to the lines
  in . A line  is stored in the node whose subtree corresponds to the
  vertices that have  below them.}
  \label{fig:tree}
\end{figure}



\begin{prop}\label{prop:tree}
Let  be a complete binary tree with height  and  nodes,
and let  be a multiple of .
Then there exists a multi-coloring of the nodes of  with the
following properties: 
(i) every node is associated with a multiset of  colors;
(ii) each color class has at most  elements; (iii) along each
root-leaf path there are at most  distinct colors,
where .
\end{prop}

\begin{proof}
Properties (i) and (ii) follow immediately from 
Proposition~\ref{prop:dualpartition} and the construction. 
For Property~(iii), observe that the lines encountered
along a root-leaf path are exactly the lines
below the vertex of  corresponding to the leaf.
\end{proof}

We can now prove the desired lower bound.

\begin{lemma}\label{lem:lowerb}
Let  be a complete binary tree with height  and  nodes, 
and let  be a multiple of .
Consider a multi-coloring of  such that
(i) every node is associated with a multiset of 
colors; and (ii) each color class has at most  elements.
Then there exists a root leaf-path with  
distinct colors.
\end{lemma}

\begin{proof}
We subdivide the nodes of  into \emph{slices}. 
The first slice consists of the first  
levels of , 
the second slice consists of the following  
levels, the third slice has the next  levels,
and so on. In general, the th slice consists of 
 consecutive levels of . 

We claim that there
exists a root-leaf path that has at least one distinct color 
for each slice that it crosses, except for the last one. 
To see this, we first consider a complete subtree  of  that is
has its root in the first level of a slice  and its
leaves in the last level of the same slice. As a complete
binary tree with  levels,  has at least  
 nodes. Therefore, our 
multi-coloring needs to assign at least  colors
in . Since each color class has size at most
, this requires at least
 \emph{distinct} colors.

We now construct the required root-leaf path slice by slice.
Throughout, we maintain the invariant that after
 slices have been considered, the path contains at least
 distinct colors. This is certainly true at the root.
Now suppose that we have constructed a partial path  that  
ends at a node  in the last level of the th slice. If 
contains at least  distinct colors, we arbitrarily extend it
to a path  that ends at the bottom of the th slice.
Otherwise, we pick an arbitrary child  of . 
As noted above, the complete subtree that is rooted at  and restricted
to the th slice contains at least  distinct colors. Thus,
we can extend  through  to a path  that goes to
the bottom of the th slice and that meets at least  distinct colors.
The claim follows. 

It remains to calculate a lower bound for the number of slices .
By construction, we must have

Now,

since clearly .
Hence, 

as desired.
\end{proof}

We now indicate how to drop the assumption that  is a multiple
of . Indeed, suppose that this is not the case,
but .  We first perform the
above construction with  
instead of . Note that since , we have . 
Then we add  suitably perturbed copies of  
(the set containing a line in conflict with all vertices of ).
Let  be the resulting set of lines. By Proposition~\ref{prop:dualpartition},
there exists a coloring of  such that each color class has at most
 elements and such that each conflict set has at most
 distinct colors. The tree  corresponding to 
has the same structure as before, but now each non-leaf node except
the leaf is associated with  colors, while the leaves
have  additional colors. 
This suffices for the argument of Lemma~\ref{lem:lowerb} to go through.

\begin{theorem}
There is a constant  such that the following holds. For every 
and , there 
exists a planar -point set 
such that the crossing number for any -partition of  is at least
. Thus,

\end{theorem}

\begin{proof}
Let  be maximum with . 
Set  and . 

From Propositions~\ref{prop:dualpartition} and \ref{prop:tree} 
and Lemma~\ref{lem:lowerb}, it follows that by taking the dual
we obtain a set  of 
 points such that
any -partition of  has crossing number at least
, for some constant .

First note that  and .
Hence,  and . Thus, we can conclude that

and

Thus, by adding at most  points that are contained in no -shallow
halfplane, we can obtain from  a point set  with  points
and crossing number at least .
Finally, observe that 

so  also has crossing number at least
, for some . 
The result follows.
\end{proof}

Note that our construction also implies a similar lower bound in 
by embedding the plane into three-dimensional space and perturbing
the points slightly. This provides an alternative proof of the result
by Afshani and Chan~\cite{AfshaniCh09}.

\section{Conclusion and Open Problems}

We have given a simple construction that give a lower bound
of  for
the crossing number of
any shallow partition of a planar point set. \matousek's 
result gives an upper bound of . Thus, there still
remains a factor of  to be settled. Can we show that
\matousek's analysis is tight? Or, perhaps more interestingly, can
we construct shallow partitions with crossing number
?

\section*{Acknowledgments}

{\small
We would like to thank Nabil Mustafa for 
drawing our attention to shallow partitions
and for enlightening conversations.
We would also like to thank Timothy Chan
for answering our questions about shallow
partitions.}


\bibliographystyle{abbrv}
\bibliography{shallow}

\end{document}
