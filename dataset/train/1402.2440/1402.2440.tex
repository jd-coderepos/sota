The following validation experiments do not only cover particular aspects as the solid-liquid phase transition or only concentrate 
on qualitative accordance but consider also the complete EBM process in a quantitative way. 
\subsection{Definitions and experimental setup}
We use the following experimental setup for the validation experiments. 
\begin{wrapfigure}{r}{5cm}
\centering
\vspace{-13pt}
\includegraphics[width=0.4\textwidth]{cuboid.eps}
\caption{Experimental setup.}
\label{fig:exp_setup}
\end{wrapfigure}
A cuboid of size (15x15x10)\,mm$^{3}$ is generated by hatching the Ti-Al6-V4 powder particles layer by layer (compare the red arrows in Fig.~\ref{fig:exp_setup}).
The hatching differs by line energy and scan velocity of the electron beam. The line energy is defined by,
\begin{equation}
 E_{L} = \frac{U\cdot I}{v_{\text{scan}}} = \frac{P_{\text{beam}}}{v_{\text{scan}}},
\label{eq:line_energy}
\end{equation}
where $U$ denotes the acceleration voltage in [V], $I$ the current in [A] and $v_{\text{scan}}$ the scan velocity in [$\frac{\text{m}}{\text{s}}$] of the electron beam. 
The parameter set $(E_{L},v_{\text{scan}})$ defines the electron beam. 
The quality of EBM products is classified into three categories, namely porous, good and swelling. Good samples have a smooth surface and 
a relative density higher than 99.5\%. If the temperature during the process is too high swelling can occur and the dimensional accuracy cannot be guaranteed. 
\begin{figure}[htbp!]
\centering
\includegraphics{range.eps}
\caption{Categories of samples.}
\label{fig:categories}
\end{figure}
On the other side, if the line energy is too low and the relative density is smaller than
99.5\%, then the sample is porous (cf. Fig.~\ref{fig:categories}). 
Next, a experimental process window is shown in Fig.~\ref{fig:pw_real} which is compared with numerical simulation results. 

\subsection{Experimental data}
Fig.~\ref{fig:pw_real} shows the quality of samples with different line energies in [$\frac{\text{kJ}}{\text{m}}$] and scan velocities in [$\frac{\text{m}}{\text{s}}$] of the electron beam. 
\begin{figure}[htbp!]
 \centering
 \includegraphics[width=0.7\textwidth]{pw_exp_12kW.eps}
 \caption{Process window of experimental data. Built temperature 1000\,K, line offset 100\,$\mu$m, layer thickness 50\,$\mu$m, focused electron beam.}
 \label{fig:pw_real}
\end{figure}
The red circles stand for a porous parameter set, the blue squares for an optimal, good parameter set and the yellow rhombus for a sample where 
swelling occurs because of too high temperatures at the surface. It can be observed that the higher the scan velocity the lower the embedded line energy has to be to get an optimal sample
result. 

\subsection{Numerical results}
We validate our 3D numerical EBM model against these experimental data of Fig.~\ref{fig:pw_real}. 
Because of the high computational cost of the 3D simulations we only model the hatching of one powder 
layer instead of the multiple layers.
\begin{wrapfigure}{r}[0cm]{6.4cm}
\centering 
 \includegraphics[width=0.5\textwidth]{sketchScenario.eps}
\caption{Sketch of simulation scenario.}\label{fig:SketchSimScenario}
 \vspace{-10pt}
\end{wrapfigure} 
One simplified scenario for the exemplary parameter set of $(3.2\,\frac{\text{m}}{\text{s}},0.2\,\frac{\text{J}}{\text{m}})$ does already require 8 compute 
nodes of \textsc{LiMa}\footnote{http://www.rrze.uni-erlangen.de/dienste/arbeiten-rechnen/hpc/systeme/lima-cluster.shtml} for 4 wall clock hours.  
\textsc{LiMa} is the compute cluster where all 3D simulations are done. However, this simplification is in good agreement with the situation
expected for the full setup as it will be shown in this Section. 
We also minimize the simulated powder particle layer, i.e., we focus only a rectangular powder layer and have seven hatching 
lines (cf. Fig.~\ref{fig:SketchSimScenario}). Thus, a beam offset per layer is defined where the electron beam is on but outside 
the simulated powder particle bed. 

In Fig.~\ref{fig:pw_num} the quality of the numerical experiments with different line energy and scan velocity is shown. 
In order to determine if the numerical sample is porous the relative density of the sample is measured and if it is less than 99.5\% the 
sample is porous. The definition of numerical swelling is more difficult. Our model does not include evaporation and thus no evaporation pressure. 
Hence, numerical swelling effects cannot occur. We have to find another way how we can determine regions numerically where swelling exists experimentally. 
Therefore, we use the numerical computed temperature which is higher than in real EB melting processes because the cooling effect during the process of evaporation is missing.  
Furthermore, numerical artefacts cause outliers in the temperature field. 
In order to overcome these numerical influences we use an averaging process for the temperature values. If these averaged values are still higher than 7500\,K 
we assume that swellings occur in the experimental setup.

\begin{figure}[htpb!]
\centering
 \includegraphics[width=0.7\textwidth]{pw_numerical.eps}
\caption{Process window of numerical data. $\Delta x=5\cdot10^{-6}$\,m, $\Delta t = 1.75\cdot10^{-7}$\,s, simulated powder domain (1.44x0.64x0.24)$\cdot10^{-3}\,\text{m}^{3}$, beam offset =13.56$\cdot10^{-3}$\,m.}
\label{fig:pw_num}
\end{figure}
In comparison with the experimental data in Fig.~\ref{fig:pw_real} it is observed that all porous samples 
are consistent in the simulation results (cf. Fig.~\ref{fig:pw_num}). Furthermore, the numerical and experimental results for scan velocities 
of 3.2\,$\frac{\text{m}}{\text{s}}$, 4.0\,$\frac{\text{m}}{\text{s}}$ and 6.4\,$\frac{\text{m}}{\text{s}}$ agree with each other.
Small differences between numerical and experimental data are only found for 0.8\,$\frac{\text{m}}{\text{s}}$, 1.6\,$\frac{\text{m}}{\text{s}}$, 2.4\,$\frac{\text{m}}{\text{s}}$, 
4.8\,$\frac{\text{m}}{\text{s}}$ and 5.6\,$\frac{\text{m}}{\text{s}}$ for higher line energies; for $v_{\text{scan}}=$~0.8\,$\frac{\text{m}}{\text{s}}$ in experimental data swelling
already occur for 0.4 and 0.5\,$\frac{\text{kJ}}{\text{m}}$ but in numerical data no swelling is measured. The same effect is seen for scan velocities 1.6 and 2.4\,$\frac{\text{m}}{\text{s}}$ 
where swelling occurs for smaller line energies experimentally than numerically. For higher scan velocities of 4.8 and 5.6\,$\frac{\text{m}}{\text{s}}$ 
the reverse effect is observed, i.e., more swelling occur numerically than experimentally. For $v_{\text{scan}}=4.8\,\frac{\text{m}}{\text{s}}$ swelling arises numerically already for a line 
energy of 0.25\,$\frac{\text{kJ}}{\text{m}}$ while the experimental setup is still good. The same is seen for $v_{\text{scan}}=5.6\,\frac{\text{m}}{\text{s}}$; line energies higher than
0.25\,$\frac{\text{kJ}}{\text{m}}$ lead to swelling in numerical simulations, but in experimental data only energies equal 0.3\,$\frac{\text{kJ}}{\text{m}}$ arise swelling. 

Both effects can be explained by the focus of the electron beam gun. The focus is more precise for smaller scan velocities imposed by a smaller electron beam power $P_{\text{beam}}$, e.g, 
$v_{\text{scan}}=1.6\,\frac{\text{m}}{\text{s}}$ or $v_{\text{scan}}=2.4\,\frac{\text{m}}{\text{s}}$ lead to a precise, small focus and thus, energy is brought onto a smaller area. 
For these velocities swelling occurs for smaller line energies experimentally than numerically because the EB gun focus is constant for different beam power and scan velocities, respectively. 
For higher scan velocities (larger $P_{\text{beam}}$) 
like 4.8\,$\frac{\text{m}}{\text{s}}$ the focus of the gun spreads and, subsequently, the same amount of energy is brought into a larger area of powder particles and the maximum 
temperature is smaller. As a consequence in experimental data less swelling occurs for higher scan velocities (larger $P_{\text{beam}}$) than in numerical data. 


\begin{figure}[htpb]
\vspace{-15pt}
\centering
 \subfigure[t =  0.20$\cdot10^{-3}$\,s \label{fig:hatching_1}]{\includegraphics[width = 0.3\textwidth]{hatch1movieiii02.eps}}  \hspace{2ex}
 \subfigure[t =  4.27$\cdot10^{-3}$\,s \label{fig:hatching_2}]{\includegraphics[width = 0.3\textwidth]{hatch1movieiii40.eps}}  \hspace{2ex}
 \subfigure[t = 12.91$\cdot10^{-3}$\,s \label{fig:hatching_3}]{\includegraphics[width = 0.3\textwidth]{hatch1movieiii133.eps}} \hspace{2ex}
 \vspace{-13pt}
\caption{Hatching of one layer, (0.15$\,\frac{\text{kJ}}{m}$, 6.4\,$\frac{\text{m}}{\text{s}}$).}
\label{fig:Hatching}
\vspace{-15pt}
\end{figure}
Fig.~\ref{fig:Hatching} shows hatching of one line with parameter set (0.15$\frac{\text{kJ}}{\text{m}}$, 6.4\,$\frac{\text{m}}{\text{s}}$) for three 
different time steps. The inverse Gaussian distributed powder particles~\cite{Ammer2013} are demonstrated the  and the free surface is visualized by an isosurface. 
In Fig.~\ref{fig:hatching_1} the electron beam melted one line and powder particles in this first line are not yet completely melted. 
The next Fig.~\ref{fig:hatching_2} shows the second melted line by a coming-back 
electron beam. Here, the particles of the first line are now melted completely. In the third Fig.~\ref{fig:hatching_3} the electron beam has melted 
seven hatching lines and the whole powder layer is melted. 
%