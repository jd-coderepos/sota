\documentclass[11pt]{llncs}
\pdfoutput=1

\usepackage{graphicx}
\usepackage{epsfig}
\usepackage{xcolor}
\usepackage{amssymb}
\usepackage{xspace}
\usepackage{multirow}
\usepackage[]{geometry}
\usepackage{subfigure}
\usepackage{wrapfig}

\newtheorem{Lemma}{Lemma}
\newtheorem{Theorem}{Theorem}
\newtheorem{Definition}{Definition}
\newtheorem{Corollary}{Corollary}
\newtheorem{Remark}{Remark}
\newtheorem{Observation}{Observation}

\def\comment#1{}\def\ournote{\comment}\def\withcomments{\newcounter{mycommentcounter}\def\comment##1{\refstepcounter{mycommentcounter}\ifhmode \unskip {\dimen1=\baselineskip \divide\dimen1 by 2 \raise\dimen1\llap{\tiny
	{-\themycommentcounter-}}}\fi \marginpar[{\renewcommand{\baselinestretch}{0.8}\hspace*{0em}\begin{minipage}{9em}\footnotesize [\themycommentcounter]:\raggedright ##1\end{minipage}}]{\renewcommand{\baselinestretch}{0.8}\begin{minipage}{9em}\footnotesize [\themycommentcounter]: \raggedright ##1\end{minipage}}}}

\newcommand{\chihhung}[1]{\comment{\textcolor{red}{\textbf{CL:}} #1}}
\newcommand{\andreas}[1]{\comment{\textcolor{green}{\textbf{AG:}} #1}}



\newcommand{\R}{\ensuremath{\mathbb{R}}}
\renewcommand{\P}{\ensuremath{\mathcal{P}}}
\newcommand{\SPM}{\ensuremath{\mathcal{SPM}}}
\newcommand{\FV}{\ensuremath{\mathcal{FV}}}
\newcommand{\kth}{\ensuremath{k^{\mathrm{th}}}\xspace}
\newcommand{\kthorder}{\kth-order\xspace}
\newcommand{\rephrase}[3]{\noindent\textbf{#1~#2.}~\emph{#3}}

\setlength{\floatsep}{8pt}

\setlength{\textfloatsep}{16pt}

\setlength{\intextsep}{8pt}



\newcommand{\deleted}[1]{}

\begin{document}

\title{Higher Order City Voronoi Diagrams}
\author{Andreas Gemsa \and D. T. Lee \and Chih-Hung Liu \and Dorothea Wagner}
\institute{Karlsruhe Institute of Technology, Germany \and Academia Sinica, Taiwan \and National Chung Hsing University, Taiwan}

\date{}
\maketitle

\begin{abstract}


We investigate higher-order Voronoi diagrams
in the \emph{city metric}.
This metric is induced by quickest paths in the  metric in the presence of an accelerating transportation network of axis-parallel line segments.
For the structural complexity of \kthorder city Voronoi diagrams of  point sites, we show an upper bound of  and a lower bound of , where  is the complexity of the transportation network.
This is quite different from the bound  in the Euclidean metric \cite{Lee-82}.
For the special case where  the complexity in the Euclidean metric is ,
while that in the city metric is .
Furthermore, we develop an -time iterative algorithm to compute the -order city Voronoi diagram
and an -time divide-and-conquer algorithm to compute the farthest-site city Voronoi diagram.

\deleted{
We investigate higher-order Voronoi diagrams
in the presence of a transportation network on the  plane, which is commonly referred to as \emph{city metric}.
More specifically, we derive structural complexities and develop algorithms.
For the structural complexity of \kthorder city Voronoi diagrams
we show a lower bound of  and an upper bound of , where  is the complexity of the transportation network.
This is quite different from the bound  in the Euclidean metric \cite{Lee-82}.
In particular, for , the complexity in the Euclidean metric is ,
while that in the city metric is . \andreas{I'm not sure this belongs in an abstract}
Furthermore, we develop an -time iterative algorithm for the -order city Voronoi diagram
and an -time divide-and-conquer algorithm for the farthest-site city Voronoi diagram.
Our results further indicates that the impact of the transportation network increases with the value of  rather than being a constant,
and the underlying distance metric will affect the structural complexity of higher-order Voronoi diagrams a lot.
}




\deleted{
We address  nearest neighbor problems in the presence of transportation networks on the  plane, which is commonly referred to as city metric.
More specifically, we investigate the higher-order city Voronoi diagrams
to derive the complexities and develop algorithms.
For the structural complexity of \kthorder city Voronoi diagrams
we show a lower bound of  and an upper bound of , where  is the complexity of the transportation network.
This is quite different from the bound  in the Euclidean metric \cite{Lee-82}.
In particular, for , the complexity in the Euclidean metric is ,
while that in the city metric is .
Furthermore, we develop an iterative algorithm to compute the -order city Voronoi diagram
in  time
and a divide-and-conquer algorithm to compute the farthest-site city Voronoi diagram in  time.
Our complexity results further indicates that the impact of the transportation network is not a constant
but increases with the value of , and the underlying distance metric will affect the structural complexity of higher-order Voronoi diagram a lot. }

\end{abstract}



\section{Introduction}

In many modern cities, e.g., Manhattan, the layout of the road network resembles a grid.
Most roads are either horizontal or vertical, and thus pedestrians can move only either horizontally or vertically.
Large, modern cities also have a public transportation network (e.g., bus and rail systems)
to ensure easy and fast travel between two places.
Traveling in such cities can be modeled well by the \emph{city metric}.
This metric is induced by quickest paths in the  metric in the presence of an accelerating transportation network.
We assume that the traveling speed on the transportation network is a given parameter .
The speed while traveling off the network is~1.
Further, we assume that the transportation network can be accessed at any point.
Then
the distance between two points is the minimum time required to travel between them.

For a given set  of  point sites (i.e., a set of  coordinates) and a transportation network in the plane,
\emph{the -order city Voronoi diagram} 
partitions the plane into \emph{Voronoi regions}
such that all points in a Voronoi region share the same  nearest sites
with respect to the city metric.

The \kthorder city Voronoi diagram can be used to resolve the following situation:
a pedestrian wants to know the  nearest facilities (e.g.,  stores, or  hospitals)
such that he can make a well-informed decision as to which facility to go to.
For this kind of scenario,
the \kthorder city Voronoi diagram
provides a way to determine the  nearest facilities, by modeling the facilities as point sites.





The nearest-site (first-order) city Voronoi diagram
has already been well-studied \cite{AAP-04,BC-05,BKC-09,GSW-08}.
Its structural complexity (the size)
has been proved to be  \cite{AAP-04},
where  is the complexity of the transportation network.
Such a Voronoi diagram can be constructed in  time \cite{BKC-09}.
However, to the best of our knowledge
there is no existing work regarding -order
or farthest-site (i.e., -order) city Voronoi diagrams.


Contrary to \kthorder city Voronoi diagrams,
\kthorder Euclidean Voronoi diagrams
have been studied extensively for over thirty years.
Their structural complexity has been shown to be  \cite{Lee-82}.
They can be computed by an iterative construction method in  time \cite{Lee-82}
or by a different approach based on geometric duality and arrangements in  time~\cite{CE-87}.
Additionally, there are several randomized algorithms \cite{ABMS-98,Mulmuley-91}
and on-line algorithms \cite{AS-92,BDT-93}.



One of the most significant differences between the Euclidean metric and the city metric that influences the computation and complexity of Voronoi diagrams is the complexity of a bisector between two points.
In the Euclidean or the  metric such a bisector has constant complexity, while in the city metric the complexity may be ~\cite{AAP-04} and can even be a closed curve.
Since the properties of a bisector between two points significantly affect the properties of Voronoi diagrams,
a \kthorder city Voronoi diagram can be very different from a Euclidean one.
First, this property makes it non-trivial to apply existing approaches for constructing Euclidean Voronoi diagrams to the city Voronoi diagrams.
Secondly, this property also indicates that the complexity of \kthorder Voronoi diagrams may depend significantly on the complexity of the transportation network.

\begin{table}[tb]
\caption{\small{Comparison between the Euclidean and the city metric. Our results are marked by .}}\label{tb-comparison}
\centering
\hspace*{-1.2 cm}\includegraphics[clip, width=15cm]{figure/tb-contribution}
\end{table}


In this paper, we derive bounds for the structural complexity of the \kthorder Voronoi diagram
and develop algorithms for computing the \kthorder city Voronoi diagram.
The remainder of this paper is organized as follows.
In Section~\ref{sec-prelim},
we introduce two important concepts, wavefront propagation \cite{AAP-04} and shortest path maps \cite{BKC-09},
which are essential for the proofs in the subsequent sections.
In Section~\ref{sec-complexity},
we adopt the wavefront propagation to introduce a novel interpretation of the iterative construction method of Lee \cite{Lee-82},
and use this interpretation to derive an upper bound of  for the structural complexity of \kthorder city Voronoi diagrams,
where~ is the complexity of the transportation network.
Then, we construct a worst-case example to obtain a lower bound of .
Finally,
we extend the insights of Section~\ref{sec-complexity} to develop an
iterative algorithm to compute \kthorder city Voronoi diagrams in  time
(see Section~\ref{sec-algorithms}).
Moreover, we give a divide-and-conquer approach to compute farthest-site city Voronoi diagrams in  time.
We conclude the paper in Section~\ref{sec-conclusion}.

For an overview of our contribution and a comparison between Euclidean and city metric see Table~\ref{tb-comparison}.





\deleted{
\begin{table}[tb]
\caption{\small{Comparison between the Euclidean and the City metrics. Our results are marked by an asterisk.}}\label{tb-comparison}
\centering

\begin{tabular}{|c||c|c||c|c|}
\hline
 \multirow{2}{*}{}& \multicolumn{2}{c||}{Euclidean} & \multicolumn{2}{c|}{City}\\
 \cline{2-5}
 & Structural Complexity & Time Complexity & Structural Complexity & Time Complexity\\
 \hline
 \hline
 nearest-site &  &  &  &  \\
 \hline
 farthest-site &  &  & * & *\\
 \hline
 \multirow{2}{*}{\kthorder} & \multirow{2}{*}{} &  & upper bound: * & \multirow{2}{*}{*}\\
  & &   & lower bound:  * & \\
 \hline

\end{tabular}

\end{table}
}




\section{Preliminaries}\label{sec-prelim}
In this section we introduce the notation used throughout this paper
for \kthorder city Voronoi diagrams.
Then, we introduce two well-established concepts in the context of Voronoi diagrams,
which are important for the proofs in the subsequent sections.

A transportation network is a planar straight-line graph  with  \emph{isothetic} edges only, i.e., edges that are either horizontal or vertical,
and all transportation edge have identical speed .
We define , and since the degree of a vertex in  is at most four,
 is .
We denote the distance of two points in the  metric by  and in the city metric by .
Similarly, we denote the bisector between two points by  and  for the  and city metric, respectively.
Additionally, for the city metric we define the distance between a point  and a set of points  to be
.
This allows us to define the bisector  between two sets of points  and  .




By  we denote a \emph{Voronoi region} of  associated with a -element subset .
The common boundary between two adjacent Voronoi regions  and 
is called a \emph{Voronoi edge}.
This Voronoi edge is a part of 
where  and  \cite{Lee-82}.
The common intersection among more than two Voronoi regions
is called a \emph{Voronoi vertex}.
Without loss of generality,
we assume that no point in the plane is equidistant from four sites in  with respect to the city metric,
ensuring that the degree of a Voronoi vertex is
exactly three.


\paragraph{\textbf{Wavefront Propagation.}}
The wavefront propagation
is a well-established model to define Voronoi diagrams \cite{AAP-04}.
In Section~\ref{sec-complexity},
we will use this concept to interpret the formation of 
and analyze its structural complexity.


For a fixed site , let .
This means that for a fixed  the wavefront  is the circle centered at  with radius .
We call  the \emph{source} of .
Note that we can view  as the wavefront at time  of the wave that originated in  at time 0.
We refer to such a wavefront as  if the value of  is unimportant.


\begin{figure}[t]
\begin{center}
\begin{minipage}[b]{0.5\textwidth}
 \centering
 \includegraphics[width=6.5cm]{figure/wavefront}
 \caption{Wavefront Propagation.}
 \label{fig-wavefront}
\end{minipage}
\hfill
\begin{minipage}[b]{0.49\textwidth}
 \centering
 \includegraphics[width=4cm]{figure/needles_2}
 \caption{ 1-needle, 2-needle and 3-needle. }
 \label{fig-needles}
\end{minipage}
\end{center}
\end{figure}

\deleted{
\begin{figure}[bt]
\centering
\subfigure[]{\includegraphics[clip, width=8cm]{figure/wavefront}\label{fig-wavefront}}
\subfigure[]{\includegraphics[clip, width=4cm]{figure/needles_2}\label{fig-needles}}
\caption{(a) Wavefront Propagation. (b) 1-needle, 2-needle and 3-needle.}
\end{figure}}

Initially, the wavefront  is a diamond. When it touches a part of the transportation network for the first time it changes its propagation speed and, hence its shape; see Fig.~\ref{fig-wavefront}.
Certain points on the transportation network play an important role to determine the structural complexity of \kthorder city Voronoi diagrams.
\chihhung{Since each point will be touched a wavefront finally,
``the points on the transportation network that are touched by a wavefront'' is not clear.
I make some changes.}
Thus, we introduce the following definitions.
For a point ,
let  denote the \emph{isothetic projection} of  onto the transportation network, i.e.,
we shoot an isothetic half-ray starting at  in each of the four directions and for each half-ray we add its first intersection with an edge of the transportation network to . It is easy to see that there are at most four such intersections.
For a set , we denote the \emph{isothetic projection of the set } as .
For a site , we call the set  \emph{activation points}
(we added  to the list for ease of argumentation in some of our proofs).


As shown by Aichholzer et al. \cite{AAP-04}, the wavefront
 changes its propagation speed only if it hits a vertex in .
Since the shape of  can become very complex after it hits multiple activation points,
we make the following simplification for the remainder of this paper:
if a wavefront  touches a point  we do not change the propagation speed of~.
Instead, we start a new wavefront at , which, in turn,
starts new wavefronts at points in  if it reaches them earlier than any other wavefront.
Hereafter, the start of the propagation of a new wavefront is called an \emph{activation event}, or we say a wavefront is \emph{activated}.
\chihhung{We did not use the term ``originate'' throughout the paper.
So use it in the proof of Lemma~\ref{lem-mix-region} or remove the sentence.}
The shape of such a new wavefront depends on the position of  on the transportation network.
It can be categorized into one of three different shapes: 1-needle, 2-needle, and 3-needle \cite{AAP-04} (see Fig.~\ref{fig-needles}).
To simplify things, we treat a 2-needle (3-needle) as two (three) 1-needles (see Fig.~\ref{fig-needle-and-bisector}(a)).

\begin{figure}[bt]
\centering
\includegraphics[clip, width=12cm]{figure/needle-and-bisector}
\caption{(a) A 2-needle is two 1-needles. (b) Wavefront propagation of a 1-needle. (c) .}\label{fig-needle-and-bisector}
\end{figure}

When a 1-needle reaches the end of the corresponding network segment,
as shown in Fig.~\ref{fig-needle-and-bisector}(b),
its shape changes (permanently) \cite{AAP-04}.
In order to interpret the propagation of a 1-needle,
Bae et al. \cite{BKC-09} introduced a structure called \emph{needle}.
A needle  is a network segment  with weight ,
where  and .
Propagating a wavefront from 
is equivalent to propagating a 1-needle from  on the network segment  at time .
If  is obvious or unimportant we may refer to  as .
Bae et al. also defined
the  distance  between a needle  and a point 
as  plus the length of a quickest path from  to  accelerated by .
Thus, the bisector
 between two needles
 and  is well defined (see Fig.~\ref{fig-needle-and-bisector}(c)).


\deleted{
\andreas{removed something. I dont think we really need this. If you disagree I find a shorter way of explaining this.}

In order to analyze the complexity of the \kthorder Voronoi diagrams,
for each  where  and ,
we create a copy  for any point , and ensure that  is only in  and  is only in .
Therefore, for a site 
the  changes the speed at a point 
only if . \chihhung{I still think the size of , , and .
If the space is enough, maybe we can consider it.}
}

\deleted{ is significantly more difficult if for two sites ,
 is non-empty as shown in Fig.~\ref{fig-UnActived}(a) or
if for two points ,  the intersection  is nonempty as shown in Fig.~\ref{fig-UnActived}(b).
To simplify matters, we create a copy  for any point , and ensure that  is only in  and  is only in .
Therefore, for a site 
the  changes the speed at a point 
only if .
Under these modifications,
it is still true that
, , and  for each site .
\andreas{check if we really need this}}

\deleted{
\begin{figure}[bt]
\centering
\includegraphics[clip, width=12cm]{figure/duplicate}
\caption{(a) Both  and  change the speed at , i.e. two activation events occur in ,
            since .
         (b)  changes the speed at  because  is in  not because  is in .
         }\label{fig-UnActived}
\end{figure}
}




\paragraph{\textbf{Shortest Path Map.}}
We use the wavefront model to define shortest path maps \cite{BKC-09},
and use this concept to explain the formation of \emph{mixed vertices} in Section~\ref{sub-mix},
which are important for deriving the structural
complexity of the \kthorder city Voronoi diagram.

For a site  its shortest path map  is a planar subdivision that can be obtained as follows:
start by propagating a wavefront from the site .
When a point  is touched for the first time by a wavefront,
propagate an additional wavefront from .
Eventually,
each point  is touched for the first time by a wavefront propagated from a needle ,
where  and ,
and  is called the \emph{predecessor} of .
This induces .
In detail,
 partitions the plane into at most  regions 
such that all points  share the same predecessor  and  is on a quickest path
from  to , i.e., .
\chihhung{minor rephrasing}
As proved in \cite{BKC-09}, the common edge between  and  where 
belongs to the bisector .
Fig.~\ref{fig-Bisectors} illustrates an example of the function of shortest path maps where
the two Voronoi regions of  are partitioned by  and , respectively.
\deleted{Without loss of generality,
in order to guarantee that each vertex of  has exactly a degree of 3,
we assume no point in the plane has the same distance to four needles , .
\chihhung{The number of regions in  generated by , i.e.  is trivially at most 4.
Therefore, the number of such regions in all   is .
This is another viewpoint to understand the upper bound.}}




\section{Complexity}\label{sec-complexity}

In this section we
derive an upper and a lower bound of the structural complexity of the \kthorder city Voronoi diagram .
In Section~\ref{sub-mix}, we first introduce a special degree-2 vertex on a Voronoi edge called \emph{mixed vertex}
which is similar to the mixed Voronoi vertices of Cheong et al.~\cite{CEGGHLLN-11} for farthest-polygon Voronoi diagrams.
Then we derive an upper bound of the structural complexity of  in terms of the number of mixed vertices and Voronoi regions.
In Section~\ref{sub-upper}, we adopt the wavefront concept to introduce a new interpretation for the iterative construction of  by Lee \cite{Lee-82}. This yields an upper bound for the structural complexity of .
In Section~\ref{sub-lower} we construct a worst-case example to obtain a lower bound for the structural complexity of .
\chihhung{We repeat the structural complexity of a \kthorder Voronoi diagram many times above.
Is it possible to give it a short name?}
\andreas{that would be good...}







\subsection{Mixed Vertices}\label{sub-mix}

\deleted
{
\begin{figure}[bt]
\centering
\includegraphics[clip, width=8cm]{figure/Bisectors-Without-Label}
\caption{ , where  for  is a mixed vertices.}\label{fig-Bisectors}
\end{figure}
}



\begin{Definition}[Mixed Vertex]\label{def-mix-voronoi-vertex}
For two sites  and a Voronoi edge  which is part of ,
a point  on  is \emph{a mixed vertex}
if there are  and  such that
.
\end{Definition}


For instance,
Fig.~\ref{fig-Bisectors} shows a first-order city Voronoi diagram , where the mixed vertices
are marked with a square and denoted by .
The vertex  is a mixed vertex because it is in .
Definition~\ref{def-mix-voronoi-vertex} yields the following.




\begin{figure}
\centering
\includegraphics[width=.5\textwidth]{figure/Bisector}
\caption{  (solid thin edge), where  are mixed vertices.}\label{fig-Bisectors}
\end{figure}


\newcommand{\lemmVvtext}{If a Voronoi edge  contains  mixed vertices,
its complexity is .
}

\begin{Lemma}\label{lem-mVv}
\lemmVvtext
\end{Lemma}
\begin{proof}
Suppose  is part of a bisector , where .
Consider two consecutive mixed vertices  and  on ,
where  and 
(see Fig.~\ref{fig-Bisectors}).
Consider each point  on  between  and .
Since  belongs to ,
we have  and .
Together with ,
 belongs to  (recall Fig~\ref{fig-needle-and-bisector}(c)).
As a result, if a Voronoi edge  contains  mixed Voronoi vertices,
 consists of  parts, each of which belongs to a bisector between two needles.
Since the complexity of an  bisector between two needles is  \cite{BC-05},
the complexity of  is .
Note that the complexity of a bisector between two points in the city metric is ,
while the complexity of an  bisector between two needles is .
\qed
\end{proof}

\newcommand{\lemmixuppertext}{An upper bound for the structural complexity of a \kthorder city Voronoi diagram 
is , where  is the total number of mixed vertices.
}

\begin{Lemma}\label{lem-mix-upper}
\lemmixuppertext
\end{Lemma}
\begin{proof}
Lee \cite{Lee-82} proved that the number of Voronoi regions in the \kthorder Voronoi diagram is 
in any distance metric satisfying the triangle inequality, and so is the number of Voronoi edges.
By Lemma~\ref{lem-mVv}, if a Voronoi edge  contains  mixed Voronoi vertices, its complexity is .
Suppose a city Voronoi diagram  contains a set  of Voronoi edges,
and each edge  contains  mixed Voronoi vertices.
Then, the complexity of all edges, i.e., the structural complexity of ,
is .
Since , it follows that
.
\qed
\end{proof}

For the proof in Section~\ref{sub-upper},
we further categorize the mixed vertices.
Let  be a mixed vertex on the Voronoi edge between  and ,
where  and .
We call   an \emph{interior mixed vertex} of 
if , for some  and ;
otherwise, we call  an \emph{exterior mixed vertex} of .
For example, in Fig.\ref{fig-Bisectors}
the vertices  and  both are interior mixed vertices of  and exterior mixed vertices of .


\subsection{Upper Bound}\label{sub-upper}
Throughout this subsection,
we consider a Voronoi region  of a -order Voronoi diagram ,
where  and .
Let  have  adjacent Voronoi regions  for .
Note that the subsets  and  differ in exactly one element~\cite{Lee-82}. In the following let
, , and .

Lee \cite{Lee-82} proved that in any distance metric satisfying the triangle inequality,
,
and thus computing  for all the Voronoi regions  of  yields ,
leading to an iterative construction for  for any .
Fig.~\ref{fig-Euclidean} illustrates this iteration technique for the Euclidean metric:
solid segments form  and dashed segments form .
Since the gray region is part of  and also part of ,
all points in the gray region share the same two nearest sites  and ,
implying that the gray region is part of .


\begin{figure}[t]
\begin{center}
\begin{minipage}[b]{0.5\textwidth}
 \centering
 \includegraphics[width=0.7\textwidth]{figure/Euclidean}
 \caption{
 where , , and .}
 \label{fig-Euclidean}
\end{minipage}
\hfill
\begin{minipage}[b]{0.43\textwidth}
 \centering
 \includegraphics[width=0.7\textwidth]{figure/cardinality}
 \caption{ where , , and .  }
 \label{fig-cardinality}
\end{minipage}
\end{center}
\end{figure}




We adopt wavefront propagation to interpret this iterative construction
in a new way, which will lead to the main proof of this section.
Let us imagine that a wavefront is
propagated from each site  into the Voronoi region .
If a point  is first touched by the wavefront that propagated from ,
 belongs to .


Note that when ,  is not necessarily the number of adjacent regions,
i.e., .
Fig.~\ref{fig-cardinality} illustrates an example for the Euclidean metric:
 has 6 adjacent Voronoi regions but  is only 4.
This is because for a site ,
 may consist of more than one Voronoi edge,
where  is a Euclidean bisector between  and  (similar to  defined in Section~\ref{sec-prelim}).
For instance, as shown in Fig.~\ref{fig-cardinality},
 consists of two Voronoi edges  and 




Now we transfer our new interpretation to the city metric.
Let  be  for some site .
If  contains  \textbf{exterior} mixed vertices with respect to ,
 intersects  regions in .
We denote these regions by  for . Note that all  must be in . Then,
instead of propagating a single wavefront from  into  (as in the Euclidean metric), we propagate
 wavefronts, namely one from each  into .


As a result, if  contains  exterior mixed vertices,
 wavefronts will be propagated into .
During the process, when a point  is \emph{first} touched by a wavefront
propagated from ,
we propagate a new wavefront from , i.e., an activation event occurs,
if (i) 
or (ii)  and .
These two conditions amount to ,
but this classification will help us to derive the number of mixed vertices.
This is due to the fact that during the  iterations for computing  from  for ,
 contributes  activation events, but  only contributes .
\chihhung{One problem is that Lemma~\ref{lem-mix-vertices} has be moved to the appendix.}

\newcommand{\lemmixregiontext}{If  contains  exterior mixed vertices,
then  contains at most  mixed vertices,
where 
and  is the number of activation events associated with points in .
}

\begin{Lemma}\label{lem-mix-region}
\lemmixregiontext
\end{Lemma}

\begin{proof}
According to the above discussion,
we propagate  wavefronts into .
All those wavefronts combined generate at most  new wavefronts from points in ,
and  new wavefronts from points in .
Note that  (condition~(i)) but  (condition~(ii))
\chihhung{This sentence can be removed.}.
\andreas{ok}
\chihhung{maybe move condition~(i) and condition~(ii) to the statement of this lemma.}
Let  be the set of the  wavefronts.
For each point , if  is first touched by a wavefront  it is associated with .
This will partition  into   regions.
We view those regions as a special Voronoi diagram .
Note that  of those regions are unbounded.

 is a subgraph of 
since if a point  is first touched by a wavefront in  propagated from 
\chihhung{or a point  belongs to },
 belongs to .
Without loss of generality,
we assume every vertex of  has degree~3.
According to Euler's formula it holds that
,
where ,  and 
are the numbers of vertices, regions, and unbounded regions, respectively.
Since  contains  unbounded regions and  bounded regions,
 contains  vertices.
By \cite{Lee-82},
since ,
there are  Voronoi vertices in .
Therefore,  contains at most  mixed vertices.
\qed
\end{proof}

\deleted{
By applying Lemma~\ref{lem-mix-region} to each region of ,
we obtain Lemma~\ref{lem-mix-diagram}. Lemma~\ref{lem-mix-diagram} indicates
a recursive formula: , and thus leads to Lemma~\ref{lem-mix-vertices}.
Lemma~\ref{lem-mix-upper} and Lemma~\ref{lem-mix-vertices} gives an upper bound
for the structural complexity of  in Theorem~\ref{thm-upper}.
}

Applying Lemma~\ref{lem-mix-region} to each region of , yields a recursive formula for the total number of mixed vertices  in :  (see Lemma~\ref{lem-mix-diagram}). In Lemma~\ref{lem-mix-vertices} we show that this formula can be bounded by  for  iterations of this iterative approach. Finally, in Theorem~\ref{thm-upper} we combine the insights of Lemma~\ref{lem-mix-upper} and Lemma~\ref{lem-mix-vertices} to give an upper bound for the structural complexity of . 


\newcommand{\lemmixdiagramtext}{ contains  mixed vertices
where  is the number of mixed vertices of  and
 is the number of activation events associated with points in  during the computation of  from .
}

\begin{Lemma}\label{lem-mix-diagram}
\lemmixdiagramtext
\end{Lemma}
\begin{proof}

For a Voronoi region ,
let  be the number of its exterior mixed vertices,
let  be  and let  be number of activation events associated
with vertices in  during the computation.
If  is empty, .
By Lemma~\ref{lem-mix-region},
 contains at most  mixed Voronoi vertices.
Therefore, the total number of mixed vertices of  is bounded by:

\qed


\end{proof}

\newcommand{\lemmixverticestext}
{
The number of mixed vertices of  is .
}

\begin{Lemma}\label{lem-mix-vertices}
\lemmixverticestext
\end{Lemma}
\begin{proof}
\andreas{to me: proof lacks structure}
\chihhung{I need your help to organize the structure.}
Let  be the total number of mixed Voronoi vertices of 
and let  be the number of activation events associated with vertices in  during the computation of  from , described by our algorithm. Then, by Lemma~\ref{lem-mix-diagram} the following holds:


Now, we show an upper bound for the complexity of .
For a vertex  where ,
let the  iteration be the first time when  is activated by a wavefront propagated from ,
i.e.  will propagate a wavefront,
and let  belong to .
Due to this and since the points in  are the  nearest sites of~,
 is the  nearest site of .
Therefore, for , if , ,
implying that  will not be activated by a wavefront propagated from  again after the  iteration.
In other words,  causes at most one activation event due to the wavefront propagation of  during the  iterations,
and the  causes  activation events, i.e.,  .
Furthermore,  has been proved to be \cite{AAP-04,BKC-09,GSW-08}.
Therefore, . \qed
\end{proof}


\begin{Theorem}\label{thm-upper}
The structural complexity of  is .
\end{Theorem}




\subsection{Lower Bound}\label{sub-lower}

\deleted{
\begin{figure}
\centering
\includegraphics[clip, width=8cm]{figure/worst-case}
\caption{A worst-case example
where , ,  leads to a lower bound .
The bold solid segments depict the transportation network,
and the dashed segments compose .
The right part is also the farthest-site city Voronoi diagram of ,
where all points in Region~i share the same farthest site .
}\label{fig-worst}
\end{figure}
}



We construct a worst-case example (see Fig.~\ref{fig-worst}) to derive a lower bound for the structural complexity of the \kthorder city Voronoi diagram .
The example consists of a left part and a right part which are placed with a sufficiently large distance between them.
We place one vertical network segment in the left part
and build a stairlike transportation network in the right part.
Then, we place  sites in the right part and the remaining  sites in the left part.
Since the distance between the left and right part is extremely large,
the  sites in the left part hardly influence the formation of  in the right part.
Therefore,  in the right part forms the farthest-site city Voronoi diagram of the  sites,
because sharing the same  nearest sites among  sites
is equivalent to sharing the same farthest site among the  sites.

\begin{figure}[tb]
\center
\includegraphics[clip, width=.47\textwidth]{figure/worst-case}
\caption{This worst-case example
(here with , , ) leads to a lower bound of .
The bold solid segments depict the transportation network,
and the dashed segments compose .
The right part is also the farthest-site city Voronoi diagram of ,
where all points in Region~ share the same farthest site .
}\label{fig-worst}
\end{figure}


By construction,
as shown in the right part of Fig.~\ref{fig-worst},
all the points in Region~ share the same farthest site .
Since we can set the speed  to be large enough,
for each point  in Region~2,
the shortest path between  and  () moves along the transportation network counterclockwise,
and thus .
The common Voronoi edge between Regions  and ()
contains at least  (here: 7) segments
since the transportation network forms  rectangles
and each rectangle except the first one contains two vertices of the Voronoi edge.
Therefore, in the right part,
 contains at least  segments.
Together with the  in the left part, we obtain the following lower bound.


\newcommand{\thmlowerdiagramtext}{The structural complexity of  is .
}


\begin{Theorem}\label{thm-lower-diagram}
\thmlowerdiagramtext
\end{Theorem}
\begin{proof}
We need to distinguish two cases:

i) :
This implies that the left part contains  segments, and thus,  contains  segments.

ii) :
This implies that the left part is empty, and thus,
 contains  segments.

This concludes the proof. \qed
\end{proof}






\section{Algorithms}
\label{sec-algorithms}

In this section we present an iterative algorithm to compute \kthorder city Voronoi diagrams in
 time. Its main idea has already been introduced in the complexity considerations in Section~\ref{sub-upper}.
For the special case of the
farthest-site Voronoi diagram, i.e., the -order Voronoi diagram,
this algorithm takes  time. However, for the farthest-site city Voronoi diagram we present a divide-and-conquer algorithm which requires only  time.


\subsection{Iterative Algorithm for -Order City Voronoi Diagrams}

We describe an algorithm to compute \kthorder city Voronoi diagrams  based on
the ideas in Section~\ref{sub-upper} and
Bae et al.'s \cite{BKC-09} -time
algorithm for the first-order city Voronoi diagram .
Bae et al.'s approach views each point site in  as a needle with zero-weight and zero-length,
and simulates the wavefront propagation from those needles to compute .
Since their approach can handle general needles,
we adopt it to simulate the wavefront propagation of Section~\ref{sub-upper} to compute  from .



\paragraph{Algorithm.}
We give the description of our algorithm for a single Voronoi region .
All four steps have to be repeated for each Voronoi region of .

Let  have  adjacent regions  with  for 
and let .
Our algorithm computes  as follows:
\begin{enumerate}
  \item Compute a new set  of sites (needles):
        For , if the Voronoi edge between  and  intersect  regions  in , , insert every  into .
  \item Construct a new transportation network  from :
        For each point ,
        if  is located on an edge  of , insert  into .
  \item Perform Bae et al.'s wavefront-based approach
         to compute  under the new transportation network .
         The approach can intrinsically handle needles as weighted sites.
  \item Determine  from :
		Consider each edge  in .
        Let~ be an edge between  and 
        where ,  and .
        If , then  is part of .
\end{enumerate}

Note that Step 2 is used only to reduce the runtime of the algorithm.
Lemma~\ref{lem-compute-region} shows the correctness and the run time of this algorithm for a single Voronoi region.

\newcommand{\lemcomputeregion}{ can be computed in  time,
where~ is the number of Voronoi edges,  is the number of mixed vertices,
and .
}



\begin{Lemma}\label{lem-compute-region}
\lemcomputeregion
\end{Lemma}
\begin{proof}
We begin by proving correctness.
Since  is exactly  \cite{Lee-82},
it is sufficient to prove that the algorithm correctly computes .
If the algorithm fails to compute ,
it must fail to propagate a wavefront from a needle , where  belongs to  and
 is nonempty.
We prove that this cannot occur by contradiction. Assume that the algorithm does not propagate a wavefront from an  for some  in  and  is nonempty.
However, either  and  must be nonempty, then,
Step~1 will include  in .
Or , then
Step~2 will include the corresponding network segment in ,
and thus  will be activated to propagate a wavefront.
Both possibilities contradict the initial assumption.
Therefore, the algorithm correctly computes .


We proceed by giving time complexity considerations.
It is clear that  is .
The run time of step~1 is linear in the complexity of the boundary of 
and thus is .
Since by definition  and ,
both  and  are in .
Since  and  ,
Step~3 takes  time \cite{BKC-09}.
Step~4 takes the time linear in the complexity of .
The activation events associated with vertices in  are only associated to vertices in , we know that .
Therefore, since there are , , and  wavefronts
due to , , and , respectively,
the complexity of  is .
Since  it holds that .
We conclude that the total running time is  .
\qed
\end{proof}



Applying Lemma~\ref{lem-compute-region} to each region of  combined with Theorem~\ref{thm-upper}
leads to Lemma~\ref{lem-compute-diagram}.
The summation of  in Lemma~\ref{lem-compute-diagram} for 
gives Theorem~\ref{thm-kth-time}.

\newcommand{\lemcomputediagramtext}
{
 can be computed from  in  time.
}

\begin{Lemma}\label{lem-compute-diagram}
\lemcomputediagramtext
\end{Lemma}
\begin{proof}
For a Voronoi region ,
let  be the number of Voronoi edges,
 be the number of mixed vertices
\chihhung{the prime is to distinguish from the ``exterior'' mixed vertices.},
and  be .
By Lemma~\ref{lem-compute-region},
the time complexity of computing  from  is

By Theorem~\ref{thm-upper},

\chihhung{Although each mixed vertices will be counted twice during the summation,
it still holds.}.
It is also clear that .
Therefore,
the total time complexity is .
The correctness follows from the correctness proof of Lemma~\ref{lem-compute-region}.\qed
\end{proof}



\newcommand{\thmkthtimetext}{
 can be computed in  time.
}

\begin{Theorem}\label{thm-kth-time}
\thmkthtimetext
\end{Theorem}
\begin{proof}
By Lemma~\ref{lem-compute-diagram},
the total time complexity is .
\qed
\end{proof}


\subsection{Divide-and-Conquer Algorithm for Farthest-Site City Voronoi Diagram}

In this section we describe a divide-and-conquer approach to compute the farthest-site city Voronoi diagram .
Since there are  Voronoi regions in  and each of them is associated with a site ,
we denote such a region by .


The idea behind this algorithm is as follows: To compute ,
divide  into two equally-sized sets  and ,
compute  and ,
and then merge the two diagrams into .
Now, suppose we have already computed  and .
Then, the edges of a Voronoi region  in  stem from three sources: i) contributed by , ii) contributed by , and iii) contributed by two points, one in  and the other in , that have the same distance to two farthest site.
In fact, the union of all of the third kind of edges is .
Each connected component of 
is called a \emph{merge curve}. A merge curve can be either a closed or open simple curve.




If all the merge curves are computed,
merging  and 
takes time linear in the complexity of .
\deleted{
\begin{enumerate}
  \item Find at one point on each merge curve.
  \item Trace out the merge curve from the discovered point.
\end{enumerate}
}
To compute the merge curves, we first need to find a point on each merge curve, and then trace out the merge curves from these discovered points.


In order to compute a merge curve,
we modify Cheong et al.'s divide-and-conquer algorithm \cite{CEGGHLLN-11}
for farthest-polygon Voronoi diagrams in the Euclidean metric to satisfy our requirements.
Given a set  of disjoint polygons, ,
of total complexity~,
the farthest-polygon Voronoi diagram  partitions the plane
into Voronoi regions such that all points in a Voronoi region share the same farthest polygon in .
Let  be the number of vertices of a polygon 
and let  be .

Their algorithm computes the medial-axis  for each polygon 
and refines  by .
 partitions the plane into regions such that
all points in a region share the same closest element of ,
where an element is a vertex or an edge of .
In other words,
for each point ,
 provides a shortest path between  and .
Therefore,
the medial axes for ,
with ,
have the same function as the shortest path maps ,
with  in the city metric.
By replacing  and  with  and  respectively,
the divide-and-conquer algorithm of Cheong et al. \cite{CEGGHLLN-11}
can be modified to compute 
with respect to the city metric.


\deleted{The major difference between the farthest-polygon Voronoi diagram and
the farthest-site city Voronoi diagram is the structural complexity.
First, the complexity of  is ,
while that of  is .
Second, if , ,
while .
Therefore, ,
while .
\andreas{this paragraph seems to be out of place...}
\chihhung{Find another feasible place or just remove it.
The motivation is to clarify a potential question,
why  can be constructed in  time
but  requires  time.
The answer is their different sizes.}}

Cheong et al. \cite{CEGGHLLN-11} pointed out
the bottleneck with respect to running time is to find for each closed merge curve
a point that lies on it.
In order to overcome the bottleneck,
the authors use some specific point location data structures
\cite{EGS-86,Mulmuley-90}.
Let  be divided into two sets  and  ,
where .
Cheong et al. \cite{CEGGHLLN-11} construct the point location data structures for  and .
For each polygon  and each vertex ,
they perform a point location query in   and  (likewise for each polygon ).
Each point location query requires  primitive operations,
and each operation tests for  points and takes  time.
Hence, one point location query takes  time.
Since ,
merging  and 
takes  time.


Since in our case
,
we perform  point location queries,
each of which takes  time.
Therefore,
merging  and 
takes  time.
We conclude:

\newcommand{\thmfarthesttimetext}{ can be computed in  time.
}

\begin{Theorem}~\label{thm-farthest-time}
\thmfarthesttimetext
\end{Theorem}

\begin{proof}
In the beginning, for each site ,
 is exactly .
Computing  takes  time \cite{BKC-09},
implying that computing , for all , takes  time.
Consider the merge process at some level .
The set  is divided into  subsets,
and each of them contains at most  sites.
Therefore, the merging process at level 
takes  time.
Since there are  levels,
 can be computed in  time.
\qed
\end{proof}



\section{Conclusion}
\label{sec-conclusion}

We contribute two major results for the \kthorder city Voronoi diagram.
First, we prove that its structural complexity is  and .
This is quite different from the  bound in the Euclidean metric~\cite{Lee-82}.
It is especially noteworthy that when , i.e., the farthest-site Voronoi diagram,
its structural complexity in the Euclidean metric is ,
while in the city metric it is .
Secondly, we develop the first algorithms that compute the \kthorder city Voronoi diagram
and the farthest-site Voronoi diagram.
Our algorithms show that traditional techniques can be applied to the city metric.
Furthermore,
since the complexity of the first-order city Voronoi diagram is ,
one may think that the complexity the transportation network contributes to the complexity of the \kthorder city Voronoi diagram is independent of .
However, our results show that the impact of the transportation network increases with the value of 
rather than being constant.


\deleted{
\chihhung{observation is a little weak. Maybe find a new term more contributive.}.
First, in most cases, e.g. general sites (line segments and polygons),
the complexity of the farthest-site Euclidean Voronoi diagram is still linear.
}
\deleted{
However, recently, Bae and Chwa \cite{BC-09} proved that complexity of farthest-site geodesic Voronoi diagram
is , where  is the total complexity of obstacles.
First, the results indicate that
the underlying distance metric significantly influences the complexity of the corresponding Voronoi diagrams.
Second,
since the complexity of the first-order city Voronoi diagram is ,
one may think that the complexity the transportation network contributes to the complexity of the \kthorder city Voronoi diagram is independent of .
However, our results show that the impact of the transportation network increases with the value of 
rather than being constant.
Besides, with a transportation network on the Euclidean plane, i.e., the Euclidean city metric,
the size of the corresponding nearest-site Voronoi diagram is already .
From our results, we make a conjecture that the complexity of the \kthorder Voronoi diagram in the Euclidean city metric is 
\chihhung{We also can remove the conjecture.
I originally want to mention a conjecture and prove in the journal version.}.}


\section*{Acknowledgment}
A. Gemsa received financial support by the \emph{Concept for the Future} of KIT within the
framework of the German Excellence Initiative. D. T. Lee and C.-H. Liu are supported by the National Science Council, Taiwan under grants No. NSC-98-2221-E-001-007-MY3 and No. NSC-99-2911-I-001-506.




\begin{thebibliography}{7}

\bibitem{AAP-04}
O. Aichholzer, F. Aurenhammer, and B. Palop,
``Quickest paths, straight skeletons, and the city Voronoi diagram,''
\emph{Discrete Comput. Geom.},
vol. 31, pp. 17--35, 2004.

\bibitem{ABMS-98}
P. K. Agarwal, M. de Berg, J.~Matou{\v s}ek, and O, Schwarzkopf, ``Constructing levels in arrangements and higher order Voronoi diagrams,'' \emph{SIAM J. Comput.}, Vol. 27, No.3, pp. 654--667, 1998.

\bibitem{AS-92}
F. Aurenhammer and O. Schwarzkopf, ``A simple on-line randomized incremental algorithm for computing higher order Voronoi diagrams,''
\emph{Internat. J. Comput. Geom. Appl.}, Vol. 2, pp. 363--381, 1992.

\bibitem{BC-05}
S. W. Bae and K.-Y. Chwa,
``Shortest paths and Voronoi diagrams with transportation networks under general distances,''
 \emph{Proc. Internat. Symp. on Algorithm and Comput.},
 pp. 1007--1018, 2005.

\bibitem{BDT-93}
J. D. Boissonnat, O. Devillers, and M. Teillaud, ``A semidynamic construction of higher-order Voronoi diagrams and its randomized analysis,'' Algorithmica, vol. 9, pp. 329--356, 1993.

\bibitem{BKC-09}
S. W. Bae, J.-H. Kim, K-Y. Chwa,
``Optimal construction of the city Voronoi diagram,''
\emph{Internat. J. Comput. Geom. Appl.},
vol. 19, no. 2, pp. 95--117, 2009.

\bibitem{CE-87}
B. Chazelle and H. Edelsbrunner, ``An improved algorithm for constructing th-order Voronoi diagrams,''
\emph{IEEE Transactions on Computers}, vol. 36, No.11, pp. 1349--1454, 1987.

\bibitem{CEGGHLLN-11}
O. Cheong, H. Everett, M. Glisse, J. Gudmundsson,
S. Hornus, S. Lazard, M. Lee, and H.-S. Na,
``Farthest-polygon Voronoi diagrams,''
\emph{Comput. Geom. Theory Appl.},
vol. 44, pp. 234--247, 2011.


\bibitem{EGS-86}
H. Edelsbrunner, L. J. Guibas, and J. Stolfi,
``Optimal point location in a monotone subdivision,''
\emph{SIAM J. Comput.},
vol. 15, no. 2, pp. 317--340, 1986.

\bibitem{GSW-08}
R. G\"{o}rke, C.-S. Shin, and A. Wolff,
``Constructing the city Voronoi diagram faster,''
\emph{Internat. J. Comput. Geom. Appl.},
vol. 18, no. 4, pp. 275--294, 2008.

\bibitem{Lee-82} D. T. Lee, ``On -nearest neighbor Voronoi diagrams in the plane,''
\emph{IEEE Trans. Comput.}, vol. 31, no. 6, pp. 478--487, 1982.

\bibitem{Mulmuley-90}
K. Mulmuley,
``A fast planar partition algorithm, I,''
\emph{J. Symbolic Comput.},
vol. 10, no. 3--4, pp. 253--280, 1990.

\bibitem{Mulmuley-91}
K. Mulmuley, ``On levels in arrangements and Voronoi diagrams,''\emph{Discrete Comput. Geom.}," vol. 6, pp. 307--338, 1991.




\end{thebibliography}

\deleted{
\newpage
\appendix
\section*{Appendix}




\rephrase{Lemma}{\ref{lem-mix-diagram}}{\lemmixdiagramtext}




\rephrase{Lemma}{\ref{lem-mix-vertices}}{\lemmixverticestext}



\rephrase{Theorem}{\ref{thm-lower-diagram}}{\thmlowerdiagramtext}




\rephrase{Lemma}{\ref{lem-compute-region}}{\lemcomputeregion}


\rephrase{Lemma}{\ref{lem-compute-diagram}}{\lemcomputediagramtext}


\rephrase{Theorem}{\ref{thm-kth-time}}{\thmkthtimetext}



\rephrase{Theorem}{\ref{thm-farthest-time}}{\thmfarthesttimetext}



}

\end{document}
