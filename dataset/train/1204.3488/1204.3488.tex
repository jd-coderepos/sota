\documentclass[preprint,12pt]{elsarticle}
\usepackage{epsfig}
\usepackage{amssymb}
\usepackage{amsthm}
\usepackage{amsmath}
\biboptions{sort&compress}



\journal{Theoretical Computer Science}

\newcommand{\keywords}[1]{\par\addvspace\baselineskip
\noindent\keywordname\enspace\ignorespaces#1}

\newtheorem{thm}{Theorem}
\newtheorem{lem}[thm]{Lemma}
\newproof{pf}{Proof}

\newcommand{\I}{\hspace{0.45cm}}
\newcommand{\II}{\I\I} \newcommand{\III}{\II\I}
\newcommand{\IIII}{\II\II} \newcommand{\IIIII}{\IIII\I}
\newcommand{\IIIIII}{\IIII\II} \newcommand{\IIIIIII}{\IIII\III}
\newcommand{\IIIIIIII}{\IIII\IIIII}

\newcommand{\CC}{\mathcal{C}}
\newcommand{\RE}{\mathbb{R}}
\newcommand{\cc}{\mathbf{c}}
\newcommand{\com}[1]{[\textsc{#1}]}
\newcommand{\z}{\mathbf{4}}

\begin{document}

\begin{frontmatter}

\title{Efficient sub- approximations for\\minimum dominating sets in unit disk graphs\tnoteref{ALL}}
\tnotetext[ALL]{An extended abstract of this paper appeared in the proceedings of the 10th Workshop on Approximation and Online Algorithms (WAOA'12), \emph{Lecture Notes in Computer Science} \textbf{7846} (2013), 82--92. Research partially supported by FAPERJ and CNPq grants.}

\author[gf]{G.D. da Fonseca}
\ead{fonseca@uniriotec.br}

\author[ch]{C.M.H. de Figueiredo}
\ead{celina@cos.ufrj.br}

\author[vs]{V.G. Pereira de S\'a}
\ead{vigusmao@dcc.ufrj.br}

\author[rm]{R.C.S. Machado}
\ead{rcmachado@inmetro.gov.br}

\address[gf]{Universidade Federal do Estado do Rio de Janeiro, Brazil}

\address[ch,vs]{Universidade Federal do Rio de Janeiro, Brazil}

\address[rm]{Instituto Nacional de Metrologia, Qualidade e Tecnologia, Brazil}

\begin{abstract}
A unit disk graph is the intersection graph of  congruent disks in the plane. Dominating sets in unit disk graphs are widely studied due to their applicability in wireless ad-hoc networks. Because the minimum dominating set problem for unit disk graphs is \textbf{NP}-hard, numerous approximation algorithms have been proposed in the literature, including some PTASs.
However, since the proposal of a linear-time -approximation algorithm in 1995, the lack of efficient algorithms attaining better approximation factors has aroused attention.
We introduce an  algorithm that takes the usual adjacency representation of the graph as input and outputs a -approximation. This approximation factor is also attained by a second algorithm, which takes the geometric representation of the graph as input and runs in  time regardless of the number of edges.
Additionally, we propose a -approximation which can be obtained in  time given only the graph's adjacency representation. It is noteworthy that the dominating sets obtained by our algorithms are also independent sets.
\end{abstract}

\begin{keyword}
approximation algorithms; dominating set; unit disk graph.
\end{keyword}

\end{frontmatter}


\section{Introduction}

A \emph{unit disk graph}  is a graph whose  vertices can be mapped to points in the plane and whose  edges are defined by pairs of points within Euclidean distance at most  from one another. Alternatively, one can regard the vertices of  as mapped to coplanar disks of unit diameter, so that two vertices are adjacent whenever the corresponding disks intersect.

A \emph{dominating set}  is a subset of the vertices of a graph such that every vertex not in  is adjacent to some vertex in . An \emph{independent set} is a subset of mutually non-adjacent vertices. An \emph{independent dominating set} is a dominating set which is also an independent set. Note that any maximal independent set is an independent dominating set.

Dominating sets in unit disk graphs are widely studied due to their application in wireless ad-hoc networks~\cite{heuristics}. Since it is \textbf{NP}-hard to compute a minimum dominating set of a unit disk graph~\cite{udg}, several approximation algorithms have been proposed~\cite{cccg,erlebach10,esa-Gibson,ptas-geometric,heuristics,ptas-graph-journal,zou11}. Such algorithms are of two main types. \emph{Graph-based algorithms} receive as input the adjacency representation of the graph and assume no knowledge of the point coordinates, whereas \emph{geometric algorithms} work in the Real RAM model of computation and receive solely the vertex coordinates as input\footnote{The Real RAM model is a technical necessity, otherwise storing the coordinates of the vertices would require an exponential number of bits~\cite{integer}.}. 

If the coordinates of the  disk centers are known, the  edges of the corresponding graph  can be figured out easily. It can be done in  time under the Real RAM model with floor function and constant-time hashing, and in  time without those operations~\cite{bentley}. Thus, for the price of a conversion step, graph-based algorithms can be used when the input is a unit disk realization of . However, unless \textbf{P}=\textbf{NP}, no efficient algorithm exists to decide whether a given graph admits a unit disk realization~\cite{breu}, let alone exhibit one. As a consequence, geometric algorithms cannot be efficiently transformed into graph-based algorithms. In this paper, we introduce approximation algorithms of both types, benefiting from the same approximation factor analysis. The proposed graph-based algorithm runs in  time, and the geometric algorithm runs in  time regardless of .

\paragraph*{Previous algorithms}
A graph-based -approximation algorithm that runs in  time was presented in~\cite{heuristics}. The algorithm computes a maximal independent set, which turns out to be a -approximation because unit disk graphs contain no  as induced subgraphs, as shown in that same paper.\footnote{The graph  consists of a vertex with  pendant neighbors.}

Polynomial-time approximation schemes (PTAS) were first presented as geometric algorithms~\cite{ptas-geometric} and later as graph-based algorithms~\cite{ptas-graph-journal}. Also, a graph-based PTAS for the more general disk graphs was proposed in~\cite{esa-Gibson}. Unfortunately, the complexities of the existing PTASs are high-degree polynomials. For example, the PTAS presented in~\cite{ptas-graph-journal} takes  time to obtain a -approximation (\mbox{using} the analysis from~\cite{cccg}). Although its analysis is not tight, the running time is too high even for moderately large graphs. The reason is that these PTASs invoke a subroutine that verifies by brute force whether a graph admits a dominating set with  vertices. The verification takes  time, and it is unlikely that this can be improved (unless \textbf{FPT}=\textbf{W[1]}, as proved in~\cite{marx}). Such a subroutine is applied to several subgraphs, and the value of  grows as the approximation error decreases. A similar strategy was used in~\cite{ids-ptas} to obtain a PTAS for the minimum independent dominating set.

The lack of fast algorithms with approximation factor less than  was recently noticed in~\cite{cccg}, where geometric algorithms with approximation factors of  and  and running times respectively  and  were presented. While a significant step towards approximating large instances, those algorithms require the geometric representation of the graph, and their running times are still polynomials of rather high degrees. Linear and near-linear-time approximation algorithms constitute an active topic of research, even for problems that can be solved exactly in polynomial time, such as maximum flow and maximum matching~\cite{maxflow,matchings}.

It is useful to contrast the minimum dominating set problem with the maximum independent set problem. While a maximal independent set is a -ap\-prox\-i\-ma\-tion to both problems, it is easy to obtain a geometric -ap\-prox\-i\-ma\-tion to the maximum independent set problem in  time~\cite{nieberg}. In the graph-based version, a related strategy takes roughly  time, though. No similar results are known for the minimum dominating set problem.

The existing PTASs for the minimum dominating set problem in unit disk graphs are based on some packing constraints
that apply to unit disk graphs.\footnote{In \emph{packing problems}, one usually wants to enclose non-{over\-lapping} objects into a recipient  
covering the greatest possible fraction of the recipient area.} 
One of these constraints is the \emph{bounded growth property}: the size of an independent set formed by vertices within distance  of a given vertex, in a unit disk graph, is at most . Note, however, that the bounded growth property is not tight. For example, for , it gives an upper bound of  vertices where the actual maximum size is . Since the bounded growth property is strongly connected to the problem of packing circles in a circle, obtaining exact values for all  seems unlikely~\cite{Fodor}.


\paragraph*{Our contribution}
Our main result consists of the two approximation algorithms given in Section~\ref{s:algorithm}: a graph-based algorithm, which runs in linear  time, and its geometric counterpart, which runs in  time in the Real RAM model, regardless of the number of edges.
The approximation factor of both algorithms is . The strategy in both cases is to construct a -approximate solution using the algorithm from~\cite{heuristics}, and then perform local improvements to that initial dominating set. Our main lemma (Lemma~\ref{l:irreducible}) uses forbidden subgraphs to show that a solution that admits no local improvement is a -approximation. Since the dominating sets produced by our algorithms are independent sets, the same approximation factor holds for the independent dominating set problem.

Proving that a certain graph is \emph{not} a unit disk graph (and is therefore a forbidden induced subgraph) is no easy feat\footnote{The fastest known algorithm to decide whether a given graph is a unit disk graph is doubly exponential~\cite{spinrad}.}. We make use of an assortment of results from discrete geometry in order to prove properties of unit disk graphs that are interesting \textit{per se}.
For example, we use universal covers and disk packings to show that the neighborhood of a clique in a unit disk graph contains at most  independent vertices.
These properties, along with a tighter version of the bounded growth property, are collected in Section~\ref{s:forbidden}, and
allow us to show that certain graphs are not unit disk graphs. Consequently, the analyses of our algorithms employ a broader set of forbidden subgraphs which include, but are not limited to, the .

Additionally, in Section~\ref{s:partial}, we show that a possible, somewhat natural refinement to our graph-based algorithm leads to a tighter -approximation, albeit for the price of an extra  multiplying factor in the time complexity of the algorithm.


\section{Forbidden subgraphs} \label{s:forbidden}

In this section, we introduce some lemmas about the structure of unit disk graphs. These lemmas will be applied to prove our approximation factors in Sections~\ref{s:algorithm} and \ref{s:partial}. We start by stating three previous results from the area of discrete geometry. The first lemma comes from the study of universal covers  (for a recent survey see~\cite{constants}).
\begin{lem}[P\'al~\cite{pal}]\label{l:universal_cover}
If a set of points  has diameter , then  can be enclosed by a circle of radius .
\end{lem}

Packing congruent disks in a circle is a well-studied problem. Exact bounds on the radius of the smallest circle packing  unitary disks are known for some small values of , namely  and ~\cite{Fodor}. The bound for  will be useful to us.
\begin{lem}[Fodor~\cite{Fodor}]\label{l:pack13}
The radius of the smallest circle enclosing  points with mutual distances at least  is .
\end{lem}

The \emph{density} of a packing is the ratio between the covered area and the total area. The following general upper bound is useful when no exact bound is known.
\begin{lem}[Fejes T\'oth~\cite{density}]\label{l:density}
Every packing of two or more congruent disks in a convex region has density at most .
\end{lem}

Given a graph  and a vertex , let  denote the \emph{open neighborhood} of  and let  denote the \emph{closed neighborhood} of~. More generally, the \emph{open -neighborhood} of a vertex  is the set of vertices  such that the distance between  and  in  is exactly , while the \emph{closed -neighborhood} of a vertex  is the set of vertices  such that the distance between  and  in  is at most . For a set , we let  and . Finally, given a subgraph  of , the closed neighborhood of , denoted , is the set of vertices that belong to the closed neighborhood of some vertex of . The following two lemmas concern neighborhoods in unit disk graphs.
\begin{lem} \label{l:clique}
The closed neighborhood of a clique in a unit disk graph contains at most  independent vertices.
\end{lem}
\begin{pf}
By Lemma~\ref{l:universal_cover}, the points which define a clique in a unit disk graph can be enclosed by a circle of radius . Therefore, the points corresponding to the closed neighborhood of such a clique are enclosed by a circle of radius . By Lemma~\ref{l:pack13},
a circle enclosing  points with mutual distances at least  has radius at least . Since , the lemma follows.\qed \end{pf}
\begin{lem} \label{l:2neighborhood}
Given an integer , the closed -neighborhood of a vertex in a unit disk graph contains at most  independent vertices.
\end{lem}
\begin{pf}
All  disks of diameter  corresponding to the closed -neighborhood of a vertex  must be enclosed by a circle  of radius  centered on . Each disk of diameter  has area  and  has area . Using Lemma~\ref{l:density}, we have , and the lemma follows.\qed
\end{pf}

We say that a graph  is \emph{-pendant} if there is a vertex  in  with ~vertices of degree~ in the open neighborhood of  and  vertices of degree~ in the open -neighborhood of . We refer to  as a \emph{generator} of the -pendant graph. The following lemma bounds the value of the parameter  for -pendant unit disk graphs.

\begin{lem} \label{l:4l-pendant}
If  is a -pendant unit disk graph, then .
\end{lem}
\begin{pf}
Let  be a generator of . Since  is a forbidden induced subgraph~\cite{heuristics} and  has  neighbors of degree , we have that the remaining neighbors of  together with  itself form a clique. By Lemma~\ref{l:clique}, it follows that .\qed
\end{pf}

Next, we consider the case of -pendant unit disk graphs.

\begin{lem} \label{l:3l-pendant}
If  is a -pendant unit disk graph, then .
\end{lem}
\begin{pf}
Let  be a generator of . Since two vertices are adjacent if and only if their Euclidean distance is at most , the closed neighborhood of  lies inside a circle of radius  centered at . Let  be a neighbor of  with \linebreak degree . We divide the aforementioned circle into six congruent sectors  in such a way that  sits on the boundary of two adjacent sectors . Since the diameter of each sector is  and  has degree , we have that  only contains vertex . By the same argument, the remaining two neighbors of  that have degree  are contained in two of the remaining four sectors, which we call  (note that the sectors do not necessarily appear in the order ). Therefore, only sectors  may contain the neighbors of  that have degree greater than .

Notice that the neighbors of the vertices in  should be located within Euclidean distance at most  from . Consequently, the disks of unit diameter corresponding to such vertices should be completely contained in the region defined as the Minkowski sum of  and a disk of radius , i.e.~the region within distance at most  from . We compute upper bounds to the area of such a region by considering two cases, depending on whether  and  are opposite sectors (bounded by the same pair of straight lines).

\begin{figure}[t]
 \centering
 \includegraphics[scale = .8]{sectors.eps}
 \caption{\label{f:sectors}Upper bounds to the area of the Minkowski sum of  and a disk of radius  in the two different scenarios discussed in the proof of Lemma~\ref{l:3l-pendant}: in (a), the dark area corresponds to a semicircle containing ; in (b), it corresponds to the convex hull of~.}
\end{figure}

First, we consider the case when  and  are \emph{not} opposite sectors. In this scenario,  is contained in a semicircle of radius  centered at the generator , as represented in Figure~\ref{f:sectors}(a). We define a region  as the locus of points within distance at most  from the aforementioned semicircle of radius . The area  of  is therefore .

By Lemma~\ref{l:density}, the number of disks of unit diameter contained in  is at most 
Since  of these disks are the degree-1 neighbors of , we have .

Second, we consider the case when  and  are opposite sectors. In this scenario, the region  is not convex. In order to obtain a convex region , we define  as the locus of points within distance at most  from the convex hull of  (see Figure~\ref{f:sectors}(b)). The area  of  is therefore .

By Lemma~\ref{l:density}, the number of disks of unit diameter contained in  is at most 
Since  of these disks are the degree-1 neighbors of , we have .
\qed
\end{pf}

Finally, the following lemma holds for the general case.

\begin{lem} \label{l:kl-pendant}
If  is a -pendant unit disk graph, then .
\end{lem}
\begin{pf}
Immediately from Lemma~\ref{l:2neighborhood} with .\qed
\end{pf}


\section{Linear-time 44/9-approximation} \label{s:algorithm}

In this section, we present two 44/9-approximation algorithms. The key property to analyze the approximation factor is presented in Lemma~\ref{l:irreducible}, while the running time analyses are presented in Sections~\ref{s:graph-alg} and~\ref{s:geo-alg}.

Hereafter, let  be a unit disk graph, and let  be an independent dominating set of . If  and , we say that  \emph{dominates}  and, conversely, that  \emph{is dominated} by .

As already mentioned, unit disk graphs are free of induced . Therefore, at most  vertices of  may belong to the closed neighborhood of any given vertex . A \emph{corona} is a set  consisting of exactly  neighbors of some vertex . Such a vertex , which is not necessarily unique, is called a \emph{core} of the corona , whereas the  vertices of the corona are referred to as the corona's \emph{petals}. Notice that the subgraph induced by a corona  and a corresponding core  is a .

Given a dominating set , a corona  is said to be \emph{reducible} if there is a core  of  such that  is a dominating set. 
We refer to the operation that converts  into the smaller dominating set  as a \emph{reduction} of  with respect to . If there is no core allowing for a reduction of , than  is dubbed \emph{irreducible}. 
If  is an irreducible corona, then, for every core  of , 
there must exist a vertex , such that:
\begin{enumerate}[(i)]
\item  is not adjacent to ;
\item  is only dominated, in , by vertices that belong to .
\end{enumerate}
We call  a \emph{witness} of , conveying the idea that the corona having  as a core cannot be reduced with respect to  due to the existence of . 

\begin{lem} \label{l:irreducible}
Let  be a unit disk graph,  an independent dominating set in , and  a minimum dominating set of . If all coronas in  are irreducible, then .
\end{lem}
\begin{pf}We use a charging argument to bound the ratio between the cardinalities of  and . Consider that each vertex  splits a \emph{unit charge} evenly among the vertices in the closed neighborhood . The function  below corresponds to the total charge assigned to each vertex , accumulating the (fractional) charges that  receives from the vertices in :


Note that, since  and  are dominating sets, neither  nor  are ever empty, and . Such function  allows us to write the cardinality of  as


Since

is precisely the average value of  over the elements of , we obtain the desired bound  by showing that the existence of vertices  \linebreak in  with  is counterbalanced by a sufficiently large number of vertices  in  with .

Before we continue, note that  means that , because the summation in (\ref{eq:fv}) has at most  terms, all of which are of the form  for some integer .
Thus, let  be a vertex in  with . Clearly, , otherwise , because  is an independent set. Moreover,  must have exactly  neighbors in , since a larger number of neighbors in  would imply the existence of an induced  in , which is not possible, and a smaller number would imply , a contradiction. Therefore, vertex  is a core.

Now let  be the corona of which  is a core. Because there are no reducible coronas in , the core  must have a witness .
Note that, for all petals , the only vertex in  is the core~. Otherwise, the contribution of some  in the summation yielding  --- given by (\ref{eq:fv}) --- would be at most , and  would be at most , a contradiction. In particular, the above implies that the witness , which is adjacent to at least one vertex in , cannot belong to . But  is a dominating set, so there must exist a vertex  that is adjacent \linebreak to , and  because a witness  is not adjacent to the corresponding core by definition. We call  a \emph{reliever} of~. Figure~\ref{f:witness} illustrates this situation.

\begin{figure}
 \centering
 \includegraphics[scale = .8]{witness.eps}
 \caption{\label{f:witness} Figure for the proof of Lemma~\ref{l:irreducible}. A square indicates a vertex of , a solid circle a vertex of the corona , and a hollow circle a vertex not in . Vertices  and  are respectively witness and reliever of core .}
\end{figure}

We now show that . For sake of contradiction, assume \linebreak . Because  is free of induced , such number must be \linebreak exactly , so that  is the core of a corona . Such a corona must be disjoint from corona , otherwise there would be a vertex in \mbox{} adjacent to more than one vertex in , namely  and , contradicting the fact that the only neighbor in  of any petal of  is the core~. Since, by definition, the witness  is only dominated in  by vertices of , we have . Hence,  is an independent set of , constituting, along with the core , an induced  in . This is a contradiction, because  is a unit disk graph. Thus, . Since , we have .

We have just shown that the existence of a vertex  in  with \mbox{} implies the existence of a vertex  such that . Were this correspondence one-to-one, we would be able to state that the average \linebreak of  over the elements of  was no greater than . Unfortunately, this correspondence is not necessarily one-to-one, as exemplified by the graph in Figure~\ref{f:badgraph}, for which a disk model is given in Figure~\ref{f:badgraph_model} with coordinates presented in Table~\ref{t:coordinates}.

\begin{figure}[t!]
 \centering
 \includegraphics[scale = .8]{badgraph.eps}
 \caption{\label{f:badgraph} Unit disk graph where  distinct cores  share the same reliever .}
\bigskip
\bigskip
\bigskip
 \centering
 \includegraphics[scale = .26]{badgraph_model.pdf}
 \caption{\label{f:badgraph_model} Geometric model for the graph in Figure~\ref{f:badgraph}. Due to scale/precision issues, some disks appear to touch one another when in fact they do not. For clarity, disks that actually touch one another have their centers connected by a straight line. Moreover, there are no concentric disks among those for the witnesses . The coordinates of the centers are given in Table~\ref{t:coordinates}.}  
\end{figure}
 
Still, the lemmas in Section~\ref{s:forbidden} allow us to bound the ratio between the number of vertices  with  and the number of vertices  for which the values of  are significantly lower. Let  be a reliever. In order to obtain the claimed bound, we consider two cases depending on the size \linebreak of :

\smallskip

(i)  By Lemma~\ref{l:2neighborhood}, the closed -neighborhood of  contains at most  independent vertices. Since each corona contains  independent vertices, at most  cores may share a common reliever\footnote{We would like to thank an anonymous referee for this simplified argument.}. To derive an upper bound, let  denote such cores. If , then the average value of  among  is at most
 

(ii)  By Lemma~\ref{l:4l-pendant}, if , then at most  cores  may have  as their common reliever, for otherwise we obtain \linebreak a -pendant graph, which cannot be a unit disk graph. Thus, the average value of  among  is at most


The worst case is , and therefore , concluding the proof.\qed
\end{pf}

\begin{table}
\begin{center}
\begin{tabular}{|lll|}
\hline
  & & \\
  &  & \\
  &  & \\
  &  & \\
  &  & \\
\multicolumn{3}{|l|}{remaining vertices:}\\
~ &  &  \\
~ &  &  \\
~ &  &  \\
~ &  &  \\
\hline
\end{tabular}
\caption{\label{t:coordinates} Coordinates of the centers of the disks in Figure~\ref{f:badgraph_model}. All diameters are equal to~.}
\end{center}
\end{table}

\subsection{Graph-based algorithm} \label{s:graph-alg}

By Lemma~\ref{l:irreducible}, an independent dominating set with no reducible coronas is a -approximation to the minimum dominating set. We now describe how to obtain such a set in linear time given the adjacency list representation of the graph.

We can easily compute a maximal independent set , which is a -approx\-i\-ma\-tion to the minimum dominating set~\cite{heuristics}, in  time. An independent dominating set with no reducible coronas can then be obtained by iteratively performing reductions. However, naively performing such reductions leads to a running time of , since (i) there are  candidate cores for a reducible corona, (ii) detecting whether a vertex  is in fact the core of a reducible corona by inspecting the -neighborhood of  takes  time, and (iii) we may need to reduce a total of  coronas. Fortunately, the following algorithm modifies the set  and returns an independent dominating set with no reducible coronas in  time.

\begin{enumerate}[(1)]
\item For each vertex , compute .

\item For each vertex , if , add  to the list of coronas  (unless it is already there).

\item Let . For each corona , if there is a vertex  such that  is a dominating set, then add  to the set .

\item Choose a maximal subset  of  such that the pairwise distance of the vertices in  is at least .

\item For each vertex , perform a reduction .

\item Repeat all the steps above until .
\end{enumerate}

The algorithm is correct since all changes made to  along its execution preserve the property that  is an independent dominating set. Notice that, in step (4), we only reduce coronas that are sufficiently far from each other, in order to guarantee that we do not reduce a corona that may have ceased to be reducible due to a previous reduction. Moreover, the algorithm always terminates because the size of  decreases at every iteration, except for the last one. 

Next, we show that the running time is . 
Step (1) can be easily implemented to run in  time. To execute step (2) in  time, we must determine in constant time whether a corona is already in the list . This can be achieved by indexing each corona  by an arbitrary vertex  (say, the one with the lowest index), and by storing with  a list of coronas that are in  and whose index is . Note that, because of the packing constraints inherent to unit disk graphs, the number of coronas that contain a given vertex is constant.

Step (3) can be implemented as follows (for each corona ):

\begin{enumerate}[(3a)]
\item Let  be the union of the open neighborhoods of the  petals of .

\item Let  be the set of vertices dominated only be vertices of , i.e.,  contains every vertex  such that .

\item Let  be the intersection of the closed neighborhoods  of all .

\item If , then add an arbitrary vertex of  to the set .
\end{enumerate}

The steps above take  total time when executed for all coronas , because the number of coronas that contain or are adjacent to a given vertex is also constant by packing constraints.

It is easy to perform steps (4) and (5) in linear time. It remains to show that the whole process is only repeated for a constant number of iterations. Let  denote the set of reducible coronas at each iteration of the algorithm with . Note that the reductions performed in step (5) never create a new reducible corona. Therefore . Let  denote a corona that was reduced in the last iteration . If  was not reduced during a previous iteration , then another corona within distance  from  was reduced at that very iteration . Since, again by packing constraints, the maximum number of coronas within constant distance from  is itself a constant, we have .

The following theorem summarizes the result from Section~\ref{s:graph-alg}.

\begin{thm} \label{thm:graph-alg}
Given the adjacency list representation of a unit disk graph with  vertices and  edges, it is possible to find a -approximation to the minimum dominating set problem in  time.
\end{thm}


\subsection{Geometric algorithm} \label{s:geo-alg}

In this section, we describe how to obtain an independent dominating set with no reducible corona in  time given the geometric representation of the graph. The input is therefore a set  of  points. Without loss of generality, we assume that the corresponding unit disk graph is connected (otherwise, we can compute the connected components in  time using a Delaunay triangulation~\cite{cg}). We use terms related to vertices of the graph and to the corresponding points interchangeably. For example, we say a set of points is independent if all pairwise distances are greater than .

We want the points of  to be structured in a suitable fashion. Thus, as a preliminary step, we sort the points by -coordinates and by -coordinates separately (such orderings will also be useful later on), and we partition the points of  according to an infinite grid with unitary square cells by performing two sweeps on the sorted points. Without loss of generality, we assume that no point lies on the boundary of a grid cell. Given , let  denote the grid cell that contains~. We refer to the set of at most  non-empty grid cells surrounding a cell  as the \emph{open vicinity} of , denoted , and to the union of  and its open vicinity as the \emph{closed vicinity} of , denoted . Note that a point  can only be adjacent to points in the closed vicinity of , that is, . Each point  stores a pointer to its containing cell . Also, each cell stores the list of points it contains and pointers to the cells in its open vicinity. 
Since the graph is connected, the diameter of the point set is at most , and thus this whole step can be done in  time.

We are now able to show how to compute a maximal independent set  efficiently. We begin by making a copy  of , and by letting . Then we repeat the two following steps while set  is non-empty. (i)~Choose an arbitrary point  and add it to set . (ii)~For each point  in the closed vicinity of , remove  from  if . When  becomes empty,  is an independent dominating set.
This process takes  time due to the two following facts. First, a cell belongs to the closed vicinity of a constant number of cells. Second, the maximum number of points inside a cell with pairwise distances greater than  is also a constant.

We now have that  is a maximal independent set, and therefore \linebreak a -approximation to the minimum dominating set. Next, we show how to modify  in order to produce an independent dominating set with no reducible corona, therefore a -approximation to the minimum dominating set. The algorithm mirrors the one in Section~\ref{s:graph-alg}, but each step takes no more than   time using the geometric representation of the graph.

Since  is an independent set and a grid cell  has side , a simple packing argument shows that . We store the set  in the corresponding cell . In order to compute , it suffices to inspect the at most  points in  for . We can then build a list of coronas in  time (steps (1) and (2) of Section~\ref{s:graph-alg}).

To perform step (3), we need to find out whether there is a core  such that  is a dominating set, for each corona . First, we make  the union of  for . Then, we let  be the subset of  containing only the points  with . These first two steps are similar to steps (3a) and (3b) in Section~\ref{s:graph-alg}. The remaining sub-steps of step (3) are significantly different, though.

We proceed by making . We need to determine whether there is a point  that is adjacent to all points in . For each , let  denote the disk of radius  centered at . Let  denote the convex region defined by the intersection of  for all . A point  is adjacent to all points in  if and only if . We can compute the region  in  time using divide-and-conquer in a manner analogous to half-plane intersection~\cite{cg}. We can then test whether each point  belongs to the region  in logarithmic time using binary search (remember the points were previously sorted). If there is at least one point , then we add  to the set . Therefore, the whole step (3) takes  time.

In step (4) of the geometric algorithm, we choose an alternative set  which can be computed in  time as follows. For each , we add  to  and then remove from  all points that are contained in the cells within Euclidean distance at most  of . Since by packing constraints there are  points in the intersection of  and the closed vicinity of a cell, we can easily perform step (5) in  time. Finally, the number of repetitions triggered by step (6) is constant by an argument identical to the one given for the graph-based algorithm.

The following theorem summarizes the result from Section~\ref{s:geo-alg}.

\begin{thm} \label{thm:geo-alg}
Given a set of  points representing a unit disk graph, it is possible to find a -approximation to the minimum dominating set problem in  time in the Real RAM model of computation.
\end{thm}

\section{Achieving a 43/9-approximation} \label{s:partial}

In the previous section, a 44/9-approximation was obtained by reducing coronas of a maximal independent set  of graph , that is, by subsequently replacing  petals with  core in  as long as that operation preserved dominance. A natural step to tighten the approximation factor is to allow for \emph{weak reductions}, whereby the  petals of a corona  are removed from the independent dominating set , yet not only is a core  of  inserted into  but also some mutually non-adjacent witnesses of , as long as their number is no greater than  and they dominate all witnesses of . By doing so, the weak reduction of  (with respect to ) preserves dominance and still shaves off at least one unit from the size of . If such operation is possible on a corona , then  is said to be \emph{weakly reducible}.
A core  which has  (or more) mutually non-adjacent witnesses is said to be an \emph{overwhelmed} core.\footnote{If  is an overwhelmed core of a corona , then it might be the case that a weak reduction on  with respect to  is still possible. If the subgraph  induced by the set  of witnesses of  admits an independent dominating set  of size no greater than , then  is still an independent dominating set of , and its cardinality is strictly less than . However, one cannot decide in (close to) linear time whether such a dominating set exists, and that will not be required by our algorithm.}

We consider the graph-based algorithm presented in Section~\ref{s:graph-alg} with some modifications to cope with weak reductions. The whole modified algorithm becomes:

\begin{enumerate}[(1)]
\item For each vertex , compute .

\item For each vertex , if , add  to the list of coronas  (unless it is already there), and add  to the list  containing the cores of .

\item Let  be an initially empty mapping of cores onto sets of witnesses. For each corona , if there is a vertex  and an independent set  with at most  witnesses of  such that  is a dominating set, then add  to . 

\item Choose a maximal subset  of the cores in  such that the pairwise distance of the vertices in  is at least .

\item For each vertex , perform a (weak) reduction .

\item Repeat all the steps above until .
\end{enumerate}


The new step (3) can be implemented as follows (for each corona ):

\begin{enumerate}[(3a)]
\item Let  be the union of the open neighborhoods of the  petals of .

\item Let  be the set of all vertices  with .

\item For each core , greedily obtain a maximal independent set  of . If , add  to  and break.
\end{enumerate}

Because each core is evaluated separately as to whether it allows for a weak reduction (whereas in the algorithm of Section~\ref{s:graph-alg} a constant number of set operations per corona was executed), step (3c) dominates the complexity of the whole sequence of steps (1) to (5), with a total  time for running on all coronas . Another important difference, as far as time complexity goes, is that in this modified algorithm the number of iterations of the main loop --- steps (1) to (5) --- is no longer , due to the fact that new (weakly) reducible coronas can be created, as illustrated in Figure~\ref{f:newcoronas}. However, the number of iterations is certainly , because the size \linebreak of  decreases by at least  in each iteration. Hence the overall time complexity of the modified algorithm is .

Next, we establish the approximation factor of the modified algorithm. Note that step (3c) asserts that the coronas that are not (weakly) reduced by the algorithm present only overwhelmed cores.

\begin{lem} \label{l:irreducible_partial}
Let  be a unit disk graph,  an independent dominating set in , and  a minimum dominating set of . If all coronas \linebreak in  have only overwhelmed cores, then .
\end{lem}

\begin{pf}
We follow a strategy similar to that in the proof of Lemma~\ref{l:irreducible} (also using the concept of reliever defined therein): by employing the same function
  defined in (\ref{eq:fv}),
we prove there is an appropriate balance among vertices
with high () and low () images under , thus yielding an average value for  that is no greater than  --- the claimed approximation factor.

\begin{figure}
 \centering
 \includegraphics[scale = .82]{cria-graph.eps}
 \caption{\label{f:newcoronas}Example of graph with a maximal independent set (represented by solid circles) containing only  weakly reducible corona. After every weak reduction performed by the modified algorithm, except for the last one, a new weakly reducible corona is created.}
\end{figure}

Let  be a vertex in  with . Because , and all (at most ) terms in the summation that yields  are either  or no greater than , we have that  implies . So  is a core.
By hypothesis,  is overwhelmed. Hence,  possesses at least  mutually non-adjacent witnesses , each one implying the existence of a reliever  with . Surely, such relievers need not be all distinct, and different cores may share a common reliever.
Still, geometric properties of unit disk graphs allow us to derive upper bounds for the core-to-reliever ratio. We can thus obtain an upper bound to the average value of  in , and consequently to the approximation factor of the algorithm.

Suppose there are  cores  such that . For , we let  be the corona having  as a core,  be a set of (at least 4) mutually non-adjacent witnesses of , and  be the set of relievers of . Now, we construct a bipartite multigraph  as follows. The parts of  are  and . The multiset  contains, between each core  and reliever , a number  of parallel edges that is equal to the number of petals of  adjacent to witnesses  \linebreak (of ) such that  is a neighbor of .
Thus, the total number of edges incident to a core  is



Analogously,
the total number of edges incident to a reliever  is



We now obtain an upper bound  for the average value of  \linebreak over . Observe that  contains all vertices  of  such \linebreak that , hence the average value  of  over the whole set  cannot be any greater.

Of course the average value we are interested in depends on the core-to-reliever ratio  in : the more cores (respectively, the fewer relievers) in  (in ), the greater the average. Therefore, in order to obtain the desired upper bound , we must consider the case in which the elements of  (respectively, of ) have degrees in  that are as low (as high) as possible.

First, notice that if  is a core of corona  and , then it is not possible that more than two non-adjacent witnesses of  sharing a common reliever  are adjacent to the same petal . Otherwise, let  and  be such witnesses. Since  and  are non-adjacent (due to the image of  under  being ), the subgraph of  induced by  is a . This is a contradiction, because the graph  is not a unit disk graph~\cite{vanLeeuwen}. Thus, since  has at least four witnesses, the number of petals of  adjacent to witnesses of  --- and therefore the degree of each core in  --- is at least . 

The above lower bound can be improved for cores  having a \mbox{reliever } with exactly four neighbors in . If a reliever  has four neighbors in (the independent set) , then the remaining neighbors of  form a clique. Since there is a witness  of  among these remaining neighbors of , then the other (at least) three mutually non-adjacent witnesses of  must have relievers distinct from . Moreover, by an argument analogous to the one used in the previous paragraph, those witnesses must be adjacent to at least  petals of the corona having  as a core. This means there are at least two edges connecting  to its relievers in , plus one edge connecting  to . Hence, the degree \linebreak of  in  is at least .

The maximum degrees of vertices  depend on the number  of neighbors of  in , and now we employ some of the geometric lemmas of Section~\ref{s:forbidden} to infer suitable upper bounds. Recall that any set of petals is an independent set.
\begin{itemize}
\item If  (implying ), then, by Lemma~\ref{l:4l-pendant}, there are at most  petals in the -neighborhood of . Hence, .
\item If  (implying ), then, by Lemma~\ref{l:3l-pendant}, there are at most  petals in the -neighborhood of . Hence, .
\item If  (implying ), then, by Lemma~\ref{l:kl-pendant}, there are at most  petals in the -neighborhood of . Hence, .
\item If  (implying ), then, by Lemma~\ref{l:kl-pendant}, there are at most  petals in the -neighborhood of . Hence, .
\end{itemize}

For , let  denote the number of relievers in  containing exactly  neighbors in , so that . Let also  be the set of cores having at least one reliever with exactly  neighbors in ,\linebreak and  be the set of cores whose relievers have at most three neighbors in . Finally, we let  and , so that .

Since  is bipartite, the number of edges incident to  and the number of edges incident to  are the same. Consequently, 

 and


The same reasoning holds for the subgraph of  induced by the cores \linebreak in  and their relievers in , so

By dividing both sides of (\ref{eq:3}) by  and adding it to (\ref{eq:2}), we obtain

As for the average  of  over the elements of , we can write



Now, substituting  in the expression above by the upper bound obtained from (\ref{eq:4}), we have



Thus,  is bounded by a multivariate rational function whose maximum is , achieved when . 
\qed


\end{pf}

The following theorem summarizes the result from Section~\ref{s:partial}.

\begin{thm} \label{thm:graph-alg-partial}
Given the adjacency list representation of a unit disk graph with  vertices and  edges, it is possible to find a -approximation to the minimum dominating set problem in  time.
\end{thm}


\section{Conclusion and open problems} \label{s:conclusion}

We introduced novel efficient algorithms for approximating the minimum dominating set and minimum independent dominating set in unit disk graphs.

On one hand, a linear-time algorithm was devised attaining a sub-5 approximation factor, namely  Nevertheless, the best lower bound we know for the proposed algorithm is , which corresponds to the unit disk graph given in Figure~\ref{f:badgraph}.
Closing this gap would likely require the development of new tools to prove that certain graphs are not unit disk graphs, for which computer generated proofs may be useful.

\begin{figure}[t!]
 \centering
 \includegraphics[scale = .25]{lowerbound425_disks_nonames.pdf}
 \caption{\label{f:lowerbound425}Lower bound of  to the approximation factor of the modified graph-based algorithm. The coordinates of the centers are given in Table~\ref{t:coordinates2}.}  
\bigskip
~
\end{figure}

\begin{table}[h!]
\begin{center}
\begin{tabular}{|lll|}
\hline
~ &  & \\
~ &  &  \\
~ &  &  \\
~ &  &  \\
~ &  &  \\
~ &  &  \\
~ &  &  \\
~ &  &  \\
~ &  & \\
\hline
\end{tabular}
\caption{\label{t:coordinates2} Coordinates of the centers of the disks in Figure~\ref{f:lowerbound425}. All diameters are equal to~.}
\end{center}
\end{table}

On the other hand, an enhanced approximation factor of  was obtained by allowing for more local replacements, yet a lower bound of~, corresponding to the unit disk graph given in Figure~\ref{f:lowerbound425} (with coordinates in Table~\ref{t:coordinates2}), is the best we are aware of.  
Notwithstanding the  factor increase on its time complexity,
such a modified algorithm is still much faster than, say, the state-of-the-art 4-approximation algorithm from~\cite{cccg}, which requires a geometric model as input and runs in  time. Moreover, since the number of (weak) reductions that are performed remains linear, it may be possible to conceive either a refined analysis or a smarter implementation.

\bibliographystyle{elsarticle-num}


\begin{thebibliography}{10}
\expandafter\ifx\csname url\endcsname\relax
  \def\url#1{\texttt{#1}}\fi
\expandafter\ifx\csname urlprefix\endcsname\relax\def\urlprefix{URL }\fi
\expandafter\ifx\csname href\endcsname\relax
  \def\href#1#2{#2} \def\path#1{#1}\fi

\bibitem{bentley}
J.L. Bentley, D.F. Stanat, E. Hollins Williams Jr.,
The complexity of finding fixed-radius near neighbors,
Information Processing Letters 6 (6) (1977) 209--212.

\bibitem{cg}
M. de~Berg, O. Cheong, M. van Kreveld, M. Overmars, Computational Geometry:
  Algorithms and Applications, Springer, 2010.

\bibitem{breu}
H. Breu, D.G. Kirkpatrick, Unit disk graph recognition is NP-hard,
Computational Geometry 9~(1--2) (1998) 3--24.

\bibitem{udg}
B. N. Clark, C.J. Colbourn, D.S. Johnson, Unit disk graphs, Discrete
  Mathematics 86~(1--3) (1990) 165--177.

\bibitem{maxflow}
P. Christiano, J.A. Kelner, A. Madry, D.A. Spielman, S.-H. Teng, Electrical
  flows, {L}aplacian systems, and faster approximation of maximum flow in
  undirected graphs, in: Proc. 43rd annual ACM Symp. on Theory of Computing
  (STOC), 2011, pp. 273--282.


\bibitem{cccg}
M. De, G. Das, S. Nandy, Approximation algorithms for the discrete piercing set
  problem for unit disk, in: Proc. 23rd Canadian Conference on Computational
  Geometry (CCCG), 2011, pp. 375--380.


\bibitem{erlebach10}
T. Erlebach, M. Mihal\'{a}k, A ()-approximation for the
  minimum-weight dominating set problem in unit disk graphs, in: Proc. 7th
  Workshop on Approximation and Online Algorithms, Vol. 5893 of Lecture Notes
  in Computer Science, 2010, pp. 135--146.


\bibitem{density}
L. Fejes T\'oth, Lagerungen in der Ebene, auf der Kugel und im Raum,
  Springer-Verlag, 1953.

\bibitem{Fodor}
F. Fodor, The densest packing of 13 congruent circles in a circle,
  Contributions to Algebra and Geometry 44 (2) (2003) 431--440.


\bibitem{constants}
S.R. Frinch, Mathematical Constants, no.~94 in Encyclopedia of Mathematics and
  its Applications, Cambridge, 2003.

\bibitem{esa-Gibson}
M. Gibson, I. Pirwani, Algorithms for dominating set in disk graphs: Breaking
  the  barrier, in: Proc. 18th Annual European Symposium on Algorithms
  (ESA), Vol. 6346 of Lecture Notes in Computer Science, 2010, pp. 243--254.

\bibitem{ptas-geometric}
H.B. {Hunt III}, M.V. Marathe, V. Radhakrishnan, S. Ravi, D.J. Rosenkrantz,
  R.E. Stearns, {NC}-approximation schemes for {NP}- and {PSPACE}-hard
  problems for geometric graphs, Journal of Algorithms 26 (1998) 238--274.


\bibitem{ids-ptas}
J.L. Hurink, T. Nieberg, Approximating minimum independent dominating sets in
  wireless networks, Information Processing Letters 109~(2) (2008) 155--160.


\bibitem{vanLeeuwen}
E.J. van Leeuwen, Approximation algorithms for unit disk graphs, Tech. Rep.
  UU-CS-2004-066, Institute of Information and Computing Sciences, Utrecht
  University (2004).
  
\bibitem{heuristics}
M.V. Marathe, H.Breu, H.B. {Hunt III}, S.S. Ravi, D.J. Rosenkrantz, Simple
  heuristics for unit disk graphs, Networks 25~(2) (1995) 59--68.

\bibitem{marx}
D. Marx, Parameterized complexity of independence and domination on
geometric graphs,
in: Proc. 2nd International Workshop on Parameterized and Exact
Computation (IWPEC),
Vol. 4169 of  Lecture Notes in Computer Science, 2006, pp. 154--165.


\bibitem{integer}
C. McDiarmid, T. Muller, Integer realizations of disk and segment graphs,
  preprint arXiv:1111.2931.


\bibitem{nieberg}
T. Nieberg, Independent and dominating sets in wireless communication graphs,
  Ph.D. thesis, University of Twente (2006).
  
\bibitem{ptas-graph-journal}
T. Nieberg, J. Hurink, W. Kern, Approximation schemes for wireless networks,
  ACM Transactions on Algorithms 4~(4) (2008) 49:1--49:17.


\bibitem{pal}
J. P\'al, Ein minimumprobleme f\"ur ovale, Math. Annalen 83 (1921) 311--319.



\bibitem{spinrad}
J. Spinrad, Efficient Graph Representations, Fields Inst.~monographs, AMS,
  2003.

\bibitem{matchings}
D.E.D. Vinkemeier, S. Hougardy, A linear-time approximation algorithm for
  weighted matchings in graphs, ACM Transactions on Algorithms 1 (2005)
  107--122.

\bibitem{zou11}
F. Zou, Y. Wang, X.-H. Xu, X. Li, H. Du, P. Wan, W. Wu, New approximations for
  minimum-weighted dominating sets and minimum-weighted connected dominating
  sets on unit disk graphs, Theoretical Computer Science 412~(3) (2011)
  198--208.


\end{thebibliography}




\end{document}
