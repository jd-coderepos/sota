

\documentclass{llncs}
\usepackage{amsmath}
\usepackage{graphicx}
\usepackage{subfigure}
\usepackage{ifthen}
\usepackage{enumitem}
\setlist{nolistsep} 

\def\MST{\mbox{MST}}\def\OPT{\mbox{OPT}}\def\Left{\mbox{left}}\def\In#1{\mbox{in}} \def\Out#1{\mbox{out}} \def\Net#1{\mbox{net}} \def\xeP{x_{(e,P)}} 

\title{Light Spanners in Bounded Pathwidth Graphs}
\author{Michelangelo Grigni \and  Hao-Hsiang Hung}
\institute{Dept.\ of Math \&\ CS, Emory University,
\textsl{\{mic,hhung2\}@mathcs.emory.edu}}

\begin{document}
\maketitle



\begin{abstract}

Given an edge-weighted graph  and , a
-spanner is a spanning subgraph  whose shortest path
distances approximate those of  within a factor of .  
For  from certain graph families (such as bounded genus graphs and apex
graphs), we know that \emph{light} spanners exist.  That is, we can
compute a -spanner  with total edge weight at most a
constant times the weight of a minimum spanning tree.  This constant
may depend on  and the graph family, but not on the
particular graph  nor on the edge weighting.  The existence of
light spanners is essential in the design of approximation schemes for
the metric TSP (the traveling salesman problem) and similar
graph-metric problems.

In this paper we make some progress towards the conjecture that light
spanners exist for every minor-closed graph family: we show that light
spanners exist for graphs with bounded pathwidth, and they are computed
by a greedy algorithm.  We do this via the intermediate construction of
light \emph{monotone} spanning trees in such graphs.
\end{abstract}


\section{Introduction}


\subsection{Light Spanners}

Suppose  is a connected undirected graph where each edge  has
length (or weight) .  Let  denote the length
of the shortest path between vertices  and~.  Suppose  is a
spanning subgraph of , where each edge of  inherits its weight
from ; evidently .  Fix .
If  (for all ),
then we say that  is a \emph{-spanner} of .  In
other words, the metric  closely approximates the
metric~.

Let  denote the total edge weight of , and let 
denote the minimum weight of a spanning tree in .  We are
interested in conditions on  that guarantee the existence of a
-spanner  with bounded .
Suppose  is a family of undirected graphs.  We
say \emph{ has light spanners} if the following
holds: for every  there is a bound , so that
for any edge-weighted  from ,  has a
-spanner  with .
Less formally, we say that  is a \emph{light spanner} for .
Note  depends on  and , but not on
 or~.

We know that if a graph family has unbounded clique minors, then it
does not have light spanners; just consider a clique with uniform edge
weights.  We conjecture the
converse~\cite{Grigni:2000:ATG:646253.686316}:
\begin{conjecture}\label{conj:main}
Any graph family with a forbidden minor has light spanners.
\end{conjecture}

Our pursuit of this conjecture is guided by the Robertson-Seymour
theory \cite{DBLP:journals/jct/RobertsonS03a}, which characterizes 
minor-closed graph families using four elements: bounded genus graphs,
apices, vortices, and repeated clique-sums.  We already know that if
 has bounded genus or is an apex graph, then it has light
spanners~\cite
{Grigni:2000:ATG:646253.686316,Grigni:2002:LSA:545381.545492}.
In particular vortices are bounded pathwidth subgraphs, stitched inside
the faces of a bounded genus graph.


\subsection{Motivation}

Conjecture~\ref{conj:main} seems like a natural question, and its
proof would address the ``main difficulty'' discussed in the
concluding remarks of Demaine \emph{et
al.}~\cite{Demaine:2007:AAV:1283383.1283413}, in the general context
of approximation algorithms on weighted graphs.  As a specific
motivating problem, we review some results on the \emph{metric TSP},
the Traveling Salesman Problem with triangle inequality.  (For some
other problems, see
\cite{Berger05approximationschemes,DBLP:conf/icalp/BergerG07}.)

 We are given an edge-weighted graph , and we seek a cyclic
order of its vertices with minimum total distance as measured by
.  Equivalently, we want a minimum weight cyclic tour in 
visiting each vertex at least once.  Let  denote the minimum
tour weight; it is well known that .  We seek an approximation scheme: an algorithm which
takes as inputs the weighted graph  and , and which
outputs a tour with weight at most .

The problem is MAX SNP-hard~\cite{PY-tsp12-93}, so we consider
approximation schemes where the input graph  is restricted to some
graph family  (e.g., planar graphs).  We would like a
PTAS (an approximation scheme running in time ,
for some function ), or better yet an EPTAS (an approximation
scheme running in time , where the constant
 is independent of ).

Suppose  is a graph family, and that for any  we can compute a -spanner  with .  Then we may attempt to design a PTAS
(or an EPTAS) for the metric TSP on , as follows:
\begin{enumerate}

\item On input  and , first compute , a
  -spanner of , with weight at most
  .

\item Choose .
Apply some algorithm finding a tour in  with cost at most
.

\item Return the tour, with cost at most .  (For other metric optimization problems,
it may be less trivial to lift a solution from  back to .)

\end{enumerate}
Step 2 looks like the original problem, except now we allow an error term
proportional to  instead of .   This
approach has already succeeded for planar graphs~\cite
{DBLP:conf/soda/AroraGKKW98,KleinTSP2005} and bounded genus
graphs~\cite
{Demaine:2007:AAV:1283383.1283413,Grigni:2000:ATG:646253.686316}.

A recent result of Demaine \textit{et
al.}~\cite[Thm.~2]{ContractionMinorFree_STOC2011} implies a PTAS for
metric TSP when  is \emph{any} graph class with a fixed
forbidden minor.  Since we do not know that  has light
spanners, for step~1 they substitute a looser result~\cite
{Grigni:2002:LSA:545381.545492}, finding a -spanner 
with weight  (the hidden constant
depending on ).  In step~2 their algorithm runs in time
. Their  is
, so their
running time is .  If we could compute light
spanners for , then  would improve to something
independent of , and this would yield an EPTAS for metric TSP on
. (Or alternatively, it would yield an approximation
scheme allowing  to slowly approach zero, as long as
 stays .)

\subsection{Our Work}

In this paper we make some progress towards
Conjecture~\ref{conj:main}: we show that light spanners exist for
bounded pathwidth graphs.
\begin{theorem}\label{thm:main}
Bounded pathwidth graphs have light spanners, computable by a greedy
algorithm.
\end{theorem}
We prove this in Section~\ref{sec:mainarg}.
This result is not algorithmically interesting by itself, since metric
TSP (and many other problems) is exactly solvable in polynomial time
when  has bounded pathwidth, or even bounded treewidth.  Rather, we
regard this as progress towards the conjecture, and towards an EPTAS
for metric TSP (and similar problems) on graphs with forbidden minors.
See Section~\ref{sec:further} for some further remarks.

\section{Preliminaries}

\subsection{Charging Schemes}

In order to exhibit light spanners in a weight-independent way, we use
\emph{charging schemes}~\cite{Grigni:2002:LSA:545381.545492}.  (We use
the notion called ``0-schemes'' in
\cite{Grigni:2002:LSA:545381.545492}, not the more general
``-schemes'' required for apex graphs.)  Suppose each edge
of graph  can hold some quantity of \emph{charge}, initially zero.
A \emph{detour} is an edge  and a path  such that  is
a simple cycle in .  For each detour  we introduce a
variable .  Each  describes a
\emph{charging move}: it subtracts  units of charge from
edge , and adds  units of charge to each edge of .
When , we say `` charges ''.


Given graph , a spanning tree , and a number , a
\emph{charging scheme from  to  of value } is an assignment
of nonnegative values to the  variables (i.e., a fractional
sum of detours) meeting the three conditions listed below.  Here 
 denotes the total charge subtracted from edge ,
 denotes the total charge added to  (as part of various detour
paths),
and  is the total charge on  after all
the moves are done:

Note ``'' means  is an edge of  but not .
As we'll see in Theorem~\ref{thm:greedy}, charging schemes imply light
spanners.
\begin{definition} 
An \emph{acyclic scheme} is a charging scheme with two additional
conditions:
\begin{description}
\item[\textnormal{(4)}] If edge  charges some path, then .
\item[\textnormal{(5)}] There is an ordering of the edges such that whenever
edge  charges a path containing edge~,  precedes~.
\end{description}
\end{definition}
For example, planar graphs have integral acyclic
schemes of value ~\cite{Althofer:1993:SSW:156252.156258}.
\begin{definition}
Suppose we have detours  and , with 
and .  Their \emph{shortcut} is the detour ,
where  is the path derived from  by replacing  with
, and then reducing that walk to a simple path.
\end{definition}

\begin{lemma}\label{lem:shortcut} 
Suppose we have an acyclic scheme of value  from  to , and
an edge  in .  Then there is an acyclic scheme of value 
from  to .
\end{lemma}

\begin{proof}
Let .  While  is positive, we find some  charging
a path  containing .  Since ,  also charges some path .  
cannot contain , since the scheme is acyclic.  Let
.  Now reduce both
 and  by , and increase
 (their shortcut) by .  After this change all the
conditions are still satisfied, except possibly for condition (1) at .
Repeat until  reaches zero.  Finally remove  and any
remaining charges out of~.  \qed
\end{proof}

\begin{theorem}
\label{thm:greedy}
Suppose  is a graph with spanning tree , and we have an acyclic
scheme from  to  of value .  Then for any , and
for any non-negative edge-weighting  on , a simple greedy
algorithm finds a -spanner  in  containing ,
with total weight .
\end{theorem}
We use the following greedy algorithm of Alth\"{o}fer 
\emph{et al.}~\cite{Althofer:1993:SSW:156252.156258},
modified to force the edges of  into~:
\begin{quote}
\begin{tabbing}
xxx \= xxx \= xxx \= \kill
Spanner(, , ): \\
\>  \\
\> for each edge , in non-decreasing  order \\
\>\> if  then \\
\>\>\> add edge  to \\
\>return 
\end{tabbing}
\end{quote}
The proof of Theorem~\ref{thm:greedy} is a variant of previous 
arguments by LP duality~\cite
{Grigni:2000:ATG:646253.686316,Grigni:2002:LSA:545381.545492},
for completeness we sketch it here.
\begin{proof}
Since  is computed by the greedy algorithm, it is clearly a
-spanner of  containing ; the issue is to bound
its weight .  
By Lemma~\ref{lem:shortcut} we have an acyclic scheme from  to
 of value~.  

Consider a detour  in  with . We claim  (to see this, compare 
with the last edge inserted by the algorithm on the cycle ). 
Multiply through by  and we have this:

When  this is still valid, since .  Now sum over all
detours :

So .
\qed
\end{proof}


\subsection{Bounded Pathwidth and Monotone Trees}
\label{def:bpmt}
Suppose  is a graph,  is a path (disjoint from ), and
 is a collection of subsets of  (bags) indexed
by vertices  in .  We call the pair  a
\emph{path decomposition} of  if the following conditions hold: (1)
; (2) for every edge , there
is at least one bag  with ; (3) for
every ,  is connected
(an interval) in .  The \emph{pathwidth} of the decomposition is
the maximum bag size minus one, and the pathwidth of  is the
minimum pathwidth of any path decomposition of .

Given , we may lay out  on the line, and regard
 as a subgraph of an interval graph.  That is, for each vertex 
we have a line interval  (corresponding to an interval in ),
and we have  whenever ,
and at most  intervals overlap at any point of the line.  For
convenience we may eliminate ties via small perturbation, so that all the interval endpoints
are distinct.  In particular, let  denote the leftmost point
of .   



Suppose  is a rooted tree in .  We say  is a \emph{monotone tree} if
for every vertex  in  with parent , we have .  When  is a path rooted at an endpoint, we say it is a
\emph{monotone path}.  In particular if  is a monotone spanning
tree in , then from any vertex , we can find a monotone path in
 from  to the root of  (the vertex with the leftmost
interval).  For this process, it is convenient to imagine that edges
connect intervals at their leftmost intersection point.

\section{Main Argument}
\label{sec:mainarg}

We are given , a connected edge-weighted graph  with
 vertices, and an interval representation  of  with
pathwidth .  We want to find a -spanner  in 
of low weight.  First we apply some reductions to simplify :


\emph{Nice Decomposition}. We may assume that each pair of consecutive bags (as vertex
sets) differ by only one vertex.  This can be enforced by an argument
similar to the construction of \emph{nice} tree-decompositions
\cite{DBLP:books/sp/Kloks94}: if two consecutive bags differ on  vertices, we introduce  intermediate bags, in such a way that
each pair differs on only one vertex, and we do not increase the
maximum bag size. This does not modify  at all.

\emph{Bounded Degree Assumption}. We may assume each vertex appears in  bags, and so the
maxdegree of  is .  To enforce this, we copy the bags of 
from left to right.  After each group of  original bags, ending
with a bag , we insert  ``replacer'' bags, each of which
replaces one vertex  with a copy , connected to 
by an edge of length zero. This ends with a bag , where every
vertex  has been replaced by a copy .
See Figure~\ref{fig:nicepath}.
We continue in this way (using the copies in place of the originals)
across the entire path decomposition.  If we aren't careful the
pathwidth may increase by one, but this does not matter for our
asymptotic results.  The original graph is obtained by contracting a
set  of weight-zero edges in the modified graph.  So given a
spanner  in this modified graph, we may contract  in 
to recover a spanner (of no greater weight) in the original.

\emph{Completion Assumption}. We may assume that  is \emph{completed}; that is, it contains
all edges allowed by its overlapping intervals.  In other words: we
have a clique in each bag,  is an interval graph.  For each absent
edge , we simply add it with weight  equal to the
shortest path length .  This does not change  at all.
Given a spanner  in the completed graph, we recover a spanner in
the original graph by replacing each completion edge by the
corresponding shortest path.

\begin{figure}[ht]
\begin{center}
{\includegraphics[width=6cm]{nicepath.png}}
\caption{Each vertex  in bag  is replaced by  in bag .}
\label{fig:nicepath}
\end{center}
\end{figure}


\begin{proof}[of Theorem~\ref{thm:main}]
We assume all the above reductions have been applied: the input graph 
is a connected edge-weighted interval graph of width , each bag in its path 
decomposition introduces at most one vertex, and each vertex of  
has degree .

By Lemma~\ref{lem:monoT} (below), we compute a monotone spanning tree
 with .
By Lemma~\ref{lem:T2scheme} (below), we exhibit an acyclic charging
scheme from  to  of value .
Finally we apply the greedy algorithm, which computes
a ()-spanner .
By Theorem~\ref{thm:greedy}, .
\qed
\end{proof}

\begin{lemma}\label{lem:monoT} Given  as above, it contains a monotone
 spanning tree  with . 
\end{lemma}
\begin{proof}
Choose a minimum spanning tree , so .
Let  and  be the leftmost and rightmost intervals.
Let  be a shortest path from  to ; since  is completed,
we may assume  is monotone, as in Figure~\ref{fig:tree1}.
Note .

Consider the components  of .
Let  be an edge connecting the leftmost point of  to a
vertex of  (it exists by completion).
For each , we recursively compute a monotone spanning tree 
of . Finally, .

It is clear that  is monotone, but we must account for the total
weight of .  For each component , let  be an edge
of  connecting  to  (there must be at least one).
By triangle inequality, we see  is at most , where  is a subpath of  from
the endpoint of  to the endpoint of .  Note the 's and
's are disjoint parts of , but the subpaths may overlap inside
.

\begin{figure}[ht]
\begin{center}
{\includegraphics[width=7cm]{tree1.png}}
\caption{ and the  subtrees. Each 
 in  is replaced by an  in .}
\label{fig:tree1}
\end{center}
\end{figure}

An edge  appears in at most  of the
 subpaths, since each subpath witnesses another vertex (from
) that must appear in the bag with~.  So .  Since  and , a simple
depth- recursion finishes our bound.  \qed
\end{proof}

\noindent\emph{Remark:} we do not have to compute  as in
Lemma~\ref{lem:monoT}; it suffices to use any light enough monotone
spanning tree.  A natural choice is to let  be the \emph{lightest}
monotone spanning tree, which we compute as follows.  Start with just
the root (in the leftmost bag), and grow the tree in a left-to-right
scan of the bags: each time a bag  introduces a new vertex , add
an edge connecting  to its nearest neighbor in  (which is
already in~).

In the completed , a \emph{triangle move} is a charging move where
a non-tree edge  charges a path  of length two, where at most
one edge of  is not in .  We now define , a graph whose
edges represent triangle moves.  Each vertex  of 
corresponds to an edge  in .  We also represent the vertex
 by the interval .  To define the edges of
, we first define a \emph{parent} for each vertex .  If
 is an edge of , then  has no parent.  Otherwise,
suppose  (else swap them), and let  be the
parent of  in ;  must exist since  is not the root.  Note
 is a triangle in .  Now we say the parent of  is
, and we add the edge  in .  Note
, so these parent links are acyclic. Thus
 is a forest, with each component rooted at a vertex
corresponding to an edge of .  Figure~\ref{fig:pages} illustrates a
simple monotone tree  and its forest .

\begin{figure}[ht]
\begin{center}
\mbox{
\subfigure[a monotone tree ]{\includegraphics[width=5cm]{tree2.png}}
\subfigure[ produced from ]{\includegraphics[width=5cm]{tree4.png}}
}
\caption{Horizontal lines are intervals, dashed verticals are edges of .}
\label{fig:pages}
\end{center}
\end{figure}

\begin{lemma}\label{lem:T2scheme} Given  as above with a monotone
spanning tree , there is an acyclic 
charging scheme from  to  of value .
\end{lemma}

\begin{figure}[ht]
\begin{center}
\mbox{\subfigure[some edges of T (solid) and G-T (dashed)]{\includegraphics[width=7cm]{tri3.png}}
\subfigure[a shortcut tour in , ending at ac]{\includegraphics[width=7cm]{tri4.png}}}
\caption{Edges of  (left) are vertices of  (right)}
\label{fig:trees}
\end{center}
\end{figure}

\begin{proof}
Recall  is a forest.  Fix a component  of ; it is
a tree, rooted at a vertex  corresponding to an edge of , and
that is the only such vertex in .  Consider a directed Euler tour
of , traversing each edge twice.  Delete each tour edge out of ,
so we get a list of directed paths, each of the form

where each vertex  corresponds to some edge of .  Since 
is a tree, these paths are vertex disjoint (except at ).  However,
a vertex may appear more than once on the same path; call an
appearance  a \emph{repeat} if the same vertex appeared earlier
on the path.  Let  be the collection of all these paths,
from all components of .

We now propose a charging scheme (which fails to be acyclic).  Recall
how we constructed edges in : we connect each vertex 
(corresponding to an edge of ) to its parent .  If a path in
 traverses this edge in the direction , we add the triangle move where edge  charges one unit
to path .  If a path traverses this edge in the other
direction  (so  is not a tree edge), we
add the triangle move where edge  charges one unit to path .
In either direction, the tree edge  is charged.

For an edge , the corresponding vertex appears at least
once on a path, and it has at least as many out-edges as in-edges, so
our proposed scheme satisfies conditions (1) and (2).  For an edge , we must bound the number of times it is charged.  Since  has
maxdegree ,  appears in  distinct triangles, and it is
charged at most twice per triangle (this includes the charges it
receives in its role as ).  So if we choose , condition (3) is satisfied.
Also there are no charges out of tree edges, so condition (4) is also satisfied.

However, this charging scheme does not satisfy condition (5); if a vertex
(corresponding to an edge ) has a repeat appearance on its
path, then there is no consistent way to order the edges.  To fix
this, we eliminate all ``repeat'' appearances using shortcuts.
That is, whenever we have a sequence  where  is a repeat, we shortcut out .
Note such shortcuts can be combined.  For example if we have a sequence
,
corresponding to four triangle moves, it is possible to shortcut out
 (in any order), and the result is a single charge from
 to a path containing  (the rest of the charged path is all
tree edges).  After eliminating all repeats by shortcuts, we get the
desired acyclic scheme.
\qed
\end{proof}

\section{Conclusion and Further Work}
\label{sec:further}

Regarding our main result, it is not clear whether we really need
to force the edges of a monotone  in the greedy spanner
computation.  Also, we might hope to reduce the  factor to
something smaller.  

The next obvious target is bounded treewidth graphs, a prerequisite
for handling clique sums as in the Robertson-Seymour characterization.


There are several obvious directions to try extending the current approach to
further minor-closed graph families.  First, as extensions of
Theorem~\ref{thm:main}, we propose two open problems: show light
spanners for a planar graph with a single vortex, and show light
spanners for a path-like clique-sum of planar graphs.  For these cases
it may help to compose multiple charging schemes into an
``-scheme'', as was necessary for apex
graphs~\cite{Grigni:2002:LSA:545381.545492}.  As usual, the main
difficulty is that we have no control over the MST topology; if the MST
has a nice topology (e.g.\ some form of monotonicity),
then we would be done.


Given a bounded treewidth graph, we can still define the notion of a
monotone spanning tree .
We choose roots in the decomposition tree and ; whenever vertex 
has parent  in , we require  to be in the bag containing  which
is closest to the root.
If we can find such a  that is light
enough, then we could repeat the rest of our argument from the bounded
pathwidth case.  However, there is an obstacle: the light monotone
tree might not exist.
\begin{theorem}\label{thm:nomono}
There is an edge-weighted graph  with a bounded treewidth decomposition,
such that any monotone spanning tree  in the completion of 
has weight .
\end{theorem}

\begin{figure}[ht]
\begin{center}
{\includegraphics[width=7cm]{treecoex.png}}
\caption{A bounded treewidth graph with no light monotone tree. 
The solid path  is the minimum spanning tree. The horizontal edge
in each leaf bag has weight one, all other solid edges of  have
weight zero.  All other edges (in particular, the dashed ones) have
weight equal to the distance in  between its endpoints.}
\label{fig:treeex}
\end{center}
\end{figure}
\begin{proof}
We construct  as follows (see Figure~\ref{fig:treeex}).   We start
with a balanced binary tree with  nodes, think of this as our tree
of bags.  We assign 3 nodes to each internal bag, and 2 to each leaf bag.  
We connect these vertices by a path  as shown; each edge of  in a leaf
bag has weight one, all other edges in  have weight zero.  (Note 
we must grow our bags a bit to support all these edges of .)

Now in any monotone tree  for , for each internal bag, the
``bottom'' vertex of the three must be connected to one of the
other two (its parent in ); in other words, we must pick one of the
dashed edges shown in each internal bag.

The main observation is that if we sum up the weights of these
selected edges over one level of the decomposition tree, their total
is already a constant fraction of .  Summing over all levels,
the total weight  is , as claimed.
\qed
\end{proof}




\bibliographystyle{plain}
\bibliography{ref}

\end{document}
