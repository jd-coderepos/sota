\subsection{The case of Layer-3 VPNs}
In layer-3 VPNs, a VPN site has a Customer Edge (CE) router that aggregates its traffic, and the provider network has a set of provider edge (PE) routers that connect the different enterprise CE routers. With respect to the model in Sec.~\ref{model}, the site id is one (or more) IP prefix(es) that address the VPN site network. Each PE router has a set of physical interfaces, each of them connects a VPN customer site.
These physical interfaces correspond to the interface in our reference model. We mainly find, in the literature, two approaches in layer-3 VPN solutions, the virtual router approach~\cite{VR:draft} and the BGP/MPLS IP VPNs one~\cite{rfc:4364}. They both have the same pattern as the generic solution depicted above in Sec.~\ref{solution}.

\subsubsection{BGP MPLS IP VPNs}\label{bgp_mpls}
Fig.~\ref{fig:Image9} illustrates the BGP/MPLS IP VPN approach to the VPN service on top of the provider network shown in Fig.~\ref{fig:interface_matrix}. This approach is built around the four elements depicted above:


	\vspace{1mm}\noindent \textbf{1.~A per interface routing table}: A PE associates by configuration a \emph{virtual routing and forwarding} (VRF) table per interface\footnote{In practice, the same VRF can be associated to more interfaces if they all belong to the same VPN}. The VRF table contains the mapping information (site id, interface locator) necessary for VPN routing. Fig.~\ref{fig:Image9} zooms on $PE_{1}$ and shows the content of $VRF_{1}$ and $VRF_{3}$ tables, corresponding respectively to $I_{1}$ and $I_{3}$. 
Interface locators have two components: one that globally identifies the egress PE, and one that locally identifies the egress interface within the PE. This decomposition in two components is due to the way two interfaces are connected: we will see in point 3. below how this is realized. 


	\vspace{1mm}\noindent \textbf{2.~A method to populate per interface routing tables}: In BGP MPLS VPNs, this is done using multi protocol extensions of BGP (MP-BGP)~\cite{mpbgp-rfc4760}. 
PEs run MP-BGP between each other, and use it for encoding and distributing the mappings in order to populate VRF tables.  
	
\vspace{1mm}\noindent \textbf{2.1.~Associating interfaces to VPNs}: This task consists in tagging similarly interfaces that are part of the same VPN. In BGP MPLS VPNs, each interface is associated by configuration to a VRF, and each VRF is associated by configuration to a VPN customer. This is done by tagging, by configuration, each VRF with one or more numerical tags, called the Route Targets (RT). Two VRFs that are part of the same VPN are tagged with the same Route Targets\footnote{In practice, two different types of Route targets are needed to implement complex VPN topologies (e.g. hub and spoke).}.
		
\vspace{1mm}\noindent \textbf{2.2.~Mappings Distribution}: The mapping (Site IP, PE address or label, interface label) relative to a VPN interface is encoded in a BGP route and is flooded to populate distant VRFs of the same VPN. The PE address is stored in the BGP Next Hop attribute. The site IP together with the internal label are encoded in the BGP NLRI field. Finally, the Route Target (RT) of the VPN interface is appended to the BGP route (encoded as an extended BGP community). This BGP route is then flooded (distributed to all the PEs in the provider backbone). However, only distant VRFs that are tagged with the same Route Target will ultimately install it. 

\vspace{1mm}\noindent \textbf{3.~Connecting an ingress interface with a specific egress interface}: This is done using MPLS thanks to the stacking of two MPLS labels. The traffic that comes from one ingress interface is encapsulated in an MPLS tunnel with two labels that are pushed. The outer label allows the traffic to be routed through the provider network till the egress PE router, whereas the inner label allows the egress PE to identify to which egress interface the traffic should be forwarded. This explains why the interface locator is defined by two labels: one global egress PE identifier, and one egress interface label that is local to the PE. 


\vspace{1mm}\noindent \textbf{4.~Learning site ids behind provider interfaces}: In BGP MPLS VPNs, this is done thanks to the PE-CE routing protocol that runs between the VPN site CE router and the provider network PE router that connects it (can be a link state protocol like ISIS or OSPF, or a protocol like BGP, or even static routing).


\begin{figure}[h]
	\centering
		\includegraphics[scale=0.25]{interface_matrix_full_BGP.eps}
	\caption{A BGP MPLS IP VPN implementation}\label{fig:Image9}
\end{figure}











































\subsubsection{The virtual router approach}
The virtual router approach~\cite{VR:draft} is another proposal that offers the same IP VPN service. Fig.~\ref{fig:vr1} illustrates a Virtual router implementation of the VPN service on top of our simple provider network model. We show how it is also centered around the elements that we depicted in the generic solution of Sec.~\ref{solution}.

\vspace{1mm}\noindent \textbf{1.~A per interface routing table}. The virtual router approach associates a Virtual Router (VR) per interface. A \textit{virtual router} is an emulation of a physical router. It has therefore its same capabilities including a routing table.
	
	\vspace{1mm}\noindent \textbf{2.~A method to populate per interface routing tables}: In the virtual routing approach, the CE routers together with the VRs of the same VPN form a Virtual Network. As such, They can run whatever routing protocol to populate the VR (and therefore the per-interface) routing tables.
	
\vspace{1mm}\noindent \textbf{2.1.~Associating interfaces to VPNs}: In the VR approach, tagging similarly interfaces that are part of the same VPN results in connecting the VRs together to form the VPN topology. The VR approach proposes to use an auto discovery mechanism~\cite{vpn-bgp} (with Route Targets similarly to BGP MPLS VPNs), to allow each virtual router to discover which distant VRs belong to the same VPN. In short, VRs that are part of the same VPN are configured with Route Targets. BGP messages containing VR identifiers (interface locators) and their Route Targets are flooded/exchanged in the provider network so that each VR discovers all the other VRs that are part of its VPN.
		 
		\vspace{1mm}\noindent \textbf{2.2.~Mappings Distribution}: Once VRs discover which distant VRs belong to the same VPN, they use one of the tunneling techniques (in point 3. below) to exchange both routing information and data. VRs of the same VPN form a virtual network and can run any routing protocol between each other to build the mappings between site ids and interface locators (These mappings are the routing entries of the routing table of each VR). 


	\vspace{1mm}\noindent \textbf{3.~Connecting an ingress interface with a specific egress interface}: Connecting two interfaces is done with tunneling like with BGP MPLS VPNs. However, the VR approach did not impose a tunneling technique, which can be IPSec~\cite{ipsec}, IP-in-IP~\cite{ipinip}, GRE~\cite{gre}, MPLS~\cite{MPLS_rfc}, or even a layer-2 connections (e.g. ATM). 


	\vspace{1mm}\noindent \textbf{4.~Learning site ids behind provider interfaces}: The routing protocol that runs between the VR and the CE allows the VR to learn site ids.






\begin{figure}[tb]
	\centering
\includegraphics[scale=0.25]{interface_matrix_VR_noVRB.eps}
		\caption{The virtual router approach (VR-to-VR model)}\label{fig:vr1}
\end{figure} 










\subsection{The case of Layer-2 VPNs}
In layer-2 VPNs~\cite{rfc4664}, similarly to layer 3 VPNs, the provider backbone has a set of PE routers. Each PE has a set of interfaces that connect customer sites. 
There are mainly two types of layer-2 VPNs, virtual private wire service (VPWS) and virtual private LAN service (VPLS).
With VPWS, a provider backbone offers a point-to-point layer-2 connection (e.g to transport ATM or Ethernet) between a pair of sites that need to be interconnected. VPLS is instead a point-to-multipoint layer-2 Ethernet VPN service that emulates a LAN.
They both exhibit the same pattern as in the generic solution. However, due to lack of space, we consider only the VPLS approach.











	






\subsubsection{Virtual Private LAN Service (VPLS)}
Fig.~\ref{fig:vpls} shows a VPLS implementation of the VPN service on the top of the provider network described in Fig.~\ref{fig:interface_matrix}. With respect to our model, site ids are the set of MAC addresses of the site. We show that also VPLS comprise the four elements depicted in our generic solution.
\begin{figure}[h]
	\centering
		\includegraphics[scale=0.25]{interface_matrix_full_VPLS.eps}
		\caption{VPLS implementation}\label{fig:vpls}
\end{figure}

\vspace{1mm}\noindent \textbf{1.~A per interface routing table}. In VPLS, each interface has a Virtual Switching Instance (VSI), a table that contains mappings between site ids (destination MAC addresses) and the corresponding interface locators (PWs) to which the traffic should be forwarded.
	
	\vspace{1mm}\noindent \textbf{2.~A method to populate per interface routing tables}: In VPLS also, this is done in the two steps that we depicted earlier.
	
			\vspace{1mm}\noindent \textbf{2.1.~Associating interfaces to VPNs}: tagging similarly interfaces that are part of the same VPN is done using either LDP~\cite{rfc4762} or BGP~\cite{rfc4762} (two dedicated RFCs were issued for the VPLS case). If BGP is used, this is done exactly similarly to the VR approach in layer 3 VPNs. The result of this step is that each ingress interface is associated with a list of egress interfaces to which it is connected.
			


		\vspace{1mm}\noindent \textbf{2.2.~Mappings Distribution}: In the classical VPLS standard, the mappings are not explicitly distributed as with BGP MPLS VPNs. Instead, each VSI acts as a layer 2 switch and performs MAC learning to associate site ids (MAC addresses) to the corresponding interface locators (PWs).
			In theory, the mappings distribution could be done explicitly, exactly the same way as with BGP MPLS VPNs, by encoding (site ids, interface locators) in BGP routes. This is the purpose of a very recent solution that is currently being developed~\cite{MAC-VPN} in the IETF.
		
	\vspace{1mm}\noindent \textbf{3.~Connecting an ingress interface with a specific egress interface}: This is done similarly to BGP MPLS VPNs with a stacking of two MPLS labels.
	


	\vspace{1mm}\noindent \textbf{4.~Learning site ids behind provider interfaces}: VSIs perform MAC learning to learn site ids (MAC addresses) of the VPN site.




\subsection{The case of Layer-1 VPNs}
The goal of Layer-1 VPNs is to provide point-to-point transport connections between pairs of distant VPN sites. 
The advent of such a solution is a consequence of the automation of layer 1 connections set up. Indeed, GMPLS~\cite{gmpls} proposes to use an IP control plane to remotely control GMPLS capable transport devices and instruct them to fast establish layer-1 connections.   
Work on layer-1 VPNs was carried jointly within both the ITU-T and the IETF. The ITU-T first standardized layer-1 VPN requirements and high level architecture~\cite{l1vpn:itu2,l1vpn:itu} and the IETF worked later on protocol aspects~\cite{rfc5251,bgp:l1}. We now show how also layer 1 VPNs exhibit the same pattern as our generic solution. In layer-1 VPNs, customer sites have ports that connect to the provider backbone through provider ports. The customer port has a Customer Port Identifier (CPI) that is the equivalent of the site id in our generic model. The provider port is equivalent to the provider VPN interface in our model. It can be identified in a unique way within the provider backbone. Each provider port has a provider port identifier (PPI), it is the equivalent to the interface locator in our model.


\vspace{1mm}\noindent \textbf{1.~A per interface routing table}. In layer-1 VPN, a PE stores a port information table (PIT) per VPN port (equivalent to the VRF in BGP/MPLS IP VPNs and to the VSI in layer-2 VPLS). It contains (CPI,PPI) mappings.
	
	\vspace{1mm}\noindent \textbf{2.~A method to populate per interface routing tables}: This is done very similarly to BGP MPLS VPNs using extensions to BGP for layer 1 VPNs~\cite{bgp:l1}.

\vspace{1mm}\noindent \textbf{2.1.~Associating interfaces to VPNs}: tagging similarly interfaces that are part of the same VPN is done using Route Targets as with BGP MPLS VPNs.

		\vspace{1mm}\noindent \textbf{2.2.~Mappings Distribution}: (CPI,PPI) mappings are encoded in BGP routes together with Route Target information exactly as done with BGP MPLS VPNs.

\vspace{1mm}\noindent \textbf{3.~Connecting an ingress interface with a specific egress interface}: Thanks to GMPLS, it is possible, through IP based signaling mechanisms, to automatically build a layer-1 physical link between any two given provider ports. 

	\vspace{1mm}\noindent \textbf{4.~Learning site ids behind provider interfaces}: This is done thanks to the GMPLS IP Control Channel (IPCC) between the provider equipment and the customer equipment.

















%
