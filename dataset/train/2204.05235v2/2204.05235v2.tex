\documentclass{article}
\usepackage{arxiv}
\usepackage[square,sort,comma,numbers]{natbib}
\usepackage[utf8]{inputenc} \usepackage[T1]{fontenc}    \usepackage[shortlabels]{enumitem}
\usepackage[obeyspaces]{url}
\usepackage{wrapfig}
\usepackage{booktabs}       \usepackage{amsfonts}       \usepackage{nicefrac}       \usepackage{microtype}      \usepackage{lipsum}		\usepackage{graphicx}
\usepackage{natbib}
\usepackage{doi}
\usepackage{multirow}
\usepackage{tabularx}
\usepackage{tabulary}
\usepackage{makecell}
\usepackage{tabularx,ragged2e}
\usepackage{fancyhdr, blindtext}
\usepackage{subfigure}
\usepackage{float}
\usepackage{hyperref}
\usepackage{hhline}
\usepackage{array}
\usepackage{amsmath}
\usepackage{diagbox}
\usepackage{color}
\usepackage{textcomp}
\usepackage{comment}
\usepackage{relsize}
\usepackage[symbol]{footmisc}
\usepackage{caption}
\usepackage{chngcntr}

\setcitestyle{square}
\newcommand{\triplet}[1]{\textlangle{\textit{#1}}\textrangle{}}



\title{Data Splits and Metrics for Method Benchmarking on Surgical Action Triplet Datasets}



\date{April 12, 2022}

\author{ 
\href{http://orcid.org/0000-0003-4777-0857}{\includegraphics[scale=0.06]{orcid.pdf}\hspace{1mm}Chinedu Innocent Nwoye}\\
	ICube Laboratory, CNRS,\\
	University of Strasbourg, France \\
	\texttt{nwoye@unistra.fr} \\
\And
	\href{https://orcid.org/0000-0002-5010-4137}{\includegraphics[scale=0.06]{orcid.pdf}\hspace{1mm}Nicolas Padoy} \\
	IHU Strasbourg, France\\
	ICube, University of Strasbourg, CNRS, France\\
\texttt{npadoy@unistra.fr} \\
}




\renewcommand{\headeright}{A Benchmark Study}
\renewcommand{\undertitle}{A Benchmark Study}
\renewcommand{\shorttitle}{CholecT45 \& CholecT50 Dataset Splits and Metrics}

\hypersetup{
pdftitle={CholecT50 Dataset Splits and Metrics},
pdfsubject={q-bio.NC, q-bio.QM},
pdfauthor={Chinedu Innocent Nwoye, Nicolas Padoy},
pdfkeywords={Surgical activity recognition, Tool-tissue interaction, Action triplet, CholecT40, CholecT45, CholecT50},
}







\begin{document}
\maketitle

\begin{abstract}
In addition to generating data and annotations, devising sensible data splitting strategies and evaluation metrics is essential for the creation of a benchmark dataset. This practice ensures consensus on the usage of the data, homogeneous assessment, and uniform comparison of research methods on the dataset.
This study focuses on CholecT50, which is a 50 video surgical dataset that formalizes surgical activities as triplets of \triplet{instrument, verb, target}.
In this paper, we introduce the standard splits for the CholecT50 and CholecT45 datasets and show how they compare with existing use of the dataset. CholecT45 is the first public release of 45 videos of CholecT50 dataset.
We also develop a metrics library, \path{ivtmetrics}, for model evaluation on surgical triplets.
Furthermore, we conduct a benchmark study by reproducing baseline methods in the most predominantly used deep learning frameworks (PyTorch and TensorFlow) to evaluate them using the proposed data splits and metrics and release them publicly to support future research.
The proposed data splits and evaluation metrics will enable global tracking of research progress on the dataset and facilitate optimal model selection for further deployment.
\end{abstract}
\keywords{: Surgical activity recognition \and Tool-tissue interaction \and Action triplet \and CholecT40 \and CholecT45 \and CholecT50\
    \label{metrics:AP}
    AP = \int_{0}^{1} p(r)dr .

   mAP = \frac{1}{N} \sum_{i=1}^N AP_i .

    \label{eqn:filter}
    Y_{d} = \biggl[~\max_{d^k \in \{i,v,t\}}~ {Y}_{ivt} ~\biggl]\quad\forall ~ ivt \in IVT, ~ k: 0\leq k < C_d.

        TP =  (\hat{y}_{_{I}} == y_{_{I}}) + \frac{\hat{b_I} \cap b_I}{\hat{b_I} \cup b_I} \geq \theta.
    
        TP =  (\hat{y}_{_{T}} == y_{_{T}}) + \frac{\hat{b_T} \cap b_T}{\hat{b_T} \cup b_T} \geq \theta
    
        TP =  (\hat{y}_{_{IVT}} == y_{_{IVT}}) + \frac{\hat{b_I} \cap b_I}{\hat{b_I} \cup b_I} \geq \theta ~~ \left [~+~ \frac{\hat{b_T} \cap b_T}{\hat{b_T} \cup b_T} \geq \theta~\right]
    
    \label{metrics:pr}
    \begin{split}
        p &= \frac{TP}{TP+FP}~,\\
        r &= \frac{TP}{TP+FN}~,\\
    \end{split}

        X_j(\%) = \frac{X_j}{\sum^N_{i=0}{X_i}} \times 100 ,
0.1in]
 \section{Benchmark Study Design}
\begin{figure*}[t]
\includegraphics[width=1.0\linewidth]{figure/models.pdf}
\caption{Reproduced models for surgical action triplet recognition: (a) Tripnet \cite{nwoye2020recognition}, (b) Rendezvous \cite{nwoye2021rendezvous}.}
\label{fig:methods}
\end{figure*}

To provide a benchmark study on CholecT45 and CholecT50 datasets using our proposed data splits and metrics, we re-implement and reproduce three proposed methods in PyTorch and TensorFlow. 

\subsection{Methods}
We summarize the reproduced methods as follows:
\begin{enumerate}
    \item {\bf Tripnet}: As shown in Fig. \ref{fig:methods}(a), Tripnet \cite{nwoye2020recognition} is a multi-task learning method that uses activation maps resulting from the instrument branch to enhance verb and target feature encoding in a new module known as class activation guide (CAG). It is followed by a 3D interaction space where relationships between instruments-verbs-targets components are resolved to triplets. Code and weights are publicly released on GitHub \url{https://github.com/CAMMA-public/tripnet}.
    
    \item {\bf Attention Tripnet}: This is an upgrade of Tripnet with an attention mechanism. The main difference is the use of class activation guided attention mechanism (CAGAM) \cite{nwoye2021rendezvous} over CAG where the verb and target feature discovery are obtained by channel and position attention processes respectively. Code and weights are publicly released on GitHub \url{https://github.com/CAMMA-public/attention-tripnet}.
    
    \item {\bf Rendezvous (RDV)}: In this model \cite{nwoye2021rendezvous}, the network encoder uses a weakly supervised approach to localize the instruments and the CAGAM module to detect the verb and target components of the triplets as shown in Fig. \ref{fig:methods}(b). The association part is achieved by both self and cross attention mechanisms in a new module known as multi-head of mixed attention (MHMA), and terminated by a simple classifier after 8 successive layers of association decoding. Code and weights are publicly released on GitHub \url{https://github.com/CAMMA-public/rendezvous}.\0.1in]
\end{enumerate}



\subsection{Implementation Details}
We made few changes in the original implementations \cite{nwoye2020recognition,nwoye2021rendezvous} as follows:
\begin{enumerate}
    \item {\bf Output resolution}: The original implementation lowered the strides of the last two blocks of the ResNet by one pixel to provide higher resolution () output. However, using original strides of size 2, we implement a faster version (size: ), trading-off precise localization to speed.
    \item {\bf Attention normalization}: The huge parameter layer-norms in the AddNorm layers of the attention module are replaced with batch-norms without affecting model accuracy.
    \item {\bf Loss function}: We integrates warmup parameters within the auxilliary task's cross-entropies without requiring additional uncertainty loss balancing \cite{kendall2018multi} as in \cite{nwoye2021rendezvous}. This removes the excessive parameters introduced by the uncertainty loss.
\end{enumerate} \section{Benchmark Results and Discussion}




\begin{table}[ht]
\centering
    \setlength{\tabcolsep}{9pt}
    \captionsetup{skip=0pt,singlelinecheck=off,justification=raggedright}
    \caption{Benchmark triplet recognition AP (\%) on CholecT50 dataset for different frameworks using RDV split.}
    \label{tab:results:rdv}
    \resizebox{\textwidth}{!}{\begin{tabular}{@{}llclcrclcr@{}}
            \toprule
            \multicolumn{1}{l}{\multirow{2}{*}{Framework}}&
            \multicolumn{1}{l}{\multirow{2}{*}{Method}}&
            \multicolumn{4}{r}{Component detection~~~~~}&\phantom{abc}&
            \multicolumn{3}{c}{Triplet association}\\ \cmidrule{4-6} \cmidrule{8-10} 
            \multicolumn{2}{l}{} && \multicolumn{1}{l}{~~~~} & \multicolumn{1}{c}{~~~~} & \multicolumn{1}{c}{~~~~} && \multicolumn{1}{c}{~~~} & \multicolumn{1}{c}{~~} & \multicolumn{1}{r@{}}{} \\ \midrule
            \multirow{3}{*}{TensorFlow} 
            & Tripnet \cite{nwoye2020recognition} &&  {\bf 92.1} & 54.5 & 33.2 && 29.7 & 26.4 & 20.0  \\ 
            & Attention Tripnet \cite{nwoye2021rendezvous} && 92.0 &  {60.2} & {38.5} && {31.1} & {29.8} & {23.4} \\ 
            & Rendezvous (RDV) \cite{nwoye2021rendezvous} && 92.0 & {60.7} & 38.3 && {39.4} & {\bf 36.9} & {\bf 29.9}  \\ \midrule
\multirow{3}{*}{PyTorch} 


& Tripnet \cite{nwoye2020recognition} && 88.7 & 59.2 & 39.3 && 31.9 & 27.9 & 21.6 \\
& Attention Tripnet \cite{nwoye2021rendezvous} && 87.9 & 59.7 & 40.6 && 34.2  & 29.0 & 23.2 \\


& Rendezvous \cite{nwoye2021rendezvous} && 89.1 & \bf 62.3 & \bf 43.8 && \bf 40.0 & 35.8 & 29.5 \\ \bottomrule 
        \end{tabular}
    }
\end{table}


 
\subsection{Quantitative Results on CholecT50 using Rendezvous Split}
All the 100 triplet classes are evaluated in this setup.
The benchmark results on the RDV split is presented in Table \ref{tab:results:rdv}. The results follow the same trend as in the original paper \cite{nwoye2021rendezvous} as it is observed that the Attention Tripnet leverages CAGAM to improve the verb and target detections while Rendezvous utilizes its transformer-inspired MHMA to improve the triplet association performance.
We show the performance across deep learning frameworks in Table \ref{tab:results:rdv}. The PyTorch models approximates the performance of their TensorFlow counterparts (version 1). We observe that these results are comparable in some of the sub tasks.










\subsection{Quantitative Results on CholecT50 using CholecTriplet Split}
The challenge rule excludes the 6 null triplet classes (IDs: 94-99) from evaluation. The results are presented in Table \ref{table:results:t50}.
It is observed that AP is higher than in the RDV split for each model, likely due to the reduced number of triplet classes (94 v 100).
Also, the direct outputs () for the individual components (i.e.:  or ) of the triplets () are not provided by the challenge approaches, instead they are filtered from the main triplet predictions () following the filtering formula in Section \ref{disentagling_function}.
As shown in Table \ref{table:results:t50}, the AP performances on the filtered prediction are generally lower compared to AP on directly predicted probabilities of those components when provided, however, they are more informative and a better representation of how a model understands the triplet's composition.




\begin{table}[htp]
\centering
    \setlength{\tabcolsep}{9pt}
    \captionsetup{skip=0pt,singlelinecheck=off,justification=raggedright}
    \caption{Benchmark triplet recognition AP (\%) on CholecT50 dataset using CholecTriplet split.}
    \label{table:results:t50}
    \resizebox{\textwidth}{!}{\begin{tabular}{@{}llclcrclcr@{}}
            \toprule
            \multicolumn{1}{l}{\multirow{2}{*}{Framework}}&
            \multicolumn{1}{l}{\multirow{2}{*}{Method}}&
            \multicolumn{4}{r}{Component detection~~~~~}&\phantom{abc}&
            \multicolumn{3}{c}{Triplet association}\\ \cmidrule{4-6} \cmidrule{8-10} 
            \multicolumn{2}{l}{} && \multicolumn{1}{l}{~~~~} & \multicolumn{1}{c}{~~~~} & \multicolumn{1}{c}{~~~~} && \multicolumn{1}{c}{~~~} & \multicolumn{1}{c}{~~} & \multicolumn{1}{r@{}}{}  \\ \midrule
            \multirow{3}{*}{TensorFlow} 
            & Tripnet \cite{nwoye2020recognition} && 74.6 &	42.9 & 32.2 &&	27.0 & 28.0 & 23.4 \\
            & Attention Tripnet \cite{nwoye2021rendezvous} && 77.1 & 43.4 & 30.0 && 32.3 & 29.7 & 25.5 \\ 
            & Rendezvous \cite{nwoye2021rendezvous} && \bf 77.5	& \bf 47.5 & 37.7 && 34.4 & \bf 38.2 & {32.7} \\
            \midrule
\multirow{3}{*}{PyTorch} 
& Tripnet \cite{nwoye2020recognition} && 73.9 & 41.6 & 32.1 && 28.8 & 29.2 & 27.4 \\ 
            & Attention Tripnet  \cite{nwoye2021rendezvous} && 73.7 & 43.7 & 35.3 && 31.6 & 31.9 & 27.7 \\
& Rendezvous  \cite{nwoye2021rendezvous} && 75.9 & 46.0 & \bf 38.7 && \bf 35.9 & 37.1 & \bf 32.8 \\
\bottomrule 
        \end{tabular}
    }
\end{table}


 










\subsection{Quantitative Results on CholecT50 using Cross-Validation Split}
The benchmarking results on the CholecT50 cross-validation split are presented in Table \ref{table:results:cv50} along with the standard deviation (std) over the folds.
All the 100 triplet classes are evaluated in this setup.
This presents a less biased and less optimistic estimate of the models with confidence intervals.
Their standard deviation (std) spread shows the extent of performances approximation of the three models positioning  Tripnet as the least and Rendezvous as the best in terms of performance.

\begin{table}[ht]
\centering
    \setlength{\tabcolsep}{9pt}
    \captionsetup{skip=0pt,singlelinecheck=off,justification=raggedright}
    \caption{Benchmark triplet recognition AP (\%) on CholecT50 dataset using the official cross-validation split.}
    \label{table:results:cv50}
    \resizebox{\textwidth}{!}{\begin{tabular}{@{}lrlcrrlcr@{}}
            \toprule
            \multirow{2}{*}{Method (in PyTorch)} & \phantom{abc}
            &\multicolumn{3}{c}{{Component detection}} & \phantom{abc}
            &\multicolumn{3}{c}{{Triplet association}} \\ \cmidrule{3-5} \cmidrule{7-9}
            && \multicolumn{1}{c}{} & \multicolumn{1}{c}{} & \multicolumn{1}{c}{} && \multicolumn{1}{c}{} & \multicolumn{1}{c}{} & \multicolumn{1}{c}{}
            \\ \midrule   
Tripnet \cite{nwoye2020recognition} &&  89.1±1.7 & 58.8±3.1 & 38.4±1.3 && 32.7±2.4 & 29.0±0.8 & 25.3±2.4 \\
            Attention Tripnet \cite{nwoye2021rendezvous}  && 88.7±1.3 & \textbf{61.1±2.0} & \textbf{40.7±3.2} && 33.1±2.7 & 30.3±1.6	& 27.2±2.9 \\
            Rendezvous \cite{nwoye2021rendezvous}  && \textbf{89.4±2.0} & 60.4±2.8 & 40.3±2.2  &&  \textbf{34.5±2.8} & \textbf{31.8±1.0} & \textbf{29.4±2.5} \\
            \bottomrule
        \end{tabular}
    }
\end{table} 








\subsection{Quantitative Results on CholecT45 using Cross-Validation Split}
Similarly, the benchmarking results on the CholecT45 cross-validation split, presented in Table \ref{table:results:cv45}, justifies the use of attention mechanisms for surgical action triplet recognition. The analysis shows that the results obtained on the CholecT45 CV approximates the ones of the CholecT50 CV in all the sub-tasks, justifying its use/sufficiency in the absence of the complete CholecT50 dataset.

\begin{table*}[ht]
\centering
    \setlength{\tabcolsep}{9pt}
    \captionsetup{skip=0pt,singlelinecheck=off,justification=raggedright}
    \caption{Benchmark triplet recognition AP (\%) on CholecT45 dataset using the official cross-validation split.}
    \label{table:results:cv45}
    \resizebox{\textwidth}{!}{\begin{tabular}{@{}lrlcrrlcr@{}}
            \toprule
            \multirow{2}{*}{Method (in PyTorch)} & \phantom{abc}
            &\multicolumn{3}{c}{{Component detection}} & \phantom{abc}
            &\multicolumn{3}{c}{{Triplet association}} \\ \cmidrule{3-5} \cmidrule{7-9}
            && \multicolumn{1}{c}{} & \multicolumn{1}{c}{} & \multicolumn{1}{c}{} && \multicolumn{1}{c}{} & \multicolumn{1}{c}{} & \multicolumn{1}{c}{}
            \\ \midrule   
Tripnet \cite{nwoye2020recognition} &&  \textbf{89.9±1.0} & 59.9±0.9 & 37.4±1.5  && 31.8±4.1 & 27.1±2.8 & 24.4±4.7 \\
            Attention Tripnet \cite{nwoye2021rendezvous} && 89.1±2.1 & 61.2±0.6 & {\bf 40.3±1.2} && 33.0±2.9 & 29.4±1.2 & 27.2±2.7 \\
            Rendezvous \cite{nwoye2021rendezvous}  && 89.3±2.1 & \textbf{62.0±1.3} & 40.0±1.4  &&  \textbf{34.0±3.3} & \textbf{30.8±2.1} & \textbf{29.4±2.8} \\
            \bottomrule
        \end{tabular}
    }
\end{table*} 



\subsection{Class-wise Performances of the Benchmark Models}
We present the per-class performance for the triplet components (Tables \ref{table:results:instruments} - \ref{table:results:targets}) and their association (Table \ref{table:results:triplets}) recognition using the cross-validation dataset splitting strategy on both CholecT45 and CholecT50. We observe similar performance pattern across the two datasets in all classes of each sub task showing the reliability of cross-validation split approach in model evaluation.

On instrument presence detection, as shown in Table \ref{table:results:instruments}, grasper and hook are the most detected owing to their highest occurrence frequencies in the dataset. However, hook is a little better detected than grasper owing to its uniqueness unlike the grasper which sometimes share some similarities with clipper, bipolar and scissor. The scissors and irrigator, on the other hand, are the least detected likely due to their low occurrence distributions in the datasets.


\begin{table}[ht]
    \captionsetup{skip=0pt,singlelinecheck=off,justification=raggedright}
    \caption{Per-class instrument presence detection AP (\%) on cross-validation splits (Method in PyTorch)}
    \label{table:results:instruments}
    \centering
    \setlength{\tabcolsep}{11pt}
    \resizebox{\textwidth}{!}{\begin{tabular}{@{}lrlcrrlcr@{}}
        \toprule
        \multirow{2}{*}{Classes} & \phantom{abc} &
        \multicolumn{3}{c}{CholecT45} & \phantom{abc} &
        \multicolumn{3}{c}{CholecT50} \\ \cmidrule{3-5} \cmidrule{7-9}
        & & Tripnet & \makecell[r]{Attention\\Tripnet} & RDV && Tripnet & \makecell[c]{Attention\\Tripnet} & RDV \\
        \midrule


        grasper && 96.5±0.4 & 96.4±0.7 & 96.6±0.6 && 96.4±0.7 & 96.5±0.6 & 96.6±0.6 \\
        bipolar && 88.4±4.2 & 86.0±4.2 & 87.4±4.7 && 89.0±3.4 & 88.2±4.2 & 88.5±4.1 \\
        hook && 97.5±1.6 & 97.1±1.3 & 97.4±1.5 && 97.5±1.3 & 97.3±1.3 & 97.6±1.3 \\
        scissors && 80.3±6.0 & 79.6±8.4 & 78.4±5.4 && 82.8±4.9 & 81.3±5.8 & 82.6±6.5 \\
        clipper && 91.2±3.9 & 90.1±3.9 & 90.9±3.8 && 89.6±5.6 & 89.8±5.8 & 89.9±5.0 \\
        irrigator && 86.0±4.1 & 85.3±2.8 & 84.5±6.8 && 79.6±6.5 & 79.1±5.7 & 81.3±4.9 \\
        \midrule
        Mean && \bf 89.9±1.0 & 89.1±2.1 & 89.3±2.1 && 89.1±1.7 & 88.7±1.3 & \bf 89.4±2.0 \\
         \bottomrule
    \end{tabular}
    }
\end{table} 
For the verb recognition, grasp, retract, dissect, coagulate, clip, cut, and aspirate are better detect above 50\% at all time as shown in Table \ref{table:results:verbs}. This is due to their strong affinities with unique instrument classes. Pack and irrigate are very challenging to discriminate from the dominant verbs of their instruments namely retract and aspirate. Null action, being a compendium of unconsidered actions, is the least recognized verb.


\begin{table}[ht]
    \captionsetup{skip=0pt,singlelinecheck=off,justification=raggedright}
    \caption{Per-class verb recognition AP (\%) on cross-validation splits (Method in PyTorch)}
    \label{table:results:verbs}
    \centering
    \captionsetup{skip=0pt,singlelinecheck=off,justification=raggedright}
    \setlength{\tabcolsep}{9pt}
    \resizebox{\textwidth}{!}{\begin{tabular}{@{}llrlcrrlcr@{}}
        \toprule
        \multirow{2}{*}{Classes} & \phantom{abc}& \phantom{abc} &
        \multicolumn{3}{c}{CholecT45} & \phantom{abc} &
        \multicolumn{3}{c}{CholecT50} \\ \cmidrule{4-6} \cmidrule{8-10}
        & & & Tripnet & \makecell[c]{Attention\\Tripnet} & RDV && Tripnet & \makecell[c]{Attention\\Tripnet} & RDV \\
        \midrule
        grasp &&& 70.5±5.8 & 60.5±9.9 & 69.8±3.7 && 67.1±3.4 & 66.1±5.4 & 68.3±3.0 \\
        retract &&& 90.5±5.4 & 84.0±9.8 & 89.7±7.2 && 86.7±5.1 & 85.8±5.4 & 86.7±5.8 \\
        dissect &&& 93.0±2.8 & 86.5±9.9 & 93.2±3.9 && 90.9±2.4 & 90.6±2.4 & 91.0±3.3 \\
        coagulate &&& 67.2±6.1 & 56.5±9.9 & 68.7±5.5 && 67.9±5.0 & 68.5±6.2 & 69.7±6.1 \\
        clip &&& 85.4±6.4 & 67.8±9.8 & 85.5±3.7 && 85.5±6.3 & 86.1±5.4 & 86.5±5.5 \\
        cut &&& 70.5±9.1 & 57.7±9.9 & 72.0±4.8 && 74.9±3.4 & 72.3±6.1 & 74.9±7.6 \\
        aspirate &&& 60.7±9.2 & 47.1±9.9 & 57.8±9.9 && 57.4±4.9 & 57.1±7.3 & 56.7±5.5 \\
        irrigate &&& 29.6±8.2 & 17.4±9.7 & 25.7±5.8 && 27.6±9.4 & 25.4±7.9 & 25.1±9.2 \\
        pack &&& 32.1±9.9 & 25.8±9.9 & 31.2±9.9 && 26.8±9.9 & 33.2±9.9 & 20.0±9.9 \\
        null-verb &&& 23.0±2.4 & 21.1±5.0 & 24.0±4.1 && 24.5±1.8 & 25.5±4.0 & 24.9±2.6 \\
        \midrule
        Mean &&& 59.9±0.9 & 61.2±0.6 & \bf 62.0±1.3 && 58.8±3.1 & \bf 61.1±2.0 & 60.4±2.8 \\
         \bottomrule
    \end{tabular}
    }
\end{table}

%
 
The per-class target detection reveals the most interesting areas of improvement. The predominant targets such as gallbladder, liver, and specimen-bag are well detected above 70\% as shown in Table \ref{table:results:targets}. The main challenge comes in detecting tiny anatomical structures such as cystic-artery, peritoneum, cystic-plate, cytic-pedicle, etc. as against conspicuous structures such as liver, omentum, cystic-duct, fluid, etc. Some anatomies with no clear boundaries such as abdominal wall, cystic-artery, etc. are fairly detected.


\begin{table}[ht]
    \captionsetup{skip=0pt,singlelinecheck=off,justification=raggedright}
    \caption{Per-class target recognition AP (\%) on cross-validation splits (Method in PyTorch)}
    \label{table:results:targets}
    \centering
    \captionsetup{skip=0pt,singlelinecheck=off,justification=raggedright}
    \setlength{\tabcolsep}{7.5pt}
    \resizebox{\textwidth}{!}{\begin{tabular}{@{}lrlcrllcr@{}}
        \toprule
        \multirow{2}{*}{Classes} & \phantom{abc} &
        \multicolumn{3}{c}{CholecT45} & \phantom{abc} &
        \multicolumn{3}{c}{CholecT50} \\ \cmidrule{3-5} \cmidrule{7-9}
        & & Tripnet & \makecell[c]{Attention\\Tripnet} & RDV && Tripnet & \makecell[c]{Attention\\Tripnet} & RDV \\
        \midrule
        gallbladder && 93.6±1.1 & 91.2±6.5 & 93.7±1.2 && 93.6±1.3 & 93.8±1.0 & 93.6±1.6 \\
        cystic-plate && 11.6±3.9 & 10.1±2.0 & 11.0±3.5 && 11.1±2.2 & 11.2±2.7 & 09.9±3.3 \\
        cystic-duct && 47.2±5.8 & 41.9±9.9 & 47.1±2.8 && 47.6±5.6 & 48.1±3.1 & 47.2±3.9 \\
        cystic-artery && 31.9±3.7 & 29.6±9.7 & 31.2±2.2 && 35.0±4.7 & 34.6±2.8 & 35.6±4.6 \\
        cystic-pedicle && 04.0±2.4 & 08.7±6.0 & 13.4±7.8 && 10.3±5.0 & 06.9±8.2 & 10.4±6.5 \\
        blood-vessel && 08.4±5.6 & 15.6±9.9 & 06.7±6.2 && 12.7±9.9 & 23.5±9.9 & 18.7±9.9 \\
        fluid && 58.4±9.2 & 48.9±9.9 & 58.0±9.9 && 56.3±5.1 & 54.5±6.8 & 57.0±5.1 \\
        abdominal-wall/cavity && 30.0±4.6 & 20.4±9.9 & 25.9±7.2 && 25.7±3.4 & 28.5±5.0 & 31.3±9.9 \\
        liver && 71.8±5.5 & 65.3±9.9 & 72.9±2.5 && 72.7±6.5 & 73.5±4.4 & 74.8±4.2 \\
        adhesion && 04.2±0.3 & 13.9±3.3 & 07.2±0.5 && 05.2±0.0 & 33.3±9.9 & 11.3±2.8 \\
        omentum && 46.7±8.6 & 44.4±9.9 & 48.0±9.9 && 45.8±8.0 & 45.8±9.9 & 46.2±9.9 \\
        peritoneum && 17.7±5.5 & 24.1±9.9 & 26.6±4.5 && 19.5±9.9 & 28.8±9.9 & 25.7±8.1 \\
        gut && 10.7±7.7 & 09.6±6.9 & 09.5±6.9 && 13.6±8.3 & 14.4±6.6 & 15.5±7.4 \\
        specimen-bag && 85.8±2.5 & 70.1±9.9 & 84.4±1.2 && 85.5±1.8 & 84.1±1.3 & 84.6±1.2 \\
        null-target && 22.8±2.3 & 21.1±5.4 & 23.5±4.1 && 24.5±2.1 & 25.5±3.9 & 25.2±2.6 \\
        \midrule
        Mean & & 37.4±1.4 & \bf 40.3±1.2 & 40.0±1.4 && 38.4±1.3 & \bf 40.7±3.2 & 40.3±2.2\\
         \bottomrule
    \end{tabular}
    }
\end{table}
%
 
A presentation of class-wise results of the complete 100 triplet classes is only possible using the cross-validation approach. 
As shown in Table \ref{table:results:triplets}, the models recognizes the most important triplets such as grasper retracting gallbladder or grasping specimen-bag, hook dissecting either gallbladder or omentum, bipolar coagulating liver, clipper clipping cystic-artery and -duct with scissors cutting the same, and irrigator aspirating fluid or irrigating the cystic-duct. 
The good performance on these specific triplets are expected from the models' high performance on their specific triplet component classes.
Surprising, the irrigator rare use in dissecting cystic-pedicle is well detected by RDV model.











\begin{table}[htp]
    \captionsetup{skip=0pt,singlelinecheck=off,justification=raggedright}
    \caption{Per-class triplet recognition AP (\%) on cross-validation splits (Method in PyTorch)}
    \label{table:results:triplets}
    \centering
    \captionsetup{skip=0pt,singlelinecheck=off,justification=raggedright}
    \setlength{\tabcolsep}{2pt}
    \resizebox{\textwidth}{!}{\begin{tabular}{@{}lr lcr r lcr r lr lcr r lcr@{}}
        \toprule
        \multirow{2}{*}{Classes} & \phantom{abc} &
        \multicolumn{3}{c}{CholecT45} & \phantom{abc} &
        \multicolumn{3}{c}{CholecT50} & \phantom{abc} &
        \multicolumn{1}{c}{\multirow{2}{*}{classes}} & \phantom{abc} &
        \multicolumn{3}{c}{CholecT45} & \phantom{abc} &
        \multicolumn{3}{c}{CholecT50}\\ \cmidrule{3-5} \cmidrule{7-9} \cmidrule{13-15} \cmidrule{17-19}
        & & Tripnet & \makecell[c]{Attention\\Tripnet} & RDV && Tripnet  & \makecell[c]{Attention\\Tripnet} & RDV &&&& Tripnet  & \makecell[c]{Attention\\Tripnet} & RDV && Tripnet  & \makecell[c]{Attention\\Tripnet} & RDV\\
        \midrule
        grasper,dissect,cystic-plate && 01.5±0.3 & 01.5±0.4 & 02.2±1.2 && 03.3±3.4 & 01.7±0.8 & 01.9±0.1 && hook,coagulate,cystic-plate && 00.5±0.1 & 00.0±0.0 & 00.3±0.1 && 01.8±0.1 & 00.4±0.1 & 00.4±0.1\\ 
        grasper,dissect,gallbladder && 09.5±9.5 & 04.6±5.6 & 05.0±4.3 && 06.2±9.7 & 07.7±8.3 & 11.6±9.9 && hook,coagulate,gallbladder && 06.1±8.2 & 03.2±1.7 & 05.6±6.8 && 08.6±9.2 & 03.8±2.9 & 06.7±5.8\\ 
        grasper,dissect,omentum && 01.3±1.2 & 05.1±5.4 & 03.4±4.0 && 03.1±3.7 & 03.0±1.8 & 01.8±1.1 && hook,coagulate,liver && 01.9±1.2 & 02.5±2.1 & 07.5±4.8 && 05.7±4.1 & 04.0±2.9 & 06.1±4.2\\ 
        grasper,grasp,cystic-artery && 01.5±0.3 & 01.3±0.3 & 02.0±0.6 && 01.9±1.3 & 01.3±0.3 & 02.3±0.4 && hook,coagulate,omentum && 07.6±7.6 & 10.3±6.1 & 06.3±7.4 && 13.6±9.9 & 04.8±3.1 & 05.0±5.1\\ 
        grasper,grasp,cystic-duct && 07.9±4.9 & 12.8±6.5 & 20.8±9.9 && 08.0±5.8 & 07.4±3.1 & 18.7±9.9 && hook,cut,blood-vessel && 00.0±0.0 & 00.0±0.0 & 00.0±0.0 && 01.6±0.1 & 01.2±0.1 & 01.0±0.1\\ 
        grasper,grasp,cystic-pedicle && 05.2±1.9 & 20.1±0.1 & 02.8±0.8 && 13.5±9.9 & 02.2±0.2 & 24.0±9.9 && hook,cut,peritoneum && 00.0±0.0 & 00.0±0.0 & 00.0±0.0 && 07.3±0.1 & 05.2±0.1 & 03.4±0.1\\ 
        grasper,grasp,cystic-plate && 20.3±9.9 & 21.6±9.9 & 23.8±9.9 && 07.8±3.8 & 06.0±5.6 & 06.3±4.5 && hook,dissect,blood-vessel && 00.7±0.1 & 01.2±0.1 & 00.9±0.1 && 01.4±0.1 & 00.7±0.1 & 00.8±0.1\\ 
        grasper,grasp,gallbladder && 23.8±9.9 & 22.2±9.9 & 30.5±9.9 && 28.6±9.9 & 28.2±9.9 & 29.4±9.9 && hook,dissect,cystic-artery && 20.4±4.9 & 19.7±6.6 & 20.7±4.3 && 25.5±5.6 & 22.7±4.6 & 26.8±6.9\\ 
        grasper,grasp,gut && 00.4±0.1 & 00.9±0.1 & 00.3±0.1 && 02.1±2.3 & 00.7±0.6 & 01.1±0.4 && hook,dissect,cystic-duct && 37.4±4.4 & 38.7±3.6 & 39.1±3.1 && 37.8±5.5 & 42.5±2.9 & 38.6±3.8\\ 
        grasper,grasp,liver && 02.5±2.0 & 02.3±3.1 & 16.9±9.9 && 07.8±7.7 & 02.1±2.0 & 06.2±4.6 && hook,dissect,cystic-plate && 14.4±6.5 & 11.5±5.1 & 18.3±9.9 && 13.6±7.5 & 17.3±9.9 & 14.5±7.7\\ 
        grasper,grasp,omentum && 04.5±5.8 & 06.1±4.2 & 26.7±9.9 && 08.7±6.2 & 03.8±5.8 & 11.4±9.7 && hook,dissect,gallbladder && 78.7±2.6 & 78.3±3.6 & 78.3±2.2 && 77.5±4.4 & 77.8±4.9 & 77.3±4.2\\ 
        grasper,grasp,peritoneum && 09.2±9.9 & 04.2±4.7 & 03.0±2.4 && 12.3±9.9 & 17.0±9.9 & 15.1±9.9 && hook,dissect,omentum && 62.5±9.9 & 65.1±7.0 & 63.9±8.6 && 67.4±9.9 & 66.5±9.9 & 67.2±9.9\\ 
        grasper,grasp,specimen-bag && 85.3±2.3 & 85.7±1.9 & 84.5±1.1 && 85.3±1.3 & 85.2±1.4 & 84.9±1.4 && hook,dissect,peritoneum && 15.1±2.8 & 11.9±8.5 & 27.3±3.8 && 19.8±6.3 & 32.5±7.4 & 26.9±9.0\\ 
        grasper,pack,gallbladder && 30.9±9.9 & 33.9±9.9 & 35.2±9.3 && 28.4±9.9 & 37.2±7.5 & 28.2±6.9 && hook,retract,gallbladder && 14.8±9.9 & 17.0±5.8 & 23.8±9.9 && 17.9±9.9 & 17.0±8.6 & 21.3±9.9\\ 
        grasper,retract,cystic-duct && 26.9±0.1 & 00.0±0.0 & 45.0±0.1 && 24.1±0.1 & 21.9±0.1 & 38.7±0.1 && hook,retract,liver && 05.0±6.5 & 12.1±3.4 & 19.2±9.9 && 06.8±5.5 & 11.3±9.9 & 11.3±7.9\\ 
        grasper,retract,cystic-pedicle && 00.8±0.1 & 02.1±0.1 & 01.4±0.1 && 01.5±0.1 & 01.2±0.1 & 02.0±0.1 && scissors,coagulate,omentum && 00.8±0.1 & 04.0±0.1 & 01.1±0.1 && 03.1±0.1 & 00.8±0.1 & 01.4±0.1\\ 
        grasper,retract,cystic-plate && 16.0±9.9 & 15.9±1.1 & 17.8±1.3 && 14.7±7.1 & 24.7±9.9 & 15.9±8.7 && scissors,cut,adhesion && 07.9±0.1 & 12.4±0.1 & 10.4±0.1 && 06.1±0.1 & 13.8±0.1 & 11.8±0.1\\ 
        grasper,retract,gallbladder && 83.4±7.6 & 86.5±4.4 & 83.9±9.6 && 79.6±6.9 & 78.3±8.2 & 79.2±8.9 && scissors,cut,blood-vessel && 19.1±9.9 & 36.7±9.9 & 33.4±9.9 && 01.9±1.7 & 10.2±1.3 & 37.5±9.9\\ 
        grasper,retract,gut && 08.5±5.2 & 10.5±6.0 & 10.8±5.1 && 18.3±8.9 & 13.7±6.2 & 17.4±9.9 && scissors,cut,cystic-artery && 50.6±9.9 & 62.1±4.8 & 57.3±5.6 && 56.1±5.8 & 58.9±9.4 & 58.9±7.2\\ 
        grasper,retract,liver && 69.7±6.7 & 72.1±6.8 & 72.0±2.6 && 71.0±6.2 & 71.0±4.7 & 74.1±3.9 && scissors,cut,cystic-duct && 51.8±9.9 & 56.3±5.6 & 59.0±7.3 && 56.4±5.8 & 56.7±5.7 & 58.6±4.2\\ 
        grasper,retract,omentum && 44.9±9.9 & 43.0±9.9 & 45.5±9.9 && 42.1±9.9 & 42.6±9.9 & 47.9±9.9 && scissors,cut,cystic-plate && 01.5±1.6 & 16.0±2.5 & 22.9±6.1 && 25.5±9.9 & 09.3±4.3 & 48.5±9.9\\ 
        grasper,retract,peritoneum && 17.7±9.9 & 31.3±9.9 & 43.5±9.9 && 24.0±9.9 & 50.5±9.9 & 46.5±9.9 && scissors,cut,liver && 02.1±0.1 & 14.9±0.1 & 08.8±0.1 && 25.4±0.1 & 06.3±0.1 & 23.9±0.1\\ 
        bipolar,coagulate,abdominal-wall-cavity && 41.2±9.9 & 40.0±9.9 & 35.6±9.9 && 39.4±9.9 & 45.9±8.1 & 41.1±9.9 && scissors,cut,omentum && 01.9±0.1 & 00.0±0.0 & 07.9±0.1 && 04.9±5.5 & 01.7±0.8 & 21.0±9.9\\ 
        bipolar,coagulate,blood-vessel && 05.5±4.1 & 12.2±9.1 & 24.0±9.9 && 23.2±9.9 & 50.8±9.9 & 41.3±9.9 && scissors,cut,peritoneum && 02.7±0.1 & 07.4±0.1 & 42.9±0.1 && 02.8±0.1 & 04.2±0.1 & 23.4±0.1\\ 
        bipolar,coagulate,cystic-artery && 21.3±1.5 & 03.9±0.1 & 15.4±4.9 && 11.3±6.4 & 09.4±3.7 & 26.2±9.9 && scissors,dissect,cystic-plate && 00.4±0.1 & 00.4±0.1 & 02.0±0.1 && 00.5±0.1 & 00.6±0.1 & 00.8±0.1\\ 
        bipolar,coagulate,cystic-duct && 02.6±0.1 & 03.8±0.1 & 07.7±0.1 && 00.8±0.1 & 01.3±0.1 & 01.4±0.1 && scissors,dissect,gallbladder && 02.3±0.1 & 00.0±0.0 & 03.7±0.1 && 02.7±0.1 & 00.9±0.1 & 01.4±0.1\\ 
        bipolar,coagulate,cystic-pedicle && 27.6±9.9 & 36.9±9.9 & 45.5±9.9 && 32.2±9.9 & 32.3±9.9 & 50.0±9.9 && scissors,dissect,omentum && 04.5±0.1 & 15.8±0.1 & 06.6±0.1 && 08.3±0.1 & 04.7±0.1 & 48.4±0.1\\ 
        bipolar,coagulate,cystic-plate && 29.3±9.9 & 25.6±9.9 & 40.5±9.9 && 31.7±9.9 & 35.8±9.9 & 33.3±9.9 && clipper,clip,blood-vessel && 13.6±5.8 & 15.5±9.9 & 17.4±9.9 && 12.1±9.9 & 20.9±9.9 & 24.9±3.6\\ 
        bipolar,coagulate,gallbladder && 36.6±9.9 & 52.4±9.9 & 43.7±9.9 && 43.0±9.9 & 41.3±9.9 & 48.5±9.9 && clipper,clip,cystic-artery && 58.7±4.1 & 61.2±9.9 & 66.5±4.0 && 57.9±9.5 & 61.6±9.9 & 67.4±3.5\\ 
        bipolar,coagulate,liver && 77.6±7.4 & 79.7±5.9 & 78.2±6.8 && 79.8±4.5 & 79.7±7.6 & 80.9±7.8 && clipper,clip,cystic-duct && 65.2±9.3 & 70.0±5.7 & 70.7±6.1 && 68.6±6.4 & 70.6±8.2 & 73.0±4.9\\ 
        bipolar,coagulate,omentum && 33.3±9.9 & 42.8±9.9 & 37.0±9.9 && 49.5±9.9 & 44.7±9.9 & 46.9±9.9 && clipper,clip,cystic-pedicle && 03.5±0.1 & 05.2±0.1 & 26.8±0.1 && 12.5±0.1 & 00.6±0.1 & 00.8±0.1\\ 
        bipolar,coagulate,peritoneum && 08.3±0.1 & 00.0±0.0 & 22.5±0.1 && 10.3±6.6 & 61.7±9.9 & 33.2±9.9 && clipper,clip,cystic-plate && 02.4±0.9 & 12.0±9.3 & 16.3±9.1 && 10.7±8.3 & 16.3±9.9 & 19.9±9.9\\ 
        bipolar,dissect,adhesion && 07.0±0.1 & 00.0±0.0 & 05.1±0.1 && 07.8±0.1 & 02.0±0.1 & 07.9±0.1 && irrigator,aspirate,fluid && 58.9±9.9 & 57.3±3.0 & 57.4±9.9 && 57.7±4.4 & 56.0±5.5 & 58.9±3.6\\ 
        bipolar,dissect,cystic-artery && 07.0±5.1 & 29.8±9.9 & 22.3±9.9 && 15.8±9.9 & 20.3±9.9 & 32.5±9.9 && irrigator,dissect,cystic-duct && 04.7±0.1 & 00.0±0.0 & 18.1±0.1 && 13.3±0.1 & 11.8±0.1 & 04.5±0.1\\ 
        bipolar,dissect,cystic-duct && 25.9±9.9 & 25.9±3.5 & 08.9±4.5 && 07.3±5.7 & 18.0±9.9 & 22.9±9.9 && irrigator,dissect,cystic-pedicle && 18.8±0.5 & 39.5±9.9 & 60.6±9.9 && 36.3±9.9 & 52.0±9.9 & 51.0±5.4\\ 
        bipolar,dissect,cystic-plate && 04.5±1.9 & 08.7±0.1 & 03.5±1.3 && 16.6±9.9 & 05.2±2.7 & 19.1±1.7 && irrigator,dissect,cystic-plate && 01.2±0.1 & 02.5±0.1 & 02.0±0.1 && 00.4±0.1 & 04.3±0.1 & 02.2±0.1\\ 
        bipolar,dissect,gallbladder && 23.0±9.9 & 39.3±9.9 & 20.5±6.3 && 12.2±9.9 & 20.0±9.9 & 19.5±9.9 && irrigator,dissect,gallbladder && 02.3±2.0 & 11.1±7.5 & 19.6±9.9 && 03.0±3.0 & 05.2±6.5 & 11.4±3.8\\ 
        bipolar,dissect,omentum && 11.2±0.1 & 00.0±0.0 & 26.0±0.1 && 09.2±0.1 & 53.2±0.1 & 33.0±0.1 && irrigator,dissect,omentum && 03.4±3.4 & 13.5±9.9 & 08.8±4.1 && 04.2±5.0 & 13.3±9.9 & 11.8±7.7\\ 
        bipolar,grasp,cystic-plate && 00.8±0.1 & 00.3±0.1 & 00.5±0.1 && 00.6±0.1 & 00.3±0.1 & 00.5±0.1 && irrigator,irrigate,abdominal-wall-cavity && 23.0±9.9 & 17.8±9.9 & 28.2±5.8 && 26.3±9.9 & 28.8±6.3 & 25.1±9.9\\ 
        bipolar,grasp,liver && 03.4±0.1 & 95.5±0.1 & 15.2±0.1 && 21.2±0.1 & 99.5±0.1 & 29.8±0.1 && irrigator,irrigate,cystic-pedicle && 92.4±2.6 & 96.1±6.0 & 92.8±2.0 && 94.0±4.6 & 91.1±0.6 & 01.8±0.8\\ 
        bipolar,grasp,specimen-bag && 23.3±9.9 & 25.6±9.9 & 26.7±9.9 && 17.0±9.9 & 23.7±9.9 & 16.6±9.9 && irrigator,irrigate,liver && 13.3±9.9 & 26.9±9.9 & 18.0±8.0 && 21.1±7.9 & 19.5±8.3 & 21.6±7.7\\ 
        bipolar,retract,cystic-duct && 00.2±0.1 & 00.2±0.1 & 01.7±0.1 && 00.2±0.1 & 00.3±0.1 & 01.4±0.1 && irrigator,retract,gallbladder && 19.1±9.9 & 21.9±9.9 & 48.5±9.9 && 42.8±9.9 & 07.1±6.1 & 33.3±9.9\\ 
        bipolar,retract,cystic-pedicle && 00.8±0.1 & 37.6±0.1 & 38.2±0.1 && 00.5±0.1 & 01.4±0.1 & 07.3±0.1 && irrigator,retract,liver && 16.7±3.2 & 24.7±9.9 & 27.5±6.0 && 15.2±6.1 & 21.7±9.9 & 24.5±4.9\\ 
        bipolar,retract,gallbladder && 01.0±0.2 & 01.9±1.5 & 01.7±0.7 && 00.8±0.5 & 03.5±2.8 & 18.0±9.9 && irrigator,retract,omentum && 05.1±6.7 & 02.8±2.7 & 11.0±9.9 && 06.8±7.2 & 23.6±9.9 & 03.2±1.9\\ 
        bipolar,retract,liver && 15.7±9.9 & 11.0±5.4 & 13.3±6.5 && 13.4±4.3 & 12.4±5.2 & 12.9±6.0 && grasper,null-verb,null-target && 22.6±4.8 & 23.0±4.9 & 24.4±5.6 && 24.5±3.2 & 25.2±4.1 & 24.6±3.5\\ 
        bipolar,retract,omentum && 05.6±4.6 & 14.8±6.7 & 17.1±9.4 && 20.4±8.4 & 21.6±9.9 & 17.9±4.2 && bipolar,null-verb,null-target && 13.0±4.0 & 14.8±7.7 & 14.0±7.4 && 16.6±9.9 & 14.5±8.0 & 19.0±9.9\\ 
        hook,coagulate,blood-vessel && 01.2±1.3 & 00.5±0.1 & 01.7±1.4 && 01.0±0.6 & 01.9±2.7 & 01.6±1.8 && hook,null-verb,null-target && 15.8±4.6 & 17.5±2.3 & 17.0±2.7 && 17.4±1.9 & 18.0±4.3 & 19.7±1.7\\ 
        hook,coagulate,cystic-artery && 00.5±0.1 & 00.6±0.1 & 01.3±0.1 && 00.8±0.1 & 01.3±0.1 & 00.4±0.1 && scissors,null-verb,null-target && 06.8±2.5 & 23.9±9.9 & 15.1±9.9 && 15.4±9.5 & 18.4±8.7 & 18.2±9.9\\ 
        hook,coagulate,cystic-duct && 02.6±3.7 & 00.5±0.5 & 02.8±1.8 && 01.4±1.1 & 01.6±1.4 & 05.1±7.3 && clipper,null-verb,null-target && 25.4±9.9 & 22.6±9.9 & 33.0±9.9 && 15.9±8.5 & 20.4±8.7 & 24.9±9.9\\ 
        hook,coagulate,cystic-pedicle && 00.9±0.4 & 00.5±0.2 & 05.6±7.5 && 01.3±1.3 & 00.7±0.4 & 00.5±0.1 && irrigator,null-verb,null-target && 15.2±9.0 & 14.7±6.6 & 13.1±3.8 && 16.4±8.6 & 16.8±9.2 & 14.0±8.9 \\
        \midrule
        Mean &&&&&&&&&&&& 24.4±04.7 & 27.2±02.7 & \bf29.4±02.8 && 25.3±02.4 & 27.2±02.9 & \bf 29.4±02.5 \\
         \bottomrule
    \end{tabular}
    }
\end{table}


 



 \section{Conclusion} 
With the first public release of the CholecT45 and dataset to support research on surgical action triplet recognition, we present in this paper a standard practice for splitting the dataset to enable a uniform comparison of research methods. These splits remain relevant in the release of the entire CholecT50 dataset.
We also design, implement, and publicly release a python packaged metrics library, {\it ivtmetrics} for the evaluation of surgical action triplet recognition and localization on the dataset.
For the benchmark study, we re-implement the current state-of-the-art models in two predominant deep learning frameworks, PyTorch and TensorFlow, and then train and evaluate them on the proposed data splits.
Owing to the well-articulated rationale for dataset splits and exhaustive cross-validation, results obtained reflect better the generalization capability of the models.
This study sets a rich foundation for fair comparison of methods researched on the CholecT45 and CholecT50 datasets using the same data splits and metrics.
Future work will extend the metrics library to include more statistical evaluations.
 \subsection*{Acknowledgements: }
This work was supported by French state funds managed within the Investissements d`Avenir program by BPI France (project CONDOR) and by the ANR under references ANR-11- LABX-0004 (Labex CAMI), ANR-16-CE33-0009 (DeepSurg), ANR-10-IAHU-02 (IHU Strasbourg) and ANR-20-CHIA-0029-01 (National AI Chair AI4ORSafety). 
It was granted access to the HPC resources of Unistra Mesocentre and GENCI-IDRIS (Grant 2021-AD011011638R1).
The authors also thank the IHU and IRCAD research teams for their help with the initial data annotation during the CONDOR project.

This paper is constantly updated with methods (+ results) that follows the recommended data splits and metrics for surgical action triplet recognition and detection. \bibliographystyle{IEEEtran}
\bibliography{main} 




















































\end{document}
