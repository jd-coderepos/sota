\documentclass[letter,11pt]{article}
\usepackage{amsmath,amssymb,amsthm,color,graphicx}
\usepackage{euscript}
\usepackage{times}


\setlength{\textwidth}{6.5in}
\setlength{\topmargin}{-0.4in}
\setlength{\headheight}{0in}
\setlength{\headsep}{0.0in}
\setlength{\textheight}{9in}
\setlength{\oddsidemargin}{0in}
\setlength{\evensidemargin}{0in}








\newcommand{\seclab}[1]{\label{sec:#1}}
\newcommand{\theolab}[1]{\label{theo:#1}}
\newcommand{\chaplab}[1]{\label{chap:#1}}
\newcommand{\eqnlab}[1]{\label{eqn:#1}}
\newcommand{\corlab}[1]{\label{cor:#1}}
\newcommand{\figlab}[1]{\label{fig:#1}}
\newcommand{\tablab}[1]{\label{tab:#1}}


\newtheorem{theorem}{Theorem}[section]
\newtheorem{corollary}[theorem]{Corollary}
\newtheorem{proposition}[theorem]{Proposition}
\newtheorem{lemma}[theorem]{Lemma}
\newtheorem{claim}[theorem]{Claim}
\graphicspath{{Figs/}}

\def\NN{\EuScript{N}}
\def\A{\mathcal{A}}
\def \reals{{\mathbb R}}
\def \L{{L}}
\def \sphere{{\mathbb S}}
\def\Re{{\mathbb R}}
\def\dirtour{{\mathcal D}}
\def\xitour {{\mathcal V}}
\def\W{{\mathcal W}}
\def\bd{{\partial}}
\def\eps{{\varepsilon}}
\def\eps{{\varepsilon}}
\def\poly{\diamond}
\def\php{\varphi^\poly_j}
\def\C{\EuScript{C}}
\def\S{\mathcal{S}}
\def\E{\mathcal{E}}
\def\F{\mathcal{F}}
\def\K{\EuScript{K}}
\def\T{\EuScript{T}}
\def\Triag{\mathbb{T}}
\def\PT{\mathbb{PT}}
\def\KK{\mathbb{K}}
\def\Chain{\mathcal{C}}
\def\CC{\mathbb{C}}
\def\TT{{\cal T}}
\newcommand{\ignore}[1]{}

\def\inprod#1#2{\langle #1, #2\rangle}

\def\bisect{b}
\def\Bisect{\EuScript{B}}
\def\Graph{\mathbb{H}}
\def\CH{\mathbb{CH}}
\def\Nbrs{N}
\def\tour{\K}

\def\indset#1#2#3{#1_{#2}^{(#3)}}

\def\distfn{\varphi}
\def\tree{\EuScript{T}}
\def\canonical{\C}
\def\extrpt{\psi}
\def\extrset{\Psi}
\def\extrpair{\xi}
\def\nn{\nu}
\def\X{\mathcal{X}}
\def\Y{\mathcal{Y}}
\def\conv{\mathcal{CH}}
\def\DDG{\mathop{\mathrm{DDG}}}
\def\G{{\sf G}}
\def\SDG{\mathop{\mathrm{SDG}}}
\def\PSDG{\mathop{\mathrm{PSDG}}}
\def\DT{\mathop{\mathrm{DT}}}
\def\VD{\mathop{\mathrm{VD}}}
\def\Vor{\mathop{\mathrm{Vor}}}
\def\intr{\mathop{\mathrm{int}}}

\def\marrow{{\marginpar[\hfill]{}}}

\def\leo#1{{\sc Leo says: }{\marrow\sf #1}}
\def\jie#1{{\sc Jie says: }{\marrow\sf #1}}
\def\pankaj#1{{\sc Pankaj says: }{\marrow\sf #1}}
\def\vladlen#1{{\sc Vladlen says: }{\marrow\sf #1}}
\def\micha#1{{\sc Micha says: }{\marrow\sf #1}}
\def\haim#1{{\sc Haim says: }{\marrow\sf #1}}
\def\natan#1{{\sc Natan says: }{\marrow\sf #1}}

\begin{document}

\title{On topological changes in the Delaunay triangulation of moving points}
\author{Natan Rubin\thanks{Department of Mathematics and Computer Science, Freie Universit\"{a}t Berlin, Takustr. 9, 14195, Berlin, Germany.
Email: {\tt rubinnat@tau.ac.il}. Work on this paper was partly supported by Minerva Postdoctoral Fellowship from Max Planck Society, and by Grant 338/09 from the Israel Science Fund.}}


\maketitle
\begin{abstract}
Let  be a collection of  points moving along pseudo-algebraic trajectories in the plane.\footnote{So, in particular, threre are constants  such that any four points are co-circular at most  times, and any three points are collinear at most  times.} 
One of the hardest open problems in combinatorial and computational geometry is to obtain a nearly quadratic upper bound, or at least a subcubic bound, on the maximum number of discrete changes that the Delaunay triangulation  of  experiences during the motion of the points of .

In this paper we obtain an upper bound of , for any , under the assumptions that (i) any four points can be co-circular at most twice, and (ii) either no triple of points can be collinear more than twice, or no ordered triple of points can be collinear more than once.
\end{abstract}

\section{Introduction}\label{Sec:Intro}
\paragraph{Delaunay triangulations.} 
Let  be a finite set of points in the plane. 
Let  and  denote the Voronoi diagram and Delaunay
triangulation of , respectively. For a point , let
 denote the Voronoi cell of . 
The Delaunay triangulation  consists of all 
triangles spanned by  whose circumcircles do not contain points of  in their
interior. A pair of points  is connected by a Delaunay edge if and only if
there is a circle passing through 
and  that does not contain any point of  in its interior.
Delaunay triangulations and  their duals, Voronoi diagrams, are fundamental to much 
of computational geometry and its applications. 
See \cite{AK,Ed2} for a survey and a
textbook on these structures.

In many applications of Delaunay/Voronoi methods (e.g., mesh generation and kinetic collision detection) the points of the input set  are moving continuously, so
these diagrams need to be efficiently updated during the motion.
Even though the motion of the points is continuous, the combinatorial and topological structure of the Voronoi and
Delaunay diagrams change only at
discrete times when certain critical events occur. 


For the purpose of kinetic maintenance, Delaunay triangulations are 
nice structures, because, as mentioned above, they admit local 
certifications associated with individual triangles.  This makes 
it simple to maintain  under point motion: an update is 
necessary only when one of these empty circumcircle conditions 
fails---this corresponds to co-circularities of certain subsets of
four points.\footnote{We assume that the motion of the points is sufficiently generic, so that no more than four points can become co-circular at any given time.} Whenever such an event happens, 
a single edge flip easily restores Delaunayhood. 



Let  be the number of moving points in . 
We assume that the
points move with so-called pseudo-algebraic motions of constant description complexity, meaning (in particular) that any four points are co-circular at most  times, for some constant .  
By using lower-envelope techniques in certain parametric planes, 
Fu and Lee \cite{FuLee} and
Guibas et al.~\cite{gmr-vdmpp-92} show roughly cubic upper bounds on the number of discrete (also known as \textit{topological}) changes in . 
The latter study~\cite{gmr-vdmpp-92} obtains an upper bound of
, where  is the maximum length
of an -Davenport-Schinzel sequence~\cite{SA95}.


If each point of  is moving along a straight line and with the same speed, a slightly better upper bound of  can be established for the number of discrete changes experienced by  (see, e.g., \cite{Vladlen}). A substantial gap exists between these upper bounds
and the best known quadratic lower bound~\cite{SA95}. Closing this gap has been in the 
computational geometry lore for many years, and is considered as one of the major (and very difficult) problems in the field; see \cite{TOPP}.



The instances of the general problem for which the number of discrete changes in  is provably sub-cubic, are strikingly few. It is worth mentioning the result of Koltun \cite{Vladlen} which deals with sets of points moving along straight lines with equal speeds such that all points start their motion from a fixed line. In this particular case one can show that any four points are co-circular at most twice, and any three points are co-linear at most once. (The result of \cite{Vladlen} is not topological and relies on the equations of point trajectories.)

Due to the very slow progress on the above general problem, several alternative lines of study have emerged in the last two decades.

Chew \cite{Chew} proved that the Voronoi diagram undergoes only a near-quadratic number of discrete changes if it is defined with respect to a so called ``polygonal" distance function. The dual representation of such a diagram  yields a proper triangulation of a certain connected subregion of the convex hull of . Agarwal et al.~\cite{Stable} use the above polygonal structures
to efficiently maintain the so called {\it -stable} subgraph , whose edges are robust with respect to small changes in the underlying norm. 

Another line of research \cite{ABGHZ,AWY,KRS} asks if one can define (and efficiently maintain) a proper triangulation of the convex hull of , which would change only near-quadratically many times during the motion of . 
The most recent such study \cite{KRS} provides a (relatively) simple such triangulation which undergoes, in expectation, only  discrete changes (where  is the maximum possible number of collinearities defined by any three points of ).



\smallskip
\noindent{\bf Our result.}
We study the case in which (i) any four points of  are co-circular at most {\it twice} during the motion, and (ii) either every unordered triple can be collinear at most twice or every ordered triple of points can be collinear at most once\footnote{That is, there can be only one collinearity of an ordered triple  so that the points appear in this order along the common line.}, and derive a nearly tight upper bound of , for any , on the number of discrete changes experienced by  during the motion in either of these cases.
We believe that our results constitute a substantial progress towards establishing nearly quadratic (or just sub-cubic) bounds for more general instances of the problem, such as the simple and natural instance of points moving along straight lines with equal speeds. In this case any four points admit at most {\it three} co-circularities, and any triple of points can be collinear at most twice. We believe that the tools developed in this paper can be extended to tackle this instance, and possibly also the general case.

\smallskip
\noindent{\bf Proof overview and organization.}
The majority of the discrete changes in  occur at moments  when some four points  are co-circular, and the corresponding circumdisc contains no other points of . We refer to these events as Delaunay co-circularities. Suppose that  appear along their common circumcircle in this order, so  and  form the chords of the quadrilateral spanned by these points. Right before , one of the chords, say , is Delaunay and thus admits a -empty disc whose boundary contains  and . 
Right after time , the edge  is replaced in  by .  
Informally, this happens because the Delaunayhood of  is violated by  and : Any disc whose boundary contains  and  contains at least one of the points . 
If  does not re-enter  after time , we can charge the event at time  to the edge . We thus assume that  is again Delaunay at some moment . In particular, at that moment the Delaunayhood of  is no longer violated by  and . Before this happens, either at least one of  or  must hit , or an additional co-circularity of  must occur during . Using our assumption that  induce at most two co-circularity events, we can guarantee (up to a reversal of the time axis) that the co-circularity at time  is the last co-circularity of these four points. Thus, one of , let it be , must cross  during .

Our goal is to derive a recurrence formula for the maximum number  of such Delaunay co-circularities induced by any set  of  points (whose motion satisfies the above conditions). 

As a preparation, we study, in Section \ref{Sec:Prelim}, the set of all co-circularities that involve the disappearing Delaunay edge  and some other pair of points of  and occur during the period  when  is absent from .
This is done in a fairly general setting, where any four points of  can be co-circular, and any three points 
of  can be collinear, at most constantly many times. 
Along the way, we establish several structural results which (as we believe) are of independent interest.

In Section \ref{Sec:DelCocircs} we use the general machinery of Section \ref{Sec:Prelim} to obtain a recurrence formula for  in the case where any four points of  are co-circular at most twice.
Recall that  leaves  at such a Delaunay co-circularity, at some time , in order to re-enter  at some later time , and  is hit by the point  in the interval  of its non-Delaunayhood.

If we find at least  ``shallow" co-circularities\footnote{Each of these co-circularities would become a Delaunay co-circularity after removal of at most  points of .}, whose respective circumdiscs (i) touch  and , and (ii) contain at most  points of , we charge them for the disappearance of . We use the standard probabilistic technique of Clarkson and Shor \cite{CS} to show that the number of Delaunay co-circularities, for which our simple charging works, is .
Informally, such Delaunay co-circularities contribute a nearly quadratic term to the overall recurrence formula (see, e.g., \cite{ASS} and \cite{ConstantLines}). Similarly, if we find a ``shallow" collinearity of  and another point (one halfplane bounded by the line of collinearity contains at most  points) we charge the disappearance of  to this collinearity. A combination of the Clarkson-Shor technique with the known near-quadratic bound on the number of topological changes in the convex hull of  (see \cite[Section 8.6.1]{SA95}) yields a near-quadratic bound in this case.

It thus remains to bound the number of Delaunay co-circularities for which  and  participate in fewer ``shallow" co-circularities and in no ``shallow" collinearity during . In this case, using the general properties established in Section \ref{Sec:Prelim}, one can restore the Delaunayhood of  throughout  by removal of some subset  of  points of . In particular, the point , which crosses , must belong to .
In the smaller Delaunay triangulation , the edge  undergoes a complex process referred to as a {\it Delaunay crossing by} . 

In Section \ref{Sec:CrossOnce}, we derive a recurrence formula for the number of these Delaunay crossings. 
This is achieved by establishing several structural properties of these novel configurations.
Combined with the analysis of Section \ref{Sec:Prelim}, this yields the desired ``near-quadratic" recurrence for the number of Delaunay co-circularities. 



\section{Preliminaries}\label{Sec:Prelim}
In this section we define the basic notions regarding Delaunay triangulations of moving points, and introduce some of the key techniques which will be repeatedly used in the rest of the paper. 


\paragraph{Delaunay co-circularities.} Let  be a collection of  points moving along pseudo-algebraic trajectories in the plane. That is, there exist constants  and  so that any four points are co-circular at most  times, and any three points are collinear at most  times. (As far as this section is concerned, we do not impose any further restrictions on the choice of  and , except for their being constant.)

We may assume, without loss of generality, that the trajectories of the points of  satisfy all the standard general position assumptions. That is, no five points can become co-circular during the motion, no four points can become collinear, no two points can coincide, and no two events of either a co-circularity of four points or of collinearity of three points can occur simultaneously.
In addition, we assume that in every co-circularity event involving some four points , each of the points, say , crosses the circumcircle of the other three points ; that is, it lies outside the circle right before the event and inside right afterwards, or vice versa. Similarly, we assume that in every collinearity event involving some triple of points of , each of the points crosses the line through the remaining two points.
Degeneracies in the point trajectories of the above kinds can be handled, both algorithmically and combinatorially, by any of the standard symbolic perturbation techniques, such as simulation of simplicity \cite{EM}; for combinatorial purposes, a sufficiently small generic perturbation of the motions will get rid of any such degeneracy, without decreasing the number of topological changes in the diagram.

\begin{figure}[htbp]
\begin{center}
\input{DelaunayCocirc.pstex_t}\hspace{2cm}\begin{picture}(0,0)\includegraphics{HullEvent.pstex}\end{picture}\setlength{\unitlength}{3158sp}\begingroup\makeatletter\ifx\SetFigFont\undefined \gdef\SetFigFont#1#2#3#4#5{\reset@font\fontsize{#1}{#2pt}\fontfamily{#3}\fontseries{#4}\fontshape{#5}\selectfont}\fi\endgroup \begin{picture}(1741,1432)(1898,-1410)
\put(2411,-1124){\makebox(0,0)[lb]{\smash{{\SetFigFont{9}{10.8}{\rmdefault}{\mddefault}{\updefault}{\color[rgb]{0,0,.56}}}}}}
\put(3072,-1355){\makebox(0,0)[lb]{\smash{{\SetFigFont{9}{10.8}{\rmdefault}{\mddefault}{\updefault}{\color[rgb]{1,0,0}}}}}}
\put(1914,-845){\makebox(0,0)[lb]{\smash{{\SetFigFont{9}{10.8}{\rmdefault}{\mddefault}{\updefault}{\color[rgb]{1,0,0}}}}}}
\end{picture} \caption{\small{Left: A Delaunay co-circularity of . An old Delaunay edge  is replaced by the new edge . Right: A collinearity of  right before  ceases being a vertex on the boundary of the convex hull. 
}} \label{Fig:DelaunayEvents}
\end{center}
\end{figure} 

The Delaunay triangulation  changes at discrete time moments  when one of the following two types of events occurs.

(i) Some four points   of  become co-circular, so that the cicrumdisc of  is {\it empty}, i.e., does not contain any point of  in its interior. We refer to such events as {\it Delaunay co-circularities}, to distinguish them from non-Delaunay co-circularities, for which the circumdisc of  is nonempty, that is, contains one or more points of  in its interior.\footnote{Strictly speaking,  is not a triangulation at the time  of such a co-circularity, because it contains then a pair of {\it crossing} edges, say  and  (as depicted in Figure \ref{Fig:DelaunayEvents} (left)).} See Figure \ref{Fig:DelaunayEvents} (left).


In what follows, we shall use  to denote the maximum possible number of Delaunay co-circularities induced by {\it any} set  of  points whose motion satisfies the above general assumptions.\footnote{In the subsequent sections, we shall impose additional restrictions on the pseudo-algebraic motions of the points of , thereby  redefining .}

(ii) Some three points  of  become collinear on the boundary of the convex hull of . Assume that  lies between  and . In this case, if  moves into the interior of the hull, then, right after this event, the triangle  becomes a new Delaunay triangle. Similarly, if  moves outside and becomes a new vertex, then, right before this event, the old Delaunay triangulation  contained the old Delaunay triangle , which has shrinked to a segment and disappeared at the event. See Figure \ref{Fig:DelaunayEvents} (right).
The number of such collinearities on the convex hull boundary is known to be at most nearly quadratic; see, e.g., \cite[Section 8.6.1]{SA95} and below.




\paragraph{Shallow co-circularities and the Clarkson-Shor argument.}
We say that a co-circularity event has {\it level}  if its corresponding circumdisc contains exactly  points of  in its interior. In particular, the Delaunay co-circularities have level . The co-circularities having level at most  are called {\it -shallow}.

We can express the maximum possible number of -shallow co-circularities in  in terms of the more elementary quanitity  via the following fairly general argument, first introduced by Clarkson and Shor \cite{CS}.
(With no loss of generalty, we assume that , for otherwise we can trivially bound the maximum number of Delaunay, that is, -shallow co-circularities in  by .)

Let  be the time of a -shallow co-circularity which involves some four points  in , and let  denote the set of at most  points that lie at time  in the interior of the common circumdisc of . Note that the above co-circularity is Delaunay with respect to , and with respect to any subset  of  which contains .

We sample at random (and without replacement) a subset  of  points. As is easy to check, the following two events occur {\it simultaneously} with probability at least : (1) the sample  contains the four points , and (2) none of the points of  belongs to . (An explicit calculation of the above probability can be found in several classical texts, such as \cite{CS} or \cite{SA95}.)

In the case of success, the aforementioned -shallow co-circularity in  becomes a Delaunay co-circularity with respect to . 
Hence, the overall number of -shallow co-circularities in  is .


\paragraph{Shallow collinearities.} Similar notations apply to collinearities of triples of points . A collinearity of  is called {\it -shallow} if the number of points of  to the left, or to the right, of the line through  is at most . 

The (essentially) same probabilistic argument implies that the number of such events, for , is , where  denote the maximum number of discrete changes on the convex hull of an -point subset of . (The difference in the exponent of  follows because now each configuration at hand involves only three points.)

As shown, e.g., in \cite[Section 8.6.1]{SA95},  , where  is an extremely slowly growing function.\footnote{Specifically,
, where  is the maximum length of an -Davenport-Schinzel sequence (see Section \ref{Sec:Intro}), and  is the maximum number of collinearities of any fixed triple of points. The pseudo-algebraicity of the motion implies that  is a constant, but we will restrict  further; see below.} We thus get that the number of -shallow collinearities is . 

\begin{figure}[htbp]
\begin{center}
\input{Circumdisc.pstex_t}\hspace{2cm}\begin{picture}(0,0)\includegraphics{RedBlueFuncs.pstex}\end{picture}\setlength{\unitlength}{3552sp}\begingroup\makeatletter\ifx\SetFigFont\undefined \gdef\SetFigFont#1#2#3#4#5{\reset@font\fontsize{#1}{#2pt}\fontfamily{#3}\fontseries{#4}\fontshape{#5}\selectfont}\fi\endgroup \begin{picture}(2530,2086)(948,-1902)
\put(2482, 14){\makebox(0,0)[lb]{\smash{{\SetFigFont{10}{12.0}{\rmdefault}{\mddefault}{\updefault}{\color[rgb]{0,0,.56}}}}}}
\put(3176,-846){\makebox(0,0)[lb]{\smash{{\SetFigFont{10}{12.0}{\rmdefault}{\mddefault}{\updefault}{\color[rgb]{1,0,0}}}}}}
\put(963,-1359){\makebox(0,0)[lb]{\smash{{\SetFigFont{10}{12.0}{\rmdefault}{\mddefault}{\updefault}{\color[rgb]{0,0,.56}}}}}}
\put(1636,-1838){\makebox(0,0)[lb]{\smash{{\SetFigFont{10}{12.0}{\rmdefault}{\mddefault}{\updefault}{\color[rgb]{0,0,0}}}}}}
\put(3223,-353){\makebox(0,0)[lb]{\smash{{\SetFigFont{10}{12.0}{\rmdefault}{\mddefault}{\updefault}{\color[rgb]{1,0,0}}}}}}
\put(2041,-1539){\makebox(0,0)[lb]{\smash{{\SetFigFont{10}{12.0}{\rmdefault}{\mddefault}{\updefault}{\color[rgb]{0,0,0}}}}}}
\put(2791,-474){\makebox(0,0)[lb]{\smash{{\SetFigFont{10}{12.0}{\rmdefault}{\mddefault}{\updefault}{\color[rgb]{0,0,0}}}}}}
\put(1148,-953){\makebox(0,0)[lb]{\smash{{\SetFigFont{10}{12.0}{\rmdefault}{\mddefault}{\updefault}{\color[rgb]{0,0,.56}}}}}}
\put(3217,-1275){\makebox(0,0)[lb]{\smash{{\SetFigFont{10}{12.0}{\rmdefault}{\mddefault}{\updefault}{\color[rgb]{1,0,0}}}}}}
\put(1711,-241){\makebox(0,0)[lb]{\smash{{\SetFigFont{10}{12.0}{\rmdefault}{\mddefault}{\updefault}{\color[rgb]{0,0,.56}}}}}}
\put(2483,-1493){\makebox(0,0)[lb]{\smash{{\SetFigFont{10}{12.0}{\rmdefault}{\mddefault}{\updefault}{\color[rgb]{1,0,0}}}}}}
\end{picture} \caption{\small Left: The circumdisc  of  and  moves continuously as long as these three points are not collinear, and then flips over to the other side of the line of collinearity after the collinearity. 
Right: A snapshot at moment . In the depicted configuration we have .}
\label{Fig:RedBlueFuncs}
\vspace{-0.5cm}
\end{center}
\end{figure} 



\paragraph{The red-blue arrangement.} For every pair of points  of  we construct a two-dimensional arrangement which ``encodes" all the collinearities and co-circularities that involve  and  (together with one or two additional points). This is done as follows.

For every ordered pair  of points of , we
denote by  the line passing through  and  and oriented from  to . 
Define  (resp., ) to be the halfplane to the left (resp., right) of .
Notice that  moves continuously with  and  (since, by assumption,  and  never coincide during the motion). Note also that  and  are oppositely oriented and that  and .
Accordingly, we orient the edge  connecting  and  from  to , so that the edges  and  have opposite orientations.


Any three points  span a circumdisc  which moves continuously with  as long as  are not collinear. See Figure \ref{Fig:RedBlueFuncs} (left). When  become collinear, say, when  crosses  from  to , the circumdisc  changes instantly from being all of  to all of .
Similarly, when  crosses  from  to  {\it outside} , the circumdisc changes instantly from  to . Symmetric changes occur when  crosses  from  to .

 
For a fixed ordered pair , we call a point  of  {\it red} (with respect to the oriented edge ) if ; otherwise it is {\it blue}.

As in \cite{gmr-vdmpp-92}, we define, for each , a pair of partial functions  over the time axis as follows.
If  at time  then  is undefined, and  is the signed distance of the center  of  from ; it is positive (resp., negative) if  lies in  (resp., in ). A symmetric definition applies when . Here too  is positive (resp., negative) if the center of  lies in  (resp., in ). We refer to  as the {\it red function} of  (with respect to ) and to  as the {\it blue function} of . Note that at all times when  are not collinear, exactly one of  is defined. See Figure \ref{Fig:RedBlueFuncs} (right).
The common points of discontinuity of  occur at moments when  crosses . Specifically,  tends to  before  crosses  from  to  outside the segment , and it tends to  when  does so within ; the behavior of  is fully symmetric. 




Let  denote the lower envelope of the red functions, and let  denote the upper envelope of the blue functions. The edge  is a Delaunay edge at time  if and only if . Any disc whose bounding circle passes through  and  which is centered anywhere in the interval  along the perpendicular bisector of  is empty at time , and thus serves as a witness to  being Delaunay.
If  is not Delaunay at time , there is a pair of a red function  and a blue function  such that .
For example, we can take  (resp., ) to be the function attaining  (resp., ) at time . In such a case, we say that the Delaunayhood of  is {\it violated} by the pair of points  which define . See Figure \ref{Fig:Envelopes}. Note that in general there can be many pairs  that violate  (quadratically many in the worst case).




\begin{figure}[htbp]
\begin{center}
\begin{picture}(0,0)\includegraphics{Envelopes.pstex}\end{picture}\setlength{\unitlength}{3355sp}\begingroup\makeatletter\ifx\SetFigFont\undefined \gdef\SetFigFont#1#2#3#4#5{\reset@font\fontsize{#1}{#2pt}\fontfamily{#3}\fontseries{#4}\fontshape{#5}\selectfont}\fi\endgroup \begin{picture}(1811,1571)(1524,-1750)
\put(2619,-867){\makebox(0,0)[lb]{\smash{{\SetFigFont{9}{10.8}{\rmdefault}{\mddefault}{\updefault}{\color[rgb]{1,0,0}}}}}}
\put(2041,-1539){\makebox(0,0)[lb]{\smash{{\SetFigFont{9}{10.8}{\rmdefault}{\mddefault}{\updefault}{\color[rgb]{0,0,0}}}}}}
\put(2791,-474){\makebox(0,0)[lb]{\smash{{\SetFigFont{9}{10.8}{\rmdefault}{\mddefault}{\updefault}{\color[rgb]{0,0,0}}}}}}
\put(2134,-452){\makebox(0,0)[lb]{\smash{{\SetFigFont{9}{10.8}{\rmdefault}{\mddefault}{\updefault}{\color[rgb]{0,0,.56}}}}}}
\end{picture} \caption{\small Snapshot at a fixed moment : The red envelope  coincides with the red function . The blue envelope coincides with the blue function . Note that  is not a Delaunay edge because  (represented by the hollow center) is smaller than  (represented by the shaded center).}
\label{Fig:Envelopes}
\end{center}
\end{figure} 



Hence, at any time when the edge  joins or leaves , via a Delaunay co-circularity involving , , and two other points of , we have . In this case the two other points, , are such that one of them, say , lies in  and  lies in , and .


\smallskip
\noindent{\it Remark.} 
Right before the edge  is crossed by a red point , the corresponding function  lies below all the blue functions  (if they exist), so the Delaunayhood of  is violated by each of the subsequent pairs . 
In other words, the edge  cannot be Delaunay right before (resp., after) being hit by a point of , unless  joins or leaves the convex hull of . 

\smallskip

Let  denote the arrangement of the  functions , for , drawn in the parametric -plane, where  is the time and  measures signed distance along the perpendicular bisector of . We label each vertex of  as red-red, blue-blue, or red-blue, according to the colors of the two functions meeting at the vertex. Note that our general position assumptions imply that  is also in general position, so that no three function pass through a common vertex, and no pair of functions are tangent to each other. Note also that the functions forming  have in general discontinuities, at the corresponding collinearities. At each such collinearity, a red function  tends to  or  on one side of the critical time, and is replaced on the other side by the corresponding blue function  which tends to  or , respectively.

An intersection between two red functions  corresponds to a co-circularity event which involves  and , occurring when both  and  lie in .
Similarly, an intersection of two blue functions  corresponds to a co-circularity event
involving  where both  and  lie in . Also, an intersection of a red fuction  and a blue function  represents a co-circularity of , where  and . We label these co-circularities, as we labeled the vertices of , as red-red, blue-blue, and red-blue (all with respect to ), depending on the respective colors of  and .


It is instructive to note that in any co-circularity of four points of  there are exactly two pairs (the opposite pairs in the co-circularity)
with respect to which the co-circularity is red-blue, and four pairs (the adjacent pairs) with respect to which the co-circularity is ``monochromatic". Suppose that the above co-circularity is Delaunay. Then the two pairs for which the co-circularity is red-blue are those that enter or leave the Delaunay triangulation  (one pair enters and one leaves). The Delaunayhood of pairs for which the co-circularity is monochromatic is not affected by the co-circularity, which appears in the corresponding arrangement as a {\it breakpoint} of either  or of .



\begin{figure}[htbp]
\begin{center}
\input{RedRedCocirc.pstex_t}\hspace{2cm}\begin{picture}(0,0)\includegraphics{RedBlueCocirc.pstex}\end{picture}\setlength{\unitlength}{3158sp}\begingroup\makeatletter\ifx\SetFigFont\undefined \gdef\SetFigFont#1#2#3#4#5{\reset@font\fontsize{#1}{#2pt}\fontfamily{#3}\fontseries{#4}\fontshape{#5}\selectfont}\fi\endgroup \begin{picture}(1537,1570)(1798,-1734)
\put(1922,-747){\makebox(0,0)[lb]{\smash{{\SetFigFont{9}{10.8}{\rmdefault}{\mddefault}{\updefault}{\color[rgb]{0,0,.56}}}}}}
\put(2041,-1539){\makebox(0,0)[lb]{\smash{{\SetFigFont{9}{10.8}{\rmdefault}{\mddefault}{\updefault}{\color[rgb]{0,0,0}}}}}}
\put(3217,-1275){\makebox(0,0)[lb]{\smash{{\SetFigFont{9}{10.8}{\rmdefault}{\mddefault}{\updefault}{\color[rgb]{1,0,0}}}}}}
\put(2791,-465){\makebox(0,0)[lb]{\smash{{\SetFigFont{9}{10.8}{\rmdefault}{\mddefault}{\updefault}{\color[rgb]{0,0,0}}}}}}
\end{picture} \caption{\small Intersections between two red functions  and  (left), or a blue function  and a red function  (right), correspond to red-red or red-blue co-circularities.}
\label{Fig:RedBlueCocirc}
\end{center}
\vspace{-0.4cm}
\end{figure} 


The main weakness of the previous approaches \cite{FuLee,gmr-vdmpp-92} is that they study only the lower envelope  of red functions, and the upper envelope  of blue functions, which are merely substructures within the above arrangement .
This yields a roughly linear upper bound on the number of monochromatic co-circularities with respect to the edge  under consideration.
Repeating the same argument for the  possible (oriented) edges  then results in a far too high, near-cubic upper bound on the number of Delaunay co-circularities.

Instead, we exploit the underlying structure of  in order to establish the following
main technical result of this section.

\begin{theorem}\label{Thm:RedBlue}
Let  be a collection of  points moving as described above. Suppose that an edge  belongs to  at (at least) one of two moments  and , for .
Let  be some sufficiently large constant.\footnote{The constants in the  and  notations do not depend on .}
Then one of the following conditions holds:\\
\indent (i) There is a -shallow collinearity which takes place during , and involves ,  and another point .\\
\indent (ii) There are  -shallow red-red, red-blue, or blue-blue co-circularities (with respect to ) which occur during .\\
\indent (iii) There is a subset  of fewer than  points whose removal guarantees that  belongs to  throughout .
\end{theorem}

Notice that we do not assume in Theorem \ref{Thm:RedBlue} that  leaves  at any moment during . 
Nevertheless, suppose that  is the time of a Delaunay co-circularity at which  leaves , and  is the first time after  when  re-enters .
Then Theorem \ref{Thm:RedBlue} relates such Delaunay co-circularities to -shallow collinearities and co-circularities which occur in  when the edge  under consideration is not Delaunay.
Therefore, this theorem can be regarded, in its own right, as one of the main contributions of this paper.






The proof of Theorem \ref{Thm:RedBlue} is based on the following simple idea. Assume that the edge  does not belong to  at a fixed time . If the Delaunayhood of  is violated by  red-blue pairs , then we encounter, during ,  co-circularities (each involving  and the corresponding pair ), or at least  points  change their color there by crossing . 
Finally, if the Delaunayhood of  is violated at time  by only  pairs, then it can be restored by removing a subset  of cardinality at most . 


Impatient readers may safely skip the full proof of Theorem \ref{Thm:RedBlue}, which involves a fairly routine 
planar analysis in the above arrangement  of red and blue curves. (A very similar argument was used in \cite{ASS} to address a totally different problem.)


\paragraph{Proof of Theorem \ref{Thm:RedBlue}.}
Without loss of generality, we assume that the edge  is Delaunay at time . (If  is Delaunay at time  then we can argue in a fully symmetrical fashion.)

Consider the portion of the red-blue arrangement associated with  within the time interval . As above, refer to the parametric plane in which this arrangement is represented as the -plane, where  is the time axis and  measures signed distances  from .
We define the {\it red} (resp., {\it blue}) {\it level} of a point  in this parametric  as the number of red (resp., blue) functions that lie below (resp., above)  (in the -direction). See Figure \ref{Fig:RedBlueLevels}.
It is easily checked that the level of a co-circularity event at time , with circumcenter at distance  from , is the sum of the red and the blue levels of .


\begin{figure}[htbp]
\begin{center}
\input{RedBlueLevels.pstex_t}\hspace{3cm}\begin{picture}(0,0)\includegraphics{RedBlueLevels1.pstex}\end{picture}\setlength{\unitlength}{3552sp}\begingroup\makeatletter\ifx\SetFigFont\undefined \gdef\SetFigFont#1#2#3#4#5{\reset@font\fontsize{#1}{#2pt}\fontfamily{#3}\fontseries{#4}\fontshape{#5}\selectfont}\fi\endgroup \begin{picture}(1959,1751)(1378,-1760)
\put(2771,-471){\makebox(0,0)[lb]{\smash{{\SetFigFont{9}{10.8}{\rmdefault}{\mddefault}{\updefault}{\color[rgb]{0,0,0}}}}}}
\put(2046,-1494){\makebox(0,0)[lb]{\smash{{\SetFigFont{9}{10.8}{\rmdefault}{\mddefault}{\updefault}{\color[rgb]{0,0,0}}}}}}
\put(2575,-1099){\makebox(0,0)[lb]{\smash{{\SetFigFont{9}{10.8}{\rmdefault}{\mddefault}{\updefault}{\color[rgb]{0,0,0}}}}}}
\put(2532,-1385){\makebox(0,0)[lb]{\smash{{\SetFigFont{9}{10.8}{\rmdefault}{\mddefault}{\updefault}{\color[rgb]{1,0,0}}}}}}
\put(2710,-895){\makebox(0,0)[lb]{\smash{{\SetFigFont{9}{10.8}{\rmdefault}{\mddefault}{\updefault}{\color[rgb]{1,0,0}}}}}}
\put(2380,-727){\makebox(0,0)[lb]{\smash{{\SetFigFont{9}{10.8}{\rmdefault}{\mddefault}{\updefault}{\color[rgb]{0,0,.56}}}}}}
\put(2167,-867){\makebox(0,0)[lb]{\smash{{\SetFigFont{9}{10.8}{\rmdefault}{\mddefault}{\updefault}{\color[rgb]{0,0,.56}}}}}}
\put(2051,-1088){\makebox(0,0)[lb]{\smash{{\SetFigFont{9}{10.8}{\rmdefault}{\mddefault}{\updefault}{\color[rgb]{0,0,.56}}}}}}
\end{picture} \caption{\small Left: The point  lies below three blue functions and above two red functions, so its blue and red levels are  and , respectively. Right: The circumdisc centered at (signed) distance  from  and touching  and  at time  contains the three corresponding blue points and two red points.}
\label{Fig:RedBlueLevels}
\end{center}
\end{figure} 

We distinguish between the following (possibly overlapping) cases:

\smallskip
\noindent {\bf (a)}  and  participate in a -shallow collinearity with a third point  at some moment during . That is, condition (i) is satisfied. (Note that here we do not care whether  crosses  or .)

Suppose that this does not happen. That is, each time when a point  changes its color from red to blue or vice versa, the number of points on each side of  is larger than . Hence, either the number of points on each side of  is always larger than  (during ), or the sets of red and blue points remain fixed throughout  (no crossing takes place), and the size of one of them is at most . More concretely, either one of the sets contains fewer than  points at the start of , and then no crossing can ever occur during , or both sets contain at least  points at the start of , and this property is maintained during , by assumption. In the former case condition (iii) trivially holds, since removal of all points in  or in  guarantees that  is a hull edge throughout , and thus belongs to the Delaunay triangulation. Hence, we may assume that the number of red points, and the number of blue points, are always both larger than  during .

\begin{figure}[htbp]
\begin{center}
\input{DeepDeep.pstex_t} \hspace{2cm} \begin{picture}(0,0)\includegraphics{CrossOutside.pstex}\end{picture}\setlength{\unitlength}{3947sp}\begingroup\makeatletter\ifx\SetFigFont\undefined \gdef\SetFigFont#1#2#3#4#5{\reset@font\fontsize{#1}{#2pt}\fontfamily{#3}\fontseries{#4}\fontshape{#5}\selectfont}\fi\endgroup \begin{picture}(1484,2101)(1785,-1917)
\put(2962,-196){\makebox(0,0)[lb]{\smash{{\SetFigFont{10}{12.0}{\rmdefault}{\mddefault}{\updefault}{\color[rgb]{1,0,0}}}}}}
\put(1907,-1857){\makebox(0,0)[lb]{\smash{{\SetFigFont{10}{12.0}{\rmdefault}{\mddefault}{\updefault}{\color[rgb]{0,0,0}}}}}}
\put(2753,-545){\makebox(0,0)[lb]{\smash{{\SetFigFont{10}{12.0}{\rmdefault}{\mddefault}{\updefault}{\color[rgb]{0,0,0}}}}}}
\put(2118,-1472){\makebox(0,0)[lb]{\smash{{\SetFigFont{10}{12.0}{\rmdefault}{\mddefault}{\updefault}{\color[rgb]{0,0,0}}}}}}
\end{picture} \caption{\small Left: Case (b). The disc  contains at least  red points, and at least  blue points. If  lies at red level at most , it belongs to . Hence, the circumdisc  contains at least  blue points, so the blue level of  is at least . Right: Case (c). The setup right after time  when  crosses .  contains at least  red points and no blue points.}
\label{Fig:DeepDisc}
\end{center}
\end{figure} 

\smallskip
\noindent {\bf (b)} At some moment  there is a disc  that touches  and , and contains at least  red points and at least  blue points.
In particular, for each of the  shallowest red functions  at time , its respective red point  belongs to  and similarly for the  shallowest blue functions.
See Figure \ref{Fig:DeepDisc} (left). Before we use the existence of  we first conduct the following structural analysis.

Let  be a red function which is defined at time , and whose red level is then at most . (Recall that, at time , the blue level of any red function is  since  belongs to .) We claim that either  is defined and continuous throughout  and its red level is always at most , or  participates in at least  red-red and/or red-blue co-circularities, all of which are -shallow. 

Indeed, the circumdisc  contains at most  red points (and no blue points) at time , and it moves continuously as long as  remains in . By the time at which either (the graph of)  reaches red level  or  hits , this disc ``swallows" either at least  red points (either in the former case or in the latter case when  crosses ) or at least  blue points (in the latter case when  crosses ). (Recall that, by assumption, the number of red points and the number of blue points is always larger than  during .) We thus obtain at least  -shallow red-red or red-blue co-circularities involving  and a fourth (red or blue) point. 

To recap, if at least  red functions, which at time  are among the  shallowest red functions, reach red level at least , or have a discontinuity at  or  (at a crossing of  by the corresponding point), then we encounter  co-circularities (involving  and ) which are -shallow, so condition (ii) holds.

Hence, we may assume that at least  red functions  that are among the  shallowest red functions at time , are defined throughout , and their red level always remains at most .
Fix any such red function .
Clearly, the red point  that defines  belongs to  at time , and the circumdisc  contains at least  blue points. See Figure \ref{Fig:DeepDisc} (left).
This implies that the blue level of  reaches  so (since the blue level was  at time )  participates in at least  -shallow co-circularities during . Repeating this argument for each of the remaining  such red functions, we conclude that condition (ii) is again satisfied.

\smallskip
\noindent {\bf(c)} Suppose that neither of the two cases (a), (b) holds. 
Let  (resp., ) be the subset of all points  whose red (resp., blue) functions  (resp., ) appear at red (resp., blue) level at most  at some moment during . 

Since the situation in (b) does not occur, we can restore the Delaunayhood of , throughout the entire interval , by removing all points in . To see this, suppose that  is not Delaunay (in ) at some time . This is witnessed by a disc  whose boundary passes through  and  and which contains a red point  and a blue point . Since the red level of  is greater than  at time ,  must also contain the  red points corresponding to the  shallowest red functions at time , and, symmetrically, also the  blue points corresponding to the  shallowest blue functions at time . But then  satisfies the condition (b), contrary to assumption. 



Let  (resp., ) be the set of  points whose red (resp., blue) functions are shallowest at time .
It remains to consider the case where at least  points  in  belong to neither of , for otherwise condition (iii) is trivially satisfied, with a removed set of size at most .
Fix such a point  and consider the first time  when its red function  has red level at most , or its blue function  has blue level at most . Without loss of generality, suppose that at time  the red function  has red level at most .
We claim that  does not cross  during . Indeed, if there were such a crossing from  to  then the blue function  would tend to  right before the crossing, and its blue level would then be  even before , contrary to the choice of . Similarly, if the crossing were from  to  then the red level of  would be  just before the crossing, again contradicting the choice of . 

First, assume that  does not cross  during , so the graph of  is continuous during this time interval. Hence, the motion of the circumdisc  is also continuous.
Since , at time  the circumdisc  contains at least  red points and no blue points. At time ,
 contains  red points and fewer than  blue points (otherwise Case (b) would occur). 
Hence, we encounter at least  -shallow co-circularities during , each involving  and some other point of .


Now, suppose  crosses  during , and consider the last time  when this happens.
We can use exactly the same argument as in the ``continuous" case but now starting from . Indeed,  is continuous during  and, right after , the circumdisc  contains (all the red points and thus) at least  red points, and no blue points. See Figure \ref{Fig:DeepDisc} (right).


Repeating this argument for all such points , we get  -shallow co-circularities which occur during  and involve  and . Hence, condition (ii) is again satisfied. This completes the proof of Theorem \ref{Thm:RedBlue}. 

\paragraph{Combinatorial charging schemes.}
To conclude this section, we briefly review the following general paradigm, which is widely used in computational geometry to bound the combinatorial complexity of various substructures in arrangements of (mostly non-linear) objects; see, e.g., \cite{Envelopes3D,ConstantLines} and \cite[Section 7]{SA95}.

Suppose that we are given two collections  and  of geometric configurations, and wish to upper-bound the cardinality  of  in terms of the cardinality  of .
Note that the configurations in  and  are usually of different types. For example,  and  can consist, respectively, of Delaunay co-circularities and of -shallow collinearities.

The most elementary class of charging schemes (which we shall use throughout this paper) is prescribed by a function  which maps each element  to a subset . 
We then say that that every element  is {\it charged} by .
We also say that a configuration  is charged  times if  contains exactly  configurations  whose respective images  contain . Furthermore, we say that  is charged {\it uniquely} if there is exactly one  whose image  contains  (so  is {\it uniquely determined} by the choice of ).


The resulting relation between  and  depends on the following two parameters  and  associated with our charging rule . The first parameter  denotes the minimum cardinality  of  (over all possible choices of ).
 The second parameter  denotes the maximal possible number  of configurations  whose images contain a given configuration  (where the maximum is taken over all choices of ).
In other words,  denotes the minimum number of configurations in  that can be charged by the same  , and  denotes the maximum number of configurations  that can charge the same .
With the above definitions, a standard double counting argument immediately shows that .

Therefore, in order to obtain the best possible upper estimate of , we seek to maximize , and to minimize 
. 
In all our charging schemes, the mapping  will be constructed explicitly, so the value of  will be clear from the construction (and, most often, equal to , with one significant exception). Thus, the main challenge will be to keep the value of  under control (i.e., make sure that each configuration  is charged by relatively {\it few} members of ).

\section{The Number of Delaunay Co-circularities}\label{Sec:DelCocircs}
In what follows, 
we assume that any four points in the underlying set  are co-circular at most {\it twice} during their pseudo-algebraic motion. 
In this section we show that the maximum possible number  of Delaunay co-circularities in a set , as above, is asymptotically dominated (if it is at least super-quadratic) by
the number of certain carefully defined configurations which will be referred to as {\it Delaunay crossings}. The analysis of Delaunay crossings will be postponed to Section \ref{Sec:CrossOnce}, where we shall impose additional restrictions on the collinearities that can be performed by triples of points in .

\smallskip
\noindent{\it Definition.} We say that a co-circularity event at time  involving  has {\it index}  (resp., ) if this is the first (resp., second) co-circularity involving . 

\smallskip
To bound the maximum possible number of Delaunay co-circularities in , we fix one such event at time , at which an edge  of  is replaced by another edge , because of a red-blue co-circularity (with respect to , and, for that matter, also with respect to ) of level .
Assume first that the co-circularity of  has index ; the case of index  is handled fully symmetrically, by reversing the direction of the time axis.
 
There are at most  such events for which the vanishing edge  never reappears in , so we focus on the Delaunay co-circularities (of index ) whose corresponding edge  rejoins  at some future moment  (or right after it).

Specifically,  experiences at time  either a Delaunay co-circularity or a hull event (at which  is hit by some point of ). In the latter case,  is not strictly Delaunay at time  and appears in  only {\it right after} this event. 

Note that in this case, if the co-circularity at time  involved two other points , then at least one of  must cross  between  and  otherwise  and  would have to become co-circular again, in order to ``free"  from non-Delaunayhood, which is impossible since our co-circularity is assumed to be {last} co-circularity of .

More generally, we have the following topological lemma:
\begin{lemma}\label{Lemma:MustCross}
Assume that the Delaunayhood of  is violated at time  (or rather right after it) by the points  and . 
Furthermore, suppose that  enters  at some future time .
Then at least one of the followings occurs during : 

\medskip
(1) The point  crosses  from  to .\\
\indent(2) The point  crosses  from  to .\\
\indent(3) The four points  are involved in another co-circularity (which is also red-blue with respect to ). \end{lemma}

A symmetric version of Lemma \ref{Lemma:MustCross} applies if the Delaunayhood of  is violated at time  (or right before it) by  and , and this edge is Delaunay at an {\it earlier} time .

\begin{proof}
Refer to Figure \ref{Fig:StaysViolated}.
Clearly, the Delaunayhood of  remains violated by  and  after time  as long as
 remains within the cap , and  remains within the cap  (as depicted in the left figure). 

Consider the first time  when the above state of affairs ceases to hold. Notice that  is intersected by  throughout the interval .
Assume without loss of generality that  leaves the the cap . If  crosses , then the first scenario holds. Otherwise,  can leave the above cap only through the boundary of  (as depicted in the right figure), so the third scenario occurs.
\end{proof}
\begin{figure}[htb]
\begin{center}
\input{AfterLastCocirc.pstex_t}\hspace{2.5cm}\input{CrossWithin.pstex_t}\hspace{2.5cm}\begin{picture}(0,0)\includegraphics{NotOutside.pstex}\end{picture}\setlength{\unitlength}{3552sp}\begingroup\makeatletter\ifx\SetFigFont\undefined \gdef\SetFigFont#1#2#3#4#5{\reset@font\fontsize{#1}{#2pt}\fontfamily{#3}\fontseries{#4}\fontshape{#5}\selectfont}\fi\endgroup \begin{picture}(1052,1257)(1896,-1545)
\put(2791,-474){\makebox(0,0)[lb]{\smash{{\SetFigFont{10}{12.0}{\rmdefault}{\mddefault}{\updefault}{\color[rgb]{0,0,0}}}}}}
\put(2127,-628){\makebox(0,0)[lb]{\smash{{\SetFigFont{10}{12.0}{\rmdefault}{\mddefault}{\updefault}{\color[rgb]{0,0,0}}}}}}
\put(2759,-1444){\makebox(0,0)[lb]{\smash{{\SetFigFont{10}{12.0}{\rmdefault}{\mddefault}{\updefault}{\color[rgb]{0,0,0}}}}}}
\put(2086,-1463){\makebox(0,0)[lb]{\smash{{\SetFigFont{10}{12.0}{\rmdefault}{\mddefault}{\updefault}{\color[rgb]{0,0,0}}}}}}
\end{picture} \caption{\small Proof of Lemma \ref{Lemma:MustCross}. Left: The setup right after time . Center and right: the point  can leave  in two possible ways.}
\label{Fig:StaysViolated}
\end{center}
\vspace{-0.3cm}
\end{figure} 




Notice, however, that the points of  can define  collinearities, so a naive charging of Delaunay co-circularities to collinearities of type (1) or (2) in Lemma \ref{Lemma:MustCross} will not lead to a near-quadratic upper bound. (In other words, the universe of {\it all} collinearity events is far too large for our purposes.) Therefore, before we get to charging collinearities, we perform several preliminary charging steps, which will account for some Delaunay co-circularities of index  (thus removing their corresponding collinearities from consideration). 

As a preparation, we fix a constant parameter , apply Theorem \ref{Thm:RedBlue} to the edge  over the interval  of its absence from . We distinguish between three possible alternatives provided by that theorem.

\smallskip
\noindent{\bf (i)} If the first condition of Theorem \ref{Thm:RedBlue} is satisfied, we can charge the co-circularity of  and  to a -shallow collinearity that occurs in  and involves  and some third point of . As argued in Section \ref{Sec:Prelim}, the overall number of -shallow collinearities is .

Clearly, any collinearity event is charged at most a constant number of times. Namely, it can be charged only for the disappearances of edges  whose two vertices  participate in the event, and only for the disappearance immediately preceding the event, without any in-between reappearance.

To conclude, the number of Delaunay co-circularities that fall into case (i) of Theorem \ref{Thm:RedBlue} does not exceed .

\smallskip
\noindent{\bf (ii)} If the second condition of Theorem \ref{Thm:RedBlue} is satisfied, then we charge the Delaunay co-circularity at time  to  -shallow co-circularities, each occurring in  and involving  together with some two other points of .

As argued in Section \ref{Sec:Prelim}, the overall number of -shallow co-circularities is . Once again, each -shallow co-circularity is charged by only  Delaunay co-circularities in this manner, because  is the last disappearance of  before the charged event. Hence, at most  Delaunay co-circularities can fall into this case.

The above two cases account for at most  Delaunay co-circularities (of index ). If left to themselves, they would result in a recurrence of , with a nearly quadratic solution (see below for details, and see, e.g., \cite{ASS} for similar situations).
Unfortunately, this scheme does not always work because there might exist Delaunay co-circularities for which the respective red-blue arrangement (of the disappearing edge ) contains relatively few -shallow co-circularities, and no -shallow collinearities.
Such instances fall into the third case of Theorem \ref{Thm:RedBlue}, which is far more complicated to handle.

\smallskip
\noindent{\bf (iii)} There is a set  of at most  points (necessarily including at least one of  or ) whose removal ensures the Delaunayhood of  throughout . Recall that, by Lemma \ref{Lemma:MustCross}, at least one the two points , let it be , crosses  during . In the reduced triangulation , the collinearity of  and  is of a special type, and we refer to it as a {\it Delaunay crossing}. 

\medskip
\noindent{\bf Delaunay crossings.}
A {\it Delaunay crossing} is a triple , where  and  is a time interval, such that 
\begin{enumerate}
\item  
leaves  at time , and returns at time  (and  does not belong to  during ),
\item  crosses the segment  {\it at least} once during , and
\item  is an edge of  during  (i.e., removing  restores the Delaunayhood of  during the entire time interval ).
\end{enumerate}


\begin{figure}[htbp]
\begin{center}
\input{DelaunayCrossing.pstex_t}\hspace{3cm}\begin{picture}(0,0)\includegraphics{DelaunayCrossing1.pstex}\end{picture}\setlength{\unitlength}{3158sp}\begingroup\makeatletter\ifx\SetFigFont\undefined \gdef\SetFigFont#1#2#3#4#5{\reset@font\fontsize{#1}{#2pt}\fontfamily{#3}\fontseries{#4}\fontshape{#5}\selectfont}\fi\endgroup \begin{picture}(2145,2125)(1560,-1941)
\put(2675,-1435){\makebox(0,0)[lb]{\smash{{\SetFigFont{9}{10.8}{\rmdefault}{\mddefault}{\updefault}{\color[rgb]{1,0,0}}}}}}
\put(2067,-1484){\makebox(0,0)[lb]{\smash{{\SetFigFont{9}{10.8}{\rmdefault}{\mddefault}{\updefault}{\color[rgb]{0,0,0}}}}}}
\put(2748,-499){\makebox(0,0)[lb]{\smash{{\SetFigFont{9}{10.8}{\rmdefault}{\mddefault}{\updefault}{\color[rgb]{0,0,0}}}}}}
\put(3045,-650){\makebox(0,0)[lb]{\smash{{\SetFigFont{9}{10.8}{\rmdefault}{\mddefault}{\updefault}{\color[rgb]{1,0,0}}}}}}
\end{picture} \caption{\small A Delaunay crossing of  by  from  to . Several snapshots of the continuous motion of  before and after  crosses  are depicted (in the left and right figures, respectively). Hollow points specify the positions of  when . The solid circle in the left (resp., right) figure is the Delaunay co-circularity that starts (resp., ends) .}
\label{Fig:DelaunayCrossing}
\end{center}
\end{figure} 

Note that we allow Delaunay crossings, where the point  hits  at one (or both) of the times . In this case, the crossed edge  leaves the convex hull of  at time , or enters it at time . Clearly, the overall number of such ``degenerate" crossings is bounded by .

It is easy to see that the third condition is equivalent to the following condition, expressed in terms of the red-blue arrangement  associated with : The point  participates only in red-blue co-circularites during the interval , and these are the only red-blue co-circularities that occur during .

More specifically, note that  is red during some portion of  and is blue during the complementary portion (both portions are nonempty, unless  hits  when  begins or ends). During the former portion the graph of  coincides with the red lower envelope  (otherwise  would hold sometime during  even after removal of ), so it can only meet the graphs of blue functions. Similarly, during the latter portion  coincides with the blue upper envelope , so it can only meet the graphs of red functions.
See Figure \ref{Fig:DelaunayCrossing} for a schematic illustration of this behavior.

Notice that no points, other than , cross  during  (any such crossing would clearly contradict the third condition at the very moment when it occurs).
Moreover,  does not cross  outside  during ; otherwise  would belong to  when  belongs to .






\medskip
\noindent{\bf Back to case (iii).} We can now express the number of remaining Delaunay co-circularities of index  in terms of the maximum possible number of Delaunay crossings. To achieve this, we again resort to a probabilistic argument, in the spirit of Clarkson and Shor. Recall that for each such co-circularity there is a set  of at most  points whose removal restores the Delaunayhood of  throughout . In addition, we assume that  hits  during , and then .

We sample at random (and without replacement) a subset  of  points, and notice that the following two events occur simultaneously with probability at least : (1) the points  belong to , and (2) none of the points of  belong to . 

Since  crosses  during , and  is Delaunay at time  and (right after) time , the sample  induces a Delaunay crossing , for some time interval . (If  crosses  more than once, there may be several such crossings which occur at disjoint  sub-intervals of , but it may also be the case that all these crossing form a single Delaunay crossing, in the way it was defined above. This depends on whether  manages to become Delaunay in  in between these crossings.)

We charge the disappearance of  from  to the above crossing in  (or to the first such crossing if there are several) and note that the charging is unique (i.e., every Delaunay crossing  in  is charged by at most one disappearance of the respective edge  from ).
Hence, the number of Delaunay co-circularities of this kind is bounded by , where  denotes the maximum number of Delaunay crossings induced by any collection  of  points whose motion satisfies the above assumptions.


If the Delaunay co-circularity of  has index , we reverse the direction of the time axis and argue as above for the edge  instead of . We thus obtain the following recurrence for the number of Delaunay co-circularities:


for some absolute constant  which is independent of .

Informally, (\ref{Eq:FirstRecurrence}) implies that the maximum number of Delaunay co-circularities is asymptotically dominated by the maximum number of Delaunay crossings.

\paragraph{Discussion.} In the above analysis, we have used Theorem \ref{Thm:RedBlue} for the edge , which vanishes at the Delaunay co-circularity, in order to decompose the universe of all such events into three sub-classes (which correspond to the respective three cases of the theorem). Within each sub-class of Delaunay co-circularities, we have devised an entirely different charging scheme. In all cases, the (almost-)uniqueness of charging has been guaranteed through the careful choice of the interval , over which Theorem \ref{Thm:RedBlue} has been applied. Additional applications of this paradigm can be found in Section \ref{Sec:CrossOnce}.


\paragraph{The number of Delaunay co-circularities--wrap-up.}
In Section \ref{Sec:CrossOnce} we shall obtain the following recurrence for the maximum number  of Delaunay crossings:



where  and  are any two constants that satisfy , and
 is another constant which is independent of .

Our analysis will rely on the following additional assumption on the pseudo-algebraic motions of  (which was not necessary to establish (\ref{Eq:FirstRecurrence})):

\smallskip
{\it Either (i) no triple of points can be collinear more than twice, or (ii) no ordered triple of points can be collinear more than once}. 

\smallskip
Substituting the inequality (\ref{Eq:AllCrossings}) into (\ref{Eq:FirstRecurrence}), we obtain the following recurrence for , in which we choose : 

where  is a constant factor which does not depend on the choice of .

Arguing as in earlier solutions of similar charging-based recurrences (see, e.g., \cite{Envelopes3D,ConstantLines}, or \cite[Section 7.3.2]{SA95}), the recurrence solves to
, for any . (Specifically, for a given , we choose the parameters  as functions of , and establish the bound  with a constant of proportionality depending on , using induction on .) 

In conclusion, we have the following main result of this paper.

\begin{theorem}\label{Thm:OverallBound}
Let  be a collection of  points moving along pseudo-algebraic trajectories in the plane, so that any four points of  are co-circular at most twice. Assume also that either (i) no triple of points can be collinear more than twice, or (ii) no {\it ordered} triple of points can be collinear more than once. Then the Delaunay triangulation  of  experiences at most  discrete changes throughout the motion, for any .
\end{theorem}


\section{The Number of Delaunay Crossings}\label{Sec:CrossOnce}

In this section we complete the proof of Theorem \ref{Thm:OverallBound}. 
Throughout this section, we assume that any four points in the underlying set  of  moving points are {\it co-circular at most twice}, and that either {\it (i) no triple of points can be collinear more than twice}, or {\it (ii) no ordered triple of them can be collinear more than once}.\footnote{The last condition (ii) is equivalent to the following one: There can be at most one collinearity of an ordered triple  at which  hits .}
With these assumptions, we show that the maximum possible number  of Delaunay crossings in any set  as above satifies
the recurrence relation (\ref{Eq:AllCrossings}) asserted in the end of Section \ref{Sec:DelCocircs}.



  
Let  be a Delaunay crossing, as defined in the previous section. Specifically,  disappears from  at time , rejoins  at time , and remains Delaunay throughout  in . To distinguish between the notion of a Delaunay crossing , which lasts for the full time interval , and the instance where  actually lies on the segment , we refer to the latter event by saying that  {\it hits} .

\paragraph{Types of Delaunay crossings.}
Notice that  can hit the edge  at most twice during the above crossing , for otherwise the ordered triple  will be collinear at least three times.

A Delaunay crossing  is called {\it single} if the point  hits  only once during .
Otherwise (i.e., if  hits  exactly twice during ), we say that  is a {\it double} Delaunay crossing. 

Note that double Delaunay crossings can only arise if no three points in  can be collinear more than twice. (That is, double crossings are simply {\it impossible} in the second setting, where no ordered triple in  can be collinear more than once.)

In Section \ref{Subsec:Single} (namely, in Theorem \ref{Thm:OrdinaryCrossings}), we show that the maximum possible number  of single Delaunay crossings in the above set  satisfies the following recurrence:


where  and  are any two constants that satisfy , and the constant of proportionality in  does not depend on . Curiously enough, our analysis of single Delaunay crossings is equally valid given {\it any} of the two alternative assumptions (i), (ii) concerning the collinearities performed by the points of .

In Section \ref{Subsec:Double} (namely, in Theorem \ref{Thm:SpecialCrossings}) we show that any set  of  points, whose pseudo-algebraic motions satisfy the above assumptions, admits at most  double Delaunay crossings. Specifically, we argue that any double Delaunay crossing  can be {\it uniquely} (or almost-uniquely) charged to one of its respective edges .
In our analysis of double Delaunay crossings we can rely on the assumption that no three points of  can be collinear more than twice, because otherwise such crossings do not arise at all.

The overall Recurrence (\ref{Eq:AllCrossings}) for , asserted in the end of Section \ref{Sec:DelCocircs}, will follow immediately by combining the above two bounds.






\smallskip
Both Sections \ref{Subsec:Single} and \ref{Subsec:Double} use the following simple lemma.


\begin{lemma}\label{Lemma:Crossing}
If  is a Delaunay crossing then each of the edges  belongs to  throughout .
\end{lemma}
\paragraph{Remark:}
Most applications of Lemma \ref{Lemma:Crossing} (especially in Section \ref{Subsec:Single}) 
rely only on the fact that the edges  and 
are Delaunay at times  and .
To establish the Delaunayhood of  and  at time  it is sufficient to observe that, at that moment, there occurs a Delaunay co-circularity involving  and some other point ; moreover, this co-circularity is red-blue with respect to . Hence,  contains the triangle  right before , so the edges ,  are then Delaunay.\footnote{For degenerate crossings (which begin with a collinearity of ), the edge  is replaced on the convex hull of  by  and .}
A symmetric argument shows that  and  are Delaunay at time . 
The stronger form of the lemma is used mainly in Section \ref{Subsec:Double}.
\begin{proof}
We prove the claim only for the edge  and for  at which  lies in ; the complementary portion of , and the corresponding treatment of , are handled symmetrically. The crucial observation is that, during the chosen portion of ,  coincides with the blue upper envelope  (defined with respect to ). Indeed, let  be any red point so that the Delaunayhood of  is violated at time  by  and . Then the Delaunayhood of  is also violated there by  and any blue point  whose respective blue function  coincides with , implying that .
Therefore, the cap  has -empty interior throughout the chosen portion of . 

Suppose that  is not Delaunay at some time  that belongs to the chosen portion of .
We now consider the red-blue arrangement of  at that moment. Let  be the point whose function  coincides with the red lower envelope  (with respect to ). In particular, we have  (as is easily checked, , when ). Clearly,  cannot be equal to , for, otherwise, the disc  would have -empty interior. Indeed, we argued that  is -empty, and a similar argument shows that  would also have to be empty if  and  coincide, from which the emptiness of the whole interior follows. It follows that  is Delaunay at time ,
contradicting the definition of a Delaunay crossing. See Figure \ref{Fig:CrossingLemma}.
Moreover,  cannot lie in , for it would then have to lie in , which is impossible since this portion of  is -empty.

\begin{figure}[htbp]
\begin{center}
\begin{picture}(0,0)\includegraphics{StayDelaunay.pstex}\end{picture}\setlength{\unitlength}{4342sp}\begingroup\makeatletter\ifx\SetFigFont\undefined \gdef\SetFigFont#1#2#3#4#5{\reset@font\fontsize{#1}{#2pt}\fontfamily{#3}\fontseries{#4}\fontshape{#5}\selectfont}\fi\endgroup \begin{picture}(1604,1513)(1529,-1603)
\put(2762,-1015){\makebox(0,0)[lb]{\smash{{\SetFigFont{12}{14.4}{\rmdefault}{\mddefault}{\updefault}{\color[rgb]{1,0,0}}}}}}
\put(2041,-1539){\makebox(0,0)[lb]{\smash{{\SetFigFont{12}{14.4}{\rmdefault}{\mddefault}{\updefault}{\color[rgb]{0,0,0}}}}}}
\put(2791,-474){\makebox(0,0)[lb]{\smash{{\SetFigFont{12}{14.4}{\rmdefault}{\mddefault}{\updefault}{\color[rgb]{0,0,0}}}}}}
\put(1834,-760){\makebox(0,0)[lb]{\smash{{\SetFigFont{12}{14.4}{\rmdefault}{\mddefault}{\updefault}{\color[rgb]{0,0,.56}}}}}}
\put(2092,-225){\makebox(0,0)[lb]{\smash{{\SetFigFont{12}{14.4}{\rmdefault}{\mddefault}{\updefault}{\color[rgb]{0,0,.56}}}}}}
\end{picture} \caption{\small Proof of Lemma \ref{Lemma:Crossing}.}
\label{Fig:CrossingLemma}
\end{center}
\end{figure} 


Since  is not Delaunay, the disc  contains another point , which is easily seen to lie in  and in .
We can expand  from  until its boundary touches ,  and , and its interior contains . This implies that  does not belong to  at time , which contradicts the definition of a Delaunay crossing. 
\end{proof}









\subsection{The number of single Delaunay crossings}\label{Subsec:Single}
In this subsection we establish Recurrence (\ref{Eq:Single}) for the maximum possible number  of single Delaunay crossings in a set  of  points whose pseudo-algebraic motions satisfy the above assumptions.
To facilitate the proof of this main result, which is asserted in the culminating Theorem \ref{Thm:OrdinaryCrossings}, we begin by introducing some additional notation, and by establishing several auxiliary lemmas.

\paragraph{Notational conventions.}
Recall from Section \ref{Sec:Prelim} that every edge  is oriented from  to , and its corresponding line  splits the plane into halfplanes  and .

Without loss of generality, we assume in what follows that, for any single Delaunay crossing , the point  crosses  from  to  during . Recall that  cannot cross  outside  during , so this is the {\it only} collinearity of  in .
If  crosses  in the opposite direction, we regard this crossing as .

Note that every such Delaunay crossing  is uniquely determined by the respective ordered triple , because there can be at most one collinearity\footnote{If  hits  twice, which is allowed only if no three points of  can be collinear more than twice, then the other crossing of  by  is from  back to .} where  crosses the line  {\it within } from  to .

For a convenience of reference, we label each such crossing  as {\it a clockwise -crossing}, and as {\it a counterclockwise -crossing}, with an obvious meaning of these labels.



The following lemma lies at the heart of our analysis.

\begin{lemma}\label{Lemma:OnceCollin}
Let  be a Delaunay crossing. Then, with the above conventions, for any  the points  define a red-blue co-circularity with respect to , which takes place during  when the point  either enters the cap , or leaves the opposite cap .
\end{lemma}
\begin{proof}
By definition,  crosses  at some (unique) time , say from  to . The disc  is -empty at  and at  and moves continuously throughout  and .
Just before ,  is the entire , so every point  at time  must have entered  during , forming a co-circularity with  at the time it enters the disc.\footnote{If  then
there are no red points when  hits , so we consider only the second interval. The case of  is treated symmetrically.} See Figure \ref{Fig:BeforeCrossing} (left). (As mentioned in Section \ref{Sec:Prelim}, this co-circularity of  is red-blue with respect to , that is, the point  enters  through .) A symmetric argument (in which we reverse the direction of the time axis) shows that the same holds for all the points  that lie in  at time ; see Figure \ref{Fig:BeforeCrossing} (right). 
\end{proof}





\begin{figure}[htbp]
\begin{center}
\input{BeforeCrossing.pstex_t}\hspace{4cm}\begin{picture}(0,0)\includegraphics{AfterCrossing.pstex}\end{picture}\setlength{\unitlength}{2842sp}\begingroup\makeatletter\ifx\SetFigFont\undefined \gdef\SetFigFont#1#2#3#4#5{\reset@font\fontsize{#1}{#2pt}\fontfamily{#3}\fontseries{#4}\fontshape{#5}\selectfont}\fi\endgroup \begin{picture}(3519,2191)(85,-1956)
\put(100,-1611){\makebox(0,0)[lb]{\smash{{\SetFigFont{11}{13.2}{\rmdefault}{\mddefault}{\updefault}{\color[rgb]{1,0,0}}}}}}
\put(2126,-1474){\makebox(0,0)[lb]{\smash{{\SetFigFont{12}{14.4}{\rmdefault}{\mddefault}{\updefault}{\color[rgb]{0,0,0}}}}}}
\put(2388,-1197){\makebox(0,0)[lb]{\smash{{\SetFigFont{12}{14.4}{\rmdefault}{\mddefault}{\updefault}{\color[rgb]{1,0,0}}}}}}
\put(2766,-554){\makebox(0,0)[lb]{\smash{{\SetFigFont{12}{14.4}{\rmdefault}{\mddefault}{\updefault}{\color[rgb]{0,0,0}}}}}}
\end{picture} \caption{\small Left: Right before  crosses , the circumdisc  contains all points in . Right: After  crosses ,  contains all points in .}
\label{Fig:BeforeCrossing}
\end{center}
\end{figure} 




\begin{lemma}\label{Lemma:TwiceCollin}
The number of triples of points  for which there exist two time intervals  such that both  and  are single Delaunay crossings, is at most . Furthermore, the lemma still holds if we reverse 
, or , or both. 
\end{lemma}

In other words, the lemma asserts that  contains at most quadratically many triples  that perform two single Delaunay crossings of distinct order types.

\begin{proof}
Assume, with no loss of generality, that the Delaunay crossing of  (or ) by  ends after the crossing of  (or ) by ; that is  ends after  (note that  and  need not be disjoint).
Let  be a point of .
By Lemma \ref{Lemma:OnceCollin}, the four points  define a co-circularity event during .
Similarly, the same four points  define a co-circularity event during . We claim that the above two co-circularities are distinct. 
Indeed, the former co-circularity of  is red-blue with respect to the edge  (which is crossed by  during ), so  and  are not adjacent in the co-circularity. On the other hand, the latter co-circularity is red-blue with respect to  (which is crossed by  during ), so  and  are not adjacent in the co-circularity. However, both non-adjacencies cannot occur simultaneously in the same co-circularity, so these two co-circularities of  must be distinct.


Hence, the points 
induce at least (by our assumption, exactly) two common co-circularity events before  re-enters . 

Thus, we cannot have a Delaunay crossing of any of  by  after  re-enters , for otherwise this would lead, according to Lemma \ref{Lemma:OnceCollin} to a third co-circularity event involving  and . 
Since this holds for every point , the crossing of  (or ) by  is the last Delaunay crossing of  (or ), so it can be charged uniquely to this edge. (Clearly, any two Delaunay crossings of the same edge  take place at disjoint time intervals.)
\end{proof}






\smallskip

Our overall strategy is to show that, for an average choice of , there exist only few -crossings of a given orientation type (which can be either clockwise or counterclockwise).
In other words, we are to show that most single Delaunay crossings  can be almost-uniquely charged to either one of its edges  and . (As a matter of fact, it is sufficient that we can charge  to only {\it one} of its edges . As explained in Section \ref{Subsec:Double}, this simple charging succeeds for all {\it double} Delaunay crossings.)

Unfortunately, there can be arbitrary many single -crossings, of both orientation types. In such cases, we resort to more intricate charging arguments (see the proof of Theorem \ref{Thm:OrdinaryCrossings}). Note that the respective intervals  and  of any pair of such crossings, say  and , may overlap.
The following lemma defines a natural order on -crossings of a given orientation (clockwise or counterclockwise).



\begin{lemma}\label{Lemma: OrderOrdinaryCrossings}
Let  and  be clockwise -crossings, and suppose that  hits  (during ) before it hits  (during ). Then  begins (resp., ends) before the beginning (resp., end) of . Clearly, the converse statements hold too. Similar statements also hold for pairs of counterclockwise -crossings.
\end{lemma}

\begin{proof}
In the configuration considered in the main statement of the lemma,  crosses  from  to , and it crosses  from  to .
We only prove the part of the lemma concerning the ending times of the crossings, because the proof about the starting times is fully symmetric (by reversing the direction of the time axis). The statement clearly holds if  and  are disjoint; the interesting situation is when they partially overlap.
Note that  enters  only once during the Delaunay crossing of  by , namely, right after  hits . Indeed, by assumption,  cannot exit  by crossing  again during , and it cannot cross  because at that time , which is Delaunay in , would be Delaunay also in the presence of , contrary to the definition of a Delaunay crossing.
Hence, we may assume that  still lies in  when it hits  during the Delaunay crossing of that edge. Indeed, otherwise the crossing of  would by then be over, so the claim would hold trivially, as noted above. In particular,  lies clockwise to  at that time.

\begin{figure}[htbp]
\begin{center}
\input{orderord1.pstex_t}\hspace{3cm}\begin{picture}(0,0)\includegraphics{OrderCrossings.pstex}\end{picture}\setlength{\unitlength}{4342sp}\begingroup\makeatletter\ifx\SetFigFont\undefined \gdef\SetFigFont#1#2#3#4#5{\reset@font\fontsize{#1}{#2pt}\fontfamily{#3}\fontseries{#4}\fontshape{#5}\selectfont}\fi\endgroup \begin{picture}(1197,1443)(2024,-1730)
\put(3111,-607){\makebox(0,0)[lb]{\smash{{\SetFigFont{11}{13.2}{\rmdefault}{\mddefault}{\updefault}{\color[rgb]{0,0,0}}}}}}
\put(3206,-994){\makebox(0,0)[lb]{\smash{{\SetFigFont{12}{14.4}{\rmdefault}{\mddefault}{\updefault}{\color[rgb]{0,0,0}}}}}}
\put(2743,-1675){\makebox(0,0)[lb]{\smash{{\SetFigFont{12}{14.4}{\rmdefault}{\mddefault}{\updefault}{\color[rgb]{0,0,0}}}}}}
\put(2039,-1490){\makebox(0,0)[lb]{\smash{{\SetFigFont{12}{14.4}{\rmdefault}{\mddefault}{\updefault}{\color[rgb]{0,0,0}}}}}}
\put(2754,-454){\makebox(0,0)[lb]{\smash{{\SetFigFont{12}{14.4}{\rmdefault}{\mddefault}{\updefault}{\color[rgb]{0,0,0}}}}}}
\end{picture} \caption{\small Proof of Lemma \ref{Lemma: OrderOrdinaryCrossings}. Left: if  remains in  after  and before it crosses , then  lies in  before that last collinearity. Right: The second co-circularity of  which occurs when  leaves . This is a red-red co-circularity with respect to , so the crossing of  is already over.}
\label{Fig:OrderOrdinaryCrossings}
\end{center}
\end{figure} 

It suffices to prove that the co-circularity of , which (by Lemma \ref{Lemma:OnceCollin}) occurs during the Delaunay crossing of  by , takes place when the crossing of  by  is already finished (and, in particular, after the co-circularity of  that occurs during the crossing of ). 

Before the Delaunayhood of  is restored, we have a co-circularity  in which  leaves . (This is argued in the proof of Lemma \ref{Lemma:OnceCollin}: Right after the crossing, the point  lies in , as in Figure \ref{Fig:OrderOrdinaryCrossings} (left), and has to leave that disc before it becomes empty; it cannot cross  during , when this edge undergoes the Delaunay crossing by ). Notice that this is a red-blue co-circularity with respect to , and a red-red co-circularity with respect to ; see Figure \ref{Fig:OrderOrdinaryCrossings} (right). Since no red-red or blue-blue co-circularities occur during a Delaunay crossing of an edge, the crossing of  is already over.
\end{proof}

\ignore{
\begin{proof}
Recall that, by our convention,  crosses  from  to , and it crosses  from  to .
We only prove the part of the lemma concerning the ending times of the clockwise -crossings, because the proof about the starting times is fully symmetric (by reversing the direction of the time axis). 
By Lemma \ref{Lemma:OnceCollin}, the four points  are involved in at least one red-blue co-circularity with respect to , which occurs at some time  when  either enters the cap  or leaves the opposite cap .
It suffices to prove that the crossing of  by  is already finished by the above time . 


\begin{figure}[htbp]
\begin{center}
\input{OrderCrossingsReturn.pstex_t}\hspace{3cm}\begin{picture}(0,0)\includegraphics{OrderCrossings.pstex}\end{picture}\setlength{\unitlength}{4342sp}\begingroup\makeatletter\ifx\SetFigFont\undefined \gdef\SetFigFont#1#2#3#4#5{\reset@font\fontsize{#1}{#2pt}\fontfamily{#3}\fontseries{#4}\fontshape{#5}\selectfont}\fi\endgroup \begin{picture}(1197,1443)(2024,-1730)
\put(3111,-607){\makebox(0,0)[lb]{\smash{{\SetFigFont{11}{13.2}{\rmdefault}{\mddefault}{\updefault}{\color[rgb]{0,0,0}}}}}}
\put(3206,-994){\makebox(0,0)[lb]{\smash{{\SetFigFont{12}{14.4}{\rmdefault}{\mddefault}{\updefault}{\color[rgb]{0,0,0}}}}}}
\put(2743,-1675){\makebox(0,0)[lb]{\smash{{\SetFigFont{12}{14.4}{\rmdefault}{\mddefault}{\updefault}{\color[rgb]{0,0,0}}}}}}
\put(2039,-1490){\makebox(0,0)[lb]{\smash{{\SetFigFont{12}{14.4}{\rmdefault}{\mddefault}{\updefault}{\color[rgb]{0,0,0}}}}}}
\put(2754,-454){\makebox(0,0)[lb]{\smash{{\SetFigFont{12}{14.4}{\rmdefault}{\mddefault}{\updefault}{\color[rgb]{0,0,0}}}}}}
\end{picture} \caption{\small Proof of Lemma \ref{Lemma: OrderOrdinaryCrossings}. Left: The co-circularity of  at time  occurs
when  enters the cap . Then  must have re-entered  after  and before .
Right: The co-circularity of  at time  occurs when  leaves the cap . This is a red-red co-circularity with respect to , and a red-blue co-circularity with respect to , so the crossing of  is already over.}
\label{Fig:OrderCrossings}
\end{center}
\vspace{-0.3cm}
\end{figure} 



Assume first that  occurs when  enters the cap , so
 lies at that moment in  (i.e., this is a blue-blue co-circularity with respect to ).
See Figure \ref{Fig:OrderCrossings} (left). Also note that Since  can hit  only once during ,  must have re-entered  after  and before , and we are done.

Assume, then, that  occurs when  leaves the cap .
Notice that this is a red-red co-circularity with respect to the edge , and a red-blue co-circularity with respect to the edge  whose Delaunayhood is violated right afterwards by  and ; see Figure \ref{Fig:OrderCrossings} (right). Since no red-red or blue-blue co-circularities occur during a Delaunay crossing of an edge (or, alternatively, since  is Delaunay throughout , by Lemma \ref{Lemma:Crossing}), the crossing  of  is already over by time .
\end{proof}
}

Lemma \ref{Lemma: OrderOrdinaryCrossings} implies that, for any pair of points  in , all the clockwise -crossings can be linearly ordered by the starting times of their intervals, or by the ending times of their intervals, or by the times when  hits the corresponding edges that emanate from , and all three orders are indentical. Clearly, a symmetric order exists for counterlockwise -crossings too.







The following theorem provides the long-awaited recursive bound on the maximum number of Delaunay crossings.

\begin{theorem}\label{Thm:OrdinaryCrossings}
Let  be a pair of constants. Then the maximum possible number  of single Delaunay crossings in any set  of  points, whose pseudo-algebraic motions in  respects the above assumptions, satisfies the following recurrence:

where the constant of proportionality in  is independent of .
\end{theorem}
\begin{proof}Fix a single Delaunay crossing  as above. If this is the last clockwise -crossing in the order implied by Lemma \ref{Lemma: OrderOrdinaryCrossings}, then we can charge  to the edge . Clearly, this accounts for at most quadratically many single crossings. 

Otherwise, let  be the clockwise -crossing that follows immediately after . That is, we have  and , and no clockwise -crossings begin in the interval  or end in the symmetric interval . Refer to Figure \ref{Fig:ChooseNext}.
Note that  is uniquely determined by the choice of , and vice versa.
We thus have reduced our problem to bounding the maximum possible number of such ``consecutive" pairs ,   of clockwise -crossings (over all ). 










\begin{figure}[htbp]
\begin{center}
\input{ConsecutiveCrossings.pstex_t}\hspace{2cm}\input{ConsecutiveDisjoint.pstex_t}\hspace{2cm}\begin{picture}(0,0)\includegraphics{ConsecutiveOverlap.pstex}\end{picture}\setlength{\unitlength}{4342sp}\begingroup\makeatletter\ifx\SetFigFont\undefined \gdef\SetFigFont#1#2#3#4#5{\reset@font\fontsize{#1}{#2pt}\fontfamily{#3}\fontseries{#4}\fontshape{#5}\selectfont}\fi\endgroup \begin{picture}(1933,693)(792,-1238)
\put(2165,-1174){\makebox(0,0)[lb]{\smash{{\SetFigFont{11}{13.2}{\rmdefault}{\mddefault}{\updefault}{\color[rgb]{0,0,0}}}}}}
\put(1442,-804){\makebox(0,0)[lb]{\smash{{\SetFigFont{11}{13.2}{\rmdefault}{\mddefault}{\updefault}{\color[rgb]{0,0,0}}}}}}
\put(1785,-1168){\makebox(0,0)[lb]{\smash{{\SetFigFont{11}{13.2}{\rmdefault}{\mddefault}{\updefault}{\color[rgb]{0,0,0}}}}}}
\put(2686,-897){\makebox(0,0)[lb]{\smash{{\SetFigFont{11}{13.2}{\rmdefault}{\mddefault}{\updefault}{\color[rgb]{0,0,0}}}}}}
\put(1842,-816){\makebox(0,0)[lb]{\smash{{\SetFigFont{11}{13.2}{\rmdefault}{\mddefault}{\updefault}{\color[rgb]{0,0,0}}}}}}
\put(807,-1147){\makebox(0,0)[lb]{\smash{{\SetFigFont{11}{13.2}{\rmdefault}{\mddefault}{\updefault}{\color[rgb]{0,0,0}}}}}}
\put(2292,-680){\makebox(0,0)[lb]{\smash{{\SetFigFont{11}{13.2}{\rmdefault}{\mddefault}{\updefault}{\color[rgb]{0,0,0}}}}}}
\end{picture} \caption{\small The pair  of consecutive clockwise -crossings.
Left:  crosses the edges  (during , from  to ) and  (during , from  to ), in this order. Center and right: We have  and , so the intervals  and  are either disjoint or {\it partly} overlapping.
No clockwise -crossings begin in  or end in .}
\label{Fig:ChooseNext}
\end{center}
\vspace{-0.3cm}
\end{figure} 







\medskip
\noindent{\bf Charging events in .} By Lemma \ref{Lemma:Crossing},  is Delaunay in each of the intervals  and .
If these intervals overlap, then  is Delaunay throughout . Otherwise, as a preparation to the main analysis, we consider the red-blue arrangement  associated with the edge  during the gap  between  and . 
Since  is Delaunay at both times  and , we can apply Theorem \ref{Thm:RedBlue} over , with the first threshold value . Refer to Figure \ref{Fig:ApplyRBPr}.

\begin{figure}[htbp]
\begin{center}
\begin{picture}(0,0)\includegraphics{ApplyRBPr.pstex}\end{picture}\setlength{\unitlength}{4736sp}\begingroup\makeatletter\ifx\SetFigFont\undefined \gdef\SetFigFont#1#2#3#4#5{\reset@font\fontsize{#1}{#2pt}\fontfamily{#3}\fontseries{#4}\fontshape{#5}\selectfont}\fi\endgroup \begin{picture}(1929,801)(796,-1338)
\put(2068,-1147){\makebox(0,0)[lb]{\smash{{\SetFigFont{12}{14.4}{\rmdefault}{\mddefault}{\updefault}{\color[rgb]{0,0,0}}}}}}
\put(2686,-897){\makebox(0,0)[lb]{\smash{{\SetFigFont{12}{14.4}{\rmdefault}{\mddefault}{\updefault}{\color[rgb]{0,0,0}}}}}}
\put(2252,-672){\makebox(0,0)[lb]{\smash{{\SetFigFont{12}{14.4}{\rmdefault}{\mddefault}{\updefault}{\color[rgb]{0,0,0}}}}}}
\put(811,-1142){\makebox(0,0)[lb]{\smash{{\SetFigFont{12}{14.4}{\rmdefault}{\mddefault}{\updefault}{\color[rgb]{0,0,0}}}}}}
\put(1320,-805){\makebox(0,0)[lb]{\smash{{\SetFigFont{12}{14.4}{\rmdefault}{\mddefault}{\updefault}{\color[rgb]{0,0,0}}}}}}
\put(1808,-1283){\makebox(0,0)[lb]{\smash{{\SetFigFont{12}{14.4}{\rmdefault}{\mddefault}{\updefault}{\color[rgb]{0,0,0}}}}}}
\put(1664,-801){\makebox(0,0)[lb]{\smash{{\SetFigFont{12}{14.4}{\rmdefault}{\mddefault}{\updefault}{\color[rgb]{0,0,0}}}}}}
\put(1879,-803){\makebox(0,0)[lb]{\smash{{\SetFigFont{12}{14.4}{\rmdefault}{\mddefault}{\updefault}{\color[rgb]{0,0,0}}}}}}
\put(2385,-1147){\makebox(0,0)[lb]{\smash{{\SetFigFont{12}{14.4}{\rmdefault}{\mddefault}{\updefault}{\color[rgb]{0,0,0}}}}}}
\end{picture} \caption{\small Applying Theorem \ref{Thm:RedBlue} in  over the gap  between  and . Note that  is Delaunay throughout each of the intervals . Unless we manage to charge the pair  within , we end up with a subset  of at most  points whose removal extends Delaunayhood of  to .}
\label{Fig:ApplyRBPr}
\end{center}
\vspace{-0.3cm}
\end{figure} 


In cases (i) and (ii) of Theorem \ref{Thm:RedBlue}, we charge the pair  either to  -shallow co-circularities, or to a -shallow collinearity (within the arrangement of ). 

In both chargings, each shallow co-circularity or collinearity is charged at most a constant number of times. Indeed, consider the moment  when the charged event occurs, and notice that it involves  and  (together with one or two additional points of ). The choice of  ensures that the moment  (when the crossing of  by  ends) is the last time before  when a clockwise -crossing is completed. Hence, having guessed  and  (in  ways),  is uniquely determined.
Therefore, using the upper bounds on the number of -shallow collinearities and co-circularities established in Section \ref{Sec:Prelim}, we get that the overall number of such consecutive pairs , for which the red-blue arrangement of  (during ) satisfies condition (i) or (ii) of Theorem \ref{Thm:RedBlue}, is 





To conclude, we can assume in what follows that either the intervals  and  overlap, or condition (iii) of Theorem \ref{Thm:RedBlue} holds. In the latter case, there exists a subset  of at most  points (possibly including  and/or ) so that  belongs to  throughout the interval . As a matter of fact,  then belongs to  throughout an even larger interval . 


Notice that reversing the direction of the time axis simply switches the order of  and , so their respective points  and  will play symmetrical roles in our case analysis. Recall also that  is a counterclockwise -crossing, and  is a counterclockwise -crossing (this in addition to their being clockwise -crossings).

\paragraph{The subsequent chargings--overview.} The rest of the proof is organized as follows.
We distinguish between three possible cases (a)--(c), ruling them out one by one.

In case (a) we assume that  is hit by one of  in the gap  between  and , so the respective triple  or  performs two single Delaunay crossings in, respectively,  or . Hence, our analysis bottoms out via Lemma \ref{Lemma:TwiceCollin}.


In case (b) we assume that the edge  is never Delaunay in the interval , or that the edge  is never Delaunay in the symmetric interval .
In the first sub-scenario, we show that  is among the last  counterclockwise -crossings (with respect to the order implied by Lemma \ref{Lemma: OrderOrdinaryCrossings}). Notice that, by Lemma \ref{Lemma:Crossing}, no such crossings can begin or end after  (where  is not even Delaunay), so we are only to show that at most  -crossings  end after  and before  (which is done in Proposition \ref{Prop:Balanced}).
In the second sub-scenario, a fully symmetric argument implies that  is among the first  counterclockwise -crossings. In both sub-scenarios, the overall number of such consecutive pairs  is easily seen to be .
 
Finally, in case (c) we may assume that there exists a time  which is the {\it first} such time when  belongs to , and that
there exists a time  which is the {\it last} such time when  belongs to .
We argue that the edge  is hit in  by one of , and that the edge  is hit in the symmetric interval  by one of .
We then invoke Theorem \ref{Thm:RedBlue} and try to charge  within one of the red-blue arrangements . In the case of failure, each of the above additional crossings of  and  yields a Delaunay crossing with respect a suitably reduced subset of , so at least one of the triples  is involved in two Delaunay crossings. Hence, our analysis again bottoms out via Lemma \ref{Lemma:TwiceCollin}.









\paragraph{Case (a).} The above intervals  and  are disjoint and (at least) one of the points  hits the edge  in the interval . 
(By Lemma \ref{Lemma:Crossing}, none of  can hit  in  or .) Refer to Figure \ref{Fig:CrossPr}.

Assume, with no loss of generality, that  is hit in  by .
Since  is Delaunay at both times  and , the edge  (or its reversely oriented copy ) undergoes a Delaunay crossing by  within the smaller triangulation  during some sub-interval of . 
This is in addition to the inherited single Delaunay crossing of  by , which is easily checked to occur in  too.
Recalling the assumptions on the possible collinearities in , we get that both of these crossings in  must be single Delaunay crossings.
Lemma \ref{Lemma:TwiceCollin}, combined with the probabilistic argument of Clarkson and Shor \cite{CS}, in a manner similar to that used in Section \ref{Sec:Prelim}, provides an upper bound of  on the number of such triples . Clearly, this also bounds the overall number of such consecutive pairs . 

\begin{figure}[htbp]
\begin{center}
\input{SecondCrossing.pstex_t}\hspace{2cm}\input{SecondCrossingReverse.pstex_t}\hspace{2cm}\begin{picture}(0,0)\includegraphics{SchemeA.pstex}\end{picture}\setlength{\unitlength}{4934sp}\begingroup\makeatletter\ifx\SetFigFont\undefined \gdef\SetFigFont#1#2#3#4#5{\reset@font\fontsize{#1}{#2pt}\fontfamily{#3}\fontseries{#4}\fontshape{#5}\selectfont}\fi\endgroup \begin{picture}(1485,680)(1260,-1352)
\put(2730,-978){\makebox(0,0)[lb]{\smash{{\SetFigFont{12}{14.4}{\rmdefault}{\mddefault}{\updefault}{\color[rgb]{0,0,0}}}}}}
\put(1320,-805){\makebox(0,0)[lb]{\smash{{\SetFigFont{11}{13.2}{\rmdefault}{\mddefault}{\updefault}{\color[rgb]{0,0,0}}}}}}
\put(1664,-801){\makebox(0,0)[lb]{\smash{{\SetFigFont{11}{13.2}{\rmdefault}{\mddefault}{\updefault}{\color[rgb]{0,0,0}}}}}}
\put(1562,-1301){\makebox(0,0)[lb]{\smash{{\SetFigFont{10}{12.0}{\rmdefault}{\mddefault}{\updefault}{\color[rgb]{0,0,0}}}}}}
\put(1832,-783){\makebox(0,0)[lb]{\smash{{\SetFigFont{10}{12.0}{\rmdefault}{\mddefault}{\updefault}{\color[rgb]{0,0,0} crossed by  or }}}}}
\put(2473,-1118){\makebox(0,0)[lb]{\smash{{\SetFigFont{11}{13.2}{\rmdefault}{\mddefault}{\updefault}{\color[rgb]{0,0,0}}}}}}
\put(2142,-1111){\makebox(0,0)[lb]{\smash{{\SetFigFont{11}{13.2}{\rmdefault}{\mddefault}{\updefault}{\color[rgb]{0,0,0}}}}}}
\end{picture} \caption{\small Case (a). Left and center: The edge  is hit by (at least) one of  during . Right: The edge  undergoes a Delaunay crossings by  or  within an appropriate triangulation  or .}
\label{Fig:CrossPr}
\end{center}
\end{figure} 


Symmetrically, if  is hit in the interval  by , the triple  performs two single Delaunay crossings in the triangulation . By Lemma \ref{Lemma:TwiceCollin}, and again using the probabilistic argument of Clarkson and Shor, the overall number of such crossings  (and, hence, of such consecutive pairs ) too cannot exceed .


\bigskip
To conclude, in each of the subsequent cases (b)--(c) we may assume that the preceding scenario (a) does not occur. In addition, we continue to assume that, unless the intervals  and  overlap, there is a subset  of at most  points whose removal restores the Delaunayhood of  in the gap  between  and . 


\begin{proposition}\label{Prop:Balanced}
With the above assumptions, at most  counterclockwise -crossings  end in the interval , and at most  counterclockwise -crossings  begin in the symmetric interval .
\end{proposition}
\begin{proof}
With no loss of generality, we focus on counterclockwise -crossings. The counterclockwise -crossings are handled symmetrically, by reversing the direction of the time axis. 

Let  be a counterclockwise -crossing that ends in . In particular,  ends after  so, by the counterclockwise variant of Lemma \ref{Lemma: OrderOrdinaryCrossings},  also begins after the beginning  of . Therefore, we get that .
We claim that the intervals  and  are disjoint, and the respective point  of  belongs to the above set  of at most  points whose removal restores the Delaunayhood of  throughout . This will imply that the overall number of such crossings  cannot exceed . 

\begin{figure}[htbp]
\begin{center}
\input{SchemeBalanced.pstex_t}\hspace{1cm}\input{BalancedCross.pstex_t}\hspace{1.5cm}\begin{picture}(0,0)\includegraphics{BalancedCross1.pstex}\end{picture}\setlength{\unitlength}{3947sp}\begingroup\makeatletter\ifx\SetFigFont\undefined \gdef\SetFigFont#1#2#3#4#5{\reset@font\fontsize{#1}{#2pt}\fontfamily{#3}\fontseries{#4}\fontshape{#5}\selectfont}\fi\endgroup \begin{picture}(1379,1785)(1911,-1860)
\put(1926,-1457){\makebox(0,0)[lb]{\smash{{\SetFigFont{11}{13.2}{\rmdefault}{\mddefault}{\updefault}{\color[rgb]{0,0,0}}}}}}
\put(2639,-1716){\makebox(0,0)[lb]{\smash{{\SetFigFont{11}{13.2}{\rmdefault}{\mddefault}{\updefault}{\color[rgb]{0,0,0}}}}}}
\put(2819,-222){\makebox(0,0)[lb]{\smash{{\SetFigFont{10}{12.0}{\rmdefault}{\mddefault}{\updefault}{\color[rgb]{0,0,0}}}}}}
\put(2529,-374){\makebox(0,0)[lb]{\smash{{\SetFigFont{11}{13.2}{\rmdefault}{\mddefault}{\updefault}{\color[rgb]{0,0,0}}}}}}
\put(3163,-1796){\makebox(0,0)[lb]{\smash{{\SetFigFont{11}{13.2}{\rmdefault}{\mddefault}{\updefault}{\color[rgb]{0,0,0}}}}}}
\put(3200,-1005){\makebox(0,0)[lb]{\smash{{\SetFigFont{11}{13.2}{\rmdefault}{\mddefault}{\updefault}{\color[rgb]{0,0,0}}}}}}
\end{picture} \caption{\small Proof of Proposition \ref{Prop:Balanced}. Left: The crossing  occurs within . The points  are co-circular at times  and . The latter co-circularity (at time ) is red-blue with respect to , so it occurs in the gap  between  and . Center: A possible trajectory of  during  if  lies in  at time . Right: A possible trajectory of  during  if  lies in  at time .}
\label{Fig:Balanced}
\end{center}
\end{figure} 

Indeed, by Lemma \ref{Lemma:OnceCollin}, the four points  are involved in (at least) one co-circularity during the single Delaunay crossing of  by , and in another co-circularity during the similar crossing of  by . Refer to Figure \ref{Fig:Balanced} (left).
Specifically, the former co-circularity is red-blue with respect to the two diagonal edges  and . By Lemma \ref{Lemma:Crossing},  is Delaunay throughout , so this co-circularity occurs at some time . Similarly, the other co-circularity of  is red-blue with respect to the edges  and , so it occurs at some later time . 
Since the latter co-circularity, occurring at time , is red-blue with respect to , it cannot occur during during the interval  (where  is Delaunay). Hence,  must occur in the gap  between the intervals  and , which then cannot overlap. 

We next argue that  hits  in the above gap interval , which will immediately\footnote{Clearly, the Delaunayhood of  in  cannot be restored before we remove from  every point that crosses  in that interval.} imply that .
Since , the times  and  cannot coincide, so  is the {\it only} co-circularity of  in . To obtain the asserted crossing of , we distinguish between the following two sub-cases:

\medskip
\noindent (i) Assume first that  lies in  at time . As prescribed in Lemma \ref{Lemma:OnceCollin}, this co-circularity occurs when  leaves the cap , so the Delaunayhood of  is violated right after time  by  and . See Figure \ref{Fig:Balanced} (center).
By Lemma \ref{Lemma:MustCross}, and keeping in mind that  is Delaunay at time  (and no further co-circularities of  can occur in ), the edge  is hit by at least one of the two points  at some moment in . 
Since case (a) is excluded,  cannot hit  during that interval (which is contained in ), so it must be the case that  hits  during the time interval . 

\smallskip
\noindent (ii) Assume, then, that  lies in  at time . As prescribed in Lemma \ref{Lemma:OnceCollin}, this co-circularity occurs when  enters the cap , so the Delaunayhood of  is violated right {\it before} time  by  and . See Figure \ref{Fig:Balanced} (right).
Since  is Delaunay at time  (and no further co-circularities of  can take place in ), we can apply Lemma \ref{Lemma:MustCross} from  in the reverse direction of the time axis to get that  is hit by at least one of  at some moment in . Since case (a) is excluded, it must be the case that  hits  during the time interval .

\smallskip
To conclude, we have shown that  is hit by  in the gap interval  between  and .
Therefore,  belongs to the above set  of at most  points whose removal restores the Delaunayhood of  throughout the inerval , so the overall number of such counterclockwise -crossings  cannot exceed .

Repeating the above analysis in the reverse direction of the time axis shows that, if  is a counterclockwise -crossing starting in , its respective point  crosses  in the gap  and, therefore, again belongs to .
Hence, the overall number of such crossings  is at most  too.
\end{proof}













\smallskip
\noindent {\bf Case (b).} The edge  is never Delaunay in the interval , or the edge  is never Delaunay in the symmetric interval .

If  is never Delaunay in , then no counterclockwise -crossings  can occur (i.e., begin or end) after , for, by Lemma \ref{Lemma:Crossing},  must belong to  when any such Delaunay crossing takes place. 
Combining this with Proposition \ref{Prop:Balanced}, we conclude that  is among the  counterclockwise -crossings  that end the latest. In other words,  is among the last  counterclockwise -crossings with respect to the order implied by Lemma \ref{Lemma: OrderOrdinaryCrossings}. Clearly, this scenario can happen for at most  consecutive pairs  of Delaunay crossings. 

A fully symmetric argument applies if  is never Delaunay in .
In that case, we get that  is among the first  counterclockwise -crossings , which can happen for at most  pairs .

\medskip
\noindent {\bf Case (c).} Neither of the previous two cases holds. 
In particular, there exists a time  which is the {\it first} such time when  belongs to .
Similarly, there exists a time  which is the {\it last} such time when  belongs to . See Figure \ref{Fig:SchemeCrossRq}.

More precisely, if  is Delaunay at time , then we have . Otherwise, if , this is one of the critical times when  enters . As reviewed in Section \ref{Sec:Prelim},  experiences then either a Delaunay co-circularity or a hull event (where  is hit by some point of ). In the latter case,  is not strictly Delaunay at time  and appears in  only {\it right after} this event. The time  has fully symmetrical properties. For simplicity of presentation, we consider the edges  and  to be Delaunay at the respective times  and .

\begin{figure}[htbp]
\begin{center}
\begin{picture}(0,0)\includegraphics{SchemeCrossRq.pstex}\end{picture}\setlength{\unitlength}{5526sp}\begingroup\makeatletter\ifx\SetFigFont\undefined \gdef\SetFigFont#1#2#3#4#5{\reset@font\fontsize{#1}{#2pt}\fontfamily{#3}\fontseries{#4}\fontshape{#5}\selectfont}\fi\endgroup \begin{picture}(1786,752)(1184,-1205)
\put(2143,-801){\makebox(0,0)[lb]{\smash{{\SetFigFont{12}{14.4}{\rmdefault}{\mddefault}{\updefault}{\color[rgb]{0,0,0}}}}}}
\put(1474,-1139){\makebox(0,0)[lb]{\smash{{\SetFigFont{12}{14.4}{\rmdefault}{\mddefault}{\updefault}{\color[rgb]{0,0,0}}}}}}
\put(1217,-1138){\makebox(0,0)[lb]{\smash{{\SetFigFont{12}{14.4}{\rmdefault}{\mddefault}{\updefault}{\color[rgb]{0,0,0}}}}}}
\put(2413,-1159){\makebox(0,0)[lb]{\smash{{\SetFigFont{12}{14.4}{\rmdefault}{\mddefault}{\updefault}{\color[rgb]{0,0,0}}}}}}
\put(1958,-564){\makebox(0,0)[lb]{\smash{{\SetFigFont{12}{14.4}{\rmdefault}{\mddefault}{\updefault}{\color[rgb]{0,0,0}}}}}}
\put(1724,-786){\makebox(0,0)[lb]{\smash{{\SetFigFont{12}{14.4}{\rmdefault}{\mddefault}{\updefault}{\color[rgb]{0,0,0}}}}}}
\put(2955,-966){\makebox(0,0)[lb]{\smash{{\SetFigFont{12}{14.4}{\rmdefault}{\mddefault}{\updefault}{\color[rgb]{0,0,0}}}}}}
\put(2652,-807){\makebox(0,0)[lb]{\smash{{\SetFigFont{12}{14.4}{\rmdefault}{\mddefault}{\updefault}{\color[rgb]{0,0,0}}}}}}
\put(1684,-1138){\makebox(0,0)[lb]{\smash{{\SetFigFont{12}{14.4}{\rmdefault}{\mddefault}{\updefault}{\color[rgb]{0,0,0}}}}}}
\put(1305,-794){\makebox(0,0)[lb]{\smash{{\SetFigFont{12}{14.4}{\rmdefault}{\mddefault}{\updefault}{\color[rgb]{0,0,0}}}}}}
\end{picture} \caption{\small Proposition \ref{Prop:ExtraCollin}. The edge  is hit in  by at least one of . Symmetrically, the edge  is hit in  by at least one of .
The four points  are co-circular at times  and .
Note that  is Delaunay at both times , and  is Delaunay at both times .}
\label{Fig:SchemeCrossRq}
\end{center}
\vspace{-0.6cm}
\end{figure} 



\begin{proposition}\label{Prop:ExtraCollin}
With the above assumptions, the edge  is hit in the interval  by at least one of the points , and, symmetrically, the edge  is hit in the interval  by at least one of the points .
\end{proposition}
\begin{proof}
Clearly, it is sufficient to show  is hit in  by  and/or . The symmetric crossing of  by  and/or  is then obtained by repeating the same analysis in the time-reversed frame (thereby switching the roles of  and ).

Indeed, applying Lemma \ref{Lemma:OnceCollin} for the two crossings  and  shows that the four points  are co-circular in each of the intervals  and . Specifically, the former co-circularity (in ) is red-blue with respect to the edges  and , so it occurs at some time  (because  is Delaunay throughout ). The latter co-circularity is red-blue with respect to  and , so it must occur at some time  in the symmetric interval . See Figure \ref{Fig:SchemeCrossRq} for a schematic summary.



Clearly, the above two co-circularities of  cannot coincide, so  is the {\it only} co-circularity of  in the interval  (which contains ). 
To obtain the asserted crossing of  by  or/and , we distinguish between the following two sub-cases.


\begin{figure}[htbp]
\begin{center}
\input{CrossQr.pstex_t}\hspace{4cm}\begin{picture}(0,0)\includegraphics{CrossQr1.pstex}\end{picture}\setlength{\unitlength}{4144sp}\begingroup\makeatletter\ifx\SetFigFont\undefined \gdef\SetFigFont#1#2#3#4#5{\reset@font\fontsize{#1}{#2pt}\fontfamily{#3}\fontseries{#4}\fontshape{#5}\selectfont}\fi\endgroup \begin{picture}(2693,1511)(918,-1765)
\put(933,-718){\makebox(0,0)[lb]{\smash{{\SetFigFont{11}{13.2}{\rmdefault}{\mddefault}{\updefault}{\color[rgb]{0,0,0}}}}}}
\put(3206,-994){\makebox(0,0)[lb]{\smash{{\SetFigFont{12}{14.4}{\rmdefault}{\mddefault}{\updefault}{\color[rgb]{0,0,0}}}}}}
\put(2740,-1710){\makebox(0,0)[lb]{\smash{{\SetFigFont{12}{14.4}{\rmdefault}{\mddefault}{\updefault}{\color[rgb]{0,0,0}}}}}}
\put(1973,-1452){\makebox(0,0)[lb]{\smash{{\SetFigFont{12}{14.4}{\rmdefault}{\mddefault}{\updefault}{\color[rgb]{0,0,0}}}}}}
\put(3374,-539){\makebox(0,0)[lb]{\smash{{\SetFigFont{12}{14.4}{\rmdefault}{\mddefault}{\updefault}{\color[rgb]{0,0,0}}}}}}
\put(2626,-401){\makebox(0,0)[lb]{\smash{{\SetFigFont{12}{14.4}{\rmdefault}{\mddefault}{\updefault}{\color[rgb]{0,0,0}}}}}}
\end{picture} \\
\vspace{0.5cm}
\input{CrossQrBefore.pstex_t}\hspace{4cm}\begin{picture}(0,0)\includegraphics{CrossQrBefore1.pstex}\end{picture}\setlength{\unitlength}{4342sp}\begingroup\makeatletter\ifx\SetFigFont\undefined \gdef\SetFigFont#1#2#3#4#5{\reset@font\fontsize{#1}{#2pt}\fontfamily{#3}\fontseries{#4}\fontshape{#5}\selectfont}\fi\endgroup \begin{picture}(1901,1563)(1571,-1830)
\put(3131,-934){\makebox(0,0)[lb]{\smash{{\SetFigFont{12}{14.4}{\rmdefault}{\mddefault}{\updefault}{\color[rgb]{0,0,0}}}}}}
\put(2614,-390){\makebox(0,0)[lb]{\smash{{\SetFigFont{12}{14.4}{\rmdefault}{\mddefault}{\updefault}{\color[rgb]{0,0,0}}}}}}
\put(2870,-1677){\makebox(0,0)[lb]{\smash{{\SetFigFont{12}{14.4}{\rmdefault}{\mddefault}{\updefault}{\color[rgb]{0,0,0}}}}}}
\put(1892,-1315){\makebox(0,0)[lb]{\smash{{\SetFigFont{12}{14.4}{\rmdefault}{\mddefault}{\updefault}{\color[rgb]{0,0,0}}}}}}
\put(1586,-1766){\makebox(0,0)[lb]{\smash{{\SetFigFont{11}{13.2}{\rmdefault}{\mddefault}{\updefault}{\color[rgb]{0,0,0}}}}}}
\put(3457,-1415){\makebox(0,0)[lb]{\smash{{\SetFigFont{12}{14.4}{\rmdefault}{\mddefault}{\updefault}{\color[rgb]{0,0,0}}}}}}
\end{picture} \caption{\small Proof of Proposition \ref{Prop:ExtraCollin}. Arguing that  is hit by (at least) one of  in . Top: Possible trajectories of  if it leaves the cap  at time . The asserted crossing of  occurs in . Bottom: Possible trajectories of  if it enters the opposite cap  at time , so the asserted crossing occurs in .}
\label{Fig:ExtraCollin}
\end{center}
\end{figure} 

\smallskip
\noindent(i) The point  lies at time  in .
As prescribed in Lemma \ref{Lemma:OnceCollin}, this co-circularity occurs when  leaves the cap , so the Delaunayhood of  is violated right after time  by  and .
By Lemma \ref{Lemma:MustCross}, and keeping in mind that  is Delaunay at time , the edge  is hit by at least one of the two points  at some moment in . See Figure \ref{Fig:ExtraCollin} (top).

\smallskip
\noindent(ii) The point  lies at time  in .
As prescribed in Lemma \ref{Lemma:OnceCollin}, this co-circularity occurs when  enters the cap , so the Delaunayhood of  is violated right {\it before} time  by  and .
By Lemma \ref{Lemma:MustCross}, and keeping in mind that  is Delaunay at time , the edge  is hit by at least one of the two points  at some moment in . See Figure \ref{Fig:ExtraCollin} (bottom).
\end{proof}

Combining the new collinearities in Proposition \ref{Prop:ExtraCollin} with the already existing crossings of  and  by  shows that at least one of the triples , ,  performs two collinearities, of distinct order types.
If we manage to amplify the above additional crossings of  and  into full-fledged Delaunay crossings (as we did in Section \ref{Sec:DelCocircs} and in case (a)), then some sub-triple in  will necessarily perform two single Delaunay crossings, so our analysis will bottom out via Lemma \ref{Lemma:TwiceCollin}.

As a preparation,
we first apply Theorem \ref{Thm:RedBlue} in  over the interval , and then apply it in  over , both times with the second constant parameter  instead of . (We again emphasize that  is Delaunay at both endpoints of , and  is Delaunay at both endpoints of .)

\smallskip
\noindent{\bf Charging events in .}
Consider the first application of Theorem \ref{Thm:RedBlue}. Refer to Figure \ref{Fig:ExtendRq}.
If one of the first two conditions of Theorem \ref{Thm:RedBlue} holds, we can charge the pair  within  either to 
-shallow co-circularities, or to a -shallow collinearity. Here the crucial observation is that every co-circularity or collinearity (which occurs in  and involves  and ) is charged in this manner at most  times. Indeed, by Proposition \ref{Prop:Balanced},
at most  counterclockwise -crossings end in . Moreover, unless , no -crossings can even partly overlap (let alone end in) , until  returns to  at time .
Thus,  is among the  counterclockwise -crossings that are the latest to end before any of the charged collinearity or co-circularity events (all occurring during ). 
Arguing as in the previous chargings, the number of consecutive pairs  for which such a charging applies is at most .

Finally, if condition (iii) of Theorem \ref{Thm:RedBlue} holds then the Delaunayhood of  can be restored, throughout the interval , by removing a set  of at most  points of  (including  and/or ). (By Lemma \ref{Lemma:Crossing},  is also Delaunay throughout , so its Delaunayhood extends, in , to an even larger interval .)
Recalling Proposition \ref{Prop:ExtraCollin}, we distinguish between the following two subcases.

\begin{figure}[htbp]
\begin{center}
\begin{picture}(0,0)\includegraphics{ExtendRq.pstex}\end{picture}\setlength{\unitlength}{5526sp}\begingroup\makeatletter\ifx\SetFigFont\undefined \gdef\SetFigFont#1#2#3#4#5{\reset@font\fontsize{#1}{#2pt}\fontfamily{#3}\fontseries{#4}\fontshape{#5}\selectfont}\fi\endgroup \begin{picture}(1786,612)(1239,-1184)
\put(3010,-970){\makebox(0,0)[lb]{\smash{{\SetFigFont{12}{14.4}{\rmdefault}{\mddefault}{\updefault}{\color[rgb]{0,0,0}}}}}}
\put(1830,-877){\makebox(0,0)[lb]{\smash{{\SetFigFont{12}{14.4}{\rmdefault}{\mddefault}{\updefault}{\color[rgb]{0,0,0}}}}}}
\put(2762,-866){\makebox(0,0)[lb]{\smash{{\SetFigFont{12}{14.4}{\rmdefault}{\mddefault}{\updefault}{\color[rgb]{0,0,0}}}}}}
\put(2505,-887){\makebox(0,0)[lb]{\smash{{\SetFigFont{12}{14.4}{\rmdefault}{\mddefault}{\updefault}{\color[rgb]{0,0,0}}}}}}
\put(2167,-683){\makebox(0,0)[lb]{\smash{{\SetFigFont{12}{14.4}{\rmdefault}{\mddefault}{\updefault}{\color[rgb]{0,0,0}}}}}}
\put(2413,-1138){\makebox(0,0)[lb]{\smash{{\SetFigFont{12}{14.4}{\rmdefault}{\mddefault}{\updefault}{\color[rgb]{0,0,0}}}}}}
\put(1564,-796){\makebox(0,0)[lb]{\smash{{\SetFigFont{12}{14.4}{\rmdefault}{\mddefault}{\updefault}{\color[rgb]{0,0,0}}}}}}
\put(1378,-803){\makebox(0,0)[lb]{\smash{{\SetFigFont{12}{14.4}{\rmdefault}{\mddefault}{\updefault}{\color[rgb]{0,0,0}}}}}}
\put(2027,-1110){\makebox(0,0)[lb]{\smash{{\SetFigFont{12}{14.4}{\rmdefault}{\mddefault}{\updefault}{\color[rgb]{0,0,0}}}}}}
\put(1684,-1138){\makebox(0,0)[lb]{\smash{{\SetFigFont{12}{14.4}{\rmdefault}{\mddefault}{\updefault}{\color[rgb]{0,0,0}}}}}}
\end{picture} \caption{\small Applying Theorem \ref{Thm:RedBlue} in  over --a schematic summary. The edge  is Delaunay at both times . In cases (i), (ii), each -shallow event is charged only  times because  is among the last  counterclockwise -crossings to end before the respective time  of the event. In case (iii) we have a subset  of at most  points whose removal extends the Delaunayhood of  to .}
\label{Fig:ExtendRq}
\end{center}
\end{figure} 

If  is hit in  by , then
the smaller set  yields a Delaunay crossing of  (or of its reversely oriented copy ) by . This is in addition to the inherited Delaunay crossing of  by . As in case (a), it is easy to check that both of these crossings in  must be single Delaunay crossings. 
Hence, Lemma \ref{Lemma:TwiceCollin}, combined with the Clarkson-Shor argument \cite{CS}, in a manner similar to that used in Section \ref{Sec:Prelim} and the previous cases, provides an upper bound of  on the number of such triples . Clearly, this also bounds the overall number of such consecutive pairs  of Delaunay crossings.

To conclude, we may assume that  is hit in  by , so the smaller set  yields a Delaunay crossing of  by .


\smallskip
\noindent{\bf Charging events in .} The second application of Theorem \ref{Thm:RedBlue} in  over  is fully symmetric to the first one. Refer to Figure \ref{Fig:ExtendRa}. If at least one of conditions (i), (ii) is satisfied, we charge the pair  within  either to  -shallow co-circularities or to an -shallow collinearity that occur in  during that interval. The crucial observation is that  is among the first  counterclockwise -crossings to begin after each charged event, which also involves  and . Hence, every collinearity or co-circularity is charged at most  times, so this charging accounts for at most  pairs.

For each of the remaining pairs  we have a set  of at most  points (possibly including  and/or ) whose removal restores the Delaunayhood of  throughout . (By Lemma \ref{Lemma:Crossing},  is also Delaunay throughout , so its Delaunayhood extends, in , to an even larger interval .)
To complete the proof of Theorem \ref{Thm:OrdinaryCrossings}, we again recall Proposition \ref{Prop:ExtraCollin} and distinguish between the two possible crossings of .

\begin{figure}[htbp]
\begin{center}
\begin{picture}(0,0)\includegraphics{ExtendRa.pstex}\end{picture}\setlength{\unitlength}{5526sp}\begingroup\makeatletter\ifx\SetFigFont\undefined \gdef\SetFigFont#1#2#3#4#5{\reset@font\fontsize{#1}{#2pt}\fontfamily{#3}\fontseries{#4}\fontshape{#5}\selectfont}\fi\endgroup \begin{picture}(1784,582)(925,-1279)
\put(1378,-803){\makebox(0,0)[lb]{\smash{{\SetFigFont{12}{14.4}{\rmdefault}{\mddefault}{\updefault}{\color[rgb]{0,0,0}}}}}}
\put(2027,-1110){\makebox(0,0)[lb]{\smash{{\SetFigFont{12}{14.4}{\rmdefault}{\mddefault}{\updefault}{\color[rgb]{0,0,0}}}}}}
\put(1791,-806){\makebox(0,0)[lb]{\smash{{\SetFigFont{12}{14.4}{\rmdefault}{\mddefault}{\updefault}{\color[rgb]{0,0,0}}}}}}
\put(1255,-1233){\makebox(0,0)[lb]{\smash{{\SetFigFont{12}{14.4}{\rmdefault}{\mddefault}{\updefault}{\color[rgb]{0,0,0}}}}}}
\put(1184,-840){\makebox(0,0)[lb]{\smash{{\SetFigFont{12}{14.4}{\rmdefault}{\mddefault}{\updefault}{\color[rgb]{0,0,0}}}}}}
\put(2694,-964){\makebox(0,0)[lb]{\smash{{\SetFigFont{12}{14.4}{\rmdefault}{\mddefault}{\updefault}{\color[rgb]{0,0,0}}}}}}
\put(1719,-1126){\makebox(0,0)[lb]{\smash{{\SetFigFont{12}{14.4}{\rmdefault}{\mddefault}{\updefault}{\color[rgb]{0,0,0}}}}}}
\put(1063,-1050){\makebox(0,0)[lb]{\smash{{\SetFigFont{12}{14.4}{\rmdefault}{\mddefault}{\updefault}{\color[rgb]{0,0,0}}}}}}
\put(2413,-1138){\makebox(0,0)[lb]{\smash{{\SetFigFont{12}{14.4}{\rmdefault}{\mddefault}{\updefault}{\color[rgb]{0,0,0}}}}}}
\put(1564,-796){\makebox(0,0)[lb]{\smash{{\SetFigFont{12}{14.4}{\rmdefault}{\mddefault}{\updefault}{\color[rgb]{0,0,0}}}}}}
\end{picture} \caption{\small Applying Theorem \ref{Thm:RedBlue} in  over --a schematic summary. The edge  is Delaunay at both times . In cases (i), (ii), each -shallow event is charged only  times because  is among the first  counterclockwise -crossings to begin after the respective time  of the event. In case (iii) we have a subset  of at most  points whose removal extends the Delaunayhood of  to .}
\label{Fig:ExtendRa}
\end{center}
\end{figure} 


If  is hit in  by , then
the smaller set  yields a Delaunay crossing of  (or of its reversely oriented copy ) by , and a Delaunay crossing of  by , which are easily checked to be single Delaunay crossings. 
Hence, Lemma \ref{Lemma:TwiceCollin}, combined with the Clarkson-Shor argument \cite{CS}, provides an upper bound of  on the number of such triples , which also bounds the overall number of such consecutive pairs .

Finally, if  is hit in  by , the triple  performs two Delaunay crossings in the triangulation , that is, the crossing of  by  (occurring entirely within ), and the crossing of  by  (occurring entirely within ). The standard assumptions on the possible collinearities in  readily imply that both of these crossings are in fact single Delaunay crossings.
Combining Lemma \ref{Lemma:TwiceCollin} with the probabilistic argument of Clarkson and Shor \cite{CS}, as above, we get that the number of such triples  is at most . Note, though, that our goal is to bound the number of possible pairs  (or, alternatively, ) rather than . However, recall that  hits  during the time interval , and  is then among the  counterclockwise -crossings that end latest before that collinearity of . Hence, any triple  can arise in the charging for at most  triples . In conclusion, the number of consecutive pairs  that fall into this final subcase is at most . 

Adding up the bounds obtained in cases (a)--(c),  and in the preparatory charging of -shallow events in , the theorem follows.
\end{proof}

\subsection{The number of double Delaunay crossings}\label{Subsec:Double}
In this subsection we show that any set  of  points moving as above in  admits at most  double Delaunay crossings.
Since double Delaunay crossings are not possible if no ordered triple of points can be collinear more than once
(i.e., if for any  the third point  can hit the segment  at most once), we may assume throughout this subsection that no triple of points in  can be collinear more than twice.

Without loss of generality, we only bound the number of such double Delaunay crossings  whose point  crosses through  from  to  during the first collinearity of  (and then returns back to  during the second collinearity).
Indeed, if the crossing  does not satisfy the above condition then they are satisfied by .
Our goal is to show that (on average) a point  of  is involved in only few Delaunay crossings of edges that share the same endpoint .


The following theorem provides certain structural properties of two double crossings that share the same crossing point () and one endpoint () of the crossed edges.

\begin{figure}[htbp]
\begin{center}
\begin{picture}(0,0)\includegraphics{DoubleCrossing.pstex}\end{picture}\setlength{\unitlength}{4539sp}\begingroup\makeatletter\ifx\SetFigFont\undefined \gdef\SetFigFont#1#2#3#4#5{\reset@font\fontsize{#1}{#2pt}\fontfamily{#3}\fontseries{#4}\fontshape{#5}\selectfont}\fi\endgroup \begin{picture}(1556,1313)(1871,-1568)
\put(3128,-1326){\makebox(0,0)[lb]{\smash{{\SetFigFont{11}{13.2}{\rmdefault}{\mddefault}{\updefault}{\color[rgb]{0,0,0}}}}}}
\put(1886,-889){\makebox(0,0)[lb]{\smash{{\SetFigFont{11}{13.2}{\rmdefault}{\mddefault}{\updefault}{\color[rgb]{0,0,0}}}}}}
\put(2575,-391){\makebox(0,0)[lb]{\smash{{\SetFigFont{11}{13.2}{\rmdefault}{\mddefault}{\updefault}{\color[rgb]{0,0,0}}}}}}
\put(2038,-1471){\makebox(0,0)[lb]{\smash{{\SetFigFont{11}{13.2}{\rmdefault}{\mddefault}{\updefault}{\color[rgb]{0,0,0}}}}}}
\end{picture} \caption{\small The trace of  according to Theorem \ref{Thm:OrderSpecialCrossings}. The four points  are involved during  in two co-circularities, which are red-blue with respect to the edges  and .}
\label{Fig:StayDelaunay}
\end{center}
\end{figure} 

\begin{theorem}\label{Thm:OrderSpecialCrossings}
Let  and  be two double Delaunay crossings of -edges (that is, edges incident to )  by the same point . 
Assume that
the first collinearity of  occurs before the first collinearity of .
Then the following properties hold (with the conventions assumed above):\\
\indent(i)  lies in  at both times when  hits .\\
\indent (ii)  lies in  at both times when  hits .\\
\indent (iii) The points  are involved during  in two co-circularities, both of them red-blue with respect to  and occurring when  and . \\
\indent (iv) One of the two co-circularities in (iii) occurs before the beginning of ; right before it the Delaunayhood of  is violated by  and . A symmetric such co-circularity occurs after the end of ;
right after it the Delaunayhood of  is again violated by  and .
In particular, .
\end{theorem}
The schematic description of the motion of  during , according to the above theorem, is depicted in Figure \ref{Fig:StayDelaunay} (right).
Clearly, a suitable variant of Theorem \ref{Thm:OrderSpecialCrossings} exists also for similar pairs of double crossings of incoming -edges  that are oriented {\it towards}  (again, by the same point ).
\begin{proof}
We first establish Part (ii) of the theorem.
The crucial observation is that the first collinearity of  occurs when  lies in  (i.e., during the interval between the two collinearities of ). 
Indeed, otherwise the point  must lie in  at both collinearities of , and  must lie in  at both collinearities of . 
We shall prove that, in this hypothetical setup, the points  are involved in two co-circularities during  which are red-blue with respect to , and in a symmetric pair of co-circularities during , both of them red-blue with respect to . That will clearly contradict the assumption that any four points can be co-circular at most twice.

Indeed, in the above situation the point  lies in the cap  shortly before the first collinearity of , and shortly after their second collinearity. 
Since  contains no points at the beginning of , the point  must have entered this cap before the first collinearity of . Moreover,  can enter this cap only through the boundary of , for otherwise it would hit  during , and no point of  can hit  during its Delaunay crossing by . This argument gives us the first of the promised two red-blue co-circularities that  define with respect to . The second such co-circularity is symmetric to the first one, and occurs when  leaves the cap  (and after  returns to  through ). See Figure \ref{Fig:DoubleFour} (left).
The other pair of co-circularities, both red-blue with respect to , is obtained by applying a fully symmetric argument to the cap  and the point . See Figure \ref{Fig:DoubleFour} (center). (For example, we can switch the roles of  and  by reversing the direction of the time axis.)
Finally, all four co-circularities are distinct, because the same co-circularity cannot be red-blue with respect to two edges  with a common endpoint. 

\begin{figure}[htbp]
\begin{center}
\input{DoubleFourCocircs1.pstex_t}\hspace{2cm}\input{DoubleFourCocircs.pstex_t}\hspace{2cm}\begin{picture}(0,0)\includegraphics{CannotReturn.pstex}\end{picture}\setlength{\unitlength}{4539sp}\begingroup\makeatletter\ifx\SetFigFont\undefined \gdef\SetFigFont#1#2#3#4#5{\reset@font\fontsize{#1}{#2pt}\fontfamily{#3}\fontseries{#4}\fontshape{#5}\selectfont}\fi\endgroup \begin{picture}(1601,1470)(1952,-1548)
\put(2621,-1493){\makebox(0,0)[lb]{\smash{{\SetFigFont{11}{13.2}{\rmdefault}{\mddefault}{\updefault}{\color[rgb]{0,0,0}}}}}}
\put(2038,-1471){\makebox(0,0)[lb]{\smash{{\SetFigFont{11}{13.2}{\rmdefault}{\mddefault}{\updefault}{\color[rgb]{0,0,0}}}}}}
\put(2546,-405){\makebox(0,0)[lb]{\smash{{\SetFigFont{11}{13.2}{\rmdefault}{\mddefault}{\updefault}{\color[rgb]{0,0,0}}}}}}
\put(3162,-1136){\makebox(0,0)[lb]{\smash{{\SetFigFont{11}{13.2}{\rmdefault}{\mddefault}{\updefault}{\color[rgb]{0,0,0}}}}}}
\end{picture} \caption{\small Proof of Theorem \ref{Thm:OrderSpecialCrossings}. Left and center: The hypothetical case where  first hits  within , after twice hitting . The points  are involved in a pair of co-circularities during , and in a symmetric pair of co-circularities during . Right: The hypothetical traces of  if it enters  before  (and before the second collinearity of  occurs).}
\label{Fig:DoubleFour}
\end{center}
\end{figure} 




Hence, we can assume, from now on, that the first time when  hits  occurs when both points lie in .
To complete the proof of Part (ii), it suffices to show that the points  and  still remain in  during the second collinearity of the triple .
Indeed, otherwise  must lie in  when  hits  for the second time, because, untill it crosses  again,  lies in  which coincides with  at the second crossing of  by . See Figure \ref{Fig:DoubleFour} (right).
That is,  must cross  from  to  while  still remains in , and before  hits the edges  for the second time. In particular, the above collinearity of  must occur during . Clearly, the point  can potentially cross  in three ways.
If  crosses  within , this contradicts the definition of  as the interval of the Delaunay crossing of  by . If  hits  within the ray emanating from  then (at that very moment)  hits , which contradicts the definition of . Finally,  cannot hit  within the outer ray emanating from  before an additional (and forbidden) collinearity of  takes place.  This establishes part (ii), and the analysis given above immediately implies part (i) two.

Part (i) follows immediately from Part (ii), because  lies in  during both collinearities of .

Parts (iii) and (iv) follow from Parts (i) and (ii). Indeed, recall that the open disc  contains no points of  at the beginning of . Right before  hits  for the first time, the right cap  of this disc contains . Clearly,  first enters this cap through the corresponding portion of . This determines the first red-blue co-circularity with respect to , right before which the Delaunayhood of  is violated by  and . The symmetric such co-circularity occurs during  when the point  leaves the cap , after the second collinearity of . Clearly, the Delaunayhood of  is violated right after that co-circularity by  and . By Lemma \ref{Lemma:Crossing}, neither of these co-circularities can occur during , because  remains Delaunay throughout . Hence, the former one occurs, according to the previously established Parts (i) and (ii), before , and the latter one occurs after . This establishes parts (iii) and (iv), and completes the proof.
\end{proof}

\begin{theorem}\label{Thm:SpecialCrossings}
Let  be a set of  points, whose motion in  respects the following conventions: (i) any four points can be co-circular at most twice, and (ii) no three points can be collinear more than twice. Then  admits at most  double Delaunay crossings.
\end{theorem}
\begin{proof}
We fix a pair of points  in . Our strategy is to show that, for an average such pair, there is at most a constant number of double Delaunay crossings of -edges by .
Indeed, let   be the complete list of such double Delaunay crossings of -edges by , and assume that  hits the edges , for the first time, in this same order.
By Theorem \ref{Thm:OrderSpecialCrossings}, the respective intervals of the above double crossings form a nested sequence .


\begin{figure}[htbp]
\begin{center}
\input{QuadraticDouble1.pstex_t}\hspace{2cm}\begin{picture}(0,0)\includegraphics{QuadraticDouble2.pstex}\end{picture}\setlength{\unitlength}{4539sp}\begingroup\makeatletter\ifx\SetFigFont\undefined \gdef\SetFigFont#1#2#3#4#5{\reset@font\fontsize{#1}{#2pt}\fontfamily{#3}\fontseries{#4}\fontshape{#5}\selectfont}\fi\endgroup \begin{picture}(1592,1672)(1835,-1568)
\put(2749,-471){\makebox(0,0)[lb]{\smash{{\SetFigFont{11}{13.2}{\rmdefault}{\mddefault}{\updefault}{\color[rgb]{0,0,0}}}}}}
\put(2038,-1471){\makebox(0,0)[lb]{\smash{{\SetFigFont{11}{13.2}{\rmdefault}{\mddefault}{\updefault}{\color[rgb]{0,0,0}}}}}}
\put(1850,-31){\makebox(0,0)[lb]{\smash{{\SetFigFont{11}{13.2}{\rmdefault}{\mddefault}{\updefault}{\color[rgb]{0,0,0}}}}}}
\put(2156,-773){\makebox(0,0)[lb]{\smash{{\SetFigFont{11}{13.2}{\rmdefault}{\mddefault}{\updefault}{\color[rgb]{0,0,0}}}}}}
\put(3128,-1326){\makebox(0,0)[lb]{\smash{{\SetFigFont{11}{13.2}{\rmdefault}{\mddefault}{\updefault}{\color[rgb]{0,0,0}}}}}}
\end{picture} \caption{\small Proof of Theorem \ref{Thm:SpecialCrossings}. Left: If the double crossing  ends before the end of  then the second co-circularity of  occurs during . Right: If the double crossing  ends after  then the second co-circularity of  occurs during .}
\label{Fig:QuadDouble}
\end{center}
\end{figure} 

Clearly, the first crossing  can be uniquely charged to the pair .
Now assume that . We show that each of the additional double Delaunay crossings , for , 
can be uniquely charged to the corresponding pair .
Specifically, we show that no double Delaunay crossing of incoming -edges  (that is, -edges that are oriented towards ), by , can end after . In other words,  is the ``last" such double crossing.

Indeed, fix  as above. We first show that no double crossing of the form  can end during the interval which lasts from the end of  and to the end of . Indeed, suppose to the contrary that such a situation occurs, and apply a suitable variant of Theorem \ref{Thm:OrderSpecialCrossings} to the double Delaunay crossings of -edges  and  by . 
By Part (iv) of that theorem,  is contained in , and the four points  are involved in a red-blue co-circularity with respect to  during the second portion of . See Figure \ref{Fig:QuadDouble} (left). Right after that co-circularity, the Delaunayhood of  is violated by  and . If  ends before the end of , the above co-circularity must occur during  (as ), which contradicts Lemma \ref{Lemma:Crossing} (applied to the crossing of  by ).

It remains to show that no double Delaunay crossing , as above, can end after the end of . Indeed, by Part (iv) of Theorem \ref{Thm:OrderSpecialCrossings} (now applied to the double crossings of the -edges  and of , by ), the points  are involved in a co-circularity during the second portion of . Right after this co-circularity, the Delaunayhood of  is violated by  and . 
If the interval  (which contains ) ends after the end of , the aforementioned co-circularity must occur during ; see Figure \ref{Fig:QuadDouble} (right). However, this is another contradiction to Lemma \ref{Lemma:Crossing} (now applied to the crossing of  by , which takes place during ).

We have shown that every double Delaunay crossing can be uniquely charged to an (ordered) pair of points of , so their number is , as asserted.
\end{proof}











\section{Conclusion} We have studied the number of discrete changes in the Delaunay triangulation of a set  of  points moving along pseudo-algebraic trajectories in the plane, so that any four points of  can be co-circular at most {\it twice} during the motion. 
We have introduced a new concept of Delaunay crossings, and established several interesting structural properties of these crossings.
In our analysis we have used Theorem \ref{Thm:RedBlue} to reduce the problem of bounding the number of Delaunay co-circularities to the more specific problem of bounding the number of Delaunay crossings. Notice that the proof of Theorem \ref{Thm:RedBlue} did not rely on any assumptions concerning the motion of the points of  (except for its being pseudo-algebraic of constant degree). Moreover, the aforementioned reduction easily extends to 
the case in which any four points of  can be co-circular at most {\it three times} during the motion.
For these reasons, the author believes that the techniques introduced in this paper can be used to establish sub-cubic upper bounds for more general instances of the problem, such as the instance where the points are moving along straight lines with equal speeds.

\section{Acknowledgements} I would like to thank my former Ph.D. advisor Micha Sharir 
whose help made this work possible.
In particular, I would like to thank him for the insightful discussions, and, especially, for his invaluable help in the preparation of this paper. 

In addition, I would like to thank the anonymous
DCG referees for valuable suggestions that helped to improve the
presentation.









\begin{thebibliography}{10}










\bibitem{ABGHZ}
P.~K. Agarwal, J.~Basch, L.~J. Guibas, J.~Hershberger, and L.~Zhang,
Deformable free-space tilings for kinetic collision detection,
{\em Internat. J. Robotics Research} 21 (3) (2002), 179--197.

\bibitem{ASS}
P. K. Agarwal, O. Cheong and M. Sharir, The overlay of lower
envelopes in 3-space and its applications, {\it Discrete Comput.
Geom.} 15 (1996), 1--13.


\bibitem{Stable}
P. K. Agarwal, J. Gao, L. Guibas, H. Kaplan, V. Koltun, N. Rubin and M. Sharir,
Kinetic stable Delaunay graphs, \textit{Proc. 26th Annu. Symp. on Comput. Geom.} (2010), 127--136.

\bibitem{AWY}
P. K. Agarwal, Y. Wang and H. Yu,
A 2D kinetic triangulation with near-quadratic topological changes,
\textit{Discrete Comput. Geom.} 36 (2006), 573--592.







\bibitem{AK}
F. Aurenhammer and R. Klein,
Voronoi diagrams,
in {\it Handbook of Computational Geometry},
J.-R. Sack and J. Urrutia, Eds.,
Elsevier, Amsterdam, 2000,
pages 201--290.



\bibitem{Chew}
L. P. Chew,
Near-quadratic bounds for the  Voronoi diagram of moving points,
{\em Comput. Geom. Theory Appl.}  7 (1997), 73--80.

\bibitem{CD}
L. P. Chew and R. L. Drysdale,
Voronoi diagrams based on convex distance functions,
{\em Proc. First Annu. ACM Sympos. Comput. Geom.}, 1985, pp.~235--244.

\bibitem{CS}
K.~Clarkson and P.~Shor, Applications of random sampling in
computational geometry, II, \emph{Discrete Comput. Geom.} 4 (1989),
387--421.



\bibitem{d-slsv-34}
B.~Delaunay,
Sur la sph{\`e}re vide. {A} la memoire de {Georges} {Voronoi},
{\em Izv. Akad. Nauk SSSR, Otdelenie Matematicheskih i Estestvennyh
Nauk} 7 (1934), 793--800.

\bibitem{TOPP}
E.~D.~Demaine, J.~S.~B.~Mitchell, and J.~O'Rourke,\\
The Open Problems Project,
\texttt{http://www.cs.smith.edu/\~{ }orourke/TOPP/}.

\bibitem{Ed2}
H. Edelsbrunner,
{\em Geometry and Topology for Mesh Generation},
Cambridge University Press, Cambride, 2001.

\bibitem{EM} H. Edelsbrunner and E. P. M\"{u}cke, Simulation of simplicity: a technique to cope with degenerate cases in geometric algorithms, {\it ACM Transactions on Graphics} 9 (1990), 66--104.

\bibitem{FuLee} J.-J. Fu and R. C. T. Lee, Voronoi diagrams of moving points in the plane, {\it Int. J. Comput. Geom. Appl.} 1(1) (1991), 23-32.





\bibitem{gmr-vdmpp-92}
L.~J. Guibas, J.~S.~B. Mitchell and T.~Roos,
Voronoi diagrams of moving points in the plane,
{\em Proc. 17th Internat. Workshop Graph-Theoret. Concepts Comput.
Sci.}, volume 570 of {\em Lecture Notes Comput. Sci.}, pages 113--125.
Springer-Verlag, 1992.

\bibitem{Envelopes3D}
D. Halperin and M. Sharir, New bounds for lower envelopes in three dimensions, with applications to visbility in terrains, {\it Discrete Comput. Geom.} 12 (1994), 313--326.

\bibitem{IKLM}
C. Icking, R. Klein, N.-M. L\^{e} and L. Ma,
Convex distance functions in 3-space are different, {\it Fundam. Inform.} 22 (4) (1995), 331--352.

\bibitem{KRS}
H. Kaplan, N. Rubin and M. Sharir, 
A kinetic triangulation scheme for moving points in the plane, {\it Comput. Geom. Theory Appl.} 44 (2011), 191--205.

\bibitem{Vladlen} V. Koltun,
Ready, Set, Go! The Voronoi diagram of moving points that start from a line, {\it Inf. Process. Lett.} 89(5), (2004), 233--235.

\bibitem{ConstantLines} V. Koltun and M. Sharir, 3-dimensional Euclidean Voronoi diagrams of lines with a fixed number of orientations, {\it SIAM J. Comput.} 32 (3) (2003), 616--642.



\bibitem{SA95}
M.~Sharir and P.~K. Agarwal,
{\em Davenport-Schinzel Sequences and Their Geometric Applications},
Cambridge University Press, New York, 1995.

\end{thebibliography}
\end{document}