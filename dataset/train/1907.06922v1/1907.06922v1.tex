\documentclass[10pt,twocolumn,letterpaper]{article} 

\usepackage{avss}
\usepackage{times}
\usepackage{epsfig}
\usepackage{graphicx}
\usepackage{amsmath}
\usepackage{amssymb}
\usepackage{tipa}
\usepackage{svg}
\usepackage{booktabs}
\usepackage{subfig}


\newcommand\blfootnote[1]{\begingroup
  \renewcommand\thefootnote{}\footnote{#1}\addtocounter{footnote}{-1}\endgroup
}

\PassOptionsToPackage{hyphens}{url}\usepackage[pagebackref=true,breaklinks=true,letterpaper=true,colorlinks,bookmarks=false]{hyperref}

\newif\ifnotes
\notestrue
\newcommand{\todo}[1]{\ifnotes\textcolor[rgb]{1,0,0}{TODO: #1}\fi}
\newcommand{\note}[1]{\ifnotes\textcolor[rgb]{0,0,1}{NOTE: #1}\fi}


\widowpenalty=1000

\avssfinalcopy 

\def\avssPaperID{102} \def\httilde{\mbox{\tt\raisebox{-.5ex}{\symbol{126}}}}
\newcommand{\baseline}{Baseline-50\xspace}

\ifavssfinal\pagestyle{empty}\fi
\begin{document}

\title{Human Pose Estimation for Real-World Crowded Scenarios}
\author{
Thomas Golda\textsuperscript{1,2,*}\quad
Tobias Kalb\textsuperscript{2,*}\quad
Arne Schumann\textsuperscript{2}\quad
J\"urgen Beyerer\textsuperscript{1,2}\\
\begin{minipage}[t]{0.5\linewidth}
\vspace{.05cm}
\begin{center}
\textsuperscript{1}Vision and Fusion Lab\\
Karlsruhe Institute of Technology KIT\\
c/o Technologiefabrik, Haid-und-Neu-Strasse 7\\
76131 Karlsruhe, Germany
\end{center}
\end{minipage}
\begin{minipage}[t]{0.5\linewidth}
\vspace{.05cm}
\begin{center}
\textsuperscript{2}Fraunhofer Institute for Optronics, System\\Technologies and Image Exploitation IOSB\\
Fraunhoferstrasse 1\\
76131 Karlsruhe, Germany
\end{center}
\vspace{.05cm}
\end{minipage}\\
{\tt\small firstname.lastname@iosb.fraunhofer.de}\\
}

\maketitle
\begin{abstract}
Human pose estimation has recently made significant progress with the adoption of deep convolutional neural networks and many applications have attracted tremendous interest in recent years.
However, many of these applications require pose estimation for human crowds, which still is a rarely addressed problem.
For this purpose this work explores methods to optimize pose estimation for human crowds, focusing on challenges introduced with larger scale crowds like people in close proximity to each other, mutual occlusions, and partial visibility of people due to the environment.
In order to address these challenges, multiple approaches are evaluated including: the explicit detection of occluded body parts, a data augmentation method to generate occlusions and the use of the synthetic generated dataset JTA~\cite{fabbri2018learning}.
In order to overcome the transfer gap of JTA originating from a low pose variety and less dense crowds, an extension dataset is created to ease the use for real-world applications.
\end{abstract}

\iffalse
Human pose estimation has recently made significant progress with the adoption of deep convolutional neural networks.
Its many applications have attracted tremendous interest in recent years.
However, many practical applications require pose estimation for human crowds, which still is a rarely addressed problem.
In this work, we explore methods to optimize pose estimation for human crowds, focusing on challenges introduced with dense crowds, such as people in close proximity to each other, mutual occlusions, and partial visibility of people due to the environment. In order to address these challenges, we evaluate three aspects of a pose detection approach: i) a data augmentation method to introduce robustness to occlusions, ii) the explicit detection of occluded body parts,  and iii) the use of the synthetic generated datasets.
The first approach to improve the accuracy in crowded scenarios is to generate occlusions at training time using person and object cutouts from the object recognition dataset COCO (Common Objects in Context).
Furthermore, the synthetically generated dataset JTA (Joint Track Auto) is evaluated for the use in real-world crowd applications.
In order to overcome the transfer gap of JTA originating from a low pose variety and less dense crowds, an extension dataset is created to ease the use for real-world applications.
Additionally, the occlusion flags provided with JTA are utilized to train a model, which explicitly distinguishes between occluded and visible body parts in two distinct branches.
The combination of the proposed additions to the baseline method help to improve the overall accuracy by 4.7\% AP and thereby provide comparable results to current state-of-the-art approaches on the respective dataset.
\fi \section{Introduction}
Articulated human pose estimation has been a commonly addressed problem in computer vision for many years.
It has attracted broad interest for its many applications such as video surveillance, action recognition, human computer interaction and motion capturing.
Recent advances in human pose estimation have been achieved by harnessing the power of deep convolutional neural networks (CNNs) and large-scale pose estimation datasets like COCO~\cite{LinMBHPRDZ14} and MPII~\cite{andriluka14cvpr}.
Up until now most approaches focus on pose estimation for images with few people in uncrowded scenarios.
\iffalse
\blfootnote{978-1-5386-9294-3/18/\S_{\text{flow}}\alphak,l\in\mathbb{N}_0k + l = nP_1^{\text{vis}}P_k^{\text{vis}}P_1^{\text{occ}}P_l^{\text{occ}}G_iP_iiC=\min\{\frac{1}{N} \sum_{i=1}^{N}\frac{N_a^i}{N_b^i},1.0\}NN_b^iiN_a^ij \neq i[0.0,0.1)[0.1,0.8)[0.8,1.0]0.910.69256 \times 19264 \times 48\alpha1.5\text{AP}_{\text{Easy}}\text{AP}_{\text{Med}}\text{AP}_{\text{Hard}}10^{-3}\text{AP}\text{AP}_{\text{Easy}}\text{AP}_{\text{Med}}\text{AP}_{\text{Hard}}10^{-3}\text{AP}\text{AP}_{\text{Easy}}\text{AP}_{\text{Med}}\text{AP}_{\text{Hard}}\text{AP}\text{AP}_{\text{Easy}}\text{AP}_{\text{Med}}\text{AP}_{\text{Hard}}10^{-3}\text{AP}\text{AP}_{\text{Easy}}\text{AP}_{\text{Med}}\text{AP}_{\text{Hard}}$ \\
		\midrule
		Xiao \etal \cite{xiao2018simple}                    & 60.8        & 71.4               & 61.2                 & 51.2 \\ 
		Li \etal \cite{li2018crowdpose}              & \textbf{66.6}        & \textbf{75.7}               & 66.3                & \textbf{57.4}               \\ \midrule
		Ours                    & 65.5        & 75.2               & \textbf{66.6}                 & 53.1               \\
		\bottomrule
		            
	\end{tabular}
	\end{center}
\end{table} \section{Conclusion}
In this work, we examined the impact of different extensions to the training of human pose estimators for crowded scenes.
We show that significant improvements were accomplished by utilizing simple data augmentation techniques like adding COCO~\cite{LinMBHPRDZ14} objects to the input images from crowded scenarios thus simulating occlusions. 
On the other hand, we observe that adding person instances or body parts is less effective and also requires additional attention to avoid creating erroneous poses in a top-down method. 
Our results further prove that JTA~\cite{fabbri2018learning} is a valuable addition to the training data for crowd-level pose estimation, especially since no large datasets are available for pose estimation in crowded scenarios. 
The created extension of JTA, which includes a higher variety of poses and denser crowds, further improves the accuracy on CrowdPose. 
However, solely relying on synthetic training data is not sufficient, due to the domain gap between synthetic and real-world data, originating from limited pose variety and visual limitations of a real-time rendering engine. 
Finally, the combination of the investigated approaches improved the baseline model and achieves results comparable to state-of-the-art methods on the real-world dataset CrowdPose.\\
In future work, we will especially focus on the problem arising by the gap between synthetic and real-world data. 

{\small
\bibliographystyle{ieee}
\bibliography{egbib}
}





\end{document}
