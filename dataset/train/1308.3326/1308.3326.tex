\documentclass{llncs}
\usepackage{fullpage}
\usepackage{makeidx}
\usepackage{algorithm}
\usepackage{algpseudocode}
\usepackage{amsmath}
\usepackage{verbatim}
\usepackage{graphicx}

\begin{document}

\title{Sorted Range Reporting Revisited\thanks{This work is a course project for CS 763 - Computational Geometry, instructed by Timothy Chan.}}

\author{Gelin Zhou}
\institute{David R. Cheriton School of Computer Science, \\ University of Waterloo, Canada. \\
\email{g5zhou@uwaterloo.ca}}

\maketitle

\bibliographystyle{splncs03}

\begin{abstract}
    We consider the two-dimensional sorted range reporting problem.
    Our data structure requires  words of space and  query time,
    where  is the number of points in the query range.
    This data structure improves a recent result of Nekrich and Navarro~\cite{DBLP:conf/swat/NekrichN12} by a factor of  in query time,
    and matches the state of the art for unsorted range reporting~\cite{DBLP:conf/compgeom/ChanLP11}.
\end{abstract}

\section{Introduction}

The orthogonal range searching problems is well-known in the communities of computational geometry and data structures.
For these problems, we need maintain a point set  in -dimensional space,
such that certain functions over points in a given query rectangle  can be computed efficiently.
In this paper we study a variant of the two-dimensional orthogonal range reporting problem,
for which points in the query range are sorted in increasing order of their
-coordinates~\footnote{Increasing/decreasing /-coordinate ordering can be easily supported via coordinate changes.}.
In addition, our data structures can work in an online fashion:
points from  are reported in increasing order of -coordinates
until the query-answering procedure is terminated or all points in  are output. \\
\indent The sorted range reporting problem was proposed by Nekrich and Navarro~\cite{DBLP:conf/swat/NekrichN12}.
Let  denote the number of points in the given query range.
Their linear space data structure requires  query time,
and their data structure with optimal query time occupies  words of space.
These results match the state of the art for unsorted range reporting~\cite{DBLP:conf/compgeom/ChanLP11}.
However, their data structure using  words of space requires  query time,
which is slower than the corresponding result for unsorted range reporting~\cite{DBLP:conf/compgeom/ChanLP11} by a factor of .
In this paper we present a data structure using the same amount of space but only  query time. \\
\indent The sorted range reporting problem is closely related to the orthogonal range successor problem
(sometimes referred to as the range next-value problem)~\cite{DBLP:conf/stacs/IliopoulosCKRW08,DBLP:journals/comgeo/YuHW11,DBLP:conf/swat/NekrichN12}.
Nekrich and Navarro's data structures~\cite{DBLP:conf/swat/NekrichN12} can be used directly to support range successor queries.
Their linear space data structure requires  query time,
and their data structure achieving the optimal  query time occupies  words of space.
Our data structure requires only  words of space to achieve the optimal query time. \\
\indent We assume that the given point set is in an  grid, or \emph{rank space}.
Every two points have different /-coordinates.
The underlying computational model throughout this work is the standard word RAM model with word size . \\
\indent The rest of this paper is organized as follows:
In Section~\ref{section_preliminary} we review Chan, Larsen and Patrascu's data structures~\cite{DBLP:conf/compgeom/ChanLP11} for (unsorted) range reporting
and Nekrich and Navarro's results~\cite{DBLP:conf/swat/NekrichN12} for the sorted variant.
In Section~\ref{section_main} we present our data structures for sorted range reporting and range successor queries.

\section{Preliminary}
\label{section_preliminary}

For completeness, we describe Chan et al.'s work~\cite{DBLP:conf/compgeom/ChanLP11} for unsorted range reporting
in addition to Nekrich and Navarro's work~\cite{DBLP:conf/swat/NekrichN12} for sorted range reporting.
We may modify their presentations for the sake of consistency.

\subsection{Unsorted Range Reporting}
\label{subsection_unsorted}

Chan et al.'s data structures~\cite{DBLP:conf/compgeom/ChanLP11} for range reporting
are based on the \emph{wavelet tree}~\cite{DBLP:journals/siamcomp/Chazelle88,DBLP:conf/soda/GrossiGV03}.
A conceptual range tree  is built on , where  is assumed to be a power of two.
Every node  in  has an associated range.
Let  denote the set of points whose -coordinates are in this range.
It is clear that  contains a point only for every leaf node  at the bottom of .
An internal node  has two children  and , whose associated ranges are a disjoint union of that of .
Points in  are conceptually listed as  in increasing order of -coordinates.
For each of these point, we write down a 0-bit if this point is also in , or a 1-bit otherwise.
These bits are concatenated and maintained as a bit vector,
such that rank/select operations can be performed in constant time~\cite{DBLP:conf/soda/ClarkM96}. \\
\indent To achieve efficient query time,
Chan et al.~\cite{DBLP:conf/compgeom/ChanLP11} formulated the following two operations as the \emph{ball-inheritance problem}:
Given a node  in  and a range  on the -axis,
 returns the range  such that
 is the first one whose -coordinate is 
and  is the last one whose -coordinate is .
Given a node  in  and an index ,
 returns the coordinates of .
The following lemma addresses their results.
\begin{lemma}[\cite{DBLP:journals/siamcomp/Chazelle88,DBLP:conf/compgeom/ChanLP11}]
    \label{lemma_ball_inheritance}
    Using  words of space,
    one can support operations  and  with  and  query time, respectively,
    where
    \begin{enumerate}
        \item  and ;
        \item  and ;
        \item  and .
    \end{enumerate}
\end{lemma}

\indent A range reporting query  over the point set  can be answered as follows.
First we find node  in , which is the lowest common ancestor of the leaf nodes that correspond to  and .
Let  and  denote the children of .
It is clear that the associated range of  contains ,
and the associated ranges of  and  both intersect .
Therefore,  is decomposed into  and .
We consider how to support  only.
We compute  using .
Thus, we need only report all the points in  whose -coordinates .
To perform this step efficiently, an index for range minimum queries over  is built.
This index returns the position of the point with the smallest -coordinate in  using constant time
and  bits of space~\cite{DBLP:journals/siamcomp/FischerH11}.
The following paragraph shows how to report  points in  time: \\
\indent Initially, we set .
We query  on the index for range minimum queries, and  is returned.
Using Lemma~\ref{lemma_ball_inheritance}, we verify if the -coordinate of  is no greater than .
We terminate if the -coordinate is greater.
Otherwise, we report  and recurse on  and . \\
\indent It is noteworthy that the first point returned by this algorithm is the lowest point in .
On the other hand, the first point returned by the other 3-sided query is the highest point in .
We cannot simply obtain the lowest or highest point in . \\
\indent The following lemma summarizes the discussions in this section.
\begin{lemma}[\cite{DBLP:conf/compgeom/ChanLP11}]
    \label{lemma_unsorted_range_reporting}
    Using  words of space,
    one can support two-dimensional range reporting queries with  query time,
    where
    \begin{enumerate}
        \item  and ;
        \item  and ;
        \item  and .
    \end{enumerate}
\end{lemma}

\subsection{Suboptimal Range Successor}

\indent Nekrich and Navarro~\cite{DBLP:conf/swat/NekrichN12} showed that
Chan et al.'s data structures~\cite{DBLP:conf/compgeom/ChanLP11} can support range successor queries after certain modifications.
For simplicity, we swap -/-coordinates and return the lowest point (i.e., the one with the smallest -coordinate)
in the query range .
Let  denote the path from the root of the conceptual range tree  to the leaf that corresponds to .
The basic idea is to find the lowest node  on  such that .
Let  denote the lowest common ancestor of the leaf nodes that correspond to  and .
It is clear that, for every node  on  that is deeper than , its associated range contains  but not .
It implies that .
One can determine if  by determining if the 3-sided query contains any point.
Therefore  can be computed using binary search on .
This requires to compute  3-sided emptiness queries. \\
\indent If  is a leaf node, then the only point in  is the answer.
If  is an internal node, the child of  on  must be the left child.
Otherwise the associated range of the left child would not intersect ,
and the assumption on  would be contracted.
Since the left child of  does not correspond to any point in the query range ,
the right child must correspond to at least a point in .
Using the algorithm described in the previous section, we can find such a point using range minimum queries.
The first point returned is by chance the lowest one. \\
\indent The following lemma summarizes the above discussions.
\begin{lemma}[\cite{DBLP:conf/swat/NekrichN12}]
    Using  words of space,
    one can support two-dimensional range successor queries with  query time,
    where
    \begin{enumerate}
        \item  and ;
        \item  and .
    \end{enumerate}
\end{lemma}

One can support sorted range reporting queries using range successor queries.
Suppose the lowest point in  has -coordinate .
We can find the second lowest point in  by querying .
The procedure is repeated until the query range contains no point.
Thus we have the following lemma:
\begin{lemma}[\cite{DBLP:conf/swat/NekrichN12}]
    Using  words of space,
    one can support two-dimensional sorted range reporting queries with  query time,
    where
    \begin{enumerate}
        \item  and ;
        \item  and .
    \end{enumerate}
\end{lemma}

\section{Optimal Range Successor Queries in Less Space}
\label{section_main}

We present the main result in this paper:
a data structure for range successor queries using  words of space and  query time.
This data structure is obtained by modifying Nekrich and Navarro's third data structure~\cite{DBLP:conf/swat/NekrichN12} for sorted range reporting.
We consider 3-sided range successor queries,
for which the leftmost point in  is returned.
A preliminary result of Nekrich and Navarro is addressed in the following lemma.
\begin{lemma}[Lemma 5 in \cite{DBLP:conf/swat/NekrichN12}]
    \label{lemma_three_sided}
    Given a set of  points, one can support 3-sided range successor queries
    using  bits of space and  query time.
\end{lemma}

\indent Now we describe our data structures.
We first build the same data structures as the second variant of Lemma~\ref{lemma_unsorted_range_reporting}.
Let  denote a 3-sided query range.
We further construct auxiliary data structures on every  such that
the leftmost point in  can be returned in  time, using  bits of additional space. \\
\indent As we have mentioned, points in  are conceptually listed in increasing order of -coordinates.
These points are divided into blocks  of size  (the last block may contain less).
 stores the lowest point in each block explicitly,
and is maintained using Lemma~\ref{lemma_three_sided} such that 3-sided range successor queries on  can be supported in  time.
This auxiliary data structure occupies  bits of space. \\
\indent For some constant , points in every block  are further divided into sub-blocks
 of size  (the last sub-block may contain less).
In addition, their /-coordinates are rewritten as the /-ranks within this block.
A rank can be represented in  bits, and all the ranks require  bits over all blocks.
 stores the lowest point in every sub-block explicitly,
and is also maintained using Lemma~\ref{lemma_three_sided} to support 3-sided range successor queries on  in  time.
This auxiliary data structure requires only 
bits of additional space.
Thus the space cost of all 's is  bits. \\
\indent Now we show how to answer the 3-sided range successor query  over .
First we compute  using , which requires  time.
Let  and  be the blocks that contains  and , respectively.
Thus  spans over blocks .
We first attempt to find the leftmost point in  (we will show how to do it later).
If such a point exists, our algorithm terminates and the point is returned.
Otherwise, we query  to find the leftmost block among  that intersect .
Let  denote the block.
We query  and returns the result.
If such  does not exist, we query . \\
\indent The remaining issue is to find the leftmost point in  that is contained in a given 3-sided range  efficiently.
We first compute the ranks of ,  and  within this block.
Let them be ,  and , respectively.
 and  can be computed directly from  and .
The computation of  requires succinct indices for predecessor search~\cite{DBLP:conf/stacs/GrossiORR09}, using  time.
Similar to the procedure described in the previous paragraph,
we find the leftmost point in  in  time.
The only difference is that we find the leftmost point within a sub-block using table-lookup.
This can be done using constant time and a global lookup table of  bits of additional space,
since there are only  different sub-blocks,
for some positive constant . \\
\indent Summarizing the discussion above, we can find the leftmost point in  in  time.
We also construct auxiliary data structures on every  for 3-sided range successor queries
of the form .
Combining both parts and following the approach in Section~\ref{subsection_unsorted},
we can find the leftmost point in a 4-sided query range in  time.
\begin{theorem}
    Using  words of space, one can support two-dimensional range successor queries in  time.
    Repeatedly using this data structure, one can support two-dimensional sorted range reporting queries in  time,
    where  is the size of output.
\end{theorem}

\begin{comment}
\section{Making it Adaptive}
\label{section_adaptive}

We can make the data structure presented in Section~\ref{section_main} adaptive at almost no cost.
We build the adaptive range counting data structure of Chan and Wilkinson~\cite{DBLP:conf/soda/ChanW13}.
This data structure supports range counting queries using  words of space and  query time,
where  is the number of points in the query range.
The key idea in their work is \emph{shallow cuttings}~\cite{DBLP:journals/comgeo/Matousek92}.
A -shallow cutting consists of  cells, each containing a subset of  points .

\cite{DBLP:journals/siamcomp/Chan00}
\end{comment}

\bibliography{ref}

\end{document}
