\documentclass{article}
\usepackage{amsthm}
\usepackage{amssymb}
\usepackage{epsfig}
\usepackage[dvips]{color}
\newcommand{\RR}{{\cal R}}
\newcommand{\R}{\mathrm{R}}
\newcommand{\QQ}{{\cal Q}}
\newcommand{\inter}{\mbox{Int}}
\newcommand{\C}{\mbox{Cyl}}
\newcommand{\claim}[2]{}


\newcount\claimno
\claimno=0
\def\mytextindent#1{\indent\indent\llap{\rm#1\enspace}\ignorespaces}
\def\myitem{\par\hangindent30pt\mytextindent}
\def\myclaim#1#2{
   \global\advance\claimno by 1\relax
   \bigskip\noindent\rlap{\rm(\the\claimno)}\ignorespaces
   \global\expandafter\edef\csname CLAIMLABEL#1\endcsname{(\the\claimno)}\relax
\hangindent=33pt\hskip30pt{\sl#2}\bigskip}
\def\refclaim#1{\csname CLAIMLABEL#1\endcsname}
\def\endclaim{\medskip}
\def\junk#1{}
\let\ppar=\par
\def\rt#1{#1}

\def\mylabel#1{{\label{#1}}}
\newtheorem{theorem}{Theorem}
\newtheorem{corollary}[theorem]{Corollary}
\newtheorem{lemma}[theorem]{Lemma}
\begin{document}
\title{Three-coloring triangle-free graphs on surfaces I. 
       Extending a coloring to a disk with one triangle}
\author{Zden\v{e}k Dvo\v{r}\'ak\thanks{Computer Science Institute,
           Charles University, 
           Malostransk{\'e} n{\'a}m{\v e}st{\'\i} 25, 118 00 Prague, 
           Czech Republic. E-mail: {\tt rakdver@iuuk.mff.cuni.cz}.
           Supported by the Center of Excellence -- Inst. for Theor. Comp. Sci., Prague (project P202/12/G061 of Czech Science Foundation).}
 \and
     Daniel Kr{\'a}l'\thanks{Mathematics Institute, DIMAP and Department of Computer Science, University of Warwick, Coventry CV4 7AL. E-mail: {\tt D.Kral@warwick.ac.uk}.}
 \and
        Robin Thomas\thanks{School of Mathematics, 
        Georgia Institute of Technology, Atlanta, GA 30332. 
        E-mail: {\tt thomas@math.gatech.edu}.
        Partially supported by NSF Grant No.~DMS-0701077.}
}
\date{March 4, 2016}
\maketitle
\begin{abstract}
Let  be a plane graph with exactly one triangle  and all 
other cycles of length at least , and let  be a facial cycle of 
of length at most six.
We prove that a -coloring of  does not extend to a -coloring of  
if and only if  has length exactly six and there is a color  such that
either  has an edge joining two vertices of  colored ,
or  is disjoint from  and every vertex of  is adjacent to a vertex
of  colored .
This is a lemma to be used in a future paper of this series.
\end{abstract}

\section{Introduction}


This is the first paper in a series aimed at studying the -colorability
of graphs on a fixed surface that are either triangle-free, or have their
triangles restricted in some way.
All graphs in this paper are simple, with no loops or parallel edges.

The subject of coloring graphs on surfaces goes back to 1890 and the work
of Heawood~\cite{Heawood}, who proved that if  is not the
sphere, then every graph in 
is -colorable as long as
.
Here and later  is the {\em Euler genus of },
defined as  when , the orientable surface
of genus , and  when , the non-orientable surface
with  cross-caps.
Incidentally, the assertion holds for the sphere as well, by the
Four-Color Theorem~\cite{AppHak1,AppHakKoc,AppHak89,RobSanSeyTho4CT}.
Ringel and Youngs (see~\cite{Ringel}) proved that the bound is
best possible for all surfaces except the Klein bottle, for which
the correct bound is .
Dirac~\cite{Dirmap} and Albertson and Hutchinson~\cite{AlbHut}
improved Heawood's result by showing that
every graph in  is actually -colorable,
unless it has a subgraph isomorphic to the complete graph
on  vertices.

For triangle-free graphs there does not seem to be a similarly nice formula,
but Gimbel and Thomassen~\cite{gimbel} gave very good bounds: they
proved that the maximum chromatic number of a triangle-free graph drawn
in a surface of Euler genus  is at least 
and at most  for some absolute constants 
 and .

In this series we adopt a more modern approach to coloring graphs on surfaces,
following the seminal work of 
Thomassen~\cite{Tho5torus,ThoCritical,thom-surf}.
The basic premise is that while Heawood's formula is best possible
for all surfaces except the Klein bottle, only relatively few graphs attain
the bound or even come close.
To make this assertion more precise let us recall that a graph  is
called {\em -critical}, where  is an integer, if every proper
subgraph of  is -colorable, but  itself is not.
It follows easily from Euler's formula that if  is a fixed surface,
and  is a sufficiently big graph drawn in , then  has 
a vertex of degree at most six. 
It follows that for every  the graph  is not -critical,
and hence there
are only finitely many -critical graphs that can be drawn in .
It is not too hard to extend this result to .
In fact, it can be extended to  by the following deep theorem
of Thomassen~\cite{ThoCritical}.

\begin{theorem}
\mylabel{thm:thomcrit}
For every surface  there are only finitely many -critical 
graphs that can be drawn in .
\end{theorem}

\noindent
The lists of -critical graphs are explicitly known for the
projective plane~\cite{AlbHut}, the torus~\cite{Tho5torus} and the 
Klein bottle~\cite{ChePosStrThoYer,KawKraKynLid}.
An immediate consequence is that for every surface  there is
a polynomial-time (in fact, linear-time) algorithm to test whether
an input graph drawn in  is -colorable.
Theorem~\ref{thm:thomcrit} does not hold for -critical graphs,
because of an elegant construction of Fisk~\cite{Fis}.
For -colorability an algorithm as above does not exist, unless ,
because testing -colorability is NP-hard even for planar 
graphs~\cite{GarJoh}.
It is an open problem whether there is a polynomial-time algorithm for testing
-colorability of graphs in  when  is a fixed
surface other than \rt{the} sphere.
The techniques currently available do not give much hope for a positive
resolution in the near future. 

How about triangle-free graphs?
Similarly as above, if  is a sufficiently large triangle-free graph
in a fixed surface , then  has a vertex of degree at most four.
Thus  is not -critical, and the argument can be strengthened to show
that  is not -critical.
Thus for a fixed integer  testing -colorability of triangle-free graphs
drawn in a fixed surface can be done in linear time, as before.
That brings us to testing -colorability of triangle-free graphs
in a fixed surface, the subject of this series of papers.
The question has been raised by
Gimbel and Thomassen~\cite{gimbel} and we resolve it later
in this series, after we develop some necessary theory.

Historically the first result in this direction is the following
classical theorem of Gr\"otzsch~\cite{grotzsch}.

\begin{theorem}
\label{grotzsch}
Every triangle-free planar graph is -colorable.
\end{theorem}

Thomassen \cite{thom-torus,Tho3list,ThoShortlist}
found three reasonably simple proofs, and extended Theorem~\ref{grotzsch}
to other surfaces.  
Recently, two of us, in joint work with Kawarabayashi~\cite{DvoKawTho}
were able to design a linear-time algorithm to -color triangle-free
planar graphs, and as a by-product found perhaps a yet simpler proof
of Theorem~\ref{grotzsch}.
The statement of Theorem~\ref{grotzsch}
cannot be extended to any surface other than the sphere.
In fact, for every non-planar surface  there are infinitely many
-critical triangle-free graphs that can be drawn in .
For instance, the graphs obtained from an odd cycle of length five or more
by applying Mycielski's
construction \cite[Section~8.5]{BonMur} have that property.
Thus an algorithm for testing -colorability of triangle-free graphs
on a fixed surface will have to involve more than just testing the
presence of finitely many obstructions.

The situation is different for graphs of girth at least five
by another deep theorem of Thomassen~\cite{thom-surf}.

\begin{theorem}
\mylabel{thm:thomgirth5}
For every surface  there are only finitely many -critical
graphs of girth at least five that can be drawn in .
\end{theorem}

Thus the -colorability problem on a fixed surface
has a polynomial-time algorithm
for graphs of girth at least five, but the presence of cycles of
length four complicates matters.
Let us remark that there are no -critical graphs of girth at least five
on the projective plane and the torus~\cite{thom-torus} and
on the Klein bottle~\cite{thomwalls}.

The only non-planar surface for which the -colorability problem
for triangle-free graphs is fully characterized is the projective plane.
Building on earlier work of Youngs~\cite{Youngs}, Gimbel and
Thomassen~\cite{gimbel} obtained the following elegant characterization.
A graph drawn in a surface is a {\em quadrangulation} if every face
is bounded by a cycle of length four.

\begin{theorem}
\mylabel{thm:gimtho}
A triangle-free graph drawn in the projective plane is -colorable if and only
if it has no subgraph isomorphic to a non-bipartite
quadrangulation of the projective plane.
\end{theorem}

For other surfaces there does not seem to be a similarly nice characterization,
but in a later paper of this series we will present a polynomial-time
algorithm to decide whether a triangle-free graph in a fixed surface
is -colorable.
The algorithms naturally breaks into two steps.
The first is when the graph is
a quadrangulation, except perhaps for a bounded number of larger faces
of bounded size, which will be allowed to be precolored.
In this case there is a simple topological obstruction to the existence
of a coloring extension based on the so-called ``winding number" of
the precoloring.
Conversely, if the obstruction is not present and the graph is highly
``locally planar", then we can show that the precoloring can be
extended to a -coloring of the entire graph.
This can be exploited to design a polynomial-time algorithm.
With additional effort the algorithm can be made to run in linear time.

The second step covers the remaining case, when the graph has either many faces
of size at least five, or one large face, and the same holds for every
subgraph.
In that case we show that the graph is -colorable.
That is a consequence of the following theorem, which will form the
cornerstone of this series.

\begin{theorem}
\mylabel{thm:corner}
There exists an absolute constant  with the following property.
Let  be a  graph drawn in a surface  of Euler genus 
with no non-contractible cycles of length at most four,  and let  be
the number of triangles in .
If  is -critical, 
then ,
where the summation is over all faces  of  of length at least five.
\end{theorem}

If  has girth at least five, then  and every face has length
at least five.
Thus Theorem~\ref{thm:corner} implies Theorem~\ref{thm:thomgirth5},
and, in fact, improves the bound given by the proof of 
Theorem~\ref{thm:thomgirth5} in~\cite{thom-surf}.
The fact that our bound in Theorem~\ref{thm:corner} is linear in the
number of triangles is needed in our solution of a problem of 
Havel~\cite{conj-havel}, as follows.

\begin{theorem}
\mylabel{havel}
There exists an absolute constant  such that if  is a planar
graph and every two distinct triangles in  are at distance at least ,
then  is -colorable.
\end{theorem}

To prove Theorem~\ref{thm:corner} we actually prove a stronger result.
If  is a graph and  is a subgraph of , then we say that 
is {\em -critical} if  and for every proper subgraph  of
 that includes  there exists a -coloring of  that extends to
a -coloring of , but does not extend to a -coloring of .
The stronger version of Theorem~\ref{thm:corner} applies to -critical
graphs, where  has bounded size.
A special case that we need in order to carry out our inductive argument
is the following.

\begin{theorem}
\mylabel{thm:speccase}
There exists an absolute constant  with the following property.
Let  be a planar graph with two distinct facial cycles  and ,
where  has length six and  has length four.
Assume that every cycle in  of length at most four separates  and 
, and that every cycle in  other than  of length exactly four is disjoint
from .
If  is -critical, then it has at most  vertices.
\end{theorem}

The machinery we develop in the second paper~\cite{proof-druhy} 
of this series can be
applied to prove Theorem~\ref{thm:speccase}, with one notable
exception: when  is disconnected, and the component  of 
that contains  has a face bounded by a triangle
(in which case this face includes the component of  containing ).
We need to show that the size of  is bounded, but we have not been able to do
so using the same methods we use for the rest of the proof of 
Theorem~\ref{thm:corner}.
Instead, we use a different method, which allows us to characterize
the components .
Since the proof method is different, we separate it from the other
arguments and present it in this paper.
Thus our main result is as follows.
The two outcomes are illustrated in Figure~\ref{fig-6plus3}.

\begin{figure}
\begin{center}\epsfbox{bunch1.eps}\end{center}
\caption{Critical graphs with a precolored -cycle and one triangle.}
\label{fig-6plus3}
\end{figure}

\begin{theorem}
\mylabel{thm-6plus3}
Let  be a plane graph with a facial cycle  of length at most six,
let  be a triangle in , and assume that
every cycle in  other than  and  has length at least five.
Let  be a -coloring of  that does not extend to a -coloring
of . Then  has length exactly six and either
\myitem{(a)} for two distinct vertices  that are
adjacent in , or
\myitem{(b)} for three pairwise distinct
vertices , where each  is adjacent to a different
vertex of .
\end{theorem}

Finally, let us mention a related interesting conjecture due to
Steinberg~\cite{conj-stein}, who conjectured that every planar graph 
without - and -cycles is -colorable.
This conjecture is still open. 
Currently the best result of 
Borodin, Glebov, Montassier and Raspaud~\cite{BorGleMonRas}
shows that excluding cycles 
of lengths ,  and  suffices to guarantee -colorability. 
(A proof of the same result by Xu~\cite{xu} is refuted in~\cite{BorGleMonRas}.)

\section{Auxiliary results}


In this short section we present several results that will be needed later.
Let  be a graph, and let  be a subgraph of .
\rt{Let us recall} that  is {\em -critical} if  and 
for every proper subgraph  of  that includes  as a subgraph
there exists a -coloring of  that extends to a -coloring of ,
but not to one of .
Theorem~\ref{grotzsch} admits the following strengthening.

\begin{theorem}\label{grotzsch5cycle}
There is no -critical triangle-free plane graph , where 
 is a cycle in  of length at most five.
\end{theorem}

This result was later strengthened in several ways.  
Gimbel and Thomassen~\cite{gimbel} extended this to cycles of length six:

\begin{theorem}\label{thm-gimbel}
Let  be a plane triangle-free graph with a facial cycle  of length six. 
If  is -critical, then all faces of  distinct from  have length four.
\end{theorem}

For graphs of girth at least five the above results can be strengthened, 
as shown by Thomassen~\cite{thom-surf} and Walls~\cite{WalPhD}.

\junk{
More can be said about the critical graphs of girth .  
shows the following characterization\footnote{Theorem~\ref{thm-planechar}
was originally formulated in \cite{thom-surf} in a slightly different
setting:  the described graphs have the property that there exists a precoloring of  that does
not extend to , but extends to every proper subgraph of  that contains .  However,
every -critical graph  has a subgraph  with this property.  Also, every face of  of
length at most  is also a face of  (see Lemma~\ref{lemma-crsub} below).
We conclude that  is one of the graphs described in Theorem~\ref{thm-planechar}.}
of critical graphs with precolored face of length at most :
}

\begin{theorem}
\mylabel{thm-planechar}
Let  be a plane graph of girth at least five,
and let  be a facial cycle in  of length  .
If  is -critical, then 
\begin{itemize}
\item[(a)] ,  and  is not induced, or
\item[(b)] ,  is a tree with at most  vertices, 
           and every vertex of  has degree three in , or
\item[(c)]  and  is a connected graph with at most  vertices
           containing exactly one cycle, and the length of this cycle is five.
           Furthermore, every vertex of  has degree three in .
\end{itemize}
\end{theorem}

\noindent
{\bf Proof.}
Since  is -critical, there exists a -coloring of  that
does not extend to a -coloring of .
Theorem~2.5 of \cite{thom-surf} (as well as independently proved Theorem~3.0.2 of \cite{WalPhD})
states that there exists a subgraph  of 
such that  is a subgraph of  and  satisfies one of (a)--(c).
We claim that . Indeed, otherwise the -criticality of 
implies that there exists a -coloring of  that does not extend to
a -coloring of , but extends to a -coloring  of .
By applying~\cite[Theorem~2.5]{thom-surf} or~\cite[Theorem~3.0.2]{WalPhD}  
to every face of 
we deduce that  extends to every face of , and hence extends
to a -coloring of , a contradiction.~\qed
\bigskip

We will need a version of Theorem~\ref{grotzsch5cycle} that allows
the existence of a triangle.
Such a result was attempted by Gr\"unbaum~\cite{grunbaum}, but his proof
is not correct.
A correct proof was found by Aksionov~\cite{aksenov}.  
The result of Aksionov together with Theorem~\ref{thm-gimbel}
gives the following characterization of critical graphs with a triangle 
and a precolored face of length at most five.

\begin{theorem}\label{thm-planetr}
Let  be a plane graph with a facial cycle  of length at most five and at 
most one triangle  distinct from .
If  is -critical, then  has length exactly five, 
 shares at least one edge
with  and all faces of  distinct from  and  have length exactly four.
\end{theorem}

\section{Graphs with one triangle}\label{sec-6plus3}


To prove Theorem~\ref{thm-6plus3} we prove, for the sake of the
inductive argument, the following slightly more general result.
Theorem~\ref{thm-6plus3} will be an immediate corollary.

\begin{theorem}
\mylabel{thm-6plus3new}
Let  be a plane graph with outer cycle  of length at most six
and assume that
\myitem{()}there exists a point  in the plane such that for every cycle  in  
   of length at most four, the open disk bounded by  contains .\ppar
\noindent
If  is -critical, then 
 has length exactly six
and  is isomorphic to one of the graphs depicted in Figure~\ref{fig-6plus3}.
\end{theorem}

\noindent
{\bf Proof}.
Let  be as stated, and suppose for a contradiction that it 
is not isomorphic to either of the two graphs depicted in
Figure~\ref{fig-6plus3}.
By Theorems~\ref{grotzsch5cycle} and~\ref{thm-gimbel} 
the graph  has a triangle .
We may assume that  is minimal in the sense that the theorem holds
for every graph with fewer vertices.
A vertex  will be called {\em internal}. 
The -criticality of  implies that

\myclaim{Z}{every internal vertex of  has degree at least three.}

If  is a cycle in , then by ins we denote the subgraph of 
consisting of all vertices and edges drawn in the closed disk bounded by .
More generally, suppose that  is a closed walk in  such that the
subgraph of  consisting of the vertices and edges of  has a
bounded face  incident with all the edges of .  It follows that
 is unique and homeomorphic to an open disk.  In this case, we say that 
\emph{bounds the open disk } and we define ins to be the subgraph of 
consisting of all vertices and edges drawn in the closure of .



\myclaim{Y}{\rt{Let  be a closed walk in  that bounds an open disk
    , and assume that  includes at least one vertex or
    edge of . Then}
    {\rm ins} is a -critical graph.}

To prove \refclaim{Y} let  be as stated, let 
 be obtained from  by deleting every vertex and edge of 
drawn in , and
let  be a proper subgraph of ins that includes .
Then some -coloring of  extends to a -coloring  of
, but does not extend to a -coloring of .
It follows that the restriction of  to  extends to ,
but not to ins, as desired.
This proves~\refclaim{Y}.  
\medskip

It follows from (), \refclaim{Y} and Theorem~\ref{thm-planetr} that

\myclaim{B}{ bounds a face and all other cycles in  have length at least five.}

Consequently,  is connected, as otherwise it would contain a component
disjoint from  and this component would be -colorable by Theorem~\ref{thm-planetr},
contrary to the assumption that  is -critical.
Next we constrain cycles in  of length at most seven:

\myclaim{A}{ Let  be a cycle in  of length at most seven that does 
    not bound a face. Then  has length at least six, and the closed
    disk bounded by  includes .}

To prove \refclaim{A} let  be as stated.
By the minimality of  and~\refclaim{Y} we deduce that  has length
at least six.
If  is not contained in the closed disk  bounded by , then 
\refclaim{B} implies that ins has girth at least five,
contrary to~\refclaim{Y} and Theorem~\ref{thm-planechar}.
Thus  is contained in , and~\refclaim{A} follows.
\endclaim

The same argument and the minimality of  imply the following  claim.




\myclaim{A6}{ Let  be a cycle in  of length six that does
    not bound a face. Then {\rm ins} is isomorphic to one of the
    graphs depicted in Figure~\ref{fig-6plus3}.}


\myclaim{AA}{Let  be a closed walk in  of length 
  bounding an open disk  disjoint from , and let
  . 
If , then  satisfies the conclusion of Theorem~\ref{thm-planechar}.}

To prove~\refclaim{AA} we modify , converting  to a cycle,
and then apply Theorem~\ref{thm-planechar}.
First we replace every edge of  that is traversed by  twice
by a pair of parallel edges bounding a face of length two, thus
creating a multigraph  and a closed walk  in 
such that  uses no edge more than once.
Thus every vertex of  is incident with an even number of edges of .
Let  be a vertex of  incident with  edges of .
Then the edges and faces of  incident with  can be numbered
 in the clockwise cyclic
order of appearance around  in such a way
that for  the edges  are consecutive
in  and .
We split  into  vertices  as follows.
Let , and let 
be the edges of  incident with  listed in clockwise cyclic order
around . Then the edges  will
be incident with  in the new graph in the clockwise order listed.
By repeating this construction for every vertex of  incident with
at least four edges of  we arrive at a plane graph  such
that the walk  corresponding to  is a cycle.
By~\refclaim{Y} the graph  is -critical, and since the above construction
preserves criticality we deduce that  is -critical.
Thus~\refclaim{AA} follows from Theorem~\ref{thm-planechar}.
\endclaim







From \refclaim{B} and Theorem~\ref{thm-planetr} it follows that

\myclaim{D}{  has length six,}

\noindent
and since every cycle of length at most five bounds a face by~\refclaim{A}
we deduce that

\myclaim{DD}{ the graph  has no subgraph  with outer face  
    such that  is isomorphic to either of the two
    graphs depicted in Figure~\ref{fig-6plus3};
    in particular,  is induced.}

Next we claim that

\myclaim{E}{ every internal vertex has at most one neighbor in .}

To prove~\refclaim{E} suppose for a contradiction that
an internal vertex  has two neighbors .
Let  denote the path , and let  be the three
cycles of .
By~\refclaim{B} either one of  is  and
the other has length seven, or  both have length five.
In either case it follows from~\refclaim{A} that  both bound
faces of , and hence  has degree two, contrary to~\refclaim{Z}.
This proves~\refclaim{E}.
\endclaim

\myclaim{F}{ The cycle  is disjoint from .}

To prove~\refclaim{F} suppose for a contradiction that .
By~\refclaim{DD} and~\refclaim{E}  is the only vertex of .
The graph  has a face bounded by a walk  of length nine.
By~\refclaim{AA} combined with \refclaim{B}, at least one of the vertices of
 has degree two, which contradicts~\refclaim{Z}.
This proves~\refclaim{F}.
\endclaim

Let us fix an orientation of the plane, and
let  and  be numbered in clockwise cyclic 
order according to the drawing of .

\myclaim{cl-onetr}{  has at most one edge joining  to .}

To prove~\refclaim{cl-onetr}
suppose that say  for some .
By~\refclaim{B} we have .
Let .  As  has degree at least three,
 does not bound a face; thus  has length at least eight 
by~\refclaim{A}, and we conclude that .
Thus  has length exactly eight, and hence by~\refclaim{AA}
ins
consists of  and at most one chord.  
Since  has degree
at least three, this chord exists and joins  with , and 
hence  has a subgraph isomorphic to the graph depicted in
Figure~\ref{fig-6plus3}(b), contrary to~\refclaim{DD}.
This proves~\refclaim{cl-onetr}.
\endclaim


\myclaim{cl-no53}{  does not contain a -face incident only with 
     internal vertices of degree three.}

To prove~\refclaim{cl-no53} suppose
for a contradiction that  contains such a -face .
For , let  be the neighbor of  not belonging to  
(each  has such a neighbor,
because  \rt{bounds a face by~\refclaim{B} and  bounds a face by definition}
and each vertex of  has degree three).  
Since  is disjoint from  by~\refclaim{F} and
 contains no -cycles by~\refclaim{B}, 
it follows that at most three of the vertices 
 belong to .  
Without loss of generality we may assume that  is internal.
Note also that , as  does not contain a -cycle.
By the symmetry between  and  we may assume that if
 is adjacent to a vertex of , then so is .
Let  be the graph obtained from  by adding the edge 
drawn in the same way as the path  in .
Observe that every -coloring of  extends to a -coloring of : 
given a -coloring of , every vertex in  has a list of two available
colors, and the lists of  and  are different.  




Our next objective is to show that  satisfies ().
To that end let  be a cycle in  of length at most four.
Then  includes the edge  by~\refclaim{B}.
Consider the cycle  in  obtained from  by replacing the edge
 by the path .
Note that  has length at most seven, and that it does not bound a face 
\rt{of }
(since the edges  and  are drawn on the opposite sides of ).
Thus by~\refclaim{A}  is a subgraph of ins, and since the edge  of 
is drawn in the same way as the path  of , it follows that the point 
is contained in the open disk bounded by .
We conclude that  satisfies (), as desired.

\begin{figure}
\begin{center}
\epsfbox{disk1.eps}
\end{center}
\caption{A configuration obtained in the proof of \refclaim{cl-no53}.}
\label{fig-12}
\end{figure}

Let  be a minimal subgraph of  such that  is a subgraph
of  and 
every -coloring of  that extends to a -coloring of  extends
to a -coloring of .  
Then , for otherwise every -coloring of  extends to
a -coloring of , and hence to a -coloring of , 
and therefore to one of , contrary to the -criticality of .
We conclude that  is -critical.
The minimality of  implies that  is isomorphic to one of the
graphs depicted in Figure~\ref{fig-6plus3}.
But  is an induced subgraph of  by~\refclaim{DD} 
and  is internal, and hence 
is isomorphic to the graph of Figure~\ref{fig-6plus3}(b).
Let  be the triangle of . 
By \refclaim{cl-onetr} we have , 
and hence  is an edge of .
Let  be the third vertex of .
We may assume that  is adjacent to ,  is adjacent to 
and  is adjacent to , where the adjacencies take place in
 and .
Let  be the face boundary of the -face of  incident with the edge 
, and let  be the -cycle of  obtained
from  by replacing the edge  by the path  (see Figure~\ref{fig-12}).  
Let  be the -cycle in  obtained from  by
replacing the edge  by the path .  
By~\refclaim{A}  lies in the closed disk bounded by , and 
since  is adjacent to  it follows that
ins includes no cycle of length at most four.  
By~\refclaim{AA}
no vertex of  lies in the open disk bounded by , 
and hence  and  lie in the open disk bounded by .
Since  has no -cycles we deduce that ,
and , for otherwise  and the cycle
 includes the edge  in its inside but not ,
contrary to~\refclaim{A}.
Since  is adjacent to , the choice of  implies that
 is adjacent to a vertex of , contrary to the planarity of .
This proves~\refclaim{cl-no53}.
\endclaim







\myclaim{cl-disttr}{ The distance between  and  is at least two.}

To prove~\refclaim{cl-disttr} 
suppose for a contradiction that the distance between  and  is at most one.
Then it is exactly one by~\refclaim{F}, and so we may assume that 
say .
\rt{Let  denote the closed walk .
Then  bounds an open disk. Let .
Since  is -critical, we see that .
By~\refclaim{AA} the graph  satisfies (a), (b) or (c) of
Theorem~\ref{thm-planechar}.}
If it satisfies (a), then by~\refclaim{Z},  has a neighbor in ,
contrary to~\refclaim{cl-onetr}.
If  satisfies (b), then
 is a tree  with at most three vertices,
each of degree three. Both  and 
have a neighbor in , and hence  includes the
vertex-set of a -cycle, contrary to~\refclaim{cl-no53}.
Finally,  cannot satisfy (c), because the cycle referenced
in~(c) would contradict~\refclaim{cl-no53}.
This proves~\refclaim{cl-disttr}.

\myclaim{cl-no22r}{ No two vertices of degree two are adjacent in }.

To prove~\refclaim{cl-no22r}
suppose for a contradiction that  has two adjacent vertices of degree two.
By~\refclaim{Z} they belong to , and so we may assume that 
say  and  have degree two.  
The edge  is not contained in any -cycle,
as otherwise  would have a chord or an internal vertex would have two 
neighbors in , contrary to~\refclaim{DD}
and~\refclaim{E}.
Let  be the graph obtained from  by contracting the edge , 
and let  be the corresponding outer cycle of .
Then  has no cycle of length at most  distinct from .
Furthermore, every -coloring  of  can be modified to a 
-coloring  of  such that  for ,
and  extends to  if and only if  extends to a -coloring of . 
It follows that  is -critical, contrary to Theorem~\ref{thm-planetr}.  
This proves~\refclaim{cl-no22r}.
\endclaim

\myclaim{cl-no3chord}{ For every path  with  
  and  internal and 
  there exists  such that  bounds a -face.}

To prove~\refclaim{cl-no3chord}
consider a path  with  and  internal 
and ,
and let  and  be the cycles of   other than  such that 
lies in the closed disk bounded by .  
Since  does not bound a face, it has length at least six by~\refclaim{A}.
Thus  has length at most six, and hence bounds a face by~\refclaim{A},
and therefore has length five by~\refclaim{cl-no22r}.
This proves~\refclaim{cl-no3chord}.
\endclaim

\myclaim{cl-faces}{ All faces of  distinct from  and  have 
        length exactly five.} 

To prove~\refclaim{cl-faces}
consider a face  of length  in .  
By~\refclaim{E} we may assume without loss of generality that
 and  are internal.  Furthermore, if , then not
all of , ,  and  may belong to , by \refclaim{cl-no22r},
and hence, by symmetry, we may assume that either  or  is internal.
Let  if  and  if .
Let  be the graph obtained from  by identifying the vertices of  
to a new vertex 
and deleting all resulting parallel edges (the drawing of the new vertex  is placed inside the
face  of  and the drawings of the edges incident with the vertices of  in  are first
shifted infinitesimally so that they do not intersect and then joined to  through ). 
Thus .
By~\refclaim{Z} and~\refclaim{A} the vertices of  are pairwise non-adjacent;
thus the identifications created no loops.
Observe that every -coloring  of  gives rise to a -coloring
of  (color the vertices of  using ).
It follows that some -coloring of  does not extend to a -coloring of .
Let  be a minimal subgraph of  such that  is a subgraph of 
and every -coloring of  that extends to a -coloring of 
also extends to a -coloring of ;
then  is -critical.

Next we show that  satisfies ().
As a first step we prove that  does not have a triangle other than .
To that end let  be a triangle in .
Recall that .
Two of the edges of  are incident in  with distinct vertices 
.
Let  be the corresponding -cycle in , obtained from  by replacing  with
the two-edge path between  and  with internal vertex in
.
Observe that  does not bound a face in , contrary to~\refclaim{A}.
Therefore,  does not have a triangle distinct from .  
Consider now a -cycle  in .
The corresponding cycle  in  (constructed in the same way as )
has length six.  
As  does not bound a face we can apply~\refclaim{A6} to the cycle ;
it follows that  is contained in the closed disk bounded by .  Note that the point
 is contained in the open disk bounded by  and since  does not lie inside ,
the choice of the drawing of  ensures that  is contained in the open disk bounded by .
It follows that  satisfies ().

Since  has fewer vertices than ,
 is one of the graphs drawn in Figure~\ref{fig-6plus3}.
Furthermore, the first result of the previous paragraph implies
that  is the unique triangle of .
However, this implies that the distance between  and  in
 is at most one, contradicting \refclaim{cl-disttr}.
This proves~\refclaim{cl-faces}.
\endclaim

\myclaim{cl-onet4}{At least one vertex of  has degree at least four.}

To prove~\refclaim{cl-onet4} suppose for a contradiction
that all vertices of  have degree at most three.
By~\refclaim{Z} and \refclaim{F} they have degree exactly three.
For , let  be the neighbor
of  that does not belong to .  As  is the only cycle of length
at most four in  \rt{by~\refclaim{B}}, 
these vertices are distinct and pairwise non-adjacent,
and by \refclaim{cl-disttr}, they are internal.

Suppose first that each of ,  and  has a neighbor in .
Let  be the subgraph of  
\rt{consisting of  and the six edges joining  to
 and ,}
and let ,  and  be the cycles bounding the faces of  distinct from  and .
Note that  and 
.
At most one of ,  and  bounds a face of , as otherwise ,  or  would be an internal vertex
of degree two, contrary to~\refclaim{Z}.
\rt{From the symmetry we may assume that  and  do not bound a face.}
From~\refclaim{A} \rt{it follows} that .
This implies that  and .  
However, \refclaim{AA} implies that  has a chord, 
contradicting~\refclaim{DD}, \refclaim{E} \rt{or~\refclaim{cl-disttr}}.

Therefore, by symmetry we assume that  has no neighbor in .
Let  be the graph obtained from  by adding the edge , where the edge is drawn along the
path  of .
\rt{Since  is -critical, there exists a -coloring  of 
that does not extend to a -coloring of .}
Note that every coloring of  extends to a coloring of , since each vertex of  has a list of two available colors
and the lists of  and  are different.  
Thus  does not extend to  a -coloring of .
Let  be a smallest subgraph of  such that  is a subgraph of
 and  does not extend to .
Then  is -critical.

Consider a cycle  in  of length at most four.  Note that
 is an edge of , and let  be the cycle obtained from  by
replacing this edge by the path .  The cycle  has length six or seven,
and thus by~\refclaim{AA} and \refclaim{cl-faces},  is contained in the closed disk bounded by .
By the choice of the drawing of the edge , we conclude that the point  is contained in the
open disk bounded by .  Therefore,  satisfies ().

As  has fewer vertices than ,  is isomorphic to one of the graphs in Figure~\ref{fig-6plus3}.
As  is not a subgraph of , the triangle of  contains the edge , and thus 
is at distance at most one from  both in  and in .  This contradicts the choice of ,
finishing the proof of~\refclaim{cl-onet4}.
\endclaim



\myclaim{cl-excl6}{If  is a -cycle of  distinct from , then the open disk bounded by 
contains no vertices and .}

\rt{To prove~\refclaim{cl-excl6} let  be as stated.
By~\refclaim{cl-faces} it does not bound a face, and by~\refclaim{A6} ins
is isomorphic to one of the graphs depicted in Figure~\ref{fig-6plus3}.
By~\refclaim{cl-onet4} it is not isomorphic to the second of those two
graphs, and hence~\refclaim{cl-excl6} follows.}
\endclaim

\myclaim{cl-5faces}{ If  is a -face incident with four internal vertices of 
  degree three, then \rt{some edge of  is incident with }.}

To prove~\refclaim{cl-5faces}
suppose for a contradiction that  contains a -face , 
where , ,  and  are internal vertices of degree three,
and that \rt{no edge of  is incident with .} 
By \refclaim{cl-no53} the degree of  is at least four:
this follows directly from~\refclaim{cl-no53} if  is internal;
otherwise  has two neighbors in  and two internal neighbors 
\rt{incident with} .
Let , ,  and  be the neighbors of , ,  
and , respectively, outside of .
If , then  is internal since  has only one neighbor
in  by~\refclaim{E}, and
 and  are internal by \refclaim{cl-no3chord} and \refclaim{Z}.
Also, not all of , ,  and  belong to ,
as \rt{no edge of  is incident} with .  
Thus we may assume that at least one of  and  and
at least one of  and  is internal.  
As  does not share an edge with , the vertices , ,  and  
are distinct and pairwise non-adjacent by~\refclaim{Z}, \refclaim{B} and~\refclaim{A}.  
Let  be the graph obtained from  by identifying
 with  to a new vertex  and  with  to a new vertex 
(with both  and  drawn inside the original face  and the edges
incident with , ,  and  extended towards them in the natural way,
without changing their position with respect to the point ).
Note that any coloring  of  extends to a coloring of : 
Give  and  the color 
and  and  the color .  
If , then color the vertices of 
in the order , ,  and .  
Otherwise, color  by  and then color ,  and  in order.
It follows that some -coloring of  does not extend to a 
-coloring of .
Let  be a minimal subgraph of  such that  is a subgraph of 
and every -coloring of  that extends to a -coloring of 
also extends to a -coloring of .
Then  is -critical.


Next we show that  satisfies ().
Consider  a cycle  of  of length at most four distinct from ,
and let  be the corresponding cycle obtained by replacing 
 by \rt{ or  or} , \rt{as appropriate, and
replacing}  by \rt{ or  or} , \rt{as appropriate}.
If we \rt{added both  and }, 
then  has length at most 
and it has two chords  and .
Thus one of them must belong to , contradicting
the assumption that \rt{no edge of  is incident with .} 
Therefore, we expanded only one vertex in  \rt{into a path}, 
and hence .  
By \refclaim{cl-faces},  does not bound a face.
By~\refclaim{A}  is a subgraph of ins.  The choice of the drawing of 
ensures that the point  (contained in the open disk bounded by ) belongs to
the open disk bounded by ; hence,  satisfies (), as claimed.

Since  has fewer vertices than ,
we conclude that  is isomorphic to one of the graphs from 
Figure~\ref{fig-6plus3}.  
Let  be the unique triangle of .
Using \refclaim{cl-disttr}, we conclude that .  
\rt{From~\refclaim{B}, \refclaim{A} and the fact that no edge of  is 
incident with  we deduce that  and  are not adjacent in .
It follows that exactly one of  belongs to .}
Let  be the corresponding cycle of length six in .
Since , \refclaim{cl-excl6} implies that .
Let us label the vertices of  so that ,
where  is either  or .
Since \rt{no edge of  is incident with } and , ,  and 
have degree three, it follows that  is an edge of .
Let  be the index such that  and  are identified
into .  
\rt{Then , because  and  are not 
adjacent in . Thus one of  is not adjacent to 
(because  is the only triangle in ),
and so we may assume that  is not adjacent to .}
Since  is isomorphic to one of the graphs from Figure~\ref{fig-6plus3},
 is at distance at most one from  in , and,
\rt{since  is not adjacent to }, we conclude that
 is also at distance at most one from  in .
This contradicts~\refclaim{cl-disttr} and proves~\refclaim{cl-5faces}.
\endclaim

\myclaim{cl-no232}{ The cycle  has no subpath  
       with  and .}

To prove~\refclaim{cl-no232} suppose for a contradiction that 
say  and  have degree two and 
has degree three.  
By \refclaim{cl-no22r} the vertices  and  have degree at least three.
By~\refclaim{cl-faces} the face incident with  distinct from the 
outer face is bounded by a -cycle, say .  
Similarly, there is a face bounded by a -cycle ,
where  by~\refclaim{Z}.
Let  be the -cycle .
By~\refclaim{Z} and~\refclaim{A6} the graph ins
is isomorphic to one of the graphs in Figure~\ref{fig-6plus3},
contrary to~\refclaim{cl-disttr}.
This proves~\refclaim{cl-no232}.
\endclaim

\myclaim{cl-no233233}{ If  has at least two vertices of degree two, 
   then it has at least one vertex of degree at least four.}

To prove~\refclaim{cl-no233233}
suppose for a contradiction that  has at least two 
vertices of degree  two and the remaining vertices of degree at most three.  
By \refclaim{cl-no22r} and \refclaim{cl-no232}  has exactly two vertices
of degree two, and the distance in  between them is three. 
We may therefore assume that
 and  have degree two, and 
have degree three.  
By \refclaim{cl-faces},  has 
a -cycle  such that 
.
By~\refclaim{A6} the graph ins is isomorphic to
one of the graphs in Figure~\ref{fig-6plus3}.
It follows that either  or  has degree two, contrary to~\refclaim{Z}.
This proves~\refclaim{cl-no233233}.
\endclaim

We are now ready to complete the proof of Theorem~\ref{thm-6plus3new} using
the so-called discharging argument.
Let us assign charges to the vertices and faces of  in the following way:
Each face  of length  not bounded by  or  gets 
a charge of , 
the face bounded by  gets charge ,
and the face bounded by  gets charge .
A vertex  of degree two gets charge ,
a vertex  of degree three gets charge ,
and all other vertices  get charge .


\myclaim{bubbles}{ The total sum of the charges is at most .}

To prove~\refclaim{bubbles} we deduce from Euler's formula 
the sum of the charges is at most
, where  is the number of vertices of degree two
and  is the number of  vertices of  of degree three in . 
By \refclaim{cl-no22r} .  
By \refclaim{cl-no232}, if  then .
By \refclaim{cl-no233233}, if , then .
It follows that , and hence the sum
of the charges is at most , as desired.
This proves~\refclaim{bubbles}.
\endclaim

\junk{
By \refclaim{cl-no22r}, .  
If , then , and the claim holds.
If , then either  and the claim again holds, 
or , in which case  has a bubble by~\refclaim{cl-no233233},
and hence .
Finally, if , then either , in which case
 and the claim holds; 
or , in which case  has a bubble by~\refclaim{cl-no22r}
and~\refclaim{cl-no232} and hence the claim holds;
or , in which case  has at least two bubbles by~\refclaim{cl-no22r}
and two applications of~\refclaim{cl-no232}, and hence the claim holds.
}

Let us now redistribute the charge according to the following rules: every face
distinct from  sends  to each incident vertex of degree two
and each incident internal vertex of degree three.  The face  sends  to each
face that shares an edge with it.  
The final charge of each vertex and of the faces  and 
is clearly non-negative.
Since the sum of the final charges is equal to the sum of the initial charges,
it follows from~\refclaim{bubbles} that  has a face  of strictly negative final
charge. The face  has length five; let  be the
incident vertices in order.

If say  were a vertex of degree two, then by \refclaim{cl-no22r}, 
 and  would be  vertices of 
of degree at least three, and hence  would send no charge to them,
contrary to the fact that the final charge of  is strictly negative.
It follows that all vertices of  have degree at least three, and since the 
final charge of  is negative,  sends charge to at least four of them.  
Therefore, at least four of the vertices incident with 
are internal and have degree three.  
The fifth vertex has degree at least four
by \refclaim{cl-no53}: this is clear if it is internal, and otherwise
it has two neighbors on  and two neighbors on .
By \refclaim{cl-5faces}  shares an edge with .  
However,  sends  to each of its incident vertices of degree three
and nothing to the fifth vertex,
and receives  from ; hence the final charge of  is non-negative,
a contradiction.~\qed
\bigskip

We are now ready to prove Theorem~\ref{thm-6plus3}.
\bigskip

\noindent
{\bf Proof of Theorem~\ref{thm-6plus3}.}
Let  and  be as in Theorem~\ref{thm-6plus3}.
We may assume that  bounds the outer face.  Let  be any point of the
plane contained in the open disk bounded by .
Let  be a minimal subgraph of  such that  is a subgraph of 
and  does not extend to a -coloring of .
It follows that  is -critical.
Note that  satisfies hypothesis () of  Theorem~\ref{thm-6plus3new}.
By Theorem~\ref{thm-6plus3new} the graph  is isomorphic to one of
the graphs depicted in Figure~\ref{fig-6plus3}.
If neither of the two outcomes of Theorem~\ref{thm-6plus3} holds,
then  extends to a -coloring of , a contradiction.~\qed

\def\JCTB{{\it J.~Combin.\ Theory Ser.\ B}}
\def\JGT{{\it J.~Graph Theory}}

\begin{thebibliography}{99}
\bibitem{aksenov} V.~A.~Aksionov,
On continuation of -colouring of planar graphs (in Russian),
{\em Diskret. Anal.  Novosibirsk} {\bf 26} (1974), 3--19.

\bibitem{AlbHut} M.~Albertson and J.~Hutchinson,
The three excluded cases of Dirac's map-color theorem,
 Second International Conference on Combinatorial Mathematics
(New York, 1978),  pp.~7--17,
{\it Ann.\ New York Acad.\ Sci.} {\bf 319}, New York Acad.\ Sci.,
New York, 1979.

\bibitem{AppHak1} K. Appel and W. Haken,
Every planar map is four colorable, Part I: discharging,
{\it Illinois J. of Math.} {\bf 21} (1977), 429--490.

\bibitem{AppHakKoc} K. Appel, W. Haken and J. Koch,
Every planar map is four colorable, Part II: reducibility,
{\it Illinois J. of Math.} {\bf 21} (1977), 491--567.

\bibitem{AppHak86} K.~Appel and W.~Haken,
The four color proof suffices,
{\it The Mathematical Intelligencer \bf 8} (1986), 10--20.

\bibitem{AppHak89} K. Appel and W. Haken,
Every planar map is four colorable,
{\it Contemp. Math.} {\bf 98} (1989).

\bibitem{BonMur} J.~A.~Bondy and U.~S.~R.~Murty,
Graph Theory with Applications,
North-Holland, New York, Amsterdam, Oxford, 1976.

\bibitem{BorGleMonRas} O.~V.~Borodin, A.~N.~Glebov, M.~Montassier, A.~Raspaud,
Planar graphs without 5- and 7-cycles and without adjacent triangles are 3-colorable,
\JCTB\ {\bf99} (2009), 668--673.

\bibitem{ChePosStrThoYer} N.~Chenette, L.~Postle, N.~Streib, R.~Thomas and C.~Yerger,
Five-coloring graphs on the Klein bottle,
\JCTB\ {\bf 102} (2012), 1067--1098.

\bibitem{Dirmap} G. A. Dirac, Map color theorems,
{\it Canad.\ J.~Math.} {\bf 4} (1952), 480--490.

\bibitem{DvoKawTho} Z.~Dvo\v{r}\'ak, K.~Kawarabayashi and R.~Thomas,
Three-coloring triangle-free planar graphs in linear time,
{\it ACM Transactions on Algorithms} {\bf 7} (2011), article no. 41.

\bibitem{proof-druhy} Z.~Dvo\v{r}\'ak, D.~Kr\'al' and R.~Thomas,
Three-coloring triangle-free graphs on surfaces II. 
-critical graphs in a disk, {\tt arXiv:1302.2158}.

\bibitem{proof-lincrit} Z.~Dvo\v{r}\'ak, D.~Kr\'al' and R.~Thomas,
Three-coloring triangle-free graphs on surfaces III. Graphs of girth five, 
{\tt arXiv:1402.4710}.

\bibitem{proof-havel} Z.~Dvo\v{r}\'ak, D.~Kr\'al' and R.~Thomas,
Three-coloring triangle-free graphs on surfaces V. Coloring planar graphs with distant anomalies,
{\tt arXiv:0911.0885}.

\bibitem{Fis} S.~Fisk, The nonexistence of colorings,
\JCTB\ {\bf 24} (1978), 247--248.

\bibitem{GarJoh} M.~R.~Garey and D.~S.~Johnson,
Computers and intractability. A guide
to the theory of NP-completeness, W. H. Freeman, San Francisco, 1979.

\bibitem{gimbel} J.~Gimbel and C.~Thomassen,
Coloring graphs with fixed genus and girth,
{\em Trans. Amer. Math. Soc.} {\bf 349} (1997), 4555--4564.

\bibitem{grotzsch} H.~Gr\"otzsch,
Ein Dreifarbensatz f\"ur dreikreisfreie Netze auf der Kugel,
{\em Wiss. Z. Martin-Luther-Univ. Halle-Wittenberg Math.-Natur. Reihe}
{\bf 8} (1959), 109--120.

\bibitem{grunbaum} B.~Gr\"unbaum,
Gr\"tzsch's theorem on -colorings,
{\it Michigan Math. J.} {\bf 10} (1963), 303--310.

\bibitem{conj-havel} I.~Havel,
On a conjecture of Gr\"unbaum,
\JCTB\ {\bf 7} (1969), 184--186.

\bibitem{Heawood} P.~J.~Heawood, Map-color theorem,
{\it Quart.\ J.~Pure Appl.\ Math.} {\bf 24} (1890), 332--338.

\bibitem{KawKraKynLid} K.~Kawarabayashi, D.~Kral, J.~Kyn\v{c}l, and B.~Lidick\'y,
6-critical graphs on the Klein bottle,
{\it SIAM J.~Discrete Math.}  {\bf23}  (2008/09), 372--383.

\bibitem{Ringel} G.~Ringel, Map Color Theorem,
Springer-Verlag, Berlin, 1974.

\bibitem{RobSanSeyTho4CT}
N.~Robertson, D.~P.~Sanders, P.~D.~Seymour and R.~Thomas,
The four-colour theorem, \JCTB\ {\bf 70} (1997), 2-44.

\bibitem{thomwalls} R.~Thomas and B.~Walls,
Three-coloring Klein bottle graphs of girth five,
\JCTB\ {\bf  92}  (2004), 115--135.

\bibitem{Tho5torus} C. Thomassen, Five-coloring graphs on the torus,
\JCTB\ {\bf62} (1994), 11--33.

\bibitem{thom-torus} C.~Thomassen,
Gr\"otzsch's 3-Color Theorem and Its Counterparts for the Torus and the Projective Plane,
\JCTB\ {\bf 62} (1994), 268--279.

\bibitem{Tho3list} C.~Thomassen,
-list coloring planar graphs of girth 5,
\JCTB\ {\bf64} (1995), 101--107.

\bibitem{ThoCritical} C. Thomassen, Color-critical graphs on a fixed surface,
\JCTB\ {\bf70} (1997), 67--100.

\bibitem{ThoShortlist} C. Thomassen,
A short list color proof of Grotzsch's theorem,
\JCTB\ {\bf88} (2003), 189--192.

\bibitem{thom-surf} C.~Thomassen,
The chromatic number of a graph of girth 5 on a fixed surface,
{\em J.~Combin. Theory Ser. B} {\bf 87} (2003), 38--71.

\bibitem{conj-stein} R.~ Steinberg,
The state of the three color problem. Quo Vadis, Graph Theory?,
{\em Ann. Discrete Math.} {\bf 55} (1993), 211--248.

\bibitem{WalPhD} B.~Walls, Coloring girth restricted graphs on surfaces,
Ph.D.\ dissertation, Georgia Institute of Technology, 1999.

\bibitem{xu} B.~Xu,
On -colorable plane graphs without - and -cycles, {\em J.~Combin. Theory Ser. B} {\bf 96} (2006), 958--963.

\bibitem{Youngs} D.~A.~Youngs, -chromatic projective graphs,
\JGT\ {\bf21} (1996), 219--227.

\end{thebibliography}

\end{document}
