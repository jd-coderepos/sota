The main result in this section is the following theorem, 
which we establish by a sequence of lemmas. 

\begin{theorem}
\label{theorem:correct}
Algorithm~\ref{algo:semd} 
is a wait-free linearizable implementation 
of the type $\ty{Queue}$ for one enqueuer and any number of dequeuers 
from the types $\ty{Fetch\&Add}$ and $\ty{Register}$. 
\end{theorem}

The following lemma implies (bounded) wait-freedom. 

\begin{lemma}
\label{lemma:waitfree}
There is a constant $U$ such that in all executions, 
$\meth{enq}$ and $\meth{deq}$ operations complete in $U$ steps or less. 
\end{lemma}
\begin{proof}
For the $\meth{enq}$ method, which has no loops, this is clear. 
The $\meth{deq}$ method has one loop, 
but upon further examination, 
we find that in the worst case, 
the loop body executes in its entirety at most twice.
If a dequeuer executes the loop body without returning, 
the local variable $\var{k}$ is nonzero, 
and $\var{itemTaken}[\var{k}].\meth{f\&a}(1)$ returns a nonzero value. 
Another dequeuer, then, must set $\var{k}$ to the same value 
and perform $\var{itemTaken}[\var{k}].\meth{f\&a}(1)$ first. 
Both dequeuers read the value of $\var{k}$ 
from locations in the array $\var{itemIndex}$, 
and since each location is accessed by at most one dequeuer, 
this value is written to two different locations. 
Any value written to two locations in the array $\var{itemIndex}$ 
is the largest written to one row 
and the smallest written to the next, 
so it is impossible for a $\meth{deq}$ operation, 
which reads values from only one row, 
to read more than two such values. 
\end{proof}

More difficult is showing that Algorithm~\ref{algo:semd} is linearizable. 
Any execution that is not linearizable 
has a finite prefix that is also not linearizable, 
that is, linearizability is a safety property. 
Moreover, by wait-freedom, 
any finite execution has a finite continuation 
in which processes finish their current queue operations 
without starting new ones. 
If the longer execution is linearizable, 
then so is its prefix, by the same order of operations. 
It thus suffices to show that any finite execution 
where all operations finish is linearizable. 

Fix a particular finite execution where, without loss of generality, 
all operations finish and all enqueued items are distinct. 
We construct a linearization order $\prec$ as follows. 
An $\meth{enq}$ operation $e$ \concept{matches} 
a $\meth{deq}$ operation $d$ 
if $e$ enqueues the item that $d$ dequeues. 
For $\meth{deq}$ operations $d$, 
let $\loc(d)$ be the last location $(i, j)$ 
of $\var{itemIndex}$ read by $d$. 
For $\meth{enq}$ operations $e$ 
that write exactly one location $(i, j)$ 
in the array $\var{itemIndex}$, let $\loc(e) = (i, j)$. 
For $\meth{enq}$ operations $e$ 
that write two locations $(i, j)$ and $(i + 1, 0)$, 
there is a unique $\meth{deq}$ operation $d$ 
that writes $\var{deqActive}[i, j]$. 
Let $\loc(e) = (i, j)$ if $e$ matches $d$ 
and let $\loc(e) = (i + 1, 0)$ otherwise. 
For operations $o$, let $\row(o)$ be the first coordinate of $\loc(o)$. 

\begin{lemma}
\label{lemma:unique-match}
No operation matches more than one other operation. 
\end{lemma}
\begin{proof}
By assumption, 
no item is enqueued more than once, 
so no item is written to two locations in the array $\var{item}$. 
In order to return an item $\var{item}[k]$, 
a $\meth{deq}$ operation $d$ must be the first 
to access $\var{itemTaken}[k]$, 
ensuring that $d$ and the $\meth{enq}$ operation 
that writes $\var{item}[k]$ are uniquely matched. 
\end{proof}

\begin{lemma}
\label{lemma:match-loc}
If an $\meth{enq}$ operation $e$ 
matches a $\meth{deq}$ operation $d$, 
then $d$ does not precede $e$ and $\loc(e) = \loc(d)$. 
\end{lemma}
\begin{proof}
The operation $d$ reads the index of the enqueued item 
from the same location to which $e$ writes that index. 
Consequently, $d$ cannot precede $e$, 
and $\loc(e) = \loc(d)$ by definition. 
\end{proof}

For $\meth{enq}$ operations $e$, 
let $\orderpt(e) = \lstart(e)$ be the time 
at which $e$ executes line~\ref{line:enq-start}, 
where the \concept{time} at which a step is taken 
is the total number of steps that are taken before it. 
For $\meth{deq}$ operations $d$, 
let $\lalloc(d)$ be the latest time 
at which $d$ executes line~\ref{line:deq-alloc}. 
If $d$ matches an $\meth{enq}$ operation $e$, 
let $\orderpt(d) = \max(\lalloc(d), \lstart(e) + \frac{1}{2})$; 
otherwise, let $\orderpt(d) = \lalloc(d)$. 
For operations $o_1$ and $o_2$, 
write $o_1 \prec o_2$ 
if $(\row(o_1), \orderpt(o_1)) \ltlex (\row(o_2), \orderpt(o_2))$, 
where the symbol $\ltlex$ denotes lexicographic order. 

\begin{lemma}
\label{lemma:total-order}
The relation $\prec$ is a total order. 
\end{lemma}
\begin{proof}
It suffices to show that the function $\orderpt$ is one-to-one. 
For operations $o$, 
either $o$ is unique in taking a step at time $\orderpt(o)$, 
or $o$ is a $\meth{deq}$ operation 
that matches an $\meth{enq}$ operation $e$ 
and $\orderpt(o) = \lstart(e) + \frac{1}{2}$. 
In the latter case, 
no operation $o' \neq o$ satisfies $\orderpt(o') = \orderpt(o)$, 
since by Lemma~\ref{lemma:unique-match}, 
the only operation that matches $e$ is $o$. 
\end{proof}

\begin{lemma}
\label{lemma:precedes}
If $o_1$ and $o_2$ are operations such that $o_1$ precedes $o_2$, 
then $o_1 \prec o_2$. 
\end{lemma}
\begin{proof}
Assume that $o_1 \not\prec o_2$. 
If $\row(o_1) > \row(o_2)$, then $o_1$ does not precede $o_2$, 
since the value of $\var{row}$ is nondecreasing. 
Otherwise, $\row(o_1) = \row(o_2)$ and $\orderpt(o_1) \ge \orderpt(o_2)$. 
For all operations $o$, 
the time $\orderpt(o)$ occurs during $o$, 
since either $o$ takes a step at that time, 
or $o$ is a $\meth{deq}$ operation 
that matches an $\meth{enq}$ operation $e$, 
in which case $o$ ends after time $\orderpt(e) = \lstart(e)$ 
by Lemma~\ref{lemma:match-loc}. 
It follows that $o_1$ does not precede $o_2$. 
\end{proof}

\begin{lemma}
\label{lemma:loc-order}
If $e_1$ and $e_2$ are $\meth{enq}$ operations, 
then $e_1 \prec e_2$ if and only if $\loc(e_1) \ltlex \loc(e_2)$. 
If $d_1$ and $d_2$ are $\meth{deq}$ operations, 
then $(\row(d_1), \lalloc(d_1)) \ltlex (\row(d_2), \lalloc(d_2))$ 
if and only if $\loc(d_1) \ltlex \loc(d_2)$. 
\end{lemma}
\begin{proof}
There is only one enqueuer, 
and the pair $(\var{row}, \var{head})$ 
increases lexicographically with each $\meth{enq}$ operation. 
Line~\ref{line:deq-alloc} is the invocation of $\meth{f\&a}$ 
where $\loc(d)$ is obtained. 
\end{proof}

\begin{lemma}
\label{lemma:complete}
If $d$ is a $\meth{deq}$ operation, 
then for all $\meth{enq}$ operations $e'$ 
with $\loc(e') \ltlex \loc(d)$, 
there exists a $\meth{deq}$ operation $d'$ 
that matches $e'$. 
\end{lemma}
\begin{proof}
Fix an $\meth{enq}$ operation $e'$ with $\loc(e') \ltlex \loc(d)$. 
It suffices to show that some process reads the index written by $e'$, 
since it follows that some $\meth{deq}$ operation matches $e'$. 
If $e'$ writes exactly one location $(i, j)$ 
in the array $\var{itemIndex}$, 
then no dequeuer reads that location beforehand, 
as otherwise the enqueuer would read $\val{true}$ 
from $\var{deqActive}[i, j]$. 
Nevertheless, some $\meth{deq}$ operation does perform the read. 
In each row, 
the set of locations read by dequeuers is a prefix of the row, 
and some dequeuer reads a location to the right of $e'$. 
If $i < \row(d)$, 
a suitable witness is the $\meth{deq}$ operation 
that causes the variable $\var{row}$ to be incremented; 
if $i = \row(d)$, a suitable witness is $d$ itself. 
When $e'$ writes two locations of the array $\var{itemIndex}$, 
the second write necessarily precedes any corresponding read, 
since it is performed before the enqueuer increments $\var{row}$. 
The remaining arguments parallel the one-write case, 
with one complication: 
it may be the case that $\loc(d)$ is between 
the locations of the first and second write. 
In this case, $\loc(e') < \loc(d)$ if and only if 
the $\meth{deq}$ operation that triggered the second write matches $e$. 
\end{proof}

\begin{lemma}
\label{lemma:linearize}
The order $\prec$ is a valid linearization order. 
\end{lemma}
\begin{proof}
Given Lemmas~\ref{lemma:total-order}~and~\ref{lemma:precedes}, 
the only property remaining to be established 
is that the return values are consistent 
with the sequential execution determined by the order $\prec$. 
We prove this by induction on the number of operations. 

Specifically, 
the inductive hypothesis is that through $m$ operations, 
all return values are correct, 
and the contents of the queue 
are the items that have been enqueued but not dequeued, 
in the order in which they were enqueued. 
The basis $m = 0$ is trivial. 
Assuming the inductive hypothesis for $m$, 
if the next operation is an $\meth{enq}$ operation, 
the inductive hypothesis holds for $m + 1$, 
since by Lemma~\ref{lemma:match-loc}, 
$\meth{enq}$ operations are not preceded 
by matching $\meth{deq}$ operations. 
If the next operation is a $\meth{deq}$ operation $d$, 
then by Lemma~\ref{lemma:complete}, 
every $\meth{enq}$ operation $e'$ with $\loc(e') \ltlex \loc(d)$ 
has a matching $\meth{deq}$ operation $d'$. 
Each such $d'$ satisfies $\lalloc(d') < \lalloc(d)$ 
by Lemma~\ref{lemma:loc-order}. 
If $d$ returns $\bot$, 
then by the definition of $\prec$, 
it is the case that $e' \prec d$ if and only if $d' \prec d$, 
so the queue is empty and remains empty. 
If $d$ matches an $\meth{enq}$ operation $e$, 
then $e$ is the first $\meth{enq}$ operation not yet matched, 
by a similar argument. 
\end{proof}

We can now prove Theorem~\ref{theorem:correct}. 

\begin{proof}[Proof of Theorem~\ref{theorem:correct}]
Algorithm~\ref{algo:semd} is wait-free by Lemma~\ref{lemma:waitfree} 
and is a linearizable implementation of \ty{Queue} by Lemma~\ref{lemma:linearize}. 
\end{proof}

\begin{theorem}
Algorithm~\ref{algo:semd} can be implemented by any type of consensus number $2$. 
\end{theorem}
\begin{proof}
By the results of Afek, Weisberger, and Weisman~\cite{DBLP:journals/jal/AfekW99,DBLP:conf/podc/AfekWW92}, any type of consensus number $2$ can implement $\ty{Fetch\&Add}$. 
\end{proof}
