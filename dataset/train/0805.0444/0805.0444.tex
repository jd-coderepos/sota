The main result in this section is the following theorem, 
which we establish by a sequence of lemmas. 

\begin{theorem}
\label{theorem:correct}
Algorithm~\ref{algo:semd} 
is a wait-free linearizable implementation 
of the type  for one enqueuer and any number of dequeuers 
from the types  and . 
\end{theorem}

The following lemma implies (bounded) wait-freedom. 

\begin{lemma}
\label{lemma:waitfree}
There is a constant  such that in all executions, 
 and  operations complete in  steps or less. 
\end{lemma}
\begin{proof}
For the  method, which has no loops, this is clear. 
The  method has one loop, 
but upon further examination, 
we find that in the worst case, 
the loop body executes in its entirety at most twice.
If a dequeuer executes the loop body without returning, 
the local variable  is nonzero, 
and  returns a nonzero value. 
Another dequeuer, then, must set  to the same value 
and perform  first. 
Both dequeuers read the value of  
from locations in the array , 
and since each location is accessed by at most one dequeuer, 
this value is written to two different locations. 
Any value written to two locations in the array  
is the largest written to one row 
and the smallest written to the next, 
so it is impossible for a  operation, 
which reads values from only one row, 
to read more than two such values. 
\end{proof}

More difficult is showing that Algorithm~\ref{algo:semd} is linearizable. 
Any execution that is not linearizable 
has a finite prefix that is also not linearizable, 
that is, linearizability is a safety property. 
Moreover, by wait-freedom, 
any finite execution has a finite continuation 
in which processes finish their current queue operations 
without starting new ones. 
If the longer execution is linearizable, 
then so is its prefix, by the same order of operations. 
It thus suffices to show that any finite execution 
where all operations finish is linearizable. 

Fix a particular finite execution where, without loss of generality, 
all operations finish and all enqueued items are distinct. 
We construct a linearization order  as follows. 
An  operation  \concept{matches} 
a  operation  
if  enqueues the item that  dequeues. 
For  operations , 
let  be the last location  
of  read by . 
For  operations  
that write exactly one location  
in the array , let . 
For  operations  
that write two locations  and , 
there is a unique  operation  
that writes . 
Let  if  matches  
and let  otherwise. 
For operations , let  be the first coordinate of . 

\begin{lemma}
\label{lemma:unique-match}
No operation matches more than one other operation. 
\end{lemma}
\begin{proof}
By assumption, 
no item is enqueued more than once, 
so no item is written to two locations in the array . 
In order to return an item , 
a  operation  must be the first 
to access , 
ensuring that  and the  operation 
that writes  are uniquely matched. 
\end{proof}

\begin{lemma}
\label{lemma:match-loc}
If an  operation  
matches a  operation , 
then  does not precede  and . 
\end{lemma}
\begin{proof}
The operation  reads the index of the enqueued item 
from the same location to which  writes that index. 
Consequently,  cannot precede , 
and  by definition. 
\end{proof}

For  operations , 
let  be the time 
at which  executes line~\ref{line:enq-start}, 
where the \concept{time} at which a step is taken 
is the total number of steps that are taken before it. 
For  operations , 
let  be the latest time 
at which  executes line~\ref{line:deq-alloc}. 
If  matches an  operation , 
let ; 
otherwise, let . 
For operations  and , 
write  
if , 
where the symbol  denotes lexicographic order. 

\begin{lemma}
\label{lemma:total-order}
The relation  is a total order. 
\end{lemma}
\begin{proof}
It suffices to show that the function  is one-to-one. 
For operations , 
either  is unique in taking a step at time , 
or  is a  operation 
that matches an  operation  
and . 
In the latter case, 
no operation  satisfies , 
since by Lemma~\ref{lemma:unique-match}, 
the only operation that matches  is . 
\end{proof}

\begin{lemma}
\label{lemma:precedes}
If  and  are operations such that  precedes , 
then . 
\end{lemma}
\begin{proof}
Assume that . 
If , then  does not precede , 
since the value of  is nondecreasing. 
Otherwise,  and . 
For all operations , 
the time  occurs during , 
since either  takes a step at that time, 
or  is a  operation 
that matches an  operation , 
in which case  ends after time  
by Lemma~\ref{lemma:match-loc}. 
It follows that  does not precede . 
\end{proof}

\begin{lemma}
\label{lemma:loc-order}
If  and  are  operations, 
then  if and only if . 
If  and  are  operations, 
then  
if and only if . 
\end{lemma}
\begin{proof}
There is only one enqueuer, 
and the pair  
increases lexicographically with each  operation. 
Line~\ref{line:deq-alloc} is the invocation of  
where  is obtained. 
\end{proof}

\begin{lemma}
\label{lemma:complete}
If  is a  operation, 
then for all  operations  
with , 
there exists a  operation  
that matches . 
\end{lemma}
\begin{proof}
Fix an  operation  with . 
It suffices to show that some process reads the index written by , 
since it follows that some  operation matches . 
If  writes exactly one location  
in the array , 
then no dequeuer reads that location beforehand, 
as otherwise the enqueuer would read  
from . 
Nevertheless, some  operation does perform the read. 
In each row, 
the set of locations read by dequeuers is a prefix of the row, 
and some dequeuer reads a location to the right of . 
If , 
a suitable witness is the  operation 
that causes the variable  to be incremented; 
if , a suitable witness is  itself. 
When  writes two locations of the array , 
the second write necessarily precedes any corresponding read, 
since it is performed before the enqueuer increments . 
The remaining arguments parallel the one-write case, 
with one complication: 
it may be the case that  is between 
the locations of the first and second write. 
In this case,  if and only if 
the  operation that triggered the second write matches . 
\end{proof}

\begin{lemma}
\label{lemma:linearize}
The order  is a valid linearization order. 
\end{lemma}
\begin{proof}
Given Lemmas~\ref{lemma:total-order}~and~\ref{lemma:precedes}, 
the only property remaining to be established 
is that the return values are consistent 
with the sequential execution determined by the order . 
We prove this by induction on the number of operations. 

Specifically, 
the inductive hypothesis is that through  operations, 
all return values are correct, 
and the contents of the queue 
are the items that have been enqueued but not dequeued, 
in the order in which they were enqueued. 
The basis  is trivial. 
Assuming the inductive hypothesis for , 
if the next operation is an  operation, 
the inductive hypothesis holds for , 
since by Lemma~\ref{lemma:match-loc}, 
 operations are not preceded 
by matching  operations. 
If the next operation is a  operation , 
then by Lemma~\ref{lemma:complete}, 
every  operation  with  
has a matching  operation . 
Each such  satisfies  
by Lemma~\ref{lemma:loc-order}. 
If  returns , 
then by the definition of , 
it is the case that  if and only if , 
so the queue is empty and remains empty. 
If  matches an  operation , 
then  is the first  operation not yet matched, 
by a similar argument. 
\end{proof}

We can now prove Theorem~\ref{theorem:correct}. 

\begin{proof}[Proof of Theorem~\ref{theorem:correct}]
Algorithm~\ref{algo:semd} is wait-free by Lemma~\ref{lemma:waitfree} 
and is a linearizable implementation of \ty{Queue} by Lemma~\ref{lemma:linearize}. 
\end{proof}

\begin{theorem}
Algorithm~\ref{algo:semd} can be implemented by any type of consensus number . 
\end{theorem}
\begin{proof}
By the results of Afek, Weisberger, and Weisman~\cite{DBLP:journals/jal/AfekW99,DBLP:conf/podc/AfekWW92}, any type of consensus number  can implement . 
\end{proof}
