

\documentclass[12pt]{article}



\usepackage{amsmath}\usepackage{amssymb}\usepackage[breaklinks]{hyperref} \usepackage{graphicx}
\usepackage[active]{srcltx} \usepackage{euscript}\usepackage{xspace}\usepackage{color}\usepackage{amsmath}\usepackage{paralist}\usepackage{theorem}\usepackage{picins}\usepackage{boxedminipage}
\usepackage{enumerate}
\usepackage{picins}
\usepackage{multirow}

\setlength{\textwidth}{6.5in}
\setlength{\topmargin}{0.0in}
\setlength{\headheight}{0in}
\setlength{\headsep}{0.0in}
\setlength{\textheight}{9in}
\setlength{\oddsidemargin}{0in}
\setlength{\evensidemargin}{0in}



\ifx\SOCG\undefined
\newcommand{\InConfVer}[1]{}
\newcommand{\InFullVer}[1]{#1}
\else
\newcommand{\InConfVer}[1]{#1}
\newcommand{\InFullVer}[1]{}
\fi



\newcommand{\xparagraph}[1]{{\smallskip\textsf{#1}}}


\newcommand{\Opt}{\mathrm{opt}}\newcommand{\OptLP}{\mathrm{opt}_{\text{LP}}}
\newcommand{\PTAS}{\Term{P{T}AS}\xspace}

\newcommand{\ts}{\hspace{0.6pt}}\renewcommand{\th}{\si{th}\xspace}

\newcommand{\scl}{\Delta}\newcommand{\sclC}{\alpha}

\newcommand{\Circle}{C}

\providecommand{\si}[1]{#1}

\newcommand{\AnneThanks}[1]{\thanks{Department of Information and Computing Sciences; Utrecht
      University; The Netherlands;
      \texttt{anne}\hspace{0cm}\texttt{\atgen{}cs.uu.nl}. #1}}

\newcommand{\BenThanks}[1]{\thanks{Department of Computer Science;
      University of Illinois; 201 N. Goodwin Avenue; Urbana, IL, 61801, USA; {\tt \si{raichel}2\atgen{}\si{uiuc}.\si{edu}}; {\tt \url{\si{http://www.cs.uiuc.edu/\string~\si{raichel2}}}}. #1}}

\newcommand{\atgen}{\symbol{'100}}
\newcommand{\SarielThanks}[1]{\thanks{Department of Computer Science;
      University of Illinois; 201 N. Goodwin Avenue; Urbana, IL,
      61801, USA; {\tt \si{sariel}\atgen{}\si{uiuc.edu}}; {\tt \url{http://www.uiuc.edu/\string~sariel/}.} #1}}

\newcommand{\emphind}[1]{\emph{#1}\index{#1}}
\definecolor{blue25}{rgb}{0,0,0.55}\newcommand{\emphic}[2]{\textcolor{blue25}{\textbf{\emph{#1}}}\index{#2}}

\newcommand{\emphi}[1]{\emphic{#1}{#1}}
\newcommand{\pth}[2][\!]{#1\left({#2}\right)}
\newcommand{\rpth}[2][]{#1\left({#2}\right)}

\definecolor{red25}{rgb}{0.4,0,0.0}

\newcommand{\PStyle}[1]{\textcolor{red25}{\textsc{#1}}}
\newcommand{\PackRegions}       {\PStyle{{Pack{}Regions}}\xspace}
\newcommand{\PackPoints}        {\PStyle{{Pack{}Points}}\xspace}
\newcommand{\PackHGraph}        {\PStyle{{HGraph{}Packing}}\xspace}\newcommand{\PackHalfspaces}    {\PStyle{{Pack{}Halfspaces}}\xspace}
\newcommand{\PackRaysInPlanes}  {\PStyle{{Pack{}Rays{}In{}Planes{}}}\xspace}
\newcommand{\PackPointsInDisks} {\PStyle{{Pack{}Points{}In{}Disks}}\xspace}
\newcommand{\PackRectsInPnts}   {\PStyle{{Pack{}Re{}ct{}s{}In{}Points}}\xspace}
\newcommand{\PackBoxesInPnts}   {\PStyle{{Pack{}Boxes{}In{}Points}}\xspace}
\newcommand{\PackPntsInSkyline} {\PStyle{{Pack{}Pnts{}In{}Skyline}}\xspace}
\newcommand{\PackPntsInRects}   {\PStyle{\si{PackPntsInRects}}\xspace}
\newcommand{\PackPntsInFTri}    {\PStyle{\si{PackPntsInFatTriangs}}\xspace}



\newtheorem{theorem}{Theorem}[section]
\newtheorem{lemma}[theorem]{Lemma}\newtheorem{definition}[theorem]{Definition}
\newtheorem{defn}[theorem]{Definition}
\newtheorem{corollary}[theorem]{Corollary}
\newtheorem{problem}[theorem]{Problem}\newtheorem{fact}[theorem]{Fact}

{\theorembodyfont{\rm} \newtheorem{observation}[theorem]{Observation}}
{\theorembodyfont{\rm} \newtheorem{remark}[theorem]{Remark}}
   {\theorembodyfont{\rm} \newtheorem{example}[theorem]{Example}}

\newcommand{\brc}[1]{\left\{ {#1} \right\}}

\newcommand{\west}{west\xspace}
\newcommand{\east}{east\xspace}
\newcommand{\south}{south\xspace}
\newcommand{\north}{north\xspace}
\newcommand{\southeast}{south-east\xspace}
\newcommand{\northwest}{north-west\xspace}


\newcommand{\ObjSet}{\EuScript{D}}\newcommand{\ObjSetA}{X}


\newcommand{\Disk}{\mathsf{d}}\newcommand{\DiskSet}{\EuScript{D}}

\newcommand{\PlaneSet}{\EuScript{H}}\newcommand{\plane}{\mathsf{h}}

\newcommand{\HalfspaceSet}{\EuScript{S}}\newcommand{\Halfspace}{\mathsf{h}}\newcommand{\RaySet}{\EuScript{R}}\newcommand{\Ray}{\mathsf{r}}

\newcommand{\RectSet}{\EuScript{B}}
\newcommand{\RectSetA}{\EuScript{D}}

\newcommand{\BoxSet}{\EuScript{B}}
\newcommand{\rect}{\mathsf{b}}


\newcommand{\VSetA}{X}\newcommand{\VSetB}{Y}\newcommand{\VSetC}{Z}\newcommand{\VSetD}{U}\newcommand{\VOpt}{V_{\Opt}}

\newcommand{\Inst}{{I}}

\newcommand{\Term}[1]{\textsf{#1}}

\newcommand{\PVD}{\Term{PVD}\xspace}
\newcommand{\VC}{\Term{VC}\xspace}
\newcommand{\LP}{\Term{LP}\xspace}\newcommand{\HyperLP}{\textsc{Hypergraph-LP}\xspace}
\newcommand{\APX}{\Term{AP{X}}\xspace}\newcommand{\APXHard}{\Term{AP{X}-hard}\xspace}\newcommand{\MAXSNP}{\Term{MAX S{}NP}\xspace}\newcommand{\NPHard}{\Term{NP-Hard}\xspace}\newcommand{\NPH}{\Term{NPH}\xspace}\renewcommand{\P}{\Term{P}\xspace}\newcommand{\NP}{\Term{NP}\xspace}\newcommand{\obj}{b}\newcommand{\hedge}{f}\newcommand{\hedgeA}{z}


\newcommand{\conflict}{h}\newcommand{\conflictSet}{\mathcal{H}}\newcommand{\RSample}{\mathsf{R}}

\newcommand{\graph}{{G}}\newcommand{\hgraph}{\mathsf{G}}\newcommand{\VSet}{\mathsf{V}}\newcommand{\HESet}{\mathsf{E}}

\newcommand{\vertex}{v}\newcommand{\weight}[1]{w\pth{#1}}

\newcommand{\RSet}{\mathcal{C}}\newcommand{\OSet}{\mathcal{O}}

\newcommand{\Energy}{\EuScript{E}}

\newcommand{\EnergyX}[2][\!]{\EuScript{E}\pth[#1]{#2}}

\newcommand{\XSays}[2]{{
      {\fbox{\tt
            #1:} }
      #2
      \marginpar{#1}
      {\fbox{\tt
            end}}
      }
   }
\newcommand{\sariel}[1]{{\XSays{Sariel}{#1}}}
\newcommand{\Sariel}[1]{{\XSays{Sariel}{#1}}}


\newcommand{\cRelax}{\phi}
\newcommand{\ringX}[1]{\EuScript{R}_{#1}}
\newcommand{\resistC}{\eta}\newcommand{\resistX}[1]{\resistC\pth{#1}}\newcommand{\resistY}[2]{\resistC\pth{#1, #2}}
\newcommand{\resistZ}[3]{\resistC_{#1}\pth{#2, #3}}
\newcommand{\indep}{\Term{I{N}D{E}P}\xspace}\newcommand{\Union}[2][\!]{\mathsf{U}\pth[#1]{#2}}
\newcommand{\union}[2][\!]{\mathsf{u}\pth[#1]{#2}}
\renewcommand{\Re}{{\rm I\!\hspace{-0.025em} R}}

\newcommand{\mc}{\nu}

\newcommand{\aftermathA}{\par\vspace{-\baselineskip}}

\newcommand{\mba}     {\rule[0.0cm]{0.0cm}{0.36cm}} \newcommand{\mbb}     {\rule[0.0cm]{0.0cm}{0.38cm}} \newcommand{\mbc}     {\rule[0.0cm]{0.0cm}{0.40cm}} \newcommand{\mbd}     {\rule[0.0cm]{0.0cm}{0.42cm}} \newcommand{\mbe}     {\rule[0.0cm]{0.0cm}{0.44cm}} \newcommand{\mbf}     {\rule[0.0cm]{0.0cm}{0.46cm}} \newcommand{\mbg}     {\rule[0.0cm]{0.0cm}{0.48cm}} \newcommand{\mbh}     {\rule[0.0cm]{0.0cm}{0.50cm}} \newcommand{\mbi}     {\rule[0.0cm]{0.0cm}{0.52cm}} \newcommand{\mbj}     {\rule[0.0cm]{0.0cm}{0.54cm}} \newcommand{\MakeBig} {\rule[-.2cm]{0cm}{0.4cm}}
\newcommand{\MakeSBig}{\rule[0.0cm]{0.0cm}{0.38cm}} 

\newcommand{\seclab}[1]{\label{sec:#1}}
\newcommand{\secref}[1]{Section~\ref{sec:#1}}

\newcommand{\problab}[1]{\label{prob:#1}}
\newcommand{\probref}[1]{Problem~\ref{prob:#1}}

\newcommand{\thmlab}[1]{{\label{theo:#1}}}\newcommand{\thmref}[1]{Theorem~\ref{theo:#1}}

\providecommand{\deflab}[1]{\label{def:#1}}
\newcommand{\defref}[1]{Definition~\ref{def:#1}}


\newcommand{\corlab}[1]{\label{cor:#1}}
\newcommand{\corref}[1]{Corollary~\ref{cor:#1}}

\newcommand{\itemlab}[1]{\label{item:#1}}
\newcommand{\itemref}[1]{(\ref{item:#1})}

\newcommand{\exelab}[1]{\label{exer:#1}}
\newcommand{\exeref}[1]{Exercise~\ref{exer:#1}}

\newcommand{\lemlab}[1]{\label{lemma:#1}}
\newcommand{\lemref}[1]{Lemma~\ref{lemma:#1}}

\newcommand{\remlab}[1]{\label{rem:#1}}
\newcommand{\remref}[1]{Remark~\ref{rem:#1}}

\newcommand{\figlab}[1]{\label{figure:#1}}
\newcommand{\figref}[1]{Figure~\ref{figure:#1}}

\newcommand{\obslab}[1]{\label{observation:#1}}
\newcommand{\obsref}[1]{Observation~\ref{observation:#1}}

\newcommand{\factlab}[1]{\label{fact:#1}}
\newcommand{\factref}[1]{Fact~\ref{fact:#1}}

\newcommand{\TwoFigures}[6]{\begin{tabular}{cc}
       \begin{minipage}{0.48\linewidth}
           \begin{center}
               \centerline{\includegraphics[#1]{{#2}}}
           \end{center}
       \end{minipage}
       &
       \begin{minipage}{0.48\linewidth}
           \centerline{\includegraphics[#4]{{#5}}}
       \end{minipage}
~\
        \distGeo{\pnt}{\site} = \distGeo{\pnt}{\pntA} + \distGeo{\pntA}{\site}\geq \distGeo{\pnt}{\pntA} + \distGeo{\pntA}{\siteA}\geq \distGeo{\pnt}{\siteA},
    
        \Prob{X_i^j=1}\leq \pi r_i^2- \pi r_{i-1}^2=\frac{\pi}{2m}(2^{2i}-2^{2(i-1)})=\frac{3\pi}{2m}4^{i-1},
    
        \Prob{Y_i^j=1}&\leq \pth{1-\frac{1}{4}\pi r_{i-2}^2}^{m-1}\leq \exp\pth{-\frac{\pi}{4} r_{i-2}^2(m-1)}=\exp\pth{-\frac{\pi 4^{i-3}(m-1)}{2m}}\leq \exp(-4^{i-3}).
    
        \Ex{\sum_{j} \sum_{i>2} X_i^jY_i^j}&=\sum_{j} \sum_{i>2} \Ex{X_i^j} \Ex{Y_i^j}\leq \sum_{j} \sum_{i>2} \frac{3\pi}{2m}4^{i-1} e^{-4^{i-3}}=\frac{3\pi}{2} \sum_{i>2} 4^{i-1} e^{-4^{i-3}} =O(1),
    
        \distGeo{\pnt}{\pntA}\leq \slFactor \distX{\pnt}{\pntA}\leq \slFactor 2 r_{i-2} \leq r_{i-1} - r_{i-2}<\distX{\pntA}{\pntB} \leq \distGeo{\pntA}{\pntB}.
    
        \Prob{X_i^j=1}&\leq \frac{\text{area on terrain of } \ringX{i}}{\text{area of
              terrain}}\leq \slFactor \pth{ \pi r_i^2- \pi r_{i-1}^2 }\leq \frac{\pi \slFactor}{2m} \pth{ (2\slFactor +1)^{2i} -
           (2\slFactor +1)^{2i-2} }\\&\leq \frac{\pi \slFactor}{2m} (2\slFactor +1)^{2i}.
    
        \Prob{Y_i^j=1}&\leq \pth{1-\frac{\pi r_{i-2}^2}{4\slFactor^2 } }^{m-1}\leq \exp \pth{-\frac{\pi r_{i-2}^2}{4\slFactor^2 }(m-1) }\leq \exp \pth{- (2\slFactor + 1)^{2i-5} }.
    
        \Ex{\sum_{j} \sum_{i>2} X_i^jY_i^j}\leq \sum_{j=1}^m \sum_{i>2} \frac{\pi \slFactor}{2m} (2\slFactor
        +1)^{2i} \cdot \exp \pth{- (2\slFactor + 1)^{2i-5} }= O(1).
    
        \Intersections \leq \constA \sum_{i \geq 1}^{M}
        \pth{\slConst k_i \cardin{\Edges_i} + \ldConst
           \cardin{\Chords_i}} = \constA \sum_{i \geq 1}^{M}
        \pth{\slConst k_i\pth{\cardin{\Edges_i^{> l}}
              +\cardin{\Edges_i^{\leq l}}} +
           \ldConst\pth{\cardin{\Chords_i^{> l}} +
              \cardin{\Chords_i^{\leq l}}}}.
    
        \Intersections \leq \constA \pth{\sum_{i \geq 1}^{M} \slConst
           k_i\cardin{\Edges_i^{\leq l}} +
           \ldConst\cardin{\Chords_i^{\leq l}} + c_2\ldConst\slConst
           k_i}\leq \constA \pth{\sum_{i \geq 1}^{M} \slConst
           k_i\cardin{\Edges_i^{\leq l}} + c_2\ldConst\slConst
           k_i}+\constA \ldConst 4\cardin{\Chords},
    
        \Ex{\Intersections} &\leq \constA \Ex{\sum_{i \geq 1}^{M} \pth{ \slConst k_i\cardin{
                 \Edges_i^{\leq l}} + c_2\ldConst\slConst k_i}} +
        \constA \ldConst 4\cardin{\Chords}=\constA \sum_{i \geq 1}^{M} \pth{ \slConst
           \Ex{k_i}\cardin{\Edges_i^{\leq l}} + c_2\ldConst\slConst
           \Ex{k_i}}
        + \constA \ldConst 4\cardin{\Chords} \\
        &\leq \constC \slConst \slTFactor \ldConst \pth{\sum_{i \geq 1}^{M}
           \pth{\cardin{\Edges_i^{\leq l}} + 1}+ 4\cardin{\Chords}}
        \leq \constC \slConst \slTFactor \ldConst
        (3\cardin{\Edges}+M+4\cardin{\Chords})
    
    r_i &=\alpha_{i,a}+ \underbrace{\frac{{x_i + \alpha_{i,a}}}{2}}_{\alpha_{i,b}}+ \underbrace{\frac{1}{\sqrt{m}} - 2\alpha_{i,a}}_{\alpha_{i,c}} + \underbrace{w - \pth{x_i-\frac{{x_i +
                \alpha_{i,a}}}{2}}}_{\alpha_{i,d}}=\frac{1}{\sqrt{m}} + w.

        \alpha_{X_i} &=\prod_{j=1}^{X_i} \Prob{ \MakeBig d_j\geq r-i/m}\geq \pth[]{1 - \frac{(r-i/m)^2 \frac{\pi}{4}} {\mathrm{area}(f)}
        }^{X_i}\geq \exp\pth{-m(r-i/m)^2\frac{\pi X_i}{2}}.
    
        \exp\pth{-m\sum_{i=1}^{\floor{ rm }} (r-i/m)^2\pi (X_i+Y_i)/2}
    
        &\hspace{-1cm} \Prob{\MakeBig \text{ is alive} \sep{ X_1,
              \dots, X_{\floor{ rm }}, Y_1, \dots, Y_{\floor{ rm }}}}=\pth{\prod_{i=1}^{\floor{ rm }} \alpha_{X_i}}
        \pth{\prod_{i=1}^{\floor{ rm }} \alpha_{Y_i}} \\
        &\geq \pth{\prod_{i=1}^{\floor{ rm }} \exp\pth{-m(r-i/m)^2\pi
              X_i/2} } \pth{\prod_{i=1}^{\floor{ rm }}
           \exp\pth{-m(r-i/m)^2\pi Y_i/2}}\\&=\exp\pth{-m\sum_{i=1}^{\floor{ rm }} (r-i/m)^2 \pi(X_i+Y_i)/2}
    
        \Ex{T}&=\Ex{\sum_{i=1}^{\floor{ rm }} m(r-i/m)^2 \pi(X_i+Y_i)/2} =\sum_{i=1}^{\floor{ rm }} m(r-i/m)^2 \pi (\Ex{X_i} +
        \Ex{Y_i})/2\\&\leq \sum_{i=1}^{\floor{ rm }} m(r-i/m)^2 \leq m(r^2m) = r^3m^2
    
        \Prob{p \text{ is alive}}\geq \frac{1}{2}\exp\pth{-2r^3m^2} \geq \frac{1}{2e^2}.
    
        \Prob{ p \text{ is alive}}&\geq \Prob{\MakeBig \pth{r(f) \leq m^{-2/3}} \cap \pth{
              p\text{ is alive}}}=\Prob{\MakeBig { \; p\text{ is alive}}
           \sep{ r(f) \leq m^{-2/3}}} \Prob{  r(f) \leq m^{-2/3} }\\
        &\geq \frac{1}{2e^2} \Prob{\MakeBig r(f) \leq m^{-2/3}} \leq \frac{1}{2e^2} \cdot \frac{\pi m}{4m^{4/3}} =
        \frac{\pi}{8e^2m^{1/3}}
    
        \alpha_k = \binom{m-1}{k} p_i^k \pth{1-p_i}^{m-1-k}.
    
        \frac{\alpha_{k+1}}{\alpha_k}=\frac{k!  (m-1-k)! p_i }{(k+1)!(m-1-k-1)! (1-p_i)}=\frac{(m-1-k)p_i}{(k+1)(1-p_i)}\leq \frac{2(m-1-k)}{i(m-1)(k+1)} \leq \frac{1}{2},
    
        \Prob{X \in [r_i, r_{i+1}] \MakeBig }&\leq \Prob{ \cardin{\disk(\pnt, r_i) \cap \pth{\PntSet \setminus
                 \brc{\pnt_1}}} \geq 2\MakeBig }=\sum_{k\geq 2} \alpha_k\leq 2 \alpha_2\\
        &\leq 2 \frac{(m-1)(m-2)}{2} \cdot \frac{1}{i^2 (m-1)^2} \leq
        \frac{1}{i^2},
    
        F=\fatX{C} \leq \frac{R_j}{r_{i+1}/4}\leq \frac{4\cdot 3^{-1/4} \, \mysqrt{j/(m-1)}}{ \sqrt{4/ \pth{(i+1) (m-1)\pi}} } \leq \frac{2\cdot 3^{-1/4} \, \mysqrt{j(i+1)}}{ \sqrt{1/ \pi} }\leq 4 \sqrt{j(i+1)}.
    
        \Ex{ \fatX{C}}&\leq \sum_{i=1}^\infty \sum_{j=1}^{L(i)} \Prob{ X_i =1} 4
        \sqrt{j(i+1)}+\sum_{i=1}^\infty \sum_{j=L(i) + 1}^\infty \Prob{ Y_j =1} 4
        \sqrt{j(i+1)}\\
        &\leq \sum_{i=1}^\infty \frac{1}{i^2} 4 L(i) \sqrt{L(i)(i+1)}+\sum_{i=1}^\infty \sum_{j=L(i) + 1}^\infty 6 e^{1-j-1} \cdot 4
        \sqrt{j(i+1)}\\
        &=\sum_{i=1}^\infty O\pth{ \frac{1}{i^{5/4}}} +\sum_{i=1}^\infty O\pth{\frac{1}{i^4}} = O(1).
    
\end{proof}



\end{document}
