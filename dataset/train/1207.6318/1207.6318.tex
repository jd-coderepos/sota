\pdfoutput=1
\documentclass[conference]{IEEEtran}
\makeatletter
\def\ps@headings{\def\@oddhead{\mbox{}\scriptsize\rightmark \hfil \thepage}\def\@evenhead{\scriptsize\thepage \hfil \leftmark\mbox{}}\def\@oddfoot{}\def\@evenfoot{}}
\makeatother
\pagestyle{headings}
\usepackage[pdftex]{graphicx}
\usepackage{amsmath}
\usepackage{amssymb}
\usepackage{array}
\usepackage{mdwmath}
\usepackage{subfigure}
\usepackage{hyperref}
\usepackage{algorithm}
\usepackage{algorithmic}
\usepackage{epsfig}
\usepackage{epstopdf}

\graphicspath{{../pdf/}{../jpeg/}}
\DeclareGraphicsExtensions{.pdf,.jpeg,.png}



\newtheorem{theorem}{Theorem}
\newtheorem{lemma}{Lemma}
\newtheorem{proposition}[theorem]{Proposition}
\newtheorem{corollary}{Corollary}
\newtheorem{definition}{Definition}
\newtheorem{remarks}{Remarks}
\newtheorem{conjecture}{Conjecture}
\newtheorem{assumption}{Assumption}


\newcommand{\rl}{r^*_{\lambda}}
\newcommand{\beqa}{}

\floatname{algorithm}{Algorithm}


\begin{document}
\title{Optimal Sequential Wireless Relay Placement \\
on a Random Lattice Path}

\author{
\IEEEauthorblockN{Abhishek Sinha\IEEEauthorrefmark{1}, Arpan Chattopadhyay\IEEEauthorrefmark{1}, K.~P.~Naveen\IEEEauthorrefmark{1}, 
Marceau Coupechoux\IEEEauthorrefmark{2}  and Anurag Kumar\IEEEauthorrefmark{1}}
\IEEEauthorblockA{\IEEEauthorrefmark{1}Dept. of Electrical Communication Engineering,
Indian Institute of Science, Bangalore 560012, India.\\
Email: \{abhishek.sinha.iisc, arpanc.ju\}@gmail.com, \{naveenkp, anurag\}@ece.iisc.ernet.in}
\IEEEauthorblockA{\IEEEauthorrefmark{2}Telecom ParisTech and CNRS LTCI, 
Dept.\ of Informatique et R\'eseaux, 23, avenue d'Italie, 75013 Paris, France.\\
Email: marceau.coupechoux@telecom-paristech.fr}
}
\maketitle

\begin{abstract}
  Our work is motivated by the need for impromptu (or ``as-you-go'')
  deployment of relay nodes (for establishing a packet communication path
  with a control centre) by firemen/commandos while operating in an
  unknown environment. We consider a model, where a deployment
  operative steps along a random lattice path whose evolution is
  Markov. At each step, the path can randomly either continue in the
  same direction or take a turn ``North'' or ``East,'' or come to an
  end, at which point a data source (e.g., a temperature sensor) has
  to be placed that will send packets to a control centre at the
  origin of the path. A decision has to be made at each step whether
  or not to place a wireless relay node. Assuming that the packet
  generation rate by the source is very low, and simple link-by-link
  scheduling, we consider the problem of relay placement so as to
  minimize the expectation of an end-to-end cost metric (a linear
  combination of the sum of convex hop costs and the number of relays
  placed). This impromptu relay placement problem is formulated as a
  total cost Markov decision process.  First, we derive the optimal
  policy in terms of an optimal placement set and show that this set
  is characterized by a boundary beyond which it is optimal to
  place. Next, based on a simpler alternative one-step-look-ahead
  characterization of the optimal policy, we propose an algorithm
  which is proved to converge to the optimal placement set in a finite
  number of steps and which is faster than the traditional value
  iteration. We show by simulations that the distance based heuristic,
  usually assumed in the literature, is close to the optimal provided
  that the threshold distance is carefully chosen.
\end{abstract}



\begin{keywords}
Relay placement, Sensor networks, Markov decision processes, One-step-look-ahead.
\end{keywords}


\section{Introduction} \label{intro} Wireless networks, such as
cellular networks or multihop ad hoc networks, would normally be
deployed via a planning and design process. There are situations,
however, that require the impromptu (or ``as-you-go'') deployment of a
multihop wireless packet network. For example, such an impromptu
approach would be required to deploy a wireless sensor network for
situational awareness in emergency situations such as those faced by
firemen or commandos (see
\cite{Fischer,howard-etal02incremental-self-deployment-algorithm}). For
example, as they attack a fire in a building, firemen might wish to
place temperature sensors on fire-doors to monitor the spread of fire,
and ensure a route for their own retreat; or commandos attempting to
flush out terrorists might wish to place acoustic or passive infra-red
sensors to monitor the movement of people in the building. As-you-go
deployment may also be of interest when deploying a multi-hop wireless
sensor network over a large terrain (such as a dense forest) in order
to obtain a first-cut deployment which could then be augmented to a
network with desired properties (connectivity and quality-of-service).

With the above larger motivation in mind, in this paper we are
concerned with the rigorous formulation and solution of a problem of
impromptu deployment of a multihop wireless network along a random
lattice path, see Fig.~\ref{impromptu_figure}. The path could
represent the corridor of a large building, or even a trail in a
forest. The objective is to create a multihop wireless path for packet
communication from the end of the path to its beginning. The problem
is formulated as an optimal sequential decision problem. The
formulation gives rise to a total cost Markov decision process, which
we study in detail in order to derive structural properties of the
optimal policy. We also provide an efficient algorithm for calculating
the optimal policy.

\begin{figure}[t]
\centering
\includegraphics[scale=0.45,angle=0]{deployment-on-latticev2}
\caption{\label{impromptu_figure} A wireless network being deployed as
  a person steps along a random lattice path. Inverted \textbf{V}:
  location of the deployment person; solid line: path already covered;
  circles: deployed relays; thick dashed path: a possible evolution of
  the remaining path. The sensor to be placed at the end is
  also shown as the black rectangle.}
\vspace{-6mm}
\end{figure}

\subsection{Related Work}
Our study is motivated by ``first responder'' networks, a concept that
has been around at least since 2001.  In
\cite{howard-etal02incremental-self-deployment-algorithm}, Howard et
al.\ provide heuristic algorithms for the problem of incremental
deployment of sensors (such as surveillance cameras) with the
objective of covering the deployment area. Their problem is related to
that of self-deployment of autonomous robot teams and to the
art-gallery problem.  Creation of a communication network that is
optimal in some sense is not an objective in
\cite{howard-etal02incremental-self-deployment-algorithm}. In a
somewhat similar vein, the work of Loukas et al.\
\cite{mobihoc.loukas-etal08robotic-wireless-network-emergency} is
concerned with the dynamic locationing of robots that, in an emergency
situation, can serve as wireless relays between the infrastructure and
human-carried wireless devices.  The problem of impromptu deployment
of static wireless networks has been considered in
\cite{mobihoc.naudts-etal07monitoring-planning-tool,
  mobihoc.souryal-etal07real-time-deployment-range-extension,
  mobihoc.souryal-etal09rapidly-deployable-mesh-network-testbed,
  mobihoc.aurisch-tlle09relay-placement-emergency-response}.  In
\cite{mobihoc.naudts-etal07monitoring-planning-tool}, Naudts et al.\
provide a methodology in which, after a node is deployed, the next
node to be deployed is turned on and begins to measure the signal
strength to the last deployed node.  When the signal strength drops
below a predetermined level, the next node is deployed and so
on. Souryal et al.\ provide a similar approach in
\cite{mobihoc.souryal-etal07real-time-deployment-range-extension,
  mobihoc.souryal-etal09rapidly-deployable-mesh-network-testbed},
where an extensive study of indoor RF link quality variation is
provided, and a system is developed and demonstrated.  The work
reported in
\cite{mobihoc.aurisch-tlle09relay-placement-emergency-response} is yet
another example of the same approach for relay deployment. More
recently, Liu et al.\ \cite{Breadcrumb} describe a ``breadcrumbs''
system for aiding firefighters inside buildings, and is similar to our
present paper in terms of the class of problems it addresses.  In a
survey article \cite{Fischer}, Fischer et al.\ describe various
localization technologies for assisting emergency responders, thus
further motivating the class of problems we consider.


In our earlier work (Mondal et al.\
\cite{mondal-etal12impromptu-deployment_NCC}) we took the first steps
towards rigorously formulating and addressing the problem of impromptu
optimal deployment of a multihop wireless network on a line.  The line
is of unknown length but prior information is available about its
probability distribution; at each step, the line can come to an end
with probability , at which point a sensor has to be placed. Once
placed, the sensor sends periodic measurement packets to a control
centre near the start of the line.  It is assumed that the measurement
rate at the sensor is low, so that (with a very high probability) a
packet is delivered to the control centre before the next packet is
generated at the sensor. This so called ``lone packet model'' is
realistic for situations in which the sensor makes a measurement every
few seconds. 

The objective of the sequential decision problem is to minimise a
certain expected per packet cost (e.g., end-to-end delay or total
energy expended by a node), which can be expressed as the sum of the
costs over each hop, subject to a constraint on the number of relays
used for the operation. It has been proved in
\cite{mondal-etal12impromptu-deployment_NCC} that an optimal placement
policy solving the above mentioned problem is a threshold rule, i.e.,
there is a threshold  such that, after placing a relay, if the
operative has walked  steps without the path ending, then a relay
must be placed at .

\subsection{Outline and Our Contributions}
In this paper, while continuing to assume (\textbf{a}) that a single
operative moves step-by-step along a path, deciding to place or to not
place a relay, (\textbf{b}) that the length of the path is a
geometrically distributed random multiple of the step size,
(\textbf{c}) that a source of packets is placed at the end of the
path, (\textbf{d}) that the lone packet traffic model applies, and
(\textbf{e}) that the total cost of a deployment is a linear
combination of the sum of convex hop costs and the number of nodes
placed, we extend the work presented in
\cite{mondal-etal12impromptu-deployment_NCC} to the two-dimensional
case. At each step, the line can take a right angle turn either to the
``East'' or to the ``North'' with known fixed probabilities. We assume
a Non-Line-Of-Sight (NLOS) propagation model, where a radio link
exists between two nodes placed anywhere on the path, see
Fig.~\ref{lattice-path-figure}.  The lone packet model is a natural
first assumption, and would be useful in low-duty cycle monitoring
applications.  Once the network has been deployed, an analytical
technique such as that presented in
\cite{rachit-kumar12performance-analysis} can be used to estimate the
actual packet carrying capacity of the network.

We will formally describe our system model and problem formulation in
Section~\ref{system_model_section}.  The following are our main
contributions:

\begin{figure}[t]
\centering
\includegraphics[width=60mm,height=40mm]{model1v2}
\caption{\label{lattice-path-figure} A depiction of relay deployment 
along a random lattice path with NLOS propagation.}
\vspace{-6mm}
\end{figure}

\begin{itemize}
\item We formulate the problem as a total cost Markov decision process
  (MDP), and characterize the optimal policies in terms of placement
  sets. We show that these optimal policies are threshold policies and
  thus the placement sets are characterized by boundaries in the
  two-dimensional lattice (Section~\ref{nlos_section}). Beyond these
  boundaries, it is optimal to place a relay.
\item Noticing that placement instants are renewal points in the
  random process, we recognize and prove the One-Step-Look-Ahead
  (OSLA) characterization of the placement sets
  (Section~\ref{OSLA_formulation_section}).
\item Based on the OSLA characterization, we propose an iterative
  algorithm, which converges to the optimal placement set in a finite
  number of steps (Section~\ref{FPI_NLOS_section}). We have observed
  that this algorithm converges much faster than value iteration.
\item In Section~\ref{numerical_work_section} we provide several
  numerical results that illustrate the theoretical development. The
  relay placement approach proposed in
  \cite{mobihoc.naudts-etal07monitoring-planning-tool,
    mobihoc.souryal-etal07real-time-deployment-range-extension,
    mobihoc.souryal-etal09rapidly-deployable-mesh-network-testbed,
    mobihoc.aurisch-tlle09relay-placement-emergency-response} would
  suggest a distance threshold based placement rule. We numerically
  obtain the optimal rule in this class, and find that the cost of
  this policy is numerically indistinguishable from that of the
  overall optimal policy provided by our theoretical development. It
  suggests that it might suffice to utilize a distance threshold
  policy. However, the distance threshold should be carefully designed
  taking into account the system parameters and the optimality
  objective.
\end{itemize}
For the ease of presentation we have moved most of the proofs
to the Appendix. 

\section{System Model} \label{system_model_section} We consider a
deployment person, whose stride length is 1 unit, moving along a
random path in the two-dimensional lattice, placing relays at some of
the lattice points of the path and finally a source node at the end of
the path. Once placed, the source node periodically generates
measurement packets which are forwarded by the successive relays in a
multihop fashion to the control centre located at ; see
Fig.~\ref{lattice-path-figure}.

\subsection{Random Path}

Let  denote the set of nonnegative integers, and
 the nonnegative orthant of the two dimensional
integer lattice. We will refer to the  direction as East
and to the  direction as North. Starting from 
there is a lattice path that takes random turns to the North or to the
East (this is to avoid the path folding back onto itself, see
Fig~\ref{lattice-path-figure}). Under this restriction, the path
evolves as a stochastic process over . When the
deployment person has reached some lattice point, the path continues
for one more step and terminates with probability , or does not
terminate with probability . In either case, the next step is
Eastward with probability  and Northward with probability .
Thus, for instance,  is the probability that the path proceeds
Eastwards without ending. The person deploying the relays is assumed
to keep a count of  and , the number of steps taken in the
 direction and in  direction, repectively,
since the previous relay was placed. He is also assumed to know the
probabilities  and .

\subsection{Cost Definition}

In our model, we assume NLOS propagation, i.e., packet transmission can take place between any two 
successive relays even if they are not on the same straight line segment of the lattice path. In the 
building context, this would correspond to the walls being radio transparent. The model is also suitable 
when the deployment region is a thickly wooded forest where the deployment person is restricted to move 
only along some narrow path (lattice edges in our model). 

For two successive relays separated by a distance , we assign a cost of  
which could be the average delay incurred over that hop (including transmission overheads and 
retransmission delays), or the power required to get a packet across the hop.
For instance, in our numerical work we use the power cost, 
, where  is the minimum power required,  
represents an SNR constraint and  is the path-loss exponent. Now suppose 
 relays are placed such that the successive 
inter-relay distances are  ( is the distance from the 
control centre at  and the first relay, and  is the distance from 
the last relay to the sensor placed at the end of the path) then the total cost 
of this placement is the sum of the one-hop costs .
The total cost being the sum of one-hop costs can be justified for the lone packet 
model since when a packet is being forwarded there is no other packet
transmission taking place. 

We now impose a few technical conditions on the one-hop cost function
: (\textbf{C1}) , (\textbf{C2})  is convex and
increasing in , and (\textbf{C3}) for any  and  the
difference  increases to .

(\textbf{C1}) is imposed considering the fact that it requires a
non-zero amount of delay or power for transmitting a packet between
two nodes, however close they may be. (\textbf{C2}) and (\textbf{C3})
are properties we require to establish our results on the optimal
policies. They are satisfied by the power cost,  ,
and also by the mean hop delay (see \cite{Prasenjit}).

We will overload the notation  by denoting the one-hop cost between 
the locations  and  as simply  instead 
of . Using the condition on  we prove the following 
convexity result of .
\begin{lemma} \label{conv} The function  is convex in ,
  where .
\end{lemma}
\begin{proof}
  This follows from the fact that  is convex, non-decreasing
  in its argument. For a formal proof, see
  Appendix~\ref{conv_appendix}.
\end{proof}
We further impose the following condition on  where
.  We allow a general cost-function  endowed
with the following property: (\textbf{C4}) The function  is
positive, twice continuously partially differentiable in variables 
and  and ,

where . These properties also hold for the mean delay and the power
functions mentioned earlier.

Finally define, for ,  and 
. 
\begin{lemma}
\label{cor1}
 and  are 
non-decreasing in both the coordinates  and . 
\end{lemma}
\begin{proof}
  This follows directly from (\ref{assumption}). See
  Appendix~\ref{cor1_appendix} for details.
\end{proof}


\subsection{Deployment Policies and Problem Formulation}
A deployment policy  is a sequence of mappings ,
where at the -th step of the path (provided that the path has not
ended thus far)  allows the deployment person to decide whether
to \emph{place} or \emph{not to place} a relay where, in general,
randomization over these two actions is allowed. The decision is based
on the entire information available to the deployment person at the
-th step, namely the set of vertices traced by the path and the
location of the previous vertices where relays were placed.  Let 
represent the set of all policies. For a given policy ,
let  represent the expectation operator under policy
. Let  denote the total cost incurred and  the total number
of relays used. We are interested in solving the following problem,

where  may be interpreted as the cost of a relay.  Solving
the problem in (\ref{eq:modified}) can also help us solve the
following constrained problem,

where  is a contraint on the average number of relays
(we will describe this procedure in Section~\ref{EN_lambda}).
First, in Sections \ref{nlos_section} to \ref{FPI_NLOS_section},
we work towards obtaining an efficient solution to the problem in (\ref{eq:modified}).





\section{MDP Formulation and Solution} 
\label{nlos_section}
In this section we formulate the problem in (\ref{eq:modified}) as a
total cost infinite horizon MDP and derive the optimal policy in terms
of optimal placement set. We show that this set is characterized by a
two-dimensional boundary, upon crossing which it is optimal to place
a relay.


\subsection{States, Actions, State-Transitions and Cost Structure} 
\label{NLOS_MDP}
We formulate the problem as a sequential decision process starting at
the origin of the lattice path. The decision to place or not place a
relay at the -th step is based on , where
 denotes the coordinates of the deployment person with
respect to the previous relay and ;
 means that at step~ the random lattice path has
ended and  means that the path will continue in the
same direction for at least one more step. Thus, the state space is
given by:

where  denotes the cost-free terminal state, i.e., the state
after the end of the path has been discovered. The action taken at
step  is denoted , where  is the action to
place a relay, and  is the action of not placing a relay. When
the state is  and when action  is taken, the
transition probabilities are given by:
\begin {itemize}
\item If  is  then,\\
(\textbf{i})     w.p.\ \\
(\textbf{ii})     w.p.\ \\
(\textbf{iii})    w.p.\ \\
(\textbf{iv})    w.p.\ .

\item If  is  then\\
(\textbf{i})  w.p.\  \\
(\textbf{ii})  w.p.\  \\
(\textbf{iii})  w.p.\  \\
(\textbf{iv})  w.p.\  .
\end{itemize}

If  then the only allowable action is  and we
enter into the state . If the current state is , we stay
in the same cost-free termination state irrespective of the control
. The one step cost when the state is  is given
by:

For simplicity we write the state  as simply .

\subsection{Optimal Placement Set }
\label{nlos_relaxed_problem_section}
Let  denote the optimal cost-to-go when the current
state is . When at some step the state is  the
deployment person has to decide whether to place or not place a relay
at the current step.   is the solution of the Bellman
equation \cite[Page~137, Prop.~1.1]{Bertsekas2},

where  and  denote the expected cost incurred when the decision
is to \emph{place} and \emph{not place} a relay, respectively.  is given by

The term  in the above expression is the one step cost which is first 
incurred when a relay is placed. The remaining terms are the average cost-to-go from 
the next step. The term  can be understood as follows:
 is the probability that the path proceeds Eastward without ending. 
Thus the state at the next step is  w.p.\ , the optimal
cost-to-go from which is, . Similarly for the term , 
 is the probability that the path will proceed, without ending, towards the North
(thus the next state is ) and   is the 
cost-to-go from the next state. Finally, in the term ,  is the probability
that the path will end, either proceeding East or North, at the next step and 
is the cost of the last link.
Following a similar explanation, the expression for  can be written as:


We define the optimal placement set  as the set of all lattice points , 
where it is optimal to place rather than to not place a relay. Formally,

In this definition, if the costs of placing and not-placing are the
same, we have arbitrarily chosen to place at that point.


The above result yields the following main theorem of this section
which characterizes the optimal placement set  in
terms of a boundary.
\begin{theorem} 
\label{placement_boundary}
The optimal placement set  is characterized by a boundary, i.e., there exist 
mappings   and   such that:

\end{theorem}
\begin{proof}[Proof Outline]
  The proof utilizes the conditions \textbf{C2} and \textbf{C3}
  imposed on the cost function . First, using (\ref{cp}) and
  (\ref{cnp}) in (\ref{defn}) and rearranging we alternatively write
   as, , where  is a constant and  is some function
  of  and . Then, we complete the proof by showing that 
  is non-decreasing in both  and . This requires us to prove
  (using an induction argument) that  is non-decreasing in  and . Also,
  Lemma~\ref{cor1} has to be used here. For a formal proof see
  Appendix~\ref{placement_boundary_appendix}.
\end{proof}


\emph{Remark:} Though the optimal placement set 
was characterized nicely in terms of a boundary  and
, a naive approach of computing this boundary, using value
iteration to obtain  (for several values of
), would be computationally intensive. Our
effort in the next section (Section~\ref{OSLA_formulation_section}) is
towards obtaining an alternate simplified representation for
 using which we propose an algorithm in
Section~\ref{FPI_NLOS_section}, which is guaranteed to return
 in a finite (in practice, small) number of
steps.

\section{Optimal Stopping Formulation} 
\label{OSLA_formulation_section}
We observe that the points where the path has not ended, and a relay is placed, 
are renewal points of the 
decision process. This motivates us to think of the decision process after a relay is placed as an 
optimal stopping problem with \emph{termination cost}  (which is the optimal cost-to-go 
from a relay placement point). Let  denote the placement set 
corresponding to the OSLA rule (to be defined next). In this section we prove our next  main result 
that .

\subsection{One-Step-Look-Ahead Stopping Set
  }
Under the OSLA rule, a relay is placed at state  if
and only if the ``cost  of \emph{stopping} (i.e., placing a
relay) at the current step'' is less than the ``cost  of
continuing (without placing relay at the current step) for one more
step, and then stopping (i.e., placing a relay at the next step)''. The
expressions for the costs  and  can be written as:

and

Then we define the OSLA placement set  as:

Substituting for  and  and simplifying we obtain:

where .
\begin{theorem} 
The OSLA rule is a threshold policy, i.e., there exist mappings  
 and  , 
which define the one-step placement set  as follows, 

\end{theorem}
\begin{proof}
  Noticing that in (\ref{OSLA_2})  is non-decreasing in
   and  is a constant, the proof
  follows along the lines of the proof of
  Theorem~\ref{placement_boundary}.
\end{proof}

Now, we present the main theorem of this section.
\begin{theorem} 
\label{optimality_OSLA_2}

\end{theorem}
\begin{proof}
See Appendix~\ref{optimality_OSLA_2_appendix}.
\end{proof}

\emph{Remark:} The characterization in (\ref{OSLA_2}) is much simpler than the one in (\ref{placement1}) 
once the value of  is given. In the following subsection, we define a function
  and express 
 as the minimum value of this function. 

\subsection{Computation of }
\label{calculation_cost_section}
Let us start by defining a collection of placement sets indexed by :

Referring to (\ref{OSLA_2}), note that . Let
 denote the cost-to-go, starting from , if the placement set  is employed. 
Then, since  is the optimal cost-to-go and 
, we have . 

To compute , we proceed by defining the boundary  of  as follows:

where .

Suppose the corridor ends at some , then only a cost of 
 is incurred. Otherwise (i.e., if the corridor reaches some  and 
continues), using a renewal argument, a cost of  is incurred, where 
 is the cost of placing a relay and  is the future cost-to-go. We can thus write:

where  is the probability of the corridor ending at  and 
 is the probability of the corridor reaching the boundary and continuing. 
Solving for , we obtain:
 
The above expression is extensively used in our algorithm proposed in the next section.

We conclude this subsection by deriving the expression for the probabilities 
 and . Let us partition the 
boundary  into three mutually disjoint sets:

For a depiction of the various boundary points, see Fig.~\ref{Boundary_figure}.
\begin{figure}[t!]
\centering
\includegraphics[width=0.6\linewidth]{Boundaryv2}
\caption{Example of placement set of the form in (\ref{NLOS_Placement}): 'o' denotes lattice 
points outside the placement set; lattice points on the boundary can be partitioned into three 
sets according to the direction, from which they can be reached.}
\label{Boundary_figure}
\end{figure}
Now,  can be written as:

This can be understood as follows. Any point  can be reached from West or South.  is the number of 
possible paths for reaching . Each such path has to go  times Eastwards (thus the 
term ) and  times Northwards (thus the term ) and finally ending at  
(thus the term ). Any point  can be reached only 
from South point . The probability of reaching  without ending is 
. Then, the corridor reaches  and ends 
with probability .  for  can 
be obtained analogously. 

Similarly,  can be written as:



\section{OSLA Based Fixed Point Iteration Algorithm} 
\label{FPI_NLOS_section}
In this section, we present an efficient fixed point iteration
algorithm (Algorithm~\ref{Algo}) using the OSLA rule in (\ref{OSLA_2})
for obtaining the optimal placement set, , and
the optimal cost-to-go, . There are two advantages of
our algorithm over the naive approach of directly trying to minimize
the function  to obtain  (recall that
):
\begin{itemize}
\item On the theoretical side, this iterative algorithm avoids
  explicit optimization altogether, which, otherwise would be
  performed numerically over a continuous range. Without any structure
  on the objective function, direct numerical minimization of
   is difficult and often unsatisfactory, as it  invariably
  uses some sort of heuristic search over this continuous range.
\item On the practical side, this algorithm is proved to converge
  within a finite number of iterations and observed to be extremely
  fast (requires 3 to 4 iterations typically). 
\end{itemize}  

The following is our Algorithm which we refer to as the 
OSLA Based Fixed Point Iteration Algorithm.
\begin{algorithm} 
\caption{OSLA Based Fixed Point Iteration Algorithm}
\begin{algorithmic}[1] 
\REQUIRE , , 
\STATE , 
\WHILE{1} 
\STATE 
\STATE Compute  using (\ref{Boundary})
\IF{} \STATE Break; \ENDIF
\STATE 
\STATE  
\ENDWHILE
\STATE {\textbf{return} , }
\end{algorithmic}
\label{Algo}
\end{algorithm}


We now prove the correctness and finite termination properties of our
algorithm.  First, we define .  Now consider a sample plot of the function  in
Fig.~\ref{func_g_figure}.  From Fig.~\ref{h_gh} observe that whenever
 (which is around 150), . Also,
Fig.~\ref{h_gh_superimposed} (where we have plotted the functions
 and ) suggests that  has a unique fixed point. We
formally prove these results.




\begin{figure}[h]
\centering
\subfigure[]{
\includegraphics[width=0.6\linewidth]{p0pt02qpt5lambda41}
       \label{h_gh}
}
\subfigure[]{
\includegraphics[width=0.6\linewidth]{p0pt02qpt5_line_cross}
        \label{h_gh_superimposed}
}
\caption{\subref{h_gh} Cost-to-go  as a function of   \subref{h_gh_superimposed} Zoom on the cost-to-go  as a function of .
These plots are for , , and .\label{func_g_figure}
}
\end{figure}
 
\begin{lemma} \label{FPI_NLOS}
If  then .
\end{lemma}
\begin{proof}
This follows from the manipulation of (\ref{NLOS_gh}). See Appendix \ref{FPI_NLOS_appendix} 
for details.
\end{proof}
\begin{lemma} \label{uniqueFP_NLOS}
  has a unique fixed point.
\end{lemma}
\begin{proof}
From (\ref{NLOS_Placement}) and (\ref{OSLA_2}), we observe that 
. 
From Theorem \ref{optimality_OSLA_2},  is the optimal 
placement set and thus the cost-to-go of using  is , 
i.e., . Hence,  is a fixed 
point of . Now, any  cannot be a fixed point since, in this case,  
from Lemma~\ref{FPI_NLOS}. On the other hand, any  is such that  because 
 is the optimal cost-to-go. Hence,  is the unique fixed point of . 
\end{proof}

We are now ready to prove the convergence property of our Algorithm.


\begin{lemma}\label{hk_nonincreasing_NLOS}
\begin{enumerate}
\item The sequence  (in Algorithm~\ref{Algo}) is non-increasing,
i.e., , with the equality sign holding if and only if .
\item The sequence  is non-increasing, i.e., , where the containment is strict whenever .
\end{enumerate}
\end{lemma}
\begin{proof}
1) Note first that  for  because . Then, for , we have either  or  . In the first case  and we can stop, whereas in the second case, from Lemma \ref{FPI_NLOS} we have  . 

2) From (\ref{NLOS_Placement}),  implies . 
Hence, as  is non-increasing (from Part~1)), 
 is also non-increasing. 

Suppose  then 
(second equality is by the definition of ), which implies  (since  has a unique fixed point, see Lemma~\ref{uniqueFP_NLOS}). Thus, .
\end{proof}

\begin{theorem}
Algorithm~\ref{Algo} returns  and  in a finite number of steps. 
\end{theorem}
\begin{proof}
Noting that  and using (\ref{NLOS_Placement}), we have 
. Either , in which case the algorithm stops. Otherwise, note that both sets,  and  contain a finite number of lattice points (from the definition of  in (\ref{NLOS_Placement})). Using Lemma~\ref{hk_nonincreasing_NLOS},  converges to  in at most  iterations. Once  converges to , the algorithm stops and returns the optimal cost-to-go .
\end{proof}




\section{Solving the Constrained Problem} 
\label{EN_lambda}
In this section, we devise a method to solve the constrained problem in (\ref{eq:main}) using the 
solution of the unconstrained problem (\ref{eq:modified}) provided by Algorithm~\ref{Algo}. 
This method is applied in Section~\ref{distance_heuristic_section} where, imposing a constraint
on the average number of relays, we compare the performance of
a distance based heuristic with the optimal.

We begin with the following standard result which relates the solutions
of the problems in (\ref{eq:modified}) and (\ref{eq:main}).
\begin{lemma}
\label{relation_lemma}
Let  be an optimal policy for the unconstrained problem in
(\ref{eq:modified}) such that . 
Then  is also optimal for the constrained problem in
(\ref{eq:main}).
\end{lemma}
However, the above lemma is useful only when we are able to exhibit a  such that
. The subsequent development in this section
is towards obtaining the solution to the more general case.

The expected number of relays used by the optimal policy, , which uses the optimal 
placement set , can be computed as:

where  is the reaching probability corresponding to  
and  is the boundary of .
A plot of  vs.\  is given in Fig.~\ref{EN_EC_lambda_figure}. 
We make the following observations about .

1)  decreases with ; this is as expected, since as 
each relay becomes ``costlier'' fewer relays are used on the average.

2) Even when ,  is finite. This is because 
, i.e., there is a positive cost for a  length link. Define the value of 
  with  to be .

3)  vs.  is a piecewise constant function. This occurs 
because the relay placement positions are discrete. For a range of values of  the 
same threshold is optimal. This structure is also evident from the results based on the 
optimal stopping formulation and the OSLA rule in Section~\ref{OSLA_formulation_section}. It 
follows that for a value of  at which there is a step in the plot, there are two 
optimal deterministic policies,  and , for the relaxed 
problem. Let  and 
.


We have the following structure of the optimal policy for the constrained problem:
\begin{theorem}
\label{constrained_solution_theorem}
\begin{enumerate}
	\item For  the optimal placement set is obtained for ,
 i.e., is .
	 \item For , if there is a  such that (a) 
 then the optimal policy is , or 
(b)  then the optimal policy is obtained by 
mixing  and .
\end{enumerate}
\end{theorem}
\begin{proof}
  1) is straight forward. For proof of 2)-(a), see
  Lemma~\ref{relation_lemma}. Considering now 2)-(b), define
   such that
  .  We obtain
  a mixing policy  by choosing  w.p.\
   and  w.p.\  at the beginning of the
  deployment. For any policy  we have the following standard
  argument:

The inequality is because  and  are both optimal for the 
problem (\ref{eq:modified}) with relay price .
Thus, we have shown that  is also optimal for the relaxed problem. Using this along 
with  in Lemma~\ref{relation_lemma}, we conclude the proof.
\end{proof}


\section{Numerical Work}
\label{numerical_work_section}
For our numerical work we use the one-hop power function
, with , .
 We first study the effect of parameter variation
on the various costs. Next, we compare the performance of 
a distance based heuristic with the optimal.

\begin{figure}[t!]
\centering
\includegraphics[width=0.6\linewidth]{EN_EC_lambda}
\caption{Average number of relays  (left) and average power cost  (right) as a function of  (,  and ).}
\label{EN_EC_lambda_figure}
\end{figure}

\begin{figure}[t]
\centering
\includegraphics[width=0.6\linewidth]{J00_lambda}
\caption{Average total cost  as a function of  (,  and ).}
\label{J00_lambda_figure}
\end{figure}




\subsection{Effect of Parameter Variation}
In Fig.~\ref{Boundary_figure}, we have already shown an optimal placement
boundary for , , and . Since  the
boundary is symmetric about the  line.

In Fig.~\ref{EN_EC_lambda_figure}, we plot  and
 vs.\ . The plot of  vs.\
 is in Fig.~\ref{J00_lambda_figure}.  These plots are for
 and . Since  is the cost per relay, as
expected,  decreases as  increases. We
observe that  and the optimal total cost
 increase as  increases. A close examination
of Fig.~\ref{EN_EC_lambda_figure} reveals that both the plots are step
functions.  This is due to the discrete placement at lattice points,
which results in the same placement boundary being optimal for a range
of  values. Thus, as seen in Section~\ref{EN_lambda}, at the
 values, where there is jump in , a
random mixture of two policies is needed.

Fig.~\ref{J00_q_figure} shows the variation of the total optimal cost
 with . The variation is symmetric about
. For a given probability  of the path ending, 
results in the path folding frequently. In such a case, since NLOS
propagation is permitted, and the path-loss is isotropic, fewer
relays are required to be placed. On the other hand, when  is close
to  or to  the path takes fewer turns and more relays are
needed, leading to larger values of the total cost.

In Fig.~\ref{boundary_variation_figure} we show the variation of optimal boundaries with 
. As  the path-loss exponent, increases the hop cost increases for a given hop 
distance. This results in relays needing to be placed more frequently. As can be seen the 
placement boundaries shrink with increasing . We also notice that the placement boundary 
for  is a straight line; indeed this provable result holds for  for any 
values of  and . 

\begin{figure}[t!]
\centering
\includegraphics[width=0.6\linewidth]{J00_q}
\caption{Average total cost  as a function of  ( and ).}
\label{J00_q_figure}
\end{figure}

\begin{figure}[t]
\centering
\includegraphics[width=0.6\linewidth]{boundary_variation}
\caption{Boundaries for various values of the path-loss exponent  (, ).}
\label{boundary_variation_figure}
\end{figure}


\subsection{Comparison with the Distance based Heuristic}
\label{distance_heuristic_section}
We recall from the literature survey in Section~\ref{intro} that prior work invariably proposed 
the policy of placing a relay after the RF signal strength from the previous relay dropped 
below a threshold. For isotropic propagation (as we have assumed in this paper), this is 
equivalent to placing the relay after a circular boundary is crossed. With this in mind, we 
obtained the \emph{optimal constant distance placement policy} (called \emph{the heuristic} 
hereafter) numerically in a manner similar to what is described in 
Section~\ref{calculation_cost_section}. A sample result is provided in 
Fig.~\ref{boundary_comparisons_figure}, for the parameters ,  and . We 
observe that if the path were to evolve roughly Eastward or Northward then the heuristic will 
result in many more relays being placed. On the other hand, if the path evolves diagonally 
(which has higher probability) then the two placement boundaries will result in similar 
placement decisions. 


\begin{figure}[t!]
\centering
\includegraphics[width=0.6\linewidth]{Boundary_comparisons}
\caption{Boundary of the optimal placement set (OSLA boundary) and boundary derived from the 
heuristic policy (,  and ). }
\label{boundary_comparisons_figure}
\vspace{-4mm}
\end{figure}

\begin{figure}[t]
\centering
\includegraphics[width=0.6\linewidth]{EC_rho_comparison}
\caption{Average total power as a function of  for the optimal policy ( and , 
which corresponds to the straight line) and for the heuristic () for  and 
.}
\label{EC_rho_comparison_figure}
\vspace{-6mm}
\end{figure}

This observation shows up in Fig.~\ref{EC_rho_comparison_figure}, where we show the cost 
incurred by the optimal policy (for  and for , which corresponds to a straight line 
corridor) and the heuristic () vs.\  for the constrained problem. As expected, the 
cost is much larger for  since the path does not fold. We find that for  the 
optimal placement boundary and the heuristic provide costs that are almost indistinguishable at 
this scale. We have performed simulations by varying the system parameters and observed the 
same good performance of the optimal constant distance placement policy. This suggests that the 
heuristic policy performs well provided that the threshold distance is optimally chosen with 
respect to the system parameters. 

\section{Conclusion}
We considered the problem of placing relays on a random lattice path
to optimize a linear combination of average power cost and average
number of relays deployed. The optimal placement policy was proved to
be of threshold nature (Theorem~\ref{placement_boundary}).  We further
proved the optimality of the OSLA rule (in
Theorem~\ref{optimality_OSLA_2}).  We have also devised an OLSA based
fixed point iteration algorithm (Algorithm~\ref{Algo}), which we have
proved to converge to the optimal placement set in a finite number of
steps.  Through numerical work we observed that the performance (in
terms of average power incurred for a given relay constraint) of the
optimal policy is closed to that of the distance threshold policy
provided that the threshold distance is optimally chosen with respect
to the system parameters.

\bibliographystyle{IEEEtran}
\bibliography{IEEEabrv,BibFile-Infocom}

\clearpage
\appendices

\section{Proof of Lemmas in Section~\ref{system_model_section}}
\subsection{Proof of Lemma \ref{conv}}
\label{conv_appendix}
\begin{proof}
Any norm is convex so that the function  is convex 
in . The delay function  is also assumed to be convex and non-decreasing in its argument. Hence by using the composition rule 
\cite[Section~3.2.4]{Boyd}, we conclude that the function  is convex in . 
\end{proof}

\subsection{Proof of Lemma \ref{cor1}}
\label{cor1_appendix}
\begin{proof}
It is easier to prove the lemma allowing the arguments  and  take values from the Real line. 
We have,

Partially differentiating both sides w.r.t. , we get

where the equality follows from the application of Lagrange's Mean Value Theorem to the function  and the inequality is due to assumption in (\ref{assumption}).
The above proves the fact that  is non-decreasing in . 

To prove that  is non-decreasing in , we partially differentiate  
w.r.t.\  and obtain
 
where the equality  follows from the application of Lagrange's Mean Value Theorem to the function  and the inequality is due to assumption in (\ref{assumption}).
This shows that the function  is non-decreasing in both the coordinates  and . In a similar way it can also be shown that  is non-decreasing in  and  under the assumption made in (\ref{assumption}). This completes the proof.
\end{proof}

\section{Proof of Theorem~\ref{placement_boundary}}
\label{placement_boundary_appendix}
We begin by defining . Substituting for 
 and  
(from (\ref{cp}) and (\ref{cnp}), respectively) into (\ref{defn}) and rearranging we obtain 
(recall the definitions of  and  from Section~\ref{system_model_section}):

\begin{lemma} 
\label{H}
For a fixed ,  is non-decreasing in both  and .
\end{lemma}
\begin{proof}
Consider a sequential relay placement problem where we have  steps to go. The corridor length is the minimum of  and of a geometric random variable with parameter . The problem be formulated as a finite horizon MDP with horizon length . For any given , ,  is obtained recursively: 

For , since a sensor must be placed at the next step, we have 

Therefore, 

From Lemma ~\ref{cor1}, it follows that  is non-decreasing in both  and .
Now we make the induction hypothesis and assume that  is non-decreasing in  and . We have:

By the induction hypothesis and Lemma ~\ref{cor1}, it follows that  is non-decreasing in both  and . The proof is complete by taking the limit as . 
\end{proof}

We are now ready to prove Theorem~\ref{placement_boundary}.
\begin{proof}[Proof of Theorem~\ref{placement_boundary}]
Referring to (\ref{placement1}), utilizing Lemma~\ref{H} and the Lemma ~\ref{cor1}, it follows that for a fixed , the LHS (Left Hand Side) of (\ref{placement1}), describing the placement set  is an increasing function of , while the RHS (Right Hand Side) is a \emph{finite} constant. Also, because of the assumed properties of the function ,  as , for any 
fixed . Hence it follows that there exists an  such that 
. Hence we may write 

The second characterization follows by similar arguments. 
\end{proof}

\section{Proof of Theorem \ref{optimality_OSLA_2}}
\label{optimality_OSLA_2_appendix}

We require the following lemmas to prove Theorem \ref{optimality_OSLA_2}.

\begin{lemma}

\end{lemma}

\begin{proof}
Suppose that . 
Then from (\ref{set11})  and from 
(\ref{set22}), . 
Since , we have from  (\ref{cp}), (\ref{cnp}) and (\ref{defn}) that 

Also we may argue that at the state , it is optimal not to place. Indeed, if it had been optimal to place at the state , at the next step, we return to the same state, viz., . Now, because of the stationarity of the optimal policy, we would keep placing relays at the same point, and since ``relay-cost''  and , the expected cost for this policy would be . Hence,
 
Since  and , we have (noticing that it is optimal to place at these points and utilizing (\ref{cp}) and (\ref{J00})),  

Now, using (\ref{J00}), (\ref{rs1}) and (\ref{rs2}) in (\ref{rs}), we obtain:

This proves that 
 and hence

\end{proof}

Using the above Lemma and from (\ref{set11}), (\ref{set22}),
(\ref{OSLA2_1}), (\ref{OSLA2_2}) we can conclude that:


\begin{lemma}
\label{lem33}
If  is such that  and 
, then 
\end{lemma}

\begin{proof}
Since , we have from (\ref{OSLA_2}),

Now , and , hence we have from (\ref{rs1}) and (\ref{rs2}):

The expression (\ref{J00}) is always true. Now using (\ref{J00}) and the above two equations in inequality (\ref{ineq}), we obtain (\ref{rs}), which proves that .
\end{proof}

\begin{lemma} \label{lemma:mplusk}
If  (resp. ), then  (resp. ) and  (resp. ) for any .
\end{lemma}
\begin{proof}
The proof follows easily because the LHS of (\ref{placement1}) is increasing in both  and  while the RHS is a constant. Similarly, the RHS of (\ref{OSLA_2}) is increasing in both  and  while the LHS is a constant. 
\end{proof}

We can now prove the main theorem. 

\begin{proof}[Proof of Theorem \ref{optimality_OSLA_2}]
We need to show that inequalities in (\ref{ineq1}) and (\ref{ineq2}) are equalities. For any , suppose that in (\ref{ineq1}) . Then we have the following inclusions: 

Let us index the collection of lattice-points  by . Since , from Lemma~\ref{lemma:mplusk}, it follows that . From (\ref{eq55}), .

Then, the optimal policy being a threshold policy, we know that there exists a finite , s.t. , i.e.,

Again from Lemma~\ref{lemma:mplusk}, since , we have for any :

Now we see that for the point , the conditions of Lemma~\ref{lem33} are satisfied. Hence . If , we already have a contradiction since . Otherwise for , using Lemma~\ref{lemma:mplusk} and , we can show that 
 is subject to the conditions of Lemma~\ref{lem33} implying that . By iteration, we finally obtain that , which contradicts (\ref{eq55}) and proves the result. 
\end{proof}





\section{Proof of Lemma~\ref{FPI_NLOS}}
\label{FPI_NLOS_appendix}

We start by showing the following lemma. 
\begin{lemma} \label{g_h_Eqn}
For \emph{any} placement set  of the form in (\ref{NLOS_Placement}), we have:

where .
\end{lemma}
\begin{proof}
We first introduce some notations and definitions.

Let us define a path  as a possible realization of the corridor, starting from  and let  be the probability of such a path. The set of all paths is denoted by . Let  denote the set of all paths that end at  and  the set of all paths that hit  and continue. 

Let us denote the set of edges whose both end vertices belong to the set  by . A path  is completely characterized by its edge set .

The reaching probability, , of a point  is defined as the probability that a random path  \emph{reaches} the point  and continues for at least one step. Hence, .

The incremental cost function   is defined as follows:


For , the incremental cost function allows us to write:

Now consider 

where by  we denote the probability that a random path goes through the edge .

Now if  is horizontal, i.e., , we have  and . Similarly if  is vertical, i.e., , we have  and . Using these relations, we may rewrite (\ref{2D_sum}) as follows:


Now consider the probability . It is the probability that a random path continues beyond the boundary . Hence we may write

Using (\ref{2D_sum_part1}) and (\ref{2D_sum_part2}) in (\ref{NLOS_gh}) and simplifying, we obtain the result. 
\end{proof}


\begin{proof}[Proof of Lemma~\ref{FPI_NLOS}]

We recall the definition of . 

Since , we immediately conclude that . From (\ref{mainEqn_2D}) in Lemma~\ref{g_h_Eqn}, we may write for the optimal placement set :

We may similarly write for the placement set :

Now, since , we may expand the LHS of (\ref{EqnPh}) as follows:

where, for the inequality, we used (\ref{defforPc}) and for (\ref{subforP*}), we have substituted the value for the quantity from (\ref{EqnP*}).
We may alternatively write the RHS of (\ref{EqnPh}) as:

Now comparing (\ref{subforP*}) and (\ref{cmpr2}) and rearranging, we may write:

Now  if and only if , i.e., . In this case we get . 
On the other hand, if , since , from the inequality (\ref{ineq_h_gh}), we conclude that .
\end{proof}











\end{document}