\documentclass{llncs}
\usepackage{amsmath, amssymb}
\usepackage{graphicx}
\usepackage{subfigure}
\usepackage{ifthen}
\usepackage{algorithm}
\usepackage{algpseudocode}
\usepackage{enumitem}









\title{Survival Network Design of Doubling Dimension Metrics}

\author{Hao-Hsiang Hung\thanks{Math \& CS Department,
        Emory University, {\tt hhung2@mathcs.emory.edu}}
}
\institute{Emory University}
\index{Hung, Hao-Hsiang}




\begin{document}
\thispagestyle{empty}
\maketitle

\begin{abstract}
We investigate the Minimum Weight 2-Edge-Connected Spanning Subgraph (2-ECSS) problem in an arbitrary metric space of doubling dimension and show a polynomial time randomized -approximation algorithm.

\end{abstract}

\section{Introduction}
Survival network design is one of the important issues in the telecommunication field, especially when fault-tolerant is required[Vazirani steiner network]~\cite{Guptasurvey}.
Such a design assumes that it is unlikely both of them would malfunction simultaneously.
The design aims to discover reliable (more than one) paths between specific terminals in case that one of the communication path breaks down, there is still another alternative path to redirect the communication flow.

There are several variations of the problems (see metaheuristics book[book number]).
We may formalized a classical survival network design as a graph problem in the following manner:
We say a edge-weighted graph  is 2-edge-connected (2-EC) if for all ,  is still connected.
Here we assume all the edge weights are non-negative, .
The \emph{2-edge-connected-spanning-subgraph} (2-ECSS) \emph{problem} is to find a 2-ECSS of  of minimum weight.
One of the reason of its proposal~\cite{Csaba02} comes from study duality of dense/sparse instances with variations of path problems related to TSP.

2-ECSS is known to be NP-hard and MAX-SNP-hard in general graphs, which is similar to metric TSP.
Note it is MAX-SNP-hard even for bounded degree graphs~\cite{Grigni07}.
In general graphs, the approximation ratio has been improved from 3 [jaja] to 2[khuller and vishkin] (the best known ratio).
About the hardness of approximation, it is NP-hard to approximate within  on graphs of maximum degree 3[citation].

The problem remains NP-hard even restricted to planar graphs.
However, additive approximation algorithms have been successfully developed by Berger and Grigni~\cite{Grigni05,Grigni07}, based on the ingredient of separators in planar graphs~\cite{Grigni00}.
In particular, the time complexity of their approach is .
Borradaile and Klein developed the first EPTAS for 2-ECSS in planar graphs with running time ~\cite{Borradaile08} and has been applied to bounded genus graphs~\cite{Borradaile09}.
There are also LP-based approach to more general -ECSS problem [rewrite this sentences]~\cite{Pritchard10}.



Doubling Dimension is proposed as a characteristic property to define an arbitrary metric space. [arora's lecture CS597D]
In particular, it defines the dimension from the concept of selecting balls of radius  around a collection of center points to cover all the points of the given metric space .
The value of dimension  is impacted by the choice of .


\emph{Our work}. We show the first PTAS for 2-ECSS problem in bounded doubling dimension.
Note the technique introduced by Bartel {et al.}~\cite{Bartal12} could be easily adapted to 2-ECSS problems.

\begin{theorem}
There is a polynomial time randomzied algorithm for computing the 2-ECSS problem in metric spaces of bounded doubling dimension. 
\end{theorem}

The paper is organzied as follows.
We define notations in Section~\ref{prelim}.
We show the approach for computing 2-ECSS in metric space of bounded doubling dimension in Section~\ref{method}.
In Section 4 we survey other more generalized survival network problems and discuss the feasibility of apply techniques from this paper.


\section{Preliminary}
\label{prelim}

A graph  is 2-edge-connected if for any pair of distinct vertices , there are at least two paths connecting them.
If  is edge-weighted and  for all , we define \textbf{2-Edge-Connected Spanning Subgraph} (2-ECSS) problem as to find a 2-ECSS of  of minimum weight.

We assume the readers are familiar with the pioneering tools use in Euclidean TSP~\cite{Arora98} and the first QPTAS for TSP in doubling dimensional metric space~\cite{Talwar04}, therefore in this section we only list the important definiton and lemmata as follows without providing proofs.

\subsection{Doubling Dimension}
We say a metric space  has \textbf{Doubling Dimension}  if for every , suppose all  points could be bounded by a ball of radius , then these points could also be bounded by  balls of radius .
Note  is the distance function defined on any  with range . It is symmetric, meeting the triangle inequality, and  when .
For any , let  be the diameter and  be the minimum interpoint distance, we define its \textbf{Aspect Ratio} as .
Suppose for any , there is a  such that , then we say  is a -covering of .
If  for all , then we say  is a -packing. 
If  is -covering of , -packing, and , then we say  is a (or )-net.\newline

\begin{lemma}[Packing Lemma]
Let  be a doubling metric with dimension ,  with aspect ratio .
The .
\end{lemma}

\noindent\textbf{Randomized Clustering Partition}. Given a doubling metric , we say  is a randomized clustering partition if points in  are grouped into collections of subsets  (we called them \textbf{Clusters}) which satisfy the following conditions: (1) ,; (2)  for all ; (3) ,  are elements of , and (4) for every ,  such that , where  is a ball with center  and radius .
We say  is a child cluster of  if .
For a cluster , we define \textbf{portals} as points in  which is a set of -nets.



\subsection{Bound the Weight of MST in Sub-metric Space}

Let  be a minimum spanning tree of a (sub)-metric space and  be the weight function of a structure.
For example,  is the total weight of edges belong to a minimum spanning tree of a metric space .

\begin{lemma}[Spanning Tree Lemma~\cite{Talwar04}]
Given a doubling metric  with dimension ,  for all .
\end{lemma}

As a useful side product of the diameter bound, we introduced the patching lemma 
(introduced in ~\cite{Arora98}) as an important tool for metric optimization problems in doubling metrics~\cite{Talwar04} 

\begin{corollary}[Patching Lemma]
In a doubling metric, for a tour  which crosses a cluster   times at crossing points , then there is another tour  such that  crosses  at most twice and .
\end{corollary}

Note the patching lemma is an application of the spanning tree lemma. 
The following simple extension shows that given a tour  which visits a cluster  contained in a ball , we can isolate another tour connecting the crossing points of  and segments outside of .

\begin{corollary}[Clustering Tour Lemma]
Suppose  is a doubling dimension, and  is a cluster contained in a ball , and  is a tour crosses  at points .
Let  such that edges in  has at most one end inside of .
Then there is another tour  such that .
\end{corollary}

Additionally, given a doubling metric  with a randomized clustering partition , and a tour  of , we can convert  into another tour  called \textbf{well-behaved tour} with total weights bounded by the weight of , and  only enters/leaves the clusters via portals.

\begin{lemma}[Well-behaved Tour~\cite{Bartal12}]
.
\end{lemma}

\begin{proof}
For any edge  which is cut by a ball of a cluster in , starting from , we go upward until we reach  such that , reaching some , and go downwards to reach .
Now the detour path becomes .
By geometric sum we know  (so is ).
Besides,  by the triangle inequality.
\qed
\end{proof}

\section{2-ECSS in Doubling Dimension}
\label{method}

As the classical approach for metrical optimization problems in doubling dimension setting, our ingredients are from ~\cite{Arora98}~\cite{Talwar04} and we also adapt the concept from ~\cite{Bartal12} for TSP in metric space of doubling dimension.
The key step of improvement from QPTAS to PTAS is the parameterized sparsity concept called -sparse (will be mentioned later), which bounds the crossing number to  (here ).
However, although it is a PTAS, the exponent might still be too large.

We sketch the high level algorithmic steps as follows.
\begin{algorithm}[htb]
  \caption{The approximation algorithm for 2-ECSS in doubling dimension}
  \label{alg:Framework}
  \begin{algorithmic}[1]
    \Require
      A metric space  with doubling dimension  and aspect ratio 
    \State Randomized decompose  into a tree of level ,
    \State with internal node (balls of radius ) at level 
    \State Set the number for sparsity  with value  and let 
    \If{ is not -sparse}
    \State Split the points into dense subsets  and sparse subsets 
    \State Recursively solve sub metric space  with aspect ratio 
    \State Combine best local tours from  and 
    \EndIf
    \State In each sparse set:
    \State Choose the proper number limit  of nets for each cluster
    \State Choose the proper crossing number limit  for each cluster
    \For{ from  to }
    \For{Each cluster of level  with portals}
    \State Retrieve local optimal tours  from children clusters (of level )
    \State Use Patching Lemma to reduce  to 2
    \State Use DP to calculate the local optimal tour of 2-ECSS by MST
    \EndFor
    \EndFor
  \end{algorithmic}
\end{algorithm}

\subsection{Randomized Clustering Partition}
We use a metric space  of doubling dimension into smaller level of clusters.
The first level include all the points in  (we assume the aspect ratio  is known and the minimum distance is 1.
Generally, for the points of each cluster in the -th level, the dimension bound guarantees that there are  balls of radius  of clusters of -th level whose union could cover all these points.
In this part the algorithm works in a randomized divide and conquer manner.
We provide a high-level of sketch of it as follows.
Assume that every vertex has its own index; at each level , we just generate a random permuation  of  and follow the ordering of  to select centers for the balls (of diameter ).
A point is assigned to the first ball that covers it.

The following lemata correspond to in~\cite{Talwar04} and~\cite{Bartal12}. 

\begin{lemma}[Cut Property]
Let  with distance  in the doubling metric  with dimension . 
The probability that  and  are divided into different clusters (of the same level ) is at most .
\end{lemma}

\subsection{Sparse Subspace}
\begin{definition}[Sparsity]
We say a tour  is -sparse if for each point  and for all level of clusters in which  belongs, the total weight . 
\end{definition}



We first show that there is a PTAS for TSP if a tour  in a metric space  is -sparse. 
If not then  can be decomposed into two subspaces  where  is -sparse and the cost for combing the optimal well-behaved subtours is not too high.

\textbf{Choice of Radius}. The range of the radius for any portal  of level  is .
Suppose we have an optimal well-behaved tour .
Classify edges of  into \textbf{short edges} (that is, edges of length at most ) and \textbf{long edges}.
We want to choose the proper range  such that the number of short edges of  cut by the ball  is fewer than .
If there is a decomposition scheme to choose portals in every level so that short edges are cut with high probability, we will have the following good property.

\begin{lemma}[Sparse Lemma]
Given a doubling dimensional metric space  which admits a well-behaved tour -sparse tour , then there is a randomized polynomial time algorithm for finding another  such that  with constant probability.
\end{lemma}
\begin{proof}
First we bound the total weight of short edges of  inside , which is at most  by the assumption that  is -sparse, and the same bound applies to short edges with one end inside .
Let  be the set of values in  such that fewer than  short edges are cut by the ball , and if the distribution of edges inside the ball is uniform, then at least  of them belong to .

For finding a proper value in  we might need resampling to fit the sparsity assumption.
The radius is chosen from a exponential distribution with density function .
By using the Padding Lemma in~\cite{abn11}, the sampled radius belong to  is at least . (illustrating the worst case scenario).

We then bound the cost of converting  to another  such that the size bound of the portals (of every level) and the allowable visiting times of portals are bounded, based on building a randomized decomposition of the metric space with portals specified for each level.
Note from lemma [ClusteringCut] the probability that an edge  is cut by a ball of level  is , and all of them needs to find a -net point, introducing additional length of at most .
Therefore for one particular  to be cut at level is , and by summing all the levels we get .

Next we bound , the number of entering and leaving of edges in  through portals in the decomposition tree of clusters based above. Again, two cases are discussed separately.
For \textbf{short} edges, we know that the radius range of cluster at level  is , and by the packing lemma, a ball of level  centered at  may neighbor to at most  neighbors balls at least at level  (and all level ), and if we overcount each of them might provide  edges to be cut by the ball centered at .
Now we set up the upper bound  for the number of short edges  as , where the  counts for the allowable levels need not patching (by going through portals).
Note such value of  guarantees that for every level , there are at least  edges must be cut by the ball (of level ).
If unfortunately more than  edges are cut by  then we have to build the spanning tree between portals and then apply patching lemma, and the cost shared by edge is .
Multiply by the probability that an edge at level  needs patching is  ( can only be larger), and multiply by all possible  levels  might be cut, the patching cost is .

On the other hand, for handling \textbf{long} edges we need  crossings at portals.
We directly count their ends inside of the ball as the portals.
We analyze it as follows.
Instead of considering a ball of center  with radius , we now consider a collection of eccentric balls of different radius: , , and  and called these balls , , and .
Since  contains , we use the number of long edges  to bound that of .
The maximum number of possible distinct -net points inside  is at most , and we may assume that by patching lemma every -net points inside  is visited at most twice.
Connecting all these edges inwards until reach the interior of , we can use them to bound the number of long edges of , that is, .

Summing the required crossing in short and long edges,  is a rough upper bound of the crossing number required.
\qed
\end{proof}

\subsection{Algorithms for 2ECSS in a Sparse Doubling Metric}

Now we provide a PTAS for 2-ECSS in doubling metrics.
For proper calculation purpose, we set  for some  in the following algorithm.
\begin{theorem}\label{longshort}
Given a doubling metric  which admits a -sparse well-behaved tour , there is a polynomial time randomized algorithm to find a tour  such that .
\end{theorem}

\begin{proof}
By the small fractional "bad events" in the sparse lemma and the exponential distribution assumption we know a successful "good" random guess for the radius is  (for balls of any level), which means we might need  times of independent guessing so that the probability of one of the successful choice is . We apply the method with union bound to determine the raidus of each net point.

We next determine the level of dynamic programming.
By applying the above choice of  and , the number of levels for the clusters is .

For combing the partial solutions from child clusters of level  to one cluster  of level , first we have to count the  random choices for the radius within range .
The number of other net points of the same level () whose radii might also cut by  is at most , therefore the possible configuration of guessing radius and cutting with neighboring net points is .
Roughly multiply  (for every level it is counted) the possible formation of  is .

We now consider another factor: the well-behaved tours. 
Because  has at most  child clusters, there are at most  possible parent-child configurations.
We do not have to consider the radii of lower level balls into cut in the current level because they are they are fixed.
The maximum access time to partial solutions from children clusters is  possibilities.

For the cost from edges coming from the -th level children clusters to the -the level cluster, we upper bound it by pairing up all the possible combination: .

The total complexity is a huge constant number, multiply by a huge polynomial:
They are:  (interacting with children clusters) and  (considering cuts between higher levels).
\end{proof}

\subsection{Decomposing a Doubling Metric into Sparse and Dense Parts}
\begin{lemma}[Sparse Decomposition] 
Given a doubling metric , and let  be the optimal well-behaved tour for .
There is a polynomial time randomized algorithm to decompose  into two parts  and  such that
\begin{enumerate}
\item , , , and ;
\item The optimal well-behaved tour of  is -sparse ();
\item . 
\end{enumerate}
\end{lemma}
\begin{proof}
For calculation purpose, we set the sparsity parameter  be .
The sparse testing starts from level  upwards until we reach some cluster in level  which is not -sparse, and we pick the densest cluster (say, with center ) in this level.
For all  in all levels  we examine if .
If no such  exists then we are done, because by the Local Tour Lemma [to be added later] we know there is a  which satisfies property 2.
Note here we let  be an arbitrary point, and , which meets property 1.
Because it is unnecessary to connect  to , property 3 definitely holds.

Consider the case when there is such a center point  of a cluster in level .
Let .
Note  always holds (independent of the value of ).
If we partition  into two disjoint subsets, the total weight of the crossing edges (subtours) may be uncontrollable, we need to bound the total length of the crossing edges via the similar idea used in Theorem~\ref{longshort}: we classify them into long and short edges, by scanning another eccentric ball of larger radius and bound of cost of these crossings.
Suppose  is the sparse part that we found, and  is the dense part.
Now  needs to be recursively partitioned (because it is still dense and we subtitute  with  with sparsity checking).
The original ball has radius , and we may pick up a value  in (we will prove the existence of  later and how to choose the best value).
A slight difference here is that we define the long crossing edges of the tour  in the layer between  and , with length more than  such that .
For patching into a well-behaved tour , we consider  traverses upwards until level  such that .
Note that we assume  and set  with a large constant.
After reaching level  we apply Well-behaved Tour lemma, and then we connect subtours between portals by Clustering Tour Lemma.
We multiply the upper bound of the Clustering Tour Lemma  by the maximum possible number of portals at level  () and the worst possible tour between every pair of portal .
The result shows patching these long edges needs additional cost

On the other hand, each short edge has one end with distance  from .
We patch them by connecting them to portals of level  such that .
Therefore we need to include portals (*portal net points interchangeably check) of level  from .
Once we obtain the temporary sub-tour for portals of level , the additional patch could be done by applying the Well-behaved Tour Lemma.

To bound the patching cost for short edges, we count the number of -level portals between the chosen radius with range .
Here the principle is to choose  such that the total weight of the collection of MST with radius  is minimized (to bound the cost of patching by converting the original tour to a well-behaved one).
The maximum possible number of -level portals is bounded by .
By a similar analysis in the Sparse Lemma, for each -level portals, we can bound the total cost of patching for short edges (they all have to modify to leave the ball from these portals) with .
If we choose the best  so that the total weight of the MST with radius  is minimized, we obtain
the following: 

Note we denote  as the collection of  portals in the layers between two eccentric balls of radii  and  respectively. 
In addition, if we denote  means the number of edges inside of the layers of eccentric balls between radius  and , then we know , because each ball  centered from every  intersects at most  other balls centered at portals of  level:  (suppose ).

Because our goal is to minimize  by choosing the proper , we want to know how heavy the best setting of  could still weigh.
Recall that we assume the -sparse test fails (of radius  at some point maximizing ), then we enlarge the radius to  to so that we can bound the total weight of the MST inside the ball of radius  by the sub-paths of a well-behaved tour inside of the ball of , plus the weight of some long edges.
By the packing property, it turns out that  can be covered by at most  balls of radius , and the patching cost of connecting the centers of these balls together is at most , so , where  because  is the worst point which fails the sparsity test in ( is also in ).
Therefore we obtain  which bounds the patching cost for edges between the extension of  (including some points from ) and the extension of , and since we set , by substitution we get

Summing the above equations, the total cost of patching is bounded by 

where  is the total weight of a well-behaved tour in .

We apply similar patching and analysis to the extension of  (we also have to add portals of level  for long crossing edges and of  for short crossing edges), and once we know the partial subtours we simply join them together (because they overlap).

To prove the well-behaved tour of  (union its extension) is -sparse where , first note that we wanted to find a  of level  such that .
Before we reached level  (assume it exists), for each ball of each previous lower level  ( and ),  (for some ).
By a common-known fact that a tour of points in  is bounded by , combined with the Well-behaved Tour Lemma, we know a well-behaved tour of  inside of  (here by  we mean all pairwise paths of points in the ball) is bounded by .
Summing all  we bound the new parameter for sparsity.

Last we show property 3 holds.
The union of two well-behaved subtours  and  is definitely a well-behaved tour for  because they both visited portals in level .
The only additional cost comes from counting the crossing edges from  to  by taking the paths through the level- portals, which is at most  by equation (4).
\end{proof}

\begin{lemma}
Suppose the well-behaved subtours of  and of  are known and  is the optimal well-behaved tour for TSP in the subset of points , and  is the solution tour by the above computation.
Then .
\end{lemma}
\begin{proof}
We first prove , then we apply the Well-behaved Tour Lemma.
In the trivial case  contains one point so no decomposition is required.
By induction we suppose .
By the Sparse Decomposition Lemma we know 

Combined with the hypothesis, we get ; then we apply the Well-behaved Tour Lemma to obtain , where  is adjustable to get the target approximation ratio.
\end{proof}




\section{Discussion}

Because of the wide applications of the portal framework, the approach of Bartel {et al.}~\cite{Bartal12} in solving metric TSP in doubling metrics could be easily adapted for some classical survivable network design problems.
Note that a question asked by Berger and Grigni~\cite{Grigni07} about algorithms for Steiner version of 2-ECSS in planar graphs has been answered~\cite{Borradaile08} and even extended to bounded genus case~\cite{Borradaile09}.
We are interested if the Steiner 2-ECSS could be found in doubling metrics.

Another natural generalization is for -connectivity, which has been studied well in geometric graphs~\cite{Czumaj06}.
Note in their patching lemma, they reduce the number of crossing by using Steiner points, based on the assumptions that they could remove the ends of crossing edges and replace them with collection of points and they assume the cost of edges between these points are zero(or infinitesimally small).
Such a scheme might not well fit into the doubling metrics and new technique is needed.



























\bibliographystyle{abbrv}
\bibliography{2ecss}






\end{document}
