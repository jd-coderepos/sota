
\documentclass{article} \usepackage{iclr2020_conference,times}



\usepackage{amsmath,amsfonts,bm}

\newcommand{\figleft}{{\em (Left)}}
\newcommand{\figcenter}{{\em (Center)}}
\newcommand{\figright}{{\em (Right)}}
\newcommand{\figtop}{{\em (Top)}}
\newcommand{\figbottom}{{\em (Bottom)}}
\newcommand{\captiona}{{\em (a)}}
\newcommand{\captionb}{{\em (b)}}
\newcommand{\captionc}{{\em (c)}}
\newcommand{\captiond}{{\em (d)}}

\newcommand{\newterm}[1]{{\bf #1}}


\def\figref#1{figure~\ref{#1}}
\def\Figref#1{Figure~\ref{#1}}
\def\twofigref#1#2{figures \ref{#1} and \ref{#2}}
\def\quadfigref#1#2#3#4{figures \ref{#1}, \ref{#2}, \ref{#3} and \ref{#4}}
\def\secref#1{section~\ref{#1}}
\def\Secref#1{Section~\ref{#1}}
\def\twosecrefs#1#2{sections \ref{#1} and \ref{#2}}
\def\secrefs#1#2#3{sections \ref{#1}, \ref{#2} and \ref{#3}}
\def\eqref#1{equation~\ref{#1}}
\def\Eqref#1{Equation~\ref{#1}}
\def\plaineqref#1{\ref{#1}}
\def\chapref#1{chapter~\ref{#1}}
\def\Chapref#1{Chapter~\ref{#1}}
\def\rangechapref#1#2{chapters\ref{#1}--\ref{#2}}
\def\algref#1{algorithm~\ref{#1}}
\def\Algref#1{Algorithm~\ref{#1}}
\def\twoalgref#1#2{algorithms \ref{#1} and \ref{#2}}
\def\Twoalgref#1#2{Algorithms \ref{#1} and \ref{#2}}
\def\partref#1{part~\ref{#1}}
\def\Partref#1{Part~\ref{#1}}
\def\twopartref#1#2{parts \ref{#1} and \ref{#2}}

\def\ceil#1{\lceil #1 \rceil}
\def\floor#1{\lfloor #1 \rfloor}
\def\1{\bm{1}}
\newcommand{\train}{\mathcal{D}}
\newcommand{\valid}{\mathcal{D_{\mathrm{valid}}}}
\newcommand{\test}{\mathcal{D_{\mathrm{test}}}}

\def\eps{{\epsilon}}


\def\reta{{\textnormal{}}}
\def\ra{{\textnormal{a}}}
\def\rb{{\textnormal{b}}}
\def\rc{{\textnormal{c}}}
\def\rd{{\textnormal{d}}}
\def\re{{\textnormal{e}}}
\def\rf{{\textnormal{f}}}
\def\rg{{\textnormal{g}}}
\def\rh{{\textnormal{h}}}
\def\ri{{\textnormal{i}}}
\def\rj{{\textnormal{j}}}
\def\rk{{\textnormal{k}}}
\def\rl{{\textnormal{l}}}
\def\rn{{\textnormal{n}}}
\def\ro{{\textnormal{o}}}
\def\rp{{\textnormal{p}}}
\def\rq{{\textnormal{q}}}
\def\rr{{\textnormal{r}}}
\def\rs{{\textnormal{s}}}
\def\rt{{\textnormal{t}}}
\def\ru{{\textnormal{u}}}
\def\rv{{\textnormal{v}}}
\def\rw{{\textnormal{w}}}
\def\rx{{\textnormal{x}}}
\def\ry{{\textnormal{y}}}
\def\rz{{\textnormal{z}}}

\def\rvepsilon{{\mathbf{\epsilon}}}
\def\rvtheta{{\mathbf{\theta}}}
\def\rva{{\mathbf{a}}}
\def\rvb{{\mathbf{b}}}
\def\rvc{{\mathbf{c}}}
\def\rvd{{\mathbf{d}}}
\def\rve{{\mathbf{e}}}
\def\rvf{{\mathbf{f}}}
\def\rvg{{\mathbf{g}}}
\def\rvh{{\mathbf{h}}}
\def\rvu{{\mathbf{i}}}
\def\rvj{{\mathbf{j}}}
\def\rvk{{\mathbf{k}}}
\def\rvl{{\mathbf{l}}}
\def\rvm{{\mathbf{m}}}
\def\rvn{{\mathbf{n}}}
\def\rvo{{\mathbf{o}}}
\def\rvp{{\mathbf{p}}}
\def\rvq{{\mathbf{q}}}
\def\rvr{{\mathbf{r}}}
\def\rvs{{\mathbf{s}}}
\def\rvt{{\mathbf{t}}}
\def\rvu{{\mathbf{u}}}
\def\rvv{{\mathbf{v}}}
\def\rvw{{\mathbf{w}}}
\def\rvx{{\mathbf{x}}}
\def\rvy{{\mathbf{y}}}
\def\rvz{{\mathbf{z}}}

\def\erva{{\textnormal{a}}}
\def\ervb{{\textnormal{b}}}
\def\ervc{{\textnormal{c}}}
\def\ervd{{\textnormal{d}}}
\def\erve{{\textnormal{e}}}
\def\ervf{{\textnormal{f}}}
\def\ervg{{\textnormal{g}}}
\def\ervh{{\textnormal{h}}}
\def\ervi{{\textnormal{i}}}
\def\ervj{{\textnormal{j}}}
\def\ervk{{\textnormal{k}}}
\def\ervl{{\textnormal{l}}}
\def\ervm{{\textnormal{m}}}
\def\ervn{{\textnormal{n}}}
\def\ervo{{\textnormal{o}}}
\def\ervp{{\textnormal{p}}}
\def\ervq{{\textnormal{q}}}
\def\ervr{{\textnormal{r}}}
\def\ervs{{\textnormal{s}}}
\def\ervt{{\textnormal{t}}}
\def\ervu{{\textnormal{u}}}
\def\ervv{{\textnormal{v}}}
\def\ervw{{\textnormal{w}}}
\def\ervx{{\textnormal{x}}}
\def\ervy{{\textnormal{y}}}
\def\ervz{{\textnormal{z}}}

\def\rmA{{\mathbf{A}}}
\def\rmB{{\mathbf{B}}}
\def\rmC{{\mathbf{C}}}
\def\rmD{{\mathbf{D}}}
\def\rmE{{\mathbf{E}}}
\def\rmF{{\mathbf{F}}}
\def\rmG{{\mathbf{G}}}
\def\rmH{{\mathbf{H}}}
\def\rmI{{\mathbf{I}}}
\def\rmJ{{\mathbf{J}}}
\def\rmK{{\mathbf{K}}}
\def\rmL{{\mathbf{L}}}
\def\rmM{{\mathbf{M}}}
\def\rmN{{\mathbf{N}}}
\def\rmO{{\mathbf{O}}}
\def\rmP{{\mathbf{P}}}
\def\rmQ{{\mathbf{Q}}}
\def\rmR{{\mathbf{R}}}
\def\rmS{{\mathbf{S}}}
\def\rmT{{\mathbf{T}}}
\def\rmU{{\mathbf{U}}}
\def\rmV{{\mathbf{V}}}
\def\rmW{{\mathbf{W}}}
\def\rmX{{\mathbf{X}}}
\def\rmY{{\mathbf{Y}}}
\def\rmZ{{\mathbf{Z}}}

\def\ermA{{\textnormal{A}}}
\def\ermB{{\textnormal{B}}}
\def\ermC{{\textnormal{C}}}
\def\ermD{{\textnormal{D}}}
\def\ermE{{\textnormal{E}}}
\def\ermF{{\textnormal{F}}}
\def\ermG{{\textnormal{G}}}
\def\ermH{{\textnormal{H}}}
\def\ermI{{\textnormal{I}}}
\def\ermJ{{\textnormal{J}}}
\def\ermK{{\textnormal{K}}}
\def\ermL{{\textnormal{L}}}
\def\ermM{{\textnormal{M}}}
\def\ermN{{\textnormal{N}}}
\def\ermO{{\textnormal{O}}}
\def\ermP{{\textnormal{P}}}
\def\ermQ{{\textnormal{Q}}}
\def\ermR{{\textnormal{R}}}
\def\ermS{{\textnormal{S}}}
\def\ermT{{\textnormal{T}}}
\def\ermU{{\textnormal{U}}}
\def\ermV{{\textnormal{V}}}
\def\ermW{{\textnormal{W}}}
\def\ermX{{\textnormal{X}}}
\def\ermY{{\textnormal{Y}}}
\def\ermZ{{\textnormal{Z}}}

\def\vzero{{\bm{0}}}
\def\vone{{\bm{1}}}
\def\vmu{{\bm{\mu}}}
\def\vtheta{{\bm{\theta}}}
\def\va{{\bm{a}}}
\def\vb{{\bm{b}}}
\def\vc{{\bm{c}}}
\def\vd{{\bm{d}}}
\def\ve{{\bm{e}}}
\def\vf{{\bm{f}}}
\def\vg{{\bm{g}}}
\def\vh{{\bm{h}}}
\def\vi{{\bm{i}}}
\def\vj{{\bm{j}}}
\def\vk{{\bm{k}}}
\def\vl{{\bm{l}}}
\def\vm{{\bm{m}}}
\def\vn{{\bm{n}}}
\def\vo{{\bm{o}}}
\def\vp{{\bm{p}}}
\def\vq{{\bm{q}}}
\def\vr{{\bm{r}}}
\def\vs{{\bm{s}}}
\def\vt{{\bm{t}}}
\def\vu{{\bm{u}}}
\def\vv{{\bm{v}}}
\def\vw{{\bm{w}}}
\def\vx{{\bm{x}}}
\def\vy{{\bm{y}}}
\def\vz{{\bm{z}}}

\def\evalpha{{\alpha}}
\def\evbeta{{\beta}}
\def\evepsilon{{\epsilon}}
\def\evlambda{{\lambda}}
\def\evomega{{\omega}}
\def\evmu{{\mu}}
\def\evpsi{{\psi}}
\def\evsigma{{\sigma}}
\def\evtheta{{\theta}}
\def\eva{{a}}
\def\evb{{b}}
\def\evc{{c}}
\def\evd{{d}}
\def\eve{{e}}
\def\evf{{f}}
\def\evg{{g}}
\def\evh{{h}}
\def\evi{{i}}
\def\evj{{j}}
\def\evk{{k}}
\def\evl{{l}}
\def\evm{{m}}
\def\evn{{n}}
\def\evo{{o}}
\def\evp{{p}}
\def\evq{{q}}
\def\evr{{r}}
\def\evs{{s}}
\def\evt{{t}}
\def\evu{{u}}
\def\evv{{v}}
\def\evw{{w}}
\def\evx{{x}}
\def\evy{{y}}
\def\evz{{z}}

\def\mA{{\bm{A}}}
\def\mB{{\bm{B}}}
\def\mC{{\bm{C}}}
\def\mD{{\bm{D}}}
\def\mE{{\bm{E}}}
\def\mF{{\bm{F}}}
\def\mG{{\bm{G}}}
\def\mH{{\bm{H}}}
\def\mI{{\bm{I}}}
\def\mJ{{\bm{J}}}
\def\mK{{\bm{K}}}
\def\mL{{\bm{L}}}
\def\mM{{\bm{M}}}
\def\mN{{\bm{N}}}
\def\mO{{\bm{O}}}
\def\mP{{\bm{P}}}
\def\mQ{{\bm{Q}}}
\def\mR{{\bm{R}}}
\def\mS{{\bm{S}}}
\def\mT{{\bm{T}}}
\def\mU{{\bm{U}}}
\def\mV{{\bm{V}}}
\def\mW{{\bm{W}}}
\def\mX{{\bm{X}}}
\def\mY{{\bm{Y}}}
\def\mZ{{\bm{Z}}}
\def\mBeta{{\bm{\beta}}}
\def\mPhi{{\bm{\Phi}}}
\def\mLambda{{\bm{\Lambda}}}
\def\mSigma{{\bm{\Sigma}}}

\DeclareMathAlphabet{\mathsfit}{\encodingdefault}{\sfdefault}{m}{sl}
\SetMathAlphabet{\mathsfit}{bold}{\encodingdefault}{\sfdefault}{bx}{n}
\newcommand{\tens}[1]{\bm{\mathsfit{#1}}}
\def\tA{{\tens{A}}}
\def\tB{{\tens{B}}}
\def\tC{{\tens{C}}}
\def\tD{{\tens{D}}}
\def\tE{{\tens{E}}}
\def\tF{{\tens{F}}}
\def\tG{{\tens{G}}}
\def\tH{{\tens{H}}}
\def\tI{{\tens{I}}}
\def\tJ{{\tens{J}}}
\def\tK{{\tens{K}}}
\def\tL{{\tens{L}}}
\def\tM{{\tens{M}}}
\def\tN{{\tens{N}}}
\def\tO{{\tens{O}}}
\def\tP{{\tens{P}}}
\def\tQ{{\tens{Q}}}
\def\tR{{\tens{R}}}
\def\tS{{\tens{S}}}
\def\tT{{\tens{T}}}
\def\tU{{\tens{U}}}
\def\tV{{\tens{V}}}
\def\tW{{\tens{W}}}
\def\tX{{\tens{X}}}
\def\tY{{\tens{Y}}}
\def\tZ{{\tens{Z}}}


\def\gA{{\mathcal{A}}}
\def\gB{{\mathcal{B}}}
\def\gC{{\mathcal{C}}}
\def\gD{{\mathcal{D}}}
\def\gE{{\mathcal{E}}}
\def\gF{{\mathcal{F}}}
\def\gG{{\mathcal{G}}}
\def\gH{{\mathcal{H}}}
\def\gI{{\mathcal{I}}}
\def\gJ{{\mathcal{J}}}
\def\gK{{\mathcal{K}}}
\def\gL{{\mathcal{L}}}
\def\gM{{\mathcal{M}}}
\def\gN{{\mathcal{N}}}
\def\gO{{\mathcal{O}}}
\def\gP{{\mathcal{P}}}
\def\gQ{{\mathcal{Q}}}
\def\gR{{\mathcal{R}}}
\def\gS{{\mathcal{S}}}
\def\gT{{\mathcal{T}}}
\def\gU{{\mathcal{U}}}
\def\gV{{\mathcal{V}}}
\def\gW{{\mathcal{W}}}
\def\gX{{\mathcal{X}}}
\def\gY{{\mathcal{Y}}}
\def\gZ{{\mathcal{Z}}}

\def\sA{{\mathbb{A}}}
\def\sB{{\mathbb{B}}}
\def\sC{{\mathbb{C}}}
\def\sD{{\mathbb{D}}}
\def\sF{{\mathbb{F}}}
\def\sG{{\mathbb{G}}}
\def\sH{{\mathbb{H}}}
\def\sI{{\mathbb{I}}}
\def\sJ{{\mathbb{J}}}
\def\sK{{\mathbb{K}}}
\def\sL{{\mathbb{L}}}
\def\sM{{\mathbb{M}}}
\def\sN{{\mathbb{N}}}
\def\sO{{\mathbb{O}}}
\def\sP{{\mathbb{P}}}
\def\sQ{{\mathbb{Q}}}
\def\sR{{\mathbb{R}}}
\def\sS{{\mathbb{S}}}
\def\sT{{\mathbb{T}}}
\def\sU{{\mathbb{U}}}
\def\sV{{\mathbb{V}}}
\def\sW{{\mathbb{W}}}
\def\sX{{\mathbb{X}}}
\def\sY{{\mathbb{Y}}}
\def\sZ{{\mathbb{Z}}}

\def\emLambda{{\Lambda}}
\def\emA{{A}}
\def\emB{{B}}
\def\emC{{C}}
\def\emD{{D}}
\def\emE{{E}}
\def\emF{{F}}
\def\emG{{G}}
\def\emH{{H}}
\def\emI{{I}}
\def\emJ{{J}}
\def\emK{{K}}
\def\emL{{L}}
\def\emM{{M}}
\def\emN{{N}}
\def\emO{{O}}
\def\emP{{P}}
\def\emQ{{Q}}
\def\emR{{R}}
\def\emS{{S}}
\def\emT{{T}}
\def\emU{{U}}
\def\emV{{V}}
\def\emW{{W}}
\def\emX{{X}}
\def\emY{{Y}}
\def\emZ{{Z}}
\def\emSigma{{\Sigma}}

\newcommand{\etens}[1]{\mathsfit{#1}}
\def\etLambda{{\etens{\Lambda}}}
\def\etA{{\etens{A}}}
\def\etB{{\etens{B}}}
\def\etC{{\etens{C}}}
\def\etD{{\etens{D}}}
\def\etE{{\etens{E}}}
\def\etF{{\etens{F}}}
\def\etG{{\etens{G}}}
\def\etH{{\etens{H}}}
\def\etI{{\etens{I}}}
\def\etJ{{\etens{J}}}
\def\etK{{\etens{K}}}
\def\etL{{\etens{L}}}
\def\etM{{\etens{M}}}
\def\etN{{\etens{N}}}
\def\etO{{\etens{O}}}
\def\etP{{\etens{P}}}
\def\etQ{{\etens{Q}}}
\def\etR{{\etens{R}}}
\def\etS{{\etens{S}}}
\def\etT{{\etens{T}}}
\def\etU{{\etens{U}}}
\def\etV{{\etens{V}}}
\def\etW{{\etens{W}}}
\def\etX{{\etens{X}}}
\def\etY{{\etens{Y}}}
\def\etZ{{\etens{Z}}}

\newcommand{\pdata}{p_{\rm{data}}}
\newcommand{\ptrain}{\hat{p}_{\rm{data}}}
\newcommand{\Ptrain}{\hat{P}_{\rm{data}}}
\newcommand{\pmodel}{p_{\rm{model}}}
\newcommand{\Pmodel}{P_{\rm{model}}}
\newcommand{\ptildemodel}{\tilde{p}_{\rm{model}}}
\newcommand{\pencode}{p_{\rm{encoder}}}
\newcommand{\pdecode}{p_{\rm{decoder}}}
\newcommand{\precons}{p_{\rm{reconstruct}}}

\newcommand{\laplace}{\mathrm{Laplace}} 

\newcommand{\E}{\mathbb{E}}
\newcommand{\Ls}{\mathcal{L}}
\newcommand{\R}{\mathbb{R}}
\newcommand{\emp}{\tilde{p}}
\newcommand{\lr}{\alpha}
\newcommand{\reg}{\lambda}
\newcommand{\rect}{\mathrm{rectifier}}
\newcommand{\softmax}{\mathrm{softmax}}
\newcommand{\sigmoid}{\sigma}
\newcommand{\softplus}{\zeta}
\newcommand{\KL}{D_{\mathrm{KL}}}
\newcommand{\Var}{\mathrm{Var}}
\newcommand{\standarderror}{\mathrm{SE}}
\newcommand{\Cov}{\mathrm{Cov}}
\newcommand{\normlzero}{L^0}
\newcommand{\normlone}{L^1}
\newcommand{\normltwo}{L^2}
\newcommand{\normlp}{L^p}
\newcommand{\normmax}{L^\infty}

\newcommand{\parents}{Pa} 

\DeclareMathOperator*{\argmax}{arg\,max}
\DeclareMathOperator*{\argmin}{arg\,min}

\DeclareMathOperator{\sign}{sign}
\DeclareMathOperator{\Tr}{Tr}
\let\ab\allowbreak
 
\usepackage{url}

\usepackage{graphicx} \usepackage{subfigure} 
\usepackage{multirow}
\usepackage{xcolor}
\usepackage{mathtools}
\usepackage{relsize}
\usepackage{nicefrac}
\usepackage{xspace}
\usepackage{amsmath,amssymb,enumerate}
\usepackage{amsthm,cancel}
\usepackage{natbib}
\usepackage{enumitem}
\usepackage{dsfont}
\usepackage[colorlinks=true,citecolor=blue]{hyperref}
\usepackage[utf8]{inputenc} \usepackage[T1]{fontenc}    

\newtheorem{lemma}{Lemma}
\newtheorem{theorem}{Theorem}
\newtheorem{corollary}{Corollary}
\newtheorem{remark}{Remark}
\newtheorem{definition}{Definition}
\newtheorem{assumption}{Assumption}

\newtheorem*{lemma*}{Lemma}
\newtheorem*{theorem*}{Theorem}
\newtheorem*{assumption*}{Assumption}
\newtheorem*{corollary*}{Corollary}
\newtheorem*{remark*}{Remark}
\newtheorem*{definition*}{Definition}

\newcommand{\resnet}{\textsc{ResNet}}
\newcommand{\bert}{\textsc{Bert}\xspace}


\newcommand{\sag}{\textsc{Sag}}
\newcommand{\sgd}{\textsc{Sgd}\xspace}
\newcommand{\sdca}{\textsc{Sdca}}
\newcommand{\csgd}{\textsc{CSgd}}
\newcommand{\dsgd}{\textsc{DSgd}}
\newcommand{\saga}{\textsc{Saga}}
\newcommand{\stgd}{\textsc{S2gd}}
\newcommand{\svrg}{\textsc{Svrg}\xspace}
\newcommand{\msvrg}{\textsc{Msvrg}}
\newcommand{\gd}{\textsc{GradientDescent}\xspace}
\newcommand{\adagrad}{\textsc{Adagrad}}
\newcommand{\nadam}{\textsc{Nadam}}
\newcommand{\adadelta}{\textsc{Adadelta}}
\newcommand{\adam}{\textsc{Adam}}
\newcommand{\adamnc}{\textsc{AdamNc}}
\newcommand{\adamw}{\textsc{AdamW}}
\newcommand{\rmsprop}{\textsc{RMSprop}}
\newcommand{\amsgrad}{\textsc{AMSGrad}}
\newcommand{\yogi}{\textsc{Yogi}}
\newcommand{\lamb}{\textsc{Lamb}}
\newcommand{\lars}{\textsc{Lars}}
\newcommand{\avsgrad}{\textsc{AVSGrad}}
\newcommand{\cifarnet}{\textsc{Cifarnet}}
\newcommand{\diag}{\text{diag}\xspace}
\newcommand{\sml}[1]{{\small #1}}
\newcommand{\fromto}[3]{\sml{}}

\newcommand{\reals}{\mathbb{R}}
\newcommand{\note}[1]{\textcolor{red}{#1}}
\newcommand{\ip}[2]{\left\langle #1, #2\right\rangle}
\newcommand{\nlsum}{\sum\nolimits}
\newcommand{\Fc}{\mathcal{F}}

\renewcommand{\baselinestretch}{0.97}
\frenchspacing
\sloppy

\usepackage{hyperref}
\usepackage{url}

\usepackage[utf8]{inputenc} \usepackage[T1]{fontenc}    \usepackage{url}            \usepackage{booktabs}       \usepackage{amsfonts}       \usepackage{nicefrac}       \usepackage{microtype}      \usepackage{algorithm}
\usepackage{algorithmic}

\title{Large Batch Optimization for Deep Learning: Training BERT in 76 minutes}



\author{\small Yang You, 
Jing Li, Sashank Reddi, Jonathan Hseu, Sanjiv Kumar, Srinadh Bhojanapalli\\{\bf \small Xiaodan Song}, {\bf \small James Demmel}, {\bf \small Kurt Keutzer}, {\bf \small Cho-Jui Hsieh}\\
{\scriptsize Yang You was a student researcher at Google Brain. This project was done when he was at Google Brain.}
\\
\normalsize{Google},
\normalsize{UC Berkeley},
\normalsize{UCLA}\\
\scriptsize{\{youyang, demmel, keutzer\}@cs.berkeley.edu, \{jingli, sashank, jhseu, sanjivk, bsrinadh, xiaodansong, chojui\}@google.com}
}



\newcommand{\fix}{\marginpar{FIX}}
\newcommand{\new}{\marginpar{NEW}}

\iclrfinalcopy \begin{document}


\maketitle

\begin{abstract}
Training large deep neural networks on massive datasets is  computationally very challenging. There has been recent surge in interest in using \emph{large batch} stochastic optimization methods to tackle this issue. The most prominent algorithm in this line of research is , which by  employing \emph{layerwise adaptive} learning rates trains  on ImageNet in a few minutes. However,  performs poorly for attention models like , indicating that its performance gains are \emph{not} consistent across tasks. In this paper, we first study a principled layerwise adaptation strategy to accelerate training of deep neural networks using large mini-batches. Using this strategy, we develop a new layerwise adaptive large batch optimization technique called ; we then provide convergence analysis of  as well as , showing convergence to a stationary point in general nonconvex settings. Our empirical results demonstrate the superior performance of  across various tasks such as  and -50 training with very little hyperparameter tuning. In particular, for  training, our optimizer enables use of very large batch sizes of 32868 without any degradation of performance.  By increasing the batch size to the memory limit of a TPUv3 Pod,  training time can be reduced from 3 days to just 76 minutes (Table \ref{table:results}). The  implementation is available online\footnote{\url{https://github.com/tensorflow/addons/blob/master/tensorflow_addons/optimizers/lamb.py}}.
\end{abstract}

\iffalse
Training large deep neural networks on massive datasets is  computationally very challenging. There has been recent surge in interest in using \emph{large batch} stochastic optimization methods to tackle this issue. The use of large batch of gradients (which are computed in parallel) in these methods enable them to use much larger learning rates, thereby, drastically reducing the training time. However, current methods require extensive tuning of learning rates and their performance gains are \emph{not} consistent across tasks. In this paper, we study a principled adaptation strategy to accelerate training of deep neural networks using large mini-batches. Using this strategy, we develop a new layer-wise adaptive large batch optimization technique called . We provide a formal convergence analysis of  as well as the previous published layerwise optimizer , showing convergence to a stationary point in general nonconvex settings. Our empirical results demonstrate the superior performance of  across various tasks such as BERT and ResNet-50 training with very little tuning. In particular, for BERT training, our optimizer enables use of very large batch sizes of 32868 without any degradation of performance.  By increasing the batch size to the memory limit of a TPUv3 Pod, BERT training time can be reduced from 3 days to just 76 minutes (Table \ref{table:results}).
\fi


\iffalse
\begin{abstract}
Training large deep neural networks on massive datasets is  computationally very challenging. One promising approach to tackle this issue is through the use of \emph{large batch} parallel stochastic optimization. However, our understanding of this approach in the context of deep learning is still very limited. Furthermore, the current approaches in this direction are heavily hand-tuned. To this end, we first study a general adaptation strategy to accelerate training of deep neural networks using large mini-batches. Using this strategy, we develop a new layer-wise adaptive large batch optimization technique called . We also provide a formal convergence analysis of  as well as the previous published layerwise optimizer , showing convergence to a stationary point in general nonconvex settings. Our empirical results demonstrate the superior performance of  for BERT and ResNet-50 training.  In particular, for BERT training, our optimizer enables use of very large batches sizes of 32868; thereby, requiring just 8599 iterations to train (as opposed to 1 million iterations in the original paper).  By increasing the batch size to the memory limit of a TPUv3 Pod,  BERT training time can be reduced from 3 days to 76 minutes (Table \ref{table:results}). Finally, we demonstrate that  outperforms previous large-batch training algorithms for ResNet-50 on ImageNet; obtaining state-of-the-art performance in just a few minutes.
More results are in the appendix of this paper.
\end{abstract}
\fi

\section{Introduction}


With the advent of large scale datasets, training large deep neural networks, even using computationally efficient optimization methods like Stochastic gradient descent , has become particularly challenging. For instance, training state-of-the-art deep learning models like \bert and ResNet-50 takes 3 days on 16 TPUv3 chips and 29 hours on 8 Tesla P100 gpus respectively \citep{devlin2018bert,he2016deep}. Thus, there is a growing interest to develop optimization solutions to tackle this critical issue. The goal of this paper is to investigate and develop optimization techniques to accelerate training large deep neural networks, mostly focusing on approaches based on variants of \sgd. 

Methods based on  iteratively update the parameters of the model by moving them in a scaled (negative) direction of the gradient calculated on a minibatch.   However, 's scalability is limited by its inherent sequential nature. Owing to this limitation, traditional approaches to improve \sgd training time in the context of deep learning largely resort to distributed asynchronous setup~\citep{dean2012large,recht2011hogwild}. However,  the implicit staleness introduced due to the asynchrony limits the parallelization of the approach, often leading to degraded performance. The feasibility of computing gradient on \emph{large minibatches} in parallel due to recent hardware advances has seen the resurgence of simply using synchronous \sgd with large minibatches as an alternative to asynchronous \sgd. However, na\"ively increasing the batch size typically results in degradation of generalization performance and reduces computational benefits \citep{goyal2017accurate}.

Synchronous  on large minibatches benefits from reduced variance of the stochastic gradients used in . This allows one to use much larger learning rates in , typically of the order square root of the minibatch size. Surprisingly, recent works have demonstrated that up to certain minibatch sizes, linear scaling of the learning rate with minibatch size can be used to further speed up the training \cite{goyal2017accurate}. These works also elucidate two interesting aspects to enable the use of linear scaling in large batch synchronous : (i) linear scaling of learning rate is harmful during the initial phase; thus, a hand-tuned warmup strategy of slowly increasing the learning rate needs to be used initially, and (ii) linear scaling of learning rate can be detrimental beyond a certain batch size. Using these tricks, \cite{goyal2017accurate} was able to drastically reduce the training time of ResNet-50 model from 29 hours to 1 hour using a batch size of 8192. While these works demonstrate the feasibility of this strategy for reducing the wall time for training large deep neural networks, they also highlight the need for an adaptive learning rate mechanism for large batch learning. 

Variants of  using layerwise adaptive learning rates have been recently proposed to address this problem. The most successful in this line of research is the  algorithm \citep{you2017scaling}, which was initially proposed for training . Using , ResNet-50 can be trained on ImageNet in just a few minutes! However, it has been observed that its performance gains are \emph{not} consistent across tasks. For instance,  performs poorly for attention models like . Furthermore, theoretical understanding of the adaptation employed in  is largely missing. To this end, we study and develop new approaches specially catered to the large batch setting of our interest.

{\bf Contributions.} More specifically, we make the following main contributions in this paper.

\begin{itemize}
\item Inspired by , we investigate a general adaptation strategy specially catered to large batch learning and provide intuition for the strategy.
\item Based on the adaptation strategy, we develop a new optimization algorithm (\lamb) for achieving adaptivity of learning rate in . Furthermore, we provide convergence analysis for both  and  to achieve a stationary point in nonconvex settings. We highlight the benefits of using these methods for large batch settings.
\item We demonstrate the strong empirical performance of  across several challenging tasks. Using  we scale the batch size in training  to more than 32k without degrading the performance; thereby, cutting the time down from 3 days to 76 minutes. Ours is the first work to reduce  training wall time to less than couple of hours.
\item We also demonstrate the efficiency of  for training state-of-the-art image classification models like . To the best of our knowledge, ours is first adaptive solver that can achieve state-of-the-art accuracy for -50 as adaptive solvers like Adam fail to obtain the accuracy of  with momentum for these tasks.
\end{itemize}

\subsection{Related Work}

The literature on optimization for machine learning is vast and hence, we restrict our attention to the most relevant works here. Earlier works on large batch optimization for machine learning mostly focused on convex models, benefiting by a factor of square root of batch size using appropriately large learning rate. Similar results can be shown for nonconvex settings wherein using larger minibatches improves the convergence to stationary points; albeit at the cost of extra computation. However, several important concerns were raised with respect to generalization and computational performance in large batch nonconvex settings. It was observed that training with extremely large batch was difficult~\citep{keskar2016large, hoffer2017train}. Thus, several prior works carefully hand-tune training hyper-parameters, like learning rate and momentum, to avoid degradation of generalization performance \citep{goyal2017accurate, li2017scaling, you2018imagenet, shallue2018measuring}. 

\citep{krizhevsky2014one} empirically found that simply scaling the learning rate linearly with respect to batch size works better up to certain batch sizes. To avoid optimization instability due to linear scaling of learning rate, \citet{goyal2017accurate} proposed a highly hand-tuned learning rate which involves a warm-up strategy that gradually increases the LR to a larger value and then switching to the regular LR policy (e.g. exponential or polynomial decay). Using LR warm-up and linear scaling, \citet{goyal2017accurate} managed to train -50 with batch size 8192 without loss in generalization performance. However, empirical study \citep{shallue2018measuring} shows that learning rate scaling heuristics with the batch size do not hold across all problems or across all batch sizes.

More recently, to reduce hand-tuning of hyperparameters, adaptive learning rates for large batch training garnered significant interests. Several recent works successfully scaled the batch size to large values using adaptive learning rates without degrading the performance, thereby, finishing -50 training on ImageNet in a few minutes \citep{you2018imagenet,iandola2016firecaffe,codreanu2017scale,akiba2017extremely,jia2018highly,smith2017don,martens2015optimizing,devarakonda2017adabatch,mikami2018imagenet,osawa2018second,you2019large,yamazaki2019yet}.
To the best of our knowledge, the fastest training result for -50 on ImageNet is due to \cite{ying2018image}, who achieve 76+\% top-1 accuracy. By using the  optimizer and scaling the batch size to 32K  on a TPUv3 Pod, \citet{ying2018image} was able to train -50 on ImageNet in 2.2 minutes. However, it was empirically observed that none of these performance gains hold in other tasks such as BERT training (see Section~\ref{sec:experiments}).  


\iffalse
\subsection{Related Work}

The literature on optimization for machine learning is vast and hence, we restrict our attention to the works on large batch settings that are most relevant to our paper. Earlier works on large batch optimization for machine learning mostly focused on convex models. It is known that for general stochastic convex objective functions, the convergence of  with minibatch  is .  If a more complex optimization problem is solved in each iteration, the convergence rate can be improved to , which improves when batch size  is large. Similar results can be shown for nonconvex settings wherein using larger minibatches improves the convergence to stationary points; albeit at the cost of extra computation. However, several important concerns were raised with respect to generalization and computational performance in large batch nonconvex settings. It was observed that training with extremely large batch was difficult~\citep{keskar2016large, hoffer2017train}. The researchers need to carefully tune training hyper-parameters, like learning rate and momentum, to avoid losing test accuracy \citep{goyal2017accurate, li2017scaling, you2018imagenet, shallue2018measuring}. 

\citet{krizhevsky2014one} introduced some practical schemes for training with large batches. One important rule is to increase the LR (learning rate) by  when batch size is scaled by  since  the variance of the gradient estimation decreases by a factor of . In practice, \citep{krizhevsky2014one} found that  linear scaling works better up to certain batch sizes. To avoid optimization instability due to high learning rate, \citet{goyal2017accurate} proposed to use a highly hand-tuned  learning rate warm-up strategy which starts with a small LR and then gradually increases the LR to a larger value. After warm-up period (usually a few epochs) one switches to the regular LR policy (multi-steps, exponential or polynomial decay etc). Using LR warm-up and linear scaling, \citet{goyal2017accurate} managed to train -50 with batch size 8192 without loss in test accuracy. However, empirical study \citep{shallue2018measuring} shows that learning rate scaling heuristics with the batch size do not hold across all problems or across all batch sizes.

More recently, to reduce hand-tuning of hyperparameters, adaptive learning rates for large batch training garnered significant interests. Several recent works successfully scaled the batch size to large values using adaptive learning rates without degrading the performance, thereby, finishing -50 training on ImageNet in a few minutes \citep{you2018imagenet,iandola2016firecaffe,codreanu2017scale,akiba2017extremely,jia2018highly,smith2017don,martens2015optimizing,devarakonda2017adabatch,mikami2018imagenet,osawa2018second,you2019large,yamazaki2019yet}.
To the best of our knowledge, the fastest training result for -50 on ImageNet is due to \cite{ying2018image}, who achieve 76+\% top-1 accuracy. By using the  optimizer and scaling the batch size to 32K  on a TPUv3 Pod, \citet{ying2018image} was able to train -50 on ImageNet in 2.2 minutes. 

\fi 
\section{Preliminaries}


\paragraph{Notation.}  
For any vector , either  or  are used to denote its  coordinate where .  Let  be the  identity matrix, and let  be its decomposition into column submatrices . For , let  be the block of variables corresponding to the columns of  i.e.,   for
. For any function , we use  to denote the gradient with respect to . For any vectors , we use  and  to denote elementwise square and division operators respectively.
We use  and  to denote -norm and -norm of a vector respectively.

We start our discussion by formally stating the problem setup.  In this paper, we study nonconvex stochastic optimization problems of the form

where  is a smooth (possibly nonconvex) function and  is a probability distribution on the domain . 
Here,  corresponds to model parameters,  is the loss function and  is an unknown data distribution. 

We assume function  is -\emph{smooth} with respect to , i.e.,  there exists a constant  such that

for all . We use  to denote the -dimensional vector of Lipschitz constants. 
We use  and  to denote  and  respectively.  We assume the following bound on the variance in stochastic gradients:  for all  and .    
Furthermore, we also assume  for all  and .  
We use  and  to denote the vectors of standard deviations of stochastic gradient per layer and per dimension respectively.  
Finally, we assume that the gradients are bounded i.e.,  for all ,  and . 
Note that such assumptions are typical in the analysis of stochastic first-order methods (cf. \citep{Ghadimi13,Ghadimi14}). 

Stochastic gradient descent (\sgd) is one of the simplest first-order algorithms for solving problem in Equation~\ref{eq:1}. The update at the  iteration of  is of the following form:

where  is set of  random samples drawn from the  distribution . For very large batch settings, the following is a well-known result for .
\begin{theorem}[\citep{ghadimi2013stochastic}]
With large batch  and using appropriate learning rate, we have the following for the iterates of :

where  is an optimal solution to the problem in \eqref{eq:1} and  is an iterate uniformly randomly chosen from .
\label{thm:sgd-conv}
\end{theorem}
However, tuning the learning rate  in , especially in large batch settings, is difficult in practice. Furthermore, the dependence on  (the maximum of smoothness across dimension) can lead to significantly slow convergence. In the next section, we discuss algorithms to circumvent this issue.  
\section{Algorithms}


In this section, we first discuss a general strategy to adapt the learning rate in large batch settings.  Using this strategy, we discuss two specific algorithms in the later part of the section. Since our primary focus is on deep learning, our discussion is centered around training a -layer neural network.

{\bf General Strategy.} Suppose we use an iterative \emph{base} algorithm  (e.g.  or ) in the small batch setting with the following layerwise update rule:
 
where  is the update made by  at time step . We propose the following two changes to the update for large batch settings:
\begin{enumerate}
\item The update is normalized to unit -norm. This is ensured by modifying the update to the form . Throughout this paper, such a normalization is done layerwise i.e., the update for each layer is ensured to be unit -norm.
\item The learning rate is scaled by  for some function . Similar to the normalization, such a scaling is done layerwise.
\end{enumerate}
Suppose the base algorithm  is , then the modification results in the following update rule:

for all layers  and where  and  are the parameters and the gradients of the  layer at time step .  The normalization modification is similar to one typically used in normalized gradient descent except that it is done layerwise. Note that the modification leads to a biased gradient update; however, in large-batch settings, it can be shown that this bias is small.  It is intuitive  that such a normalization provides robustness to exploding gradients (where the gradient can be arbitrarily large) and plateaus (where the gradient can be arbitrarily small). Normalization of this form essentially ignores the size of the gradient and is particularly useful in large batch settings where the direction of the gradient is largely preserved.

The scaling term involving  ensures that the norm of the update is of the same order as that of the parameter. We found that this typically ensures faster convergence in deep neural networks. In practice, we observed that a simple function of  works well. It is instructive to consider the case where . In this scenario, the overall change in the learning rate is  , which can also be interpreted as an estimate on the inverse of Lipschitz constant of the gradient (see~\eqref{eq:l-const}). We now discuss different instantiations of the strategy discussed above. 
In particular, we focus on two algorithms:  (\ref{Subsection:lars}) and the proposed method,  (\ref{Subsection:lamb}).

\subsection{ Algorithm}
\label{Subsection:lars}

The first instantiation of the general strategy is  algorithm \citep{you2017scaling}, which is obtained by using momentum optimizer as the base algorithm  in the framework.  was earlier proposed for large batch learning for  on ImageNet. In general, it is observed that the using (heavy-ball) momentum, one can reduce the variance in the stochastic gradients at the cost of little bias. The pseudocode for  is provide in Algorithm~\ref{alg:lars}.

\begin{figure}
\begin{minipage}[b]{.48\textwidth}
\begin{algorithm}[H]\small
	\caption{}
	\label{alg:lars}
	\begin{algorithmic}
		\STATE {\bfseries Input:} , learning rate , parameter , scaling function , 
		\STATE Set 
		\FOR{ {\bfseries to} }
		\STATE Draw b samples  from 
        \STATE Compute 
        \STATE 
		\STATE  for all 
		\ENDFOR
	\end{algorithmic}
\end{algorithm}
\end{minipage}\hfill \begin{minipage}[b]{.5\textwidth}
\begin{algorithm}[H]\small
	\caption{}
	\label{alg:lamb}
	\begin{algorithmic}
		\STATE {\bf Input:} , learning rate ,  parameters , scaling function , 
		\STATE Set , 
		\FOR{ {\bf to} }
		\STATE Draw b samples  from .
        \STATE Compute .
		\STATE   
		\STATE  
		\STATE  
        \STATE 
		\STATE Compute ratio 
		\STATE 
		\ENDFOR
	\end{algorithmic}
\end{algorithm}
\end{minipage}
\end{figure}

We now provide convergence analysis for  in general nonconvex setting stated in this paper. For the sake of simplicity, we analyze the case where  and  in Algorithm~\ref{alg:lars}. However, our analysis should extend to the general case as well. We will defer all discussions about the convergence rate to the end of the section.

\begin{theorem}
\label{thm:lars-conv}
Let  \ for all , ,  for all  where . Then for  generated using  (Algorithm~\ref{alg:lars}), we have the following bound

where  is an optimal solution to the problem in \eqref{eq:1} and  is an iterate uniformly randomly chosen from .
\end{theorem}

\subsection{ Algorithm}
\label{Subsection:lamb}

The second instantiation of the general strategy is obtained by using  as the base algorithm .  optimizer is popular in deep learning community and has shown to have good performance for training state-of-the-art language models like . Unlike , the adaptivity of  is two-fold: (i) per dimension normalization with respect to the square root of the second moment used in  and (ii) layerwise normalization  obtained due to layerwise adaptivity. The pseudocode for  is provided in Algorithm~\ref{alg:lamb}. When  and , the algorithm reduces to be Sign \sgd where the learning rate is scaled by square root of the layer dimension \citep{signsgd}.

The following result provides convergence rate for  in general nonconvex settings. Similar to the previous case, we focus on the setting where  and . As before, our analysis extends to the general case; however, the calculations become messy.

\begin{theorem}
\label{thm:lamb-conv}
Let  \ for all , ,  for all , and  for all  where . Then for  generated using  (Algorithm~\ref{alg:lamb}), we have the following bounds:
\begin{enumerate}
    \item When , we have
    

    \item When , we have
    

\end{enumerate}
where  is an optimal solution to the problem in \eqref{eq:1} and  is an iterate uniformly randomly chosen from .
\end{theorem}

{\bf Discussion on convergence rates.} We first start our discussion with the comparison of convergence rate of  with that of  (Theorem~\ref{thm:sgd-conv}). The convergence rates of  and  differ in two ways: (1) the convergence criterion is  as opposed to  in  and (2) the dependence on  and  in the convergence rate. Briefly, the convergence rate of  is better than  when the gradient is denser than curvature and stochasticity. This convergence rate comparison is similar in spirit to the one obtained in \citep{signsgd}. Assuming that the convergence criterion in Theorem~\ref{thm:sgd-conv} and Theorem~\ref{thm:lars-conv} is of similar order (which happens when gradients are fairly dense), convergence rate of  and  depend on  instead of  and are thus, significantly better than that of . A more quantitative comparison is provided in Section~\ref{sec:conv-compare} of the Appendix. The comparison of  (with  = 0) with  is along similar lines. We obtain slightly worse rates for the case where ; although, we believe that its behavior should be better than the case . We leave this investigation to future work.  
\section{Experiments}
\label{sec:experiments}
We now present empirical results comparing  with existing optimizers on two important large batch training tasks:  and -50 training. 
We also compare  with existing optimizers for small batch size () and small dataset (e.g. CIFAR, MNIST) (see Appendix).

{\bf Experimental Setup. } To demonstrate its robustness, we use very minimal hyperparameter tuning for the  optimizer. Thus, it is possible to achieve better results by further tuning the hyperparameters. The parameters  and  in Algorithm~\ref{alg:lamb} are set to  and  respectively in all our experiments; we only tune the learning rate. We use a polynomially decaying learning rate of  in Algorithm~\ref{alg:lamb}), which is the same as in  baseline. This setting also works for all other applications in this paper.
Furthermore, for  and -50  training, we did not tune the hyperparameters of  while increasing the batch size. We use the square root of LR scaling rule to automatically adjust learning rate and linear-epoch warmup scheduling. We use TPUv3 in all the experiments.
A TPUv3 Pod has 1024 chips and can provide more than 100 petaflops performance for mixed precision computing.
To make sure we are comparing with solid baselines, we use grid search to tune the hyper-parameters for 
, ,  ( with weight decay), and . We also tune weight decay for . All the hyperparameter tuning settings are reported in the Appendix. Due to space constraints, several experimental details are relegated to the Appendix.




\subsection{ Training}
We first discuss empirical results for speeding up  training. For this experiment, we use the same dataset as \cite{devlin2018bert}, which is a concatenation of Wikipedia and BooksCorpus with 2.5B and 800M words respectively. We specifically focus on the SQuAD task\footnote{https://rajpurkar.github.io/SQuAD-explorer/} in this paper. 
The F1 score on SQuAD-v1 is used as the accuracy metric in our experiments. All our comparisons are with respect to the baseline  model by \cite{devlin2018bert}. To train , \citet{devlin2018bert} first train the model for 900k iterations using a sequence length of 128 and then switch to a sequence length of 512 for the last 100k iterations. This results in a training time of around 3 days on 16 TPUv3 chips. The baseline  model\footnote{Pre-trained BERT model can be downloaded  from https://github.com/google-research/bert} achieves a F1 score of 90.395. To ensure a fair comparison, we follow the same SQuAD fine-tune procedure of~\cite{devlin2018bert} without modifying any configuration (including number of epochs and hyperparameters). As noted earlier, we could get even better results by changing the fine-tune configuration. For instance, by just slightly changing the learning rate in the fine-tune stage, we can obtain a higher F1 score of 91.688 for the batch size of 16K using . We report a F1 score of 91.345 in Table \ref{table:results}, which is the score obtained for the untuned version. Below we describe two different training choices for training  and discuss the corresponding speedups.

\begin{table}[ht]
\renewcommand{\arraystretch}{1.3}
\caption{ We use the F1 score on SQuAD-v1 as the accuracy metric. The baseline F1 score is the score obtained by the pre-trained model (-Large) provided on 's public repository (as of February 1st, 2019). We use TPUv3s in our experiments. We use the same setting as the baseline: the first 9/10 of the total epochs used a sequence length of 128 and the last 1/10 of the total epochs used a sequence length of 512. All the experiments run the same number of epochs. Dev set means the test data. It is worth noting that we can achieve better results by manually tuning the hyperparameters. The data in this table is collected from the untuned version.}
\centering
 
\begin{tabular}{|c|c|c|c|c|c|}
\hline
Solver & batch size & steps & F1 score on dev set & TPUs & Time\\
\hline
\hline
Baseline & 512 & 1000k & 90.395 & 16 & 81.4h\\
\hline
 & 512 & 1000k & 91.752 & 16 & 82.8h\\
\hline
 & 1k & 500k & 91.761 & 32 & 43.2h\\
\hline
 & 2k & 250k & 91.946 & 64 & 21.4h\\
\hline
 & 4k & 125k & 91.137 & 128 & 693.6m\\
\hline
 & 8k & 62500 & 91.263 & 256 & 390.5m\\
\hline
 & 16k & 31250 & 91.345 & 512 & 200.0m\\
\hline
 & 32k & 15625 & 91.475 & 1024 & 101.2m\\
\hline
\hline
 & 64k/32k & 8599 & 90.584 & 1024 & 76.19m\\
\hline
\end{tabular}
\label{table:results}
\end{table}










For the first choice, we maintain the same training procedure as the baseline except for changing the training optimizer to . We run with the same number of epochs as the baseline but with batch size scaled from 512 to 32K. The choice of 32K batch size (with sequence length 512) is mainly due to memory limits of TPU Pod. Our results are shown in Table \ref{table:results}. By using the  optimizer, we are able to achieve a F1 score of 91.460 in 15625 iterations for a batch size of 32768 (14063 iterations for sequence length 128 and 1562 iterations for sequence length 512).
With 32K batch size, we reduce  training time from 3 days to around 100 minutes. 
We achieved 49.1 times speedup by 64 times computational resources (76.7\% efficiency).
We consider the speedup is great because we use the synchronous data-parallelism. 
There is a communication overhead coming from transferring of the gradients over the interconnect.
For -50, researchers are able to achieve 90\% scaling efficiency because -50 has much fewer parameters (\# parameters is equal to \#gradients) than  (25 million versus 300 million).


To obtain further improvements, we use the {\bf Mixed-Batch Training} procedure with . 
Recall that  training involves two stages: the first 9/10 of the total epochs use a sequence length of 128, while the last 1/10 of the total epochs use a sequence length of 512. 
For the second stage training, which involves a longer sequence length, due to memory limits, a maximum batch size of only 32768 can be used on a TPUv3 Pod. However, we can potentially use a larger batch size for the first stage because of a shorter sequence length. 
In particular, the batch size can be increased to 131072 for the first stage. 
However, we did not observe any speedup by increasing the batch size from 65536 to 131072 for the first stage, thus, we restrict the batch size to 65536 for this stage. 
By using this strategy, we are able to make full utilization of the hardware resources throughout the training procedure. 
Increasing the batch size is able to warm-up and stabilize the optimization process \citep{smith2017don}, but decreasing the batch size brings chaos to the optimization process and can cause divergence.
In our experiments, we found a technique that is useful to stabilize the second stage optimization.
Because we switched to a different optimization problem, it is necessary to re-warm-up the optimization.
Instead of decaying the learning rate at the second stage, we ramp up the learning rate from zero again in the second stage (re-warm-up).
As with the first stage, we decay the learning rate after the re-warm-up phase.
With this method, we only need 8599 iterations and finish  training in 76 minutes (100.2\% efficiency).




\paragraph{Comparison with  and .}
To ensure that our approach is compared to a solid baseline for the  training, we tried three different strategies for tuning : (1)  with default hyperparameters (see \cite{devlin2018bert}) (2)  with the same hyperparameters as , and (3)  with tuned hyperparameters.  stops scaling at the batch size of 16K because it is not able to achieve the target F1 score (88.1 vs 90.4). 
The tuning information of  is shown in the Appendix.
For 64K/32K mixed-batch training, even after extensive tuning of the hyperparameters, we fail to get any reasonable result with  optimizer. 
We conclude that  does not work well in large-batch  training or is at least hard to tune.
We also observe that  performs better than  for all batch sizes (see Table \ref{table:lars_lamb_bert}).

\begin{table}[ht]
\renewcommand{\arraystretch}{1.3}
\caption{  achieves a higher performance (F1 score) than  for all the batch sizes. The baseline achieves a F1 score of 90.390. Thus,  stops scaling at the batch size of 16K.}
\centering

\begin{tabular}{|c|c|c|c|c|c|c|c|}
\hline
Batch Size & 512 & 1K & 2K & 4K & 8K & 16K & 32K\\
\hline
\hline
 & 90.717 & 90.369 & 90.748 & 90.537 & 90.548 & 89.589 & diverge \\
\hline
 & 91.752 & 91.761 & 91.946 & 91.137 & 91.263 & 91.345 & 91.475 \\
\hline
\end{tabular}
\label{table:lars_lamb_bert}
\end{table}

\subsection{ImageNet Training with ResNet-50.}
ImageNet training with ResNet-50 is an industry standard metric that is being used in MLPerf\footnote{https://mlperf.org/}. 
The baseline can get 76.3\% top-1 accuracy in 90 epochs \citep{goyal2017accurate}.
All the successful implementations are based on momentum SGD \citep{he2016deep, goyal2017accurate} or  optimizer \citep{ying2018image, jia2018highly, mikami2018imagenet, you2018imagenet,yamazaki2019yet}.
Before our study, we did not find any paper reporting a state-of-the-art accuracy achieved by , , or  optimizer.
In our experiments, even with comprehensive hyper-parameter tuning, // (with batch size 16K) only achieves 55.38\%/66.04\%/67.27\% top-1 accuracy.
After adding learning rate scheme of \cite{goyal2017accurate},
the top-1 accuracy of // was improved to 72.0\%/73.48\%/73.07\%.
However, they are still much lower than 76.3\%.
The details of the tuning information are in the Appendix.
Table \ref{table:resnet50_acc} shows that  can achieve the target accuracy.
Beyond a batch size of 8K, 's accuracy is higher than the momentum.
's accuracy is also slightly better than .
At a batch size of 32K,  achieves 76.4\% top-1 accuracy while  achieves 76.3\%.
At a batch size of 2K,  is able to achieve 77.11\% top-1 accuracy while  achieves 76.6\%.

\iffalse
\begin{figure*}[tb]
\vspace{5pt}
\centering
\includegraphics[width=0.88\textwidth]{figs/imagenet_resnet50.png}
\caption{ is able to achieve state-of-the-art accuracy in ImageNet/ResNet-50 training for large-batch training. The performance of momentum solver was reported by \citep{goyal2017accurate}.  means adding the learning rate scheme of \cite{goyal2017accurate} to : (1) 5-epoch warmup to stablize the initial stage; and (2) multiply the learning rate by 0.1 at 30th, 60th, and 80th epoch. The target accuracy is around 0.763 \citep{goyal2017accurate}. All the adaptive solvers were comprehensively tuned. The tuning information was in the appendix of this paper.}
\label{fig:resnet50_acc}
\vspace{-10pt}
\end{figure*}
\fi

\begin{table}[ht]
\renewcommand{\arraystretch}{1.3}
\caption{ Top-1 validation accuracy of ImageNet/-50 training at the batch size of 16K (90 epochs). The performance of momentum was reported by \citep{goyal2017accurate}. + means adding the learning rate scheme of \cite{goyal2017accurate} to the optimizer: (1) 5-epoch warmup to stablize the initial stage; and (2) multiply the learning rate by 0.1 at 30th, 60th, and 80th epoch. The target accuracy is around 0.763 \citep{goyal2017accurate}. All the adaptive solvers were comprehensively tuned. The tuning information was in the Appendix.}
\centering

\begin{tabular}{|c|c|c|c|c|c|}
\hline
optimizer & adagrad/adagrad+ & adam/adam+ & adamw/adamw+ & momentum & lamb \\
\hline
\hline
Accuracy & 0.5538/0.7201 & 0.6604/0.7348 & 0.6727/0.7307 & 0.7520 & 0.7666  \\
\hline
\end{tabular}
\label{table:resnet50_acc}
\end{table}

\subsection{Hyperparameters for scaling the batch size}
For  and ImageNet training, we did not tune the hyperparameters of  optimizer when increasing the batch size. We use the square root LR scaling rule and linear-epoch warmup scheduling to automatically adjust learning rate. The details can be found in Tables \ref{table:hyper_parameters} and \ref{table:resnet_hyper_parameters}

\begin{table}[ht]
\renewcommand{\arraystretch}{1.3}
\caption{Untuned  for  training across different batch sizes (fixed \#epochs). We use square root LR scaling and linear-epoch warmup. For example, batch size 32K needs to finish 15625 iterations. It uses 0.215625 = 3125 iterations for learning rate warmup. 's baseline achieved a F1 score of 90.395. We can achieve an even higher F1 score if we manually tune the hyperparameters.}

\centering
\begin{tabular}{|c|c|c|c|c|c|c|c|}
\hline
Batch Size & 512 & 1K & 2K & 4K & 8K & 16K & 32K\\
\hline
\hline
Learning Rate &  &  &  &  &  &  & \\
\hline
Warmup Ratio &  &  &  &  &  &  & \\
\hline
F1 score & 91.752 & 91.761 & 91.946 & 91.137 & 91.263 & 91.345 & 91.475 \\
\hline
Exact Match & 85.090 & 85.260 & 85.355 & 84.172 & 84.901 & 84.816 & 84.939 \\
\hline
\end{tabular}
\label{table:hyper_parameters}
\end{table}

\begin{table}[ht]
\renewcommand{\arraystretch}{1.3}
\caption{Untuned  for ImageNet training with -50 for different batch sizes (90 epochs). We use square root LR scaling and linear-epoch warmup. The baseline \cite{goyal2017accurate} gets 76.3\% top-1 accuracy in 90 epochs. Stanford DAWN Bench \citep{coleman2017dawnbench} baseline achieves 93\% top-5 accuracy.  achieves both of them.  can achieve an even higher accuracy if we manually tune the hyperparameters.}
\centering
\begin{tabular}{|c|c|c|c|c|c|c|c|}
\hline
Batch Size & 512 & 1K & 2K & 4K & 8K & 16K & 32K\\
\hline
\hline
Learning Rate &  &  &  &  &  &  & \\
\hline
Warmup Epochs & 0.3125 & 0.625 & 1.25 & 2.5 & 5 & 10 & 20\\
\hline
Top-5 Accuracy & 0.9335 & 0.9349 & 0.9353 & 0.9332 & 0.9331 & 0.9322 & 0.9308 \\
\hline
Top-1 Accuracy & 0.7696 & 0.7706 & 0.7711 & 0.7692 & 0.7689 & 0.7666 & 0.7642 \\
\hline
\end{tabular}
\label{table:resnet_hyper_parameters}
\end{table} 
\section{Conclusion}
Large batch techniques are critical to speeding up deep neural network training. In this paper, we propose the  optimizer, which supports adaptive elementwise updating and layerwise learning rates. Furthermore,  is a general purpose optimizer that works for both small and large batches. We also provided theoretical analysis for the  optimizer, highlighting the cases where it performs better than standard .  achieves a better performance than existing optimizers for a wide range of applications.  By using , we are able to scale the batch size of  pre-training to 64K without losing accuracy, thereby, reducing the  training time from 3 days to around 76 minutes.  is also the first large batch adaptive solver that can achieve state-of-the-art accuracy on ImageNet training with -50. 
\vspace{-2mm}
\section{Acknowledgement}
\vspace*{-3mm}
We want to thank the comments from George Dahl and Jeff Dean.
We want to thank Michael Banfield, Dehao Chen, Youlong Cheng, Sameer Kumar, and Zak Stone for TPU Pod support.
%
 

\bibliography{iclr2020_conference}
\bibliographystyle{iclr2020_conference}

\appendix
\section*{Appendix}

\section{Proof of Theorem~\ref{thm:lars-conv}}

\begin{proof}
We analyze the convergence of  for general minibatch size here. Recall that the update of   is the following

for all . For simplicity of notation, we reason the 

Since the function  is -smooth, we have the following:

The first inequality follows from the lipschitz continuous nature of the gradient. Let . Then the above inequality can be rewritten in the following manner:


Using Cauchy-Schwarz inequality in the above inequality, we have:


Taking expectation, we obtain the following:

Summing the above inequality for  to  and using telescoping sum, we have the following inequality:
 
Rearranging the terms of the above inequality, and dividing by , we have:


\end{proof}

\section{Proof of Theorem~\ref{thm:lamb-conv}}

\begin{proof}
We analyze the convergence of  for general minibatch size here. Recall that the update of   is the following

for all . For simplicity of notation, we reason the 

Since the function  is -smooth, we have the following:

The above inequality simply follows from the lipschitz continuous nature of the gradient. We bound term  in the following manner:

This follows from the fact that  and . If , then  can be bounded as follows:

The rest of the proof for  is similar to argument for the case , which is shown below. Taking expectation, we have the following:

Using the bound on the probability that the signs differ, we get:

Substituting the above bound on  in \eqref{eq:lamb-conv-eq1}, we have the following bound:

Summing the above inequality for  to  and using telescoping sum, we have the following inequality:
 


Rearranging the terms of the above inequality, and dividing by , we have:

\end{proof}


\section{Comparison of Convergence Rates of  and }
\label{sec:conv-compare}

Inspired by the comparison used by \citep{signsgd} for comparing SIGN  with , we define the following quantities:

Then  convergence rate can be written in the following manner:

If  and  then  (i.e., gradient is more denser than curvature or stochasticity), we gain over . Otherwise, 's upper bound on convergence rate is better.


\begin{figure}
\begin{minipage}[b]{.48\textwidth}

\begin{algorithm}[H]\small
	\caption{N-LAMB}
	\label{alg:nlamb}
	\begin{algorithmic}
		\STATE {\bf Input:} , learning rate ,  parameters , scaling function , , parameters 
		\STATE Set , 
		\FOR{ {\bf to} }
		\STATE Draw b samples  from .
        \STATE Compute .
\STATE   
		\STATE 
		\STATE  
		\STATE 
		\STATE Compute ratio 
		\STATE 
		\ENDFOR
	\end{algorithmic}
\end{algorithm}

\end{minipage}\hfill \begin{minipage}[b]{.5\textwidth}

\begin{algorithm}[H]\small
	\caption{NN-LAMB}
	\label{alg:nnlamb}
	\begin{algorithmic}
		\STATE {\bf Input:} , learning rate ,  parameters , scaling function , , parameters 
		\STATE Set , 
		\FOR{ {\bf to} }
		\STATE Draw b samples  from .
        \STATE Compute .
\STATE   
		\STATE 
		\STATE  
\STATE 
		\STATE Compute ratio 
		\STATE 
		\ENDFOR
	\end{algorithmic}
\end{algorithm}

\end{minipage}
\end{figure}

\section{N-LAMB: Nesterov Momentum for LAMB}

\cite{sutskever2013importance} report that Nesterov’s accelerated gradient (NAG) proposed by \cite{nesterov1983method} is conceptually and empirically better than the regular momentum method for convex, non-stochastic objectives.
\cite{dozat2016incorporating} incorporated Nesterov’s momentum into Adam optimizer and proposed the Nadam optimizer. Specifically, only the first moment of Adam was modified and the second moment of Adam was unchanged. The results on several applications (Word2Vec, Image Recognition, and LSTM Language Model) showed that Nadam optimizer improves the speed of convergence and the quality of the learned models. We also tried using Nesterov’s momentum to replace the regular momentum of LAMB optimizer's first moment. In this way, we got a new algorithm named as N-LAMB (Nesterov LAMB). The complete algorithm is in Algorithm \ref{alg:nlamb}. We can also Nesterov’s momentum to replace the regular momentum of LAMB optimizer's second moment. We refer to this algorithm as NN-LAMB (Nesterov's momentum for both the first moment and the second moment). The details of NN-LAMB were shown in Algorithm \ref{alg:nnlamb}.

\cite{dozat2016incorporating} suggested the best performance of Nadam was achieved by  = 0.975,  = 0.999, and  = 1e-8. We used the same settings for N-LAMB and NN-LAMB. We scaled the batch size to 32K for ImageNet training with ResNet-50. Our experimental results show that N-LAMB and NN-LAMB can achieve a comparable accuracy compared to LAMB optimizer. Their performances are much better than momentum solver (Figure \ref{fig:n_lamb}).

\begin{figure*}[tb]
\vspace{5pt}
\centering
\includegraphics[width=0.88\textwidth]{figs/n_lamb.png}
\caption{This figure shows N-LAMB and NN-LAMB can achieve a comparable accuracy compared to LAMB optimizer. Their performances are much better than momentum solver. The result of momentum optimizer was reported by \cite{goyal2017accurate}. For Nadam, we use the learning rate recipe of \citep{goyal2017accurate}: (1) 5-epoch warmup to stablize the initial stage; and (2) multiply the learning rate by 0.1 at 30th, 60th, and 80th epoch. The target accuracy is around 0.763 \citep{goyal2017accurate}. We also tuned the learning rate of Nadam in \{1e-4, 2e-4, ..., 9e-4, 1e-3, 2e-3, ..., 9e-3, 1e-2\}.}
\label{fig:n_lamb}
\vspace{-10pt}
\end{figure*}

\section{LAMB with learning rate correction}

There are two operations at each iteration in original Adam optimizer (let us call it adam-correction):


It has an impact on the learning rate by .
According to our experimental results, adam-correction essentially has the same effect as learning rate warmup (see Figure \ref{fig:adam_correct}). The warmup function often was implemented in the modern deep learning system. Thus, we can remove adam-correction from the LAMB optimizer. We did not observe any drop in the test or validation accuracy for BERT and ImageNet training.

\begin{figure*}[tb]
\vspace{5pt}
\centering
\includegraphics[width=0.96\textwidth]{figs/adam_correct.png}
\caption{The figure shows that adam-correction has the same effect as learning rate warmup. We removed adam-correction from the LAMB optimizer. We did not observe any drop in the test or validation accuracy for BERT and ImageNet training.}
\label{fig:adam_correct}
\vspace{-10pt}
\end{figure*}

\section{LAMB with different norms}
We need to compute the matrix/tensor norm for each layer when we do the parameter updating in the LAMB optimizer.
We tried different norms in LAMB optimizer. However, we did not observe a significant difference in the validation accuracy of ImageNet training with ResNet-50. In our experiments, the difference in validation accuracy is less than 0.1 percent (Figure \ref{fig:lamb_norm}). We use L2 norm as the default.

\begin{figure*}[tb]
\vspace{5pt}
\centering
\includegraphics[width=0.96\textwidth]{figs/lamb_norm.png}
\caption{We tried different norms in LAMB optimizer. However, we did not observe a significant difference in the validation accuracy of ImageNet training with ResNet-50. We use L2 norm as the default.}
\label{fig:lamb_norm}
\vspace{-10pt}
\end{figure*}

\section{Regular Batch Sizes for Small Datasets: MNIST and CIFAR-10.}
According to DAWNBench, DavidNet (a custom 9-layer Residual ConvNet) is the fastest model for CIFAR-10 dataset (as of April 1st, 2019)\footnote{https://dawn.cs.stanford.edu/benchmark/CIFAR10/train.html}.
The baseline uses the momentum SGD optimizer.
Table \ref{table:cifar10_davidnet} and Figure \ref{fig:cifar10_davidnet} show the test accuracy of CIFAR-10 training with DavidNet. The PyTorch implementation (momentum SGD optimizer) on GPUs was reported on Standford DAWNBench's website, which achieves 94.06\% in 24 epochs. The Tensorflow implementation (momentum SGD optimizer) on TPU achieves a 93.72\% accuracy in 24 epochs\footnote{https://github.com/fenwickslab/dl\_tutorials/blob/master/tutorial3\_cifar10\_davidnet\_fix.ipynb}. We use the implementation of TensorFlow on TPUs.  optimizer is able to achieve 94.08\% test accuracy in 24 epochs, which is better than other adaptive optimizers and momentum SGD.
Even on the smaller tasks like MNIST training with LeNet,  is able to achieve a better accuracy than existing solvers (Table \ref{table:mnist_results}).


\begin{figure*}[tb]
\vspace{5pt}
\centering
\includegraphics[width=0.9\textwidth]{figs/cifar10_davidnet.png}
\caption{ is better than the existing solvers (batch size = 512). We make sure all the solvers are carefully tuned. The learning rate tuning space of Adam, AdamW, Adagrad and LAMB is \{0.0001, 0.0002, 0.0004, 0.0006, 0.0008, 0.001, 0.002, 0.004, 0.006, 0.008, 0.01, 0.02, 0.04, 0.06, 0.08, 0.1, 0.2, 0.4, 0.6, 0.8, 1, 2, 4, 6, 8, 10, 15, 20, 25, 30, 35, 40, 45, 50\}. The momentum optimizer was tuned by the baseline implementer. The weight decay term of AdamW was tuned by \{0.0001, 0.001, 0.01, 0.1, 1.0\}.}
\label{fig:cifar10_davidnet}
\vspace{-10pt}
\end{figure*}


\begin{table}[ht]
\renewcommand{\arraystretch}{1.3}
\caption{ CIFAR-10 training with DavidNet (batch size = 512). All of them run 24 epochs and finish the training under one minute on one cloud TPU. We make sure all the solvers are carefully tuned. The learning rate tuning space of Adam, AdamW, Adagrad and LAMB is \{0.0001, 0.0002, 0.0004, 0.0006, 0.0008, 0.001, 0.002, 0.004, 0.006, 0.008, 0.01, 0.02, 0.04, 0.06, 0.08, 0.1, 0.2, 0.4, 0.6, 0.8, 1, 2, 4, 6, 8, 10, 15, 20, 25, 30, 35, 40, 45, 50\}. The momentum optimizer was tuned by the baseline implementer. The weight decay term of AdamW was tuned by \{0.0001, 0.001, 0.01, 0.1, 1.0\}.}
\centering
\begin{tabular}{|c|c|c|c|c|c|}
\hline
Optimizer &  &  &  & momentum &  \\
\hline
\hline
Test Accuracy & 0.9074 & 0.9225 & 0.9271 & 0.9372 & 0.9408 \\
\hline
\end{tabular}
\label{table:cifar10_davidnet}
\end{table}




\begin{table}[ht]
\renewcommand{\arraystretch}{1.3}
\caption{ Test Accuracy by MNIST training with LeNet (30 epochs for Batch Size = 1024). The tuning space of learning rate for all the optimizers is \{0.0001, 0.001, 0.01, 0.1\}. We use the same learning rate warmup and decay schedule for all of them.}
\centering
\begin{tabular}{|c|c|c|c|c|c|}
\hline
Optimizer & Momentum & Addgrad &  &  &  \\
\hline
\hline
Average accuracy over 5 runs & 0.9933  & 0.9928 & 0.9936 & 0.9941 & 0.9945  \\
\hline
\end{tabular}
\label{table:mnist_results}
\end{table}

\section{Implementation Details and Additional Results}

There are several hyper-parameters in  optimizer. 
Although users do not need to tune them, we explain them to help users to have a better understanding.
 is used for decaying the running average of the gradient.
 is used for decaying the running average of the square of gradient.
The default setting for other parameters: weight decay rate =0.01, =0.9, =0.999, =1e-6.
We did not tune  and .
However, our experiments show that tuning them may get a higher accuracy.

Based on our experience, learning rate is the most important hyper-parameter that affects the learning efficiency and final accuracy.
\cite{bengio2012practical} suggests that it is often the single most important hyper-parameter and that it always should be tuned.
Thus, to make sure we have a solid baseline, we carefully tune the learning rate of , , , and momentum 

In our experiments, we found that the validation loss is not reliable for large-batch training. A lower validation loss does not necessarily lead to a higher validation accuracy (Figure \ref{fig:not_trust_val_loss}).
Thus, we use the test/val accuracy or F1 score on dev set to evaluate the optimizers.





\begin{figure*}[tb]
\vspace{5pt}
\centering
\includegraphics[width=0.88\textwidth]{figs/do_not_trust_val_loss.png}
\caption{Our experiments show that even the validation loss is not reliable in the large-scale training. A lower validation loss may lead to a worse accuracy. Thus, we use the test/val accuracy or F1 score on dev set to evaluate the optimizers.}
\label{fig:not_trust_val_loss}
\vspace{-10pt}
\end{figure*}



\subsubsection{BERT}
Table \ref{table:adamw_16k} shows some of the tuning information from BERT training with  optimizer.
 stops scaling at the batch size of 16K. The target F1 score is 90.5.  achieves a F1 score of 91.345. The table shows the tuning information of . In Table \ref{table:adamw_16k}, we report the best F1 score we observed from our experiments.

\begin{table}[ht]
\renewcommand{\arraystretch}{1.3}
\caption{  stops scaling at the batch size of 16K. The target F1 score is 90.5.  achieves a F1 score of 91.345. The table shows the tuning information of . In this table, we report the best F1 score we observed from our experiments.}
\centering
\begin{tabular}{|c|c|c|c|c|c|}
\hline
Solver & batch size & warmup steps & LR & last step infomation & F1 score on dev set\\
\hline
\hline
 & 16K & 0.0531250 & 0.0001 & loss=8.04471, step=28126 & diverged\\
\hline
 & 16K & 0.0531250 & 0.0002 & loss=7.89673, step=28126 & diverged\\
\hline
 & 16K & 0.0531250 & 0.0003 & loss=8.35102, step=28126 & diverged\\
\hline
 & 16K & 0.1031250 & 0.0001 & loss=2.01419, step=31250 & 86.034 \\
\hline
 & 16K & 0.1031250 & 0.0002 & loss=1.04689, step=31250 & 88.540 \\
\hline
 & 16K & 0.1031250 & 0.0003 & loss=8.05845, step=20000 & diverged \\
\hline
 & 16K & 0.2031250 & 0.0001 & loss=1.53706, step=31250 & 85.231 \\
\hline
 & 16K & 0.2031250 & 0.0002 & loss=1.15500, step=31250 & 88.110 \\
\hline
 & 16K & 0.2031250 & 0.0003 & loss=1.48798, step=31250 & 85.653 \\
\hline
\end{tabular}
\label{table:adamw_16k}
\end{table}

The loss curves of BERT training by  for different batch sizes are shown in Figure \ref{fig:bert_train_loss}.
We observe that the loss curves are almost identical to each other, which means our optimizer scales well with the batch size. 

\begin{figure*}[tb]
\vspace{5pt}
\centering
\includegraphics[width=1.08\textwidth]{figs/bert_train_loss.png}
\caption{This figure shows the training loss curve of  optimizer. We just want to use this figure to show that  can make the training converge smoothly. Even if we scale the batch size to the extremely large cases, the loss curves are almost identical to each other.}
\label{fig:bert_train_loss}
\vspace{-10pt}
\end{figure*}

The training loss curve of BERT mixed-batch pre-training with LAMB is shown in Figure \ref{fig:bert_lamb_loss}.
This figure shows that LAMB can make the training converge smoothly at the batch size of 64K.

\begin{figure*}[tb]
\vspace{5pt}
\centering
\includegraphics[width=0.88\textwidth]{figs/64k_32k_loss.png}
\caption{This figure shows the training loss curve of LAMB optimizer. This figure shows that LAMB can make the training converge smoothly at the extremely large batch size (e.g. 64K).}
\label{fig:bert_lamb_loss}
\vspace{-10pt}
\end{figure*}

Figure \ref{fig:bert_scaling} shows that we can achieve 76.8\% scaling efficiency by scaling the batch size (49.1 times speedup by 64 times computational resources) and 101.8\% scaling efficiency with mixed-batch (65.2 times speedup by 64 times computational resources)

\begin{figure*}[tb]
\vspace{5pt}
\centering
\includegraphics[width=0.90\textwidth]{figs/bert_scaling.png}
\caption{We achieve 76.8\% scaling efficiency (49 times speedup by 64 times computational resources) and 101.8\% scaling efficiency with a mixed, scaled batch size (65.2 times speedup by 64 times computational resources). 1024-mixed means the mixed-batch training on 1024 TPUs.}
\label{fig:bert_scaling}
\vspace{-10pt}
\end{figure*}





\subsubsection{ImageNet}

Figures \ref{fig:lamb_ratio_1} - \ref{fig:lamb_ratio_200} show the LAMB trust ratio at different iterations for ImageNet training with ResNet-50.
From these figures we can see that these ratios are very different from each other for different layers.
LAMB uses the trust ratio to help the slow learners to train faster.

\subsection{Baseline tuning details for ImageNet training with ResNet-50}
If you are not interested in the baseline tuning details, please skip this section.

\citet{goyal2017accurate} suggested a proper learning rate warmup and decay scheme may help improve the ImageNet classification accuracy. 
We included these techniques in Adam/AdamW/AdaGrad tuning.
Specifically, we use the learning rate recipe of \cite{goyal2017accurate}: (1) 5-epoch warmup to stablize the initial stage; and (2) multiply the learning rate by 0.1 at 30th, 60th, and 80th epoch. The target accuracy is around 76.3\% \citep{goyal2017accurate}.
There techniques help to improve the accuracy of Adam/AdamW/AdaGrad to around 73\%.
However, even with these techniques, Adam/AdamW/AdaGrad stil can not achieve the target validation accuracy.

To make sure our baseline is solid, we carefully tuned the hyper-parameters.
Table \ref{table:imagenet_adagrad_tuning_1} shows the tuning information of standard Adagrad.
Table \ref{table:imagenet_adagrad_tuning_2} shows the tuning information of adding the learning rate scheme of \cite{goyal2017accurate} to standard Adagrad.
Table \ref{table:imagenet_adam_tuning_1} shows the tuning information of standard Adam.
Table \label{table:imagenet_adam_tuning_2} shows the tuning information of adding the learning rate scheme of \cite{goyal2017accurate} to standard Adam.
It is tricky to tune the AdamW optimizer since both the L2 regularization and weight decay have the effect on the performance.
Thus we have four tuning sets.

The first tuning set is based on AdamW with default L2 regularization.
We tune the learning rate and weight decay.
The tuning information is in Figures \ref{table:imagenet_adamw_default_l2_1}, \ref{table:imagenet_adamw_default_l2_2}, \ref{table:imagenet_adamw_default_l2_3}, and \ref{table:imagenet_adamw_default_l2_4}.

The second tuning set is based on AdamW with disabled L2 regularization.
We tune the learning rate and weight decay.
The tuning information is in Figures \ref{table:imagenet_adamw_default_1}, \ref{table:imagenet_adamw_default_2}, \ref{table:imagenet_adamw_default_3}, and \ref{table:imagenet_adamw_default_4}.

Then we add the learning rate scheme of \cite{goyal2017accurate} to AdamW and refer to it as AdamW+.

The third tuning set is based on AdamW+ with default L2 regularization.
We tune the learning rate and weight decay.
The tuning information is Figure \ref{table:imagenet_adam_tuning_l2_1} and \ref{table:imagenet_adam_tuning_l2_2}.

The fourth tuning set is based on AdamW+ with disabled L2 regularization.
We tune the learning rate and weight decay.
The tuning information is in Figures \ref{table:imagenet_adam_tuning_nol2_1}, \ref{table:imagenet_adam_tuning_nol2_2}, \ref{table:imagenet_adam_tuning_nol2_3}.

Based on our comprehensive tuning results, we conclude the existing adaptive solvers do not perform well on ImageNet training or at least it is hard to tune them.



\begin{figure*}[tb]
\vspace{5pt}
\centering
\includegraphics[width=0.88\textwidth]{figs/lamb_ratio_1.png}
\caption{The LAMB trust ratio.}
\label{fig:lamb_ratio_1}
\vspace{-10pt}
\end{figure*}

\begin{figure*}[tb]
\vspace{5pt}
\centering
\includegraphics[width=0.88\textwidth]{figs/lamb_ratio_4.png}
\caption{The LAMB trust ratio.}
\label{fig:lamb_ratio_4}
\vspace{-10pt}
\end{figure*}

\begin{figure*}[tb]
\vspace{5pt}
\centering
\includegraphics[width=0.88\textwidth]{figs/lamb_ratio_10.png}
\caption{The LAMB trust ratio.}
\label{fig:lamb_ratio_10}
\vspace{-10pt}
\end{figure*}

\begin{figure*}[tb]
\vspace{5pt}
\centering
\includegraphics[width=0.88\textwidth]{figs/lamb_ratio_50.png}
\caption{The LAMB trust ratio.}
\label{fig:lamb_ratio_50}
\vspace{-10pt}
\end{figure*}

\begin{figure*}[tb]
\vspace{5pt}
\centering
\includegraphics[width=0.88\textwidth]{figs/lamb_ratio_100.png}
\caption{The LAMB trust ratio.}
\label{fig:lamb_ratio_100}
\vspace{-10pt}
\end{figure*}

\begin{figure*}[tb]
\vspace{5pt}
\centering
\includegraphics[width=0.88\textwidth]{figs/lamb_ratio_200.png}
\caption{The LAMB trust ratio.}
\label{fig:lamb_ratio_200}
\vspace{-10pt}
\end{figure*}

\begin{table}[ht]
\renewcommand{\arraystretch}{1.3}
\caption{The accuracy information of tuning default AdaGrad optimizer for ImageNet training with ResNet-50 (batch size = 16384, 90 epochs, 7038 iterations).}
\centering
\begin{tabular}{|c|c|c|c|c|c|c|}
\hline
Learning Rate & Top-1 Validation Accuracy \\
\hline
\hline
0.0001 & 0.0026855469 \\
\hline
0.001 & 0.015563965 \\
\hline
0.002 & 0.022684732 \\
\hline
0.004 & 0.030924479 \\
\hline
0.008 & 0.04486084 \\
\hline
0.010 & 0.054158527 \\
\hline
0.020 & 0.0758667 \\
\hline
0.040 & 0.1262614 \\
\hline
0.080 & 0.24037679 \\
\hline
0.100 & 0.27357993 \\
\hline
0.200 & 0.458313 \\
\hline
0.400 & {\bf 0.553833} \\
\hline
0.800 & 0.54103595 \\
\hline
1.000 & 0.5489095 \\
\hline
2.000 & 0.47680664 \\
\hline
4.000 & 0.5295207 \\
\hline
6.000 & 0.36950684 \\
\hline
8.000 & 0.31081137 \\
\hline
10.00 & 0.30670166 \\
\hline
12.00 & 0.3091024 \\
\hline
14.00 & 0.3227946 \\
\hline
16.00 & 0.0063680015 \\
\hline
18.00 & 0.11287435 \\
\hline
20.00 & 0.21602376 \\
\hline
30.00 & 0.08315023 \\
\hline
40.00 & 0.0132039385 \\
\hline
50.00 & 0.0009969076 \\
\hline
\end{tabular}
\label{table:imagenet_adagrad_tuning_1}
\end{table}

\begin{table}[ht]
\renewcommand{\arraystretch}{1.3}
\caption{The accuracy information of tuning AdaGrad optimizer for ImageNet training with ResNet-50 (batch size = 16384, 90 epochs, 7038 iterations). We use the learning rate recipe of \citep{goyal2017accurate}: (1) 5-epoch warmup to stablize the initial stage; and (2) multiply the learning rate by 0.1 at 30th, 60th, and 80th epoch. The target accuracy is around 0.763 \citep{goyal2017accurate}.}
\centering
\begin{tabular}{|c|c|c|c|c|c|c|}
\hline
Learning Rate & Top-1 Validation Accuracy \\
\hline
\hline
0.0001 & 0.0011189779 \\
\hline
0.001 & 0.00793457 \\
\hline
0.002 & 0.012573242 \\
\hline
0.004 & 0.019022623 \\
\hline
0.008 & 0.027079264 \\
\hline
0.010 & 0.029012045 \\
\hline
0.020 & 0.0421346 \\
\hline
0.040 & 0.06618246 \\
\hline
0.080 & 0.10970052 \\
\hline
0.100 & 0.13429768 \\
\hline
0.200 & 0.26550293 \\
\hline
0.400 & 0.41918945 \\
\hline
0.800 & 0.5519816 \\
\hline
1.000 & 0.58614093 \\
\hline
2.000 & 0.67252606 \\
\hline
4.000 & 0.70306396 \\
\hline
6.000 & 0.709493 \\
\hline
8.000 & 0.7137858 \\
\hline
10.00 & 0.71797687 \\
\hline
12.00 & 0.7187703 \\
\hline
14.00 & {\bf 0.72007245} \\
\hline
16.00 & 0.7194214 \\
\hline
18.00 & 0.7149251 \\
\hline
20.00 & 0.71293133 \\
\hline
30.00 & 0.70458984 \\
\hline
40.00 & 0.69085693 \\
\hline
50.00 & 0.67976886 \\
\hline
\end{tabular}
\label{table:imagenet_adagrad_tuning_2}
\end{table}

\begin{table}[ht]
\renewcommand{\arraystretch}{1.3}
\caption{The accuracy information of tuning default Adam optimizer for ImageNet training with ResNet-50 (batch size = 16384, 90 epochs, 7038 iterations). The target accuracy is around 0.763 \citep{goyal2017accurate}.}
\centering
\begin{tabular}{|c|c|c|c|c|c|c|}
\hline
Learning Rate & Top-1 Validation Accuracy \\
\hline
\hline
0.0001 & 0.5521 \\
\hline
0.0002 & 0.6089 \\
\hline
0.0004 & 0.6432 \\
\hline
0.0006 & 0.6465 \\
\hline
0.0008 & 0.6479 \\
\hline
0.001 & {\bf 0.6604} \\
\hline
0.002 & 0.6408 \\
\hline
0.004 & 0.5687 \\
\hline
0.006 & 0.5165 \\
\hline
0.008 & 0.4812 \\
\hline
0.010 & 0.3673 \\
\hline
\end{tabular}
\label{table:imagenet_adam_tuning_1}
\end{table}

\begin{table}[ht]
\renewcommand{\arraystretch}{1.3}
\caption{The accuracy information of tuning Adam optimizer for ImageNet training with ResNet-50 (batch size = 16384, 90 epochs, 7038 iterations). We use the learning rate recipe of \citep{goyal2017accurate}: (1) 5-epoch warmup to stablize the initial stage; and (2) multiply the learning rate by 0.1 at 30th, 60th, and 80th epoch. The target accuracy is around 0.763 \citep{goyal2017accurate}.}
\centering
\begin{tabular}{|c|c|c|c|c|c|c|}
\hline
Learning Rate & Top-1 Validation Accuracy\\
\hline
\hline
0.0001 & 0.410319 \\
\hline
0.0002 & 0.55263263 \\
\hline
0.0004 & 0.6455485 \\
\hline
0.0006 & 0.6774495 \\
\hline
0.0008 & 0.6996867 \\
\hline
0.001 & 0.71010333 \\
\hline
0.002 & {\bf 0.73476154} \\
\hline
0.004 & 0.73286945 \\
\hline
0.006 & 0.72648114 \\
\hline
0.008 & 0.72214764 \\
\hline
0.010 & 0.71466064 \\
\hline
0.012 & 0.7081502 \\
\hline
0.014 & 0.6993001 \\
\hline
0.016 & 0.69108075 \\
\hline
0.020 & 0.67997235 \\
\hline
0.040 & 0.58658856 \\
\hline
0.060 & 0.51090497 \\
\hline
0.080 & 0.45174155 \\
\hline
0.100 & 0.40297446 \\
\hline
\end{tabular}
\label{table:imagenet_adam_tuning_2}
\end{table}


\begin{table}[ht]
\renewcommand{\arraystretch}{1.3}
\caption{The accuracy information of tuning default AdamW optimizer for ImageNet training with ResNet-50 (batch size = 16384, 90 epochs, 7038 iterations). The target accuracy is around 0.763 \citep{goyal2017accurate}.}
\centering
\begin{tabular}{|c|c|c|c|c|c|c|}
\hline
learning rate & weight decay & L2 regularization & Top-1 Validation Accuracy\\
\hline
\hline
0.0001 & 0.00001 & default (0.01) & 0.53312176 \\
\hline
0.0002 & 0.00001 & default (0.01) & 0.5542806 \\
\hline
0.0004 & 0.00001 & default (0.01) & 0.48769125 \\
\hline
0.0006 & 0.00001 & default (0.01) & 0.46317545 \\
\hline
0.0008 & 0.00001 & default (0.01) & 0.40903726 \\
\hline
0.001 & 0.00001 & default (0.01) & 0.42401123 \\
\hline
0.002 & 0.00001 & default (0.01) & 0.33870444 \\
\hline
0.004 & 0.00001 & default (0.01) & 0.12339274 \\
\hline
0.006 & 0.00001 & default (0.01) & 0.122924805 \\
\hline
0.008 & 0.00001 & default (0.01) & 0.08099365 \\
\hline
0.010 & 0.00001 & default (0.01) & 0.016764322 \\
\hline
0.012 & 0.00001 & default (0.01) & 0.032714844 \\
\hline
0.014 & 0.00001 & default (0.01) & 0.018147787 \\
\hline
0.016 & 0.00001 & default (0.01) & 0.0066731772 \\
\hline
0.018 & 0.00001 & default (0.01) & 0.010294597 \\
\hline
0.020 & 0.00001 & default (0.01) & 0.008260091 \\
\hline
0.025 & 0.00001 & default (0.01) & 0.008870442 \\
\hline
0.030 & 0.00001 & default (0.01) & 0.0064493814 \\
\hline
0.040 & 0.00001 & default (0.01) & 0.0018107096 \\
\hline
0.050 & 0.00001 & default (0.01) & 0.003540039 \\
\hline
\end{tabular}
\label{table:imagenet_adamw_default_l2_1}
\end{table}


\begin{table}[ht]
\renewcommand{\arraystretch}{1.3}
\caption{The accuracy information of tuning default AdamW optimizer for ImageNet training with ResNet-50 (batch size = 16384, 90 epochs, 7038 iterations). The target accuracy is around 0.763 \citep{goyal2017accurate}.}
\centering
\begin{tabular}{|c|c|c|c|c|c|c|}
\hline
learning rate & weight decay & L2 regularization & Top-1 Validation Accuracy\\
\hline
\hline
0.0001 & 0.0001 & default (0.01) & 0.55489093 \\
\hline
0.0002 & 0.0001 & default (0.01) & {\bf 0.56514484} \\
\hline
0.0004 & 0.0001 & default (0.01) & 0.4986979 \\
\hline
0.0006 & 0.0001 & default (0.01) & 0.47595215 \\
\hline
0.0008 & 0.0001 & default (0.01) & 0.44685873 \\
\hline
0.001 & 0.0001 & default (0.01) & 0.41029868 \\
\hline
0.002 & 0.0001 & default (0.01) & 0.2808024 \\
\hline
0.004 & 0.0001 & default (0.01) & 0.08111572 \\
\hline
0.006 & 0.0001 & default (0.01) & 0.068115234 \\
\hline
0.008 & 0.0001 & default (0.01) & 0.057922363 \\
\hline
0.010 & 0.0001 & default (0.01) & 0.05222575 \\
\hline
0.012 & 0.0001 & default (0.01) & 0.017313639 \\
\hline
0.014 & 0.0001 & default (0.01) & 0.029785156 \\
\hline
0.016 & 0.0001 & default (0.01) & 0.016540527 \\
\hline
0.018 & 0.0001 & default (0.01) & 0.00575765 \\
\hline
0.020 & 0.0001 & default (0.01) & 0.0102335615 \\
\hline
0.025 & 0.0001 & default (0.01) & 0.0060831704 \\
\hline
0.030 & 0.0001 & default (0.01) & 0.0036417644 \\
\hline
0.040 & 0.0001 & default (0.01) & 0.0010782877 \\
\hline
0.050 & 0.0001 & default (0.01) & 0.0037638347 \\
\hline
\end{tabular}
\label{table:imagenet_adamw_default_l2_2}
\end{table}

\begin{table}[ht]
\renewcommand{\arraystretch}{1.3}
\caption{The accuracy information of tuning default AdamW optimizer for ImageNet training with ResNet-50 (batch size = 16384, 90 epochs, 7038 iterations). The target accuracy is around 0.763 \citep{goyal2017accurate}.}
\centering
\begin{tabular}{|c|c|c|c|c|c|c|}
\hline
learning rate & weight decay & L2 regularization & Top-1 Validation Accuracy\\
\hline
\hline
0.0001 & 0.001 & default (0.01) & 0.21142578 \\
\hline
0.0002 & 0.001 & default (0.01) & 0.4289144 \\
\hline
0.0004 & 0.001 & default (0.01) & 0.13537598 \\
\hline
0.0006 & 0.001 & default (0.01) & 0.33803305 \\
\hline
0.0008 & 0.001 & default (0.01) & 0.32611084 \\
\hline
0.001 & 0.001 & default (0.01) & 0.22194417 \\
\hline
0.002 & 0.001 & default (0.01) & 0.1833903 \\
\hline
0.004 & 0.001 & default (0.01) & 0.08256022 \\
\hline
0.006 & 0.001 & default (0.01) & 0.020507812 \\
\hline
0.008 & 0.001 & default (0.01) & 0.018269857 \\
\hline
0.010 & 0.001 & default (0.01) & 0.007507324 \\
\hline
0.012 & 0.001 & default (0.01) & 0.020080566 \\
\hline
0.014 & 0.001 & default (0.01) & 0.010762532 \\
\hline
0.016 & 0.001 & default (0.01) & 0.0021362305 \\
\hline
0.018 & 0.001 & default (0.01) & 0.007954915 \\
\hline
0.020 & 0.001 & default (0.01) & 0.005859375 \\
\hline
0.025 & 0.001 & default (0.01) & 0.009724935 \\
\hline
0.030 & 0.001 & default (0.01) & 0.0019124349 \\
\hline
0.040 & 0.001 & default (0.01) & 0.00390625 \\
\hline
0.050 & 0.001 & default (0.01) & 0.0009969076 \\
\hline
\end{tabular}
\label{table:imagenet_adamw_default_l2_3}
\end{table}

\begin{table}[ht]
\renewcommand{\arraystretch}{1.3}
\caption{The accuracy information of tuning default AdamW optimizer for ImageNet training with ResNet-50 (batch size = 16384, 90 epochs, 7038 iterations). The target accuracy is around 0.763 \citep{goyal2017accurate}.}
\centering
\begin{tabular}{|c|c|c|c|c|c|c|}
\hline
learning rate & weight decay & L2 regularization & Top-1 Validation Accuracy\\
\hline
\hline
0.0001 & 0.01 & default (0.01) & 0.0009765625 \\
\hline
0.0002 & 0.01 & default (0.01) & 0.0009969076 \\
\hline
0.0004 & 0.01 & default (0.01) & 0.0010172526 \\
\hline
0.0006 & 0.01 & default (0.01) & 0.0009358724 \\
\hline
0.0008 & 0.01 & default (0.01) & 0.0022379558 \\
\hline
0.001 & 0.01 & default (0.01) & 0.001566569 \\
\hline
0.002 & 0.01 & default (0.01) & 0.009480794 \\
\hline
0.004 & 0.01 & default (0.01) & 0.0033569336 \\
\hline
0.006 & 0.01 & default (0.01) & 0.0029907227 \\
\hline
0.008 & 0.01 & default (0.01) & 0.0018513998 \\
\hline
0.010 & 0.01 & default (0.01) & 0.009134929 \\
\hline
0.012 & 0.01 & default (0.01) & 0.0022176106 \\
\hline
0.014 & 0.01 & default (0.01) & 0.0040690103 \\
\hline
0.016 & 0.01 & default (0.01) & 0.0017293295 \\
\hline
0.018 & 0.01 & default (0.01) & 0.00061035156 \\
\hline
0.020 & 0.01 & default (0.01) & 0.0022379558 \\
\hline
0.025 & 0.01 & default (0.01) & 0.0017089844 \\
\hline
0.030 & 0.01 & default (0.01) & 0.0014241537 \\
\hline
0.040 & 0.01 & default (0.01) & 0.0020345051 \\
\hline
0.050 & 0.01 & default (0.01) & 0.0012817383 \\
\hline
\end{tabular}
\label{table:imagenet_adamw_default_l2_4}
\end{table}

\begin{table}[ht]
\renewcommand{\arraystretch}{1.3}
\caption{The accuracy information of tuning default AdamW optimizer for ImageNet training with ResNet-50 (batch size = 16384, 90 epochs, 7038 iterations). The target accuracy is around 0.763 \citep{goyal2017accurate}.}
\centering
\begin{tabular}{|c|c|c|c|c|c|c|}
\hline
learning rate & weight decay & L2 regularization & Top-1 Validation Accuracy\\
\hline
\hline
0.0001 & 0.00001 & disable & 0.48917642 \\
\hline
0.0002 & 0.00001 & disable & 0.58152264 \\
\hline
0.0004 & 0.00001 & disable & 0.63460284 \\
\hline
0.0006 & 0.00001 & disable & 0.64849854 \\
\hline
0.0008 & 0.00001 & disable & 0.6598918 \\
\hline
0.001 & 0.00001 & disable & 0.6662801 \\
\hline
0.002 & 0.00001 & disable & {\bf 0.67266846} \\
\hline
0.004 & 0.00001 & disable & 0.6692708 \\
\hline
0.006 & 0.00001 & disable & 0.6573079 \\
\hline
0.008 & 0.00001 & disable & 0.6639404 \\
\hline
0.010 & 0.00001 & disable & 0.65230304 \\
\hline
0.012 & 0.00001 & disable & 0.6505534 \\
\hline
0.014 & 0.00001 & disable & 0.64990234 \\
\hline
0.016 & 0.00001 & disable & 0.65323895 \\
\hline
0.018 & 0.00001 & disable & 0.67026776 \\
\hline
0.020 & 0.00001 & disable & 0.66086835 \\
\hline
0.025 & 0.00001 & disable & 0.65425617 \\
\hline
0.030 & 0.00001 & disable & 0.6476237 \\
\hline
0.040 & 0.00001 & disable & 0.55478925 \\
\hline
0.050 & 0.00001 & disable & 0.61869305 \\
\hline
\end{tabular}
\label{table:imagenet_adamw_default_1}
\end{table}


\begin{table}[ht]
\renewcommand{\arraystretch}{1.3}
\caption{The accuracy information of tuning default AdamW optimizer for ImageNet training with ResNet-50 (batch size = 16384, 90 epochs, 7038 iterations). The target accuracy is around 0.763 \citep{goyal2017accurate}.}
\centering
\begin{tabular}{|c|c|c|c|c|c|c|}
\hline
learning rate & weight decay & L2 regularization & Top-1 Validation Accuracy\\
\hline
\hline
0.0001 & 0.0001 & disable & 0.5033366 \\
\hline
0.0002 & 0.0001 & disable & 0.5949707 \\
\hline
0.0004 & 0.0001 & disable & 0.62561035 \\
\hline
0.0006 & 0.0001 & disable & 0.6545207 \\
\hline
0.0008 & 0.0001 & disable & 0.66326904 \\
\hline
0.001 & 0.0001 & disable & 0.6677043 \\
\hline
0.002 & 0.0001 & disable & {\bf 0.67244464} \\
\hline
0.004 & 0.0001 & disable & 0.6702881 \\
\hline
0.006 & 0.0001 & disable & 0.66033936 \\
\hline
0.008 & 0.0001 & disable & 0.66426593 \\
\hline
0.010 & 0.0001 & disable & 0.66151935 \\
\hline
0.012 & 0.0001 & disable & 0.6545817 \\
\hline
0.014 & 0.0001 & disable & 0.65509033 \\
\hline
0.016 & 0.0001 & disable & 0.6529338 \\
\hline
0.018 & 0.0001 & disable & 0.65651447 \\
\hline
0.020 & 0.0001 & disable & 0.65334064 \\
\hline
0.025 & 0.0001 & disable & 0.655009 \\
\hline
0.030 & 0.0001 & disable & 0.64552814 \\
\hline
0.040 & 0.0001 & disable & 0.6425374 \\
\hline
0.050 & 0.0001 & disable & 0.5988159 \\
\hline
\end{tabular}
\label{table:imagenet_adamw_default_2}
\end{table}

\begin{table}[ht]
\renewcommand{\arraystretch}{1.3}
\caption{The accuracy information of tuning default AdamW optimizer for ImageNet training with ResNet-50 (batch size = 16384, 90 epochs, 7038 iterations). The target accuracy is around 0.763 \citep{goyal2017accurate}.}
\centering
\begin{tabular}{|c|c|c|c|c|c|c|}
\hline
learning rate & weight decay & L2 regularization & Top-1 Validation Accuracy\\
\hline
\hline
0.0001 & 0.001 & disable & 0.4611206 \\
\hline
0.0002 & 0.001 & disable & 0.0076293945 \\
\hline
0.0004 & 0.001 & disable & 0.29233804 \\
\hline
0.0006 & 0.001 & disable & 0.57295734 \\
\hline
0.0008 & 0.001 & disable & 0.5574748 \\
\hline
0.001 & 0.001 & disable & 0.5988566 \\
\hline
0.002 & 0.001 & disable & 0.586263 \\
\hline
0.004 & 0.001 & disable & {\bf 0.62076825} \\
\hline
0.006 & 0.001 & disable & 0.61503094 \\
\hline
0.008 & 0.001 & disable & 0.4697876 \\
\hline
0.010 & 0.001 & disable & 0.619751 \\
\hline
0.012 & 0.001 & disable & 0.54243976 \\
\hline
0.014 & 0.001 & disable & 0.5429077 \\
\hline
0.016 & 0.001 & disable & 0.55281574 \\
\hline
0.018 & 0.001 & disable & 0.5819295 \\
\hline
0.020 & 0.001 & disable & 0.5938924 \\
\hline
0.025 & 0.001 & disable & 0.541097 \\
\hline
0.030 & 0.001 & disable & 0.45890298 \\
\hline
0.040 & 0.001 & disable & 0.56193036 \\
\hline
0.050 & 0.001 & disable & 0.5279134 \\
\hline
\end{tabular}
\label{table:imagenet_adamw_default_3}
\end{table}

\begin{table}[ht]
\renewcommand{\arraystretch}{1.3}
\caption{The accuracy information of tuning default AdamW optimizer for ImageNet training with ResNet-50 (batch size = 16384, 90 epochs, 7038 iterations). The target accuracy is around 0.763 \citep{goyal2017accurate}.}
\centering
\begin{tabular}{|c|c|c|c|c|c|c|}
\hline
learning rate & weight decay & L2 regularization & Top-1 Validation Accuracy\\
\hline
\hline
0.0001 & 0.01 & disable & 0.0009969076 \\
\hline
0.0002 & 0.01 & disable & 0.0008951823 \\
\hline
0.0004 & 0.01 & disable & 0.00095621747 \\
\hline
0.0006 & 0.01 & disable & 0.0012817383 \\
\hline
0.0008 & 0.01 & disable & 0.016886393 \\
\hline
0.001 & 0.01 & disable & 0.038146973 \\
\hline
0.002 & 0.01 & disable & 0.0015258789 \\
\hline
0.004 & 0.01 & disable & 0.0014241537 \\
\hline
0.006 & 0.01 & disable & 0.081441246 \\
\hline
0.008 & 0.01 & disable & 0.028116861 \\
\hline
0.010 & 0.01 & disable & 0.011820476 \\
\hline
0.012 & 0.01 & disable & 0.08138021 \\
\hline
0.014 & 0.01 & disable & 0.010111491 \\
\hline
0.016 & 0.01 & disable & 0.0041910806 \\
\hline
0.018 & 0.01 & disable & 0.0038248699 \\
\hline
0.020 & 0.01 & disable & 0.002746582 \\
\hline
0.025 & 0.01 & disable & 0.011555989 \\
\hline
0.030 & 0.01 & disable & 0.0065104165 \\
\hline
0.040 & 0.01 & disable & 0.016438803 \\
\hline
0.050 & 0.01 & disable & 0.007710775 \\
\hline
\end{tabular}
\label{table:imagenet_adamw_default_4}
\end{table}

\begin{table}[ht]
\renewcommand{\arraystretch}{1.3}
\caption{The accuracy information of tuning AdamW optimizer for ImageNet training with ResNet-50 (batch size = 16384, 90 epochs, 7038 iterations). We use the learning rate recipe of \citep{goyal2017accurate}: (1) 5-epoch warmup to stablize the initial stage; and (2) multiply the learning rate by 0.1 at 30th, 60th, and 80th epoch. The target accuracy is around 0.763 \citep{goyal2017accurate}.}
\centering
\begin{tabular}{|c|c|c|c|c|c|c|}
\hline
learning rate & weight decay & L2 regularization & Top-1 Validation Accuracy\\
\hline
\hline
0.0001 & 0.01 & default (0.01) & 0.0009969076\\
\hline
0.0002 & 0.01 & default (0.01) & 0.0009969076\\
\hline
0.0004 & 0.01 & default (0.01) & 0.0009969076\\
\hline
0.0006 & 0.01 & default (0.01) & 0.0009358724\\
\hline
0.0008 & 0.01 & default (0.01) & 0.0009969076\\
\hline
0.001 & 0.01 & default (0.01) & 0.0009765625\\
\hline
0.002 & 0.01 & default (0.01) & 0.0010172526\\
\hline
0.004 & 0.01 & default (0.01) & 0.0010172526\\
\hline
0.006 & 0.01 & default (0.01) & 0.0010172526\\
\hline
0.008 & 0.01 & default (0.01) & 0.0010172526\\
\hline
0.0001 & 0.001 & default (0.01) & 0.0010172526\\
\hline
0.0002 & 0.001 & default (0.01) & 0.0010172526\\
\hline
0.0004 & 0.001 & default (0.01) & 0.0010172526\\
\hline
0.0006 & 0.001 & default (0.01) & 0.0009969076\\
\hline
0.0008 & 0.001 & default (0.01) & 0.0010172526\\
\hline
0.001 & 0.001 & default (0.01) & 0.0010172526\\
\hline
0.002 & 0.001 & default (0.01) & 0.0010172526\\
\hline
0.004 & 0.001 & default (0.01) & 0.0038452148\\
\hline
0.006 & 0.001 & default (0.01) & 0.011881511\\
\hline
0.008 & 0.001 & default (0.01) & 0.0061442056\\
\hline
\end{tabular}
\label{table:imagenet_adam_tuning_l2_1}
\end{table}

\begin{table}[ht]
\renewcommand{\arraystretch}{1.3}
\caption{The accuracy information of tuning AdamW optimizer for ImageNet training with ResNet-50 (batch size = 16384, 90 epochs, 7038 iterations). We use the learning rate recipe of \citep{goyal2017accurate}: (1) 5-epoch warmup to stablize the initial stage; and (2) multiply the learning rate by 0.1 at 30th, 60th, and 80th epoch. The target accuracy is around 0.763 \citep{goyal2017accurate}.}
\centering
\begin{tabular}{|c|c|c|c|c|c|c|}
\hline
learning rate & weight decay & L2 regularization & Top-1 Validation Accuracy \\
\hline
\hline
0.0001 & 0.0001 & default (0.01) & 0.3665975 \\
\hline
0.0002 & 0.0001 & default (0.01) & 0.5315755 \\
\hline
0.0004 & 0.0001 & default (0.01) & 0.6369222 \\
\hline
0.0006 & 0.0001 & default (0.01) & 0.6760457 \\
\hline
0.0008 & 0.0001 & default (0.01) & 0.69557697 \\
\hline
0.001 & 0.0001 & default (0.01) & 0.7076009 \\
\hline
0.002 & 0.0001 & default (0.01) & {\bf 0.73065186} \\
\hline
0.004 & 0.0001 & default (0.01) & 0.72806805 \\
\hline
0.006 & 0.0001 & default (0.01) & 0.72161865 \\
\hline
0.008 & 0.0001 & default (0.01) & 0.71816 \\
\hline
0.0001 & 0.00001 & default (0.01) & 0.49804688 \\
\hline
0.0002 & 0.00001 & default (0.01) & 0.6287028 \\
\hline
0.0004 & 0.00001 & default (0.01) & 0.6773885 \\
\hline
0.0006 & 0.00001 & default (0.01) & 0.67348224 \\
\hline
0.0008 & 0.00001 & default (0.01) & 0.6622111 \\
\hline
0.001 & 0.00001 & default (0.01) & 0.6468709 \\
\hline
0.002 & 0.00001 & default (0.01) & 0.5846761 \\
\hline
0.004 & 0.00001 & default (0.01) & 0.4868978 \\
\hline
0.006 & 0.00001 & default (0.01) & 0.34969077 \\
\hline
0.008 & 0.00001 & default (0.01) & 0.31193033 \\
\hline
\end{tabular}
\label{table:imagenet_adam_tuning_l2_2}
\end{table}


\begin{table}[ht]
\renewcommand{\arraystretch}{1.3}
\caption{The accuracy information of tuning AdamW optimizer for ImageNet training with ResNet-50 (batch size = 16384, 90 epochs, 7038 iterations). We use the learning rate recipe of \citep{goyal2017accurate}: (1) 5-epoch warmup to stablize the initial stage; and (2) multiply the learning rate by 0.1 at 30th, 60th, and 80th epoch. The target accuracy is around 0.763 \citep{goyal2017accurate}.}
\centering
\begin{tabular}{|c|c|c|c|c|c|c|}
\hline
learning rate & weight decay & L2 regularization & Top-1 Validation Accuracy \\
\hline
\hline
0.0001 & 0.01 & disable & 0.0010172526 \\
\hline
0.0002 & 0.01 & disable & 0.0009765625 \\
\hline
0.0004 & 0.01 & disable & 0.0010172526 \\
\hline
0.0006 & 0.01 & disable & 0.0009969076 \\
\hline
0.0008 & 0.01 & disable & 0.0010172526 \\
\hline
0.001 & 0.01 & disable & 0.0009765625 \\
\hline
0.002 & 0.01 & disable & 0.0009969076 \\
\hline
0.004 & 0.01 & disable & 0.0009969076 \\
\hline
0.006 & 0.01 & disable & 0.0009765625 \\
\hline
0.008 & 0.01 & disable & 0.0010172526 \\
\hline
0.0001 & 0.001 & disable & 0.0009765625 \\
\hline
0.0002 & 0.001 & disable & 0.0010172526 \\
\hline
0.0004 & 0.001 & disable & 0.0010172526 \\
\hline
0.0006 & 0.001 & disable & 0.0010172526 \\
\hline
0.0008 & 0.001 & disable & 0.0010172526 \\
\hline
0.001 & 0.001 & disable & 0.0009969076 \\
\hline
0.002 & 0.001 & disable & 0.0010579427 \\
\hline
0.004 & 0.001 & disable & 0.0016886393 \\
\hline
0.006 & 0.001 & disable & 0.019714355 \\
\hline
0.008 & 0.001 & disable & 0.1329956 \\
\hline
\end{tabular}
\label{table:imagenet_adam_tuning_nol2_1}
\end{table}

\begin{table}[ht]
\renewcommand{\arraystretch}{1.3}
\caption{The accuracy information of tuning AdamW optimizer for ImageNet training with ResNet-50 (batch size = 16384, 90 epochs, 7038 iterations). We use the learning rate recipe of \citep{goyal2017accurate}: (1) 5-epoch warmup to stablize the initial stage; and (2) multiply the learning rate by 0.1 at 30th, 60th, and 80th epoch. The target accuracy is around 0.763 \citep{goyal2017accurate}.}
\centering
\begin{tabular}{|c|c|c|c|c|c|c|}
\hline
learning rate & weight decay & L2 regularization & Top-1 Validation Accuracy \\
\hline
\hline
0.0001 & 0.0001 & disable & 0.28515625 \\
\hline
0.0002 & 0.0001 & disable & 0.44055176 \\
\hline
0.0004 & 0.0001 & disable & 0.56815594 \\
\hline
0.0006 & 0.0001 & disable & 0.6234741 \\
\hline
0.0008 & 0.0001 & disable & 0.6530762 \\
\hline
0.001 & 0.0001 & disable & 0.6695964 \\
\hline
0.002 & 0.0001 & disable & 0.70048016 \\
\hline
0.004 & 0.0001 & disable & 0.71698 \\
\hline
0.006 & 0.0001 & disable & 0.72021484 \\
\hline
0.008 & 0.0001 & disable & {\bf 0.7223918} \\
\hline
0.010 & 0.0001 & disable &  0.72017413\\
\hline
0.012 & 0.0001 & disable &  0.72058105\\
\hline
0.014 & 0.0001 & disable &  0.7188924\\
\hline
0.016 & 0.0001 & disable &  0.71695966\\
\hline
0.018 & 0.0001 & disable &  0.7154134\\
\hline
0.020 & 0.0001 & disable &  0.71358234\\
\hline
0.025 & 0.0001 & disable &  0.7145386\\
\hline
0.030 & 0.0001 & disable &  0.7114258\\
\hline
0.040 & 0.0001 & disable &  0.7066447\\
\hline
0.050 & 0.0001 & disable &  0.70284015\\
\hline
\end{tabular}
\label{table:imagenet_adam_tuning_nol2_2}
\end{table}


\begin{table}[ht]
\renewcommand{\arraystretch}{1.3}
\caption{The accuracy information of tuning AdamW optimizer for ImageNet training with ResNet-50 (batch size = 16384, 90 epochs, 7038 iterations). We use the learning rate recipe of \citep{goyal2017accurate}: (1) 5-epoch warmup to stablize the initial stage; and (2) multiply the learning rate by 0.1 at 30th, 60th, and 80th epoch. The target accuracy is around 0.763 \citep{goyal2017accurate}.}
\centering
\begin{tabular}{|c|c|c|c|c|c|c|}
\hline
learning rate & weight decay & L2 regularization & Top-1 Validation Accuracy \\
\hline
\hline
0.0001 & 0.00001 & disable & 0.31247965 \\
\hline
0.0002 & 0.00001 & disable & 0.4534912 \\
\hline
0.0004 & 0.00001 & disable & 0.57765704 \\
\hline
0.0006 & 0.00001 & disable & 0.6277669 \\
\hline
0.0008 & 0.00001 & disable & 0.65321857 \\
\hline
0.001 & 0.00001 & disable & 0.6682129 \\
\hline
0.002 & 0.00001 & disable & 0.69938153 \\
\hline
0.004 & 0.00001 & disable & 0.7095947 \\
\hline
0.006 & 0.00001 & disable & 0.710612 \\
\hline
0.008 & 0.00001 & disable & 0.70857745 \\
\hline
0.010 & 0.00001 & disable & 0.7094116 \\
\hline
0.012 & 0.00001 & disable & 0.70717365 \\
\hline
0.014 & 0.00001 & disable & {\bf 0.7109375} \\
\hline
0.016 & 0.00001 & disable & 0.7058309 \\
\hline
0.018 & 0.00001 & disable & 0.7052409 \\
\hline
0.020 & 0.00001 & disable & 0.7064412 \\
\hline
0.025 & 0.00001 & disable & 0.7035319 \\
\hline
0.030 & 0.00001 & disable & 0.6994629 \\
\hline
0.040 & 0.00001 & disable & 0.6972656 \\
\hline
0.050 & 0.00001 & disable & 0.6971232 \\
\hline
\end{tabular}
\label{table:imagenet_adam_tuning_nol2_3}
\end{table} 
\end{document}
