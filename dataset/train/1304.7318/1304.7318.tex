





\ifx\STACS\undefined \documentclass[12pt]{article}\else \documentclass[runningheads,a4paper]{llncs}
\bibliographystyle{splncs}\fi






\ifx\STACS\undefined
\newcommand{\InConfVer}[1]{}
\newcommand{\InFullVer}[1]{#1}
\else
\newcommand{\InConfVer}[1]{#1}
\newcommand{\InFullVer}[1]{}
\fi


\usepackage[breaklinks]{hyperref} \usepackage{graphicx}
\usepackage[active]{srcltx} \InFullVer{\usepackage{fullpage}}\InFullVer{\usepackage{euscript}}

\usepackage{xspace}\usepackage{caption}\usepackage{color}\usepackage{amsmath}\usepackage{paralist}\usepackage{picins}\usepackage{boxedminipage}
\usepackage{enumerate}
\usepackage{picins}
\usepackage{multirow}
\InConfVer{\usepackage{microtype}}
\usepackage[noend]{algorithmic}
\InFullVer{\usepackage{ntheorem}\theoremseparator{.}}
\InConfVer{\def\ProofInAppendix{TRUE}
}

\usepackage{proofapnd}


\InFullVer{\setlength{\textwidth}{6.5in}
\setlength{\topmargin}{0.0in}
\setlength{\headheight}{0in}
\setlength{\headsep}{0.0in}
\setlength{\textheight}{9in}
\setlength{\oddsidemargin}{0in}
\setlength{\evensidemargin}{0in}
}
\newcommand{\WSPD}{\Term{WSPD}\index{well-separated pair
      decomposition}\xspace}

\newcommand{\PRPOTSAT}{\Term{PRPOT3SAT}\xspace}
\newcommand{\ThreeSAT}{\Term{3SAT}\xspace}
\newcommand{\True}{\Term{TRUE}\xspace}
\newcommand{\False}{\Term{FALSE}\xspace}
\newcommand{\NN}{\Term{NN}\xspace}
\newcommand{\LowerBoundedCenter}   {\PStyle{{Lower{}Bounded{}Center}}\xspace}
\newcommand{\kCenter} {\PStyle{k-Center}\xspace}
\newcommand{\lbc}{\PStyle{LBC}\xspace}
\newcommand{\lbcnn}   {\PStyle{{LBCNN}}\xspace}
\newcommand{\ConstrainedCenter}   {\PStyle{{Constrained{}Center}}\xspace}
\newcommand{\CC}   {\PStyle{CC}\xspace}


\newcommand{\xparagraph}[1]{{\smallskip\textsf{#1}}}

\newcommand{\Opt}{\mathrm{opt}}\newcommand{\OptLP}{\mathrm{opt}_{\text{LP}}}
\newcommand{\PTAS}{\Term{P{T}AS}\xspace}

\renewcommand{\th}{\si{th}\xspace}

\newcommand{\scl}{\Delta}\newcommand{\sclC}{\alpha}

\newcommand{\Circle}{C}

\providecommand{\si}[1]{#1}

\newcommand{\AlinaThanksInner}[1]{{Department of Computer Science; University of Illinois; 201 N. Goodwin Avenue; Urbana, IL, 61801, USA; {\tt \si{ene}1\atgen{}\si{uiuc}.\si{edu}}; {\tt
         \si{\url{http://www.cs.uiuc.edu/\string~ene1/}}.} #1}}
\newcommand{\AlinaThanks}[1]{\thanks{\AlinaThanksInner{#1}}}

\newcommand{\BenThanksInner}[1]{{Department of Computer Science;
      University of Illinois; 201 N. Goodwin Avenue; Urbana, IL, 61801, USA; {\tt \si{raichel}2\atgen{}\si{uiuc}.\si{edu}}; {\tt \url{\si{http://www.cs.uiuc.edu/\string~\si{raichel2}}}}. #1}}

\newcommand{\BenThanks}[1]{\thanks{\BenThanksInner{#1}}}

\newcommand{\atgen}{\symbol{'100}}
\newcommand{\SarielThanksInner}[1]{{Department of Computer Science;
      University of Illinois; 201 N. Goodwin Avenue; Urbana, IL,
      61801, USA; {\tt \si{sariel}\atgen{}\si{uiuc.edu}}; {\tt \url{http://www.uiuc.edu/\string~sariel/}.} #1}}
\newcommand{\SarielThanks}[1]{\thanks{\SarielThanksInner{#1}}}

\newcommand{\emphind}[1]{\emph{#1}\index{#1}}
\definecolor{blue25}{rgb}{0,0,0.55}\newcommand{\emphic}[2]{\textcolor{blue25}{\textbf{\emph{#1}}}\index{#2}}

\newcommand{\emphi}[1]{\emphic{#1}{#1}}
\newcommand{\pth}[2][\!]{#1\left({#2}\right)}
\newcommand{\rpth}[2][]{#1\left({#2}\right)}

\definecolor{red25}{rgb}{0.4,0,0.0}

\newcommand{\PStyle}[1]{\textcolor{red25}{\textrm{\textsf{#1}}}}
\newcommand{\PackRegions}       {\PStyle{{Pack{}Regions}}\xspace}
\newcommand{\PackPoints}        {\PStyle{{Pack{}Points}}\xspace}
\newcommand{\PackHGraph}        {\PStyle{{HGraph{}Packing}}\xspace}\newcommand{\PackHalfspaces}    {\PStyle{{Pack{}Halfspaces}}\xspace}
\newcommand{\PackRaysInPlanes}  {\PStyle{{Pack{}Rays{}In{}Planes{}}}\xspace}
\newcommand{\PackPointsInDisks} {\PStyle{{Pack{}Points{}In{}Disks}}\xspace}
\newcommand{\PackRectsInPnts}   {\PStyle{{Pack{}Re{}ct{}s{}In{}Points}}\xspace}
\newcommand{\PackBoxesInPnts}   {\PStyle{{Pack{}Boxes{}In{}Points}}\xspace}
\newcommand{\PackPntsInSkyline} {\PStyle{{Pack{}Pnts{}In{}Skyline}}\xspace}
\newcommand{\PackPntsInRects}   {\PStyle{\si{PackPntsInRects}}\xspace}
\newcommand{\PackPntsInFTri}    {\PStyle{\si{PackPntsInFatTriangs}}\xspace}



\InFullVer{\newtheorem{theorem}{Theorem}[section]
   \newtheorem{lemma}[theorem]{Lemma}\newtheorem{definition}[theorem]{Definition}
   \newtheorem{defn}[theorem]{Definition}
   \newtheorem{corollary}[theorem]{Corollary}

   {\theorembodyfont{\rm} \newtheorem{observation}[theorem]{Observation}}
   {\theorembodyfont{\rm} \newtheorem{remark}[theorem]{Remark}}
}
\InFullVer{
   \newtheorem{problem}[theorem]{Problem}}

\newcommand{\brc}[1]{\left\{ {#1} \right\}}


\newcommand{\ObjSet}{\EuScript{D}}\newcommand{\ObjSetA}{X}

\newcommand{\constA}{\alpha}\newcommand{\constB}{\beta}\newcommand{\constC}{\gamma}

\newcommand{\distChar}{\mathsf{d}}\newcommand{\distX}[2]{\distChar\pth{#1, #2}}

\newcommand{\nRad}{\ell}

\newcommand{\linelab}[1]{\label{line:#1}}\newcommand{\lineref}[1]{Line~\ref{line:#1}}\renewcommand{\algorithmiccomment}[1]{\hspace{2em} //{#1}}
\algsetup{indent=2em}

\newcommand{\distE}[1]{\left\| {#1}  \right\|}

\newcommand{\Disk}{\mathsf{d}}\newcommand{\DiskSet}{\EuScript{D}}

\newcommand{\PlaneSet}{\EuScript{H}}\newcommand{\plane}{\mathsf{h}}

\newcommand{\HalfspaceSet}{\EuScript{S}}\newcommand{\Halfspace}{\mathsf{h}}\newcommand{\RaySet}{\EuScript{R}}\newcommand{\Ray}{\mathsf{r}}

\newcommand{\RectSet}{\EuScript{B}}
\newcommand{\RectSetA}{\EuScript{D}}

\newcommand{\BoxSet}{\EuScript{B}}
\newcommand{\rect}{\mathsf{b}}


\newcommand{\VSetA}{X}\newcommand{\VSetB}{Y}\newcommand{\VSetC}{Z}\newcommand{\VSetD}{U}\newcommand{\VOpt}{V_{\Opt}}

\newcommand{\Inst}{{I}}

\newcommand{\Term}[1]{\textsf{#1}}

\newcommand{\VC}{\Term{VC}\xspace}
\newcommand{\LP}{\Term{LP}\xspace}\newcommand{\HyperLP}{\textsc{Hypergraph-LP}\xspace}
\newcommand{\APX}{\Term{AP{X}}\xspace}\newcommand{\APXHard}{\Term{AP{X}-hard}\xspace}\newcommand{\MAXSNP}{\Term{MAX S{}NP}\xspace}\newcommand{\NPHard}{\Term{NP-Hard}\xspace}\newcommand{\NPH}{\Term{NPH}\xspace}\renewcommand{\P}{\Term{P}\xspace}\newcommand{\NP}{\Term{NP}\xspace}\newcommand{\obj}{b}\newcommand{\hedge}{f}\newcommand{\hedgeA}{z}

\newcommand{\conflict}{h}\newcommand{\conflictSet}{\mathcal{H}}\newcommand{\RSample}{\mathsf{R}}

\newcommand{\graph}{{G}}\newcommand{\hgraph}{\mathsf{G}}\newcommand{\VSet}{\mathsf{V}}\newcommand{\HESet}{\mathsf{E}}

\newcommand{\vertex}{v}\newcommand{\weight}[1]{w\pth{#1}}

\newcommand{\RSet}{\mathcal{C}}\newcommand{\OSet}{\mathcal{O}}

\newcommand{\Energy}{\EuScript{E}}

\newcommand{\EnergyX}[2][\!]{\EuScript{E}\pth[#1]{#2}}

\newcommand{\XSays}[2]{{
      {\fbox{\tt
            #1:} }
      #2
      \marginpar{#1}
      {\fbox{\tt
            end}}
      }
   }
\newcommand{\sariel}[1]{{\XSays{Sariel}{#1}}}
\newcommand{\Sariel}[1]{{\XSays{Sariel}{#1}}}


\newcommand{\cRelax}{\phi}
\newcommand{\forceX}[2][\!]{\rho\pth[#1]{#2}}
\newcommand{\forceY}[2]{\rho_{#1}\pth{#2}}\newcommand{\resistC}{\eta}\newcommand{\resistX}[1]{\resistC\pth{#1}}\newcommand{\resistY}[2]{\resistC\pth{#1, #2}}
\newcommand{\resistZ}[3]{\resistC_{#1}\pth{#2, #3}}
\newcommand{\indep}{\Term{I{N}D{E}P}\xspace}\newcommand{\Union}[2][\!]{\mathsf{U}\pth[#1]{#2}}
\newcommand{\union}[2][\!]{\mathsf{u}\pth[#1]{#2}}
\renewcommand{\Re}{{\rm I\!\hspace{-0.025em} R}}

\newcommand{\mc}{\nu}

\newcommand{\aftermathA}{\par\vspace{-\baselineskip}}

\newcommand{\mba}     {\rule[0.0cm]{0.0cm}{0.36cm}} \newcommand{\mbb}     {\rule[0.0cm]{0.0cm}{0.38cm}} \newcommand{\mbc}     {\rule[0.0cm]{0.0cm}{0.40cm}} \newcommand{\mbd}     {\rule[0.0cm]{0.0cm}{0.42cm}} \newcommand{\mbe}     {\rule[0.0cm]{0.0cm}{0.44cm}} \newcommand{\mbf}     {\rule[0.0cm]{0.0cm}{0.46cm}} \newcommand{\mbg}     {\rule[0.0cm]{0.0cm}{0.48cm}} \newcommand{\mbh}     {\rule[0.0cm]{0.0cm}{0.50cm}} \newcommand{\mbi}     {\rule[0.0cm]{0.0cm}{0.52cm}} \newcommand{\mbj}     {\rule[0.0cm]{0.0cm}{0.54cm}} \newcommand{\MakeBig} {\rule[-.2cm]{0cm}{0.4cm}}
\newcommand{\MakeSBig}{\rule[0.0cm]{0.0cm}{0.38cm}} 

\newcommand{\seclab}[1]{\label{sec:#1}}
\newcommand{\secref}[1]{Section~\ref{sec:#1}}

\newcommand{\problab}[1]{\label{prob:#1}}
\newcommand{\probref}[1]{Problem~\ref{prob:#1}}

\newcommand{\thmlab}[1]{{\label{theo:#1}}}\newcommand{\thmref}[1]{Theorem~\ref{theo:#1}}

\providecommand{\deflab}[1]{\label{def:#1}}
\newcommand{\defref}[1]{Definition~\ref{def:#1}}


\newcommand{\corlab}[1]{\label{cor:#1}}
\newcommand{\corref}[1]{Corollary~\ref{cor:#1}}

\newcommand{\itemlab}[1]{\label{item:#1}}
\newcommand{\itemref}[1]{(\ref{item:#1})}

\newcommand{\apndlab}[1]{\label{apnd:#1}}
\newcommand{\apndref}[1]{Appendix~\ref{apnd:#1}}

\newcommand{\lemlab}[1]{\label{lemma:#1}}
\newcommand{\lemref}[1]{Lemma~\ref{lemma:#1}}

\newcommand{\remlab}[1]{\label{rem:#1}}
\newcommand{\remref}[1]{Remark~\ref{rem:#1}}

\newcommand{\figlab}[1]{\label{figure:#1}}
\newcommand{\figref}[1]{Figure~\ref{figure:#1}}

\newcommand{\obslab}[1]{\label{observation:#1}}
\newcommand{\obsref}[1]{Observation~\ref{observation:#1}}

\newcommand{\TwoFigures}[6]{\begin{tabular}{cc}
       \begin{minipage}{0.48\linewidth}
           \begin{center}
               \centerline{\includegraphics[#1]{{#2}}}
           \end{center}
       \end{minipage}
       &
       \begin{minipage}{0.48\linewidth}
           \centerline{\includegraphics[#4]{{#5}}}
       \end{minipage}
~\1cm]
        \end{minipage}
    \end{tabular}
    \vspace{-1.6cm}

    \begin{tabular}{cc}
        \begin{minipage}{0.6\linewidth}
            ~\-0.5cm]
        (A) & (B) & (C)
    \end{tabular}
    \begin{center}
        \begin{minipage}{0.99\linewidth}
            If the signal arrives on two of the three wires, then
            again, one of the points must be exposed. It is easy to
            verify that in any of these cases, there is a portion of
            the clause between two gates that needs covering, but it
            contains an odd number of points -- thus the points can
            not be covered with disks of radius one.
        \end{minipage}
    \end{center}
    \begin{tabular}{cc}
        \begin{minipage}{0.3\linewidth}
            \includegraphics[page=6,scale=.95]{figs/clause_assignment}
        \end{minipage}
        &
        \begin{minipage}{0.6\linewidth}
            If the signal arrives on all three wires then the points
            of the clause can not be covered, by the same
            argumentation of case (B) above (for the reader's
            enjoyment, we show a different covering pattern in the
            figure).
        \end{minipage}
    \end{tabular}
     \caption{The clause gadget can be satisfied \si{iff} the signal
        arrives on only one wire. }
    \figlab{case:analysis}
\end{figure}



\InsertAppendixOfProofs




\end{document}
