\documentclass[journal, onecolumn]{IEEEtran}

\usepackage{graphicx}    
\usepackage{float}
\usepackage{epstopdf} 
\usepackage{cite}  
\usepackage[labelfont=bf]{caption}
\usepackage{subcaption}

\usepackage{amsmath} 
\usepackage{amssymb} 
\usepackage{amsfonts}

\usepackage{color}

\newtheorem{assumption}{Assumption}
\newtheorem{remark}{Remark}
\newtheorem{lem}{Lemma}
\newtheorem{prop}{Proposition}
\newtheorem{pf}{Proof}

\newcommand{\overbar}[1]{\mkern 1.5mu\overline{\mkern-1.5mu#1\mkern-1.5mu}\mkern 1.5mu}
    
\newcommand{\stas}[1]{{\color{red} #1}}    
\newcommand{\rea}{\mathrm{R}}
\newcommand{\img}{\mathrm{i}}
\newcommand{\col}[1]{\mathrm{col}(#1)}
\newcommand{\iinb}{i\in\overbar{N}}
\newcommand{\hth}{\hat{\theta}}
\newcommand{\tth}{\tilde{\theta}}
    
\begin{document}

\title{Improved Transients in Multiple Frequencies Estimation via Dynamic Regressor Extension and Mixing} 
\author{Stanislav~Aranovskiy,
        Alexey~Bobtsov,
				Romeo~Ortega,
        Anton~Pyrkin\thanks{Stanislav Aranovskiy is with the NON-A team, INRIA-LNE, Parc Scientifique de la Haute Borne 40, avenue Halley Bat.A, Park Plaza, 59650 Villeneuve d'Ascq, France}		
\thanks{Stanislav Aranovskiy, Alexey Bobtsov and Anton Pyrkin are with the Department of Control Systems and Informatics, ITMO University, Kronverkskiy av. 49, Saint Petersburg, 197101, Russia :  {\tt\small aranovskiysv@niuitmo.ru}}
\thanks{Romeo Ortega is with the LSS-Supelec, 3, Rue Joliot-Curie, 91192 Gif--sur--Yvette, France : {\tt\small ortega@lss.supelec.fr}}
}

\maketitle 

\begin{abstract}                A problem of performance enhancement for multiple frequencies estimation is studied. First, we consider a basic gradient-based estimation approach with global exponential convergence. Next, we apply dynamic regressor extension and mixing technique to improve transient performance of the basic approach and ensure non-strict monotonicity of estimation errors. Simulation results illustrate benefits of the proposed solution.
\end{abstract}



\section{Introduction}
A problem of frequency identification for sinusoidal signals attracts researchers' attention both in control and signal processing communities due to its practical importance. Indeed, frequency identification methods are widely used in fault detection systems \cite{goupil2010oscillatory}, for periodic disturbance attenuation \cite{landau2011adaptive, bobtsov2011compensation}, in naval applications \cite{belleter2013globally} and so on.

Many online frequency estimation methods are currently available in literature, e.g. a phase-locked loop (PLL) proposed in \cite{wu2003magnitude}, adaptive notch filters \cite{regalia1991improved, mojiri2004adaptive}. Another popular approach is to find a parametrization yielding a linear regression model, which parameters are further identified with pertinent estimation techniques, see \cite{xia, chen2014robust, fedele2014frequency}. However, the most of online methods are focused on stability studies and local or global convergence analysis; transients performance is not usually considered and is only demonstrated with simulations. On the other hand, it is well-known that many gradient-based estimation methods can exhibit poor transients even for relatively small number of estimated parameters, the transients can oscillate or even display a peaking phenomena. A method to increase frequency estimation performance with adaptive band-pass filters was proposed in \cite{Aranovskiy2015Cascade} but for a single frequency case only. Thus, the problem of performance improvement for multiple frequencies estimation remains open.

A novel way to improve transient performance for linear regression parameters estimation was proposed in \cite{Aranovskiy2015DREM}; the approach is based on extension and mixing of the original vector regression in order to obtain a set of scalar equations. In this paper we apply this approach to the problem of multiple frequencies estimation. It is shown that under some reasonable assumptions and neglecting fast-decaying terms, \emph{non-strict monotonicty} can be provided for estimates of parameters avoiding any oscillatory or peaking behavior. 

The paper is organized as follows. First, in Section \ref{sec:PS} a multiple frequencies estimation problem is stated. A basic method to solve the problem is presented in Section \ref{sec:basic}. Next, in Section \ref{sec:DREM} we consider dynamic regressor extension and mixing (DREM) procedure and apply it to the previously proposed method. Illustrative results are given in Section \ref{sec:sim} and the paper is wrapped up with Conclusion.

\section{Problem Statement} \label{sec:PS}
Consider the measured scalar signal 

where  is time, , , and  are unknown amplitudes, phases, and frequencies, respectively, ,  is the number of the frequencies in the signal. 

\begin{assumption} \label{as:as1}
All the frequencies , , are distinguished, i.e.

\end{assumption}

\begin{remark} \label{rem:theta}
The signal \eqref{eq:u} can be seen as an output of a marginally stable linear signal generator 

where  and . The characteristic polynomial of the matrix  is given by 

where the parameters  are such that roots of the polynomial  are , where , . Obviously, given a vector , the frequencies can be univocally (up to numerical procedures accuracy) defined, and \emph{vice versa}. Thus, in many multiple frequencies estimation methods the vector  is identified instead of separate frequencies values. In our paper we follow this approach and assume that the frequencies are estimated if the vector  is obtained. The problem of \emph{direct} frequency identification is considered, for example, in  \cite{Pin2015Direct}.
\end{remark}

\emph{Frequencies Estimation Problem.} The goal is to find mappings  and , such that the following estimator 

ensures


\section{A Basic frequencies identification method} \label{sec:basic}
In this section we consider a multiple frequencies estimation method, proposed in \cite{aranovskiy2010identification} and further extended in \cite{bobtsov2012switched, pyrkin2015estimation}. This method is based on State-Variable Filter (SVF) approach, see \cite{young1981parameter, Garnier2003SVF}. 

\begin{lem} \label{lem:svf}
Consider the following SVF

where ,

, , are coefficients of a Hurwitz polynomial 

Define

Then the following holds:

where 

 is defined in Remark \ref{rem:theta}, and  is an exponentially decaying term. \end{lem}
The proof is straightforward and follows the proof presented in \cite{pyrkin2015estimation}.

Using Lemma \ref{lem:svf} we can propose a multiple frequencies estimator.

\begin{prop} \label{prop:estim}
Consider the signal \eqref{eq:u} satisfying Assumption \ref{as:as1}, the SVF \eqref{eq:SVF}, and the signals  and , defined by \eqref{eq:y} and \eqref{eq:phi}, respectively. Then the estimator

where , , ensures the goal \eqref{eq:goal}. Moreover, the estimation error  converges to zero exponentially fast. \end{prop}

\begin{remark}
	The proposed estimator can be also written in form \eqref{eq:estim} as (the argument of time is omitted):
	
\end{remark}

\emph{Sketch of the proof.} The proof of Proposition \ref{prop:estim} follows the proof given in \cite{pyrkin2015estimation}. Substituting \eqref{eq:reg}, it is easy to show that the estimation error  obeys the following differential equation

where  is bounded and exponentially decays. Since signal \eqref{eq:u} consists of  sinusoidal components with distinguished frequencies, the vector  satisfies \emph{persistant excitation} condition \cite{sastry2011adaptive}, that is 

for some  and for all , which will be denoted as . Thus, the linear time-varying system \eqref{eq:tildtheta} is exponentially stable and 
 

The estimation algorithm \eqref{eq:hattheta} ensures global exponential convergence of , but do not guarantee performance transients. It is known from practice that for  behavior of the estimator \eqref{eq:hattheta} becomes oscillatory and can exhibit peaking phenomena. However, these limitations can be overcome with DREM technique presented in the next section.

\section{Enchancing the basic algorithm via DREM procedure} \label{sec:DREM}
In this section we first present the DREM procedure proposed in \cite{Aranovskiy2015DREM}, and then apply it to the basic frequencies estimation algorithm studied in Section \ref{sec:basic}.

\subsection{Dynamic Regressor Extension and Mixing}
 Consider the basic linear regression 

where  and  are measurable bounded signals and  is the vector of unknown constant parameters to be estimated. The standard gradient estimator, equivalent to \eqref{eq:hattheta},

with a positive definite adaptation gain  yields the error equation

where  is the parameters estimation error.

We propose the following dynamic regressor extension and mixing procedure. The first step in DREM is to introduce  {\em linear, -stable} operators , whose output, for any bounded input, may be decomposed as

with  is a (generic) exponentially decaying term. For instance, the operators  may be simple, exponentially stable {\em LTI filters} of the form

with , ; in this case  accounts for the effect of initial conditions of the filters. Another option of interest are {\em delay operators}, that is

where .

Now, we apply these operators to the regressor equation \eqref{eq:regrho} to get the filtered regression\footnote{To simplify the presentation in the sequel we will neglect the  terms. However, it is incorporated in the analysis and proofs given in \cite{Aranovskiy2015DREM}.}


Combining the original regressor equation \eqref{eq:regrho} with the  filtered regressors we can construct the extended regressor system

where  and  are defined as

Note that, because of the --stability assumption of ,  and  are bounded. Premultiplying \eqref{eq:RM} by the {\em adjunct matrix} of  we get  scalar regressions of the form 

with , where we defined the determinant of  as

and the vector 


\begin{prop} \label{prop:DREM}
Consider the --dimensional linear regression \eqref{eq:regrho}, where  and  are known, bounded functions of time and  is a vector of unknown parameters. Introduce   linear, --stable operators  verifying \eqref{eq:H}. Define the vector  and the matrix  as given in \eqref{eq:ReMe}. 
Next consider the estimator

where ,  and  are defined in \eqref{eq:psim} and \eqref{eq:R}, respectively. The following implications holds:

Moreover, if , then  tends to zero exponentially fast. \end{prop}

\begin{remark}
It is well--known \cite{sastry2011adaptive} that the zero equilibrium of the linear time--varying system \eqref{eq:tilder} is (uniformly) exponentially stable if and only if the regressor vector . However, the implication \eqref{eq:notL2} proposes a novel criterion for \emph{assymptotic} convergence which is not necessary uniform for . This criterion, namely , is established not for the regressor  itself, but for a determinant of the extended matrix , and do not coincide with the condition . For more details and illustrative examples see \cite{Aranovskiy2015DREM}.
\end{remark}

\begin{remark} \label{rem:mono}
It is easy to show that error dynamics is given by

It follows that all the transients are \emph{non-strictly monotonic} and do not exhibit oscillatory behavior. 
\end{remark}

\subsection{Applying DREM for frequencies estimation}

Following the proposed procedure we introduce  linear, --stable delay operators ,  , where  and  for , and define  filtered signals

Next coupling these signals with  and  we construct
	
where  is a  vector and  is a  matrix. Defining 

and

we result with a set of  scalar equations

Next the basic differentiator  is replaced with

where , .

Following the Proposition \ref{prop:DREM} and Remark \ref{rem:mono}, we are now in position to establish the main result.

\begin{prop} \label{prop:DREMEstim}
Consider the signal \eqref{eq:u} and the SVF \eqref{eq:SVF}. Define  and  as \eqref{eq:y} and \eqref{eq:phi}, respectively. Choose  parameters ,  and compute  and  as \eqref{eq:filtphiy},\eqref{eq:YePhie}. If the parameters  are chosen such that , where  is defined in \eqref{eq:psiphi}, then  estimation algorithm \eqref{eq:estimdrem} with  defined in \eqref{eq:Yphi} guarantees for 
\begin{itemize}
	\item ;
	\item  is non-strictly monotonic and  is non-increasing.
\end{itemize}
Moreover, if , then  converges to  exponentially fast. \end{prop}

The main novelty of Proposition \ref{prop:DREMEstim} in compare with the basic algorithm given in Proposition \ref{prop:estim} consists in guaranteed non-strict monotonicity of the transients . Obviously, the second statement of Proposition \ref{prop:DREMEstim} is only valid neglecting exponentially decaying terms in SVF transients, namely  in \eqref{eq:reg}. However, these transients depend on our choice of SVF matrix  in \eqref{eq:SVF}, and, practically, are significantly faster then the estimation process.
\section{An Example} \label{sec:sim}
As an illustrative example we consider the case , i.e.

First we are to choose the tuning parameters
\begin{itemize}
\item SVF \eqref{eq:SVF} with the characteristic polynomial of the matrix 

where ;
\item the linear delay operator , where ;
\item the tunning gains . 
\end{itemize}

Next we construct ,  ,  as \eqref{eq:y}, , and the matrices




Applicability of the DREM procedure is stated in the following proposition.
\begin{prop} \label{prop:delN2}
The condition

is sufficient to ensure .\end{prop}
\begin{remark}
The condition \eqref{eq:N2cond} is not, actually, restrictive, since in many practical scenarios it is reasonable to assume a known upper bound, i.e.  . Then \eqref{eq:N2cond} is satisfied for .
\end{remark}
\begin{pf}
Neglecting exponentially decaying terms, for the states of SVF we have

where the parameters  and  depend on the choice of  and parameters of the signal \eqref{eq:uN2}.

Define the function

Tedious but straightforward trigonometric computations yield
 
where the \emph{linear} term is

 is a bounded periodic term and  is a constant. The condition  is equivalent to  as , that is satisfied if . Noting that 

follows from \eqref{eq:N2cond} and recalling Assumption \ref{as:as1} complete the proof. \end{pf}

It is worth noting that condition  implies that  is not a period of the signals , , or a half of period if the signals have half-wave symmetry, ; otherwise the matrix  is singular for all . The inequality \eqref{eq:N2cond} guarantees that  is smaller then a have of the smallest period among sinusoidal components with the frequencies ; it is sufficient but conservative estimate. 

\begin{figure}[t]
	\centering
	\includegraphics[width=0.48\linewidth]{fig1.eps}
	\caption{Transients of the basic estimator \eqref{eq:hattheta} for the input signal \eqref{eq:SimU} with , .}
	\label{fig:basic}
	\includegraphics[width=0.48\linewidth]{fig2.eps}
	\caption{Transients of the estimator with DREM \eqref{eq:estimdrem} for the input signal \eqref{eq:SimU} with , , .}
	\label{fig:drem}
\end{figure}

The both estimators \eqref{eq:hattheta} and \eqref{eq:estimdrem} were simulated for the input signal

with the following parameters
\begin{itemize}
\item ,  for the estimator \eqref{eq:hattheta};
\item , ,  for the estimator \eqref{eq:estimdrem}.
\end{itemize}
Zero initial conditions are chosen for the both estimators, , that implies . To separate transients of the estimator and of the SVF both the estimators are turned on at  seconds.

Transients  of the estimator \eqref{eq:hattheta} are presented in Fig.\ref{fig:basic}, while transients  of the estimator \eqref{eq:estimdrem} are presented in Fig.\ref{fig:drem}; note the difference in gains  and  and in transient time. Transients of the estimator \eqref{eq:estimdrem} with ,  and different values  are given in Fig. \ref{fig:difgamma} and illustrate the impact of the gains.

\begin{figure}[t]
	\centering
\subcaptionbox{}{\includegraphics[width=0.48\linewidth]{fig3_a.eps}}
\subcaptionbox{}{\includegraphics[width=0.48\linewidth]{{fig3_b.eps}}}
	\caption{Transients comparision of the estimator with DREM \eqref{eq:estimdrem} for the input signal \eqref{eq:uN2} with ,  and different gains.}
	\label{fig:difgamma}
\end{figure}

We also present simulation results for  and  with zero initial condition, SVF \eqref{eq:SVF} with  and , and start time  seconds. The transients  are given in Fig. \ref{fig:N3_1} for the estimator \eqref{eq:hattheta} with 

and in Fig. \ref{fig:N3_2} for the estimator \eqref{eq:estimdrem} with , , ; note the difference in time scales.

\begin{figure}[t]
	\centering
	\includegraphics[width=0.48\linewidth]{fig_N3_1.eps}
	\caption{Transients of the basic estimator \eqref{eq:hattheta} for  with , .}
	\label{fig:N3_1}
\end{figure}

\begin{figure}[t]
	\centering
	\includegraphics[width=0.48\linewidth]{fig_N3_2.eps}
	\caption{Transients of the estimator with DREM \eqref{eq:estimdrem} for  with , , , .}
	\label{fig:N3_2}
\end{figure}

\section{Conclusion}
The problem of transients improving for multiple frequencies estimation was considered. The dynamic regressor extension and mixing (DREM) procedure, which allows to translate the original vector estimation problem to a set of scalar sub-problems, was successfully applied to enhance the basic estimation algorithm;  as a benefit of this translation the non-strict monotonicity can be ensured. Significant transients improvement is illustrated with simulation results. 

\bibliographystyle{IEEETran}
\bibliography{sin_drem}             

\end{document}
