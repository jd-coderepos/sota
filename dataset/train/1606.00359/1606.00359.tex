\documentclass[runningheads]{llncs}
\usepackage{graphicx}
\usepackage{epsfig}
\usepackage{array}
\usepackage{graphics}
\usepackage{amsmath}
\usepackage{algorithm}
\usepackage{algorithmic}

\usepackage{amssymb}



\usepackage{tikz}
\usepackage{epstopdf}
\usepackage{fullpage}
\usepackage{tikz}
\DeclareGraphicsExtensions{.pdf,.jpg,.png}

\date{}
\newtheorem{fact}{Fact}
\newtheorem{obs}{Observation:}
\title{ON STRICTLY CHORDALITY- GRAPHS}
\author{ S.Dhanalakshmi and N.Sadagopan } 
\institute{Indian Institute of Information Technology, Design and Manufacturing, Kancheepuram, Chennai, India. \\
\email{}}


\begin{document}
\maketitle
\pagenumbering{arabic}
\pagestyle{plain}




\begin{abstract}
Strictly Chordality- graphs ( graphs) are graphs which are either cycle free or every induced cycle is exactly , for some fixed . Note that  and  are precisely the Chordal graphs and Chordal Bipartite graphs, respectively. In this paper, we initiate a structural and an algorithmic study of  graphs. 
\\ \\
\textbf{Keywords:} Girth = Chordality = , Minimal vertex separator, Treewidth.
\end{abstract}


\section{Introduction}
The study of graphs with forbidden graph structures has attracted researchers from the field of mathematics and theory of computing. The popular ones are chordal and chordal bipartite graphs. Interestingly, these graphs find applications in computer architecture to factorize sparse matrix \cite{app1}, solving indefinite linear equations \cite{app2} and the study of linear programming \cite{app3}. A graph is chordal if every cycle of length at least 4 has a chord. Chordal graphs were introduced by Hajnal and Suranyi in 1958 \cite{Hajnal}. Dirac \cite{dirac} presented a structural characterization of chordal graphs with respect to minimal vertex separators and showed that chordal graphs are precisely the graph class in which every minimal vertex separator is a clique. A vertex is a \emph{simplicial vertex} if its neighborhood induces a clique. Interestingly, Dirac observed that every chordal graph has a simplicial vertex. Further, Fulkerson and Gross \cite{fulkerson} showed that all chordal graphs have a simplicial ordering (Perfect Elimination Ordering). On the time complexity front, chordal graphs can be recognized in polynomial time \cite{fulkerson,tarjan}.

Like chordal graphs, a related graph class, namely chordal bipartite graph received a considerable attention in the literature. A bipartite graph is chordal bipartite if every cycle of length at least 6 has a chord in it. Similar to chordal graphs, Golumbic and Goss \cite{GolumbicGoss} showed that a graph is chordal bipartite if and only if every minimal edge separator is a complete bipartite graph. Further, they can be recognized in polynomial time due to the existence of perfect edge elimination ordering \cite{GolumbicGoss}.

Both chordal and chordal bipartite graphs have received a good attention in the last four decades due to their nice structural and algorithmic characterizations. We also highlight that many classical combinatorial problems such as Vertex cover \cite{tarjan,gavril}, Clique cover \cite{corneil,Hoang}, Independent set \cite{gavril}, Treewidth \cite{bodlaender,kloks} are polynomial-time solvable when the input is restricted to chordal and chordal bipartite graphs, which are NP-Complete on general graphs. In some sense, these two graphs help to identify the gap between polynomial-time solvable input instances and the input instances that cause NP-Hardness. Other notable combinatorial problems such as Dominating-set \cite{booth,mueller}, Hamiltonian path \cite{colbourn,muellerh} remain NP-Complete on chordal and chordal bipartite graphs. It is important to highlight that chordal and chordal bipartite graphs are well studied graphs in the literature as it is clearly evident from some of the recent results on Join colorings \cite{join}, Contractibility problems \cite{contractibility}, Strong Chromatic index \cite{strong}, Enumeration of minimal dominating sets \cite{enumeration}, Reconfiguration graphs for vertex colourings \cite{reconfiguration} restricted to  chordal and chordal bipartite graphs.



A relook on the definition reveals that chordal graphs (chordal bipartite graphs) are graphs which are either cycle free or every induced cycle is  (induced cycle is  for chordal bipartite graphs). It is natural to ask, what is the graph class which are either cycle free or every induced cycle is  and we call them as Strictly Chordality-5 graphs ( graphs). Interestingly, these graphs have the additional property that the girth (the length of the shortest cycle) equals the chordality (the length of the longest induced cycle). We shall explore this question in a larger dimension and initiate the study of \emph{Strictly Chordality- graphs} ( graphs), girth = chordality = , for some . Thus, in this paper, we shall investigate a structural and an algorithmic study of  graphs and we believe that this investigation has not been done in the literature.\\
\textbf{Our Contributions:} In the context of strictly chordality- graphs, , we show the following results: 
\begin{itemize}
\item[1.] Every minimal vertex separator in  graphs, , is of cardinality at most two.
\item[2.] Every minimal vertex separator in  graphs, , is of cardinality at most , where  is the size of the maximum \emph{cage}. 
\item[3.] We show that in every  graphs, there exists a special vertex or special . Further, we show a special ordering among the vertices and cycles of .
\item[4.] Recognizing  graph can be done in polynomial-time.
\item[5.] We show that every  graphs, , is hamiltonian if and only if it is connected, - pyramid free and -cage free.
\item[6.] Every  graph,  is 2-colorable if  is even and 3-colorable if  is odd.
\item[7.] We establish that tree-width of  graphs is at most two.
\item[8.] We show that minimum fill-in problem is polynomial-time solvable.
\end{itemize}
\textbf{This paper is organized as follows:} We present graph preliminaries in Section 2. Structural observations on  graphs based on minimal vertex separators are addressed in Section 3. We characterize  graphs by establishing an ordering in Section 4.  The algorithmic results like testing a graph, coloring, hamiltonicity, treewidth and minimum fill-in for  graphs are presented in Section 5.
\section{Graph Preliminaries}

Notations used in this paper are as per \cite{golumbicbook,dbwest}. The graphs considered in this paper are simple, undirected, connected and unweighted. Let  be a simple connected graph with the non-empty vertex set  and the edge set = \{\{\}   and  is adjacent to  in  and \}. The  of a vertex  of , (), is the set of vertices adjacent to  in . The degree of the vertex  is .  and  denotes the minimum and maximum degree of a graph , respectively. A graph  is said to be - if . The graph  is called a  of  if    and . The subgraph  of a graph  is said to be  , if for every pair of vertices  and  of , \{\}   if and only if \{\}   and it is denoted by .  is a \emph{path} defined on  such that . For simplicity, we use  to refer to  . The set  denotes the \emph{internal vertices} of the path .  denotes the path on  vertices. A \emph{cycle}  on -vertices is denoted as , where  and . An   is a cycle that is an induced subgraph of . A graph  is said to be \emph{cycle-free} if there is no induced cycle in . A graph  is said to be  if every pair of vertices in  has a path and if a graph is disconnected, it can be divided into disjoint connected  , , where  denotes the set of vertices in the component . Let  be a non-empty subset of  and let  denotes the induced subgraph on . The set  is said to be an \emph{independent set} if every pair of vertices of  is non-adjacent. For , a subset  is a -vertex separator if  and  lies in different connected components of .  is a minimal -vertex separator if there does not exist a -vertex separator . A vertex  of a connected graph  is said to be a \emph{cut vertex}, if  is a disconnected graph. An edge  of a connected graph  is said to be a \emph{cut-edge}, if the deletion of an edge  from  disconnects the graph . 


 
\section{Structural Observations on Strictly Chordality- Graphs}
Recall that, a graph  is said to be a strictly chordality- graph, , if every induced cycle is of length exactly  or  is cycle-free. In this section, we present some structural observations on , graphs with respect to minimal vertex separators.

\begin{lemma}
\label{intersection}
Let  be a connected , graph. For any two induced cycles  and  in , one of the following is true.
\begin{itemize}
\item[(i)] 
\item[(ii)] 
\item[(iii)]  if  is even
\end{itemize}
\end{lemma}
\begin{proof}
On the contrary, assume that there exist induced cycles  and  such that  and  and,  is even and . The only possible cycles satisfying these condition's are; If  is odd, then for every  and if  is even, then for every  and ,   and .  i.e., there exist at least two cycles  and  in  such that both contains a  in common (\emph{see Figure  \ref{sisj}}). Let  be the set of internal vertices of . The graph 
\begin{figure}[h]
\centering
\includegraphics[scale=0.35]{dhana.eps}
\caption{(a) An illustration when , where  and  (b) An example when  and }
\label{sisj}
\end{figure}
\noindent  induces . Note that, the cycle is induced because any chord from  to ,  induces either  or , for any  and . Since , neither  nor  is ,  for any  and , which contradicts the definition of  graphs and hence, the lemma follows.
\end{proof}
                                                                                                                                                                               
                                                                                                                                                                               \noindent Note that the induced cycles  and  in an  graph  is said to have \emph{vertex intersection} if  and \emph{edge intersection} if .
                                                                                                                                                                                                                                                                                                                                                               \begin{corollary}                                                                                                                                                                                                                                                                                                                                                                                                                                                                                                                                       \label{intersectionseparate} Let  be a connected  graph, . For any two induced cycles  and  in ,  either  or , if  is odd and either  or  is  or  or , if  is even.                                                                                                                                                                                   \end{corollary}                                                                                                                                                                                                                                                                                                                                                            \begin{proof}                                                                                                                                                                              Trivially follows from \emph{Lemma \ref{intersection}}.                                                                                                                                                                                   \end{proof}

This corollary acts as a powerful tool to determine the maximum size of the minimal vertex separator in an  graph as well as the structure of minimal vertex separators in  graphs which we shall present next.                                                                                                                                                                                 
                                                                                                                                                                                 
\begin{theorem}
\label{mvssc2k+1}
 Let  be a connected  graph, . The cardinality of every minimal vertex separator of  is at most 2.
\end{theorem}
\begin{proof} On the contrary, assume that there exist a minimal vertex separator  such that  = ,  3. The graph  is a disconnected graph with distinct connected components , . 
Consider the graph  induced on the set . Throughout this proof, when we refer to , we mean the shortest path  where every internal vertex belongs to , . Let  and  be any three vertices in  and let . Since  is a minimal vertex separator, every vertex in  is adjacent to at least one vertex in each component. Thus, for every pair  there exists  and  ( and  are connected components of ). Let  = (),  = (),  = (),  = (),  = () and  = (). Note that if , then  forms an induced  and if , then  forms an induced . We complete this proof using case analysis (\emph{see Table \ref{table:table1}}) by considering the cases where  is independent and not independent.

In each case, we arrive at a contradiction by exhibiting an induced cycle other than . Further, we exhibit two induced cycles  and  with  in common, which contradicts \emph{Corollary \ref{intersectionseparate}}. It follows that our assumption that there exist a minimal vertex separator of size 3 or more is wrong. Thus, the theorem is true for  and hence the super graph  as every induced cycle  is also an induced cycle in . 
\end{proof}

\begin{figure}[h]
\vspace{-0.5cm}
\begin{center}
\includegraphics[scale=0.25]{mvsk1.eps}
\vspace{-0.5cm}
\caption{An illustration for \emph{Theorem \ref{mvssc2k+1}}}
\label{mvsk1}
\end{center}

\end{figure}

\vspace{-1.4cm}
\begin{table}[h]
\centering
\label{pre}
\caption{Possible chords from vertices  and }
\begin{tabular}{|l|l|l|l|} \hline
\textbf{Type A:} &  is adjacent to a vertex in  &
\textbf{Type E:} &  is adjacent to a vertex in  \\
\textbf{Type B:} &  is adjacent to a vertex in  &
\textbf{Type F:} &  is adjacent to a vertex in  \\
\textbf{Type C:} &  is adjacent to a vertex in  &
\textbf{Type G:} &  is adjacent to a vertex in  \\
\textbf{Type D:} &  is adjacent to a vertex in  &
\textbf{Type H:} &  is adjacent to a vertex in  \\ \hline
\end{tabular}
\vspace{-0.8cm}
\end{table}

\begin{table}[H]
\caption{Case analysis for the proof of Theorem \ref{mvssc2k+1}}
\label{table:table1}
\begin{tabular}{|l|l|}
\multicolumn{2}{l}{\textbf{Case 1:}   is independent.  and  form an induced  in .}\\ \hline

\multicolumn{1}{|c|}{\textbf{Case Analysis}} & \multicolumn{1}{|c|}{\textbf{Induced cycles with justification}}\\  \hline 
\textbf{Case 1.1:}  and ; &   forms an induced , a contradiction. \\ 
& 	The cycle is induced by the following sub cases:\\ \cline{2-2} 
&\textbf{Case 1.1a:} Chord of \emph{Type C}.  and , where,  is the least indexed \\
& vertex in  such that ; Note that  and \\
& . , a contradiction. \\ 
&	Similar arguments hold good for chords of \emph{Type A, G} and \emph{E}. \\ \cline{2-2}
&\textbf{Case 1.1b:} Chord of \emph{Type D}.  and , where,  is the least indexed \\
&  vertex in  such that ; , a contradiction (\emph{see Figure 2(a)}). \\ 
&	Similar arguments can be given if chords are of \emph{Type B, F} and \emph{H}.\\
\hline
\textbf{Case 1.2:}  and ; &  forms an induced , a contradiction. \\ 
& The cycle is induced by the following sub cases:\\ \cline{2-2}

&	\textbf{Case 1.2a:} Chord of \emph{Type B}.  and , where,  is the least indexed \\
&	vertex in  such that ; , a contradiction (\emph{see Figure 2(b)}).  \\ 
&	The argument is symmetric for chords of \emph{Type F}.\\ \cline{2-2}
& \textbf{Case 1.2b:} Chord of \emph{Type A} or \emph{G}.  The argument is similar to the \emph{Case 1.1a}. \\ \cline{2-2}
& \textbf{Case 1.2c:} Chord of \emph{Type D} or \emph{H}. The argument is similar to the \emph{Case 1.1b}. \\  \hline

\textbf{Case 1.3:}  and ; &  forms an induced , a contradiction. \\
& The argument for the cycle is induced is symmetric to the \emph{Case 1.2}\\ \hline

\textbf{Case 1.4:}  and  & The induced cycles  and  have  \\
& in common, a contradiction.\\ \hline
\end{tabular}
\vspace{-0.8cm}
\end{table}

\begin{table}[H]
\begin{tabular}{|l|l|}
\multicolumn{2}{l}{\textbf{Case 2:}  is not independent and , .} \\ \hline
\textbf{Case 2.1:}  and  &  forms an induced , a contradiction. \\
&	The cycle is induced by the following sub cases:\\ \cline{2-2}

&	\textbf{Case 2.1a:} Chord of \emph{Type C}.  and , where,  is the least indexed \\
& vertex in  such that ; , a contradiction (\emph{see Figure 2(c)}). \\
&	Similar argument hold good for chord of \emph{Type A}. \\ \cline{2-2}
&	\textbf{Case 2.1b} Chords of \emph{Type E} or \emph{G}.	The argument is similar to the \emph{Case 1.1a}. \\ \cline{2-2}
&	\textbf{Case 2.1c:} Chord of \emph{Type B}.  and , where,  is the least indexed \\
 & vertex in  such that ; , a contradiction. \\ 
 & The argument is symmetric for chord of \emph{Type D}.\\ \cline{2-2}
 & \textbf{Case 2.1d:} Chord of \emph{Type F}.   and , where,  is the largest \\
 & indexed vertex in  such that ; , a contradiction.\\ 
 & Similar argument for \emph{Case H}.\\ \hline
\textbf{Case 2.2:}  and  &  forms an induced , a contradiction. \\ 
&The cycle is induced by the following sub cases:\\ \cline{2-2}

& \textbf{Case 2.2a:} Chord of \emph{Type F}.  and , where,  is the largest indexed \\ 
& vertex in  such that ; , a contradiction (\emph{see Figure 2(d)}). \\ \cline{2-2}
& \textbf{Case 2.2b:} Chord of \emph{Type B}.   and , where, \\ 
 &  is the least indexed vertex in  such that . \\ \cline{2-2}
&\textbf{Case 2.2c} Chords of \emph{Type C} or \emph{D} or \emph{G} or \emph{H:}\\
&  The arguments are similar to the sub cases of \emph{Case 2.1}. \\ \hline
\textbf{Case 2.3:}  and  &  forms an induced , a contradiction. \\
  & The argument is similar to the \emph{Case 2.2}\\ \hline
\textbf{Case 2.4:}  and  &  forms an induced , a contradiction. \\ 
& The cycle is induced by the arguments in \emph{Case 2.2a; Case 2.2b; Case 2.3}\\ \hline
\multicolumn{2}{|l|}{\textbf{Case 3:}  is not independent and , . The argument is similar to \emph{Case 2}.} \\ \hline

\multicolumn{2}{|l|}{\textbf{Case 4:}  is not independent and , . The argument is similar to \emph{Case 2}.} \\ \hline

\multicolumn{2}{l}{\textbf{ }} \\
\multicolumn{2}{l}{\textbf{Case 5:}  is not independent and ,  }\\ \hline
\textbf{Case 5.1:}  and  &  forms an induced , a contradiction. \\ \hline
& The cycle is induced by the following sub cases:\\ \cline{2-2}

& \textbf{Case 5.1a:} Chord of \emph{Type A}.  The argument is similar to the \emph{Case 2.1a.} \\ & Similar arguments can be given if chords are of \emph{Type C, E} and \emph{G}. \\ \cline{2-2}
& \textbf{Case 5.1b:} Chord of \emph{Type B}.  and , where,  is the least indexed \\ 
 &vertex in  such that ; , a contradiction (\emph{see Figure 2(e)}). \\ 
 & Similar arguments hold good for chords of \emph{Type D, F} and \emph{H}. \\ \hline  
\textbf{Case 5.2:}  and  &   forms an induced , a contradiction. \\
& The argument is similar to the \emph{Case 5.1}.\\
 \hline
\textbf{Case 5.3:}  and  &   forms an induced , a contradiction. \\
& The argument is similar to the \emph{Case 5.1}\\
 \hline
\textbf{Case 5.4:}  and  &    forms an induced , a contradiction. \\ 
& The argument is similar to the \emph{Case 5.1b}.\\
 \hline
\multicolumn{2}{|l|}{\textbf{Case 6:}  is not independent and , . The argument is similar to \emph{Case 5}. }\\ \hline
\multicolumn{2}{|l|}{\textbf{Case 7:}  is not independent and , . The argument is similar to \emph{Case 5}. }\\ \hline
\end{tabular}
\end{table}



\begin{lemma}
\label{mvssck}
Let  be a connected  graph, . For any two induced cycles  and : if either  or , then the cardinality of every minimal vertex separator of  is at most 2.
\end{lemma}
\begin{proof}
An argument similar to \emph{Theorem \ref{mvssc2k+1}} establishes this claim. 
\end{proof}

\begin{lemma}
\label{mvsindependent}
Let  be a connected  graph, . If  is a minimal vertex separator of  with , then  is an independent set.
\end{lemma}
\begin{proof}
On the contrary, assume that there exists a minimal vertex separator  such that  and  is not an independent set. Let ,  be the connected components of . Consider the graph  induced on the set . Choose any three vertices, , from  such that either  and  or  and . Since  is a minimal vertex separator, every vertex in  is adjacent to at least one vertex in each component. Thus,  and  exists and these paths create a cycle of length , say  and . Let  be a vertex in  which is adjacent to  in  and  be a vertex in  which is adjacent to  in .
\begin{description}

\item[\textbf{Case (} and \textbf{):}] 
 It is clear that,  forms an induced . Let  and --. Thus,  and . Hence,  forms an induced cycle of length greater than . The cycle is induced because the following cases are not possible by the definition of .

\begin{figure}[h]
\centering
\includegraphics[scale=0.3]{is.eps}
\caption{An illustration of the graph when (a)  and , (b)  and , and (c)  and  }
\label{fig:is1}
\end{figure}


\begin{itemize}
\item[]  If , , has adjacency in , and , , has adjacency in . Choose the least  such that , , and  forms an induced . Choose the least  such that ,  and   forms an induced . Then, either  or   forms an induced cycle of length greater than . 

\item[]  If , , has adjacency in  and , , has adjacency in . Choose the least  such that , , and  forms an induced . Choose the least  such that ,  and   forms an induced . Then, either  or   forms an induced cycle of length greater than . 

\item[] If  is adjacent to some vertices in . Pick the largest indexed vertex in , say , such that  is adjacent to . If , then  forms an induced  (\emph{see Figure  \ref{fig:is1}(a)}). If , then  creates an induced cycle of greater than . The argument is similar if  is adjacent to a vertex in .

\item[] If  has a neighbor in . Choose the least indexed vertex in , say , such that . 

\begin{itemize}
\item[-] If  does not have a neighbor in , then   forms an induced cycle of length greater than .
\item[-] If  has a neighbor in , then choose the least indexed vertex in , say , such that  is adjacent to . Since  is an  graph, . Thus, either  or  forms an induced cycle of length greater than  (\emph{see Figure  \ref{fig:is1}(b)}).
\end{itemize}

\end{itemize}


\item[\textbf{Case (} and \textbf{):}]
By the definition of ,  and  forms an induced . Thus,  forms an induced cycle of length greater than . The cycle is induced because the following cases are not possible by the definition of .


\begin{itemize}
\item[] If  is adjacent to some vertices in , then choose the largest indexed vertex in , say , such that  . If , then  forms an induced . If , then  forms an induced  (\emph{see Figure  \ref{fig:is1}(c)}). Similar argument if  has a neighbor in  and if  has an adjacency in  or in .
\end{itemize}
\end{description}
All the above cases contradict the definition of  graphs. Hence, the lemma is true. 
\end{proof}


\begin{definition}
Let . A graph  is said to be a   of size  denoted as  if there exist  such that  for all  and  is a path of length . The  is shown in \emph{Figure. \ref{fig:pp}}. A  is maximum or a maximum cage if there is no  such that  has .
\end{definition}



\begin{figure}[h]
\centering
\includegraphics[scale=0.5]{pp.eps}
\caption{}
\label{fig:pp}
\end{figure}


\begin{theorem}
\label{mvssc2ksize}
Let  be a connected  graph, . For any two induced cycles  and  in , if  i.e.,  contains , then the cardinality of every minimal vertex separator of  is at most , where  is the size of the maximum cage. 
\end{theorem}
\begin{proof}
On the contrary, assume that there exists a minimal vertex separator  of  such that , . Since ,  is an independent set, due to \emph{Lemma \ref{mvsindependent}}. We know that every minimal vertex separator is -minimal vertex separator for some non-adjacent vertices  and  in . Also, every -minimal vertex separator is -minimum vertex separator for some non-adjacent vertices  and  in . Without loss of generality, let us assume that  is a -minimum vertex separator. Thus, every vertex in  is part of a vertex disjoint path from  to . Hence, we get , where . This contradicts the maximality of . Hence the theorem. 
\end{proof}

\section{Characterization of  graphs}
Like chordal graphs has a simplicial vertex \cite{tarjan} and chordal bipartite \cite{GolumbicGoss} has a bi-simplicial edge, we shall observe that every  graph has a special vertex or a special  namely   or \emph{pendant cycle}, respectively. Thus, we can obtain an ordering called \emph{vertex cycle ordering (VCO)} for an  graph. 

\begin{definition}
Let  be an , graph. A vertex  is said to be a \emph{pendant vertex} if . A cycle  is said to be -  in  if for every cycle , , in ,  and  can have at most one cut vertex of .\\

 A cycle  in  is said to be -  if  has exactly one cut vertex , and there exist at least one induced cycle  such that  and  and  shares  in common and with every other cycle  in  . \\
 
 A cycle  in  is said to be -  if  has exactly one -vertex separator such that  and for all other cycles ,  has vertex intersection with  at  or , or edge intersection with  at , or no intersection with .\\

 A cycle  in  is said to be - , , if there exist at least one cycle  in  such that  and , say , satisfying the following conditions:
\begin{itemize}
\item[1.]  can have  or  as a cut vertex but not both.
\item[2.] there does not exist a cycle  in  such that the graph induced on  is not  and .
\end{itemize}   
\end{definition}

\begin{figure}[h]
\centering
\includegraphics[scale=0.3]{pendantegs.eps}
\caption{An example of an  graph where  and  are pendant vertices,  is a -pendant ,  is a -pendant ,  is a -pendant cycle and  is a -pendant cycle.}
\end{figure}

\begin{lemma}
\label{specialvertex2k+1}
An  graph  other than , , has any one of the following properties:
\begin{itemize}
\item[(i)] Two non-adjacent pendant vertices
\item[(ii)] Two -pendant , .
\item[(iii)] An -pendant  and a pendant vertex, .
\end{itemize}
\end{lemma}
\begin{proof}
 We shall partition the set of  graphs into  graphs with at least one minimal vertex separator of size one and  graphs with every minimal vertex separator is of size two. In both the cases, we shall prove the lemma by mathematical induction on the number of vertices  of .
\begin{description}
\item[Case 1:] There is a minimal vertex separator of size one. \\
\noindent \emph{Base cases:} 
\begin{itemize}
\item[(A)]  be a tree on  vertices, . Trivially,  has two non-adjacent pendant vertices as there are at least two leaves (degree one vertex) in any tree.
\item[(B)]  is not a tree on  vertices, . Clearly,  has two  sharing a vertex in common. So,  has two -pendant .
\item[(C)]  is a graph different from (A) and (B) on  vertices, . It is easy to see that  has either a -pendant  and a pendant vertex or two pendant vertices.
\end{itemize}
\noindent Let  be an  graph with  vertices. Let  be any minimal vertex separator of  such that . Let  and  be any two connected components in . Let  and  be the graphs induced on  and  , respectively. If both  and  are , then there are two -pendant 's in . Otherwise, by the induction hypothesis,  and  have a -pendant , , or a pendant vertex, which are also pendant in . Hence the claim.
\item[Case 2:] Every minimal vertex separator is of size two. Let  be an  graph and  be any minimal vertex separator of  such that  and .\\
\noindent \emph{Base case:} For , an  graph with  edges has two -pendant 's.\\
\noindent Let  be an  graph with  and  and  as defined before. By the hypothesis,  and  have a -pendant  which are also a -pendant  in .
\end{description}
Thus the lemma is true for all  graphs, . 

\end{proof}

\begin{lemma}
\label{specialvertex2k}
An  graph  other than , , has any one of the following properties:
\begin{itemize}
\item[(i)] Two non-adjacent pendant vertices.
\item[(ii)] Two -pendant , .
\item[(iii)] An -pendant  and a pendant vertex, .
\end{itemize}
\end{lemma}
\begin{proof}
We use induction on , the number of vertices in .\\
\noindent \emph{Base cases:} \vspace{-0.16cm}
\begin{itemize}
\item[(A)] For , any tree with  vertices has exactly two non-adjacent pendant vertices.
\item[(B)]  is not a tree on  vertices, .  has two -pendant 's, or two pendant vertices, or a -pendant  and a pendant vertex.
\item[(C)]  is not a tree on  vertices,  has two -pendant , or two pendant vertices, or a -pendant  and a pendant vertex.
\item[(D)]  is not a tree on  vertices.  has any one of the following:
\begin{itemize}
\item[] two -pendant 's.
\item[] three -pendant .
\item[] a pendant vertex and a -pendant .
\item[] a pendant vertex and a -pendant .
\item[] a -pendant  and a pendant vertex.
\item[] two pendant vertices.
\end{itemize}
 
\item[(E)]  is a graph different from (A) and (C) on  vertices, .  has either a -pendant , , and one pendant vertex or two pendant vertices.
\end{itemize}
\noindent Let  be an  graph with  vertices. Let  be any minimal vertex separator of . Let  and  be any two connected components in . Let  and  be the graphs induced on  and  , respectively. If   or  such that , by the induction hypothesis, both  and  have a pendant vertex or a -pendant , , which are also pendant in . If , then  is an independent set, by \emph{Lemma \ref{mvsindependent}}. The possible existence of , , or  in this case are as follows:

\begin{itemize}
\item[(1)] If  (as well as ) has a pendant vertex in  (as well as ), it is also a pendant vertex in .
\item[(2)] If  (as well as ) has a -pendant ,  in  (as well as ), it is also pendant in .
\item[(3)] If  and  do not have any pendant vertices and -pendant ,  in  and , respectively, and if  and  have pendant vertices only in . Since,  is a minimal vertex separator of size greater than two, the only possibility of  is . Thus,  has at least two -pendant .
\item[(4)] If either  has a pendant vertex or a -pendant ,  in , and  has neither of them in . If  itself has any one of ,  and  in , then there is nothing to prove. If  has a pendant vertex  in , then  may have any one of ,  and , by the hypothesis. If  has none of (i), (ii), and (iii) then,  is a CAGE, thus  has a pendant cycle together with , our claim follows in . If  has , then  has two pendant vertices, one is  and the other is from . If  has , then  has a pendant vertex and -pendant cycle from . If  has , then  has a pendant vertex and either a -pendant cycle or a pendant vertex from . If  has a -pendant  in , , say , then  has any one of ,  and , by the hypothesis. Let . Thus, if  has , then  has a pendant vertex from  and a -pendant  from , if  has , then  has two -pendant cycle's one from  and the other from , if  has , then  has a -pendant  from  and either a -pendant cycle or a pendant vertex from .
\end{itemize}


Thus the lemma is true for all  graphs, . 

\end{proof}


\begin{theorem}
\label{construction}
A connected graph is , if and only if it can be constructed using the following rules.
\begin{itemize}
\item[(i)]  is an  graph.
\item[(ii)]  is an  graph.
\item[(iii)] If  is an  graph, then the graph , where, ,  such that  and  is any vertex in ,  is also an  graph.
\item[(iv)] If  is an  graph, then the graph , where, ,  such that  and  is any vertex in ,  is also an  graph.
\item[(v)] If  is an  graph, then the graph , where, ,  such that  and  is any edge in ,  is also an  graph.
\item[(vi)] If  is an  graph and , then the graph , where, ,  such that  and  is any path of length  contained in no induced cycle in  or in any one induced cycle  of length  in  such that there does not exist an induced cycle  in  with ,  for some  and for at least one , .
\end{itemize}
\end{theorem} 
\begin{proof}
\begin{description}

\item[{Necessity:}] Given that  is an  graph. By \emph{Lemma \ref{specialvertex2k+1}} and \emph{Lemma \ref{specialvertex2k}},  graph has at least one pendant vertex or a -pendant , , and we denote them using the label . Consider the graph  obtained from  by removing the label , i.e., remove a pendant vertex or a -pendant , . Since  graphs respect hereditary property,  contains a label  which is a pendant vertex or a -pendant , . Repeat the previous step by removing the label . Clearly, in at most  iterations we can get an ordering among labels which we call us \emph{vertex cycle ordering(VCO)}. Clearly, the reverse of VCO gives the construction of the underlying  graph. This completes the necessity.
\item[{Sufficiency:}] Let  be a graph constructed using the rules (i) to (vi). We shall prove the theorem by mathematical induction on the number of iterations needed to construct .
\begin{description}
\item[Case 1:]  is obtained by rule (iii).
 
The vertex set and the edge set of the graph  are  and , for some , respectively. By the hypothesis,  is an  graph and the newly added edge  does not create any new cycle in . Thus,  is also an  graph.

\item[Case 2:]  is obtained by rule (iv)

For any ,  = () be the newly added . The vertex set and the edge set of the graph  are  and , respectively. By the hypothesis,  is an  graph and  does not induce a cycle other than  in . Thus,  is also an  graph.

\item[Case 3:]  obtained by rule (v)

For any edge ,  = () be the newly added . The vertex set and the edge set of the graph  are  and , respectively. By the hypothesis,  is an  graph and  does not induce a cycle other than  in . Thus,  is also an  graph.


\item[Case 4:]  obtained by rule (vi)


For any path of length ,  contained in no induced cycle in .  be the newly added . The vertex set and the edge set of the graph  are  and , respectively. By the hypothesis,  is an  graph and  does not induce a cycle other than  in . Thus,  is also an  graph. 



For any path of length ,  contained in any one induced cycle  of length  in . Let  be the newly added . The vertex set and the edge set of the graph  are  and , respectively. If there exist an induced  in  with ,  for some  and for at least one , , then the possible cases are as follows:

\begin{itemize}
\item[] If , then  forms an induced cycle of length +.

\item[] If , then   forms an induced cycle of length +.

\item[] If , , then   forms an induced cycle of length -+, which is always greater than .
\end{itemize}


All the above cases contradicts the definition of  graph. Thus, there does not exist an induced  in  with ,  for some  and for at least one , . By the hypothesis,  is an  graph and  does not induce a cycle other than  in . Thus,  is also an  graph.  
\end{description}
\end{description}
\end{proof}



\begin{lemma}
\label{mindegree}
Let  be an  graph, where . Then, the minimum degree of  is at most 2. I.e., .
\end{lemma}

\begin{proof}
Let us prove the theorem by induction on the length  of VCO of .
\\
\noindent Consider an ordering , . The label  corresponds to a vertex  or a . Let  be the graph corresponds to . Let  be the graph induced on  in . Thus,


\end{proof}

\section{Algorithmic results on  graphs}
In this section, we present a polynomial-time algorithm for testing whether an arbitrary graph is an  graph or not, for a fixed . Further, we solve the famous combinatorial problems like coloring, hamiltonicity and treewidth for a given  graph.




\subsection{Recognizing  graphs}
We shall use the ordering on  graphs to test whether the given graph is , graph or not. First, we present a decomposition theorem for , graphs followed by the algorithm for testing  graphs for any fixed . Similarly, we shall produce a decomposition theorem for , graphs along with its recognition algorithm.

\begin{definition}
A -  is a connected graph with no cut vertex. A -  of a graph  is a maximal bi-connected subgraph of . 
\end{definition}


\begin{theorem}
\label{decompositionsc2k}
A graph  is an  graph, , if and only if it can be decomposed into a set of connected components, such that each connected component is any one of the following: 
\begin{itemize}
\item a cut edge
\item a 
\item , 
\end{itemize}
\end{theorem}
\begin{proof}
\textbf{Necessity:} We shall prove the necessity by mathematical induction on the length  of VCO of .\\
\noindent Consider an ordering . The label  corresponds to either a vertex or a . 
\begin{itemize}
\item[] 
If  is a vertex , then it is a pendant vertex in  and  is an edge . Note that  is a cut edge. By the hypothesis,  has a decomposition  where each connected component is a cut edge or a  or a , . Thus,  can be decomposed into  and a cut edge .

\item[] 
If  is a -pendant , , say , then by the hypothesis, the graph obtained by the ordering  has a decomposition  where each connected component is a cut edge or a  or a , . Thus,  can be decomposed into  and a cycle .

\item[] 
If  is a -pendant , say , then by the hypothesis, the graph obtained by the ordering  has a decomposition  where each connected component is a cut edge or a  or a , . Now, combine the cycle  to the path , which belongs to an induced cycle  in one of the connected components of  and thus, the corresponding component results in a CAGE, by \emph{Theorem \ref{construction}}. Note that by introducing , either a new CAGE is created or the size of the existing CAGE increased by one. Hence, we obtained a decomposition as per the theorem.

\end{itemize}

\noindent \textbf{Sufficiency:} Given a decomposition of a graph in which every connected component is an  graph. It is clear that, any two connected components are connected either by a vertex or by an edge and this will not induce any new cycle of length, which is not equal to . Hence the claim. 
\end{proof}





\noindent From \emph{Theorem \ref{decompositionsc2k}}, we learn that the recognition of  graphs, , involves two simple steps. Given any arbitrary graph : first, find the decomposition of the graph  such that each connected component is free from the clique separators of size one and two. Now, for each connected component, check whether it is an edge or a 2-regular graph on  vertices or a , . If not,  is not an  graph. Note that computing a decomposition where each connected component is free from the clique separators of size one and two for the graph  involves three steps: (1) Find the bi-connected components of , (2) in each component  search for an edge  whose removal disconnects  (3) if the edge  exists then decompose  as follows: find , where  and add back the edge  to every connected component of . Do this process recursively in each  until there is no component with clique separators of size two. Testing whether a graph is CAGE or not involves the following steps:
\begin{itemize}
\item[1.] Search for two non-adjacent vertices with equal degree and the degree is at least three, say , and all other vertices in the graph should be of degree two. If the above check is unsuccessful, then the given graph is not a CAGE. Otherwise, proceed with the next step. 
\item[2.] Draw BFS tree  rooted at a maximum degree vertex.
\item[3.] To know whether  corresponds to , check whether the number of levels in  is , the root has degree , and there are  slanting edges between the last two levels. Further, the last level has exactly one vertex and  slanting edges are from  to all other vertices at last but one level except its parent. 
\end{itemize}

Clearly, all the above steps can be verified using the standard BFS and hence test can be done in  time, where  and  denotes the number of vertices and edges in , respectively. 


\begin{theorem}
\label{decompositionsc2k+1}
A graph  is an , graph if and only if it can be decomposed into a set of connected components, where every connected component of  is any one of the following:
\begin{itemize}
\item[(i)] a cut edge
\item[(ii)] a 
\end{itemize}
\end{theorem}
\begin{proof}
\textbf{Necessity:}  We shall prove this by mathematical induction on the length  of VCO of .
\noindent Consider an ordering . The label  corresponds to either a vertex or a . 
\begin{itemize}
\item[] 
If  is a vertex , then it is a pendant vertex in  and  is a cut edge  in . By the hypothesis,  has a decomposition  where each connected component is an edge or a . Thus,  can be decomposed into  and a cut edge .

\item[] 
If  is a -pendant , , say , then by the hypothesis, the graph obtained by the ordering  has a decomposition  where each connected component is a cut edge or a . Thus,  can be decomposed into  and a cycle .

\end{itemize}

\noindent \textbf{Sufficiency:} Given a decomposition of a graph in which every connected component is an  graph. It is clear that, any two connected components are connected either by a vertex or by an edge and this will not induce any new cycle of length, which is not equal to . Hence the claim. 
\end{proof}



\noindent From \emph{Theorem \ref{decompositionsc2k+1}}, we observe that the recognition of  graphs, , involves two simple steps. Given any arbitrary graph : first, find the decomposition of the graph  such that each connected component is free from the clique separators of size one and two. Now, for each connected component, check whether it is an edge or a 2-regular graph on  vertices. If not,  is not an  graph. Thus, we can recognize  graphs, , using BFS in  time, where  and  denotes the number of vertices and edges in , respectively.

\subsection{Structure of non-tree edges in  graphs}
\begin{definition}
Let  be a connected graph and  be the \emph{Breadth First Search} () tree of . Let  denotes the edges in the graph  and  denotes the edges in the BFS tree . The \emph{non-tree edges} are the edges in  i.e., the edges which are in graph  but not in tree .  
\end{definition}

\begin{definition}
Let  be a connected graph and  be the \emph{Breadth First Search} () tree of . The set  is called as non-tree edges. A non-tree edge,  is said to be a \emph{cross edge} if both  and  are in same levels of the tree . A non-tree edge,  is said to be a \emph{slanting edge} if both  and  are in adjacent levels of the tree .
\end{definition}

\begin{definition}
A \emph{matching} in a graph  is a set of independent edges.
\end{definition}

\begin{lemma}
\label{matching}
 Let  be the BFS tree of an , graph , then the set of non-tree edges of  forms a matching.
\end{lemma}
\begin{proof}
Construct a BFS tree  for the given graph  by fixing  as a root. Since,  is an  graph, the case where every non-tree edge in  is a slanting edge, is not possible. Now our claim is to prove that the non-tree edges of T forms a matching. On the contrary, assume that the non-tree edges of T do not form a matching. i.e., there exist at least two non-tree edges with a common vertex. We shall partition the  graphs into the graphs which has only cross edges in  and the graphs which has both cross edges and slanting edges in .

\begin{description}
\item[\bf{Case 1:}] The only non-tree edges in  are cross edges. By our assumption,  there exist cross edges  in the least level  such that  and .

\begin{figure}[h]
\begin{center}
\includegraphics[scale=0.25]{bfs.eps}
\caption{BFS Tree  of  with cross edges  and }
\label{bfs1}
\end{center}
 \vspace{-0.7cm}
\end{figure}


\begin{description}
\item[] If for some ,  and  forms an induced  in , where  and  are cross edges and all other edges are in , then  forms an induced  (\emph{see} \emph{Figure }\ref{bfs1}(a)). 

\item[] If for some ,  forms an induced  and if there exists  and  in  such that  is a common parent of  and , then  forms an induced ,   (\emph{see} \emph{Figure }\ref{bfs1}(b)). 

\item[] If for some ,  forms an induced  in , where  is a cross edge, and for some  and ,  is a cross edge, then  forms an induced cycle of even length (\emph{see} \emph{Figure }\ref{bfs1}(c)).
\end{description}


\item[\bf{Case 2:}] The non-tree edges in  contains both cross edges and slanting edges. By our assumption, there exist an edge  in level  and an edge  where  is in level  or in level  such that  is the least possible level.

\begin{figure}[h]
\begin{center}
\includegraphics[scale=0.25]{bfsslant1.eps}
\caption{BFS Tree  of  with cross edge  and slanting edge }
\label{bfs2}
\end{center}
\end{figure}

\begin{description}

\item[] If  is in level  and if for some ,  and  forms an induced  in , where  and  are cross edges,  is a slanting edge and all other edges are in , then  forms an induced  (\emph{see} \emph{Figure }\ref{bfs2}(a)). 


\item[] If  is in level  and for some ,  forms an induced  in ,  exists in  and  is the common parent of  and  where  and  are cross edges,  is a slanting edge and all other edges are in , then forms an induced  (\emph{see} \emph{Figure }\ref{bfs2}(b)).

\item[] If for some ,  forms an induced  in , where  is a slanting edge and  is a cross edge, and for some ,  forms a cross edge, then  forms an induced cycle of even length (\emph{see} \emph{Figure }\ref{bfs2}(c)). 

\item[] If for some ,  forms an induced  in , where  is a cross edge and  is a slanting edge, and for some ,  forms a slanting edge and , then  forms an induced cycle of even length (\emph{see} \emph{Figure }\ref{bfs2}(d)). 



\end{description}

\item[\textbf{Case 3:}] The non-tree edges in  contains both cross edges and slanting edges. By our assumption, there exists two slanting edges   and  where  is in level ,  is in level  and  is in level  such that  is the least possible level. 

\begin{description}
\item[] If  and  forms an induced , where  and  are cross edges, then  forms an induced  
(\emph{see} \emph{Figure }\ref{fig:bfs3}(a)).    

\item[] If for some ,  is a slanting edge and  is a cross edge such that there exist , ,  and . Thus,  forms an induced  and  forms an induced cycle of even length 
(\emph{see} \emph{Figure }\ref{fig:bfs3}(b)).    
\end{description}

\begin{figure}[h]
\begin{center}
\includegraphics[scale=0.25]{bfsslant2.eps}
\caption{BFS Tree  of  with two slanting edges  and }
\label{fig:bfs3}
\end{center}
\end{figure}

\end{description}
All the above cases contradicts the definition of  graphs. Hence our assumption, cross edges does not form a matching is wrong. Thus, cross edges in  forms a matching. 

\end{proof}




\subsection{Hamiltonicity in  graphs}

In this subsection, we provide a necessary and sufficient condition for the existence of hamiltonian cycle in  graphs.


\begin{definition}
 An  graph is said to be \emph{- pyramid} if it has  vertices,  edges, exactly two adjacent vertices of degree  and every other vertices are of degree two. A 3- pyramid is shown in Figure  \ref{fig: 3c5}.
\end{definition}

\begin{figure}[h]
\begin{center}
\includegraphics[scale=0.30]{3c5.eps}
\vspace{-0.5cm}
\caption{3- Pyramid.}
\label{fig: 3c5}
\end{center}
\end{figure}

\begin{definition}
 The graph  is \emph{Hamiltonian} if it has a spanning cycle (a cycle that contains all vertices in ), also called a Hamiltonian cycle.
 \end{definition}
 
\begin{theorem}
\emph{(Chvatal \cite{chvatal})} If a connected graph  has a Hamiltonian cycle, then for each , the graph  has at most  components.
\end{theorem}


 
\begin{lemma}
\label{ncknonhamiltonian}
A - pyramid graph is non-hamiltonian for all  and .
\end{lemma}
\begin{proof}
Let  be a - pyramid graph,  and . Let  and  be the two adjacent vertices of degree  in . Let . By Chvatal's theorem,  is not a Hamiltonian graph, as the graph  will disconnect the graph into  connected components. Hence the lemma.

\end{proof}

\begin{lemma}
\label{ncknonhamiltoniansubclass}
Any -graph  which contains - pyramid, , as an induced subgraph is non-hamiltonian.
\end{lemma}
\begin{proof}
On the contrary, assume that  is hamiltonian. Let  and  be the adjacent vertices of degree greater than or equal to  in  such that  is an edge of - pyramid. Let  be the  cycles containing the edge  in . Let  and  (\emph{for e.g.,} \emph{see} \emph{Figure } \ref{fig: 3c5}, where  and ). Since  is hamiltonian, there exist a path from  to ,  other than the path through the edge . In particular, there exist a path from  to  which does not pass through the vertices . i.e., there exist at least one path  from  to ,  which does not pass through the vertices , which contradicts the construction of  graph. Hence our assumption is wrong, which implies  is non-hamiltonian.  
\end{proof}

\begin{theorem}
 Let  be an  graph.  is Hamiltonian if and only if it is 2-connected,  free and 3- pyramid free.
\end{theorem}
\begin{proof}
\emph{Necessity:} we know that every hamiltonian graph is -connected, thus,  is -connected. Now our claim is to prove  is 3- pyramid free. On the contrary, assume that  has 3- pyramid as an induced subgraph. Thus, there exist an edge  such that  has at least three 's, say , , with the property . By \emph{Lemma \ref{ncknonhamiltonian}} and \emph{Lemma \ref{ncknonhamiltoniansubclass}},  is non-hamiltonian, which is a contradiction. Therefore,  is 3- pyramid free. Also, by the definition of CAGE it is clear that  is non-hamiltonian, thus,  is  free.
\\
\emph{Sufficiency:} Let . Consider a graph , where  and . Since,  is -connected, by \emph{Theorem \ref{construction}}, the graph  is constructed only by \emph{rule (ii)} and \emph{rule (v)}. Therefore, the graph  is an induced cycle, which is a spanning cycle in . Hence,  is hamiltonian. 
\end{proof}

\section{Treewidth of  graphs}
 A \emph{tree decomposition}\cite{kloks} of a graph  is a pair () where  is a tree and  assigns a set  to each vertex  of  such that 
\begin{itemize}
\item[(i)] ,
\item[(ii)] for every edge , there is some  such that  and
\item[(iii)] for every vertex , the set  induces a subtree of the tree .
\end{itemize}
The \emph{width} of a tree decomposition () is  and the \emph{tree-width}, , of  is the minimum width of all tree decompositions of .  

\begin{definition}
A graph is a - if every minimal vertex separator of  is of size  and every maximal clique is of size . A graph  is said to be a \emph{partial-k-tree} if it is an edge subgraph of a -tree.
\end{definition}

In this section, we present an exact bound for the treewidth followed by an algorithm which gives a tree decompositon () with  or  for the given , graph. Let  be an  graph. We know that . Since,  is the maximum clique in , . We can divide  graphs into  graphs with cycles and  graphs without cycles. It is clear that,  graphs without cycles are same as trees and we know that , i.e., . Thus, in this section, we consider  graphs with cycles. It is evident that the lower bound of  graphs is two as . We observe that the upper bound for  graphs is two by proving that  graphs are partial-2-trees, an edge subgraph of a 2-tree. Alternatively, we augment edges to the given  to produce a 2-tree and the augmentation algorithm is given below.

\begin{definition}
A \emph{minimum fill-in} of a graph  is the minimum number of edges whose addition makes the graph  chordal.
\end{definition}

\begin{algorithm}
\caption{}
\begin{algorithmic}[1]
\STATE{\textbf{INPUT:} An  graph, .}
\STATE{\textbf{OUTPUT:} A chordal graph }
\STATE{Decompose the graph  into a set of connected components as per \emph{Theorems \ref{decompositionsc2k+1} and \ref{decompositionsc2k}}.}
\STATE{Let  be the connected components of the decomposed graph.}
\FOR{ to }
		\IF{ is a }
			\STATE{Choose any one vertex  in  and make it adjacent to all the non-adjacent vertices of  in .}
		\ELSIF{ is a CAGE}
			\STATE{Choose a vertex with maximum degree and make it adjacent to all the non-adjacent vertices in the CAGE.}
		\ENDIF
\ENDFOR
\STATE{Now combine the decomposed graph into a graph  and Return .}
\end{algorithmic}
\label{alg:partial2tree}
\end{algorithm}

\begin{theorem}
\label{fill-in}
The algorithm  outputs a chordal graph, which is a partial-2-tree.
\end{theorem}
\begin{proof}
We prove this by induction on the length of the VCO of a given  graph.
In the ordering , let  be an  graph obtained after  ordering, . Our claim is to prove  is a chordal graph and a partial-2-tree.
\begin{description}
\item[\textbf{Case 1:}]  is , say .\\
By the hypothesis, it is clear that  is chordal and a partial-2-tree.

\item[\textbf{Case 2:}]  is a -pendant  or a -pendant .\\
Let  be  and  be the associated graph for the ordering . W.l.o.g, . By the induction hypothesis, when  is passed as an input to the Algorithm \ref{alg:partial2tree}, the output of Algorithm \ref{alg:partial2tree} is a chordal graph and a partial-2-tree. Now \emph{Step 7} of Algorithm \ref{alg:partial2tree} adds edges from  to all the non-adjacent vertices of . Clearly, the resulting graph is chordal and a partial-2-tree.

\item[\textbf{Case 3:}]  is a -pendant .\\
Let  be . Let  be the associated graph for the ordering  and by the induction hypothesis, when  is given as an input to the Algorithm \ref{alg:partial2tree}, the output of Algorithm \ref{alg:partial2tree} is a chordal graph and a partial-2-tree. Since,  is -pendant vertex, w.l.o.g, let . Now, augment edges from  to every non-adjacent vertex of . Clearly, the resulting graph is chordal and a partial-2-tree.

\item[\textbf{Case 4:}]  is a -pendant .\\
Let  be  and  be the associated graph for the ordering . By the induction hypothesis, when  is passed as an input to the Algorithm \ref{alg:partial2tree}, the output of Algorithm \ref{alg:partial2tree} is a chordal graph and a partial-2-tree. W.l.o.g, assume that . Now \emph{Step 9} of Algorithm \ref{alg:partial2tree} adds edges from  to all the non-adjacent vertices of . Clearly, the resulting graph is chordal and partial-2-tree. 
\end{description}
\end{proof}

\noindent From the above case analysis, it follows that . Since , .

\begin{corollary}
\label{minfillin}
Minimum fill-in of  graphs is polynomial-time solvable.
\end{corollary}
\begin{proof}
The output of Algorithm \ref{alg:partial2tree} yields a chordal graph by augmenting a minimum number of edges. Therefore, the output is precisely the minimum fill-in of  graphs. Further, minimum fill-in is polynomial-time solvable for  graphs. Note that the number of edges augmented in a given graph  by Algorithm \ref{alg:partial2tree} is , where  denotes the number of 's and  denotes the  in the decomposition of .

\end{proof}

\noindent Having given the bounds for treewidth, we now present an algorithm which gives a tree decomposition for  graphs, where .\\

\noindent \textbf{Outline of the algorithm:}
The algorithm first constructs a graph  from  as follows: to start with, every induced cycle in  is converted into a collection of 's appropriately, where the weights of the edges are assigned to be one. Next, the algorithm collects all the edges in  which are not a part of any cycle in . Now, for every element in , the algorithm creates a new vertex. Finally, the algorithm adds weighted edges among the newly formed vertices and the newly constructed 's, and the weights of the edges depends on its end vertices.  Thus, the graph  has been constructed from . Now, find the minimum spanning tree  for the weighted graph  and the algorithm outputs  as a tree decomposition for .


\begin{algorithm}
\caption{Tree Decomposition for , graphs}
\begin{algorithmic}[1]
\STATE{\textbf{Input:}  graph  with cycles, .}
\STATE{\textbf{Output:} Tree decomposition of }
\STATE{Let  be the vertex set of  and ,  be the cycles in .}
\FOR{ to }
\STATE{for every cycle  in  define  as follows.
\begin{itemize}
\item[]   is the induced  in \}, 
\item[] ,
\item[] , , .
\end{itemize}}
\ENDFOR
\STATE{Let   or }
\STATE{Let  () be the edges in .}
\FOR{ to }
\STATE{}
\ENDFOR
\STATE{}
\FOR{ to }
\FOR{ to }
\STATE{ and }
\ENDFOR
\ENDFOR
\STATE{For each , , relabel the vertices in  by the labels given for the vertices during the input.}
\STATE{Construct a graph  with vertex set,  and two vertices ,  are adjacent if any one of the following types is true:
\begin{itemize}
\item[Type 1:]  and , for some , then, . 
\item[Type 2:]  and , for some  and  and , then, .
\item[Type 3:]  and , for some  and  and , then, .
\item[Type 4:]  and  and .
\item[Type 5:]  and  and .
\end{itemize}}
\STATE{Convert the unweighted graph  to a weighted graph  by assigning the weight  for the edges of type , . }
\STATE{Find a minimum spanning tree  for the weighted graph }
\STATE{\emph{Return }}
\end{algorithmic}
\label{treedecompositionsc2k+1}
\end{algorithm}

\vspace{0.4cm}

\noindent \textbf{{Trace of the algorithm}}
\vspace{0.3cm}

\noindent We trace the steps of the Algorithm \ref{treedecompositionsc2k+1} in \emph{Figure \ref{fig:tracetreedecompositionsc2k+1}}.


\begin{figure}[H]
\centering
\includegraphics[scale=0.26]{trace1.eps} 
\vspace{-0.1cm}
\caption{Tree decomposition of an  graph.}
\label{fig:tracetreedecompositionsc2k+1}
\end{figure}



\begin{itemize}
\item[1.] Input is an  graph . For cycles ,  and  in the graph , create , , , , , , ,  and .
\item[2.] Now assign, , , , , , , , , ,  and .
\item[3.] Draw edges between  and ,  if it obeys the \emph{line 19} and assign weights for edges as in \emph{lines 19-20} (\emph{see} \emph{Figure 1}).
\item[4.] Construct a minimum weight spanning tree, , for the graph . Thus, the algorithm is complete and results a tree decomposition with minimum tree width for the given graph .
\end{itemize}

\begin{theorem}\label{prooftreedecomp2k+1}
The graph  obtained from the \emph{Algorithm \ref{treedecompositionsc2k+1}} is a tree decomposition of  such that .
\end{theorem}
\begin{proof}
 Our claim is to prove that the graph  is a tree and all the three conditions of tree decomposition are satisfied by .
\begin{description}
\item[claim 1:]  is a tree \\
It is clear from the construction of the graph , that the graph  is connected and hence  is connected. Further, the graph  is the minimum spanning tree of the graph , which proves  is acyclic. Hence,  is a tree.
\item[claim 2:] . \\
Let us partition the vertex set of  into  and , where  denotes the set of vertices which takes part in some cycle of  and  denotes the set of vertices which does not take part in any cycle of . It is evident from \emph{Steps 13-17} and from \emph{Steps 9-11}, that every element in  and  is added to , for some , respectively. Thus, . 

\item[claim 3:] For every edge , there is some  such that . \\
Every edge, which takes part in some cycle of , is added to , for some , by means of  in \emph{Steps 4-6} and every non-cycle edge is added to , for some , by means of  in \emph{Step 7}. Hence, the claim.


\item[claim 4:] For every vertex , the set  induces a subtree of the tree .\\
On the contrary, assume that there exist a vertex  such that the set  does not induce a subtree of the tree . i.e., the graph induced by the vertex set , say , is not connected. Let ,  be the connected components of . Choose a vertex  from  and  from . 
\begin{itemize}
\item[] . The weight of the edge  is 5 and hence, this edge will not create a cycle. Thus, , which is a contradiction as  and  are disjoint connected components in .
\item[]  and . The weight of the edge  is 4 and hence, this edge will not create a cycle. Thus, , which is a contradiction.
\item[]  and if the weight of the edge  is 1. Then,  since  is a minimum spanning tree of  and there can not be a cycle in  where the weights of all edges are 1.
 
\item[]  and if the weight of the edge  is 2. The edge , implies that, the edge  is part of a cycle and every other edge in the cycle is of weight one or two. Let  be the second shortest path from  to  in  and . 
\begin{itemize}
\item[-]  is , say . \\ By our assumption, . Since, the weight of  is either  or , . Similarly,  . Thus, , which is a contradiction to the construction of .
\item[-]  is , say . Since,  is a shortest path and the weight of the edge  is 2,  and . Thus,  and . The weight of the edge  is either 1 or 2, implies that, , which is a contradiction.
\end{itemize}


\item[]  and if the weight of the edge  is 3. Thus, the vertices  belongs to some , say  and the vertices in  belongs to some , say , , and both  and  has a vertex intersection. The edge , implies that, the edge  is part of a cycle and every other edge in the cycle is of weight one, two or three. Let  be the second shortest path from  to  in  and .

\begin{itemize}
\item[-]  is , say . \\ By our assumption, . If  and  or  and , then , which is a contradiction to the construction of . If  and  or  and , then the cycle belongs to , say , , contradicts the \emph{Theorem \ref{construction}}. The case where  and  is not possible by the construction of . 


\item[-]  is , say . Since,  is a shortest path and the weight of the edge  is 3,  does not belongs to any internal vertices of . If the weight of the edges  and  are 1 and 2 or 2 and 1 or 2 and 2, respectively, then there exists an edge , which is a contradiction to the minimality of . If the weight of the edges  and  are 1 and 1 or 1 and 3 or 3 and 1 or 3 and 3, then the cycle  contradicts the \emph{Theorem \ref{construction}}. 
\end{itemize}
\end{itemize}

All the above cases gives the contradiction, hence the claim. 

\end{description}
\end{proof}


\noindent Now, we present an algorithm which gives a tree decomposition for  graphs, where .\\

\noindent \textbf{Outline of the algorithm:}
The algorithm first decomposes the graph  into connected components where each component is a cut edge or a  or a CAGE. Next, the algorithm finds the tree decomposition for each connected component. Now, the algorithm combine the components based on its intersection and results in a graph . Finally, the algorithm finds a minimum spanning tree  of .


\begin{algorithm}
\caption{Tree Decomposition for , graphs}
\begin{algorithmic}[1]
\STATE{\textbf{Input:}  graph  with cycles, .}
\STATE{\textbf{Output:} Tree decomposition of }
\STATE{Let  be the vertex set of  and let .}
\STATE{Decompose the graph  into connected components as per \emph{Theorem \ref{decompositionsc2k}} and let  be the connected components in the decomposition.}
\FOR{ to }
	\IF{ is an edge}
		\STATE{ and }
	\ELSIF{ is a }
		\STATE{Let  be an induced  in  define  as follows.
\begin{itemize}
\item[]  and 
\item[]  and 
\item[]  and , ,  and .
\end{itemize}}
	\ELSIF{ is a CAGE}
		\STATE{Collect the vertices in  whose degree is equal to . CAGE has exactly two such vertices, say .}
		\STATE{Let . Then,  and  have  distinct paths of length  in . Let  be the  path between  and , .}
			\FOR{ to }
				\STATE{Define  as follows:
\begin{itemize}
\item[]  and 
\item[]  and 
\item[]  and , ,  and 
\end{itemize}		
		}
			\ENDFOR
\ENDIF
\ENDFOR
\STATE{For each , , relabel the vertices in  by the labels given for the vertices during the input.}

\STATE{Construct a graph  with vertex set,  and two vertices ,  are adjacent if any one of the following types is true:
\begin{itemize}
\item[Type 1:]  and , for some , then, . Let  be the connected components of the graph after augmenting Type 1 edges. 
\item[Type 2:]  and , for some  and  and , and if there are no edges between the vertices of  and  then, .
\item[Type 3:]  and , for some  and  and , and if there are no edges between the vertices of  and  then, .
\item[Type 4:]  and  and .
\item[Type 5:]  and  and .
\end{itemize}}
\STATE{Convert the unweighted graph  to a weighted graph  by assigning the weight  for the edges of type , .}
\STATE{Find a minimum spanning tree  for the weighted graph }
\STATE{\emph{Return }}
\end{algorithmic}
\label{treedecompositionsc2k}
\end{algorithm}

\newpage

\noindent \textbf{\large{Trace of the algorithm}}\\
\noindent We trace the steps of the Algorithm \ref{treedecompositionsc2k} in \emph{Figure \ref{fig:tracetreedecompositionsc2k}}.

\begin{figure}[H]
\centering
\includegraphics[scale=0.26]{trace2.eps} 
\vspace{-0.2cm}
\end{figure}

\begin{figure}[h]
\centering
\includegraphics[scale=0.26]{trace3.eps} 
\vspace{-0.1cm}
\caption{Tree decomposition of an  graph.}
\label{fig:tracetreedecompositionsc2k}
\end{figure}

\newpage

\begin{theorem}
The graph, , obtained from the \emph{Algorithm \ref{treedecompositionsc2k}} is a tree decomposition of  such that .
\label{prooftreedecomp2k}
\end{theorem}
\begin{proof}
In the algorithm, we decompose the graph  into connected components, where each connected component is a cut edge or a  or a CAGE. It is clear that, for each connected component, the graph constructed in \emph{Steps 5-20} is a tree decomposition of the respective component. Now, we add edges between components based on the conditions in \emph{Step 20} and we make the unweighted graph into a weighted graph  by giving weights to the edges. Finally, minimum spanning tree  is computed for the graph . The proof for  is a tree decomposition is similar to the proof in \emph{Theorem \ref{prooftreedecomp2k+1}}. Note that,  by \emph{Steps 5-15}.
\end{proof}

\begin{corollary}
\label{partial}
Let  be a connected , , graph. Then,  is a partial-2-tree.
\end{corollary}
\begin{proof}
Trivially follows from \emph{Theorem \ref{prooftreedecomp2k+1}} and \emph{Theorem \ref{prooftreedecomp2k}}.
\end{proof}


\begin{theorem}
\label{coloring}
Let  be a connected , graph. The chromatic number of  is at most \emph{three}. i.e., . Further, if  is odd then  and  if  is even then .
\end{theorem}
\begin{proof}
If  is even, then  is bipartite and hence, . If  is odd: 
let  be the maximum independent set in the graph induced on the non-tree edges of T. From \emph{Lemma \ref{matching}} (\emph{Section 5.2}), it follows that the set of non-tree edges in T forms a matching. Thus,  and  can be colored using the third color. Hence,  requires at most three colors. Therefore, we can conclude  if  is odd. We can also prove the theorem from the fact that , , graphs are partial-2-trees.  
\end{proof}




\section{Conclusions and Further Research}
In this paper, we have investigated strictly chordality  graphs, graphs in which every induced cycle is of length  or cycle-free, from both structural and algorithmic perspectives. We have obtained nice structural results based on the structure of the minimal vertex separators. Further, we have shown that testing  graphs are polynomial-time solvable using a special ordering, namely Vertex Cycle Ordering (VCO). Other results include Coloring, Hamiltonicity and Treewidth. Classical problems such as Vertex Cover, Odd Cycle Transversal, Feedback Vertex Set etc., are yet to be explored restricted to  graphs. 
\nocite{*}
\begin{thebibliography}{}

\bibitem{app1}
Jean R. S. Blair and Barry Peyton: An Introduction to Chordal Graphs and Clique Trees. In Graph Theory and Sparse Matrix Computation, Vol.56, pp.1-29, (1993).

\bibitem{app2}
I. Duff and J. Reid: The multifrontal solution of indefinite sparse symmetric linear equations. ACM Trans math. Software, Vol.9 , pp. 302 - 325, (1983).

\bibitem{app3}
R.P. Anstee, M. Farber, Characterizations of totally balanced matrices: Journal of Algorithms, Vol.5, pp.215-230, (1984).

\bibitem{Hajnal}
A. Hajnal and T. Surányi: ber die Auflsung von Graphen vollstandiger Teilgraphen, Annales Universitatis Scientarium Budapestinensis de Rolando Eötvös Nominatae
Sectio Mathematica. Math.,1 (1958).


\bibitem{dirac}
G.A. Dirac: On rigid circuit graphs. Abhandlungen aus dem Mathematischen Seminar der Universität Hamburg, Vol.25, pp.71-76, (1961).

\bibitem{fulkerson}
D.R. Fulkerson and O.A. Gross: Incidence matrices and interval graphs. Pacific Journal of Mathematics, Vol.15, pp.835-855, (1965).

\bibitem{tarjan}
Donald J.Rose,  George Lueker and R.E. Tarjan: Algorithmic aspects of vertex elimination on graphs. SIAM Journal of Applied Mathematics, Vol.34, pp.176-197, (1978).

\bibitem{GolumbicGoss}
M.C. Golumbic, C.F. Goss: Perfect elimination and chordal bipartite graphs. Journal of Graph Theory, Vol.2, pp.155-163, (1978).

 \bibitem{gavril}
Fnic Gavril: The intersection graphs of subtrees in trees are exactly the chordal graphs. Journal of Combinatorial Theory, Series B, Vol.16, pp.47-56, (1974).

\bibitem{corneil}
D.G. Corneil and J. Fonlupt: The complexity of generalized clique covering. Discrete Applied Mathematics, Vol.22, pp.109-118, (1989). 

\bibitem{Hoang}
C.T. Hoang: Efficient algorithms for minimum weighted coloring of some classes of perfect graphs. Discrete Applied Mathematics, Vol.55, pp.133-143, (1994).





\bibitem{bodlaender}
H.L. Bodlaender: A tourist guide through treewidth. Acta Cybernetica, Vol.11, pp.1-23 (1993).

\bibitem{kloks}
T. Kloks and D. Kratsch: Treewidth of Chordal Bipartite Graphs. Technical report, Utrecht University, (1992).







\bibitem{booth}
K.S. Booth and J.H. Johnson: Dominating sets in chordal graphs. SIAM Journal of Computation, Vol.11, pp.191-199, (1982).

\bibitem{mueller}
Haiko Mueller and Andreas Brandstaedt: The NP-completeness of Steiner tree and dominating set for chordal bipartite graphs. Theoretical Computer Science, Vol.53, pp.257-265, (1987).

\bibitem{colbourn}
C.J. Colbourn and L.K. Stewart: Dominating cycles in series-parallel graphs. Ars Combinatoria. 19A, pp.107-112, (1985).


\bibitem{muellerh}
H. Mueller: Hamiltonian circuits in chordal bipartite graphs. Discrete Mathematics, Vol.156, pp.291-298, (1996).

\bibitem{join}
Pavol Hell and Pei-Lan Yen: Join colourings of chordal graphs. Discrete Mathematics, Vol.338, pp.2453-2461, (2015).


\bibitem{contractibility}
Remy Belmonte, Petr A. Golovach, Pinar Heggernes, Pim van't Hof, Marcin Kaminki, and Daniel Paulusma: Detecting Fixed Patterns in Chordal Graphs
in Polynomial Time. Algorithmica, Vol.69, pp. 501-521, (2014).

\bibitem{strong}
Ton Kloks, Sheung-Hung Poon, Chin-Ting Ung and  Yue-Li Wang: On the strong chromatic index and maximum induced matching of tree-cographs, permutation graphs and chordal bipartite graphs. Journal of Discrete Algorithms, Vol. 30, pp. 21-28, (2015). 

\bibitem{enumeration}
Petr A. Golovach, Pinar Heggernes, Mamadou M. Kante, Dieter Kratsch and Yngve Villanger: Enumerating minimal dominating sets in chordal bipartite graphs. Discrete Applied Mathematics (to be published), (2015).

\bibitem{reconfiguration}
Marthe Bonamy, Matthew Johnson, Ioannis Lignos, Viresh Patel and Daniel Paulusma: Reconfiguration graphs for vertex colourings of chordal and chordal bipartite graphs. Journal of Combinatorial Optimization, Vol.27, pp. 132-143, (2012).


\bibitem{golumbicbook}
M.C.Golumbic: Algorithmic Graph Theory and Perfect Graphs. Academic Press, New York, (1980).


\bibitem{dbwest}
D.B.West: Introduction to Graph Theory. Published by Prentice Hall, (2001).

\bibitem{golumbic}
M.C.Golumbic: Dirac's theorem on triangulated graphs. Annals of the New York Academy of Sciences. 319, pp.242-246, (1979).

\bibitem{chvatal}
Chvatal V: In the travelling salesman problem: A guided tour of combinatorial optimization. Wiley, pp. 403-429, (1985).

\end{thebibliography}
\end{document}
