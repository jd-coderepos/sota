



In this section, we describe two modes in which we can leverage the computational capabilities of the analog integrated circuit for WSN applications, namely ($1$)~selective wake-up and ($2$)~selective sample mode.  We then quantify the performance gained in both cases. Finally, we demonstrate the use of our ASP-interfaced mote in a vehicle classification application and highlight the energy efficiency gained in comparison to an all-digital implementation.




\subsection{Selective Wake-Up Mode}
In this mode, we leverage the low-power processing capability of analog circuits by placing the mote into long periods of hibernation and then selectively waking the mote when a specific frequency band has been detected. This is done by keeping the analog circuit always-on and monitoring for a threshold in the selected band. Figure \ref{fig:asp_interrupts}(a) demonstrates single-band event detection.  The band of interest is 1kHz and the filter has a quality factor of $2.8$.  Signal content appears in the band at $2.6$ seconds but has noise added to it.  This noise is a combination of white noise and tones at 100Hz, 600Hz, and 10kHz. The bandpass filter focuses on the frequency of interest and the comparator trips once the RMS reaches the threshold.   Note that the subband event is detected despite having much lower amplitude than the noise and other frequencies.  In this mode, the mote samples the raw sensor signal when it wakes up and transmits it to a {\em basestation} (a data aggregation device in a wireless sensor network).  The signal received by the basestation is shown in the bottom trace. 



In order to compare the power consumed by the ASP-interfaced mote with an all-digital implementation, we implement a second-order Butterworth bandpass filter on a TelosB mote running TinyOS and measure the power consumed for detecting an event of interest. The digital filter is implemented by buffering $100$ samples at a time and then computing the filter outputs after every $100$ms.  The power consumed for this operation is measured for sampling frequencies ranging from $10$Hz to $1$kHz. In Fig.~\ref{fig:filterbankpower}, we compare the average power consumed by the digital filter over different sampling frequencies with the power drawn by an analog bandpass filter over different center frequencies. We point out that the energy consumed by our entire spectral analysis system is over $1000$ times lower than the power consumed by a single digital filter, thus signifying the energy savings compared to keeping a mote always turned on. 





\insfiga{asp_interrupts}{(a) Single-band event detection.  The frequency of interest is 1kHz and is the ``Signal'' trace in the top plot.  Broadband noise and tones at 100Hz, 600Hz, and 10kHz are added to ``Signal''.  ``Noise$+$Signal'' is the input to the analog signal processor.  The middle plot shows the response of the three stages of the processor within the 1kHz subband.  The bandpass filter cuts the undesired frequencies, while the RMS circuit tracks the magnitude.  Once the magnitude exceeds the threshold, an interrupt is generated to wake the mote.  The mote then samples the output of the sensor and transmits it to the basestation.  The received signal is plotted in the bottom plot.  (b) Sampling of pre-processed subbands and spectral analysis performed by the analog IC. (Top) The input signal is composed of two 1kHz pulses followed by a logarithmic chirp signal. (Middle) Once the trigger goes high, the ADC samples all 8 channels for a user-specified amount of time (e.g. 300msec).  (Bottom) Spectrogram of the transmitted frequency-dependent magnitude data received by the base station.  }{16cm}



\iffalse
\begin{figure*}[t]
  \begin{center}
  	\begin{tabular}{cc}
		\includegraphics[width=.47\textwidth]{eps/subbandDetect} & 
  		\includegraphics[width=.47\textwidth]{eps/spectrogram_array12pt} \\
  		(a) & (b)
  	\end{tabular}   
    \caption{(a) Sampling a single frequency band of interest. Once the 1kHz band RMS exceeds a user-defined threshold, an interrupt signal is generated to wake the mote.  The mote then samples the 1kHz subband and transmits it to the basestation.  The received signal is plotted in the bottom plot. (b) Spectral analysis performed by the analog IC. (Top) The input signal is composed of two 1kHz pulses followed by a logarithmic chirp signal. (Middle) Once the trigger goes high, the ADC samples all 8 channels for a user-specified amount of time (e.g. 300msec).  (Bottom) Spectrogram of the transmitted frequency-dependent magnitude data received by the base station. }
    \label{fig:subbandDetect}
  \end{center}
\end{figure*}
\fi

\subsection{Selective Sample Mode}
\label{sec:ss}

In the selective wake-up mode described in the previous subsection, once the mote is awake it samples the raw signal for processing or transmission. The drawback with this approach is that a low power processing platform such as the TelosB mote is unable to sample and process signals of high frequencies and is also limited in the kind of signal processing operations that can be performed (an FFT, for example, is infeasible to be performed on a TelosB mote \cite{vango}). This often warrants the use of a platform with higher processing power such as the IMote2 or Stargate \cite{stargate} for performing these signal processing operations which increases the overall power requirements of the system. In this subsection, we highlight the selective-sample mode of operation in which we leverage the ASP's ability to perform pre-ADC signal analysis. The ASP is used to perform a full spectral analysis of the input signal and the mote only samples the RMS energy of each subband. Thus we are able to reduce the computational resources required at the mote, allowing for lower power operation.

 In the experiment (Fig.~\ref{fig:asp_interrupts}b), the input signal consists of two 1kHz pulses followed by a chirp signal. The 1 kHz pulse is used to trigger the mote into sampling the RMS energy of each subband in succession, for a specified period of time. The mote scans through subbands by writing to the GIO port's output register between each sampling operation.  The frequency-decomposed RMS data obtained by the mote is transmitted to a base station and is displayed in the bottom plot. We note that by scanning through the energy of all the subband channels in succession, a complete spectral decomposition can be obtained at the mote in real time using the analog circuit. By doing so, we are also able to operate the system on signals with much higher frequencies (since we sample only the RMS amplitude of sub-bands) than would be possible with a mote alone.



\subsection{Evaluation in the context of a vehicle classification application}



In order to evaluate the accuracy and energy-efficiency of our ASP-interfaced mote in the context of an actual sensor network application, we have used an acoustic-sensor based vehicle classification scenario described in \cite{lanl} as an example. The vehicle classification system is intended for unattended monitoring of secure facilities, and the objective of the system is to accurately identify an approaching vehicle as belonging to one of multiple categories such as small, medium sized, and large vehicles and then accurately raise an alert when a vehicle of a particular type has been detected. The vehicles are assumed to appear in isolation and not concurrently with other vehicles. The system is required to be long lasting on battery sources while at the same time retaining high accuracy and low latency in classification. Arrival of any vehicle is expected to be a rare event, therefore rendering duty cycling of resources essential for energy-efficiency -- but at the same time it is critical that no vehicles are missed. We note that the chosen application is representative of typical wireless sensor network applications for monitoring such as detection of anomalies in bridges \cite{struct_rice}, unattended ground sensing by military personnel in combat situations, classification of objects for asset protection \cite{lites}, classification of animal sounds \cite{acoustic_frogs}, and monitoring of seismic activity. All these applications involve detection and classification of rare, short-lived events and demand high accuracy and high energy-efficiency. 

In this subsection, we describe the implementation of the vehicle classification system described above using our ASP-mote architecture and compare the system performance of our cooperative analog-digital implementation with that of an all-digital implementation. We specifically consider classification into two vehicle categories: car and truck.

\begin{figure}[tp]
  \begin{center}
    \includegraphics[width=.48\textwidth]{eps/timingDiagram}
    \caption{Demonstration of the stages of the detection system for one specific $10$ second test sample of a truck being classified. The truck is closest to the sensor between seconds $4-6$ of the test sample (shown in (ii)). The comparator outputs of the $8$ filter-bands are shown in (iv) and the CPLD outputs are shown in (v) and (vi). The CPLD interrupt pin goes high when a car or truck is detected. The CPLD class pin specifies the classification (high for truck, low for car) and is only valid when the interrupt pin is high.  Once the interrupt is generated, the mote is awake and starts recording and accumulating the CPLD classifications (consuming about $1.5$ mW of power). When a final decision is made, the output is transmitted via radio and this consumes $60$mW (shown in (vii)).}
    \label{truck}
  \end{center}
\end{figure}

\noindent{\bf Data collection:}~~The data collected for the experiments described in \cite{lanl} have been used for performance evaluation in this paper. The acoustic sensor used for data collection was a Samsung C01U - USB Studio Condenser Microphone. The microphone was placed 10-12 feet from the road and was mounted 1 foot off the ground.  The microphone has directional response and was mounted facing toward the roadway. Samson windshields on the microphones were used to filter out wind noise. A mid-sized car (Honda Accord LX) and a pickup truck (Chevy C4500 4x4) were considered as the two vehicle classes. Each vehicle was driven at speeds between 10 mph and 30 mph. The microphone was sampled at 4kHz. For each vehicle class, multiple observations were collected. These collected data samples were supplied by means of a standalone DAC for training and testing of our system. Ambient data was also collected using the microphone without any vehicle being present in the scene. 



\noindent{\bf Training:}~~The collected data are first normalized so that the peak amplitude of the signal across vehicle classes is uniform.  The data are then divided into two sets, one for training and the other for testing. Regions of the data corresponding to when the vehicle was and was not present were manually identified. 

Based on the short-time FFT spectra of the data (computed with Matlab), the filter bank biasing was chosen to be half-octave spacing from $100Hz$ to $1131Hz$.  Analysis was performed on all of the training samples by streaming the samples through the ASP using a DAC and recording the RMS output of each subband.  After obtaining the RMS data, the objective is to determine the combination of comparator trigger point and codeword assignments which achieves the desired classification performance.  The codewords are the 8-bit outputs from the eight comparators.  During training, each of the possible 256 codewords are associated with a class (i.e. car, truck, and no vehicle).  

The training procedure, which is performed offline, is to iterate through comparator threshold values (20 steps of $10mV$).  For each threshold, the following steps are performed: ($1$) thresholding is applied to the RMS data to obtain an 8-bit codeword for each time step, ($2$) the distribution of each class (i.e. car, truck, and no vehicle; combined across all observations of the class) across all codewords is computed, ($3$) each codeword is assigned to the class which was most likely to result in observing that codeword (i.e. the class which causes that codeword for the largest percentage of time), then ($4$) the combination of comparator threshold and codewords is evaluated by finding the percentage of time samples which are associated with the correct class.  After iterating through the threshold values, the threshold which resulted in the largest percentage of correct decisions in step 4 is chosen as the final threshold and the codeword assignments found in step 3 for that threshold are used as the final codeword assignments.

Once the comparator threshold and codeword assignments have been determined, the system is configured by ($1$) transmitting the threshold value to the mote and ($2$) by programming the codeword assignments into the CPLD via the JTAG header on the circuit board.  The CPLD is programmed such that the interrupt pin goes high whenever a codeword associated with either a car or truck is encountered, and the classification pin goes high whenever a truck is encountered.  

The ASP is susceptible to occasional false decisions depending upon the instantaneous signal-to-noise ratio. Hence, it is possible that an interrupt pin is reset despite the presence of a vehicle and the GPIO pins provide a false classification output. In order to compensate for these false decisions, we use the mote to generate the final classification output based on inputs from the ASP over a length of time. Once an interrupt has been generated by the ASP, the mote stays on and records the state of the interrupt pin and the GPIO pin until the interrupt stays low continuously for a duration of $100ms$ confirming that a vehicle is outside the sensing range. The mote then generates the final classification result as the most frequent decision from the ASP over the duration of the event.

\noindent{\bf Testing:}~~All testing was performed by streaming the samples into the ASP using a 16-bit DAC at a samping frequency of 4kHz.  The operation and the power consumption of the ASP-interfaced mote is shown in Fig.~\ref{truck} in the form of a timing diagram for one  $10$ second test sample of a truck (Fig.~\ref{truck}(i)) being classified. The truck is closest to the sensor between seconds $4-6$ of the test sample (shown in Fig.~\ref{truck}(ii)). The spectral analysis output of the event detector front-end is shown in Fig.~\ref{truck}(iii) in the form of a spectrogram.  The comparator outputs of the $8$ filter-bands are shown in Fig.~\ref{truck}(iv) and the CPLD outputs are shown in Fig.~\ref{truck}(v) and Fig.~\ref{truck}(vi). The CPLD interrupt pin goes high when a car or truck is detected. The CPLD class pin specifies the classification (high for truck, low for car) and is only valid when the interrupt pin is high. Once the interrupt is generated, the mote is awake and starts accumulating the classifications from the CPLD (consuming about $1.5$ mW of power). When a final decision is made, the output is transmitted via radio and this consumes $60$mW (Fig.~\ref{truck}(vii)). The detailed power consumption of the ASP-interfaced mote for the various operations being performed are shown in Table~\ref{tab:powTabI}. The accuracy of classification is highlighted in Table~\ref{tab:classResults}. An overall accuracy of $90\%$ is achieved with an average false alarm rate of one false positive every 50 seconds in the presence of amplified ambient wind noise.







\begin{table}[!t]
  \renewcommand{\arraystretch}{1.3}
  \caption{Vehicle Classification Results}
  \label{tab:classResults}
  \centering
  \begin{tabular}{|c|c|c|c|}
    \hline
    & \multicolumn{3}{|c|}{Ground Truth} \\ \hline
    & NULL & Car & Truck \\ 
    & (200 seconds) & (10 Samples) & (10 Samples) \\ \hline
    NULL &  & 20\% & 0\% \\ \hline
    Car & 2 false alarms & 80\% & 0\% \\ \hline
    Truck & 2 false alarms & 0\% & 100\% \\ \hline
  \end{tabular}
\end{table}

\noindent{\bf Comparison with all-digital implementation:}~~Low-power computing platforms such as the Telos mote are unable to perform spectral analysis on-board and therefore processing platforms such as the Stargate have to be used to perform signal processing. Since these devices consume significantly higher power, they are typically used in  a layered architecture in conjunction with mote platforms that act as wakeup devices to trigger the detection of an event. In \cite{lanl}, an all-digital implementation using such a layered architecture is used for the vehicle classification system described above. A low power Mica2 mote attached to a seismic sensor stays on all the time to detect the arrival of a vehicle. Upon detection of a vehicle, the mote triggers a wake-up signal to Linux based Stargate platform that performs spectral analysis for vehicle classification. The mote stays on all the time and consumes $1.5$mW of power when processing and $60$mW when transmitting. The Stargate running off of a $4.2$V battery consumes $420$-$470$ mA when processing for a duration of $8-10$ seconds per vehicle detection. In comparison, the cooperative analog-digital implementation described in this paper consumes only $417~\mu W$ of power when idle and $1.5$ mW when an event is detected. 

\noindent{\bf Discussion:}~We note that in our implementation, we used the ASP to output binary decision bits via the GPIO pins on the mote and used these bits to make the final classification output. Alternatively, the mote can be used to sample the RMS energy of each subband as described in Section~\ref{sec:ss} while the interrupt pin stays high and then use the sampled spectrogram of the signal to make a decision. Such an approach is likely to be beneficial in a more general classification scenario with much more than $2$ classes. 

