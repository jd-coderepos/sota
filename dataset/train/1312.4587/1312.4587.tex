

\documentclass[conference,10pt]{IEEEtran}











\usepackage{url}
\usepackage{float}
\usepackage{comment}
\usepackage{color}
\usepackage{graphicx}
\usepackage[font=footnotesize,labelfont=bf]{caption}
\usepackage{subcaption}
\usepackage{amsmath,amsfonts,amssymb}
\usepackage{algorithm}
\usepackage{algorithmic}
\usepackage{multirow}

\usepackage{amsthm}
\usepackage{verbatim, url}
\newtheorem{mydef}{Definition}
\newcounter{theorem}
\newcounter{def}
\setcounter{theorem}{0}
\setcounter{def}{0}
\newcounter{mylea}
\newcounter{corollary}
\setcounter{corollary}{0}
\setcounter{mylea}{0}
\usepackage{fancyhdr}



\newtheorem{definition}[def]{Definition}
\newtheorem{lemma}[mylea]{Lemma}



\newtheorem{mythe}[theorem]{Theorem}
\newtheorem{myclm}[theorem]{Claim}
\newtheorem{mycol}[corollary]{Corollary}

\renewcommand{\algorithmicrequire}{\textbf{Input:}}
\renewcommand{\algorithmicensure}{\textbf{Output:}}

\pagestyle{plain}
\pagenumbering{arabic}








\begin{document}

\title{\huge \bf FFTPL: An Analytic Placement Algorithm Using \\ Fast Fourier Transform for Density Equalization}





\author{
\IEEEauthorblockN{Jingwei Lu, Pengwen Chen, Chin-Chih Chang, Lu Sha, \\
Dennis Jen-Hsin Huang, Chin-Chi Teng, Chung-Kuan Cheng}
\IEEEauthorblockA
{Department of Computer Science and Engineering, University of California, San Diego \\
Department of Applied Mathematics, National Chung Hsing University \\
Cadence Design Systems \\
jlu@cs.ucsd.edu, pengwen@nchu.edu.tw, chinchih@cadence.com, lusha@cadence.com, \\
dhuang@cadence.com, ccteng@cadence.com, ckcheng@ucsd.edu\\}}
\maketitle
\thispagestyle{empty}
\renewcommand{\headrulewidth}{0pt}
\begin{abstract}
\label{sec:abs}
We propose a flat nonlinear placement algorithm FFTPL
using fast Fourier transform for density equalization.
The placement instance is modeled as an electrostatic system 
with the analogy of density cost to the potential energy.
A well-defined Poisson's equation is proposed for 
gradient and cost computation.
Our placer outperforms state-of-the-art placers 
with better solution quality and efficiency.
\end{abstract}



\section{Introduction}
\label{sec:intro}

Placement remains an important role in the 
VLSI physical design automation~\cite{abk} 
with impacts to congestion analysis~\cite{cp00},
clock tree synthesis~\cite{dmst}
and routing~\cite{lmgr}.
Placement quality is usually evaluated by
the total half-perimeter wirelength (HPWL), 
which correlates with timing, power, cost and 
routability.
Traditional placement methods can be generally divided 
into four categories.
{\bf Stochastic} approaches~\cite{timberwolf}
are usually based on simulated annealing. 
Despite good solution quality, the runtime is quite long. 
{\bf Min-cut} approaches~\cite{capo} comprises 
recursive problem partitioning and local optimum solution. 
However, improper partitioning would cause 
unrecoverable quality loss.
{\bf Quadratic} approaches~\cite{simpl,rql,fp3} 
approximate the net length using quadratic 
functions 
which enables gradient-based minimization.
By solving the system, 
cells are dragged away from over-filled regions 
with quadratic wirelength overhead. 
Nonetheless, the modeling accuracy remains a long-term issue.
{\bf Nonlinear} approaches~\cite{mpl6,ntupl3,aplace2} refer
to the algorithms using nonlinear optimization framework. 
Wirelength and density are modeled by smooth mathematical functions 
where gradient can be analytically calculated.
Due to the high computation complexity,
nonlinear approaches usually employ multi-level cell
clustering with quality overhead introduced. 

{\bf In this work}, 
we develop a flat nonlinear
placement algorithm 
which produces better and faster solution. 
The placement instance 
is modeled as an electrostatic system 
which induce one density constraint. 
The electric potential and field are coupled with density 
by a well-defined Poisson's equation, which is numerically 
solved using fast Fourier transform (FFT). 
Our algorithm is validated through experiments 
on the ISPD 2005 benchmark suite.

The remainder of the paper is organized as follows.
In Section~\ref{sec:ana}, 
we review the previous works 
and discuss their existing problems. 
In Section~\ref{sec:new}, 
we propose a new formulation of the density constraint 
with numerical solutions.
In Section~\ref{sec:gp} and~\ref{sec:exp}, 
we discuss and validate our placement algorithm.
We conclude the work in Section~\ref{sec:conc}.



\section{Essential Concepts and Related Works}
\label{sec:ana}

Placement instance is formulated 
as a hyper-graph  with 
nodes  nets  and region .
Let  and  denote 
movable nodes (cells) and 
fixed nodes (macros) with .
A placer determines all the cell locations 
, 
where  and 
 are the 
horizontal and veritical cell coordinates.
 is named as a placement solution.
We have the placement region  uniformly 
decomposed into  
rectangular grids (bins) 
denoted as .
The HPWL of each net  is denoted as
 while the total HPWL  
is the sum of HPWL of all the nets.
{\footnotesize
}
Analytic global placement targets minimum total HPWL 
subject to the constraint that the ratio of cell area 
to the site area of every bin  
(denoted as bin density ) 
does not exceed 
the target density 

As neither the wirelength function  
nor the density function  is differentiable, 
smoothing techniques are 
developed to improve the optimization quality.

{\bf Wirelength modeling} functions 
can be divided into two categories.
{\bf Log-Sum-Exp (LSE)} 
wirelength model is proposed 
in~\cite{naylor} and widely used in recent academic 
placers~\cite{mpl6,ntupl3,aplace2}.
{\bf Weighted-Average (WA)} function is recently proposed 
in~\cite{wa} with smaller modeling error 
compared to that of LSE, the equation for the horizontal 
wirelength is 

\cite{wa} shows that 
the function 
is strictly convex and
converges to HPWL as 
the smoothing parameter  
approaches zero.


{\bf Density modeling} techniques
generally form two categories.
{\bf Local smoothing} 
functions~\cite{naylor} 
replaces the piece-wise linear original 
density function
with a ``bell-shaped'' quadratic function~\cite{ntupl3,aplace2}.
As only local information is involved,
more iterations may be consumed before the solution 
converges. 
{\bf Global smoothing} techniques 
use elliptic PDE and have many applications 
in modern nonlinear placers~\cite{mpl6}. 
Global information incorporation enables large-scale cell motion. 
Helmholtz equation is proposed in~\cite{mpl6} as below

where  is the smoothed density distribution. 
A unique solution can be produced when the linear factor .
However, the smoothing effect becomes sensitive.

{\bf Nonlinear global placement} formulates the 
problem as an unconstrained nonlinear optimization.
In~\cite{ntupl3,aplace2} 
the density constraints are relaxed 
using quadratic penalty method

The approach in~\cite{mpl6}
assigns all the grid density  
with penalty factor .
However, this consumes longer runtime.



\section{Electrostatic System Modeling}
\label{sec:new}

We model the placement instance as an 
independent electrostatic system for 
density function transformation. 
Each node  is converted to a
positively charged particle with  
the electric quantity  equals
the node area . 
Let  and  denote the 
electric potential and field at cell ,  
We have the potential energy and electric force 
for each cell as  
and , respectively.
An example is shown in Figure~\ref{fig:dfp}.
\begin{figure*}\centering
  \begin{subfigure}[b]{0.33\textwidth}
    \centering
    \includegraphics[keepaspectratio, width=1.05\textwidth]{d50.eps}
    \caption{}
    \label{subfig:den}
  \end{subfigure}
  \begin{subfigure}[b]{0.33\textwidth}
    \centering
    \includegraphics[keepaspectratio, width=1.05\textwidth]{fx50.eps}
    \caption{}
    \label{subfig:field}
  \end{subfigure}\begin{subfigure}[b]{0.33\textwidth}
    \centering
    \includegraphics[keepaspectratio, width=1.05\textwidth]{p50.eps}
    \caption{}
    \label{subfig:phi}
  \end{subfigure}
  \caption{The distribution of electric (a) density (b) field (c) potential without filler insertion. Snapshots are extracted at iteration 50 of ADAPTEC1.} 
\label{fig:dfp}
\end{figure*}


\subsection{Electrostatic Equilibrium}

We use electric force for cell movement direction
and density equalization. 
Direct-current (DC) component is removed
from the density function such that 
under-filled regions become negatively charged. 
Cells locating at positive regions are 
attracted for neutralization.
In the end, the system reaches the electrostatic 
equilibrium state with zero bin density and potential 
energy.








\subsection{Potential Energy Computation}

Similar to~\cite{filler,mpl6}, we add disconnected "fillers" to 
induce density force for connected cells clotting thus 
interconnect shortenning.
Placement region could be irregular polygon 
(bounding box ). 
We name each non-placeable rectangular region within   
as a ``dark node''.
Cells are pushed away by the density force  
when approaching the chip boundary.
Also, the area of fixed and dark nodes
must be scaled down by the target density 
to globally balance the density force.
Let  and  denote the sets of 
fillers and dark nodes and 
, 
the potential energy is computed as




\subsection{Density Constraint Formulation}

By applying the penalty factor , 
we formulate an unconstrained optimization problem 

Compared to the quadratic penalty method~\cite{ntupl3,aplace2} or 
the multiple constraints~\cite{mpl6}, 
our method has lower complexity and better quality. 
The gradient vector is 
 where 
 and  are the electric quantity and 
field vectors of all the cells.
As the electric force always points to the steepest descent 
of system energy, 
we could dynamically balance the wirelength and density forces
using penalty factors.

\subsection{Well-Defined Poisson's Equation}

By Gauss' law, 
the potential and field 
are coupled with density by Poisson's equation.

 is the outer unit normal and 
 is the boundary of . 
 is a differential operator.
As electric force decreases to zero at boundary 
to prevent cells from moving outside, 
we select Neumann boundary condition. 
The potential integral is set to zero 
thus Poisson's equation has unique solution.

\subsection{Fast Numerical Solution}

We use fast Fourier transform 
to solve the Poisson's equation~\cite{fft_poi}.
Discrete sine transform (DST) is used  
to represent the electric field, 
which well satisfies the Neumann boundary condition. 
Therefore, potential and density functions are 
represented by discrete cosine transform (DCT).
We first mirror the function domain
from  to 
, 
then periodically extend it to infinity.
The density function can thus be expressed as 

where  and 
 are frequency components 
and  are coefficients. 

The solution to the potential function can be expressed as 

Therefore, we have the electric field distribution 
 shown as below

The above equations can be efficiently solved 
using many FFT algorithms~\cite{fft}.
Suppose we have  cells in the netlist.
In each iteration,
we reset the grid density using  time 
followed by density update using  time due to netlist traversal.
The FFT computation consumes  time. 
In Section~\ref{subsec:gp_par} we define  thus the 
total complexity is essentially .







\section{Global Placement Algorithm}
\label{sec:gp}
\vspace{-0.05in}

\begin{figure}[http]
\centering
\includegraphics[width=1.0\columnwidth, angle=0]{flow.eps}
\caption{The entire flow of initial, global and detail placement.}
\label{fig:flow01}
\end{figure}



The flow of the entire placement optimization is shown 
in Figure~\ref{fig:flow01}.
The initial placement solution  is based on 
quadratic wirelength minimization 
using bound-2-bound (B2B) net model~\cite{kw2}.
The global placement problem is solved 
using nonlinear Conjugate Gradient (CG) method.
After global placement completes, 
all the filler cells are removed from the 
solution , which is then legalized 
and discretely optimized
using FastDP~\cite{fastdp}.
with greedy flipping~\cite{capo}.


\subsection{Self-Adaptive Parameter Adjustment}
\label{subsec:gp_par}

{\bf Grid dimension}  is statically determined 
before the global placement based on the number of cells 
.
As required to be power of 2 in~\cite{fft}, 
we set  with upper-bound of .
{\bf Step length} correlates with the search interval of which the 
length is dynamically updated.
The initial value is determined as 
, where  is the grid width.
The search interval is iteratively updated as 
 and 
.
{\bf Penalty factor} is 
initially set as~\cite{ntupl3,aplace2}. 
Unlike those methods with 
constant scaling, 
we iteratively update  
to balance the wirelength and density forces.
The scaling factor is determined by 

based on HPWL variation 
. 
In practice, we set the reference variation 
 and bound  by .
{\bf Density overflow} is used to terminate the 
global placement process 
Similar to Eq.~(11) in~\cite{ntupl3},
we use the density overflow  as the stopping criterion 
The global placer terminates when .
As illustrated in Figure~\ref{fig:ovf}(a), 
system energy is consistent with the density overflow.
\begin{figure}\centering
  \begin{subfigure}[b]{0.25\textwidth}
    \centering
    \includegraphics[keepaspectratio, width=1.00\textwidth]{ovf.eps}
    \caption{}
    \label{subfig:ovf}
  \end{subfigure}\begin{subfigure}[b]{0.25\textwidth}
    \centering
    \includegraphics[keepaspectratio, width=0.90\textwidth]{wlen.eps}
    \caption{}
    \label{subfig:wlen}
  \end{subfigure}
  \caption{The illustration of (a) total overflow ratio  and potential energy  (b) total HPWL  and smoothed wirelength .}
\label{fig:ovf}
\end{figure}
{\bf Wirelength coefficient} is used together with WA model~\cite{wa} 
to smooth the HPWL as Figure~\ref{fig:ovf}(b) shows. 
The smoothing parameter  is larger at early time
to encourage global movement 
and smaller at later iterations to enable movement of 
only HPWL-insensitive cells.
We set the smoothing parameter as  
.



\subsection{Global Placement}
\label{subsec:gp_flow}

The detail flow of our global placement method FFTPL 
is shown in Algorithm~\ref{alg:gp}. 
We solve the Poisson's equation 
at line 5 by FFT library call~\cite{fft}. 
The global placement solution  is output to the legalizer 
and detail placer at line 13. 
\begin{algorithm}
\caption{FFTPL}
\label{alg:gp}
\begin{algorithmic}[1]
\REQUIRE 
initial placement solution  \\
\ENSURE
global placement solution 
\STATE initialize , . \FOR {}
\STATE 
\STATE compute density function 
\STATE FFTsolver
\STATE compute gradient 
\STATE CGsolver
\STATE update , , ,  \IF {} \STATE , break 
\ENDIF
\ENDFOR
\STATE {\bf return} 
\end{algorithmic}
\end{algorithm}










\section{Experiments and Results}
\label{sec:exp}


\begin{table*}[htb]
\caption{HPWL () and runtime (minutes) of all the placers on the ISPD 2005 benchmark suite~\cite{ispd05} 
using official script for performance evaluation.
Experiments are conducted under our 2.67GHz linux machine in single-thread mode. 
Average results are normalized to that of FFTPL (our work).}
\begin{small}
\begin{center}
\begin{tabular}{|c|r|r|r|r|r|r|r|r|r|r|r|r|r|} 
\hline
\multicolumn{2}{|c|}{Categories}   &
\multicolumn{2}{|c|}{Min-Cut}      &
\multicolumn{2}{|c|}{Quadratic}    &
\multicolumn{8}{|c|}{Nonlinear}   \\ \cline{1-14}
\multicolumn{2}{|c|}{Placers}      & 
\multicolumn{2}{|c|}{Capo10.5~\cite{capo}}     &
\multicolumn{2}{|c|}{FastPlace3.0~\cite{fp3}} & 
\multicolumn{2}{|c|}{APlace2~\cite{aplace2}}      &
\multicolumn{2}{|c|}{NTUPlace3~\cite{ntupl3}}    &
\multicolumn{2}{|c|}{mPL6~\cite{mpl6}}         & 
\multicolumn{2}{|c|}{FFTPL}       \\ \hline
Circuits~\cite{ispd05} &\#Cells& HPWL & CPU  & HPWL & CPU  & HPWL & CPU  & HPWL & CPU  & HPWL & CPU & HPWL & CPU \\ \hline
ADAPTEC1   & 211K  & 87.80& 48.33& 78.34& 2.92 & 78.35& 48.88& 80.29& 7.17 & 77.93& 23.27 & 76.46 &  9.17 \\ \hline
ADAPTEC2   & 255K  &102.66& 61.63& 93.47& 4.13 & 95.70& 68.07& 90.18& 8.22 & 92.04& 24.75 & 85.57 & 12.67 \\ \hline
ADAPTEC3   & 452K  &234.27&133.43&213.48& 9.53 &218.52&186.67&233.77& 18.53&214.16& 73.97 & 202.16& 45.40 \\ \hline
ADAPTEC4   & 496K  &204.33&141.85&196.88& 8.75 &209.28&209.60&215.02& 23.53&193.89& 71.03 & 185.83& 34.33 \\ \hline
BIGBLUE1   & 278K  &106.58& 77.90& 96.23& 4.57 &100.02& 64.05& 98.65& 14.30& 96.80& 30.05 & 91.64 & 23.63 \\ \hline
BIGBLUE2   & 558K  &161.68&150.15&154.89& 8.00 &153.75&136.43&158.27& 35.10&152.34& 79.00 & 145.54& 30.83 \\ \hline
BIGBLUE3   & 1097K &403.36&373.87&369.19& 21.05&411.59&289.78&346.33& 38.77&344.10& 104.63& 359.00& 116.67\\ \hline
BIGBLUE4   & 2177K &871.29&730.42&834.04& 40.13&945.77&779.22&829.09&106.08&829.44& 238.82& 805.90& 165.00\\ \hline \hline
\multicolumn{2}{|c|}{Average}  &1.14  &4.13  &1.05  & 0.25 & 1.10 & 4.41 & 1.07 & 0.66 & 1.04 & 1.80  & 1.00  & 1.00  \\ \hline
\end{tabular}
\label{tab:res}
\end{center}
\end{small}
\end{table*}

We implement our algorithm 
using C programming language and 
execute the program in a
Linux operating system with Intel i7 920 2.67GHz CPU
and 12GB memory. 
In our experiments, 
we use the benchmark suite from~\cite{ispd05}.
The target placement density  is set 
to be  for all the benchmarks.
There is no parameter tuning towards specific benchmarks.

We include 
five cutting-edge placers for performance comparison 
with their source code or binary obtained. 
All the results are shown in Table~\ref{tab:res}.
On average, our placer improves the total wirelength by
,
,
,
 and 
 
over
Capo10.5~\cite{capo}, 
FastPlace3.0~\cite{fp3},
APlace2~\cite{aplace2},
NTUPlace3~\cite{ntupl3} and 
mPL6~\cite{mpl6}, 
respectively.
Compared to the published results in 
RQL~\cite{rql} and SimPL~\cite{simpl}, 
our placer produces better solutions  
in six and seven out of the totally eight
benchmarks, respectively.
















\section{Conclusion}
\label{sec:conc}
\vspace{-0.05in}

In this paper, 
we propose a flat nonlinear global placement algorithm 
with improved quality and efficiency. 
The placement instance is modeled as an electrostatic system, 
where electric potential and field are computed using Poisson's equation. 
In future, we will extend our algorithm to parallel platform 
and other design objectives (timing, congestion, etc.).



\section{Acknowledgement}
\label{sec:ack}
The authors would like to acknowledge the support of NSF CCF-1017864.


\bibliographystyle{abbrv}
\bibliography{pl}

\end{document}
