\appendix
\setcounter{proposition}{0}


The appendix contains   proofs of the propositions  listed in the paper.
We also give proofs of lemmas  used in the proofs of propositions.

\section*{Propositions of Section~\ref{sec:pqe}: Partial Quantifier Elimination}
\begin{proposition}
Let  be a CNF formula such that  where
. (That is  is a solution to the QE problem.)
Then the assignments satisfying  specify the range of .
\end{proposition}
\begin{proof}
Let us show that  indeed specifies the range of . Let \pnt{z} be
a complete assignment to  that is in the range of . Then there
is an assignment  satisfying  and hence
\prob{W}{G} evaluates to 1 when variables  are assigned as in 
\pnt{z}. Hence  has to be equal to 1.  Now assume that
\pnt{z} is not in the range of . Then no assignment
 satisfies . So \prob{W}{G} evaluates to
0 for assignment \pnt{z}. Then  has to be equal to 0.
\end{proof}

\begin{proposition}
Let  be a clause depending only on input variables of .  Let
 be a CNF formula such that  where . (That is  is a solution
to the PQE problem.)  Let  and   be  a noise-free and noisy 
solution respectively.  Then
\begin{enumerate}
\item The assignments \ti{falsifying}  specify the range reduction
  in  caused by excluding inputs falsifying . That is 
  iff
\begin{itemize}
   \item there is an input \pnt{x} for which circuit  produces
   output \pnt{z} 
   \item all inputs for which  produces
   output \pnt{z} falsify 
\end{itemize}
\item \Impl{H^*}{H}
\end{enumerate}
\end{proposition}
\begin{proof}
\tb{First condition.} Let us prove that  indeed specifies the range
reduction of . Let \pnt{z} be a complete assignment to  that is
in the range of .  Assume that \pnt{z} remains in the range of 
even if the inputs falsifying clause  are excluded. Then there is
an assignment  satisfying  and
hence \prob{W}{C \wedge G} evaluates to 1 when variables of  are
assigned as in \pnt{z}. So,  has to be equal to 1.

Now assume that \pnt{z} is in the range of  but it is not in the
range of  if the inputs falsifying clause  are excluded. Then no
assignment  satisfies  and
hence \prob{W}{C \wedge G} evaluates to 0 when variables of  are
assigned as in \pnt{z}. On the other hand, since \pnt{z} is in the
range of , there is an assignment 
satisfying .  So formula \prob{W}{G} is equal to 1 when variables
of  are assigned as in \pnt{z}. Since  is
equal to 0 when variables of  are assigned as in \pnt{z}, then
 has to be equal to 0.

 Now assume that \pnt{z} is not in the range of . Then no
 assignment  satisfies . So
 both \prob{W}{G} and \prob{W}{C \wedge G} evaluate to 0. This means
 that the value of  is, in general, not defined. However,
 since we require  to be a noise-free solution,  has to
 be equal to 1.

\tb{Second condition.} As we showed above, any solution to the PQE
problem is defined uniquely for a complete assignment \pnt{z} to 
that is in the range of . So in this case,
.  If \pnt{z} is not in the range of , by
definition of a noise-free solution, .  So
 implies   and hence \Impl{H^*}{H}.
\end{proof}

\section*{Propositions of Section~\ref{sec:isol_publ_traces}: Isolated And Public Traces}
\begin{lemma}
\label{lemma:trace}
Let  be range reduction formulas computed with respect
to clause . Let  be an initialized trace \tr{s}{x}{0}{m}
such that 
\begin{itemize}
\item \IP{s_0}{x_0} falsifies  i.e.  is excluded by 
\item \IP{s_i}{x_i} falsifies , .
\end{itemize} 
  Let  be an initialized
trace \tr{s'}{x'}{0}{m} that is allowed by clause .  Then
,  and hence .
\end{lemma}
\begin{proof}
Since \pnt{s'_i} is in , it is reachable by a trace allowed by
. From Definition~\ref{def:rr_formulas} it follows that
.
\end{proof}


\begin{proposition}
Let  be a clause such that \Impl{\overline{C}}{I}. Let
 be range reduction formulas computed with respect to
clause . Let  denote an initialized trace \tr{s}{x}{0}{m} such
that
\begin{itemize}
\item \IP{s_0}{x_0} falsifies  i.e.  is excluded by 
\item \IP{s_i}{x_i} falsifies , .
\end{itemize} 
 Then  is isolated with respect to .
\end{proposition}

\begin{proof}
Assume that  is not isolated. 
Then there is an initialized  trace  such that
\begin{itemize}
\item  is allowed by 
\item \pnt{s_i}=\pnt{s'_i}, for some  such that .
\end{itemize}
The existence of such a trace contradicts Lemma~\ref{lemma:trace}.
\end{proof}



\begin{proposition}
Let  be a system with property . Let  be a clause such
that \Impl{\overline{C}}{I}. Assume that  does not exclude any
counterexample of length at most  isolated with respect to .
Then  is a -equivalent clause.
\end{proposition}
\begin{proof}
Assume the contrary i.e. every counterexample of length at most  is
excluded by  and so  is not -equivalent. Let
=\tr{s}{x}{0}{m} be a counterexample of length  excluded
by .  By assumption,  is not isolated with respect to . Then
there is an initialized trace  equal to \tr{s'}{x'}{0}{k},  such that
\begin{itemize}
\item  is allowed by 
\item \pnt{s'_k} = \pnt{s_k}.
\end{itemize}
Let  be a sequence of input pairs
\tr{s'}{x'}{0}{k},\tr{s}{x}{k+1}{m}. Since  is obtained by
stitching together two traces and \pnt{s'_k} = \pnt{s_k},  is a
trace.  Since \pnt{s'_0} is an initial state,  is an initialized
trace. Since  transitions to a bad state under
input \IP{s_m}{x_m}  is a counterexample.  Since \IP{s'_0}{x'_0}
satisfies ,  is allowed by .  So  does not exclude all
counterexamples of length at most  and we have a contradiction.
\end{proof}

\section*{Propositions of Section~\ref{sec:bug_hunting}: Bug Hunting By CRR}


\begin{proposition}
Let  be a state transition system with property . Let 
be a non-empty clause such that \Impl{\overline{C}}{I}.  Let 
denote formula equal to . Let formulas  be obtained
recursively as follows.  Let  denote formula equal to .
Let  denote formula . Here  where  and  are state and input
variables of -th time frame.  Formula  is obtained by
taking  out of the scope of quantifiers in
formula \prob{W}{H_i \wedge T_i \wedge \Phi_i} where .  That is
.  Then formulas  are range
reduction formulas.
\end{proposition}
\begin{proof}
Let us prove this proposition by induction on . This proposition is
vacuously true for . Assume that it holds for . Let
us show that then this proposition holds for .  That is one needs
to show that  is a range reduction formula and hence
 iff \pnt{s} is reachable in  transitions only
by traces excluded by .  Assume that  is not a range
reduction formula.  Then one needs to consider the two cases below.

A)  and \pnt{s} is not reachable by any
initialized trace of length . This means that \pnt{s} cannot be
extended to satisfy formula .
Hence \pnt{s} cannot be extended to satisfy formula . Then the clause of maximal length falsified
by \pnt{s} is implied by . This means that  is a
``noisy'' solution of the PQE problem and hence cannot be obtained by
a noise-free PQE solver. So we have a contradiction.

B) The set of initialized traces of length  reaching
state \pnt{s} is not empty but at least one trace of this set is
allowed by . Let  be such a trace. The fact that 
reaches \pnt{s} means that  satisfies formula . Since  is a trace allowed by  it also
satisfies . Moreover,  has to satisfy all the formulas ,
. Indeed, if  falsifies  then there is a
initialized trace of length  that is allowed by  and that
reaches a state excluded by . This means that  is not a
range reduction formula.  So  satisfies , and
hence formula  is satisfied by . This
means that formula  is not implied by . Hence 
is not equivalent to
\prob{W}{H_n \wedge T_n \wedge \Phi_n} and we have a contradiction.
\end{proof}
\begin{lemma}
\label{lemma:impl}
Let  be  CNF formulas such that
\begin{itemize}
\item 
\item 
\item \Impl{F'}{F''}
\end{itemize}
Let , be obtained by a noise-free PQE solver.
Then \Impl{H'}{H''} holds.
\end{lemma}
\begin{proof}
Let  denote the set of free variables. Assume the contrary
i.e. \Nmpl{H'}{H''}.  Then there is a complete assignment \pnt{y} to
 such that  and . The latter means
that 
\begin{enumerate}
\item \prob{X}{F'' \wedge G}=0 in subspace \pnt{y} and so every
assignment (\pnt{x},\pnt{y}) falsifies 
\item Since  is obtained by a noise-free PQE solver,
\Nmpl{G}{C} where  is the longest clause falsified by \pnt{y}.
So there is an assignment (\pnt{x},\pnt{y}) satisfying .
\end{enumerate}

The fact that every assignment (\pnt{x},\pnt{y}) falsifies  and that \Impl{F'}{F''} entails that every assignment
(\pnt{x},\pnt{y}) falsifies  as
well. So \prob{X}{F' \wedge G}=0 in subspace \pnt{y}. This means that
 in subspace \pnt{y} as well. The fact that
there is an assignment (\pnt{x},\pnt{y}) satisfying  and 
depends only on variables of  implies that .
So we have a contradiction.
\end{proof}
\begin{proposition}
Let  be formulas obtained as described in
Proposition~\ref{prop:noise_free_pqe} with only one exception. A \ti{noisy}
PQE-solver is used to obtain  by taking  out of the
scope of quantifiers in \prob{W}{H^*_i \wedge T_i \wedge \Phi^*_i}.
Here ,  and  for .
Then \Impl{H^*_i}{H_i} holds where  are range
reduction formulas.
\end{proposition}
\begin{proof}
We prove this proposition by induction on . 
\Impl{H^*_0}{H_0} holds because .
Now we prove that \Impl{H^*_i}{H_i},  entails that
\Impl{H^*_{i+1}}{H_{i+1}}. Denote by  a noise-free formula
obtained by taking  out of the scope of quantifiers
in \prob{W}{H^*_i \wedge T_i \wedge \Phi_i}. From
Lemma~\ref{lemma:impl} it follows that \Impl{Q_{i+1}}{H_{i+1}}. On the
other hand, from Proposition~\ref{prop:range_red} it follows that
\Impl{H^*_{i+1}}{Q_{i+1}}. Hence \Impl{H^*_{i+1}}{H_{i+1}}.
\end{proof}

\section*{Propositions of Section~\ref{sec:p_equiv_clauses}: Generation Of -Equivalent Clauses}
\begin{proposition}
Let  be a system with property . Let  be a clause such
that \Impl{\overline{C}}{I}. Let  be approximate
range reduction formulas computed with respect to clause  by a
noisy PQE solver.  Suppose that no formula ,~
excludes a reachable bad state \pnt{s}. Then clause  is
-equivalent.
\end{proposition}
\begin{proof}
From Proposition~\ref{prop:noisy_pqe} it follows
that \Impl{H^*_i}{H_i} where  is a range reduction formula.  So
that fact that  does not exclude a bad reachable state implies
that  does not exclude a bad state. This means that clause 
does not exclude an isolate counterexample of length at most . Then
Proposition~\ref{prop:p_equiv_clause} entails that  is
-equivalent.
\end{proof}

\begin{proposition}
Let  be a system with property . Let  be a clause such
that \Impl{\overline{C}}{I}. Let  be approximate
range reduction formulas computed with respect to clause  by a
noisy PQE solver. Suppose that every bad state excluded by ,  is unreachable. Suppose that every state (bad or good)
excluded by  is unreachable. Then clause  is
-equivalent for any .
\end{proposition}
\begin{proof}
From Proposition~\ref{prop:noisy_pqe} it follows
that \Impl{H^*_j}{H_j} where  is a range reduction formula.  So
that fact that ,  does not exclude a bad
reachable state implies that  does not exclude a bad state.  The
fact that every state excluded by  is unreachable means that
 is empty i.e. . Formula  is obtained by
taking  out of the scope of quantifiers in
formula \prob{W}{H_i \wedge T_i \wedge \Phi_i}. This means that
 and hence  does not exclude any bad states
either. So no formula ,  excludes a bad state. Hence clause
 is  equivalent for any .
\end{proof}
