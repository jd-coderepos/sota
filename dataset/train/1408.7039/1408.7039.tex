\appendix
\setcounter{proposition}{0}


The appendix contains   proofs of the propositions  listed in the paper.
We also give proofs of lemmas  used in the proofs of propositions.

\section*{Propositions of Section~\ref{sec:pqe}: Partial Quantifier Elimination}
\begin{proposition}
Let $R(Z)$ be a CNF formula such that $R \equiv \prob{W}{G}$ where
$W = X \cup Y$. (That is $R$ is a solution to the QE problem.)
Then the assignments satisfying $R(z)$ specify the range of $M$.
\end{proposition}
\begin{proof}
Let us show that $R$ indeed specifies the range of $M$. Let \pnt{z} be
a complete assignment to $Z$ that is in the range of $M$. Then there
is an assignment $(\pnt{x},\pnt{y},\pnt{z})$ satisfying $G$ and hence
\prob{W}{G} evaluates to 1 when variables $Z$ are assigned as in 
\pnt{z}. Hence $R(\pnt{z})$ has to be equal to 1.  Now assume that
\pnt{z} is not in the range of $M$. Then no assignment
$(\pnt{x},\pnt{y},\pnt{z})$ satisfies $G$. So \prob{W}{G} evaluates to
0 for assignment \pnt{z}. Then $R(\pnt{z})$ has to be equal to 0.
\end{proof}

\begin{proposition}
Let $C(X)$ be a clause depending only on input variables of $M$.  Let
$H(Z)$ be a CNF formula such that $H \wedge \prob{W}{G} \equiv
\prob{W}{C \wedge G}$ where $W = X \cup Y$. (That is $H$ is a solution
to the PQE problem.)  Let $H$ and $H^*$  be  a noise-free and noisy 
solution respectively.  Then
\begin{enumerate}
\item The assignments \ti{falsifying} $H$ specify the range reduction
  in $M$ caused by excluding inputs falsifying $C$. That is $H(\pnt{z})=0$
  iff
\begin{itemize}
   \item there is an input \pnt{x} for which circuit $M$ produces
   output \pnt{z} 
   \item all inputs for which $M$ produces
   output \pnt{z} falsify $C$
\end{itemize}
\item \Impl{H^*}{H}
\end{enumerate}
\end{proposition}
\begin{proof}
\tb{First condition.} Let us prove that $H$ indeed specifies the range
reduction of $M$. Let \pnt{z} be a complete assignment to $Z$ that is
in the range of $M$.  Assume that \pnt{z} remains in the range of $M$
even if the inputs falsifying clause $C$ are excluded. Then there is
an assignment $(\pnt{x},\pnt{y},\pnt{z})$ satisfying $C \wedge G$ and
hence \prob{W}{C \wedge G} evaluates to 1 when variables of $Z$ are
assigned as in \pnt{z}. So, $H(\pnt{z})$ has to be equal to 1.

Now assume that \pnt{z} is in the range of $M$ but it is not in the
range of $M$ if the inputs falsifying clause $C$ are excluded. Then no
assignment $(\pnt{x},\pnt{y},\pnt{z})$ satisfies $C \wedge G$ and
hence \prob{W}{C \wedge G} evaluates to 0 when variables of $Z$ are
assigned as in \pnt{z}. On the other hand, since \pnt{z} is in the
range of $M$, there is an assignment $(\pnt{x},\pnt{y},\pnt{z})$
satisfying $G$.  So formula \prob{W}{G} is equal to 1 when variables
of $Z$ are assigned as in \pnt{z}. Since $H \wedge \prob{W}{G}$ is
equal to 0 when variables of $Z$ are assigned as in \pnt{z}, then
$H(\pnt{z})$ has to be equal to 0.

 Now assume that \pnt{z} is not in the range of $M$. Then no
 assignment $(\pnt{x},\pnt{y},\pnt{z})$ satisfies $G$. So
 both \prob{W}{G} and \prob{W}{C \wedge G} evaluate to 0. This means
 that the value of $H(\pnt{z})$ is, in general, not defined. However,
 since we require $H$ to be a noise-free solution, $H(\pnt{z})$ has to
 be equal to 1.

\tb{Second condition.} As we showed above, any solution to the PQE
problem is defined uniquely for a complete assignment \pnt{z} to $Z$
that is in the range of $M$. So in this case,
$H(\pnt{z})=H^*(\pnt{z})$.  If \pnt{z} is not in the range of $M$, by
definition of a noise-free solution, $H(\pnt{z})=1$.  So
$H(\pnt{z})=0$ implies $H^*(\pnt{z})=0$  and hence \Impl{H^*}{H}.
\end{proof}

\section*{Propositions of Section~\ref{sec:isol_publ_traces}: Isolated And Public Traces}
\begin{lemma}
\label{lemma:trace}
Let $H_1,\dots,H_m$ be range reduction formulas computed with respect
to clause $C$. Let $E$ be an initialized trace \tr{s}{x}{0}{m}
such that 
\begin{itemize}
\item \IP{s_0}{x_0} falsifies $C$ i.e. $E$ is excluded by $C$
\item \IP{s_i}{x_i} falsifies $H_i$, $i=1,\dots,m$.
\end{itemize} 
  Let $E'$ be an initialized
trace \tr{s'}{x'}{0}{m} that is allowed by clause $C$.  Then
$H_i(\pnt{s'_i})=1$, $i=1,\dots,m$ and hence $\pnt{s_i} \neq
\pnt{s'_i}, i=1,\dots,m$.
\end{lemma}
\begin{proof}
Since \pnt{s'_i} is in $E'$, it is reachable by a trace allowed by
$C$. From Definition~\ref{def:rr_formulas} it follows that
$H_i(\pnt{s'_i})=1$.
\end{proof}


\begin{proposition}
Let $C(S,X)$ be a clause such that \Impl{\overline{C}}{I}. Let
$H_1,\dots,H_m$ be range reduction formulas computed with respect to
clause $C$. Let $E$ denote an initialized trace \tr{s}{x}{0}{m} such
that
\begin{itemize}
\item \IP{s_0}{x_0} falsifies $C$ i.e. $E$ is excluded by $C$
\item \IP{s_i}{x_i} falsifies $H_i$, $i=1,\dots,m$.
\end{itemize} 
 Then $E$ is isolated with respect to $C$.
\end{proposition}

\begin{proof}
Assume that $E$ is not isolated. 
Then there is an initialized  trace $E' =  \tr{s'}{x'}{0}{m}$ such that
\begin{itemize}
\item $E'$ is allowed by $C$
\item \pnt{s_i}=\pnt{s'_i}, for some $i$ such that $1 \leq i \leq m$.
\end{itemize}
The existence of such a trace contradicts Lemma~\ref{lemma:trace}.
\end{proof}



\begin{proposition}
Let $\xi$ be a system with property $P$. Let $C(S,X)$ be a clause such
that \Impl{\overline{C}}{I}. Assume that $C$ does not exclude any
counterexample of length at most $n$ isolated with respect to $C$.
Then $C$ is a $P^n$-equivalent clause.
\end{proposition}
\begin{proof}
Assume the contrary i.e. every counterexample of length at most $n$ is
excluded by $C$ and so $C$ is not $P^n$-equivalent. Let
$E$=\tr{s}{x}{0}{m} be a counterexample of length $m \leq n$ excluded
by $C$.  By assumption, $E$ is not isolated with respect to $C$. Then
there is an initialized trace $E'$ equal to \tr{s'}{x'}{0}{k}, $k \leq
m$ such that
\begin{itemize}
\item $E'$ is allowed by $C$
\item \pnt{s'_k} = \pnt{s_k}.
\end{itemize}
Let $E''$ be a sequence of input pairs
\tr{s'}{x'}{0}{k},\tr{s}{x}{k+1}{m}. Since $E''$ is obtained by
stitching together two traces and \pnt{s'_k} = \pnt{s_k}, $E''$ is a
trace.  Since \pnt{s'_0} is an initial state, $E''$ is an initialized
trace. Since $\xi$ transitions to a bad state under
input \IP{s_m}{x_m} $E''$ is a counterexample.  Since \IP{s'_0}{x'_0}
satisfies $C$, $E''$ is allowed by $C$.  So $C$ does not exclude all
counterexamples of length at most $n$ and we have a contradiction.
\end{proof}

\section*{Propositions of Section~\ref{sec:bug_hunting}: Bug Hunting By CRR}


\begin{proposition}
Let $\xi$ be a state transition system with property $P$. Let $C(S,X)$
be a non-empty clause such that \Impl{\overline{C}}{I}.  Let $H_0$
denote formula equal to $C$. Let formulas $H_1,\dots,H_n$ be obtained
recursively as follows.  Let $\Phi_0$ denote formula equal to $I$.
Let $\Phi_i,~~0 < i \leq n$ denote formula $I \wedge H_0 \wedge
T_0 \wedge \dots \wedge H_{i-1} \wedge T_{i-1}$. Here $T_j =
T(S_j,X_j,S_{j+1})$ where $S_j$ and $X_j$ are state and input
variables of $j$-th time frame.  Formula $H_{i+1}$ is obtained by
taking $H_i$ out of the scope of quantifiers in
formula \prob{W}{H_i \wedge T_i \wedge \Phi_i} where $W = S_0 \cup
X_0 \cup \dots \cup S_i \cup X_i$.  That is
$H_{i+1} \wedge \prob{W}{T_i \wedge \Phi_i} \equiv \prob{W}{H_i \wedge
T_i \wedge \Phi_i}$.  Then formulas $H_1,\dots,H_n$ are range
reduction formulas.
\end{proposition}
\begin{proof}
Let us prove this proposition by induction on $i$. This proposition is
vacuously true for $i=0$. Assume that it holds for $i=0,\dots,n$. Let
us show that then this proposition holds for $n+1$.  That is one needs
to show that $H_{n+1}$ is a range reduction formula and hence
$H_{n+1}(\pnt{s})=0$ iff \pnt{s} is reachable in $n+1$ transitions only
by traces excluded by $C$.  Assume that $H_{n+1}$ is not a range
reduction formula.  Then one needs to consider the two cases below.

A) $H_{n+1}(\pnt{s})=0$ and \pnt{s} is not reachable by any
initialized trace of length $n+1$. This means that \pnt{s} cannot be
extended to satisfy formula $I \wedge T_0 \dots \wedge T_n$.
Hence \pnt{s} cannot be extended to satisfy formula $\Phi_n \wedge
T_n \wedge H_n$. Then the clause of maximal length falsified
by \pnt{s} is implied by $\Phi_n$. This means that $H_{n+1}$ is a
``noisy'' solution of the PQE problem and hence cannot be obtained by
a noise-free PQE solver. So we have a contradiction.

B) The set of initialized traces of length $n+1$ reaching
state \pnt{s} is not empty but at least one trace of this set is
allowed by $C$. Let $E$ be such a trace. The fact that $E$
reaches \pnt{s} means that $E$ satisfies formula $I \wedge
T_0 \dots \wedge T_n$. Since $E$ is a trace allowed by $C$ it also
satisfies $C$. Moreover, $E$ has to satisfy all the formulas $H_i$,
$i=1,\dots,n$. Indeed, if $E$ falsifies $H_i$ then there is a
initialized trace of length $i$ that is allowed by $C$ and that
reaches a state excluded by $H_i$. This means that $H_i$ is not a
range reduction formula.  So $E$ satisfies $H_i$,$i=1,\dots,n$ and
hence formula $H_n \wedge T_n \wedge \Phi_n$ is satisfied by $E$. This
means that formula $H_{n+1}$ is not implied by $H_n \wedge
T_n \wedge \Phi_n$. Hence $H_{n+1} \wedge \prob{W}{T_n \wedge \Phi_n}$
is not equivalent to
\prob{W}{H_n \wedge T_n \wedge \Phi_n} and we have a contradiction.
\end{proof}
\begin{lemma}
\label{lemma:impl}
Let $F',F'',H',H'',G$ be  CNF formulas such that
\begin{itemize}
\item $H' \wedge \prob{X}{G} \equiv \prob{X}{F' \wedge G}$
\item $H'' \wedge \prob{X}{G} \equiv \prob{X}{F'' \wedge G}$
\item \Impl{F'}{F''}
\end{itemize}
Let $H'$,$H''$ be obtained by a noise-free PQE solver.
Then \Impl{H'}{H''} holds.
\end{lemma}
\begin{proof}
Let $Y$ denote the set of free variables. Assume the contrary
i.e. \Nmpl{H'}{H''}.  Then there is a complete assignment \pnt{y} to
$Y$ such that $H'(\pnt{y})=1$ and $H''(\pnt{y})=0$. The latter means
that 
\begin{enumerate}
\item \prob{X}{F'' \wedge G}=0 in subspace \pnt{y} and so every
assignment (\pnt{x},\pnt{y}) falsifies $F'' \wedge G$
\item Since $H''$ is obtained by a noise-free PQE solver,
\Nmpl{G}{C} where $C$ is the longest clause falsified by \pnt{y}.
So there is an assignment (\pnt{x},\pnt{y}) satisfying $G$.
\end{enumerate}

The fact that every assignment (\pnt{x},\pnt{y}) falsifies $F'' \wedge
G$ and that \Impl{F'}{F''} entails that every assignment
(\pnt{x},\pnt{y}) falsifies $F' \wedge G$ as
well. So \prob{X}{F' \wedge G}=0 in subspace \pnt{y}. This means that
$H' \wedge \prob{X}{G} = 0$ in subspace \pnt{y} as well. The fact that
there is an assignment (\pnt{x},\pnt{y}) satisfying $G$ and $H'$
depends only on variables of $Y$ implies that $H'(\pnt{y})=0$.
So we have a contradiction.
\end{proof}
\begin{proposition}
Let $H^*_i, i=0,\dots,n$ be formulas obtained as described in
Proposition~\ref{prop:noise_free_pqe} with only one exception. A \ti{noisy}
PQE-solver is used to obtain $H^*_{i+1}$ by taking $H^*_i$ out of the
scope of quantifiers in \prob{W}{H^*_i \wedge T_i \wedge \Phi^*_i}.
Here $\Phi^*_0 = I$, $H^*_0=C$ and $\Phi^*_i = I \wedge H^*_0 \wedge
T_0 \wedge \dots \wedge H^*_{i-1} \wedge T_{i-1}$ for $i < 0 \leq n$.
Then \Impl{H^*_i}{H_i} holds where $H_i,i=1,\dots,n$ are range
reduction formulas.
\end{proposition}
\begin{proof}
We prove this proposition by induction on $i$. 
\Impl{H^*_0}{H_0} holds because $H^*_0 = H_0 = I$.
Now we prove that \Impl{H^*_i}{H_i}, $i \geq 0$ entails that
\Impl{H^*_{i+1}}{H_{i+1}}. Denote by $Q_{i+1}$ a noise-free formula
obtained by taking $H^*_i$ out of the scope of quantifiers
in \prob{W}{H^*_i \wedge T_i \wedge \Phi_i}. From
Lemma~\ref{lemma:impl} it follows that \Impl{Q_{i+1}}{H_{i+1}}. On the
other hand, from Proposition~\ref{prop:range_red} it follows that
\Impl{H^*_{i+1}}{Q_{i+1}}. Hence \Impl{H^*_{i+1}}{H_{i+1}}.
\end{proof}

\section*{Propositions of Section~\ref{sec:p_equiv_clauses}: Generation Of $P$-Equivalent Clauses}
\begin{proposition}
Let $\xi$ be a system with property $P$. Let $C(S,X)$ be a clause such
that \Impl{\overline{C}}{I}. Let $H^*_1,\dots,H^*_n$ be approximate
range reduction formulas computed with respect to clause $C$ by a
noisy PQE solver.  Suppose that no formula $H^*_i$,~$i=1,\dots,n$
excludes a reachable bad state \pnt{s}. Then clause $C$ is
$P^n$-equivalent.
\end{proposition}
\begin{proof}
From Proposition~\ref{prop:noisy_pqe} it follows
that \Impl{H^*_i}{H_i} where $H_i$ is a range reduction formula.  So
that fact that $H^*_i$ does not exclude a bad reachable state implies
that $H_i$ does not exclude a bad state. This means that clause $C$
does not exclude an isolate counterexample of length at most $n$. Then
Proposition~\ref{prop:p_equiv_clause} entails that $C$ is
$P^n$-equivalent.
\end{proof}

\begin{proposition}
Let $\xi$ be a system with property $P$. Let $C(S,X)$ be a clause such
that \Impl{\overline{C}}{I}. Let $H^*_1,\dots,H^*_i$ be approximate
range reduction formulas computed with respect to clause $C$ by a
noisy PQE solver. Suppose that every bad state excluded by $H^*_j$, $1
\leq j < i$ is unreachable. Suppose that every state (bad or good)
excluded by $H^*_i$ is unreachable. Then clause $C$ is
$P^n$-equivalent for any $n > 0$.
\end{proposition}
\begin{proof}
From Proposition~\ref{prop:noisy_pqe} it follows
that \Impl{H^*_j}{H_j} where $H_j$ is a range reduction formula.  So
that fact that $H^*_j$, $1 \leq j < i$ does not exclude a bad
reachable state implies that $H_j$ does not exclude a bad state.  The
fact that every state excluded by $H^*_i$ is unreachable means that
$H_i$ is empty i.e. $H_i \equiv 1$. Formula $H_{i+1}$ is obtained by
taking $H_i$ out of the scope of quantifiers in
formula \prob{W}{H_i \wedge T_i \wedge \Phi_i}. This means that
$H_{i+1} = \equiv 1$ and hence $H_{i+1}$ does not exclude any bad states
either. So no formula $H_i$, $i > 0$ excludes a bad state. Hence clause
$C$ is $P^n$ equivalent for any $n > 0$.
\end{proof}
