\noindent In Sections~\ref{sec:sn}--\ref{sec:stepwise} we have devised
a characterisation of local termination in terms of monotone partial
algebras.  While this gives the general method, for the purpose of
obtaining automatable methods we strive for fruitful classes of these
algebras.  For global termination, instances of monotone algebras are
well-known.  This raises the natural question whether we can transform
a given monotone algebra for global termination in such a way that we
obtain a partial monotone algebra for local termination.

In this section we present one such approach.
We combine monotone partial models with (ordinary) monotone algebras.
The monotone partial models are roughly deterministic tree automata
that are closed under rewriting;
they describe the language of term on which we proof termination.
We search for such an automaton that accepts the starting language $\termset$
together with a monotone algebra such that the rewrite rules decrease
on the language of the automaton.
In this way monotone algebras for global termination carry over to local termination,
and we obtain an automatable method that 
is applicable for proofs of local termination.

First we give the definition of extended $\samumap$-monotone algebras as known from 
global termination of context-sensitive TRSs, see~\cite{luca:98,EWZ06}.
A mapping $\samumap \funin \asig \to \powerset{\nat}$
is called a \emph{replacement map (for $\asig$)} if for all $f \in \asig$ we have
$\amumap{f} \subseteq \{1,\ldots,\arity{f}\}$.
Let $\pair{\aset}{\sinterpret}$ be a $\asig$-algebra and $\samumap$ a replacement map.
For symbols $f \in \asig$ we say that the interpretation $\interpret{f} \funin \aset^{\arity{f}} \to \aset$
is \emph{$\samumap$-monotone} with respect to $\alggt$
if for every $a, b \in \aset$ and $i \in \amumap{f}$ with $a \alggt b$ we have:
\(\funap{f}{\underbrace{\ldots}_{i-1},a,\underbrace{\ldots}_{\arity{f}-i}} \alggt \funap{f}{\ldots,b,\ldots}\punc.\)

\begin{definition}\normalfont\label{def:globalalg}
  Let $\samumap$ be a replacement map for $\asig$.


  \noindent An \emph{extended well-founded $\samumap$-monotone $\asig$-algebra 
  $\quadruple{\aset}{\interpret{\cdot}}{{\alggt}}{{\algge}}$}
  is a $\asig$-algebra $\pair{\aset}{\interpret{\cdot}}$
  with two binary relations $\alggt$, $\algge$ on $\aset$ for which
  the following conditions hold:
\begin{enumerate}[(i)]
\item $\SNrel{{\alggt}}{\,{\algge}}$, and
\item for every $f \in \asig$ the function $\interpret{f}$ is $\samumap$-monotone with respect to $\alggt$ and $\algge$.
  \end{enumerate}
\end{definition}


A partial model $\aalg = \triple{\aset}{\sinterpret}{{\qge}}$
may contain elements $a \in \aset$
for which $\interpret{t} = a$ implies that $t$ is a normal form.
For a given partial model the set of these, which we denote by $\nf{\aset}{\atrs}$, can be computed (see below Definition~\ref{def:nfalg}).
We can exploit this knowledge as follows:
if a certain argument of a symbol $f \in \asig$
is always a normal form, then its interpretation $\interpret{f}$
does not need to be monotonic for this argument position.
The following definition gives an algorithm for computing the set $\nf{\aset}{\atrs}$.
Elements that are interpretations $\interpreta{\ell}{\alpha}$ of left-hand sides in $\atrs$
cannot belong to this set. Moreover if $a \not\in \nf{\aset}{\atrs}$
and $b = \funap{\interpret{f}}{\ldots,a,\ldots}$ then we conclude $b \not\in \nf{\aset}{\atrs}$.
This is formalised as follows:
\begin{definition}\normalfont\label{def:nfalg}
  Let $\atrs$ be a TRS over the signature $\asig$, and $\aalg = \pair{\aset}{\sinterpret}$ a partial $\asig$-algebra.
  The \emph{normal forms $\nf{\aset}{\atrs}$ of $\aalg$}
  are the largest set $\nf{\aset}{\atrs} \subseteq \core{\aalg}$ such that
  $\interpreta{\ell}{\alpha} \not\in \nf{\aset}{\atrs}$
  for every $\ell \to r \in \atrs$ and every $\alpha \funin \vars{\ell} \to \core{\aalg}$,
  and
  $\funap{\interpret{f}}{a_1,\ldots,a_n} \not\in \nf{\aset}{\atrs}$ 
  for every $f \in \asig$, $a_i \not\in \nf{\aset}{\atrs}$ and $a_1,\ldots,a_n \in \core{\aalg}$.
\end{definition}
Then by construction we obtain the following lemma:
\begin{lemma}\label{lem:nfalg}
  $\nf{\aset}{\atrs}$ consists of all $a \in \core{\aalg}$ 
  for which 
  every term $t \in \ter{\asig,\setemp}$ with $\interpret{t} = a$ is a normal form with respect to $\atrs$.
\end{lemma}
As mentioned above the interpretations do not need to be monotonic in argument positions
which are normal forms. We formalise this by defining a replacement map 
for the labelling $\lAbeltrs{\aalg}{\atrs}$ of $\atrs$ which does not contain
argument positions that are in normal form.
\begin{definition}\normalfont\label{def:qtransform}
  Let $\atrs$ be a TRS over $\asig$,
  and $\aalg = \pair{\aset}{\sinterpret}$ a partial $\asig$-algebra.
  Let the replacement map $\nfamumap{\atrs}$ be defined for 
  every symbol $f^{\lambda} \in \lAbelsig{\aalg}{\asig}$ with $\lambda = \tuple{a_1,\ldots,a_{\arity{f}}}$ as follows:
  $\funap{\nfamumap{\atrs}}{f^{\lambda}} = \{1,\ldots,\arity{f}\} \setminus \{i \where a_i \in \nf{\aset}{\atrs}\}$.
\end{definition}

As an instance of Theorem~\ref{thm:stepwise}
we obtain a method for stepwise rule removal for local termination that is based
on a combination of monotone partial models 
and extended monotone algebras.

\begin{theorem}\label{thm:quasicomb}
  Let $\atrs$, $\atrs'$ and $\ctrs$ 
  be TRSs over $\asig$, and $\termset \subseteq \ter{\asig,\setemp}$ a set of terms such that
  $\SNrels{\ctrs}{\atrs \cup \atrs'}{\termset}$ holds.
  Furthermore let 
  $\aalg = \triple{\aset}{\sinterpret}{{\qge}}$
  be a monotone partial model for $\atrs \cup \atrs' \cup \ctrs$ with $\termset \subseteq \lang{\aalg}$,
  and $\balg = \quadruple{\bset}{\interpret{\cdot}_\balg}{{\alggt}}{{\algge}}$
  an extended well-founded $\nfamumap{\atrs \cup \atrs'}$-monotone $(\lAbelsig{\aalg}{\asig})$-algebra
  such that:
  \begin{enumerate}[\em(1)]
   \item $\pair{\bset}{{\alggt}}$ is a model for $\lAbeltrs{\aalg}{\atrs'}$,
\item $\pair{\bset}{{\algge}}$ is a model for $\lAbeltrs{\aalg}{\atrs}$, and
\item for all $f \in \asig$, $\vec{a}_1\, a\, \vec{a}_2\in \aset^{\arity{f}}$, $a \qge a' \in \aset$, and $b_1,\ldots,b_{\arity{f}} \in \bset$:
        \[\funap{\interpret{\slAbel{f}{\vec{a}_1\, a\, \vec{a}_2}}_\balg}{b_1,\ldots,b_{\arity{f}}}
          \algge
          \funap{\interpret{\slAbel{f}{\vec{a}_1\, a'\, \vec{a}_2}}_\balg}{b_1,\ldots,b_{\arity{f}}}
          \punc.
        \]
  \end{enumerate}
  Then $\SNrels{(\ctrs \cup \atrs')}{\atrs}{\termset}$ holds.
\end{theorem}
\begin{proof}We construct an
  extended well-founded monotone partial
  $\asig$-algebra $\calg = \quadruple{\cset}{\interpret{\cdot}_\calg}{{\alggt}_\calg}{{\algge}_\calg}$
  fulfilling the requirements of Theorem~\ref{thm:stepwise}.
  Let $\cset = \aset \times \bset$,
  and define 
  $\pair{a_1}{b_1} \alggt_\calg \pair{a_2}{b_2} 
   \Longleftrightarrow a_1 \not\in \nf{\aset}{\atrs \cup \atrs'} \;\&\; a_1 \qge a_2 \;\&\; b_1 \alggt b_2$
  and
  $\pair{a_1}{b_1} \algge_\calg \pair{a_2}{b_2} 
   \Longleftrightarrow a_1 \not\in \nf{\aset}{\atrs \cup \atrs'} \;\&\; a_1 \qge a_2 \;\&\; b_1 \algge b_2$.
  Note that the $\nfamumap{\atrs \cup \atrs'}$-monotonicity is implemented
  by excluding elements $\pair{a_1}{b_1}$ with $a_1 \in \nf{\aset}{\atrs \cup \atrs'}$ from being sources of $\alggt \cup \algge$ steps.
  Then for each $f \in \asig$:
  $\funap{\interpret{f}_\calg}{\pair{a_1}{b_1},\ldots,\pair{a_{\arity{f}}}{b_{\arity{f}}}}
  \!=\!\pair{\funap{\interpret{f}_\aalg}{a_1,\ldots,a_{\arity{f}}}}
         {\funap{\interpret{\slAbel{f}{a_1,\ldots,a_{\arity{f}}}}_\balg}{b_1,\ldots,b_{\arity{f}}}}$
  if $\defd{\funap{\interpret{f}_\aalg}{a_1,\ldots,a_{\arity{f}}}}$, and $\undefd{}$ otherwise.
  Finally, we define the relation $\algnu$ on $\cset$ by 
  $\pair{a_1}{b_1} \algnu \pair{a_2}{b_2} \Longleftrightarrow a_1 \qge a_2$.
  Now it is straightforward to check that all requirements of Theorem~\ref{thm:stepwise}
  are fulfilled, and we conclude $\SNrels{(\ctrs \cup \atrs')}{\atrs}{\termset}$.
\end{proof}

Let us briefly elaborate on the theorem.
As an instance of Theorem~\ref{thm:stepwise}, Theorem~\ref{thm:quasicomb}
is applicable for proving local termination as well as local relative termination.
We start without knowledge $\SNrels{\setemp}{\atrs \cup \btrs}{\termset}$ and stepwise `remove' rules,
more precisely, we move rules from the right side to the left side of the slash `$/$'.
If we reach the goal $\SNrels{\atrs}{\btrs}{\termset}$, 
then the proof has been successful.

The use of partial monotone partial models for $\atrs \cup \atrs' \cup \ctrs$ 
with $\termset \subseteq \lang{\aalg}$
guarantees that the language we consider is closed under rewriting.
The set $\atrs'$ is the set of strictly decreasing rules that we are aiming to remove.
The $\nfamumap{\atrs \cup \atrs'}$-monotone $\lAbelsig{\aalg}{\asig}$-algebra $\balg$
then has the task to make all labelled rules stemming from $\atrs'$ strictly decreasing ($\alggt$),
and from $\atrs$ weakly decreasing ($\algge$).
Then we conclude that $\atrs' \cup \ctrs$ is terminating relative to 
$\atrs$ on $\termset$.

\begin{example}[Klop, see \cite{barendregt:84}, Exercise 7.4.7]
\label{ex:SK}
Example~\ref{ex:S} can be generalised 
to include the combinator $\combK$,
which has the reduction rule $\combK x y \to x$.
The initial language of flat $\combS,\combK$-terms is $\termset = (\combS|\combK)^*$; 
for example $\combS \combS \combK \combS = (((\combS \combS) \combK) \combS)$.
The partial model presented in Example~\ref{ex:Smodel}
can be extended to a monotone partial model for this generalised example 
by fixing $\interpret{\combK} = 0$
and $2 > 0$, $2 > 1$.
Note that this is not a model due to
$\interpreta{\combK x y}{\alpha} = 2 > 0 = \interpreta{x}{\alpha}$ 
for $\alpha = \mylam{z}{0}$.
For the complete proof, employing this model, we refer to: \begin{center}
\uRl{{\tt http://infinity.few.vu.nl/local/}}.
\end{center}
\end{example}



The second example illustrates the stepwise rule removal.
\begin{example}\label{ex:TM}{\newcommand{\free}{1}      \newcommand{\sright}{R}
  \newcommand{\sleft}{L}
  \newcommand{\slleft}{F}
  \newcommand{\wall}{\Box}   \newcommand{\blank}{0}     \newcommand{\start}{S}
  \newcommand{\tm}{\msf{M}}
  \newcommand{\finish}{\mit{finish}}
We use a Turing-machine-like TRS which does the following.
  Starting with its head between two symbols $\free$, the tape
  containing  a finite string of $\free$'s and further blanks ($\blank$),
  it initially puts two boxes $\wall$ left and right of its head
  and afterwards alternately runs left and right between the boxes,
  each time moving them one position further, until 
the blanks are reached:
  \begin{align*}
    &\free\free\wall\free\sright\free\free\free\free\free\wall\free\free
    \mred \free\free\wall\free\free\free\free\free\free\sright\wall\free\free\\
    \red\ &\free\free\wall\free\free\free\free\free\free\sleft\free\wall\free
    \mred \free\free\wall\sleft\free\free\free\free\free\free\free\wall\free\\
    \red\ &\free\wall\free\sright\free\free\free\free\free\free\free\wall\free \mred \ldots
  \end{align*}
This is implemented by the TRS $\atrs$ consisting of the following rules:
\begin{align*}
    \free\start\free &\to \wall\sright\wall &
    \sright\free &\to \free\sright &
    \sright\wall\free &\to \sleft\free\wall &
    \sright\wall\blank &\to \slleft\wall\blank \\
    &&
    \free\sleft &\to \sleft\free &
    \free\wall\sleft &\to \wall\free\sright &
    \blank\wall\sleft &\to \blank\wall\sright \\
    &&
    \free\slleft &\to \slleft\free &
    \free\wall\slleft &\to \wall\free\sright &
    \blank\wall\slleft &\to \finish
\end{align*}
  where all symbols apart form $\finish$ (which is a constant) are unary,
  but have been written without parenthesis for the purpose of compactness.
  Note that the construction of the TRS is similar to
 the standard translation of Turing machines
  to string rewriting systems as given in~\cite{terese:03}.

  While the Turing machine is terminating on every input,
  the TRS $\atrs$ fails to be globally terminating.
  The reason is that $\atrs$ allows for configurations with multiple heads working at the same time on the same tape:
  \begin{gather*}
   \blank\wall\sright\wall\free\slleft\wall\blank 
   \red \blank\wall\sleft\free\wall\slleft\wall\blank 
   \red \blank\wall\sleft\wall\free\sright\wall\blank
   \red^2 \blank\wall\sright\wall\free\slleft\wall\blank \to \ldots
  \end{gather*}
  We will prove that $\atrs$ is locally terminating on all terms
  containing arbitrary occurrences of the symbols $\blank$, $\free$
  and at most one occurrence of $\start$, that is,
  the language given by $\termset = \{\blank,\free\}^* \,\start\, \{\blank,\free\}^* \finish$.
As the first step we remove the rules $\free\start\free \to \wall\sright\wall$ and $\blank\wall\slleft \to \finish$.
  We do this by using a monotone model $\aalg$ consisting of only one element, accepting all terms.
  We combine this model with the $\lAbelsig{\aalg}{\asig}$-algebra $\balg$
  where $\bset = \nat$
  and $\funap{\interpret{\free}_\balg}{x} = \funap{\interpret{\blank}_\balg}{x} = x+1$,
  all other symbols are interpreted as $\mylam{x}{x}$.
  This makes the above two rules decreasing ($\alggt$ is a model for them).



  In the second step, we use a partial model $\aalg = \triple{\aset}{\sinterpret}{{\qge}}$ where $\aset = \{0,1\}$,
  $0 \ge 0$, $1 \ge 1$ (but not $1 \ge 0$),
  $\interpret{\finish} = 0$ and the other interpretations are given in Table~\ref{tab:turing}:

  \begin{table}[h!]
  \begin{center}
  { \renewcommand{\arraystretch}{1.3}
  \begin{tabular}{|@{\hspace{1ex}\extracolsep{1ex}}c@{\hspace{1ex}\extracolsep{.4ex}}|c@{\hspace{.4ex}\extracolsep{.4ex}}|c@{\hspace{.4ex}\extracolsep{.4ex}}|c@{\hspace{.4ex}\extracolsep{.4ex}}|c@{\hspace{.4ex}\extracolsep{.4ex}}|c@{\hspace{.4ex}\extracolsep{.4ex}}|c@{\hspace{.4ex}\extracolsep{.4ex}}|c@{\hspace{.4ex}\extracolsep{.4ex}}|}
    \hline
    $x$ & $\funap{\interpret{\free}}{x}$ & $\funap{\interpret{\wall}}{x}$ & $\funap{\interpret{\sright}}{x}$ & $\funap{\interpret{\sleft}}{x}$ & $\funap{\interpret{\slleft}}{x}$ & $\funap{\interpret{\blank}}{x}$ & $\funap{\interpret{\start}}{x}$ \\
    \hline
    $0$ & $0$ & $1$ & $\undefd{}$ & $\undefd{}$ & $\undefd{}$ & $0$         & $0$          \\
    \hline
    $1$ & $1$ & $0$ & $1$         & $1$         & $1$         & $\undefd{}$ & $\undefd{}$ \\
    \hline
  \end{tabular}}
  \caption{Symbol interpretations.}
  \label{tab:turing}
  \end{center}
  \end{table}

  \noindent As required by the theorem $\aalg$ is a monotone partial model for $\atrs$ including the two removed rules
  $\ctrs = \{\free\start\free \to \wall\sright\wall,\; \blank\wall\slleft \to \finish\}$
  (without them $\termset$ would consist of normal forms).
  We use this partial model together with the extended well-founded monotone $\lAbelsig{\aalg}{\asig}$-algebra 
  $\balg= \quadruple{\nat}{\interpret{\cdot}_\balg}{{\alggt}}{{\algge}}$ 
  where $\alggt$ and $\algge$ are the usual orders $>$ and $\ge$ on $\nat$, respectively.
  The interpretation $\interpret{\cdot}_\balg$ is
  $\interpret{\finish}_\balg = 7$,
  $\funap{\interpret{\slAbel{\free}{0}}_\balg}{x} = 2\cdot x + 1$,
  $\funap{\interpret{\slAbel{\free}{1}}_\balg}{x} = 2\cdot x$,
  $\funap{\interpret{\slAbel{\wall}{0}}_\balg}{x} = \funap{\interpret{\slAbel{\wall}{1}}_\balg}{x} = x$,
  $\funap{\interpret{\slAbel{\sright}{1}}_\balg}{x} = 2\cdot x$,
  $\funap{\interpret{\slAbel{\sleft}{1}}_\balg}{x} = 2\cdot x + 1$,
  $\funap{\interpret{\slAbel{\slleft}{1}}_\balg}{x} = 2\cdot x$,
  $\funap{\interpret{\slAbel{\blank}{0}}_\balg}{x} = 2\cdot x$, and
  $\funap{\interpret{\slAbel{\start}{0}}_\balg}{x} = 5\cdot x + 6$.
  Then $\atrs'$ consists of the following rules:
  $\sright\wall\free \to \sleft\free\wall$, 
  $\free\sleft \to \sleft\free$, 
  $\free\wall\sleft \to \wall\free\sright$, 
  $\blank\wall\sleft \to \blank\wall\sright$, and
  $\free\wall\slleft \to \wall\free\sright$.
  Then $\pair{\balg}{{\alggt}}$ is a model for $\lAbeltrs{\aalg}{\atrs'}$.
  For instance consider the rule $\sright\wall\free \to \sleft\free\wall$.
  The labelling
  $\slAbel{\sright}{1}\slAbel{\wall}{0}\slAbel{\free}{0} \to \slAbel{\sleft}{1}\slAbel{\free}{1}\slAbel{\wall}{0}$
  is in $\lAbeltrs{\aalg}{\atrs'}$ 
  and its interpretation in $\balg$ is:
  $\funap{\slAbel{\sright}{1}\slAbel{\wall}{0}\slAbel{\free}{0}}{x} 
   = 4\cdot x + 2 
   > 4\cdot x + 1
   = \funap{\slAbel{\sleft}{1}\slAbel{\free}{1}\slAbel{\wall}{0}}{x}$.
  The labelling
  $\slAbel{\sright}{0}\slAbel{\wall}{1}\slAbel{\free}{1} \to \slAbel{\sleft}{0}\slAbel{\free}{0}\slAbel{\wall}{1}$
  is not in $\lAbeltrs{\aalg}{\atrs'}$ since its left-hand side is undefined with respect to $\aalg$,
  thus we can ignore this rule.
  Analogously it can be verified 
  $\pair{\balg}{{\algge}}$ is a model for $\lAbeltrs{\aalg}{\atrs \setminus \atrs'}$.
Since $>$ is the empty relation on $\aset$
  the third condition of Theorem~\ref{thm:quasicomb} holds trivially.

  The three remaining rules
  $\sright\free \to \free\sright$, $\free\slleft \to \slleft\free$, and $\sright\wall\blank \to \slleft\wall\blank$
  are even globally terminating.
  This corresponds to taking a model which has only one state and accepts all terms
  together with the corresponding termination order which proves global termination.
Hence we have proven $\SNrs{\atrs}{\termset}$ by three consecutive applications of Theorem~\ref{thm:quasicomb}.
}\end{example}

Finally, we give a theorem that allows us to remove rules and forget about them.
We need to be sure that these rules do not influence the family, that is, the set of reachable terms.
This is guaranteed if all terms in the family are normal forms with 
respect to these rules.

\begin{theorem}\label{thm:qremove}
  Let $\atrs$, $\atrs'$ and $\btrs$
  be TRSs over $\asig$, and $\termset \subseteq \ter{\asig,\setemp}$.
  Let 
  $\aalg = \triple{\aset}{\sinterpret}{{\qge}}$
  be a monotone partial model for $\atrs \cup \atrs' \cup \btrs$ with $\termset \subseteq \lang{\aalg}$
  such that for all rules $\ell \to r \in \atrs'$ and $\alpha \funin \vars{\ell} \to \aset$
  we have $\undefd{\interpreta{\ell}{\alpha}}$ (the left-hand side is undefined).
  Then $\SNrels{\atrs}{\btrs}{\termset}$ implies $\SNrels{\atrs \cup \atrs'}{\btrs}{\termset}$.
\end{theorem}
\begin{proof}
  From $\famil{\atrs \cup \atrs' \cup \btrs}{\termset} \subseteq \lang{\aalg}$
  together with $\undefd{\interpreta{\ell}{\alpha}}$ for all $\ell \to r \in \atrs'$
  and $\alpha$ it follows that the rules in $\atrs'$ are not reachable. All terms in $\family{\termset}$ are normal
  forms with respect to $\atrs'$. Hence we can ignore these rules.
\end{proof}

\begin{example}{
  \newcommand{\ssuc}{\msf{s}}
  \newcommand{\suc}{\funap{\msf{s}}}
  \newcommand{\zer}{0}
  Consider the TRS $\atrs$ consisting of the following four rules:
\begin{align*}
    \funap{f}{\suc{\suc{x}}} &\to \funap{f}{\funap{o}{x}} &
    \funap{o}{\suc{\suc{x}}} &\to \suc{\suc{\funap{o}{x}}} &
    \funap{o}{\zer} &\to \zer &
    \funap{o}{\suc{\zer}} &\to \suc{\suc{\suc{\zer}}}
  \end{align*}
The TRS is not terminating: $\funap{f}{\suc{\suc{\suc{\zer}}}} \red \funap{f}{\funap{o}{\suc{\zer}}} \red \funap{f}{\suc{\suc{\suc{\zer}}}} \to \ldots$. However, the function $f$ is terminating when applied to an even number,
  that is, the language $\termset = \{\funap{f}{\funap{\ssuc^{2\cdot n}}{\zer}} \where n \in \nat\}$.
  We choose $\aalg = \triple{\{0,1\}}{\sinterpret}{{\qge}}$
  where $\interpret{\zer} = 0$, $\funap{\interpret{\ssuc}}{0} = 1$, $\funap{\interpret{\ssuc}}{1} = 0$,
  $\funap{\interpret{o}}{0} = 0$, $\undefd{\funap{\interpret{o}}{1}}$, $\funap{\interpret{f}}{0} = 0$ and $\undefd{\funap{\interpret{f}}{1}}$.
  Then $\aalg$ is a monotone partial model with $\termset \subseteq \lang{\aalg}$.
  We have $\undefd{\interpreta{\funap{o}{\suc{\zer}}}{\alpha}}$ (for all $\alpha$), thus the rule
  $\funap{o}{\suc{\zer}} \to \suc{\suc{\suc{\zer}}}$ is never applicable and can be removed.
}\end{example}
