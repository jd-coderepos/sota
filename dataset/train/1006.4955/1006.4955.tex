\noindent In Sections~\ref{sec:sn}--\ref{sec:stepwise} we have devised
a characterisation of local termination in terms of monotone partial
algebras.  While this gives the general method, for the purpose of
obtaining automatable methods we strive for fruitful classes of these
algebras.  For global termination, instances of monotone algebras are
well-known.  This raises the natural question whether we can transform
a given monotone algebra for global termination in such a way that we
obtain a partial monotone algebra for local termination.

In this section we present one such approach.
We combine monotone partial models with (ordinary) monotone algebras.
The monotone partial models are roughly deterministic tree automata
that are closed under rewriting;
they describe the language of term on which we proof termination.
We search for such an automaton that accepts the starting language 
together with a monotone algebra such that the rewrite rules decrease
on the language of the automaton.
In this way monotone algebras for global termination carry over to local termination,
and we obtain an automatable method that 
is applicable for proofs of local termination.

First we give the definition of extended -monotone algebras as known from 
global termination of context-sensitive TRSs, see~\cite{luca:98,EWZ06}.
A mapping 
is called a \emph{replacement map (for )} if for all  we have
.
Let  be a -algebra and  a replacement map.
For symbols  we say that the interpretation 
is \emph{-monotone} with respect to 
if for every  and  with  we have:


\begin{definition}\normalfont\label{def:globalalg}
  Let  be a replacement map for .


  \noindent An \emph{extended well-founded -monotone -algebra 
  }
  is a -algebra 
  with two binary relations ,  on  for which
  the following conditions hold:
\begin{enumerate}[(i)]
\item , and
\item for every  the function  is -monotone with respect to  and .
  \end{enumerate}
\end{definition}


A partial model 
may contain elements 
for which  implies that  is a normal form.
For a given partial model the set of these, which we denote by , can be computed (see below Definition~\ref{def:nfalg}).
We can exploit this knowledge as follows:
if a certain argument of a symbol 
is always a normal form, then its interpretation 
does not need to be monotonic for this argument position.
The following definition gives an algorithm for computing the set .
Elements that are interpretations  of left-hand sides in 
cannot belong to this set. Moreover if 
and  then we conclude .
This is formalised as follows:
\begin{definition}\normalfont\label{def:nfalg}
  Let  be a TRS over the signature , and  a partial -algebra.
  The \emph{normal forms  of }
  are the largest set  such that
  
  for every  and every ,
  and
   
  for every ,  and .
\end{definition}
Then by construction we obtain the following lemma:
\begin{lemma}\label{lem:nfalg}
   consists of all  
  for which 
  every term  with  is a normal form with respect to .
\end{lemma}
As mentioned above the interpretations do not need to be monotonic in argument positions
which are normal forms. We formalise this by defining a replacement map 
for the labelling  of  which does not contain
argument positions that are in normal form.
\begin{definition}\normalfont\label{def:qtransform}
  Let  be a TRS over ,
  and  a partial -algebra.
  Let the replacement map  be defined for 
  every symbol  with  as follows:
  .
\end{definition}

As an instance of Theorem~\ref{thm:stepwise}
we obtain a method for stepwise rule removal for local termination that is based
on a combination of monotone partial models 
and extended monotone algebras.

\begin{theorem}\label{thm:quasicomb}
  Let ,  and  
  be TRSs over , and  a set of terms such that
   holds.
  Furthermore let 
  
  be a monotone partial model for  with ,
  and 
  an extended well-founded -monotone -algebra
  such that:
  \begin{enumerate}[\em(1)]
   \item  is a model for ,
\item  is a model for , and
\item for all , , , and :
        
  \end{enumerate}
  Then  holds.
\end{theorem}
\begin{proof}We construct an
  extended well-founded monotone partial
  -algebra 
  fulfilling the requirements of Theorem~\ref{thm:stepwise}.
  Let ,
  and define 
  
  and
  .
  Note that the -monotonicity is implemented
  by excluding elements  with  from being sources of  steps.
  Then for each :
  
  if , and  otherwise.
  Finally, we define the relation  on  by 
  .
  Now it is straightforward to check that all requirements of Theorem~\ref{thm:stepwise}
  are fulfilled, and we conclude .
\end{proof}

Let us briefly elaborate on the theorem.
As an instance of Theorem~\ref{thm:stepwise}, Theorem~\ref{thm:quasicomb}
is applicable for proving local termination as well as local relative termination.
We start without knowledge  and stepwise `remove' rules,
more precisely, we move rules from the right side to the left side of the slash `'.
If we reach the goal , 
then the proof has been successful.

The use of partial monotone partial models for  
with 
guarantees that the language we consider is closed under rewriting.
The set  is the set of strictly decreasing rules that we are aiming to remove.
The -monotone -algebra 
then has the task to make all labelled rules stemming from  strictly decreasing (),
and from  weakly decreasing ().
Then we conclude that  is terminating relative to 
 on .

\begin{example}[Klop, see \cite{barendregt:84}, Exercise 7.4.7]
\label{ex:SK}
Example~\ref{ex:S} can be generalised 
to include the combinator ,
which has the reduction rule .
The initial language of flat -terms is ; 
for example .
The partial model presented in Example~\ref{ex:Smodel}
can be extended to a monotone partial model for this generalised example 
by fixing 
and , .
Note that this is not a model due to
 
for .
For the complete proof, employing this model, we refer to: \begin{center}
\uRl{{\tt http://infinity.few.vu.nl/local/}}.
\end{center}
\end{example}



The second example illustrates the stepwise rule removal.
\begin{example}\label{ex:TM}{\newcommand{\free}{1}      \newcommand{\sright}{R}
  \newcommand{\sleft}{L}
  \newcommand{\slleft}{F}
  \newcommand{\wall}{\Box}   \newcommand{\blank}{0}     \newcommand{\start}{S}
  \newcommand{\tm}{\msf{M}}
  \newcommand{\finish}{\mit{finish}}
We use a Turing-machine-like TRS which does the following.
  Starting with its head between two symbols , the tape
  containing  a finite string of 's and further blanks (),
  it initially puts two boxes  left and right of its head
  and afterwards alternately runs left and right between the boxes,
  each time moving them one position further, until 
the blanks are reached:
  
This is implemented by the TRS  consisting of the following rules:

  where all symbols apart form  (which is a constant) are unary,
  but have been written without parenthesis for the purpose of compactness.
  Note that the construction of the TRS is similar to
 the standard translation of Turing machines
  to string rewriting systems as given in~\cite{terese:03}.

  While the Turing machine is terminating on every input,
  the TRS  fails to be globally terminating.
  The reason is that  allows for configurations with multiple heads working at the same time on the same tape:
  
  We will prove that  is locally terminating on all terms
  containing arbitrary occurrences of the symbols , 
  and at most one occurrence of , that is,
  the language given by .
As the first step we remove the rules  and .
  We do this by using a monotone model  consisting of only one element, accepting all terms.
  We combine this model with the -algebra 
  where 
  and ,
  all other symbols are interpreted as .
  This makes the above two rules decreasing ( is a model for them).



  In the second step, we use a partial model  where ,
  ,  (but not ),
   and the other interpretations are given in Table~\ref{tab:turing}:

  \begin{table}[h!]
  \begin{center}
  { \renewcommand{\arraystretch}{1.3}
  \begin{tabular}{|@{\hspace{1ex}\extracolsep{1ex}}c@{\hspace{1ex}\extracolsep{.4ex}}|c@{\hspace{.4ex}\extracolsep{.4ex}}|c@{\hspace{.4ex}\extracolsep{.4ex}}|c@{\hspace{.4ex}\extracolsep{.4ex}}|c@{\hspace{.4ex}\extracolsep{.4ex}}|c@{\hspace{.4ex}\extracolsep{.4ex}}|c@{\hspace{.4ex}\extracolsep{.4ex}}|c@{\hspace{.4ex}\extracolsep{.4ex}}|}
    \hline
     &  &  &  &  &  &  &  \\
    \hline
     &  &  &  &  &  &          &           \\
    \hline
     &  &  &          &          &          &  &  \\
    \hline
  \end{tabular}}
  \caption{Symbol interpretations.}
  \label{tab:turing}
  \end{center}
  \end{table}

  \noindent As required by the theorem  is a monotone partial model for  including the two removed rules
  
  (without them  would consist of normal forms).
  We use this partial model together with the extended well-founded monotone -algebra 
   
  where  and  are the usual orders  and  on , respectively.
  The interpretation  is
  ,
  ,
  ,
  ,
  ,
  ,
  ,
  , and
  .
  Then  consists of the following rules:
  , 
  , 
  , 
  , and
  .
  Then  is a model for .
  For instance consider the rule .
  The labelling
  
  is in  
  and its interpretation in  is:
  .
  The labelling
  
  is not in  since its left-hand side is undefined with respect to ,
  thus we can ignore this rule.
  Analogously it can be verified 
   is a model for .
Since  is the empty relation on 
  the third condition of Theorem~\ref{thm:quasicomb} holds trivially.

  The three remaining rules
  , , and 
  are even globally terminating.
  This corresponds to taking a model which has only one state and accepts all terms
  together with the corresponding termination order which proves global termination.
Hence we have proven  by three consecutive applications of Theorem~\ref{thm:quasicomb}.
}\end{example}

Finally, we give a theorem that allows us to remove rules and forget about them.
We need to be sure that these rules do not influence the family, that is, the set of reachable terms.
This is guaranteed if all terms in the family are normal forms with 
respect to these rules.

\begin{theorem}\label{thm:qremove}
  Let ,  and 
  be TRSs over , and .
  Let 
  
  be a monotone partial model for  with 
  such that for all rules  and 
  we have  (the left-hand side is undefined).
  Then  implies .
\end{theorem}
\begin{proof}
  From 
  together with  for all 
  and  it follows that the rules in  are not reachable. All terms in  are normal
  forms with respect to . Hence we can ignore these rules.
\end{proof}

\begin{example}{
  \newcommand{\ssuc}{\msf{s}}
  \newcommand{\suc}{\funap{\msf{s}}}
  \newcommand{\zer}{0}
  Consider the TRS  consisting of the following four rules:

The TRS is not terminating: . However, the function  is terminating when applied to an even number,
  that is, the language .
  We choose 
  where , , ,
  , ,  and .
  Then  is a monotone partial model with .
  We have  (for all ), thus the rule
   is never applicable and can be removed.
}\end{example}
