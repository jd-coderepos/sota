\documentclass[11pt]{article}

\usepackage{amssymb}
\usepackage[cmex10]{amsmath}
\usepackage{amsthm}
\usepackage{units, xcolor}
\usepackage{fullpage}
\usepackage{times}
\interdisplaylinepenalty=2500
\newif\ifconf
\conffalse

\def\compactify{\itemsep=0pt \topsep=0pt \partopsep=0pt \parsep=0pt}

\newcommand {\compactMath}[1]{}

\DeclareMathOperator*{\argmax}{arg\,max}
\DeclareMathOperator*{\argmin}{arg\,min}
\DeclareMathOperator*{\esssup}{ess\,sup}
\DeclareMathOperator {\diam}  {diam}
\DeclareMathOperator {\rank}  {rank}
\DeclareMathOperator {\val}  {val}
\DeclareMathOperator {\vol}  {vol}
\DeclareMathOperator {\dem}  {dem}
\DeclareMathOperator {\spn}  {span}
\DeclareMathOperator {\sgn}  {sgn}
\DeclareMathOperator {\conv} {conv}
\DeclareMathOperator {\Var}  {Var}
\DeclareMathOperator {\cov}  {cov}
\DeclareMathOperator {\supp} {supp}


\newcommand {\h}  {h}

\newcommand {\nc}  {\frac{1}{\sqrt{2\pi}}}
\newcommand {\ncc} [2]  {\frac{#1}{\sqrt{2\pi} #2}}
\newcommand {\roundup}   [1] {{\lceil {#1} \rceil}}
\newcommand {\rounddown} [1] {{\lfloor {#1} \rfloor}}

\newcommand {\set}   [1] {\left\{ #1 \right\}}
\newcommand {\brc}   [1] {\left(#1\right)}

\newcommand {\Exp}       {\mathbb{E}}
\newcommand {\Prob}  [1] {\Pr \brc{#1 }}
\newcommand {\Probb} [2] {\Pr_{#1} \brc{#2 }}
\newcommand {\E}     [1] {\Exp\left[#1\right]}
\newcommand {\EE}    [2] {\Exp_{#1}\left[#2\right]}
\newcommand {\Varr}  [1] {\Var \left[#1\right]}
\newcommand {\iprod} [2] {\langle #1, #2 \rangle}
\newcommand {\Iprod} [2] {\left\langle #1, #2 \right\rangle}
\newcommand {\tprod} [2] {{#1 \otimes #2}}
\newcommand{\given}{\mid}

\newcommand {\bbN}    {\mathbb{N}}
\newcommand {\bbZ}    {\mathbb{Z}}
\newcommand {\bbB}    {\mathbb{B}}
\newcommand {\bbR}    {\mathbb{R}}
\newcommand {\bbC}    {\mathbb{C}}

\newcommand {\calA}   {{\cal{A}}}
\newcommand {\calB}   {{\cal{B}}}
\newcommand {\calC}   {{\cal C}}
\newcommand {\calD}  {{\cal{D}}}
\newcommand {\calE}   {{\cal{E}}}
\newcommand {\calS}   {{\cal{S}}}
\newcommand {\calH}  {{\cal{H}}}
\newcommand {\calN}  {{\cal{N}}}
\newcommand {\calP}   {{\cal{P}}}
\newcommand {\calQ}  {{\cal{Q}}}
\newcommand {\calF}   {{\cal{F}}}
\newcommand {\calZ}   {{\cal{Z}}}
\newcommand {\calU}  {{\cal{U}}}
\newcommand {\calI}    {{\cal I}}
\newcommand {\calL}   {{\cal{L}}}

\newcommand {\Tau}    {{\cal{T}}}

\newcommand {\ord}    {{\cal{O}}} 
\newcommand {\NP}   {{\cal{NP}}}


\newcommand {\TODO}[1] {\footnote{TODO: #1}}
\newcommand{\ynote}[1]{\textcolor{purple}{}\marginpar[\hspace*{-0.8in}{\parbox{0.75in}{\raggedright\tiny \textcolor{red}{Yury: #1}}}]{\parbox{0.75in}{\raggedright\tiny \textcolor{red}{TODO: #1}}}}

\newcommand {\normu}   {\tilde{u}}
\newcommand {\normv}   {\tilde{v}}
\newcommand {\Ui}      {{u_i}}

\newcommand {\ONE}      {\text{\textbf{1}}}
\newcommand {\VOne}     {v_\text{\textbf{1}}}
\newcommand {\VPerp}    {v_\text{\textbf{}}}
\newcommand {\tensor}   {\otimes}
\newcommand {\ptensor}  {\mathbin{\hat\otimes}}

\DeclareMathOperator {\Ball}{Ball}
\DeclareMathOperator {\srcost}{sr-cost}
\DeclareMathOperator {\sdpcost}{sdp-cost}
\DeclareMathOperator {\cost}{cost}
\DeclareMathOperator {\cut}{cut}


\newtheorem{theorem}{Theorem}
\newtheorem{lemma}[theorem]{Lemma}
\newtheorem{proposition}[theorem]{Proposition}
\newtheorem{claim}[theorem]{Claim}
\newtheorem{corollary}[theorem]{Corollary}
\newtheorem{definition}[theorem]{Definition}
\newtheorem{observation}[theorem]{Observation}
\newtheorem{fact}[theorem]{Fact}
\newtheorem{property}{Property}
\newtheorem{remark}[theorem]{Remark}
\newtheorem{notation}{Notation}
\newtheorem{example}{Example}
\newtheorem{algorithm}{Algorithm}
\newtheorem{conjecture}{Conjecture}
\newtheorem{question}[conjecture]{Question}

\title{Satisfiability of Ordering CSPs Above Average\\ Is Fixed-Parameter Tractable}


\author{Konstantin Makarychev \\Microsoft Research \and Yury Makarychev
\\TTIC \and Yuan Zhou \\ MIT}
\date{}
\begin{document}
\maketitle

\begin{abstract}
We study the satisfiability of ordering constraint satisfaction problems (CSPs) above average. We prove the conjecture of Gutin, van Iersel, Mnich, and Yeo that the satisfiability above average of ordering CSPs of arity  is fixed-parameter tractable for every . Previously, this was only known for  and . We also generalize this result to more general classes of CSPs, including CSPs with predicates defined by linear inequalities.

To obtain our results, we prove a new Bonami-type inequality for the Efron---Stein decomposition. The inequality applies to functions defined on arbitrary product
probability  spaces. In contrast to other variants of the Bonami Inequality, it does not depend on the mass of the smallest atom in the probability space. We believe that this inequality is of independent interest.
\end{abstract}

\section{Introduction}

In this paper, we study the satisfiability of ordering constraint satisfaction problems (CSPs) above the average value. An ordering -CSP
is defined by a set of variables  and a set of constraints . Each constraint 
is a disjunction of clauses of the form  for some distinct variables 
from a -element subset . A linear ordering  of variables  satisfies a constraint  if one of the clauses in the disjunction agrees with the linear ordering . The goal is to find an assignment  that maximizes the number of satisfied constraints.

A classical example of an ordering CSP is the Maximum Acyclic Subgraph problem,  in which constraints are of the form ``'' (the problem has arity 2). Another well-known example is the Betweenness problem, in which constraints are of the form `` or '' (the problem has arity 3). Both problems are -hard and cannot be solved exactly in polynomial--time unless ~\cite{Karp72, Opatrny79} (see also \cite{Seymour,CS,M-betw}).

There is a trivial approximation algorithm for ordering CSPs as well as other constraint satisfaction problems: output a random linear ordering of variables  (chosen uniformly among all  linear orderings). Say, if each constraint is just a clause on  variables, this algorithm satisfies each clause with probability  and thus satisfies a  fraction of all constraints in expectation.
In 1997, H\aa{}stad~\cite{Hastad} showed that for some regular (i.e., non-ordering) constraint satisfaction problems the best approximation algorithm is the \textit{random assignment} algorithm. His work raised
the following question: for which CSPs are there non-trivial approximation algorithms and
for which CSPs is the best approximation algorithm the random assignment?
This question has been extensively studied in the literature. Today, there are
many known classes of constraint satisfaction problems that do not admit non-trivial approximations
assuming the Unique Games or  conjectures (see e.g \cite{Hastad, AM, GR, Chan}).
There are also many constraint satisfaction problems for which we know non-trivial approximation algorithms.
Surprisingly, the situation is very different for ordering CSPs:
Guruswami, H\aa{}stad,
Manokaran, Raghavendra, and Charikar~\cite{GHMRC} showed that
\textit{all} ordering -CSPs do not admit non-trivial approximation assuming the Unique Games Conjecture.

A similar question has been actively studied from the fixed-parameter tractability
perspective\footnote{We refer the reader to an excellent survey of results in this area by Gutin and Yeo~\cite{GY}.}~\cite{AGKSY, CFGJRTY, CGJRS,GKSY,KW,MR,RS}: Given an instance of a CSP, can we decide whether
 for a fixed parameter ? Here,  is the value of the optimal solution for the instance,
and  is the expected value on a  random assignment.
In 2011, Alon, Gutin, Kim, Szeider, and Yeo~\cite{AGKSY} gave the affirmative answer to this question for all (regular, non-ordering) -CSPs
with a constant size alphabet. In~\cite{GKSY, GIMY, GKMY}, Gutin et al. extended this result to
2-arity and 3-arity ordering CSPs, and conjectured~\cite{GIMY} that the satisfiabilty above average is fixed-parameter tractable for ordering CSPs of any arity .
Below we state the problem formally.

\begin{definition}[Satisfiability of Ordering CSP Above Average]\label{def:above-avg}
Consider an instance  of arity  and a parameter . Let  be the number of the constraints satisfied by the optimal solution, and
 be the number of constraints satisfied in expectation by a random solution. We need to decide whether .
\end{definition}
\begin{definition}
A problem with a parameter  is fixed-parameter tractable	if there exists an
algorithm for the problem with running time , where  is an arbitrary function of ,  is a fixed polynomial (independent of ),
and  is the size of the input.
\end{definition}

In this paper, we prove the conjecture of Gutin et al.~\cite{GIMY} and show that the satisfiability above average of any ordering CSP of any arity  is fixed-parameter tractable.
\begin{theorem}\label{thm:main-intro}
There exists a deterministic algorithm that given an instance  of an ordering -CSP on  variables and a parameter , decides whether
 in time , where  is a function of ,  is a polynomial of  ( and  depend on ).
If , then the algorithm also outputs an assignment satisfying at least  constraints.
\end{theorem}
Furthermore, we prove that the problem has a \emph{kernel} of size .


\medskip

\noindent \textbf{Techniques.} Let us examine approaches used previously for ordering CSPs.
The algorithms of Gutin et al.~\cite{GKSY, GIMY, GKMY} work by applying a carefully chosen
set of reduction rules to ordering CSPs of arity 2 and 3. These rules heavily depend on the structure of
 and  CSPs. Unfortunately, the structure of ordering
CSPs of higher arities is substantially more complex. Here is a quote from~\cite{GIMY}: \emph{``it appears technically
very difficult to extend results obtained for arities  and  to }.''
In this paper, we do not use such reductions.

The papers~\cite{CMM, GZ, Mak} use an alternative approach to get an advantage over the random assignment
for special families of ordering CSPs. They first reduce the ordering -CSP to a regular -CSP with a constant size alphabet, and
then work with the obtained regular -CSP. However, this reduction, generally, does not preserve the value of the CSP. So if for the original
ordering CSP instance  we have , then for the new instance  we may
have  (we note that ).
In this paper, we do not use this reduction either.

Instead, we treat all ordering CSPs as CSPs with the continuous domain: Our goal is to arrange
all variables on the interval  so as maximize the number of satisfied constraints.
The arrangement of variables uniquely  determines their order. Moreover, if we
independently assign random values from  to variables , then the induced ordering on 's will be
uniformly distributed among all  possible orderings.
Thus, our reduction preserves the values of  and . However, we can no longer
apply Fourier analytic tools used previously in~\cite{AGKSY, GZ, Mak}. We cannot use the
(standard) Fourier analysis on , since we have no control over the Fourier coefficients
of the functions we  need to analyze. Instead, we work with the Efron---Stein decomposition~\cite{ES}
(see Sections~\ref{sec:proof-overview} and \ref{sec:EfronStein}). We show that all terms in
the Efron---Stein decomposition have a special form. We use this fact to prove that an
ordering -CSP that depends on many variables must have a large variance.
Specifically, we show that
if a -CSP instance depends on  variables, then the standard deviation of its value from the mean
(on a random assignment) is greater than  (for some  and ). As is, this claim does not imply that
 since for some assignments the value may be substantially less than .
To finish the proof of the main result, we prove a new hypercontractive inequality, which is
an analog of the Bonami Lemma~\cite{Bonami}. This inequality is one of the main
technical contributions of our paper.

\begin{theorem}[Bonami Lemma for Efron---Stein Decomposition]\label{thm:bonami-for-ES}
Consider . Let  be the Efron---Stein decomposition of . Denote the degree of the decomposition by .
Assume that for every set ,

Then

\end{theorem}
We note that hypercontractive inequalities have been extensively studied under various settings
(see e.g., \cite{Talagrand94, DS96, Wolf07, MOS}). However, all of them depend on
the \textit{mass of the smallest atom in the probability space}. In our case, the smallest atom is polynomially small in ,
which is why we cannot apply known hypercontractive inequalities.
This is also the reason why we need an extra condition~(\ref{eq:Bonami-Fourth-Moment}) on the function .
Condition (\ref{eq:Bonami-Fourth-Moment}) is a ``local'' condition in
the sense that all expectations in~(\ref{eq:Bonami-Fourth-Moment}) are over at most  variables for every set .
Consequently, as we will see below, it is very easy to verify that it holds in many cases
(in contrast to (\ref{eq:Bonami-Bound}), which is very difficult to verify directly).
Note also that  condition (\ref{eq:Bonami-Fourth-Moment}) is necessary --- if it is not satisfied, then the ratio  can
be arbitrarily large even for .

\medskip

\noindent\textbf{Extensions.}
Once we assume that the domain of every variable is the interval , we might be tempted
to write more complex constraints than before such as
\emph{``the average of ,  and  is at most
''}, or \emph{`` lies to the left of the midpoint between  and ''},
or \emph{`` is closer to  than to ''}.
 Each of these constraints can be written as a system of linear inequalities or a disjunction of clauses,
 each of which is given by a system of linear inequalities. For instance,
  \emph{`` lies to the left of the midpoint between  and ''}
  can be written as  .
In Section~\ref{sec:PPP}, we extend our results to CSPs in which every constraint is a disjunction of clauses, each of which is
a ``small'' linear program (LP).
Namely, each constraint should have arity at most , only variables that a constraint depends on should appear in the LPs that define it,
and all LP coefficients must be integers in the range  (for a fixed ).
We call this new class of CSPs -LP CSPs.

\begin{definition}\label{def:LP-CSP}
A -LP CSP is defined by a set of variables  taking values in the interval 
and a set of constraints . Each constraint  is a disjunction of clauses of the form
, where  is a matrix with integer coefficients in the range ;
 is a vector with integer coefficients in the range ; the indices of  non-zero columns of the matrix
 lie in the set  of size  (the set  is the same for all clauses in ).
The goal is to assign distinct real values to variables  so as to maximize the number of satisfied constraints.
\end{definition}

In fact, we extend our results to a much more general class of valued CSPs -- all CSPs whose
predicates lie in a lattice of functions with some natural properties
(see Sections~\ref{sec:general-framework} and~\ref{sec:PPP}  for details);
but we believe that the subclass of -LP CSPs is the most natural example of CSPs in the class.
Observe that every ordering -CSP is a -LP CSP since we can
write each clause  as the system of linear equations
 for . Similarly, every -CSP on
a finite domain  is equivalent to a -LP CSP. The reduction works as follows:
We break the interval  into  equal subintervals 
and map every value  to the -th interval. Then, we replace every
condition  with the equation 
which can be written as  and .


\medskip

\noindent \textbf{Overview.} In the next section we give an informal overview of the proof.
We formally define the problem and describe the Efron---Stein decomposition in Section~\ref{sec:prelim}.
Then, in Section~\ref{sec:ES-ordering-CSP}, we prove several claims about the Efron---Stein decomposition
of ordering CSPs. We derive the main results (Theorem~\ref{thm:main-intro} and Theorem~\ref{thm:main}) in
Section~\ref{sec:proof-main-thm}. Finally, we prove the Bonami Lemma for the Efron---Stein decomposition
in Section~\ref{sec:Bonami}. We generalize our results to all CSPs with a lattice structure in Section~\ref{sec:general-framework}
and show that -LP CSPs (as well as more general ``piecewise polynomial'' CSPs)
have a lattice structure in Section~\ref{sec:PPP}.


\section{Proof Overview}\label{sec:proof-overview}
Our high-level approach is similar to that developed by Alon et al.~\cite{AGKSY} and Gutin et al.~\cite{GKSY, GKMY, GIMY}.
As in~\cite{GKSY, GKMY, GIMY}, we design an algorithm that given an instance  of an ordering CSP does the following:
\begin{itemize}
\item It either finds a kernel (another instance of the ordering CSP)  on  variables such that  and .
Then we can decide whether  by trying out all possible permutations of variables that  depends on in time .
\item Or it certifies that .
\end{itemize}
To this end, we show that either  depends on at most  variables or the variance of  is  (where  is chosen uniformly at random). In the former case, the restriction of  to the variables it depends on is the desired kernel of size . In the latter case, .
Though our approach resembles that of~\cite{GKSY, GKMY, GIMY} at the high level, we employ very different techniques to prove our results.

We extensively use Fourier analysis and, specifically, the Efron---Stein decomposition. Fourier analysis is a very powerful tool, which works especially well with product spaces. The space of feasible solutions of an ordering CSP is not, however, a product space --- it is a discrete space
that consists of  linear orderings of variables . To overcome this problem, we define ``continuous solutions'' for an ordering
CSP (see Section~\ref{sec:csp-over-product-space}). A solution  is an assignment of real values in  to variables ; that is, it is a point in the product space . Each continuous solution defines a combinatorial solution  in a natural way:  orders variables  according to the values assigned to them (e.g., if we assign values ,  and  to ,  and  then  according to ). Thus we get an optimization problem over the product space .
Denote by  its objective function.
We consider the Efron---Stein decomposition of :  (see Section~\ref{sec:EfronStein}). Here, informally,  is the part of  that depends on variables  with . All functions  are uncorrelated:  for . We show that each  is either identically equal to  or has variance greater than some positive number, which depends only on  (see Section~\ref{sec:ES-ordering-CSP}, Lemma~\ref{lem:beta}).
We now consider two cases.

I. In the first case, there are at most   terms   not equal to . Using that  depends only on variables  with  and that there are at most  sets  such that , we get that  depends on at most  variables and we are done.

II. In the second case, there are at least   terms   not equal to . Since the variance of each term  is  and all terms  are uncorrelated, the variance of  is at least  (see Theorem~\ref{thm:variance}). Therefore,  deviates from  by at least . We then show that  satisfies the conditions of Theorem~\ref{thm:bonami-for-ES} (see Lemma~\ref{lem:bound-on-C}) and the degree of the decomposition is at most .
Thus, the ratio  is bounded by . This implies by a theorem of Alon, Gutin, Kim, Szeider, and Yeo~\cite{AGKSY} (see Theorem~\ref{thm:AGKSY} in Section~\ref{sec:proof-main-thm})
that . Hence, .

 



\section{Preliminaries}\label{sec:prelim}
\subsection{Ordering CSP}\label{sec:prelim-order-csp}
Consider a set of variables .
An ordering constraint  on
a subset of variables 
is a set of linear orderings of
.
A linear ordering  of   satisfies a constraint  on  if the restriction of  to
 is in .
We say that  depends on variables .
\begin{definition}
An instance  of an ordering constraint satisfaction problem consists of a set of variables  and a set of constraints ;
each constraint  depends on some subset of variables.
A feasible solution to  is a linear ordering of
variables . The value  of a solution  is the number of constraints in  that  satisfies. The goal of the problem is to find a solution of maximum possible value.
\end{definition}

We denote the value of the optimal solution by :

The average value  of an instance is the expected value of a solution chosen uniformly at random among  feasible solutions:

We say that  has arity  if each constraint
in  depends on at most  variables.

\begin{definition}
In the Satisfiability Above Average Problem,
we are given an instance of an ordering CSP of arity 
and a parameter . We need to decide if there is a solution  that satisfies at least  constraints, or, in other words, if .
\end{definition}

In this paper, we show that this problem is fixed-parameter tractable. To this end, we design an algorithm
that either finds a kernel on  variables or certifies that .
\begin{theorem}\label{thm:main}
There is an algorithm that given an instance of an
ordering CSP problem of arity  and a parameter ,
either finds a kernel on at most  variables
(where constant  depends only on )
or
certifies that .
The algorithm runs in time  linear in the number of constraints  and variables  (the coefficient in the -notation depends on ).	
\end{theorem}


\subsection{Ordering CSPs over }\label{sec:csp-over-product-space}

Consider an instance  of an ordering CSP on variables .
Let us say that a continuous feasible solution to  is an assignment of distinct values  to variables . Each continuous solution  defines
an ordering  of variables :  is less then  with respect to  if and only if .
We define the value of a continuous solution 
as the value of the corresponding solution (linear ordering) . We will denote the value of
solution  by
.

Note that if we sample a continuous solution
 uniformly at random, by  choosing values  independently and uniformly from , the corresponding solution 
will be uniformly distributed among  feasible solutions. Therefore,

Note that all  are distinct a.s. and thus a random point in  is a feasible
continuous solution a.s. 

\subsection{Efron---Stein Decomposition}\label{sec:EfronStein}
The main technical tool in this paper is the Efron---Stein decomposition. We refer the reader to~\cite[Section 8.3]{ODonnell} for
a detailed description of the decomposition. Now, we just remind its definition and basic properties.

The Efron---Stein decomposition can be seen as a generalization of the Fourier expansion
of Boolean functions on the Hamming cube . Consider the Fourier expansion of a function
,

where  are Fourier coefficients of . Informally,
the Fourier expansion  breaks  into pieces, , each of which depends on its own set of variables: The term
 depends on variables  and no other variables.

The Efron---Stein decomposition is an analogue of the Fourier expansion for functions defined on arbitrary product probability spaces. Consider a probability space  and the product probability space .
Let  be a function (random variable) on . Informally, the Efron---Stein decomposition of  is the decomposition of  into the sum of
functions ,

in which  depends on variables .

We formally define the  Efron---Stein decomposition  as follows.
Consider the space  of functions on  with bounded second moment. Note that .
That is, every  can be represented as

for some functions . Let  be the one-dimensional space of constant functions on . Let  be the orthogonal complement to . That is,  is the space of functions
 with . We have,  and

Expanding this decomposition, we get a representation of  as the direct sum of  spaces:

where  is the closed linear span of the set of functions of the form  where  if , and  if .
Since functions in  are constants,  equals the closed linear span of the set of functions of the form  where
.

Consider a function . Let  be the orthogonal projection of  onto .
Since the linear spaces  are orthogonal, we have

We call this decomposition the Efron---Stein decomposition of . We define the degree of  as ,
the size of the largest subset  such that   is not identically equal to  (we let the degree of  to be ).

Let  be a random element of . That is,  are  independent random elements of ; each of them is distributed according to . We write .
We will employ the following properties of the Efron---Stein decomposition (see \cite[Section 8.3]{ODonnell}).
\begin{enumerate}
\item  depends only on variables  with .
\item For every two sets  and , , we have .
\item Let  be subsets of . Suppose that there is an index  that belongs to exactly
one set . Then .
\end{enumerate}

We will also use the following equivalent and more explicit definition of the Efron---Stein decomposition.
For every subset  of indices , let


\section{Efron---Stein Decomposition of Ordering CSP Objective}\label{sec:ES-ordering-CSP}
In this section, we study the Efron---Stein
decomposition of the function .
To this end, we represent  as a sum of
``basic ordering predicates'' and then analyze the Efron---Stein of a basic ordering predicate.

\subsection{Basic Ordering Predicate}\label{subsec:bop}
Let  be a tuple of distinct
indices from  to .
Define the basic ordering predicate  for ,

Note that the indicator of each constraint
 is a sum of ordering predicates:

where the sum is over permutations of variables that
the constraint  depends on.
Since  is the sum of indicators of all
predicates  in ,  is also a sum
of basic ordering predicates :

 for some multiset .

\subsection{Efron---Stein Decomposition of Ordering Predicates}
Let  and  be the uniform measure on . We study the Efron---Stein decomposition
of a basic ordering predicate .

\ifconf\pagebreak\fi
\begin{theorem}\label{thm:ES-ordering-predicate}
Let  be a tuple of distinct indices of size . Denote . Consider the Efron---Stein decomposition of , , over  with uniform measure.
There exists a set of polynomials  with integer coefficients of degree at most  such that

where the summation is over all permutations  of . The polynomial  depends only on variables in . It is equal to  if
 is not a subset of .
\end{theorem}
\begin{proof}
We may assume without loss of generality that
.
Since  depends only on
variable ,  only if . We may therefore assume that 
for notational convenience.

Denote the elements of  by .
Define auxiliary variables
 and , and let  and .
Let  be the indicator of the event that
 for every .
Then .  Note that
.
All events for  (for )
are independent given variables . Therefore,

For each , we have

If , then . Otherwise,

We computed the probability above as follows: Given , the probability that
 for all 
equals . Then, given that 
and  for all , the probability that
 equals  as all orderings
of  are equally likely.
We get

Plugging this expression in (\ref{eq:g-subset-S}), we obtain the following formula

Observe that  divides
.
Thus

is a polynomial with integer coefficients of degree at most  . Denote this polynomial by .
Then,

where the sum is over all permutations  of .
Using the identity
,
we get a representation of  as

where  are some polynomials with integer coefficients.
\end{proof}
Since  is a sum of some basic ordering predicates (see Section~\ref{subsec:bop}), we get the following corollary.

\begin{corollary}\label{cor:ES-objective}
Let  be an instance of an ordering CSP problem of arity
at most . Let  be the value of continuous solution .
Then the Efron---Stein decomposition of  has degree at most . Moreover there exist
polynomials  with integer coefficients of degree at most  such that

where the summation is over some set  of tuples of indices in , and  depends only on \ifconf\else variables in\fi .
\end{corollary}
\subsection{Variance of Ordering CSP Objective}\label{sec:variance-CSP}
In this section, we show that the variance  if   (non-trivially) depends on at least  variables.

\begin{claim}\label{claim:compact}
There exists a sequence of positive numbers  such that for every polynomial  of degree at most  with integer coefficients we have .
\end{claim}
\begin{proof}
Consider the set  of polynomials over 
of degree at most . Let  be the set of polynomials in , whose largest
in absolute value coefficient is equal to  or .

Denote . For every , we have  since  is not identically equal to  on .
Note that  is a compact set and  is a continuous function on it.
Therefore,  attains its minimum on .
Let .

Now let  be a polynomial with integer coefficients of degree at most . Denote the absolute value of its largest
coefficient (in absolute value) by .  is a positive integer and thus . We have  and thus
\ifconf
.
\else

\fi
\end{proof}
\begin{lemma}\label{lem:beta}
The following claim holds for some positive parameters . Let  be an instance of arity at most . Let  be the Efron---Stein decomposition of . Then for every set  either  or .
\end{lemma}
\begin{proof}
Let , where  is as in Claim~\ref{claim:compact}.
Assume that . By Corollary~\ref{cor:ES-objective},

Note that all functions
 have disjoint support, and, therefore, are pairwise orthogonal.
Choose one tuple  such that .
We have, 
By Claim~\ref{claim:compact},  and hence
.
\end{proof}




We say that  depends on the variable  if there exist two vectors  and  that differ only in the -th
coordinate such that .

\begin{theorem}\label{thm:variance}
Let  be an instance of arity at most . Suppose that  depends on
at least  variables. Then  .
\end{theorem}
\begin{proof}
Consider the Efron---Stein decomposition of . Let
.
Note that  depends on all variables in  and no other variables. Thus, .
There are at least  non-empty sets  with  since each such set  contributes at most  elements to .
For , we have   and hence . By Lemma~\ref{lem:beta},
 if  and . We have,

\ifconf\else
as required.
\fi
\end{proof}

\section{Bonami Lemma for ordering CSPs}
We are going to apply Theorem~\ref{thm:bonami-for-ES} (the Bonami Lemma for the Efron---Stein decomposition) to the function , where  is the objective function of the ordering CSP problem. In order to do that, we show now that  satisfies the condition of the theorem
(Condition~(\ref{eq:Bonami-Fourth-Moment})) with some constant  that depends only on the arity of the CSP.

\begin{lemma}\label{lem:bound-on-C}
There exists a sequence of constants  such that the following holds. Let  be an instance of an ordering CSP of arity at most . Let  and  of cardinality at most . Then	

\end{lemma}
\begin{proof}
We assume that  is non-empty as otherwise both the left and right hand sides of the inequality are equal to 
(since ). Therefore, .

Since , we may assume without loss of generality that .
Let  be the set of all functions
of  of the form

where  are some polynomials of degree at most  with real coefficients. By Corollary~\ref{cor:ES-objective}, . Let . Note that  is
a compact set (since  is a finite dimensional space; and  is a norm on it). Therefore, the continuous
function

is bounded when . Denote its maximum by 
(note that  depends only on  and not on ).

Letting ,  we have,

as required.
\end{proof}


\section{Proof of Main Theorems}\label{sec:proof-main-thm}
In this section, we prove Theorems~\ref{thm:main} and~\ref{thm:main-intro}. We will need the following theorem.
\begin{theorem}[Corollary 1 from Alon, Gutin, Kim, Szeider, and Yeo~\cite{AGKSY}]\label{thm:AGKSY}
Let  be a real random variable. Suppose that , , and 
for some . Then .	
\end{theorem}

\begin{proof}[Proof of Theorem~\ref{thm:main}]
Let  be the set of variables that  depends on (see Section~\ref{sec:variance-CSP} for definitions).
By Theorem~\ref{thm:variance},  for some absolute constant .
By Lemma~\ref{lem:bound-on-C}, the function  satisfies condition (\ref{eq:Bonami-Fourth-Moment})
of the Bonami Lemma for the Efron---Stein Decomposition (Theorem~\ref{thm:bonami-for-ES}) with some absolute constant . Hence,
. Applying Theorem~\ref{thm:AGKSY}
to the random variable  with  and ,
we get that , where . Consequently,


We are now ready to state the algorithm. The algorithm computes the Efron---Stein decomposition in time .
Then, using the formula  (see Theorem~\ref{thm:variance}), it
finds the set  also in time . It considers two cases.
\begin{enumerate}
\item If , then the algorithm returns .
\item Otherwise, if , the algorithm outputs
the restriction of  to the variables in . This is a kernel for , since  depends only on the variables in .
\end{enumerate}
\vspace{-2mm}
\end{proof}
\vspace{-2mm}
To prove Theorem~\ref{thm:main-intro}, we need to show how to find an assignment satisfying  constraints if .
This can be easily done using -rankwise independent permutations. A random permutation  is -rankwise independent if
for every subset  of size , the order of elements in  induced by  is uniformly distributed (the definition
is due to~Itoh, Takei, and Tarui~\cite{ITT}). Note that any -wise independent permutation  is also an
-rankwise independent permutation. Using the result of Alon and Lovett~\cite{AL}, we can obtain
a -wise independent permutation  supported on a set of size . In Lemma~\ref{lem:rankwise} (in Appendix~\ref{sec:rankwise}),
we show that for some permutation  in the support of , we have .
Hence, to find an assignment satisfying  constraints, we need to search for the best permutation in the support of ,
which can be done in time .


\section{Bonami Lemma} \label{sec:Bonami}
In this section, we prove the Bonami Lemma for the Efron---Stein decomposition (Theorem~\ref{thm:bonami-for-ES}) stated in the introduction.
Our starting point will be the standard Bonami Lemma for Bernoulli  random variables.
\begin{lemma}[see~\cite{Bonami, ODonnell}]\label{lem:bonami-standard}
Let  be a polynomial of degree at most . Let  be independent
unbiased -Bernoulli variables. Then

\end{lemma}

We will consider the following probability distribution in this proof. Let  be a random variable equal to 3 with probability 
and to  with probability . Denote by  the probability distribution of .
We first prove a variant of the Bonami Lemma for random variables distributed according to .
\begin{lemma}\label{lem:bonami-Z}
Let  be a polynomial of degree at most . Let  be independent random variables
distributed according to . Then

\end{lemma}
\begin{proof}
Consider  Bernoulli random variables  (uniformly distributed in ).
Note that random variables  are distributed in the same way as random variables . Therefore,

Now  is a polynomial of  variables  of degree at most .
Applying Lemma~\ref{lem:bonami-standard} to , we get

and, therefore,

as required.
\end{proof}

\noindent Now let  and  be its Efron---Stein decomposition. Define polynomial
 by

We now get bounds for moments of  in terms of moments of .
\begin{claim}\label{claim:second-moment}
Note that  and thus .

Also,

\end{claim}
\begin{proof}
Note that  and thus .

Also,

Since , we have , and therefore

\end{proof}

\begin{claim}\label{claim:fourth-moment}
Let  and  be as in the condition of Theorem~\ref{thm:bonami-for-ES}. Then

\end{claim}
\begin{proof}
Write,


To prove the claim, we show that for every four sets , the following inequality holds:

Note first that if some index  appears in exactly one of the sets , , , and  then the expressions on the left and on the right
are equal to  (by Property 3 of the Efron---Stein decomposition in Section~\ref{sec:EfronStein}), and we are done. So we assume that every index  in  appears in at least 2 of the sets . Denote the number of times  appears in sets  by .

Applying the Cauchy---Schwarz inequality three times, we get,

By the condition of Theorem~\ref{thm:bonami-for-ES} and the definition of coefficients ,

We also have,

We compute  for . We get ,  and .
Thus, ,  and ,
and, consequently,

Since all coefficients  are non-negative, we get from (\ref{eq:f-product}) and (\ref{eq:Z-product})  that  inequality (\ref{enq:term-by-term}) holds.
\end{proof}

\begin{proof}[Proof of Theorem~\ref{thm:bonami-for-ES}]
By Lemma~\ref{lem:bonami-Z}, we have

From Claims~\ref{claim:second-moment} and~\ref{claim:fourth-moment}, we get

\end{proof}

\section{General Framework}\label{sec:general-framework}
\subsection{Filtered -Lattice of Functions}
In this section, we generalize the result of the paper to a more general class of constraint satisfaction problems having a lattice structure. In Section~\ref{sec:PPP},
we show that LP CSPs and valued CSPs with ``piece-wise polynomial predicates'' (see Section~\ref{sec:PPP} for the defintion) have a lattice structure.

\subsection{Discussion}
We note that in our proofs we used only few properties of ordering CSPs. Specifically, in Theorem~\ref{thm:ES-ordering-predicate}, we showed that
all functions in the Efron---Stein decomposition of the basic ordering predicate are in the set

Since  is closed under addition (the sum of any two functions in  is in ), we got that
all functions in the Efron---Stein decomposition of the ordering CSP objective  are also in .
Then in the proof of  Lemma~\ref{lem:beta}, we showed that every non-zero function in   has
variance at least  (where  depends only on ), and this was sufficient to get the result of the paper.
To summarize, we only used the following properties of the set of functions :
\begin{itemize}
\item[A.] all functions in the Efron---Stein decomposition of each predicate are in ,
\item[B.]  is closed under addition,
\item[C.] every non-zero function in  has variance at least  (for some fixed ).
\end{itemize}
\subsection{Filtered -Lattice of Functions}
We now formalize properties A, B, and C in the definitions of -lattice of functions and filtered -lattice of functions.
Recall first the definition of a lattice.
\begin{definition}
Let  be a finite-dimensional space and  be a subset of . We say that  is a lattice in  if
for some basis  of , we have .
We say that  is the basis of the lattice .
\end{definition}

Now we define an -lattice of functions.
\begin{definition}\label{def:A-lattice}
Let  be a probability space.
Consider a set  of bounded (real-valued) functions   on ; .
We say that  is an -lattice of functions of arity (at most)  on  if it satisfies the following properties.
\begin{enumerate}
\item  is a lattice in a finite dimensional subspace of .
\item If we permute arguments of a function in , we get a function in . Specifically, if  and  is a permutation of  then .
\end{enumerate}

For an -lattice , we write that a function  if  is in  after possibly renaming the arguments of 
(in other words,  is in  as an abstract function from  to ).\footnote{For example, let  be an -lattice of
functions of the form  where . Then , since after renaming  to  and  to  we get
, which is of the form  .}
\end{definition}
Clearly, every -lattice  of functions satisfies property~B. Since  is discrete, it also satisfies property~C (we will prove that formally in
Claim~\ref{claim:compact-A-CSP}). We also want to ensure that it satisfies an analog of property~A.
To this end, we consider the averaging operator , which takes the expectation of a function with respect to variable   and require
that  maps every function in the lattice to a function in the lattice.

\begin{definition}
For , let  be the averaging operator that maps a function  of arity  to a function  of arity  defined as follows:

\end{definition}
\begin{definition}
We say that a family of sets  (indexed by integer ) is a filtered -lattice of
functions of arity (at most)  if it satisfies the following properties.
\begin{enumerate}
\item  is an -lattice of functions of arity  on .
\item  is a filtration:  for .
\item For every  there exists , which we denote by , such that the operator  maps  to  (for every ).
\end{enumerate}
\end{definition}
We remark  that  is a filtered -lattice.
We are going to prove that our result for ordering CSPs holds, in fact, for any constraint satisfaction problem with predicates from a filtered -lattice.

\subsection{General -CSP(, )}
\begin{definition}
Consider a probability space . Let  be a filtered -lattice of functions and  is an integer.
An instance  of General -CSP(,  ) consists of a set of variables , taking values in ,
and a set of real-valued constraints of the form   where . The objective function  is the sum of
all the constraints.
General -CSP(, ) asks to find an assignment to variables  that maximizes
.
\end{definition}
We denote the optimal value of an instance  by  and the average
by , where  are independent random elements of  distributed according to the probability measure .
\begin{remark}\label{rem:measure-0}
Note that we follow the standard convention that two functions  are equal if  on a set of measure .
That is, we identify functions that are equal almost everywhere. Accordingly, we define  as the essential supremum  of :
  is equal to the maximum value of  such that

for every .
\end{remark}
We prove a counterpart of Theorem~\ref{thm:main} for the General -CSP problem.

\begin{theorem}\label{thm:main-A-CSP}
There is an algorithm that given an instance of General -CSP(, , ) and a parameter ,
either finds a kernel on at most  variables (where  depends only on the filtered -lattice  and numbers )
or
certifies that .
The algorithm runs in time , linear in the number of constraints  and variables  (the coefficient in the -notation depends the filtered -lattice  and numbers ).

 We assume that computing the sum of two functions in   requires constant time and that computing  requires constant time
(the time may depend on ).
\end{theorem}

We first prove analogues of Theorem~\ref{thm:ES-ordering-predicate} and Corollary~\ref{cor:ES-objective} for General -CSP.
\begin{lemma}[cf. Theorem~\ref{thm:ES-ordering-predicate}]\label{lem:ES-general-predicate}
Consider a probability space  and a filtered -lattice .
For every , there exists  so that the following holds. For every  and every ,
.
\end{lemma}
\begin{proof}
Let ,  (where  is as in the definition of a filtered -lattice), , and so on;
. Let .
Consider a function  and a set . By~(\ref{eq:ES-1}),

Denote the elements of  by  (where ). Note that

by the definition of a filtered -lattice.
Now by~(\ref{eq:ES-2}),

Since  is a lattice and  is a linear combination, with integer coefficients, of functions  (all of which are in ),
 is in .
\end{proof}
Since the set of functions  is closed under addition, we get that for every General -CSP(, ) instance  with objective
function , all functions  are also in .
\begin{corollary}[cf. Corollary~\ref{cor:ES-objective}]
Consider a probability space  and a filtered -lattice .
Let  be an instance of General -CSP(, ) and let  be its objective functions.
Then for every subset  of size at most , ;
for every subset  of size greater than , .

Furthermore, the Efron---Stein decomposition  of  can be computed in time .
\end{corollary}
\subsection{Compactness Properties of Filtered -Lattices}
We now prove counterparts of Claim~\ref{claim:compact} and Lemma~\ref{lem:bound-on-C} for General A-CSP.
\begin{claim}[cf. Claim~\ref{claim:compact}]\label{claim:compact-A-CSP}
Consider a probability space . Let  be an -lattice of functions of arity  on .
There exists a positive number  such that for every function , .
\end{claim}
\begin{proof}
Let  be the basis of lattice . Consider the linear span  of functions  (the set of all linear combinations with real coefficients).
Vector space  is finite dimensional. Let . Note that  is a compact set. All functions in  are non-zero (since
 are linearly independent),
and, therefore,  for every . Since  is compact,  . Denote .

Now consider a non-zero function . Write . Let . Since  and all coefficients  are integer, .
Note that . We have,

as required.
\end{proof}

\begin{lemma}[cf.~Lemma~\ref{lem:bound-on-C}]\label{lem:bound-on-C-A-CSP}
Consider a probability space . Let   be an -lattice of functions of arity  on .
There exists a constant  such that the following holds. Let  be a functions of arity at most , which
depends on a subset of variables . Assume that  (see Definition~\ref{def:A-lattice}). Then

\end{lemma}
\begin{proof}
Since function  depends on at most  variables among , we may assume without loss of generality
that it depends on a subset of . Then .
Let  be the linear span of .
Note that  is a finite dimensional vector space of functions.

Define . Since  is
a compact set, the continuous
function

is  bounded when . Denote its maximum by .

Letting , we have,

as required.
\end{proof}

\subsection{Variance of -CSP Objective}
We now prove a counterpart of Theorem~\ref{thm:variance} for General -CSP(, , ).
\begin{lemma}[cf. Theorem~\ref{thm:variance}]\label{lem:variance-A-CSP}
Consider a probability space . Let   be a filtered -lattice and  be an integer.
There exists a number , which depends only on  and  such that the following holds.
Let  be an instance of General -CSP(, ) and   be a parameter.
Either  has a kernel on at most  variables or . Moreover,
there is an algorithm that either finds a kernel on at most  variables or certifies that
. The algorithm runs in time , where  is the number of variables and  is the number of constraints.
\end{lemma}
\begin{proof}
Let  be as in Lemma~\ref{lem:ES-general-predicate}. Since  is an -lattice, by Claim~\ref{claim:compact-A-CSP},
there exists  such that  for every non-zero .
Note that  does not depend on  and .

Consider the Efron---Stein decomposition of . Let .
Function  depends only on variables in . Therefore, the restriction of  to variables in  is a kernel for . Let .
If , then we are done. So let us assume that .
There are at least  non-empty sets  with  since each such set  contributes at most  variables to .
Note that  for  and hence . Since ,
, if  and . We have,

as required.

Note that we can compute the Efron---Stein decomposition of  in time  and then find the set  in time . If , we output the restriction of  to  (which we compute in time ). Otherwise, we output that .
\end{proof}

\subsection{Proof of Theorem~\ref{thm:main-A-CSP}}
We are ready to prove Theorem~\ref{thm:main-A-CSP}.
\smallskip
\begin{proof}
Let  be as in Lemma~\ref{lem:variance-A-CSP} and  be as in Lemma~\ref{lem:bound-on-C-A-CSP}.
Let . Denote  Note that .

By Lemma~\ref{lem:variance-A-CSP}, either  has a kernel on at most  variables or
. In the former case, we output the kernel, and we are done. In the latter case, we show that . Assume that . By
Theorem~\ref{thm:bonami-for-ES} and Lemma~\ref{lem:bound-on-C-A-CSP},
.
By Theorem~\ref{thm:AGKSY}, , and hence .

The algorithm only executes the algorithm from Lemma~\ref{lem:variance-A-CSP}, so its running time is .
\end{proof}

\section{Piecewise Polynomial Predicates}\label{sec:PPP}
In this section, we present an interesting example of a filtered -lattice, the set of piecewise polynomial functions.  As a corollary,
we get that  the problem of maximizing the objective over average for a CSP with piecewise polynomial functions is fixed-parameter tractable.

\begin{definition}
Let us say that a subset  of  is -polyhedral if it is defined by a set of linear inequities on , in which all coefficients are bounded by  in absolute value.
 In other words,  is a -polyhedral set if for some  there exist a  matrix  and vector  (with  coordinates) such that
   (here, the inequality  is understood coordinate-wise), and  every entry of  and coordinate of  is bounded by  in absolute value.
 We denote the indicator function of a polyhedral set  by .
\end{definition}
\begin{definition}
We denote the set of polynomials  with real coefficients  of degree at most  by ;
we denote the set of polynomials   with integer coefficients of degree at most   by .
\end{definition}
\begin{definition}
We say that a function  is piecewise polynomial on polyhedral sets or -PPP if  is the sum of terms of the form , where  and  is a -polyhedral set.
\end{definition}

We note that every -PPP function can be written in the following ``canonical form''. Consider all hyperplanes in  of the form ,
in which . They partition  into polyhedrons. We call these polyhedrons elementary polyhedrons and denote the set of all
elementary polyhedrons by . Note that each -polyhedral set is a union of elementary polyhedrons.
Thus we can write every -PPP function  as follows:

where .


\begin{theorem}\label{thm:PPP-is-FAL}
Let  and  be the uniform measure on .
Let

Then  is a filtered -lattice of functions.
\end{theorem}
\begin{proof}
First, we prove  that each set  is an -lattice.
It follows from (\ref{eq:canonical-form}) that  is a lattice with basis ,
where  and  is a monomial of degree at most  (i.e.,  is of the form ).
Since every monomial  is bounded on , every basis function is bounded, and, therefore, all functions in  are bounded.
The definition of  is symmetric with respect to , hence if we permute the arguments of any function ,
we get a function in .

Now we show that  is a filtered -lattice. The inclusion  for  is immediate.
It remains to show that for every  there exists  such that  maps  to .
Let  and  (we note that, in fact, we can choose a much smaller value of ;
 however, we use this value to simplify the exposition). Observe  that all integer numbers between  and  divide .

It is sufficient to prove that  sends every basis function
 to .
Moreover, since the set of functions  is invariant under permutation of function arguments, we may assume without loss of generality that .
Denote , where .
Consider the set of linear inequalities  that define polyhedron . All coefficients in each of the inequalities are bounded by  in absolute value.
Let  be those inequalities that do not depend on  and  be those that do depend on .
We rewrite every inequality in  as follows. Consider an inequality in . Let  be the coefficient of  in it.
We multiply the inequality by , and if , we change the comparison sign in the inequality to the opposite.
Finally, we move all terms in the inequality other than  to the right hand side.
We get an equivalent inequality of the form either  or ,
where  and  are linear functions with integer coefficients bounded by  in absolute value.
Denote the inequalities of the form  by

and the inequalities of the form  by


Let  be the set of points  such that
 or 
for some . Note that  has measure .

Define  polyhedrons  in . For  and , let  be the
polyhedron defined by the following inequalities:
\begin{enumerate}
\item all inequalities in ,
\item inequality ,
\item inequalities  for every ,
\item inequalities  for every .
\end{enumerate}
Inequalities in items 2--4 are equivalent to the following condition (except for points in ):


Note that if  then  for

Also note that polyhedrons  are disjoint.
Now let


Let  be a point in . Consider two cases.
\medskip

\paragraph{Case 1} First, assume that  for some  and . Then

The point  satisfies all inequalities in  since .
Hence,  if and only if it satisfies all inequalities in , which are equivalent to

Combining this with~(\ref{ineq:Pjj-constraints}), we get that  if and only if

Therefore,


\paragraph{Case 2} Now assume that  for every  and .  Then there is no  such that . Therefore,

(The equality holds on a set of full measure; see Remark~\ref{rem:measure-0}.)

We conclude that

All coefficients in the inequalities that define  are bounded by  in absolute value, and
. Therefore,
 is an -PPP function. Thus .
\end{proof}

From Theorems~\ref{thm:main-A-CSP} and~\ref{thm:PPP-is-FAL}, we get the following corollary.
\begin{corollary}
For every ,  and , there is an algorithm that given an instance of a constraint satisfaction problem on  variables 
with  real-valued constraints, each of which is a -PPP function of arity , and a parameter ,
either finds a kernel on at most  variables or certifies that .
The algorithm runs in time . (The coefficient  and the coefficient in the -notation depend only on , , and ).
\end{corollary}
\begin{proof}
Let .
We apply Theorem~\ref{thm:main-A-CSP} to filtered -lattice  from Theorem~\ref{thm:PPP-is-FAL} and get the corollary.
\end{proof}

\smallskip
Since every constraint in a  -LP CSP problem is a -PPP function of arity  (see Definition~\ref{def:LP-CSP}), we get the following corollary.
\begin{corollary}
For every  and , there is an algorithm that given an instance of -LP CSP
either finds a kernel on at most  variables
or certifies that .
The algorithm runs in time  (The coefficient  and the coefficient in the -notation depend only on  and ).
\end{corollary}
\begin{remark}
Note that for an instance of -LP CSP, we have

since all LP constraints are strict. If we were to use non-strict ``less-than-or-equal-to'' and ``greater-than-or-equal-to'' LP constraints, we would
have to define
 as  , and not as ,
since, in general,  might not be equal to .
For example, consider an instance of -LP CSP with two constraints  and ; we have

but

(The maximum is attained on a set of measure , where .)
\end{remark}
 
\section*{Acknowledgement}
The authors thank Matthias Mnich for valuable comments. Yury Makarychev was supported by NSF CAREER award CCF-1150062 and NSF award IIS-1302662.


\begin{thebibliography}{WW}
\bibitem{AGKSY} N. Alon, G. Gutin, E. J. Kim, S. Szeider, and A. Yeo. Solving Max -SAT Above a Tight Lower Bound. Algorithmica 61 (3), pp.~638--655 (2011).	

\bibitem{AL}
N. Alon and S. Lovett. Almost -wise vs. -wise independent permutations, and uniformity for general group actions.
Approximation, Randomization, and Combinatorial Optimization. Algorithms and Techniques. Springer Berlin Heidelberg, pp.~350--361, 2012.

\bibitem{AM}
P. Austrin and E. Mossel. Approximation resistant predicates from pairwise independence. Computational Complexity, 18(2), pp.~249--271, 2009.


\bibitem{Bonami}
A. Bonami. \'{E}tude des coefficients Fourier des fonctions de . Annales de
l' Institut Fourier, 20(2):335--402, 1970.

\bibitem{Chan}
S. Chan. Approximation resistance from pairwise independent subgroups.
STOC 2014, pp.~447--456.

\bibitem{CMM}
M. Charikar, K. Makarychev, and Y. Makarychev.
On the Advantage over Random for Maximum Acyclic Subgraph.
FOCS 2007, pp.~625--633.



\bibitem{CFGJRTY} R. Crowston, M. Fellows, G. Gutin, M. Jones, F. Rosamond, S. Thomass{\'e} and A. Yeo.
Simultaneously Satisfying Linear Equations Over : MaxLin2 and Max--Lin2
Parameterized Above Average. In FSTTCS 2011, LIPICS Vol. 13, 229--240.

\bibitem{CGJRS}
R. Crowston, G. Gutin, M. Jones, V. Raman, and S. Saurabh. Parameterized complexity of MaxSat above average. In the Proceedings of LATIN 2012, pp. 184--194.

\bibitem{CS}
B.~Chor and M.~Sudan. A geometric approach to betweenness. SIAM Journal on Discrete Mathematics, vol. 11, no. 4 (1998): pp.~511--523.

\bibitem{DS96}
P. Diaconis and L. Saloff-Coste. Logarithmic Sobolev inequalities for finite Markov chains. Ann. Appl. Probab.~6
(1996), no. 3, 695--750.

\bibitem{ES} B.~Efron and C.~Stein. The jackknife estimate of variance. Annals of
Statistics, 9(3):586--596, 1981.

\bibitem{ODonnell} R. O'Donnell. Analysis of Boolean Functions. Cambridge University Press. 2014. ISBN 9781107038325.

\bibitem{GHMRC}
V. Guruswami, J. H\aa{}stad, R. Manokaran, P. Raghavendra, and M. Charikar. Beating the random ordering is hard: Every ordering CSP is approximation resistant. SIAM Journal on Computing 40, no. 3 (2011): 878--914.


\bibitem{GR}
V.~Guruswami and P.~Raghavendra. Constraint Satisfaction over a Non-Boolean Domain:
Approximation Algorithms and Unique-Games Hardness. APPROX 2008, pp.~77-90.

\bibitem{GZ}
V. Guruswami and Y. Zhou.
Approximating Bounded Occurrence Ordering CSPs. APPROX-RANDOM 2012, pp.~158--169.

\bibitem{GIMY}
G. Gutin, L. van Iersel, M. Mnich, and A. Yeo. All Ternary Permutation Constraint Satisfaction Problems Parameterized Above Average Have Kernels with Quadratic Number of Variables. J. Comput. Syst. Sci. 78 (2012), 151--163.

\bibitem{GKMY}
G. Gutin, E. J. Kim, M. Mnich, and A. Yeo. Betweenness parameterized above tight lower bound. J. Comput. Syst. Sci., 76: 872--878, 2010.

\bibitem{GKSY}
G. Gutin, E.J. Kim, S. Szeider, and A. Yeo. A Probabilistic Approach to Problems Parameterized
Above or Below Tight Bounds. J. Comput. Syst. Sci. 77 (2011), 422--429.


\bibitem{GY} G.~Gutin and A.~Yeo. Constraint satisfaction problems parameterized above or below tight bounds: a survey. In The Multivariate Algorithmic Revolution and Beyond,
pp.~257--286, 2012.

\bibitem{Hastad}
J. H{\aa}stad. Some optimal inapproximability results. STOC 1997.


\bibitem{ITT}
T.~Itoh, Y.~Takei, and J.~Tarui.
On Permutations with Limited Independence.
SODA 2000, pp.~137--146.

\bibitem{Karp72}
R.~Karp.
Reducibility among combinatorial problems. In Complexity of Computer Computations.
New York: Plenum, 1972, pp. 85--103.

\bibitem{KW} E.~J.~Kim and R.~Williams. Improved parameterized algorithms for above average constraint satisfaction. In Parameterized and Exact Computation, pp. 118--131, 2012.

\bibitem{MR} V. Mahajan and V. Raman. Parameterizing above Guaranteed Values: MaxSat and MaxCut. Journal of Algorithms, Vol. 31, Issue 2, May 1999, pp. 335--354.

\bibitem{Mak} K.~Makarychev. Local Search is Better than Random Assignment for Bounded Occurrence Ordering -CSPs. STACS 2013, pp.~139--147.

\bibitem{M-betw} Y.~Makarychev. Simple linear time approximation algorithm for betweenness. Operations Research Letters, vol.~40, no. 6 (2012), pp.~450---452.

\bibitem{MOS}
E.~Mossel, K.~Oleszkiewicz and A.~Sen.
On Reverse Hypercontractivity
Geometric and Functional Analysis, vol 23(3), pp.~1062--1097, 2013.

\bibitem{Opatrny79}
J.~Opatrny. Total ordering problem. SIAM Journal on Computing, 8(1):111--114, Feb. 1979.

\bibitem{RS} I. Razgon and B. O'Sullivan. Almost 2-SAT is fixed-parameter tractable. J. Comput. Syst.
Sci. 75(8):435--450, 2009.

\bibitem{Seymour} P.~D.~Seymour. Packing directed circuits fractionally. Combinatorica, vol.~15, no. 2 (1995), pp.~281--288.

\bibitem{Talagrand94}
M. Talagrand. On Russo’s approximate 0-1 law , Annals of Probability 22 (1994), 1576--1587.

\bibitem{Wolf07}
P.~Wolff. Hypercontractivity of simple random variables, Studia Mathematica
180 (2007), pp.~219--326.
\end{thebibliography}
\appendix




\section{Rankwise independent permutations}\label{sec:rankwise}
In this section, we prove the following lemma.

\begin{lemma}\label{lem:rankwise}
If  is a random  rankwise independent permutation and , then for some  in the
support of , .
\end{lemma}
\begin{proof}
Let  be a permutation uniformly distributed among all  permutations. The random variables  and 
are identically distributed. Hence,
 Observe, that

since for every four predicates , we have

Hence, as in Theorem~\ref{thm:main},  and
.
Consequently, by Theorem~\ref{thm:AGKSY}, .
This concludes the proof.
\end{proof}


\end{document} 