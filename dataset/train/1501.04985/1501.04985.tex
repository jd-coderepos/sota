\documentclass[
final
]{dmtcs-episciences}




\usepackage[utf8]{inputenc}
\usepackage{subfigure}



\usepackage[round]{natbib}

\usepackage{arcs}

\usepackage[usenames,dvipsnames]{xcolor}
\usepackage{makeidx} 
\usepackage{algorithm}
\usepackage{algorithmic}
\usepackage{graphicx,tipa}
\usepackage{arcs,lmodern,fix-cm}
\usepackage{amsmath,amstext,amsthm,url,setspace, enumerate}
\usepackage{rotating}


\usepackage{fourier}

\newcommand{\barr}[1]{\overline{#1}}
\newcommand{\vect}[1]{\overrightarrow{#1}}
\newcommand{\sinn}[1]{\sin \left({#1}\right)}
\newcommand{\arcsinn}[1]{\arcsin \left({#1}\right)}
\newcommand{\coss}[1]{\cos \left({#1}\right)}
\newcommand{\RA}{\ensuremath{R_1}}
\newcommand{\RB}{\ensuremath{R_2 }}
\newcommand{\postB}{\ensuremath{t_d}} \newcommand{\postC}{\ensuremath{t_d}} \def\qed{\hfill\rule{2mm}{2mm}}
\newcommand{\ff}{face-to-face }
\newcommand{\arccc}[1]{
\widearc{#1}
}
\newcommand{\ignore}[1]{}

\newtheorem{theorem}{Theorem}[section]
\newtheorem{claim}[theorem]{Claim}
\newtheorem{proposition}[theorem]{Proposition}
\newtheorem{lemma}[theorem]{Lemma}
\newtheorem{corollary}[theorem]{Corollary}
\newtheorem{conjecture}[theorem]{Conjecture}
\newtheorem{observation}[theorem]{Observation}
\newtheorem{fact}[theorem]{Fact}
\theoremstyle{definition}
\newtheorem{example}[theorem]{Example}
\newtheorem{definition}[theorem]{Definition}
\newtheorem{assumption}[theorem]{Assumption}
\newtheorem{remark}[theorem]{Remark}
\newtheorem{problem}[theorem]{Problem}
\newtheorem*{theorem_notag}{}






\author{
J. Czyzowicz\affiliationmark{1}~\thanks{Research supported in part by NSERC of Canada.} 
\and
K. Georgiou\affiliationmark{2}~\thanks{Research supported in part by NSERC of Canada.}
\and 
E. Kranakis\affiliationmark{3}~\thanks{Research supported in part by NSERC of Canada.}
\and
L. Narayanan\affiliationmark{4}~\thanks{Research supported in part by NSERC of Canada.} \\
\and 
J. Opatrny\affiliationmark{4}~\thanks{Research supported in part by NSERC of Canada.}
\and
B. Vogtenhuber\affiliationmark{5}
}

  
  
  
\title[Evacuating Robots from a Disk Using Face-to-Face Communication]{Evacuating Robots from a Disk \\
Using Face-to-Face Communication\footnote{An extended abstract of this paper appeared in the proceedings of the 8th International Conference on Algorithms and Complexity (CIAC'15)~\cite{CzyzowiczGKNOV15}.
}
}
\affiliation{
D\'{e}pt. d'Informatique, Universit\'{e} du Qu\'{e}bec en Outaouais,
Gatineau, QC, Canada\\
  Dept. of Mathematics, 
Ryerson University, 
Toronto, ON, Canada\\
School of Computer Science, Carleton University, Ottawa ON, Canada\\
Dept. of Comp. Science and Soft. Engineering, 
Concordia University, Montreal, QC,  Canada \\
Institute of Software Technology,
Graz University of Technology, Graz, Austria}
\keywords{Disk, Evacuation, Face-to-Face Model}
\received{2020-03-13}
\revised{2020-06-12, 2020-07-22}
\accepted{2020-07-31}
\begin{document}
\publicationdetails{22}{2020}{4}{4}{6198}
\maketitle
\begin{abstract}
Assume that two robots are located at the centre of a unit disk.
Their goal is to {\em evacuate} from the disk 
through an {\em exit} at an unknown location on the boundary of the disk.
At any time the robots can move anywhere they choose on the disk, independently of each other, with maximum speed . 
The robots can cooperate by exchanging information 
whenever they meet.
We study algorithms for the two robots to minimize the {\em evacuation time}: the time when {\em both} robots reach the exit.

Czyzowicz et al.~(2014)
gave an algorithm defining trajectories for the two 
robots yielding evacuation time at most  and also proved that any algorithm has evacuation time at least .
We improve both the upper and lower bound on the evacuation time of a unit disk. 
Namely, we present a new non-trivial algorithm whose evacuation time is at most 
and show that any algorithm has evacuation time at least .
To achieve the upper bound, we designed an algorithm which proposes a forced meeting between the two robots, even if the exit has not been found by either of them. We also show that such a strategy is provably optimal for a related problem of searching for an exit placed at the vertices of a regular hexagon. 
\end{abstract}









\section{Introduction}

The goal of traditional search problems is to find an object that is
located in a specific domain. This subject of research has a long history
and there is a plethora of models 
investigated in the mathematical and theoretical
computer science literature with emphasis
on probabilistic search in~\cite{stone1975theory},
game theoretic applications in~\cite{alpern2003theory},
cops and robbers in~\cite{anthony2011game},
classical pursuit and evasion in~\cite{nahin2012chases},
search problems and group testing in ~\cite{ahlswede1987search},
and many more. 

In this paper, we investigate the problem of searching for a stationary point target called an {\em exit} at an unknown location using two robots. 
This type of collaborative search is advantageous in that it reduces the required search time by distributing the search effort between the two robots. 
In previous work on collaborative search, the goal has generally been to minimize the time taken by the {\em first} robot to find the object of the search. 
In contrast, in the work at hand, we are interested in minimizing the time when the {\em last robot} finds the exit. 
In particular, suppose two robots are in the interior of a region with a single exit. 
The robots need to evacuate the region but the location of the exit is unknown to them. 
The robots can cooperate to search for the exit, but it is not enough for one robot to find the exit, we require {\em both} robots to reach  the exit as soon as possible. 

We study the problem of two robots that  start at the same time at the centre of a unit disk and attempt to reach an exit placed  at an unknown location on the boundary of the disk. 
At any time the robots can move anywhere they choose within the disk. 
Indeed, they can take short-cuts by moving in the interior of the disk if desired. 
We assume that their maximum speed is 1. 
The robots can communicate with each other only if they are at the same point at the same time: we call this communication model {\em \ff  communication}.  
Our goal is to schedule the trajectories of the robots so as to minimize the {\em evacuation time}, which is the time it takes both robots to reach the exit (for the worst case location of the exit). 

Our main contributions pertain to improved upper and lower bounds for the evacuation problem on the disc. The main ideas are derived by the study of a related evacuation problem in which the exit lies in one of the vertices of a regular hexagon. For the latter problem, we introduce a novel and non-intuitive search strategy in which searchers are forced to detour and meet if the exit is not found early enough. Surprisingly, we show this trajectory to induce optimal evacuation cost for that problem. The implications are two-fold. First, the lower bound we obtain applies directly to the main evacuation problem on the disc. Second, we use the same ideas of a detour and forced meeting to improve the best evacuation cost known (at the time when the result was first announced; see Section~\ref{sec: related work} for newer developments). 



\subsection{Related work}
\label{sec: related work}




The most related work to ours is ~\cite{CGGKMP}, where the evacuation problem for a set of robots all
starting from the centre of a unit disk was introduced and studied.  Two communication models are introduced in \cite{CGGKMP}. In the {\em wireless} model,
the two robots can communicate at any time regardless of their locations. In particular, a robot that finds the exit can immediately communicate its location to the other
robot. The other model is called the  {\em non-wireless or local} model in \cite{CGGKMP}, and is the same as our \ff model:  two robots can only communicate when they are face to face,  that is, they are at the same point location at the same time. In \cite{CGGKMP}, for the case of 2 robots, an algorithm with  evacuation time  is  given for the wireless model; this is shown to be optimal. For the \ff  model,  they prove an upper bound of 
 and a lower bound of   on the evacuation time. 
Since the first announcement of our results in~\cite{CzyzowiczGKNOV15}, 
Brandt at al.~\cite{Watten2017} improved our upper bound from 5.628 to 5.625. Their contributions also pertain to a significant simplification of the search trajectories (that use detours but without forced meetings), along with a clever performance analysis. 
Very recently, \cite{disser2019evacuating} reported a further improvement of 5.6234 by employing and generalizing techniques introduced in~\cite{Watten2017}.
However, to the best of our knowledge, the lower bound we provide in this manuscript is the best currently known. 

Since the introduction of the problem in~\cite{CGGKMP}, a number of interesting variations emerged some of which are summarized in recent survey~\cite{CGK19search} (see also~\cite{flocchini2019distributed} for a collection of broadly related problems).
Some representative examples of related problems include 
an evacuation problem with speed bounds in the wireless model~\cite{lamprou2016fast},
evacuation from triangles~\cite{CzyzowiczKKNOS15,ChuangpishitMNO17},
evacuation from multiple rays~\cite{BrandtFRW20},
evacuation from graphs~\cite{Borowiecki0DK16},
search with terrain dependent speeds~\cite{CzyzowiczKKNOS17},
wireless evacuation from multiple exits~\cite{czyzowicz2018evacuating,PattanayakR0S18},
priority evacuation~\cite{CGKKKNOS18b,CzyzowiczGKKKNO18},
evacuation with faulty robots~\cite{czyzowicz2017evacuation,pattanayak2019chauffeuring,GKLPP19Algosensors} or with probabilistically faulty robots~\cite{bgmp2020probabilistically},	
evacuation with immobile agents~\cite{GKKa16,dmtcs:5528},
time energy tradeoffs for evacuation on the line \cite{czyzowiczICALP2019,kranakis2009time},
worst-case average-case tradeoffs for evacuation on the disc~\cite{chuangpishit2018average}, 
and searching graphs~\cite{angelopoulos2019expanding}.



In a different direction, Baeza-Yates {\em et al} posed the question of minimizing the worst-case trajectory of a single robot searching for a target point at an unknown location in the plane ~\cite{baezayates1993searching}. This was generalized to multiple robots in \cite{LS01}, and more recently has been studied  in \cite{Emekicalp2014,Lenzen2014}. However, in these papers, the robots cannot communicate, and moreover, the objective is for the first robot to find the target. 
Two seminal and influential papers (that appeared almost at the
same time) on probabilistic search that concern minimizing the {\em expected time}  for the robot to find the target are
\cite{beck1964linear}~and~\cite{bellman1963optimal}.
As for two surveys on search theory the interested reader is referred to~\cite{benkoski1991survey}~and~\cite{dobbie1968survey}.
The latter survey also contains 
an interesting classification of
search problems by search objectives, distribution of effort, point target
(stationary, large, moving), two-sided search, and other criteria.
The evacuation problem considered in our paper
is related to searching on a line, in that we are searching
on the boundary of a disk but with the
additional ability to make short-cuts in order
to enable the robots to meet sooner
and thus evacuate faster. 
Our problem is also related to the  {\em rendezvous problem}  and the problem of  {\em gathering} \cite{alpern1999asymmetric,PGNP2005}. Indeed our problem can be seen as a version of a rendezvous problem for three robots, where one of them remains stationary. 




\subsection{Preliminaries and notation}
\label{subsec:prelim}







We assume that two robots \RA\ and \RB\  are initially at the centre of a disk with radius 1, and that there is an exit at some location  on the boundary of the disk. 
The robots do not know , but do know each other's algorithms. 
The robots move at a speed subject to a maximum speed, say 1. 
They cannot communicate except if they are at the same location at the same time. 
The {\em evacuation problem} is to define trajectories for the two robots that minimize the {\em evacuation time}.
All our results (upper and lower bounds) pertain to deterministic algorithms that work against a (worst case) deterministic adversary. 


For two points  and  on the unit circle, the length of an arc  is denoted by , 
while the length of the corresponding chord (line segment) will be denoted by  
(arcs on the circle are always read clockwise, i.e., arc  together with arc  cover the whole circle). 
By  we denote the angle at  in the triangle . Finally by  we denote the vector with initial point  and terminal point , .i.e. vector  when  are understood as points in the Cartesian plane. 












\subsection{Outline, results of the paper and new improvements}

In \cite{CGGKMP} an algorithm is given defining a
trajectory for two robots in the \ff communication
model with evacuation time
 and it is also proved that any such algorithm has evacuation 
time at least .

Our main contribution in this paper is to improve both the upper and lower bounds
on the evacuation time. Namely,
we give a new algorithm whose evacuation time is at most 
(see Section~\ref{sec:Evacuation Algorithms})
and also prove that any algorithm has evacuation 
time at least 
(see Section~\ref{sec:Lower Bound}).
To prove our lower bound on the disk, we first give tight bounds for the problem of evacuating a regular hexagon where the exit is placed at an unknown vertex. 
We observe that, surprisingly, in our optimal evacuation algorithm for the hexagon, the two robots are forced to meet after visiting a subset of vertices, even if an exit has not been found at that time. 
We use the idea of such a forced meeting in the design of our disk evacuation algorithm, inducing this way an improvement of the upper bound from 5.740 to 5.628. 
It is still unknown  whether such a forced meeting is necessary. However, since the first announcement of the upper bound, Brandt at al.~\cite{Watten2017} proposed an elegant analysis of a simplified trajectory that avoids forced meetings, still improving the upper bound from 5.628 to 5.625. The authors in the latter paper even proposed a refinement of their technique involving multiple detours as the robots are searching for the exit. Their ideas were later materialized in~\cite{disser2019evacuating} that reported a further improvement of 5.6234.







\section{Evacuation Algorithms}
\label{sec:Evacuation Algorithms}


In this section we give two new evacuation algorithms for two robots in the  \ff model that take evacuation time approximately
 and  respectively. 
We begin by presenting  Algorithm  proposed by \cite{CGGKMP}
which has been shown to have evacuation time 
.
Our goal is to understand the worst possible configuration for this algorithm, and to subsequently modify it accordingly so as to improve its performance. 

 All the algorithms we present follow the same general structure: 
 The two robots \RA\ and \RB\ start by moving together to an arbitrary point  on the boundary of the disk. 
 Then \RA\ explores the arc , where  is the antipodal point of , by moving along some trajectory defined by the algorithm.
 At the same time, \RB\ explores the arc , following a trajectory that is the reflection of \RA's trajectory.
 If either of the robots finds the exit, it immediately uses the {\em Meeting Protocol} defined below to meet the other robot (note that the other robot has not yet found the exit and hence keeps exploring). 
 After meeting, the two robots travel together on the shortest path to the exit, thereby completing the evacuation. 
 At all times, the two robots travel at unit speed. 
 Without loss of generality, we assume for our analysis that \RA~ finds the exit and then {\em catches}~\RB. 



 \paragraph{Meeting Protocol for :}
If at any time ,    finds the exit at point , it 
computes the shortest additional time  such that , after traveling 
distance , is located at point  satisfying . 
Robot \RA\ moves along the segment . At time  the two robots meet 
at  and traverse  directly back to the exit at  incurring total time cost 
.

\subsection{Evacuation Algorithm  \texorpdfstring{}{Lg} of \texorpdfstring{\cite{CGGKMP}}{Lg}}


We proceed by describing the trajectories of the two robots in Algorithm  
. As mentioned above, both robots start 
from the centre  of the disk and move together to an arbitrary position  on the 
boundary of the disk.  then moves clockwise along the boundary of the disk up to distance , see the
left-hand side of Figure~\ref{fig: old-algo}, and robot  moves counter clockwise on the trajectory that is a reflection of 's trajectory with respect to the 
line passing through  and . When  finds the exit, it invokes the  {\it meeting protocol} in order to meet  , after which the evacuation is completed. 



\begin{figure}[!ht]
                \centering
                \includegraphics[scale=0.5]{FigsEvac/old-evac.pdf}
				\caption{Evacuation Algorithm  with exit position .  
				The trajectory of robot  is depicted on the left. 
				The movement paths of robots  are shown on the right, until the moment they meet at point  on the circle. 
}
                \label{fig: old-algo}
\end{figure}




The meeting-protocol trajectory of  in Algorithm  is depicted in the right-hand side of Figure~\ref{fig: old-algo}, with exit point  and meeting point . 
Clearly, for the two robots to meet, we must have . Next we want to analyze the performance of the algorithm, with respect to , i.e. the length  of the arc that  travels, before it is met by . We also set .
It follows that , and therefore\footnote{We are using the elementary fact that in a unit circle, an arc of length  corresponds to a chord of length }

In other words,  \footnote{Uniqueness of the root of the equation defining  is an easy exercise.} 
is the length of the segment  that  
needs to travel in the interior of the disk after locating the exit at , 
to meet  at point . 

Then the cost of Algorithm , given that the two robots meet at time  after they together reached the boundary of the disk at , is . Given that the distance  traveled by  until finding the exit is between  and , it directly follows that  can take any value between  and  as well. Hence, the worst case performance of Algorithm  is determined by

The next lemma, along with more details of its proof, follows from~\cite{CGGKMP}.
\begin{lemma}\label{lem: monotonicity of x+f(x)}
Expression  attains its supremum at 
(which is ). In particular,  is strictly increasing when  and strictly decreasing when .
\end{lemma}

\begin{proof}
The behavior of , as  ranges in , is shown in Figure~\ref{fig: performance old-algo}.
\begin{figure}[!ht]
                \centering
                \includegraphics[scale=0.3]{FigsEvac/OldAlgoPerformance.pdf}
                \caption{The performance of Algorithm  as a function of the meeting points of the robots.}
                \label{fig: performance old-algo}
\end{figure}
\end{proof}

By Lemma~\ref{lem: monotonicity of x+f(x)}, the evacuation time of Algorithm  is . The worst case position of the exit is attained for . 





\subsection{New evacuation algorithm \texorpdfstring{}{Lg}}


We now show how to improve Algorithm  and obtain an evacuation time of at most~.
The main idea for the improvement is to change the trajectory of the robots when the distance traveled on the boundary of the disk approaches the critical value  of Lemma~\ref{lem: monotonicity of x+f(x)}. Informally, 
robot  could meet  earlier if it makes a linear detour inside the interior of 
the disk towards  a little before traversing distance . 

We describe a generic family of algorithms that realizes this idea. The specific
trajectory of each algorithm is determined by two parameters  and   where  and , whose optimal values will be determined later. 
For ease of exposition, we assume that  finds the exit. The trajectory of  (assuming it has not yet met ) is  
partitioned into four phases that we call the \textit{deployment}, \textit{pre-detour}, \textit{detour} and \textit{post-detour} phase. 
The description of the phases relies on the left-hand side of Figure~\ref{fig: line-evac New}.\\

\noindent
{\bf Algorithm }(with a linear detour).  
The phases of robot 's trajectory (unless it is met by  to go to the exit) are: \\  \textit{Deployment phase:} 
Robot  starts from the centre  of the disk and moves to an arbitrary position  on the boundary of the disk. \\  \textit{Pre-detour phase:} 
 moves clockwise along the boundary of the disk until having explored an arc of length . 
Let  be the point on the circle in which this phase ends.\\
 \textit{Detour phase:}
Let  be the reflection of  with respect to  (where  is the antipodal point of ). 
 moves on a straight line towards the interior of the disk and towards the side where  lies, 
forming an angle of  with line  until it reaches the line  at point . 
At ,  turns around and follows the same straight line back to . 
Note that by the restrictions on ,  is indeed in the interior of the line segment . \\
 \textit{Post-detour phase:}
Robot  continues moving clockwise on the arc . \\label{equa: meeting condition line interior trajectory}
y+h(y)=\chi+q(y)

\max_{y \in [\chi - f(\chi) , \chi] } \{ y + 2 h(y) \}.
\label{equa: def g}
p(x)~:=~ \textrm{unique}~z~\textrm{satisfying}~ z= 2\sinn{x+\frac z2}.
\postB+2p(\chi) = \chi+2\sinn{\chi}/\coss{\phi} + 2p(\chi).\label{eqa: line algo performance}
1+\max
\left\{
\chi+f(\chi),~~
\max_{y \in [\chi - f(\chi) , \chi] } \{ y + 2 h(y) \},~~
\chi+2\sinn{\chi}/\coss{\phi} + 2p(\chi)
\right\},

h(y)=
\frac{
2+(\chi-y)^2 - 2\cos(\chi+y) + 2 (\chi-y) \left(\sin (\phi+y)- \sin (\phi-\chi) \right)
}{2 (\chi-y-\sin (\phi-\chi)+\sin (\phi+y))}.

\theta &= \notag
\frac \pi 2 - \angle{EHB}
= \frac \pi 2 - \frac{\arccc{BE}}{2} 
= \frac \pi 2 - \frac{\arccc{BD}+\arccc{DE}}{2}\\
&= \label{equa: value of theta}
\frac \pi 2 - \frac{2(\pi-\chi)+\chi-y}{2}
=\frac{\chi+y}{2}-\frac{\pi}{2}
\label{equa: inner products line algo}
\barr{EG}^2=\barr{EH}^2+\barr{HB}^2+\barr{BG}^2 
+ 2\left(
\vect{EH}\cdot\vect{HB}+
\vect{EH}\cdot\vect{BG}+
\vect{HB}\cdot\vect{BG}
\right).

\vect{EH}\cdot\vect{HB} &= 
\barr{EH} ~\barr{HB} \coss{\pi - \angle{EHB}}
\\
\vect{EH}\cdot\vect{BG} &=
\barr{EH} ~\barr{BG} \coss{\angle{GBL}}
\\
\vect{HB}\cdot\vect{BG} &=
\barr{HB} ~\barr{BG} \coss{\pi- \angle{GBH}}.

\coss{\pi - \angle{EHB}}&=\coss{\pi/2+\theta}= \coss{\frac{\chi+y}2}
\\
\coss{\angle{GBL}} &=\coss{\pi-\phi}=\coss{\phi}
\\
\coss{\pi- \angle{GBH}} &= \coss{\pi +\frac\pi2+\theta - \phi}=
-\coss{\frac{\chi+y}2 - \phi}.
\label{equa: meeting condition piecewise line}
\arccc{EA}+\barr{EK}=\arccc{AB}+\barr{BG}+\barr{GK}.

N(\chi, y, \lambda, \phi):=
&
~~~~2+\lambda^2+(\lambda+\chi-y)^2+2 \lambda (\sinn{\phi-\chi}-\sinn{\phi+y}) \\
&+2 (\lambda+\chi-y) (\sinn{\chi}+\sinn{y} 
	- \lambda \coss{\phi})-2 \coss{\chi+y}, \ \ \mbox{ and} \\
D(\chi, y, \lambda, \phi):=
&2 (\lambda+\chi - y +\sinn{\chi}+\sinn{y} -\lambda \coss{\phi}).

\lambda+\sinn{\chi}-\lambda\coss{\phi}+\sqrt{\sin^2(\chi)+\lambda^2\sin^2(\phi)}.

\barr{EK}^2
= & ~~ \barr{EH}^2+\barr{HB}^2+\barr{BG}^2  + \barr{GC}^2 +  \notag \\
&
 2\left(
\vect{EH}\cdot\vect{HB}+
\vect{EH}\cdot\vect{BG}+
\vect{EH}\cdot\vect{GC}+
\vect{HB}\cdot\vect{BG}+
\vect{HB}\cdot\vect{GC}+
\vect{BG}\cdot\vect{GC}
\right), 
\label{equa: inner products piecewise line algo}

\vect{EH}\cdot\vect{GK} &= 
\barr{EH} ~\barr{GK} \coss{\pi}
=- \barr{EH} ~\barr{GK},
\\
\vect{HB}\cdot\vect{GK} &=
\barr{HB} ~\barr{GK} \coss{\frac\pi2-\theta}
=\barr{HB} ~\barr{GK} \sinn{\theta}, \ \ \mbox{ and}
\\
\vect{BG}\cdot\vect{GK} &=
\barr{BG} ~\barr{GK} \coss{\phi}.

\barr{BC}^2
&=\barr{BG}^2+\barr{GC}^2 + 2\vect{BG}\cdot\vect{GC} \\
&= \lambda^2 + \left(\sinn{\chi} - \lambda \coss{\phi}\right)^2 + 2\lambda \left(\sinn{\chi} - \lambda \coss{\phi}\right) \coss{\phi} \\
& = \sin^2(\chi)+\lambda^2\sin^2(\phi),

1+\max
\left\{
\begin{array}{lr}
\chi + f(\chi) & \textit{(pre-detour phase)} \\
\max_{\chi-f(\chi) \leq y \leq \psi} \{ y + 2h(y)\}& \textit{(detour subphase-1)} \\
\max_{\psi \leq y \leq \chi} \{ y + 2h'(y)\} & \textit{(detour subphase-2)} \\
\chi + \lambda+\sinn{\chi}-\lambda\coss{\phi}+
&\sqrt{\sin^2(\chi)+\lambda^2\sin^2(\phi)} + 2p(\chi)\\
  &\textit{(post-detour phase)}
\end{array}
\right\},

h(y) &= 
\frac{
-0.5 y^2
+3.45042 y
+ a(y) + b(y) -6.61768
}{
y-0.900812 \sin (y)-0.434209 \cos (y)-3.45042
}
\\
h'(y)&=
\frac{
-0.5 y^2+3.12552 y
+(y-3.12552) \sin (y)
- 0.847858 \cos(y)
-5.74387
}{
y
-\sin (y)
-3.12552},
\max_{0.63584 \leq y \leq 0.755204} \{ y + 2h(y)\} < 4.627972\max_{0.755204 \leq y \leq 2.631865} \{ y + 2h'(y)\} < 4.627961.
This completes the proof of the theorem. 
\end{proof}


\section{Lower Bound}
\label{sec:Lower Bound}

In this section we show that any evacuation algorithm for
two robots in the \ff model takes time at least
. We first prove a result of independent interest about an evacuation problem on a hexagon. 
\begin{theorem}\label{thm:hexagon}
Consider a hexagon of radius  with an exit placed at an unknown vertex.
Then the worst case evacuation time for two robots starting at any two  arbitrary vertices of the hexagon 
is at least .
\end{theorem}
\begin{proof}
Assume an arbitrary deterministic algorithm  for the problem. 
 solves the problem for any input, i.e., any placement of the exit.
We construct two inputs for  and show that for at least one of them, the required evacuation time is at least .
First, we let  run without placing an exit at any vertex, so as to 
find out in which order the robots are exploring all the vertices
of the hexagon. We label the vertices of the hexagon according to this order (if two vertices are explored simultaneously then we just order them arbitrarily).
Let  be the time when the fifth vertex, , of the hexagon is visited by some robot, say , i.e., robot  is at vertex  at time , and four more vertices of the hexagon have already been visited. In other words,  and  are the only vertices\footnote{It might be that  and  are explored simultaneously, or that  and  are explored simultaneously. In the former case  is explored strictly after  while in the latter  is explored strictly before .} 
that are guaranteed to not have been explored at time , for any sufficiently small .
Note that we must have , since at least one of the two robots must have visited at least three vertices by time  (and hence must have walked at least the 
two segments between the first and the second, and between the second and the third vertex visited in its trajectory).
	
The first input  we construct has the exit placed at vertex . 
Until robot  reaches , the algorithm  
processes  identically to the case in which there is no exit; further, at time , robot  needs additional time at least  just to reach the exit.
If , then this input gives an evacuation time of at least . 

\begin{figure}[ht]
                \centering
                \includegraphics[scale=0.7]{FigsEvac/hex1.pdf}
                \caption{Vertices of the hexagon as visited by algorithm ;  is the time when the fifth vertex  is visited by some robot, say . One of the vertices adjacent to  has not been visited yet by a robot (here ).}
                \label{fig: hex}
\end{figure}
Hence assume that . Let , , and  be the three vertices that are non-adjacent to  in the hexagon (see Figure~\ref{fig: hex}).
Note that the minimum distance between  and any of , , and  is at least .
If  then on input ,  needs evacuation time at least , as  still has to 
reach the exit. 

Therefore, assume that  and hence 	.
Note that, since , on input , robot
 has visited at most one of , , and  at time . 
Hence, the other robot,  has visited at least two of them.
For the second input  that we construct, we place the exit on the vertex  that is the last vertex among , , and  in the visiting order of the vertices by .
Let  be the time when  reaches , and note that at least until , the algorithm  behaves identical on the two inputs  and .
As  has visited at least one vertex before visiting , we have .

Next we claim that  and  cannot meet between time  and .
The claim below states that this is impossible. Namely, we can prove:
\begin{claim}\label{claim:hex}
Let , , and  be as defined in the preceding paragraphs. 
If , then on input ,  and  do not meet between time  and time .
\end{claim}
\begin{proof} (Claim~\ref{claim:hex})
Assume on the contrary, that
on input ,  and  do meet at some time  at point , with
.
Observe that on input , robot  continues until time  as on input  but having met  at time  might continue differently after time~.
Let  be the time that  uses on input  to get from the exit  to ,  
and let  be the time that  uses on input  to get to vertex  from .  As  and~ are at distance at least , and since , we have
.  So we obtain , which contradicts the assumption that .
This proves the claim. 
\end{proof}


Having proved the claim, we conclude that on input , 
 continues until time  as on input . Hence  needs at least  time to reach the exit on input ,
which completes the proof of the theorem.
\end{proof}

It is worth noting that the lower bound from Theorem~\ref{thm:hexagon} matches the upper bound of evacuating a regular hexagon, when the initial starting vertices may be chosen by the algorithm. 
Consider a hexagon  and suppose that the trajectory of one robot, as long as no exit was found, is . 
Similarly, the other robot follows the symmetric trajectory ; cf.\ left-hand side of \figurename~\ref{fig:hex_ub}. 
By symmetry it is sufficient to consider exits at vertices , , or . 
An exit at  is reached by each robot independently, while both robots proceed to an exit at  or  after meeting at point , the intersection of the segments  and . 
Altogether, they need a total time of at most  to evacuate from the hexagon. 
An interested reader may verify that, in each case, the evacuation time of this algorithm is always upper bounded by .


\begin{figure}[!ht]
                \centering
 \vspace{-2ex}
 \includegraphics[scale=0.7,page=3]{FigsEvac/hex1.pdf} \hspace{1cm}
                \includegraphics[scale=0.7,page=2]{FigsEvac/hex1.pdf}
				\caption{The trajectories for \RA~(red) and \RB~(blue) for the hexagon evacuation algorithm having evacuation time , 					while the exit has not been found, are depicted on the left.
					Right-hand side: At time , there is regular hexagon all of whose vertices are unexplored and lie on the boundary of the disk.
				}
				\label{fig:hex_ub}
\end{figure}

In the above algorithm, the robots meet at M, regardless of  whether the exit has been already found or not. 
The idea of our algorithm for disk evacuation presented in the previous section was influenced by this non-intuitive presence of a forced meeting. \\
Combining Theorem~\ref{thm:hexagon} with the fact that hexagon edges correspond to arcs of length , we obtain the following lower bound for our evacuation problem.

\begin{theorem}
Assume you have a unit disk with an exit placed somewhere on the boundary. 
The worst case evacuation time  for two robots starting at the centre in the \ff model 
  is at least . \end{theorem}
\begin{proof}
It takes 1 time unit for the robots to reach the boundary of the disk.
By time ,
any algorithm could have explored at most  of the boundary of the disk. 
Hence for any  with , there exists a regular hexagon with all vertices on the boundary of the disk and all of whose vertices are unexplored at time ; see the right-hand side of Figure~\ref{fig:hex_ub}. 
As the exit might as well be on any vertex of this hexagon, invoking Theorem~\ref{thm:hexagon} implies a lower bound of  for the worst case of evacuating both robots. 
\end{proof}

Our lower bound technique could be extended to , and it might 
improve the lower bound value. However, the case analysis becomes 
extensive,
very complicated, and it would not provide any more insight into the 
lower bound question.



\section{Conclusion}
\label{sec:Conclusion}

In this paper we studied evacuating two robots from a disk, where the robots 
can collaborate using \ff communication.
Unlike evacuation for two robots in the wireless communication
model, for which the tight bound 
is proved in~\cite{CGGKMP}, the evacuation problem for two robots in
the \ff model is much harder to solve.
We gave a new non-trivial algorithm for the  \ff communication
model, by this improving the upper bound of 5.740
in~\cite{CGGKMP} to 
5.628
(the upper bound has since been improved 
to 5.625 in~\cite{Watten2017}
and to 5.6234 in~\cite{disser2019evacuating}
).
We used a novel, non-intuitive idea of a forced meeting between the robots, 
regardless of whether the exit was found before the meeting (although~\cite{Watten2017,disser2019evacuating} have since demonstrated that by avoiding a forced meeting one can improve the upper bound). 
We also provided a different analysis that improves the lower bound in~\cite{CGGKMP}.
Further tightening of the upper and lower bounds remains a challenging open question. Along the same lines, results pertaining to the generalization of our auxiliary problem of evacuating from an  is another interesting open problem. 


\acknowledgements
\label{sec:ack}
This work was initiated during the \emph{ Workshop on Routing} held in July 15--19,
2014 in Quer\'{e}taro, M\'{e}xico.
The authors would also like to thank the anonymous reviewers who, during the journal evaluation process, had a number of suggestions that improved the manuscript. 

\nocite{*}
\bibliographystyle{abbrvnat}
\bibliography{refs-August2019}
\label{sec:biblio}

\end{document}
