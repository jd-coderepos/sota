\subsubsection{Types as trees}
\label{sec:typingCoinductive}


Type safety, addressed in the Sec.\ref{sec:safety}, also relies on $\subtypemu$
enjoying the fundamental property of \emph{invertibility} of non-union types
(\cf Prop.~\ref{prop:subtypingIsInvertible}):
\begin{enumerate}
  \item If $\datatype{D}{A} \subtypemu \datatype{D'}{A'}$, then $D \subtypemu
  D'$ and $A \subtypemu A'$.
  \item If $\functype{A}{B} \subtypemu \functype{A'}{B'}$, then $A' \subtypemu
  A$ and $B \subtypemu B'$.
\end{enumerate}


To prove this we appeal to the standard tree interpretation of terms and
formulate an equivalent coinductive definition of equivalence and subtyping
($\subtypeco$). For the latter, invertibility of non-union types is proved
coinductively, (Lem.~\ref{lem:subtypingIsInvertible}), entailing Prop.~\ref{prop:subtypingIsInvertible}.

Consider \emphdef{type constructors} $\idatatype$ and $\ifunctype$ together
with \emphdef{type connector} $\iuniontype$ and the ranked alphabet
$\mathfrak{L} \eqdef \set{a^0 \mathrel| a \in \TypeVariable \cup \TypeConstant}
\cup \set{\idatatype^2, \ifunctype^2, \iuniontype^2}$.  We write $\Tree$ for
the set of (possibly) \emphdef{infinite types} with symbols in $\mathfrak{L}$.
This is a standard construction~\cite{terese03,journals/tcs/Courcelle83} given
by the metric completion based on a simple depth function measuring the
distance from the root to the minimum conflicting node in two trees.
Perhaps worth mentioning is that the type connector $\iuniontype$ does not
contribute to the depth (hence the reason for calling it a connector rather
than a constructor) excluding types consisting of infinite branches of
$\iuniontype$, such as $\uniontype{(\uniontype{\ldots}{\ldots})}{(\uniontype{\ldots}{\ldots})}$,
from $\Tree$. We use meta-variables $\tA, \tB, \ldots$ to denote elements of
$\Tree$.

\begin{remark}
\label{rem:maximalUnionTypes}
For any $\star \in \mathfrak{L}$, we write $\tA \neq \star$ to mean that
$\tA(\epsilon) \neq \star$, $\epsilon$ being the root position of the tree. For
example, $A \neq \iuniontype$ means that $A$ is a \emph{non-union type}. Any
type $\tA$ can be written as $\tA = \maxuniontype{i \in 1..n}{\tA_i}$ (dubbed a
\emph{maximal} union type) where $\tA_i \neq \iuniontype$ for all $i \in 1..n$
with $n\in \Natural$, irrespective of how their arguments are associated. All
such associations yield equivalent infinite types in a sense to be made precise
shortly.
\end{remark}




\subsubsection{Equivalence of Infinite Types}


\begin{figure}[t] $$
\begin{array}{c}
\RuleCo{}
       {a \eqtypeco a}
       {\ruleEqcoRefl}
\\
\\
\RuleCo{\tA \eqtypeco \tA' \quad \tB \eqtypeco \tB'}
       {\functype{\tA}{\tB} \eqtypeco \functype{\tA'}{\tB'}}
       {\ruleEqcoFunc}
\qquad
\RuleCo{\tD \eqtypeco \tD' \quad \tA \eqtypeco \tA'}
       {\datatype{\tD}{\tA} \eqtypeco \datatype{\tD'}{\tA'}}
       {\ruleEqcoComp}
\\
\\
\RuleCo{\begin{array}{ll}
          \tA_i \eqtypeco \tB_{f(i)} & \quad f : 1..n \to 1..m \\
          \tA_{g(j)} \eqtypeco \tB_j & \quad g : 1..m \to 1..n
        \end{array}
        \quad
        \tA_i, \tB_j \neq \iuniontype
        \quad
        n + m > 2}
       {\maxuniontype{i \in 1..n}{\tA_i} \eqtypeco \maxuniontype{j \in 1..m}{\tB_j}}
       {\ruleEqcoUnion}
\end{array} $$
\caption{Equivalence relation for infinite types}
\label{fig:equivalenceSchemesCo}
\end{figure}

\begin{definition}
Infinite type equivalence, written $\mathbin{\eqtypeco}$, is defined by the
coinductive interpretation of the schemes of
Fig.~\ref{fig:equivalenceSchemesCo}.
\end{definition}

Note that $\ruleEqcoUnion$ is actually a rule scheme, representing all
possible associations within maximal union types $\tA =
\maxuniontype{i \in 1..n}{\tA_i}$ and $\tB = \maxuniontype{j \in 1..m}{\tB_j}$.
Each instance of the rule states that every $A_i$ must be equivalent to some
$\tB_j$ via a function $f : 1..n \to 1..m$ and vice versa (with $g: 1..m \to
1..n$). Note that the  type connector $\iuniontype$ is seen to be not only
associative and commutative but also idempotent.

Formally, let $\Phieqtypeco : \Parts{\Tree \times \Tree} \to \Parts{\Tree
\times \Tree}$ be the functional associated to the rules in
Fig.~\ref{fig:equivalenceSchemesCo}, defined as follows: $$
\begin{array}{rcl}
\Phieqtypeco(\S) & =    & \set{\pair{a}{a} \mathrel| a \in \TypeVariable \cup \TypeConstant} \\
                 & \cup & \set{\pair{\datatype{\tD}{\tA}}{\datatype{\tD'}{\tA'}} \mathrel| \pair{\tD}{\tD'}, \pair{\tA}{\tA'} \in \S} \\
                 & \cup & \set{\pair{\functype{\tA}{\tB}}{\functype{\tA'}{\tB'}} \mathrel| \pair{\tA}{\tA'}, \pair{\tB}{\tB'} \in \S} \\
                 & \cup & \{\pair{\maxuniontype{i \in 1..n}{\tA_i}}{\maxuniontype{j \in 1..m}{\tB_j}} \mathrel| \tA_i, \tB_j \neq \iuniontype, n + m > 2 \\
                 &      & \qquad \exists f : 1..n \to 1..m \text{ s.t. } \pair{\tA_i}{\tB_{f(i)}} \in \S, \\
                 &      & \qquad \exists g : 1..m \to 1..n \text{ s.t. } \pair{\tA_{g(j)}}{\tB_j} \in \S\}
\end{array} $$

Then $\mathbin{\eqtypeco} \eqdef \nu\Phieqtypeco$. Now we show that it is
indeed an equivalence relation.



\begin{lemma}
$\eqtypeco$ is an equivalence relation (\ie reflexive, symmetric and
transitive).
\end{lemma}

\begin{proof}
The three properties are proved be showing that the sets defining them are
$\Phieqtypeco$-dense. Then we conclude by the coinductive
principle\footnote{\emph{Coinductive principle}: if $\X$ is $\Phi$-dense, then
$\X \subseteq \nu\Phi$.} that the properties hold on $\eqtypeco$.
\begin{itemize}
  \item Reflexivity: $\Refl\ \eqdef \set{\pair{\tA}{\tA} \mathrel| \tA \in
  \Tree}$. Let $\pair{\tA}{\tA} \in \Refl$. We proceed by analyzing the shape
  of $\tA$:
  \begin{itemize}
    \item $\tA = a$. Immediate since $\pair{a}{a} \in \Phieqtypeco(\Refl)$ for
    every $a \in \TypeVariable \cup \TypeConstant$.
    
    \item $\tA = \datatype{\tD}{\tA'}$. By definition of reflexivity
    $\pair{\tD}{\tD}, \pair{\tA'}{\tA'} \in \Refl$. Then $\pair{A}{A} \in
    \Phieqtypeco\!\left(\Refl\right)$.
    
    \item $\tA = \functype{\tA'}{\tA''}$. Similarly to the previous case, we
    have $\pair{\tA'}{\tA'}, \pair{\tA''}{\tA''} \in \Refl$. Hence
    $\pair{\tA}{\tA} \in \Phieqtypeco\!\left(\Refl\right)$.
    
    \item $\tA = \maxuniontype{i \in 1..n}{\tA_i}$ with $\tA_i \neq \iuniontype$
    for $i \in i..n, n > 1$. Then, since $\pair{\tA_i}{\tA_i} \in \Refl$ and
    $n + n > 2$, we conclude
    $\pair{\maxuniontype{i \in 1..n}{\tA_i}}{\maxuniontype{i \in 1..n}{\tA_i}}
    \in \Phieqtypeco\!\left(\Refl\right)$ by considering $f = g = \mathit{id}$
    (the identity function).
  \end{itemize}
  
  
  \item Symmetry: $\Symm\!\left(\S\right) \eqdef \set{\pair{\tB}{\tA} \mathrel|
  \pair{\tA}{\tB} \in \S}$. We show that $\Symm\!\left(\eqtypeco\right)
  \subseteq \mathbin{\eqtypeco}$.
  
  Let $\pair{\tA}{\tB} \in \Symm\!\left(\eqtypeco\right)$, then
  $\pair{\tB}{\tA} \in \mathbin{\eqtypeco} =
  \Phieqtypeco\!\left(\eqtypeco\right)$. By Rem.~\ref{rem:maximalUnionTypes} we
  can consider maximal union types $$
\begin{array}{r@{\quad\text{with}\quad}l}
\tA = \maxuniontype{i \in 1..n}{\tA_i} & \tA_i \neq \iuniontype, i \in 1..n \\
\tB = \maxuniontype{j \in 1..m}{\tB_j} & \tB_j \neq \iuniontype, j \in 1..m
\end{array} $$ and we have two separate cases to analyze:
  \begin{enumerate}
    \item If $n = m = 1$, then both $\tA$ and $\tB$ are non-union types. Now we
    proceed by analyzing the shape of $\tB$:
    \begin{itemize}
      \item $\tB = a$. Then $\tA = a$ by definition of $\Phieqtypeco$ and the
      result is immediate since $\pair{a}{a} \in
      \Phieqtypeco\!\left(\Symm(\eqtypeco)\right)$ for every $a \in
      \TypeVariable \cup \TypeConstant$.
      
      \item $\tB = \datatype{\tD'}{\tB'}$. Again, by definition, we have
      $\tA = \datatype{\tD}{\tA'}$ with $\pair{\tD'}{\tD}, \pair{\tB'}{\tA'}
      \in \mathbin{\eqtypeco}$. Then $\pair{\tD}{\tD'}, \pair{\tA'}{\tB'} \in
      \Symm\!\left(\eqtypeco\right)$ and we conclude $\pair{\tA}{\tB} \in
      \Phieqtypeco\!\left(\Symm(\eqtypeco)\right)$.
      
      \item $\tB = \functype{\tB'}{\tB''}$. Similarly, $\tA =
      \functype{\tA'}{\tA''}$ with $ \pair{\tB'}{\tA'}, \pair{\tB''}{\tA''} \in
      \mathbin{\eqtypeco}$. Hence $\pair{\tA'}{\tB'}, \pair{\tA''}{\tB''} \in
      \Symm\!\left(\eqtypeco\right)$ and we conclude $\pair{\tA}{\tB} \in
      \Phieqtypeco\!\left(\Symm(\eqtypeco)\right)$.
    \end{itemize}
    
    \item If not, we have $n + m > 2$ and only the rule $\ruleEqcoUnion$
    applies. Then $$
\begin{array}{r@{\quad\text{s.t.}\quad}c@{\quad\text{for every}\quad}l}
\exists g : 1..m \to 1..n & \pair{\tB_j}{\tA_{g(j)}} \in \mathbin{\eqtypeco} & j \in 1..m \\
\exists f : 1..n \to 1..m & \pair{\tB_{f(i)}}{\tA_i} \in \mathbin{\eqtypeco} & i \in 1..n
\end{array} $$ Applying symmetry we get $\pair{\tA_i}{\tB_{f(i)}},
    \pair{\tA_{g(j)}}{\tB_j} \in \Symm\!\left(\eqtypeco\right)$ for every
    $i \in 1..n, j \in 1..m$. Thus, we conclude $\pair{\tA}{\tB} \in
    \Phieqtypeco\!\left(\Symm(\eqtypeco)\right)$.
  \end{enumerate}
  
  
  \item Transitivity: $\Trans\!\left(\S\right) \eqdef \set{\pair{\tA}{\tB}
  \mathrel| \exists \tC \in \Tree. \pair{\tA}{\tC}, \pair{\tC}{\tB} \in \S}$.
  As before, we show that $\Trans\!\left(\eqtypeco\right) \subseteq
  \mathbin{\eqtypeco}$. Let $\pair{\tA}{\tB} \in
  \Trans\!\left(\eqtypeco\right)$, then there exists $\tC \in \Tree$ such that
  $\pair{\tA}{\tC}, \pair{\tC}{\tB} \in \mathbin{\eqtypeco} =
  \Phieqtypeco\!\left(\eqtypeco\right)$. Again, we resort to
  Rem.~\ref{rem:maximalUnionTypes} an consider maximal union types $$
\begin{array}{r@{\quad\text{with}\quad}l}
\tA = \maxuniontype{i \in 1..n}{\tA_i} & \tA_i \neq \iuniontype, i \in 1..n \\
\tB = \maxuniontype{j \in 1..m}{\tB_j} & \tB_j \neq \iuniontype, j \in 1..m \\
\tC = \maxuniontype{k \in 1..l}{\tC_k} & \tC_k \neq \iuniontype, k \in 1..l
\end{array} $$
  \begin{enumerate}
    \item If $n = m = l = 1$ (\ie all three are non-union types), we proceed by
    analyzing the shape of $\tC$:
  \begin{itemize}
	  \item $\tC = a$. By definition of $\Phieqtypeco$, $\tA = a$ and $\tB =
	  a$. Then $\pair{\tA}{\tB} = \pair{a}{a} \in
	  \Phieqtypeco\!\left(\Trans(\eqtypeco)\right)$.
	  
	  \item $\tC = \datatype{\tD''}{\tC'}$. Once again by definition of
	  $\Phieqtypeco$, $\tA = \datatype{\tD}{\tA'}$ with $\pair{\tD}{\tD''},
	  \pair{\tA'}{\tC'} \in \mathbin{\eqtypeco}$ and $\tB =
	  \datatype{\tD'}{\tB'}$ with $\pair{\tD''}{\tD'}, \pair{\tC'}{\tB'} \in
	  \mathbin{\eqtypeco}$. Then $\pair{\tD}{\tD'}, \pair{\tA'}{\tB'} \in
	  \Trans\!\left(\subtypeco\right)$ and we conclude $\pair{\tA}{\tB} \in
	  \Phieqtypeco\!\left(\Trans(\eqtypeco)\right)$.
	  
	  \item $\tC = \functype{\tC'}{\tC''}$. Similarly, we have $\tA =
	  \functype{\tA'}{\tA''}$ and $\tB = \functype{\tB'}{\tB''}$ with
	  $\pair{\tA'}{\tC'}, \pair{\tA''}{\tC''}, \pair{\tC'}{\tB'},
	  \pair{\tC''}{\tB''} \in \mathbin{\eqtypeco}$. By transitivity
	  $\pair{\tA'}{\tB'}, \pair{\tA''}{\tB''} \in
	  \Trans\!\left(\eqtypeco\right)$ and $\pair{\tA}{\tB} \in
	  \Phieqtypeco\!\left(\Trans(\eqtypeco)\right)$.
  \end{itemize}
  
	\item If not (\ie $n + m + l > 3$), we have three different situations to
	consider:
    \begin{inparaenum}[(i)]
      \item $n + l > 2$ and $m + l > 2$;
      \item $n > 1$ and $m = l = 1$; or
      \item $m > 1$ and $n = l = 1$.
    \end{inparaenum}
    In terms of applied rules to derive $\tA \eqtypeco \tC$ and $\tC \eqtypeco
    \tB$, in the former case the only possibility is $\ruleEqcoUnion$ on both
    sides, while in the latter two we have $\ruleEqcoUnion$ on one side and any
    of the other three rules ($\ruleEqcoRefl$, $\ruleEqcoComp$,
    $\ruleEqcoFunc$) on the other. Note that this last two cases are symmetric,
    therefore we only analyse cases (i) and (ii) below:
    \begin{enumerate}[(i)]
      \item $n + l > 2$ and $m + l > 2$. By definition of $\Phieqtypeco$ $$
\begin{array}{r@{\quad\text{s.t.}\quad}c@{\quad\text{for every}\quad}l}
\exists f : 1..n \to 1..l  & \pair{\tA_i}{\tC_{f(i)}}  \in \mathbin{\eqtypeco} & i \in 1..n \\
\exists g : 1..l \to 1..n  & \pair{\tA_{g(k)}}{\tC_k}  \in \mathbin{\eqtypeco} & k \in 1..l \\
\exists f' : 1..l \to 1..m & \pair{\tC_k}{\tB_{f'(k)}} \in \mathbin{\eqtypeco} & k \in 1..l \\
\exists g' : 1..m \to 1..l & \pair{\tC_{g'(j)}}{\tB_j} \in \mathbin{\eqtypeco} & j \in 1..m
\end{array} $$
      Then, we have $\pair{\tA_i}{\tC_{f(i)}},
      \pair{\tC_{f(i)}}{\tB_{f'(f(i))}} \in \mathbin{\eqtypeco}$ for every $i
      \in 1..n$, and $\pair{\tA_{g(g'(j))}}{\tC_{g'(j)}},
      \pair{\tC_{g'(j)}}{\tB_j} \in \mathbin{\eqtypeco}$ for every $j \in
      1..m$.
      
      Here we have two possible situations. If $n = m = 1$ (hence $l > 1$) it
      is necessarily the case $\pair{\tA}{\tC_{f(1)}}, \pair{\tC_{f(1)}}{\tB}
      \in \mathbin{\eqtypeco}$ with all three non-union types. Then we can
      safely conclude, by the previous analysis made in case 1, that
      $\pair{\tA}{\tB} \in \Phieqtypeco\!\left(\Trans(\eqtypeco)\right)$.
      
      If not (\ie $n + m > 2$), taking $f'' = f' \circ f : 1..n \to 1..m$ we
      get $\pair{\tA_i}{\tB_{f''(i)}} \in \Trans\!\left(\eqtypeco\right)$.
      Similarly, $\pair{\tA_{g''(j)}}{\tB_j} \in
      \Trans\!\left(\eqtypeco\right)$ for every $j \in 1..m$ with $g'' = g
      \circ g' : 1..m \to 1..n$. Finally we conclude by $\ruleEqcoUnion$,
      $\pair{\tA}{\tB} \in \Phieqtypeco\!\left(\Trans(\eqtypeco)\right)$.

      \item $n > 1$ and $m = l = 1$. Then, by definition of $\Phieqtypeco$,
      $f : 1..n \to 1$ is a constant function and we have $\pair{\tA_i}{\tC}
      \in \mathbin{\eqtypeco}$ for every $i \in 1..n$. On the other hand
      $\pair{\tC}{\tB} \in \mathbin{\eqtypeco}$ by hypothesis. By transitivity
      once again we get $\pair{\tA_i}{\tB} \in \Trans\!\left(\eqtypeco\right)$
      and we conclude with the same constant function $f$, $\pair{\tA}{\tB} \in
      \Phieqtypeco\!\left(\Trans(\eqtypeco)\right)$.
    \end{enumerate}
  \end{enumerate}
\end{itemize}
\end{proof}



\begin{lemma}[Equality of non-union types is invertible]
\label{lem:equalityIsInvertible}
Let $\tA \eqtypeco \tB$ be two non-union types.
\begin{enumerate}
  \item If $\tA = a$, then $\tB = a$.
  \item If $\tA = \datatype{\tD}{\tA'}$, then $\tB = \datatype{\tD'}{\tB'}$
  with $\tD \eqtypeco \tD'$ and $\tA' \eqtypeco \tB'$.
  \item If $\tA = \functype{\tA'}{\tA''}$, then $\tB = \functype{\tB'}{\tB''}$
  with $\tA' \eqtypeco \tB'$ and $\tA'' \eqtypeco \tB''$.
\end{enumerate}
\end{lemma}

\begin{proof}
Immediate from the definition of subtyping. Note that there's only one
applicable rule in each case.
\end{proof}


Along the document we often resort to the following definition and properties
of the substitution operator over infinite trees:

\begin{definition}
\label{def:treeSubstitution}
The \emphdef{substitution} of a variable $V$ by a tree $\tB$ in $\tA$ (notation
$\substitute{V}{\tB}{\tA}$) is defined as: $$
\begin{array}{r@{\quad \eqdef\quad}l@{\quad}l}
(\substitute{V}{\tB}{\tA})(\pi)     & \tA(\pi)  & \text{if $\tA(\pi)$ defined and $\tA(\pi) \neq V$} \\
(\substitute{V}{\tB}{\tA})(\pi\pi') & \tB(\pi') & \text{if $\tA(\pi)$ defined and $\tA(\pi) = V$}
\end{array} $$
\end{definition}



The following lemma provides a more convenient characterisation of the
substitution.

\begin{lemma}
\label{lem:treeSubstitution}
\begin{enumerate}[(i)]
  \item $\substitute{V}{\tB}{V} = \tB$.
  \item $\substitute{V}{\tB}{a} = a$ for $V \neq a \in \TypeVariable \cup
  \TypeConstant$.
  \item $\substitute{V}{\tB}{(\tA_1 \star \tA_2)} = \substitute{V}{\tB}{\tA_1}
  \star \substitute{V}{\tB}{\tA_2}$ for $\star \in \set{\idatatype, \ifunctype,
  \iuniontype}$.
\end{enumerate}
\end{lemma}

\begin{proof}
The three cases are by analysis of the defined positions.
\begin{enumerate}[(i)]
  \item The only defined position in $V$ is $\epsilon$. Then, for every $\pi$
  in $\tB$ we have $$(\substitute{V}{\tB}{V})(\pi) =
  (\substitute{V}{\tB}{V})(\epsilon\pi) = B(\pi)$$
  
  \item The only defined position in $a \neq V$ is $\epsilon$, thus we have
  $(\substitute{V}{\tB}{a})(\epsilon) = a(\epsilon) = a$. Any other position is
  undefined.
  
  \item Here we have $\tA = \tA_1 \star \tA_2$ with $\star \in \set{\idatatype,
  \ifunctype, \iuniontype}$. We proceed by analysing the defined positions of
  $\tA$.
  \begin{itemize}
    \item $\pi = \epsilon$. Then
    $$(\substitute{V}{\tB}{(\tA_1 \star \tA_2)})(\epsilon) =
    (\tA_1 \star \tA_2)(\epsilon) = \star = (\substitute{V}{\tB}{\tA_1} \star
    \substitute{V}{\tB}{\tA_2})(\epsilon)$$
    \item $\pi = i\pi'$. Here we have two possibilities:
    \begin{enumerate}
      \item either $\tA(\pi) \neq V$. Then $\tA_i(\pi') \neq V$ and we have $$
\begin{array}{r@{\quad=\quad}l@{\quad}l}
(\substitute{V}{\tB}{(\tA_1 \star \tA_2)})(\pi) & (\tA_1 \star \tA_2)(i\pi')         & \text{by Def.~\ref{def:treeSubstitution}} \\
                                                & \tA_i(\pi') \\
                                                & (\substitute{V}{\tB}{\tA_i})(\pi') & \text{by Def.~\ref{def:treeSubstitution}} \\
                                                & (\substitute{V}{\tB}{\tA_1} \star \substitute{V}{\tB}{\tA_2})(\pi)
\end{array} $$
      \item or $\tA(\pi) = V$. Then $\tA_i(\pi') = V$ and by definition of
      substitution we have, for every position $\pi''$ in $\tB$ $$
\begin{array}{r@{\quad=\quad}l}
(\substitute{V}{\tB}{(\tA_1 \star \tA_2)})(\pi\pi'') & \tB(\pi'') \\
                                                     & (\substitute{V}{\tB}{\tA_i})(\pi'\pi'') \\
                                                     & (\substitute{V}{\tB}{\tA_1} \star \substitute{V}{\tB}{\tA_2})(\pi\pi'')
\end{array} $$
    \end{enumerate}
  \end{itemize}
\end{enumerate}
\end{proof}



We show next that the substitution preserves the equivalent relation.

\begin{lemma}
\label{lem:substitutionOfEqtypesCo}
Let $\tA \eqtypeco \tA'$ and $\tB \eqtypeco \tB'$. Then
$\substitute{V}{\tB}{\tA} \eqtypeco \substitute{V}{\tB'}{\tA'}$.
\end{lemma}

\begin{proof}
Let $\S = \set{\pair{\substitute{V}{\tB}{\tA}}{\substitute{V}{\tB'}{\tA'}}
\mathrel| \tA \eqtypeco \tA', \tB \eqtypeco \tB'}$. We show that $\S \cup
\mathbin{\eqtypeco}$ is $\Phieqtypeco$-dense.

Let $\pair{\tC}{\tC'} \in \S \cup \mathbin{\eqtypeco}$. If $\pair{\tC}{\tC'}
\in \mathbin{\eqtypeco}$ the result is immediate by monotonicity of
$\Phieqtypeco$, since $\mathbin{\eqtypeco} =
\Phieqtypeco\!\left(\eqtypeco\right) \subseteq \Phieqtypeco\!\left(\S \cup
\mathbin{\eqtypeco}\right)$. Then we only present the case where
$\pair{\tC}{\tC'} \in \S$, $\tC = \substitute{V}{\tB}{\tA}$ and $\tC' =
\substitute{V}{\tB'}{\tA'}$ with $\tA \eqtypeco \tA'$ and $\tB \eqtypeco \tB'$.
Assume, without loss of generality $$
\begin{array}{r@{\quad\text{with}\quad}l}
\tA  = \maxuniontype{i \in 1..n}{\tA_i}  & \tA_i  \neq \iuniontype, i \in 1..n \\
\tA' = \maxuniontype{j \in 1..m}{\tA'_j} & \tA'_j \neq \iuniontype, j \in 1..m
\end{array} $$

\begin{enumerate}
  \item If $n = m = 1$ (\ie $\tA, \tA' \neq \iuniontype$), we analyze the shape
  of $\tA$:
  \begin{itemize}
    \item $\tA = a$. By Lem.~\ref{lem:equalityIsInvertible}, $\tA' = a$ and we
    have two possible cases. If $a \neq V$, by Lem.~\ref{lem:treeSubstitution}
    (ii), $\tC = a = \tC'$. If not, by Lem.~\ref{lem:treeSubstitution} (i),
    $\tC = \tB \eqtypeco \tB' = \tC'$. Both cases are immediate by definition
    of $\mathbin{\eqtypeco} \subseteq \Phieqtypeco\!\left(\S \cup
    \mathbin{\eqtypeco}\right)$.
    
    \item $\tA = \datatype{\tD}{\tA_1}$. By
    Lem.~\ref{lem:equalityIsInvertible}, $\tA' = \datatype{\tD'}{\tA'_1}$ with
    $\tD \eqtypeco \tD'$ and $\tA_1 \eqtypeco \tA'_1$. Then, by definition of
    $\S$, we have $\pair{\substitute{V}{\tB}{\tD}}{\substitute{V}{\tB'}{\tD'}}$
    and $\pair{\substitute{V}{\tB}{\tA_1}}{\substitute{V}{\tB'}{\tA'_1}} \in \S
    \cup \mathbin{\eqtypeco}$. Finally we conclude $\pair{\tC}{\tC'} \in
    \Phieqtypeco\!\left(\S \cup \mathbin{\eqtypeco}\right)$ since, by
    Lem.~\ref{lem:treeSubstitution} (iii), $$
\begin{array}{r@{\quad=\quad}c@{\quad=\quad}l}
\tC  & \substitute{V}{\tB}{(\datatype{\tD}{\tA_1})}    & \datatype{\substitute{V}{\tB}{\tD}}{\substitute{V}{\tB}{\tA_1}} \\
\tC' & \substitute{V}{\tB'}{(\datatype{\tD'}{\tA'_1})} & \datatype{\substitute{V}{\tB'}{\tD'}}{\substitute{V}{\tB'}{\tA'_1}}
\end{array} $$

    \item $\tA = \functype{\tA_1}{\tA_2}$. As before, by
    Lem.~\ref{lem:equalityIsInvertible}, we get $\tA =
    \functype{\tA'_1}{\tA'_2}$ with $\tA_1 \eqtypeco \tA'_1$ and $\tA_2
    \eqtypeco \tA'_2$. By definition $\S$ we have
    $\pair{\substitute{V}{\tB}{\tA_1}}{\substitute{V}{\tB'}{\tA'_1}}$ and
    $\pair{\substitute{V}{\tB}{\tA_2}}{\substitute{V}{\tB'}{\tA'_2}} \in \S
    \cup \mathbin{\eqtypeco}$. Thus, we conclude by
    Lem.~\ref{lem:treeSubstitution} (iii), $\pair{\tC}{\tC'} \in
    \Phieqtypeco\!\left(\S \cup \mathbin{\eqtypeco}\right)$.
  \end{itemize}
  
  \item If $n + m > 2$, by $\ruleEqcoUnion$ we have $$
\begin{array}{r@{\quad\text{s.t.}\quad}c@{\quad\text{for every}\quad}l}
\exists f : 1..n \to 1..m & \tA_i \eqtypeco \tA'_{f(i)} & i \in 1..n \\
\exists g : 1..m \to 1..n & \tA_{g(j)} \eqtypeco \tA'_j & j \in 1..m
\end{array} $$ Then,
  $\pair{\substitute{V}{\tB}{\tA_i}}{\substitute{V}{\tB'}{\tA'_{f(i)}}}$ and
  $\pair{\substitute{V}{\tB}{\tA_{g(j)}}}{\substitute{V}{\tB'}{\tA'_j}} \in \S
  \cup \mathbin{\eqtypeco}$ for every $i \in 1..n, j \in 1..m$. Once again we
  conclude by definition of $\Phieqtypeco$ and Lem.~\ref{lem:treeSubstitution}
  (iii), $\pair{\tC}{\tC'} \in
  \Phieqtypeco\!\left(\S \cup \mathbin{\eqtypeco}\right)$.
\end{enumerate}
\end{proof}


\subsubsection{Subtyping of trees}


In a similar way we have a coinductive characterization of subtyping over trees.

\begin{definition}
Infinite type subtyping, written $\subtypeco$, is defined by the coinductive
interpretation of the schemes in Fig.~\ref{fig:subtypingSchemesCo}. 
\end{definition}

The most interesting rule in Fig.~\ref{fig:subtypingSchemesCo} is
$\ruleSubcoUnion$. Here, for a maximal union type of the form $\maxuniontype{i
\in 1..n}{\tA_i}$ to be a subtype of a maximal union type $\maxuniontype{j \in
1..m}{\tB_j}$, one of the two must have at least one occurrence of the union
type construct ($n + m > 2$) and there must be a function $f : 1..n \to 1..m$
such that $\tA_i \subtypeco \tB_{f(i)}$ for each $i \in 1..n$. 

\begin{figure}[t] $$
\begin{array}{c}
\RuleCo{}
       {a \subtypeco a}
       {\ruleSubcoRefl}
\\
\\
\RuleCo{\tA' \subtypeco \tA \quad \tB \subtypeco \tB'}
       {\functype{\tA}{\tB} \subtypeco \functype{\tA'}{\tB'}}
       {\ruleSubcoFunc}
\qquad
\RuleCo{\tD \subtypeco \tD' \quad \tA \subtypeco \tA'}
       {\datatype{\tD}{\tA} \subtypeco \datatype{\tD'}{\tA'}}
       {\ruleSubcoComp}
\\
\\
\RuleCo{\tA_i \subtypeco \tB_{f(i)}
        \quad
        f : 1..n \to 1..m
        \quad
        \tA_i, \tB_j\neq \iuniontype
        \quad
        n + m > 2}
       {\maxuniontype{i \in 1..n}{\tA_i} \subtypeco \maxuniontype{j \in 1..m}{\tB_j}}
       {\ruleSubcoUnion}
\end{array} $$
\caption{Subtyping relation for infinite types}
\label{fig:subtypingSchemesCo}
\end{figure}


\begin{remark}
The rules are derived from those of Fig.~\ref{fig:subtypingSchemesMu}. More
precisely, rules $\ruleSubmuUnionRL$, $\ruleSubmuUnionRR$ and
$\ruleSubmuUnionL$ of Fig.~\ref{fig:subtypingSchemesMu} and the observation
that $\ruleSubmuUnionRL$ and $\ruleSubmuUnionRR$ can always be permuted past
$\ruleSubmuUnionL$.
\end{remark}

As above, the formal definition of the subtyping relation is given by the
associated function $\Phisubtypeco : \Parts{\Tree \times \Tree} \to
\Parts{\Tree \times \Tree}$ defined next: $$
\begin{array}{rcl}
\Phisubtypeco(\S) & =    & \set{\pair{a}{a} \mathrel| a \in \TypeVariable \cup \TypeConstant} \\
                  & \cup & \set{\pair{\datatype{\tD}{\tA}}{\datatype{\tD'}{\tA'}} \mathrel| \pair{\tD}{\tD'}, \pair{\tA}{\tA'} \in \S} \\
                  & \cup & \set{\pair{\functype{\tA}{\tB}}{\functype{\tA'}{\tB'}} \mathrel| \pair{\tA'}{\tA}, \pair{\tB}{\tB'} \in \S} \\
                  & \cup & \{\pair{\maxuniontype{i \in 1..n}{\tA_i}}{\maxuniontype{j \in 1..m}{\tB_j}} \mathrel| \tA_i, \tB_j \neq \iuniontype, n + m > 2 \\
                  &      & \qquad \exists f : 1..n \to 1..m \text{ s.t. } \pair{\tA_i}{\tB_{f(i)}} \in \S\}
\end{array} $$

Then $\mathbin{\subtypeco} = \nu\Phisubtypeco$. We now address some properties
of subtyping.

\begin{lemma}[Subtyping is a preorder]
$\subtypeco$ is a preorder (\ie reflexive and transitive).
\end{lemma}

\begin{proof}
This proof is similar to the one presented before for $\eqtypeco$.
\end{proof}



The following notion of invertibility (Lem.~\ref{lem:subtypingIsInvertible}) is
the main result of the present Section and an essential property to prove
Subject Reduction (Prop.~\ref{prop:subjectReduction}) and Progress
(Prop.~\ref{prop:progress}) for the type system proposed in
Sec.~\ref{sec:typingSystem}.

\begin{lemma}[Subtyping of non-union types is invertible]
\label{lem:subtypingIsInvertible}
Let $\tA, \tB \in \Tree$ be non-union types. Suppose $\tA \subtypeco \tB$.
\begin{enumerate}
  \item If $\tA = a$, then $\tB = a$.
  \item If $\tA = \datatype{\tD}{\tA'}$, then $\tB = \datatype{\tD'}{\tB'}$
  with $\tD \subtypeco \tD'$ and $\tA' \subtypeco \tB'$.
  \item If $\tA = \functype{\tA'}{\tA''}$, then $\tB = \functype{\tB'}{\tB''}$
  with $\tB' \subtypeco \tA'$ and $\tA'' \subtypeco \tB''$.
\end{enumerate}
\end{lemma}

\begin{remark}
In each of the three items of Lem.~\ref{lem:subtypingIsInvertible} the roles of
$\tA$ and $\tB$ can be reversed. 
\end{remark}





\begin{lemma}
\label{lem:eqImpliesSub}
$\tA \eqtypeco \tB \implies \tA \subtypeco \tB$.
\end{lemma}

\begin{proof}
We show that $\mathbin{\eqtypeco} = \Phieqtypeco\!\left(\eqtypeco\right)$ is
$\Phisubtypeco$-dense. Let $\pair{\tA}{\tB} \in \mathbin{\eqtypeco}$. By
Rem.~\ref{rem:maximalUnionTypes} we can consider maximal union types $$
\begin{array}{r@{\quad\text{with}\quad}l}
\tA = \maxuniontype{i \in 1..n}{\tA_i} & \tA_i \neq \iuniontype, i \in 1..n \\
\tB = \maxuniontype{j \in 1..m}{\tB_j} & \tB_j \neq \iuniontype, j \in 1..m
\end{array} $$ and we have two separate cases to analyze:
\begin{enumerate}
  \item If $n = m = 1$, then both $\tA$ and $\tB$ are non-union types. Now we
  proceed by analyzing the shape of $\tA$:
  \begin{itemize}
    \item $\tA = a$. Then, by definition of $\Phieqtypeco$, $\tB = a$ and the
    result is immediate since $\pair{a}{a} \in
    \Phisubtypeco\!\left(\eqtypeco\right)$ for every $a \in \TypeVariable \cup
    \TypeConstant$.
    
    \item $\tA = \datatype{\tD}{\tA'}$. Again, by definition of $\Phieqtypeco$,
    we have $\tB = \datatype{\tD'}{\tB'}$ with $\pair{\tD}{\tD'},
    \pair{\tA'}{\tB'} \in \mathbin{\eqtypeco}$. Then we conclude by definition
    of $\Phisubtypeco$, $\pair{\datatype{\tD}{\tA'}}{\datatype{\tD'}{\tB'}} \in
    \Phisubtypeco\!\left(\eqtypeco\right)$.
    
    \item $\tA = \functype{\tA'}{\tA''}$. Similarly, $\tB =
    \functype{\tB'}{\tB''}$ with $ \pair{\tA'}{\tB'}, \pair{\tA''}{\tB''} \in
    \mathbin{\eqtypeco}$. By symmetry $\pair{\tB'}{\tA'} \in
    \mathbin{\eqtypeco}$ and we conclude $\pair{\tA}{\tB} \in
    \Phisubtypeco\!\left(\eqtypeco\right)$.
  \end{itemize}
  
  \item If not (\ie $n + m > 2$), rule $\ruleEqcoUnion$ applies. Then $$
\begin{array}{r@{\quad\text{s.t.}\quad}c@{\quad\text{for every}\quad}l}
\exists f : 1..n \to 1..m & \pair{\tA_i}{\tB_{f(i)}} \in \mathbin{\eqtypeco} & i \in 1..n \\
\exists g : 1..m \to 1..n & \pair{\tA_{g(j)}}{\tB_j} \in \mathbin{\eqtypeco} & j \in 1..m
\end{array} $$ Thus, we conclude with the same function $f$, $\pair{A}{B}
  \in \Phisubtypeco\!\left(\eqtypeco\right)$.
\end{enumerate}
\end{proof}



To prove the correspondence of the coinductive formulation with the inductive
approach, it is convenient to work with finite trees (types). Thus, we
introduce a characterisation of the equivalence and subtyping relations in
terms of finite truncations of infinite trees.

We denote with $\card{\iuniontype}{\tA}$ the maximal number of adjacent union
type nodes, starting from the root of $\tA$: $$
\card{\iuniontype}{A} \eqdef \left\{
\begin{array}{l@{\quad\text{if}\quad}l}
0                                                         & \tA \neq \iuniontype \\
1 + \card{\iuniontype}{\tA_1} + \card{\iuniontype}{\tA_2} & \tA = \tA_1 \iuniontype \tA_2
\end{array}\right.
$$ Recall that, by definition of $\Tree$, a type cannot consist of infinitely
many consecutive occurrences of $\iuniontype$. Thus, the previous inductive
definition is well-founded, as well as the following:

\begin{definition}
\label{def:treeCut}
The \emphdef{truncation} of a tree $\tA$ at depth $k \in \Natural$ (notation
$\cut{\tA}{k}$) is defined inductively\footnote{Using the lexicographical
extension of the standard order to $\pair{k}{\card{\iuniontype}{\tA}}$.} as
follows: $$
\begin{array}{r@{\quad\eqdef\quad}l@{\quad}l}
\cut{\tA}{0}                          & \bullet \\
\cut{a}{k+1}                          & a                                   & \text{for $a \in \TypeVariable \cup \TypeConstant$} \\
\cut{(\tA_1 \star \tA_2)}{k+1}        & \cut{\tA_1}{k} \star \cut{\tA_2}{k} & \text{for $\star \in \set{\idatatype, \ifunctype}$} \\
\cut{(\uniontype{\tA_1}{\tA_2})}{k+1} & \uniontype{\cut{\tA_1}{k+1}}{\cut{\tA_2}{k+1}}
\end{array} $$ where $\bullet \in \TypeConstant$ is a distinguished type
constant used to identify the nodes where the tree was truncated.
\end{definition}

\begin{remark}
\label{rem:cutMaximalUnionTypes}
Given a maximal union type $\maxuniontype{i \in 1..n}{\tA_i}$, immediately from
the definition we have $\cut{(\maxuniontype{i \in 1..n}{\tA_i})}{k+1} =
\maxuniontype{i \in 1..n}{(\cut{\tA_i}{k+1})}$.
\end{remark}



\begin{lemma}
\label{lem:cutEquivalenceCo}
$\forall k \in \Natural.\cut{\tA}{k} \eqtypeco \cut{\tB}{k}$ iff $\tA \eqtypeco \tB$.
\end{lemma}

\begin{proof}
$\Rightarrow)$ We show that $\S \eqdef \set{\pair{\tA}{\tB} \mathrel| \forall
k \in \Natural. \cut{\tA}{k} \eqtypeco \cut{\tB}{k}}$ is $\Phieqtypeco$-dense.
Let $\pair{\tA}{\tB} \in \S$. Then, for every $k \in \Natural$ we have
$\cut{\tA}{k} \eqtypeco \cut{\tB}{k}$. Consider maximal union types $$
\begin{array}{r@{\quad\text{with}\quad}l}
\tA = \maxuniontype{i \in 1..n}{\tA_i} & \tA_i \neq \iuniontype, i \in 1..n \\
\tB = \maxuniontype{j \in 1..m}{\tB_j} & \tB_j \neq \iuniontype, j \in 1..m
\end{array} $$
\begin{enumerate}
  \item If $n = m = 1$ (\ie $\tA, \tB \neq \iuniontype$), we proceed by
  analyzing the shape of $\tA$:
  \begin{itemize}
    \item $\tA = a$. Then, $\cut{\tA}{k} = a$ for every $k > 0$ and, by
    Lem.~\ref{lem:equalityIsInvertible}, $\cut{\tB}{k} = a$. Hence, $\tB = a$
    and we conclude directly from the definition of $\Phieqtypeco$,
    $\pair{a}{a} \in \Phieqtypeco\!\left(\S\right)$.
    
    \item $\tA = \datatype{\tD}{\tA'}$. Similarly, we have $\cut{\tA}{k} =
    \datatype{\cut{\tD}{k-1}}{\cut{\tA'}{k-1}}$ for every $k > 0$. By
    Lem.~\ref{lem:equalityIsInvertible} once again, we get $\cut{\tB}{k} =
    \datatype{\tD'_k}{\tB'_k}$ with $\cut{\tD}{k-1} \eqtypeco \tD'_k$ and
    $\cut{\tA'}{k-1} \eqtypeco \tB'_k$. Note that for every $k$ we have
    different subtrees $\tD'_k$ and $\tB'_k$ but, since
    Lem.~\ref{lem:equalityIsInvertible} refers to tree equality (not
    equivalence) when determining the shape of $\tB$, it is immediate to see
    from the definition of the truncation that $\tB = \datatype{\tD'}{\tB'}$
    with $\tD'_k = \cut{\tD'}{k-1}$ and $\tB'_k = \cut{\tB'}{k-1}$ for every
    $k > 0$. Hence, $\cut{\tD}{k-1} \eqtypeco \cut{\tD'}{k-1}$ and
    $\cut{\tA'}{k-1} \eqtypeco \cut{\tB'}{k-1}$ for every $k > 0$. Then, by
    definition of $\S$, $\pair{\tD}{\tD'}, \pair{\tA'}{\tB'} \in \S$ and we
    conclude $\pair{\datatype{\tD}{\tA'}}{\datatype{\tD'}{\tB'}} \in
    \Phieqtypeco\!\left(\S\right)$.
    
    \item $\tA = \functype{\tA'}{\tA''}$. Analysis for this case is similar to
    the previous one. From $\cut{\tA}{k} =
    \functype{\cut{\tA'}{k-1}}{\cut{\tA''}{k-1}}$ we get $\tB =
    \functype{\tB'}{\tB''}$ with $\cut{\tA'}{k-1} \eqtypeco \cut{\tB'}{k-1}$
    and $\cut{\tA''}{k-1} \eqtypeco \cut{\tB''}{k-1}$ for every $k > 0$. Then
    we have $\pair{\tA'}{\tB'}, \pair{\tA''}{\tB''} \in \S$ and conclude
    $\pair{\functype{\tA'}{\tA''}}{\functype{\tB'}{\tB''}} \in
    \Phieqtypeco\!\left(\S\right)$.
  \end{itemize}
  
  \item If $n + m > 2$ we have $\cut{\tA}{k} =
  \maxuniontype{i \in 1..n}{(\cut{\tA_i}{k})}$ and $\cut{\tB}{k} =
  \maxuniontype{j \in 1..m}{(\cut{\tB_j}{k})}$ for every $k > 0$. From
  $\cut{\tA}{k} \eqtypeco \cut{\tB}{k}$, by $\ruleEqcoUnion$, we get $$
\begin{array}{r@{\quad\text{s.t.}\quad}c@{\quad\text{for every}\quad}l}
\exists f : 1..n \to 1..m & \cut{\tA_i}{k} \eqtypeco \cut{\tB_{f(i)}}{k} & i \in 1..n \\
\exists g : 1..m \to 1..n & \cut{\tA_{g(j)}}{k} \eqtypeco \cut{\tB_j}{k} & j \in 1..m
\end{array} $$
  Since $\cut{\tC}{0} = \bullet$ for every $\tC \in \Tree$, we have
  $\cut{\tA_i}{0} \eqtypeco \cut{\tB_{f(i)}}{0}$ and $\cut{\tA_{g(j)}}{0}
  \eqtypeco \cut{\tB_j}{0}$ by reflexivity. Thus, $\cut{\tA_i}{k} \eqtypeco
  \cut{\tB_{f(i)}}{k}$ and $\cut{\tA_{g(j)}}{k} \eqtypeco \cut{\tB_j}{k}$ for
  every $k \in \Natural$. Then, by definition of $\S$,
  $\pair{\tA_i}{\tB_{f(i)}}, \pair{\tA_{g(j)}}{\tB_j} \in \S$ for every $i \in
  1..n$, $j \in 1..m$. Finally, we conclude $\pair{\tA}{\tB} \in
  \Phieqtypeco\!\left(\S\right)$.
\end{enumerate}

$\Leftarrow)$ For this part of the proof we show that the converse relation $\S
\eqdef \set{\pair{\cut{\tA}{k}}{\cut{\tB}{k}} \mathrel| \tA \eqtypeco \tB, k
\in \Natural}$ is $\Phieqtypeco$-dense. Let $\pair{\cut{\tA}{k}}{\cut{\tB}{k}}
\in \S$. If $k = 0$, by definition of the truncation, $\cut{\tA}{k} = \bullet =
\cut{\tB}{k}$ and trivially $\pair{\bullet}{\bullet} \in
\Phieqtypeco\left(\S\right)$. We analyze next the cases where $k > 0$ given
that, by definition of $\S$, $\tA \eqtypeco \tB$. Once again we consider
maximal union types $$
\begin{array}{r@{\quad\text{with}\quad}l}
\tA = \maxuniontype{i \in 1..n}{\tA_i} & \tA_i \neq \iuniontype, i \in 1..n \\
\tB = \maxuniontype{j \in 1..m}{\tB_j} & \tB_j \neq \iuniontype, j \in 1..m
\end{array} $$ and analyze separately the cases where both $\tA$ and $\tB$ ar
non-union types.
\begin{enumerate}
  \item If $n = m = 1$ we a look at the shape of $A$:
  \begin{itemize}
    \item $\tA = a$. By Lem.~\ref{lem:equalityIsInvertible}, $\tB = a$ and
    $\cut{a}{k} = a$ for every $k > 0$. Then we conclude by definition of
    $\Phieqtypeco$, $\pair{a}{a} \in \Phieqtypeco\left(\S\right)$.
    
    \item $\tA = \datatype{\tD}{\tA'}$. By Lem.~\ref{lem:equalityIsInvertible},
    $\tB = \datatype{\tD'}{\tB'}$ with $\tD \eqtypeco \tD'$ and $\tA' \eqtypeco
    \tB'$. Then, by definition of $\S$,
    $\pair{\cut{\tD}{k-1}}{\cut{\tD'}{k-1}},
    \pair{\cut{\tA'}{k-1}}{\cut{\tB'}{k-1}} \in \S$ and we conclude
    $\pair{\cut{\tA}{k}}{\cut{\tB}{k}} =
    \pair{\datatype{\cut{\tD}{k-1}}{\cut{\tA'}{k-1}}}{\datatype{\cut{\tD'}{k-1}}{\cut{\tB'}{k-1}}}
    \in \Phieqtypeco\left(\S\right)$.
    
    \item $\tA = \functype{\tA'}{\tA''}$. Similarly to the previous case, we
    have $\tB = \functype{\tB'}{\tB''}$ with $\tA' \eqtypeco \tB'$ and $\tA''
    \eqtypeco \tB''$. Then $\pair{\cut{\tA'}{k-1}}{\cut{\tB'}{k-1}},
    \pair{\cut{\tA''}{k-1}}{\cut{\tB''}{k-1}} \in \S$ and
    $\pair{\cut{\tA}{k}}{\cut{\tB}{k}} =
    \pair{\functype{\cut{\tA'}{k-1}}{\cut{\tA''}{k-1}}}{\functype{\cut{\tB'}{k-1}}{\cut{\tB''}{k-1}}}
    \in \Phieqtypeco\left(\S\right)$.
  \end{itemize}
  
  \item If $n + m > 2$, by $\ruleEqcoUnion$ we have $$
\begin{array}{r@{\quad\text{s.t.}\quad}c@{\quad\text{for every}\quad}l}
\exists f : 1..n \to 1..m & \tA_i \eqtypeco \tB_{f(i)} & i \in 1..n \\
\exists g : 1..m \to 1..n & \tA_{g(j)} \eqtypeco \tB_j & j \in 1..m
\end{array} $$
  Then, by definition of $\S$, $\pair{\cut{\tA_i}{k}}{\cut{\tB_{f(i)}}{k}},
  \pair{\cut{\tA_{g(j)}}{k}}{\cut{\tB_j}{k}} \in \S$ for every $k > 0$. Thus,
  we conclude by resorting to Rem.~\ref{rem:cutMaximalUnionTypes},
  $\pair{\cut{\tA}{k}}{\cut{\tB}{k}} \in \Phieqtypeco\left(\S\right)$.
\end{enumerate}
\end{proof}



\begin{lemma}
\label{lem:cutSubtypingCo}
$\forall k \in \Natural.\cut{\tA}{k} \subtypeco \cut{\tB}{k}$ iff $\tA \subtypeco \tB$.
\end{lemma}

\begin{proof}
$\Rightarrow)$ Similarly to the previous lemma, we prove this part by showing
that $\S \eqdef \set{\pair{\tA}{\tB} \mathrel| \forall k \in \Natural.
\cut{\tA}{k} \subtypeco \cut{\tB}{k}}$ is $\Phisubtypeco$-dense. By hypothesis
we have $\cut{\tA}{k} \subtypeco \cut{\tB}{k}$ for every $k \in \Natural$. As
before we consider maximal union types and analyze separately the case for
non-union types $$
\begin{array}{r@{\quad\text{with}\quad}l}
\tA = \maxuniontype{i \in 1..n}{\tA_i} & \tA_i \neq \iuniontype, i \in 1..n \\
\tB = \maxuniontype{j \in 1..m}{\tB_j} & \tB_j \neq \iuniontype, j \in 1..m
\end{array} $$
\begin{enumerate}
  \item If $n = m = 1$ (\ie $\tA, \tB \neq \iuniontype$), we proceed by
  analyzing the shape of $\tA$:
  \begin{itemize}
    \item $\tA = a$. Then, $\cut{\tA}{k} = a$ for every $k > 0$ and, by
    Lem.~\ref{lem:subtypingIsInvertible}, $\cut{\tB}{k} = a$. Hence, $\tB = a$
    and we conclude directly from the definition of $\Phisubtypeco$,
    $\pair{a}{a} \in \Phisubtypeco\!\left(\S\right)$.
    
    \item $\tA = \datatype{\tD}{\tA'}$. Similarly, we have $\cut{\tA}{k} =
    \datatype{\cut{\tD}{k-1}}{\cut{\tA'}{k-1}}$ for every $k > 0$. By
    Lem.~\ref{lem:subtypingIsInvertible} once again, we get $\cut{\tB}{k} =
    \datatype{\tD'_k}{\tB'_k}$ with $\cut{\tD}{k-1} \subtypeco \tD'_k$ and
    $\cut{\tA'}{k-1} \subtypeco \tB'_k$. As in the previous lemma, in this case
    we have different subtrees $\tD'_k$ and $\tB'_k$ for every $k$ but, by
    resorting to tree equality on Lem.~\ref{lem:subtypingIsInvertible} and the
    definition of the truncation, we can assure that $\tB =
    \datatype{\tD'}{\tB'}$ with $\tD'_k = \cut{\tD'}{k-1}$ and $\tB'_k =
    \cut{\tB'}{k-1}$ for every $k > 0$. Hence, $\cut{\tD}{k-1} \subtypeco
    \cut{\tD'}{k-1}$ and $\cut{\tA'}{k-1} \subtypeco \cut{\tB'}{k-1}$ for every
    $k > 0$. Then, by definition of $\S$, $\pair{\tD}{\tD'}, \pair{\tA'}{\tB'}
    \in \S$ and we conclude $\pair{\datatype{\tD}{\tA'}}{\datatype{\tD'}{\tB'}}
    \in \Phisubtypeco\!\left(\S\right)$.
    
    \item $\tA = \functype{\tA'}{\tA''}$. Analysis for this case is similar to
    the previous one. From $\cut{\tA}{k} =
    \functype{\cut{\tA'}{k-1}}{\cut{\tA''}{k-1}}$ we get $\tB =
    \functype{\tB'}{\tB''}$ with $\cut{\tB'}{k-1} \subtypeco \cut{\tA'}{k-1}$
    and $\cut{\tA''}{k-1} \subtypeco \cut{\tB''}{k-1}$ for every $k > 0$. Note
    that, by Lem.~\ref{lem:subtypingIsInvertible}, subtyping order on the
    domains is inverted. Then we have $\pair{\tB'}{\tA'}, \pair{\tA''}{\tB''}
    \in \S$ and conclude $\pair{\functype{\tA'}{\tA''}}{\functype{\tB'}{\tB''}}
    \in \Phisubtypeco\!\left(\S\right)$.
  \end{itemize}
  
  \item If $n + m > 2$ we have $\cut{\tA}{k} = \maxuniontype{i \in
  1..n}{(\cut{\tA_i}{k})}$ and $\cut{\tB}{k} = \maxuniontype{j \in
  1..m}{(\cut{\tB_j}{k})}$ for every $k > 0$. From $\cut{\tA}{k} \subtypeco
  \cut{\tB}{k}$, by $\ruleSubcoUnion$, we get $$
\begin{array}{r@{\quad\text{s.t.}\quad}c@{\quad\text{for every}\quad}l}
\exists f : 1..n \to 1..m & \cut{\tA_i}{k} \subtypeco \cut{\tB_{f(i)}}{k} & i \in 1..n \\
\exists g : 1..m \to 1..n & \cut{\tA_{g(j)}}{k} \subtypeco \cut{\tB_j}{k} & j \in 1..m
\end{array} $$
  Since $\cut{\tC}{0} = \bullet$ for every $\tC \in \Tree$, we also have
  $\cut{\tA_i}{0} \subtypeco \cut{\tB_{f(i)}}{0}$ and $\cut{\tA_{g(j)}}{0}
  \subtypeco \cut{\tB_j}{0}$ by reflexivity. Thus, $\cut{\tA_i}{k} \subtypeco
  \cut{\tB_{f(i)}}{k}$ and $\cut{\tA_{g(j)}}{k} \subtypeco \cut{\tB_j}{k}$ for
  every $k \in \Natural$. Then, by definition of $\S$,
  $\pair{\tA_i}{\tB_{f(i)}}, \pair{\tA_{g(j)}}{\tB_j} \in \S$ for every $i \in
  1..n$, $j \in 1..m$. Finally, we conclude $\pair{\tA}{\tB} \in
  \Phisubtypeco\!\left(\S\right)$.
\end{enumerate}


$\Leftarrow)$ As before, we define $\S \eqdef
\set{\pair{\cut{\tA}{k}}{\cut{\tB}{k}} \mathrel| \tA \subtypeco \tB, k \in
\Natural}$ and show that is $\Phisubtypeco$-dense to prove this part of the
lemma. Again, if $k = 0$ the result is immediate, so lets focus on the case
where $k > 0$.

Let $\tA \subtypeco \tB$. We assume, without loss of generality, $\tA =
\maxuniontype{i \in 1..n}{\tA_i}$ and $\tB = \maxuniontype{j \in 1..m}{\tB_j}$
are maximal union types.

If $n + m > 2$ it is the case of $\ruleSubcoUnion$ and we have $\exists f :
1..n \to 1..m$ such that $\tA_i \subtypeco \tB_{f(i)}$ for every $i \in 1..n$.
Then, by definition we have $\pair{\cut{\tA_i}{k}}{\cut{\tB_{f(i)}}{k}} \in \S$
and conclude $\pair{\cut{\tA}{k}}{\cut{\tB}{k}} \in
\Phisubtypeco\left(\S\right)$.

On the other hand, if $n = 1 = m$ we analyze the form of $\tA$:
\begin{enumerate}
  \item $\tA = a$. By Lem.~\ref{lem:subtypingIsInvertible} we have $\tB = a$
  and the result is immediate.
  
  \item $\tA = \datatype{\tD}{\tA'}$. By Lem.~\ref{lem:subtypingIsInvertible},
  $\tB = \datatype{\tD'}{\tB'}$ with $\tD \subtypeco \tD'$ and $\tA' \subtypeco
  \tB'$. Then we have $\pair{\cut{\tD}{k-1}}{\cut{\tD'}{k-1}},
  \pair{\cut{\tA'}{k-1}}{\cut{\tB'}{k-1}} \in \S$ for every $k > 0$, and
  conclude by definition of $\Phisubtypeco$, $\pair{\cut{\tA}{k}}{\cut{\tB}{k}}
  \in \Phisubtypeco\left(\S\right)$.
  
  \item $\tA = \functype{\tA'}{\tA''}$. Similarly to the previous case we have
  $\tB = \functype{\tB'}{\tB''}$ with $\tB' \subtypeco \tA'$ and $\tA''
  \subtypeco \tB''$. Then we conclude by definition of $\S$ and $\Phisubtypeco$
  that $\pair{\cut{\tA}{k}}{\cut{\tB}{k}} =
  \pair{\functype{\cut{\tA'}{k-1}}{\cut{\tA''}{k-1}}}{\functype{\cut{\tB'}{k-1}}{\cut{\tB''}{k-1}}}
  \in \Phisubtypeco\left(\S\right)$.
\end{enumerate}
\end{proof}



\subsubsection{Correspondence between \texorpdfstring{$\mu$}{u}-types and infinite types}


Contractive $\mu$-types
characterize~\cite{journals/tcs/Courcelle83,DBLP:journals/toplas/AmadioC93,DBLP:journals/fuin/BrandtH98,Pierce:2002:TPL:509043}
a proper subset of $\Tree$ known as the \emphdef{regular trees} (trees whose
set of distinct subtrees is finite) and denoted $\TreeRegular$.
Given a contractive $\mu$-type $A$, $\toBTree{A}$ is the regular tree obtained
by completely unfolding all occurrences of $\rectype{V}{B}$ in $A$.
Def.~\ref{def:treeFunction} below extends that of~\cite{Pierce:2002:TPL:509043}
to union and data types. It is well-founded, relying on the lexicographical
extension of the standard order to $\pair{\length{\pi}}{\card{\irectype}{A}}$,
where $\card{\irectype}{A}$ is the number of occurrences of the $\irectype$
type constructor at the head position of $A$.

\begin{definition}
\label{def:treeFunction}
The function $\toBTree{\bullet} : \Type \to \TreeRegular$, mapping $\mu$-types to types, is defined inductively as follows: $$
\begin{array}{rcll}
\toBTree{a}(\epsilon)             & \eqdef & a \\
\toBTree{A_1 \star A_2}(\epsilon) & \eqdef & \star           & \quad \text{for $\star \in \set{\idatatype, \ifunctype, \iuniontype}$} \\
\toBTree{A_1 \star A_2}(i\pi)     & \eqdef & \toBTree{A_i}(\pi) & \quad \text{for $\star \in \set{\idatatype, \ifunctype, \iuniontype}$} \\
\toBTree{\rectype{V}{A}}(\pi)     & \eqdef & \toBTree{\substitute{V}{\rectype{V}{A}}{A}}(\pi) \\
\end{array} $$
\end{definition}



Commutation of $\toBTree{\bullet}$ with substitutions is as expected. 

\begin{lemma}
\label{lem:substitutionOfTrees}
$\toBTree{\substitute{V}{B}{A}} = \substitute{V}{\toBTree{B}}{\toBTree{A}}$.
\end{lemma}

\begin{proof}
We actualy prove the equivalente result $$\forall{k \in \Natural}.
\cut{\toBTree{\substitute{V}{B}{A}}}{k} =
\cut{(\substitute{V}{\toBTree{B}}{\toBTree{A}})}{k}$$ and conclude by
reflexivity of $\eqtypeco$ and Lem.~\ref{lem:cutEquivalenceCo}.

The proof is by induction on the lexicographical extension of the standard
order to
$\pair{h(\cut{\toBTree{\substitute{V}{B}{A}}}{k})}{\card{\irectype\iuniontype}{A}}$,
where $h : \TreeFinite \to \Natural$ is the height function for finite trees
and $\card{\irectype\iuniontype}{A}$ is the number of occurrences of both
$\irectype$ and $\iuniontype$ at the head of $A$.

We proceed by analyzing the possible forms of $A$ and assuming $k > 0$ since
the result for that case is immediate.
\begin{itemize}
  \item $A = V$: then $\cut{\toBTree{\substitute{V}{B}{V}}}{k} =
  \cut{\toBTree{B}}{k} = \cut{(\substitute{V}{\toBTree{B}}{V})}{k}$ by
  Lem.~\ref{lem:treeSubstitution}.
  
  \item $A = a \neq V$: then $\cut{\toBTree{\substitute{V}{B}{a}}}{k} =
  \cut{\toBTree{a}}{k} = a = \cut{(\substitute{V}{\toBTree{B}}{a})}{k}$ by
  definition of the interpretation and Lem.~\ref{lem:treeSubstitution}.
  
  \item $A = \datatype{D}{A'}$: then $$\kern-3em
\begin{array}{r@{\quad=\quad}l@{\quad}l}
\cut{\toBTree{\substitute{V}{B}{A}}}{k} & \cut{\toBTree{\datatype{\substitute{V}{B}{D}}{\substitute{V}{B}{A'}}}}{k} \\
                                        & \datatype{\cut{\toBTree{\substitute{V}{B}{D}}}{k-1}}{\cut{\toBTree{\substitute{V}{B}{A'}}}{k-1}} & \text{by Def.~\ref{def:treeFunction} and~\ref{def:treeCut}} \\
                                        & \datatype{\cut{(\substitute{V}{\toBTree{B}}{\toBTree{D}})}{k-1}}{\cut{(\substitute{V}{\toBTree{B}}{\toBTree{A'}})}{k-1}} & \text{by IH} \\
                                        & \cut{(\datatype{\substitute{V}{\toBTree{B}}{\toBTree{D}}}{\substitute{V}{\toBTree{B}}{\toBTree{A'}}})}{k} & \text{by Def.~\ref{def:treeCut}} \\
                                        & \cut{(\substitute{V}{\toBTree{B}}{\toBTree{\datatype{D}{A'}}})}{k} & \text{by Lem.~\ref{lem:treeSubstitution} and Def.~\ref{def:treeFunction}}
\end{array} $$
  
  \item $A = \functype{A'}{A''}$: this case is similar to the previous one.
  
  \item $A = \uniontype{A_1}{A_2}$: analysis for this case is similar to the
  previous ones but notice that we get the same $k$ when resorting to
  Def.~\ref{def:treeCut} (instead of $k-1$) before applying the inductive
  hypothesis. However, we are in conditions to apply it anyway since
  $$h(\cut{\toBTree{\substitute{V}{B}{A}}}{k}) \geq
  h(\cut{\toBTree{\substitute{V}{B}{A_i}}}{k}) \quad\text{but}\quad
  \card{\irectype\iuniontype}{A} > \card{\irectype\iuniontype}{A_i}$$ Hence, it
  is safe to conclude $\cut{\toBTree{\substitute{V}{B}{A}}}{k} =
  \cut{(\substitute{V}{\toBTree{B}}{\toBTree{A}})}{k}$.
  
  \item $A = \rectype{W}{A'}$: without loss of generality we can assume
  $\rename{V}{B} \avoids W$\footnote{We use the predicate $\sigma \avoids V$ to
  mean that there is no collition at all between $V$ and the variables in
  $\sigma$ (\ie $V \notin \dom{\sigma} \cap (\bigcup_{x \in \dom{\sigma}}
  \fv{\sigma x})$).}. Then $$
\begin{array}{r@{\quad=\quad}l@{\quad}l}
\cut{\toBTree{\substitute{V}{B}{A}}}{k} & \cut{\toBTree{\rectype{W}{\substitute{V}{B}{A'}}}}{k} \\
                                        & \cut{\toBTree{\substitute{W}{\rectype{W}{\substitute{V}{B}{A'}}}{\substitute{V}{B}{A'}}}}{k} & \text{by Def.~\ref{def:treeFunction}} \\
                                        & \cut{\toBTree{\substitute{V}{B}{\substitute{W}{A}{A'}}}}{k} \\
                                        & \cut{(\substitute{V}{\toBTree{B}}{\toBTree{\substitute{W}{A}{A'}}})}{k} & \text{by IH} \\
                                        & \cut{(\substitute{V}{\toBTree{B}}{\toBTree{A}})}{k} & \text{by Def.~\ref{def:treeFunction}}
\end{array} $$ Here we are in condition to apply the indutive hypothesis since
  $\card{\irectype\iuniontype}{A} >
  \card{\irectype\iuniontype}{\substitute{W}{A}{A'}} $ by contractiveness.
\end{itemize}
\end{proof}



The finite unfolding of a contractive $\mu$-type $A$ consists of recursively
replacing all occurrences of a bounded variable $V$ by $A$ itself a finite
number of times. We formalize a slightly more general variation of this idea in
the following lemma and prove its relation with $\toBTree{A}$.

\begin{lemma}
\label{lem:cutFiniteUnfolding}
Let $A = \rectype{V}{A'}$, $B$ any other $\mu$-type and $\sigma$ a
substitution. Define $$\unfoldf{A}{\sigma}{0} \eqdef B \qquad
\unfoldf{A}{\sigma}{n+1} \eqdef (\sigma \uplus
\rename{V}{\unfoldf{A}{\sigma}{n}})A'$$ Then, $\forall k \in \Natural.
\cut{\toBTree{\unfoldf{A}{\sigma}{k}}}{k} \eqtypeco \cut{\toBTree{\sigma
A}}{k}$.
\end{lemma}

\begin{proof}
By induction on $k$. We assume without loss of generality that $\sigma \avoids
V$.
\begin{itemize}
  \item $k = 0$. Then $\cut{\toBTree{B}}{0} = \bullet = \cut{\toBTree{\sigma
  A}}{0}$ by definition of the truncation.
  
  \item $k > 0$. By inductive hypothesis we have
  $\cut{\toBTree{\unfoldf{A}{\sigma}{k-1}}}{k-1} \eqtypeco
  \cut{\toBTree{\sigma A}}{k-1}$. Moreover, since $A = \rectype{V}{A'}$ is
  contractive, the first appearance of $V$ in $A'$ is at depth $n > 1$. So we
  have $k \leq k - 1 + n$ and, by Lem.~\ref{lem:substitutionOfEqtypesCo}
  and~\ref{lem:substitutionOfTrees}, we may conclude $$
\begin{array}{rcl@{\quad}l}
\cut{\toBTree{\unfoldf{A}{\sigma}{k}}}{k} & =         & \cut{\toBTree{(\sigma \uplus \rename{V}{\unfoldf{A}{\sigma}{k-1}})A'}}{k} \\
                                          & =         & \cut{(\substitute{V}{\toBTree{\unfoldf{A}{\sigma}{k-1}}}{\toBTree{\sigma A'}})}{k}            & \text{by Lem.~\ref{lem:substitutionOfTrees}} \\
                                          & =         & \cut{(\substitute{V}{\cut{\toBTree{\unfoldf{A}{\sigma}{k-1}}}{k-1}}{\toBTree{\sigma A'}})}{k} & k \leq k - 1 + n \\
                                          & \eqtypeco & \cut{(\substitute{V}{\cut{\toBTree{\sigma A}}{k-1}}{\toBTree{\sigma A'}})}{k}                 & \text{by Lem.~\ref{lem:substitutionOfEqtypesCo}} \\
                                          & =         & \cut{(\substitute{V}{\toBTree{\sigma A}}{\toBTree{\sigma A'}})}{k}                            & k \leq k - 1 + n \\
                                          & =         & \cut{\toBTree{(\sigma \uplus \rename{V}{\sigma A})A'}}{k}                                  & \quad\text{by Lem.~\ref{lem:substitutionOfTrees}} \\
                                          & =         & \cut{\toBTree{\sigma A}}{k}
\end{array} $$
\end{itemize}
\end{proof}

\begin{remark}
\label{rem:cutFiniteUnfolding}
It follows immediately from the previous result that for every $n \geq k$,
$\cut{\toBTree{\unfoldf{A}{\sigma}{n}}}{k} \eqtypeco
\cut{\toBTree{\sigma A}}{k}$.
\end{remark}



One of the main results of this section is the correspondence between the
equivalence relations $\eqtypemu$ and $\eqtypeco$ via the function
$\toBTree{\bullet}$. It follows from the lemma below that relates two
$\mu$-equivalent types with the truncation of their respective trees:

\begin{lemma}
\label{lem:cutEquivalenceMu}
$A \eqtypemu B$ iff $\forall k \in \Natural.\cut{\toBTree{A}}{k} \eqtypeco
\cut{\toBTree{B}}{k}$.
\end{lemma}

\begin{proof}
$\Rightarrow)$ This part of the proof is by induction on $A \eqtypemu B$
analyzing the last rule applied. Note that $\cut{\toBTree{A}}{0} = \bullet =
\cut{\toBTree{B}}{0}$ by definition of the truncation, so we only analyze the
cases where $k > 0$.
\begin{itemize}
  \item $\ruleEqmuRefl$: then $B = A$ and we conclude by reflexivity of
  $\eqtypeco$, $\cut{\toBTree{A}}{k} \eqtypeco \cut{\toBTree{A}}{k}$ for every
  $k > 0$.

  \item $\ruleEqmuTrans$: then $A \eqtypemu C$ and $C \eqtypemu B$. By
  inductive hypothesis $\cut{\toBTree{A}}{k} \eqtypeco \cut{\toBTree{C}}{k}$
  and $\cut{\toBTree{C}}{k} \eqtypeco \cut{\toBTree{B}}{k}$ for every $k > 0$.
  Then we conclude by transitivity of $\eqtypeco$.
  
  \item $\ruleEqmuSymm$: then $B \eqtypemu A$. By inductive hypothesis
  $\cut{\toBTree{B}}{k} \eqtypeco \cut{\toBTree{A}}{k}$ for every $k > 0$ and
  we conclude by symmetry of $\eqtypeco$.
  
  \item $\ruleEqmuFunc$: then $A = \functype{A'}{A''}, B = \functype{B'}{B''}$
  with $A' \eqtypemu B'$ and $A'' \eqtypemu B''$. By inductive hypothesis
  $\cut{\toBTree{A'}}{k} \eqtypeco \cut{\toBTree{B'}}{k}$ and
  $\cut{\toBTree{A''}}{k} \eqtypeco \cut{\toBTree{B''}}{k}$ for every $k > 0$.
  Then $$\cut{\toBTree{A}}{k} =
  \functype{\cut{\toBTree{A'}}{k-1}}{\cut{\toBTree{A''}}{k-1}} \eqtypeco
  \functype{\cut{\toBTree{B'}}{k-1}}{\cut{\toBTree{B''}}{k-1}} =
  \cut{\toBTree{B}}{k}$$
  
  \item $\ruleEqmuComp$: then $A = \datatype{D}{A'}, B = \datatype{D'}{B'}$
  with $A' \eqtypemu B'$ and $A'' \eqtypemu B''$. This case is similar to the
  previous one. We conclude directly from the inductive hypothesis and the
  definition of the truncation $$\cut{\toBTree{\datatype{D}{A'}}}{k} \eqtypeco
  \cut{\toBTree{\datatype{D'}{B'}}}{k}$$
  
  \item $\ruleEqmuUnionIdem$: then $A = \uniontype{B}{B}$. In this case we need
  to take into account that $B$ may be a union type as well and, when working
  with $\eqtypeco$, we must consider maximal union types. Let
  $\cut{\toBTree{A}}{k} = \maxuniontype{i \in 1..n}{\tA_i}$ and
  $\cut{\toBTree{B}}{k} = \maxuniontype{j \in 1..m}{\tB_j}$ with $\tA_j, \tB_j
  \neq \iuniontype$. It is immedate to see from the equality above that
  $n = 2*m$ and $\tA_j = \tA_{2*j} = \tB_j$ for every $j \in 1..m$. Finally we
  conclude by reflexivity of $\eqtypeco$ and $\ruleEqcoUnion$ $$
\begin{array}{rcl}
\cut{\toBTree{A}}{k} & =         & \maxuniontype{i \in 1..n}{\tA_i} \\
                     & =         & \uniontype{(\maxuniontype{j \in 1..m}{\tB_j})}{(\maxuniontype{j \in 1..m}{\tB_j})} \\
                     & \eqtypeco & \maxuniontype{j \in 1..m}{\tB_j} \\
                     & =         & \cut{\toBTree{B}}{k}
\end{array} $$
  
  \item $\ruleEqmuUnionComm$: then $A = \uniontype{C_1}{C_2}$ and $B =
  \uniontype{C_2}{C_1}$. As in the previous case consider $\cut{A}{k} =
  \maxuniontype{i \in 1..n}{\tA_i}$ and $\cut{B}{k} = \maxuniontype{j \in
  1..m}{\tB_j}$ with $\tA_i, \tB_j \neq \iuniontype$. Here $n = m > 1$, hence
  $n + m > 2$. Moreover, assuming $\tA_k$ is the last component of $C_1$ ($k
  \in 1..(n-1)$), we have $\tA_i = \tB_{i+k}$ if $i \leq n-k$, and $\tA_i =
  \tB_{i-(n-k)}$ if $i > n-k$. Thus, we conclude by reflexivity of $\eqtypeco$
  and $\ruleEqcoUnion$, $\cut{\toBTree{A}}{k} \eqtypeco \cut{\toBTree{B}}{k}$.
  
  \item $\ruleEqmuUnionAssoc$: then $A =
  \uniontype{C_1}{(\uniontype{C_2}{C_3})}$ and $B =
  \uniontype{(\uniontype{C_1}{C_2})}{C_3}$. Considering maximal union types as
  before we have $\cut{A}{k} = \maxuniontype{i \in 1..n}{\tA_i}$ and
  $\cut{B}{k} = \maxuniontype{j \in 1..m}{\tB_j}$ with $\tA_i, \tB_j \neq
  \iuniontype$ and $n = m > 2$. In this case we may conclude by resorting to
  the identity function in $1..n$, since $\tA_i = \tB_i$. Thus, by reflexivity
  and $\ruleEqcoUnion$, $\cut{\toBTree{A}}{k} \eqtypeco \cut{\toBTree{B}}{k}$.
  
  \item $\ruleEqmuUnion$: then $A = \uniontype{A_1}{A_2}, B =
  \uniontype{B_1}{B_2}$ with $A_1 \eqtypemu B_1$ and $A_2 \eqtypemu B_2$. By
  inductive hypothesis $\cut{\toBTree{A_1}}{k} \eqtypeco
  \cut{\toBTree{B_1}}{k}$ and $\cut{\toBTree{A_2}}{k} \eqtypeco
  \cut{\toBTree{B_2}}{k}$ for every $k \in \Natural$. Assume, without loss of
  generality $$
\begin{array}{r@{\quad\text{with}\quad}l}
\cut{\toBTree{A_1}}{k} = \maxuniontype{i \in 1..n}{\tA_i} & \tA_i \neq \iuniontype, i \in 1..n \\
\cut{\toBTree{B_1}}{k} = \maxuniontype{j \in 1..m}{\tB_j} & \tB_j \neq \iuniontype, j \in 1..m
\end{array} $$

  If $n + m > 2$, there exists $f : 1..n \to 1..m$, $g : 1..m \to 1..n$ such
  that $\tA_i \eqtypeco \tB_{f(i)}$ and $\tA_{g(j)} \eqtypeco \tB_j$. If not
  (\ie $n = m = 1$), we simply take $f = g = \mathit{id}$.
  
  Likewise, for $A_2$ and $B_2$ we have $$
\begin{array}{r@{\quad\text{with}\quad}l}
\cut{\toBTree{A_2}}{k} = \maxuniontype{i \in 1..n'}{\tA'_i} & \tA'_i \neq \iuniontype, i \in 1..n' \\
\cut{\toBTree{B_2}}{k} = \maxuniontype{j \in 1..m'}{\tB'_j} & \tB'_j \neq \iuniontype, j \in 1..m'
\end{array} $$ and there exists $f' : 1..n' \to 1..m'$, $g' : 1..m' \to 1..n'$
  such that $\tA'_i \eqtypeco \tB'_{f'(i)}$ and $\tA'_{g'(j)} \eqtypeco
  \tB'_j$.

  Finally, since $(n + n' + m + m') > 2$, we can apply $\ruleEqcoUnion$ to
  conclude $$
\begin{array}{rcl}
\cut{\toBTree{A}}{k} & =         & \uniontype{\cut{\toBTree{A_1}}{k}}{\cut{\toBTree{A_2}}{k}} \\
                     & =         & \uniontype{(\maxuniontype{i \in 1..n}{\tA_i})}{(\maxuniontype{i \in 1..n'}{\tA'_i})} \\
                     & \eqtypeco & \uniontype{(\maxuniontype{j \in 1..m}{\tB_j})}{(\maxuniontype{j \in 1..m'}{\tB'_j})} \\
                     & =         & \uniontype{\cut{\toBTree{B_1}}{k}}{\cut{\toBTree{B_2}}{k}} \\
                     & =         & \cut{\toBTree{B}}{k}
\end{array} $$
  
  \item $\ruleEqmuRec$: then $A = \rectype{V}{A'}, B = \rectype{V}{B'}$ with
  $A' \eqtypemu B'$. By inductive hypothesis $\cut{\toBTree{A'}}{k} \eqtypeco
  \cut{\toBTree{B'}}{k}$ and, by Lem.~\ref{lem:cutEquivalenceCo}, $\toBTree{A'}
  \eqtypeco \toBTree{B'}$.
  
  Now we consider the definition of $A_\sigma^n$ and $B_\sigma^n$ as in
  Lem.~\ref{lem:cutFiniteUnfolding} with $A_\sigma^0 \eqdef \sigma A'$ and
  $B_\sigma^0 \eqdef \sigma B'$. We claim that $\toBTree{A_{id}^n} \eqtypeco
  \toBTree{B_{id}^n}$ for every $n \in \Natural$. To prove this we proceed by
  induction on $n$
  \begin{itemize}
    \item $n = 0$. Then we have $\toBTree{A_{id}^0} = \toBTree{A'} \eqtypeco
    \toBTree{B'} = \toBTree{B_{id}^0}$ that holds by hypothesis.
    
    \item $n > 0$. By reflexivity
    $\substitute{V}{\toBTree{A_{id}^{n-1}}}{\toBTree{A'}} \eqtypeco
    \substitute{V}{\toBTree{A_{id}^{n-1}}}{\toBTree{A'}}$. Also, by inductive
    hypothesis, $\toBTree{A_{id}^{n-1}} \eqtypeco \toBTree{B_{id}^{n-1}}$ and,
    by hypothesis, $\toBTree{A'} \eqtypeco \toBTree{B'}$. Then we can apply
    Lem.~\ref{lem:substitutionOfEqtypesCo} and~\ref{lem:substitutionOfTrees},
    and conclude $$\toBTree{A_{id}^n} =
    \substitute{V}{\toBTree{A_{id}^{n-1}}}{\toBTree{A'}} \eqtypeco
    \substitute{V}{\toBTree{B_{id}^{n-1}}}{\toBTree{B'}} = \toBTree{B_{id}^n}$$
  \end{itemize}
  
  Finally, by Lem.~\ref{lem:cutEquivalenceCo}, $\cut{\toBTree{A_{id}^n}}{k}
  \eqtypeco \cut{\toBTree{B_{id}^n}}{k}$ for every $k, n \in \Natural$. Thus we
  conclude by Lem.~\ref{lem:cutFiniteUnfolding} $$\cut{\toBTree{A}}{k}
  \eqtypeco \cut{\toBTree{A_{id}^k}}{k} \eqtypeco \cut{\toBTree{B_{id}^k}}{k}
  \eqtypeco \cut{\toBTree{B}}{k}$$
  
  \item $\ruleEqmuFold$: then $A = \rectype{V}{A'}$ and $B =
  \substitute{V}{\rectype{V}{A'}}{A'}$. The result is immediate by definition
  of the interpretation, $\toBTree{A} = \toBTree{\rectype{V}{A'}} =
  \toBTree{\substitute{V}{\rectype{V}{A'}}{A'}} = \toBTree{B}$. Then
  $\cut{\toBTree{A}}{k} \eqtypeco \cut{\toBTree{B}}{k}$ for every $k \in
  \Natural$ by reflexivity of $\eqtypeco$.
  
  \item $\ruleEqmuContr$: then $B = \rectype{V}{B'}$ is contractive and $A
  \eqtypemu \substitute{V}{A}{B'}$. By inductive hypothesis and
  Lem.~\ref{lem:cutEquivalenceCo}, $\toBTree{A} \eqtypeco
  \toBTree{\substitute{V}{A}{B'}}$.
  
  As in the previous case we consider $B_\sigma^n$ from
  Lem.~\ref{lem:cutFiniteUnfolding}, this time with $B_\sigma^0 \eqdef \sigma
  A$. Now we show $\toBTree{A} \eqtypeco \toBTree{B_{id}^n}$ for every $n \in
  \Natural$, by induction on $n$
  
  \begin{itemize}
    \item $n = 0$. This case is immediate since $\toBTree{B_{id}^0} =
    \toBTree{A}$ by definition.
    
    \item $n > 0$. Then, by definition and Lem.~\ref{lem:substitutionOfTrees},
    $\toBTree{B_{id}^n} =
    \substitute{V}{\toBTree{B_{id}^{n-1}}}{\toBTree{B'}}$. By inductive
    hypothesis we know $\toBTree{A} \eqtypeco \toBTree{B_{id}^{n-1}}$ and, by
    Lem.~\ref{lem:substitutionOfEqtypesCo}, $\toBTree{B_{id}^n} \eqtypeco
    \substitute{V}{\toBTree{A}}{\toBTree{B'}}$. Finally we conclude by applying
    Lem.~\ref{lem:substitutionOfTrees} and transitivity of $\eqtypeco$ with
    hypothesis $\toBTree{A} \eqtypeco \toBTree{\substitute{V}{A}{B'}}$
    $$\toBTree{B_{id}^n} \eqtypeco \toBTree{\substitute{V}{A}{B'}} \eqtypeco
    \toBTree{A}$$
  \end{itemize}
  
  Then, by Lem.~\ref{lem:cutEquivalenceCo}, $\cut{\toBTree{A}}{k} \eqtypeco
  \cut{\toBTree{B_{id}^n}}{k}$ for every $k, n \in \Natural$. On the other
  hand, by Lem.~\ref{lem:cutFiniteUnfolding}, we know
  $\cut{\toBTree{B_{id}^k}}{k} \eqtypeco \cut{\toBTree{B}}{k}$. Thus, we
  conclude $$\cut{\toBTree{A}}{k} \eqtypeco \cut{\toBTree{B_{id}^k}}{k}
  \eqtypeco \cut{\toBTree{B}}{k}$$
\end{itemize}


$\Leftarrow)$ Let $\cut{\toBTree{A}}{k} \eqtypeco \cut{\toBTree{B}}{k}$ for
every $k \in \Natural$. Given $B = \rectype{V}{B'}$ it is immediate to see that
$\toBTree{\rectype{V}{B'}} = \toBTree{\substitute{V}{B}{B'}}$ while $B
\eqtypemu \substitute{V}{B}{B'}$, by definition of the interpretation and
$\ruleEqmuFold$ respectively. Moreover, since $\mu$-types are contractive, we
can assure that $\card{\irectype}{\substitute{V}{B}{B'}} <
\card{\irectype}{B}$. By a simple induction on $\card{\irectype}{B}$ we can
prove that for every $B \in \Type$ there exists $C \in \Type$ such that
$\card{\irectype}{C} = 0$, $B \eqtypemu C$ and $\toBTree{B} = \toBTree{C}$. It
is important to note that we are resorting to tree equality on this argument.
Thus, without loss of generality, we consider during the proof only the cases
where $\card{\irectype}{B} = 0$.

This proof is by induction on the lexicographical extension of the standard
order to $\pair{h(\cut{\toBTree{A}}{k})}{\card{\irectype}{A}}$, where $h :
\TreeFinite \to \Natural$ is the height function for finite trees. We proceed
by analyzing the possible forms of $A$.

Given $A, B \in \Type$ we can assume $$
\begin{array}{r@{\quad\text{with}\quad}l}
\toBTree{A} = \maxuniontype{i \in 1..n}{\tA_i} & \tA_i \neq \iuniontype, i \in 1..n \\
\toBTree{B} = \maxuniontype{j \in 1..m}{\tB_j} & \tB_j \neq \iuniontype, j \in 1..m
\end{array} $$ by Rem.~\ref{rem:maximalUnionTypes}. Moreover, since
$\card{\irectype}{B} = 0$ and by definition of the interpretation, we have
$B = \maxuniontype{j \in 1..m}{B_j}$ with $\toBTree{B_j} = \tB_j$ for every $j
\in 1..m$ (note that $B_j$ is a non-union type for every $j \in 1..m$).

Then, we can divide this proof in two cases, either
\begin{inparaenum}[(i)]
  \item $A$ and $B$ are both non-union types and thus $n = m = 1$; or
  \item at least one of them is a union type (\ie $n + m > 2$).
\end{inparaenum}

\begin{enumerate}[(i)]
  \item If $n = m = 1$. Here we analyze the shape of $A$:
  \begin{itemize}
    \item $A = a$. Then $\cut{\toBTree{A}}{k} = a$ for every $k > 0$ and, by
    Lem.~\ref{lem:equalityIsInvertible}, $\cut{\toBTree{B}}{k} = \cut{\tB_1}{k}
    = a$. Thus, by definition of the interpretation and tree truncation with
    the assumption $\card{\irectype}{B} = 0$, we have $B = a$ and conclude with
    $\ruleEqmuRefl$.
    
    \item $A = \datatype{D}{A'}$. Here we have $\cut{\toBTree{A}}{k} =
    \datatype{\cut{\toBTree{D}}{k-1}}{\cut{\toBTree{A'}}{k-1}}$ for every $k >
    0$ and, by Lem.~\ref{lem:equalityIsInvertible} once again,
    $\cut{\toBTree{B}}{k} = \datatype{\tB'_k}{\tB''_k}$ with
    $\cut{\toBTree{D}}{k-1} \eqtypeco \tB'_k$ and $\cut{\toBTree{A'}}{k-1}
    \eqtypeco \tB''_k$. With a similar analysis to the one made in
    Lem.~\ref{lem:cutEquivalenceCo}, by definition of the interpretation and
    tree truncation with the assumption $\card{\irectype}{B} = 0$, we can
    assure that $B = \datatype{D'}{B'}$ such that $\tB'_k =
    \cut{\toBTree{D'}}{k-1}$ and $\tB''_k = \cut{\toBTree{B'}}{k-1}$ for every
    $k > 0$. Then, we have $\cut{\toBTree{D}}{k-1} \eqtypeco
    \cut{\toBTree{D'}}{k-1}$ and $\cut{\toBTree{A'}}{k-1} \eqtypeco
    \cut{\toBTree{B'}}{k-1}$ and we can apply the inductive hypothesis to get
    $D \eqtypemu D'$ and $A' \eqtypemu B'$. Finally we conclude by
    $\ruleEqmuComp$, $\datatype{D}{A'} \eqtypemu \datatype{D'}{B'}$.

    \item $A = \functype{A'}{A''}$. Analysis for this case is similar to the
    previous one. From $\cut{\toBTree{A}}{k} =
    \functype{\cut{\toBTree{A'}}{k-1}}{\cut{\toBTree{A''}}{k-1}}$ we get $B =
    \functype{B'}{B''}$ with $\cut{\toBTree{A'}}{k-1} \eqtypeco
    \cut{\toBTree{B'}}{k-1}$ and $\cut{\toBTree{A''}}{k-1} \eqtypeco
    \cut{\toBTree{B''}}{k-1}$ for every $k > 0$. Then, by inductive hypothesis
    $A' \eqtypemu B'$ and $A'' \eqtypemu B''$. Thus we conclude with
    $\ruleEqmuFunc$, $\functype{A'}{A''} \eqtypemu \functype{B'}{B''}$.

    \item $A = \rectype{V}{A'}$ with $A'$ a non-union type. By definition of
    the interpretation we have $\cut{\toBTree{A}}{k} =
    \cut{\toBTree{\substitute{V}{A}{A'}}}{k} \eqtypeco \cut{\toBTree{B}}{k}$.
    Here we may apply the inductive hypothesis as
    $\card{\irectype}{\substitute{V}{A}{A'}} < \card{\irectype}{A}$. Then,
    $\substitute{V}{\rectype{V}{A'}}{A'} \eqtypemu B$. On the other hand,
    $\rectype{V}{A'} \eqtypemu \substitute{V}{\rectype{V}{A'}}{A'}$ by
    $\ruleEqmuFold$. Finally we conclude with $\ruleEqmuTrans$,
    $\rectype{V}{A'} \eqtypemu B$.
  \end{itemize}
  
  \item If $n + m > 2$. Then the last rule applied to derive
  $\cut{\toBTree{A}}{k} \eqtypeco \cut{\toBTree{B}}{k}$ is necessarily
  $\ruleEqcoUnion$. Then, there exists $f : 1..n \to 1..m, g : 1..m \to 1..n$
  such that $\cut{\tA_i}{k} \eqtypeco \cut{\toBTree{B_{f(i)}}}{k}$ and
  $\cut{\tA_{g(j)}}{k} \eqtypeco \cut{\toBTree{B_j}}{k}$ for every $i \in 1..n,
  j \in 1..m$.
  
  If $\card{\irectype}{A} \neq 0$, then $A = \rectype{V}{A'}$, $\toBTree{A} =
  \toBTree{\substitute{V}{A}{A'}}$ by definition and
  $\card{\irectype}{\substitute{V}{A}{A'}} < \card{\irectype}{A}$ by
  contractivity. Thus we can conclude directly from the inductive hypothesis
  with $\ruleEqmuFold$ and $\ruleEqmuTrans$ as before.
  
  If $\card{\irectype}{A} = 0$, by definition of the interpretation we have $A
  = \maxuniontype{i \in 1..n}{A_i}$ with $\toBTree{A_i} = \tA_i$ for every $i
  \in 1..n$. Hence, $\cut{\toBTree{A_i}}{k} \eqtypeco
  \cut{\toBTree{B_{f(i)}}}{k}$ and $\cut{\toBTree{A_{g(j)}}}{k} \eqtypeco
  \cut{\toBTree{B_j}}{k}$.
  
  Moreover, since $\tA_i, \tB_j \neq \iuniontype$, we are in the same situation
  as case (i) of this proof, so we can assure $A_i \eqtypemu B_{f(i)}$ and
  $A_{g(j)} \eqtypemu B_j$ for every $i \in 1..n, j \in 1..m$.
  
  Finally, we are under the hypothesis of Lem.~\ref{lem:unionEquivalence}, thus
  we conclude $\maxuniontype{i \in 1..n}{A_i} \eqtypemu \maxuniontype{j \in
  1..m}{B_j}$.
\end{enumerate}
\end{proof}



\begin{proposition}
\label{prop:eqtypeSoundnessAndCompleteness}
$A \eqtypemu B$ iff $\toBTree{A} \eqtypeco \toBTree{B}$.
\end{proposition}

\begin{proof}
This proposition follows from previous results shown on
Lem.~\ref{lem:cutEquivalenceCo} and~\ref{lem:cutEquivalenceMu}: $A \eqtypemu B$
iff $\forall k \in \Natural.\cut{\toBTree{A}}{k} \eqtypeco
\cut{\toBTree{B}}{k}$ iff $\toBTree{A} \eqtypeco \toBTree{B}$.
\end{proof}



To prove the correspondence between the subtyping relations we need to verify
that all variable assumptions in the subtyping context can be substituted by
convenient $\mu$-types before applying $\toBTree{\bullet}$.

\begin{lemma}
\label{lem:subtypeSoundnessWithHypothesis}
Let $\Sigma = \set{V_i \subtypemu W_i}_{i \in 1..n}$ be a subtyping context
and $\sigma$ a substitution such that $\dom{\sigma} = \set{V_i, W_i}_{i \in
1..n}$, $\sigma(V_i) = A_i$ and $\sigma(W_i) = B_i$ with $\dom{\sigma} \cap
\fv{\set{A_i, B_i}_{i \in 1..n}} = \varnothing$, $\toBTree{A_i} \subtypeco
\toBTree{B_i}$ and $A_i, B_i \in \Type$ for every $i \in 1..n$.
If $\sequTE{\Sigma}{A \subtypemu B}$, then $\toBTree{\sigma A} \subtypeco
\toBTree{\sigma B}$.
\end{lemma}

\begin{proof}
By induction on $\sequTE{\Sigma}{A \subtypemu B}$ analyzing the last rule
applied.
\begin{itemize}
  \item $\ruleSubmuRefl$: $A = B$ and the result is immediate by reflexivity of
  $\subtypeco$.
  
  \item $\ruleSubmuTrans$: $\sequTE{\Sigma}{A \subtypemu C}$ and
  $\sequTE{\Sigma}{C \subtypemu B}$ for some $C \in \Type$. By inductive
  hypothesis $\toBTree{\sigma A} \subtypeco \toBTree{\sigma C}$ and
  $\toBTree{\sigma C} \subtypeco \toBTree{\sigma B}$ for every $\sigma$
  satisfying the hypothesis of the lemma. Then we conclude by transitivity of
  $\subtypeco$.
  
  \item $\ruleSubmuHyp$: $A = V$ and $B = W$ with $\Sigma = \Sigma', V
  \subtypemu W$. Then $\sigma A = A_n$, $\sigma B = B_n$ and the result is
  immediate since, by hypothesis of the lemma, $\toBTree{A_i} \subtypeco
  \toBTree{B_i}$ for every $i \in 1..n$.
  
  \item $\ruleSubmuEq$: $\sequTE{}{A \eqtypemu B}$ and, since $\eqtypemu$ is a
  congruence, we have $\sequTE{}{\sigma A \eqtypemu \sigma B}$ for every
  substitution. So we can take $\sigma$ satisfying the hypothesis of the lemma.
  Then, by Prop.~\ref{prop:eqtypeSoundnessAndCompleteness}, $\toBTree{\sigma A}
  \eqtypeco \toBTree{\sigma B}$ and we conclude by Lem.~\ref{lem:eqImpliesSub},
  $\toBTree{\sigma A} \subtypeco \toBTree{\sigma B}$.
  
  \item $\ruleSubmuFunc$: $A = \functype{A'}{A''}$ and $B = \functype{B'}{B''}$
  with $\sequTE{\Sigma}{B' \subtypemu A'}$ and $\sequTE{\Sigma}{A'' \subtypemu
  B''}$. By inductive hypothesis we have $\toBTree{\sigma B'} \subtypeco
  \toBTree{\sigma A'}$ and $\toBTree{\sigma A''} \subtypeco \toBTree{\sigma
  B''}$. Then $$
\begin{array}{rcl}
\toBTree{\sigma A} & =          & \toBTree{\functype{\sigma A'}{\sigma A''}} \\
                   & =          & \functype{\toBTree{\sigma A'}}{\toBTree{\sigma A''}} \\
                   & \subtypeco & \functype{\toBTree{\sigma B'}}{\toBTree{\sigma B''}} \\
                   & =          & \toBTree{\functype{\sigma B'}{\sigma B''}} \\
                   & =          & \toBTree{\sigma B}
\end{array} $$
  
  \item $\ruleSubmuComp$: $A = \datatype{D}{A'}$ and $B = \datatype{D'}{B'}$
  with $\sequTE{\Sigma}{D \subtypemu D'}$ and $\sequTE{\Sigma}{A' \subtypemu
  B'}$. Similarly to the previous case we conclude from the inductive
  hypothesis that $\toBTree{\datatype{\sigma D}{\sigma A'}} \subtypeco
  \toBTree{\datatype{\sigma D'}{\sigma B'}}$.
  
  \item $\ruleSubmuUnionL$: $A = \uniontype{A'}{A''}$ with $\sequTE{\Sigma}{A'
  \subtypemu B}$ and $\sequTE{\Sigma}{A'' \subtypemu B}$. By inductive
  hypothesis $\toBTree{\sigma A'} \subtypeco \toBTree{\sigma B}$ and
  $\toBTree{\sigma A''} \subtypeco \toBTree{\sigma B}$. Let $$
\begin{array}{r@{\ =\ }l@{\qquad}r@{\ \neq\ }l}
\toBTree{\sigma A'}  & \maxuniontype{i \in 1..m}{\tA'_i}   & \tA'_i  & \iuniontype \\
\toBTree{\sigma A''} & \maxuniontype{j \in 1..m'}{\tA''_j} & \tA''_j & \iuniontype \\
\toBTree{\sigma B}   & \maxuniontype{k \in 1..l}{\tB_k}    & \tB_k   & \iuniontype
\end{array} $$
Now we need to consider the following situations:
  \begin{enumerate}
    \item $m = m' = l = 1$. Then we conclude directly from the inductive
    hypothesis by applying $\ruleSubcoUnion$,
    $\uniontype{\toBTree{\sigma A'}}{\toBTree{\sigma A''}} =
    \uniontype{\tA'_1}{\tA''_1} \subtypeco \tB_1 = \toBTree{\sigma B}$.
    
    \item $m + l > 2$. Then there exists $f : 1..m \to 1..l$ such that $\tA'_i
    \subtypeco \tB_{f(i)}$ and there are two possible cases:
    \begin{enumerate}
      \item $m' = l = 1$. Then $\tA'_i \subtypeco \tB_1$ (\ie $f$ is a constant
      function) and $\tA''_1 \subtypeco \tB_1$. Then we conclude by
      $\ruleSubcoUnion$ $$\uniontype{\toBTree{\sigma A'}}{\toBTree{\sigma A''}}
      = \uniontype{(\maxuniontype{i \in 1..m}{\tA'_i})}{\tA''_1} \subtypeco
      \tB_1 = \toBTree{\sigma B}$$
      
      \item $m' + l > 2$. Then there exists $g : 1..m' \to 1..l$ such that
      $\tA''_j \subtypeco \tB_{g(j)}$. Once again we conclude by
      $\ruleSubcoUnion$ $$
\begin{array}{rcl}
\uniontype{\toBTree{\sigma A'}}{\toBTree{\sigma A''}} & =          & \uniontype{(\maxuniontype{i \in 1..m}{\tA'_i})}{(\maxuniontype{j \in 1..m'}{\tA''_j})} \\
                                                      & \subtypeco & \maxuniontype{k \in 1..l}{\tB_k} \\
                                                      & =          & \toBTree{\sigma B}
\end{array} $$
    \end{enumerate}
    
    \item The only case left to analyze is $m = l = 1$ and $m' + l > 2$ that
    are similar to one where $m' = l = 1$ and $m + l > 2$.
  \end{enumerate}
  So we conclude that $\toBTree{\sigma A} =
  \uniontype{\toBTree{\sigma A'}}{\toBTree{\sigma A''}} \subtypeco
  \toBTree{\sigma B}$.
  
  \item $\ruleSubmuUnionRL$: $B = \uniontype{B'}{B''}$ with $\sequTE{\Sigma}{A
  \subtypemu B'}$. By inductive hypothesis $\toBTree{\sigma A} \subtypeco
  \toBTree{\sigma B'}$. Let $$
\begin{array}{r@{\ =\ }l@{\qquad}r@{\ \neq\ }l}
\toBTree{\sigma A}   & \maxuniontype{i \in 1..m}{\tA_i}    & \tA_i   & \iuniontype \\
\toBTree{\sigma B'}  & \maxuniontype{j \in 1..l}{\tB'_j}   & \tB'_j  & \iuniontype \\
\toBTree{\sigma B''} & \maxuniontype{k \in 1..l'}{\tB''_k} & \tB''_k & \iuniontype
\end{array} $$
Here there are two possible situations:
  \begin{enumerate}
    \item $m = l = 1$. Then $\tA_1 \subtypeco \tB'_1$ and we conclude by
    $\ruleSubcoUnion$ $$\toBTree{\sigma A} = \tA_1 \subtypeco
    \uniontype{\tB'_1}{(\maxuniontype{k \in 1..l'}{\tB''_k})} =
    \uniontype{\toBTree{\sigma B'}}{\toBTree{\sigma B''}}$$
    \item $m + l > 2$. Then there exists $f : 1..m \to 1..l$ such that $\tA_i
    \subtypeco \tB'_{f(i)}$. We are again in a situation where all the
    conditions for $\ruleSubcoUnion$ hold $$
\begin{array}{rcl}
\toBTree{\sigma A} & =          & \maxuniontype{i \in 1..m}{\tA_i} \\
                   & \subtypeco & \uniontype{(\maxuniontype{j \in 1..l}{\tB'_j})}{(\maxuniontype{k \in 1..l'}{\tB''_k})} \\
                   & =          & \uniontype{\toBTree{\sigma B'}}{\toBTree{\sigma B''}}
\end{array} $$
  \end{enumerate}
  So we conclude that $\toBTree{\sigma A} \subtypeco \uniontype{\toBTree{\sigma B'}}{\toBTree{\sigma B''}} = \toBTree{\sigma B}$.
  
  \item $\ruleSubmuUnionRR$: this case is similar to the previous one, with
  $B = \uniontype{B'}{B''}$ and $\sequTE{\Sigma}{A \subtypemu B''}$.
  
  \item $\ruleSubmuRec$: $A = \rectype{V}{A'}, B = \rectype{W}{B'}$ with
  $\sequTE{\Sigma, V \subtypemu W}{A' \subtypemu B'}, W \notin \fv{A'}$ and $V
  \notin \fv{B'}$. Let $\sigma$ be a substitution satisfying the hypothesis of
  the lemma $$
 \begin{array}{cl}
(1) & \dom{\sigma} = \set{V_i, W_i}_{i \in 1..n} \\
(2) & \sigma(V_i) = A_i \text{ and } \sigma(W_i) = B_i \\
(3) & \set{V_i, W_i}_{i \in 1..n} \cap \fv{\set{A_i, B_i}_{i \in 1..n}} = \varnothing \\
(4) & \toBTree{A_i} \subtypeco \toBTree{B_i}
\end{array} $$
  Now consider $\unfoldf{A}{\sigma}{m}$ and $\unfoldf{B}{\sigma}{m}$ as in
  Lem.~\ref{lem:cutFiniteUnfolding}, recall $$
\begin{array}{rcl}
\unfoldf{A}{\sigma}{0} & \eqdef & \bullet \\
\unfoldf{B}{\sigma}{0} & \eqdef & \bullet
\end{array}
\hspace{.5cm}
\begin{array}{rcl}
\unfoldf{A}{\sigma}{m+1} & \eqdef & (\sigma \uplus \rename{V}{\unfoldf{A}{\sigma}{m}})A' \\
\unfoldf{B}{\sigma}{m+1} & \eqdef & (\sigma \uplus \rename{W}{\unfoldf{B}{\sigma}{m}})B'
\end{array} $$ and also the substitution $\sigma_m = (\sigma \uplus
  \rename{V}{\unfoldf{A}{\sigma}{m}} \uplus
  \rename{W}{\unfoldf{B}{\sigma}{m}})$ for each $m \in \Natural$. Notice that
  $\sigma_m A' = \unfoldf{A}{\sigma}{m+1}$ since $W \notin \fv{A'}$. Similarly,
  $\sigma_m B' = \unfoldf{B}{\sigma}{m+1}$.
  
  It is immediate to see that $\sigma_0$ satisfies the hypothesis of the lemma
  for the extended context $\Sigma, V \subtypemu W$, taking $A_{n+1} =
  \unfoldf{A}{\sigma}{0} = \bullet = \unfoldf{B}{\sigma}{0} = B_{n+1}$. This
  allow us to apply the inductive hypothesis and conclude that
  $\toBTree{\unfoldf{A}{\sigma}{1}} = \toBTree{\sigma_0 A'} \subtypeco
  \toBTree{\sigma_0 B'} = \toBTree{\unfoldf{B}{\sigma}{1}}$, and once again we
  are under the hypothesis of the lemma, this time with $\sigma_1$. Thus,
  directly from the inductive hypothesis (applied as many times as needed) we
  have $\toBTree{\unfoldf{A}{\sigma}{m}} \subtypeco
  \toBTree{\unfoldf{B}{\sigma}{m}}$ for every $m \in \Natural$.
  
  Then, by Lem.~\ref{lem:cutSubtypingCo},
  $\cut{\toBTree{\unfoldf{A}{\sigma}{m}}}{k} \subtypeco
  \cut{\toBTree{\unfoldf{B}{\sigma}{m}}}{k}$ for every $k \in \Natural$.
  Moreover, by Lem.~\ref{lem:cutFiniteUnfolding} we have
  $\cut{\toBTree{\sigma A}}{k} \eqtypeco
  \cut{\toBTree{\unfoldf{A}{\sigma}{k}}}{k}$ and
  $\cut{\toBTree{\unfoldf{B}{\sigma}{k}}}{k} \eqtypeco \cut{\toBTree{\sigma
  B}}{k}$. Finally, by Lem.~\ref{lem:eqImpliesSub} and transitivity of
  subtyping we get $\cut{\toBTree{\sigma A}}{k} \subtypeco \cut{\toBTree{\sigma
  B}}{k}$ and conclude with Lem.~\ref{lem:cutSubtypingCo}.
\end{itemize}
\end{proof}



Finally, as mentioned above, the following proposition and
Lem.~\ref{lem:subtypingIsInvertible} allows us to prove
Prop.~\ref{prop:subtypingIsInvertible}.

\begin{proposition}
\label{prop:subtypeSoundnessAndCompleteness}
$A \subtypemu B$ iff $\toBTree{A} \subtypeco \toBTree{B}$.
\end{proposition}

\begin{proof}
$\Rightarrow)$ This part of the proof follows directly from
Lem.~\ref{lem:subtypeSoundnessWithHypothesis}, taking $\Sigma$ an empty
subtyping context and thus $\sigma$ results in the identity substitution. Hence
from $A \subtypemu B$ we get $\toBTree{A} \subtypeco \toBTree{B}$.

$\Leftarrow)$ For the converse we prove the equivalent result: if $\forall k
\in \Natural.\cut{\toBTree{A}}{k} \subtypeco \cut{\toBTree{B}}{k}$ then, $A
\subtypemu B$. And finally conclude by Lem.~\ref{lem:cutSubtypingCo}.

Let $\cut{\toBTree{A}}{k} \subtypeco \cut{\toBTree{B}}{k}$ for every $k \in
\Natural$. As in the proof for Lem.~\ref{lem:cutEquivalenceMu}, we only
consider the cases where $\card{\irectype}{B} = 0$ and proceed by induction on
the lexicographical extension of the standard order to
$\pair{h(\cut{\toBTree{A}}{k})}{\card{\irectype}{A}}$, analyzing the possible
forms of $A$.\
\begin{itemize}
  \item $A = a$. By definition of of the interpretation and tree truncation we
  have $\cut{\toBTree{A}}{k} = a$ for every $k > 0$. Now, by definition of
  $\subtypeco$, only two rules apply:
  \begin{itemize}
    \item $\ruleSubcoRefl$: in this case we have $\cut{\toBTree{B}}{k} = a =
    B$, by definition of the interpretation, and we conclude with
    $\ruleSubmuRefl$.
    
    \item $\ruleSubcoUnion$: by definition of the interpretation once again, we
    have $B = \maxuniontype{1 \in 1..n}{B_i}$ and $$\cut{\toBTree{a}}{k}
    \subtypeco \maxuniontype{i \in 1..n}{\cut{\toBTree{B_i}}{k}}$$ with
    $\cut{\toBTree{a}}{k} \subtypeco \cut{\toBTree{B_j}}{k} \neq \iuniontype$
    for some $j \in 1..n$, $n > 1$. Now the only applicable rule is
    $\ruleSubcoRefl$, thus $\cut{\toBTree{B_j}}{k} = a = B_j$. Then, by
    $\ruleSubmuRefl$, $\ruleSubmuUnionRL$ and $\ruleSubmuUnionRR$, we conclude
    $A \subtypemu \maxuniontype{i \in 1..n}{B_i}$.
  \end{itemize}
  
  \item $A = \datatype{D}{A'}$. As before, by definition of the interpretation
  and tree truncation with $k > 0$, $\cut{\toBTree{A}}{k} =
  \datatype{\cut{\toBTree{D}}{k-1}}{\cut{\toBTree{A'}}{k-1}} \subtypeco
  \cut{\toBTree{B}}{k}$. The only two possible cases here are:
  \begin{itemize}
    \item $\ruleSubcoComp$: by definition of the interpretation and tree
    truncation once again, we have $B = \datatype{D'}{B'}$ with
    $\cut{\toBTree{D}}{k-1} \subtypeco \cut{\toBTree{D'}}{k-1}$ and
    $\cut{\toBTree{A'}}{k-1} \subtypeco \cut{\toBTree{B'}}{k-1}$. Then, by
    inductive hypotesis, $D \subtypemu D'$ and $A' \subtypemu B'$. Finally we
    conclude by $\ruleSubmuComp$, $\datatype{D}{A'} \subtypemu
    \datatype{D'}{B'}$.
    
    \item $\ruleSubcoUnion$: with a similar analysis as the case
    $\ruleSubcoUnion$ for $A = a$, we have $B = \maxuniontype{i \in 1..n}{B_i}$
    and $$\cut{\toBTree{\datatype{D}{A'}}}{k} \subtypeco \maxuniontype{i \in
    1..n}{\cut{\toBTree{B_i}}{k}}$$ with $\cut{\toBTree{\datatype{D}{A'}}}{k}
    \subtypeco \cut{\toBTree{B_j}}{k} \neq \iuniontype$ for some $j \in 1..n$,
    $n > 1$. Then, by definition of $\subtypeco$, it is necessarily the case
    $B_j = \datatype{D'}{B'}$ with $\cut{\toBTree{D}}{k} \subtypeco
    \cut{\toBTree{D'}}{k}$ and $\cut{\toBTree{A'}}{k} \subtypeco
    \cut{\toBTree{B'}}{k}$. Now, as in the previous case, we have
    $\datatype{D}{A'} \subtypemu B_j$ by inductive hypothesis. Finally, with
    $\ruleSubmuUnionRL$ and $\ruleSubmuUnionRR$, we conclude $\datatype{D}{A'}
    \subtypemu \maxuniontype{i \in 1..n}{B_i}$.
  \end{itemize}
  
  \item $A = \functype{A'}{A''}$. The only two applicable rules here are
  $\ruleSubcoFunc$ and $\ruleSubcoUnion$. Both cases are similar to the ones
  exposed for $\idatatype$, concluding directly from the inductive hypothesis
  and the application of $\ruleSubmuFunc$ in the former while
  $\ruleSubmuUnionRL$ and $\ruleSubmuUnionRR$ are used in the latter.
  
  \item $A = \maxuniontype{i \in 1..n}{A_i}$ with $A_i$ a non-union type for
  every $i \in 1..n$, $n > 1$. This case is slightly simpler than the others as
  the only applicable rule is $\ruleSubcoUnion$. Let $B = \maxuniontype{j \in
  1..m}{B_j}$ with $B_j$ a non-union type for $j \in 1..m$. Note that $m$ is
  not necessarily greater then 1. By definition of the interpretation and tree
  truncation we have, from $\ruleSubcoUnion$, $\exists f : 1..n \to 1..m$ such
  that $\cut{\toBTree{A_i}}{k} \subtypeco \cut{\toBTree{B_{f(i)}}}{k}$ for
  every $i \in 1..n$. Then, by inductive hypothesis, $A_i \subtypemu B_{f(i)}$
  for every $i \in 1..n$. Now, by properly applying $\ruleSubmuUnionRL$ and
  $\ruleSubmuUnionRR$ on each case, we get $A_i \subtypemu B$ for every $i \in
  1..n$. Finally we conclude by multiple applications of $\ruleSubmuUnionL$,
  $\maxuniontype{i \in 1..n}{A_i}$.
  
  \item $A = \rectype{V}{A'}$. Then $\cut{\toBTree{A}}{k} =
  \cut{\toBTree{\substitute{V}{\rectype{V}{A'}}{A'}}}{k} \subtypeco
  \cut{\toBTree{B}}{k}$. By inductive hypothesis, with
  $\card{\irectype}{\substitute{V}{A}{A'}} < \card{\irectype}{A}$, we have
  $\substitute{V}{\rectype{V}{A'}}{A'} \subtypemu B$. On the other hand, by
  $\ruleEqmuFold$ and $\ruleSubmuEq$, we get $\rectype{V}{A'} \subtypemu
  \substitute{V}{\rectype{V}{A'}}{A'}$ and we conclude by $\ruleSubmuTrans$,
  $\rectype{V}{A'} \subtypemu B$.
\end{itemize}
\end{proof}



\begin{proposition}
\label{prop:subtypingIsInvertible}
\begin{enumerate}
  \item If $\datatype{D}{A} \subtypemu \datatype{D'}{A'}$, then $D \subtypemu
  D'$ and $A \subtypemu A'$.
  
  \item If $\functype{A}{B} \subtypemu \functype{A'}{B'}$, then $A' \subtypemu
  A$ and $B \subtypemu B'$.
\end{enumerate}
\end{proposition}

\begin{proof}
This result follows immediately from Lem.~\ref{lem:subtypingIsInvertible} and
Prop.~\ref{prop:subtypeSoundnessAndCompleteness}.
\end{proof}



\subsubsection{Further properties on \texorpdfstring{$\mu$}{u}-types}


We conclude the section with a simple but useful result on the preservation of
the structure of non-union types by means of subtyping. Define the set of
\emphdef{union contexts} as the expressions generated by the following grammar
$$\U ::= \Box \mathrel| \uniontype{\U}{A} \mathrel| \uniontype{A}{\U}$$

\begin{lemma}
\label{lem:noMuOnHeadPosition}
For every type $A \in \Type$ there exists $A' \in \Type$ such that $A \eqtypemu
A'$ and $\card{\irectype}{A'} = 0$. Moreover, if $\card{\irectype}{A} = 0$ then
$A$ and $A'$ have the same outermost type constructor.
\end{lemma}

\begin{proof}
By induction in $\card{\irectype}{A}$.
\begin{itemize}
  \item $\card{\irectype}{A} = 0$: the result is immediate taking $A' = A$.
  Notice that the second part of the statement holds trivially.
  
  \item $\card{\irectype}{A} > 0$: then $A = \rectype{V}{A''}$ and by rule
  $\ruleEqmuFold$ $A \eqtypemu \substitute{V}{A}{A''}$. Since $\mu$-types are
  contractive we have $\card{\irectype}{\substitute{V}{A}{A''}} <
  \card{\irectype}{A}$. Then, by inductive hypothesis, there exists $A' \in
  \Type$ such that $A \eqtypemu A'$, $\card{\irectype}{A'} = 0$. Finally we
  conclude by rule $\ruleEqmuTrans$.
\end{itemize}
\end{proof}




\begin{lemma}\label{lem:supertypesOfNonUnionTypes}
If $\U[A] \subtypemu B$ and $A$ is a non-union type, then there exists a
non-union type $A' \in \Type$ such that
\begin{inparaenum}[(i)]
  \item $B \eqtypemu \U'[A']$;
  \item $A \subtypemu A'$; and
  \item $A$ and $A'$ have the same outermost type constructor.
\end{inparaenum}
\end{lemma}



\begin{proof}
By induction on the union context $\U$. Without loss of generality we can
assume $\card{\irectype}{A} = 0$, by Lem.~\ref{lem:noMuOnHeadPosition}.
\begin{itemize}
  \item $\U = \Box$. We have $A \subtypemu B$. By
  Prop.~\ref{prop:subtypeSoundnessAndCompleteness}, $\toBTree{A} \subtypeco
  \toBTree{B}$ where $\toBTree{A} \neq \iuniontype$ by hypothesis. Let
  $\toBTree{B} = \maxuniontype{i \in 1..n}{\tB_i}$ with $\tB_i \neq
  \iuniontype$ for $i \in i..n$. Note that $\tB_i$ is a subtree of the regular
  tree $\toBTree{B}$, thus it is regular too. Then, for every $i \in 1..n$
  there exists $C_i \in \Type$ such that $\toBTree{C_i} = \tB_i$. Moreover,
  taking $C = \maxuniontype{i \in 1..n}{C_i}$ we have $\toBTree{C} =
  \toBTree{B}$, hence $C \eqtypemu B$ by
  Prop.~\ref{prop:eqtypeSoundnessAndCompleteness}.
  \begin{itemize}
    \item If $n = 1$ (\ie $\toBTree{B} = B_1 \neq \iuniontype$) the only
    applicable rules are $\ruleSubcoRefl$, $\ruleSubcoFunc$ or
    $\ruleSubcoComp$, hence both trees have the same type constructor on the
    root. Applying Lem.~\ref{lem:noMuOnHeadPosition} on $B$ yields a type $A'$
    such that $B \eqtypemu A'$, thus proving the first item with $U' = \Box$.
    This type $A'$ has the same outermost type constructor as $B$, which we
    already saw is the same as $A$, hence proving item (iii). We are left to
    prove the second item. This follows from $A \subtypemu B$, $B \eqtypemu A'$
    by rules $\ruleEqmuTrans$ and $\ruleSubmuEq$.
    
    \item If $n > 1$, then the only applicable rule is $\ruleSubcoUnion$ and we
    have $\toBTree{A} \subtypeco \toBTree{C_j} = \tB_j \neq \iuniontype$ and,
    by Prop.~\ref{prop:subtypeSoundnessAndCompleteness}, $A \subtypemu C_j$ for
    some $j \in 1..n$. Note that both trees must have the same constructor in
    the root since neither of them is a union type
    (Lem.~\ref{lem:subtypingIsInvertible}).
    Then we take the union context $$\U' =
    \uniontype{\uniontype{\uniontype{\uniontype{C_1}{\ldots}}{\Box_j}}{\ldots}}{C_n}$$
    and, by Lem.~\ref{lem:noMuOnHeadPosition}, there exists $A' \in \Type$ such
    that $A' \eqtypemu C_j$, $\card{\irectype}{A'} = 0$ and has the same
    outermost type constructor than $C$. Finally, we have $$B \eqtypemu C
    \eqtypemu \U[A']$$ while $A \subtypemu A'$ and both have the same outermost
    type constructor.
  \end{itemize}

  \item $\U =
  \uniontype{\uniontype{\uniontype{\uniontype{C_1}{\ldots}}{\Box_k}}{\ldots}}{C_m}$
  with $m > 1$, where $C_k$ with $k \in 1..m$ is the position of $\Box$ within $\U$
  (\ie $C_k = A$ in $\U[A]$). We can assume without loss of generality that $C_j$
  is a non-union type for every $j \in 1..m$.
  
  From $\U[A] \subtypemu B$ and
  Prop.~\ref{prop:subtypeSoundnessAndCompleteness} we have $\toBTree{\U[A]}
  \subtypeco \toBTree{B}$. By definition $$\toBTree{\U[A]} =
  \uniontype{\uniontype{\uniontype{\uniontype{\toBTree{C_1}}{\ldots}}{\toBTree{A}}}{\ldots}}{\toBTree{C_m}}$$
  with $\toBTree{C_j} \neq \iuniontype$ for every $j \in 1..m$.
  
  Assume once again $\toBTree{B} = \maxuniontype{i \in 1..n}{\tB_i}$ with $B_i
  \neq \iuniontype$ for $i \in 1..n$. The only subtyping rule that applies
  here is $\ruleSubcoUnion$ since $m > 1$, hence $n + m > 2$. Then there exists
  $f : 1..m \to 1..n$ such that $\toBTree{C_j} \subtypeco \tB_{f(j)}$ for every
  $j \in 1..m$.

  Notice that $\U = \uniontype{\U''}{C_n}$ or $\U = \uniontype{C_1}{\U''}$ for
  some proper union context $\U''$. Hence, by construction $$\U'' =
  \uniontype{\uniontype{\uniontype{\uniontype{C_1}{\ldots}}{\Box_k}}{\ldots}}{C_{m-1}}
  \quad\text{or}\quad \U'' =
  \uniontype{\uniontype{\uniontype{\uniontype{C_2}{\ldots}}{\Box_k}}{\ldots}}{C_m}$$
  In either case, by rule $\ruleSubcoUnion$, we have $\toBTree{\U''[A]}
  \subtypeco \toBTree{B}$, hence $\U''[A] \subtypemu B$ by
  Prop.~\ref{prop:subtypeSoundnessAndCompleteness}.
  
  Finally, we can apply the inductive hypothesis to conclude that $B \eqtypemu
  \U'[A']$ with $A' \in \Type$ a non-union type such that $A \subtypemu A'$ and
  both have the same outermost type constructor. 
\end{itemize}
\end{proof}



