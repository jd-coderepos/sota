\subsubsection{Types as trees}
\label{sec:typingCoinductive}


Type safety, addressed in the Sec.\ref{sec:safety}, also relies on 
enjoying the fundamental property of \emph{invertibility} of non-union types
(\cf Prop.~\ref{prop:subtypingIsInvertible}):
\begin{enumerate}
  \item If , then  and .
  \item If , then  and .
\end{enumerate}


To prove this we appeal to the standard tree interpretation of terms and
formulate an equivalent coinductive definition of equivalence and subtyping
(). For the latter, invertibility of non-union types is proved
coinductively, (Lem.~\ref{lem:subtypingIsInvertible}), entailing Prop.~\ref{prop:subtypingIsInvertible}.

Consider \emphdef{type constructors}  and  together
with \emphdef{type connector}  and the ranked alphabet
.  We write  for
the set of (possibly) \emphdef{infinite types} with symbols in .
This is a standard construction~\cite{terese03,journals/tcs/Courcelle83} given
by the metric completion based on a simple depth function measuring the
distance from the root to the minimum conflicting node in two trees.
Perhaps worth mentioning is that the type connector  does not
contribute to the depth (hence the reason for calling it a connector rather
than a constructor) excluding types consisting of infinite branches of
, such as ,
from . We use meta-variables  to denote elements of
.

\begin{remark}
\label{rem:maximalUnionTypes}
For any , we write  to mean that
,  being the root position of the tree. For
example,  means that  is a \emph{non-union type}. Any
type  can be written as  (dubbed a
\emph{maximal} union type) where  for all 
with , irrespective of how their arguments are associated. All
such associations yield equivalent infinite types in a sense to be made precise
shortly.
\end{remark}




\subsubsection{Equivalence of Infinite Types}


\begin{figure}[t] 
\caption{Equivalence relation for infinite types}
\label{fig:equivalenceSchemesCo}
\end{figure}

\begin{definition}
Infinite type equivalence, written , is defined by the
coinductive interpretation of the schemes of
Fig.~\ref{fig:equivalenceSchemesCo}.
\end{definition}

Note that  is actually a rule scheme, representing all
possible associations within maximal union types  and .
Each instance of the rule states that every  must be equivalent to some
 via a function  and vice versa (with ). Note that the  type connector  is seen to be not only
associative and commutative but also idempotent.

Formally, let  be the functional associated to the rules in
Fig.~\ref{fig:equivalenceSchemesCo}, defined as follows: 

Then . Now we show that it is
indeed an equivalence relation.



\begin{lemma}
 is an equivalence relation (\ie reflexive, symmetric and
transitive).
\end{lemma}

\begin{proof}
The three properties are proved be showing that the sets defining them are
-dense. Then we conclude by the coinductive
principle\footnote{\emph{Coinductive principle}: if  is -dense, then
.} that the properties hold on .
\begin{itemize}
  \item Reflexivity: . Let . We proceed by analyzing the shape
  of :
  \begin{itemize}
    \item . Immediate since  for
    every .
    
    \item . By definition of reflexivity
    . Then .
    
    \item . Similarly to the previous case, we
    have . Hence
    .
    
    \item  with 
    for . Then, since  and
    , we conclude
     by considering 
    (the identity function).
  \end{itemize}
  
  
  \item Symmetry: . We show that .
  
  Let , then
  . By Rem.~\ref{rem:maximalUnionTypes} we
  can consider maximal union types  and we have two separate cases to analyze:
  \begin{enumerate}
    \item If , then both  and  are non-union types. Now we
    proceed by analyzing the shape of :
    \begin{itemize}
      \item . Then  by definition of  and the
      result is immediate since  for every .
      
      \item . Again, by definition, we have
       with . Then  and we conclude .
      
      \item . Similarly,  with . Hence  and we conclude .
    \end{itemize}
    
    \item If not, we have  and only the rule 
    applies. Then  Applying symmetry we get  for every
    . Thus, we conclude .
  \end{enumerate}
  
  
  \item Transitivity: .
  As before, we show that . Let , then there exists  such that
  . Again, we resort to
  Rem.~\ref{rem:maximalUnionTypes} an consider maximal union types 
  \begin{enumerate}
    \item If  (\ie all three are non-union types), we proceed by
    analyzing the shape of :
  \begin{itemize}
	  \item . By definition of ,  and . Then .
	  
	  \item . Once again by definition of
	  ,  with  and  with . Then  and we conclude .
	  
	  \item . Similarly, we have  and  with
	  . By transitivity
	   and .
  \end{itemize}
  
	\item If not (\ie ), we have three different situations to
	consider:
    \begin{inparaenum}[(i)]
      \item  and ;
      \item  and ; or
      \item  and .
    \end{inparaenum}
    In terms of applied rules to derive  and , in the former case the only possibility is  on both
    sides, while in the latter two we have  on one side and any
    of the other three rules (, ,
    ) on the other. Note that this last two cases are symmetric,
    therefore we only analyse cases (i) and (ii) below:
    \begin{enumerate}[(i)]
      \item  and . By definition of  
      Then, we have  for every , and  for every .
      
      Here we have two possible situations. If  (hence ) it
      is necessarily the case  with all three non-union types. Then we can
      safely conclude, by the previous analysis made in case 1, that
      .
      
      If not (\ie ), taking  we
      get .
      Similarly,  for every  with . Finally we conclude by ,
      .

      \item  and . Then, by definition of ,
       is a constant function and we have  for every . On the other hand
       by hypothesis. By transitivity
      once again we get 
      and we conclude with the same constant function , .
    \end{enumerate}
  \end{enumerate}
\end{itemize}
\end{proof}



\begin{lemma}[Equality of non-union types is invertible]
\label{lem:equalityIsInvertible}
Let  be two non-union types.
\begin{enumerate}
  \item If , then .
  \item If , then 
  with  and .
  \item If , then 
  with  and .
\end{enumerate}
\end{lemma}

\begin{proof}
Immediate from the definition of subtyping. Note that there's only one
applicable rule in each case.
\end{proof}


Along the document we often resort to the following definition and properties
of the substitution operator over infinite trees:

\begin{definition}
\label{def:treeSubstitution}
The \emphdef{substitution} of a variable  by a tree  in  (notation
) is defined as: 
\end{definition}



The following lemma provides a more convenient characterisation of the
substitution.

\begin{lemma}
\label{lem:treeSubstitution}
\begin{enumerate}[(i)]
  \item .
  \item  for .
  \item  for .
\end{enumerate}
\end{lemma}

\begin{proof}
The three cases are by analysis of the defined positions.
\begin{enumerate}[(i)]
  \item The only defined position in  is . Then, for every 
  in  we have 
  
  \item The only defined position in  is , thus we have
  . Any other position is
  undefined.
  
  \item Here we have  with . We proceed by analysing the defined positions of
  .
  \begin{itemize}
    \item . Then
    
    \item . Here we have two possibilities:
    \begin{enumerate}
      \item either . Then  and we have 
      \item or . Then  and by definition of
      substitution we have, for every position  in  
    \end{enumerate}
  \end{itemize}
\end{enumerate}
\end{proof}



We show next that the substitution preserves the equivalent relation.

\begin{lemma}
\label{lem:substitutionOfEqtypesCo}
Let  and . Then
.
\end{lemma}

\begin{proof}
Let . We show that  is -dense.

Let . If  the result is immediate by monotonicity of
, since . Then we only present the case where
,  and  with  and .
Assume, without loss of generality 

\begin{enumerate}
  \item If  (\ie ), we analyze the shape
  of :
  \begin{itemize}
    \item . By Lem.~\ref{lem:equalityIsInvertible},  and we
    have two possible cases. If , by Lem.~\ref{lem:treeSubstitution}
    (ii), . If not, by Lem.~\ref{lem:treeSubstitution} (i),
    . Both cases are immediate by definition
    of .
    
    \item . By
    Lem.~\ref{lem:equalityIsInvertible},  with
     and . Then, by definition of
    , we have 
    and . Finally we conclude  since, by
    Lem.~\ref{lem:treeSubstitution} (iii), 

    \item . As before, by
    Lem.~\ref{lem:equalityIsInvertible}, we get  with  and . By definition  we have
     and
    . Thus, we conclude by
    Lem.~\ref{lem:treeSubstitution} (iii), .
  \end{itemize}
  
  \item If , by  we have  Then,
   and
   for every . Once again we
  conclude by definition of  and Lem.~\ref{lem:treeSubstitution}
  (iii), .
\end{enumerate}
\end{proof}


\subsubsection{Subtyping of trees}


In a similar way we have a coinductive characterization of subtyping over trees.

\begin{definition}
Infinite type subtyping, written , is defined by the coinductive
interpretation of the schemes in Fig.~\ref{fig:subtypingSchemesCo}. 
\end{definition}

The most interesting rule in Fig.~\ref{fig:subtypingSchemesCo} is
. Here, for a maximal union type of the form  to be a subtype of a maximal union type , one of the two must have at least one occurrence of the union
type construct () and there must be a function 
such that  for each . 

\begin{figure}[t] 
\caption{Subtyping relation for infinite types}
\label{fig:subtypingSchemesCo}
\end{figure}


\begin{remark}
The rules are derived from those of Fig.~\ref{fig:subtypingSchemesMu}. More
precisely, rules ,  and
 of Fig.~\ref{fig:subtypingSchemesMu} and the observation
that  and  can always be permuted past
.
\end{remark}

As above, the formal definition of the subtyping relation is given by the
associated function  defined next: 

Then . We now address some properties
of subtyping.

\begin{lemma}[Subtyping is a preorder]
 is a preorder (\ie reflexive and transitive).
\end{lemma}

\begin{proof}
This proof is similar to the one presented before for .
\end{proof}



The following notion of invertibility (Lem.~\ref{lem:subtypingIsInvertible}) is
the main result of the present Section and an essential property to prove
Subject Reduction (Prop.~\ref{prop:subjectReduction}) and Progress
(Prop.~\ref{prop:progress}) for the type system proposed in
Sec.~\ref{sec:typingSystem}.

\begin{lemma}[Subtyping of non-union types is invertible]
\label{lem:subtypingIsInvertible}
Let  be non-union types. Suppose .
\begin{enumerate}
  \item If , then .
  \item If , then 
  with  and .
  \item If , then 
  with  and .
\end{enumerate}
\end{lemma}

\begin{remark}
In each of the three items of Lem.~\ref{lem:subtypingIsInvertible} the roles of
 and  can be reversed. 
\end{remark}





\begin{lemma}
\label{lem:eqImpliesSub}
.
\end{lemma}

\begin{proof}
We show that  is
-dense. Let . By
Rem.~\ref{rem:maximalUnionTypes} we can consider maximal union types  and we have two separate cases to analyze:
\begin{enumerate}
  \item If , then both  and  are non-union types. Now we
  proceed by analyzing the shape of :
  \begin{itemize}
    \item . Then, by definition of ,  and the
    result is immediate since  for every .
    
    \item . Again, by definition of ,
    we have  with . Then we conclude by definition
    of , .
    
    \item . Similarly,  with . By symmetry  and we conclude .
  \end{itemize}
  
  \item If not (\ie ), rule  applies. Then  Thus, we conclude with the same function , .
\end{enumerate}
\end{proof}



To prove the correspondence of the coinductive formulation with the inductive
approach, it is convenient to work with finite trees (types). Thus, we
introduce a characterisation of the equivalence and subtyping relations in
terms of finite truncations of infinite trees.

We denote with  the maximal number of adjacent union
type nodes, starting from the root of :  Recall that, by definition of , a type cannot consist of infinitely
many consecutive occurrences of . Thus, the previous inductive
definition is well-founded, as well as the following:

\begin{definition}
\label{def:treeCut}
The \emphdef{truncation} of a tree  at depth  (notation
) is defined inductively\footnote{Using the lexicographical
extension of the standard order to .} as
follows:  where  is a distinguished type
constant used to identify the nodes where the tree was truncated.
\end{definition}

\begin{remark}
\label{rem:cutMaximalUnionTypes}
Given a maximal union type , immediately from
the definition we have .
\end{remark}



\begin{lemma}
\label{lem:cutEquivalenceCo}
 iff .
\end{lemma}

\begin{proof}
 We show that  is -dense.
Let . Then, for every  we have
. Consider maximal union types 
\begin{enumerate}
  \item If  (\ie ), we proceed by
  analyzing the shape of :
  \begin{itemize}
    \item . Then,  for every  and, by
    Lem.~\ref{lem:equalityIsInvertible}, . Hence, 
    and we conclude directly from the definition of ,
    .
    
    \item . Similarly, we have  for every . By
    Lem.~\ref{lem:equalityIsInvertible} once again, we get  with  and
    . Note that for every  we have
    different subtrees  and  but, since
    Lem.~\ref{lem:equalityIsInvertible} refers to tree equality (not
    equivalence) when determining the shape of , it is immediate to see
    from the definition of the truncation that 
    with  and  for every
    . Hence,  and
     for every . Then, by
    definition of ,  and we
    conclude .
    
    \item . Analysis for this case is similar to
    the previous one. From  we get  with 
    and  for every . Then
    we have  and conclude
    .
  \end{itemize}
  
  \item If  we have  and  for every . From
  , by , we get 
  Since  for every , we have
   and  by reflexivity. Thus,  and  for
  every . Then, by definition of ,
   for every , . Finally, we conclude .
\end{enumerate}

 For this part of the proof we show that the converse relation  is -dense. Let . If , by definition of the truncation,  and trivially . We analyze next the cases where  given
that, by definition of , . Once again we consider
maximal union types  and analyze separately the cases where both  and  ar
non-union types.
\begin{enumerate}
  \item If  we a look at the shape of :
  \begin{itemize}
    \item . By Lem.~\ref{lem:equalityIsInvertible},  and
     for every . Then we conclude by definition of
    , .
    
    \item . By Lem.~\ref{lem:equalityIsInvertible},
     with  and . Then, by definition of ,
     and we conclude
    .
    
    \item . Similarly to the previous case, we
    have  with  and . Then  and
    .
  \end{itemize}
  
  \item If , by  we have 
  Then, by definition of ,  for every . Thus,
  we conclude by resorting to Rem.~\ref{rem:cutMaximalUnionTypes},
  .
\end{enumerate}
\end{proof}



\begin{lemma}
\label{lem:cutSubtypingCo}
 iff .
\end{lemma}

\begin{proof}
 Similarly to the previous lemma, we prove this part by showing
that  is -dense. By hypothesis
we have  for every . As
before we consider maximal union types and analyze separately the case for
non-union types 
\begin{enumerate}
  \item If  (\ie ), we proceed by
  analyzing the shape of :
  \begin{itemize}
    \item . Then,  for every  and, by
    Lem.~\ref{lem:subtypingIsInvertible}, . Hence, 
    and we conclude directly from the definition of ,
    .
    
    \item . Similarly, we have  for every . By
    Lem.~\ref{lem:subtypingIsInvertible} once again, we get  with  and
    . As in the previous lemma, in this case
    we have different subtrees  and  for every  but, by
    resorting to tree equality on Lem.~\ref{lem:subtypingIsInvertible} and the
    definition of the truncation, we can assure that  with  and  for every . Hence,  and  for every
    . Then, by definition of ,  and we conclude .
    
    \item . Analysis for this case is similar to
    the previous one. From  we get  with 
    and  for every . Note
    that, by Lem.~\ref{lem:subtypingIsInvertible}, subtyping order on the
    domains is inverted. Then we have  and conclude .
  \end{itemize}
  
  \item If  we have  and  for every . From , by , we get 
  Since  for every , we also have
   and  by reflexivity. Thus,  and  for
  every . Then, by definition of ,
   for every , . Finally, we conclude .
\end{enumerate}


 As before, we define  and show that is -dense to prove this part of the
lemma. Again, if  the result is immediate, so lets focus on the case
where .

Let . We assume, without loss of generality,  and 
are maximal union types.

If  it is the case of  and we have  such that  for every .
Then, by definition we have 
and conclude .

On the other hand, if  we analyze the form of :
\begin{enumerate}
  \item . By Lem.~\ref{lem:subtypingIsInvertible} we have 
  and the result is immediate.
  
  \item . By Lem.~\ref{lem:subtypingIsInvertible},
   with  and . Then we have  for every , and
  conclude by definition of , .
  
  \item . Similarly to the previous case we have
   with  and . Then we conclude by definition of  and 
  that .
\end{enumerate}
\end{proof}



\subsubsection{Correspondence between \texorpdfstring{}{u}-types and infinite types}


Contractive -types
characterize~\cite{journals/tcs/Courcelle83,DBLP:journals/toplas/AmadioC93,DBLP:journals/fuin/BrandtH98,Pierce:2002:TPL:509043}
a proper subset of  known as the \emphdef{regular trees} (trees whose
set of distinct subtrees is finite) and denoted .
Given a contractive -type ,  is the regular tree obtained
by completely unfolding all occurrences of  in .
Def.~\ref{def:treeFunction} below extends that of~\cite{Pierce:2002:TPL:509043}
to union and data types. It is well-founded, relying on the lexicographical
extension of the standard order to ,
where  is the number of occurrences of the 
type constructor at the head position of .

\begin{definition}
\label{def:treeFunction}
The function , mapping -types to types, is defined inductively as follows: 
\end{definition}



Commutation of  with substitutions is as expected. 

\begin{lemma}
\label{lem:substitutionOfTrees}
.
\end{lemma}

\begin{proof}
We actualy prove the equivalente result  and conclude by
reflexivity of  and Lem.~\ref{lem:cutEquivalenceCo}.

The proof is by induction on the lexicographical extension of the standard
order to
,
where  is the height function for finite trees
and  is the number of occurrences of both
 and  at the head of .

We proceed by analyzing the possible forms of  and assuming  since
the result for that case is immediate.
\begin{itemize}
  \item : then  by
  Lem.~\ref{lem:treeSubstitution}.
  
  \item : then  by
  definition of the interpretation and Lem.~\ref{lem:treeSubstitution}.
  
  \item : then 
  
  \item : this case is similar to the previous one.
  
  \item : analysis for this case is similar to the
  previous ones but notice that we get the same  when resorting to
  Def.~\ref{def:treeCut} (instead of ) before applying the inductive
  hypothesis. However, we are in conditions to apply it anyway since
   Hence, it
  is safe to conclude .
  
  \item : without loss of generality we can assume
  \footnote{We use the predicate  to
  mean that there is no collition at all between  and the variables in
   (\ie ).}. Then  Here we are in condition to apply the indutive hypothesis since
   by contractiveness.
\end{itemize}
\end{proof}



The finite unfolding of a contractive -type  consists of recursively
replacing all occurrences of a bounded variable  by  itself a finite
number of times. We formalize a slightly more general variation of this idea in
the following lemma and prove its relation with .

\begin{lemma}
\label{lem:cutFiniteUnfolding}
Let ,  any other -type and  a
substitution. Define  Then, .
\end{lemma}

\begin{proof}
By induction on . We assume without loss of generality that .
\begin{itemize}
  \item . Then  by definition of the truncation.
  
  \item . By inductive hypothesis we have
  . Moreover, since  is
  contractive, the first appearance of  in  is at depth . So we
  have  and, by Lem.~\ref{lem:substitutionOfEqtypesCo}
  and~\ref{lem:substitutionOfTrees}, we may conclude 
\end{itemize}
\end{proof}

\begin{remark}
\label{rem:cutFiniteUnfolding}
It follows immediately from the previous result that for every ,
.
\end{remark}



One of the main results of this section is the correspondence between the
equivalence relations  and  via the function
. It follows from the lemma below that relates two
-equivalent types with the truncation of their respective trees:

\begin{lemma}
\label{lem:cutEquivalenceMu}
 iff .
\end{lemma}

\begin{proof}
 This part of the proof is by induction on 
analyzing the last rule applied. Note that  by definition of the truncation, so we only analyze the
cases where .
\begin{itemize}
  \item : then  and we conclude by reflexivity of
  ,  for every
  .

  \item : then  and . By
  inductive hypothesis 
  and  for every .
  Then we conclude by transitivity of .
  
  \item : then . By inductive hypothesis
   for every  and
  we conclude by symmetry of .
  
  \item : then 
  with  and . By inductive hypothesis
   and
   for every .
  Then 
  
  \item : then 
  with  and . This case is similar to the
  previous one. We conclude directly from the inductive hypothesis and the
  definition of the truncation 
  
  \item : then . In this case we need
  to take into account that  may be a union type as well and, when working
  with , we must consider maximal union types. Let
   and
   with . It is immedate to see from the equality above that
   and  for every . Finally we
  conclude by reflexivity of  and  
  
  \item : then  and . As in the previous case consider  and  with . Here , hence
  . Moreover, assuming  is the last component of  (), we have  if , and  if . Thus, we conclude by reflexivity of 
  and , .
  
  \item : then  and . Considering maximal union types as
  before we have  and
   with  and . In this case we may conclude by resorting to
  the identity function in , since . Thus, by reflexivity
  and , .
  
  \item : then  with  and . By
  inductive hypothesis  and  for every . Assume, without loss of
  generality 

  If , there exists ,  such
  that  and . If not
  (\ie ), we simply take .
  
  Likewise, for  and  we have  and there exists , 
  such that  and .

  Finally, since , we can apply  to
  conclude 
  
  \item : then  with
  . By inductive hypothesis  and, by Lem.~\ref{lem:cutEquivalenceCo}, .
  
  Now we consider the definition of  and  as in
  Lem.~\ref{lem:cutFiniteUnfolding} with  and
  . We claim that  for every . To prove this we proceed by
  induction on 
  \begin{itemize}
    \item . Then we have  that holds by hypothesis.
    
    \item . By reflexivity
    . Also, by inductive
    hypothesis,  and,
    by hypothesis, . Then we can apply
    Lem.~\ref{lem:substitutionOfEqtypesCo} and~\ref{lem:substitutionOfTrees},
    and conclude 
  \end{itemize}
  
  Finally, by Lem.~\ref{lem:cutEquivalenceCo},  for every . Thus we
  conclude by Lem.~\ref{lem:cutFiniteUnfolding} 
  
  \item : then  and . The result is immediate by definition
  of the interpretation, . Then
   for every  by reflexivity of .
  
  \item : then  is contractive and . By inductive hypothesis and
  Lem.~\ref{lem:cutEquivalenceCo}, .
  
  As in the previous case we consider  from
  Lem.~\ref{lem:cutFiniteUnfolding}, this time with . Now we show  for every , by induction on 
  
  \begin{itemize}
    \item . This case is immediate since  by definition.
    
    \item . Then, by definition and Lem.~\ref{lem:substitutionOfTrees},
    . By inductive
    hypothesis we know  and, by
    Lem.~\ref{lem:substitutionOfEqtypesCo}, . Finally we conclude by applying
    Lem.~\ref{lem:substitutionOfTrees} and transitivity of  with
    hypothesis 
    
  \end{itemize}
  
  Then, by Lem.~\ref{lem:cutEquivalenceCo},  for every . On the other
  hand, by Lem.~\ref{lem:cutFiniteUnfolding}, we know
  . Thus, we
  conclude 
\end{itemize}


 Let  for
every . Given  it is immediate to see that
 while , by definition of the interpretation and
 respectively. Moreover, since -types are contractive, we
can assure that . By a simple induction on  we can
prove that for every  there exists  such that
,  and . It
is important to note that we are resorting to tree equality on this argument.
Thus, without loss of generality, we consider during the proof only the cases
where .

This proof is by induction on the lexicographical extension of the standard
order to , where  is the height function for finite trees. We proceed
by analyzing the possible forms of .

Given  we can assume  by Rem.~\ref{rem:maximalUnionTypes}. Moreover, since
 and by definition of the interpretation, we have
 with  for every  (note that  is a non-union type for every ).

Then, we can divide this proof in two cases, either
\begin{inparaenum}[(i)]
  \item  and  are both non-union types and thus ; or
  \item at least one of them is a union type (\ie ).
\end{inparaenum}

\begin{enumerate}[(i)]
  \item If . Here we analyze the shape of :
  \begin{itemize}
    \item . Then  for every  and, by
    Lem.~\ref{lem:equalityIsInvertible}, . Thus, by definition of the interpretation and tree truncation with
    the assumption , we have  and conclude with
    .
    
    \item . Here we have  for every  and, by Lem.~\ref{lem:equalityIsInvertible} once again,
     with
     and . With a similar analysis to the one made in
    Lem.~\ref{lem:cutEquivalenceCo}, by definition of the interpretation and
    tree truncation with the assumption , we can
    assure that  such that  and  for every
    . Then, we have  and  and we can apply the inductive hypothesis to get
     and . Finally we conclude by
    , .

    \item . Analysis for this case is similar to the
    previous one. From  we get  with  and  for every . Then, by inductive hypothesis
     and . Thus we conclude with
    , .

    \item  with  a non-union type. By definition of
    the interpretation we have .
    Here we may apply the inductive hypothesis as
    . Then,
    . On the other hand,
     by
    . Finally we conclude with ,
    .
  \end{itemize}
  
  \item If . Then the last rule applied to derive
   is necessarily
  . Then, there exists 
  such that  and
   for every .
  
  If , then ,  by definition and
   by
  contractivity. Thus we can conclude directly from the inductive hypothesis
  with  and  as before.
  
  If , by definition of the interpretation we have  with  for every . Hence,  and .
  
  Moreover, since , we are in the same situation
  as case (i) of this proof, so we can assure  and
   for every .
  
  Finally, we are under the hypothesis of Lem.~\ref{lem:unionEquivalence}, thus
  we conclude .
\end{enumerate}
\end{proof}



\begin{proposition}
\label{prop:eqtypeSoundnessAndCompleteness}
 iff .
\end{proposition}

\begin{proof}
This proposition follows from previous results shown on
Lem.~\ref{lem:cutEquivalenceCo} and~\ref{lem:cutEquivalenceMu}: 
iff  iff .
\end{proof}



To prove the correspondence between the subtyping relations we need to verify
that all variable assumptions in the subtyping context can be substituted by
convenient -types before applying .

\begin{lemma}
\label{lem:subtypeSoundnessWithHypothesis}
Let  be a subtyping context
and  a substitution such that ,  and  with ,  and  for every .
If , then .
\end{lemma}

\begin{proof}
By induction on  analyzing the last rule
applied.
\begin{itemize}
  \item :  and the result is immediate by reflexivity of
  .
  
  \item :  and
   for some . By inductive
  hypothesis  and
   for every 
  satisfying the hypothesis of the lemma. Then we conclude by transitivity of
  .
  
  \item :  and  with . Then ,  and the result is
  immediate since, by hypothesis of the lemma,  for every .
  
  \item :  and, since  is a
  congruence, we have  for every
  substitution. So we can take  satisfying the hypothesis of the lemma.
  Then, by Prop.~\ref{prop:eqtypeSoundnessAndCompleteness},  and we conclude by Lem.~\ref{lem:eqImpliesSub},
  .
  
  \item :  and 
  with  and . By inductive hypothesis we have  and . Then 
  
  \item :  and 
  with  and . Similarly to the previous case we conclude from the inductive
  hypothesis that .
  
  \item :  with  and . By inductive
  hypothesis  and
  . Let 
Now we need to consider the following situations:
  \begin{enumerate}
    \item . Then we conclude directly from the inductive
    hypothesis by applying ,
    .
    
    \item . Then there exists  such that  and there are two possible cases:
    \begin{enumerate}
      \item . Then  (\ie  is a constant
      function) and . Then we conclude by
       
      
      \item . Then there exists  such that
      . Once again we conclude by
       
    \end{enumerate}
    
    \item The only case left to analyze is  and  that
    are similar to one where  and .
  \end{enumerate}
  So we conclude that .
  
  \item :  with . By inductive hypothesis . Let 
Here there are two possible situations:
  \begin{enumerate}
    \item . Then  and we conclude by
     
    \item . Then there exists  such that . We are again in a situation where all the
    conditions for  hold 
  \end{enumerate}
  So we conclude that .
  
  \item : this case is similar to the previous one, with
   and .
  
  \item :  with
   and . Let  be a substitution satisfying the hypothesis of
  the lemma 
  Now consider  and  as in
  Lem.~\ref{lem:cutFiniteUnfolding}, recall  and also the substitution  for each . Notice that
   since . Similarly,
  .
  
  It is immediate to see that  satisfies the hypothesis of the lemma
  for the extended context , taking . This
  allow us to apply the inductive hypothesis and conclude that
  , and once again we
  are under the hypothesis of the lemma, this time with . Thus,
  directly from the inductive hypothesis (applied as many times as needed) we
  have  for every .
  
  Then, by Lem.~\ref{lem:cutSubtypingCo},
   for every .
  Moreover, by Lem.~\ref{lem:cutFiniteUnfolding} we have
   and
  . Finally, by Lem.~\ref{lem:eqImpliesSub} and transitivity of
  subtyping we get  and conclude with Lem.~\ref{lem:cutSubtypingCo}.
\end{itemize}
\end{proof}



Finally, as mentioned above, the following proposition and
Lem.~\ref{lem:subtypingIsInvertible} allows us to prove
Prop.~\ref{prop:subtypingIsInvertible}.

\begin{proposition}
\label{prop:subtypeSoundnessAndCompleteness}
 iff .
\end{proposition}

\begin{proof}
 This part of the proof follows directly from
Lem.~\ref{lem:subtypeSoundnessWithHypothesis}, taking  an empty
subtyping context and thus  results in the identity substitution. Hence
from  we get .

 For the converse we prove the equivalent result: if  then, . And finally conclude by Lem.~\ref{lem:cutSubtypingCo}.

Let  for every . As in the proof for Lem.~\ref{lem:cutEquivalenceMu}, we only
consider the cases where  and proceed by induction on
the lexicographical extension of the standard order to
, analyzing the possible
forms of .\
\begin{itemize}
  \item . By definition of of the interpretation and tree truncation we
  have  for every . Now, by definition of
  , only two rules apply:
  \begin{itemize}
    \item : in this case we have , by definition of the interpretation, and we conclude with
    .
    
    \item : by definition of the interpretation once again, we
    have  and  with
    
    for some , . Now the only applicable rule is
    , thus . Then, by
    ,  and , we conclude
    .
  \end{itemize}
  
  \item . As before, by definition of the interpretation
  and tree truncation with , . The only two possible cases here are:
  \begin{itemize}
    \item : by definition of the interpretation and tree
    truncation once again, we have  with
     and
    . Then, by
    inductive hypotesis,  and . Finally we
    conclude by , .
    
    \item : with a similar analysis as the case
     for , we have 
    and  with  for some ,
    . Then, by definition of , it is necessarily the case
     with  and . Now, as in the previous case, we have
     by inductive hypothesis. Finally, with
     and , we conclude .
  \end{itemize}
  
  \item . The only two applicable rules here are
   and . Both cases are similar to the ones
  exposed for , concluding directly from the inductive hypothesis
  and the application of  in the former while
   and  are used in the latter.
  
  \item  with  a non-union type for
  every , . This case is slightly simpler than the others as
  the only applicable rule is . Let  with  a non-union type for . Note that  is
  not necessarily greater then 1. By definition of the interpretation and tree
  truncation we have, from ,  such
  that  for
  every . Then, by inductive hypothesis, 
  for every . Now, by properly applying  and
   on each case, we get  for every . Finally we conclude by multiple applications of ,
  .
  
  \item . Then . By inductive hypothesis, with
  , we have
  . On the other hand, by
   and , we get  and we conclude by ,
  .
\end{itemize}
\end{proof}



\begin{proposition}
\label{prop:subtypingIsInvertible}
\begin{enumerate}
  \item If , then  and .
  
  \item If , then  and .
\end{enumerate}
\end{proposition}

\begin{proof}
This result follows immediately from Lem.~\ref{lem:subtypingIsInvertible} and
Prop.~\ref{prop:subtypeSoundnessAndCompleteness}.
\end{proof}



\subsubsection{Further properties on \texorpdfstring{}{u}-types}


We conclude the section with a simple but useful result on the preservation of
the structure of non-union types by means of subtyping. Define the set of
\emphdef{union contexts} as the expressions generated by the following grammar


\begin{lemma}
\label{lem:noMuOnHeadPosition}
For every type  there exists  such that  and . Moreover, if  then
 and  have the same outermost type constructor.
\end{lemma}

\begin{proof}
By induction in .
\begin{itemize}
  \item : the result is immediate taking .
  Notice that the second part of the statement holds trivially.
  
  \item : then  and by rule
   . Since -types are
  contractive we have . Then, by inductive hypothesis, there exists  such that , . Finally we
  conclude by rule .
\end{itemize}
\end{proof}




\begin{lemma}\label{lem:supertypesOfNonUnionTypes}
If  and  is a non-union type, then there exists a
non-union type  such that
\begin{inparaenum}[(i)]
  \item ;
  \item ; and
  \item  and  have the same outermost type constructor.
\end{inparaenum}
\end{lemma}



\begin{proof}
By induction on the union context . Without loss of generality we can
assume , by Lem.~\ref{lem:noMuOnHeadPosition}.
\begin{itemize}
  \item . We have . By
  Prop.~\ref{prop:subtypeSoundnessAndCompleteness},  where  by hypothesis. Let
   with  for . Note that  is a subtree of the regular
  tree , thus it is regular too. Then, for every 
  there exists  such that . Moreover,
  taking  we have , hence  by
  Prop.~\ref{prop:eqtypeSoundnessAndCompleteness}.
  \begin{itemize}
    \item If  (\ie ) the only
    applicable rules are ,  or
    , hence both trees have the same type constructor on the
    root. Applying Lem.~\ref{lem:noMuOnHeadPosition} on  yields a type 
    such that , thus proving the first item with .
    This type  has the same outermost type constructor as , which we
    already saw is the same as , hence proving item (iii). We are left to
    prove the second item. This follows from , 
    by rules  and .
    
    \item If , then the only applicable rule is  and we
    have  and,
    by Prop.~\ref{prop:subtypeSoundnessAndCompleteness},  for
    some . Note that both trees must have the same constructor in
    the root since neither of them is a union type
    (Lem.~\ref{lem:subtypingIsInvertible}).
    Then we take the union context 
    and, by Lem.~\ref{lem:noMuOnHeadPosition}, there exists  such
    that ,  and has the same
    outermost type constructor than . Finally, we have  while  and both have the same outermost
    type constructor.
  \end{itemize}

  \item 
  with , where  with  is the position of  within 
  (\ie  in ). We can assume without loss of generality that 
  is a non-union type for every .
  
  From  and
  Prop.~\ref{prop:subtypeSoundnessAndCompleteness} we have . By definition 
  with  for every .
  
  Assume once again  with  for . The only subtyping rule that applies
  here is  since , hence . Then there exists
   such that  for every
  .

  Notice that  or  for
  some proper union context . Hence, by construction 
  In either case, by rule , we have , hence  by
  Prop.~\ref{prop:subtypeSoundnessAndCompleteness}.
  
  Finally, we can apply the inductive hypothesis to conclude that  with  a non-union type such that  and
  both have the same outermost type constructor. 
\end{itemize}
\end{proof}



