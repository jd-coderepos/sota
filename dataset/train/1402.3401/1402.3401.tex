\documentclass{sig-alternate-2013}




\newfont{\mycrnotice}{ptmr8t at 7pt}
\newfont{\myconfname}{ptmri8t at 7pt}
\let\crnotice\mycrnotice \let\confname\myconfname 

\makeatletter
\let\@copyrightspace\relax
\makeatother


\clubpenalty=10000 
\widowpenalty = 10000



\usepackage{microtype} \usepackage{graphicx}
\usepackage{times}
\usepackage{subfigure}

\usepackage{xspace}
\usepackage[bf]{caption}


\usepackage{booktabs}

\usepackage{paralist}
\usepackage{enumitem}



\usepackage{subfigure}
\usepackage{multirow}
\usepackage[hang,flushmargin]{footmisc}
\usepackage{algpseudocode}
\usepackage[hyphens]{url}
\usepackage{balance}
\usepackage{color}



\renewcommand{\footnotesize}{\fontsize{9}{10}\selectfont}




\makeatletter
\def\url@leostyle{\@ifundefined{selectfont}{\def\UrlFont{}}{\def\UrlFont{}}}
\makeatother	
\urlstyle{leo}

\usepackage{breakurl}
\usepackage{fancyvrb}
\usepackage{amssymb}
\newcommand{\beri}[1]{{\color{blue}{#1}}}

\newcommand{\red}[1] {{\color{black}{#1}}}
\newcommand{\mat}[1] {{\color{blue}{#1}}}
\newcommand{\descr}[1]{\smallskip\noindent{\bf #1}}
\def\obs{\textit{Observed\xspace}}
\def\den{\textit{Denied\xspace}}
\def\prox{\textit{Proxied\xspace}}
\def\acc{\textit{Accepted\xspace}}
\def\df{\xspace}
\def\dd{\xspace}
\def\dc{\xspace}

\def\du{\xspace}
\def\ds{\xspace}
\def\dipv4{\xspace}
 \def\cd{\xspace}
 \def\lsyriacom{\xspace}
 \def\loni{\xspace}

\usepackage{listings}
\lstset{breaklines=true} 
\newcommand{\policydenied}{\emph{policy\_denied}\xspace}
\newcommand{\policyredirect}{\emph{policy\_redirect}\xspace}
\newcommand{\tcperror}{\emph{tcp\_error}\xspace}
\newcommand{\invalidrequest}{\emph{invalid\_request}\xspace}
\newcommand{\dnsserverfailure}{\emph{dns\_server\_failure}\xspace}
\newcommand{\dnsunresolvedhostname}{\emph{dns\_unresovled\_hostname}\xspace}
\usepackage{paralist}






\pagenumbering{arabic}

\sloppy

\begin{document}




\title{Censorship in the Wild:\hspace{-0.1cm} Analyzing Internet Filtering in Syria}

\numberofauthors{6} 
\author{
\alignauthor{
Abdelberi Chaabane\\
       \affaddr{INRIA Rh\^{o}ne-Alpes}\\
       \affaddr{Montbonnot, France}}
\alignauthor{
Terence Chen\\
       \affaddr{NICTA}\\
       \affaddr{Sydney, Australia}}
\alignauthor{
Mathieu Cunche\\
       \affaddr{University of Lyon \& INRIA}\\
       \affaddr{Lyon, France}}
\and
\alignauthor{
Emiliano De Cristofaro\\
       \affaddr{University College London}\\
       \affaddr{London, United Kingdom}}
\alignauthor{
Arik Friedman\\
       \affaddr{NICTA}\\
       \affaddr{Sydney, Australia}}
\alignauthor{
Mohamed Ali Kaafar\\
       \affaddr{NICTA \& INRIA Rh\^{o}ne-Alpes}\\
       \affaddr{Sydney, Australia}}
}




\maketitle

\begin{abstract}
Internet censorship is enforced by numerous governments worldwide, however, due to the lack of publicly available information, as well as the inherent risks of performing active measurements, it is often hard for the research community to investigate censorship practices in the wild. Thus, the leak of 600GB worth of logs from 7 Blue Coat SG-9000 proxies, deployed in Syria to filter Internet traffic at a country scale, represents a unique opportunity to provide a detailed snapshot of a real-world censorship ecosystem. 

This paper presents the methodology and the results of a measurement analysis of the leaked Blue Coat logs, revealing a relatively stealthy, yet quite targeted, censorship. We find that traffic is filtered in several ways: using IP addresses and domain names to block subnets or websites, and keywords or categories to target specific content. We show that keyword-based censorship produces some collateral damage as many requests are blocked even if they do not relate to sensitive content.  We also discover that Instant Messaging is heavily censored, while filtering of social media is limited to specific pages. Finally, we show that Syrian users try to evade censorship by using web/socks proxies, Tor, VPNs, and BitTorrent. To the best of our knowledge, our work provides the first analytical look into Internet filtering in Syria.
\end{abstract}








\section{Introduction}
As the relation between society and technology evolves, so does censorship---the practice of suppressing ideas and information that certain individuals, groups or government officials may find objectionable, dangerous, or detrimental.
Censors increasingly target access to, and dissemination of, electronic information, for instance, 
aiming to restrict freedom of speech, control knowledge available to the masses, or enforce religious/ethical principles. 

Even though the research community dedicated a lot of attention to censorship and its circumvention, knowledge and understanding of filtering technologies is inherently limited, as it is challenging and risky to conduct measurements from countries operating censorship, while logs pertaining to filtered traffic are obviously hard to come by.
Prior work has analyzed censorship practices in China~\cite{king2012censorship,knockel2011three,park2010empirical,winter2012great,xu2011internet}, Iran~\cite{anderson2012hidden,aryan2013internet,verkamp2012inferring}, 
Pakistan~\cite{nabi2013anatomy}, and a few Arab countries~\cite{dalek2013method},
however, mostly based on probing, i.e., {\em inferring} what information is being censored by generating requests and observing what content is blocked.  While providing valuable insights, these methods suffer from two main limitations: (1) only a limited number of requests can be observed, thus providing a skewed representation of the censorship policies due to the inability to enumerate all censored keywords, and (2) it is hard to assess the actual extent of the censorship, e.g., 
what kind/proportion of the overall traffic is being censored. 
	

\descr{Roadmap.} In this paper, we present a measurement analysis of Internet filtering in Syria: we study a set of logs extracted from 7 Blue Coat SG-9000 proxies, which were deployed to monitor, filter and block traffic of Syrian users. The logs (600GB worth of data) were leaked by a ``hacktivist'' group, Telecomix, in October 2011, and relate to a period of 9 days between July and August 2011~\cite{leak}.
By analyzing the logs, we provide a detailed snapshot of how censorship was operated in Syria. As opposed to probing-based methods, the analysis of actual logs allows us to extract information about processed requests for \textit{both} censored and allowed traffic and provide a detailed snapshot of Syrian censorship practices. 


\descr{Main Findings.}  Our measurement-based analysis uncovers several interesting findings. First, we observe that a few different techniques are employed by Syrian authorities: IP-based filtering to block access to entire subnets (e.g., Israel), domain-based to block specific websites, keyword-based to target specific kinds of traffic (e.g., filtering-evading technologies, such as socks proxies), and category-based to target specific content and pages.
As a side-effect of keyword-based censorship, many requests are blocked even if they do not relate to any sensitive content or anti-censorship technologies. Such collateral damage affects Google toolbar's queries as well as a few ads delivery networks that are blocked as they generate requests containing the word {\em proxy}.





Also, the logs highlight that Instant Messaging software (like Skype) is heavily censored, while filtering of social media is limited to specific pages. Actually, most social networks (e.g., Facebook and Twitter) are not blocked, but only certain pages/groups are (e.g., the Syrian Revolution Facebook page).
We find that proxies have specialized roles and/or slightly different configurations, as some of them tend to censor more traffic than others. For instance, one particular proxy blocks Tor traffic for several days, while other proxies do not.
Finally, we show that Syrian Internet users not only try to evade censorship and surveillance using well-known web/socks proxies, Tor, and VPN software, but also use P2P file sharing software (BitTorrent) to fetch censored content. 





Our analysis shows that, compared to other countries (e.g., China and Iran), censorship in Syria seems to be less pervasive, yet quite targeted. Syrian censors particularly target Instant Messaging, information related to the political opposition (e.g., pages related to the ``Syrian Revolution''), and Israeli subnets.
Arguably, less evident censorship does not necessarily mean minor information control or less ubiquitous surveillance. In fact, Syrian users seem to be aware of this and do resort to censorship- and surveillance-evading software, as we show later in the paper, which seems to confirm reports about Syrian users enganging in self-censorship to avoid being arrested~\cite{selfcensorship,bbc,oni}.  


\descr{Contributions.} To the best of our knowledge, we provide the first detailed snapshot of Internet filtering in Syria and the first set of large-scale measurements of actual filtering proxies' logs.
We show how censorship was operated via a statistical overview of censorship activities and an analysis of temporal patterns, proxy specializations, and filtering of social network sites. Finally, we provide some details on the usage of surveillance- and  censorship-evading tools.


\descr{Remarks.} Logs studied in this paper date back to July-August 2011, thus, our work is not intended to provide insights to the {\em current} situation in Syria as censorship might have evolved in the last two years. According to Bloomberg~\cite{bloomberg}, Syria has invested \\ldots\sim\pm\alpha\neq<0.5\%\neq\in\not\inD_{full}D_{sample}D_{user}D_{denied})D_{full}10^{-5}D_{sample}D_{sample}A_iB_iiAB\mathcal{C}\mathcal{A}w\mathcal{C}N_\mathcal{C}N_\mathcal{A}w\mathcal{C}\mathcal{A}N_\mathcal{C} >> 1N_\mathcal{A} = 0\mathcal{C}wwTor_{http}Tor_ {onion}<>Tor_{http}Tor_{http}Tor_{onion}Tor_{onion}Tor_{http}kkk|\cdot|R_{filter}R_{filter}(k)kR_{filter}(k)kR_{filter}$ shows that the censorship is inconsistent. In fact, the beginning of this plot is either dominated by red circles (all traffic is allowed) or few small peaks showing a very mild censorship. Then few successive peaks with a high variance of blockage appear for several hours, followed by a lull at the night of 03 of August. The same scenario is repeated several times, alternating between aggressive censorship and more mild period. This behavior  is hard to explain as some IPs are alternating between blocked and allowed status. One explanation might be a testing phase that is carried in a single router and for a short period of time to test a new censorship approach (in this scenario the newly deployed Tor censorship).  Or, this censorship is based on fields that are not logged by the appliance and hence inaccessible to us.  


\begin{figure}[t!]
 \begin{center}
 \includegraphics[width=0.35\textwidth]{tor_sg44_fraction_denied_ips.pdf}
\caption{Ratio of (re)censored IPs SG-44.}
\label{fig:tor_sg44_spec}
 \vspace{-0.2cm}
\end{center}
\end{figure}


\subsection{Web Proxies and VPNs}
As already mentioned, access to web/socks proxies is censored, as demonstrated by an aggressive filtering of requests including the keyword {\em proxy}.
In the following, we use ``proxies'' to refer to the Blue Coat appliances whose logs we study in this paper, and ``web proxies'' to refer to services used to circumvent censorship.

To use web proxies, end-users need to configure their browsers or their network interfaces, or rely on tools (e.g., Ultrasurf) that automatically redirect all HTTP traffic to the web proxy. Some web proxies support encryption, and create a SSL-based encrypted HTTP tunnel between the user and the web proxy. Similarly, VPN tools (e.g., Hotspot Shield) are often used to circumvent censorship, again, by relaying traffic through a VPN server. 

We analyze the usage of VPN and web proxy tools and find that 
a few of them are very popular among Internet users in Syria. However, as discussed in Section \ref{subsec:cat_string_ip_censorship}, keywords like {\em ultrasurf} and {\em hotspotshield} are heavily monitored and censored.
Nonetheless, some web proxies and VPN software (such as Freegate, GTunnel and GPass) do not include the keyword {\em proxy} in request URLs  and are therefore not censored. Similarly, we do not observe any censorship activity triggered by the keyword {\em VPN}. 

Next, we focus on the domains that are categorized as ``Anonymizers'' by McAfee's TrustedSource tool,  including both web proxies and VPN-related hosts.
In the \ds\ dataset, there are 821 ``Anonymizer'' domains, which are the target of 122K requests (representing 0.4\% of the total number of requests). 92.7\% of these hosts (accounting for 25\% of the requests) are never filtered. Figure~\ref{fig:cdf_anonymizers} shows the CDF of the number of requests sent to each of those allowed hosts. Less than 10\% of these hosts receive more than 100 requests, suggesting that only a few popular services attract a high number of the requests.      






Finally, we look at the 7.3\% of the identified ``Anonymizer'' hosts, for which some of the requests are censored. We calculate the ratio between the number of allowed requests in \df and the number of censored requests in \dd. Figure~\ref{fig:proxy_inconsistency} shows the CDF of this censorship ratio. We observe a  non-consistent policy for whether a request is allowed or censored, with more than 50\% of the proxies showing a higher number of allowed requests than censored requests. This suggests that these requests are not censored based on their IP or hostname, but rather on other criteria, e.g., the inclusion of a blacklisted keyword in the request.   





\begin{figure}[t!]
  \begin{center}
  \subfigure[]{
            \includegraphics[width=0.35\textwidth]{allowed_proxies.pdf}
\label{fig:cdf_anonymizers}
                }
           \subfigure[]{
            \includegraphics[width=0.35\textwidth]{proxy_inconsistency2.pdf}
 			\label{fig:proxy_inconsistency}
        }
        \vspace*{-0.2cm}
        \caption{(a) CDF of the number of requests for identified ``anonymizer'' hosts; (b) Ratio of Allowed versus Censored number of requests for identified ``anonymizer'' hosts.}
	\end{center}
\vspace*{-0.2cm}
\end{figure}


In conclusion, while some services (such as Hotspot Shield) are  heavily censored, other, less known, services are not, unless related requests contain blacklisted keywords. This somehow introduces a trade-off between 
the ability to bypass censorship and promoting censorship- and surveillance-evading tools (e.g., in web searches) by including keywords such as {\em proxy} in the URL.











\subsection{Peer-to-Peer networks}

The distributed architecture of peer-to-peer networks makes them, by nature, more resilient to censorship: users obtain content from peers and not from a server, which makes it harder to locate and block content. Shared data is usually identified by a unique identifier (e.g., info hash in BitTorrent), and these identifiers are useless to censors unless mapped back to, e.g., the description of the corresponding files. However, resolving these identifiers is not trivial: content can be created by anyone at any time, and the real description can be distributed in many different ways, publicly and privately. 

To investigate the use of peer-to-peer networks as a way to access censored content, we look for signs of BitTorrent traffic. We find a total of 338,168 BitTorrent announce requests from 38,575 users (\red{BitTorrent uses a 20-byte peer\_id field to identify peers, which we use to count unique users}) for 35,331 unique contents in the \df\ dataset.\footnote{BitTorrent clients send announce requests to BitTorrent servers (aka trackers) to retrieve a list of IP addresses from which the requested content can be downloaded.} Most of these requests (99.97\%) are allowed. Censored requests can be explained with the occurrence of blacklisted keywords, e.g., {\em proxy}, in the request URL. For instance, all announce requests sent to the tracker on \emph{tracker-proxy.furk.net} are censored. 

Using the hashes of torrent files provided in the announce messages, we crawl  \url{torrentz.eu} and \url{torrentproject.com} to extract the titles of these torrent files, achieving a success rate of 77.4\%. 
The five blacklisted keywords, reported in Table \ref{tab:keyword_blacklist}, are actually present in the titles of some of the BitTorrent files, yet the associated announce requests are allowed. We do not find any content that can be directly associated with sensitive topics like ``Syrian revolution'' or ``Arab spring'' (these files may still be shared via BitTorrent without publicly announcing the content), however, we identify content relating to anti-censorship software, such as UltraSurf (2,703 requests for all versions), HideMyAss (176 requests), Auto Hide IP (532 requests) and anonymous browsers (393 requests). Our findings suggest that peer-to-peer networks are indeed used by users inside Syria to circumvent censorship to a certain extent. Also note that BitTorrent is used to download Instant Messaging software, such as Skype, MSN messenger, and Yahoo! Messenger, which cannot be downloaded directly from the official download pages due to censorship.

\subsection{Google Cache}

When searching for terms in Google's search engine, the result pages allow access to  cached versions of suggested pages.  While Google Cache is not intended as an anti-censorship tool, a simple analysis of the logs shows that it provides a way to access content that is otherwise censored. 

We identify a total of 4,860 requests accessing Google's cache on  \url{webcache.googleusercontent.com} in the \df dataset. Only 12 of them are censored due to an occurrence of a blacklisted keyword in the URL, and a single request to retrieve a cached version of \url{http://ar-ar.facebook.com/SYRIANREVOLUTION.K.N.N} has \policydenied, although it is not categorized as a ``Blocked Site.'' However, the rest of the requests are allowed. Interestingly, some of the allowed requests, although small in number, relate to cached versions of webpages that are otherwise censored, such as \url{www.panet.co.il}, \url{aawsat.com}, \url{www.facebook.com/Syrian.Revolution}, and \url{www.free-syria.com}. 
While the use of Google cache to access censored content is obviously limited in scope, the logs actually suggest that it is very effective. Thus, when properly secured with HTTPS, Google cache could actually serve as a way to access censored content.


\subsection{Summary}
The logs highlighted that Syrian users do resort to censorship circumvention tools, with a relatively high effectiveness. While some tools and websites are monitored and blocked (e.g., Hotspot Shield), many others are successful in bypassing censorship. Our study also shows that 
some tools that were not necessarily designed as circumvention tools, such as BitTorrent and Google cache, could provide additional ways to access censored content if proper precautions are taken, especially considering that Syrian ISPs started to block Tor relays and bridges in December 2012.


\section{Discussion}
\label{sec:discussion}


\descr{Economics of Censorship.} Our analysis shows that Syrian authorities deploy several techniques to filter Internet traffic, ranging from blocking entire subnets to filtering based on specific keywords. This range of techniques can be explained by the cost/benefit tradeoff of censorship, as described, e.g., by Danezis and 
Anderson~\cite{danezis}. While censoring the vast majority of the Israeli network -- regardless of the actual content -- can be explained on geo-political grounds, completely denying the access to social networks, such as Facebook, could generate unrest. For instance, facing the ``Arab spring'' uprisings, the Syrian authorities decided to allow access to Facebook, Twitter, and Youtube in February 2011.
Nonetheless, these websites are monitored and selectively censored. Censors might aim at a more subtle control of the Internet, by only denying access to a predefined set of websites, as well as a set of keywords. This shift is achievable as the proxy appliances seamlessly support Deep Packet Inspection (DPI), thus allowing fine-grained censorship in real-time.    

\descr{Censorship's target.} Censored traffic encompasses a large variety of content, mostly aiming to prevent users from using Instant Messaging software (e.g., Skype), video sharing websites (e.g., \url{metacafe.com},  \url{upload.youtube.com}), Wikipedia, as well as sites related to news and opposition parties (e.g., \url{islammemo.cc}, \url{alquds.co.uk}). Censors also deliberately block any requests related to a set of predefined anti-censorship tools (e.g., `proxy'). This mechanism, however, has several side effects as it denies the access to any page containing these keywords, including those that have nothing to do with censorship circumvention. 



\descr{Censorship Circumvention.} We also highlighted that users attempt to circumvent censorship. One interesting way is using BitTorrent to download anti-censorship tools such as UltraSurf as well as Instant Messaging software. Users also rely on known censorship-evading technologies, such as web/socks proxies and Tor. 





\section{Conclusion}\label{sec:conclusion}

This paper presented a measurement analysis of Internet filtering in Syria.
We analyzed 600GB worth of logs produced by 7 Blue Coat SG-9000 proxies in Summer 2011 and, 
by extracting  information about processed requests for both censored and allowed traffic, we provided a detailed, first-of-a-kind snapshot of a real-world censorship ecosystem. We uncovered the presence of a relatively stealthy yet quite targeted filtering, which relies on IP addresses to block access to entire subnets, on domains to block specific websites, and on keywords to target specific content. 
Keyword-based censorship (e.g., denying all requests containing the word `proxy') also produces  collateral damage, as many requests are blocked even if they do not relate to sensitive content.
Finally, we showed that Instant Messaging software is heavily censored, that filtering of social media appears to be limited to specific pages, and that Syrian users try to circumvent censorship using web/socks proxies, Tor, VPNs, and BitTorrent. 



\balance	
\begin{thebibliography}{10}

\bibitem{wikileaks}
{Blue Coat ProxySG -- Configuration and Management Guide}.
\newblock
  \url{http://wikileaks.org/spyfiles/files/0/226_BLUECOAT-SGOS_CMG_4.1.4.pdf}.

\bibitem{anderson2012hidden}
C.~Anderson.
\newblock {The Hidden Internet of Iran: Private Address Allocations on a
  National Network}.
\newblock {\em ArXiv 1209.6398}, 2012.

\bibitem{selfcensorship}
{Arabic Network for Human Rights Information}.
\newblock {Online Syria, Offline Syrians}.
\newblock \url{http://old.openarab.net/en/node/1625}, 2013.

\bibitem{aryan2013internet}
S.~Aryan, H.~Aryan, and J.~A. Halderman.
\newblock {Internet Censorship in Iran: A First Look}.
\newblock In {\em FOCI}, 2013.

\bibitem{bbc}
{BBC News}.
\newblock {Syrian jailed for Internet usage}.
\newblock \url{http://news.bbc.co.uk/1/hi/world/middle_east/3824595.stm}.

\bibitem{bluecoat-update}
{Blue Coat}.
\newblock
  \url{https://www.bluecoat.com/company/news/update-blue-coat-devices-syria},
  2011.

\bibitem{crandall2007conceptdoppler}
J.~R. Crandall, D.~Zinn, M.~Byrd, E.~T. Barr, and R.~East.
\newblock {ConceptDoppler: A Weather Tracker for Internet Censorship}.
\newblock In {\em CCS}, 2007.

\bibitem{dainotti2011analysis}
A.~Dainotti, C.~Squarcella, E.~Aben, K.~C. Claffy, M.~Chiesa, M.~Russo, and
  A.~Pescap{\'e}.
\newblock {Analysis of country-wide internet outages caused by censorship}.
\newblock In {\em IMC}, 2011.

\bibitem{dalek2013method}
J.~Dalek, B.~Haselton, H.~Noman, A.~Senft, M.~Crete-Nishihata, P.~Gill, and
  R.~J. Deibert.
\newblock {A Method for Identifying and Confirming the Use of URL Filtering
  Products for Censorship}.
\newblock In {\em IMC}, 2013.

\bibitem{danezis}
G.~Danezis and R.~Anderson.
\newblock {The Economics of Censorship Resistance}.
\newblock In {\em WEIS}, 2004.

\bibitem{bonneau}
S.~Egelman, J.~Bonneau, S.~Chiasson, D.~Dittrich, and S.~E. Schechter.
\newblock {It's Not Stealing If You Need It: A Panel on the Ethics of
  Performing Research Using Public Data of Illicit Origin}.
\newblock In {\em Financial Cryptography}, 2012.

\bibitem{StatisticalSignificance91}
R.~Jain.
\newblock {The art of computer systems performance analysis: Techniques for
  experimental design, measurements, simulation, and modeling}.
\newblock {\em Wiley}, April 1991.

\bibitem{king2012censorship}
G.~King, J.~Pan, and M.~Roberts.
\newblock {How censorship in China allows government criticism but silences
  collective expression}.
\newblock In {\em APSA Annual Meeting}, 2012.

\bibitem{knockel2011three}
J.~Knockel, J.~R. Crandall, and J.~Saia.
\newblock {Three Researchers, Five Conjectures: An Empirical Analysis of
  TOM-Skype Censorship and Surveillance}.
\newblock In {\em FOCI}, 2011.

\bibitem{leberknight2012taxonomy}
C.~S. Leberknight, M.~Chiang, H.~V. Poor, and F.~Wong.
\newblock {A Taxonomy of Internet Censorship and Anti-Censorship}.
\newblock Princeton Tech Report,
  \url{http://www.princeton.edu/~chiangm/anticensorship.pdf}, 2012.

\bibitem{citizenlab}
M.~Marquis-Boire et~al.
\newblock {Planet Blue Coat: Mapping Global Censorship and Surveillance Tools}.
\newblock
  \url{https://citizenlab.org/wp-content/uploads/2013/01/Planet-Blue-Coat.pdf},
  2013.

\bibitem{nabi2013anatomy}
Z.~Nabi.
\newblock {The Anatomy of Web Censorship in Pakistan}.
\newblock In {\em FOCI}, 2013.

\bibitem{oni}
{Open Net Initiative}.
\newblock {Internet Filtering in Syria}.
\newblock \url{https://opennet.net/research/profiles/syria}, 2009.

\bibitem{park2010empirical}
J.~C. Park and J.~R. Crandall.
\newblock {Empirical Study of a National-Scale Distributed Intrusion Detection
  System: Backbone-level Filtering of HTML Responses in China}.
\newblock In {\em ICDCS}, 2010.

\bibitem{RWB}
{Reporters Without Borders}.
\newblock {Syria}.
\newblock \url{http://en.rsf.org/syria-syria-12-03-2012,42053.html}, 2012.

\bibitem{bloomberg}
V.~Silver.
\newblock {H-P Computers Underpin Syria Surveillance}.
\newblock
  \url{http://www.bloomberg.com/news/2011-11-18/hewlett-packard-computers-underpin-syria-electonic-surveillance-project.html},
  2011.

\bibitem{leak}
Telecomix.
\newblock {\#OpSyria: Syrian Censorship Logs (Season 3)}.
\newblock \url{http://reflets.info/opsyria-syrian-censoship-log/}, 2011.

\bibitem{tor-syria}
{The Tor Project}.
\newblock {Tor's Censorship Wiki -- Syria}.
\newblock
  \url{https://trac.torproject.org/projects/tor/wiki/doc/OONI/censorshipwiki/CensorshipByCountry/Syria},
  2013.

\bibitem{valentino}
J.~Valentino-Devries, P.~Sonne, and N.~Malas.
\newblock {U.S. Firm Acknowledges Syria Uses Its Gear to Block Web}.
\newblock \url{http://preview.tinyurl.com/kvpkeos}, 2011.

\bibitem{verkamp2012inferring}
J.-P. Verkamp and M.~Gupta.
\newblock {Inferring Mechanics of Web Censorship Around the World}.
\newblock In {\em FOCI}, 2012.

\bibitem{slideshare}
{VFM Systems \& Services Ltd.}
\newblock {Blue Coat ProxySG Solution with Web Filter and Reporter}.
\newblock
  \url{http://www.slideshare.net/vfmindia/vfm-bluecoat-proxy-sg-solution-with-web-filter-and-reporter},
  2013.

\bibitem{cbs}
{Whittaker, Z.}
\newblock {Surveillance and censorship: Inside Syria's Internet}.
\newblock
  \url{http://www.cbsnews.com/news/surveillance-and-censorship-inside-syrias-internet/},
  2013.

\bibitem{winter2012great}
P.~Winter and S.~Lindskog.
\newblock {How the Great Firewall of China is Blocking Tor}.
\newblock In {\em FOCI}, 2012.

\bibitem{xu2011internet}
X.~Xu, Z.~M. Mao, and J.~A. Halderman.
\newblock {Internet censorship in China: Where does the filtering occur?}
\newblock In {\em PAM}, 2011.

\bibitem{yen:tracking}
T.-F. Yen, Y.~Xie, F.~Yu, R.~Yu, and M.~Abadi.
\newblock {Host Fingerprinting and Tracking on the Web: Privacy and Security
  Implications}.
\newblock {\em NDSS}, 2012.

\end{thebibliography}
 

\end{document}
