\documentclass{llncs}
\newcommand{\longversion}[1]{#1}
\newcommand{\shortversion}[1]{}

\usepackage{hyperref}

\usepackage{subfigure}
 \usepackage{amsmath}
 \usepackage{amssymb}
\usepackage{psfrag}
\usepackage{graphicx}
\usepackage{pifont}
\usepackage{url}
\usepackage{algorithm,myalgorithmic}
\usepackage{boxedminipage}
\usepackage{multirow}
\usepackage{tikz}
\usepackage{xspace}


\renewcommand{\algorithmicrequire}{\textbf{Input(s):}}
\renewcommand{\algorithmicensure}{\textbf{Output(s):}}
\newcommand{\defines}{\emph}
\newcommand{\mc}{\mathcal}



\newtheorem{fact}{Fact}

\newcommand{\comment}[1]{ }
\newcommand{\scl}{0.67}
\newcommand{\YES}{\textup{\textsf{YES}}}
\newcommand{\NO}{\textup{\textsf{NO}}}
\newcommand{\Oh}{{\mc{O}}}

\newcommand{\real}{{\mathbb R}}
\newcommand{\nat}{{\mathbb N}}
\renewcommand{\int}{\operatorname{int}}       \newcommand{\es}{\emptyset}
\newcommand{\sm}{\setminus}
\newcommand{\bw}{\operatorname{bw}}       \newcommand{\poss}{\text{poss}}
\newcommand{\tw}{\operatorname{tw}}       \newcommand{\degree}{\operatorname{deg}} 

\newcommand{\Pol}{\mbox{}}
\newcommand{\FPT}{\mbox{}}
\newcommand{\NP}{\mbox{}}
\newcommand{\PSPACE}{\mbox{}}
\newcommand{\PTAS}{\mbox{}}
  \newcommand{\dcup}{\uplus}

\newcommand{\paramproblem}[4]{An instance of {\sc #1}
is given by #2, and the parameter, #3. The task is: #4}
\newcommand{\paramproblemvar}[4]{An instance of {\sc #1}
is given by #2, and  #3 (the parameter). We ask: #4}

\newcommand{\emphprob}[3]
{\begin{center}\begin{boxedminipage}{0.95 \textwidth}\noindent {{\bf {\sc  #1}} \\
{\bf Given:} #2.\\
{\bf Task:} #3.}\end{boxedminipage}\end{center}\medskip}


\newcommand{\redrule}[2]{\vspace*{1ex}{\bf #1}: #2}
\pagenumbering{arabic}



\newcommand{\mist}{\textsc{MIST}\xspace}
\newcommand{\rt}{1.8669}
\newcommand{\rtk}{3.4854}
\newcommand{\rtki}{2.7321}
\newcommand{\rtkii}{2.1364}

\pagestyle{headings}

\begin{document}

\title{Exact and Parameterized Algorithms for\\ {\sc Max Internal Spanning Tree}\thanks{\longversion{This work was partially s}\shortversion{S}upported by a PPP grant between DAAD (Germany) and NFR (Norway).}}\author{
 Henning Fernau\inst1,
 Serge Gaspers\inst2,
 Daniel Raible\inst1      
}
\institute{Univ. Trier, FB 4---Abteilung Informatik, D-54286 Trier,
Germany\\
\email{\{fernau,raible\}@uni-trier.de}
\and
LIRMM -- Univ. of Montpellier 2, CNRS, 34392 Montpellier, France\\
\email{gaspers@lirmm.fr}
}


\maketitle
\begin{abstract}We consider the \NP-hard problem of finding a spanning tree with a maximum number of internal vertices. This problem is a
generalization of the famous 
 {\sc Hamiltonian Path} problem. Our dynamic-programming algorithms for general and degree-bounded graphs have running times of the form 
(). The main result, however, is a branching algorithm for graphs with maximum degree three. It only needs polynomial space and has a running
time of  when analyzed with respect to the number of vertices. We also show that its running time is  when the goal
is to find a spanning tree with at least  internal vertices. Both running time bounds are obtained via a Measure \& Conquer analysis, the latter
one being a novel use of this kind of analyses for parameterized algorithms.
\end{abstract}

\section{Introduction}
\subsubsection*{Motivation}
In this paper we investigate the following problem:
\emphprob{Max Internal Spanning Tree (MIST)}{A graph  with  vertices and  edges}{Find a spanning tree of  with a maximum number of internal vertices}
\mist is a generalization of the \longversion{famous and }well-studied {\sc Hamiltonian Path} problem\shortversion{:}\longversion{. Here, one is asked to} find a path in a graph such that every vertex is visited exactly  once. Clearly, such a path, if it exists, is also a spanning tree, namely one with a maximum number of internal vertices. Whereas the running time barrier of  has not been broken for general graphs, there are faster algorithms for cubic graphs (using only polynomial space). It is natural to ask if for the generalization, \mist, this can also be obtained.

A second issue is \longversion{if we can}\shortversion{to} find an algorithm for \mist with a running time of the form .
\footnote{\longversion{Throughout the paper, we write } if 
for some polynomial .}  
The \longversion{very }na{\"i}ve approach gives only an upper bound of .


A possible application could be the following scenario. \longversion{Suppose you have a set of}\shortversion{Consider} cities which should be connected with water pipes. The possible connections between them can be represented by a graph . It suffices to compute a spanning tree  for .  In  we may have high degree vertices that have to be implemented by branching pipes\longversion{. These branching pipes}\shortversion{ which} cause turbulences and therefore pressure may drop. To minimize the number of branching pipes one  can equivalently compute a spanning tree with the smallest number of leaves, leading to \mist. Vertices representing 
 branching pipes should not be of  arbitrarily high degree, motivating us to investigate \mist on degree-restricted graphs.

\subsubsection*{Previous Work}
It is well-known that the more restricted problem, \textsc{Hamiltonian Path}, can be solved within  steps and exponential space. This
result has been  independently obtained by Bellman~\cite{Bel62}, and Held and Karp~\cite{HelKar62}. The \textsc{Traveling Salesman} problem is very
closely related to \textsc{Hamiltonian Path}. Basically, the same algorithm solves this problem, but there has not been any improvement on the running
time since 1962. The space requirements have, however, been improved and now there 
are  algorithms needing only polynomial space. In 1977, Kohn \emph{et al.}~\cite{KohGotKoh77} gave an algorithm based on generating
functions with a running time of  and space requirements of  and in 1982 Karp~\cite{Kar77} came up with an algorithm which
improved storage requirements to  and preserved this run time by an inclusion-exclusion approach.

Eppstein~\cite{Epp03} studied the {\sc Traveling Salesman }problem on cubic graphs. He could achieve a running time of  using
polynomial space. Iwama and Nakashima~\cite{IwaNak07} could improve this to . solving {\sc Hamiltonian Path} in .
Bj{\"o}rklund \emph{et al.}~\cite{BjoHusKasKoi08} studied TSP with respect to degree-bounded graphs. Their algorithm is a variant of the classical
-algorithm and the space requirements are therefore exponential. Nevertheless, they showed that for a graph with maximum degree 
 there is a -algorithm. In particular for  there is a - and for  a -algorithm.

\mist was also studied with respect to parameterized complexity. The (standard) parameterized version of the problem is parameterized by , and asks
whether  
has a spanning tree with at least  internal vertices. Prieto and Sloper~\cite{PriSlo2003} proved a -vertex kernel for the problem showing
\FPT-membership. In \cite{\longversion{Pri2005,}PriSlo2005} the kernel size has been improved to  and in \cite{FominGST} to .
Parameterized algorithms for \mist have been studied in~\cite{CohenFGKSY09,FominGST,PriSlo2005}. Prieto and Sloper~\cite{PriSlo2005} gave the first
FPT algorithm, with running time . This result was improved by Cohen {\em et al.}~\cite{CohenFGKSY09} who solve a more
general directed version of the problem in time . The current fastest algorithm has running time ~\cite{FominGST}.

Salamon~\cite{Sal07} studied the problem considering approximation. He could achieve a -approximation. A -approximation for the node-weighted version is also a by-product. Cubic and claw-free graphs were considered by Salamon and Wiener~\cite{SalWie08}. They introduced algorithms with approximation ratios  and , respectively.

\subsubsection{Our Results} This paper gives two algorithms:\shortversion{\\}
\longversion{
\begin{enumerate}\itemsep.5pt
 \item[]}
\shortversion{(a)} A dynamic-programming algorithm solving \mist in time . We extend this algorithm and show that for any degree-bounded
graph a running time of  with  can be achieved. To our knowledge this is the first algorithm for \mist with a
running time 
bound of the form .\footnote{Before the camera-ready version of this paper was prepared, Nederlof~\cite{Nederlof09} came up with a
polynomial-space  algorithm for \mist on general graphs, answering a question in a preliminary version of this paper.}
\shortversion{\\label{one}
M[I,L]=\left\{
\begin{array}{l@{\,:\,}l}
1 & \exists x \in L \cap N(I): M[I,L \setminus  \{x\}]=1\\
1 & \exists x \in L, y \in N(x) \cap I: M[ I \setminus \{y\},(L \cup \{y\}) \setminus \{x\}]=1 \\
0 & \text{otherwise}
\end{array}
 \right.

\sum_{A \subseteq V \atop A \in {\cal C}}\sum_{I \subseteq A \atop I \in {\cal C}} 1\le \sum_{A \subseteq V \atop A \in {\cal C}} \beta_\Delta^{|A|} \le \sum_{i=0}^n \left(n \atop i\right) \beta_\Delta^i =(\beta_\Delta+1)^n
1ex]
Next we present \longversion{a sequence of}\shortversion{several} reduction rules. \longversion{Note that t}\shortversion{T}he order in which they are applied is crucial\shortversion{: B}\longversion{. We assume that b}efore a rule is applied the preceding ones were carried out exhaustively.

\begin{figure}[bt]
\psfrag{a}{}
\psfrag{b}{}
\psfrag{u}{}
\psfrag{v}{}
\psfrag{w}{}
\psfrag{x}{}
\psfrag{z}{}
\centering
\subfigure[ ]{\label{1b}
\includegraphics[scale=\scl]{redrule.eps}}\hspace*{3ex}
\subfigure[ ]{\label{1c}
\includegraphics[scale=\scl]{redrule4.eps}}
\caption{Light edges may be not present. Double edges (dotted or solid, resp.) refer to edges which are either -edges or not, resp. Edges attached to oblongs are pt-edges.}
\label{attach}
\end{figure}

\begin{enumerate}\itemsep.5pt
\item \redrule{Cycle}{Delete any edge  such that  has a cycle.}
\item \redrule{Bridge}{If there is a bridge , then add  to . }
\item \redrule{Deg1}{If there is a degree- vertex , then add its incident edge to . }
\item \redrule{Pending}{If there is a vertex  that is incident to  pt-edges, then remove its incident pt-edges. }
\item \redrule{ConsDeg2}{If there are edges  such that , then delete  from  and add the edge  to .}
\item \redrule{Deg2}{If there is an edge  such that  and , then add  to .}
\item \redrule{Attach}{If there are edges  such that , , , then delete
. .}
\item \redrule{Attach2}{If there is a vertex  with  and  such that  is incident to
a pt-edge, then delete . See Fig.~\ref{1b}}
\item \redrule{Special}{If there are two edges  with  , , and  is incident to a pt-edge, then add  to  (Fig.~\ref{1c}). }
\end{enumerate}

\begin{lemma}
The reduction rules stated above are sound.
\end{lemma}
\begin{proof}
Let  be an optimal spanning tree of . The first three rules are correct for the purpose of connectedness and acyclicity of the evolving spanning tree. 
\begin{description}\itemsep.5pt
\item[\bf Pending] is correct as the other edge incident to  (which will be added to  by a subsequent {\bf Deg1} rule) is a bridge and needs to be in any spanning tree. 
\item [\bf ConsDeg2] We implicitly assume that we can add  to . If  then . Then we can simply exchange the two edges giving a solution  with  and .  \item[\bf Deg2] Since the preceding reduction rules do not apply, we have  and there is one edge, say
, , that is not pending. Assume  
has  as a leaf. Define 
another spanning tree  by setting .
Since 
,  is also optimal.   
\item[\bf Attach] If  then  due to the acyclicity of  and as  is connected. Then by exchanging  and  we obtain a solution  with at least as many 2-vertices.
\item[\bf Attach2] Suppose . Let  be the pt-edge and  the third edge incident to  (that must exist and is not pending, since {\bf Pending} did not apply). Since {\bf Bridge} did not apply,  is not a bridge. 
Firstly, suppose . Due to the proof of Lemma~\ref{nodeg3}, there is also an optimal solution  with .
Secondly, assume . Then  is also optimal as  has become a 2-vertex.
\item[Special] Suppose . Then  where  is the third edge incident to . Let . In ,  is a 2-vertex and hence  is also optimal.\qed
\end{description}
\end{proof}


\subsection{The Algorithm}

The algorithm we describe here is recursive. It constructs a set  of edges which are selected to be in every spanning tree considered in the current recursive step. The algorithm chooses edges and considers all relevant choices for adding them to  or removing them from . It selects these edges based on priorities chosen to optimize the running time analysis. Moreover, the set  of edges will always be the union of a tree  and a set of edges  that are not incident to the tree and have one end point of degree  in  (pt-edges). We do not explicitly write in the algorithm that edges move from  to  whenever an edge is added to  that is incident to both an edge of  and an edge of . To maintain the connectivity of , the algorithm explores edges in the set  to grow .

If  every spanning tree  must have a vertex  with . Thus initially the algorithm creates an instance for every vertex  and every possibility that . Due to the degree constraint there are no more than  instances. After this initial phase, the algorithm proceeds as follows.

\begin{enumerate}
 \item[1.] Carry out each reduction rule exhaustively in the given order (until no rule applies).
 \item[2.] If  and , then  is not connected and does not admit a spanning tree. Ignore this branch.
 \item[3.] If  and , then return .
 \item[4.] Select  with  according to the following priorities (if such an edge exists):
  \begin{itemize}
   \item[a)] there is an edge ,
   \item[b)] ,
   \item[c)]  is incident to a pt-edge, or
   \item[d)] .
  \end{itemize}
  Recursively solve two instances where  is added to  or removed from  respectively, and return a spanning tree with most internal vertices.
 \item[5.] Otherwise, select  with . Let  be the other two neighbors of . Recursively solve three instances where 
  \begin{itemize}
   \item[(i)]  is removed from ,
   \item[(ii)]  and  are added to  and  is removed from , and
   \item[(iii)]  and  are added to  and  is removed from .
  \end{itemize}
  Return a spanning tree with most internal vertices.
\end{enumerate}

\longversion{\subsection{An Exact Analysis of the Algorithm}}

By a Measure \& Conquer analysis taking into account the degrees of the vertices, their number of incident edges that are in , and to some extent
the degrees of their neighbors, 
we obtain the following result.

\begin{theorem}\label{thm:exalg}
{\sc Max Internal Spanning Tree} can be solved in time  on subcubic graphs.
\end{theorem}



\shortversion{The proof is omitted for reasons of space, but we will}\longversion{Let us} provide measure we used in the following:
Let ,  and . 
Then the measure we use for our running time bound is:

with     ,     and     .\longversion{

Let , , , ,  and . We define  for , 
, .}
\longversion{
 }
The proof of the theorem is using the following result:

\begin{lemma}\label{noinc}
None of the reduction rules increase  for the given weights.
\end{lemma}

\longversion{
\begin{proof}
{\bf Bridge}, {\bf Deg1}, {\bf Deg2} and {\bf Special} add edges to . Due to the definitions of  and  and the
 choice of the weights it can be seen that  only decreases. It is also easy to see that the deletion of edges  with  is
safe with respect to . The weight of  can only decrease due to this. Nevertheless, the rules which delete edges might cause that a  will be in  afterwards. Thus, we have to prove that in this case the overall reduction is enough. A
sufficient criterion that the described scenario takes place is if degree 2 vertices are created. {\bf Cycle} may create vertices of degree 2, but
none which are adjacent to a vertex in  and are not subject to another application of {\bf Cycle}. The next reduction
rule which may create vertices of degree 2 is {\bf Attach} when . The minimum reduction is . No other reduction rule creates degree 2 vertices.\qed
\end{proof}

\begin{proof} (Theorem~\ref{thm:exalg}) 
As the algorithm deletes edges or moves edges from  to , cases 1--3 do not contribute to the exponential function in the running time of the algorithm. It remains to analyze cases 4 and 5, which we do now. Note that after applying the reduction rules exhaustively, we have that for all ,  ({\bf Deg2}) and for all ,  ({\bf Pending}).


\begin{enumerate}\itemsep.5pt
 \item[\bf 4.(a)] Obviously, , and there is a vertex  such that ; see Figure~\ref{branch1}. We must have  (due to the reduction rule {\bf Attach}).
We consider three cases.
\begin{itemize}
\item . When  is added to , {\bf Cycle} deletes . We get an amount of  and  as  drops out of  and  out of  ({\bf Deg2}). Also  will be removed from  and added to  which amounts to a reduction of at least . When  is deleted,  is added to  ({\bf Bridge}). By a symmetric argument we get a reduction of  as well. In total this yields a -branch.
\item  and there is one  pt-edge attached to . Adding  to  decreases the measure by  (from ) and  (deleting , then {\bf Deg2} on ). By Deleting   we decrease  by  and by  (from ).   This amounts to a -branch.
\item  and no pt-edge is attached to . Let  be the third edge incident to . In the first branch the measure drops by at least  from  and  ({\bf Deg2}),  from  ({\bf Deg2}). 
In the second branch we get . Observe that we also get an amount of at least  from  if . If  we get .  It results a -branch.
\end{itemize}
Note that from this point on, for all  there is no  with .\vspace*{2ex}

\item[\bf 4.(b)] As the previous case does not apply, the other neighbor  of  has , and  ({\bf Pending}), see Figure~\ref{branch1.5}. Additionally, observe that we must have  ({\bf ConsDeg2}) and that  due to {\bf Special}. We consider two subcases.
\begin{itemize}
\item[] . When we add  to , then  is also added due to {\bf Deg2}. The reduction is at least 
from , 
 from  and  from . When  is deleted,  becomes a pt-edge. There is  with ,
which is subject to a {\bf Deg2} reduction rule. We get at least  from ,  from ,  from  and
 from . This is a -branch. 
\item[] . Similarly, we obtain a -branch.
\end{itemize}
\item[\bf 4.(c)] In this case,  and there is one  pt-edge attached to , see Figure~\ref{branch2}. 
Note that  can be ruled out due to {\bf Attach2}. Thus, . Let  be such that . Due to the priorities, .  We distinguish between the cases where  is incident to a pt-edge or not.
\begin{enumerate}
 \item . First suppose . Adding  to  allows a reduction of  (due to case 4.(b) we can exclude
). Deleting  
implies that we get a reduction from  and  of  ({\bf Deg2} and {\bf Pending}). As  is added to  we reduce  by
at least  as the state of  changes. Now due to {\bf Pending} and {\bf Deg1} we include  and get  from . We have at least a -branch.\\
If  we consider the two cases for  also. These are  and . The first  entails .
 Note that when we add  we trigger {\bf Attach2}. The second is a -branch.
\item . Let  be the other neighbor of  that does not have degree . When  is added to ,  is deleted by
{\bf Attach2} and  becomes a pt-edge 
({\bf Pending} and {\bf Deg1}). The changes on  incur a measure decrease of  and those on  a measure decrease of .
When  is deleted,  is added to  ({\bf Deg2}) and  becomes a pt-edge by two applications of the {\bf Pending} and {\bf
Deg1} rules. Thus,
 the decrease of the measure is at least  in this branch. In total, we have a
-branch here.
\end{enumerate}
\item[\bf 4.(d)] Now, ,  is not incident to a pt-edge, and . See Figure~\ref{branch2}. There is also some  such that .
Note that ,  and . Otherwise either {\bf Cycle} or  cases 4.(b) or 4.(c) would have been triggered. 
From the addition of  to  we get  and from its deletion  (from  via {\bf Deg2}),  (from
) and at least  from  and thus, a -branch.

\item[\bf 5.] See Figure~\ref{branch3}. The algorithm branches in the following way:  Delete ,  add , and delete ,  add  and delete . Due to {\bf Deg2}, 
we can disregard the case when  is a leaf. Due to Lemma~\ref{nodeg3} we also disregard the case when  is a 3-vertex. Thus by branching in this manner we find at least one optimal solution.\
  \kappa := \kappa(G,F,k) := k - \omega \cdot |X| - |Y|,
 
  \kappa := \kappa(G,F,k) := k - \omega_1 \cdot |X| - |Y| - \omega_2 |Z| - \omega_3 |W|,
 1ex]
\textbf{Acknowledgment} We would like to thank Alexey A. Stepanov for useful discussions in the initial phase of this paper.




\begin{thebibliography}{1}
\bibitem{Bel62}
R. Bellman.
\newblock Dynamic programming treatment of the Travelling Salesman Problem.
\newblock {\em J. Assoc. Comput. Mach.} 9 (1962), 61--63.

\bibitem{BjoHusKasKoi08}
A.~Bj{\"o}rklund, T.~Husfeldt, P.~Kaski, and M.~Koivisto.
\newblock The Travelling Salesman Problem in bounded degree graphs.
\newblock In {\em\longversion{Automata, Languages and Programming, 35th International Colloquium,}
  ICALP 2008, \longversion{Proceedings,} Part I\longversion{: Tack A:
  Algorithms, Automata, Complexity, and Games}}, volume 5125 of {\em LNCS}, pages 198--209. Springer, 2008.
  
\bibitem{CohenFGKSY09}
N. Cohen, F.~V. Fomin, G. Gutin, E.~J. Kim, S. Saurabh, and A. Yeo.
\newblock Algorithm for finding -Vertex Out-trees and its application to -Internal Out-branching problem.
\newblock In {\em\longversion{Computing and Combinatorics, 13th Annual International Conference,}
  COCOON 2009, \longversion{Proceedings,}} to appear. Springer, 2008.

\bibitem{Epp03}
D.~Eppstein.
\newblock The Traveling Salesman problem for cubic graphs.
\newblock {\em J. Graph Algorithms Appl.} 11(1), pages 61--81, 2007.

\bibitem{FominGST}
F.~V. Fomin, S. Gaspers, S. Saurabh, and S. Thomass{\'e}.
\newblock A linear vertex kernel for Maximum Internal Spanning Tree.
\newblock {\em In preparation.}

\bibitem{HelKar62}
M. Held, R. M. Karp. 
\newblock A dynamic programming approach to sequencing problems.
\newblock {\em J. Soc. Indust. Appl. Math.} 10, pages 196--210, 1962.



\bibitem{IwaNak07}
K.~Iwama and T.~Nakashima.
\newblock An improved exact algorithm for cubic graph TSP.
\newblock In {\em\longversion{ Computing and Combinatorics, 13th Annual
  International Conference,} COCOON 2007\longversion{,
  Proceedings}}, volume 4598 of {\em LNCS}, pages
  108--117. Springer, 2007.

\bibitem{Kar77}
R.~M. Karp.
\newblock Dynamic programming meets the principle of inclusion-exclusion.
\newblock {\em Inf. Process. Lett.}, 1(2):49--51, 1982.

\bibitem{KohGotKoh77}
S.~Kohn, A.~Gottlieb, and M.~Kohn.
\newblock A generating function approach to the Traveling Salesman Problem.
\newblock In {\em Proceedings of the 1977 ACM Annual Conference (ACM 1977)}, pages
  294--300. Association for Computing Machinery, 1977.
  
\bibitem{Nederlof09} J. Nederlof.
\newblock Fast polynomial-space algorithms using Mobius inversion: Improving on Steiner Tree and related problems.
\newblock To appear in {\em\longversion{Automata, Languages and Programming, 36th International Colloquium,}
  ICALP 2009, \longversion{Proceedings,} Part I\longversion{: Tack A:
  Algorithms, Automata, Complexity, and Games}}.

\longversion{
\bibitem{Pri2005}
E.~Prieto.
\newblock {\em Systematic Kernelization in FPT Algorithm Design}.
\newblock {PhD} thesis, The University of Newcastle, Australia, 2005.}

\bibitem{PriSlo2003}
E.~Prieto and C.~Sloper.
\newblock Either/or: Using vertex cover structure in designing
  {FPT}-algorithms---the case of -internal spanning tree.
\newblock In {\em\longversion{ Proceedings of} WADS 2003\longversion{, Workshop on Algorithms and Data
  Structures}}, volume 2748 of {\em LNCS}, pages 465--483. Springer, 2003.
  
\bibitem{PriSlo2005}
E.~Prieto and C.~Sloper.
\newblock Reducing to independent set structure -- the case of -internal spanning tree. 
\newblock {\em Nord. J. Comput.}, 12(3): 308--318, 2005.

\bibitem{SalWie08}
G.~Salamon and G.~Wiener.
\newblock On finding spanning trees with few leaves.
\newblock {\em Inf. Process. Lett.}, 105(5): 164--169, 2008.

\bibitem{Sal07}
G. Salamon.
\newblock Approximation algorithms for the maximum internal spanning tree
  problem.
\newblock In {\em\longversion{ Mathematical
  Foundations of Computer Science 2007, 32nd International Symposium,} MFCS
  2007\longversion{, Proceedings}},
  volume 4708 of {\em LNCS}, pages 90--102.
  Springer, 2007.

\bibitem{Wah2007}
M.~Wahlstr{\"o}m.
\newblock {\em Algorithms, Measures and Upper Bounds for Satisfiability and
  Related Problems}.
\newblock PhD thesis, Department of Computer and Information Science,
  Link\"opings universitet, Sweden, 2007.

\end{thebibliography}

\newpage

\appendix


\end{document}
