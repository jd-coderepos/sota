\documentclass[12pt]{article}

\usepackage{amsmath}
\usepackage{amsfonts}
\usepackage{amsthm}
\usepackage{enumerate}
\usepackage{graphicx}
\usepackage{subcaption}
\usepackage{color}
\usepackage{cite}


\newtheorem{lemma}{Lemma}
\newtheorem{corollary}{Corollary}
\newtheorem{theorem}{Theorem}
\theoremstyle{definition}
\newtheorem{definition}{Definition}

\DeclareMathOperator{\next}{next}
\DeclareMathOperator{\prev}{prev}
\DeclareMathOperator{\first}{first}
\DeclareMathOperator{\last}{last}
\DeclareMathOperator{\cvx}{conv}

\title{Geometric Spanning Cycles in Bichromatic Point Sets}

\author{B. Joeris \thanks{University of Waterloo, Waterloo, Canada. Email:
bjoeris@uwateloo.ca}
\and I. Urrutia \thanks{University of Waterloo, Waterloo, Canada. Email:
ihurruti@uwaterloo.ca} \and J. Urrutia \thanks{Instituto de Matem{\'a}ticas, UNAM,
Mexico.
Email:
urrutia@matem.unam.mx. Research supported by project number 178379 Conacyt, M{\'e}xico.}}

\date{}

\begin{document}
\maketitle

\begin{abstract}
Given a set of points in the plane each colored either red or blue, we find
non-self-intersecting geometric spanning cycles of the red points and of the
blue points such that each edge of the red spanning cycle is crossed at most
three times by the blue spanning cycle and vice-versa.
\end{abstract}

\section{Introduction}
A \emph{geometric graph} is a graph embedded in the plane
with edges that are straight-line segments. A set of points is in general position if no three points of the set are collinear.
In this paper, a bichromatic point set is a finite set of points  in general position, partitioned into two disjoint color classes  and  (red and blue.) 


Several problems have been studied that involve finding geometric graphs on sets
of red and blue points. 
Alternating paths in bichromatic point sets in convex position were studied in
\cite{akiyama1990simple}. Alternating paths in general position were
studied in \cite{abellanas1999bipartite}.
Alternating paths in points with more
than two colors were studied in \cite{merino2006length}.
\cite{tokunaga1996intersection} examined non-self-intersecting geometric spanning trees of the red points and the blue
points and found a tight bound on the minimum number of intersection points
between the red and blue spanning trees.
\cite{Kano2005301} considered the case of more than two colors, and studied the
 number of intersections for monochromatic spanning trees and for
monochromatic spanning cycles.
\cite{merino2005intersection} obtained a tight bound on the number of
intersections in monochromatic perfect matchings.
\cite{kano2013discrete} looked at points and lines in the plane lattice, and
studied the number of crossings for alternating matchings and monochromatic
spanning trees.



In \cite{tokunaga1996intersection}, Tokunaga also showed that there exist
non-intersecting geometric spanning paths  and  of the red and blue points respectively, such that each
edge of  is crossed at most once by , and vice-versa. 

One may then wonder if a similar result is possible for cycles; i.e., is it possible to construct spanning cycles with ``few'' intersections on bichromatic point sets. In particular, we wonder for what values of  does the
following statement hold: there exist non-intersecting geometric spanning
cycles  and  of the red and blue points respectively, such that each
edge of  is crossed at most  times by , and vice-versa.
In \cite{tokunaga1996intersection}, Tokunaga conjectured that this statement is
true when .
It is easy to see that the statement is false with .
We show here that the statement is true when .

\begin{theorem}
\label{thm:main}
Given any bichromatic point set in general position, there exists a non-self-intersecting geometric spanning cycle of the red points and a
non-self-intersecting geometric spanning cycle of the blue points such that
each edge of the red spanning cycle is crossed by at most three edges of the blue spanning cycle, and vice-versa.
\end{theorem}

\subsection{Definitions and Notation}
If  is a set of points and  is a point outside the convex hull of , we say that  \emph{sees} a point  (with respect to ), if the line segment  intersects the convex hull of  only at .
In other words, if the convex hull of  were opaque, then  could see .
In particular,  must be a vertex of the convex hull in order for  to see it.
If  and  are two sets of points with disjoint convex hulls, and  and , we say that  and  \emph{see each other} (with respect to  and ) if  sees  with respect to  and  sees  with respect to .

Let  be a set of points in the plane and .
The \emph{radial order} of  about  is the cyclic  list  taking all the
elements  ordered clockwise around .
The \emph{interval between  and  in the radial order} is the set consisting of , if , or  if .
If  is in the interval between  and  in the radial order, then we say that  lies \emph{between  and  in the radial order}.

If  and  are disjoint sets of red and blue points, respectively, then a
\emph{blob} in the radial order of  about  is a maximal
monochromatic set of consecutive elements in the radial order.
Because the blobs are monochromatic, it is natural to speak of \emph{red blobs} and \emph{blue blobs}, containing red and blue points respectively.

Given a radial order  of  with respect to a point , for any ,  is called the point \emph{before}  and  is called the point \emph{after} , where the indices are taken modulo .
For any blob  in this radial ordering, the \emph{first point of } is the point  such that .
Similarly, the \emph{last point of } is the point  such that .
We say that  is the \emph{previous red (resp.\ blue) blob before}  if  is a blue blob and there is no red (resp.\ blue) point between the last point of  and the first point of  in the radial order.
Similarly, we say that  is the \emph{next red (resp.\ blue) blob after}  if there is no red (resp.\ blue) point between the the last point of  and the first point of  in the radial order.

\subsection{Jump Configurations}
If  is a blob, and  is the next blob after  of the same color as , then a \emph{jump edge} from  to  is a line segment  where ,  such that  and  see each other with respect to  and  and the angle between  and  with respect to the point  is at most .
This last condition ensures that each point  on the line segment  would, if added to the radial order, lie between  and . See Figure \ref{fig:jump-edges}.



\begin{figure}[h]
\centering
\includegraphics[width=5cm]{jump-edges.pdf}
\caption{Jump edges from a blob  to the next blob of the same color, .  to  is a jump edge.  to  is not a
jump edge; it intersects the interior of the convex hulls of the blobs
containing  and .
 to  is not a jump edge because the angle from  to  is
greater than .}
\label{fig:jump-edges}
\end{figure}

A \emph{jump configuration}  is a collection of jump edges, one blue edge from each blue blob to the next  blue blob and one red edge from each red blob to the next red blob, such that the blue edges do not cross each other, the red edges do not cross each other, and, for each blob , the two edges with endpoints in  do not share a common endpoint unless  contains only one point.

In this section we show that a jump configuration can be completed to a pair of non-intersecting spanning cycles of the two color classes by adding a spanning path within each blob, such that each edge of these cycles will be crossed at most three times, except where a specific structure, called a 4-forcing, appears.
4-forcings will be defined later in this section.

\begin{lemma}
  \label{lem:jump-config-blob-order}
  If there is a jump configuration  in the radial order with respect to , then for all blobs , the angle from the first point in  to the last point in  is strictly less than .
  Hence, for any  in the convex hull of , if  were added to the radial order, it would lie in the interval from the first point in  to the last point in , which implies that the convex hulls of the blobs are disjoint.

  \begin{proof}
    Let  be the jump edge in  between the blob before  and the blob after .
    By the definition of a jump edge, the angle from  to  is strictly less than .
    Also,  is contained in the interval between  and  in the radial order, so the angle from the first point in  to the last point in  is strictly less than the angle from  to , which is less than .
  \end{proof}
\end{lemma}


The following two lemmas describe how jump edges in a jump configuration can intersect, and how jump edges can intersect the convex hulls of blobs.

\begin{lemma}
  \label{lem:jump-config-crossing}
  If  is a blob, and, for ,  is the next blob after , then for any jump configuration , the only jump edges in  that can cross the jump edge from  to  are the jump edges from  to  and from  to .
  In particular, no jump edges of the same color cross, and each jump edge is crossed by at most two jump edges of the opposite color.

  \begin{proof}
    Let  be the jump edge from  to  in the jump configuration , and let  be any jump edge in  such that  and  meet at a point .
    By construction, if added to the radial order,  would lie between  and , and between  and .
    Therefore the interval  between  and  and the interval  between  and  intersect.
    If , then  must be a jump edge from  to , so .
    Otherwise, either  or , so, without loss of generality,  lies between  and  in the radial order.
    Therefore .
    If  and  is in the previous blob of the same color, then  and  do not cross by construction, and similarly if  and  is in the next blob of the same color.
    If , then  is either the jump edge from  to  or the jump edge from  to .
  \end{proof}
\end{lemma}

\begin{lemma}
  \label{lem:jump-config-cross-blob}
  If  is a blob, and, for ,  is the next blob after , then for any jump configuration , the only jump edges that can intersect the convex hull of  are the jump edges from  to , from  to , and from  to .

  \begin{proof}
    Let  be the jump edge between blobs  and  in the jump configuration  such that  intersects the convex hull of  at a point .
    Then , if added to the radial order, would lie between the first point of  and the last point of .
    But  lies between a point in  and a point in  in the radial order, so  must be , , or the blob between  and  in the radial order.
  \end{proof}
\end{lemma}

When a jump edge passes through the convex hull of a blob  (which can only happen when it is the jump edge from the previous blob before  to the next blob after ), we want to find a spanning path within  that crosses the jump edge as few times as possible. The following lemma gives a construction for such a path.

\begin{lemma}
  \label{lem:spanning-path-crossing}
  If  is a finite set of points in general position,  are distinct points, and  is a line, then there exists a non-self-intersecting spanning path  of  such that
  \begin{enumerate}[(i)]
  \item if  lies entirely on one side of , then  does not cross ;
  \item if  and  lie on opposite sides of , then  crosses  exactly once;
  \item if  and  lie on the same side of , and  contains points on both sides of , then  crosses  exactly twice.
  \end{enumerate}
  
  \begin{proof}
    By induction on .
    If , then , and so the one-edge path from  to  crosses  zero times, if  and  are on the same side of ; or once if  and  are on opposite sides of .

    If , then  contains at least two points, and so  sees at least two points of .
    One of these points, , is not .
    If there are multiple choices for , choose  to lie on the same side of  as .
    By the induction hypothesis, there exists a spanning path  of  from  to  satisfying (i) (ii) and (iii).
     lies entirely within the convex hull of , and the edge from  to  intersects the convex hull only at , so  is a non-self-intersecting spanning path of  from  to .
    
    If  lies entirely on one side of , then  lies entirely on one side of  because it is contained in the convex hull of , so (i) holds.

    Now suppose  and  lie on opposite sides of .
    If  lies on the same side of  as  then  crosses  as many times as , which is once.
    If  lies on the opposite side of  from , then  cannot contain any point on the same side of  as , or else  would see some point  on the same side of , and  because  is on the opposite side of , so  would have been chosen over .
    Therefore when  lies on the opposite side of  from ,  does not cross , so  crosses  exactly once.
    
    Finally, suppose  and  lie on the same side of , and that  contains some point  on the opposite side of  from  and .
    If  lies on the same side of  as  and , then  crosses  as many times as , which is twice.
    If  lies on the opposite side of  from  and , then  crosses  once, so  crosses  twice.
  \end{proof}
\end{lemma}



We now show how to construct a pair of monochromatic spanning cycles from a jump configuration by adding spanning paths within each blob.

The next lemma shows that there is exactly one case in which adding spanning paths can force our monochromatic spanning cycles to have edges that are crossed 4 times. We call this case a \emph{4-forcing} and define it as follows:
Suppose we are given a jump configuration . Let  be consecutive blobs such that the jump edge  between  and  in  intersects the convex hull of  and crosses both the jump edge from  to  and the jump edge from  to , and such that the endpoints in  of the jump edges from  and to  both lie on the same side of .
Then this is called a \emph{4-forcing} in , and  is called the \emph{center edge} of the 4-forcing.
See Figure \ref{fig:4-forcing}.

\begin{figure}[h]
\centering
\includegraphics[width=6cm]{4-forcing.pdf}
\caption{4-forcing}
\label{fig:4-forcing}
\end{figure}

\begin{lemma}
  \label{lem:complete-jump-config}
  Given any jump configuration ,
  spanning paths of each blob can be added to  to construct a pair of non-self-intersecting spanning cycles of the red and blue points respectively such that, if an edge  is crossed more than three times by the opposite color cycle, then  is in  (and not in one of the spanning paths), and  is the center edge of a 4-forcing.
  \begin{proof}
For each blob , let  be the point incident with the jump edge to  from the previous blob of the same color, and let  be the point incident with the jump edge from  to the next blob of the same color.
    If  is the previous blob before  and  is the next blob after , let  be the line through  and , and note that  is the only jump edge that can cross through the convex hull of .
    Let  be a spanning path of  from  to  minimizing the number of crossings of the .

    By Lemma \ref{lem:jump-config-cross-blob}, no jump edge of the same color as  can cross the convex hull of , and hence no jump edge can cross .
    Also, for any blob , the convex hulls of  and  are disjoint, so  and  cannot cross.
    By Lemma \ref{lem:jump-config-crossing}, two jump edges of the same color cannot cross, so the edges of  X form a pair of non-self-intersecting spanning-cycles.

    Let  be an edge of one of these spanning cycles which is crossed at least 4 times by the other cycle.
    By Lemma \ref{lem:jump-config-cross-blob}, any edge of  can only be crossed by the jump edge , and hence cannot be crossed 4 times, so  is a jump edge.
    Without loss of generality, suppose  is the jump edge between  and .
    By Lemma \ref{lem:jump-config-crossing},  is crossed by at most two other jump edges (the two jump edges to and from ).
    The only blob for which  could cross the convex hull is , so the only blob whos spanning-path  could cross is .
    By Lemma \ref{lem:spanning-path-crossing},  crosses  at most two times, with equality only if  and  lie on the same side of .
    Therefore  must be crossed by both of the jump edges incident with , so  is the center edge of a 4-forcing.
  \end{proof}
\end{lemma}

In the remainder of the paper, we focus on finding a good jump configuration such that when we add spanning cycles of each blob, as in the previous lemma, we have no 4-crossings \textemdash this will result in a pair of monochromatic spanning cycles in which each edge is crossed at most 3 times. In the following sections we show that it is possible to avoid 4-crossings by choosing  carefully.

\section{Monster-Jumps}

If  are consecutive blobs such that  and  are blue and  and  are red, then we say that  to  is a \emph{red monster-jump} if  and the angle from the \emph{second} point in  to the \emph{first} point in  is at least , and the line segment between the last point in  and the first point in  intersects the convex hull of . See Figure \ref{fig:red-monster-jump}.

If  are consecutive blobs such that  and  are blue and  and  are red, then we say that  to  is a \emph{blue monster-jump} if  and the angle from the \emph{last} point in  to the \emph{second to last} point in  (i.e., the point before the last point in ) is at least , and the line segment between the last point in  and the first point in  intersects the convex hull of . See Figure \ref{fig:blue-monster-jump}.

Note the slight asymmetry between the definition of red and blue monster-jump.
In particular, if the colors are reversed and the plane reflected, then red monster-jumps become blue monster-jumps, and vice-versa.

\begin{figure}
\centering
\begin{subfigure}[b]{0.4\textwidth}
\centering
\includegraphics[width=\textwidth]{red-monster-jump.pdf}
\caption{Red monster-jump between  and .}
\label{fig:red-monster-jump}
\end{subfigure}
\quad
\begin{subfigure}[b]{0.4\textwidth}
\centering
\includegraphics[width=\textwidth]{blue-monster-jump.pdf}
\caption{Blue monster-jump between  and .}
\label{fig:blue-monster-jump}
\end{subfigure}
\caption{Red and blue monster-jumps}
\label{fig:monster-jumps}
\end{figure}

In this section we show that if  is chosen such that the radial ordering about  has no monster-jumps, then we can construct a jump configuration which can be completed to a pair of monochromatic spanning cycles with no 4-crossings. 

\subsection{Monster-Jumps and 4-forcings}
\begin{lemma}
  \label{lem:jump-config-exists}
  Suppose that, for each blob , the angle from the last point in  to the first point in the next blob of the same color is less than . Then the collection of jump-edges consisting of, for each blob , the edge from the last point of  to the first point of the next blob of the same color, is a valid jump configuration.
  \begin{proof}
    For each blob , let  be the first point in , and  be the last point in .
    Then for each blob , if  is the next blob of the same color, then the angle from  to  is less than .
    Therefore every point on the line segment  would, if added to the radial order, lie in the interval between  and .
    If  is the blob before  and  is the blob after , then  is contained in the interval between the last point of  and the first point of , which has angle less than , so, for any point  in the convex hull of , if  was added to the radial order,  would lie between the first point of  and the last point of .
    Hence,  can only intersect the convex hull of  and .
    Similarly,  can only intersect the convex hull of  at .
    Therefore  is a valid jump edge.

    If  is the next blob of the same color after , then the interval between  and  and the interval between  and  are either disjoint, if , or meet at the point , if .
    Therefore the line segments  and  are either disjoint, if ; or meet at , if .
    Thus, the collection  of jump edges consisting of, for each blob  with next blob of the same color , , is a valid jump configuration.
  \end{proof}
\end{lemma}

\begin{figure}
\centering
\includegraphics[width=9cm]{no-4-forcing-proof.pdf}
\caption{In the proof of Lemma \ref{lem:no-4-forcing}, if there is a 4-forcing on the red jump edge between  and , then the blue-red crossing between the jump edge from  to  and the jump edge from  to  can be removed by replacing  by  and replacing  by .}
\label{fig:no-4-forcing-proof}
\end{figure}

\begin{lemma}
  \label{lem:no-4-forcing}
  If a radial order contains no red monster-jump and no blue monster-jump and, for each blob , the angle from the last point in  to the first point in the next blob of the same color is less than , then there exists a jump configuration which contains no 4-forcing.
  \begin{proof}
    A \emph{blue-red crossing} in a jump configuration occurs when there are consecutive blobs  such that  and  are blue and  and  are red, and the jump edge from  to  crosses the jump edge from  to .

    For each red blob , let  be the first point in .
    For each blue blob , let  be the last point in .

    For each red blob  choose , and for each blue blob  choose  such that the collection  of edges consisting of, for each blob , , where  is the next blob of the same color, is a valid jump configuration, and that the number of blue-red crossings is minimized, with ties broken by minimizing the number of 4-crossings.
    Note that such a jump configuration exists by Lemma \ref{lem:jump-config-exists}.

    Suppose that the jump configuration  contains a 4-forcing.
    By swapping the colors and reflecting the plane, if necessary (so that we preserve the fact that there are no red or blue monster-jumps), we may assume that there are consecutive blobs  such that  are blue and  are red, and that the red jump edge  from  to  is the center edge of a 4-forcing.
    Then  and  both cross ,  and  are on the same side of the line  through  and , and  contains points on both sides of .
    
    If  is the last point in , then replacing  with  in  will give another valid jump configuration , and cannot increase the number of blue-red crossings, because the only blue-red crossing that a jump edge from  to  could be involved in is with the jump edge from  to , which  already crosses.
    Therefore, by minimality of ,  must be the center edge of a 4-forcing in .

    This means that  and  are both on the same side of the line  through  and , and  contains points on both sides of , and that  and  are on the opposite side of  from  and .
    There is some point  on the same side of  as , which  sees with respect to .
    The angle from  to  is at most the angle from  to the second to last point in , which is less than  because  to  is not a blue monster-jump.
    Thus  sees  with respect to , so  is a valid jump edge, and replacing  by  in  gives a valid jump configuration .
    The only blue-red crossing which  can be involved in is with a jump edge from  to , and  does not cross , so  has fewer blue-red crossings than the chosen jump configuration , a contradiction.
    See Figure \ref{fig:no-4-forcing-proof}.
  \end{proof}
\end{lemma}


\subsection{Avoiding Monster-Jumps}
We have shown that if we choose  such that the radial order about  gives us a jump configuration with no monster-jumps, then we can complete this jump configuration to a pair of monochromatic spanning cycles with no 4-crossings. In this section we show that it is possible to choose a point  that avoids monster-jumps.
This is broken down into two cases: when the blue and red convex hulls properly overlap (Lemma \ref{lem:no-monster-jump-overlap}) and when the red convex hull contains the blue convex hull (Lemma \ref{lem:no-monster-jump-contains}).
The only other alternative is that the red and blue convex hulls are disjoint, in which case the desired spanning cycles exist trivially.

\begin{figure}
\centering
\includegraphics[width=4cm]{overlap-case.pdf}
\caption{The choice of  when the red and blue convex hulls properly overlap (Lemma \ref{lem:no-monster-jump-overlap}).}
\label{fig:overlap-case}
\end{figure}

\begin{lemma}
  \label{lem:no-monster-jump-overlap}
  Suppose the bichromatic point set  contains at least three red points and at least three blue points and that the convex hulls of the red points and of the blue points properly overlap; i.e., the intersection of the convex hulls is non-empty, and the blue convex hull is not contained in the red convex hull, nor vice-versa.
  Then after possibly swapping the color classes, there exists a point  in the intersection of the interior of the convex hulls such that the radial order about  contains neither a red nor a blue monster-jump.
  \begin{proof}
    Note that because the convex hulls properly overlap, the boundaries of the convex hulls must intersect at some point .
    Let  be the red segment containing , such that  follows  in the clockwise ordering of the vertices of the red convex hull.
    Similarly, let  be the blue segment containing , such that  follows  in the clockwise ordering of the vertices of the blue convex hull.
    By swapping colors if necessary, we may assume that  appear in that order in the (clockwise) radial order about .
    Therefore  is outside the blue convex hull and  is outside the red convex hull.

    For all , there is a point  in the intersection of the interiors of the red and blue convex hulls such that .

    We will show that for  sufficiently small, there is no red or blue monster-jump in the radial order about .
    If  is sufficiently small, then the radial orders of the bichromatic point set with respect to  and with respect to  coincide.
    
    Any point between  and  in the radial order with respect to  is outside the blue convex hull, and hence must be red.
    Therefore there is exactly one red blob  between  and , and .
    Similarly there is exactly one blue blob, , between  and , and .

    Let  be the blue blob containing  and  be the red blob containing .
    Note that  is the last point in  and  is the first point in , and the blob  lies entirely on one side of the line through  and , so  to  is not a red monster-jump.
    Similarly,  is the last point in ,  is the first point in , and  lies entirely on one side of the line between  and , so  to  is not a blue monster-jump.
    
    Let  be any blue blob which is not , and let  be the next blue blob after .
    If , then  to  is not a blue monster-jump, so suppose .
    Then the last point, , in  and the second to last point, , in  lie in the interval between  and  in the radial order with respect to .
    Note that  because  and , because  is the not the second to last point in .
    Therefore if  is the point after  in the radial order with respect to , then the angle from  to  is at most the angle from  to , which, for  sufficiently small, is less than .
    So,  to  is not a blue monster-jump.
    By a symmetric argument, if  is a red blob which is not  and  is the next red blob after  and  is sufficiently small, then  to  is not a red monster-jump.
    Therefore, if  is sufficiently small, then the radial order of the bichromatic point set with respect to  contains no red or blue monster-jump.
  \end{proof}
\end{lemma}

\begin{lemma}
  \label{lem:no-monster-jump-contains}
  Suppose the bichromatic point set  contains at least three red points and at least three blue points and that the red convex hull contains the blue convex hull.
  Then there exists a point  in the interior of the blue convex hull such that the radial order of the bichromatic point set with respect to  contains no red or blue monster-jump.

  \begin{proof}
    Let  be the vertices of the blue convex hull in clockwise order.
    For , let  be the open half-plane bounded by the line through  and  which contains no blue points.
    Note that  is the complement of the blue convex hull, and so, for some ,  contains a red point.
    Without loss of generality, there is some red point .
    Because  is in the interior of the red convex hull, there is some red point in ; choose a red point  in  minimizing the angle from  to  with respect to . 

\begin{figure}
\centering
\includegraphics[width=4.5cm]{containment-case-1.pdf}
\caption{Case 1: The blue convex hull is contained in the red convex hull, and
there are red points  in  and  in .}
\label{fig:containment-case-1}
\end{figure}

    \textbf{Case 1:}  (see Figure \ref{fig:containment-case-1}). For , let  be the vector  rotated counter-clockwise by , and, for , let , and consider the radial order of the bichromatic points with respect to .
    For , let  be the blue blob containing .
    For , let  be the red blob containing .

    If  and  are sufficiently small,  lies entirely on one side of the line through  and , which are the first point of  and last point of  respectively, so  to  is not a red monster-jump.
    Again, if  and  are sufficiently small,  lies entirely on one side of the line through  and , so  to the next red blob is not a red monster-jump.
    If  is a red blob that is not  or  and  is the next red blob and  and  are sufficiently small, then  lies entirely on the same side of the line through  and  as , so the angle between the second point of  and the first point of  is at most the angle between the second point of  and , which is less than .
    Therefore the radial order with respect to  contains no red monster-jump.

    For  and  sufficiently small, the points before and after  are both red, so .
    Therefore  to  is not a blue monster-jump.
    Let  be the previous blue point before  in the radial order.
    For any blue blob  that is not , if  is the next blue blob, then the angle between the last point  and the second last point of  is at most the angle between  and , which is less than  if  and  are sufficiently small.
    Therefore the radial order with respect to  contains no blue monster-jump.

\begin{figure}
\centering
\includegraphics[width=5cm]{containment-case-2.pdf}
\caption{Case 2: The blue convex hull is contained in the red convex hull and
there are red points  in  and  in .}
\label{fig:containment-case-2}
\end{figure}


    \textbf{Case 2:}  (see Figure \ref{fig:containment-case-2}).
    Let  be the line through  and .
    Because  is in the interior of the red convex hull, there is a red point  on the opposite side of  from .
    However,  cannot lie between  and  in the radial order with respect to , so  lies between  and  in the radial order with respect to , and the angle from  to  with respect to  is less than .
    
    For , let  be the vector from  to  rotated counter-clockwise by .
    As in case 1, for , let , and consider the radial order of the bichromatic points with respect to .
    For , let  be the blue blob containing , and, for , let  be the red blob containing .

    If  and  are sufficiently small, then  lies entirely on one side of the line between  and , which are the last point of  and first point of  respectively, so  to the next red blob is not a red monster-jump.
    For  and  sufficiently small,  is the point before  and  is the point after  so . Therefore,  to the next red blob is not a red monster-jump.
    If  is a red blob that is neither  nor , and  is the next red blob, then  lies entirely on the same side of the line through  and  as , so the angle from the second point in  to the first point in  is at most the angle from  to , which is less than .
    Therefore the radial order with respect to  contains no blue monster-jump.

    If  and  are sufficiently small, then the points before and after  are both red, so .
    Therefore  to  is not a blue monster-jump.
    If  and  are sufficiently small, then  lies between  and , and the angle from  to  is less than .
    The blob  ends before , so the angle from  to any point in  is less than , and  to  is not a blue monster-jump.
    If  is a blue blob that is not  or , and  is the next blue blob, then  and  lie between  and  in the radial order, so the angle from any point in  to any point in  is less than , and hence  to  is not a blue monster-jump.
    Therefore the radial order with respect to  contains no blue monster-jump.

    In both cases,  can be chosen such that the radial order with respect to  contains no red monster-jump and no blue monster-jump.
  \end{proof}
\end{lemma}

\begin{proof}[Proof of Theorem \ref{thm:main}]
If the red convex hull and the blue convex hull are disjoint, then any pair of red and blue spanning cycle will be disjoint, so assume the red and blue convex hulls intersect.
By Lemma \ref{lem:complete-jump-config}, it suffices to show that for some point , the radial order about  contains a jump-configuration with no 4-forcing.
By Lemma \ref{lem:no-4-forcing}, it suffices to show that there exists a point  such that the radial order about  contains no red or blue monster-jump.

Either the red and blue convex hulls properly overlap, or the blue convex hull is contained in the red convex hull, or the red convex hull is contained in the blue convex hull.
If the red and blue convex hulls properly overlap, then by Lemma \ref{lem:no-monster-jump-overlap}, there is a point  such that the radial order with respect to  contains no red or blue monster-jump.
If the red convex hull is contained in the blue convex hull, we may swap the colors, so that the blue convex hull is contained in the red convex hull, and in that case, by Lemma \ref{lem:no-monster-jump-contains}, there is a point  such that the radial order with respect to  contains no red or blue monster-jump.
\end{proof}

\bibliography{references}{}
\bibliographystyle{plain}
\end{document}