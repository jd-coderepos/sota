\documentclass{eptcs} 

\providecommand{\event}{}
\providecommand{\volume}{??}
\providecommand{\anno}{20??}
\providecommand{\firstpage}{1}



\usepackage{amssymb}
\usepackage{url}
\usepackage{times}
\usepackage{epsfig}
\usepackage{breakurl}
\usepackage{color}

\makeatletter

\def\hmo{{\cal O}}
\def\hmofin{\hmo_{FIN}}
\def\hmofdp{\hmo_{FDP}}

\def\hml{{\it HML}}
\def\hmli{{\it HML}}
\def\hmlfin{{\it HML}_{\it FIN}}
\def\hmlbd{{\it HML}_{\it FDP}}

\def\proc{\mathbb{P}}

\def\eqhmlfin{\sim_{\hml}^{\it FIN}}
\def\eqhmli{\sim_{\hml}}
\def\eqhml{\sim_{\hml}}
\def\eqhmlfdp{\sim_{\hml}^{\it FDP}}

\def\eqhmofin{\sim_{\hmo}^{\it FIN}}
\def\eqhmo{\sim_{\hmo}}
\def\eqhmofdp{\sim_{\hmo}^{\it FDP}}

\def\nat{\mathbb{N}}
\def\hmeq{\sim_{\cal O}}
\def\hmoi{{\cal O}^{\infty}}
\def\enc{\partial}
\def\encone{\partial^{1}}
\def\enca{\partial_{a}}
\def\transa{\stackrel{a}{\rightarrow}}
\def\transaeen{\stackrel{a_1}{\rightarrow}}
\def\transatwee{\stackrel{a_2}{\rightarrow}}
\def\transan{\stackrel{a_n}{\rightarrow}}
\def\transa{\stackrel{a}{\rightarrow}}
\def\transb{\stackrel{b}{\rightarrow}}
\def\proofend{\begin{flushright}\framebox{}\end{flushright}}
\def\iff{\Leftrightarrow}
\def\iffdef{\stackrel{\rm def}\Leftrightarrow}
\def\eqdef{\stackrel{\rm def}=}
\def\implies{\Rightarrow}
\def\true{{\sf T}}
\def\false{{\sf F}}

\newtheorem{defi}{Definition}
\newtheorem{theo}{Theorem}
\newtheorem{prop}{Proposition}
\newtheorem{lemm}{Lemma}
\newtheorem{coro}{Corollary}

\newenvironment{definition}{\begin{defi} \rm }{\end{defi}}
\newenvironment{theorem}{\begin{theo} \rm }{\end{theo}}
\newenvironment{proposition}{\begin{prop} \rm }{\end{prop}}
\newenvironment{lemma}{\begin{lemm} \rm }{\end{lemm}}
\newenvironment{corollary}{\begin{coro} \rm }{\end{coro}}
\newenvironment{proof}{\begin{trivlist} \item[\hspace{\labelsep}\bf Proof:]}{\hfill  \end{trivlist}}

\def\foralli{\forall_{i \in I}}

\newcommand{\diam}[1]{\langle#1\rangle}
\newcommand{\lmerge}[2]{#1 ||_{\_} #2}

\title{Congruence from the Operator's Point of View:\\
Compositionality Requirements on Process Semantics}
\author{Maciej Gazda \& Wan Fokkink
\institute{Vrije Universiteit\\
Department of Computer Science\\
De Boelelaan 1081a, 1081 HV Amsterdam, Netherlands}
\email{m.w.gazda@student.vu.nl,~wanf@cs.vu.nl}
}
\def\authorrunning{M. Gazda and W. Fokkink}
\def\titlerunning{Compositionality Requirements on Modal Characterizations of Process Semantics}

\begin{document}
\maketitle
\begin{abstract}
One of the basic sanity properties of a behavioural semantics is that it constitutes
a congruence with respect to standard process operators. This 
issue has been traditionally addressed by the development of rule formats 
for transition system specifications that define process algebras. In 
this paper we suggest a novel, orthogonal approach. Namely, we focus on a 
number of process operators, and for each of them attempt to find the 
widest possible class of congruences. To this end, we impose restrictions 
on sublanguages of Hennessy-Milner logic, so that a semantics whose 
modal characterization satisfies a given criterion is guaranteed to be 
a congruence with respect to the operator in question. We investigate action 
prefix, alternative composition, two restriction operators, and parallel composition.
\end{abstract}

\section{Introduction}

Congruence is one of the most important properties of a behavioural semantics. The reason is that the fundamental issue in process algebra - providing sound and complete axiomatisations for collections of process operators - requires that these operators are compositional. Only then we can use equational logic priciples and provide sound axioms.

There is a large amount of research to find ways of ensuring the congruence property. The basic methodology is to impose restrictions on operator definitions; there is a notion of a rule format for transition system specifications which provide operational semantics for process algebras. If a process operator is defined with rules that fit within a format, then the semantics in question is a congruence with respect to this operator. Examples include the panth format for bisimulation semantics \cite{Ver95} and formats designed specifically for several decorated trace semantics \cite{BlFoGl04}. The focus here is on \textit{semantics}; rule formats are most often defined with one particular process semantics in mind. Interestingly, in \cite{BlFoGl04}, the modal characterization of a process semantics is taken as starting point to derive the syntactic constraints of the congruence format for this semantics. A modal characterization of a semantics is a sublanguage of Hennessy-Milner logic such that two processes are semantically equivalent if and only if they satisfy exactly the same formulas in the modal characterization of the semantics. For almost all process semantics in van Glabbeek's spectrum \cite{Gla01} there is a corresponding modal characterization.
 
In this paper, we attempt to look at the compositionality issue from an \textit{operator's} point of view. For a number of basic process operators, we determine conditions that a process semantics should satisfy in order to be congruence with respect to such an operator. To be more precise, given a process operator, we develop syntactic constraints on modal characterizations; if the modal characterization of a process semantics satisfies these constraints, then the process operator is guaranteed to be compositional with respect to this semantics. So instead of going from a process semantics to a class of transition system specifications for which that semantics is a congruence, we go from the transition rules of a process operator to a class of process semantics for which this operator is compositional. This approach gives us an orthogonal view on compositionality, and provides further insight into connections between process algebra and modal logic.

\section{Preliminaries}

We work in the usual setting of labelled transition systems (LTSs), which consist of a set  of states  (also called processes), a set 
of actions , and a set of transitions .


\subsection{Hennessy-Milner logic}
\label{sec:hml}

Hennessy-Milner logic () \cite{HeMi85} is a modal logic for specifying properties of states in an LTS.
There exist different versions of  \cite{HeMi85,Gla01,BlRiVe01}. The choice of syntax is important here,
even if two logics have the same expressivity; compositionality requirements established for some version of 
(e.g.\ with diamond, conjunction and negation only) may become insufficient when we add other operators
(e.g.\ box), because these extra operators may require syntactic requirements of their own.
Our point of departure is the infinitary  variant without box and disjunction. The  syntax is therefore as follows:
\begin{center}

\end{center}
where  is an arbitrary index set, and  ranges over the set {\it Act} of actions.
Furthermore, we use  as an abbreviation for . We introduce some additional notations, based on the standard notion of context. A context, notation , is a  formula with one occurrence of . A \textit{multicontext } is a  formula containing one or more  symbols, indexed by the elements from . For a (multi)context , a formula is obtained by replacing the  symbols with formulas .
Finally, we introduce an \textit{-level context}, which means that the context symbol has  diamond operators above it. It is defined inductively as follows:
 \begin{itemize}
 \item  is a -level context;
 \item if  is an -level context, then  and  are -level contexts;
 \item if  is an -level context, then  is an ()-level context.
 \end{itemize}
An example of a -level context is , while  is a -level context.

A sublanguage  of  gives rise to a process equivalence by identifying those processes which satisfy exactly the same formulas from :
\begin{center}
.
\end{center}
We call  a  modal characterization of . Below, examples of modal characterizations of standard process equivalences from the literature are given (see \cite{Gla01}):
\begin{itemize}
	\item \textit{trace observations}:\\
	 
	\item \textit{completed trace obervations}:\\
	 
	\item \textit{failures observations}:\\
	 
  \item \textit{readiness observations}:\\	 
	
	\item \textit{failure trace observations}:\\	 
	
	\item \textit{ready trace observations}:\\	
	
	\item \textit{simulation observations}:\\	
	
	\item \textit{ready simulation observations}:\\	 
	
	\item \textit{-nested simulation observations for }:\\	 
	
	\item \textit{ bisimulation observations}:\\
	
	\end{itemize}

\vspace{2mm}
We write  if  for any process  in any LTS.
Given an , we write  for the set of HML formulas  for which there
exists a  with .


\subsection{BCCSP}

Any LTS isomorphic with a finite tree can be described with the following process algebra BCCSP, consisting of three operators:
	
\begin{itemize}
\item a nullary process  which does not have any behaviour;
\item action prefix  for : a unary operator which represents execution of a single action followed by the process given as the argument, defined by the transition rule
\begin{center}

\end{center}
\item alternative composition (), a nondeterministic choice between two processes, defined by the transition rules
\begin{center}

\end{center}
\end{itemize}
	
In this paper we focus on several process operators from the literature, and try to establish which syntactic properties a modal language  should satisfy to guarantee that the induced equivalence is a congruence with respect to the given operator. That is, given a process operator , we will search for a syntactic condition  such that if  satisfies , then  is compositional with respect to .


\section{Basic operators}

\subsection {Alternative composition}

We start with alternative composition, which expresses a nondeterministic choice between two processes. We want to find a general property of a modal language that would guarantee congruence of the induced equivalence with respect to alternative composition. Our first observation is that the behaviour of an alternative composition  \textit{after performing the first step} is completely determined by the behaviour of one of the components. For example,  if and only if either  or . The only potential problem can occur when there is a formula with a conjunction at level 0 (i.e., not in the scope of an action prefix). For instance, consider .  We have  and , but . As it turns out, it suffices to simply close the language on sub-conjunctions at level 0.
  
\begin{theorem} 
Let . If for any 0-level context  and  for ,
\begin{center}
(AC)~~~~~ implies that ,
\end{center}
then  is a congruence with respect to alternative composition ().
\end{theorem}
\begin{proof} Assume a modal language  with the AC property. Let  and . We show that for any :
\begin{center}

\end{center}
(the converse implication "" is symmetric).  We apply induction on the structure of . The base case () is trivial. We proceed with the inductive step. Assume that . We have to consider the following cases:
 
\begin{itemize}
\item : then either  or . From the equivalence of components we have either  or , which yields .

\item : we have  (AC + inductive hypothesis) .

\item : let  be the outermost subformula of  which does not begin with a "" symbol (so ). Then  is logically equivalent to either  or . The case  is trivial. Also, the case where  can be handled analogously as the first two cases. We thus have to consider two possibilities:

\begin{itemize}
\item : we have  (equivalence of components) .

\item : we have  (AC + inductive hypothesis) .
\end{itemize}

\end{itemize}
 
\end{proof} 


For example, consider . The language  satisfies the AC requirement, and so the corresponding equivalence  is a congruence with respect to . 

\vspace{2mm}

Almost all modal characterizations of standard process semantics from Section \ref{sec:hml} fulfill AC. The only exception is the modal characterization of completed trace equivalence,
although we can provide an alternative characterization that meets the AC requirement:
\begin{center}
 
\end{center}
The characterization  is the same as , except that it includes formulas . Clearly this does not change the corresponding semantics.
	 

\subsection{Action prefix}

In the case of action prefix, it is easy to obtain a sufficient congruence requirement; the crucial observatioin is that  if and only if , so we need to make sure that for each formula , the subformula  also belongs to the language . If this is not the case, an equivalence might not be a congruence. For instance, if , then , but .

\begin{theorem} Let  and fix . If for any 0-level context  and ,
\begin{center}
(AP)~~~~~ implies that ,
\end{center}
then  is a congruence with respect to the action prefix operator .
 \end{theorem}
\begin{proof}
Let . We need to show that for any , .

Take any . Let  such that the multicontext  does not contain any action prefix symbols. That is, the  for  are all action prefix subformulas of  that appear at level zero. Since  is built from only , conjunction and negation, whether a process satisfies  is completely determined by the satisfiability of  for  by this process. In other words, if , then . Coming back to our setting with  and , take an arbitrary . We have:\\
\\
\\
 (AP + )\\
.\\
The choice of  was arbitrary, hence the earlier remark yields: .
\end{proof}
The AP condition is satisfied by all modal characterizations from Section \ref{sec:hml}.


\section{Restriction operators: projection and encapsulation}

We now consider projection and encapsulation operators. The th projection of a process , for , mimicks the behaviour of  up to level :
\begin{center}

\end{center}
Applying encapsulation with parameter  removes all transitions whose labels are in  from the process:
\begin{center}

\end{center}
More generally, we consider unary restriction operators  such that given a process , the process  can be viewed as a subgraph of . Below we will give a precise description of which restriction operators are covered. For the projection operator  as well as for the encapsulation operator , given any  formula we can deduce in advance which of its subformulas  will always yield false, regardless of the process  or  for which the  formula is evaluated. In case of a process , any subformula  that appears at level  can be replaced by . And in case of a process , any subformula  with  can be replaced by .

We cannot reason in this way about any restriction operator. For example, consider the priority operator , which assumes a partial order  on the set of actions and allows us to execute an action only if no action with higher priority is executable at the same time:
\begin{center}

\end{center}
Suppose that  and there is a process  of which we only know that it satisfies . This knowledge is not sufficient to determine whether .

Let  be a unary operator such as  or . We would like to define for each formula  a corresponding formula  in which every subformula  which is known in advance to be unsatisfiable when evaluating any process  is replaced by . Actually this means that either we can replace a larger subformula by , or the entire formula becomes . Namely, we can replace the first innermost negation symbol (closest to the introduced ) and the following subformula by ; if the  symbol does not appear within the scope of a negation symbol, then the whole formula yields . If a language  is closed under , then it induces a congruence with respect to . The whole idea is made formal below.

\begin{lemma}
\label{lem:cut1} Let  be a unary process operator. Suppose there exists a function   such that for any process  and ,
\begin{center}
\begin{tabular}{l l}
(CUT)~~ & \\
\end{tabular}
\end{center}
Then for any language  satisfying
\begin{center}

\end{center}
the corresponding equivalence  is a congruence with respect to .
\end{lemma}
\begin{proof} Suppose  and . We have
 (CUT)  (either because  and , or because )  (CUT).
\end{proof}
The next lemma gives an explicit condition for a modal language to induce a congruence in case  formulas are obtained from the original ones by turning certain subformulas  into .

\begin{lemma}
\label{lem:cut2}
Assume  and  are as in Lem.~\ref{lem:cut1}, and satisfy CUT. Suppose that for each  there exists a multicontext  such that 
 and .
Then for each language  that satisfies for any context  and ,
\begin{center}
\begin{tabular}{l l}
(RES) &  implies ,
\end{tabular}
\end{center}
the corresponding equivalence  is a congruence with respect to .
\end{lemma}
\begin{proof} By Lem.\ \ref{lem:cut1} it suffices to prove that for all  either  or . Take any  such that . By assumption,  for some multicontext . Since , clearly each occurrence of  in this formula must be within the scope of a negation symbol.
Hence , where we can choose contexts  for  such that in each ,  is not within the scope of a negation.
Then . Since  satisfies RES, . Hence .
\end{proof}
We have provided a compositionality framework for a general class of restriction operators. What remains is to provide  functions for the projection and encapsulation operators.

\begin{lemma} 
\label{lem:cut3}
The functions  defined below are proper cutting functions (i.e., they satisfy condition CUT of Lem.~\ref{lem:cut1}).\vspace{4mm}\\
a) For the projection operators  with :\vspace{2mm}\\
\begin{tabular}{lll}
 &
 &
 \vspace{2mm} \\
 & 
\end{tabular}
\\ \\ \\
b) For the encapsulation operators  with :\vspace{2mm}\\
\begin{tabular}{lll}
 &
 &
 \vspace{2mm} \\
 if  &
 if 
\end{tabular}
\end{lemma}
\begin{proof}
a) We prove CUT by induction on the structure of .
\begin{itemize}
\item :

  and .

\item :

We distinguish the cases  and .
Clearly  and .

If , then  (transition rule for )
 (structural induction)
 (definition of ).

\item :

 (structural induction)
 (definition of ).

\item :

 (structural induction)
  (definition of ).
\end{itemize}



b) Again we use structural induction on .
\begin{itemize}
\item :

  and .

\item :

Suppose first that . Then  (transition rule for )
and  (definition of ).

Suppose now that .
Then 
 (transition rule for )
 (structural induction)
 (definition of ).

\item :

 (structural induction)
 (definition of ).

\item :

 ~
 (structural induction)
  (definition of ).
\end{itemize}

\end{proof}
\begin{theorem}
For any language  satisfying RES, the corresponding equivalence  is a congruence with respect to the projection operators  (for ) and the encapsulation operators  (for ).
\end{theorem}

\begin{proof}
By Lem.\ \ref{lem:cut3}, the functions  and  satisfy CUT.
Observe that the  and  functions defined in the Lem.~\ref{lem:cut3} only replace certain subformulas  of the original formula with . So they meet the requirements of Lem.~\ref{lem:cut2}. Congruence is thus an immediate consequence of Lem.\ \ref{lem:cut2}.
\end{proof}

To demonstrate that the RES requirement is essential, consider the following counterexamples.
\begin{itemize}
\item
For projection, take . We have , but .
\item
 For encapsulation, take . We have , but .
\end{itemize}

The RES requirement is satisfied by every characterization from Section \ref{sec:hml}, except for completed trace observations. Completed trace equivalence is a congruence with respect to projection operators, but not encapsulation. Take for instance the completed trace equivalent processes  and . We have .



\section{Parallel composition ()}

We now consider the parallel composition operator (without communication). That is,  behaves as  where the left-merge operator is defined by
\begin{center}

\end{center}

Let us restrict for a moment to only trace formulas (meaning that conjunctions are disregarded). The following example shows that the requirement AP and even being closed under substrings is not sufficient (by a substring of  we mean a subsequence constisting of elements appearing \textit{consecutively} in ). 
Take . This language not only satisfies AP, but is also closed under prefixes and substrings (but not arbitrary subsequences). However, we have , but  while  does not satisfy this formula.

This example suggests that if a trace  belongs to , then all \textit{subsequences} of  must belong to the language as well. This is not unexpected; the behaviour of parallel composition consists of all possible interleavings of the component processes, and all of these interleavings should be described in the modal characterization.

It is also necessary to close the language on subconjunctions. Indeed, take , a language which does not meet this condition. We have , but  while  does not satisfy this formula.

In case of general  formulas, we first define a generalization of a subsequence for an arbitrary formula  by specifying a set of subformulas with possible replacement from a lower level. We thus define  as the smallest set of  formulas satisfying:
\begin{itemize}
\item
;
\item
.
\end{itemize}

We now define a tool to infer satisfaction of modal formulas by a parallel composition  from the formulas satisfied by the component processes  and . This is accomplished by the function , which given  and , returns the collection of formulas that are certainly satisfied by a parallel composition of two processes satisfying  and  respectively. One can view  as parallel composition operator on collections of modal formulas.

Formally,  is defined with induction on the structure of formulas.
\begin{itemize}
\item  
\item 
 

\item 

\item 

\end{itemize}

By abuse of notation, we let  also denote the formula .

\begin{lemma} Let .
\label{lem:par}
\begin{center}
.
\end{center}
\end{lemma}

\begin{proof} 
 We use induction on the structure of formulas. The base case () is immediate. We proceed with the inductive step:
 
: Assume that . We prove that .

\begin{itemize}
\item : Without loss of generality, suppose  (the case  is symmetric), so . From the inductive hypothesis we know that there are  such that . We take  and .

\item : By the inductive hypothesis, for each  there are  such that . We can take  and .

\item : We have:\\
  \\
 (inductive hypothesis)\\
.\\
We define:\\
\\
\\
These are the  and  we are looking for.
\end{itemize}

: Suppose that . We prove that .
\begin{itemize}

\item : From  we have 
. Without loss of generality suppose that . Then . From  and the inductive hypothesis we have . Since , we finally obtain .

\item : According to the definition of  we have . The inductive hypothesis yields , and hence .

\item :  We have . Suppose, towards a contradiction, that . Then according to the inductive hypothesis there exist  such that ,  and . But from the earlier remark, we have either  or . This contradicts the fact that  and .

\end{itemize}
\end{proof}

\begin{theorem} For any language  satisfying 
\begin{center}
(PAR)~~~~~,
\end{center}
  is a congruence with respect to parallel composition .
\end{theorem}
\begin{proof}
Suppose  and . Suppose that . According to Lem.~\ref{lem:par}, there exist  such that  and . Since , by condition PAR, . Since  and , it follows that  and . According to Lem.~\ref{lem:par} this implies .

\end{proof}

As an example, if we want to define a modal language that would be a congruence with respect to parallel composition, which includes behaviour described by a formula , we should include the following formulas in the characterization (we omit irrelevant formulas like ): 
,
,
,
,
,

.

All basic equivalences except for completed trace have modal characterizations that satisfy the condition PAR. We note that parallel composition is compositional with respect to completed trace equivalence.

\section{Conclusions and future work}

We have presented, for a number of process operators from the literature, general conditions that guarantee congruence of process equivalences defined by means of a modal characterization. To the best of our knowledge it is the first such attempt.

Our conditions are sufficient, but by no means necessary. We believe that it is difficult (if not impossible) to provide a syntactic restriction on a modal language that would characterize the class of congruences for a given operator (strictly speaking, languages that induce congruences). We aimed at clear and comprehensible rather than slightly relaxed but more complicated conditions.

As the next step, we would like to investigate other process operators (e.g.\ sequential composition, renaming, merge with communication), consider the setting of weak semantics and different modal languages. In the last case, if we consider e.g.\  with recursion or the -calculus, we may attempt to combine our work with existing results on characteristic formulas \cite{AcInSa09}. In that setting, instead of modal language properties, we could focus on compositionality of single formulas.

\begin{thebibliography}{11}

\bibitem{AcFoVe01}
{L. Aceto, W.J. Fokkink \& C. Verhoef} (2001):
\newblock {\em Structural operational semantics.}
In (J.A. Bergstra, A. Ponse and S.A. Smolka, eds)
\newblock {\sl Handbook of Process Algebra}, Elsevier, pp.\ 197--292.

\bibitem{AcInSa09}
{L. Aceto, A. Ingolfsdottir \& J. Sack} (2009):
\newblock {\em Characteristic Formulae for Fixed-Point Semantics: A General Framework.}
In (D. Gorla and S. Fr\"oschle, eds)  \newblock {\sl Proc. EXPRESS'09}, EPTCS 8, pp.\ 1--15.

\bibitem{BlRiVe01}
{P. Blackburn, M. de Rijke \& Y. Venema} (2001):
\newblock {\em Modal Logic.}
\newblock Cambridge University Press.

\bibitem{BlFoGl04}
{B. Bloom, W.J. Fokkink \& R.J. van Glabbeek} (2004):
\newblock {\em Precongruence formats for decorated trace semantics.}
\newblock {\sl ACM Transactions on Computational Logic} 5(1), pp.\ 26--78.

\bibitem{FoGlWi06}
{W.J. Fokkink, R.J. van Glabbeek \& P. de Wind} (2006):
\newblock {\em Compositionality of Hennessy-Milner logic by structural operational semantics.}
\newblock {\sl Theoretical Computer Science}, 354(3), pp.\ 421--440. 

\bibitem{Gla01}
{R.J.~van Glabbeek} (2001):
\newblock {\em The linear time -- branching time spectrum {I}; the semantics of concrete, sequential processes.}
\newblock In J.A. Bergstra, A.~Ponse \& S.A. Smolka, editors: {\sl Handbook of Process Algebra}, Elsevier, pp.\ 3--99.

\bibitem{HeMi85}
{M. Hennessy \& R. Milner} (1985):
\newblock {\em Algebraic laws for non-determinism and concurrency.}
\newblock {\sl Journal of the ACM} 32(1), pp.\ 137--161.

\bibitem{Ver95}
{C. Verhoef} (1995):
\newblock {\em A congruence theorem for structured operational semantics with
predicates and negative premises.}
\newblock {\sl Nordic Journal of Computing} 2, pp.\ 274--302.

\end{thebibliography}
\end{document}
