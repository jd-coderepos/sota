\documentclass[final]{cvpr}

\usepackage{times}
\usepackage{epsfig}
\usepackage{graphicx}
\usepackage{amsmath}
\usepackage{amssymb}


\usepackage{subfigure}
\usepackage{appendix}
\usepackage{xcolor}
\usepackage{color}
\usepackage{mathtools}
\usepackage{multirow}
\usepackage{pifont}
\usepackage[procnames]{listings}
\usepackage{wrapfig}
\usepackage[numbers,compress,sort]{natbib}
\usepackage{booktabs}
\usepackage{ulem}
\newcommand{\snl}[1]{{\color{red} #1}}


\usepackage[pagebackref=true,breaklinks=true,colorlinks,bookmarks=false]{hyperref}


\def\confYear{CVPR 2021}

\begin{document}

ayer in the neural network, as the fused signal is propagated until the end. With PONO as the normalization method, we observe that the first layer ($\ell=1$) usually leads to the best result. In contrast, we find that \methodname{} is more suited for later layers when using IN~\citep{ulyanov2016instance}, GN~\citep{wu2018group}, or LN~\citep{ba2016layer} for moment extraction. 
Please see \shortautoref{sec:discussion} for a detailed ablation study. 
The inherent randomness of mini-batches allows us to implement \methodname{} very efficiently. For each input instance in the mini-batch $\mathbf{x}_i$ we compute the normalized features and moments  $\hat{\mathbf{h}}_i,\boldsymbol{\mu}_i,\boldsymbol{\sigma}_i$. Subsequently we sample a random permutation $\pi$ and apply \methodname{} with a random pair within the mini-batch
\(
    \mathbf{h}_i^{\left(\pi(i)\right)}\leftarrow 
    F^{-1}({\hat{\mathbf{h}}}_i,\boldsymbol{\mu}_{\pi(i)},\boldsymbol{\sigma}_{\pi(i)}).
\)
See Algorithm 1 in the Appendix for an example implementation in PyTorch~\citep{paszke2017automatic}. Note that all computations are extremely fast and only introduce negligible overhead during training. 


\textbf{Hyper-parameters.}
To control the intensity of our data augmentation, we perform \methodname{} during training with some probability $p$. In this way, the model can still see the original features with probability $1 - p$. In practice we found that $p = 0.5$ works well on most datasets except that we set $p = 1$ for ImageNet where we need stronger data augmentation.
The interpolation weight $\lambda$ is another hyper-parameter to be tuned. Empirically, we find that 0.9 works well across data sets. The reason can be that the moments contain less information than the normalized features.
Please see Appendix for a detailed ablation study. 


\textbf{Properties.} \methodname{} is performed entirely at the feature level inside the neural network and can be readily combined with other augmentation methods that operate on the raw input (pixels or words). For instance,  Cutmix~\cite{yun2019cutmix} typically works best when applied on the input pixels directly. We find that the improvements of \methodname{} are complimentary to such prior work and recommend to use \methodname{} in combination with established data augmentation methods. 

\vspace{-0.1in}



 \section{Experiments}
\label{sec:exp}

We evaluate the efficacy of \methodname{} thoroughly across several tasks and data modalities. Our implementation will be released as open source upon publication. 


\subsection{Image Classification on CIFAR}

\begin{table}[t]
    \centering
    \footnotesize
\begin{tabular}{c|c|c|c}
\toprule
Model & \#param. & CIFAR10 & CIFAR100 \\  
\midrule\midrule
ResNet-110 (3-stage) & 1.7M &  6.82$\pm$0.23& 26.28$\pm$0.10 \\
+\methodname{} & 1.7M & \textbf{6.03$\pm$0.24}& \textbf{25.47$\pm$0.09} \\

\midrule
DenseNet-BC-100 (k=12) & 0.8M & 4.67$\pm$0.10
& 22.61$\pm0.17$   \\
+\methodname{} &0.8M&\textbf{4.58$\pm$0.03} &  \textbf{21.38$\pm$0.18}\\
\midrule
ResNeXt-29 (8$\times$64d) & 34.4M &4.00$\pm$0.04 &18.54$\pm$0.27  \\
+\methodname{} &34.4M&\textbf{3.64$\pm$0.07}&\textbf{17.08$\pm$0.12}\\
\midrule
WRN-28-10 & 36.5M &  3.85$\pm$0.06& 18.67$\pm0.07$ \\
+\methodname{} & 36.5M & \textbf{3.31$\pm$0.03} &  \textbf{17.69$\pm$0.10} \\
\midrule
DenseNet-BC-190 (k=40) & 25.6M &3.31$\pm$0.04 &17.10$\pm$0.02  \\
+\methodname{} &25.6M&\textbf{2.87$\pm$0.03}& \textbf{16.09$\pm$0.14}\\
\midrule
PyramidNet-200 ($\tilde{\alpha}=240$) & 26.8M &3.65$\pm$0.10  & 16.51$\pm0.05$ \\
+\methodname{} & 26.8M & \textbf{3.44$\pm$0.03} &  \textbf{15.50$\pm$0.27} \\
\bottomrule
\end{tabular}
\caption{
Classification results (Err (\%)) on CIFAR-10, CIFAR-100 in comparison with various competitive baseline models. See text for details.
}
\label{tab:moex_cifar100_result1}     \vspace{-0.1in}
\end{table}

\begin{table}[t]
    \centering
    \begin{tabular}{l|c}
\toprule
PyramidNet-200 ($\tilde{\alpha}=240$) &Top-1 / Top-5\\
(\# params: 26.8 M) & Error (\%)\\
\midrule
\midrule
    Baseline &16.45 / 3.69 \\
    Manifold Mixup~\cite{zhang2017mixup}&16.14 / 4.07 \\
    StochDepth~\cite{huang2016deep}& 15.86 / 3.33\\
    DropBlock~\cite{ghiasi2018dropblock}&15.73 / 3.26 \\
    Mixup~\cite{zhang2017mixup}&15.63 / 3.99 \\
    ShakeDrop~\cite{yamada2018shakedrop} & 15.08 /\textbf{ 2.72}\\
    \methodname{}&\textbf{15.02} / 2.96\\
\midrule
    Cutout~\cite{devries2017cutout}&16.53 / 3.65  \\
    Cutout + \methodname{}& \textbf{15.11} / \textbf{3.23} \\
    \midrule
    CutMix~\cite{yun2019cutmix}& 14.47 / 2.97\\
    CutMix + \methodname{}& \textbf{13.95} / \textbf{2.95} \\
    \midrule
    CutMix + ShakeDrop~\cite{yamada2018shakedrop} & 13.81 / 2.29\\
    CutMix + ShakeDrop + \methodname{}&    \textbf{13.47} / \textbf{2.15} \\
\bottomrule
\end{tabular}
\caption{Combining \methodname{} with other regularization methods on CIFAR-100 following the setting of~\cite{yun2019cutmix}. The best numbers in each group are \textbf{bold}.}
\label{tab:moex_cifar100_result2}     \vspace{-0.2in}
\end{table}

\label{subsec:cifar}
CIFAR-10 and CIFAR-100~\cite{krizhevsky2009learning} are benchmark datasets containing 50K training and 10K test colored images at 32x32 resolution. We evaluate our method using various model architectures~\cite{He2015,huang2017densely,xie2017aggregated,zagoruyko2016wide,han2017deep} on CIFAR-10 and CIFAR-100. We follow the conventional setting\footnote{\href{https://github.com/bearpaw/pytorch-classification}{https://github.com/bearpaw/pytorch-classification}} with random translation as the default data augmentation and apply \methodname{} to the features after the first layer. 
Furthermore, to justify the compatibility of \methodname{} with other regularization methods, we follow the official setup\footnote{\href{https://github.com/clovaai/CutMix-PyTorch}{https://github.com/clovaai/CutMix-PyTorch}} of \citep{yun2019cutmix} and apply \methodname{} jointly with several regularization methods to PyramidNet-200~\cite{han2017deep} on CIFAR-100.

\autoref{tab:moex_cifar100_result1} displays the classification results on CIFAR-10 and CIFAR-100 with and without \methodname{}. We report mean and standard error over three runs~\cite{gurland1971simple}. \methodname{} consistently enhances the performance of all the baseline models. 
\autoref{tab:moex_cifar100_result2} demonstrates the CIFAR-100 classification results on the basis of PyramidNet-200. 
Compared to other augmentation methods, PyramidNet trained with \methodname{} obtains the lowest error rates in all-but one settings. However, significant additional improvements are achieved when \methodname{}  is combined with existing methods  --- setting a new  state-of-the-art for this particular benchmark task when combined with the two best performing alternatives, CutMix and ShakeDrop.






\subsection{Image Classification on ImageNet}

\begin{table}[t]
    \centering
    \begin{tabular}{c|c|c|c}
\toprule
& \multicolumn{1}{c}{\# of} & \multicolumn{2}{|c}{Test  Error (\%)} \\
\cline{3-4}
Model & epochs & Baseline &+\methodname{}  \\  
\midrule\midrule
ResNet-50   & 90 &23.6 &\textbf{23.1}  \\
ResNeXt-50 (32$\times$4d)& 90 & 22.2 &\textbf{21.4}   \\
DenseNet-265 &90&21.9 & \textbf{21.6}    \\
\midrule
ResNet-50 &  300& 23.1&\textbf{21.9} \\
ResNeXt-50 (32$\times$4d) &300 & 22.5 &\textbf{22.0}  \\
DenseNet-265 &300& 21.5&\textbf{20.9}  \\
\bottomrule
\end{tabular}
\caption{Classification results (Test Err (\%)) on ImageNet in comparison with various models.
Note: The ResNeXt-50 (32$\times$4d) models trained for 300 epochs overfit. They have higher training accuracy but lower test accuracy than the 90-epoch ones.}
\label{tab:moex_imagenet_baseline_result} \end{table}

\begin{table}[t]
    \centering
    \begin{tabular}{l|c|c}
    \toprule
    ResNet50  & \# of & Top-1 / Top-5\\
   (\# params:  25.6M) & epochs & Error (\%)\\
    \midrule
    \midrule
    ISDA~\cite{wang2019implicit} & 90 & 23.3 / 6.8 \\
    Shape-ResNet~\cite{geirhos2018imagenet} & 105 & 23.3 / 6.7 \\
    Mixup~\cite{zhang2017mixup} & 200 & 22.1 / 6.1  \\
    AutoAugment~\cite{cubuk2019autoaugment} & 270 & 22.4 / 6.2 \\
    Fast AutoAugment~\cite{lim2019fast} & 270 & 22.4 / 6.3 \\
    DropBlock~\cite{ghiasi2018dropblock} & 270 & 21.9 / 6.0 \\
    Cutout~\cite{devries2017cutout}& 300 & 22.9 / 6.7  \\ 
    Manifold Mixup~\cite{zhang2017mixup} & 300 & 22.5 / 6.2 \\
    Stochastic Depth~\cite{huang2016deep} & 300 & 22.5 / 6.3 \\ 
    CutMix~\cite{yun2019cutmix} & 300 & 21.4 / 5.9 \\
    \midrule
    Baseline & 300 & 23.1 / 6.6 \\  
    \methodname{} & 300 & 21.9 / 6.1 \\
    CutMix & 300 & 21.3 / \textbf{5.7} \\
    CutMix + \methodname{} & 300 & \textbf{20.9} / \textbf{5.7} \\
    \bottomrule
\end{tabular}
\caption{Comparison of regularization and augmentation methods on ImageNet. Stochastic Depth and Cutout results are from \cite{yun2019cutmix}.}
\label{tab:imagenet_da_result}     \vspace{-0.2in}
\end{table}

\label{subsec:imagenet}
We evaluate on ImageNet~\cite{deng2009imagenet} (ILSVRC 2012 version), which consists of 1.3M training images and 50K validation images of various resolutions. For faster convergence, we use NVIDIA's mixed-precision training code base\footnote{\href{https://github.com/NVIDIA/apex/tree/master/examples/imagenet}{https://github.com/NVIDIA/apex/tree/master/examples/imagenet}} with batch size 1024, default learning rate $0.1 \times \mbox{batch\_size} / 256$, cosine annealing learning rate scheduler~\cite{loshchilov2016sgdr} with linear warmup~\cite{goyal2017accurate} for the first 5 epochs. 
As the model might require more training updates to converge with data augmentation, we apply \methodname{} to ResNet-50, ResNeXt-50 (32$\times$4d), DenseNet-265 and train them for 90 and 300 epochs. For a fair comparison, we also report Cutmix~\cite{yun2019cutmix} under the same setting.
Since the test server of ImageNet is no longer available to the public, we follow the common practice~\cite{xie2017aggregated,huang2017densely,zhang2017mixup,yun2019cutmix,ghiasi2018dropblock} reporting the validation scores of the baselines as well as our method.


\autoref{tab:moex_imagenet_baseline_result} shows the test error  rates on the ImageNet data set. 
\methodname{} is able to improve the classification performance throughout, regardless of model architecture. Similar to the previous CIFAR experiments, we observe in  \autoref{tab:imagenet_da_result} that \methodname{} is  highly competitive when compared to existing regularization methods and truly shines when it  is combined with them. When applied jointly with CutMix (the  strongest alternative), we obtain our lowest Top-1 and Top-5 error of 20.9/5.7  respectively. Beyond, we apply \methodname{} to   EfficientNet-B0~\cite{tan2019efficientnet} and follow the official training scheme, we find \methodname{} is able to help reduce the error rate of baseline from 22.9 to 22.3. 



\subsection{Fintuneing Imagenet pretrained models on Pascal VOC for Object Detection}
\label{object_dectection}
To demonstrate that \methodname{} encourages models to learn better image representations, we apply models pre-trained on ImageNet with \methodname{} to downstream tasks including object detection on Pascal VOC 2007 dataset.
We use the Faster R-CNN~\citep{ren2015faster} with C4 or FPN~\citep{lin2017feature} backbones implemented in Detectron2~\citep{wu2019detectron2} and following their default training configurations.
We consider three ImageNet pretrained models: the ResNet-50 provided by \citet{He2015}, our ResNet-50 baseline trained for 300 epochs, our ResNet-50 trained with CutMix~\citep{yun2019cutmix}, and our ResNet-50 trained with \methodname{}.
A Faster R-CNN is initialized with these pretrained weights and finetuned on Pascal VOC 2007 + 2012 training data, tested on Pascal VOC 2007 test set, and evaluated with the PASCAL VOC style metric: average precision at IoU 50\% which we call AP\textsubscript{VOC} (or AP50 in detectron2). We also report MS COCO~\citep{lin2014microsoft} style average precision metric AP\textsubscript{COCO} which is recently considered as a better choice.
Notably, \methodname{} is not applied during finetuning.

\autoref{tab:voc} shows the average precision of different initializations. We discover that \methodname{} provides a better initialization than the baseline ResNet-50 and is competitive against CutMix\citep{yun2019cutmix} for the downstream cases and leads slightly better performance regardless of backbone architectures. 

\begin{table}[h]
    \centering
    \small
\begin{tabular}{c|l|c|c}
    \toprule
    Backbone & Initialization & AP\textsubscript{VOC} & AP\textsubscript{COCO} \\
    \midrule
    \multirow{3}{*}{C4}
    & ResNet-50 (default)  & 80.3  & 51.8 \\
    & ResNet-50 (300 epochs) & 81.2 & 53.5\\
    & ResNet-50 + CutMix & \textbf{82.1} & 54.3 \\
    & ResNet-50 + \methodname{} & 81.6 & \textbf{54.6} \\
    \midrule
    \multirow{3}{*}{FPN}
    & ResNet-50 (default) & 81.8 & 53.8 \\
    & ResNet-50 (300 epochs) & 82.0 & 54.2 \\
    & ResNet-50 + CutMix & 82.1 & \textbf{54.3}  \\
    & ResNet-50 + \methodname{} & \textbf{82.3} & \textbf{54.3} \\
    \bottomrule
\end{tabular}
\caption{Object detection on PASCAL VOC 2007 test set using Faster R-CNN whose backbone is initialized with different pretrained weights. We use either the original C4 or feature pyramid network~\citep{lin2017feature} backbone.}
\label{tab:voc}     \vspace{-0.15in}
\end{table}

\subsection{3D model classification on ModelNet}
\label{subsec:modelnet}
We conduct experiments on Princeton ModelNet10 and ModelNet40 datasets~\cite{wu20153d} for 3D model classification. This task aims to classify 3D models encoded as 3D point clouds into 10 or 40 categories. As a proof of concept, we use PointNet++ (SSG)~\cite{qi2017pointnet++} implemented efficiently in PyTorch Geometric\footnote{\href{https://github.com/rusty1s/pytorch_geometric/blob/master/examples/pointnet2_classification.py}{https://github.com/rusty1s/pytorch\_geometric}}~\cite{Fey/Lenssen/2019} as the baseline. It does not use surface normal as additional inputs.
We apply \methodname{} to the features after the first set abstraction layer in PointNet++. Following their default setting, all models are trained with ADAM~\cite{kingma2014adam} at batch size 32 for 200 epochs. The learning rate is set to 0.001. We tune the hyper-parameters of \methodname{} on ModelNet-10 and apply the same hyper-parameters to ModelNet-40.  We choose $p = 0.5$, $\lambda = 0.9$, and InstanceNorm\footnote{We do hyper-parameter search from $p \in \{0.5, 1\}$, $\lambda \in \{0.5, 0.9\}$ and whether to use PONO or InstanceNorm. } for this task, which leads to slightly better results. 



\autoref{tab:pointnet} summarizes the results out of three runs, showing mean error rates with standard errors.  \methodname{} reduces the classification errors from 6.0\% to 5.3\% and 9.2\% to 8.8\% on ModelNet10 and ModelNet40, respectively. 


\begin{table}[h]
    \centering
    \begin{tabular}{l|c|c}
    \toprule
    Model & ModelNet10 & ModelNet40 \\
    \midrule
    \midrule
    PointNet++ & 6.02$\pm$0.10 & 9.16$\pm$0.16\\
    + \methodname{} & \textbf{5.25$\pm$0.18 }& \textbf{8.78$\pm$0.28}\\
    \bottomrule
\end{tabular}
\caption{Classification errors (\%) on ModelNet10 and ModelNet40. The mean and standard error out of 3 runs are reported.}
\label{tab:pointnet}     \vspace{-0.2in}
\end{table}









 \input{sections/discussion.tex}
\input{sections/conclusion.tex}


\input{sections/acknowledgement.tex}


{\small
\bibliographystyle{ieee_fullname}
\bibliography{reference}
}

\clearpage
\appendix
\input{sections/appendix.tex}


\end{document}
