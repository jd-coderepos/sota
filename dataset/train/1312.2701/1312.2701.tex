
\myparagraph{Logical layer} We propose an embedding of our analysis
into Hennessy Milner Logic (HML), together with soundness and
completeness results. The analysis in~\cite{BDY12} be seen as the
superposition of two analyses: a session type system and a logical
layer. The former ensures that a process is able to perform some
visible actions described by the specification and can be
mechanically, yet tediously encoded in HML, for instance, by using a
``surely/then'' modality \cite{DBLP:conf/icalp/BergerHY08}.  Our
contribution focuses on the embedding of the latter, namely on
\emph{predicate safety}, ensuring that stateful predicates will be
satisfied.  As consequence, the completeness result we propose
(Proposition~\ref{prop:hml:cscomp}), states that if a process abides
to the session-type component  of a local assertion  (obtained by
erasing all predicates in ) and satisfies the logical encoding of
, then it is provable against .

\paragraph{MPSAs} We focus here on local assertions, 
each referring to a specific role and deriving, via projection, from a
global assertion as in~\cite{BDY12} -- e.g., as (\ref{eq:assertion}).
Local assertions are defined by the grammar below and are ranged over
by .

 {\small

}
 
Selection 
models an interaction where the role sends  a branch label 
and a message  of sort  (e.g., ,
, etc., and local assertion for delegation) and continues as , with being 
predicates\footnote{As in~\cite{BHTY10,BDY12} we assume that the
  validity of closed formulae is decidable.} and  state
updates. Branching  is dual to selection. 
We use guarded recursion defining a recursion parameter  initially set equal to a value satisfying the initialisation predicate , 
where  is the free variable of , and with  being an invariant predicate. 
Recursive call  instantiates a new iteration of  where the recursion parameter takes a value satisfying , with  free variable of . 

\cite{BDY12} uses local assertions as a basis for the verification of a processes, ranged over by . 
\newcommand{\ptp}[1]{\mathtt{#1}}

1mm]
&&  \PRG \PAR \PRGQ ~~
\mid
(\mu X(x).\PRG) \langle{e}\rangle
\mid
X\langle{e}\rangle
   \end{array}
\begin{array}{rll}
\ell & ::= & s[\p,\q](x) \mid \out{s[\p,\q]}{x} \mid E
\end{array}

\begin{array}{rll}
 \phi  ::=  \mathtt{true} \mid \phi \land \phi \mid \phi \impl \phi \mid
 \Hfa{\ell}{\phi} \mid A \mid \forall x:S.\phi  
\end{array}
\begin{array}{ccc}
\inferrule{P,\sigma \models \phi_1 \and P,\sigma \models \phi_2}{P,\sigma
  \models \phi_1 \land \phi_2}
\quad \inferrule{~}{P,\sigma \models \mathtt{true}}
\quad \inferrule{\text{ if }P,\sigma \models \phi_1\text{ then
  }P,\sigma\models \phi_2}{P,\sigma \models \phi_1 \impl \phi_2}\\
\inferrule{\text{For all } P',\sigma'\text{ s.t. } P,\sigma
  \TRANS{\ell} P',\sigma' ,  P',\sigma'\models \phi }{P \models
  \Hfa{\ell}{\phi}}
\quad \inferrule{\sigma \provlog{\emptyset}{A}}{P,\sigma \models A}
\quad \inferrule{\text{For all values } v \text{ of type }T, P,\sigma
  \models \phi\SUBS{v}{x}}{P,\sigma \models \forall x:T.\phi}
\end{array} \label{emb1}\embed{\lsumOut{\q}{l_i}{x_i:S_i}{A_i}{E_i}{\LA_i}
_{i \in I}}{s[\p]} =  \bigwedge_{i \in I} \forall x_i:S_i,
  \Hfa{\out{s[\p,\q]}{x_i}}{(A_i \land
    \Hfa{E_i}{\embed{\LA_i}{s[\p]}})} 
 \label{emb2}
\embed{\lsumIn{\q}{l_j}{x_j:S_j}{A_j}{E_j}{\LA_j}_{j \in
    J}}{s[\p]}  =  \bigwedge_{j \in J} \forall x_j:S_j,
\Hfa{s[\q,\p](x_j)}{(A_j \impl
  \embed{\LA_j}{s[\p]})} 
\begin{array}{ccc}
 \Hfa{\ell_1}{\phi_1} \shfl
\Hfa{\ell_2}{\phi_2}   =  \Hfa{\ell_1}{(\phi_1 \shfl
  \Hfa{\ell_2}{\phi_2})} \land \Hfa{\ell_2}{(\Hfa{\ell_1}{\phi_1}
  \land \phi_2)}\\
\Hfa{\ell_1}{\phi_1} \shfl (\phi_{2,1} \land \phi_{2,2})  = 
\Hfa{\ell_1}{(\phi_1 \shfl \phi_{2,1})} \land \Hfa{\ell_1}{(\phi_1
  \shfl \phi_{2,2})}\\
\phi \shfl \mathtt{true}  =  \phi\\
\phi \shfl (\phi_1 \land \phi_2)  =  (\phi \shfl \phi_1) \land
(\phi \shfl \phi_2)\\
(\phi_1 \land \phi_2) \shfl \phi  =  (\phi_1 \shfl \phi) \land
(\phi_2 \shfl \phi)\\
\forall x:T.\phi_1 \shfl \phi_2\\
(A \impl \phi_1) \shfl \phi_2  =  A \impl (\phi_1 \shfl
\phi_2)
\end{array}\Formu{s_1[\p_1],\dots,s_n[\p_n];a_1:\mathtt{I}(\mathcal{G}_1),
    \dots,a_m:\mathtt{I}(\mathcal{G}_m)}
= \embed{T_1}{s_1[\p_1]}
  \shfl \dots \shfl \embed{T_n}{s_n[\p_n]} \shfl 
\phi_1 \shfl \dots \shfl \phi_m

 \begin{array}{l}\pembed{x_1 \mapsto v_1, \dots, x_n \mapsto v_n}{}
  = \quad\quad \out{a_1}{v_1} \PA \dots \PA \out{a_n}{v_n} \PA
  \\ \quad !x_1(e).a_1(y_1)\dots
  a_n(y_n).(\out{a_1}{\mathtt{eval}(e\SUBS{y_1 \dots y_n}{x_1 \dots
      x_n})} \PA \out{a_2}{y_2} \PA \dots \PA \out{a_n}{y_n} ) \PA
  \dots \PA \\ \quad !x_n(e).a_1(y_1)\dots a_n(y_n).(\out{a_1}{y_1} \PA
  \dots \PA \out{a_{n-1}}{y_{n-1}} \PA
  \out{a_n}{\mathtt{eval}(e\SUBS{y_1 \dots y_n}{x_1 \dots x_n})})
\end{array}\begin{array}{lll}
\pembed{\Hfa{E}{\phi}} & = & \Hfa{\pembed{E}}{\pembed{\phi}}\\
\pembed{A} & = & \Hfa{\out{x_1}{v_1}}{\dots\Hfa{\out{x_n}{v_n}}
  {A\substi{v_1,\dots,v_n}{x_1,\dots,x_n}}}
  \end{array}\begin{array}{l} \mu A.([1]\mu B.([2]A \land [3]\mu
C.([4]B \land [2]([1]C \land [4].A) )) \land  \\
 ~[3]\mu D.([4].A \land
[1]\mu E.([2]D \land [4]([2]A \land [3]E))))
  \end{array}}







We use the following method to obtain a matching HML formula. We use a
translation through finite automata. Here is a sketch of the method,
which takes as arguments a set session environment :
\begin{enumerate}
\item Encode every session judgement  of  into
  a formula , using .
\item Translate every formula  into a finite automata
  , one state corresponds to a syntactic point between
  two modalities or a , one transition correponds to either
   (output) or  (input). Every automata is \emph{directed} with a
  source state corresponding to the head of the formula and leaf
  states corresponding to recursion variables (or end of protocols).
\item Compute the automata , the parallel composition of
  all the , which is still \emph{directed}.
\item Expand the automata , in order to obtain an
  equivalent branch automata, that is, an automata such that there is
  a root (the starting state) and transitions form a tree (back
  transitions are allowed but only on the same branch). This could be
  done by recursively replacing sub-automata with several copies of
  this sub-automata.
\item Translate back the automata into a formula, every state with
  more than two incoming transition is encoded as a recursion
  operator.
\end{enumerate}

On our example, we obtain the formulas  and ,
each one giving an automaton with 2 states (initial and between 
(resp. ) and  (resp. )).
Merging yields automata with  states: the initial one, one after
, one after , one after both  and . These automata
are diamond-shaped (hence not tree-shaped).
Expansion yields an automaton with  states, which is then translated
in the formula described above.
The preciseness proof relies on the fact that the operation described
in 3. and 4. give equivalent automata, and that two formulas
translated into two equivalent automata are equivalent for the HML
satisfaction relation.












