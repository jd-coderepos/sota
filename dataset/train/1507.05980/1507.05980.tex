
\documentclass[11pt,twoside]{article}
\usepackage{graphicx,hyperref,color}
\usepackage[utf8]{inputenc}
\usepackage[margin=1in]{geometry}
\usepackage{marvosym}
\usepackage{cite}
\usepackage{verbatim}
\usepackage{amsmath,amssymb,amsthm}
\usepackage{amsmath,amssymb}
\usepackage{chngcntr}
\usepackage[stable]{footmisc}
\usepackage{float}
\usepackage{url}
\usepackage{xspace}
\usepackage{paralist}
\usepackage{multienum}

\usepackage{thmtools}
\usepackage{thm-restate}
\usepackage{hyperref}
\usepackage{cleveref}

\usepackage{authblk}


\newcommand{\EMPH}[1]{\emph{#1}}







\newtheorem{theorem}{Theorem}[section]
\newtheorem{corollary}[theorem]{Corollary}
\newtheorem{lemma}[theorem]{Lemma}
\newtheorem{remark}[theorem]{Remark}
\newtheorem{definition}[theorem]{Definition}
\newtheorem{observation}[theorem]{Observation}


\newcommand{\OPT}{OPT}
\newcommand{\terminals}{}
\newcommand{\opts}{}
\newcommand{\spath}[1]{P_{#1}}
\newcommand{\opt}[1]{Q_{#1}}
\newcommand{\region}[1]{R_{#1}}
\newcommand{\border}[2]{B_{#1,#2}}
\newcommand{\len}[1]{\ell(#1)}

\newcommand{\eric}{Colin~de Verdi{\`e}re}


\newcommand{\anote}[1]{\footnote{\color{red} Amir:{#1}}}
\newcommand{\aNote}[1]{{\color{red} Amir:{#1}}}
\newcommand{\cnote}[1]{\footnote{\color{blue} Cora: {#1}}}
\newcommand{\cNote}[1]{{\color{blue} Cora: {#1}}}
\newcommand{\fnote}[1]{\footnote{\color{green} Farzad: {#1}}}
\newcommand{\fNote}[1]{{\color{green} Farzad: {#1}}}



\title{Towards single face shortest vertex-disjoint paths in undirected planar graphs}





\author{Glencora Borradaile}
\author{Amir Nayyeri}
\author{Farzad Zafarani}
\affil{School of Electrical Engineering and Computer Science \\ Oregon State University\\
	\{glencora, nayyeria, zafaranf\}@eecs.oregonstate.edu}




\begin{document}
\maketitle

\begin{abstract}
Given  pairs of terminals \terminals\ in a graph , the min-sum  vertex-disjoint paths problem is to find a collection  of vertex-disjoint paths with minimum total length, where  is an -to- path between  and .  We consider the problem in planar graphs, where little is known about computational tractability, even in restricted cases. Kobayashi and Sommer propose a polynomial-time algorithm for  in undirected planar graphs assuming all terminals are adjacent to at most two faces.
\eric\ and Schrijver give a polynomial-time algorithm when all the sources are on the boundary of one face and all the sinks are on the boundary of another face and ask about the existence of a polynomial-time algorithm provided all terminals are on a common face.  

We make progress toward \eric\ and Schrijver's open question by giving an  time algorithm for undirected planar graphs when \terminals\ are in counter-clockwise order on a common face.
\end{abstract}

\newpage
\section{Introduction}

Given  pairs of terminals \terminals, the \EMPH{ vertex-disjoint} paths problem asks for a set of  disjoint paths \opts, in which  is a path between  and  for all .
As a special case of the multi-commodity flow problem, computing vertex disjoint paths has found several applications, for example in VLSI design\cite{kramer1984complexity}, or network routing \cite{ogier1993distributed,srinivas2005finding}.  
It is one of Karp's NP-hard problems \cite{karp1974} even for undirected planar graphs if  is part of the input \cite{middendorf1993disjPathComplexity}.  
However, there are polynomial time algorithms if  is a constant for general undirected graphs~\cite{robertson1995graph, Kawarabayashi2010shorterProofGraphMinorAlg}.
In general directed graphs, the -vertex-disjoint paths problem is NP-hard even for ~\cite{fortune1980directedSubGraphHom} but is fixed parameter tractable with respect to parameter  in directed planar graphs~\cite{schrijver1994finding, cygan2013planarDirDisjFPT}.

Surprisingly, much less is known for the optimization variant of the problem, \EMPH{minimum-sum  vertex-disjoint paths problem (-min-sum)}, where a set of disjoint paths with \emph{minimum total length} is desired. 
For example, the -min-sum problem and the -min-sum problem are open in directed and undirected planar graphs, respectively, even when the terminals are on a common face; neither polynomial-time algorithms nor hardness results are known for these problems~\cite{kobayashi2010shortest}. Bjorklund and Husfeldt gave a randomized polynomial time algorithm for the min-sum two vertex-disjoint paths problem in general undirected graphs \cite{bjorklund2014shortest}. Kobayashi and Sommer provide a comprehensive list of similar open problems (Table 2~\cite{kobayashi2010shortest}).



One of a few results in this context is due to \eric~and Schrijver~\cite{verdiere2011shortest}: a polynomial time algorithm for the -min-sum problem in a (directed or undirected) planar graph, given all sources are on one face and all sinks are on another face~\cite{verdiere2011shortest}.  In the same paper, they ask about the existence of a polynomial time algorithm provided all the terminals (sources and sinks) are on a common face. If the sources and sinks are ordered so that they are in the order  around the boundary, then the -min-sum problem can be solved by finding a min-cost flow from  to .  For  in undirected planar graphs with the terminals in arbitrary order around the common face, Kobayashi and Sommer give an  algorithm~\cite{kobayashi2010shortest}\footnote{Kobayashi and Sommer also describe algorithms for the case where terminals are on two different faces, and .}.
In this paper, we give the first polynomial-time algorithm for an arbitrary number of terminals on the boundary of a common face, which we call , so long as the terminals alternate along the boundary.  Formally, we prove:
\begin{theorem}
There exists an  time algorithm to solve the -min-sum problem, provided that the terminals  are in counter-clockwise order on the boundary of the graph.
\end{theorem}

\paragraph{Definitions and assumptions} We use standard notation for graphs and planar graphs.  For simplicity, we assume that the terminal vertices are distinct.  One could imagine allowing ; our algorithm can be easily modified to handle this case.  We also assume that the shortest path between any two vertices of the input graph is unique as it significantly simplifies the presentation of our result; this assumption can be enforced using a perturbation technique~\cite{MVV87}.


\section{Preliminaries}

\paragraph{Walks and paths.} Let  be a graph, and let  be a subgraph of .  We use  and  to denote the vertex set and the edge set of , respectively.  For , we use \EMPH{} to denote the subgraph of  induced by , whose vertex set is  and whose edge set is all edges of  with both endpoints in .  A \EMPH{walk}  in  is a sequence of vertices such that for all , we have .  For any , \EMPH{} is the (sub-)walk .
A \EMPH{path} is a walk with no repeated vertices.  Sometimes, we view a walk as its set of edges, and use set operations on walks.  For example, given two walks  and , we define  to be their symmetric difference when it is clear from the context that  is a walk.  Given a length function , the length of  is denoted by \EMPH{}, and it is .

\paragraph{Planarity.}
An \EMPH{embedding} of a graph  into the Euclidean plane is a mapping of vertices of  into different points of , and edges of  into internally disjoint simple curves in  such that the endpoints of the image of  are the images of vertices  and .  If such an embedding exists then  is a \EMPH{planar graph}.  A \EMPH{plane graph} is a planar graph and an embedding of it.  The \EMPH{faces} of a plane graph  are the maximal connected components of  that are disjoint from the image of .  If  is connected it contains only one unbounded face.  This is called the \EMPH{outer face} of  and we denote the boundary of  by .
Let  be a walk and let  be the set of edges that are used an odd number of times in .  
 is a collection of simple cycles.  For any point , we say that  is inside  if  is on the image of  in the plane, or  is inside an odd number of cycles of .
When it is clear from the context, we use the same notation to refer to the subgraph composed of the vertices and edges whose images are completely inside . 


\section{Structural properties}
In this section, we present fundamental properties of the optimum solution that we exploit in our algorithm.
To simplify the exposition, we search for pairwise disjoint walks rather than simple paths and refer to a set of pairwise disjoint walks connecting corresponding pair of terminals as a \emph{feasible} solution.
Indeed, in an optimal solution, the walks are simple paths.

Let  be an optimal solution, where  is a -to- path and let  be the set of shortest paths, where  is the -to- shortest path.
These shortest paths together with the boundary of the graph, , define internally disjoint regions of the plane.  Specifically, we define  to be the subset of the plane bounded by the cycle , where  is the -to- subpath of  that does not contain other terminal vertices.
The following lemmas constrain the behavior of the optimal paths.

\begin{lemma}
\label{lem:sp_in_region}
For all , the path  is inside .
\end{lemma}
\begin{proof}
Suppose, to derive a contradiction, that .
So, there is a vertex .  Let  be the maximal path containing  that is internally disjoint from .  Let  be endpoints of , and observe that .  Also, by uniqueness of shortest paths, we have .  Thus,  is shorter than .  

It remains to show that  for all , .  But, by the construction,  is inside , and all terminals other than  and  are outside .  So, by Jordan curve theorem, for any  to intersect , it has to intersect , too.  Thus,  is a shorter solution than the optimum, which is a contradiction.
\end{proof}

\noindent We take the vertices of  and  to be ordered along these paths from  to .

\begin{lemma}
\label{lem:pqSameOrder}
For ,  precedes  in  if and only if  precedes  in .
\end{lemma}
\begin{proof}
Suppose, to derive a contradiction, that  and  have different orders on  and , and assume, without loss of generality, that  precedes  in , but  precedes  in .  So,  can be decomposed into three subpaths (1)  is a -to- path, (2)  is a -to- path, and (3)  is a -to- path.  If  contains  then  is not a simple path, visiting  at least twice. Otherwise, The Jordan curve theorem implies that  or  must intersect .  Again  is not simple, so, it is not a walk in the optimum solution.
\end{proof}

\begin{figure}[tbh]
  \centering
    \includegraphics[height=1.2in]{RegionBorder}\qquad\qquad
    \includegraphics[height=1.2in]{TypePaths}
      \caption{(left) A -min sum instance; regions are shaded and borders are green. (right) A feasible solution; Type~I and Type~II subpaths are blue and red, respectively.}
  \label{fig:4-Disj}
\end{figure}

We call  the \emph{border} of  and  and denote it .   Note that a border can be a single vertex. Since we assume shortest paths are unique,  is a single (shortest) path.
Figure~\ref{fig:4-Disj} illustrates borders for a -min-sum instance.  The following lemma bounds the total number of borders.

\begin{lemma}
\label{lem:bounded_number_borders}
There are  border paths. 
\end{lemma}
\begin{proof}
Let  be the graph whose vertices correspond to regions in , and there is an edge between two vertices of  if their corresponding regions in  share a border.  Let , and observe that  is a subgraph of the planar dual of .  Thus,  is planar.

Since there is a bijection between  and the set of regions of , we have .
Additionally, there is a bijection between  and borders in .  Because,  is planar we conclude that the number of borders in  is .
\end{proof}

Consider a region  and consider the borders along , .  Observe that the intersections of the regions  with  must be in a {\em clockwise} order.  Let  be the subsequence of  of indices to regions that intersect .  For , let  and  be the first and last vertex of  in .  Additionally, define  and . We partition  into a collection of subpaths of two types as follows.
\begin{enumerate}[Type~I :]
\item For ,  is a {\em Type~I} subpath in region .
\item For ,  is a {\em Type~II} subpath in region .  We say that  is on the border  containing  and .
\end{enumerate}
By this definition, all Type~I paths are internally disjoint from all borders. By Lemma~\ref{lem:pqSameOrder}, each Type~II path is internally disjoint from all borders except possibly the border that contains its endpoints, with which it may have several intersecting points.  See Figure~\ref{fig:4-Disj} for an illustration of Type~I~and~II paths.

The following lemma demonstrates a key property of Type~I paths, implying that (given their endpoints) they can be computed efficiently via a shortest path computation:
\begin{lemma} 
\label{lem:alpha_opt}
Let  be a Type~I subpath in region .  
Then  is the shortest path between its endpoints in  that is internally disjoint from all borders.
\end{lemma}
\begin{proof}
Let  be the endpoints of , and let  be the shortest -to- path in  that is internally disjoint from all borders.
Also, let .  
The path  has the same endpoints as , and it is internally disjoint from all  if .
Also, by Lemma~\ref{lem:sp_in_region}, for each ,  is inside .
Therefore,  is disjoint from  for all  and .
Thus,  is a set of  pairwise disjoint walks, with total length  where  is the total length of the optimal paths.
So, , which implies .
Therefore  must be a shortest  path in  that avoids boundaries.  In fact, uniqueness of shortest paths implies .
\end{proof}


A Type~II path has a similar property if it is the only Type~II path on the border that contains its endpoints. The proof of the following lemma is almost exactly the same as the previous proof.
\begin{lemma} 
\label{lem:beta_opt0}
Let  be a Type~II subpath in region  on border .  Suppose there is no Type~II path on  inside .
Then  is the subpath of  between its endpoints.  \end{lemma}

The following lemma reveals a relatively more sophisticated structural property of Type~II paths on shared borders.

\begin{lemma}
\label{lem:beta_opt}
Let  and  be  Type~II subpaths in  and  on , respectively, let  and  be the endpoints of , and let  and  be the endpoints of .
Then,  is the pair of paths with \emph{minimum total length} with the following properties:
\begin{enumerate}[(1)]
\item  is an -to- path inside , and it is internally disjoint from all borders except possibly .
\item  is an -to- path inside , and it is internally disjoint from all borders except possibly .  \end{enumerate}
\end{lemma}
\begin{proof}
Properties (1) and (2) of  and  are implied by the definition of Type~II paths and because they are internally disjoint from all borders except .  It remains to show that their total length is minimum.

Let  be the pair of paths with minimum total length that has properties (1) and (2). 
Let , and let .

By construction,  is internally disjoint from all borders except (possibly) .  Additionally, the endpoints of  are the same as the endpoints of .
Therefore, intersection points of  with borders of  that are not in  are all in .
Similarly, intersection points of  with borders of  that are not in  are all on .
It immediately follows that  and  are disjoint for any .
Similarly,  and  are disjoint for any .

Furthermore,  and  are disjoint by their construction.  Thus,  is a set of  pairwise disjoint walks connecting the terminals.  The total length of this new set is .  Consequently, , in fact, .  Thus,  are a pair of path with minimum total length that has properties (1) and (2).
\end{proof}



\section{Algorithmic toolbox} 

In this section, we describe algorithms to compute paths of Type~I~and~II for given endpoints.
These algorithms are key ingredients of our strongly polynomial time algorithm described in the next section.
More directly, they imply an  time algorithm via enumerating the endpoints, which is sketched at the end of this section.

Each Type I path can be computed in linear time using the algorithms of Henzinger et al.~\cite{henzinger1997planarShortestPaths}; they can also be computed in bulk in  time using the multiple-source shortest path algorithm of Klein~\cite{klein2005multiSourceShPaths} (although other parts of our algorithms dominate the shortest path computation).
Similarly, a Type~II path on a border  can be computed in linear time provided it is the only path on .
Computing pairs of Type~II paths on a shared border is slightly more challenging.  
To achieve this, our algorithm reduces this problem into a -min sum problem that can be solved in linear time via a reduction to the minimum-cost flow problem.  
The following lemma is implicit in the paper of Kobayashi and Sommer~\cite{kobayashi2010shortest}. 

\begin{lemma}
\label{lem:2minSum}
There exists a linear time algorithm to solve the -min sum problem on an undirected planar graph, provided the terminals are on the outer face.
\end{lemma}
\begin{proof}
Consider an instance of the -min sum problem with terminals . This problem reduces to a minimum-cost flow problem as follows.
Because of the symmetry for terminals in undirected graphs, we can assume that  and  are next to each other on the outer face:  there is a -to- path on the boundary of the outer face that does not intersect . We make the graph directed by replacing each undirected edge  with edges  and . For edge  in the directed graph, we assign its length using length function . For every vertex  in , we split it into two vertices  and  and connect them with a zero length edge that has unit capacity. For each edge , we connect  to , and for each edge , we connect  to . We introduce a dummy source vertex , with two edges , and  of unit capacity and zero length.  Also, we introduce a dummy sink vertex , with edges  and  of unit capacity and zero length.  We assign capacity one to all other edges.  The lengths of the other edges are specified by the length function  of . Since,  and  (also  and ) are next to each other, the graph remains planar after adding , , and their incident edges.
Now, it is straight forward to see that our -min sum problem is equivalent to a minimum cost -to- flow problem of value .
This minimum cost flow problem in turn reduces to two shortest path computations from  to , which can be done in linear time~\cite{henzinger1997planarShortestPaths}.
\end{proof}

We reduce the computation of Type II paths to -min sum.  The following lemma is a slightly stronger form of this reduction, which finds application in our strongly polynomial time algorithm. 

\begin{lemma}
 \label{lem:2minSumplus}
Let  and  be two regions with border  and let  and .  A pair of paths  with total minimum length and with the following properties can be computed in linear time.
\begin{enumerate}
\item  is an -to- path inside , and it is internally disjoint from all borders except possibly .
\item  is an -to- path inside , and it is internally disjoint from all borders except possibly .  
\end{enumerate}
\end{lemma}
\begin{proof}
Let the graph  be the induced subgraph by the vertices of  inside .
We obtain  from  by performing the following operations:
\begin{enumerate}
\item deleting all vertices of  that belong to borders other than ,
\item deleting all edges in  that are incident to  but not incident to , and
\item deleting all edges in  that are incident to  but not incident to .
\end{enumerate}
Note that  is not necessarily connected.  For an illustration of  and , see Figure~\ref{fig:mincost}.

\begin{figure}[tbh]
  \centering
    \includegraphics[height=1.5in]{betaComputation}
      \caption{(left) The graph  induced by vertices in ; regions are shaded and borders are green. (right) The graph  defined in the proof of Lemma~\ref{lem:2minSumplus}}
  \label{fig:mincost}
\end{figure}


Since  and  are intact in , they can be computed in  instead of .
Furthermore, observe that  are on the boundary of .  If  is disconnected, then  and  are a shortest paths in their own connected components.  So they can be computed in linear time using the algorithm of Henzinger et al.~\cite{henzinger1997planarShortestPaths}.

So, suppose  is connected, and observe that  are on the boundary of the outer face of .  By Lemma~\ref{lem:2minSum}, there is a linear time algorithm to compute a pair of disjoint paths of minimum total length between corresponding terminals.

Let  and  be -to- path and -to- path computed via solving a -min sum problem. 
It remains to prove that  and  have properties (1) and (2) of the lemma.
That is , and .  This can be done through a replacing path argument similar to Lemma~\ref{lem:sp_in_region}, as  (so, any subpath of it) is a shortest path.
\end{proof}

\subsection{An  time algorithm}

The properties of Type~I~and~II paths imply a na\"ive  time algorithm, which we sketch here.
An optimal solution defines the endpoints of Type~I and Type~II paths, so we can simply enumerate over which borders contain endpoints of Type~I and~II paths and then enumerate over the choices of the endpoints.  
Consequently, there are zero, two, or four (not necessarily distinct) endpoints of Type~I and~II paths on , or

possibilities, which is  since .  Since there are  borders (Lemma~\ref{lem:bounded_number_borders}), there are  endpoints to guess.  Given the set of endpoints, we compute a feasible solution composed of the described Type~I and~II paths or determines that no such solution exists.  Since Type~I and~II paths can be computed in polynomial time, the overall 
algorithm runs in  time.




\section{A fully polynomial time algorithm}

We give an -time algorithm via dynamic programming over the regions.  For two regions  and  that have a shared border ,  and  separate the terminal pairs into two sets: those terminals  for  and  for  (for ).  Any -to- path that is in region  cannot touch any -to- path that is in region  since  and  are vertex disjoint.  Therefore any influence the -to- path has on the -to- path occurs indirectly through the -to- and -to- paths.  Our dynamic program is indexed by the shared borders  and pairs of vertices on (a subpath of)  and (a subpath of) ; the vertices on  and  will indicate a {\em last point} on the boundary of  and  that a (partial) feasible solution uses. 

We use a tree to order the shared borders for processing by the dynamic program. Since there are  borders (Lemma~\ref{lem:bounded_number_borders}), the dynamic programming table has  entries.  We formally define the dynamic programming table below and show how to compute each entry in  time.

\subsection{Dynamic programming tree}
Let  and  be the set of all borders between all pairs of regions.
We assume, without loss of generality, that  is connected, otherwise we  split the problem into independent subproblems, one for each connected component of .

We define a graph  (that we will argue is a tree) whose edges are the shared borders  between the regions .
Two distinct borders  and  are incident in this graph if there is an endpoint  of  and  of   that are connected by an -to- curve in  that does not touch any region  except at its endpoints  and ;  see Figure~\ref{fig:T}.  Note that this curve may be trivial (i.e.\ ).  The vertices of  (in a non-degenerate instance) correspond one-to-one with components of  (plus some additional trivial components if three or more regions intersect at a point), or {\em non-regions}. The edges of  cannot form a cycle, since by the Jordan Curve Theorem this would define a disk that is disjoint from ;  an edge  in the cycle bounds two regions, one of which would be contained by the disk, contradicting that each region shares a boundary with .  Therefore  is indeed a tree.   We use an embedding of  that is derived naturally from the embedding of  according to this construction.  We use this tree to guide the dynamic program.


\begin{figure}[ht]
  \centering
        \includegraphics[height=1.2in]{dpTree}
      \caption{(left) Thick green segments are borders, thin green curves are in  connecting borders that are incident in . (right) The directed tree  used for dynamic programming.}
\label{fig:T}
\end{figure}

 
By the correspondence of the vertices of  to non-regions, we have:
\begin{observation}\label{obs:region-path-T}
  The borders  along a given region  form a path in .  The order of the borders from  to  along  is the same as in the path in .
\end{observation}
Consider two edges  and  that are incident to the same vertex  of  and that are consecutive in a cyclic order of the edges incident to  in 's embedding.  By Observation~\ref{obs:region-path-T} and the embedding of , there is a labeling of  such that:
\begin{observation}\label{obs:first-last-border}
  Either  or  is the last border of  along  and  is the first border of  along .
\end{observation}





Root  at an arbitrary leaf.  Since  is connected, the leaf of  is non-trivial; that is, it has an edge  incident to it.  By Observation~\ref{obs:region-path-T},  is (w.l.o.g.) the last border of  along  and the first border of  along .  By the correspondence of the vertices of  to non-regions, and the connectivity of , either  or  and .
For ease of notation, we assume that the terminals are numbered so that  and .  We get:
\begin{observation}\label{obs:leaf-edge}
  Every non-root leaf edge of  corresponds to a border .
\end{observation}

We consider the borders to be both paths in  and edges in . In  we orient the borders toward the root.  In , this gives a well defined start  and endpoint  of the corresponding path  (note that  is possible).  By our choice of terminal numbering and orientation of the edges of , from  to  along ,  is visited before , and from  to  along ,  is visited before .

\subsection{Dynamic programming table}
We populate a dynamic programming table  for each border .   is indexed by two vertices  and :  is a vertex of   and  is a vertex of .   is defined to be the minimum length of a set of vertex-disjoint paths that connect:
  \begin{center}
     to ,  to , and  to  for every 
  \end{center}
These paths are illustrated in Figure~\ref{fig:dp-table}.
We interpret  as the last vertex of  that is used in this sub-solution and we interpret  as the first vertex of  that can be used in this sub-solution (or, more intuitively, the last vertex of the reverse of ).  By Lemma~\ref{lem:sp_in_region}, each of the paths defining  are contained by their respective region.
\begin{figure}[ht]
  \centering
    \includegraphics[height=1.4in]{dpSub}
  \caption{An illustration of the paths defined by .}
  \label{fig:dp-table}
\end{figure}

\paragraph{Optimal solution}  Given , we can compute the value of the optimal solution.  By Lemma~\ref{lem:alpha_opt},  and  contain a shortest (possibly trivial) path from  to a vertex  on , and from a vertex  on  to , respectively.  Let  be the last vertex of  that  contains and let  be the first vertex of  that  contains.  Then, by Lemma~\ref{lem:alpha_opt}, the optimal solution has length  where  is a Type~I path between the given vertices.  The optimal solution can be computed in  time by enumerating over all choices of  and .  Computing all such Type~I paths takes  since there are  such paths to compute, and each path can be found using the linear-time shortest paths algorithm for planar graphs~\cite{henzinger1997planarShortestPaths}.

\paragraph{Base case: Leaf edges of }  Consider a non-root leaf edge of , which, by Observation~\ref{obs:leaf-edge}, is  for some .  Then  is the length of minimum vertex disjoint -to- and -to- paths in .
By Lemma~\ref{lem:2minSumplus},  can be computed in  time  for any  and  and so  can be populated in  time. 






\subsection{Non-base case of the dynamic program}

Consider a border  and consider the edges of  that are children of .  These edges considered counter-clockwise around their common node of  correspond to borders  where .  For simplicity of notation, we additionally let  and .
Then, by Observation~\ref{obs:first-last-border}, either  or  is the last border on  and  is the first border on  for .


\paragraph{An acyclic graph  to piece together sub-solutions}
To populate  we create a directed acyclic graph  with sources corresponding to vertices of  and sinks corresponding to .  A source-to-sink (-to-) path in  will correspond one-to-one with vertex disjoint paths from:
\begin{center}
   to ,  to , and  to  for every 
\end{center}
Here  and  do not correspond to the vertices  and  that index ; to these vertex disjoint paths, we will need to append vertex disjoint -to- and -to- paths (which can be found using a minimum cost flow computation by Lemma~\ref{lem:2minSumplus}).

The arcs of  are of two types: (a) Type~I arcs and (b) sub-problem arcs.  Directed paths in  alternate between these two types of arcs.  The Type~I arcs correspond to Type~I paths and the endpoints of the Type~I arcs correspond to the endpoints of the Type~I paths.
Sub-problem arcs correspond to the sub-solutions from the dynamic programming table and the endpoints of the sub-problem arcs correspond to the indices of the dynamic programming table (and so are the endpoints of the {\em incomplete} paths represented by the table).  Note that vertices of a border may appear as either the first or second index to the dynamic programming table; in , two copies of the border vertices are included so the endpoints of the resulting sub-solution arcs are distinct.  Formally:
\begin{itemize}
\item Type~I arcs go from vertices of  to vertices of  for .  Consider regions  and .  There are two cases depending on whether or not .
  \begin{itemize}
  \item If , then for every vertex  of  and every vertex  of , we define a Type~I arc from  to  with length equal to the length of the -to- Type~I path.
  \item If , then for every vertex  of  and every vertex  of , we define a Type~I arc from  to  with length equal to the sum of the lengths of the -to- and -to- Type~I paths.
  \end{itemize}
\item Sub-problem arcs go from vertices of  to vertices of  for . For every  and every vertex  of  and vertex  of  (that are not duplicates of each other), we define a sub-problem arc from  to  with length equal to .
\end{itemize}


\paragraph{Shortest paths in }
By construction of  and the definition of , for a source  and sink , we have:
\begin{observation}
  There is a -to- path in  with length  if and only if there are vertex disjoint paths of total length  from  to ,  to , and  to  for every .
\end{observation}
See Figure~\ref{fig:one-to-one} for an illustration of the paths in  that correspond to a source-to-sink path in . Let  denote the shortest -to- path in  (for a source  and a sink ).  We will need to compute  for every pair of sources and sinks.  Since every vertex in  appears at most twice in , the size of  is  and for a given sink and for all sources, the shortest source-to-sink paths can be found in time linear in the size of  using dynamic programming.  Repeating for all sinks results in an  running time to compute  for every pair of sources and sinks.\footnote{Computing the length of the Type~I paths is dominated by , but can be improved to  time by running Klein's boundary shortest path algorithm~\cite{klein2005multiSourceShPaths} in all regions, resulting in an  time to construct .}

\begin{figure}[ht]
  \centering
  \includegraphics[height=1.7in]{one2one}
  \caption{Mutually disjoint walks represented by a directed path in  for a set of incident borders (green).  The blue arcs correspond to the walks represented in sub-problems and the solid red paths correspond to the Type~I paths represented by Type~I arcs.  The dotted red paths represent vertex disjoint -to- and -to- paths that will be added via a min-cost flow computation.}
\label{fig:one-to-one}
\end{figure}


\paragraph{Handling vertices that appear in more than two regions}  As indicated, a vertex  may appear in more than two regions; this occurs when two or more borders share an endpoint.  In the construction above, if  appears in only two regions, then,  can only be used as the endpoint of two sub-paths (whose endpoints meet to form a part of an -to- path in the global solution).  However, suppose for example that  appears as an endpoint of both  and  and so 4 copies of  are included in  (two copies for each of these borders).  On the other hand, one need only {\em guess} which -to- path  should belong to first and construct  accordingly.  There are only  possibilities to try.

Unfortunately, there may be  shared vertices among the borders \\  involved in populating .
  It seems that for each of these  shared vertices, one would need to guess which -to- path it belongs to, resulting in an exponential dependence on . 

Here we recall the structure of : the nodes of  correspond to {\em non-regions}: disks (or points) surrounded by regions.  If there are several shared vertices among the borders, then there is an order of these vertices around the boundary of the non-region.  That is, for a vertex shared by a set of borders, these borders must be contiguous subsets of .  In terms of the construction of , there is a contiguous set of levels that a given shared vertex appears in and distinct shared vertices participate in non-overlapping sets of levels.  For one set of these levels, we can create different copies of the corresponding section of .  In each copy we modify the directed graph to reflect which -to- path the corresponding shared vertex may belong to (see Figure~\ref{fig:copies}).  As we have argued, since distinct shared vertices participate in non-overlapping sets of levels, this may safely be repeated for every shared vertex.  The resulting graph has size  since there are  borders and shared vertices are shared by borders.  The resulting running time for computing all source-to-sink shortest paths in the resulting graph is then .

\begin{figure}[ht]
  \centering
  \includegraphics[height=4in]{dag}
\caption{(left)  constructed without handling the fact that vertices  and  (black) may appear more than twice.  The arcs are all directed upwards (arrows are not shown); green arcs are sub-problem arcs and red arcs are Type~I arcs. (right) The levels that  appears in are duplicated, and only one pair of copies of  is kept in each copy of these levels. The vertex  may only be visited twice now on a source to sink path.}
\label{fig:copies}
\end{figure}

\paragraph{Computing  from }
To compute , we consider all possible  on  and  on  and compute the minimum-length vertex disjoint -to- path and -to- path that only use vertices that are interior to  (that is vertices of  may be used); by Lemma~\ref{lem:2minSumplus}, these paths can be computed in linear time.  Let  be the cost of these paths.  Then

As there are  choices for  and  and  can be computed in linear time,  can be computed in  time given that distances in  have been computed.

\paragraph{Overall running time}  For each border ,  is constructed and shortest source-to-sink paths are computed in  time.  For each ,  is computed in  time.  Since there are  pairs of vertices in ,  is computed in  time (dominating the time to construct and compute shortest paths in ).  Since there are  borders (Lemma~\ref{lem:bounded_number_borders}),  the overall time for the dynamic program is .




\paragraph{Acknowledgements}  This material is based upon work supported by the National Science Foundation under Grant No.\ CCF-1252833.


\bibliographystyle{alpha}
\bibliography{dispaths}

\end{document}
