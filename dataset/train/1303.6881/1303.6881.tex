
DOAT uses coordinates to obtain a measure of proximity. In the rest of the paper,
we assume network delay coordinates, however, coordinates
with more rich semantics like \emph{load-aware network coordinates} \cite{LANC}
are also possible. 
In order to minimise query time, DOAT operates on the principle of creating
paths with logarithmically decreasing distances to the destination, a
technique similar to \cite{stoicachord}. This principle underlies the
protocols for establishing connections between the overlay nodes, 
for exchanging routing information and for
forwarding queries for a particular group. 


\subsection{Overlay Topology Construction}
\label{subsect:neighbors}
Before connecting to the overlay, a node determines its position
using a distributed coordinate system such as \cite{vivaldi}.  This
allows the node to calculate its coordinates
in a multi-dimensional delay space $\mathcal{X}$ where metric distance
between peers in the space is directly correlated with
network delay between the peers in the network. Thus, short metric
distances imply low delay.

Coordinates in
$\mathcal{X}$ are mapped to a single-dimensional \emph{DOAT coordinate}
which becomes the identifier of the node, used to determine its neighbours in
the overlay.
The DOAT
coordinate has the property that, if two nodes are ``close" in it, 
then they are close in $\mathcal{X}$. 
Note that the opposite is not true: closeness in
$\mathcal{X}$ does not guarantee closeness in the single dimensional DOAT
space. This has the drawback that the closest node in the single-dimensional 
space might not be the closest node in the multi-dimensional
space, but coarser locality information will be preserved.

The mapping from  $\mathcal{X}$ to a 
single-dimensional space is done by
first using a linear transform to map $\mathcal{X}$ coordinates into the
unit square, and
then mapping the unit square to a single-dimensional coordinate using a number of
iterations
of a space filling curve.  The curve used here is the
\emph{H-curve}~\cite{hcurve}, which is known
to have good locality preserving properties.
The obtained single-dimensional coordinate is a wrapping coordinate in the
range of $[0,1)$, that positions the nodes in a ring 
(see figure \ref{fig:doat}).
For the purpose of this paper we will treat $\mathcal{X}$ as a
two-dimensional space, note however that the \emph{H-curve} trivially generalises 
to the multi-dimensional space.


\begin{figure}[h]
\begin{center}
\includegraphics[width=7cm]{doat-ring}
\caption{DOAT Topology Construction using Space Filling Curve}
\label{fig:doat}
\end{center}
\end{figure}


After obtaining its DOAT coordinate, a node establishes peering connections
with other DOAT nodes to exchange routing information. 
Following the principle of logarithmically decreasing
distances, a node first connects to the furthest neighbour (closest to the
opposite point of the ring at $0.5$ distance), then to the two at half
distance ($0.25$) in either direction of the ring, and so on.
This process terminates 
when the immediately closest node is found at each direction. 

To connect to a neighbour $n_2$ at a given distance and direction on the ring, 
a DOAT node $n_1$ calculates a target coordinate for $n_2$. For example,
from $n_1$ at $0.43$, the neighbours at a distance of $0.25$ have target
coordinates of $0.68$ (clockwise) and $0.18$ (anti-clockwise).  
Node $n_1$ then sends a message with the target coordinate $n_2$ to any known
DOAT node $n_3$
, which then forwards the message to its
neighbour that is closest to the target coordinate $n_2$.
This is done recursively, until the message reaches the actual DOAT node
$n_2'$ that is closest to the target coordinate $n_2$. This is the node that
has no neighbour closer to $n_2$ than the node itself.



\subsection{Registering Membership and Updating Routing Tables}
\label{subsect:registering}

When a host becomes a member of a group, it has already
calculated reliably its position in $\mathcal{X}$. It then
sends a registration message to any known DOAT node, which converts the
position of the new group member to the corresponding DOAT coordinate. 
The registration is then forwarded to the DOAT neighbour node 
which is closest to this DOAT coordinate,
until it reaches the DOAT node which is closest to the DOAT coordinate of
the new group member. This node installs the new membership entry at its
local registry.

Each DOAT node maintains a routing table to forward queries for groups
with no members in its local registry. The routing table contains one entry for
the local registry, and one entry for each of its
neighbours. Each entry includes
information on the identity of the next hop (neighbour DOAT node), the
distance to the next hop along the DOAT ring, and the set of groups
reachable through it. Routing entries are sorted in ascending
order of distance to the next hop, with first the entry corresponding
to the local registry. 
Figure \ref{fig:doat} shows the routing table of
the node at DOAT coordinate $0.43$.

The group identifiers cannot be inherently aggregated as it is the case
with, for example, IP addresses. To preserve the scalability of routing
tables and routing update messages, group identifiers are aggregated using
Bloom filters. 

A route announcement from a node $n_1$ to one of its neighbours $n_2$ contains
the groups present in the local registry of $n_1$, and the groups that $n_1$
can reach through other DOAT nodes $n_i$, where the distance between 
$n_1$ and $n_i$ is less than the distance between $n_1$ and $n_2$. In other words, the node
announces all the anycast groups for which there are member hosts in its
local area, to nodes further away.  The area Bloom filter sent to
each neighbour $n_i$ is the aggregate of the Bloom filter that corresponds to the local registry, and the Bloom
filters received from all the neighbours which are closer than $n_i$,
i.e. appearing before $n_i$ in the routing table. 
When a new group is registered with the node $n_1$, the latter sends an update to all its neighbours.
When the node $n_1$ forwards a routing update received from another neighbour $n_2$
then it propagates the update only to these nodes further away than $n_2$.

To reduce the message overhead, a DOAT node may not send routing updates
synchronously with local changes. Instead, a minimum interval may be enforced
between two consecutive updates sent to a neighbour. There is an
accuracy penalty incurred by delaying updating the neighbours
(see Section \ref{subsect:evalperiodic}).
Additionally, instead of sending the entire
Bloom filter every time, a node may send only the difference since the last
update and other compression techniques are possible.


A routing update message also contains the
immediately closest neighbour of the node in both directions. 
These DOAT nodes can be used as alternative neighbours in case the
node fails and does not follow the procedure of gracefully removing itself
from the overlay.

\subsection{Query Forwarding}
\label{subsect:querying}

When a DOAT node receives a query, it will either return the appropriate
group member IP address if the group appears in its local registry, or
forward the query to another DOAT node, according to the Bloom filter
matches on its routing table.

To determine the next hop to forward the query, the DOAT node
will iterate its routing table in increasing distance order, attempting to
match the queried group identifier with the Bloom filters at each step.
The next hop will be the first neighbour whose Bloom filter returns a positive
match with the group identifier. The query is then forwarded to the
associated next hop neighbour. Because the routing table is sorted in ascending order
of distance to the next hop, this is the nearest next hop that has
announced a route for this group. The process is repeated at
each hop until a node with group members in its 
local registry is found.

Because of the neighbour selection and route forwarding mechanisms, the
path the query follows is composed by hops of exponentially decreasing
distances, starting with a maximum of $0.5$ on the DOAT coordinate, i.e.
the maximum network delay between any two DOAT nodes.




\subsection{Node Insertion and Deletion}

The DOAT nodes may be statically appointed hosts, operated by a single
administration. However, the system is designed to work in a peer-to-peer
environment, where any \emph{stable} peer (high uptime, stable network
delay coordinates, considerable storage, bandwidth and processing
resources) can be elected to become a DOAT node
\footnote{We assume that
peers participate in DOAT altruistically or that there is an
incentives mechanism in place \cite{Incentives}.}
(more elaborate methods can be adopted to minimise
churn \cite{godfrey}).

Not every peer becomes a DOAT node and a peer does not decide in isolation
to become a DOAT node. When the load on a DOAT
node increases 
beyond a
locally defined threshold, it chooses the most stable peer in its local
area and sends a
DOAT invite message.


When a node leaves the system
it informs all its neighbours with a message that also includes its
direct neighbours in both directions. This way, all its neighbours can automatically
substitute the removed neighbour with one of its direct neighbours.
Routing updates also carry this information, in case the node fails without
warning. When a node detects that a neighbour has failed, it can
immediately forward queries and routing updates to the 
replacing neighbour of the failed node.

Node removal and re-insertion can be also triggered by network coordinate
drift \cite{wild}, or by changes in the prevailing network congestion and
delay patterns. As the position of a node in the overlay is dependent on
its DOAT coordinate (which itself is calculated from the its network
coordinates), if it moves too far away from its original position, the node
will remove itself from the DOAT and re-insert itself in the new position.
This is necessary because network delay changes that are not reflected in the
DOAT coordinate can reduce the accuracy of the closest group
member discovery (see Section \ref{subsect:evalcoordinates} for an evaluation of the
impact of coordinates accuracy on the DOAT accuracy).