\pdfoutput=1 \documentclass[12pt]{elsarticle}





\usepackage{graphicx}
\usepackage{wrapfig}

\usepackage{amssymb}
\usepackage{amsmath}

\usepackage{xspace}









\newtheorem{defin}{Definition}
  \newenvironment{definition}{\begin{defin} \sl}{\end{defin}}
\newtheorem{theo}[defin]{Theorem}
  \newenvironment{theorem}{\begin{theo} \sl}{\end{theo}}
\newtheorem{lem}[defin]{Lemma}
  \newenvironment{lemma}{\begin{lem} \sl}{\end{lem}}
\newtheorem{coro}[defin]{Corollary}
  \newenvironment{corollary}{\begin{coro} \sl}{\end{coro}}
\newenvironment{proof}{\emph{Proof.}}{\hfill  \medskip\\}


\newcommand{\etal}{\emph{et~al.}\xspace}
\newcommand{\from}[1]{{\emph{\textbf{(#1)}}}}

\journal{Computational Geometry: Theory and Applications}

\begin{document}

\begin{frontmatter}





\title{Making triangulations 4-connected using flips\tnoteref{thanks}}
\tnotetext[thanks]{This research was partially supported by NSERC, GraDR EUROGIGA project No. GIG/11/E023, MTM2009-07242, Gen. Cat. DGR 2009SGR1040 and ESF EUROCORES programme EuroGIGA, CRP ComPoSe: Fonds National de la Recherche Scientique (F.R.S.-FNRS) - EUROGIGA NR 13604.}


\author[carleton]{Prosenjit Bose}
\ead{jit@scs.carleton.ca}

\author[carleton]{Dana Jansens}
\ead{dana@cg.scs.carleton.ca}

\author[carleton]{Andr\'e van Renssen}
\ead{andre@cg.scs.carleton.ca}

\author[maria]{Maria Saumell}
\ead{maria.saumell.m@gmail.com}

\author[carleton]{Sander Verdonschot}
\ead{sander@cg.scs.carleton.ca}

\address[carleton]{School of Computer Science, Carleton University, 5302 Herzberg Laboratories,\\
1125 Colonel By Drive, Ottawa, Ontario, K1S 5B6, Canada.}
\address[maria]{Computer Science Department, Universit\'e Libre de Bruxelles,\\
Boulevard du Triomphe - CP 212, 1050 Brussels, Belgium.}

\begin{abstract}
We show that any combinatorial triangulation on  vertices can be transformed into a 4-connected one using at most  edge flips. We also give an example of an infinite family of triangulations that requires this many flips to be made 4-connected, showing that our bound is tight. In addition, for , we improve the upper bound on the number of flips required to transform any 4-connected triangulation into the canonical triangulation (the triangulation with two dominant vertices), matching the known lower bound of . Our results imply a new upper bound on the diameter of the flip graph of , improving on the previous best known bound of .
\end{abstract}

\begin{keyword}
diagonal flip \sep flip graph \sep triangulation \sep 4-connected triangulation \sep Hamiltonian triangulation


\end{keyword}

\end{frontmatter}



\section{Introduction}
\label{sec:introduction}

\noindent Given a triangulation (a maximal planar simple graph) on a set of  vertices, we define an \emph{edge flip} as removing an edge  from the graph and replacing it with the edge , where  and  are the other vertices of the two triangular faces that had  as an edge. Flipping an edge is allowed if and only if it does not create a duplicate edge. Figure~\ref{fig:flip} shows an example.

\begin{figure}[b]
 \centering
 \includegraphics{./figures/FlipNonGeometric.pdf}
 \caption{An example triangulation before and after flipping edge .}
 \label{fig:flip}
\end{figure}

Flips have been studied mostly in two different settings: the \emph{geometric} setting, where we are given a fixed set of points in the plane and edges are straight line segments, and the \emph{combinatorial} setting, where we are only given the clockwise order of edges around each vertex (a combinatorial embedding). In this paper, we concern ourselves with the number of flips required to transform one triangulation into another in the combinatorial setting. We give a brief overview of previous work on this problem. A more detailed overview, including full proofs of the previous bounds, can be found in a recent survey by Bose and Verdonschot~\cite{bose2012history}. For a broader overview on the topic of flips, including applications and related work, we refer the reader to a survey by Bose and Hurtado~\cite{bose2009flips}.

Given a set of  vertices, we define its \emph{flip graph} as the graph with a vertex for each distinct triangulation and an edge between two vertices if their corresponding triangulations differ by a single flip. Two triangulations are considered distinct if they are not isomorphic. In his seminal paper, Wagner~\cite{wagner1936bemerkungenzum} showed that there always exists a sequence of  flips that transforms a given triangulation into any other triangulation on the same number of vertices. In terms of the flip graph, Wagner showed that it is connected and has diameter . Recently, efforts have been made to find better bounds on the diameter of the flip graph. Komuro~\cite{komuro1997diagonal} was the first to show that the diameter is linear and Mori~\etal~\cite{mori2003diagonal} currently have the strongest upper bound of .

These results all show how to transform any triangulation into a fixed \emph{canonical} triangulation, which is the  triangulation with two dominant vertices (adjacent to every other vertex). Transformation of one triangulation into another is then straightforward. We transform the first triangulation into the canonical one and transform it into the second triangulation by reversing the sequence of flips that transforms the second triangulation into the canonical one. Mori~\etal's algorithm to transform a triangulation into the canonical one consists of two steps. They first make the given triangulation 4-connected using at most  flips. Since a 4-connected triangulation is always Hamiltonian~\cite{whitney1931theorem}, they then show how to transform any Hamiltonian triangulation into the canonical one by at most  flips, using a decomposition into two outerplanar graphs that share a Hamiltonian cycle as their outer faces.

The problem of making triangulations 4-connected has also been studied in the setting where many edges may be flipped simultaneously. Bose~\etal~\cite{bose2006simultaneous} showed that any triangulation can be made 4-connected by one simultaneous flip and that  simultaneous flips are sufficient and sometimes necessary to transform between two given triangulations.

In Section~\ref{sec:ub}, we show that any triangulation can be made 4-connected using at most  flips. This improves the first step of the construction by Mori~\etal For , we also improve the bound on the second step of their algorithm to match the lower bound by Komuro~\cite{komuro1997diagonal}. This results in a new upper bound on the diameter of the flip graph of . We then show in Section~\ref{sec:lb} that, when  is a multiple of 5, there are triangulations that require  flips to be made 4-connected, showing that our bound is tight. Section~\ref{sec:lemmas} contains proofs for various technical lemmas that are used in the proof of the upper bound. Section~\ref{sec:conclusions} contains a discussion of our results and some remaining open problems.

\section{Upper bound}
\label{sec:ub}

\noindent In this section we prove an upper bound on the number of flips required to make any given triangulation 4-connected. Specifically, we show that \mbox{} flips always suffice. The proof references several technical lemmas whose proofs can be found in Section~\ref{sec:lemmas}. We also prove that any 4-connected triangulation can be transformed into the canonical form using a worst-case optimal number of  flips.

We are given a triangulation , along with a combinatorial embedding specifying the clockwise order of edges around each vertex of . In addition, one of the faces of  is marked as the \emph{outer face}. If an edge of the outer face is flipped, one of the two new faces is designated as the new outer face. A \emph{separating triangle}  is a cycle in  of length three whose removal splits  into two (non-empty) connected components. We call the component that contains vertices of the outer face the \emph{exterior} of , and the other component the \emph{interior} of . A vertex in the interior of  is said to be \emph{inside}  and likewise, a vertex in the exterior of  is \emph{outside} . An edge is inside a separating triangle if at least one of its endpoints is inside. A separating triangle  \emph{contains} another separating triangle  if and only if the interior of  is a subgraph of the interior of  with a strictly smaller vertex set. If  contains ,  is called the \emph{containing} triangle. A separating triangle that is contained by the largest number of separating triangles in  is called \emph{deepest}. Since containment is transitive, a deepest separating triangle cannot contain any separating triangles, as these would have a higher number of containing triangles. Finally, we call an edge that does not belong to any separating triangle a \emph{free edge}. Free edges have the following useful property.

\begin{lemma}
 \label{lem:freeedge}
 In a triangulation, every vertex  of a separating triangle  is incident to at least one free edge inside .
\end{lemma}
\begin{proof}
 Consider one of the edges of  incident to . Since  is separating, its interior cannot be empty and since  is part of a triangulation, there is a triangular face inside  that uses this edge. Now consider the other edge  of this face that is incident to .

 The remainder of the proof is by induction on the number of separating triangles contained in . For the base case, assume that  does not contain any other separating triangles. Then  must be a free edge and we are done.

 For the induction step, there are two further cases. If  does not belong to a separating triangle, we are again done, so assume that  belongs to a separating triangle . Since  is itself a separating triangle contained in  and containment is transitive, the number of separating triangles contained in  must be strictly smaller than the number contained in . Since  is also a vertex of , our induction hypothesis tells us that there is a free edge incident to  inside . Since  is contained in , this edge is also inside .
\end{proof}

We will remove all separating triangles from  by repeatedly flipping an edge of a separating triangle. This makes  4-connected, as a triangulation is 4-connected if and only if it has no separating triangles. This technique was also used by Mori~\etal~\cite{mori2003diagonal}, who proved the following lemma.

\begin{lemma}
 \label{lem:flip}
 \from{Mori~\etal~\cite{mori2003diagonal}}
 In a triangulation on  vertices, flipping any edge of a separating triangle  will remove that separating triangle. This never introduces a new separating triangle, provided that the selected edge belongs to multiple separating triangles or none of the edges of  belong to multiple separating triangles.
\end{lemma}

\noindent With this in mind, our algorithm works as follows.
\
  V_i~~=~~V_{i-1} + 5 L_{i-1}~~=~~10 + 5 \sum_{j=2}^{i} L_{j-1} \label{eq:v}
 
  S_i~~=~~S_{i-1} + 3 L_{i-1}~~=~~4 + 3 \sum_{j=2}^{i} L_{j-1} \label{eq:s}
  \sum_{j=2}^{i} L_{j-1}~~=~~\frac{V_i - 10}{5}  S_i~~=~~4 + 3 \cdot \frac{V_i - 10}{5}~~=~~\frac{3 V_i - 10}{5} 
 Since each flip removes only the separating triangle that the edge belongs to, we need  flips to make this triangulation 4-connected. Constructions for multiples of 5 between  and  can be obtained by recursing on a subset of the triangles in the final recursion step.
\end{proof}

\noindent Note that this triangulation is not useful for a lower bound on the diameter of the flip graph in general, since it is already Hamiltonian. Thus, even though it is not 4-connected, we know that it can be transformed into the canonical triangulation by at most  flips from the proof by Mori~\etal~\cite{mori2003diagonal}.

\section{Lemmas and proofs}
\label{sec:lemmas}

\noindent This section contains proofs for the technical lemmas used in the proof of Theorem~\ref{thm:4-connected}.

\begin{lemma}
 \label{lem:interior}
 If a separating triangle  contains a separating triangle , then there is a vertex of  inside  and no vertex of  can lie outside .
\end{lemma}
\begin{proof}
 Let  be a vertex in the interior of  and let  be a vertex of  that is not shared with . Since the interior of  is a subgraph of the interior of  and  is not inside ,  must be outside . Since every triangulation is 3-connected, there is a path from  to  that stays inside . This path connects the interior of  to the exterior, so there must be a vertex of  on the path and hence inside .
 
 Now suppose that there is another vertex of  outside . Since all vertices of a triangle are connected by an edge, there is an edge between this vertex and the vertex of  inside . This contradicts the fact that  is a separating triangle, so no such vertex can exist.
\end{proof}
\vspace{-0.5em}
\begin{lemma}
 \label{lem:interiorvertex}
 If a vertex  of a separating triangle  is inside a separating triangle , then  contains .
\end{lemma}
\begin{proof}
 Let  be a vertex of  that is not shared with . There is a path from  to the outer face that stays in the exterior of . There can be no vertex of  on this path, since this would create an edge between the interior and exterior of . Therefore  is outside .
 
 Now suppose that  does not contain . Then there is a vertex  inside  that is not inside . There must be a path from  to  that stays inside . Since  is inside , there must be a vertex of  on this path. But since  is outside , this would create an edge between the interior and exterior of . Therefore  must contain .
\end{proof}
\vspace{-0.5em}
\begin{lemma}
 \label{lem:onecontainingtriangle}
 A separating triangle can share at most one edge with containing triangles.
\end{lemma}
\begin{proof}
 Suppose we have a separating triangle  that shares two of its edges with separating triangles that contain it. First of all, these triangles cannot be the same, since then they would be forced to share the third edge as well, which means that they are . Since a triangle does not contain itself, this is a contradiction. So call one of these triangles  and call one of the triangles that shares the other edge . Let ,  and  be the vertices of , such that  is shared with  and ,  is shared only with  and  is shared only with .

 By Lemma~\ref{lem:interior},  must be inside , while  must be inside , since in both cases the other two vertices of  are shared and therefore not in the interior. But then by Lemma~\ref{lem:interiorvertex},  contains  and  contains . This is a contradiction, since by transitivity it would imply that the interior of  is a subgraph of itself with a strictly smaller vertex set.
\end{proof}
\vspace{-0.5em}
\begin{lemma}
 \label{lem:onecontaining}
 A separating triangle  that shares no edge with containing triangles can share at most one vertex with containing triangles.
\end{lemma}
\begin{proof}
 Suppose that  shares two of its vertices with containing triangles. First, both vertices cannot be shared with the same containing triangle, since then the edge between these two vertices would also be shared. Now let  be one of the containing triangles and let  be one of the containing triangles sharing the other vertex. By Lemma~\ref{lem:interior}, there must be a vertex of  inside . So then both vertices of  that are not shared with  must be inside , otherwise there would be an edge between the interior and the exterior of . In particular, the vertex shared by  and  lies inside , which by Lemma~\ref{lem:interiorvertex} means that  contains . But the reverse is also true, so  contains  as well, which is a contradiction.
\end{proof}
\vspace{-0.5em}
\begin{lemma}
 \label{lem:unsharedvertex}
 A separating triangle that shares an edge with a containing triangle cannot share the unshared vertex with another containing triangle.
\end{lemma}
\begin{proof}
 Suppose we have a separating triangle  that shares an edge  with a containing triangle  and the other vertex  with another containing triangle . By Lemma~\ref{lem:interior}, at least one of  and  has to be inside . Since these are vertices of , by Lemma~\ref{lem:interiorvertex},  contains . Similarly,  has to be inside  and since it is a vertex of ,  contains . This is a contradiction.
\end{proof}
\vspace{-0.5em}
\begin{lemma}
 \label{lem:containingshared}
 Given two separating triangles  and  that share an edge , any separating triangle that contains  but not  must use .
\end{lemma}
\begin{proof}
 Suppose that we have a separating triangle  that contains , but not  and that does not use one of the vertices  of . By Lemma~\ref{lem:interior},  must be inside . But then  would also contain  by Lemma~\ref{lem:interiorvertex}, as  is a vertex of  as well. Therefore  must share both vertices of  and hence  itself.
\end{proof}
\vspace{-0.5em}
\begin{lemma}
 \label{lem:outertriangle}
 A separating triangle  that uses an edge  of the outer face cannot be contained in a separating triangle that does not share .
\end{lemma}
\begin{proof}
 Suppose  is contained in a separating triangle . If  does not share , by Lemma~\ref{lem:interior}, at least one of the vertices of  must be inside . But since  is part of the outer face, this is a contradiction.
\end{proof}

\section{Conclusions and future work}
\label{sec:conclusions}

\noindent We showed that any triangulation on  vertices can be made 4-connected using at most  flips, while there are triangulations that require  flips when  is a multiple of 5. This shows that our bound is tight for an infinite family of values for , although a slight improvement to  is still possible. We believe that this is the true bound. We also showed that any 4-connected triangulation on  vertices can be transformed into the canonical form using at most  flips. This matches the lower bound by Komuro~\cite{komuro1997diagonal} in the worst case where the graph has maximum degree 6 and results in a new upper bound of  on the diameter of the flip graph. It also means that both steps of the algorithm, when considered individually, are now tight in the worst case. Therefore, any further improvement must either merge the two steps in some fashion or employ a different technique.

One potential direction for improvement is to see if fewer flips are required to make triangulations Hamiltonian than to make them 4-connected. While 4-connectivity is a sufficient condition for Hamiltonicity, it is not required. Aichholzer~\etal~\cite{oswin08} gave a lower bound of  on the number of flips required to make a triangulation Hamiltonian.

Further, all of the current algorithms use the same, single, canonical form. Surprisingly, the lower bound of  flips to the canonical form is also the best known lower bound on the diameter of the flip graph. This suggests that at least one of the two bounds still has significant room for improvement. So is there another canonical form that gives a better upper bound? Or can we get a better bound by using multiple canonical forms and picking the closest?

Another interesting problem is to minimize the number of flips to make a triangulation 4-connected. We showed that our technique is worst-case optimal, but there are cases where fewer flips would suffice. There is a natural formulation of the problem as an instance of 3-hitting set, where the subsets correspond to the edges of separating triangles and we need to pick a minimal set of edges such that we include at least one edge from every separating triangle. This gives a simple 3-approximation algorithm that picks an arbitrary separating triangle and flips all shared edges or an arbitrary edge if there are no shared edges. However, it is not clear whether the problem is even NP-hard, as not all instances of 3-hitting set can be encoded as separating triangles in a triangulation. Therefore it might be possible to compute the optimal sequence in polynomial time.









\bibliographystyle{unsrt}
\bibliography{papers}













\end{document}
