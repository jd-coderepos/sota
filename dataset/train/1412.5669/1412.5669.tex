
\documentclass[11pt]{amsart}

\usepackage{amsfonts, amsmath, amssymb}
\usepackage{amsaddr}
\usepackage{graphicx,color}
\usepackage{listings}
\usepackage{algorithm}
\usepackage{algpseudocode}
\usepackage{cancel} \usepackage{wrapfig}
\usepackage{color}
\usepackage{tikz}
\usepackage{url}
\usepackage{stmaryrd}

\newtheorem{theorem}{Theorem}[section]
\newtheorem{corollary}[theorem]{Corollary}
\newtheorem{question}[theorem]{Question}
\newtheorem{conjecture}[theorem]{Conjecture}
\newtheorem{lemma}[theorem]{Lemma}
\newtheorem{proposition}[theorem]{Proposition}

\theoremstyle{definition}
\newtheorem{definition}[theorem]{Definition}
\newtheorem{remark}[theorem]{Remark}
\newtheorem{remarks}[theorem]{Remarks}
\newtheorem{example}[theorem]{Example}
\newtheorem{examples}[theorem]{Examples}

\newcommand{\LLL}{\mathfrak{L}}
\newcommand{\DDD}{\mathfrak{D}}
\newcommand{\RRR}{\mathfrak{R}}
\newcommand{\ARRR}{\mathfrak{R}^t}
\newcommand{\IARRR}{\mathfrak{R}_{\infty}^{t}}
\newcommand{\PARRR}{\mathfrak{R}_\textnormal{per}^{t}}
\newcommand{\SSS}{\mathfrak{S}}

\newcommand{\TTTSSS}{\mathbf{TS}}

\newcommand{\ta}{\mathrm{TA}}
\newcommand{\dta}{\mathrm{DTA}}
\newcommand{\nta}{\mathrm{NTA}}
\newcommand{\taeps}{\mathrm{eTA}}
\newcommand{\ntaeps}{\mathrm{eNTA}}

\newcommand{\Actions}{\Sigma}
\newcommand{\ActionsA}{\Sigma^{A}}
\newcommand{\eActions}{\Sigma_{\epsilon}}
\newcommand{\eActionsA}{\Sigma^{A}_{\epsilon}}
\newcommand{\eActionsB}{\Sigma^{B}_{\epsilon}}
\newcommand{\eActionsAparB}{\Sigma^{A \parallel B}_{\epsilon}}
\newcommand{\InputActions}{\Sigma_{I}}
\newcommand{\OutputActions}{\Sigma_{O}}
\newcommand{\SharedActions}{\Sigma_{shr}}

\newcommand{\Locs}{\mathcal{Q}}
\newcommand{\Clocks}{\mathcal{C}}
\newcommand{\ResetClocks}{\mathcal{C}_{rst}}
\newcommand{\Invariants}{\mathcal{I}}
\newcommand{\clk}{c}
\newcommand{\Qinit}{\mathcal{Q}_{init}}
\newcommand{\Qacc}{\mathcal{Q}_{accept}}
\newcommand{\Qracc}{\mathcal{Q}_{raccept}}
\newcommand{\Qlacc}{\mathcal{Q}_{laccept}}
\newcommand{\Trans}{\mathcal{T}}
\newcommand{\Guards}{\mathcal{G}}
\newcommand{\SharedGuards}{\mathcal{G}_{shr}}
\newcommand{\automaton}{\mathcal{A}}
\newcommand{\mutant}[2]{M_{#1}(#2)}
\newcommand{\sink}{\textrm{sink}}
\newcommand{\untime}{\textrm{untime}}
\newcommand{\States}{\mathcal{S}}
\newcommand{\BS}{\mathcal{B}}
\newcommand{\Kbound}{\mathcal{K}}

\newcommand{\telapse}{\tau}
\newcommand{\ClockVal}{\mathcal{V}}
\newcommand{\zerov}{\textbf{0}}
\newcommand{\proj}[1]{\textsf{P}#1}

\newcommand{\Naturals}{\mathbb{N}}
\newcommand{\ZNaturals}{\mathbb{N}_0}
\newcommand{\Integers}{\mathbb{Z}}
\newcommand{\Reals}{\mathbb{R}}
\newcommand{\PReals}{\mathbb{R}_{\geq 0}}

\newcommand{\frc}[1]{\left\{{#1}\right\}}
\newcommand{\cycles}[1]{\textsf{cycles} \left({#1}\right) }
\newcommand{\lcm}{\mathrm{lcm}}

\newcommand{\BLanInc}{\textsc{BLI}}
\newcommand{\LanInc}{\textsc{LI}}
\newcommand{\PLICE}{\textsc{PLICE}}
\newcommand{\Unfold}{\textsc{Unfold}}
\newcommand{\Complete}{\textsc{Complete}}
\newcommand{\MakeNonaccept}{\textsc{MakeNonaccept}}
\newcommand{\MakeRightNonaccept}{\textsc{MakeRightNonaccept}}
\newcommand{\MakeLeftNonaccept}{\textsc{MakeLeftNonaccept}}
\newcommand{\RightAcceptPath}{\textsc{RightAcceptPath}}
\newcommand{\BuildParallel}{\textsc{BuildParallel}}
\newcommand{\LanAccept}{\textsc{Accept}}

\newcommand{\BLI}{\textrm{BLI}}
\newcommand{\LI}{\textrm{LI}}
\newcommand{\Init}{\textrm{Init}}
\newcommand{\Accept}{\textrm{Accept}}
\newcommand{\timeProject}{\textrm{timeProject}}
\newcommand{\actProject}{\textrm{actProject}}
\newcommand{\genStep}{\textrm{genStep}}
\newcommand{\Step}{\textrm{Step}}
\newcommand{\eStep}{\textrm{eStep}}
\newcommand{\Delay}{\textrm{Delay}}
\newcommand{\Jump}{\textrm{Jump}}
\newcommand{\Path}{\textrm{Path}}
\newcommand{\ePath}{\textrm{ePath}}
\newcommand{\accPath}{\textrm{accPath}}
\newcommand{\lAccPath}{\textrm{lAccPath}}
\newcommand{\rAccPath}{\textrm{rAccPath}}
\newcommand{\rAccept}{\textrm{rAccept}}
\newcommand{\lAccept}{\textrm{lAccept}}

\newcommand{\pass}{\textbf{pass}}
\newcommand{\fail}{\textbf{fail}}
\newcommand{\inconclusive}{\textbf{inc}}

\newcommand{\RReset}{\textrm{Reset}}

\newcommand{\ActionCode}[1]{\textrm{Act}_{#1}}

\newcommand{\powerset}[1]{\mathcal P \left({#1}\right) }

\renewcommand{\algorithmicrequire}{\textbf{Input:}}
\renewcommand{\algorithmicensure}{\textbf{Output:}}

\newcommand\pfun{\mathrel{\ooalign{\hfil\hfil\cr\cr}}}

\newcommand{\incremental}{\textsf{Incremental}}
\newcommand{\classic}{\textsf{Classic}}

\newcommand{\shortw}{\textsf{Short}}
\newcommand{\longw}{\textsf{Long}}

\usepackage{color}
\newcommand{\FIXME}[1]{{\color{red}FIXME: #1}}

\begin{document}

\title{The Timestamp of Timed Automata}
\author{Amnon Rosenmann}
\address{Graz University of Technology, Steyrergasse 30, A-8010 Graz, Austria}
\email{rosenmann@math.tugraz.at}



\date{}
\maketitle

\begin{abstract}
Let  be the class of non-deterministic timed automata with silent transitions. Given , we effectively compute its timestamp: the set of all pairs (time value, action) of all observable timed traces of . We show that the timestamp is eventually periodic and that one can compute a simple deterministic timed automaton with the same timestamp as that of . As a consequence, we have a partial method, not bounded by time or number of steps, for the general language non-inclusion problem for . We also show that the language of  is periodic with respect to suffixes. 
\end{abstract}
\section{Introduction}
\label{sec:intro}
Timed automata () are finite automata extended with clocks that measure the time that elapsed since past events in order to control the triggering of future events.
They were defined by Alur and Dill in their seminal paper \cite{ta} as abstract models of real-time systems and were implemented in tools like UPPAAL \cite{uppaal}, Kronos \cite{Kronos98}, RED \cite{Wang04} and PRISM\cite{Prism11}.

A fundamental problem in this area is the reachability problem, which in its basic form asks whether a given location of a timed automaton is reachable from the initial location.
The set of states of the system (i.e., locations and valuation to the clocks) is, in general, an infinite uncountable set.
However, through the construction of a region automaton, which contains finitely-many equivalence classes of regions \cite{ta}, the reachability problem becomes a decidable problem (though of complexity PSPACE-complete).


Research on the reachability problem went beyond the above basic question.
In \cite{CY92} it is shown that the problem of the minimum and maximum reachability time is also PSPACE-complete.
In another work, \cite{CJ99}, which is more of a theoretical nature, the authors show that some problems on the relations between states may be defined in the decidable theory of the domain of real numbers equipped with the addition operation.
In particular, the reachability problem between any two states is decidable.
For other aspects of the reachability problem, also in the context of variants and extensions of timed automata (e.g. with game and probability characteristics) we refer to \cite{CY92},\cite{member-ta-ha}, \cite{TY01},  \cite{WZP03}, \cite{control_ta}, \cite{HP06}, \cite{CHKM11}, \cite{HOW12}.
In this paper we generalize the reachability problem in another direction.
We show that the problem of computing the set of all time values on which any observable transition occurs (and thus, a location is reached by an observable transition) is solvable.
This set, called the \emph{timestamp} of the automaton  and denoted , is more precisely defined to be the set of all pairs  that appear in the observable timed traces of .
Note that for this definition it does not matter whether we consider infinite runs or finite ones.

We show that the timestamp is in the form of a union of action-labeled open intervals with integral end-points, and action-labeled points of integral values.
When the timestamp is unbounded in time then it is eventually periodic.
 
The set of languages defined by the class  of deterministic timed automata is strictly included in the set of languages defined by the class  of non-deterministic timed automata \cite{ta}, \cite{Finkel06}, and the latter is strictly included in the set of languages defined by the class  of non-deterministic timed automata with silent transitions \cite{ta-eps}.
The fundamental problem of inclusion of the language accepted by a timed automaton  (e.g. the implementation) in the language accepted by the timed automaton  (e.g. the specification) is undecidable for the class  but decidable for the class .
On the other hand, for special sub-classes or modifications it was shown that decidability exists (see \cite{bbbb,ta-eps,era,updatable-ta,Ouak_one_clk,Ouak_time_bound,bounded-time,Dec_TA_Survey,bound-det} for a partial list).
However, the abstraction (or over-approximation) represented in the form of a timestamp is a discrete object, in which questions like inclusion of timestamps or universality are decidable.
In fact, we show that for any given non-deterministic timed automaton with silent transitions, one can construct a simple deterministic timed automaton having the same timestamp. 

The computation of the timestamp is done through the construction of a periodic augmented region automaton .
It is a region automaton augmented with a global non-resetting clock  and containing periodic regions and periodic transitions: they are defined modulo a time period .
This kind of abstraction demonstrates a periodic nature which is absent, in general, from timed traces: there are timed automata with no timed traces that are eventually periodic (see Example~\ref{ex:non-period}).
Periodic transitions were introduced in \cite{periodic}, where it was shown that they increase the expressiveness of , though they are less expressive than silent transitions.

The construction of the periodic automaton is preceded by defining the infinite augmented region automaton , in which the values of the clock  are unbounded.
Then, after exhibiting the existence of a pattern that repeats itself every  time units, we fold the infinite automaton into a finite one according to this periodic structure.

Our construction shows that the language of a timed automaton  is periodic with respect to suffixes: for every run  with suffix  that occurs after passing a fixed computable time there are infinitely-many runs of  with the same suffix , but with the suffix shifted in time by multiples of .
Note that this result does not follow from the pumping lemma, which does not hold in general in timed automata \cite{pumping}. 

In Section~\ref{sec:ta} basic definitions concerning timed automata are given.
Then, in Section~\ref{sec:sing_path} we describe the trail and timestamp of a single path of a timed automaton, more from a geometric than from an algebraic point of view, after treating the absolute-time clock  as part of the system.
The augmented and infinite augmented region automaton,  and , are presented in Section~\ref{sec:IARA}, and then, in Section~\ref{sec:per}, we explore the time-periodicity in them, so that  can be folded into the finite periodic augmented region automaton  (Section~\ref{sec:APARA}).
In last section (Section~\ref{sec:timestamp}) we construct the entire eventually periodic timestamp.
As for the general language inclusion problem in , the timestamp, or better - the more informative automaton , may serve as a tool in demonstrating the non-inclusion relation between the languages of two members of s.
\section{Timed Automata with Silent Transitions}
\label{sec:ta}
A timed automaton is an abstract model aiming at capturing the temporal behavior of real-time systems.
It is a finite automaton extended with a finite set of clocks defined over , the set of non-negative real numbers.
It consists of a finite set of \emph{locations}  with a finite set of \emph{transitions}  between the locations, while time, measured by the clocks, is continuous.
A transition at time  can occur only if the condition expressed as a \emph{transition guard}
is satisfied at .
The transition is immediate - no clock is advancing in time. However, some of the clocks may be reset to zero.

There are two sorts of transitions: \emph{observable} transitions, which can be traced by an outside observer, and \emph{silent} transitions, which are inner transitions and thus cannot be observed from the outside.
There are finitely-many types of observable transitions, each type labeled by a unique \emph{action} , whereas all the silent transitions have the same label .
In , the class of non-deterministic timed automata, there exist states in which two transitions from the same location  can be taken at the same time and with the same action but to two different locations  and .
When this situation cannot happen, the TA is deterministic.


Let  and let  be the power set of a set .
A transition guard is a conjunction of constraints of the form , where  is a clock,  and .
A formal definition of  is as follows.
\begin{definition}[]
\label{def:ntaeps}
A \emph{non-deterministic timed automaton with silent transitions}  is a tuple , where:
\begin{enumerate}
\item  is a finite set of locations and  is the initial location;
\item  is a finite set of transition labels, called actions, where  refers to the observable actions and  represents a silent transition;
\item  is a finite set of clock variables;
\item  is a finite set of transitions of the form , where:
\begin{enumerate}
\item  are the source and the target locations respectively;
\item  is the transition action;
\item  is the \emph{transition guard};
\item  is the subset of clocks to be reset.
\end{enumerate}
\end{enumerate}
\end{definition}

A clock \emph{valuation}  is a function . 
We denote by  the set of all clock valuations and by
 the valuation which assigns the value  to every clock.
Given a valuation  and , we define  to be the valuation  for every .
The valuation , , is defined to be  for  and  for .

The \emph{semantics} of  is given by the \emph{timed transition system}
, where:
\begin{enumerate}
\item  is the set of states, with
 the initial state;
\item  is the transition relation.
The set  consists of
\begin{enumerate}
\item \emph{Timed transitions (delays):} , where ;
\item \emph{Discrete transitions (jumps):} , where  and there exists a transition  in , such that for each clock ,  satisfies the constraints of  regarding , and .
\end{enumerate}
\end{enumerate}

A (finite) \emph{run}  of  is a sequence of alternating timed and discrete transitions of the form 

and \emph{duration} .
The run  of  induces the \emph{timed trace} (\emph{timed word})

with  and . 
From the latter we can extract the \emph{observable timed trace} (\emph{observable timed word}), which is obtained by deleting from  all the pairs containing silent transitions.
Note that when the TA is deterministic then each timed trace refers to a unique run.
We remark that we did not include for a location  the location invariants in the definition of timed automata since these invariants can be incorporated in the guards of the transitions to  (for the clocks that are not reset at the transitions) and in those emerging from .
We also do not distinguish between accepting and non-accepting locations as they do not change the analysis and results concerning the reachability problems that are dealt with here.
Thus, the \emph{language}  of  refers here to the set of observable timed traces of  without restricting it to those observable timed traces of runs that end in acceptable locations.
\section{The Trail and Timestamp of a Single Path}
\label{sec:sing_path}
In this section we describe the trail and timestamp of a single path of a TA.
Given a timed automaton  over  clocks , we add to it a non-resetting global clock  that displays absolute time.
A finite \emph{path} in  has the form  of alternating locations and transitions, with  the initial location and  a transition between  and , , that is, a path here refers to the standard definition in a directed graph.
A run of the TA induces a \emph{trajectory} in the non-negative part of the -space that is a piecewise-linear curve (the discontinuity is the clocks reset).
\begin{definition}[Trajectory of a run]
	Let  be an ordered set of clocks of . 
	Let  be a run of duration  of .
	The \emph{trajectory} of  is the set of points  in the -space visited during , where .
\end{definition}
Next, we define the \emph{trail} of a path.
\begin{definition}[Trail of a path]
	The \emph{trail of a path} 
is the union of the trajectories of all feasible runs along , that is, runs that follow the locations and discrete transitions of .
\end{definition}
The \emph{trail legs}, the parts of the trail between clocks reset, are in the form of \emph{zones} \cite{zones}, a conjunction of diagonal constraints  or , , bounded by transition constraints , where , .
Each trail leg can be further partitioned into \emph{simplicial trails}, which are (possibly unbounded) parallelotopes consisting of a sequence of \emph{regions} \cite{ta} arranged along the directional vector .
Each region  is in the form of an open (unless it is a point) simplex  that is a hyper-triangle of dimension .
The simplex  is characterized by the fractional values  of the clock variables, and each point in the simplex satisfies the same fixed ordering of the form
 
 where .
The integral point  consists of the integral parts of the values of the clocks , and it indicates the lowest point in 
the -space of the boundary of the region.  
Each region has a unique \emph{immediate time-successor}, which is the next region along the directional vector , as long as no clock is reset on an event.

When the simplicial trail  is -dimensional then the immediate time-successor of an open -simplex (a simplex of dimension , ) is a -simplex and vice-versa, where each -simplex is a face of its neighbouring -simplices.
A region which is in the form of a -simplex refers to the case where the fractional parts of the clocks are all non-zero, and then its immediate time-successor is a -simplex, in which the integral part of the clock with maximal fractional part is increased by 1 while its fractional part is set to zero. The order between the other clocks remains as before.
The switch from a -simplex into a -simplex occurs when a clock of fractional part 0 turns into a positive fractional part and the order of the fractional parts of the clocks as well as their integral values remains as before.  
Thus, at each switch there is a cyclic shift in the fractional parts of the clocks, which results in a periodic sequence of simplices along a simplicial trail.

Let  be the feasible \emph{duration} of the -th event along a path .
That is, , where  is the supremum, over the runs along , of the time at which the -th event of the run occurs, and  is defined as the infimum of the same set.
In case of an automaton with a single clock , if  resets on this transition then the size of the temporal part of the timestamp of the -th event increases by , resulting in an increase in the width of the parallelogram that represents the trail of  after the -th event, and possibly increasing the dimension of the trail from 1 to 2.
Otherwise, the width remains as before.
In case of multiple clocks, the dimension of the trail can increase, decrease or stay the same after an event with reset of clocks: clocks with the same fractional part can be separated, resulting in an increase of the dimension, while clocks whose fractional parts become identical (namely, ) contribute to a decrease of the dimension.

Let us look at a simple example of the trail and timestamp of a path in an automaton with a single clock.
\begin{example}
	In Fig.~\ref{fig:trail_and_ts_path}(a) a TA is drawn, and in Fig.~\ref{fig:trail_and_ts_path}(b) we see the trail and timestamp of the finite path , where '-timestamp' refers to the projection on the -axis of the elements  of the timestamp, and similarly for '-timestamp'.
	The first event occurs when  and the timestamp is .
	Then  resets and the trail (a straight line of slope ) continues from the -axis.
	Event 2 occurs when  with timestamp  and a reset of .
	After that event the trail is 2-dimensional (a parallelogram).
	Event  occurs when  without clock reset, and the orthogonal projection to the -axis gives the timestamp  (here  is the open interval ).
	The fourth event happens when  and its timestamp is .
	The timestamp of  is the union of the above sets, that is, , with  and .
	\begin{figure}[htb]
\centering
		\scalebox{0.5}{ \begin{picture}(0,0)\includegraphics{trail_and_ts_path.pdf}\end{picture}\setlength{\unitlength}{3947sp}\begingroup\makeatletter\ifx\SetFigFont\undefined \gdef\SetFigFont#1#2#3#4#5{\reset@font\fontsize{#1}{#2pt}\fontfamily{#3}\fontseries{#4}\fontshape{#5}\selectfont}\fi\endgroup \begin{picture}(11280,7912)(-4739,-10550)
\put(151,-6436){\makebox(0,0)[lb]{\smash{{\SetFigFont{14}{16.8}{\rmdefault}{\bfdefault}{\updefault}{\color[rgb]{0,0,0}-timestamp}}}}}
\put(-2249,-7186){\makebox(0,0)[lb]{\smash{{\SetFigFont{17}{20.4}{\rmdefault}{\mddefault}{\updefault}{\color[rgb]{0,0,0}}}}}}
\put(1726,-2761){\makebox(0,0)[lb]{\smash{{\SetFigFont{12}{14.4}{\rmdefault}{\mddefault}{\updefault}{\color[rgb]{0,0,0}}}}}}
\put(6526,-5236){\makebox(0,0)[lb]{\smash{{\SetFigFont{12}{14.4}{\rmdefault}{\mddefault}{\updefault}{\color[rgb]{0,0,0}}}}}}
\put(-3674,-5761){\makebox(0,0)[lb]{\smash{{\SetFigFont{14}{16.8}{\rmdefault}{\mddefault}{\updefault}{\color[rgb]{0,0,0}}}}}}
\put(-2024,-5761){\makebox(0,0)[lb]{\smash{{\SetFigFont{14}{16.8}{\rmdefault}{\mddefault}{\updefault}{\color[rgb]{0,0,0}}}}}}
\put(-374,-5761){\makebox(0,0)[lb]{\smash{{\SetFigFont{14}{16.8}{\rmdefault}{\mddefault}{\updefault}{\color[rgb]{0,0,0}}}}}}
\put(-1649,-5311){\makebox(0,0)[lb]{\smash{{\SetFigFont{14}{16.8}{\rmdefault}{\mddefault}{\updefault}{\color[rgb]{0,0,0}}}}}}
\put(-1124,-6211){\makebox(0,0)[lb]{\smash{{\SetFigFont{14}{16.8}{\rmdefault}{\mddefault}{\updefault}{\color[rgb]{0,0,0}}}}}}
\put(-3674,-4111){\makebox(0,0)[lb]{\smash{{\SetFigFont{14}{16.8}{\rmdefault}{\mddefault}{\updefault}{\color[rgb]{0,0,0}}}}}}
\put(-1199,-5086){\makebox(0,0)[lb]{\smash{{\SetFigFont{14}{16.8}{\rmdefault}{\mddefault}{\updefault}{\color[rgb]{0,0,0}}}}}}
\put(4113,-9706){\makebox(0,0)[lb]{\smash{{\SetFigFont{11}{13.2}{\rmdefault}{\mddefault}{\updefault}{\color[rgb]{0,0,0}}}}}}
\put(-644,-7361){\makebox(0,0)[lb]{\smash{{\SetFigFont{11}{13.2}{\rmdefault}{\mddefault}{\updefault}{\color[rgb]{0,0,0}}}}}}
\put(1351,-10486){\makebox(0,0)[lb]{\smash{{\SetFigFont{17}{20.4}{\rmdefault}{\mddefault}{\updefault}{\color[rgb]{0,0,0}}}}}}
\put(3751,-7186){\makebox(0,0)[lb]{\smash{{\SetFigFont{17}{20.4}{\rmdefault}{\mddefault}{\updefault}{\color[rgb]{0,0,0}}}}}}
\put(-2924,-5986){\makebox(0,0)[lb]{\smash{{\SetFigFont{14}{16.8}{\rmdefault}{\mddefault}{\updefault}{\color[rgb]{0,0,0}}}}}}
\put(-3524,-6211){\makebox(0,0)[lb]{\smash{{\SetFigFont{14}{16.8}{\rmdefault}{\mddefault}{\updefault}{\color[rgb]{0,0,0}}}}}}
\put(-4724,-4936){\makebox(0,0)[lb]{\smash{{\SetFigFont{14}{16.8}{\rmdefault}{\mddefault}{\updefault}{\color[rgb]{0,0,0}}}}}}
\put(-4349,-4711){\makebox(0,0)[lb]{\smash{{\SetFigFont{14}{16.8}{\rmdefault}{\mddefault}{\updefault}{\color[rgb]{0,0,0}}}}}}
\put(-1574,-6436){\makebox(0,0)[lb]{\smash{{\SetFigFont{14}{16.8}{\rmdefault}{\mddefault}{\updefault}{\color[rgb]{0,0,0}}}}}}
\put(151,-5836){\makebox(0,0)[lb]{\smash{{\SetFigFont{14}{16.8}{\rmdefault}{\bfdefault}{\updefault}{\color[rgb]{0,0,0}-timestamp}}}}}
\end{picture} }
		\caption{Trail, timestamp and regions of a path (single clock)}
		\label{fig:trail_and_ts_path}
\end{figure} 
\end{example}
\begin{definition}[Timestamp of a run]
	The \emph{timestamp of a run}  is the set of pairs  of the observable timed trace induced by .
\end{definition}
A finite path in  has the form  of alternating locations and transitions, and we always assume that  is the initial location.
Such a path is an abstraction of a run since the temporal part is omitted.
Given a path  in , there may be many possible runs along , and we say that  is \emph{feasible} when there is at least one run along it.
\begin{definition}[Timestamp of a path]
	The \emph{timestamp of a feasible path}  of  is the union of the timestamps of all runs  along .
\end{definition}
Each instance of a transition along  is an \emph{event}.
That is, a transition is a static object which joins two locations of the TA, whereas an event refers to a specific occurrence of a transition within the path .
Hence, several events along a path may refer to the same transition of the TA.
\begin{definition}[Timestamp of an event in a path]
	The \emph{timestamp of an event} in a path  is the union of the timestamps of that event of all runs along .
	It is the part of the timestamp of the path that refers to that event.
\end{definition}
\begin{proposition}
\label{pr:mult_ev_timestamp}
The timestamp of each event is either a labeled integral point or a labeled (open, closed or half-open) interval between points  and , ,  and .
\end{proposition}
\begin{proof} 
	The trail of each path is composed of simplices as in \eqref{eq:simplex} residing on the integral grid.
	The intersection of such a simplex  with a domain satisfying a transition constraint of the form , where ,  is either the whole of  or the empty set.
	A possible reset of clocks  during an event results in mapping  to another simplex , which may be of smaller dimension.
	Thus, it suffices to show that the timestamp of a single simplex  is of the required form.
	But the temporal part of the timestamp of  is the set , were  and  is the set of values of the clock  in .
	Since  is either  or the open interval , we get that the timestamp of  is either an action-labeled integral point  or an action-labeled open unit interval .

	Another way of proving the claim is via linear programming.
	Suppose that a path  contains  events and that the time of event , , is recorded by the variable . Then we can represent  as satisfying equalities and inequalities over the integers: instead of referring to a regular clock  in the constraint of the -th transition along , we refer to the variable , where the -th transition along  was the last time that the clock  was reset. The result then follows by the fact that the corresponding maximum and minimum linear programming problems have integer solutions.	
\end{proof}	
\begin{definition}[Timestamp of a timed automaton]
	The timestamp  of a timed automaton  is the set of all pairs , such that an observable transition with action  occurs at time  in some run of .
\end{definition}
\section{Augmented and Infinite Augmented Region Automaton}
\label{sec:IARA}
\subsection{Infinite Augmented Region Automaton}
Given a (finite) timed automaton , the region automaton  \cite{ta} is a finite \emph{discretized} version of , such that time is abstracted and both automata define the same untimed language.
Each vertex in  records a location  in  and a region , which is either in the form of a simplex (as described in Section~\ref{sec:sing_path}) or an unbounded region, in which the value of at least one of the clocks is , meaning that it passed the maximal integer value  that appears in the transition guards.
The regions partition the space of clock valuations into equivalence classes, where two valuations belong to the same equivalence class if and only if they agree on the clocks with  value and on the integral parts and the order among the fractional parts of the other clocks.
The edges of  are labeled by the transition actions, and they correspond to the actual transitions that occur in the runs of .
Using the time-successor relation over the clock regions (see \cite{ta}), the region automaton can be effectively constructed.
As shown in \cite{ta}, through the region automaton the questions of reachable locations and states of  and the actions along the (possibly infinitely-many) paths that lead to these locations, i.e. the untimed language of , become decidable.

Now we define the \emph{infinite augmented region automaton} .
First, we add to  a clock  that measures absolute time, does not appear in the transition guards, is never reset to  and does not affect the runs and timed traced of .
Next, we construct the region automaton augmented with .
The construction is similar to the construction of the standard region automaton with respect to the regular clocks (all clocks except for ) and the maximal bound , that is, the time regions of each regular clock  are , , , the latter being unbounded and refers to all values of  greater than .
The integration of the clock  is as follows. The construction of regions is as usual by considering the integral parts and the order of the fractional parts of all clocks, including . The only difference is that the integral part of  is in  and not bounded by . Thus, the infinitely-many time-regions associated with  are the alternating point and open unit interval: , , ,  (see Fig.~\ref{fig:APTA_a}(b)).
Hence,   contains information about absolute time that is lacking from the standard region automaton.
\begin{definition}[Infinite augmented region automaton]
	\label{def:inf_aug_region_automaton}
	Given  extended with the clock  that measures absolute time, a corresponding \emph{infinite augmented region automaton}  is a tuple , where:
	\begin{enumerate}
		\item  is an infinite (in general) set of vertices of the form , where  is a location of  and the pair  is a region, with
		
		containing the integral parts of the clocks , and 
		is the simplex defined by the order of the fractional parts of the clocks.
		\item  is the initial vertex with  the initial location of  and with all clocks having integral part and fractional part equal to 0.
		\item  is the set of edges.
		There is an edge
		
		labeled with  in  if and only if there is a run of   which contains a timed transition followed by a discrete transition of the form
		
		such that the clock valuation  over  represents a point in the region  and the clock valuation  represents a point in the region .
		\item  is the finite set of actions that are edge labels.
	\end{enumerate}
\end{definition}
We note that there may be infinitely-many edges going-out of the same region in  (see Fig.~\ref{fig:APTA_a}(b)).
\begin{proposition}
	\label{prop:finite_IARA}
	For each positive integer , one can effectively construct the part of  which contains all regions with  and all in-coming edges of these regions. 
\end{proposition}
\begin{proof}
	There are finitely-many regions obeying the constraint .
	These regions and their in-coming edges can be constructed the same way as a standard region automaton is constructed, starting with the initial location and proceeding step by step according to the immediate time-successor regions (which include the clock ) and according to the transitions of .
	Indeed, the additional clock  is only responsible for a finer partition of regions, but its introduction does not affect the transition guards of .  
	Note also that since the clock  never resets, there are no edges connecting regions with  to regions with .
	Hence, the number of edges of  restricted to  is finite.   
\end{proof}

The benefit of introducing the clock  into the region automaton is that we can know approximately at what absolute time an action occurs. 
For example, suppose that  has a single clock  and that  is reset on a transition from location  to location .
Then, in the corresponding region automaton the information about the time spent at location  before moving to  is lost.
In , however, if we take the absolute time at which an action occurs to be  when entering a region whose time-region (the value of ) is the open interval , and the absolute time  when entering a region whose time-region is exactly , then it is possible to construct from it an (infinite) approximate timed automaton with a single clock and which differs from  by at most  time units at each action.

The timestamp of the TA , denoted , is the union of the timestamps of all observable transitions of , that is, the set of all pairs , such that an observable transition with action  occurs at time  in some run of .
We define also the timestamp of .
\begin{definition}[Timestamp of ]
	The timestamp of , denoted , is the union of sets , where  is a time-region of  (an integral point  or an open unit interval ) that is part of a region of a vertex of  and  is a label of an edge of  that is directed towards this vertex.
\end{definition}

\begin{proposition}
	\label{prop:eq_timestamp}
	.
\end{proposition}
\begin{proof}
	By definition of the infinite augmented region automaton , its regions are exactly the clock-regions which are visited by runs of the TA  extended with the clock .
	In particular, the time-regions of  are the time-regions that are visited by the runs on the extended TA.
	Thus, .
	By Proposition~\ref{pr:mult_ev_timestamp}, this is an equality since for each open interval  representing absolute time that is visited in some run of  on an action , the set of all runs of  cover all the points of this interval with the same action .
\end{proof}
\subsection{Augmented Region Automaton}
A second construction is the augmented region automaton, denoted , in which we consider only the fractional part of  and ignore its integral part.
 is a finite folding of , obtained by identifying vertices that contain the same data except for the integral part of , and the corresponding edges. Thus,  has only two time-regions:  and .
As a compensation, we assign weights to the edges of , as explained below.


\begin{definition}[Augmented region automaton]
	\label{def:aug_region_automaton}
	Given a non-deterministic timed automaton with silent transitions , extended with the absolute-time clock , a corresponding (finite) \emph{augmented region automaton}  is a tuple , where:
	\begin{enumerate}
		\item  is the set of vertices.
		Each vertex is a triple ,	where  is a location of  and the pair  is a region, with
		
		containing the integral parts of the clocks , and 
		is the simplex defined by the fractional parts of the clocks .
		\item  is the initial vertex.
		\item  is the set of edges.
		There is an edge  labeled with action  if and only if there is a run of   which contains a timed transition followed by a discrete transition of the form , such that, when ignoring the integral part of the time measured by , the clock valuation  represents a point in the region  and the clock valuation  represents a point in the region .
		
		\item  is the finite set of actions.
		\item  is the set of weights on the edges.
		Each weight , possibly marked with , is , where  is the integral part of the value of  in the target location and  - in the source location in the corresponding run of .
	\end{enumerate}
\end{definition}
	There may be more than one edge between two vertices of , each one with a distinguished weight.
	A marked weight  represents infinitely-many consecutive values  as weights between the same two vertices, with  being the minimal value of such a sequence.
	It refers to a transition to or from a region  in which all regular clocks have passed the maximal integer  appearing in a transition guard.


\begin{example}
	In Fig.~\ref{fig:APTA_a}(a) we see a very simple TA  containing a transition to an unbounded region.
	The corresponding infinite augmented region automaton  is shown in Fig.~\ref{fig:APTA_a}(b).
	Each vertex of  is represented by a rounded rectangle containing the original location of  (circled, on the left), the integral values of  and of  (in the top of the rectangle) and the simplex (in the bottom).
	Notice that when the value of  is greater than  it is marked by  and its fractional part is ignored.
	To the left of  we see the discretization of time  into time-regions, and each vertex of  is drawn in the level of its time-region.
	In  Fig.~\ref{fig:APTA_a}(c) the augmented region automaton  is shown.
	Here the integral part of the value of  is ignored.
	The edge labeled by  represents the infinitely-many differences in the integral parts of the values of : .
	Similarly, the edge labeled with  refers to the differences .
	\begin{figure}[t]
\centering
		\scalebox{0.45}{ \begin{picture}(0,0)\includegraphics{APTA_a.pdf}\end{picture}\setlength{\unitlength}{3947sp}\begingroup\makeatletter\ifx\SetFigFont\undefined \gdef\SetFigFont#1#2#3#4#5{\reset@font\fontsize{#1}{#2pt}\fontfamily{#3}\fontseries{#4}\fontshape{#5}\selectfont}\fi\endgroup \begin{picture}(13077,9300)(1261,-4366)
\put(7876,3764){\makebox(0,0)[lb]{\smash{{\SetFigFont{17}{20.4}{\rmdefault}{\mddefault}{\updefault}{\color[rgb]{0,0,0}}}}}}
\put(10051,-1111){\makebox(0,0)[lb]{\smash{{\SetFigFont{14}{16.8}{\rmdefault}{\mddefault}{\updefault}{\color[rgb]{0,0,0}}}}}}
\put(9676,-961){\makebox(0,0)[lb]{\smash{{\SetFigFont{14}{16.8}{\rmdefault}{\mddefault}{\updefault}{\color[rgb]{0,0,0}}}}}}
\put(9676,-2161){\makebox(0,0)[lb]{\smash{{\SetFigFont{14}{16.8}{\rmdefault}{\mddefault}{\updefault}{\color[rgb]{0,0,0}}}}}}
\put(10351,-1561){\makebox(0,0)[lb]{\smash{{\SetFigFont{14}{16.8}{\rmdefault}{\mddefault}{\updefault}{\color[rgb]{0,0,0}}}}}}
\put(10651,-1561){\makebox(0,0)[lb]{\smash{{\SetFigFont{14}{16.8}{\rmdefault}{\mddefault}{\updefault}{\color[rgb]{0,0,0}}}}}}
\put(10426,-811){\makebox(0,0)[lb]{\smash{{\SetFigFont{14}{16.8}{\rmdefault}{\mddefault}{\updefault}{\color[rgb]{0,0,0}}}}}}
\put(10351,-2311){\makebox(0,0)[lb]{\smash{{\SetFigFont{14}{16.8}{\rmdefault}{\mddefault}{\updefault}{\color[rgb]{0,0,0}}}}}}
\put(11701,-4261){\makebox(0,0)[lb]{\smash{{\SetFigFont{17}{20.4}{\rmdefault}{\mddefault}{\updefault}{\color[rgb]{0,0,0}}}}}}
\put(11701,-2161){\makebox(0,0)[lb]{\smash{{\SetFigFont{14}{16.8}{\rmdefault}{\mddefault}{\updefault}{\color[rgb]{0,0,0}}}}}}
\put(9676,-3361){\makebox(0,0)[lb]{\smash{{\SetFigFont{14}{16.8}{\rmdefault}{\mddefault}{\updefault}{\color[rgb]{0,0,0}}}}}}
\put(11851,-811){\makebox(0,0)[lb]{\smash{{\SetFigFont{14}{16.8}{\rmdefault}{\mddefault}{\updefault}{\color[rgb]{0,0,0}}}}}}
\put(12751,-1111){\makebox(0,0)[lb]{\smash{{\SetFigFont{14}{16.8}{\rmdefault}{\mddefault}{\updefault}{\color[rgb]{0,0,0}}}}}}
\put(12376,-961){\makebox(0,0)[lb]{\smash{{\SetFigFont{14}{16.8}{\rmdefault}{\mddefault}{\updefault}{\color[rgb]{0,0,0}}}}}}
\put(13126,-811){\makebox(0,0)[lb]{\smash{{\SetFigFont{14}{16.8}{\rmdefault}{\mddefault}{\updefault}{\color[rgb]{0,0,0}}}}}}
\put(12001,-2161){\makebox(0,0)[lb]{\smash{{\SetFigFont{14}{16.8}{\rmdefault}{\mddefault}{\updefault}{\color[rgb]{0,0,0}}}}}}
\put(10351,-3511){\makebox(0,0)[lb]{\smash{{\SetFigFont{14}{16.8}{\rmdefault}{\mddefault}{\updefault}{\color[rgb]{0,0,0}}}}}}
\put(10201,-2011){\makebox(0,0)[lb]{\smash{{\SetFigFont{14}{16.8}{\rmdefault}{\mddefault}{\updefault}{\color[rgb]{0,0,0}}}}}}
\put(10201,-3211){\makebox(0,0)[lb]{\smash{{\SetFigFont{14}{16.8}{\rmdefault}{\mddefault}{\updefault}{\color[rgb]{0,0,0}}}}}}
\put(8326,-2161){\makebox(0,0)[lb]{\smash{{\SetFigFont{14}{16.8}{\rmdefault}{\mddefault}{\updefault}{\color[rgb]{0,0,0}}}}}}
\put(8476,-3361){\makebox(0,0)[lb]{\smash{{\SetFigFont{14}{16.8}{\rmdefault}{\mddefault}{\updefault}{\color[rgb]{0,0,0}}}}}}
\put(8476,-961){\makebox(0,0)[lb]{\smash{{\SetFigFont{14}{16.8}{\rmdefault}{\mddefault}{\updefault}{\color[rgb]{0,0,0}}}}}}
\put(8476,-286){\makebox(0,0)[lb]{\smash{{\SetFigFont{20}{24.0}{\rmdefault}{\mddefault}{\updefault}{\color[rgb]{0,0,0}}}}}}
\put(10051,3014){\makebox(0,0)[lb]{\smash{{\SetFigFont{14}{16.8}{\rmdefault}{\mddefault}{\updefault}{\color[rgb]{0,0,0}}}}}}
\put(9676,3164){\makebox(0,0)[lb]{\smash{{\SetFigFont{14}{16.8}{\rmdefault}{\mddefault}{\updefault}{\color[rgb]{0,0,0}}}}}}
\put(9676,1964){\makebox(0,0)[lb]{\smash{{\SetFigFont{14}{16.8}{\rmdefault}{\mddefault}{\updefault}{\color[rgb]{0,0,0}}}}}}
\put(10351,2564){\makebox(0,0)[lb]{\smash{{\SetFigFont{14}{16.8}{\rmdefault}{\mddefault}{\updefault}{\color[rgb]{0,0,0}}}}}}
\put(10651,2564){\makebox(0,0)[lb]{\smash{{\SetFigFont{14}{16.8}{\rmdefault}{\mddefault}{\updefault}{\color[rgb]{0,0,0}}}}}}
\put(10426,3314){\makebox(0,0)[lb]{\smash{{\SetFigFont{14}{16.8}{\rmdefault}{\mddefault}{\updefault}{\color[rgb]{0,0,0}}}}}}
\put(10351,1814){\makebox(0,0)[lb]{\smash{{\SetFigFont{14}{16.8}{\rmdefault}{\mddefault}{\updefault}{\color[rgb]{0,0,0}}}}}}
\put(11851,3314){\makebox(0,0)[lb]{\smash{{\SetFigFont{14}{16.8}{\rmdefault}{\mddefault}{\updefault}{\color[rgb]{0,0,0}}}}}}
\put(12751,3014){\makebox(0,0)[lb]{\smash{{\SetFigFont{14}{16.8}{\rmdefault}{\mddefault}{\updefault}{\color[rgb]{0,0,0}}}}}}
\put(12376,3164){\makebox(0,0)[lb]{\smash{{\SetFigFont{14}{16.8}{\rmdefault}{\mddefault}{\updefault}{\color[rgb]{0,0,0}}}}}}
\put(13126,3314){\makebox(0,0)[lb]{\smash{{\SetFigFont{14}{16.8}{\rmdefault}{\mddefault}{\updefault}{\color[rgb]{0,0,0}}}}}}
\put(10201,2114){\makebox(0,0)[lb]{\smash{{\SetFigFont{14}{16.8}{\rmdefault}{\mddefault}{\updefault}{\color[rgb]{0,0,0}}}}}}
\put(12376,1964){\makebox(0,0)[lb]{\smash{{\SetFigFont{14}{16.8}{\rmdefault}{\mddefault}{\updefault}{\color[rgb]{0,0,0}}}}}}
\put(13051,1814){\makebox(0,0)[lb]{\smash{{\SetFigFont{14}{16.8}{\rmdefault}{\mddefault}{\updefault}{\color[rgb]{0,0,0}}}}}}
\put(12901,2114){\makebox(0,0)[lb]{\smash{{\SetFigFont{14}{16.8}{\rmdefault}{\mddefault}{\updefault}{\color[rgb]{0,0,0}}}}}}
\put(11626,1064){\makebox(0,0)[lb]{\smash{{\SetFigFont{17}{20.4}{\rmdefault}{\mddefault}{\updefault}{\color[rgb]{0,0,0}}}}}}
\put(11851,3014){\makebox(0,0)[lb]{\smash{{\SetFigFont{14}{16.8}{\rmdefault}{\mddefault}{\updefault}{\color[rgb]{0,0,0}}}}}}
\put(12001,2489){\makebox(0,0)[lb]{\smash{{\SetFigFont{14}{16.8}{\rmdefault}{\mddefault}{\updefault}{\color[rgb]{0,0,0}}}}}}
\put(12301,2639){\makebox(0,0)[lb]{\smash{{\SetFigFont{14}{16.8}{\rmdefault}{\mddefault}{\updefault}{\color[rgb]{0,0,0}}}}}}
\put(4051,2564){\makebox(0,0)[lb]{\smash{{\SetFigFont{14}{16.8}{\rmdefault}{\mddefault}{\updefault}{\color[rgb]{0,0,0}}}}}}
\put(4126,1364){\makebox(0,0)[lb]{\smash{{\SetFigFont{14}{16.8}{\rmdefault}{\mddefault}{\updefault}{\color[rgb]{0,0,0}}}}}}
\put(4126,164){\makebox(0,0)[lb]{\smash{{\SetFigFont{14}{16.8}{\rmdefault}{\mddefault}{\updefault}{\color[rgb]{0,0,0}}}}}}
\put(4126,-1036){\makebox(0,0)[lb]{\smash{{\SetFigFont{14}{16.8}{\rmdefault}{\mddefault}{\updefault}{\color[rgb]{0,0,0}}}}}}
\put(4126,-2236){\makebox(0,0)[lb]{\smash{{\SetFigFont{14}{16.8}{\rmdefault}{\mddefault}{\updefault}{\color[rgb]{0,0,0}}}}}}
\put(4126,-3436){\makebox(0,0)[lb]{\smash{{\SetFigFont{14}{16.8}{\rmdefault}{\mddefault}{\updefault}{\color[rgb]{0,0,0}}}}}}
\put(2176,-2836){\makebox(0,0)[lb]{\smash{{\SetFigFont{14}{16.8}{\rmdefault}{\mddefault}{\updefault}{\color[rgb]{0,0,0}}}}}}
\put(2176,-1636){\makebox(0,0)[lb]{\smash{{\SetFigFont{14}{16.8}{\rmdefault}{\mddefault}{\updefault}{\color[rgb]{0,0,0}}}}}}
\put(2176,764){\makebox(0,0)[lb]{\smash{{\SetFigFont{14}{16.8}{\rmdefault}{\mddefault}{\updefault}{\color[rgb]{0,0,0}}}}}}
\put(2176,-436){\makebox(0,0)[lb]{\smash{{\SetFigFont{14}{16.8}{\rmdefault}{\mddefault}{\updefault}{\color[rgb]{0,0,0}}}}}}
\put(2926,-2686){\makebox(0,0)[lb]{\smash{{\SetFigFont{14}{16.8}{\rmdefault}{\mddefault}{\updefault}{\color[rgb]{0,0,0}}}}}}
\put(2926,-1486){\makebox(0,0)[lb]{\smash{{\SetFigFont{14}{16.8}{\rmdefault}{\mddefault}{\updefault}{\color[rgb]{0,0,0}}}}}}
\put(2926,-286){\makebox(0,0)[lb]{\smash{{\SetFigFont{14}{16.8}{\rmdefault}{\mddefault}{\updefault}{\color[rgb]{0,0,0}}}}}}
\put(2926,914){\makebox(0,0)[lb]{\smash{{\SetFigFont{14}{16.8}{\rmdefault}{\mddefault}{\updefault}{\color[rgb]{0,0,0}}}}}}
\put(2176,3164){\makebox(0,0)[lb]{\smash{{\SetFigFont{14}{16.8}{\rmdefault}{\mddefault}{\updefault}{\color[rgb]{0,0,0}}}}}}
\put(2851,-1786){\makebox(0,0)[lb]{\smash{{\SetFigFont{14}{16.8}{\rmdefault}{\mddefault}{\updefault}{\color[rgb]{0,0,0}}}}}}
\put(2851,614){\makebox(0,0)[lb]{\smash{{\SetFigFont{14}{16.8}{\rmdefault}{\mddefault}{\updefault}{\color[rgb]{0,0,0}}}}}}
\put(1276,1964){\makebox(0,0)[lb]{\smash{{\SetFigFont{14}{16.8}{\rmdefault}{\mddefault}{\updefault}{\color[rgb]{0,0,0}}}}}}
\put(1276,-436){\makebox(0,0)[lb]{\smash{{\SetFigFont{14}{16.8}{\rmdefault}{\mddefault}{\updefault}{\color[rgb]{0,0,0}}}}}}
\put(1276,-2836){\makebox(0,0)[lb]{\smash{{\SetFigFont{14}{16.8}{\rmdefault}{\mddefault}{\updefault}{\color[rgb]{0,0,0}}}}}}
\put(2176,1964){\makebox(0,0)[lb]{\smash{{\SetFigFont{14}{16.8}{\rmdefault}{\mddefault}{\updefault}{\color[rgb]{0,0,0}}}}}}
\put(4351,3314){\makebox(0,0)[lb]{\smash{{\SetFigFont{14}{16.8}{\rmdefault}{\mddefault}{\updefault}{\color[rgb]{0,0,0}}}}}}
\put(5626,3314){\makebox(0,0)[lb]{\smash{{\SetFigFont{14}{16.8}{\rmdefault}{\mddefault}{\updefault}{\color[rgb]{0,0,0}}}}}}
\put(5251,3014){\makebox(0,0)[lb]{\smash{{\SetFigFont{14}{16.8}{\rmdefault}{\mddefault}{\updefault}{\color[rgb]{0,0,0}}}}}}
\put(4876,3164){\makebox(0,0)[lb]{\smash{{\SetFigFont{14}{16.8}{\rmdefault}{\mddefault}{\updefault}{\color[rgb]{0,0,0}}}}}}
\put(1426,764){\makebox(0,0)[lb]{\smash{{\SetFigFont{14}{16.8}{\rmdefault}{\mddefault}{\updefault}{\color[rgb]{0,0,0}}}}}}
\put(1426,-1636){\makebox(0,0)[lb]{\smash{{\SetFigFont{14}{16.8}{\rmdefault}{\mddefault}{\updefault}{\color[rgb]{0,0,0}}}}}}
\put(2926,2114){\makebox(0,0)[lb]{\smash{{\SetFigFont{14}{16.8}{\rmdefault}{\mddefault}{\updefault}{\color[rgb]{0,0,0}}}}}}
\put(2851,1814){\makebox(0,0)[lb]{\smash{{\SetFigFont{14}{16.8}{\rmdefault}{\mddefault}{\updefault}{\color[rgb]{0,0,0}}}}}}
\put(2926,3314){\makebox(0,0)[lb]{\smash{{\SetFigFont{14}{16.8}{\rmdefault}{\mddefault}{\updefault}{\color[rgb]{0,0,0}}}}}}
\put(2551,3014){\makebox(0,0)[lb]{\smash{{\SetFigFont{14}{16.8}{\rmdefault}{\mddefault}{\updefault}{\color[rgb]{0,0,0}}}}}}
\put(2851,-586){\makebox(0,0)[lb]{\smash{{\SetFigFont{14}{16.8}{\rmdefault}{\mddefault}{\updefault}{\color[rgb]{0,0,0}}}}}}
\put(2851,-2986){\makebox(0,0)[lb]{\smash{{\SetFigFont{14}{16.8}{\rmdefault}{\mddefault}{\updefault}{\color[rgb]{0,0,0}}}}}}
\put(1426,3164){\makebox(0,0)[lb]{\smash{{\SetFigFont{14}{16.8}{\rmdefault}{\mddefault}{\updefault}{\color[rgb]{0,0,0}}}}}}
\put(1426,3839){\makebox(0,0)[lb]{\smash{{\SetFigFont{20}{24.0}{\rmdefault}{\mddefault}{\updefault}{\color[rgb]{0,0,0}}}}}}
\put(2926,-4261){\makebox(0,0)[lb]{\smash{{\SetFigFont{17}{20.4}{\rmdefault}{\mddefault}{\updefault}{\color[rgb]{0,0,0}}}}}}
\put(8926,4289){\makebox(0,0)[lb]{\smash{{\SetFigFont{14}{16.8}{\rmdefault}{\mddefault}{\updefault}{\color[rgb]{0,0,0}}}}}}
\put(7276,4289){\makebox(0,0)[lb]{\smash{{\SetFigFont{14}{16.8}{\rmdefault}{\mddefault}{\updefault}{\color[rgb]{0,0,0}}}}}}
\put(7801,4514){\makebox(0,0)[lb]{\smash{{\SetFigFont{14}{16.8}{\rmdefault}{\mddefault}{\updefault}{\color[rgb]{0,0,0}}}}}}
\put(8026,4739){\makebox(0,0)[lb]{\smash{{\SetFigFont{14}{16.8}{\rmdefault}{\mddefault}{\updefault}{\color[rgb]{0,0,0}}}}}}
\end{picture} }
		\caption{ ;  The infinite augmented region automaton ;   The augmented region automaton ;  A periodic augmented region automaton . Each rectangle represents a vertex containing the location of  (circled, left), the integral values of  and  (top) and the simplex (bottom).}
		\label{fig:APTA_a}
\end{figure}
\end{example}

	The languages  of  and  of   consist of all observable timed traces but, in contrast to the language  of , in each pair  the time  is not exact: it is either an exact integer  or an arbitrary value of an interval  that satisfies .
	Thus,  and   are less abstract than the untimed language  of the region automaton  but are more abstract than : one cannot, in general, distinguish between a transition that occurs without any time delay, e.g. when , and a transition that demands a time delay, e.g. when .
	When comparing  and  then, since  may be obtained from , it is clear that  cannot be less abstract than .	  
But, in fact, these region automata are equally informative: for each positive integer , one can effectively construct  up to time , as in Proposition~\ref{prop:finite_IARA}, by unfolding  and recovering absolute time  by summing up the weights of the edges along the taken paths.
	Indeed, since the transitions in  do not rely on , by taking the quotient of  by 'forgetting' the integral part of , the only loss of information is the time difference in  between the target and source regions, but then this information is regained in the form of weight on the corresponding edge of .
	Thus, we have the following.
\begin{proposition}
	\label{prop:eq_info}
	 .
\end{proposition}

As with , we can construct from  an approximate automaton, this time a finite and deterministic one, which approximates  with a maximal error of  time units at each observed transition.
This automaton has only one clock and this clock resets at every transition.
The maximal error  could be further reduced to  by allowing transitions in the approximate automaton to occur at times ,  and only on such times.

\section{Eventual Periodicity}
\label{sec:per}
In this section we address the main topic of this paper: exploring the time-periodic property of .
In addition to demonstrating its existence, we show how one can actually compute the parameters of a period.
\subsection{Non-Zeno Cycles in }
 is in the form of a finite connected directed graph with an initial vertex.
Every edge of  corresponds to a feasible transition in  (contained in a run of ).
In what follows, a `path' in  is a directed path that starts at the initial vertex , unless otherwise stated.
\begin{definition}[Duration of a path]
Given a path  in , its minimal integral \emph{duration}, or simply duration,  is the sum of the weights on its edges, where a weight  is counted as .
\end{definition}
\begin{definition}[(Non)-Zeno cycle]
A  cycle of  of duration  is called a \emph{Zeno cycle} .
Otherwise, it is a \emph{non-Zeno cycle}.
\end{definition}
A path is called \emph{simple} if no vertex of it repeats itself, and we let  be the maximal duration of a simple path in .
\begin{lemma}
\label{lem:reg_time}
There exists a minimal positive integer , the non-Zeno threshold time, such that every path  of  that is of (minimal) duration  or more contains a vertex belonging to some non-Zeno cycle.
\end{lemma}
\begin{proof}
Indeed, if  does not contain non-Zeno cycles then we can take  and the claim holds vacuously.
So, suppose that  contains non-Zeno Cycles.
Then each path of duration  must contain non-Zeno cycles because otherwise the Zeno cycles could have been removed, without changing the duration of the path, resulting in a simple path of duration  - a contradiction.
\end{proof}
In order to compute  we can explore the simple paths of , say in a breadth-first manner, up to the time  in which each such path either cannot be extended to a path of a larger duration or any extension of it hits a vertex belonging to some non-Zeno cycle.
Then , which may be much smaller than .
\subsection{A Period of }
A set  is \emph{minimal} with respect to some property if for every element  the set  does not satisfy the property.
\begin{definition}[Covering set of non-Zeno cycles]
A set  of non-Zeno cycles of  is called a \emph{covering set of non-Zeno cycles} if every path  of  whose duration  is at least  intersects a cycle in  in a common vertex.
\end{definition}
Without loss of generality, we may assume that a covering set of non-Zeno cycles is minimal.
\begin{definition}[Period of ]
A time \emph{period} (or just period)  of  is a common multiple of the set of durations , , for
some fixed (minimal) covering set of non-Zeno cycles .
For convenience, we also set  to be greater than , unless  does not contain non-Zeno cycles, in which case we define  to be 0.
\end{definition}

We remark that if we want to compute a minimal period  we need to conduct a thorough exploration of the duration of cycles in , taking into account their common factors, but this computation is not needed for the results presented here.

\subsection{Eventual Periodicity of }
Let  be as above, with  fixed.
We denote by  the subgraph of  that starts at time-level , that is, the set of vertices of  with absolute time  and their out-going edges.
\begin{definition}[-shift in time]
	Given a subgraph  of , an \emph{-shift in time} of , denoted , is the graph obtained by adding the value  to each value of the integral part of the clock  in  and leaving the rest of the data unaltered.
	We also denote by  the -shift in time for the set of vertices of , with  in case .
\end{definition}
\begin{lemma}
	\label{lem:forward_period}
	If  is not bounded in time then 
	
\end{lemma}
\begin{proof}
	First we show that the inclusion holds for the set of vertices of the above subgraphs.
	Let  be a path of  which terminates in a vertex .
	Let  be the image of  under the projection to .
	If  contains an edge  whose image  is labeled by a marked weight  then we can replace  by another edge  whose delay is greater by  than the delay of .
	So, suppose that  starts in the vertex  and terminates in . Then  starts in  and terminates in the vertex  and then the path continues as in  but with an -shift in time, terminating in the vertex .
Otherwise, no edge of  has a marked weight.
	Since  then by Lemma~\ref{lem:reg_time} and the definition of ,  contains a vertex  that belongs to a non-Zeno cycle  and whose duration is a factor of .
	Hence, by a 'pumping' argument, we can extend  with  cycles of  that start and end in  and then reach the vertex  in the pre-image in  of this extended path.
	
	The inclusion of the out-going edges follows from the fact that the out-going edges do not depend on the value of .
\end{proof}
Let us denote by , , the set of vertices

\begin{theorem}
\label{th:eventual_period}
If the infinite augmented region automaton  is not bounded in time then it is eventually periodic: there exists an integral time  such that 

\end{theorem}
\begin{proof}
By Lemma~\ref{lem:forward_period}, , for .
But there is a bound on the number of possible vertices of  since  is bounded, hence the sequence  eventually stabilizes.
The result then follows since for the out-going edges the same argument given in the proof of Lemma~\ref{lem:forward_period} holds also here.
\end{proof}
When  is finite then we can set  to be , where  is the maximal integral time of .
By the following proposition, a possible value for  can be effectively computed when  is infinite.
\begin{proposition}
\label{prop:cons_eq}
if  for some  then we can set
.
\end{proposition}
\begin{proof}
The equalities  are equivalent to  and . 
By induction, it suffices to show that these equalities imply that , that is, .
Let .
We need to show that there exists  such that .

Suppose that  is reached by an edge from a vertex .
Since , there exists a vertex  and this vertex is connected to a vertex  .

Otherwise,  is reached by an edge  from a vertex  in  or earlier, and the time difference  between  and  is greater than .
This implies that the projection  is of unbounded time delay .
Since  then .
Hence, there is another edge  in , that is also a pre-image of , and which joins  to a vertex , where .
\end{proof}
\begin{example}
	This example refers to the TA of Fig.~\ref{fig:complex2} (a).
	In order to make the analysis of its time-periodic structure simpler, we changed the guard on the transition from location  to location  to be simpler (Fig.~\ref{fig:complex} (a)), so that in the resulting infinite augmented region automaton  (Fig.~\ref{fig:complex} (b)) we can clearly see two different cycles of period  (circled in dotted lines) (the edges with label  are only partly shown).
	We then added the original guard between locations  and  (Fig.~\ref{fig:complex2} (a)).
	In the additional part in  (Fig.~\ref{fig:complex2} (b)) we see two more cycles, one of period  and one of period .
	We can still use a period of length  for this more complex automaton, but the existence of cycles of other lengths results in a longer time until reaching the repeated periodic part of the entire automaton.
\begin{figure}[t]
\centering
		\scalebox{0.4}{ \begin{picture}(0,0)\includegraphics{complex.pdf}\end{picture}\setlength{\unitlength}{3947sp}\begingroup\makeatletter\ifx\SetFigFont\undefined \gdef\SetFigFont#1#2#3#4#5{\reset@font\fontsize{#1}{#2pt}\fontfamily{#3}\fontseries{#4}\fontshape{#5}\selectfont}\fi\endgroup \begin{picture}(12224,19133)(4486,-14451)
\put(13201,-7486){\makebox(0,0)[lb]{\smash{{\SetFigFont{14}{16.8}{\rmdefault}{\mddefault}{\updefault}{\color[rgb]{0,0,0}}}}}}
\put(4576,2564){\makebox(0,0)[lb]{\smash{{\SetFigFont{14}{16.8}{\rmdefault}{\mddefault}{\updefault}{\color[rgb]{0,0,0}}}}}}
\put(4726,1364){\makebox(0,0)[lb]{\smash{{\SetFigFont{14}{16.8}{\rmdefault}{\mddefault}{\updefault}{\color[rgb]{0,0,0}}}}}}
\put(4726,3764){\makebox(0,0)[lb]{\smash{{\SetFigFont{14}{16.8}{\rmdefault}{\mddefault}{\updefault}{\color[rgb]{0,0,0}}}}}}
\put(4576,164){\makebox(0,0)[lb]{\smash{{\SetFigFont{14}{16.8}{\rmdefault}{\mddefault}{\updefault}{\color[rgb]{0,0,0}}}}}}
\put(4726,-1036){\makebox(0,0)[lb]{\smash{{\SetFigFont{14}{16.8}{\rmdefault}{\mddefault}{\updefault}{\color[rgb]{0,0,0}}}}}}
\put(4726,-3436){\makebox(0,0)[lb]{\smash{{\SetFigFont{14}{16.8}{\rmdefault}{\mddefault}{\updefault}{\color[rgb]{0,0,0}}}}}}
\put(4726,4439){\makebox(0,0)[lb]{\smash{{\SetFigFont{20}{24.0}{\rmdefault}{\mddefault}{\updefault}{\color[rgb]{0,0,0}}}}}}
\put(4576,-4636){\makebox(0,0)[lb]{\smash{{\SetFigFont{14}{16.8}{\rmdefault}{\mddefault}{\updefault}{\color[rgb]{0,0,0}}}}}}
\put(4576,-7036){\makebox(0,0)[lb]{\smash{{\SetFigFont{14}{16.8}{\rmdefault}{\mddefault}{\updefault}{\color[rgb]{0,0,0}}}}}}
\put(4726,-5836){\makebox(0,0)[lb]{\smash{{\SetFigFont{14}{16.8}{\rmdefault}{\mddefault}{\updefault}{\color[rgb]{0,0,0}}}}}}
\put(4576,-9436){\makebox(0,0)[lb]{\smash{{\SetFigFont{14}{16.8}{\rmdefault}{\mddefault}{\updefault}{\color[rgb]{0,0,0}}}}}}
\put(4726,-8236){\makebox(0,0)[lb]{\smash{{\SetFigFont{14}{16.8}{\rmdefault}{\mddefault}{\updefault}{\color[rgb]{0,0,0}}}}}}
\put(4801,-2236){\makebox(0,0)[lb]{\smash{{\SetFigFont{14}{16.8}{\rmdefault}{\mddefault}{\updefault}{\color[rgb]{0,0,0}}}}}}
\put(4801,-5236){\makebox(0,0)[lb]{\smash{{\SetFigFont{14}{16.8}{\rmdefault}{\mddefault}{\updefault}{\color[rgb]{0,0,0}}}}}}
\put(4576,-10636){\makebox(0,0)[lb]{\smash{{\SetFigFont{14}{16.8}{\rmdefault}{\mddefault}{\updefault}{\color[rgb]{0,0,0}}}}}}
\put(4501,-11836){\makebox(0,0)[lb]{\smash{{\SetFigFont{14}{16.8}{\rmdefault}{\mddefault}{\updefault}{\color[rgb]{0,0,0}}}}}}
\put(4501,-13036){\makebox(0,0)[lb]{\smash{{\SetFigFont{14}{16.8}{\rmdefault}{\mddefault}{\updefault}{\color[rgb]{0,0,0}}}}}}
\put(4801,-10036){\makebox(0,0)[lb]{\smash{{\SetFigFont{14}{16.8}{\rmdefault}{\mddefault}{\updefault}{\color[rgb]{0,0,0}}}}}}
\put(4801,-11236){\makebox(0,0)[lb]{\smash{{\SetFigFont{14}{16.8}{\rmdefault}{\mddefault}{\updefault}{\color[rgb]{0,0,0}}}}}}
\put(4801,-12436){\makebox(0,0)[lb]{\smash{{\SetFigFont{14}{16.8}{\rmdefault}{\mddefault}{\updefault}{\color[rgb]{0,0,0}}}}}}
\put(15751,-6886){\makebox(0,0)[lb]{\smash{{\SetFigFont{14}{16.8}{\rmdefault}{\mddefault}{\updefault}{\color[rgb]{0,0,0}}}}}}
\put(15676,-7186){\makebox(0,0)[lb]{\smash{{\SetFigFont{14}{16.8}{\rmdefault}{\mddefault}{\updefault}{\color[rgb]{0,0,0}}}}}}
\put(16051,-7186){\makebox(0,0)[lb]{\smash{{\SetFigFont{14}{16.8}{\rmdefault}{\mddefault}{\updefault}{\color[rgb]{0,0,0}}}}}}
\put(16426,-6286){\makebox(0,0)[lb]{\smash{{\SetFigFont{14}{16.8}{\rmdefault}{\mddefault}{\updefault}{\color[rgb]{0,0,0}}}}}}
\put(12601,-2236){\makebox(0,0)[lb]{\smash{{\SetFigFont{14}{16.8}{\rmdefault}{\mddefault}{\updefault}{\color[rgb]{0,0,0}}}}}}
\put(13951,-886){\makebox(0,0)[lb]{\smash{{\SetFigFont{14}{16.8}{\rmdefault}{\mddefault}{\updefault}{\color[rgb]{0,0,0}}}}}}
\put(13876,-1186){\makebox(0,0)[lb]{\smash{{\SetFigFont{14}{16.8}{\rmdefault}{\mddefault}{\updefault}{\color[rgb]{0,0,0}}}}}}
\put(14251,-1186){\makebox(0,0)[lb]{\smash{{\SetFigFont{14}{16.8}{\rmdefault}{\mddefault}{\updefault}{\color[rgb]{0,0,0}}}}}}
\put(12181,-886){\makebox(0,0)[lb]{\smash{{\SetFigFont{14}{16.8}{\rmdefault}{\mddefault}{\updefault}{\color[rgb]{0,0,0}}}}}}
\put(12481,-1186){\makebox(0,0)[lb]{\smash{{\SetFigFont{14}{16.8}{\rmdefault}{\mddefault}{\updefault}{\color[rgb]{0,0,0}}}}}}
\put(12106,-1186){\makebox(0,0)[lb]{\smash{{\SetFigFont{14}{16.8}{\rmdefault}{\mddefault}{\updefault}{\color[rgb]{0,0,0}}}}}}
\put(10351,3914){\makebox(0,0)[lb]{\smash{{\SetFigFont{14}{16.8}{\rmdefault}{\mddefault}{\updefault}{\color[rgb]{0,0,0}}}}}}
\put(10276,3614){\makebox(0,0)[lb]{\smash{{\SetFigFont{14}{16.8}{\rmdefault}{\mddefault}{\updefault}{\color[rgb]{0,0,0}}}}}}
\put(10651,3614){\makebox(0,0)[lb]{\smash{{\SetFigFont{14}{16.8}{\rmdefault}{\mddefault}{\updefault}{\color[rgb]{0,0,0}}}}}}
\put(12151,3914){\makebox(0,0)[lb]{\smash{{\SetFigFont{14}{16.8}{\rmdefault}{\mddefault}{\updefault}{\color[rgb]{0,0,0}}}}}}
\put(12076,3614){\makebox(0,0)[lb]{\smash{{\SetFigFont{14}{16.8}{\rmdefault}{\mddefault}{\updefault}{\color[rgb]{0,0,0}}}}}}
\put(12451,3614){\makebox(0,0)[lb]{\smash{{\SetFigFont{14}{16.8}{\rmdefault}{\mddefault}{\updefault}{\color[rgb]{0,0,0}}}}}}
\put(10801,-2236){\makebox(0,0)[lb]{\smash{{\SetFigFont{14}{16.8}{\rmdefault}{\mddefault}{\updefault}{\color[rgb]{0,0,0}}}}}}
\put(10351,-3286){\makebox(0,0)[lb]{\smash{{\SetFigFont{14}{16.8}{\rmdefault}{\mddefault}{\updefault}{\color[rgb]{0,0,0}}}}}}
\put(10651,-3586){\makebox(0,0)[lb]{\smash{{\SetFigFont{14}{16.8}{\rmdefault}{\mddefault}{\updefault}{\color[rgb]{0,0,0}}}}}}
\put(10276,-3586){\makebox(0,0)[lb]{\smash{{\SetFigFont{14}{16.8}{\rmdefault}{\mddefault}{\updefault}{\color[rgb]{0,0,0}}}}}}
\put(10351,314){\makebox(0,0)[lb]{\smash{{\SetFigFont{14}{16.8}{\rmdefault}{\mddefault}{\updefault}{\color[rgb]{0,0,0}}}}}}
\put(10651, 14){\makebox(0,0)[lb]{\smash{{\SetFigFont{14}{16.8}{\rmdefault}{\mddefault}{\updefault}{\color[rgb]{0,0,0}}}}}}
\put(10276, 14){\makebox(0,0)[lb]{\smash{{\SetFigFont{14}{16.8}{\rmdefault}{\mddefault}{\updefault}{\color[rgb]{0,0,0}}}}}}
\put(10351,-886){\makebox(0,0)[lb]{\smash{{\SetFigFont{14}{16.8}{\rmdefault}{\mddefault}{\updefault}{\color[rgb]{0,0,0}}}}}}
\put(10651,-1186){\makebox(0,0)[lb]{\smash{{\SetFigFont{14}{16.8}{\rmdefault}{\mddefault}{\updefault}{\color[rgb]{0,0,0}}}}}}
\put(10276,-1186){\makebox(0,0)[lb]{\smash{{\SetFigFont{14}{16.8}{\rmdefault}{\mddefault}{\updefault}{\color[rgb]{0,0,0}}}}}}
\put(9826,1364){\makebox(0,0)[lb]{\smash{{\SetFigFont{14}{16.8}{\rmdefault}{\mddefault}{\updefault}{\color[rgb]{0,0,0}}}}}}
\put(12151,3164){\makebox(0,0)[lb]{\smash{{\SetFigFont{14}{16.8}{\rmdefault}{\mddefault}{\updefault}{\color[rgb]{0,0,0}}}}}}
\put(12451,2414){\makebox(0,0)[lb]{\smash{{\SetFigFont{14}{16.8}{\rmdefault}{\mddefault}{\updefault}{\color[rgb]{0,0,0}}}}}}
\put(12151,2714){\makebox(0,0)[lb]{\smash{{\SetFigFont{14}{16.8}{\rmdefault}{\mddefault}{\updefault}{\color[rgb]{0,0,0}}}}}}
\put(12076,2414){\makebox(0,0)[lb]{\smash{{\SetFigFont{14}{16.8}{\rmdefault}{\mddefault}{\updefault}{\color[rgb]{0,0,0}}}}}}
\put(12751,3314){\makebox(0,0)[lb]{\smash{{\SetFigFont{14}{16.8}{\rmdefault}{\mddefault}{\updefault}{\color[rgb]{0,0,0}}}}}}
\put(11551,3914){\makebox(0,0)[lb]{\smash{{\SetFigFont{14}{16.8}{\rmdefault}{\mddefault}{\updefault}{\color[rgb]{0,0,0}}}}}}
\put(10801,1964){\makebox(0,0)[lb]{\smash{{\SetFigFont{14}{16.8}{\rmdefault}{\mddefault}{\updefault}{\color[rgb]{0,0,0}}}}}}
\put(11851,1364){\makebox(0,0)[lb]{\smash{{\SetFigFont{14}{16.8}{\rmdefault}{\mddefault}{\updefault}{\color[rgb]{0,0,0}}}}}}
\put(12601,764){\makebox(0,0)[lb]{\smash{{\SetFigFont{14}{16.8}{\rmdefault}{\mddefault}{\updefault}{\color[rgb]{0,0,0}}}}}}
\put(13801,1364){\makebox(0,0)[lb]{\smash{{\SetFigFont{14}{16.8}{\rmdefault}{\mddefault}{\updefault}{\color[rgb]{0,0,0}}}}}}
\put(14551,-286){\makebox(0,0)[lb]{\smash{{\SetFigFont{14}{16.8}{\rmdefault}{\mddefault}{\updefault}{\color[rgb]{0,0,0}}}}}}
\put(8626,-2836){\makebox(0,0)[lb]{\smash{{\SetFigFont{14}{16.8}{\rmdefault}{\mddefault}{\updefault}{\color[rgb]{0,0,0}}}}}}
\put(7126,-7186){\makebox(0,0)[lb]{\smash{{\SetFigFont{14}{16.8}{\rmdefault}{\mddefault}{\updefault}{\color[rgb]{0,0,0}}}}}}
\put(8851,-7186){\makebox(0,0)[lb]{\smash{{\SetFigFont{14}{16.8}{\rmdefault}{\mddefault}{\updefault}{\color[rgb]{0,0,0}}}}}}
\put(8551,-6886){\makebox(0,0)[lb]{\smash{{\SetFigFont{14}{16.8}{\rmdefault}{\mddefault}{\updefault}{\color[rgb]{0,0,0}}}}}}
\put(8476,-7186){\makebox(0,0)[lb]{\smash{{\SetFigFont{14}{16.8}{\rmdefault}{\mddefault}{\updefault}{\color[rgb]{0,0,0}}}}}}
\put(8551,-8086){\makebox(0,0)[lb]{\smash{{\SetFigFont{14}{16.8}{\rmdefault}{\mddefault}{\updefault}{\color[rgb]{0,0,0}}}}}}
\put(8476,-8386){\makebox(0,0)[lb]{\smash{{\SetFigFont{14}{16.8}{\rmdefault}{\mddefault}{\updefault}{\color[rgb]{0,0,0}}}}}}
\put(8851,-8386){\makebox(0,0)[lb]{\smash{{\SetFigFont{14}{16.8}{\rmdefault}{\mddefault}{\updefault}{\color[rgb]{0,0,0}}}}}}
\put(8551,-9286){\makebox(0,0)[lb]{\smash{{\SetFigFont{14}{16.8}{\rmdefault}{\mddefault}{\updefault}{\color[rgb]{0,0,0}}}}}}
\put(8476,-9586){\makebox(0,0)[lb]{\smash{{\SetFigFont{14}{16.8}{\rmdefault}{\mddefault}{\updefault}{\color[rgb]{0,0,0}}}}}}
\put(8851,-9586){\makebox(0,0)[lb]{\smash{{\SetFigFont{14}{16.8}{\rmdefault}{\mddefault}{\updefault}{\color[rgb]{0,0,0}}}}}}
\put(8551,-10486){\makebox(0,0)[lb]{\smash{{\SetFigFont{14}{16.8}{\rmdefault}{\mddefault}{\updefault}{\color[rgb]{0,0,0}}}}}}
\put(8476,-10786){\makebox(0,0)[lb]{\smash{{\SetFigFont{14}{16.8}{\rmdefault}{\mddefault}{\updefault}{\color[rgb]{0,0,0}}}}}}
\put(8851,-10786){\makebox(0,0)[lb]{\smash{{\SetFigFont{14}{16.8}{\rmdefault}{\mddefault}{\updefault}{\color[rgb]{0,0,0}}}}}}
\put(6226,-8761){\makebox(0,0)[lb]{\smash{{\SetFigFont{14}{16.8}{\rmdefault}{\mddefault}{\updefault}{\color[rgb]{0,0,0}}}}}}
\put(7126,-8761){\makebox(0,0)[lb]{\smash{{\SetFigFont{14}{16.8}{\rmdefault}{\mddefault}{\updefault}{\color[rgb]{0,0,0}}}}}}
\put(9001,-8836){\makebox(0,0)[lb]{\smash{{\SetFigFont{14}{16.8}{\rmdefault}{\mddefault}{\updefault}{\color[rgb]{0,0,0}}}}}}
\put(8401,-8686){\makebox(0,0)[lb]{\smash{{\SetFigFont{14}{16.8}{\rmdefault}{\mddefault}{\updefault}{\color[rgb]{0,0,0}}}}}}
\put(7951,-6886){\makebox(0,0)[lb]{\smash{{\SetFigFont{14}{16.8}{\rmdefault}{\mddefault}{\updefault}{\color[rgb]{0,0,0}}}}}}
\put(7801,-8761){\makebox(0,0)[lb]{\smash{{\SetFigFont{14}{16.8}{\rmdefault}{\mddefault}{\updefault}{\color[rgb]{0,0,0}}}}}}
\put(9376,-2236){\makebox(0,0)[lb]{\smash{{\SetFigFont{14}{16.8}{\rmdefault}{\mddefault}{\updefault}{\color[rgb]{0,0,0}}}}}}
\put(10426,-4636){\makebox(0,0)[lb]{\smash{{\SetFigFont{14}{16.8}{\rmdefault}{\mddefault}{\updefault}{\color[rgb]{0,0,0}}}}}}
\put(7126,2414){\makebox(0,0)[lb]{\smash{{\SetFigFont{14}{16.8}{\rmdefault}{\mddefault}{\updefault}{\color[rgb]{0,0,0}}}}}}
\put(8851,2414){\makebox(0,0)[lb]{\smash{{\SetFigFont{14}{16.8}{\rmdefault}{\mddefault}{\updefault}{\color[rgb]{0,0,0}}}}}}
\put(8551,2714){\makebox(0,0)[lb]{\smash{{\SetFigFont{14}{16.8}{\rmdefault}{\mddefault}{\updefault}{\color[rgb]{0,0,0}}}}}}
\put(8476,2414){\makebox(0,0)[lb]{\smash{{\SetFigFont{14}{16.8}{\rmdefault}{\mddefault}{\updefault}{\color[rgb]{0,0,0}}}}}}
\put(8551,1514){\makebox(0,0)[lb]{\smash{{\SetFigFont{14}{16.8}{\rmdefault}{\mddefault}{\updefault}{\color[rgb]{0,0,0}}}}}}
\put(8476,1214){\makebox(0,0)[lb]{\smash{{\SetFigFont{14}{16.8}{\rmdefault}{\mddefault}{\updefault}{\color[rgb]{0,0,0}}}}}}
\put(8851,1214){\makebox(0,0)[lb]{\smash{{\SetFigFont{14}{16.8}{\rmdefault}{\mddefault}{\updefault}{\color[rgb]{0,0,0}}}}}}
\put(8551,314){\makebox(0,0)[lb]{\smash{{\SetFigFont{14}{16.8}{\rmdefault}{\mddefault}{\updefault}{\color[rgb]{0,0,0}}}}}}
\put(8476, 14){\makebox(0,0)[lb]{\smash{{\SetFigFont{14}{16.8}{\rmdefault}{\mddefault}{\updefault}{\color[rgb]{0,0,0}}}}}}
\put(8851, 14){\makebox(0,0)[lb]{\smash{{\SetFigFont{14}{16.8}{\rmdefault}{\mddefault}{\updefault}{\color[rgb]{0,0,0}}}}}}
\put(8551,-886){\makebox(0,0)[lb]{\smash{{\SetFigFont{14}{16.8}{\rmdefault}{\mddefault}{\updefault}{\color[rgb]{0,0,0}}}}}}
\put(8476,-1186){\makebox(0,0)[lb]{\smash{{\SetFigFont{14}{16.8}{\rmdefault}{\mddefault}{\updefault}{\color[rgb]{0,0,0}}}}}}
\put(8851,-1186){\makebox(0,0)[lb]{\smash{{\SetFigFont{14}{16.8}{\rmdefault}{\mddefault}{\updefault}{\color[rgb]{0,0,0}}}}}}
\put(9001,3539){\makebox(0,0)[lb]{\smash{{\SetFigFont{14}{16.8}{\rmdefault}{\mddefault}{\updefault}{\color[rgb]{0,0,0}}}}}}
\put(6226,839){\makebox(0,0)[lb]{\smash{{\SetFigFont{14}{16.8}{\rmdefault}{\mddefault}{\updefault}{\color[rgb]{0,0,0}}}}}}
\put(7126,839){\makebox(0,0)[lb]{\smash{{\SetFigFont{14}{16.8}{\rmdefault}{\mddefault}{\updefault}{\color[rgb]{0,0,0}}}}}}
\put(8401,914){\makebox(0,0)[lb]{\smash{{\SetFigFont{14}{16.8}{\rmdefault}{\mddefault}{\updefault}{\color[rgb]{0,0,0}}}}}}
\put(6751,2714){\makebox(0,0)[lb]{\smash{{\SetFigFont{14}{16.8}{\rmdefault}{\mddefault}{\updefault}{\color[rgb]{0,0,0}}}}}}
\put(6676,2414){\makebox(0,0)[lb]{\smash{{\SetFigFont{14}{16.8}{\rmdefault}{\mddefault}{\updefault}{\color[rgb]{0,0,0}}}}}}
\put(6601,3314){\makebox(0,0)[lb]{\smash{{\SetFigFont{14}{16.8}{\rmdefault}{\mddefault}{\updefault}{\color[rgb]{0,0,0}}}}}}
\put(7951,2714){\makebox(0,0)[lb]{\smash{{\SetFigFont{14}{16.8}{\rmdefault}{\mddefault}{\updefault}{\color[rgb]{0,0,0}}}}}}
\put(7801,839){\makebox(0,0)[lb]{\smash{{\SetFigFont{14}{16.8}{\rmdefault}{\mddefault}{\updefault}{\color[rgb]{0,0,0}}}}}}
\put(8551,1964){\makebox(0,0)[lb]{\smash{{\SetFigFont{14}{16.8}{\rmdefault}{\mddefault}{\updefault}{\color[rgb]{0,0,0}}}}}}
\put(9001,764){\makebox(0,0)[lb]{\smash{{\SetFigFont{14}{16.8}{\rmdefault}{\mddefault}{\updefault}{\color[rgb]{0,0,0}}}}}}
\put(8626,-436){\makebox(0,0)[lb]{\smash{{\SetFigFont{14}{16.8}{\rmdefault}{\mddefault}{\updefault}{\color[rgb]{0,0,0}}}}}}
\put(8626,-5836){\makebox(0,0)[lb]{\smash{{\SetFigFont{14}{16.8}{\rmdefault}{\mddefault}{\updefault}{\color[rgb]{0,0,0}}}}}}
\put(8626,-7636){\makebox(0,0)[lb]{\smash{{\SetFigFont{14}{16.8}{\rmdefault}{\mddefault}{\updefault}{\color[rgb]{0,0,0}}}}}}
\put(8626,-10036){\makebox(0,0)[lb]{\smash{{\SetFigFont{14}{16.8}{\rmdefault}{\mddefault}{\updefault}{\color[rgb]{0,0,0}}}}}}
\put(6751,-6886){\makebox(0,0)[lb]{\smash{{\SetFigFont{14}{16.8}{\rmdefault}{\mddefault}{\updefault}{\color[rgb]{0,0,0}}}}}}
\put(6676,-7186){\makebox(0,0)[lb]{\smash{{\SetFigFont{14}{16.8}{\rmdefault}{\mddefault}{\updefault}{\color[rgb]{0,0,0}}}}}}
\put(6601,-6286){\makebox(0,0)[lb]{\smash{{\SetFigFont{14}{16.8}{\rmdefault}{\mddefault}{\updefault}{\color[rgb]{0,0,0}}}}}}
\put(8656,-12466){\makebox(0,0)[lb]{\smash{{\SetFigFont{14}{16.8}{\rmdefault}{\mddefault}{\updefault}{\color[rgb]{0,0,0}}}}}}
\put(8626,-11236){\makebox(0,0)[lb]{\smash{{\SetFigFont{14}{16.8}{\rmdefault}{\mddefault}{\updefault}{\color[rgb]{0,0,0}}}}}}
\put(8581,-12916){\makebox(0,0)[lb]{\smash{{\SetFigFont{14}{16.8}{\rmdefault}{\mddefault}{\updefault}{\color[rgb]{0,0,0}}}}}}
\put(8506,-13216){\makebox(0,0)[lb]{\smash{{\SetFigFont{14}{16.8}{\rmdefault}{\mddefault}{\updefault}{\color[rgb]{0,0,0}}}}}}
\put(8881,-13216){\makebox(0,0)[lb]{\smash{{\SetFigFont{14}{16.8}{\rmdefault}{\mddefault}{\updefault}{\color[rgb]{0,0,0}}}}}}
\put(8581,-4486){\makebox(0,0)[lb]{\smash{{\SetFigFont{14}{16.8}{\rmdefault}{\mddefault}{\updefault}{\color[rgb]{0,0,0}}}}}}
\put(8506,-4786){\makebox(0,0)[lb]{\smash{{\SetFigFont{14}{16.8}{\rmdefault}{\mddefault}{\updefault}{\color[rgb]{0,0,0}}}}}}
\put(8881,-4786){\makebox(0,0)[lb]{\smash{{\SetFigFont{14}{16.8}{\rmdefault}{\mddefault}{\updefault}{\color[rgb]{0,0,0}}}}}}
\put(8581,-11686){\makebox(0,0)[lb]{\smash{{\SetFigFont{14}{16.8}{\rmdefault}{\mddefault}{\updefault}{\color[rgb]{0,0,0}}}}}}
\put(8506,-11986){\makebox(0,0)[lb]{\smash{{\SetFigFont{14}{16.8}{\rmdefault}{\mddefault}{\updefault}{\color[rgb]{0,0,0}}}}}}
\put(8881,-11986){\makebox(0,0)[lb]{\smash{{\SetFigFont{14}{16.8}{\rmdefault}{\mddefault}{\updefault}{\color[rgb]{0,0,0}}}}}}
\put(12151,-3286){\makebox(0,0)[lb]{\smash{{\SetFigFont{14}{16.8}{\rmdefault}{\mddefault}{\updefault}{\color[rgb]{0,0,0}}}}}}
\put(12076,-3586){\makebox(0,0)[lb]{\smash{{\SetFigFont{14}{16.8}{\rmdefault}{\mddefault}{\updefault}{\color[rgb]{0,0,0}}}}}}
\put(12451,-3586){\makebox(0,0)[lb]{\smash{{\SetFigFont{14}{16.8}{\rmdefault}{\mddefault}{\updefault}{\color[rgb]{0,0,0}}}}}}
\put(12151,-5686){\makebox(0,0)[lb]{\smash{{\SetFigFont{14}{16.8}{\rmdefault}{\mddefault}{\updefault}{\color[rgb]{0,0,0}}}}}}
\put(12076,-5986){\makebox(0,0)[lb]{\smash{{\SetFigFont{14}{16.8}{\rmdefault}{\mddefault}{\updefault}{\color[rgb]{0,0,0}}}}}}
\put(12451,-5986){\makebox(0,0)[lb]{\smash{{\SetFigFont{14}{16.8}{\rmdefault}{\mddefault}{\updefault}{\color[rgb]{0,0,0}}}}}}
\put(12601,-4636){\makebox(0,0)[lb]{\smash{{\SetFigFont{14}{16.8}{\rmdefault}{\mddefault}{\updefault}{\color[rgb]{0,0,0}}}}}}
\put(13951,-3286){\makebox(0,0)[lb]{\smash{{\SetFigFont{14}{16.8}{\rmdefault}{\mddefault}{\updefault}{\color[rgb]{0,0,0}}}}}}
\put(13876,-3586){\makebox(0,0)[lb]{\smash{{\SetFigFont{14}{16.8}{\rmdefault}{\mddefault}{\updefault}{\color[rgb]{0,0,0}}}}}}
\put(14251,-3586){\makebox(0,0)[lb]{\smash{{\SetFigFont{14}{16.8}{\rmdefault}{\mddefault}{\updefault}{\color[rgb]{0,0,0}}}}}}
\put(13951,-4486){\makebox(0,0)[lb]{\smash{{\SetFigFont{14}{16.8}{\rmdefault}{\mddefault}{\updefault}{\color[rgb]{0,0,0}}}}}}
\put(13876,-4786){\makebox(0,0)[lb]{\smash{{\SetFigFont{14}{16.8}{\rmdefault}{\mddefault}{\updefault}{\color[rgb]{0,0,0}}}}}}
\put(14401,-2236){\makebox(0,0)[lb]{\smash{{\SetFigFont{14}{16.8}{\rmdefault}{\mddefault}{\updefault}{\color[rgb]{0,0,0}}}}}}
\put(14251,-4786){\makebox(0,0)[lb]{\smash{{\SetFigFont{14}{16.8}{\rmdefault}{\mddefault}{\updefault}{\color[rgb]{0,0,0}}}}}}
\put(14551,-3886){\makebox(0,0)[lb]{\smash{{\SetFigFont{14}{16.8}{\rmdefault}{\mddefault}{\updefault}{\color[rgb]{0,0,0}}}}}}
\put(14026,-4036){\makebox(0,0)[lb]{\smash{{\SetFigFont{14}{16.8}{\rmdefault}{\mddefault}{\updefault}{\color[rgb]{0,0,0}}}}}}
\put(12751,-2686){\makebox(0,0)[lb]{\smash{{\SetFigFont{14}{16.8}{\rmdefault}{\mddefault}{\updefault}{\color[rgb]{0,0,0}}}}}}
\put(9151,-11086){\makebox(0,0)[lb]{\smash{{\SetFigFont{14}{16.8}{\rmdefault}{\mddefault}{\updefault}{\color[rgb]{0,0,0}}}}}}
\put(5476,-1936){\makebox(0,0)[lb]{\smash{{\SetFigFont{14}{16.8}{\rmdefault}{\mddefault}{\updefault}{\color[rgb]{0,0,0}}}}}}
\put(5476,-1636){\makebox(0,0)[lb]{\smash{{\SetFigFont{14}{16.8}{\rmdefault}{\mddefault}{\updefault}{\color[rgb]{0,0,0}}}}}}
\put(5476,-3736){\makebox(0,0)[lb]{\smash{{\SetFigFont{14}{16.8}{\rmdefault}{\mddefault}{\updefault}{\color[rgb]{0,0,0}}}}}}
\put(5476,-3436){\makebox(0,0)[lb]{\smash{{\SetFigFont{14}{16.8}{\rmdefault}{\mddefault}{\updefault}{\color[rgb]{0,0,0}}}}}}
\put(5476,-3136){\makebox(0,0)[lb]{\smash{{\SetFigFont{14}{16.8}{\rmdefault}{\mddefault}{\updefault}{\color[rgb]{0,0,0}}}}}}
\put(5476,-2836){\makebox(0,0)[lb]{\smash{{\SetFigFont{14}{16.8}{\rmdefault}{\mddefault}{\updefault}{\color[rgb]{0,0,0}}}}}}
\put(5476,-2236){\makebox(0,0)[lb]{\smash{{\SetFigFont{14}{16.8}{\rmdefault}{\mddefault}{\updefault}{\color[rgb]{0,0,0}}}}}}
\put(5476,-4036){\makebox(0,0)[lb]{\smash{{\SetFigFont{14}{16.8}{\rmdefault}{\mddefault}{\updefault}{\color[rgb]{0,0,0}}}}}}
\put(5476,-2536){\makebox(0,0)[lb]{\smash{{\SetFigFont{14}{16.8}{\rmdefault}{\mddefault}{\updefault}{\color[rgb]{0,0,0}}}}}}
\put(5476,-4336){\makebox(0,0)[lb]{\smash{{\SetFigFont{14}{16.8}{\rmdefault}{\mddefault}{\updefault}{\color[rgb]{0,0,0}}}}}}
\put(15751,1364){\makebox(0,0)[lb]{\smash{{\SetFigFont{14}{16.8}{\rmdefault}{\mddefault}{\updefault}{\color[rgb]{0,0,0}}}}}}
\put(9826,-8236){\makebox(0,0)[lb]{\smash{{\SetFigFont{14}{16.8}{\rmdefault}{\mddefault}{\updefault}{\color[rgb]{0,0,0}}}}}}
\put(8851,-14086){\makebox(0,0)[lb]{\smash{{\SetFigFont{14}{16.8}{\rmdefault}{\mddefault}{\updefault}{\color[rgb]{0,0,0}}}}}}
\put(12451,-6886){\makebox(0,0)[lb]{\smash{{\SetFigFont{14}{16.8}{\rmdefault}{\mddefault}{\updefault}{\color[rgb]{0,0,0}}}}}}
\put(14251,-5686){\makebox(0,0)[lb]{\smash{{\SetFigFont{14}{16.8}{\rmdefault}{\mddefault}{\updefault}{\color[rgb]{0,0,0}}}}}}
\put(6976,-11161){\makebox(0,0)[lb]{\smash{{\SetFigFont{20}{24.0}{\rmdefault}{\bfdefault}{\updefault}{\color[rgb]{0,0,0}}}}}}
\put(12826,1589){\makebox(0,0)[lb]{\smash{{\SetFigFont{20}{24.0}{\rmdefault}{\bfdefault}{\updefault}{\color[rgb]{0,0,0}}}}}}
\put(16126,-4036){\makebox(0,0)[lb]{\smash{{\SetFigFont{14}{16.8}{\rmdefault}{\mddefault}{\updefault}{\color[rgb]{0,0,0}}}}}}
\put(14326,-6136){\makebox(0,0)[lb]{\smash{{\SetFigFont{20}{24.0}{\rmdefault}{\bfdefault}{\updefault}{\color[rgb]{0,0,0}}}}}}
\put(13306,-10846){\makebox(0,0)[lb]{\smash{{\SetFigFont{14}{16.8}{\rmdefault}{\mddefault}{\updefault}{\color[rgb]{0,0,0}}}}}}
\put(11656,-10846){\makebox(0,0)[lb]{\smash{{\SetFigFont{14}{16.8}{\rmdefault}{\mddefault}{\updefault}{\color[rgb]{0,0,0}}}}}}
\put(14131,-10396){\makebox(0,0)[lb]{\smash{{\SetFigFont{14}{16.8}{\rmdefault}{\mddefault}{\updefault}{\color[rgb]{0,0,0}}}}}}
\put(12031,-10621){\makebox(0,0)[lb]{\smash{{\SetFigFont{14}{16.8}{\rmdefault}{\mddefault}{\updefault}{\color[rgb]{0,0,0}}}}}}
\put(14956,-10846){\makebox(0,0)[lb]{\smash{{\SetFigFont{14}{16.8}{\rmdefault}{\mddefault}{\updefault}{\color[rgb]{0,0,0}}}}}}
\put(13681,-10621){\makebox(0,0)[lb]{\smash{{\SetFigFont{14}{16.8}{\rmdefault}{\mddefault}{\updefault}{\color[rgb]{0,0,0}}}}}}
\put(13306,-12946){\makebox(0,0)[lb]{\smash{{\SetFigFont{14}{16.8}{\rmdefault}{\mddefault}{\updefault}{\color[rgb]{0,0,0}}}}}}
\put(13381,-9271){\makebox(0,0)[lb]{\smash{{\SetFigFont{14}{16.8}{\rmdefault}{\mddefault}{\updefault}{\color[rgb]{0,0,0}}}}}}
\put(12931,-9496){\makebox(0,0)[lb]{\smash{{\SetFigFont{14}{16.8}{\rmdefault}{\mddefault}{\updefault}{\color[rgb]{0,0,0}}}}}}
\put(13306,-11971){\makebox(0,0)[lb]{\smash{{\SetFigFont{14}{16.8}{\rmdefault}{\mddefault}{\updefault}{\color[rgb]{0,0,0}}}}}}
\put(10531,-11671){\makebox(0,0)[lb]{\smash{{\SetFigFont{14}{16.8}{\rmdefault}{\mddefault}{\updefault}{\color[rgb]{0,0,0}}}}}}
\put(12481,-10396){\makebox(0,0)[lb]{\smash{{\SetFigFont{14}{16.8}{\rmdefault}{\mddefault}{\updefault}{\color[rgb]{0,0,0}}}}}}
\put(12856,-13171){\makebox(0,0)[lb]{\smash{{\SetFigFont{14}{16.8}{\rmdefault}{\mddefault}{\updefault}{\color[rgb]{0,0,0}}}}}}
\put(11131,-11446){\makebox(0,0)[lb]{\smash{{\SetFigFont{14}{16.8}{\rmdefault}{\mddefault}{\updefault}{\color[rgb]{0,0,0}}}}}}
\put(15706,-11596){\makebox(0,0)[lb]{\smash{{\SetFigFont{14}{16.8}{\rmdefault}{\mddefault}{\updefault}{\color[rgb]{0,0,0}}}}}}
\put(15931,-11371){\makebox(0,0)[lb]{\smash{{\SetFigFont{14}{16.8}{\rmdefault}{\mddefault}{\updefault}{\color[rgb]{0,0,0}}}}}}
\put(13681,-11221){\makebox(0,0)[lb]{\smash{{\SetFigFont{14}{16.8}{\rmdefault}{\mddefault}{\updefault}{\color[rgb]{0,0,0}}}}}}
\put(13456,-11446){\makebox(0,0)[lb]{\smash{{\SetFigFont{14}{16.8}{\rmdefault}{\mddefault}{\updefault}{\color[rgb]{0,0,0}}}}}}
\put(14581,-11746){\makebox(0,0)[lb]{\smash{{\SetFigFont{14}{16.8}{\rmdefault}{\mddefault}{\updefault}{\color[rgb]{0,0,0}}}}}}
\put(14206,-11971){\makebox(0,0)[lb]{\smash{{\SetFigFont{14}{16.8}{\rmdefault}{\mddefault}{\updefault}{\color[rgb]{0,0,0}}}}}}
\put(15376,-9661){\makebox(0,0)[lb]{\smash{{\SetFigFont{14}{16.8}{\rmdefault}{\mddefault}{\updefault}{\color[rgb]{0,0,0}}}}}}
\put(14551,-9961){\makebox(0,0)[lb]{\smash{{\SetFigFont{14}{16.8}{\rmdefault}{\mddefault}{\updefault}{\color[rgb]{0,0,0}}}}}}
\put(13201,-13711){\makebox(0,0)[lb]{\smash{{\SetFigFont{14}{16.8}{\rmdefault}{\mddefault}{\updefault}{\color[rgb]{0,0,0}}}}}}
\end{picture} }
		\caption{(a) The simplified ; (b)  with period }
		\label{fig:complex}
\end{figure}
\begin{figure}[t]
\centering
		\scalebox{0.4}{ \begin{picture}(0,0)\includegraphics{complex2.pdf}\end{picture}\setlength{\unitlength}{3947sp}\begingroup\makeatletter\ifx\SetFigFont\undefined \gdef\SetFigFont#1#2#3#4#5{\reset@font\fontsize{#1}{#2pt}\fontfamily{#3}\fontseries{#4}\fontshape{#5}\selectfont}\fi\endgroup \begin{picture}(11565,19900)(4411,-16418)
\put(9301,-7036){\makebox(0,0)[lb]{\smash{{\SetFigFont{20}{24.0}{\rmdefault}{\bfdefault}{\updefault}{\color[rgb]{0,0,0}}}}}}
\put(10351,-2086){\makebox(0,0)[lb]{\smash{{\SetFigFont{14}{16.8}{\rmdefault}{\mddefault}{\updefault}{\color[rgb]{0,0,0}}}}}}
\put(10651,-2386){\makebox(0,0)[lb]{\smash{{\SetFigFont{14}{16.8}{\rmdefault}{\mddefault}{\updefault}{\color[rgb]{0,0,0}}}}}}
\put(10276,-2386){\makebox(0,0)[lb]{\smash{{\SetFigFont{14}{16.8}{\rmdefault}{\mddefault}{\updefault}{\color[rgb]{0,0,0}}}}}}
\put(4576,1364){\makebox(0,0)[lb]{\smash{{\SetFigFont{14}{16.8}{\rmdefault}{\mddefault}{\updefault}{\color[rgb]{0,0,0}}}}}}
\put(4726,2564){\makebox(0,0)[lb]{\smash{{\SetFigFont{14}{16.8}{\rmdefault}{\mddefault}{\updefault}{\color[rgb]{0,0,0}}}}}}
\put(4726,3239){\makebox(0,0)[lb]{\smash{{\SetFigFont{20}{24.0}{\rmdefault}{\mddefault}{\updefault}{\color[rgb]{0,0,0}}}}}}
\put(12151,-2086){\makebox(0,0)[lb]{\smash{{\SetFigFont{14}{16.8}{\rmdefault}{\mddefault}{\updefault}{\color[rgb]{0,0,0}}}}}}
\put(12076,-2386){\makebox(0,0)[lb]{\smash{{\SetFigFont{14}{16.8}{\rmdefault}{\mddefault}{\updefault}{\color[rgb]{0,0,0}}}}}}
\put(12451,-2386){\makebox(0,0)[lb]{\smash{{\SetFigFont{14}{16.8}{\rmdefault}{\mddefault}{\updefault}{\color[rgb]{0,0,0}}}}}}
\put(8581,-2686){\makebox(0,0)[lb]{\smash{{\SetFigFont{14}{16.8}{\rmdefault}{\mddefault}{\updefault}{\color[rgb]{0,0,0}}}}}}
\put(8881,-2986){\makebox(0,0)[lb]{\smash{{\SetFigFont{14}{16.8}{\rmdefault}{\mddefault}{\updefault}{\color[rgb]{0,0,0}}}}}}
\put(8476,-2986){\makebox(0,0)[lb]{\smash{{\SetFigFont{14}{16.8}{\rmdefault}{\mddefault}{\updefault}{\color[rgb]{0,0,0}}}}}}
\put(6781,-3286){\makebox(0,0)[lb]{\smash{{\SetFigFont{14}{16.8}{\rmdefault}{\mddefault}{\updefault}{\color[rgb]{0,0,0}}}}}}
\put(7081,-3586){\makebox(0,0)[lb]{\smash{{\SetFigFont{14}{16.8}{\rmdefault}{\mddefault}{\updefault}{\color[rgb]{0,0,0}}}}}}
\put(6676,-3586){\makebox(0,0)[lb]{\smash{{\SetFigFont{14}{16.8}{\rmdefault}{\mddefault}{\updefault}{\color[rgb]{0,0,0}}}}}}
\put(12151,-3886){\makebox(0,0)[lb]{\smash{{\SetFigFont{14}{16.8}{\rmdefault}{\mddefault}{\updefault}{\color[rgb]{0,0,0}}}}}}
\put(12076,-4186){\makebox(0,0)[lb]{\smash{{\SetFigFont{14}{16.8}{\rmdefault}{\mddefault}{\updefault}{\color[rgb]{0,0,0}}}}}}
\put(12451,-4186){\makebox(0,0)[lb]{\smash{{\SetFigFont{14}{16.8}{\rmdefault}{\mddefault}{\updefault}{\color[rgb]{0,0,0}}}}}}
\put(8881,-4786){\makebox(0,0)[lb]{\smash{{\SetFigFont{14}{16.8}{\rmdefault}{\mddefault}{\updefault}{\color[rgb]{0,0,0}}}}}}
\put(8581,-4486){\makebox(0,0)[lb]{\smash{{\SetFigFont{14}{16.8}{\rmdefault}{\mddefault}{\updefault}{\color[rgb]{0,0,0}}}}}}
\put(8506,-4786){\makebox(0,0)[lb]{\smash{{\SetFigFont{14}{16.8}{\rmdefault}{\mddefault}{\updefault}{\color[rgb]{0,0,0}}}}}}
\put(7051,-5386){\makebox(0,0)[lb]{\smash{{\SetFigFont{14}{16.8}{\rmdefault}{\mddefault}{\updefault}{\color[rgb]{0,0,0}}}}}}
\put(6751,-5086){\makebox(0,0)[lb]{\smash{{\SetFigFont{14}{16.8}{\rmdefault}{\mddefault}{\updefault}{\color[rgb]{0,0,0}}}}}}
\put(6676,-5386){\makebox(0,0)[lb]{\smash{{\SetFigFont{14}{16.8}{\rmdefault}{\mddefault}{\updefault}{\color[rgb]{0,0,0}}}}}}
\put(12181,-5686){\makebox(0,0)[lb]{\smash{{\SetFigFont{14}{16.8}{\rmdefault}{\mddefault}{\updefault}{\color[rgb]{0,0,0}}}}}}
\put(12481,-5986){\makebox(0,0)[lb]{\smash{{\SetFigFont{14}{16.8}{\rmdefault}{\mddefault}{\updefault}{\color[rgb]{0,0,0}}}}}}
\put(12076,-5986){\makebox(0,0)[lb]{\smash{{\SetFigFont{14}{16.8}{\rmdefault}{\mddefault}{\updefault}{\color[rgb]{0,0,0}}}}}}
\put(8581,-8086){\makebox(0,0)[lb]{\smash{{\SetFigFont{14}{16.8}{\rmdefault}{\mddefault}{\updefault}{\color[rgb]{0,0,0}}}}}}
\put(8881,-8386){\makebox(0,0)[lb]{\smash{{\SetFigFont{14}{16.8}{\rmdefault}{\mddefault}{\updefault}{\color[rgb]{0,0,0}}}}}}
\put(8476,-8386){\makebox(0,0)[lb]{\smash{{\SetFigFont{14}{16.8}{\rmdefault}{\mddefault}{\updefault}{\color[rgb]{0,0,0}}}}}}
\put(12181,-8086){\makebox(0,0)[lb]{\smash{{\SetFigFont{14}{16.8}{\rmdefault}{\mddefault}{\updefault}{\color[rgb]{0,0,0}}}}}}
\put(12106,-8386){\makebox(0,0)[lb]{\smash{{\SetFigFont{14}{16.8}{\rmdefault}{\mddefault}{\updefault}{\color[rgb]{0,0,0}}}}}}
\put(12481,-8386){\makebox(0,0)[lb]{\smash{{\SetFigFont{14}{16.8}{\rmdefault}{\mddefault}{\updefault}{\color[rgb]{0,0,0}}}}}}
\put(10381,-8086){\makebox(0,0)[lb]{\smash{{\SetFigFont{14}{16.8}{\rmdefault}{\mddefault}{\updefault}{\color[rgb]{0,0,0}}}}}}
\put(10306,-8386){\makebox(0,0)[lb]{\smash{{\SetFigFont{14}{16.8}{\rmdefault}{\mddefault}{\updefault}{\color[rgb]{0,0,0}}}}}}
\put(10681,-8386){\makebox(0,0)[lb]{\smash{{\SetFigFont{14}{16.8}{\rmdefault}{\mddefault}{\updefault}{\color[rgb]{0,0,0}}}}}}
\put(13381,-8686){\makebox(0,0)[lb]{\smash{{\SetFigFont{14}{16.8}{\rmdefault}{\mddefault}{\updefault}{\color[rgb]{0,0,0}}}}}}
\put(13306,-8986){\makebox(0,0)[lb]{\smash{{\SetFigFont{14}{16.8}{\rmdefault}{\mddefault}{\updefault}{\color[rgb]{0,0,0}}}}}}
\put(13681,-8986){\makebox(0,0)[lb]{\smash{{\SetFigFont{14}{16.8}{\rmdefault}{\mddefault}{\updefault}{\color[rgb]{0,0,0}}}}}}
\put(14581,-9286){\makebox(0,0)[lb]{\smash{{\SetFigFont{14}{16.8}{\rmdefault}{\mddefault}{\updefault}{\color[rgb]{0,0,0}}}}}}
\put(14506,-9586){\makebox(0,0)[lb]{\smash{{\SetFigFont{14}{16.8}{\rmdefault}{\mddefault}{\updefault}{\color[rgb]{0,0,0}}}}}}
\put(14881,-9586){\makebox(0,0)[lb]{\smash{{\SetFigFont{14}{16.8}{\rmdefault}{\mddefault}{\updefault}{\color[rgb]{0,0,0}}}}}}
\put(14551,-10486){\makebox(0,0)[lb]{\smash{{\SetFigFont{14}{16.8}{\rmdefault}{\mddefault}{\updefault}{\color[rgb]{0,0,0}}}}}}
\put(14476,-10786){\makebox(0,0)[lb]{\smash{{\SetFigFont{14}{16.8}{\rmdefault}{\mddefault}{\updefault}{\color[rgb]{0,0,0}}}}}}
\put(14851,-10786){\makebox(0,0)[lb]{\smash{{\SetFigFont{14}{16.8}{\rmdefault}{\mddefault}{\updefault}{\color[rgb]{0,0,0}}}}}}
\put(12151,-9886){\makebox(0,0)[lb]{\smash{{\SetFigFont{14}{16.8}{\rmdefault}{\mddefault}{\updefault}{\color[rgb]{0,0,0}}}}}}
\put(12076,-10186){\makebox(0,0)[lb]{\smash{{\SetFigFont{14}{16.8}{\rmdefault}{\mddefault}{\updefault}{\color[rgb]{0,0,0}}}}}}
\put(12451,-10186){\makebox(0,0)[lb]{\smash{{\SetFigFont{14}{16.8}{\rmdefault}{\mddefault}{\updefault}{\color[rgb]{0,0,0}}}}}}
\put(14536,-7486){\makebox(0,0)[lb]{\smash{{\SetFigFont{14}{16.8}{\rmdefault}{\mddefault}{\updefault}{\color[rgb]{0,0,0}}}}}}
\put(14461,-7786){\makebox(0,0)[lb]{\smash{{\SetFigFont{14}{16.8}{\rmdefault}{\mddefault}{\updefault}{\color[rgb]{0,0,0}}}}}}
\put(14836,-7786){\makebox(0,0)[lb]{\smash{{\SetFigFont{14}{16.8}{\rmdefault}{\mddefault}{\updefault}{\color[rgb]{0,0,0}}}}}}
\put(6751,-6886){\makebox(0,0)[lb]{\smash{{\SetFigFont{14}{16.8}{\rmdefault}{\mddefault}{\updefault}{\color[rgb]{0,0,0}}}}}}
\put(6676,-7186){\makebox(0,0)[lb]{\smash{{\SetFigFont{14}{16.8}{\rmdefault}{\mddefault}{\updefault}{\color[rgb]{0,0,0}}}}}}
\put(7051,-7186){\makebox(0,0)[lb]{\smash{{\SetFigFont{14}{16.8}{\rmdefault}{\mddefault}{\updefault}{\color[rgb]{0,0,0}}}}}}
\put(12136,-6886){\makebox(0,0)[lb]{\smash{{\SetFigFont{14}{16.8}{\rmdefault}{\mddefault}{\updefault}{\color[rgb]{0,0,0}}}}}}
\put(12061,-7186){\makebox(0,0)[lb]{\smash{{\SetFigFont{14}{16.8}{\rmdefault}{\mddefault}{\updefault}{\color[rgb]{0,0,0}}}}}}
\put(12436,-7186){\makebox(0,0)[lb]{\smash{{\SetFigFont{14}{16.8}{\rmdefault}{\mddefault}{\updefault}{\color[rgb]{0,0,0}}}}}}
\put(5476,2864){\makebox(0,0)[lb]{\smash{{\SetFigFont{14}{16.8}{\rmdefault}{\mddefault}{\updefault}{\color[rgb]{0,0,0}}}}}}
\put(5476,3164){\makebox(0,0)[lb]{\smash{{\SetFigFont{14}{16.8}{\rmdefault}{\mddefault}{\updefault}{\color[rgb]{0,0,0}}}}}}
\put(5476,1064){\makebox(0,0)[lb]{\smash{{\SetFigFont{14}{16.8}{\rmdefault}{\mddefault}{\updefault}{\color[rgb]{0,0,0}}}}}}
\put(5476,1364){\makebox(0,0)[lb]{\smash{{\SetFigFont{14}{16.8}{\rmdefault}{\mddefault}{\updefault}{\color[rgb]{0,0,0}}}}}}
\put(5476,1664){\makebox(0,0)[lb]{\smash{{\SetFigFont{14}{16.8}{\rmdefault}{\mddefault}{\updefault}{\color[rgb]{0,0,0}}}}}}
\put(5476,1964){\makebox(0,0)[lb]{\smash{{\SetFigFont{14}{16.8}{\rmdefault}{\mddefault}{\updefault}{\color[rgb]{0,0,0}}}}}}
\put(5476,2564){\makebox(0,0)[lb]{\smash{{\SetFigFont{14}{16.8}{\rmdefault}{\mddefault}{\updefault}{\color[rgb]{0,0,0}}}}}}
\put(5476,764){\makebox(0,0)[lb]{\smash{{\SetFigFont{14}{16.8}{\rmdefault}{\mddefault}{\updefault}{\color[rgb]{0,0,0}}}}}}
\put(5476,2264){\makebox(0,0)[lb]{\smash{{\SetFigFont{14}{16.8}{\rmdefault}{\mddefault}{\updefault}{\color[rgb]{0,0,0}}}}}}
\put(5476,464){\makebox(0,0)[lb]{\smash{{\SetFigFont{14}{16.8}{\rmdefault}{\mddefault}{\updefault}{\color[rgb]{0,0,0}}}}}}
\put(5476,164){\makebox(0,0)[lb]{\smash{{\SetFigFont{14}{16.8}{\rmdefault}{\mddefault}{\updefault}{\color[rgb]{0,0,0}}}}}}
\put(5476,-136){\makebox(0,0)[lb]{\smash{{\SetFigFont{14}{16.8}{\rmdefault}{\mddefault}{\updefault}{\color[rgb]{0,0,0}}}}}}
\put(5476,-436){\makebox(0,0)[lb]{\smash{{\SetFigFont{14}{16.8}{\rmdefault}{\mddefault}{\updefault}{\color[rgb]{0,0,0}}}}}}
\put(12181,-886){\makebox(0,0)[lb]{\smash{{\SetFigFont{14}{16.8}{\rmdefault}{\mddefault}{\updefault}{\color[rgb]{0,0,0}}}}}}
\put(12481,-1186){\makebox(0,0)[lb]{\smash{{\SetFigFont{14}{16.8}{\rmdefault}{\mddefault}{\updefault}{\color[rgb]{0,0,0}}}}}}
\put(12106,-1186){\makebox(0,0)[lb]{\smash{{\SetFigFont{14}{16.8}{\rmdefault}{\mddefault}{\updefault}{\color[rgb]{0,0,0}}}}}}
\put(10351,314){\makebox(0,0)[lb]{\smash{{\SetFigFont{14}{16.8}{\rmdefault}{\mddefault}{\updefault}{\color[rgb]{0,0,0}}}}}}
\put(10651, 14){\makebox(0,0)[lb]{\smash{{\SetFigFont{14}{16.8}{\rmdefault}{\mddefault}{\updefault}{\color[rgb]{0,0,0}}}}}}
\put(10276, 14){\makebox(0,0)[lb]{\smash{{\SetFigFont{14}{16.8}{\rmdefault}{\mddefault}{\updefault}{\color[rgb]{0,0,0}}}}}}
\put(10351,-886){\makebox(0,0)[lb]{\smash{{\SetFigFont{14}{16.8}{\rmdefault}{\mddefault}{\updefault}{\color[rgb]{0,0,0}}}}}}
\put(10651,-1186){\makebox(0,0)[lb]{\smash{{\SetFigFont{14}{16.8}{\rmdefault}{\mddefault}{\updefault}{\color[rgb]{0,0,0}}}}}}
\put(10276,-1186){\makebox(0,0)[lb]{\smash{{\SetFigFont{14}{16.8}{\rmdefault}{\mddefault}{\updefault}{\color[rgb]{0,0,0}}}}}}
\put(8551,-886){\makebox(0,0)[lb]{\smash{{\SetFigFont{14}{16.8}{\rmdefault}{\mddefault}{\updefault}{\color[rgb]{0,0,0}}}}}}
\put(8851,-1186){\makebox(0,0)[lb]{\smash{{\SetFigFont{14}{16.8}{\rmdefault}{\mddefault}{\updefault}{\color[rgb]{0,0,0}}}}}}
\put(4576,164){\makebox(0,0)[lb]{\smash{{\SetFigFont{14}{16.8}{\rmdefault}{\mddefault}{\updefault}{\color[rgb]{0,0,0}}}}}}
\put(4726,-1036){\makebox(0,0)[lb]{\smash{{\SetFigFont{14}{16.8}{\rmdefault}{\mddefault}{\updefault}{\color[rgb]{0,0,0}}}}}}
\put(8476,-1186){\makebox(0,0)[lb]{\smash{{\SetFigFont{14}{16.8}{\rmdefault}{\mddefault}{\updefault}{\color[rgb]{0,0,0}}}}}}
\put(4576,-2236){\makebox(0,0)[lb]{\smash{{\SetFigFont{14}{16.8}{\rmdefault}{\mddefault}{\updefault}{\color[rgb]{0,0,0}}}}}}
\put(4801,-1636){\makebox(0,0)[lb]{\smash{{\SetFigFont{14}{16.8}{\rmdefault}{\mddefault}{\updefault}{\color[rgb]{0,0,0}}}}}}
\put(10801,-1636){\makebox(0,0)[lb]{\smash{{\SetFigFont{14}{16.8}{\rmdefault}{\mddefault}{\updefault}{\color[rgb]{0,0,0}}}}}}
\put(9676,-1036){\makebox(0,0)[lb]{\smash{{\SetFigFont{14}{16.8}{\rmdefault}{\mddefault}{\updefault}{\color[rgb]{0,0,0}}}}}}
\put(10351,2714){\makebox(0,0)[lb]{\smash{{\SetFigFont{14}{16.8}{\rmdefault}{\mddefault}{\updefault}{\color[rgb]{0,0,0}}}}}}
\put(10276,2414){\makebox(0,0)[lb]{\smash{{\SetFigFont{14}{16.8}{\rmdefault}{\mddefault}{\updefault}{\color[rgb]{0,0,0}}}}}}
\put(10651,2414){\makebox(0,0)[lb]{\smash{{\SetFigFont{14}{16.8}{\rmdefault}{\mddefault}{\updefault}{\color[rgb]{0,0,0}}}}}}
\put(12151,2714){\makebox(0,0)[lb]{\smash{{\SetFigFont{14}{16.8}{\rmdefault}{\mddefault}{\updefault}{\color[rgb]{0,0,0}}}}}}
\put(12076,2414){\makebox(0,0)[lb]{\smash{{\SetFigFont{14}{16.8}{\rmdefault}{\mddefault}{\updefault}{\color[rgb]{0,0,0}}}}}}
\put(12451,2414){\makebox(0,0)[lb]{\smash{{\SetFigFont{14}{16.8}{\rmdefault}{\mddefault}{\updefault}{\color[rgb]{0,0,0}}}}}}
\put(11551,2714){\makebox(0,0)[lb]{\smash{{\SetFigFont{14}{16.8}{\rmdefault}{\mddefault}{\updefault}{\color[rgb]{0,0,0}}}}}}
\put(8851,1214){\makebox(0,0)[lb]{\smash{{\SetFigFont{14}{16.8}{\rmdefault}{\mddefault}{\updefault}{\color[rgb]{0,0,0}}}}}}
\put(8551,1514){\makebox(0,0)[lb]{\smash{{\SetFigFont{14}{16.8}{\rmdefault}{\mddefault}{\updefault}{\color[rgb]{0,0,0}}}}}}
\put(8476,1214){\makebox(0,0)[lb]{\smash{{\SetFigFont{14}{16.8}{\rmdefault}{\mddefault}{\updefault}{\color[rgb]{0,0,0}}}}}}
\put(9001,2339){\makebox(0,0)[lb]{\smash{{\SetFigFont{14}{16.8}{\rmdefault}{\mddefault}{\updefault}{\color[rgb]{0,0,0}}}}}}
\put(10801,1364){\makebox(0,0)[lb]{\smash{{\SetFigFont{14}{16.8}{\rmdefault}{\mddefault}{\updefault}{\color[rgb]{0,0,0}}}}}}
\put(12601,764){\makebox(0,0)[lb]{\smash{{\SetFigFont{14}{16.8}{\rmdefault}{\mddefault}{\updefault}{\color[rgb]{0,0,0}}}}}}
\put(11701,764){\makebox(0,0)[lb]{\smash{{\SetFigFont{14}{16.8}{\rmdefault}{\mddefault}{\updefault}{\color[rgb]{0,0,0}}}}}}
\put(8626,164){\makebox(0,0)[lb]{\smash{{\SetFigFont{14}{16.8}{\rmdefault}{\mddefault}{\updefault}{\color[rgb]{0,0,0}}}}}}
\put(8626,-1936){\makebox(0,0)[lb]{\smash{{\SetFigFont{14}{16.8}{\rmdefault}{\mddefault}{\updefault}{\color[rgb]{0,0,0}}}}}}
\put(4726,-2911){\makebox(0,0)[lb]{\smash{{\SetFigFont{14}{16.8}{\rmdefault}{\mddefault}{\updefault}{\color[rgb]{0,0,0}}}}}}
\put(4576,-3436){\makebox(0,0)[lb]{\smash{{\SetFigFont{14}{16.8}{\rmdefault}{\mddefault}{\updefault}{\color[rgb]{0,0,0}}}}}}
\put(4576,-4036){\makebox(0,0)[lb]{\smash{{\SetFigFont{14}{16.8}{\rmdefault}{\mddefault}{\updefault}{\color[rgb]{0,0,0}}}}}}
\put(4726,-4636){\makebox(0,0)[lb]{\smash{{\SetFigFont{14}{16.8}{\rmdefault}{\mddefault}{\updefault}{\color[rgb]{0,0,0}}}}}}
\put(4801,-3736){\makebox(0,0)[lb]{\smash{{\SetFigFont{14}{16.8}{\rmdefault}{\mddefault}{\updefault}{\color[rgb]{0,0,0}}}}}}
\put(4576,-5236){\makebox(0,0)[lb]{\smash{{\SetFigFont{14}{16.8}{\rmdefault}{\mddefault}{\updefault}{\color[rgb]{0,0,0}}}}}}
\put(4801,-5536){\makebox(0,0)[lb]{\smash{{\SetFigFont{14}{16.8}{\rmdefault}{\mddefault}{\updefault}{\color[rgb]{0,0,0}}}}}}
\put(7126,-1936){\makebox(0,0)[lb]{\smash{{\SetFigFont{14}{16.8}{\rmdefault}{\mddefault}{\updefault}{\color[rgb]{0,0,0}}}}}}
\put(10726,-3286){\makebox(0,0)[lb]{\smash{{\SetFigFont{14}{16.8}{\rmdefault}{\mddefault}{\updefault}{\color[rgb]{0,0,0}}}}}}
\put(6826,-4336){\makebox(0,0)[lb]{\smash{{\SetFigFont{14}{16.8}{\rmdefault}{\mddefault}{\updefault}{\color[rgb]{0,0,0}}}}}}
\put(8626,-3736){\makebox(0,0)[lb]{\smash{{\SetFigFont{14}{16.8}{\rmdefault}{\mddefault}{\updefault}{\color[rgb]{0,0,0}}}}}}
\put(4576,-5836){\makebox(0,0)[lb]{\smash{{\SetFigFont{14}{16.8}{\rmdefault}{\mddefault}{\updefault}{\color[rgb]{0,0,0}}}}}}
\put(6826,-6136){\makebox(0,0)[lb]{\smash{{\SetFigFont{14}{16.8}{\rmdefault}{\mddefault}{\updefault}{\color[rgb]{0,0,0}}}}}}
\put(8626,-5536){\makebox(0,0)[lb]{\smash{{\SetFigFont{14}{16.8}{\rmdefault}{\mddefault}{\updefault}{\color[rgb]{0,0,0}}}}}}
\put(12601,-4936){\makebox(0,0)[lb]{\smash{{\SetFigFont{14}{16.8}{\rmdefault}{\mddefault}{\updefault}{\color[rgb]{0,0,0}}}}}}
\put(10726,-6736){\makebox(0,0)[lb]{\smash{{\SetFigFont{14}{16.8}{\rmdefault}{\mddefault}{\updefault}{\color[rgb]{0,0,0}}}}}}
\put(11176,-7411){\makebox(0,0)[lb]{\smash{{\SetFigFont{14}{16.8}{\rmdefault}{\mddefault}{\updefault}{\color[rgb]{0,0,0}}}}}}
\put(12601,-3136){\makebox(0,0)[lb]{\smash{{\SetFigFont{14}{16.8}{\rmdefault}{\mddefault}{\updefault}{\color[rgb]{0,0,0}}}}}}
\put(12601,-6436){\makebox(0,0)[lb]{\smash{{\SetFigFont{14}{16.8}{\rmdefault}{\mddefault}{\updefault}{\color[rgb]{0,0,0}}}}}}
\put(4801,-8536){\makebox(0,0)[lb]{\smash{{\SetFigFont{14}{16.8}{\rmdefault}{\mddefault}{\updefault}{\color[rgb]{0,0,0}}}}}}
\put(4651,-10636){\makebox(0,0)[lb]{\smash{{\SetFigFont{14}{16.8}{\rmdefault}{\mddefault}{\updefault}{\color[rgb]{0,0,0}}}}}}
\put(4501,-8236){\makebox(0,0)[lb]{\smash{{\SetFigFont{14}{16.8}{\rmdefault}{\mddefault}{\updefault}{\color[rgb]{0,0,0}}}}}}
\put(4426,-10036){\makebox(0,0)[lb]{\smash{{\SetFigFont{14}{16.8}{\rmdefault}{\mddefault}{\updefault}{\color[rgb]{0,0,0}}}}}}
\put(4801,-10336){\makebox(0,0)[lb]{\smash{{\SetFigFont{14}{16.8}{\rmdefault}{\mddefault}{\updefault}{\color[rgb]{0,0,0}}}}}}
\put(8851,-9286){\makebox(0,0)[lb]{\smash{{\SetFigFont{14}{16.8}{\rmdefault}{\mddefault}{\updefault}{\color[rgb]{0,0,0}}}}}}
\put(12451,-11086){\makebox(0,0)[lb]{\smash{{\SetFigFont{14}{16.8}{\rmdefault}{\mddefault}{\updefault}{\color[rgb]{0,0,0}}}}}}
\put(14851,-11686){\makebox(0,0)[lb]{\smash{{\SetFigFont{14}{16.8}{\rmdefault}{\mddefault}{\updefault}{\color[rgb]{0,0,0}}}}}}
\put(15001,-10036){\makebox(0,0)[lb]{\smash{{\SetFigFont{14}{16.8}{\rmdefault}{\mddefault}{\updefault}{\color[rgb]{0,0,0}}}}}}
\put(10726,-9436){\makebox(0,0)[lb]{\smash{{\SetFigFont{14}{16.8}{\rmdefault}{\mddefault}{\updefault}{\color[rgb]{0,0,0}}}}}}
\put(4726,-7636){\makebox(0,0)[lb]{\smash{{\SetFigFont{14}{16.8}{\rmdefault}{\mddefault}{\updefault}{\color[rgb]{0,0,0}}}}}}
\put(15001,-8536){\makebox(0,0)[lb]{\smash{{\SetFigFont{14}{16.8}{\rmdefault}{\mddefault}{\updefault}{\color[rgb]{0,0,0}}}}}}
\put(4501,-7036){\makebox(0,0)[lb]{\smash{{\SetFigFont{14}{16.8}{\rmdefault}{\mddefault}{\updefault}{\color[rgb]{0,0,0}}}}}}
\put(7051,-8086){\makebox(0,0)[lb]{\smash{{\SetFigFont{14}{16.8}{\rmdefault}{\mddefault}{\updefault}{\color[rgb]{0,0,0}}}}}}
\put(13651,-7786){\makebox(0,0)[lb]{\smash{{\SetFigFont{14}{16.8}{\rmdefault}{\mddefault}{\updefault}{\color[rgb]{0,0,0}}}}}}
\put(4726,-6436){\makebox(0,0)[lb]{\smash{{\SetFigFont{14}{16.8}{\rmdefault}{\mddefault}{\updefault}{\color[rgb]{0,0,0}}}}}}
\put(8626,-7336){\makebox(0,0)[lb]{\smash{{\SetFigFont{14}{16.8}{\rmdefault}{\mddefault}{\updefault}{\color[rgb]{0,0,0}}}}}}
\put(4426,-8836){\makebox(0,0)[lb]{\smash{{\SetFigFont{14}{16.8}{\rmdefault}{\mddefault}{\updefault}{\color[rgb]{0,0,0}}}}}}
\put(4651,-9436){\makebox(0,0)[lb]{\smash{{\SetFigFont{14}{16.8}{\rmdefault}{\mddefault}{\updefault}{\color[rgb]{0,0,0}}}}}}
\put(8581,-6286){\makebox(0,0)[lb]{\smash{{\SetFigFont{14}{16.8}{\rmdefault}{\mddefault}{\updefault}{\color[rgb]{0,0,0}}}}}}
\put(8506,-6586){\makebox(0,0)[lb]{\smash{{\SetFigFont{14}{16.8}{\rmdefault}{\mddefault}{\updefault}{\color[rgb]{0,0,0}}}}}}
\put(8881,-6586){\makebox(0,0)[lb]{\smash{{\SetFigFont{14}{16.8}{\rmdefault}{\mddefault}{\updefault}{\color[rgb]{0,0,0}}}}}}
\put(12601,-9136){\makebox(0,0)[lb]{\smash{{\SetFigFont{14}{16.8}{\rmdefault}{\mddefault}{\updefault}{\color[rgb]{0,0,0}}}}}}
\put(14626,-6436){\makebox(0,0)[lb]{\smash{{\SetFigFont{14}{16.8}{\rmdefault}{\mddefault}{\updefault}{\color[rgb]{0,0,0}}}}}}
\put(11731,-12991){\makebox(0,0)[lb]{\smash{{\SetFigFont{14}{16.8}{\rmdefault}{\mddefault}{\updefault}{\color[rgb]{0,0,0}}}}}}
\put(6451,-14386){\makebox(0,0)[lb]{\smash{{\SetFigFont{14}{16.8}{\rmdefault}{\mddefault}{\updefault}{\color[rgb]{0,0,0}}}}}}
\put(9256,-13516){\makebox(0,0)[lb]{\smash{{\SetFigFont{14}{16.8}{\rmdefault}{\mddefault}{\updefault}{\color[rgb]{0,0,0}}}}}}
\put(7606,-13516){\makebox(0,0)[lb]{\smash{{\SetFigFont{14}{16.8}{\rmdefault}{\mddefault}{\updefault}{\color[rgb]{0,0,0}}}}}}
\put(7981,-13291){\makebox(0,0)[lb]{\smash{{\SetFigFont{14}{16.8}{\rmdefault}{\mddefault}{\updefault}{\color[rgb]{0,0,0}}}}}}
\put(10906,-13516){\makebox(0,0)[lb]{\smash{{\SetFigFont{14}{16.8}{\rmdefault}{\mddefault}{\updefault}{\color[rgb]{0,0,0}}}}}}
\put(9256,-15616){\makebox(0,0)[lb]{\smash{{\SetFigFont{14}{16.8}{\rmdefault}{\mddefault}{\updefault}{\color[rgb]{0,0,0}}}}}}
\put(9331,-11941){\makebox(0,0)[lb]{\smash{{\SetFigFont{14}{16.8}{\rmdefault}{\mddefault}{\updefault}{\color[rgb]{0,0,0}}}}}}
\put(8881,-12166){\makebox(0,0)[lb]{\smash{{\SetFigFont{14}{16.8}{\rmdefault}{\mddefault}{\updefault}{\color[rgb]{0,0,0}}}}}}
\put(9256,-14641){\makebox(0,0)[lb]{\smash{{\SetFigFont{14}{16.8}{\rmdefault}{\mddefault}{\updefault}{\color[rgb]{0,0,0}}}}}}
\put(8431,-13066){\makebox(0,0)[lb]{\smash{{\SetFigFont{14}{16.8}{\rmdefault}{\mddefault}{\updefault}{\color[rgb]{0,0,0}}}}}}
\put(8806,-15841){\makebox(0,0)[lb]{\smash{{\SetFigFont{14}{16.8}{\rmdefault}{\mddefault}{\updefault}{\color[rgb]{0,0,0}}}}}}
\put(7081,-14116){\makebox(0,0)[lb]{\smash{{\SetFigFont{14}{16.8}{\rmdefault}{\mddefault}{\updefault}{\color[rgb]{0,0,0}}}}}}
\put(11656,-14266){\makebox(0,0)[lb]{\smash{{\SetFigFont{14}{16.8}{\rmdefault}{\mddefault}{\updefault}{\color[rgb]{0,0,0}}}}}}
\put(11881,-14041){\makebox(0,0)[lb]{\smash{{\SetFigFont{14}{16.8}{\rmdefault}{\mddefault}{\updefault}{\color[rgb]{0,0,0}}}}}}
\put(9631,-13891){\makebox(0,0)[lb]{\smash{{\SetFigFont{14}{16.8}{\rmdefault}{\mddefault}{\updefault}{\color[rgb]{0,0,0}}}}}}
\put(9406,-14116){\makebox(0,0)[lb]{\smash{{\SetFigFont{14}{16.8}{\rmdefault}{\mddefault}{\updefault}{\color[rgb]{0,0,0}}}}}}
\put(10531,-14416){\makebox(0,0)[lb]{\smash{{\SetFigFont{14}{16.8}{\rmdefault}{\mddefault}{\updefault}{\color[rgb]{0,0,0}}}}}}
\put(12181,-12766){\makebox(0,0)[lb]{\smash{{\SetFigFont{14}{16.8}{\rmdefault}{\mddefault}{\updefault}{\color[rgb]{0,0,0}}}}}}
\put(10156,-14641){\makebox(0,0)[lb]{\smash{{\SetFigFont{14}{16.8}{\rmdefault}{\mddefault}{\updefault}{\color[rgb]{0,0,0}}}}}}
\put(9376,-13111){\makebox(0,0)[lb]{\smash{{\SetFigFont{14}{16.8}{\rmdefault}{\mddefault}{\updefault}{\color[rgb]{0,0,0}}}}}}
\put(9976,-13336){\makebox(0,0)[lb]{\smash{{\SetFigFont{14}{16.8}{\rmdefault}{\mddefault}{\updefault}{\color[rgb]{0,0,0}}}}}}
\put(9076,-16336){\makebox(0,0)[lb]{\smash{{\SetFigFont{14}{16.8}{\rmdefault}{\mddefault}{\updefault}{\color[rgb]{0,0,0}}}}}}
\put(9076,-10786){\makebox(0,0)[lb]{\smash{{\SetFigFont{14}{16.8}{\rmdefault}{\mddefault}{\updefault}{\color[rgb]{0,0,0}}}}}}
\put(12751,-4711){\makebox(0,0)[lb]{\smash{{\SetFigFont{20}{24.0}{\rmdefault}{\bfdefault}{\updefault}{\color[rgb]{0,0,0}}}}}}
\put(12601,-1636){\makebox(0,0)[lb]{\smash{{\SetFigFont{14}{16.8}{\rmdefault}{\mddefault}{\updefault}{\color[rgb]{0,0,0}}}}}}
\put(13276,-1636){\makebox(0,0)[lb]{\smash{{\SetFigFont{20}{24.0}{\rmdefault}{\bfdefault}{\updefault}{\color[rgb]{0,0,0}}}}}}
\put(7351,-5911){\makebox(0,0)[lb]{\smash{{\SetFigFont{20}{24.0}{\rmdefault}{\bfdefault}{\updefault}{\color[rgb]{0,0,0}}}}}}
\end{picture} }
		\caption{(a) The original ; (b) The additional part of  with cycles of lengths  and }
		\label{fig:complex2}
\end{figure}
\end{example}

As is known, a TA may be totally non-periodic in the sense that no single timed trace of it is eventually periodic (see Example~\ref{ex:non-period}).
However, a special kind of periodicity, which we call \emph{suffix-periodicity}, holds between different timed traces, as shown in the following theorem.
\begin{theorem}
\label{th:lang_eventual_period}
If  is not bounded in time then its language  is suffix-periodic:
if  and

is an observable timed trace of  then, for each , if  then there exists an observable timed trace
 such that 
\end{theorem}
\begin{proof}
Suppose that  is the observable timed trace of some run  of .
This run corresponds to a path in  whose -th transition reaches a vertex  with some time-region  with . 
By Theorem~\ref{th:eventual_period} there exists a path  in  which reaches a vertex .
That is, if  then  is identical to  except for the integral part of , which is increased by : 
, or, in other words, the time-region  of  is a translate by  of the time-region  of .
Hence, since  then .
As we saw in Section~\ref{sec:sing_path}, the trail of the path  (the union of the trajectories along ) is composed of regions in the form of simplices.
Thus, for every value of the time-region , in particular for ,   there exists a run  of  which reaches location  at the exact time  on an observable action .
From that time on, the run  can imitate the behavior of  by keeping a time difference  in the taken transitions.
The result then follows.
\end{proof}
\section{Periodic Augmented Region Automaton}
\label{sec:APARA}
After revealing the periodic structure of , it is natural to fold it into a finite graph according to this period, which we call periodic augmented region automaton and denote by .
The construction of  is done by first taking the subgraph of  of time  and then folding the infinite subgraph of  of time  onto the subgraph of time , which becomes the periodic subgraph.
Thus, each vertex of the periodic subgraph represents infinitely-many vertices of .
Similarly, the out-going edges of the periodic subgraph are periodic edges.
In addition, some of the edges of  are marked with () or (), as explained below.
For an edge , we denote by  and  the initial, resp. terminal, vertex of . 
\begin{definition}[Periodic augmented region automaton]
	\label{def:per_aug_region_automaton}
	Given an infinite augmented region automaton  with period  and periodicity starting time , a finite projection  of it, called \emph{periodic augmented region automaton} and denoted , is a tuple , where:
	\begin{enumerate}
		\item  is the set of vertices, with  the initial vertex.
		For each , if  then  equals  in all fields, except possibly for the integral part of .
If  then  and  is a \emph{regular} vertex.
		Otherwise,  is a \emph{periodic} vertex,  is written as , for some ,  is infinite and .
\item  is the set of edges, which are the projected edges of  under the map .
		Each edge joining two vertices of  is mapped to an edge with the same action label that joins the projected vertices. 
		Some of the edges are marked with a symbol of .
		The description below is technical and refers to the different types of edges that occur when folding : whether the source of the edge is a regular \upshape{(\textbf{R})} or a periodic \upshape{(\textbf{P})} vertex (in the latter case the preimage in  contains infinitely-many edges, one from each of the preimage vertices), whether it is unmarked \upshape{(\textbf{U})} or marked \upshape{(\textbf{M})} (in the latter case there are infinitely-many edges starting from each of the vertices in the preimage source vertices), and finally the plus sign (\textbf{+}) represents the case where in the preimage the target vertices are not of value  but .
\begin{itemize}
			\item \textnormal{\textbf{UR :}} (unmarked, regular) If  is unmarked and  is regular then ,  or    and , with  and .
\item \textnormal{\textbf{UP :}} (unmarked, periodic) If  is unmarked and  is periodic then , ,  and the preimage of  in  are the infinitely-many edges satisfying the following.
If  then , 
and if  then .
\item \textnormal{\textbf{MR :}} (marked, regular) If  is marked with  and  is regular, with  and  or   , then , that is, infinitely-many edges starting from the same vertex.
			\item \textnormal{\textbf{MP :}} (marked, periodic) If  is marked with  and  is periodic, with  and , then its preimage in  contains all the edges according to both rules \textnormal{\textbf{UP}} and \textnormal{\textbf{MR}}. 
		\item \textnormal{\textbf{MP+ :}} (marked, periodic, shifted) If  is marked with 
		then the same rules that apply to an edge marked with  hold, except that the target vertices are of -shift in time compared to those of an edge marked with . 
		\end{itemize}
		\item  is the finite set of actions.
\end{enumerate}
\end{definition}
We remark that instead of periodic time interval of type  we can define it analogously to be of type  as in Fig.~\ref{fig:APTA_a}(d), where the periodic time is .
\begin{example}
	\label{ex:non-period}
	The TA shown in Fig.~\ref{fig:ta3}(a) is taken from \cite{ta}, where it demonstrates non-periodicity: the time difference between an -transition and the following -transition is strictly decreasing along a run.
However, the periodicity among the collection of timed traces is seen in the periodic augmented region automaton, where the period here is of size 1, and the vertices in times  and  are periodic.
	Notice also that there are edges marked with () which represent infinitely-many edges with the same source.
\begin{figure}[htb]
\centering
		\scalebox{0.5}{ \begin{picture}(0,0)\includegraphics{ta3.pdf}\end{picture}\setlength{\unitlength}{3947sp}\begingroup\makeatletter\ifx\SetFigFont\undefined \gdef\SetFigFont#1#2#3#4#5{\reset@font\fontsize{#1}{#2pt}\fontfamily{#3}\fontseries{#4}\fontshape{#5}\selectfont}\fi\endgroup \begin{picture}(11652,10875)(6961,-4216)
\put(7126,4439){\makebox(0,0)[lb]{\smash{{\SetFigFont{20}{24.0}{\rmdefault}{\mddefault}{\updefault}{\color[rgb]{0,0,0}}}}}}
\put(10951,5789){\makebox(0,0)[lb]{\smash{{\SetFigFont{14}{16.8}{\rmdefault}{\mddefault}{\updefault}{\color[rgb]{0,0,0}}}}}}
\put(9301,5789){\makebox(0,0)[lb]{\smash{{\SetFigFont{14}{16.8}{\rmdefault}{\mddefault}{\updefault}{\color[rgb]{0,0,0}}}}}}
\put(12601,5789){\makebox(0,0)[lb]{\smash{{\SetFigFont{14}{16.8}{\rmdefault}{\mddefault}{\updefault}{\color[rgb]{0,0,0}}}}}}
\put(14251,5789){\makebox(0,0)[lb]{\smash{{\SetFigFont{14}{16.8}{\rmdefault}{\mddefault}{\updefault}{\color[rgb]{0,0,0}}}}}}
\put(12976,6239){\makebox(0,0)[lb]{\smash{{\SetFigFont{14}{16.8}{\rmdefault}{\mddefault}{\updefault}{\color[rgb]{0,0,0}}}}}}
\put(10051,6239){\makebox(0,0)[lb]{\smash{{\SetFigFont{14}{16.8}{\rmdefault}{\mddefault}{\updefault}{\color[rgb]{0,0,0}}}}}}
\put(11776,6239){\makebox(0,0)[lb]{\smash{{\SetFigFont{14}{16.8}{\rmdefault}{\mddefault}{\updefault}{\color[rgb]{0,0,0}}}}}}
\put(13351,6464){\makebox(0,0)[lb]{\smash{{\SetFigFont{14}{16.8}{\rmdefault}{\mddefault}{\updefault}{\color[rgb]{0,0,0}}}}}}
\put(9676,6014){\makebox(0,0)[lb]{\smash{{\SetFigFont{14}{16.8}{\rmdefault}{\mddefault}{\updefault}{\color[rgb]{0,0,0}}}}}}
\put(13501,5339){\makebox(0,0)[lb]{\smash{{\SetFigFont{14}{16.8}{\rmdefault}{\mddefault}{\updefault}{\color[rgb]{0,0,0}}}}}}
\put(12526,5114){\makebox(0,0)[lb]{\smash{{\SetFigFont{14}{16.8}{\rmdefault}{\mddefault}{\updefault}{\color[rgb]{0,0,0}}}}}}
\put(11626,6014){\makebox(0,0)[lb]{\smash{{\SetFigFont{14}{16.8}{\rmdefault}{\mddefault}{\updefault}{\color[rgb]{0,0,0}}}}}}
\put(11551,5264){\makebox(0,0)[lb]{\smash{{\SetFigFont{17}{20.4}{\rmdefault}{\mddefault}{\updefault}{\color[rgb]{0,0,0}}}}}}
\put(9451,-1036){\makebox(0,0)[lb]{\smash{{\SetFigFont{14}{16.8}{\rmdefault}{\mddefault}{\updefault}{\color[rgb]{0,0,0}}}}}}
\put(9451,1364){\makebox(0,0)[lb]{\smash{{\SetFigFont{14}{16.8}{\rmdefault}{\mddefault}{\updefault}{\color[rgb]{0,0,0}}}}}}
\put(9451,3764){\makebox(0,0)[lb]{\smash{{\SetFigFont{14}{16.8}{\rmdefault}{\mddefault}{\updefault}{\color[rgb]{0,0,0}}}}}}
\put(9826,3614){\makebox(0,0)[lb]{\smash{{\SetFigFont{14}{16.8}{\rmdefault}{\mddefault}{\updefault}{\color[rgb]{0,0,0}}}}}}
\put(7126,-3436){\makebox(0,0)[lb]{\smash{{\SetFigFont{14}{16.8}{\rmdefault}{\mddefault}{\updefault}{\color[rgb]{0,0,0}}}}}}
\put(9451,-3436){\makebox(0,0)[lb]{\smash{{\SetFigFont{14}{16.8}{\rmdefault}{\mddefault}{\updefault}{\color[rgb]{0,0,0}}}}}}
\put(9451,164){\makebox(0,0)[lb]{\smash{{\SetFigFont{14}{16.8}{\rmdefault}{\mddefault}{\updefault}{\color[rgb]{0,0,0}}}}}}
\put(9151,-436){\makebox(0,0)[lb]{\smash{{\SetFigFont{14}{16.8}{\rmdefault}{\mddefault}{\updefault}{\color[rgb]{0,0,0}}}}}}
\put(9226,764){\makebox(0,0)[lb]{\smash{{\SetFigFont{14}{16.8}{\rmdefault}{\mddefault}{\updefault}{\color[rgb]{0,0,0}}}}}}
\put(9826,1214){\makebox(0,0)[lb]{\smash{{\SetFigFont{14}{16.8}{\rmdefault}{\mddefault}{\updefault}{\color[rgb]{0,0,0}}}}}}
\put(12751,1364){\makebox(0,0)[lb]{\smash{{\SetFigFont{14}{16.8}{\rmdefault}{\mddefault}{\updefault}{\color[rgb]{0,0,0}}}}}}
\put(9826, 14){\makebox(0,0)[lb]{\smash{{\SetFigFont{14}{16.8}{\rmdefault}{\mddefault}{\updefault}{\color[rgb]{0,0,0}}}}}}
\put(10126,-2386){\makebox(0,0)[lb]{\smash{{\SetFigFont{14}{16.8}{\rmdefault}{\mddefault}{\updefault}{\color[rgb]{0,0,0}}}}}}
\put(10126,-2086){\makebox(0,0)[lb]{\smash{{\SetFigFont{14}{16.8}{\rmdefault}{\mddefault}{\updefault}{\color[rgb]{0,0,0}}}}}}
\put(10126,-3286){\makebox(0,0)[lb]{\smash{{\SetFigFont{14}{16.8}{\rmdefault}{\mddefault}{\updefault}{\color[rgb]{0,0,0}}}}}}
\put(10351,1514){\makebox(0,0)[lb]{\smash{{\SetFigFont{14}{16.8}{\rmdefault}{\mddefault}{\updefault}{\color[rgb]{0,0,0}}}}}}
\put(10351,314){\makebox(0,0)[lb]{\smash{{\SetFigFont{14}{16.8}{\rmdefault}{\mddefault}{\updefault}{\color[rgb]{0,0,0}}}}}}
\put(10351,-886){\makebox(0,0)[lb]{\smash{{\SetFigFont{14}{16.8}{\rmdefault}{\mddefault}{\updefault}{\color[rgb]{0,0,0}}}}}}
\put(10351,3914){\makebox(0,0)[lb]{\smash{{\SetFigFont{14}{16.8}{\rmdefault}{\mddefault}{\updefault}{\color[rgb]{0,0,0}}}}}}
\put(10126,-3586){\makebox(0,0)[lb]{\smash{{\SetFigFont{14}{16.8}{\rmdefault}{\mddefault}{\updefault}{\color[rgb]{0,0,0}}}}}}
\put(13126,1214){\makebox(0,0)[lb]{\smash{{\SetFigFont{14}{16.8}{\rmdefault}{\mddefault}{\updefault}{\color[rgb]{0,0,0}}}}}}
\put(13651,1514){\makebox(0,0)[lb]{\smash{{\SetFigFont{14}{16.8}{\rmdefault}{\mddefault}{\updefault}{\color[rgb]{0,0,0}}}}}}
\put(12226,1514){\makebox(0,0)[lb]{\smash{{\SetFigFont{14}{16.8}{\rmdefault}{\mddefault}{\updefault}{\color[rgb]{0,0,0}}}}}}
\put(12751,-1036){\makebox(0,0)[lb]{\smash{{\SetFigFont{14}{16.8}{\rmdefault}{\mddefault}{\updefault}{\color[rgb]{0,0,0}}}}}}
\put(16051,-1036){\makebox(0,0)[lb]{\smash{{\SetFigFont{14}{16.8}{\rmdefault}{\mddefault}{\updefault}{\color[rgb]{0,0,0}}}}}}
\put(12751,-3436){\makebox(0,0)[lb]{\smash{{\SetFigFont{14}{16.8}{\rmdefault}{\mddefault}{\updefault}{\color[rgb]{0,0,0}}}}}}
\put(12751,-2236){\makebox(0,0)[lb]{\smash{{\SetFigFont{14}{16.8}{\rmdefault}{\mddefault}{\updefault}{\color[rgb]{0,0,0}}}}}}
\put(11926,-4111){\makebox(0,0)[lb]{\smash{{\SetFigFont{17}{20.4}{\rmdefault}{\mddefault}{\updefault}{\color[rgb]{0,0,0}}}}}}
\put(13651,-886){\makebox(0,0)[lb]{\smash{{\SetFigFont{14}{16.8}{\rmdefault}{\mddefault}{\updefault}{\color[rgb]{0,0,0}}}}}}
\put(16951,-886){\makebox(0,0)[lb]{\smash{{\SetFigFont{14}{16.8}{\rmdefault}{\mddefault}{\updefault}{\color[rgb]{0,0,0}}}}}}
\put(13426,-3286){\makebox(0,0)[lb]{\smash{{\SetFigFont{14}{16.8}{\rmdefault}{\mddefault}{\updefault}{\color[rgb]{0,0,0}}}}}}
\put(13426,-2086){\makebox(0,0)[lb]{\smash{{\SetFigFont{14}{16.8}{\rmdefault}{\mddefault}{\updefault}{\color[rgb]{0,0,0}}}}}}
\put(13126,-3586){\makebox(0,0)[lb]{\smash{{\SetFigFont{14}{16.8}{\rmdefault}{\mddefault}{\updefault}{\color[rgb]{0,0,0}}}}}}
\put(13126,-2386){\makebox(0,0)[lb]{\smash{{\SetFigFont{14}{16.8}{\rmdefault}{\mddefault}{\updefault}{\color[rgb]{0,0,0}}}}}}
\put(9826,-1186){\makebox(0,0)[lb]{\smash{{\SetFigFont{14}{16.8}{\rmdefault}{\mddefault}{\updefault}{\color[rgb]{0,0,0}}}}}}
\put(13126,-1186){\makebox(0,0)[lb]{\smash{{\SetFigFont{14}{16.8}{\rmdefault}{\mddefault}{\updefault}{\color[rgb]{0,0,0}}}}}}
\put(16426,-1186){\makebox(0,0)[lb]{\smash{{\SetFigFont{14}{16.8}{\rmdefault}{\mddefault}{\updefault}{\color[rgb]{0,0,0}}}}}}
\put(8701,-2911){\makebox(0,0)[lb]{\smash{{\SetFigFont{14}{16.8}{\rmdefault}{\mddefault}{\updefault}{\color[rgb]{0,0,0}}}}}}
\put(8776,-1711){\makebox(0,0)[lb]{\smash{{\SetFigFont{14}{16.8}{\rmdefault}{\mddefault}{\updefault}{\color[rgb]{0,0,0}}}}}}
\put(9151,-2911){\makebox(0,0)[lb]{\smash{{\SetFigFont{14}{16.8}{\rmdefault}{\mddefault}{\updefault}{\color[rgb]{0,0,0}}}}}}
\put(9451,-2236){\makebox(0,0)[lb]{\smash{{\SetFigFont{14}{16.8}{\rmdefault}{\mddefault}{\updefault}{\color[rgb]{0,0,0}}}}}}
\put(9151,-1711){\makebox(0,0)[lb]{\smash{{\SetFigFont{14}{16.8}{\rmdefault}{\mddefault}{\updefault}{\color[rgb]{0,0,0}}}}}}
\put(10726,2564){\makebox(0,0)[lb]{\smash{{\SetFigFont{14}{16.8}{\rmdefault}{\mddefault}{\updefault}{\color[rgb]{0,0,0}}}}}}
\put(13426,-2836){\makebox(0,0)[lb]{\smash{{\SetFigFont{14}{16.8}{\rmdefault}{\mddefault}{\updefault}{\color[rgb]{0,0,0}}}}}}
\put(14326,-2836){\makebox(0,0)[lb]{\smash{{\SetFigFont{14}{16.8}{\rmdefault}{\mddefault}{\updefault}{\color[rgb]{0,0,0}}}}}}
\put(14026,-1636){\makebox(0,0)[lb]{\smash{{\SetFigFont{14}{16.8}{\rmdefault}{\mddefault}{\updefault}{\color[rgb]{0,0,0}}}}}}
\put(16726,-1936){\makebox(0,0)[lb]{\smash{{\SetFigFont{14}{16.8}{\rmdefault}{\mddefault}{\updefault}{\color[rgb]{0,0,0}}}}}}
\put(14026, 89){\makebox(0,0)[lb]{\smash{{\SetFigFont{14}{16.8}{\rmdefault}{\mddefault}{\updefault}{\color[rgb]{0,0,0}}}}}}
\put(14626,314){\makebox(0,0)[lb]{\smash{{\SetFigFont{14}{16.8}{\rmdefault}{\mddefault}{\updefault}{\color[rgb]{0,0,0}}}}}}
\put(12226,-886){\makebox(0,0)[lb]{\smash{{\SetFigFont{14}{16.8}{\rmdefault}{\mddefault}{\updefault}{\color[rgb]{0,0,0}}}}}}
\put(6976,2564){\makebox(0,0)[lb]{\smash{{\SetFigFont{14}{16.8}{\rmdefault}{\mddefault}{\updefault}{\color[rgb]{0,0,0}}}}}}
\put(6976,164){\makebox(0,0)[lb]{\smash{{\SetFigFont{14}{16.8}{\rmdefault}{\mddefault}{\updefault}{\color[rgb]{0,0,0}}}}}}
\put(6976,-2236){\makebox(0,0)[lb]{\smash{{\SetFigFont{14}{16.8}{\rmdefault}{\mddefault}{\updefault}{\color[rgb]{0,0,0}}}}}}
\put(7126,1364){\makebox(0,0)[lb]{\smash{{\SetFigFont{14}{16.8}{\rmdefault}{\mddefault}{\updefault}{\color[rgb]{0,0,0}}}}}}
\put(7126,-1036){\makebox(0,0)[lb]{\smash{{\SetFigFont{14}{16.8}{\rmdefault}{\mddefault}{\updefault}{\color[rgb]{0,0,0}}}}}}
\put(7126,3764){\makebox(0,0)[lb]{\smash{{\SetFigFont{14}{16.8}{\rmdefault}{\mddefault}{\updefault}{\color[rgb]{0,0,0}}}}}}
\end{picture} }
		\caption{a)  ; b) , a periodic augmented region automaton of }
		\label{fig:ta3}
\end{figure}
\end{example} 
\begin{proposition}
	\label{prop:well_def}
	 is well-defined and as informative as .
\end{proposition}
\begin{proof}
	Clearly, since  may be obtained from  then it cannot be more informative.
	It suffices then to show that for each positive integer ,  can be effectively constructed from  up to time .
	Well, for ,  is identical to .
	Then, by Theorem~\ref{th:eventual_period}, the graph of  becomes periodic in the sense that the subgraph of time  repeats itself, except for the integral part of , which progresses indefinitely in  but can be expressed modulo the period , as is done in .
	Indeed, since the transitions in  do not rely on , by taking the quotient of  modulo  from time , the only loss of information is the exact time difference in  between the target and source regions.
	But due to the periodicity in , this information can be finitely presented.
	Hence, since the edges of  whose initial vertices are of time  are translates of similar edges that start at time , it suffices to examine the latter.
		
	So, let  be an edge of  which joins a vertex  of integral time , , with a vertex  of integral time 
	, and suppose that .
	Suppose also that  is not joined to a vertex . 
	Then, since  there are only two cases: either  or .
	In order to distinguish between these cases, the latter case is marked by a plus sign that is added to the corresponding edge of  from a vertex of integral time  to a vertex of integral time .
	When  is also connected to a vertex  then we let  be of minimal integral time modulo  to which  is connected, that is , for some , and there is no edge from  to  (here  ans  are identical except for the integral time of ).
	If  is of integral time  then necessarily  is connected to infinitely-many vertices of integral time , , and all these edges are captured in  by marking with  the edge from the corresponding vertex of integral time  to a vertex of integral time . 
	
	The case where  is handled similarly.   
	 \iffalse	
	So, let  be a directed edge of  joining a vertex of integral time , to a vertex of integral time .
	Suppose also that  then there are then two possibilities.
	Either the edge represents a minimal time difference  or a minimal time difference .
	In order to distinguish between these two cases we mark the edge in the latter case with a plus sign.
	Since the period  satisfies  then if the time difference  , for some  then necessarily there exists also a similar edge with time difference , thus the minimal time difference cannot be , .
	However, we need also distinguish between the case where there is only one possible time difference  or there are infinitely-many time differences , , and in the latter case we mark the edge with .
	
	Similarly, when  then either the minimal time difference is  or a minimal time difference , and in the latter case we mark the edge with a plus sign.
	Again, we add also the  mark to represent the infinitely-many time differences , . 
	\fi
	It is now clear that in order to construct  up to time  we only need to unfold  up to this time by obeying the above rules.
\end{proof}
\subsection{Complexity}
Let  denote the number of vertices in the augmented region automaton .
If  denotes the number of clocks, including the absolute clock ,  the number of locations in  and , where  is the maximal integer appearing in a guard of , then

Indeed, the number of combinations of the integral values of the clocks is bounded by  (in fact,  is assigned a single value), there are  different orderings of the fractional parts of the clocks , and the term  refers to all possibilities of inequality or equality between each pair of adjacent  in an ordering.

Let us look now at the number of vertices in .
At each time-level the number of vertices is bounded by .
Since  then there are at most  vertices of time .
After passing  we have the subgraphs  of time length , where  is the period.
Each such subgraph has at most  vertices.
Since the number of vertices in the subgraphs forms an almost increasing sequence (until an equality occurs two consecutive times), the number of vertices from time  to time  is bounded by . 
Thus, the number  of vertices in  satisfies

as .

The largest factor in \eqref{eq:nr_vertices} may come from the period , so let us compute an upper bound of .
 is the least common multiple of the durations  of cycles that form a covering set of non-Zeno cycles.
For each such cycle ,  since  the length of a simple cycle is bounded by the number  of vertices in  and the time difference between two vertices along a path is at most .
Thus, a bound on  is given by the least common multiple of , which is by the prime number theorem

as .

\begin{example}
	When computing the period , in the worst case the numbers  are pairwise prime and the vertices of the cycles  form a disjoint union of sets which (almost) covers the set of vertices of .
	So, suppose that  is in the form of  simple cycles, where each cycle is connected to the initial vertex by an additional edge.
	Suppose also that the length of cycle  is , the -th prime number, .
	Let us assume that  and each edge is of weight 1.
	The number of vertices in  is .
Then , the primorial .
	This upper bound is closer to  than to the bound  of \eqref{eq:L_bound}. 
\end{example}
\section{The Timestamp}
\label{sec:timestamp}
Recall that the timestamp  of a timed automaton  is the set of all pairs , such that an observable transition with action  occurs at time  in some run of .
\begin{theorem}
\label{th:timestamp_eventual_period}
The timestamp of a TA  is a union of action-labeled integral points and open unit intervals with integral end-points.
It is either finite or forms an eventually periodic (with respect to time ) subset of  and is effectively computable.
\end{theorem}
\begin{proof}
	By Theorem~\ref{th:lang_eventual_period}, if the timestamp is not finite then it becomes periodic, with period , after time .
	Thus, if it can effectively be computed up to time , then in order to find whether there is an observable transition with action  at time  one only needs to check the timestamp at time .
	
	By Proposition~\ref{pr:mult_ev_timestamp}, the timestamp up to time  is a finite number of labeled integral points and open intervals between integral points and by Proposition~\ref{prop:finite_IARA}, it is effectively computable.	
\end{proof}


The timestamp of a TA is an abstraction of its language: it does not preserve the timestamps of single timed traces.
However, the timestamp is eventually periodic and computable, hence the timestamp inclusion problem is decidable.
Thus, due to the general undecidability of the language inclusion problem in non-deterministic timed automata, one may use the timestamp for refutation purpose.
\begin{corollary}
Given two timed automata  over the same alphabet (action labels), the question of non-inclusion of their timestamps is decidable, 
thus providing a decidable sufficient condition for the (in general, undecidable) question of non-inclusion of their languages: .
\end{corollary}

The timestamp is easily extracted from  (in fact, it is enough to take the subgraph of  up to level ).
We just form the union of the time-regions up to level , where each time-region is either a point  or an open interval , along with the labels of the actions of the in-going edges.
The timestamp in the interval  then repeats itself indefinitely. 
\begin{definition}
	For each , let  be the restriction of  to -actions, obtained by substituting each  with , representing the silent transition.
\end{definition}
Thus, the language of  is the 'censored' language of , which is the outcome of deleting from each word (timed trace) all pairs , .
\begin{example}
	The timestamp of the -transitions of the automaton of Fig.~\ref{fig:ta3} is , and that of the -transitions is .
\end{example}
\subsection{Timestamp Automaton}
Given a TA , one can effectively construct a deterministic TA , called a \emph{timestamp automaton} of  with the same timestamp as that of .
Such as automaton is decomposable into the timestamp automata of the automata .
\begin{definition}[Timestamp automaton]
	Given a timed automaton , a timestamp automaton  is a deterministic (finite) timed automaton with a single clock and with timestamp identical to that of .
	It is the union of the timestamp automata , , having a common initial vertex.
	Each  has the form of a single path  of positive length, which may end in a loop , thus giving  the form of a bouquet.
\end{definition}
\begin{theorem}
	\label{th:timestamp_contruct}
	Given a timed automaton , one can effectively construct a timestamp automaton .
\end{theorem}
\begin{proof}
We construct  by following the ordered connected components (intervals) of the timestamp  (here 'interval' includes also singletons ).
To each such time interval corresponds the next transition guard in , where the lower and upper constraint on the clock  in the transition guard are exactly the left and right end-points of the interval.
In case  contains a finite number of intervals (possibly the last interval of infinite length) then we are done.

Otherwise,  contains infinitely-many intervals, which form an eventually periodic sequence with respect to the sizes of the intervals and the distances among them. 
Then we need to attach a loop at the end of .
We distinguish between two cases.

Case (i): The periodic part of  contains an integral point  (not necessarily as an isolated point).
Then we first split the interval, say , to which  belongs into disjoint intervals , such that the point  belongs to a singleton. 
Then we extend  until reaching , so that the last transition of  is constrained to  while resetting . 
From that point begins the loop , which obeys the same rules as applied to , with  being reset only when finishing the loop (see Fig.~\ref{fig:ts_autom} (a)).

Case (ii): The periodic part of  does not contain an integral point, that is, it is a union of open unit intervals .
Then, if necessary, we split the last interval before starting the loop into two with the second component a unit open interval (we know that this last interval is not a singleton).
This unit interval refers to the last transition of  and we reset  on that transition. 
Then, all transitions within the loop  are forced to occur at integral times, with  being reset when completing the whole loop (see Fig.~\ref{fig:ts_autom} (b)) (hence, in both cases the clock  is reset in each  only on a transition to the vertex ).
The idea is that if we enter the loop at a fractional time, say , then all the next transitions will take place at times , but since  can be arbitrarily chosen within the open interval  then the set of all runs will cover the entire timestamp.
\end{proof}
\begin{example}
\label{ex:ts_autom}
Let  be a TA with timestamp

Then a possible timestamp automaton of  is given in Fig.~\ref{fig:ts_autom}. 
\begin{figure}[htb]
\centering
\scalebox{0.5}{ \begin{picture}(0,0)\includegraphics{ts_autom.pdf}\end{picture}\setlength{\unitlength}{3947sp}\begingroup\makeatletter\ifx\SetFigFont\undefined \gdef\SetFigFont#1#2#3#4#5{\reset@font\fontsize{#1}{#2pt}\fontfamily{#3}\fontseries{#4}\fontshape{#5}\selectfont}\fi\endgroup \begin{picture}(9920,6113)(1639,884)
\put(4501,5189){\makebox(0,0)[lb]{\smash{{\SetFigFont{17}{20.4}{\rmdefault}{\mddefault}{\updefault}{\color[rgb]{0,0,0}}}}}}
\put(3901,5564){\makebox(0,0)[lb]{\smash{{\SetFigFont{14}{16.8}{\rmdefault}{\mddefault}{\updefault}{\color[rgb]{0,0,0}}}}}}
\put(5551,5564){\makebox(0,0)[lb]{\smash{{\SetFigFont{14}{16.8}{\rmdefault}{\mddefault}{\updefault}{\color[rgb]{0,0,0}}}}}}
\put(7201,5564){\makebox(0,0)[lb]{\smash{{\SetFigFont{14}{16.8}{\rmdefault}{\mddefault}{\updefault}{\color[rgb]{0,0,0}}}}}}
\put(8401,6689){\makebox(0,0)[lb]{\smash{{\SetFigFont{14}{16.8}{\rmdefault}{\mddefault}{\updefault}{\color[rgb]{0,0,0}}}}}}
\put(9601,5564){\makebox(0,0)[lb]{\smash{{\SetFigFont{14}{16.8}{\rmdefault}{\mddefault}{\updefault}{\color[rgb]{0,0,0}}}}}}
\put(4426,5789){\makebox(0,0)[lb]{\smash{{\SetFigFont{14}{16.8}{\rmdefault}{\mddefault}{\updefault}{\color[rgb]{0,0,0}}}}}}
\put(4651,6014){\makebox(0,0)[lb]{\smash{{\SetFigFont{14}{16.8}{\rmdefault}{\mddefault}{\updefault}{\color[rgb]{0,0,0}}}}}}
\put(9076,6239){\makebox(0,0)[lb]{\smash{{\SetFigFont{14}{16.8}{\rmdefault}{\mddefault}{\updefault}{\color[rgb]{0,0,0}}}}}}
\put(9151,4889){\makebox(0,0)[lb]{\smash{{\SetFigFont{14}{16.8}{\rmdefault}{\mddefault}{\updefault}{\color[rgb]{0,0,0}}}}}}
\put(9601,5114){\makebox(0,0)[lb]{\smash{{\SetFigFont{14}{16.8}{\rmdefault}{\mddefault}{\updefault}{\color[rgb]{0,0,0}}}}}}
\put(8401,4439){\makebox(0,0)[lb]{\smash{{\SetFigFont{14}{16.8}{\rmdefault}{\mddefault}{\updefault}{\color[rgb]{0,0,0}}}}}}
\put(6826,6239){\makebox(0,0)[lb]{\smash{{\SetFigFont{14}{16.8}{\rmdefault}{\mddefault}{\updefault}{\color[rgb]{0,0,0}}}}}}
\put(7276,6464){\makebox(0,0)[lb]{\smash{{\SetFigFont{14}{16.8}{\rmdefault}{\mddefault}{\updefault}{\color[rgb]{0,0,0}}}}}}
\put(9301,6464){\makebox(0,0)[lb]{\smash{{\SetFigFont{14}{16.8}{\rmdefault}{\mddefault}{\updefault}{\color[rgb]{0,0,0}}}}}}
\put(5926,5789){\makebox(0,0)[lb]{\smash{{\SetFigFont{14}{16.8}{\rmdefault}{\mddefault}{\updefault}{\color[rgb]{0,0,0}}}}}}
\put(6301,6014){\makebox(0,0)[lb]{\smash{{\SetFigFont{14}{16.8}{\rmdefault}{\mddefault}{\updefault}{\color[rgb]{0,0,0}}}}}}
\put(6751,4889){\makebox(0,0)[lb]{\smash{{\SetFigFont{14}{16.8}{\rmdefault}{\mddefault}{\updefault}{\color[rgb]{0,0,0}}}}}}
\put(7201,5114){\makebox(0,0)[lb]{\smash{{\SetFigFont{14}{16.8}{\rmdefault}{\mddefault}{\updefault}{\color[rgb]{0,0,0}}}}}}
\put(10726,3689){\makebox(0,0)[lb]{\smash{{\SetFigFont{14}{16.8}{\rmdefault}{\mddefault}{\updefault}{\color[rgb]{0,0,0}}}}}}
\put(10801,2339){\makebox(0,0)[lb]{\smash{{\SetFigFont{14}{16.8}{\rmdefault}{\mddefault}{\updefault}{\color[rgb]{0,0,0}}}}}}
\put(7351,3239){\makebox(0,0)[lb]{\smash{{\SetFigFont{14}{16.8}{\rmdefault}{\mddefault}{\updefault}{\color[rgb]{0,0,0}}}}}}
\put(8326,2339){\makebox(0,0)[lb]{\smash{{\SetFigFont{14}{16.8}{\rmdefault}{\mddefault}{\updefault}{\color[rgb]{0,0,0}}}}}}
\put(8776,3014){\makebox(0,0)[lb]{\smash{{\SetFigFont{14}{16.8}{\rmdefault}{\mddefault}{\updefault}{\color[rgb]{0,0,0}}}}}}
\put(9976,4139){\makebox(0,0)[lb]{\smash{{\SetFigFont{14}{16.8}{\rmdefault}{\mddefault}{\updefault}{\color[rgb]{0,0,0}}}}}}
\put(11176,3014){\makebox(0,0)[lb]{\smash{{\SetFigFont{14}{16.8}{\rmdefault}{\mddefault}{\updefault}{\color[rgb]{0,0,0}}}}}}
\put(9976,1889){\makebox(0,0)[lb]{\smash{{\SetFigFont{14}{16.8}{\rmdefault}{\mddefault}{\updefault}{\color[rgb]{0,0,0}}}}}}
\put(7201,3014){\makebox(0,0)[lb]{\smash{{\SetFigFont{14}{16.8}{\rmdefault}{\mddefault}{\updefault}{\color[rgb]{0,0,0}}}}}}
\put(2251,3014){\makebox(0,0)[lb]{\smash{{\SetFigFont{14}{16.8}{\rmdefault}{\mddefault}{\updefault}{\color[rgb]{0,0,0}}}}}}
\put(3901,3014){\makebox(0,0)[lb]{\smash{{\SetFigFont{14}{16.8}{\rmdefault}{\mddefault}{\updefault}{\color[rgb]{0,0,0}}}}}}
\put(5551,3014){\makebox(0,0)[lb]{\smash{{\SetFigFont{14}{16.8}{\rmdefault}{\mddefault}{\updefault}{\color[rgb]{0,0,0}}}}}}
\put(6076,3239){\makebox(0,0)[lb]{\smash{{\SetFigFont{14}{16.8}{\rmdefault}{\mddefault}{\updefault}{\color[rgb]{0,0,0}}}}}}
\put(6301,3464){\makebox(0,0)[lb]{\smash{{\SetFigFont{14}{16.8}{\rmdefault}{\mddefault}{\updefault}{\color[rgb]{0,0,0}}}}}}
\put(4276,3239){\makebox(0,0)[lb]{\smash{{\SetFigFont{14}{16.8}{\rmdefault}{\mddefault}{\updefault}{\color[rgb]{0,0,0}}}}}}
\put(4726,3464){\makebox(0,0)[lb]{\smash{{\SetFigFont{14}{16.8}{\rmdefault}{\mddefault}{\updefault}{\color[rgb]{0,0,0}}}}}}
\put(2626,3239){\makebox(0,0)[lb]{\smash{{\SetFigFont{14}{16.8}{\rmdefault}{\mddefault}{\updefault}{\color[rgb]{0,0,0}}}}}}
\put(3076,3464){\makebox(0,0)[lb]{\smash{{\SetFigFont{14}{16.8}{\rmdefault}{\mddefault}{\updefault}{\color[rgb]{0,0,0}}}}}}
\put(8851,3689){\makebox(0,0)[lb]{\smash{{\SetFigFont{14}{16.8}{\rmdefault}{\mddefault}{\updefault}{\color[rgb]{0,0,0}}}}}}
\put(9076,3914){\makebox(0,0)[lb]{\smash{{\SetFigFont{14}{16.8}{\rmdefault}{\mddefault}{\updefault}{\color[rgb]{0,0,0}}}}}}
\put(10951,3914){\makebox(0,0)[lb]{\smash{{\SetFigFont{14}{16.8}{\rmdefault}{\mddefault}{\updefault}{\color[rgb]{0,0,0}}}}}}
\put(11026,2564){\makebox(0,0)[lb]{\smash{{\SetFigFont{14}{16.8}{\rmdefault}{\mddefault}{\updefault}{\color[rgb]{0,0,0}}}}}}
\put(7876,3464){\makebox(0,0)[lb]{\smash{{\SetFigFont{14}{16.8}{\rmdefault}{\mddefault}{\updefault}{\color[rgb]{0,0,0}}}}}}
\put(8851,2564){\makebox(0,0)[lb]{\smash{{\SetFigFont{14}{16.8}{\rmdefault}{\mddefault}{\updefault}{\color[rgb]{0,0,0}}}}}}
\put(4726,1814){\makebox(0,0)[lb]{\smash{{\SetFigFont{14}{16.8}{\rmdefault}{\mddefault}{\updefault}{\color[rgb]{0,0,0}}}}}}
\put(3826,1364){\makebox(0,0)[lb]{\smash{{\SetFigFont{14}{16.8}{\rmdefault}{\mddefault}{\updefault}{\color[rgb]{0,0,0}}}}}}
\put(5476,1364){\makebox(0,0)[lb]{\smash{{\SetFigFont{14}{16.8}{\rmdefault}{\mddefault}{\updefault}{\color[rgb]{0,0,0}}}}}}
\put(7126,1364){\makebox(0,0)[lb]{\smash{{\SetFigFont{14}{16.8}{\rmdefault}{\mddefault}{\updefault}{\color[rgb]{0,0,0}}}}}}
\put(5851,1589){\makebox(0,0)[lb]{\smash{{\SetFigFont{14}{16.8}{\rmdefault}{\mddefault}{\updefault}{\color[rgb]{0,0,0}}}}}}
\put(6451,1814){\makebox(0,0)[lb]{\smash{{\SetFigFont{14}{16.8}{\rmdefault}{\mddefault}{\updefault}{\color[rgb]{0,0,0}}}}}}
\put(4501,1589){\makebox(0,0)[lb]{\smash{{\SetFigFont{14}{16.8}{\rmdefault}{\mddefault}{\updefault}{\color[rgb]{0,0,0}}}}}}
\put(2026,2039){\makebox(0,0)[lb]{\smash{{\SetFigFont{14}{16.8}{\rmdefault}{\mddefault}{\updefault}{\color[rgb]{0,0,0}}}}}}
\put(2476,2264){\makebox(0,0)[lb]{\smash{{\SetFigFont{14}{16.8}{\rmdefault}{\mddefault}{\updefault}{\color[rgb]{0,0,0}}}}}}
\put(2026,4364){\makebox(0,0)[lb]{\smash{{\SetFigFont{14}{16.8}{\rmdefault}{\mddefault}{\updefault}{\color[rgb]{0,0,0}}}}}}
\put(2476,4589){\makebox(0,0)[lb]{\smash{{\SetFigFont{14}{16.8}{\rmdefault}{\mddefault}{\updefault}{\color[rgb]{0,0,0}}}}}}
\put(4501,2639){\makebox(0,0)[lb]{\smash{{\SetFigFont{17}{20.4}{\rmdefault}{\mddefault}{\updefault}{\color[rgb]{0,0,0}}}}}}
\put(4501,989){\makebox(0,0)[lb]{\smash{{\SetFigFont{17}{20.4}{\rmdefault}{\mddefault}{\updefault}{\color[rgb]{0,0,0}}}}}}
\end{picture} }
\caption{Timestamp automata of a) ; b) ; c) }
\label{fig:ts_autom}
\end{figure}
\end{example} 
\begin{figure}[htb]
\centering
\scalebox{0.5}{ \begin{picture}(0,0)\includegraphics{time_stamp1.pdf}\end{picture}\setlength{\unitlength}{3947sp}\begingroup\makeatletter\ifx\SetFigFont\undefined \gdef\SetFigFont#1#2#3#4#5{\reset@font\fontsize{#1}{#2pt}\fontfamily{#3}\fontseries{#4}\fontshape{#5}\selectfont}\fi\endgroup \begin{picture}(7902,2175)(8389,4634)
\put(13351,6239){\makebox(0,0)[lb]{\smash{{\SetFigFont{14}{16.8}{\rmdefault}{\mddefault}{\updefault}{\color[rgb]{0,0,0}}}}}}
\put(9976,5339){\makebox(0,0)[lb]{\smash{{\SetFigFont{14}{16.8}{\rmdefault}{\mddefault}{\updefault}{\color[rgb]{0,0,0}}}}}}
\put(9001,5864){\makebox(0,0)[lb]{\smash{{\SetFigFont{14}{16.8}{\rmdefault}{\mddefault}{\updefault}{\color[rgb]{0,0,0}}}}}}
\put(10801,5864){\makebox(0,0)[lb]{\smash{{\SetFigFont{14}{16.8}{\rmdefault}{\mddefault}{\updefault}{\color[rgb]{0,0,0}}}}}}
\put(14851,5864){\makebox(0,0)[lb]{\smash{{\SetFigFont{14}{16.8}{\rmdefault}{\mddefault}{\updefault}{\color[rgb]{0,0,0}}}}}}
\put(13201,5864){\makebox(0,0)[lb]{\smash{{\SetFigFont{14}{16.8}{\rmdefault}{\mddefault}{\updefault}{\color[rgb]{0,0,0}}}}}}
\put(14251,5264){\makebox(0,0)[lb]{\smash{{\SetFigFont{17}{20.4}{\rmdefault}{\mddefault}{\updefault}{\color[rgb]{0,0,0}}}}}}
\put(16276,6014){\makebox(0,0)[lb]{\smash{{\SetFigFont{14}{16.8}{\rmdefault}{\mddefault}{\updefault}{\color[rgb]{0,0,0}}}}}}
\put(15901,5789){\makebox(0,0)[lb]{\smash{{\SetFigFont{14}{16.8}{\rmdefault}{\mddefault}{\updefault}{\color[rgb]{0,0,0}}}}}}
\put(9526,6389){\makebox(0,0)[lb]{\smash{{\SetFigFont{14}{16.8}{\rmdefault}{\mddefault}{\updefault}{\color[rgb]{0,0,0}}}}}}
\put(9976,6614){\makebox(0,0)[lb]{\smash{{\SetFigFont{14}{16.8}{\rmdefault}{\mddefault}{\updefault}{\color[rgb]{0,0,0}}}}}}
\put(9601,5114){\makebox(0,0)[lb]{\smash{{\SetFigFont{14}{16.8}{\rmdefault}{\mddefault}{\updefault}{\color[rgb]{0,0,0}}}}}}
\put(9826,4739){\makebox(0,0)[lb]{\smash{{\SetFigFont{17}{20.4}{\rmdefault}{\mddefault}{\updefault}{\color[rgb]{0,0,0}}}}}}
\put(13951,6464){\makebox(0,0)[lb]{\smash{{\SetFigFont{14}{16.8}{\rmdefault}{\mddefault}{\updefault}{\color[rgb]{0,0,0}}}}}}
\end{picture} }
\caption{a) A non-determinizable  ; b) A timestamp automaton }
\label{fig:time_stamp1}
\end{figure} 
\begin{example}
The language of the TA  of Fig.~\ref{fig:time_stamp1} (a) is
 (supposing all locations are `accepting').
The timestamp of  is the set of all positive non-integral reals: .
 is not determinizable. Each transition occurs between the next pair of successive natural numbers.
The guard of each such transition must refer to a clock which was reset on some previous integral time.
But since all transitions occur on non-integral time, the only clock that can be referred to is a clock  that is reset at time  and hence the transition guards need to be of the form  for each , which makes the automaton infinite.
Nevertheless, the timestamp automaton associated with , seen in Fig.~\ref{fig:time_stamp1} (b), is deterministic.
\end{example}
\section{Conclusion and Future Research}
The timestamp of a non-deterministic timed automaton with silent transitions () consists of the set of all action-labeled times at which locations can be reached by observable transitions.
The problem of computing the timestamp is a generalization of the basic reachability problem, a fundamental problem in model checking, thus being of interest from the theoretical as well as from the practical point of view.
In this paper we showed that the timestamp can be effectively computed, also when the timed automata are non-deterministic and include silent transitions.

One of the major problems in testing and verification of abstract models of real-time systems is the inclusion of the language of one timed automaton in the language of another timed automaton.
This problem is, in general, undecidable.
Thus, since (non)-inclusion of timestamps of timed automata is a decidable problem, we have a tool which provides a sufficient condition for language non-inclusion in timed automata. However, the timestamp may be seen as overly abstract since it does not take into account the order in which events occur.
Another property to be considered is complexity.
We did not try to find here an efficient algorithm for the construction of the timestamp, e.g. by replacing regions with time-periodic structures like zones or other symbolic representations \cite{MPS11}
and this can be the subject of possible future research.

\subsection*{Acknowledgements.} 
\begin{small}
	This research was partly supported by the Austrian Science Fund (FWF) Project P29355-N35. \end{small}
\bibliography{ta}
\bibliographystyle{amsplain}
\end{document}
