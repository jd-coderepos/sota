\documentclass[11pt]{llncs}
\usepackage{algorithm2e}
\usepackage{algorithmicx}
\usepackage{epsfig}
\usepackage{fancybox}
\usepackage{graphics}
\usepackage{geometry}




\newcounter{qcounter}


\pagestyle{plain}
\begin{document}
\title{Generalized -skeletons
\thanks{This research is supported by the ESF EUROCORES programme EUROGIGA,
CRP VORONOI.}}
\author{
        Miros{\l}aw Kowaluk  
\and
        Gabriela Majewska  
\institute{
Institute of Informatics, University of Warsaw, Warsaw, Poland,
\texttt{kowaluk@mimum.edu.pl}
\texttt{gm248309@students.mimuw.edu.pl}
}}


\maketitle

\begin{abstract}
-skeletons, a prominent member of the neighborhood graph family, have
interesting geometric properties and various applications ranging from
geographic networks to archeology.
This paper focuses on developing a new, more general than the present one, definition
of -skeletons based only on the distance criterion.
It allows us to consider them in many different cases, e.g. for weighted graphs 
or objects other than points.
Two types of -skeletons are especially well-known: the Gabriel Graph 
(for ) and the Relative Neighborhood Graph (for ).
The new definition retains relations between those graphs and the other well-known ones
(minimum spanning tree and Delaunay triangulation).
We also show several new algorithms finding -skeletons.           
\end{abstract} 


\section{Introduction}

The -skeletons \cite{kr85} in  belong to the family of
proximity graphs, geometric graphs in which two vertices (points) are connected
with an edge if and only if they satisfy particular geometric requirements.
In the case of -skeletons, those requirements are dependent on a given parameter 
. 
 
The -skeletons are both important and popular because of many practical 
applications. The applications span a wide spectrum of areas: 
from geographic information systems to wireless ad hoc networks and machine learning. 
They also facilitate reconstructing  shapes of two-dimension objects from 
sample points, and are also useful in finding the minimum weight triangulation 
of point sets. 
  
{\em Gabriel graphs} (-skeletons), 
defined by Gabriel and Sokal \cite{gs69}, are an example of -skeletons for .


The {\em relative neighborhood graph} (\textit{RNG}) is  another example of the -skeleton
graph family, for . The \textit{RNG} was 
introduced by Toussaint \cite{tou80} in the context of their applications in pattern 
recognition. 

Kirkpatrick and Radke \cite{kr85} proved an important theorem connecting -skeletons 
with the minimum spanning tree  and Delaunay triangulation  of  :

\begin{theorem}
\label{faktinkluzja}
Let us assume that points in  are in general position.
For  following inclusions hold true: 
.
\end{theorem}

According to this theorem, many algorithms computing Gabriel graphs and relative neighborhood graphs
in subquadratic time were created \cite{ms84,su83,jk87,jky89,l94}.

There are very few algorithms for Gabriel graphs and relative neighborhood graphs
in metrics different than the euclidean one (\cite{mg11,w06}). In particular, -skeletons 
in different metrics have not been studied and this paper makes an initial effort to fill this gap.
Our main purpose of this paper is to develop a definition of -skeleton based 
only on the distance criterion.
Moreover, -skeletons defined in a new way fulfill all conditions of Theorem \ref{faktinkluzja}.

Two different forms of -neighborhoods have been studied for 
(see for example  \cite{abe98,e02})   
leading to two different families of -skeletons: lens-based -skeletons 
and circle-based -skeletons. 
The new definition of -skeletons can be used in both cases. 
However, in this work, we focus on the lens-based -skeletons.   

  










The paper is organized as follows.
The definition and basic properties of -skeletons are introduced in Section 2. 
In Section 3 we present a distance based definition of a -skeleton in  metrics. 
In Section 4 we discuss 
a problem of not uniquely defined centers of discs determining regions of a -skeleton. 
Then, we consider -skeletons in weighted graphs. In the next section we describe a case 
when the generators of the regions are not uniquely defined. In Section 7 we formulate 
the definition of the generalized -skeletons. In Section 8 we present some algorithms 
computing generalized -skeletons for cases discussed in the previous sections. 
The last section contains open problems and conclusions.




\section{Preliminaries}

We consider a two-dimensional plane  with the  metric (with distance 
function ), where . 

\begin{definition} \cite{kr85}
\label{betaskeleton}
For a given set of points   in  and parameters 
 and  we define graph 
 -- called a lens-based -skeleton -- as follows:  
two points  are connected with an edge if and only if no point 
from  belongs to the set 
 where:

\begin{enumerate} 
\item 
for ,   is equal to the segment ;
\item 
for ,  is the intersection of two discs in , 
each of them has
radius  and whose boundaries contain both  and ;
\item 
for ,  is the intersection of two  
discs, each with radius , whose centers are in points 
 and 
in , respectively;
\item 
for ,  is the unbounded strip between lines 
perpendicular to the segment  and containing  and .
\end{enumerate}

The region  is called a lens (the name of the skeleton type, used 
in the literature, i.e. lune-based, is descended from the complement of the lens to the disc 
which looks like a lune - see Figure \ref{fig:region}). 
\end{definition}

Furthermore,  we can  consider  open or closed  regions
that lead to {\em an open} or {\em closed -skeletons}. For example,  the
{\em Gabriel graph} is the closed -skeleton and the
{\em relative neighborhood graph} is the open -skeleton.

In a similar way we can define a different family of graphs by changing 
the lens-based definition for . This new family of graphs is called 
circle-based -skeletons. \\ 

\begin{definition} \cite{e02}
For a given set of points   in  and parameters 
 and  we define graph 
 -- called a circle-based -skeleton -- as follows:  
two points  are connected with an edge if and only if no point 
from  belongs to the set 
 where:
\begin{enumerate}
\item
for  we define set  the same way as for lens-based -skeleton;
\item
for  the set  is a union of two discs in , 
each with radius , whose boundaries contain both  and ;
\item
for ,  is a union of the segment  and two open 
hyperplanes defined by the line passing by  and .
\end{enumerate} 
\end{definition}

In this work we generally focus on lens-based -skeletons.\\ 

\begin{figure}[htbp]
\centering
\includegraphics[scale=0.3]{./region-v.eps}
\caption{Lenses of the -skeleton for  (a), the lens-based skeleton 
(b) and the circle-based skeleton (c) for .}
\label{fig:region}
\end{figure}








\section{Distance based definition of -skeletons }
\label{distancebased}



To describe -skeletons for  in the remaining part of this paper 
we will use the definition of Painleve-Kuratowski convergence \cite{k66}: 

\begin{definition}
\label{kuratowskiconv}
Let  be a space with metric ,  be a sequence 
of subsets in  and for any  there is . Then:
\begin{enumerate}
\item
the upper (outer) limit  of  
is defined as a such a set that:\\
;
\item
the lower (inner) limit 
 of  is such a set that:\\
;
The sequence  is said to be convergent in the sense of Painleve-Kuratowski
and denoted as  
if .
\end{enumerate}
\end{definition}







The regions  are uniquely defined in  metric for each .
Therefore  and 
 are also unique and
for any two points  we have 
 and 
.


Let us consider lens-based -skeletons in  metric, where . 
Let  (, respectively) be the center of a disc defining the lens  
and   (, 
respectively). 
It is easy to notice that  
and  .

\begin{lemma}
\label{lemata}
Let  be points in  with a  metric, where ,
 and 
 . 
Then, the points  are centers of discs determining the lens .
\end{lemma}

\begin{proof}
For  we have
.
For  we have
.
Hence, the points  are colinear.
Moreover, from the triangle inequality it follows that
in an arbitrary metric for  is 

and for  is

The same relations occur for . 
Hence  (discs defining the lens
have the radius ), i.e. the lens is .
\end{proof}

\begin{corollary}
\label{lp-distance}
Distance conditions in Definition \ref{betaskeleton} for  and in Lemma \ref{lemata}
for  correctly define lenses of -skeletons for .
The Painleve-Kuratowski convergence correctly defines lenses of -skeletons for  
and . 
\end{corollary}



\section{Not uniquely defined centers of the discs determining lens}
\label{l1-section}

Note that for metrics  and  corollary \ref{lp-distance} is not true,
since centers of discs determining a lens can be defined not uniquely.
Let us assume that points belonging to  are in general position.
For , intersection of circles centered in  and  with
radiuses  and ,
respectively, contains may points. 
If a line passing through  and  is parallel to sides of squares being discs 
in  or  then centers of discs determining a lens for  
are not uniquely defined too.
      

We will concentrate on  metric and describe what should be modified in the definition 
of the -skeleton to be able to use it also in this metric. The consideration concerning
 metric are exactly the same. 

In this case we analyze lens defined by discs with centers satisfying the conditions below.

For  and , the points  and  are centers of discs defining
 when 
 
and each shortest path connecting  and  intersects some shortest path between 
 and  (the second condition guarantees that  and 
are ends of the lens just like in ). 
Note that the shortest path between two points in  does not have to be a segment.

For  the points  and  are centers of discs defining a lens
 when  and 
 .

From now on we will alternately use a symbol  for description
of concrete lens (one element set) or for a set of lenses defined for an edge . 



\begin{lemma}
\label{l1-01}
For  and for  there is .
\end{lemma}
\begin{proof}
Note that if points  and  are not on a line parallel to sides of a square being a circle 
in  then points  and  are defined uniquely and the lens  
is a rectangle with opposite vertices in points  and sides parallel to sides of a square 
being a circle in  (see Figure \ref{fig:l_1}).\\
In other case, there are more than two points belonging to the intersection of discs with radius 
 and centers in points  and . 
However, there are only two pairs of points  for which each shortest path connecting 
 and  intersects some shortest path between  and . Both pairs define 
the same lens and .   
\end{proof}


Note that  is uniquely defined for any  
and  for .\\


\begin{lemma}
\label{l1-betabig}
For any two points  a lens  is uniquely defined. 
For  there exists a lens  such that . 
Moreover,   is contained in every lens in .
\end{lemma}
\begin{proof} 
For  we have  and .
Let  (, respectively) denote a value of -coordinate (-coordinate, respectively)
of a point . Let us assume that .
It is easy to notice that for  and centers  of discs defining a lens 
 such that
 and  we have .
Lenses in  are an rectangles. Points  and  lie on their opposite sides
(see Figure \ref{fig:l_1}). Since  we have 
. If the difference  grows then 
the opposite vertices of the lens drive away. Hence  is contained in each lens 
from . A similar argumentation in respect of -coordinates is true when 
.        
\end{proof} 

\begin{lemma}
\label{l1-twolenses}
For  and  all lenses from  are
included in .
For  there exist two lenses  such that each of them 
does not contain any lens from , 
and . 
\end{lemma}
\begin{proof}
Let us assume that  and .
Centers of discs determining lenses in  belong to the rectangle 
. Therefore each such a disc is contained in one of the discs
determining . Hence, each lens from  is included 
in . 
Let  be centers of discs determining a lens in . The lens has
minimum size when  or . Those lenses are in extreme positions
when  or . Three sides of the first and the second
lens are contained in sides of . Both lenses contain vertices  and .
Hence, their intersection is not empty and their sum is . 
A similar argumentation in respect of -coordinates is true when .
\end{proof}


\begin{figure}[htbp]
\centering
\includegraphics[scale=0.3]{./l1-regions.eps}
\caption{Lenses in  metric: (a) , (b) ,
(c) .}
\label{fig:l_1}
\end{figure}


\begin{corollary}
\label{equality}
For a given set of points  and the parameter  the graph 
 is a set of edges such that  if and only if 
there exists a lens  that no point from  belongs 
to it.  is called -skeleton in .
For any  we have  and for any  there is 
.
\end{corollary}

Finally, we can prove that:

\begin{lemma}
\label{finalequalityl1}
For  in  metric we have:\\
.
\end{lemma}
\begin{proof}  
Let  be centers of discs determining a lens .
Let  be a point such that 
and 
and  or  (one of such choices is always possible).
Then disc with center in  and radius  contains a disc with center
in  and radius . 
In the same way we define a disc center .  
Then  are centers of discs determining a lens 
and . 
Hence . \\
The other inclusions can be proved in the same way as in the paper by Kirkpatrick and Radke 
\cite{kr85}.    
\end{proof} 

\begin{corollary}
Let  and  be Gabriel graphs defined in \cite{w06} and \cite{mg11}, respectively.
Then .
\end{corollary}



\section{-skeletons for weighted graphs}

Let  be a connected, edge-weighted graph in the plane, where  
is a subset of the set of vertices of . Each edge of  is labelled with 
a positive weight . 
The distance  between two points  and  of  is the minimum total weight 
of any path connecting  and  in .
Graph  with the distance function  is a metric space.\\
The closed disc  is defined as the set of points  of  for which .\\

Abrego et al. \cite{a12} defined Minimum Spanning Tree and Delaunay Graph of  
in the following way:

\begin{definition}
A minimum spanning tree of  is a tree  such that the sum of  
over all edges  is minimal. 

The Delaunay graph of , denoted by DG(U), is the graph  such that 
 if and only if there exists a closed disc , where  is a 
point of , containing  and  and no other vertex from .

\end{definition}


Let us assume (like \cite{a12}) that the shortest paths between each pair vertices in  is uniquely 
defined. Similarly as in the previous Section we can define lenses  for edges 
of the graph .
Unfortunately, lenses for  can be defined only for particular kinds of graphs
(for each edge  connecting vertices in  a cycle of  
length has to occur). Since the total weight of edges is limited lenses for a big value 
of  do not exist either. Therefore we can define a -skeleton 
like in the Section \ref{l1-section} only in a limited range.

\begin{lemma}
\label{beta-weight}
Let  denote the length of the shortest circle containing the edge .
The -skeletons  for weighted graphs are correctly defined 
for .
\end{lemma}   
\begin{proof}
A distance between centers of discs determining a lens  should be
. Hence, , i.e.
. \\
Then for 
and any  if a lens  then there exists
a lens  such that . 
\end{proof}

Note that .
Hence  and  are correctly defined.

\begin{lemma}
For  and for every graph  the following inequalities occur: \\
, \\
where  and .
\end{lemma}
\begin{proof}
According to Lemma \ref{beta-weight} inclusions 

are true. The other inclusions were proved by Abrego et al. \cite{a12}. 
\end{proof}

\begin{figure}[htbp]
\centering
\includegraphics[scale=0.3]{./g-wagi.eps}
\caption{Lenses in the weighted graph (each edge has the same weight) for different
values of .}
\label{fig:weighted}
\end{figure}


\section{Not uniquely defined generators of lenses} 

In previous sections, we considered -skeletons such that centers of discs determining lenses
were defined depending on position and distance of points from . Those points generated sets
of lenses. What will happen when we replace points with objects, e.g. clusters of points or segments ?
We will study this problem focusing on the case of segments. \\ 

Let us consider a set  of  segments and the  metric, where .
For two segments  and a parameter  we define 
as a set of lenses , where  are points such that  
and . 

\begin{definition}
\label{betaskeleton3}
For a given set of segments  and given parameters  and  
we define a graph such that an edge between segments  and  exists 
if and only if there exists a lens in  whose intersection
with  is empty. 
\end{definition}

In order to define Delaunay triangulation we will use a definition by Brevilliers et al. \cite{bcs08}:
\begin{definition}
A segment triangulation  of  is such a partition of the convex hull  of  
in disjoint sites, edges and faces that: 
\begin{enumerate}
\item Every face of  is an open triangle whose vertices are in three distinct sites 
of S and whose open edges do not intersect S,
\item No face can be added without intersecting another one,
\item The edges of T are the (possibly two-dimensional) connected components of , where   is the union of faces of .
\end{enumerate} 
\end{definition}

A segment triangulation of  is Delaunay  if the circumcircle of each face does not contain 
any point of  in its interior (see Figure \ref{fig:segments}). 

\begin{figure}[htbp]
\centering
\includegraphics[scale=0.2]{./dt-segment.eps}
\caption{Delaunay triangulation for a set of segments and a multigraph corresponding to it.}
\label{fig:segments}
\end{figure}

Note that we can present this triangulation as a multigraph  with a set of vertices , 
and a separate edge between vertices  and  for each edge in  connecting segments 
 and  (see Figure \ref{fig:segments}). Each such edge in the graph can be labelled with 
the length of the shortest path between two points of a given edge in  belonging to opposite 
segments.

We can formulate the following lemma.

\begin{lemma}
\label{gg-dt-s}
For  we have: \\ 
,\\
where  and  is a minimum spanning tree for a set 
of segments .
\end{lemma}
\begin{proof} 
We want to show that .
Let  be such a pair of points that there is a  disc  
with diameter  containing no points from segments from  inside of it. 
We transform  by homothety in respect of  so that its image  would be tangent to  
in . Then we transform  by homothety in respect of  so that its image  would be 
tangent to  (see Figure \ref{fig:gg-dt}). The disc  lies inside of , i.e. it does not 
intersect segments from , and is tangent to  and . Hence, 
if the edge  belongs to  then it also belongs to . \\
Let  be centers of discs determining a lens .
Let  (, respectively) be an image of  (, respectively)
by homothety with the factor  in respect of point  (, respectively).
Then  are centers of discs determining a lens 
and . Hence . \\
The last inclusion can be proved in the same way as in the paper by Kirkpatrick and Radke 
\cite{kr85}.   

\begin{figure}[htbp]
\centering
\includegraphics[scale=0.35]{./okregi.eps}
\caption{.}
\label{fig:gg-dt}
\end{figure}  
\end{proof}


\section{Generalized -skeletons}

Finally, we can formulate the definition of the generalized -skeletons:

\begin{definition}
\label{betaskeletons5}
For a given set of objects  in space  with a metric  and for parameters 
 we define a graph  - called generalized -skeleton -
as follows: 
two objects  and  are connected with an edge if and only if  
at least one lens in  is not intersected by any object from 
 where:
\begin{itemize}
\item
for  a lens , where  
and , is the intersection of two discs, each of them has radius 
, whose boundaries contain both  and  and each shortest
path connecting their centers intersects some shortest path between  and , 

\item
for  a lens , where  
and , is the intersection of two discs with centers  and , respectively,
such that  and 
.

\item
for   (, respectively) a lens , 
(, respectively), where  
and , we have  
 and 
.
 
\end{itemize}
\end{definition}

Note that in the similar way we can modify the definition of circle-based -skeleton.
Moreover, the definition (in both cases) can be applied in multidimensional spaces.  
For example, for  in , where , regions  are 
determined by spheres which centers are located symmetrically in respect of the edge . 


\section{Algorithms}


In this section we will describe algorithms computing -skeletons in above considered
situations.\\

We will start from -skeletons in  with  metric. 
According to Lemma \ref{l1-01} for  the lenses are unique defined. Moreover, 
the points from  are their vertices. \\
We rotate plane to obtain axis-aligned lenses. Then we use a plane sweep algorithm.
We sweep from left to right. The event structure contains a -ordered set .
The state structure contains for each  -ordered, labelled by , list of vertices 
lying on the right from . When the sweep line reaches a vertex  we remove from each list being 
in the state structure all points lying on the opposite side of  than a vertex labelling the list. 
If  is on the list we remove them and we add an edge connecting  with a vertex being a label 
of the list to the set of solutions (see Algorithm \ref{alg-1}). \\ 


\begin{algorithm}[H]
\footnotesize{
\SetKwFunction{Front}{Front}
\SetKwFunction{SEARCH}{SEARCH}
\SetKwFunction{Y}{Y}
\SetKwFunction{return}{return}
\KwIn{rotated plane and a set of points }
\KwOut{a set  of all edges of -skeleton for  }
\BlankLine
create list  of all vertices sorted by their -coordinate from left to right\;


\For{all vertices }{
create a list  of vertices that follow  in the  list, sorted by their -coordinates\;
}
\For{ to }{
    \For{ to }{
    remove from the list  all vertices that are separated from  on the this list by \;
     \If{the first not deleted element on the list  is }{
       add  to \;
       remove  from \;
      } 
     }
}
}
\caption{Algorithm computing  for  }
\label{alg-1} 
\end{algorithm}  

\begin{theorem}
For a given  and a set  of n points in  with  metric the -skeleton
 can be computed in  time.
\end{theorem}
\begin{proof}
We can rotate and sort points of  in  time. Lists  can be prepared 
in total  time. Time needed for loops activity is , where
 is a number of deleted elements in -th turn of the first loop. 
Since , the algorithm complexity is .
\end{proof}


Based on Lemmas \ref{l1-twolenses} and \ref{l1-betabig} it follows that if an edge 
belongs to a -skeleton for  then there exist a lens 
such that  and .
Moreover, for  the lens  is one of two extremal lenses
or it lies between points eliminating those lenses. \\
Let us rotate plane to obtain axis-aligned lenses. Let us consider a lens such that points of 
generating it lie on the top and bottom sides of the lens. To solve the problem, it is sufficient 
to find leftmost point in rightmost lens and rightmost point in leftmost lens. Then we can verify
if the distance between those points is big enough. Note that any point from  inside
 eliminates an edge  from -skeleton for .
According to Lemma \ref{finalequalityl1} a -skeleton is a subset of .
Hence we have to analyze only  lenses. \\
We use a plane sweep algorithm. We will sweep a plane to the right analyzing rightmost lenses
and next to the left analyzing leftmost ones. The event structure contains positions of  
elements and sides od lenses sorted in respect of -coordinate. The state structure is 
an interval tree with  leaves corresponding to points of . In nodes we will save 
information about swept lenses. If a sweep line visit a lens we mark a node being a lowest 
ancestor of leaves corresponding to the top and bottom sides of analyzed lens 
(see Figure \ref{fig:l1-alg}). 
If a sweep line reaches a point  we find a path from a root of the tree to a leaf 
corresponding to . We allocate the point to all lenses marked on this path. Then we erase
the markers. If the sweep line leaves the lens we erase the corresponding marker from the tree
(if the marker exists).  

\begin{figure}[htbp]
\centering
\includegraphics[scale=0.4]{./l1-alg.eps}
\caption{A plane sweep algorithm for , where .
sweep line ecounters a lens side (a) or a point of  (b). }
\label{fig:l1-alg}
\end{figure}   

  
\begin{algorithm}[H]
\footnotesize{
\SetKwFunction{Front}{Front}
\SetKwFunction{SEARCH}{SEARCH}
\SetKwFunction{p}{p}
\SetKwFunction{return}{return}
\KwIn{rotated plane with rightmost lenses and a set of points }
\KwOut{leftmost points belonging to rightmost lenses}
\BlankLine

\While{there are still unvisited points from  or lens sides }{
sweep the plane from left to right\;
\If{ecounter a lens side}
{ mark a node in the interval tree corresponding to the side of the lens\;
 }
\If{ecounter a point }
{ find a path from the root of tree to the leaf corresponding to \;
 \For{each lens mark  on the path}
 {save the pair  and erase  from the tree }
}
\If{leave a lens}
{ remove mark  (if it exists) from the tree \;
}
}
}
\caption{Algorithm computing leftmost points belonging to rightmost lenses}
\label{alg-2}   
\end{algorithm}   

\begin{theorem}
For a given  and a set  of  points in  with  metric 
the -skeleton  can be computed in  time.
\end{theorem}
\begin{proof}
We use a plane sweep algorithm four times (two times in each direction after clockwise 
and counterclockwise rotation of the plane - depending on a slope of edges generating lenses).
The number of events is . The -th event needs  time, where 
 is a number of added or removed marks. Since , the algorithm complexity is 

\end{proof}



Now, we will present an algorithm computing  for a weighted graph .\\
The algorithm is natural. First we compute distances between all vertices in  using 
e.g. Johnson's algorithm \cite{clrs09}. Then we find a set of points which are candidates for centers
of disc determining lenses. For this purpose we analyze distances of two points generating lens
from point belonging to any edge. Next, we verify which pairs of candidates can create lenses
(a ratio of a distance between centers of discs determining lens to a distance between generators
is ). In the last step analyze an intersection of a set  with each lens. \\
Unfortunately, the algorithm is expensive due to a big number of possible lenses.

\begin{lemma}
A number of all candidates for centers of disc determining lenses is , where .
\end{lemma}
\begin{proof}
Let us assume that ,  is even, , , sets  and  
form cycles which every second element belongs to . One edge of  has a big weight.
Edges connecting vertices of  with vertices of  also have
a big weight (see Figure \ref{fig:sr-wag}). \\
Hence the graph  has  edges.
For a sufficiently big value of parameter  and each of  pairs vertices generating 
a lens, which are connected by  short edges, there exist  edges containing 
candidates for disc centers. Hence, the total number of candidates is 
.
\end{proof} 


\begin{figure}[htbp]
\centering
\includegraphics[scale=0.4]{./srodki-wag.eps}
\caption{ An example of many centers of disc determining lenses for generators .}
\label{fig:sr-wag}
\end{figure}   


\begin{algorithm}[H]
\footnotesize{
\SetKwFunction{Front}{Front}
\SetKwFunction{SEARCH}{SEARCH}
\SetKwFunction{A}{A}
\SetKwFunction{PotCen}{PotCen}
\SetKwFunction{return}{return}
\KwIn{weighted graph  parameter }
\KwOut{set  of all edges of -skeleton for  }
\BlankLine
compute all distances in graph  between vertices in \;
\For{every edge }{
   find all potential centers  for \;
     \For{every pair }{
       \If{}{
       add pair  to \;
          }
       }
     \For{every pair }{
      \If{lens defined by  is empty}{
      add  to \;
      }
      }
}
}
\caption{Algorithm for computing -skeleton for  for weighted graphs }  
\end{algorithm}  

\begin{theorem}
Let  denote the length of the shortest circle containing the edge  
in the weighted graph .
For 
the -skeleton  can be computed in  time, where  and .
\end{theorem}
\begin{proof}
The most time-consuming is the last part of the algorithm, i.e. verification of emptiness of lenses.
For one pair of generators there are at most  candidates for centers of disc determining lenses.
Hence, there are  lenses to verify. The total algorithm complexity
is .
\end{proof}

\begin{corollary}
For  the -skeleton  can be computed in  time.
\end{corollary}
\begin{proof}
Centers of disc determining lenses for  are uniquely defined. There are points of .
Since , there are at most  lenses. Complexity of the Johnson's algorithm
id . Hence, the total complexity of the algorithm is .   
\end{proof}


The most interesting and difficult is an algorithm computing -skeletons for a set 
of  segments in  with  metric. We will outline the solution. 
Details can be found in the paper \cite{km14}.
Let us consider a set of parametrized lines containing given segments. A line 
contains a segment  and its parametrization is 
, where  and  are
ends of the segment . Let  and  be generators of a lens.
We shoot a rays from points . Let us assume that the rays pass through
a point , where .
For  the ray reflects. The sum of angle of incidence and angle of reflection 
is equal to an inscribed angle for a lens generated for a given value of .
For  we compute a new line perpendicular to the ray such that 
a distance between an intersection point  and the point  is 
.
Both new created lines will be called reflected rays.
The reflected ray intersects line  in point .
A relation between parameters  and  is a hyperbolic function.
For a given segment , hyperbolas for points 
create hyperbolic stripe. Intersection of this stripe with a hyperbolic stripe describing relation
between parameters  and  is a polygon whose sides are parts of hyperbolas 
(see Figure \ref{fig:rays}). 

\begin{figure}[htbp]
\centering
\includegraphics[scale=0.3]{./rays.eps}
\caption{Rays for computing of  (a), hyperbolic polygon (b).}
\label{fig:rays}
\end{figure}   

\begin{theorem}
An edge generated by segments  and  belongs to -skeleton 
if and only iff a sum of polygons computed for all segments 
does not cover a square . 
\end{theorem}
\begin{proof}
If there exists an uncovered point , then a lens generated 
for points  and  does not intersect any segment in .
The opposite implication also is true.
\end{proof}

 
\begin{theorem}
The -skeleton  for a set of  segments  in  with  metric
can be computed for  in  time and for  in 
time.
\end{theorem}
\begin{proof}
A sum of  polygons can be computed in  time. A verification  pairs 
of lens generators needs  time.
According to Lemma \ref{gg-dt-s} and linearity of , for 
we have to check only  pairs of generators. Hence, a complexity of the algorithm
is . 
\end{proof}

For  we can use -order Voronoi diagrams for segments to compute .
In this case a complexity of the algorithm is  \cite{km14}.


\section{Conclusions}

In this paper we show a way of defining -skeletons in general.
We have based our proposition on a distance criterion and we described conditions
should be satisfied if the lenses are not defined uniquely.
We have focused our considerations only on a few special cases which in our opinion
well describe the idea of this general definition.
In a similar way, we can also define -skeletons for example for a set of polygons
or for the jungle river metric.
It is also easy to generalize this definition for higher dimensions. 
One can consider a couple of new problems regarding this definition.
It would be interesting to check how those changes can influence the time 
of algorithms computing -skeletons. For example, if the  for segments could be
computed faster than in  time.
Many questions can also concern properties of -skeletons which they have for different objects
and their practical applications. 


\noindent
{\bf Acknowledgements}\\
\noindent
The authors would like to thank Evanthia Papadopoulou for important discussions. 
We thank Jerzy W. Jaromczyk for his comments and suggestions.



\bibliographystyle{abbrv}
\begin{thebibliography}{99999}

\bibitem[A12]{a12} 
B.M. Ábrego, R. Fabila-Monroy, S. Fernández-Merchant, D. Flores-Penaloza, 
F. Hurtado, H. Meijer, V. Sacristán, M. Saumell, 
\textit{Proximity graphs inside large weighted graphs}, 
\textit{Networks},volume 2012, 2012

\bibitem[ABE98]{abe98} 
N. Amenta, M.W. Bern, D. Eppstein, 
\textit{The crust and the -skeleton: combinatorial curve reconstruction}, 
\textit{Graphical Models Image Processing}, 60/2 (2), 1998, 125-135

\bibitem[BCS08]{bcs08}
M. Brevilliers, N. Chevallier, D. Schmitt,
\textit{Flip Algorithm for Segment Triangulations},
\textit{Proceedings of the 33rd international symposium on Mathematical Foundations of Computer Science}, 2008, 180-192

\bibitem[CLRS]{clrs09} 
T.H. Cormen, C.E. Leiserson, R.L. Rivest, C. Stein, 
\textit{Introduction to Algorithms}, 
MIT Press, 2009

\bibitem[EKS83]{eks83} 
H. Edelsbrunner, D.G. Kirkpatrick, R. Seidel, 
\textit{On the shape of a set of points in the plane}, 
\textit{Transaction on Information Theory}, volume 29, 1983, 551-559

\bibitem[E02]{e02} 
D. Eppstein, 
\textit{-skeletons have unbounded dilation}, 
\textit{Computational Geometry}, vol. 23, 2002, 43-52


\bibitem[GS69]{gs69} 
K.R. Gabriel, R.R. Sokal, 
\textit{A new statistical approach to geographic variation analysis}, 
\textit{Systematic Zoology} 18, 1969, 259-278



\bibitem[JK87]{jk87} 
J.W. Jaromczyk, M. Kowaluk, 
\textit{A note on relative neighborhood graphs}, 
\textit{Proceedings of the 3rd Annual Symposium on Computational Geometry}, Canada, Waterloo, 
ACM Press, 1987, 233-241 

\bibitem[JKY89]{jky89} 
J.W. Jaromczyk, M. Kowaluk, F. Yao, 
\textit{An optimal algorithm for constructing -skeletons in  metric},
 \textit{Manuscript}, 1989

\bibitem[JT92]{JT92} 
J.W. Jaromczyk, G.T. Toussaint, 
\textit{Relative neighborhood graphs and their relatives}, 
\textit{Proceedings of the IEEE}, volume 80, issue 9, 1992, 1502-1517

\bibitem[KR85]{kr85} 
D.G. Kirkpatrick, J.D. Radke, 
\textit{A framework for computational morphology}, 
\textit{Computational Geometry}, North Holland, Amsterdam, 1985, 217-248



\bibitem[KM14]{km14}
M. Kowaluk, G. Majewska,
\textit{-skeletons for a set of segments in },
\textit{Manuscript prepared for CCCG 2014}, 2014

\bibitem[K66]{k66}
K. Kuratowski,
\textit{Topology},
Academic Press, New York, 1966

\bibitem[L94]{l94} 
A. Lingas. 
\textit {A linear-time construction of the relative neighborhood graph 
from the Delaunay triangulation}, 
\textit{Computational Geometry} 4, 1994, 199-208

\bibitem[MG11]{mg11} 
L. Maignan, F. Gruau. 
\textit {Gabriel Graphs in Arbitrary Metric Space and their Cellular Automaton for Many Grids}, 
\textit{ACM Transactions on Autonomous and Adaptive Systems} 6 (2), 2011, 12:1-12:13
 
\bibitem[MS84]{ms84} 
D.W. Matula, R.R. Sokal, 
\textit{Properties of Gabriel graphs relevant to geographical variation research 
and the clustering of points in plane}, 
\textit{Geographical Analysis} 12, 1984, 205-222

\bibitem[Su83]{su83} 
K.J. Supowit, 
\textit{The relative neighborhood graph, with an application to minimum spanning trees}, 
\textit{Journal of the ACM} volume 30, issue 3, 1983, 428-448

\bibitem[Tou80]{tou80} 
G.T. Toussaint, 
\textit{The relative neighborhood graph of a finite planar set}, 
\textit{Pattern Recognition} 12, 1980, 261-268 

\bibitem[W06]{w06} 
C. Wulff-Nilsen, 
\textit{Proximity structures in the fixed orientation metrics}, 
\textit{Proceedings of the EWCG 2006} 2006, 153-156

\end{thebibliography}

\end{document}
