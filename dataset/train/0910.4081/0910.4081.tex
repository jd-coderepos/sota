\documentclass{LMCS}

\usepackage{hyperref}
\usepackage{enumerate}

\usepackage{ifpdf}
\ifpdf
\usepackage[pdftex,matrix,arrow,tips,curve,color]{xy}
\else
\usepackage[dvips,matrix,arrow,tips,curve,color]{xy}
\fi
\usepackage{xspace}

\usepackage{amssymb}
\usepackage{subfig}


\theoremstyle{plain}
\newtheorem{claim}{Claim}

\theoremstyle{definition}
\newtheorem{notation}[thm]{Notation}

\SelectTips{cm}{10}

\usepackage[english]{babel}




\newcommand{\trsp}[3]{\mathcal{#1} = (#2, #3)}

\newcommand{\rew}{\rightarrow}
\newcommand{\rewt}{\rightarrow^*}
\newcommand{\trewt}{\twoheadrightarrow}
\newcommand{\trewtb}{\twoheadleftarrow}
\newcommand{\trewtp}[1]{\twoheadrightarrow^{#1}}
\newcommand{\out}{\trewt^\mathrm{out}}
\newcommand{\outs}{\rew^\mathrm{out}}

\newcommand{\trewtc}{\mathrel{(\mathord{\trewtb}\mathord{\cdot}\mathord{\trewt})^*}}

\newcommand{\simhc}{\sim_{hc}}
\newcommand{\simhcv}{\wr_{_{\scriptstyle hc}}}

\newcommand{\pos}[1]{\mathcal{P}os(#1)}

\newcommand{\rs}[1]{root(#1)}

\newcommand{\natnum}{\mathbb{N}}

\newcommand{\seper}{\; | \;}

\newcommand{\ulam}{\underline{\lambda}}

\newcommand{\iLC}{ic\xspace}

\newcommand{\dev}{\Rightarrow}

\newcommand{\pmap}{\varepsilon}
\newcommand{\pmapp}[2]{\pmap_{#1}({#2})}

\newcommand{\pme}{\mu}
\newcommand{\pmep}[2]{\pme_{#1}({#2})}



\def\doi{5 (4:3) 2009}
\lmcsheading {\doi}
{1--29}
{}
{}
{Oct.~28, 2008}
{Dec.~20, 2009}
{}   

\begin{document}
 
\title[Infinitary Combinatory Reduction Systems: Confluence]{Infinitary Combinatory Reduction Systems:\\
Confluence\rsuper*}

\author[J.~Ketema]{Jeroen Ketema\rsuper a}
\address{{\lsuper a}Research Institute of Electrical Communication, Tohoku University\\
2-1-1 Katahira, Aoba-ku, Sendai 980-8577, Japan}
\email{jketema@nue.riec.tohoku.ac.jp}
\thanks{{\lsuper a}This author was partially funded by the Netherlands Organisation for Scientific Research (NWO) under FOCUS/BRICKS grant number 642.000.502.}

\author[J.~G.~Simonsen]{Jakob Grue Simonsen\rsuper b}
\address{{\lsuper b}Department of Computer Science, University of Copenhagen (DIKU)\\
Universitetsparken 1, 2100 Copenhagen \O, Denmark}
\email{simonsen@diku.dk}

\keywords{term rewriting, higher-order computation, combinatory reduction systems, lambda-calculus, infinite computation, confluence, normal forms}
\subjclass{D.3.1, F.3.2, F.4.1, F.4.2}
\titlecomment{{\lsuper*}Parts of this paper have previously appeared as \cite{JJ05b}}

\begin{abstract}
We study confluence in the setting of higher-order infinitary rewriting, in particular for infinitary Combinatory Reduction Systems (iCRSs). We prove that 
fully-extended, orthogonal iCRSs are confluent modulo identification of hypercollapsing subterms. As a corollary, we obtain that fully-extended, orthogonal iCRSs have the normal form property and the unique normal form property (with respect to reduction). We also show that, unlike the case in first-order infinitary rewriting, almost non-collapsing iCRSs are not necessarily confluent.
\end{abstract}

\maketitle

\tableofcontents

\section{Introduction}

\noindent This paper is part of a series outlining the fundamental
theory of higher-order infinitary rewriting in the guise of
\emph{infinitary Combinatory Reduction Systems} (iCRSs). In
preliminary papers \cite{JJ05a,JJ05b} we outlined basic motivation and
definitions, and gave a number of introductory results. Moreover, we
lifted a number of results from first-order infinitary rewriting to
the setting of iCRSs. In particular, staple results such as
compression and existence of complete developments of sets of redexes
(subject to certain conditions) were proved.

The purpose of iCRSs is to extend infinitary term rewriting to encompass higher-order rewrite systems.
This allows us, for instance, to reason about the behaviour of the well-known  functional 
when it is applied to infinite lists. The  functional
and the usual constructors and destructors for lists can be represented by the below iCRS:


Systems such the above may satisfy certain simple criteria: being \emph{orthogonal} (rules do not overlap syntactically) and \emph{fully-extended} (if a variable is bound, then every meta-variable in its scope must be applied to it). We show that systems satisfying these two criteria are confluent modulo identification of a certain class of `meaningless' subterms: Subterms that are \emph{hypercollapsing}. As an example,  above, when applied to any infinite list , will yield identical results no matter how it is computed, except when applied to lists that will never yield a proper result \emph{irrespective} of the evaluation order.

{\bf A succinct description for researchers familiar with infinitary rewriting}: In the current paper, we employ the methods developed in previous papers to show that fully-extended, orthogonal iCRSs are confluent modulo identification of hypercollapsing subterms. As a corollary, we obtain that fully-extended, orthogonal iCRSs have the normal form property and the unique normal form property (with respect to reduction). Finally, we show that, unlike the case in first-order infinitary rewriting, almost non-collapsing iCRSs are not necessarily confluent.  

Parts of this paper have previously appeared as \cite{JJ05b}; the current paper \emph{corrects} the results of that paper and \emph{extends} them: We now allow rules with infinite right-hand sides, not only finite right-hand sides.
The present paper requires some of the results proved in the previously published, peer-reviewed papers \cite{JJ05a,JJ05b}. A much-updated and extended version of these results is available as \cite{paper_i}. 







\subsection{Overview and roadmap to confluence}

The contents of the paper are as follows: Section \ref{sec:preliminaries} introduces preliminary notions. Section \ref{sec:pp} on projection pairs recapitulates in an abstract way the fundamental results on \emph{essential rewrite steps}, the primary method used to prove confluence in the higher-order infinitary setting. Section \ref{sec:confluence} provides
proofs of our main results on confluence.
Section \ref{sec:normal_form_properties} considers
the normal form property, the unique normal form property, and the
unique normal form with respect to reduction property. 
Section \ref{sec:conclusion} concludes.



The main result of the paper is Theorem \ref{the:confl_modulo}:
Fully-extended, orthogonal iCRSs are confluent modulo identification of hypercollapsing subterms. To aid the reader we give a roadmap of the most important auxiliary results leading up to that theorem in Figure \ref{fig:road}.


\begin{figure}
Confluence modulo )} & 
}

(\lambda x. \Box)\{(\lambda y . x) z / \Box\}
\rew_\beta \lambda x.x \, .

(\lambda x .\Box)\{(\lambda y . x) z / \Box\}
\rew_\beta \lambda w . x \, ,

\label{pbetaeq}
(\ulam \vec{x} . s)(t_1, \ldots, t_n) = s[\vec{x} := \vec{t}] \, .

(\ulam \vec{x}. s)(t_1, \ldots, t_n) \rew s[\vec{x} := \vec{t}] \, .

f([x]g(x)) \rew g(f([x]g(x)) \rew \cdots \rew g^n(f([x]g(x))) \rew g^{n + 1}(f([x]g(x))) \rew \cdots

\xymatrix{
s \ar@{=>}[r]^{\mathcal{V}} \ar@{=>}[d]^{\mathcal{U}} & t'
\ar@{=>}[d]^{\mathcal{U}/(s \dev^\mathcal{V} t')} \\
t \ar@{=>}[r]_{\mathcal{V}/(s \dev^\mathcal{U} t)} & s'
}

\xymatrix{
s_{0, 0} \ar[r] \ar[d]
    & s_{0, 1} \ar@{.}[r] \ar@{->>}[d]
    & s_{0, \delta} \ar[r] \ar@{->>}[d]
    & s_{0, \delta + 1} \ar@{.}[r] \ar@{->>}[d]
    & s_{0, \beta} \ar@{->>}[d] \\
s_{1, 0} \ar@{->>}[r] \ar@{.}[d]
    & s_{1, 1} \ar@{.}[r] \ar@{.}[d]
    & s_{1, \delta} \ar@{->>}[r] \ar@{.}[d]
    & s_{1, \delta + 1} \ar@{.}[r] \ar@{.}[d]
    & s_{1, \beta} \ar@{.}[d] \\
s_{\gamma, 0} \ar@{->>}[r] \ar@{->}[d]
    & s_{\gamma, 1} \ar@{.}[r] \ar@{->>}[d]
    & s_{\gamma, \delta} \ar@{->>}[r] ^{T_{\gamma, \delta}}
                         \ar@{->>}[d]_{S_{\gamma, \delta}}
    & s_{\gamma, \delta + 1} \ar@{.}[r] \ar@{->>}[d]
    & s_{\gamma, \beta} \ar@{->>}[d] \\
s_{\gamma + 1, 0} \ar@{->>}[r] \ar@{.}[d]
    & s_{\gamma + 1, 1} \ar@{.}[r] \ar@{.}[d]
    & s_{\gamma + 1, \delta} \ar@{->>}[r] \ar@{.}[d]
    & s_{\gamma + 1,\delta + 1} \ar@{.}[r] \ar@{.}[d]
    & s_{\gamma + 1, \beta} \ar@{.}[d] \\
s_{\alpha, 0} \ar@{->>}[r]
    & s_{\alpha, 1} \ar@{.}[r]
    & s_{\alpha, \delta} \ar@{->>}[r]
    & s_{\alpha, \delta + 1} \ar@{.}[r]
    & s_{\alpha, \beta}
}

f^\omega \rew f^\omega \rew \cdots

g([x]g(x)) \rew g([x]g(x)) \rew \cdots \, .

s = s_0 \trewt s'_0 \rew s_1 \trewt s'_1 \rew s_2 \trewt \cdots  \, ,

s = t_0 \rewt t'_0 \rew t_1 \rewt t'_1 \rew t_2 \rewt \cdots  \, ,

s_0 \rewt s'_0 \rew s_1 \rewt s'_1 \rew s_2 \rewt \cdots  \, ,

\xymatrix{
s_0 \ar[d]^{u} \ar[r]^{*} & s'_0 \ar@{=>}[d]^{\mathcal{U}'_0} \ar[r] & s_1
\ar@{=>}[d]^{\mathcal{U}_1} \ar[r]^{*} & s'_1 \ar@{=>}[d]^{\mathcal{U}'_1}
\ar[r] & s_2 \ar@{=>}[d]^{\mathcal{U}_2} \ar[r]^{*} & \cdot \ar@{:>}[d]
\ar@{.}[r] & \\
t_0 \ar@{->>}[r] & t'_0 \ar@{->>}[r] & t_1 \ar@{->>}[r] & t'_1 \ar@{->>}[r]
& t_2 \ar@{->>}[r] & \cdot \ar@{.}[r] &
}

\xymatrix@=0.5cm{
 & s \ar@{->>}[dl] \ar@{}[r]|*\txt{} & t \ar@{->>}[dr] &    \\
s' \ar@{->>}[dr] & & & t' \ar@{->>}[dl] \\
 & s'' \ar@{}[r]|*\txt{} & t'' & \\
}

\xymatrix@!0@C=1.2cm@R=0.8cm{
*+[gray]{s_{\gamma, \delta}} \ar@{->>}@[gray][rrrr]^[gray]{out}
                                \ar@{=>}@[gray][dd]
                                \ar@{.}[dr]
    & & &
    & *+[gray]{s_{\gamma, \delta + 1}} \ar@{=>}@[gray]'[d][dd]
                                          \ar@{.}[dr] \\
& t_{\gamma, \delta} \ar@{=>}[rr] \ar@{->>}[dddd]_{out}
    &
    & t'_{\gamma, \delta + 1} \ar@{->>}[dddd]
    & \simhc & t_{\gamma, \delta + 1} \ar@{->>}[dddd]_{out} \\
*+[gray]{s'_{\gamma + 1, \delta}} \ar@{->>}@[gray]'[r]'[rrr][rrrr]
    & & &
    & *+[gray]{s'_{\gamma + 1, \delta + 1}} \\
*[gray]{\simhcv}
    & & &
    & *[gray]{\simhcv} \\
*+[gray]{s_{\gamma + 1, \delta}}
                            \ar@{->>}@[gray]'[r]'[rrr]^[gray]{out}[rrrr]
                            \ar@{.}[dr]
    & & &
    & *+[gray]{s_{\gamma + 1, \delta + 1}} \ar@{.}[dr] \\
& t_{\gamma + 1, \delta} \ar@{=>}[rr]
    &
    & t'_{\gamma + 1, \delta + 1} & \simhc & t_{\gamma + 1, \delta + 1}
}

\xymatrix{
    & s \ar@{}[r]|*\txt{} \ar@{->>}[dl] \ar@{}[d]|{\mathrm{(1)}}
    & t \ar@{=}[r] \ar@{->>}[dl]^{out} \ar@{->>}[dr]_{out}
    & t \ar@{->>}[dr] \ar@{}[d]|{\mathrm{(2)}} \\
s' \ar@{}[r]|*\txt{} \ar@{->>}[d]_{out} \ar@{}[dr]|{\mathrm{(4)}}
    & t'_0 \ar@{->>}[d]^{out} \ar@{}[drr]|{\mathrm{(3)}}
    & 
    & t'_1 \ar@{}[r]|*\txt{} \ar@{->>}[d]_{out}
    & t' \ar@{->>}[d]^{out} \ar@{}[dl]|{\mathrm{(5)}} \\
s'' \ar@{}[r]|*\txt{}
    & t^*_0 \ar@{}[rr]|*\txt{}
    &
    & t^*_1 \ar@{}[r]|*\txt{}
    & t''
}

\mathtt{map}([z]F(z),\mathtt{cons}(X,XS)) & \rew
\mathtt{cons}(F(X),\mathtt{map}([z]F(z),XS)) \\
\mathtt{map}([z]F(z),\mathtt{nil}) & \rew \mathtt{nil}\\
\mathtt{hd}(\mathtt{cons}(X,XS)) & \rew X \\
\mathtt{tl}(\mathtt{cons}(X,XS)) & \rew XS

f^n([x]Z(x),Z') \rightarrow g(c^\omega,f^{n+1}([x]Z(x),Z(Z'))) \qquad n \geq 1

f([x]Z(x)) \rew Z(f([x]Z(x)) \, .

f([x]f([y]x)) \rew f([x]x) \, ,

f([x]f([y]x)) \rew f([y]f([x]f([y]x))) \rew f([y]f([y]f([x]f([y]x)))) \rew \cdots \, s \, ,

a & \rew f(b) & b & \rew b\\
a & \rew g(c) & c & \rew c

\noindent No term in which a redex occurs has a normal form. Hence, NF is immediate. However, confluence modulo hypercollapsing subterms does not hold, as  reduces to  and  --- both of which only reduce to themselves --- and as .

\section{Conclusion and suggestions for future work}
\label{sec:conclusion}

\noindent We have extended confluence modulo identification of
hypercollapsing subterms to higher-order infinitary rewriting by
employing the proof techniques of earlier papers in the series on
iCRSs as well as extending the known proof methods from
\cite{T03_KV}. Our results properly generalise similar results for
iTRSs and \iLC, and the paper develops and extends the proof methods
employed in earlier papers on these subjects.

Two major open questions related to confluence of iCRSs and higher-order infinitary rewriting in general remain. We invite the reader to consider these: 

\begin{enumerate}[]

\item Can a characterisation be given of the subclass of confluent iCRSs, generalising the first-order result that almost-non-collapsing systems are confluent? As we reason in Section \ref{sec:almost_non_collapsing}, a generalisation will likely not be easy to come by.

\item The current proof of confluence modulo hypercollapsing subterms requires
orthogonality. Is it possible to replace orthogonality
by weak orthogonality?

\end{enumerate}

In the greater context of infinitary rewriting, this paper is part of an account of the general theory of iCRSs. We believe that the results and proof methods laid out will contribute to the further development of infinitary rewriting and equational reasoning involving infinite terms.

\section*{Acknowledgement}

\noindent The authors extend their thanks to the anonymous referees
for their diligent work and many comments that have led to substantial
improvements in the readability of the paper.


\bibliographystyle{abbrv}
\bibliography{icrs}

\newpage

\appendix

\section{Proof of Lemma \ref{lem:funky_bind}}
\label{app:bound}

We prove Lemma \ref{lem:funky_bind}.

\proof
Suppose that a variable bound by an abstraction in the redex pattern of  occurs in . We reason by transfinite induction on , the length of the reduction . In case , the result is immediate since bound variables can only occur below the abstraction by which they are bound and since subterms are substituted for the variables that occur in the subterm at the position of  in .

In case  is a successor ordinal, suppose either that (a) , or that (b) , but  
does not occur in the reduct  of . We have the following:

\begin{enumerate}[]
\item
In case , we have that  does not occur below . Hence, for some  a nesting is created by contracting a redex at a prefix position of  and  and, by definition of valuations, if a variable is bound by the redex pattern of , then it cannot occur in .

\item
In case , but  does not occur in the reduct of  at  in , there is a position  in  such that  is a reduct of a subterm not strictly below . Hence, for some  a nesting is created by contracting a redex at a prefix position of  and . In the term  a residual of  occurs below  and above  and, by definition of valuations, if a variable is bound by the redex pattern of , then it cannot occur in the residual and, hence, in . 
\end{enumerate}

Thus, in both cases it follows for some  that  nests a variable bound by  in . By the definition of valuations, we have for the redex contracted in , say  at position , that
\begin{enumerate}[(1)]
\item
a variable bound by an abstraction in the redex pattern of  occurs in , and that
\item
a variable bound by an abstraction in the redex pattern of  occurs in . 
\end{enumerate}
Since it follows by assumption that  is the residual of a redex  in  for all , we have by the induction hypothesis that  for all . Hence, for every  with  the contracted redex is a residual of a redex in  in case it occurs at a prefix position of . By the induction hypothesis we now have:
\begin{enumerate}[]
\item
In case , it follows that  and . Hence, , a contradiction.

\item
In case , but  does not occur in the reduct of  at  in , it follows that  occurs in the reduct of  at  in  and that  occurs in the reduct of  at  in . Hence,  occurs in the reduct of  at  in , a contradiction.
\end{enumerate}
Hence, the result follows if  is a successor ordinal.

In case  is a limit ordinal, the result is immediate by strong convergence and the induction hypothesis, since residuals occur at finite depth. \qed

\end{document}
