
\thmsimu


\begin{proof}
  Let  be a binary relation on processes defined as follows
  
  Let us show that  is a simulation.
  \begin{itemize}
  \item If 
    then 
    and by Lemmas~\ref{lem:gvsproj} and~\ref{lem:gvssys}
    we have that 
    and thus 

    Now we have to show that , i.e.
    
    Pose
    
    by Lemma~\ref{lem:subred}, ,
    and by Corollary~\ref{cor:subred}, we have
    , as required.

\item If 
    then 
    and by Lemmas~\ref{lem:gvsproj} and~\ref{lem:gvssys}
    we have that 
    and thus 
    
    Now we have to show that , i.e.
    
    Pose
    
    and 
    
    
    by Lemma~\ref{lem:subred}, ,
    and by Corollary~\ref{cor:subred}, we have
    , as required.
    
  \item If  
    then  and ,
    thus 
    and by Lemma~\ref{lem:partis} this means that ,
    we then have .
  \end{itemize}
\end{proof}





\begin{lemma}\label{lem:sysproj}
  The following holds:
  \begin{enumerate}
  \item \label{en:lem:sysproj1}
    If 
    then 
   
  \item \label{en:lem:sysproj2}
    If 
    then either ,
    or  
  \end{enumerate}
\end{lemma}
\begin{proof}
  The proof of~\ref{en:lem:sysproj1} is by Lemma~\ref{lem:svsg},
  and the proof of~\ref{en:lem:sysproj2}
  is by Lemma~\ref{lem:svsg} and rule \rulename{}.
\end{proof}
\medskip






\thmbisim

\begin{proof}
For this proof we pose
.

\proocase{ sends}
Assume ,
 and
.
We have

and 

By Lemma~\ref{lem:sysproj}, ,
thus ,
and by Lemma~\ref{lem:qproj} .

We then have (note that )

and by definition of ,



Let us now show that

By Lemma~\ref{lem:subred} we know that ,
we have that
, by Corollary~\ref{cor:subred}, and
 by Lemma~\ref{lem:qproj}.

\proocase{ sends}
Assume ,  and 
. We have

and


By Lemma~\ref{lem:simu}, we have ,
and since, by Lemma~\ref{lem:chans}, , there is a queue  in . Note that a queue  can always make
a transition  (regardless of ).
By Lemma~\ref{lem:qproj}, .

Therefore, we must have


And by definition of , we have


Finally, we have

by Lemma~\ref{lem:subred}.

\proocase{ receives}
Assume ,
 and
.
We have

and 

By Lemma~\ref{lem:sysproj} and~\eqref{eq:corr11}, we have either

or

However, by assumption we have , with  as in~\eqref{eq:corr11},
therefore~\eqref{eq:corr14} cannot hold by Lemma~\ref{lem:qbranch}.

By Lemma~\ref{lem:qproj}, we have that 
and by~\eqref{eq:corr13}, .
By definition of , we have


Let us now show that

By Lemma~\ref{lem:subred} we know that ,
we have that
, by Corollary~\ref{cor:subred}, and
 by Lemma~\ref{lem:qproj}.

\proocase{ receives}
Assume ,  and 
. We have

and

\end{proof}
By Lemma~\ref{lem:simu}, we have ,
and by Lemma~\ref{lem:qproj}, we have ,
therefore

and 


We now have to show that

as before, we have  by Corollary~\ref{cor:subred} and
 by Lemma~\ref{lem:qproj}.

\proocase{End}
If , then  and ,
and vice versa if  .


\begin{lemma}\label{lem:qbranch}

  If 
  
  is derivable then
   is \emph{not} derivable.
\end{lemma}

\begin{proof}
  Assume .
  We show that we must have
  
  and the derivation of  must have the following form
  
where we must have
\begin{itemize}
\item  or  and
   or .
\item  (the second part of the split does not matter)
\item  (by Lemma~\ref{lem:twooutsQ})
\item  and 
  because of rules \rulename{} and \rulename{}
\end{itemize}

Now let us discuss a derivation for .
Since we have  , we must have
 such that ,
and ,
if the split does exist.
If it does, we have
 such that . Therefore, the
split for the rest of the system is the same as in the other derivation.

Again, we can divide the system using \rulename{} if need be such that we get

with  therefore no rule is applicable for
this judgement, and the derivation does not exist.

\end{proof}



\begin{lemma}\label{lem:twoouts}
  Let  a system such , and
  
  the following is \emph{not} derivable
  
\end{lemma}
\begin{proof}
  We show this by contradiction.
  Given  as above, the only rule applicable is \rulename{} on 
  either selecting the branch on  or on .
  Therefore, the following should be derivable
  
and, we must have  and
.
And the other derivation as the form:

where we have  and
.
However, this is clearly in contradiction with Theorem~\ref{thm:RUNIQUE}, i.e.\

\end{proof}

\begin{lemma}\label{lem:twooutsQ}
  Let  a system such , and
  
  the following is \emph{not} derivable
  
\end{lemma}
\begin{proof}
  The proof is similar to the one of Lemma~\ref{lem:twooutsQ}
  where  replaces  
  .
\end{proof}

%