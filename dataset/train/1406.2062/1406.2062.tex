\documentclass[copyright,creativecommons]{eptcs}

\pdfoutput=1

\pdfmapfile{+newtx.map}

\usepackage[natbib=false,typeface=Times]{wjtools}

\hypersetup{hidelinks}

\usepackage{tikz}

\ifmodernengine\else

    \PreloadUnicodePage{0}
    \PreloadUnicodePage{3}
    \PreloadUnicodePage{32}
    \PreloadUnicodePage{33}
    \PreloadUnicodePage{34}
    \PreloadUnicodePage{37}
    \PreloadUnicodePage{39}
    \PreloadUnicodePage{468}
    \PreloadUnicodePage{471}

    \DeclareUnicodeCharacter{949}{\ensuremath\varepsilon}
    \DeclareUnicodeCharacter{952}{\ensuremath\theta}
    \DeclareUnicodeCharacter{954}{\ensuremath\kappa}
    \DeclareUnicodeCharacter{960}{\ensuremath\pi}
    \DeclareUnicodeCharacter{961}{\ensuremath\rho}
    \DeclareUnicodeCharacter{963}{\ensuremath\sigma}
    \DeclareUnicodeCharacter{966}{\ensuremath\varphi}
    \DeclareUnicodeCharacter{1013}{\ensuremath\epsilon}
    \DeclareUnicodeCharacter{977}{\ensuremath\vartheta}
    \DeclareUnicodeCharacter{1008}{\ensuremath\varkappa}
    \DeclareUnicodeCharacter{982}{\ensuremath\varpi}
    \DeclareUnicodeCharacter{1009}{\ensuremath\varrho}
    \DeclareUnicodeCharacter{962}{\ensuremath\varsigma}
    \DeclareUnicodeCharacter{981}{\ensuremath\phi}
    \new\let\mathbbm=\mathbb


    \DeclareUnicodeCharacter{9655}{\ensuremath\rhd}
    \DeclareUnicodeCharacter{9671}{\ensuremath\Diamond}

    \new\let\mathscr=\mathcal

\fi

\DeclareMathSymbol{.}{\mathrel}{letters}{"3A}
\DeclareMathSymbol{!}{\mathord}{operators}{"21}
\DeclareMathSymbol{?}{\mathord}{operators}{"3F}


\newcommand{\relwithsizeof}[2]{
    \mathrel{
        \text{\makebox[0mm][c]{\phantom{}\makebox[0mm][c]{}}\phantom{}}
    }
}

\tikzset{diagram/.style={row sep=7ex,column sep=4em}}
\tikzset{object/.style={}}
\tikzset{morphism/.style={-stealth}}
\tikzset{left morphism/.style={morphism,bend left=40}}
\tikzset{right morphism/.style={morphism,bend right=40}}
\newcommand{\edges}{\AtBeginMath\scriptstyle}

\newcommand{\Ob}{\mathrm{Ob}}
\newcommand{\Mor}{\mathrm{Mor}}
\DeclareMathOperator{\Hom}{Hom}
\newcommand{\id}{\mathrm{id}}
\newcommand{\Id}{\mathrm{Id}}
\newcommand{\op}[1]{#1^{\mathrm{op}}}
\newcommand{\Set}{\mathbf{Set}}
\newcommand{\timebound}{t_{\mathrm b}}
\newcommand{\obstime}{t_{\mathrm o}}
\newcommand{\rset}{\mathbf{RSet}}
\newcommand{\ifix}{\mathbf{ifix}}

\newcommand{\event}{MSFP~’14}

\title{Categorical Semantics for Functional Reactive Programming\\
       with Temporal Recursion and Corecursion}
\author{Wolfgang Jeltsch
        \institute{TTÜ Küberneetika Instituut\\Tallinn, Estonia}
        \email{wolfgang@cs.ioc.ee}}
\def\titlerunning{Categorical Semantics for FRP with Temporal Recursion and
                  Corecursion}
\def\authorrunning{Wolfgang Jeltsch}

\begin{document}

\maketitle

\begin{abstract}

Functional reactive programming (FRP) makes it possible to express temporal
aspects of computations in a declarative way. Recently we developed two kinds of
categorical models of FRP: abstract process categories (APCs) and concrete
process categories (CPCs). Furthermore we showed that APCs generalize CPCs. In
this paper, we extend APCs with additional structure. This structure models
recursion and corecursion operators that are related to time. We show that the
resulting categorical models generalize those CPCs that impose an additional
constraint on time scales. This constraint boils down to ruling out
-supertasks, which are closely related to Zeno’s paradox of Achilles and the
tortoise.

\end{abstract}



\extsection{introduction}{Introduction}

Functional reactive programming (FRP) makes it possible to express temporal
aspects of computations in a declarative way. Traditional FRP is based on
behaviors and events, which denote time-varying values and values attached to
times, respectively. There is a Curry–Howard correspondence between traditional
FRP and an intuitionistic temporal logic with “always” and “eventually”
modalities~\cite{jeltsch:entcs-286,jeffrey:plpv-2012}. Thereby the type
constructor for behaviors corresponds to “always,” and the type constructor for
events corresponds to “eventually.”

Extending the temporal logic with “until” operators leads to an extended variant
of FRP. Thereby proofs of “until” propositions correspond to a new class of FRP
constructs, which we call processes. Processes in the FRP sense combine
continuous and discrete aspects and generalize behaviors and events
\cite[Section~2]{jeltsch:plpv-2013}. We give an introduction to FRP with
processes in \sectionref{functional-reactive-programming-with-processes}.

We have developed two kinds of models of FRP with processes, which can also be
used to model temporal logic with “until:”
\begin{description}

\item[Abstract process categories (APCs)~\cite{jeltsch:plpv-2014}]

are defined purely axiomatically. They are an extension of temporal
categories~\cite{jeltsch:entcs-286}, which in turn build on categorical models
of intuitionistic S4~\cite{kobayashi:tcs-175-1,bierman:sl-65-3}.

\item[Concrete process categories (CPCs)~\cite{jeltsch:plpv-2013}]

are not defined in a purely axiomatic manner, but use concrete constructions to
express time-dependence of type inhabitance and causality of FRP operations.

\end{description}
Abstract process categories generalize concrete process categories. We describe
APCs in \sectionref{abstract-process-categories} and CPCs in
\sectionref{concrete-process-categories}.

The aim of this paper is to extend APCs and CPCs in order to model recursion and
corecursion on processes. We make the following contributions:
\begin{itemize}

\item

In \sectionref{apcs-with-recursion-and-corecursion}, we develop APCs with
recursion and corecursion (ℛ-APCs). ℛ-APCs extend APCs with recursive comonad
and completely iterative monad structures that model recursion and corecursion
on processes. The extensions to APCs arise naturally as extensions of ideal
comonad and ideal monad structures that APCs already contain.

\item

In \sectionref{cpcs-with-recursion-and-corecursion}, we develop CPCs with
recursion and corecursion (ℛ-CPCs). ℛ-CPCs differ from CPCs in that they impose
an additional constraint on time scales. This constraint boils down to ruling
out -supertasks~\cite{laraudogoitia:supertasks}. We show that ℛ-APCs
generalize ℛ-CPCs.

\end{itemize}

We discuss related work in \sectionref{related-work} and give conclusions and an
outlook on further work in \sectionref{conclusions-and-further-work}.

\extsection{functional-reactive-programming-with-processes}
           {Functional Reactive Programming with Processes}

The FRP dialect we consider here is based on a linear notion of time. However
the time scale does not need to be discrete; so the time domain can be any
totally ordered set. Type inhabitance is time-dependent, which means that every
FRP type corresponds to a time-indexed family of sets of inhabitants.

Besides the ordinary type constructors , , , , and~, our FRP
dialect has two binary type constructors  and~, which we call the
strong and the weak basic process type constructor. Both have higher precedence
than the other binary type constructors.

A value that inhabits a type  at a time~ corresponds to a tuple
with the following elements:
\begin{itemize}

\item

a time~

\item

a function~ that maps each time  with  to a value that
inhabits  at~

\item

a value~ that inhabits  at~

\end{itemize}
The function~ denotes a time-varying value that exists between times 
and~. We call this time-varying value the continuous part of the process.
The pair  denotes an event that occurs at time~ and carries the
value~. We call this event the terminal event of the process.

A value that inhabits a type  at a time~ is either a process that
inhabits  at~ or a time-varying value of type~ that starts
immediately after~ and persists forever. We regard constructs of the latter
kind as special processes that never terminate and thus have an infinite
continuous part and no terminal event.


An inhabitant of a type  or  has a continuous part that
assigns values only to future times. We define type constructors  and~
for processes that start at the present time:

A pair  represents a process that starts with~ as the initial value
of its continuous part, and continues like the process~.

An inhabitant of a type  or  can only terminate in the
future. We define type constructors  and~ for processes that may
terminate at the present time:

A value  represents a process whose terminal event occurs immediately and
carries the value~, while a value  represents the process~, which
does not terminate immediately.

From the process type constructors, we derive the type constructors , ,
, and~ as follows:

Inhabitants of types  and  are non-terminating time-varying values and
are called behaviors. Inhabitants of types  and  are events.

The FRP dialect described above corresponds to an intuitionistic temporal logic
via a Curry–Howard isomorphism. Thereby time-dependent type inhabitance is
related to time-dependent trueness of temporal propositions. The type
constructors  and~ correspond to the strong and weak “until” operators
 and~ from linear-time temporal logic (LTL), and the type constructors 
and~ correspond to the “always” and “eventually” modalities.

\extsection{abstract-process-categories}{Abstract Process Categories}

Abstract process categories (APCs) are axiomatically defined categorical models
of FRP with processes, which we developed recently~\cite{jeltsch:plpv-2014}.
They are an extension of temporal categories, which are models of FRP with
behaviors and events, but without processes~\cite{jeltsch:entcs-286}. In this
section, we give an introduction to APCs.

\extsubsection{the-basics}{The Basics}

An APC is a category~ with some additional structure. FRP types are modeled
by objects of~. If objects  and~ model FRP types  and~, the
morphisms from~ to~ model operations that turn every value that
inhabits~ at some time into a value that inhabits~ at the same time.

Since FRP has the usual type constructors for finite products, finite sums, and
function spaces, we require~ to be a cartesian closed category (CCC) with
finite coproducts.

\extsubsection{temporal-functors}{Temporal Functors}

Let  denote the interval category, that is, the category with exactly two
objects  and~ and a single non-identity morphism . Let
furthermore  be the category . We require the existence of a
functor , which we call the basic temporal functor. We use the
notation  for . We model the process type constructors
 and~ by the partial functor applications  and~, which are
themselves functors from  to~.

Every inhabitant of a type  corresponds to an inhabitant of . So there should be an operation that performs a type conversion from 
to~. We use the natural transformation  to model this
operation.

From the functor~, we derive functors  that model the type
constructors , , , and~ as well as functors  that model the type constructors for behaviors and events:

We call the functors introduced in \eqref{equation:derived-process-functors}
through~\eqref{equation:event-functors} derived temporal functors.

APCs do not necessarily use the category~ to model how weak a process type
constructor is. Any category~ with finite products is allowed instead. In
particular, we can use any category that corresponds to a partially ordered set
with finite meets. This makes it possible to model more advanced termination
properties. An example of such a property is termination with an upper bound on
the termination time. We discuss this example in
\subsectionref{the-basic-temporal-functor}.

\extsubsection{process-expansion-and-joining}{Process Expansion and Joining}

An APC contains a natural transformation~ with

This natural transformation models an FRP operation that turns a process~
into a process~ such that the following holds:
\begin{itemize}

\item

The process~ terminates if and only if~ terminates.

\item

The terminal event of~, if any, is the terminal event of~.

\item

The value of the continuous part of~ at a time~ is the suffix of~ that
starts at~.

\end{itemize}
We call this FRP operation process expansion. We derive variants of~ that
work with  and~ instead of~:
\topsep]
θ_{W, A, B}  & \relwithsizeof=: A ▷_W B → \left(A ▷′_W B\right) ▷_W B                      \notag\\
\label{equation:no-prime-theta}
θ_{W, A, B}  & =                \id_B + θ′_{W, A, B}

ϑ″_{W, A, B} : A ▷″_W (A ▷_W B) → A ▷″_W B\enspace.

ϑ′_{W, A, B} & \relwithsizeof=: A ▷′_W (A ▷_W B) → A ▷′_W B                   \notag\\
\label{equation:single-prime-vartheta}
ϑ′_{W, A, B} & =                \id_A × ϑ″_{W, A, B}                          \

The natural transformations  and~ have to fulfill certain coherence
conditions. Regarding , we require that for all  and ,  is an ideal comonad.

\begin{extdefinition}{ideal-comonad}[Ideal comonad]
Let  be a category with binary products, let  be an endofunctor on~,
and let  be a natural transformation. We define ,
, and~ as follows:

The pair  is an ideal comonad on~ if and only if the diagram in
\figureref{coherence-of-an-ideal-comonad} commutes.
\end{extdefinition}
\begin{extfigure}{coherence-of-an-ideal-comonad}{Coherence of an ideal comonad}

\begin{tikzpicture}[diagram]

\matrix{
\node [object] (epsilon end)           {};   &
\node [object] (inside source)         {};  &
\node [object] (delta end)             {}; \\
                                                 &
\node [object] (start)                 {};   &
\node [object] (outside delta source)  {};  \\
};

\edges

\path (start)                edge [morphism] node [auto=right] {}       (inside source)
      (inside source)        edge [morphism] node [auto=right] {}      (epsilon end)
      (start)                edge [morphism] node [auto=left]  {} (epsilon end)
      (inside source)        edge [morphism] node [auto=left]  {}      (delta end)
      (start)                edge [morphism] node [auto=right] {}       (outside delta source)
      (outside delta source) edge [morphism] node [auto=right] {}      (delta end);

\end{tikzpicture}

\end{extfigure}

Regarding , we require that the diagram in
\figureref{coherence-of-process-joining} commutes. This coherence condition
implies that every pair  is an ideal
monad, that is, an ideal comonad on~. Note that we obtain a diagram for
the coherence condition of such an ideal monad by taking the diagram in
\figureref{coherence-of-process-joining} and replacing  and~ with 
and~.
\begin{extfigure}{coherence-of-process-joining}{Coherence of process joining}

\begin{tikzpicture}[diagram,column sep=6.3em]

\matrix{
\node [object] (eta start)              {};                 &
\node [object] (inside destination)     {};         &
\node [object] (mu start)               {}; \\
                                                                      &
\node [object] (end)                    {};                 &
\node [object] (outside mu destination) {};         \\
};

\edges

\path (eta start)              edge [morphism] node [auto=left]  {}          (inside destination)
      (inside destination)     edge [morphism] node [auto=left]  {}       (end)
      (eta start)              edge [morphism] node [auto=right] {}     (end)
      (mu start)               edge [morphism] node [auto=right] {} (inside destination)
      (mu start)               edge [morphism] node [auto=left]  {} (outside mu destination)
      (outside mu destination) edge [morphism] node [auto=left]  {}       (end);

\end{tikzpicture}

\end{extfigure}

Finally we require that the diagram in
\figureref{coherence-between-process-expansion-and-joining} commutes. This
ensures that  and~ interact properly.
\begin{extfigure}{coherence-between-process-expansion-and-joining}
                 {Coherence between process expansion and joining}

\begin{tikzpicture}[diagram,column sep=7em]

\matrix{
\node [object] (nested on the right)        {};                                                    &
\node [object] (deeper nested on the left)  {};                        \\
\node [object] (not nested)                 {};                                                            &
                                                                                                                     \\
\node [object] (nested on the left)         {};                                        &
\node [object] (deeper nested on the right) {}; \\
};

\edges

\path (nested on the right)        edge [morphism] node [auto=left]  {}            (deeper nested on the left)
      (deeper nested on the left)  edge [morphism] node [auto=left]  {} (deeper nested on the right)
      (deeper nested on the right) edge [morphism] node [auto=left]  {}           (nested on the left)
      (nested on the right)        edge [morphism] node [auto=right] {}                  (not nested)
      (not nested)                 edge [morphism] node [auto=right] {}                  (nested on the left);

\end{tikzpicture}

\end{extfigure}

\extsubsection{process-merging-and-the-canonical-nonterminating-process}
              {Process Merging and the Canonical Nonterminating Process}

There are additional constraints on APCs, which ensure that an APC also models
two other operations: process merging and the construction of the sole value
of~, which is called the canonical nonterminating process. We do not
discuss these additional constraints here, because they are not fundamentally
related to our extension of APCs. We just mention them in the definition of APCs
in the next subsection.

\extsubsection{abstract-process-categories-summary}{Summary}

Let us now state the complete definition of abstract process categories.

\begin{extdefinition}{abstract-process-category}[Abstract process category]
Let  be a category with finite products, and let  be a CCC with finite
coproducts. Furthermore let  be a functor from~ to~, let 
and~ be defined as in \eqref{equation:derived-process-functors}, and let 
and~ be natural transformations with the following typings:

The tuple  is an abstract process category (APC) if and only
if the following propositions hold:
\begin{itemize}

\item

For all  and , 
is an ideal comonad.

\item

For all  and , the diagrams in Figures
\ref{figure:coherence-of-process-joining}
and~\ref{figure:coherence-between-process-expansion-and-joining} commute where
, , and~ are defined as in \eqref{equation:no-prime-theta},
\eqref{equation:single-prime-vartheta}, and~\eqref{equation:no-prime-vartheta}.

\item

The natural transformation  with

is an isomorphism.

\item

The morphism  is an isomorphism.

\end{itemize}
\end{extdefinition}

The third and fourth of the above propositions refer to process merging and the
canonical nonterminating process mentioned in
\subsectionref{process-merging-and-the-canonical-nonterminating-process}.

\extsection{apcs-with-recursion-and-corecursion}
           {APCs with Recursion and Corecursion}

In \subsectionref{process-expansion-and-joining}, we saw that every APC
comprises ideal comonads and ideal monads based on temporal functors. There
exists a special kind of ideal comonads, called recursive comonads, which
captures a form of comonadic recursion. Likewise there exists a special kind of
ideal monads, called completely iterative monads, which captures a form of
monadic corecursion. In this section, we introduce APCs with recursion and
corecursion (ℛ-APCs), which comprise recursive comonads and completely iterative
monads that model recursion and corecursion on processes.

\extsubsection{corecursion-on-processes}{Corecursion on Processes}

If  is an APC, then every pair  is an ideal monad. We extend the structure of APCs by requiring
that every such pair is a completely iterative monad. Completely iterative
monads are defined, for example, by Milius
\cite[Definition~5.5]{milius:ic-196-1}. We use a definition that is different
from the one by Milius, but nevertheless equivalent to it. We compare both
definitions in
\appendixref{changes-in-the-completely-iterative-monad-definition}, where we
also list the advantages that our definition has in our opinion.

\begin{extdefinition}{completely-iterative-monad}[Completely iterative monad]
Let  be a category with binary coproducts. A pair  is a completely
iterative monad on~ if and only if it is an ideal monad on~, and for every
morphism , there exists a unique morphism  for
which the diagram in
\figureref{condition-for-a-morphism-f-infinity-of-a-completely-iterative-monad}
commutes.
\end{extdefinition}
\begin{extfigure}{condition-for-a-morphism-f-infinity-of-a-completely-iterative-monad}
                 {Condition for a morphism~ of a completely iterative monad}

\begin{tikzpicture}[diagram,column sep=7em]

\matrix{
\node [object] (start)            {};           &
\node [object] (end)              {};         \\
\node [object] (after expansion)  {};   &
\node [object] (after inner call) {}; \\
};

\edges

\path (start)            edge [morphism] node [auto=left]  {}             (end)
      (start)            edge [morphism] node [auto=right] {}               (after expansion)
      (after expansion)  edge [morphism] node [auto=right] {} (after inner call)
      (after inner call) edge [morphism] node [auto=right] {}            (end);

\end{tikzpicture}

\end{extfigure}

By requiring that every pair  is a
completely iterative monad, we ensure that for every morphism , there is a corresponding morphism . Let  and~
be the FRP operations that  and~ model, and let  be a value from the
domain of  and~. The diagram in
\figureref{condition-for-a-morphism-f-infinity-of-a-completely-iterative-monad}
tells us how the process  is defined:
\begin{itemize}

\item

If  terminates, and its terminal event carries a value , then
 terminates at the same time as , the continuous part of 
is the continuous part of , and the terminal event of  carries the
value~.

\item

If  terminates, and its terminal event carries a value , then
 is the result of concatenating the continuous part of  and the
process~.

\item

If  does not terminate, then  does not terminate, and the
continuous part of  is the continuous part of .

\end{itemize}
Note that in the second case,  is defined in terms of , but
only the suffix of  that starts at the termination time of 
actually depends on . The operator~ models corecursion on
processes.

We derive a variant of~ that works with~ instead of~. We use the
notation  also for this variant. This overloading of notation is
possible, because we can always deduce from the types which variant of~
is meant. The -variant turns every morphism  into a
morphism~ that is defined as follows:

Note that the  on the right-hand side refers to the original
-variant. There is also a variant of~ that works with~. It turns
every morphism  into a morphism~ that is
defined as follows:

Again the  on the right-hand side refers to the original -variant.

\extsubsection{recursion-on-processes}{Recursion on Processes}

We saw in \subsectionref{process-expansion-and-joining} that every pair
 is an ideal comonad. We extend the
structure of APCs by requiring that every such pair is a recursive comonad. The
notion of recursive comonad is dual to the notion of completely iterative monad.
So a recursive comonad on~ is just a completely iterative monad on~.
We give an explicit definition nevertheless.

\begin{extdefinition}{recursive-comonad}[Recursive comonad]
Let  be a category with binary products. A pair  is a recursive
comonad on~ if and only if it is an ideal comonad on~, and for every
morphism , there exists a unique morphism  for
which the diagram in
\figureref{condition-for-a-morphism-f-star-of-a-recursive-comonad} commutes.
\end{extdefinition}
\begin{extfigure}{condition-for-a-morphism-f-star-of-a-recursive-comonad}
                 {Condition for a morphism~ of a recursive comonad}

\begin{tikzpicture}[diagram,column sep=7em]

\matrix{
\node [object] (end)               {};           &
\node [object] (start)             {};         \\
\node [object] (before collapsing) {};   &
\node [object] (before inner call) {}; \\
};

\edges

\path (start)             edge [morphism] node [auto=right] {}                        (end)
      (before collapsing) edge [morphism] node [auto=left]  {}                          (end)
      (before inner call) edge [morphism] node [auto=left]  {} (before collapsing)
      (start)             edge [morphism] node [auto=left]  {}                       (before inner call);

\end{tikzpicture}

\end{extfigure}

By requiring that every pair  is a
recursive comonad, we ensure that for every morphism ,
there is a corresponding morphism . Let  and~ be the
FRP operations that  and~ model, and let  be a process from the
domain of~. The diagram in
\figureref{condition-for-a-morphism-f-star-of-a-recursive-comonad} tells us that
the value  is  where  is defined by the
following statements:
\begin{itemize}

\item

The process~ terminates if and only if~ terminates.

\item

The terminal event of~, if any, is the terminal event of~.

\item

The value of the continuous part of~ at a time~ is  where
 is the value of the continuous part of~ at~, and  is the suffix
of~ that follows~.

\end{itemize}
Note that  is defined in terms of , but  is a proper suffix
of~. The operator~ models recursion on processes.

We derive a variant of~ that works with~ instead of~. We use the
notation  also for this variant. The -variant turns every morphism  into a morphism~ that is defined as follows:

Note that this construction is dual to the construction of the -variant
of~.

\extsubsection{apcs-with-recursion-and-corecursion-summary}{Summary}

We formulate the definition of ℛ-APCs based on the above explanations.

\begin{extdefinition}{apc-with-recursion-and-corecursion}
                     [APC with recursion and corecursion]
An APC with recursion and corecursion (ℛ-APC) is an APC  for
which the following holds:
\begin{itemize}

\item

For all  and , 
is a recursive comonad.

\item

For all  and , 
is a completely iterative monad.

\end{itemize}
\end{extdefinition}

\extsection{concrete-process-categories}{Concrete Process Categories}

Concrete process categories (CPCs) are models of FRP with processes that use
concrete categorical constructions to express time-dependence of type
inhabitance and causality of FRP operations. They are described in detail in an
earlier publication of ours \cite[Section~3]{jeltsch:plpv-2013}. Here we only
give a short introduction to CPCs.

\extsubsection{core-structure}{Core Structure}

Let  be a totally ordered set, and let  be a CCC with finite
coproducts. We use  to model the time scale, and  to model ordinary
types and functions. Based on , we construct a category~, which we
call the temporal index category of .

\begin{extdefinition}{temporal-index-category}[Temporal index category]
The temporal index category of a totally ordered set  is the
category~ for which the following holds:

\end{extdefinition}

We model FRP types and FRP operations by the functor category~. Let  be
an object of~ that models an FRP type~. For every object 
of~, the object  of~ deals with the FRP values that
inhabit~ at~; it describes the type whose inhabitants give the information
we have about these FRP values at~. We call  the observation
time. For every morphism  of~, the
morphism  models the function that turns
information we have at~ into information we have at~ by
forgetting all information acquired after~

The definition of CPCs requires the category~ to have some additional
properties, which are necessary for modeling function types and process types in
FRP.

\begin{extdefinition}{concrete-process-category}[Concrete process category]
Let  be a totally ordered set, let  be its temporal index category,
and let  be a CCC with finite coproducts that has all products and coproducts
of families indexed by intervals  and all ends of the form  where . Then the tuple  is a concrete process category (CPC). The actual category that  denotes is the functor category~.
\end{extdefinition}

\extsubsection{the-basic-temporal-functor}{The Basic Temporal Functor}

The basic temporal functor of a CPC \cite[Subsection~4.3]{jeltsch:plpv-2014}
models the weak basic process type constructor as well as process type
constructors that put upper bounds on termination times.

We define the totally ordered set  such that  and

Let  be the category of . We use  to model constraints
regarding termination. Thereby any object  stands for
termination at or before the time~, and the object~ stands for
the absence of any termination guarantees. We define the basic temporal functor
as follows.

\begin{extdefinition}{basic-temporal-functor-of-a-cpc}
                     [Basic temporal functor of a CPC]
Let  be a CPC, and let  be defined as above. The basic temporal
functor of~ is the functor  with

where for every~,  is defined as follows:

\end{extdefinition}

The above definition is actually incomplete. A complete definition would also
define the following:
\begin{itemize}

\item

morphisms  where , , and 

\item

morphisms  where , , and 

\item

morphisms  where , , and 

\end{itemize}
However defining these things is a mostly mechanical task. In particular, the
definition of morphisms  can be derived from
\eqref{equation:process-information-objects}
and~\eqref{equation:process-information-objects-helper} by replacing  with
 and object expressions of the form  with morphism
expressions of the form .

\extsubsection{relationship-to-abstract-process-categories}
              {Relationship to Abstract Process Categories}

APCs generalize CPCs. This is expressed by the following theorem.

\begin{exttheorem}{apc-from-cpc}
Let  be a CPC, let  be the temporal index category of ,
let  be the basic temporal functor of , and let  be the
category of the totally ordered set  that is defined as described in
\subsectionref{the-basic-temporal-functor}. Then there exist natural
transformations  and~ such that  is an
APC.
\end{exttheorem}

Please see our earlier work \cite[proof of Theorem~10]{jeltsch:plpv-2014} for a
proof of this theorem.

\extsection{cpcs-with-recursion-and-corecursion}
           {CPCs with Recursion and Corecursion}


According to \theoremref{apc-from-cpc}, every CPC gives rise to an APC. However
there is no guarantee that this APC is also an ℛ-APC. In this section, we
develop CPCs with recursion and corecursion (ℛ-CPCs). ℛ-CPCs are a special kind
of CPCs whose corresponding APCs are ℛ-APCs.

\extsubsection{enabling-corecursion-on-processes}
              {Enabling Corecursion on Processes}

Let  be a CPC. Let furthermore  be
the APC it induces according to \theoremref{apc-from-cpc}. This APC models
corecursion on processes if and only if for every morphism , there exists a unique morphism~ for which the
following holds:

From the definition of~ in \eqref{equation:derived-process-functors} and the
definition of the basic temporal functor in
\subsectionref{the-basic-temporal-functor}, it follows that
\eqref{equation:f-infinity-for-apc} is true if and only if for all  with , we have

where  is defined as follows:
\medskipamount]
    \left(∐_{t′ ∈ (t, \obstime]} \left(\id × \left(\id + f^∞_{(t′, \obstime)}\right)\right)\right) + \id &
        \text {if }
    \end{cases}

f^* & \relwithsizeof=: A ▷″_{\timebound} B → C                                  \notag\\
\label{equation:f-star-for-apc}
f^* & =                f                                                      ∘
                       \left((\id_A × f^*) ▷″_{\id_{\timebound}} \id_B\right) ∘
                       θ″_{\timebound, A, B}

\label{equation:f-star-element-definition}
f^*_{(t, \obstime)} = f_{(t, \obstime)}                                  ∘
                      h                                                  ∘
                      \left(θ″_{\timebound, A, B}\right)_{(t, \obstime)}

h      & = \begin{cases}
\id_0                                                                                          &
               \text {if }                                                                \\
           ∐_{t′ ∈ (t, \timebound]} s_{t′}                                                                &
               \text {if }                                                     \\
           ∐_{t′ ∈ (t, \obstime]} s_{t′} + ∏_{t′ ∈ (t, \obstime]} \left(\id × f^*_{(t′, \obstime)}\right) &
               \text {if }
\end{cases}                                                                                       \\
\label{equation:last-f-star-element-definition-helper}
s_{t′} & = \left(∏_{t″ ∈ (t, t′)} \left(\id × f^*_{(t″, \obstime)}\right)\right) × \id

g^† & \relwithsizeof=: C → B + T′B                                                         \notag\\
g^† & =                \left(\id_B + \left(μ′_B ∘ T′\left[ι₁, g^†\right]\right)\right) ∘ g

f & \relwithsizeof=: T′(B + C) → T′(B + T′(B + C)) \notag\\
\label{equation:f-in-equivalence-proof-1}
f & =                T′[ι₁, g]

f^∞ & \relwithsizeof=: T′(B + C) → T′B            \notag\\
\label{equation:f-infinity-in-equivalence-proof-1}
f^∞ & =                μ′_B ∘ T′(\id_B + f^∞) ∘ f

\begin{split}
μ′_B ∘ T′(\id_B + f^∞) ∘ f & = μ′_B ∘ T′(\id_B + f^∞) ∘ T′[ι₁, g]        \\
                           & = μ′_B ∘ T′[ι₁, (\id_B + f^∞) ∘ g]\enspace,
\end{split}

f^∞ = μ′_B ∘ T′[ι₁, (\id_B + f^∞) ∘ g]

& q \relwithsizeof=: \Hom(T′(B + C), T′B) → \Hom(C, B + T′B) \notag\\
& q(a) =             (\id_B + a) ∘ g                      \
We know that there exists a unique  with

and we have to prove that there exists a unique  with

From \eqref{equation:f-infinity-simplified-in-equivalence-proof-1}, it follows
that

so  fulfills
\eqref{equation:g-dagger-simplified-in-equivalence-proof-1}. On the other hand,
if we have a  that fulfills
\eqref{equation:g-dagger-simplified-in-equivalence-proof-1}, we know that

Because there is only one  that fulfills
\eqref{equation:f-infinity-simplified-in-equivalence-proof-1}, it follows that
. From this, we get

So  is the unique solution of
\eqref{equation:g-dagger-simplified-in-equivalence-proof-1}.
\end{extproof}

\begin{exttheorem}{milius-cim-is-jeltsch-cim}
If  is a Milius-style completely iterative monad on a category~,
then it is a completely iterative monad on~ according to
\definitionref{completely-iterative-monad}.
\end{exttheorem}

\begin{extproof}
Let  be a Milius-style completely iterative monad, and let  be an
arbitrary morphism with . We have to show that there exists a
unique morphism~ for which the following holds:

We define  as follows:

Since  is a Milius-style completely iterative monad, there exists a
unique  for which the following holds:

According to \eqref{equation:g-in-equivalence-proof-2}, we have

so \eqref{equation:g-dagger-in-equivalence-proof-2} is equivalent to the
following equation:

We define the functions  and~ on morphisms as follows:
\topsep]
& r \relwithsizeof=: \Hom(C, T′B) → \Hom(C, B + T′B) \notag\\
& r(b) =             ι₂ ∘ b

g^† = r\left(q\left(g^†\right)\right)\enspace,

\label{equation:f-infinity-simplified-in-equivalence-proof-2}
f^∞ = q(r(f^∞))\enspace.

Using reasoning analogous to the one at the end of the previous proof, we can
deduce that  is the unique solution of
\eqref{equation:f-infinity-simplified-in-equivalence-proof-2}.
\end{extproof}

\section*{Acknowledgments}

I thank Tarmo Uustalu for all the helpful discussions about the topics of this
paper, and in particular, for introducing me to completely iterative monads and
recursive comonads.

This work was supported by the target-financed research theme No.~0140007s12 of
the Estonian Ministry of Education and Research and by the ERDF through the
Estonian Center of Excellence in Computer Science (EXCS) and the national ICTP
project \emph{Coinduction for Semantics, Analysis, and Verification of
Communicating and Concurrent Reactive Software}. I thank all tax payers in the
European Union for funding the ERDF.

\bibliographystyle{eptcs}
\bibliography{references/references}

\end{document}
