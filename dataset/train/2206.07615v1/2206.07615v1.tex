\pdfoutput=1


\documentclass[11pt]{article}

\usepackage[]{acl}

\usepackage{times}
\usepackage{latexsym}

\usepackage[T2A,T1]{fontenc}


\usepackage[utf8]{inputenc}
\usepackage[russian,english]{babel}

\usepackage{microtype}
\usepackage{tabularx}
\usepackage{multirow}

\newcommand{\todo}[1]{\textcolor{red}{(\textbf{TODO:} #1)}}

\usepackage{gb4e} \noautomath

\newcommand{\md}[1]{{\texttt{#1}}}

\usepackage{tikz}
\def\checkmark{\tikz\fill[scale=0.4](0,.35) -- (.25,0) -- (1,.7) -- (.25,.15) -- cycle;} 


\title{The SIGMORPHON 2022 Shared Task on Morpheme Segmentation}



\author{Khuyagbaatar Batsuren \and G{\'a}bor Bella \and Aryaman Arora \and {\bf Viktor Martinovic} \\ {\bf Kyle Gorman} \and {\bf Zdeněk Žabokrtský} \and {\bf Amarsanaa Ganbold} \and {\bf Šárka Dohnalová} \\ {\bf Magda Ševčíková} \and {\bf Kateřina Pelegrinová} \and {\bf Fausto Giunchiglia} \and {\bf Ryan Cotterell} \\ \and {\bf Ekaterina Vylomova} \\ National University of Mongolia \hspace{.1cm} University of Trento \hspace{.1cm} Georgetown University \\ University of Vienna \hspace{.1cm} Graduate center, City University Of New York \hspace{.1cm} Charles University \\ University of Ostrava \hspace{.1cm} ETH Zürich \hspace{.1cm} University of Melbourne \\ 
}

\begin{document}
\maketitle
\begin{abstract}
The SIGMORPHON 2022 shared task on morpheme segmentation challenged systems to decompose a word into a sequence of morphemes and covered most types of morphology: compounds, derivations, and inflections. Subtask 1, word-level morpheme segmentation, covered 5 million words in 9 languages (Czech, English, Spanish, Hungarian, French, Italian, Russian, Latin, Mongolian) and received 13 system submissions from 7 teams and the best system averaged 97.29\% F1 score across all languages, ranging English (93.84\%) to Latin (99.38\%). Subtask 2, sentence-level morpheme segmentation, covered 18,735 sentences in 3 languages (Czech, English, Mongolian), received 10 system submissions from 3 teams, and the best systems outperformed all three state-of-the-art subword tokenization methods (BPE, ULM, Morfessor2) by 30.71\% absolute.
To facilitate error analysis and support any type of future studies, we released all system predictions, the evaluation script, and all gold standard datasets.\footnote{\url{https://github.com/sigmorphon/2022SegmentationST}}

\end{abstract}

\section{Introduction}
Many NLP applications, such as machine translation or question answering, require \emph{subword tokenization}, i.e. splitting words into a sequence of substrings \cite{mielke2021between}. Such tokenizers are trained by an unsupervised algorithm, usually either Byte-Pair Encoding (BPE; \citealt{gage1994new,sennrich2016neural}) or Unigram Language Modeling (ULM; \citealt{kudo-2018-subword}). To give a few examples, contemporary language models RoBERTa \cite{liu2019roberta} and GPT-3 \cite{brown2020language} use a byte-level BPE \cite{radford2019language} while XLNet \cite{yang2019xlnet} relies on ULM. These subword tokenization algorithms are not linguistically motivated but are rather based on statistical co-occurrences. Therefore, unsupervised and semi-supervised methods for morphological segmentation \cite{creutz2005unsupervised} have emerged in parallel, state-of-the-art methods of this kind being Morfessor variants \cite{gronroos2014morfessor,gronroos2020morfessor}. \citet{ataman2017linguistically} and \citet{schwartz2020neural} find that Morfessor-based language models can outperform BPE-based ones. \citet{matthews2018using,nzeyimana2022kinyabert} show that enriching BPE with morphological analyzers can be beneficial for translation, while many others \cite{domingo2018much,machavcek2018morphological,schwartz2020neural,saleva-lignos-2021-effectiveness} find no conclusive improvements over BPE for machine translation. 


\begin{table}[t]
\small
\newcolumntype{R}{>{\raggedleft\arraybackslash}X}
\newcolumntype{C}{>{\centering\arraybackslash}X}
\newcolumntype{L}{>{\raggedright\arraybackslash}X}\centering
\begin{tabularx}{\linewidth}{l|ll|r}

System & type   & motivation   & segmentation             \\
\hline
\hline
BPE  & surface & sta. & in | val | uable       \\
Morfessor2  & surface    & sta. \& lin. & in | valuable         \\
DeepSPIN-3  & canonical & sta. \& lin. & in | value | able      \\

\hline
\end{tabularx}
\caption{\label{tab:examples} Structural differences of subword tokenization (BPE), morphological segmentation (Morfessor2), and morpheme segmentation (DeepSPIN-3 -- subtask 1 winning system); acronyms: sta. - statistics and lin. - linguistic}
\end{table}

One of the core problems is that the state-of-the-art morphological segmentation and subword tokenization algorithms provide  ``surface-level'' segmentation, which has several theoretical drawbacks  with respect to ``canonical'' segmentation (e.g., segmented substrings are not considered as meaningful as morphemes). \citet{cotterell2016joint} provided formal definitions for both:
given a word , its ``surface'' segmentation is a sequence of \textit{surface substrings} the concatenation of which is~, e.g., \textit{funniest} → \textit{funn-i-est}. The purpose of canonical segmentation \cite{kann-etal-2016-neural,Task2-TueSeg}, on the other hand, is not only computing surface segmentation but also restoring standardized forms of morphemes, e.g., \textit{funniest} → \textit{fun-y-est}. More detailed structural distinctions between these segmentation types are shown in Table~\ref{tab:examples}. 

However, state-of-the-art studies in canonical segmentations have been limited to very low numbers of languages with sufficiently rich morphological resources \cite{kurimo2010morpho,kurimo2010proceedings,cotterell2016joint,kann-etal-2018-fortification}. With the goal of advancing research in this direction, we present a \textit{morpheme segmentation shared task} and provide large-scale datasets over nine languages, evaluation metrics, and morphological annotations of five million word formations. In this, we rely on the latest release of UniMorph \cite{batsuren2022unimorph} which has introduced morpheme segmentations and derivational data from MorphyNet \cite{batsuren-etal-2021-morphynet}. The resulting shared task is a follow-up to past morphological segmentation shared tasks such as ``MorphoChallenge'' \cite{kurimo2007unsupervised,kurimo2008overview,kurimo2009overview} or ``Multilingual parsing''  \cite[where lemmatization as segmentation is a subtask]{zeman2017conll}.












\begin{table}[t]
\small
\newcolumntype{R}{>{\raggedleft\arraybackslash}X}
\newcolumntype{C}{>{\centering\arraybackslash}X}
\newcolumntype{L}{>{\raggedright\arraybackslash}X}\centering
\begin{tabularx}{\linewidth}{clLc}
\hline
Lang & word      & segmentation             & category \\
\hline
\multirow{2}{*}{eng}  & sheepiness & sheep @@y @@ness   & 010      \\
  & pokers     & poke @@er @@s      & 110      \\
\hline
\multirow{2}{*}{hun}  & időpontod  & idő @@pont @@od    & 101      \\
  & szőttetek  & sző @@tt @@etek    & 100      \\
\hline
\multirow{2}{*}{mon}  & \foreignlanguage{russian}{харах}      & \foreignlanguage{russian}{харах}              & 000     \\
  & \foreignlanguage{russian}{гэмтлийг}   & \foreignlanguage{russian}{гэмтэх @@л @@ийг} & 110      \\

\hline
\end{tabularx}
\caption{\label{tab:training_examples} Training samples for Subtask 1. Each sample consists of a word, its canonical segmentation, and a category encoding word formation processes.}
\end{table}



\begin{table*}[h]
\small
\newcolumntype{R}{>{\raggedleft\arraybackslash}X}
\newcolumntype{C}{>{\centering\arraybackslash}X}
\newcolumntype{L}{>{\raggedright\arraybackslash}X}\centering
\begin{tabularx}{\textwidth}{|c|ccc|l|L|}
\hline
Category & Infl. & Deri. & Comp. & Description  & English example (input ==\textgreater ~ output)     \\
\hline
000        & - & - & - &Root words (free morphemes)                     & progress ==\textgreater ~progress                  \\
100        & \checkmark & - & - & Inflection only                  & prepared ==\textgreater  ~ prepare @@ed                   \\
010        & - & \checkmark & - &Derivation only                  & intensive ==\textgreater ~intense @@ive                   \\
001        & - & - & \checkmark &Compound only                    & hotpot ==\textgreater ~hot @@pot                   \\
101        & \checkmark & - & \checkmark & Inflection and Compound          & wheelbands ==\textgreater ~wheel @@band @@s        \\
011        & - & \checkmark & \checkmark & Derivation and Compound          & tankbuster ==\textgreater ~tank @@bust @@er        \\
110        & \checkmark & \checkmark & - & Inflection and Derivation        & urbanizes ==\textgreater ~urban @@ize @@s          \\
111        & \checkmark & \checkmark & \checkmark & Inflection, Derivation, Compound & trackworkers ==\textgreater ~track @@work @@er @@s\\
\hline
\end{tabularx}
\caption{\label{tab:categories}Morphological categories and descriptions of segmented words in subtask 1}
\end{table*}

\begin{table*}[ht]
\small
\newcolumntype{R}{>{\raggedleft\arraybackslash}X}
\newcolumntype{C}{>{\centering\arraybackslash}X}
\newcolumntype{L}{>{\raggedright\arraybackslash}X}\centering
\begin{tabularx}{0.95\textwidth}{|C|rrrrrrrrr|}
\hline
Category & English & Spanish & Hungarian & French & Italian & Russian & Czech & Latin  & Mongolian \\
\hline
000         & 101938  & 15843   & 6952      & 13619  & 21037   & 2921    & -     & 50338  & 1604      \\
100         & 126544  & 502229  & 410662    & 105192 & 253455  & 221760  & -     & 831991 & 7266      \\
010         & 203102  & 18449   & 24923     & 67983  & 41092   & 72970   & -     & 0      & 2201      \\
001         & 16990   & 248     & 3320      & 1684   & 431     & 259     & -     & 0      & 5         \\
101         & 13790   & 458     & 101189    & 478    & 317     & 1909    & -     & 0      & 35        \\
011         & 5381    & 82      & 1654      & 506    & 140     & 328     & -     & 0      & 0         \\
110         & 106570  & 346862  & 323119    & 126196 & 237104  & 481409  & -     & 0      & 7855      \\
111         & 3059    & 343     & 54279     & 186    & 158     & 2658    & -     & 0      & 0         \\
\hline
total words & 577374  & 884514  & 926098    & 382797 & 553734  & 784214  & 38682 & 882329 & 18966    \\
\hline
\end{tabularx}
\caption{\label{tab:task1stats}Word statistics across morphological categories on subtask 1} 
\end{table*}



\section{Task and Evaluation Details}
\subsection{Subtask 1: Word-level Morpheme Segmentation}
In subtask 1, participating systems were asked to segment a given word into a sequence of morphemes. The participants were initially provided with examples of segmentation to train and fine-tune their systems, as shown in Table~\ref{tab:training_examples}. Each instance in the training set is a triplet consisting of a word, a sequence of morphemes, and a morphological category specifying the types of word formation (see Table~\ref{tab:categories}). The morphological category is an optional feature that can only be used to oversample or undersample the training dataset (the word frequencies are imbalanced across the morphological categories, e.g., Italian has 431 compound words and 253K inflections). The test data only contained the initial word itself.


Key points of this subtask are:
\begin{itemize}
\item The task is focusing on canonical segmentation, i.e. given an input word, participants had to predict \emph{a sequence of morphemes}. In canonical segmentation, the participating systems need to reconstruct internal morphophonological processes involved in word formation. For example, the word ``intensive'' will be decomposed into the base form ``intens\textit{\textbf{e}}''  and the adjectival siffix `@@ive'' (note that the ending `\textit{\textbf{e}}' of the base word is inferred here);
\item As shown in Table~\ref{tab:task1stats}, the task is multilingual, with seven high-resource languages (English, Spanish, Hungarian, French, Italian, Russian, Latin) and two low-resource languages (Czech and Mongolian);
\item The annotated corpus data represents a variety of morphological phenomena, including inflection, derivation, compounding (Table \ref{tab:task1stats});
\item A large-scale coverage as segmentations of five million words.
\end{itemize}






\subsection{Subtask 2: Sentence-level Morpheme Segmentation}
The second subtask is a context-dependent morpheme segmentation and focuses on resolving ambiguity in segmentations. Consider the following example containing a Mongolian homonym:
\begin{exe}
\ex
\glll \foreignlanguage{russian}{Гэрт} \foreignlanguage{russian}{эмээ} \foreignlanguage{russian}{хоол} \foreignlanguage{russian}{хийв}\\
\foreignlanguage{russian}{Гэр @@т} \foreignlanguage{russian}{эмээ} \foreignlanguage{russian}{хоол} \foreignlanguage{russian}{хийх @@в} \\
Home.\texttt{DAT} grandma meal cook.\texttt{PRS.PRF} \\
\glt `Grandma just cooked a meal at home.'
\end{exe}

\begin{exe}
\ex
\glll \foreignlanguage{russian}{Би} \foreignlanguage{russian}{өдөр} \foreignlanguage{russian}{эмээ} \foreignlanguage{russian}{уусан }\\
\foreignlanguage{russian}{Би} \foreignlanguage{russian}{өдөр} \foreignlanguage{russian}{эм @@ээ} \foreignlanguage{russian}{уух @@сан} \\
I afternoon medicine.\texttt{PSSD} take.\texttt{PST} \\
\glt `Afternoon I took my medicine.'
\end{exe}

\noindent where ``\foreignlanguage{russian}{эмээ}'' is a homonym of two different words; in the first sentence, it is ``grandmother'', and in the second sentence --- an inflected form of ``medicine''. Thus, the form in the second case can be segmented. However, the modern subword segmentation tools consider no contextual differences in word forms.


Key points of this subtask are:
\begin{itemize}
\item Morpheme segmentation is context-dependent;
\item We organize it for three languages: English, Czech, and Mongolian;
\item For Czech and Mongolian we asked native speakers to manually annotate the data. The details of data collection are provided in Section~\ref{sec:Data}.

\end{itemize}


\begin{table}[t]
\newcolumntype{R}{>{\raggedleft\arraybackslash}X}
\newcolumntype{C}{>{\centering\arraybackslash}X}
\newcolumntype{L}{>{\raggedright\arraybackslash}X}\centering
\begin{tabularx}{0.9\linewidth}{lRRR}
\hline
Language & train & dev  & test \\
\hline
Czech     & 1,000  & 500  & 500  \\
English   & 11,007 & 1,783 & 1,845 \\
Mongolian & 1,000  & 500  & 600 \\

\hline
\end{tabularx}
\caption{\label{tab:task2stats}The number of samples in each language in Subtask 2.}
\end{table}


\subsection{Evaluation}
In order to evaluate and compare the systems, we used four metrics: (i) \textit{\textbf{precision}}, the ratio of correctly predicted morphemes over all predicted morphemes; (ii) \textit{\textbf{recall}}, the ratio of correctly predicted morphemes over all gold-label morphemes; (iii) \textit{\textbf{f-measure}},  the harmonic mean of the precision and recall;
(iv) \textit{\textbf{edit distance}} - average Levenshtein distance between the predicted output and the gold instance. For convenience, we provided the python tool\footnote{\url{https://github.com/sigmorphon/2022SegmentationST/tree/main/evaluation}} to evaluate these metrics on both subtasks. In addition, for subtask 1 this tool also provided detailed results across the morphological categories. 


\section{Data}
\label{sec:Data}
We collected our morphological data from various sources to account for all types of morphology: derivational, inflectional, compounding. We also collected base forms. For derivational and inflectional morphology, we have used the segmentation data from UniMorph 4.0 \cite{batsuren2022unimorph} and MorphyNet \cite{batsuren-etal-2021-morphynet}. UniMorph contains inflectional paradigms collected from linguistic sources as well as Wiktionary, while MorphyNet represents derivations scraped from various editions of Wiktionary. Compounds and base forms were also extracted from Wiktionary (see  Section~\ref{sub:extraction} for more details on the data extraction). We then used the data to produce morpheme segmentations for seven high-resource languages. For Czech and Mongolian, as low-resource languages, we asked native speakers and linguists to develop the resources (Section~\ref{sub:LRL} provides more details). For English sentence data, we have used the universal dependency treebank of English \cite{silveira14gold}. 

\subsection{Data Statistics}
The data for the shared task was moderately multilingual, containing nine unique languages of five genera including Germanic, Italic, Slavic, Mongolic, and Uralic. In subtask 1, we have over 5 million samples of morpheme segmentations that cover nine languages over nine morphological categories, as shown in Table~ \ref{tab:task1stats}. In  subtask 2, Table~\ref{tab:task2stats} displays the data statistics of three languages. 

\subsection{Extraction from Wiktionary}
\label{sub:extraction}
Language-specific editions of Wiktionary contain a considerably large amount of derivations and compounds. 

\emph{Compound extraction rules} were applied to the etymology sections of Wiktionary entries to collect the Morphology template usages, such as for the English \emph{newspaper}:
\begin{center}
     Equivalent to \textbf{news} + \textbf{paper}.
\end{center}
where we have a morphology entry from the Wiktionary XML dump as follows:
\begin{center}
\{\{compound~|~en~|~news~|~paper\}\} 
\end{center}
Most of compound entries use ``compound'' etymology template while some cases use ``affix`` templates, e.g., \emph{basketball} and \emph{volleyball}. 

\emph{Root (and base) word extraction} is a two-step procedure. In the first step we collected words, inherited from earlier phases of corresponding languages. For example, English `book' is traced back to the Middle English `bok', according to the etymology section of Wiktionary. We extracted 279,173 words from 6 languages from CogNet, a cognate database containing 8.1 million cognate pairs of 335 languages from Wiktionary  \cite{batsuren2019cognet,batsuren2021large}. In the second step, we filtered out 116,863 words from the earlier extracted derivational and compound data, resulting in 162,310 root words in 6 languages. Similar Wiktionary data extraction procedures have been applied to a wide range of linguistic data, e.g., etymology  \cite{fourrier2020methodological}, multilingual lexicons - DBnary \cite{serasset2015dbnary} and Yawipa \citep{wu-yarowsky-2020-computational}.  
\begin{table*}[t]
\small
\newcolumntype{R}{>{\raggedleft\arraybackslash}X}
\newcolumntype{C}{>{\centering\arraybackslash}X}
\newcolumntype{L}{>{\raggedright\arraybackslash}X}\centering
\begin{tabularx}{\textwidth}{l|l|l|CcCcc}
     &  &      & \multicolumn{5}{c}{System features} \\
Team     & Description & System     & Neural & Ensemble & Data+ & Multilingual & Multi-task \\
\hline
\hline
\multirow{3}{*}{Baseline} & \cite{schuster2012japanese}           & \md{WordPiece*}        & -      & -        & -     & -            & -          \\
 & \cite{kudo-2018-subword}           & \md{ULM*}        & -      & -        & -     & -            & -          \\
 & \cite{virpioja2013morfessor}           & \md{Morfessor2*}        & -      & -        & -     & -            & -          \\
\hline
\hline
\multirow{6}{*}{AUUH}     &    \multirow{6}{*}{\cite{auuh22sigmorphon}}       & \md{AUUH\_A*}    & \checkmark    &    -      & \checkmark   & \checkmark          & \checkmark        \\
         &             & \md{AUUH\_B*}    & \checkmark    &     -     & -   & \checkmark          & \checkmark        \\
         &             & \md{AUUH\_C}    & \checkmark    &       -   & \checkmark   &          -    & \checkmark        \\
         &             & \md{AUUH\_D}    & \checkmark    &      -    & -   &        -      & \checkmark        \\
         &             & \md{AUUH\_E*}    & \checkmark    &       -   & \checkmark   &        -      &     -       \\
         &             & \md{AUUH\_F*}    & \checkmark    &     -     & -   &        -      &     -       \\
\hline
\hline
\multirow{4}{*}{CLUZH}    &   \multirow{4}{*}{\cite{cluzh_sig22}}          & \md{CLUZH}      & \checkmark    & \checkmark      &    -   &        -      &     -       \\
         &             & \md{CLUZH-1}    & \checkmark    & \checkmark      &   -    &        -      &    -        \\
         &             & \md{CLUZH-2}    & \checkmark    & \checkmark      &     -  &         -     &    -        \\
         &             & \md{CLUZH-3}    & \checkmark    & \checkmark      &    -   &          -    &   -         \\
\hline
\hline
\multirow{3}{*}{DeepSPIN} &  \multirow{3}{*}{\cite{DeepSPIN2022}}           & \md{DeepSPIN-1} & \checkmark    &     -     &  -     &      -        &      -      \\
         &             & \md{DeepSPIN-2} & \checkmark    &    -     &     -  &     -         &   -         \\
         &             & \md{DeepSPIN-3} & \checkmark    &   -       &  -     &       -       &      -      \\
\hline
\hline
\multirow{2}{*}{GU}      &  \multirow{2}{*}{\cite{GU2022}}           & \md{GU-1}       & \checkmark    &     -     & \checkmark   &        -      &     -       \\
         &             & \md{GU-2}       & \checkmark    &      -    & \checkmark   &      -        &   -         \\
\hline
\hline
NUM DI   &  \cite{Task2_NUMDI}           & \md{NUM DI}     & \checkmark    &     -     &     -  &           -   &    -        \\
\hline
\hline
JB132    &  \cite{JB132}          & \md{JB132}      &    -   &       -   &     -  &      -        &         -   \\
\hline
\hline
\multirow{2}{*}{Tü Seg}  &  \multirow{2}{*}{\cite{Task2-TueSeg}}        & \md{Tü\_Seg-1}    & \checkmark    &    -      &     -  &       -     & -

\\ 
&           &  \md{Tü\_Seg-2} & \checkmark    &    -      &     -  &         -     & \checkmark  
\end{tabularx}
\caption{\label{tab:systems}The list of participating systems submitted to the shared task and baseline systems; Systems marked with * are submitted to both subtasks}
\end{table*}

\subsection{Collecting data for Czech and Mongolian}
\label{sub:LRL}
We had two languages with limited amount of data, Czech and Mongolian. For each  language, we used a different development methodology than for the other seven languages
(with larger amount of available data).

\textbf{Mongolian}: we asked two linguists (who are also native speakers of Mongolian) to annotate morpheme segmentations of 3,810 words from Mongolian WordNet \cite{batsuren-etal-2019-building}. After manual annotation, we received 1,604 base forms, 2201 derived forms, and 5 compounds. To account for inflectional morphology, we have used the Mongolian transducer tool \cite{munkhjargal2016morphological} to generate inflected forms of the 3,810 annotated words. In total, we collected morpheme segmentations of 18,966~Mongolian words for subtask~1. For subtask~2, the same two linguists annotated 2,100~Mongolian sentences.

\textbf{Czech}: we merged hand-segmented word forms from four sources for the purpose of subtask 1:
(a) segmentations previously created within DeriNet \cite{derinet-2019}, a project aimed at capturing derivational relations in Czech  (9,508 word forms),
(b) segmentations of Czech verb lemmas imported from a partially digitized version of a printed dictionary (\citealt{slavickova-2017}; 13,162 word forms in addition, i.e. not counting overlaps),
(c) segmentations available in the MorfCzech dataset  \cite{morfoczech-data-2022}, mostly extracted from dictionaries and  grammar books existing for Czech (additional 11,137 word forms), and
(d) word forms that we annotated newly in order to reach complete coverage of Czech subtask 2 sentences (see below; additional 4,887 word forms). In total, the subtask~1 dataset contains 38,694 unique Czech word forms segmented to morphs.

All annotations were performed by native speakers with linguistic education, and underwent careful harmonization if the input resources disagreed, as well as numerous consistency checks. However, because of rich allomorphy in Czech, we have not been able to merge allomorph sets  under more abstract umbrella morphemes so far, and thus words are represented as sequences of morphs (whose concatenation perfectly matches the original word forms), not of morphemes. 

The Czech subtask~2 dataset contains in total 2,000 sentences from the Czech subset of Universal Dependencies (\citealt{ud-cl-2021}; more specifically, 1000, 500, and 500 first sentences from the train, dev, and test sections, respectively, of the Prague Dependency Treebank subset of UD 2.9). Given that homonymy resulting in different morph boundaries is extremely rare in Czech, words are segmented basically regardless of their contexts. 


\subsection{Data Splits}
From each language's collection of morpheme segmentations in subtask 1, we sampled 80\% for the training,  10\% for development, and 10\% for test sets.\footnote{All the data splits can be obtained from \url{ https://github.com/sigmorphon/2022SegmentationST/tree/main/data}} All splits of subtask 1 are balanced w.r.t. the nine morphological categories, described in Table~ \ref{tab:categories}. While sampling the training and development sets for the subtask 1, we excluded words that were present in the test sentences of subtask 2. This was done in order to avoid situations when the subtask 1 data could directly influence the results of subtask 2 (since we allowed the multi-task learnings between both subtasks).  

\section{Baseline Systems}
The shared task provided predictions and results of baseline systems to participants that covered all languages and both subtasks. We chose three baseline systems: 
First is \texttt{WordPiece},  one of the state-of-the-art subword tokenization algorithms used in BERT \cite{devlin-etal-2019-bert}, which is based on \citet{schuster2012japanese} and somewhat resembles BPE \cite{sennrich2016neural}. Second is \texttt{ULM} (Unigram Language Model \citet{kudo-2018-subword}), another popular subword tokenization, used in XLNet \cite{yang2019xlnet}. Third is \texttt{Morfessor2}, one of the state-of-the-art unsupervised morphological segmentations \cite{virpioja2013morfessor}. 

In future shared tasks, we aim to include more state-of-the-art tokenization tools including other Morfessor variants \cite{gronroos2014morfessor,ataman2017linguistically,gronroos2020morfessor}, BPE-dropout \cite{provilkov2019bpe}, dynamic programming encoding (DPE) \cite{he2020dynamic} or its variant \cite{hiraoka2021joint,song2022self}, multi-view subword regularization \cite{wang2021multi},  Charformer \cite{tay2021charformer}, space-treatment variants of BPE and ULM \cite{gow2022improving}.
\begin{table*}[t]
\small
\newcolumntype{R}{>{\raggedleft\arraybackslash}X}
\newcolumntype{C}{>{\centering\arraybackslash}X}
\newcolumntype{L}{>{\raggedright\arraybackslash}X}\centering
\begin{tabularx}{\textwidth}{l|RRRRRRRRR|R}
     &             &             &             &             &             &             &             &             &             & macro      \\
System     & \multicolumn{1}{c}{ces}            & \multicolumn{1}{c}{eng}            & \multicolumn{1}{c}{fra}            & \multicolumn{1}{c}{ita}            & \multicolumn{1}{c}{lat}            & \multicolumn{1}{c}{rus}            & \multicolumn{1}{c}{mon}            & \multicolumn{1}{c}{hun}            & \multicolumn{1}{c|}{spa}            & avg.     \\
\hline
\hline
WordPiece       & 20.42          & 23.06          & 12.66          & 9.08           & 8.84           & 13.81          & 14.58          & 24.00          & 16.57          & 15.89          \\
ULM      & 23.71 & 32.32 & 16.08 & 10.65 & 10.42 & 15.67 & 25.82 & 31.27 & 19.58 & 20.61
          \\
Morfessor2 & 29.43 & 37.65 & 22.38 & 9.02 & 14.53 & 17.71 & 37.80 & 40.96 & 20.64 & 25.57\\

\hline
\hline
AUUH\_A*    & 93.65          & 92.32          & -              & -              & -              & -              & 98.19          & -              & -              & 94.72          \\
AUUH\_B*    & 93.85          & 93.20          & -              & -              & -              & -              & 98.31          & -              & -              & 95.12          \\
AUUH\_E*   & 90.71          & 87.10          & 90.78          & 92.39          & 98.71          & 94.33          & 96.06          & -              & -              & 92.87          \\
AUUH\_F    & 90.28          & 86.40          & 90.81          & 92.56          & 98.85          & 93.68          & 95.32          & 98.34          & 97.25          & \textbf{93.72} \\
\hline
\hline
CLUZH   & 93.81          & 92.70          & 94.80          & 96.93          & 99.37          & 98.62          & 98.12          & 98.54          & 98.74          & \textbf{96.85} \\
\hline
\hline
DeepSPIN-1 & 93.42          & 92.29          & 91.66          & 96.01          & 99.37          & 98.75          & 98.03          & 98.56          & 98.79          & 96.32          \\
DeepSPIN-2 & \textbf{93.88} & 93.39          & 95.29          & \textbf{97.47} & 99.36          & 99.30          & 98.00          & 98.68          & 99.02          & 97.15          \\
DeepSPIN-3 & 93.84          & \textbf{93.63} & \textbf{95.73} & 97.43          & \textbf{99.38} & \textbf{99.35} & \textbf{98.51} & \textbf{98.72} & \textbf{99.04} & \textbf{97.29} \\
\hline
\hline
GU-1*      & -              & -              & 83.44          & 88.69          & -              & -              & -              & -              & -              & 86.07          \\
GU-2*     & -              & -              & 83.38          & 87.49          & -              & -              & -              & -              & 95.95          & 88.94          \\
\hline
\hline
JB132      & 64.65          & 65.43          & 46.20          & 33.44          & 91.39          & 50.55          & 57.82          & 72.64          & 43.39          & \textbf{58.39} \\
\hline
\hline
NUM DI*     & -          & 83.56          & -              & 89.55          & -              & -              & 85.59          & 95.91          & -              & 88.65         \\
\hline
\hline
Tü\_Seg-1    & 93.38          & 90.51          & 93.76          & 95.73          & 99.37          & 98.21          & 97.02          & 98.59          & 97.93          & \textbf{96.06}
\end{tabularx}
\caption{\label{tab:subtask1:all}Subtask 1 word-level results by system: The f-measure performance of systems by language; and macro average f-measure of all languages in the last column. Systems marked with * are partial submissions of a specific language set. The performances in bold are best performance of corresponding languages.}
\end{table*}


\section{System Descriptions}
The SIGMORPHON 2022 Shared Task on Morpheme Segmentation received submissions from 7 teams with members from 10 universities and institutes. Many teams submitted more than one system while some focused on a specific set of languages like Romance. In total, we had 24 unique systems over two subtasks, including the baseline system. More system details can be seen in Table~\ref{tab:systems}.

\vspace{1em} \noindent \textbf{AUUH} Researchers at the Aalto University and the University of Helsinki produced six submission systems: two were transformer models and four were bidirectional GRU models created with several innovations of Morfessor feature enrichment, multi-task learning, and multilingual learning. Morfessor \cite{creutz2002unsupervised,creutz2007unsupervised} is the famous language-independent unsupervised and semi-supervised segmentation tool and has a big family of Morfessor variants \cite{virpioja2013morfessor,gronroos2014morfessor,ataman2017linguistically,gronroos2020morfessor}. They have used the first variant of Morfessor \cite{creutz2005unsupervised} for enriching input words along with their Morfessor subword segmentations. AUUH\_A, AUUH\_C, AAUH\_E systems used this Morfessor-based feature enrichment. The key innovation of AUUH systems was multilingual and multi-task traning. They used a similar preprocessing technique \cite{johnson2017google} to distinguish tasks and languages from one another, and then trained multilingual neural models which work on both subtasks. Their transformer-based multilingual and multi-task model, AUUH\_B was the subtask 2 winning system (by its macro average f-measure) and also quite competitive with the subtask 1 winning systems on its partial three-language submissions. 

\vspace{1em} \noindent \textbf{CLUZH} Researchers at the University of Zurich ensembled four submissions \cite{cluzh_sig22} by extending their previous neural hard-attention transducer models \cite{makarov2018uzh,makarov2018imitation,makarov-clematide-2020-cluzh}. For subtask 1, they submit the following strong ensemble \textbf{CLUZH} composed of 3 models without encoder dropout and 2 models with encoder dropout of 0.15. In the sentence-level subtask 2, they submitted three ensembles, and treated this problem as the word-level problem by tokenizing sentences into words. They have also used POS tags as additional features to provide a light for the context of words. All individual models have an encoder dropout probability of 0.25 and vary only in their use of features: \textbf{CLUZH-1} with 3 models without POS features, \textbf{CLUZH-2} with 3 models with POS tag features, and \textbf{CLUZH-3} with combined all the models from CLUZH-1 and CLUZH-2. In overall, the \textbf{CLUZH-3} system was the subtask 2 winning system (by winning two out of three languages) and in subtask 1 \textbf{CLUZH} was the only system, outranked one (DeepSPIN-1) of three DeepSPIN systems. 

\vspace{1em} \noindent \textbf{DeepSPIN} Researchers submitted three neural seq2seq models: (1) \textbf{DeepSPIN-1}, a character-level LSTM with soft attention \cite{bahdanau2014neural} with softmax trained with cross-entropy loss; (2) \textbf{DeepSPIN-2}, a character-level LSTM with soft attention in which softwax is replaced with its sparser version,  1.5-entmax \cite{peters2019sigmorphon}; (3) \textbf{DeepSPIN-3}, a subword-level transformer \cite{vaswani2017attention} with the proposed 1.5- entmax, in which subword segments are  modelled using ULM \cite{kudo-2018-subword}. This design was one of most innovative architectures among all submitted systems. The authors previously experimented with the 1.5-entmax function on other tasks, demonstrating its utility, especially in the tasks with less uncertainty in the search space (e.g., compared to language modelling or machine translation) such as morphological and phonological modelling \cite{peters-martins-2020-one}. The final results of this year's shared task confirm these observations: \textbf{DeepSPIN-2} and \textbf{DeepSPIN-3} achieve superior results and are the winner of the shared task.

\begin{table*}[!h]
\scriptsize
\newcolumntype{R}{>{\raggedleft\arraybackslash}X}
\newcolumntype{C}{>{\centering\arraybackslash}X}
\newcolumntype{L}{>{\raggedright\arraybackslash}X}\centering
\begin{tabularx}{\textwidth}{CCc|lllllll|l}
\hline
inf.                & drv.                & cmp.                & eng        & fra        & ita        & rus        & mon         & hun        & spa        & macro avg. \\
\hline
\hline
\multirow{2}{*}{-}  & \multirow{2}{*}{-}  & \multirow{2}{*}{-}  & \textbf{83.80}      & 84.08      & 82.69*      & 82.56*      & 93.37       & \textbf{85.52}      & 83.58      & 83.6       \\
                     &                      &                      & CLUZH      & DeepSPIN-3 & DeepSPIN-3 & DeepSPIN-1 & JB132       & DeepSPIN-3 & DeepSPIN-2 & DeepSPIN-3 \\
\hline
\hline
\multirow{2}{*}{-}  & \multirow{2}{*}{-}  & \multirow{2}{*}{\checkmark} & 93.23      & \textbf{81.80}      & \textbf{58.10}*      & \textbf{77.67}      & 100.00      & 85.89      & \textbf{57.89}*      & \textbf{78.60}      \\
                     &                      &                      & AUUH\_A    & CLUZH      & CLUZH      & DeepSPIN-2 & all systems & DeepSPIN-3 & DeepSPIN-3 & DeepSPIN-3 \\
\hline
\hline
\multirow{2}{*}{-}  & \multirow{2}{*}{\checkmark} & \multirow{2}{*}{-}  & 94.12      & 87.36*      & 94.62      & 91.4       & \textbf{92.41}       & 94.96      & 92.47      & 92.48      \\
                     &                      &                      & DeepSPIN-3 & DeepSPIN-3 & DeepSPIN-3 & DeepSPIN-3 & DeepSPIN-3  & DeepSPIN-3 & DeepSPIN-3 & DeepSPIN-3 \\
\hline
\hline
\multirow{2}{*}{\checkmark} & \multirow{2}{*}{-}  & \multirow{2}{*}{-}  & 91.29*      & 96.37      & 96.27      & 99.75      & 99.66       & 98.31      & 98.81      & 96.97      \\
                     &                      &                      & CLUZH      & CLUZH      & CLUZH      & DeepSPIN-3 & DeepSPIN-3  & DeepSPIN-3 & DeepSPIN-2 & DeepSPIN-3 \\
\hline
\hline
\multirow{2}{*}{-}  & \multirow{2}{*}{\checkmark} & \multirow{2}{*}{\checkmark} & 95.74      & 80.61      & 70.59*      & 92.13      & -           & 89.82      & 97.3       & 87.65      \\
                     &                      &                      & DeepSPIN-2 & DeepSPIN-3 & DeepSPIN-3 & DeepSPIN-3 & -           & DeepSPIN-3 & DeepSPIN-3 & DeepSPIN-3 \\
\hline
\hline
\multirow{2}{*}{\checkmark} & \multirow{2}{*}{-}  & \multirow{2}{*}{\checkmark} & 96.89      & 96.60      & 94.97      & 100        & 100         & 98.71      & 96.15      & 97.45      \\
                     &                      &                      & DeepSPIN-3 & DeepSPIN-2 & DeepSPIN-3 & DeepSPIN-3 & all systems & DeepSPIN-3 & DeepSPIN-1 & DeepSPIN-3 \\
\hline
\hline
\multirow{2}{*}{\checkmark} & \multirow{2}{*}{\checkmark} & \multirow{2}{*}{-}  & 97.54      & 99.03      & 99.23      & 99.97      & 99.74       & 99.41      & 99.75      & 99.24      \\
                     &                      &                      & DeepSPIN-3 & DeepSPIN-3 & DeepSPIN-3 & DeepSPIN-3 & DeepSPIN-3  & DeepSPIN-2 & DeepSPIN-3 & DeepSPIN-3 \\
\hline
\hline
\multirow{2}{*}{\checkmark} & \multirow{2}{*}{\checkmark} & \multirow{2}{*}{\checkmark} & 97.13      & 100        & 100        & 99.88      & -           & 99.28      & 97.04      & 98.23      \\
                     &                      &                      & DeepSPIN-3 & DeepSPIN-3 & DeepSPIN-2 & DeepSPIN-2 & -           & DeepSPIN-2 & DeepSPIN-2 & DeepSPIN-2 \\
\hline
\end{tabularx}
\caption{\label{tab:besttask1:all}Subtask 1 word-level results by morphological category: f-measure performance of best performing system on a corresponding language and a category; Numbers in bold are worst performance of their corresponding language. Performances marked with * are worst performances of their morphological category.}
\end{table*}


\vspace{1em} \noindent \textbf{GU}  One team from Georgetown University produced two submissions for three Romance languages of the word-level subtask, based on the GRU-based encoder-decoder model \cite{GU2022}. In initial attempts, they tried to use additional features from the Wiktionary lists of prefixes and suffixes to train the model. However, such additional features decreased the main performances across morphological categories, so they excluded these features from the final submissions. Later on, they focus on data sharing between Romance languages. In French, the training data were augmented with four morphological category data from Italian and Spanish training and development datasets. These categories include non-inflection categories of \texttt{000}, \texttt{001}, \texttt{010}, \texttt{011}. With these experiments, they made minor improvements to these three languages. For these results, more research is needed to understand that transfer learning is useful.

\vspace{1em} \noindent \textbf{NUM DI} A single submission from the National University of Mongolia \cite{Task2_NUMDI} is a transformer-based neural model. Their model architecture is simple as single-layered encoder-decoder classic architecture. All the hyper-parameter settings are same as fairseq's standard tutorial tool. Their submission is also limited by four languages of subtask 1 due to human error.

\vspace{1em} \noindent \textbf{JB132} The Charles University team \cite{JB132} designed the Hidden Markov model, trained with the expectation-maximization algorithm. This model architecture has two sub-models. The first sub-model takes words as input and converts them into candidate morphemes. The second sub-model takes candidate morphemes and generates morphs as output. The first sub-model has three generators for accounting prefixes, root words, and suffixes. It is the only system not using neural methods among all submitted systems and the system's prediction is interpretable and can be useful for error analysis.

\vspace{1em} \noindent \textbf{Tü Seg} The University of Tübingen \cite{Task2-TueSeg} team submitted two systems for each of subtasks. Both systems extend the sequence-labeling method proposed by \cite{hellwig2018sanskrit,li2022word}. Their systems are very innovative and unique among all other neural models for considering the main segmentation task as a sequence-labeling task. All other neural systems used seq2seq architecture. Their neural model used a plain two-layer BiLSTM architecture. By its design, Tü Seg systems have at least two advantages over the main seq2seq alternative: (a) the number of parameters is much fewer, so the model can be trained fast and process quickly; (b) the system predictions are more interpretable compared to other neural systems and can help with the error analyses of high-resource datasets.  

\begin{figure*}[!h]
\begin{center}
\includegraphics[width=\textwidth]{images/category_systems_f1.v5.pdf}
\caption{Impact of training sizes over languages and morphological categories: Results from top5-ranked systems of word-level subtask 1}
\label{fig:size}
\end{center}
\end{figure*}

\begin{figure*}[!h]
\begin{center}
\includegraphics[width=\textwidth]{images/length_deepspin3_f_measure.v3.pdf}
\caption{Impact of word length over languages and morphological categories: Results from DeepSPIN-3, the winning system of subtask 1, word-level morpheme segmentation}
\label{fig:word_len}
\end{center}
\end{figure*}

\section{The System Results}
All system results can be found and downloaded from the shared task GitHub page.\footnote{\url{https://github.com/sigmorphon/2022SegmentationST/tree/main/results}}

\subsection{Subtask 1 word-level results}
Relative system performance of subtask 1 is provided in Table~\ref{tab:subtask1:all} which shows each system's f-measure by languages. The best performance of each language from submitted systems is in bold. 

Two teams exploited external resources in some form: AUUH and GU. In general, any relative performance gained was minimal. AUUH submitted two systems that used additional resources, they received extra ~1\% compared to the team's other systems. Similarly, GU and their submitted systems saw some minimal improvements over the performances. This details can be seen from their system description paper \cite{GU2022}. 

Only two of all the systems submitted to subtask 1 were multilingual and multi-task learning at same time. These two systems were proposed by AUUH team, but partial-language submissions were for English, Czech, and Mongolian. The important insight from this experiment is that the multi-task and multilingual learning approaches are quite beneficial for the task because their partial performances are quite competitive with the winning systems, DeepSPIN-3, DeepSPIN-2, and CLUZH. 




\vspace{1em} \noindent \textbf{Impact of training size:} In subtask 1, the training datasets' sizes vary across languages and morphological categories. It might have impacted the top-ranked systems. Therefore, we plotted the top5-ranked systems over training size and f-measure performance across morphological categories, as shown in Figure \ref{fig:size}. Here, in high-resource setting (as greater than ) in all morphological categories, any of the top5-ranked systems always achieves 80\% f-measure greater than 80\%. 

The root words are present in all types of resources settings from high to low. All the systems in this category of root words achieved no more than 85.5\% f-measure except for Mongolian. 

The two inflectional categories \texttt{100} and \texttt{110} are always in high-resource setting, having more than  training instances (except for two low-resource languages Czech and Mongolian). All systems achieved their best system performance over these two categories, compared to other categories.

\vspace{1em} \noindent \textbf{Impact of word length:} 
In many NLP tasks, the length of the input sequence is strongly correlated with the difficulty of their tasks \cite{yin2017comparative,wu2018phrase}. So, we present how the DeepSPIN-3's (subtask 1 winning system) performance relates to the word length across languages and morphological categories. Figure \ref{fig:word_len} shows various related facts: (i) for root words \texttt{000}, overall performance decreases across languages with increasing word length; (ii) inflectional morphology is systematically far more productive than other morphological categories, so this fact is reproduced here: the main inflectional category \texttt{100} has consistently high performance across languages and word lengths. 
\begin{table*}[!h]
\small
\newcolumntype{R}{>{\raggedleft\arraybackslash}X}
\newcolumntype{C}{>{\centering\arraybackslash}X}
\newcolumntype{L}{>{\raggedright\arraybackslash}X}\centering
\begin{tabularx}{\textwidth}{l|RRRr|RRRr|RRRr|rr}
\multirow{2}{*}{System} & \multicolumn{4}{c|}{Czech}     & \multicolumn{4}{c|}{English}   & \multicolumn{4}{c|}{Mongolian} & \multicolumn{2}{c}{Macro avg.} \\
\cline{2-15}
                        & \multicolumn{1}{c}{P}     & \multicolumn{1}{c}{R}     & \multicolumn{1}{c}{}    & \multicolumn{1}{c|}{Lev.}  & \multicolumn{1}{c}{P}     & \multicolumn{1}{c}{R}     & \multicolumn{1}{c}{}    & \multicolumn{1}{c|}{Lev.}  & \multicolumn{1}{c}{P}     & \multicolumn{1}{c}{R}     & \multicolumn{1}{c}{}    & \multicolumn{1}{c|}{Lev.}  & \multicolumn{1}{c}{}    & \multicolumn{1}{c}{Lev.}          \\
\hline
\hline
WordPiece                & 38.47 & 31.45 & 34.61 & 17.88 & 62.02 & 65.13 & 63.53 & 5.54  & 19.82 & 29.20 & 23.62 & 29.19 & 40.59          & 17.54         \\
ULM & 41.98 & 30.39 & 35.26 & 16.39 & 62.32 & 69.24 & 65.60 & 5.68 & 38.79 & 35.58 & 37.12 & 20.76 & 45.99 & 14.28 \\
Morfessor2 & 49.89 & 36.95 & 42.45 & 13.09 & 54.61 & 69.75 & 61.25 & 6.00 & 50.88 & 45.91 & 48.26 & 17.16 & 50.65 & 12.08 \\

\hline
\hline
AUUH\_A                 & 89.70 & 87.53 & 88.60 & 4.97  & 96.66 & 95.78 & 96.22 & 1.86  & 83.49 & 80.94 & 82.19 & 5.42  & 89.00          & 4.08          \\
AUUH\_B                 & 91.89 & 89.00 & 90.42 & 3.96  & \textbf{96.82} & \textbf{95.79} & \textbf{96.31} & \textbf{1.39}  & 83.74 & 81.46 & 82.59 & 5.16  & \textbf{89.77}          & \textbf{3.50}          \\
AUUH\_C                 & 50.60 & 69.19 & 58.45 & 71.37 & 84.77 & 71.67 & 77.67 & 19.13 & 79.07 & 73.45 & 76.15 & 17.33 & 70.76          & 35.94         \\
AUUH\_D                 & 45.07 & 67.82 & 54.15 & 80.67 & 93.29 & 83.41 & 88.07 & 10.58 & 77.99 & 74.15 & 76.02 & 17.88 & 72.75          & 36.38         \\
AUUH\_E                 & 57.39 & 67.22 & 61.92 & 55.92 & 95.23 & 76.82 & 85.04 & 12.36 & 73.34 & 72.01 & 72.67 & 24.88 & 73.21          & 31.05         \\
AUUH\_F                 & 62.36 & 43.82 & 51.47 & 61.84 & 91.50 & 74.84 & 82.34 & 13.30 & 75.50 & 59.22 & 66.38 & 33.91 & 66.73          & 36.35         \\
\hline
\hline
CLUZH-1                 & 92.03 & 90.69 & 91.35 & 1.93  & 89.74 & 89.20 & 89.47 & 9.86  & 82.98 & 81.48 & 82.22 & 5.28  & 87.68          & 5.69          \\
CLUZH-2                 & 92.41 & 91.13 & 91.76 & 1.87  & 89.71 & 89.22 & 89.47 & 9.79  & 83.29 & 81.83 & 82.55 & 5.19  & 87.93          & 5.62          \\
CLUZH-3                 & \textbf{92.63} & \textbf{91.35} & \textbf{91.99} & \textbf{1.80}  & 89.83 & 89.25 & 89.54 & 9.84  & \textbf{83.71} & \textbf{82.07} & \textbf{82.88} & \textbf{5.10}  & 88.14          & 5.58          \\
\hline
\hline
Tü\_Seg-2                  & 89.52 & 88.42 & 88.97 & 2.50  & 87.83 & 89.58 & 88.69 & 1.78  & 69.59 & 67.55 & 68.55 & 9.85  & 82.07          & 4.71         
\end{tabularx}
\caption{\label{tab:subtask2:all}Subtask 2 sentence-level results: F-measure across 3 languages}
\end{table*}

\vspace{1em} \noindent \textbf{Difficulty of morphological categories:} Even though the top-ranking systems perform very well on their own, other systems may have some complementary information across morphological categories. Therefore, we listed the best-performing systems for combinations of each language and each morphological category in Table~ \ref{tab:besttask1:all}. In the table, the lowest scores in corresponding languages are provided in bold. For instance, English root words (83.80 f-measure) are much harder to predict than other morphological categories in English. The hardest morphological categories are roots \texttt{000}, compounds \texttt{001}, and derivation and compound words \texttt{011}. The winning system, DeepSPIN-3 (marked with + in Figure~\ref{fig:size}), is consistently winning in these three categories across languages. Another observation from Figure~\ref{fig:word_len} is that compound and root words are getting harder to predict across languages with the increase of word length. Also, identifying inflections from short words (word length~\textless~5) is one of the unsolved challenges in all languages (except for English), as shown in Figure~\ref{fig:word_len}. 

\subsection{Subtask 2 sentence-level results}
Relative system performance is described in Table~\ref{tab:subtask2:all}, showing all four evaluation metrics by each combination of system and language. In the sentence-level subtask 2, we have two winners: CLUZH-3 (won two out of three languages) and AUUH\_B (F1 89.77 as maximum macro- average among submissions).

The performance of systems in the sentence-level subtask significantly decreased by 15\% in Mongolian compared to the results of the word-level subtask. One reason is that all submitted systems treated this problem as a zero-shot solution of word-level subtask 1, and mostly ignored its context by their design. 










\section{Future Directions}
The submitted systems achieved unexpectedly high accuracy across nine languages. This result suggests that the neural systems may have more capabilities beyond segmenting morphemes. For the next year, we plan to modify the task design and enrich the dataset with more fine-grained analysis. For example, \textit{truckdrivers} → \textit{truck @@drive @@er @@s} → \textit{truck \drive @@er \#\#s} where \ is compound, @@ is derivation, and \#\# is inflection.
In another direction, we will explore possibilities of adapting other morphological resources including word-formation resources \cite{zeller2013derivbase,talamo2016derivatario,derinet-2019,vodolazsky2020derivbase} or segmentation resources, UniSegments \cite{unisegments-lrec-2022,unisegments-data-2022}. Our shared task team welcomes continued contributions from the community. 






\section{Conclusion}
The SIGMORPHON 2022 Shared Task on Morpheme Segmentation significantly expanded the problem of morphological segmentation, making it more linguistically plausible. In this task, seven teams submitted 23 systems for two subtasks in total of nine languages, achieving at minimum F1 30.71 improvement over the three baselines of the state-of-the-art subword tokenization and morphological segmentation tools, being used to train large language models, e.g., XLNet \cite{yang2019xlnet}. The results suggest many directions for improving morpheme segmentation shared task.

\nocite{Ando2005,borschinger-johnson-2011-particle,andrew2007scalable,rasooli-tetrault-2015,goodman-etal-2016-noise,harper-2014-learning}

\section*{Acknowledgements}
We thank Garrett Nicolai and Eleanor Chodroff for their advice and support. The authors also thank Ben Peters and Simon Clematide for their invaluable contributions and advice, including developing the evaluation tool and early detection of data errors. 

\bibliography{anthology,custom}
\bibliographystyle{acl_natbib}

\end{document}
