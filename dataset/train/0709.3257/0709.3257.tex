\documentclass{article}
\usepackage{amsfonts,amsmath,amsthm,amstext,epsfig,mathbbol}
\usepackage{subfigure}
\RequirePackage{ifthen}
\RequirePackage[usenames]{pstcol}
\RequirePackage{pst-node}
\RequirePackage{pst-plot}
\RequirePackage{pst-coil}
\RequirePackage{multido}
\RequirePackage{pst-3d}
\RequirePackage{color}
\RequirePackage{calc}


\newcommand{\SQRTwo}{0.717}
\newcommand{\SQRThree}{1.732}
\newcommand{\GoldMean}{0.618}
\newcommand{\GoldMeanI}{1.618}
\newcommand{\SQRGoldMeanI}{1.272}
\newcommand{\Vaucanson}{\textsc{VC\raisebox{.3ex}{}S}}
\newcommand{\IOL}[2]{#1\!\mid\! #2}
\newcommand{\StackTwoLabels}[2]{\renewcommand{\arraystretch}{0.75}\begin{array}{c}#1\\ #2 \end{array}\renewcommand{\arraystretch}{1.333}}
\newcommand{\StackThreeLabels}[3]{\renewcommand{\arraystretch}{0.75}\begin{array}{c}#1\\ #2 \\ #3 \end{array}\renewcommand{\arraystretch}{1.333}}
\newcommand{\StackTwoLabelsP}[2]{\renewcommand{\arraystretch}{0.75}\begin{array}{c}#1\\ + \\ #2 \end{array}\renewcommand{\arraystretch}{1.333}}
\newcommand{\StackThreeLabelsP}[3]{\renewcommand{\arraystretch}{0.75}\begin{array}{c}#1\\ + \\ #2 \\ + \\ #3 \end{array}\renewcommand{\arraystretch}{1.333}}
\newcommand{\LineTwoLabelsP}[2]{#1 + #2} 
\newcommand{\LineThreeLabelsP}[3]{#1 + #2 + #3}
\newcommand{\LargeScale}{0.85}
\newcommand{\MediumScale}{0.6}
\newcommand{\SmallScale}{0.5}
\newcommand{\TinyScale}{0.42}
\newlength{\MediumStateDiameter}
\newlength{\SmallStateDiameter}
\newlength{\LargeStateDiameter}
\newlength{\VerySmallStateDiameter}
\setlength{\MediumStateDiameter}{.9cm}
\setlength{\SmallStateDiameter}{.6cm}
\setlength{\LargeStateDiameter}{1.2cm}
\setlength{\VerySmallStateDiameter}{.3cm}
\newlength{\StateLineWidth}        \setlength{\StateLineWidth}{1.8pt}
\newcommand{\StateLineStyle}{solid} \newcommand{\StateLineColor}{black}
\newif\ifStateLineDbl \StateLineDblfalse 
\newcommand{\StateLineDblCoef}{0.6} 
\newcommand{\StateLineDblSep}{0.4} 
\newcommand{\VSStateLineCoef}{.6} \newcommand{\StateFillStatus}{solid} 
\newcommand{\StateFillColor}{white}
\newcommand{\StateLabelColor}{black}
\newcommand{\StateLabelScale}{1.7}
\newcommand{\SmallStateFillStatus}{solid} \newcommand{\SmallStateFillColor}{white} \newcommand{\DimStateLineStyle}{solid} 
\newcommand{\DimStateLineCoef}{1} \newcommand{\DimStateLineColor}{gray}
\newcommand{\DimStateLabelColor}{gray}
\newcommand{\DimStateFillColor}{white}
\newlength{\EdgeLineWidth}
\setlength{\EdgeLineWidth}{1pt}
\newcommand{\EdgeLineStyle}{solid}
\newif\ifEdgeLineDbl \EdgeLineDblfalse 
\newcommand{\EdgeLineDblCoef}{0.5} 
\newcommand{\EdgeLineDblSep}{0.6} 
\newcommand{\EdgeLineColor}{black}
\newlength{\EdgeArrowWidth}\setlength{\EdgeArrowWidth}{5pt}
\newlength{\EdgeDblArrowWidth}\setlength{\EdgeDblArrowWidth}{5.5pt}
\newcommand{\EdgeArrowLengthCoef}{1.4}
\newcommand{\EdgeDblArrowLengthCoef}{1.7}
\newcommand{\EdgeArrowInset}{0.1}
\newcommand{\EdgeArrowStyle}{->}
\newcommand{\EdgeRevArrowStyle}{<-}
\newcommand{\EdgeLineBorderCoef}{2}
\newcommand{\EdgeLineBorderColor}{white}
\newcommand{\EdgeLabelColor}{black}
\newcommand{\EdgeLabelScale}{1.7}
\newcommand{\DimEdgeLineCoef}{1.2} 
\newcommand{\DimEdgeLineStyle}{solid} 
\newcommand{\DimEdgeLineColor}{gray}
\newcommand{\DimEdgeLabelColor}{gray}
\newlength{\ZZSize}
\setlength{\ZZSize}{.9cm}\newcommand{\ZZShape}{0.5}
\newcommand{\ZZLineWidth}{1.7}\newcommand{\TransLabelZZCoef}{0.6}\newlength{\EdgeOffset}
\setlength{\EdgeOffset}{0pt}
\newcommand{\ForthBackEdgeOffset}{5}\newcommand{\VaucArcAngle}{15}
\newcommand{\VaucArcCurvature}{0.8}
\newlength{\VaucArcOffset}
\setlength{\VaucArcOffset}{1pt}
\newcommand{\VaucLArcAngle}{30}
\newcommand{\VaucLArcCurvature}{0.8}
\newlength{\LoopOffset}\setlength{\LoopOffset}{0pt}
\newlength{\LoopVarOffset}\setlength{\LoopVarOffset}{.7pt}
\newcommand{\LoopAngle}{30}
\newcommand{\CLoopAngle}{22}
\newcommand{\LoopVarAngle}{28}
\newcommand{\LoopOnMediumState}{7}
\newcommand{\LoopOnSmallState}{9.6} 
\newcommand{\LoopOnLargeState}{5.8} 
\newcommand{\LoopOnVariableState}{5.1} 
\newcommand{\LoopOnVerySmallState}{15} 
\newcommand{\CLoopOnMediumState}{8}
\newcommand{\CLoopOnSmallState}{12}
\newcommand{\CLoopOnLargeState}{6}
\newcommand{\CLoopOnVariableState}{5.2}
\newlength{\TransLabelSep}\setlength{\TransLabelSep}{3.5pt}
\newcommand{\EdgeLabelPosit}{.45}\newcommand{\EdgeLabelRevPosit}{.55}
\newcommand{\ArcLabelPosit}{.4}\newcommand{\ArcLabelRevPosit}{.6}
\newcommand{\LArcLabelPosit}{.4}\newcommand{\LArcLabelRevPosit}{.6}
\newcommand{\LoopLabelPosit}{.25}\newcommand{\LoopLabelRevPosit}{.75}
\newcommand{\CLoopLabelPosit}{.25}\newcommand{\CLoopLabelRevPosit}{.75}
\newcommand{\InitStateLabelPosit}{.1}\newcommand{\InitStateLabelRevPosit}{.9}
\newcommand{\FinalStateLabelPosit}{.9}\newcommand{\FinalStateLabelRevPosit}{.1}
\newcommand{\ArrowOnStateCoef}{}
\newcommand{\ArrowOnMediumState}{1.5}
\newcommand{\ArrowOnSmallState}{1.7} 
\newcommand{\ArrowOnLargeState}{1.3}
\newcommand{\ArrowOnVerySmallState}{5} 
\newlength{\VertShiftH} \settoheight{\VertShiftH}{}
\newlength{\VertShiftD} \settodepth{\VertShiftD}{}
\newlength{\VertShift}
\setlength{\VertShift}{.5\VertShiftD-.5\VertShiftH}
\newif\ifVCFrame
\newcommand{\HideFrame}{\VCFramefalse}
\newcommand{\ShowFrame}{\VCFrametrue}
\newif\ifVCGrid
\newcommand{\HideGrid}{\VCGridfalse}
\newcommand{\ShowGrid}{\VCGridtrue}
\newif\ifVCRigidLabel
\newcommand{\RigidLabel}{\VCRigidLabeltrue}
\newcommand{\SwivelLabel}{\VCRigidLabelfalse}
\newif\ifVCStateLabelBaseLine
\newcommand{\AlignedLabel}{\VCStateLabelBaseLinetrue}
\newcommand{\FloatingLabel}{\VCStateLabelBaseLinefalse}
\psset{unit=1cm}
\newpsstyle{VaucFrameStyle}{arrows=-,framesep=0pt,linewidth=0.6pt,linecolor=black,linestyle=solid,doubleline=false,fillcolor=white,fillstyle=none,cornersize=relative,framearc=0}
\newcommand{\FrameStyle}{\psset{style=VaucFrameStyle}}
\newpsstyle{VaucGridStyle}{gridwidth=0.6pt,griddots=10,subgriddiv=1,gridlabels=7pt}
\newcommand{\GridStyle}{\psset{style=VaucGridStyle}}
\newenvironment{VCPicture}[2][.5]{\settoheight{\VertShiftH}{}\settodepth{\VertShiftD}{}\setlength{\VertShift}{.5\VertShiftD-.5\VertShiftH}\begin{pspicture}[#1]#2\ifVCFrame \FrameStyle \psframe#2\fi \ifVCGrid \FrameStyle\GridStyle \psgrid#2\fi}{\RstState\RstEdge \end{pspicture}}
\newcommand{\VCScale}{}\newcommand{\VCGridScale}{1}\newcommand{\FixVCScale}[1]{\renewcommand{\VCScale}{#1}}\newcommand{\LargePicture}{\FixVCScale{\LargeScale}}
\newcommand{\MediumPicture}{\FixVCScale{\MediumScale}}
\newcommand{\SmallPicture}{\FixVCScale{\SmallScale}}
\newcommand{\TinyPicture}{\FixVCScale{\TinyScale}}
\newcommand{\FixVCGridScale}[1]{\renewcommand{\VCGridScale}{#1}}\newcommand{\VCDirectory}{}
\newcommand{\SetVCDirectory}[1]{\renewcommand{\VCDirectory}{#1}}
\newif\ifVCName
\newcommand{\HideName}{\VCNamefalse}
\newcommand{\ShowName}{\VCNametrue}
\newcommand{\VCDraw}[2][\VCGridScale]{\psset{unit=#1cm}\ifVCName\makebox[0pt][r]{\fbox{{\scriptsize #2}}}\fi \scalebox{\VCScale}{#2}\psset{unit=1cm}}
\newcommand{\VCCall}[2][\VCGridScale]{\psset{unit=#1cm}\ifVCName\makebox[0pt][r]{\fbox{{\scriptsize #2}}}\fi \scalebox{\VCScale}{\input{\VCDirectory #2}}\psset{unit=1cm}}
\newcommand{\VCPut}[3][0]{\rput{#1}#2{#3}}\newlength{\StateLineWid}
\setlength{\StateLineWid}{\StateLineWidth}
\newcommand{\StateLineSty}{\StateLineStyle} 
\newcommand{\StateLineCol}{\StateLineColor}
\newcommand{\StateLineDblWid}{\StateLineDblWidth}
\newcommand{\StateLineDblSp}{\StateLineDblSep}
\newcommand{\StateFillCol}{\StateFillColor}
\newcommand{\StateFillSta}{\StateFillStatus} 
\newcommand{\StateLabelSca}{1}
\newcommand{\StateLabelCol}{\StateLabelColor}
\newcommand{\StateDimen}{outer}
\newcommand{\StateDblDimen}{middle}
\newcommand{\VCIFflag}{}\newcommand{\VCIFflagtemp}{}
\newcommand{\PlainState}{\renewcommand{\VCIFflag}{0}\renewcommand{\VCIFflagtemp}{0}}
\newcommand{\FullState}{\renewcommand{\VCIFflag}{2}\renewcommand{\VCIFflagtemp}{2}}
\newcommand{\IFState}{\renewcommand{\VCIFflag}{1}}
\newcommand{\IFXState}{\renewcommand{\VCIFflag}{2}}
\newcommand{\RstVCIF}{\renewcommand{\VCIFflag}{\VCIFflagtemp}}
\newif\ifVCShowState
\newcommand{\HideState}{\VCShowStatefalse}
\newcommand{\ShowState}{\VCShowStatetrue}
\ShowState \newpsstyle{VaucStateStyle}{framesep=0pt,linewidth=\StateLineWid,linecolor=\StateLineCol,linestyle=\StateLineSty,doubleline=false,fillcolor=\StateFillCol,fillstyle=\StateFillSta,border=0pt,dimen=\StateDimen,cornersize=relative,framearc=1,framesep=0pt}
\newpsstyle{VaucStateDblStyle}{framesep=0pt,linewidth=\StateLineDblCoef\StateLineWid,linecolor=\StateLineCol,linestyle=\StateLineSty,doubleline=true,doublesep=\StateLineDblSep\StateLineWid,fillcolor=\StateFillCol,fillstyle=\StateFillSta,border=0pt,dimen=\StateDblDimen,cornersize=relative,framearc=1,framesep=0pt}
\newpsstyle{VaucHiddenStateStyle}{framesep=0pt,linewidth=\StateLineWid,linecolor=\StateLineCol,linestyle=none,fillcolor=\StateFillCol,fillstyle=none,border=0pt,dimen=outer,cornersize=relative,framearc=1,framesep=0pt}
\newcommand{\StateStyle}{\ifVCShowState \ifStateLineDbl\psset{style=VaucStateDblStyle}\else\psset{style=VaucStateStyle}\fi \else\psset{style=VaucHiddenStateStyle}\fi}
\newcommand{\VaucStateRBLabel}[1]{\textcolor{\StateLabelCol}{\scalebox{\StateLabelSca}{\scalebox{\StateLabelScale}{\rput[B]{0}(0,\VertShift){}}}}}\newcommand{\VaucStateLabel}[1]{\ifVCShowState \ifVCRigidLabel \ifVCStateLabelBaseLine \textcolor{\StateLabelCol}{\scalebox{\StateLabelSca}{\scalebox{\StateLabelScale}{\rput[B]{*0}(0,\VertShift){}}}}\else
                 \textcolor{\StateLabelCol}{\scalebox{\StateLabelSca}{\scalebox{\StateLabelScale}{\rput{*0}(0,0){}}}}\fi
        \else
                 \textcolor{\StateLabelCol}{\scalebox{\StateLabelSca}{\scalebox{\StateLabelScale}{}}}\fi
     \else \textcolor{white}{\scalebox{\StateLabelSca}{\scalebox{\StateLabelScale}{}}}\fi}
\newcommand{\VCPutStateLabel}[2]{\rput#1{\scalebox{\StateLabelSca}{}}}\newcommand{\ChgStateLineStyle}[1]{\renewcommand{\StateLineSty}{#1}}
\newcommand{\RstStateLineStyle}{\ChgStateLineStyle{\StateLineStyle}}
\newcommand{\SetStateLineStyle}[1]{\renewcommand{\StateLineStyle}{#1}\RstStateLineStyle}\newcommand{\StateLineDouble}{\StateLineDbltrue}
\newcommand{\StateLineSimple}{\StateLineDblfalse}
\newcommand{\ChgStateLineWidth}[1]{\setlength{\StateLineWid}{#1\StateLineWidth}}\newcommand{\RstStateLineWidth}{\ChgStateLineWidth{1}}\newcommand{\SetStateLineWidth}[1]{\setlength{\StateLineWidth}{#1}\RstStateLineWidth}
\newcommand{\ChgStateLineColor}[1]{\renewcommand{\StateLineCol}{#1}}
\newcommand{\RstStateLineColor}{\ChgStateLineColor{\StateLineColor}}
\newcommand{\SetStateLineColor}[1]{\renewcommand{\StateLineColor}{#1}\RstStateLineColor}
\newcommand{\ChgStateFillStatus}[1]{\renewcommand{\StateFillSta}{#1}}
\newcommand{\RstStateFillStatus}{\ChgStateFillStatus{\StateFillStatus}}
\newcommand{\SetStateFillStatus}[1]{\renewcommand{\StateFillStatus}{#1}\RstStateFillStatus}
\newcommand{\ChgStateFillColor}[1]{\renewcommand{\StateFillCol}{#1}}
\newcommand{\RstStateFillColor}{\ChgStateFillColor{\StateFillColor}}
\newcommand{\SetStateFillColor}[1]{\renewcommand{\StateFillColor}{#1}\RstStateFillColor}\newcommand{\ChgStateLabelColor}[1]{\renewcommand{\StateLabelCol}{#1}}
\newcommand{\RstStateLabelColor}{\ChgStateLabelColor{\StateLabelColor}}
\newcommand{\SetStateLabelColor}[1]{\renewcommand{\StateLabelCol}{#1}\RstStateLabelColor}
\newcommand{\ChgStateLabelScale}[1]{\renewcommand{\StateLabelSca}{#1}}
\newcommand{\RstStateLabelScale}{\ChgStateLabelScale{1}}
\newcommand{\SetStateLabelScale}[1]{\renewcommand{\StateLabelScale}{#1}\RstStateLabelScale}
\newcommand{\FixStateLineDouble}[2]{\renewcommand{\StateLineDblCoef}{#1}\renewcommand{\StateLineDblSep}{#2}}
\newcommand{\FixDimState}[5]{\renewcommand{\DimStateLineStyle}{#1}\renewcommand{\DimStateLineCoef}{#3}\renewcommand{\DimStateLineColor}{#2}\renewcommand{\DimStateLabelColor}{#4}\renewcommand{\DimStateFillColor}{#5}}\newcommand{\RstState}{\RstStateLineStyle\RstStateLineWidth \RstStateLineColor \RstStateFillStatus\RstStateFillColor \RstStateLabelColor\RstStateLabelScale}\newcommand{\DimState}{\ChgStateLineStyle{\DimStateLineStyle}\ChgStateLineWidth{\DimStateLineCoef}\ChgStateLineColor{\DimStateLineColor}\ChgStateFillColor{\DimStateFillColor}\ChgStateLabelColor{\DimStateLabelColor}}\newlength{\StateDiam}
\newlength{\VaucAOS}\newlength{\VaucAOSdiag}
\newcommand{\StateSizeFlag}{}
\newcommand{\SetAOS}{\setlength{\VaucAOS}{\ArrowOnStateCoef\StateDiam}\setlength{\VaucAOSdiag}{\SQRTwo\VaucAOS}}
\newlength{\VariableStateIntDiam}
\newlength{\VariableStateWidth}
\newlength{\VariableStateITPos}
\newcommand{\SetStateIntDiam}{\setlength{\VariableStateIntDiam}{\StateDiam}\addtolength{\VariableStateIntDiam}{-2\StateLineWid}}\newcommand{\LoopSize}{}\newcommand{\LoopSi}{}
\newcommand{\LoopVarSize}{}\newcommand{\LoopVarSi}{}
\newcommand{\CLoopSize}{}\newcommand{\CLoopSi}{}
\newcommand{\ChgLoopSize}[1]{\renewcommand{\LoopSi}{#1}}
\newcommand{\RstLoopSize}{\ChgLoopSize{\LoopSize}}
\newcommand{\SetLoopSize}[1]{\renewcommand{\LoopSize}{#1}\RstLoopSize}
\newcommand{\ChgCLoopSize}[1]{\renewcommand{\CLoopSi}{#1}}
\newcommand{\RstCLoopSize}{\ChgCLoopSize{\CLoopSize}}
\newcommand{\SetCLoopSize}[1]{\renewcommand{\CLoopSize}{#1}\RstCLoopSize}
\newcommand{\ChgLoopVarSize}[1]{\renewcommand{\LoopVarSi}{#1}}
\newcommand{\RstLoopVarSize}{\ChgLoopVarSize{\LoopVarSize}}
\newcommand{\SetLoopVarSize}[1]{\renewcommand{\LoopVarSize}{#1}\RstLoopVarSize}
\newcommand{\SetStateDiam}[4]{\setlength{\StateDiam}{#1}\renewcommand{\ArrowOnStateCoef}{#2}\SetLoopSize{#3}\SetLoopVarSize{#3}\SetCLoopSize{#4}\SetAOS\SetStateIntDiam}
\newcommand{\FixStateDiameter}[1]{\setlength{\StateDiam}{#1}\SetStateIntDiam \SetAOS}
\newcommand{\MediumState}{\SetStateDiam{\MediumStateDiameter}{\ArrowOnMediumState}{\LoopOnMediumState}{\CLoopOnMediumState}\renewcommand{\StateSizeFlag}{0}}
\newcommand{\SmallState}{\SetStateDiam{\SmallStateDiameter}{\ArrowOnSmallState}{\LoopOnSmallState}{\CLoopOnSmallState}\renewcommand{\StateSizeFlag}{1}}
\newcommand{\LargeState}{\SetStateDiam{\LargeStateDiameter}{\ArrowOnLargeState}{\LoopOnLargeState}{\CLoopOnLargeState}\renewcommand{\StateSizeFlag}{2}}
\newcommand{\RstStateSize}{\ifthenelse{\equal{\StateSizeFlag}{0}}{\MediumState}{\ifthenelse{\equal{\StateSizeFlag}{1}}{\SmallState}{\LargeState}}}\newcommand{\VaucState}[3][{}]{\rput#2{\Cnode[radius=.5\StateDiam](0,0){#3}\ifVCShowState \nput[labelsep=-.5\StateDiam]{0}{#3}{\makebox[0pt]{\VaucStateLabel{#1}}}\fi
      \ifthenelse{\equal{\VCIFflag}{0}}{}{\pnode(-\VaucAOS,0){#3w}\pnode(\VaucAOS,0){#3e}\pnode(0,\VaucAOS){#3n}\pnode(0,-\VaucAOS){#3s}\ifthenelse{\equal{\VCIFflag}{1}}{}{\pnode(-\VaucAOSdiag,\VaucAOSdiag){#3nw}\pnode(\VaucAOSdiag,\VaucAOSdiag){#3ne}\pnode(-\VaucAOSdiag,-\VaucAOSdiag){#3sw}\pnode(\VaucAOSdiag,-\VaucAOSdiag){#3se}}}}}
\newcommand{\State}[3][{}]{\StateStyle\VaucState[#1]{#2}{#3}}
\newcommand{\FinalState}[3][{}]{\psset{style=VaucStateDblStyle}\VaucState[#1]{#2}{#3}}
\newcommand{\StateIF}[3][{}]{\IFState\State[#1]{#2}{#3}\RstVCIF}\newcommand{\StateIFX}[3][{}]{\IFXState\State[#1]{#2}{#3}\RstVCIF}\newcommand{\VSState}[2]{\renewcommand{\ArrowOnStateCoef}{\ArrowOnVerySmallState}\FixStateDiameter{\VerySmallStateDiameter}\ChgStateLineWidth{\VSStateLineCoef}\State{#1}{#2}\RstStateLineWidth\RstStateSize}\newcommand{\WB}{\textcolor{white}{\{\!\!\!}}\newcommand{\HS}{}
\newlength{\ExtraSpace}
\setlength{\ExtraSpace}{1em}
\newcommand{\StateVar}[3][]{\StateStyle \settowidth{\VariableStateWidth}{\scalebox{\StateLabelSca}{\scalebox{\StateLabelScale}{}}}\addtolength{\VariableStateWidth}{\ExtraSpace}
  \ifthenelse{\lengthtest{\VariableStateWidth < \VariableStateIntDiam}}{\setlength{\VariableStateWidth}{\VariableStateIntDiam}}{}\setlength{\VariableStateITPos}{\ArrowOnStateCoef\StateDiam}\addtolength{\VariableStateITPos}{0.5\VariableStateWidth}\addtolength{\VariableStateITPos}{-0.5\StateDiam}\rput#2{\pnode(\VariableStateITPos,0){#3e}\pnode(-\VariableStateITPos,0){#3w}\pnode(0,\ArrowOnStateCoef\StateDiam){#3n}\pnode(0,-\ArrowOnStateCoef\StateDiam){#3s}}\rput#2{\rnode{#3}{\psframebox{\protect\rule[-.5\VariableStateIntDiam]{0pt}{\VariableStateIntDiam}\protect\rule{\VariableStateWidth}{0pt}}}}
  \rput#2{\VaucStateRBLabel{#1}}}\newcommand{\StateSqr}[3][]{\StateStyle \settowidth{\VariableStateWidth}{\scalebox{\StateLabelSca}{\scalebox{\StateLabelScale}{}}}\addtolength{\VariableStateWidth}{\ExtraSpace}
  \ifthenelse{\lengthtest{\VariableStateWidth < \VariableStateIntDiam}}{\setlength{\VariableStateWidth}{\VariableStateIntDiam}}{}\setlength{\VariableStateITPos}{\ArrowOnStateCoef\StateDiam}\addtolength{\VariableStateITPos}{0.5\VariableStateWidth}\addtolength{\VariableStateITPos}{-0.5\StateDiam}\rput#2{\pnode(\VariableStateITPos,0){#3e}\pnode(-\VariableStateITPos,0){#3w}\pnode(0,\ArrowOnStateCoef\StateDiam){#3n}\pnode(0,-\ArrowOnStateCoef\StateDiam){#3s}}\rput#2{\rnode{#3}{\psframebox[framearc=0]{\protect\rule[-.5\VariableStateIntDiam]{0pt}{\VariableStateIntDiam}\protect\rule{\VariableStateWidth}{0pt}}}}
  \rput#2{\VaucStateRBLabel{#1}}}\newcommand{\VarLoopOn}{\ChgLoopOffset{\LoopVarOff}\ChgLoopSize{\LoopVarSi}\ChgLoopAngle{\LoopVarAng}}
\newcommand{\VarLoopOff}{\RstLoopOffset \RstLoopSize \RstLoopAngle}
\newlength{\EdgeLineWid}
\setlength{\EdgeLineWid}{\EdgeLineWidth}
\newcommand{\EdgeLineSty}{\EdgeLineStyle}
\newcommand{\EdgeLineCol}{\EdgeLineColor}
\newcommand{\EdgeLabelSca}{1}
\newcommand{\EdgeLabelCol}{\EdgeLabelColor}
\newlength{\EdgeArrowSZDim}
\setlength{\EdgeArrowSZDim}{\EdgeArrowWidth}
\newcommand{\EdgeArrowSZNum}{\EdgeArrowLengthCoef}
\newcommand{\EdgeArrowSty}{\EdgeArrowStyle}
\newcommand{\EdgeArrowIns}{\EdgeArrowInset}
\newlength{\EdgeLineBord}\setlength{\EdgeLineBord}{0pt}
\newlength{\ZZSiZ}
\setlength{\ZZSiZ}{\ZZSize}\newcommand{\ZZLineWid}{\ZZLineWidth}\newlength{\EdgeOff}
\setlength{\EdgeOff}{\EdgeOffset}
\newcommand{\VaucArcAng}{\VaucArcAngle}
\newcommand{\VaucLArcAng}{\VaucLArcAngle}
\newlength{\VaucArcOff}\setlength{\VaucArcOff}{\VaucArcOffset}
\newcommand{\VaucArcCurv}{\VaucArcCurvature}
\newcommand{\VaucLArcCurv}{\VaucLArcCurvature}
\newcommand{\LoopAng}{\LoopAngle}
\newcommand{\CLoopAng}{\CLoopAngle}
\newcommand{\LoopVarAng}{\LoopVarAngle}
\newlength{\LoopOff}\setlength{\LoopOff}{\LoopOffset}
\newlength{\LoopVarOff}\setlength{\LoopVarOff}{\LoopVarOffset}
\newlength{\TransLabelSP}\setlength{\TransLabelSP}{\TransLabelSep}
\newcommand{\EdgeLabelPos}{\EdgeLabelPosit}
\newcommand{\ArcLabelPos}{\ArcLabelPosit}
\newcommand{\LArcLabelPos}{\LArcLabelPosit}
\newcommand{\LoopLabelPos}{\LoopLabelPosit}
\newcommand{\CLoopLabelPos}{\CLoopLabelPosit}
\newcommand{\InitStateLabelPos}{\InitStateLabelPosit}
\newcommand{\FinalStateLabelPos}{\FinalStateLabelPosit}
\newpsstyle{VaucEdgeStyle}{arrows=\EdgeArrowSty,arrowsize=\EdgeArrowSZDim,arrowlength=\EdgeArrowSZNum,arrowinset=\EdgeArrowIns,linewidth=\EdgeLineWid,linecolor=\EdgeLineCol,linestyle=\EdgeLineSty,doubleline=false,bordercolor=\EdgeLineBorderColor,border=\EdgeLineBord,fillstyle=none,offset=\EdgeOff,labelsep=\TransLabelSP,nodesep=0pt}
\newpsstyle{VaucEdgeDblStyle}{arrows=\EdgeArrowSty,arrowsize=\EdgeArrowSZDim,arrowlength=\EdgeArrowSZNum,arrowinset=\EdgeArrowIns,linewidth=\EdgeLineDblCoef\EdgeLineWid,linecolor=\EdgeLineCol,linestyle=\EdgeLineSty,doubleline=true,doublesep=\EdgeLineDblSep\EdgeLineWid,bordercolor=\EdgeLineBorderColor,border=\EdgeLineBord,fillstyle=none,offset=\EdgeOff,labelsep=\TransLabelSP,nodesep=0pt}
\newpsstyle{VaucArcR}{ncurv=\VaucArcCurv,arcangle=-\VaucArcAng,labelsep=\TransLabelSP,offset=-\VaucArcOff}
\newpsstyle{VaucArcL}{ncurv=\VaucArcCurv,arcangle=\VaucArcAng,labelsep=\TransLabelSP,offset=\VaucArcOff}
\newpsstyle{VaucLArcR}{ncurv=\VaucLArcCurv,arcangle=-\VaucLArcAng,labelsep=\TransLabelSP,offset=-\VaucArcOff}
\newpsstyle{VaucLArcL}{ncurv=\VaucLArcCurv,arcangle=\VaucLArcAng,labelsep=\TransLabelSP,offset=\VaucArcOff}
\newpsstyle{VaucZigzagStyle}{linewidth=\ZZLineWid\EdgeLineWid,labelsep=\TransLabelSP,nodesep=0pt,coilwidth=1.2\ZZSiZ,coilarmA=0.1\ZZSiZ,coilarmB=0.3\ZZSiZ,coilheight=\ZZShape,linearc=1.6pt}
\newcommand{\EdgeStyle}{\ifEdgeLineDbl\psset{style=VaucEdgeDblStyle}\else\psset{style=VaucEdgeStyle}\fi}
\newcommand{\ZigzagStyle}{\addtolength{\TransLabelSP}{\TransLabelZZCoef\ZZSiZ}\psset{style=VaucZigzagStyle}\addtolength{\TransLabelSP}{-\TransLabelZZCoef\ZZSiZ}}
\newcommand{\ChgEdgeOffset}[1]{\setlength{\EdgeOff}{#1}}
\newcommand{\RstEdgeOffset}{\ChgEdgeOffset{\EdgeOffset}}
\newcommand{\SetEdgeOffset}[1]{\setlength{\EdgeOffset}{#1}\RstEdgeOffset}
\newcommand{\ForthBackOffset}{\setlength{\EdgeOff}{\ForthBackEdgeOffset\EdgeLineWid}}
\newcommand{\ChgArcAngle}[1]{\renewcommand{\VaucArcAng}{#1}}
\newcommand{\RstArcAngle}{\ChgArcAngle{\VaucArcAngle}}
\newcommand{\SetArcAngle}[1]{\renewcommand{\VaucArcAngle}{#1}\RstArcAngle}
\newcommand{\ChgLArcAngle}[1]{\renewcommand{\VaucLArcAng}{#1}}
\newcommand{\RstLArcAngle}{\ChgLArcAngle{\VaucLArcAngle}}
\newcommand{\SetLArcAngle}[1]{\renewcommand{\VaucLArcAngle}{#1}\RstLArcAngle}
\newcommand{\ChgArcCurvature}[1]{\renewcommand{\VaucArcCurv}{#1}}
\newcommand{\RstArcCurvature}{\ChgArcCurvature{\VaucArcCurvature}}
\newcommand{\SetArcCurvature}[1]{\renewcommand{\VaucArcCurvature}{#1}\RstArcCurvature}
\newcommand{\ChgLArcCurvature}[1]{\renewcommand{\VaucLArcCurv}{#1}}
\newcommand{\RstLArcCurvature}{\ChgLArcCurvature{\VaucLArcCurvature}}
\newcommand{\SetLArcCurvature}[1]{\renewcommand{\VaucLArcCurvature}{#1}\RstLArcCurvature}
\newcommand{\ChgArcOffset}[1]{\setlength{\VaucArcOff}{#1}}
\newcommand{\RstArcOffset}{\setlength{\VaucArcOff}{\VaucArcOffset}}
\newcommand{\SetArcOffset}[1]{\renewcommand{\VaucArcOffset}{#1}\RstArcOffset}
\newcommand{\ChgLoopOffset}[1]{\setlength{\LoopOff}{#1}}
\newcommand{\RstLoopOffset}{\setlength{\LoopOff}{\LoopOffset}}
\newcommand{\SetLoopOffset}[1]{\renewcommand{\LoopOffset}{#1}\RstLoopOffset}
\newcommand{\ChgLoopAngle}[1]{\renewcommand{\LoopAng}{#1}}
\newcommand{\RstLoopAngle}{\ChgLoopAngle{\LoopAngle}}
\newcommand{\SetLoopAngle}[1]{\renewcommand{\LoopAngle}{#1}\RstLoopAngle}
\newcommand{\ChgCLoopAngle}[1]{\renewcommand{\CLoopAng}{#1}}
\newcommand{\RstCLoopAngle}{\ChgCLoopAngle{\CLoopAngle}}
\newcommand{\SetCLoopAngle}[1]{\renewcommand{\CLoopAngle}{#1}\RstCLoopAngle}
\newcommand{\ChgEdgeLineColor}[1]{\renewcommand{\EdgeLineCol}{#1}}
\newcommand{\RstEdgeLineColor}{\ChgEdgeLineColor{\EdgeLineColor}}
\newcommand{\SetEdgeLineColor}[1]{\renewcommand{\EdgeLineColor}{#1}\RstEdgeLineColor}
\newcommand{\ChgEdgeLineStyle}[1]{\renewcommand{\EdgeLineSty}{#1}}  
\newcommand{\RstEdgeLineStyle}{\ChgEdgeLineStyle{\EdgeLineStyle}}
\newcommand{\SetEdgeLineStyle}[1]{\renewcommand{\EdgeLineStyle}{#1}\RstEdgeLineStyle}
\newcommand{\ChgEdgeLineWidth}[1]{\setlength{\EdgeLineWid}{#1\EdgeLineWidth}}
\newcommand{\RstEdgeLineWidth}{\ChgEdgeLineWidth{1}}
\newcommand{\SetEdgeLineWidth}[1]{\setlength{\EdgeLineWidth}{#1}\RstEdgeLineWidth}
\newcommand{\EdgeLineDouble}{\EdgeLineDbltrue \ChgEdgeArrowWidth{\EdgeDblArrowWidth}
    \ChgEdgeArrowLengthCoef{\EdgeDblArrowLengthCoef}}
\newcommand{\EdgeLineSimple}{\EdgeLineDblfalse \RstEdgeArrowWidth \RstEdgeArrowLengthCoef}
\newcommand{\ChgEdgeLabelColor}[1]{\renewcommand{\EdgeLabelCol}{#1}}
\newcommand{\RstEdgeLabelColor}{\ChgEdgeLabelColor{\EdgeLabelColor}}
\newcommand{\SetEdgeLabelColor}[1]{\renewcommand{\EdgeLabelColor}{#1}\RstEdgeLabelColor}
\newcommand{\ChgEdgeLabelScale}[1]{\renewcommand{\EdgeLabelSca}{#1}}
\newcommand{\RstEdgeLabelScale}{\ChgEdgeLabelScale{1}}
\newcommand{\SetEdgeLabelScale}[1]{\renewcommand{\EdgeLabelScale}{#1}\RstEdgeLabelScale}
\newcommand{\FixDimEdge}[4]{\renewcommand{\DimEdgeLineStyle}{#1}\renewcommand{\DimEdgeLineCoef}{#2}\renewcommand{\DimEdgeLineColor}{#3}\renewcommand{\DimEdgeLabelColor}{#4}}\newcommand{\ChgEdgeArrowStyle}[1]{\renewcommand{\EdgeArrowSty}{#1}}
\newcommand{\RstEdgeArrowStyle}{\ChgEdgeArrowStyle{\EdgeArrowStyle}}
\newcommand{\SetEdgeArrowStyle}[1]{\renewcommand{\EdgeArrowStyle}{#1}\RstEdgeArrowStyle}
\newcommand{\ChgEdgeArrowWidth}[1]{\setlength{\EdgeArrowSZDim}{#1}} \newcommand{\RstEdgeArrowWidth}{\ChgEdgeArrowWidth{\EdgeArrowWidth}}
\newcommand{\SetEdgeArrowWidth}[1]{\setlength{\EdgeArrowWidth}{#1} \RstEdgeArrowWidth}
\newcommand{\ChgEdgeArrowLengthCoef}[1]{\renewcommand{\EdgeArrowSZNum}{#1}}
\newcommand{\RstEdgeArrowLengthCoef}{\ChgEdgeArrowLengthCoef{\EdgeArrowLengthCoef}}
\newcommand{\SetEdgeArrowLengthCoef}[1]{\renewcommand{\EdgeArrowLengthCoef}{#1}\RstEdgeArrowLengthCoef}
\newcommand{\ChgEdgeArrowInsetCoef}[1]{\renewcommand{\EdgeArrowIns}{#1}}
\newcommand{\RstEdgeArrowInsetCoef}{\ChgEdgeArrowInsetCoef{\EdgeArrowInset}}
\newcommand{\SetEdgeArrowInsetCoef}[1]{\renewcommand{\EdgeArrowInset}{#1}\RstEdgeArrowInsetCoef}
\newcommand{\ReverseArrow}{\ChgEdgeArrowStyle{\EdgeRevArrowStyle}\renewcommand{\EdgeLabelPos}{\EdgeLabelRevPosit}\renewcommand{\ArcLabelPos}{\ArcLabelRevPosit}\renewcommand{\LArcLabelPos}{\LArcLabelRevPosit}\renewcommand{\LoopLabelPos}{\LoopLabelRevPosit}\renewcommand{\CLoopLabelPos}{\CLoopLabelRevPosit}\renewcommand{\InitStateLabelPos}{\InitStateLabelRevPosit}\renewcommand{\FinalStateLabelPos}{\FinalStateLabelRevPosit}}
\newcommand{\StraightArrow}{\ChgEdgeArrowStyle{\EdgeArrowStyle}\renewcommand{\EdgeLabelPos}{\EdgeLabelPosit}\renewcommand{\ArcLabelPos}{\ArcLabelPosit}\renewcommand{\LArcLabelPos}{\LArcLabelPosit}\renewcommand{\LoopLabelPos}{\LoopLabelPosit}\renewcommand{\CLoopLabelPos}{\CLoopLabelPosit}\renewcommand{\InitStateLabelPos}{\InitStateLabelPosit}\renewcommand{\FinalStateLabelPos}{\FinalStateLabelPosit}}
\newcommand{\FixEdgeLineDouble}[2]{\renewcommand{\EdgeLineDblCoef}{#1}\renewcommand{\EdgeLineDblSep}{#2}}
\newcommand{\FixEdgeBorder}[2]{\renewcommand{\EdgeLineBorderCoef}{#1}\renewcommand{\EdgeLineBorderColor}{#2}}
\newcommand{\EdgeBorder}{\setlength{\EdgeLineBord}{\EdgeLineBorderCoef\EdgeLineWid}}
\newcommand{\EdgeBorderOff}{\setlength{\EdgeLineBord}{0pt}}
\newcommand{\DimEdge}{\ChgEdgeLineStyle{\DimEdgeLineStyle}\ChgEdgeLineWidth{\DimEdgeLineCoef}\ChgEdgeLineColor{\DimEdgeLineColor}\ChgEdgeLabelColor{\DimEdgeLabelColor}}
\newcommand{\ChgZZSize}[1]{\setlength{\ZZSiZ}{#1}}
\newcommand{\RstZZSize}{\setlength{\ZZSiZ}{\ZZSize}}
\newcommand{\SetZZSize}[1]{\setlength{\ZZSiZe}{#1}\RstZZSize}
\newcommand{\ChgZZLineWidth}[1]{\renewcommand{\ZZLineWid}{#1}}
\newcommand{\RstZZLineWidth}{\ChgZZLineWidth{\ZZLineWidth}}
\newcommand{\SetZZLineWidth}[1]{\renewcommand{\ZZLineWidth}{#1}\RstZZLineWidth}
\newcommand{\VaucEdgeLabel}[1]{\textcolor{\EdgeLabelCol}{\scalebox{\EdgeLabelSca}{\scalebox{\EdgeLabelScale}{}}}}
\newcommand{\RstEdge}{\RstEdgeOffset\RstArcAngle\RstLArcAngle \RstArcCurvature\RstLArcCurvature \RstArcOffset\RstLoopOffset\RstLoopSize \RstEdgeLineColor\RstEdgeLineStyle\RstEdgeLineWidth\EdgeLineSimple \StraightArrow \RstEdgeLabelScale\RstEdgeLabelColor}
\newcommand{\InitialDir}{w}\newcommand{\FinalDir}{e}
\newcommand{\Initial}[2][\InitialDir]{\EdgeStyle\ncline{#2#1}{#2}}
\newcommand{\Final}[2][\FinalDir]{\EdgeStyle\ncline{#2}{#2#1}}
\newcommand{\InitFinal}[2][\InitialDir]{\EdgeStyle\ncline[arrows=<->]{#2#1}{#2}}
\newcommand{\InitialL}[4][{\InitStateLabelPos}]{\EdgeStyle\ncline{#3#2}{#3}\naput[npos=#1]{\VaucEdgeLabel{#4}}}
\newcommand{\InitialR}[4][{\InitStateLabelPos}]{\EdgeStyle\ncline{#3#2}{#3}\nbput[npos=#1]{\VaucEdgeLabel{#4}}}
\newcommand{\FinalL}[4][{\FinalStateLabelPos}]{\EdgeStyle\ncline{#3}{#3#2}\naput[npos=#1]{\VaucEdgeLabel{#4}}}
\newcommand{\FinalR}[4][{\FinalStateLabelPos}]{\EdgeStyle\ncline{#3}{#3#2}\nbput[npos=#1]{\VaucEdgeLabel{#4}}}
\newcommand{\EdgeL}[4][{\EdgeLabelPos}]{\EdgeStyle \ncline{#2}{#3} \naput[npos=#1]{\VaucEdgeLabel{#4}}}
\newcommand{\EdgeR}[4][{\EdgeLabelPos}]{\EdgeStyle \ncline{#2}{#3} \nbput[npos=#1]{\VaucEdgeLabel{#4}}}
\newcommand{\ArcL}[4][{\ArcLabelPos}]{\EdgeStyle \psset{style=VaucArcL}\ncarc{#2}{#3} \naput[npos=#1]{\VaucEdgeLabel{#4}}}
\newcommand{\ArcR}[4][{\ArcLabelPos}]{\EdgeStyle \psset{style=VaucArcR}\ncarc{#2}{#3} \nbput[npos=#1]{\VaucEdgeLabel{#4}}}
\newcommand{\LArcL}[4][{\LArcLabelPos}]{\EdgeStyle \psset{style=VaucLArcL}\ncarc{#2}{#3} \naput[npos=#1]{\VaucEdgeLabel{#4}}}
\newcommand{\LArcR}[4][{\LArcLabelPos}]{\EdgeStyle \psset{style=VaucLArcR}\ncarc{#2}{#3} \nbput[npos=#1]{\VaucEdgeLabel{#4}}}
\newcounter{anglea}\newcounter{angleb}
\newcommand{\LoopXR}[7]{{\setcounter{anglea}{#2-#4}}{\setcounter{angleb}{#2+#4}}{\EdgeStyle \psset{angleA=\theanglea,angleB=\theangleb,offset=#5,ncurv=#6}\nccurve{#3}{#3} \nbput[npos=#1]{\VaucEdgeLabel{#7}}}}
\newcommand{\LoopXL}[7]{{\setcounter{anglea}{#2+#4}}{\setcounter{angleb}{#2-#4}}{\EdgeStyle \psset{angleA=\theanglea,angleB=\theangleb,offset=-#5,ncurv=#6}\nccurve{#3}{#3} \naput[npos=#1]{\VaucEdgeLabel{#7}}}}
\newcommand{\LoopR}[4][{\LoopLabelPos}]{\LoopXR{#1}{#2}{#3}{\LoopAng}{\LoopOff}{\LoopSi}{#4}}
\newcommand{\LoopL}[4][{\LoopLabelPos}]{\LoopXL{#1}{#2}{#3}{\LoopAng}{\LoopOff}{\LoopSi}{#4}}
\newcommand{\CLoopR}[4][{\CLoopLabelPos}]{\LoopXR{#1}{#2}{#3}{\CLoopAng}{\LoopOff}{\LoopSi}{#4}}
\newcommand{\CLoopL}[4][{\CLoopLabelPos}]{\LoopXL{#1}{#2}{#3}{\CLoopAng}{\LoopOff}{\LoopSi}{#4}}
\newcommand{\LoopVarR}[4][{\LoopLabelPos}]{\LoopXR{#1}{#2}{#3}{\LoopVarAng}{\LoopVarOff}{\LoopVarSi}{#4}}
\newcommand{\LoopVarL}[4][{\LoopLabelPos}]{\LoopXL{#1}{#2}{#3}{\LoopVarAng}{\LoopVarOff}{\LoopVarSi}{#4}}
\newcommand{\LoopW}[3][{\LoopLabelPos}]{\LoopR[#1]{180}{#2}{#3}}
\newcommand{\LoopE}[3][{\LoopLabelPos}]{\LoopL[#1]{0}{#2}{#3}}
\newcommand{\LoopN}[3][{\LoopLabelPos}]{\LoopL[#1]{90}{#2}{#3}}
\newcommand{\LoopS}[3][{\LoopLabelPos}]{\LoopR[#1]{-90}{#2}{#3}}
\newcommand{\LoopNW}[3][{\LoopLabelPos}]{\LoopR[#1]{135}{#2}{#3}}
\newcommand{\LoopNE}[3][{\LoopLabelPos}]{\LoopL[#1]{45}{#2}{#3}}
\newcommand{\LoopSW}[3][{\LoopLabelPos}]{\LoopL[#1]{-135}{#2}{#3}}
\newcommand{\LoopSE}[3][{\LoopLabelPos}]{\LoopR[#1]{-45}{#2}{#3}}
\newcommand{\CLoopW}[3][{\CLoopLabelPos}]{\CLoopR[#1]{180}{#2}{#3}}
\newcommand{\CLoopE}[3][{\CLoopLabelPos}]{\CLoopL[#1]{0}{#2}{#3}}
\newcommand{\CLoopN}[3][{\CLoopLabelPos}]{\CLoopL[#1]{90}{#2}{#3}}
\newcommand{\CLoopS}[3][{\CLoopLabelPos}]{\CLoopR[#1]{-90}{#2}{#3}}
\newcommand{\CLoopNW}[3][{\CLoopLabelPos}]{\CLoopR[#1]{135}{#2}{#3}}
\newcommand{\CLoopNE}[3][{\CLoopLabelPos}]{\CLoopL[#1]{45}{#2}{#3}}
\newcommand{\CLoopSW}[3][{\CLoopLabelPos}]{\CLoopL[#1]{-135}{#2}{#3}}
\newcommand{\CLoopSE}[3][{\CLoopLabelPos}]{\CLoopR[#1]{-45}{#2}{#3}}
\newcommand{\LoopVarN}[3][{\CLoopLabelPos}]{\LoopVarL[#1]{90}{#2}{#3}}
\newcommand{\LoopVarS}[3][{\CLoopLabelPos}]{\LoopVarR[#1]{-90}{#2}{#3}}
\newcommand{\ZZEdge}[2]{\EdgeStyle\ZigzagStyle\nczigzag{#1}{#2}}\newcommand{\ZZEdgeL}[4][{\EdgeLabelRevPosit}]{\EdgeStyle\ZigzagStyle\nczigzag{#2}{#3}\naput[npos=#1]{\VaucEdgeLabel{#4}}}
\newcommand{\ZZEdgeR}[4][{\EdgeLabelRevPosit}]{\EdgeStyle\ZigzagStyle\nczigzag{#2}{#3}\nbput[npos=#1]{\VaucEdgeLabel{#4}}}
\newcommand{\Point}[2]{\pnode#1{#2}}
\newcommand{\Edge}[2]{\EdgeStyle\ncline{#1}{#2}}
\newcommand{\VArcL}[5][{\ArcLabelPos}]{\EdgeStyle \psset{style=VaucLArcL}\ncarc[#2]{#3}{#4} \naput[npos=#1]{\VaucEdgeLabel{#5}}}
\newcommand{\VArcR}[5][{\ArcLabelPos}]{\EdgeStyle \psset{style=VaucLArcR}\ncarc[#2]{#3}{#4} \nbput[npos=#1]{\VaucEdgeLabel{#5}}}
\newcommand{\VCurveL}[5][{\ArcLabelPos}]{\EdgeStyle \psset{angleA=0,angleB=180,ncurv=1}\nccurve[#2]{#3}{#4} \naput[npos=#1]{\VaucEdgeLabel{#5}}}
\newcommand{\VCurveR}[5][{\ArcLabelPos}]{\EdgeStyle \psset{angleA=0,angleB=0,ncurv=1}\nccurve[#2]{#3}{#4} \nbput[npos=#1]{\VaucEdgeLabel{#5}}}
\newcommand{\LabelL}[2][{\EdgeLabelPos}]{\naput[npos=#1]{\VaucEdgeLabel{#2}}}
\newcommand{\LabelR}[2][{\EdgeLabelPos}]{\nbput[npos=#1]{\VaucEdgeLabel{#2}}}
\SetStateLabelColor{black}              \SetStateLabelScale{1.7}                \SetStateLineStyle{solid}               \SetStateLineWidth{1.8pt}               \SetStateLineColor{black}               \SetStateFillStatus{solid}              \SetStateFillColor{white}               \FixDimState{solid}{gray}{1}{gray}{white}  \FixStateLineDouble{0.6}{0.4}               \SetEdgeLabelColor{black}               \SetEdgeLabelScale{1.7}                 \SetEdgeLineStyle{solid}                \SetEdgeLineWidth{1pt}                  \SetEdgeLineColor{black}                \SetArcAngle{15}                        \SetLArcAngle{30}                       \SetArcCurvature{0.8}                   \SetEdgeOffset{0pt}                     \SetArcOffset{1pt}                      \SetLoopOffset{0pt}                     \renewcommand{\ForthBackEdgeOffset}{5}  \FixDimEdge{solid}{1.2}{gray}{gray}     \FixEdgeBorder{2}{white}                \FixEdgeLineDouble{0.4}{0.8}            \setlength{\ZZSize}{.9cm}               \renewcommand{\ZZShape}{0.5}            \SetZZLineWidth{1.7}                    \renewcommand{\TransLabelZZCoef}{0.6}   \renewcommand{\LargeScale}{0.85}        \renewcommand{\MediumScale}{0.6}        \renewcommand{\SmallScale}{0.5}         \renewcommand{\TinyScale}{0.42}         \setlength{\MediumStateDiameter}{.9cm}          \setlength{\SmallStateDiameter}{.6cm}           \setlength{\LargeStateDiameter}{1.2cm}          \setlength{\VerySmallStateDiameter}{.3cm}       \renewcommand{\VSStateLineCoef}{.6}             \renewcommand{\ArrowOnMediumState}{1.5}         \renewcommand{\ArrowOnSmallState}{1.7}          \renewcommand{\ArrowOnLargeState}{1.3}          \renewcommand{\ArrowOnVerySmallState}{5}        \renewcommand{\LoopOnMediumState}{7}            \renewcommand{\LoopOnSmallState}{9.6}           \renewcommand{\LoopOnLargeState}{5.8}           \renewcommand{\LoopOnVariableState}{5.1}                \renewcommand{\CLoopOnMediumState}{8}           \renewcommand{\CLoopOnSmallState}{12}           \renewcommand{\CLoopOnLargeState}{6}            \renewcommand{\CLoopOnVariableState}{5.2}               \renewcommand{\EdgeLabelPosit}{.45}   \renewcommand{\EdgeLabelRevPosit}{.55}
\renewcommand{\ArcLabelPosit}{.40}
\renewcommand{\ArcLabelRevPosit}{.60}
\renewcommand{\LArcLabelPosit}{.40}
\renewcommand{\LArcLabelRevPosit}{.60}
\renewcommand{\LoopLabelPosit}{.25}
\renewcommand{\LoopLabelRevPosit}{.75}
\renewcommand{\CLoopLabelPosit}{.25}
\renewcommand{\CLoopLabelRevPosit}{.75}
\renewcommand{\InitStateLabelPosit}{.10}
\renewcommand{\InitStateLabelRevPosit}{.90}
\renewcommand{\FinalStateLabelPosit}{.90}
\renewcommand{\FinalStateLabelRevPosit}{.10}
\SetEdgeArrowWidth{5pt}         \SetEdgeArrowLengthCoef{1.4}            \setlength{\EdgeDblArrowWidth}{5.5pt}           \renewcommand{\EdgeDblArrowLengthCoef}{1.7}     \SetEdgeArrowInsetCoef{0.1}                     \SetEdgeArrowStyle{->}                          \renewcommand{\EdgeRevArrowStyle}{<-}           \renewcommand{\StateDimen}{outer}               \renewcommand{\StateDblDimen}{middle}           \SetVCDirectory{}        \FixVCScale{.6}
\HideFrame
\HideGrid
\MediumPicture
\HideName
\RigidLabel
\FullState                              \MediumState


\parindent 0pt

\newtheorem{thrm}{Theorem}[section]
 \newtheorem{prpstn}[thrm]{Proposition}
 \newtheorem{dfntn}[thrm]{Definition}
\newtheorem{crllr}[thrm]{Corollary}


\newcommand{\mrm}[1]{\text{\rm #1}}
\newcommand{\supp}[1]{\text{Supp }#1}

\def\cA{{\mathcal A}}
\def\cU{{\mathcal U}}
\def\cV{{\mathcal V}}
\def\cN{{\mathcal N}}
\def\cI{{\mathcal I}}
\def\cT{{\mathcal T}}
\def\cJ{{\mathcal J}}
\def\cD{{\mathcal D}}
\def\cO{{\mathcal O}}
\def\cF{{\mathcal F}}
\def\cE{{\mathcal E}}
\def\cB{{\mathcal B}}
\def\cG{{\mathcal G}}
\def\cR{{\mathcal R}}
\def\cP{{\mathcal P}}
\def\cS{{\mathcal S}}
\def\cH{{\mathcal H}}
\def\cF{{\mathcal F}}
\def\cC{{\mathcal C}}
\def\fC{{\mathfrak C}}
\def\cM{{\mathcal M}}
\def\cL{{\mathcal L}}
\def\cK{{\mathcal K}}
\def\cM{{\mathcal M}}
\def\fT{{\mathfrak T}}
\def\fD{{\mathfrak D}}
\def\fU{{\mathfrak U}}
\def\fA{{\mathfrak A}}


\newcommand{\D}{\mathcal{D}}
\newcommand{\Bc}{\mathcal{B}}
\newcommand{\spart}{\mathcal{P}}
\newcommand{\Sc}{\mathcal{S}}
\newcommand{\Uc}{\mathcal{U}}

\newcommand{\intro}{\marginpar{mettre dans l'intro}}

\newcommand{\N} {\ensuremath{\mathbb{N}}}
\newcommand{\Z} {\ensuremath{\mathbb{Z}}}
\newcommand{\Q} {\ensuremath{\mathbb{Q}}}
\newcommand{\R} {\ensuremath{\mathbb{R}}}
\newcommand{\K} {\ensuremath{\mathbb{K}}}
\newcommand{\B} {\ensuremath{\mathbb{B}}}
\newcommand{\Nmin} {\N_{\min}}
\newcommand{\Nmax} {\N_{\max}}
\newcommand{\Zmin} {\Z_{\min}}
\newcommand{\Zmax} {\Z_{\max}}
\newcommand{\Qmin} {\Q_{\min}}
\newcommand{\Qmax} {\Q_{\max}}
\newcommand{\Rmin} {\R_{\min}}
\newcommand{\Rmax} {\R_{\max}}

\def\plus{\oplus}
\def\bigplus{\bigoplus}
\def\fois{\otimes}
\def\zero{\mathbf{0}}
\def\un{\mathbf{1}}
\def\vide{\epsilon}
\def\emptyword{\epsilon}
\newcommand{\1}{\mathbb{1}}
\newcommand{\0}{\mathbb{0}}
\def\w{\omega}
\def\Rw{\R\cup\{-\w\}}

\newcommand{\flech}[1]{\overset{#1}{\longrightarrow}}
\newcommand{\fleche}[2]{\mathchoice
         {\xrightarrow{#1\mid #2}}
         {\xrightarrow{\smash{\lower1pt\hbox{}}}}
         {\text{Erreur}}
         {\text{Erreur}}}
\newcommand{\chemin}[2]{\left[#1\right]_{#2}}
\newcommand{\val}[2]{\mrm{val}(#1)}
\newcommand{\wght}[1]{\textit{weight}\left(#1\right)}
\newcommand{\wghti}[1]{\wght{\rightarrow#1}}
\newcommand{\wghtf}[1]{\wght{\rightarrow#1\!\rightarrow}}

\def\io{{\text{\tiny }}}
\newcommand{\init}[2][]{^{\text{\tiny}\io}\!#2}
\newcommand{\final}[2][]{{#2}_{\io#1}}

\newcommand{\coef}[2]{\langle #1, #2\rangle}
\newcommand{\rat}[2]{#1\text{Rat}\langle\! \langle #2 \rangle\! \rangle}

\def\Rat{\text{Rat}}
\def\FAmb{\text{FAmb}}
\def\FUSeq{\text{FSeq}}
\def\NAmb{\text{NAmb}}
\def\Seq{\text{Seq}}
\def\Lip{\text{Lip}}

\def\ab{\Sigma}
\newcommand{\uab}{1_\ab}

\newcommand{\Past}[2][\A]{\mathsf{Past}_{#1}(#2)}
\newcommand{\Fut}[2][\A]{\mathsf{Fut}_{#1}(#2)}


\def\A{\mathcal{A}}
\def\P{\mathcal{P}} \newcommand{\ind}[1]{^{\llcorner\raisebox{.4ex}{\footnotesize\!\!}}}
\newcommand{\p}[1]{\mathbf{#1}}



\newcommand{\mylabel}[1]{\label{#1}\marginpar{\fbox{#1}}}
\newcommand{\imglabel}[1]{\label{#1}\fbox{#1}}




\begin{document}
\title{Series which are both max-plus and min-plus rational are
  unambiguous}
\author{Sylvain Lombardy and Jean Mairesse\thanks{LIAFA (UMR 7089), CNRS - Universit\'e Paris 7,
2 place Jussieu, 75251 Paris cedex 05, France,
e-mail: {\small{\tt \{lombardy,mairesse\}@liafa.jussieu.fr}}}}

\maketitle

\begin{abstract}
Consider partial maps  with a rational
domain. We show that two families of such series are actually the
same: the unambiguous rational series on the one hand, and 
the max-plus and min-plus rational series on the other hand.
The decidability of equality was known to hold in both families with
different proofs, so the above unifies the picture. 
We give an effective procedure to build an unambiguous automaton from
a max-plus automaton and a min-plus one that recognize the same series.
\end{abstract}

\section{Introduction}

A max-plus automaton is an automaton with multiplicities in the
semiring . Roughly, the transitions
of the automaton have a {\em label} in a finite alphabet  and a {\em
  weight} in the semiring. The weight of a word  in  is the maximum over all
successful paths of label  of the sum of the weights along the path. 
The series {\em recognized} by the
automaton  is the resulting map . The set of series recognized by a max-plus automaton 
is denoted by . 

These automata, or the variants obtained by considering the min-plus
semi\-ring  or subsemirings such as 
 or the {\em tropical semiring} , have been studied
under various names: distance automata, cost automata, finance
automata... 
The motivations range from complexity issues in formal language
theory~\cite{simo88}, 
to automatic speech recognition \cite{mohr}, via the modeling of Tetris
heaps \cite{GaMa98b}. 

In Krob~\cite{krob94}, the following question was raised: characterize the
series which are recognized both by a max-plus and a min-plus
automaton. That is, characterize the class . Here, we answer the question by showing that these
series are precisely the unambiguous max-plus (equivalently, min-plus)
series. Given a finitely ambiguous max-plus automaton, it is decidable if the
corresponding series is unambiguous~\cite{KLMP04}. On the other hand,
the status of the same problem
starting from an infinitely ambiguous max-plus automaton is unknown. 


Apart from an interest in terms of classification, this result 
clarifies the status of the equality problem for 
max-plus series. The equality problem is to determine if ``'',
where  and  are series recognized by given max-plus
automata. The equality problem is already undecidable in  and
for two letters alphabet~\cite{krob}, but it is decidable for
finitely ambiguous automata over ~\cite{HIJi,webe94}. Also, the following result
is proved in \cite{krob94}: if  is an automaton over , and
 an automaton over , then the problem ``'' is
decidable, so the equality problem is decidable 
in  (see Proposition \ref{pr-maxmin}). We can now
conclude that the decidability result in \cite{krob94} is a particular
case of the one in \cite{HIJi,webe94}. 

\medskip

The paper is organized as follows. In \S \ref{se-krob}, we extend several
results of \cite{krob94} from  to , in particular the
so-called Fatou property. The
results are then used in \S \ref{se-minmax} to obtain the characterization of
. 

\medskip

Below, the results on decidability and
complexity should be interpreted under the assumption that two real
numbers can be added or compared in constant time. 

\section{Preliminaries}\label{se-prel}


Let  be any semiring and denote the neutral element of
the additive, resp. multiplicative, law by ,
resp. . 
Let  be a finite set and  a finite alphabet.
A finite linear representation indexed by  over the alphabet 
and the semiring 
is a triple , where
, resp. , is a row, resp. columm, vector
of  and  is a morphism from  into 
(for ).
The (formal power) series {\it recognized} by  is the
series~ such that .
By the Sch\"{u}tzenberger
Theorem, the set of series that can be
recognized by a finite linear representation is precisely the set of
rational series. We denote it by .

\medskip

Let  be a finite linear representation indexed by 
over the semiring . This representation can be viewed as an automaton
with set of states :
for every~ in  and every letter  in
, if , there is a transition from  to 
with label  and weight . For every  in ,
if , (resp. ),
the state  is initial with weight  (resp. terminal
with weight ). In the sequel, we identify the linear representation
with the corresponding automaton.
As usual we transfer the terminology of graph theory to automata, {\it e.g.}
(simple) path or circuit of an automaton. A path which is both starting with an ingoing arc and 
ending with an outgoing arc is called a {\em successful path}.  
The {\em label of a path} is the concatenation of the labels of the
successive arcs (transitions). The {\em weight of a 
  path} is the product (with respect to the multiplicative law of the
semiring) of the weights of the successive 
arcs (including the ingoing and the outgoing arc, need it be).
We denote by  the weight of the path .

Two automata are {\em equivalent} if they recognize the same series. 

\medskip

The {\em support} of a series~ is the set of words~ such that
 . We denote the support of  by . The
 characteristic series of a language  is the series 
   such that  if , and
   otherwise.

\medskip

The {\em max-plus semiring}  is the semiring formed by the set
 with  as 
the additive operation and  as the multiplicative operation.
In the sequel, we sometimes denote  and  respectively by  and
; the neutral elements for these operations are respectively
 and . This semiring
is naturally ordered by the usual order on  extended by: . 
The {\em min-plus semiring}  is obtained by replacing
 by  and  by 
in the definition of .
The subsemirings , , , , \ldots, are defined in the natural
way. 


The subsemiring  of  is
the Boolean semiring. 
There exists a
morphism from  
onto  that maps  onto  and
any other element onto .

\medskip

An automaton over  is called a {\em max-plus
automaton}, the corresponding series is called a {\em max-plus
(rational) series}.
Let  be a max-plus
  rational series recognized by
. Then  is the regular language 
recognized
by the Boolean automaton obtained from  by
applying to each coefficient the canonical morphism from  onto
. 

\medskip

An automaton is {\em unambiguous} if, for every word ,
there is at most one successful path labeled by .
An automaton is {\em 1-valued} if, for every word ,
all the successful paths labeled by  have the same weight.

In , a triple  is unambiguous if, for every word ,

In , a triple  is 1-valued if,
for every word ,

Analogous definitions hold for triples over . 
A max-plus, resp. min-plus, series is {\em unambiguous} if there exists 
an unambiguous max-plus, resp. min-plus, automaton recognizing it.

\medskip

The operations on matrices over  are defined classically
with respect to the operations of , e.g.:
. We usually write  for . 
Given  and , set . 

\medskip

Consider a matrix . The matrix  is {\em
  irreducible} if the graph of  (nodes ,  if
  ) is strongly connected. A  scalar 
and a column vector  such that 

are called respectively an {\em eigenvalue} and an {\em
  eigenvector} of . 
The number of eigenvalues is at least one and at most
  , and it is exactly one if  is irreducible. 
The max-plus spectral theory is the study of
  these eigenvalues and eigenvectors. 
In the sequel, we only need the result in Theorem \ref{th-spectral}.
For a more complete picture, as
  well as proofs and bibliographic references, see for instance \cite{BCOQ}. 

\begin{thrm}[Max-plus spectral theory]\label{th-spectral}
Consider . Let   be the maximal eigenvalue
of . We have:

In words,  is the maximal mean weight of a simple circuit of
(the graph of) . \end{thrm}

\section{Some decidability results}\label{se-krob}

In this section, we reconsider the various results proved by
Krob~\cite{krob94} for series in  and we extend
them to . The proofs are different since they use
the max-plus spectral theory. The results are then used in \S
\ref{se-minmax}. Obviously, analogous results hold for . 

\medskip

The decidability part of Proposition \ref{pr-ineq} is
given in \cite[Corollary 4.3]{krob94} for
series in .
The proof in \cite{krob94} is different and relies
on the fact that  is a constructive Fatou
extension of . We prove a generalization of
this last 
result for   in Proposition
\ref{pr-fatou} below. Using Proposition \ref{pr-fatou}, we can then 
recover the decidability in Proposition \ref{pr-ineq} in the same way
as in \cite{krob94}. 
Observe however that the proof of Proposition \ref{pr-ineq} 
given below provides a polynomial
procedure. 

In contrast with Proposition \ref{pr-ineq}, the problem ``'' is undecidable even for , see
\cite{krob}. 

\begin{prpstn}
\label{pr-ineq}
Consider the following problem:
\begin{tabbing} 
de la marge \= \kill
\> {\bf Instance:} \hspace*{0.4cm} \=  \\
\>  {\bf Problem:} \>  \:.
\end{tabbing}
This problem can be decided with an algorithm of polynomial time
complexity in the size of an automaton recognizing . 
\end{prpstn}

\begin{proof}
Let  be a trim automaton recognizing  with
set of states . Set 

Let  be the maximal eigenvalue of . 
By the Max-plus Spectral Theorem \ref{th-spectral}, there exist
 and  such that . 
It implies that there exists  such that 
. 
Clearly, we have  for all . 
Since the automaton is 
trim, there exist  such that  and  .  
Assume that , then 
by choosing  large enough, we get the following contradiction

Assume now that . By the Max-plus Spectral Theorem
\ref{th-spectral}, 
it implies that all the circuits in the automaton have a weight which
is non-positive. 
Assume that there exists a word  such
that . Let  be a successful path of label  and 
maximal weight in the automaton. If  contains a circuit, then 
the path  obtained by removing the circuit is still a successful
path. In particular, if  is the label of ,
we have . 
So we can choose, without loss of generality, a word  such that
 and . 
Now notice that we have for all ,

Summarizing the results obtained so far, we get

where  is the identity matrix defined by: . 

\paragraph{Complexity.}
Computing the matrix  has a time complexity . 
Computing  can be done using Karp algorithm \cite[Theorem
2.19]{BCOQ} in 
time . Computing  for all  requires also a time complexity . 
\end{proof}














Proposition \ref{pr-fatou} is proved for series in
 in \cite[Proposition 4.2]{krob94}. It is not obvious to
extend the approach of \cite{krob94} to series in
. We propose a quite different proof. 

\begin{prpstn}[Fatou property]\label{pr-fatou}
Consider a series  in . Then we have

Furthermore, given an automaton  over  recognizing
, one can effectively compute an automaton  over 
recognizing . The procedure to get  from  has a
polynomial time complexity in the size of . 
\end{prpstn}

\begin{proof}
Let  be a trim triple recognizing  with set of
states . 
Define the matrix . Since , it follows from (\ref{eq-crit}) that
. In particular any circuit has non-positive weight. It
follows immediately that:

where  is the identity matrix of dimension  defined by
. 
Since , it follows that .
Set  and define the diagonal matrix (the non-diagonal coefficients
being ) . Define

Clearly, the automaton  
recognizes the  series .  
We have: ; and also: . At last, we have:
,

where we have used that .
Hence the triple 
is defined over the semiring . This completes the proof. 



\paragraph{Complexity.}
The matrix  is computed in time . 
Then, computing  requires  operations. Knowing ,
computing 
 requires
 operations. 
\end{proof}

Proposition \ref{pr-eq} is proved for series in
 in \cite[Proposition 5.1]{krob94}.
The proof relies on the Fatou property. Since
we have extended this last property to , the
proof of Krob carries over unchanged. In the proof below, we present
the arguments in a slightly different way. 

\begin{prpstn}
\label{pr-eq}
The following problem is decidable:
\begin{tabbing} 
de la marge \= \kill
\> {\bf Instance:} \hspace*{0.4cm} \=  and
 \\
\>  {\bf Problem:} \>  \ \
(i.e. )\:.
\end{tabbing}
\end{prpstn} 

\begin{proof}
First of all, it is enough to prove the result for . Indeed,
testing if  is equivalent to testing if
 where  is  
the series defined by . And it is
straightforward to get a triple recognizing  from a triple
recognizing . 

According to 
Proposition \ref{pr-ineq}, we can decide if  belongs to
. If not, then we have . If  then, by Proposition \ref{pr-fatou}, there exists an
effectively computable automaton  over  recognizing . 
We define an automaton
 as follows: 

with  and  being defined from
 and  in the same
way. 
The important property is that for , 

Let us set 
. Obviously,  is a submonoid
of the finite monoid . In particular,
 is finite and can be effectively
constructed. In view of (\ref{eq-imp}), we have

Since  is finite and effectively computable,
the property on the right can be checked algorithmically. 
\end{proof}

\paragraph{Complexity.}
In contrast with Proposition \ref{pr-ineq}, we do not get a polynomial
procedure in Proposition \ref{pr-eq}. 
Deciding if the right-hand side in (\ref{eq-univ}) holds is PSPACE-complete with respect to the
dimension of the triple, see for instance \cite[Theorem 13.14 and
  Exercise 13.25]{HoUl}.  This is known as 
the {\em universality problem}.


\begin{prpstn}
\label{pr-eq-part}
The following problem is decidable:
\begin{tabbing} 
de la marge \= \kill
\> {\bf Instance:} \hspace*{0.4cm} \=  and
 \\
\>  {\bf Problem:} \> \;.
\end{tabbing}
\end{prpstn}
\begin{proof}
The proof is the same as in Proposition~\ref{pr-eq}.
Instead of deciding the right-hand side of (\ref{eq-univ}), it must be decided whether
 and 
have the same support.
\end{proof}

\paragraph{Complexity.}
The complexity of this problem is PSPACE-complete. 
Indeed,  is obtained
from  by deleting some transitions. 
And deciding whether the language accepted by a non-deterministic automaton
remains the same after the deletion of some transitions is
PSPACE-complete. We briefly explain why. 
First, the equivalence problem for non-deterministic Boolean automata
is PSPACE-complete~\cite{StMe},
and thus our problem is in PSPACE.
Next, let  be a non-deterministic automaton and
let  be the automaton obtained from 
by adding a state, initial and terminal, with loops labelled by evey letter.
Deciding whether  is equivalent to  is equivalent
to deciding whether  accepts every word (universality problem),
which is PSPACE-complete.
Thus our problem is PSPACE-hard.

\medskip

Proposition \ref{pr-maxmin} is proved for series in
 and  in
\cite[Proposition 5.3]{krob94}. As discussed in the Introduction, a
consequence of Proposition \ref{pr-maxmin} is that the equality problem is decidable in
. Quoting
\cite{krob94}: ``the problem remains to characterize (such)
series''. This is done in \S \ref{se-minmax}. 

\begin{prpstn}
\label{pr-maxmin}
The following problem is decidable:
\begin{tabbing} 
de la marge \= \kill
\> {\bf Instance:} \hspace*{0.4cm} \=   \\ 
\>  {\bf Problem:} \> \:.
\end{tabbing}
The above equality should be interpreted as:  and
. 
\end{prpstn} 
\begin{proof}
Define the series  with coefficients in  by
 for all . Clearly . The above problem is equivalent to:
 
Point (a) is the problem of equivalence of rational languages and is thus
decidable. 
The series  is the Hadamard max-plus product of  and ;
it is recognized by the tensor product of triples recognizing  and :

Let  (resp. ) be a trim triple
recognizing  (resp. ) with set of states 
(resp. ). 
Let  be the triple defined on  by:

By Proposition~\ref{pr-eq-part}, (b) is decidable.
\end{proof}

Using the same proof, one also shows that ``'' is
decidable. On the other hand, ``'' is already undecidable for
 and , see \cite{krob}. 

\section{Max-plus and min-plus rational implies
  unambiguous}\label{se-minmax}

To prove that a series recognized by a max-plus and a min-plus
automaton is also recognized by an unambiguous 
automaton, we use an intermediate step which is to prove that it is
recognized by a 1-valued automaton. 

\medskip

Recall that the notion of 1-valuedness of a max-plus automaton has
been defined in \S \ref{se-prel}. 
This notion clearly extends to any automaton with multiplicities over
an idempotent semiring, in particular to a transducer. 
A {\em transducer}  is an automaton over the semiring
. 
The transducer  is {\em 1-valued} (or {\em functional}) if 
 for all . 
Next result is classical and due to Eilenberg \cite{eile} and
Sch\"utzenberger \cite{schu76}, see \cite[Chapter IV.4]{bers79}: 
a 1-valued transducer can be effectively transformed into an
equivalent unambiguous one. 
The proof of Eilenberg and Sch\"utzenberger easily extends 
to a 1-valued automaton with multiplicities in an idempotent
semiring. 
Here we give a different and simple proof of the same result. 
The argument is basically the same one as in
\cite[Section 4]{KLMP04}.  

\begin{prpstn}\label{pr-weber}
For any max-plus or min-plus 1-valued automaton, there exists 
an unambiguous automaton which recognizes the same series.
\end{prpstn}

\begin{proof}
Let~ be a 1-valued automaton and  the underlying
Boolean automaton.
 Let  be the determinized automaton of
  obtained by the subset construction. Let
  be the tensor product of  and :
 
The automaton  is the {\em Sch\"{u}tzenberger covering} of ,
see \cite{saka98}. 
There is a {\it competition} in  if:
\b) there exist ,  and  such that ,
   and .

Let  be  any automaton obtained from  by
removing the minimal number of transitions and/or terminal arrows such
that there is no more competition. 
We claim that  is an unambiguous automaton equivalent to .
The proof of this claim
can be found in~\cite[Section 4]{KLMP04}.
\end{proof}

As a side remark, the above proof is also clearly valid in any idempotent semiring. 

\medskip

We now have all the ingredients to prove the main result. 

\begin{prpstn}
\label{pr-maxmin2}
Let   be a series in . The series  is in
 if and only if the series  is unambiguous. 
\end{prpstn}

\begin{proof}
Let , resp. , be a triple
that recognizes , resp. .
Let   be the triple on the semiring  and with set of states  defined by:

This triple recognizes the series .

For every vector or matrix  with coefficients in ,
for  in , we denote , the projection of  with
respect to the -th coordinate.

By Proposition~\ref{pr-fatou} there exists an
automaton  
equivalent to  and such that
 is over  (the first
ccordinate is unmodified: ).
We define an automaton
 over the semiring
 as follows:

with  and  being defined from
 and  in the same way. 
We claim that  is a
1-valued automaton that recognizes . 

For every word~, every successful path of
 labeled by  has a weight equal to the first coordinate 
of the weight  of a successful path of  such that .
It means that  is the weight of a successful path labeled by  in 
and that  is the weight of a successful path labeled by  in .
Hence,  and ,
and so .
Therefore, every successful path of 
labeled by  has a weight equal to~.

Conversely, every word  in  labels a successful path in .
Indeed, there is a successful path labeled by  with weight  in ,
and a successful path labeled by  with weight  in .
The product of the two paths gives a successful path in
 labeled by  with a weight
having a second coordinate equal to , hence, after applying
Proposition~\ref{pr-fatou}, the weight of every transition along this path has
a second coordinate equal to . 

Therefore  recognizes the same series as . We complete the
proof by applying Proposition \ref{pr-weber}. 
\end{proof}

There is a canonical bijection from  to  that consists
in mapping every  different from  onto itself and 
onto . This bijection is obviously {\it not} an isomorphism.
With some abuse, we say that a series  of  is also
in  if its image with respect to the canonical bijection
is in .

\begin{crllr}
\label{pr-maxmin3}
A series  is in  if and
only if it is unambiguous. Starting from a pair formed by a max-plus
and a min-plus automaton recognizing , one can effectively compute an
unambiguous automaton recognizing . 
\end{crllr}

Observe that given a pair
formed by a max-plus and a min-plus automaton, it can be checked if
they indeed recognize the same series using Proposition
\ref{pr-maxmin}. 

\begin{proof}
Since  is in ,
 is in  (there is an {\it isomorphism}
from  onto  that maps  onto ). This result is
therefore equivalent to Proposition~\ref{pr-maxmin2}. The effective
computation of an unambiguous automaton recognizing  is done in the
proof of Proposition~\ref{pr-maxmin2}. 
\end{proof}

\paragraph{Complexity.} In Corollary \ref{pr-maxmin3}, one gets 
a 1-valued automaton recognizing  of dimension the product of the dimensions
of the max-plus and min-plus automata. This follows directly from the proof of
Proposition \ref{pr-maxmin2}. The time complexity to construct it is
also clearly polynomial. 
On the other hand, the dimension of an unambiguous automaton
recognizing  may be exponential with respect to the dimension of
the 1-valued automaton. 

\section{Examples}

Let  be the series defined by .
This series is obviously max-plus rational. In~\cite{KLMP04}, it is proved 
that  is not unambiguous (section~),
and with a different argument that it is not min-plus rational (section~).
We know now that both statements are equivalent.

\medskip

We consider now a simple example on which we illustrate the different
steps of our proof. 

\medskip

\begin{figure}[ht]
\subfigure[Two automata and their product]{\FixVCScale{.5}
\VCDraw[.8]{\begin{VCPicture}{(-4.5,-10.5)(10,5)}
\renewcommand{\ForthBackEdgeOffset}{-3}
\renewcommand{\LoopLabelPosit}{.5}
\ForthBackOffset
\VCPut{(-4.5,4)}{\scalebox{2}{}}
\VCPut{(-2.5,5.5)}{\scalebox{2}{}}
\VCPut[-90]{(-3,0)}{\State{(0,0)}{A}\State{(8,0)}{B}}
\EdgeR{A}{B}{a|1} \EdgeR{B}{A}{a|1}
\Initial[s]{A}\Final[se]{A}
\LoopN{A}{\StackTwoLabels{a|2}{b|1}}
\LoopS{B}{b|3}
\VCPut[0]{(0,4.5)}{\State{(0,0)}{A}\State{(8,0)}{B}}
\Initial{A}\Final[se]{A}\FinalL{e}{B}{|1}
\EdgeR{A}{B}{a|1} \EdgeR{B}{A}{\StackTwoLabels{a|1}{b|2}}
\LoopN{A}{b|1}
\LoopN{B}{\StackTwoLabels{a|0}{b|1}}
\State{(0,0)}{A}\State{(8,0)}{B}\State{(0,-8)}{C}\State{(8,-8)}{D}
\Initial{A}\Final[s]{A}\FinalL{e}{B}{|1,1}
\LoopN{A}{b|1,0}
\LoopN[.5]{B}{\StackTwoLabels{a|0,-2}{b|1,0}}
\LoopS{C}{b|1,-2}
\LoopS{D}{b|1,-2}
\EdgeR{A}{B}{a|1,-1} \EdgeR{B}{A}{\StackTwoLabels{a|1,-1}{b|2,1}}
\EdgeR{B}{D}{a|0,-1} \EdgeR{D}{B}{a|0,-1}
\EdgeBorder
\EdgeR[.3]{A}{D}{a|1,0}
\EdgeR[.3]{B}{C}{a|1,0} 
\EdgeR[.3]{D}{A}{a|1,0}
\EdgeR[.3]{C}{B}{a|1,0}
\RstEdgeOffset
\EdgeL{D}{C}{b|2,-1}
\end{VCPicture}}}
\quad\vline
\subfigure[Application of the Fatou property]{\FixVCScale{.5}
\VCDraw[.8]{\begin{VCPicture}{(-4,-10.5)(12,5)}
\renewcommand{\LoopLabelPosit}{.5}
\renewcommand{\ForthBackEdgeOffset}{-3}
\State{(0,0)}{A}\State{(8,0)}{B}\State{(0,-8)}{C}\State{(8,-8)}{D}
\Initial{A}\Final[s]{A}\FinalL{e}{B}{|1,0}
\LoopN{A}{b|1,0}
\LoopN{B}{\StackTwoLabels{a|0,-2}{b|1,0}}
\LoopS{C}{b|1,-2}
\LoopS{D}{b|1,-2}
\EdgeL{D}{C}{b|2,0}
\ForthBackOffset
\EdgeR{A}{B}{a|1,0} \EdgeR{B}{A}{\StackTwoLabels{a|1,-2}{b|2,0}}
\EdgeR{B}{D}{a|0,-2} \EdgeR{D}{B}{a|0,0}
\EdgeBorder
\EdgeR[.3]{A}{D}{a|1,0}
\EdgeR[.3]{B}{C}{a|1,0} 
\EdgeR[.3]{D}{A}{a|1,0}
\EdgeR[.3]{C}{B}{a|1,0}
\end{VCPicture}}}
\caption{Getting an unambiguous automaton (I)}\label{fig:ex}
\end{figure}

Let  and  be the two automata drawn on
Figure~\ref{fig:ex}-(a) (the weights equal to 0 on ingoing or outgoing
arrows have been omitted).
The automaton  is a max-plus automaton, while
the automaton  is a min-plus automaton. Their product, performed as
in the proof of Proposition~\ref{pr-maxmin2}, is drawn on the same figure.
The automata are equivalent only if the weight with respect to the second coordinate is
non-positive on every successful path. Hence, we can apply the Fatou property
(Proposition~\ref{pr-fatou}) to get an equivalent automaton on which the
weight ot the second coordinate is non-positive on every arc (transitions, and initial
and final arrows). The result is shown on Figure~\ref{fig:ex}-(b). After
deleting the arcs that have a second coordinate weight different
from~, and remembering only the first coordinate, we get the 
-valued automaton of Figure~\ref{fig:ex2}-(a). As this automaton has the
same support as  and , we can conclude that
 and  are indeed equivalent. We can then turn this
-valued automaton into an unambiguous one (Figure~\ref{fig:ex2}-(b)),
using the construction of Proposition~\ref{pr-weber}.

\begin{figure}[t]
\subfigure[The 1-valued automaton]{\FixVCScale{.5}
\VCDraw{\begin{VCPicture}{(-2,-11)(10,4)}
\renewcommand{\LoopLabelPosit}{.5}
\renewcommand{\ForthBackEdgeOffset}{-3}
\State{(0,0)}{A}\State{(8,0)}{B}\State{(0,-8)}{C}\State{(8,-8)}{D}
\Initial{A}\Final[s]{A}\FinalL{e}{B}{|1}
\LoopN{A}{b|1}
\LoopN{B}{b|1}
\EdgeL{D}{C}{b|2} \EdgeR{D}{B}{a|0}
\ForthBackOffset
\EdgeR{A}{B}{a|1} \EdgeR{B}{A}{b|2}
\EdgeBorder
\EdgeR[.3]{A}{D}{a|1}
\EdgeR[.3]{B}{C}{a|1} 
\EdgeR[.3]{D}{A}{a|1}
\EdgeR[.3]{C}{B}{a|1}
\end{VCPicture}}}
\vline
\subfigure[The unambiguous automaton]{\FixVCScale{.4}
\VCDraw[1.4]{\begin{VCPicture}{(-4.5,-1)(9.5,14)}
\State{(-3,12)}{A}
\State{(-3,9)}{B}
\State{(-3,6)}{C}
\State{(-3,3)}{D}
\State{(-3,0)}{E}
\Initial{A}\Final{A}\Final{B}\Final{C}\Final{D}\Final{E}
\LoopN{A}{b}
\EdgeR{A}{B}{a}
\EdgeR{B}{C}{a,b}
\ArcR{C}{D}{a}
\ArcR{D}{C}{a,b}
\LArcR{C}{E}{b}
\EdgeR{E}{D}{a}
\LoopS{E}{b}
\State{(0,15)}{A}
\State{(3,15)}{B}
\State{(6,15)}{C}
\State{(9,15)}{D}
\Initial[nw]{B}\Final[ne]{B}\FinalL[.6]{s}{C}{|1}
\LoopN{B}{b|1}
\LoopN{C}{b|1}
\LArcR[.85]{A}{C}{a|0}
\LArcR[.85]{A}{D}{b|2}
\renewcommand{\ForthBackEdgeOffset}{3}
\ForthBackOffset
\EdgeL{B}{C}{a|1}
\EdgeL[.7]{C}{B}{b|2}
\EdgeL[.5]{A}{B}{a|1}
\EdgeL[]{B}{A}{}
\EdgeL[.5]{C}{D}{a|1}
\EdgeL[]{D}{C}{}
\RstEdgeOffset
\State{(3,12)}{A}
\State{(6,9)}{B}
\State{(0,9)}{C}
\State{(3,6)}{D}
\State{(9,6)}{F}
\State{(3,0)}{G}
\State{(6,3)}{H}
\State{(0,3)}{I}
\Initial{A}\Final{A}\FinalL{e}{B}{|1}\Final{D}\FinalL{s}{H}{|1}\Final{G}
\EdgeBorder
\LoopN{A}{b|1}
\EdgeL{A}{B}{a|1}
\EdgeR{A}{C}{a|1}
\EdgeR[.2]{B}{D}{b|2}
\EdgeL{B}{F}{a|1}
\EdgeR{C}{D}{a|1}
\EdgeL[.2]{C}{F}{b|2}
\EdgeR[.7]{D}{G}{b|1}
\LoopS{G}{b|1}
\EdgeR{G}{H}{a|1}
\EdgeL{G}{I}{a|1}
\EdgeR[.1]{I}{F}{b|2}
\renewcommand{\ForthBackEdgeOffset}{3}
\ForthBackOffset
\EdgeL[.2]{D}{H}{a|1}
\EdgeL[.2]{H}{D}{b|2}
\EdgeL{F}{H}{a|1}
\EdgeL[.2]{H}{F}{a|1}
\EdgeL{D}{I}{a|1}
\EdgeL{I}{D}{a|1}
\end{VCPicture}}}
\caption{Getting an unambiguous automaton (II)}\label{fig:ex2}
\end{figure}

This example is ``artificial''. For instance, we can 
get an equivalent two states unambiguous automaton from the max-plus one
only by deleting some transitions. This does not imply that there always
exists an unambiguous automaton that has a number of states less or equal to
the number of states of either the max-plus or the min-plus automaton.
We now give an example that enhances this point.

\medskip

Recall first that every max-plus or min-plus series over a one-letter alphabet is
unambiguous~\cite{BoKr,moll}.
We now make the following claim (the proof is not difficult): 
If  is a max-plus rational series over the one-letter alphabet
, and if the sequence  is periodic
of minimal period , then the smallest 1-valued automaton
recognizing  is of dimension , and is deterministic. 

Let , and  be four distinct prime numbers. For ,
define the series  on  by:

If  is not in , set
 with the convention that 
is neutral for both  and  and absorbing for .
We then consider the series  defined by:

The series  and , and therefore , are unambiguous, so
they belong to . 
\begin{figure}[ht]

\caption{A max-plus automaton recognizing .}\label{fi-last}
\end{figure}
The series  is recognized by the max-plus
automaton of dimension  given in Figure \ref{fi-last}. 
A min-plus (and determinisic) automaton recognizing  is the
following one (for ): 

States: ; transitions: ; initial state:
; final states:
 for , and
 for .

\medskip

And similarly for , the small
automaton being the min-plus one. 
Therefore, the series  is recognized by a max-plus automaton of
dimension , and a min-plus one of dimension . Now
observe that  is periodic of minimal period . 
Using the above claim, the smallest 1-valued (or unambiguous, or
deterministic) automaton recognizing  is of dimension . 



\begin{thebibliography}{1}

\bibitem{BCOQ}
F.~Baccelli, G.~Cohen, G.J. Olsder, and J.P. Quadrat.
\newblock {\em Synchronization and Linearity}.
\newblock John Wiley \& Sons, New York, 1992.

\bibitem{bers79}
J.~Berstel.
\newblock {\em Transductions and context-free languages}.
\newblock B. G. Teubner, 1979.

\bibitem{BoKr}
A.~Bonnier-Rigny and D.~Krob.
\newblock A complete system of identities for one-letter rational expressions
  with multiplicities in the tropical semiring.
\newblock {\em Theoretical Computer Science}, 134:27--50, 1994.



\bibitem{eile}
S.~Eilenberg.
\newblock {\em Automata, languages and machines}, volume~A.
\newblock Academic Press, New York, 1974.



\bibitem{GaMa98b}
S.~Gaubert and J.~Mairesse.
\newblock Modeling and analysis of timed {P}etri nets using heaps of pieces.
\newblock {\em IEEE Trans. Aut. Cont.}, 44(4):683--698, 1999.

\bibitem{HIJi}
K.~Hashigushi, K.~Ishiguro, and S.~Jimbo.
\newblock Decidability of the equivalence problem for finitely ambiguous
  finance automata.
\newblock {\em Int. J. Algebra Comput.}, 12(3):445--461, 2002.

\bibitem{HoUl}
J.~Hopcroft and J.~Ullman.
\newblock {\em Introduction to automata theory, languages, and computation}.
\newblock Addison-Wesley Publishing Co., 1979.

\bibitem{KLMP04}
I.~Klimann, S.~Lombardy, J.~Mairesse, and C.~Prieur.
\newblock Deciding unambiguity and sequentiality from a finitely ambiguous
  max-plus automaton.
\newblock {\em Theoret. Comput. Sci.}, 2004.
\newblock To appear. Short version in LNCS 2710:373--385, 2003.

\bibitem{krob}
D.~Krob.
\newblock The equality problem for rational series with multiplicities in the
  tropical semiring is undecidable.
\newblock {\em Int. J. Algebra Comput.}, 4(3):405--425, 1994.

\bibitem{krob94}
D.~Krob.
\newblock Some consequences of a {F}atou property of the tropical semiring.
\newblock {\em J. Pure Appl. Algebra}, 93(3):231--249, 1994.

\bibitem{mohr}
M.~Mohri.
\newblock Finite-state transducers in language and speech processing.
\newblock {\em Comput. Linguist.}, 23(2):269--311, 1997.

\bibitem{moll}
P.~Moller.
\newblock {\em Th\'eorie alg\'ebrique des syst\`emes \`a \'ev\'enements
  discrets}.
\newblock PhD thesis, \'Ecole des Mines, Paris, 1988.

\bibitem{saka98}
J.~Sakarovitch.
\newblock A construction on finite automata that has remained hidden.
\newblock {\em Theoret. Comput. Sci.}, 204(1-2):205--231, 1998.

\bibitem{schu76}
M.-P. Sch{\"u}tzenberger.
\newblock Sur les relations rationnelles entre mono\"{\i}des libres.
\newblock {\em Theoret. Comput. Sci.}, 3(2):243--259, 1976/77.

\bibitem{simo88}
I.~Simon.
\newblock Recognizable sets with multiplicities in the tropical semiring.
\newblock In {\em Mathematical Foundations of Computer Science, Proc. 13th
  Symp.}, number 324 in LNCS, pages 107--120, 1988.

\bibitem{StMe}
L.~Stockmeyer and A.~Meyer.
\newblock Word problems requiring exponential time: preliminary report.
\newblock In {\em Fifth Annual ACM Symposium on Theory of Computing}, pages
  1--9. Assoc. Comput. Mach., New York, 1973.

\bibitem{webe94}
A.~Weber.
\newblock Finite-valued distance automata.
\newblock {\em Theor. Comput. Sci.}, 134(1):225--251, 1994.

\end{thebibliography}

\end{document}
