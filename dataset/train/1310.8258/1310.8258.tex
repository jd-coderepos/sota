\chapter{Information Centric Networks}
\label{chapt:icn}
 
In this chapter, we present several Information Centric Network proposals encountered in the literature.
We first present DONA, an effort to illustrate drawbacks related to name resolution systems.
We classify the remainder proposals into two groups: $(i)$ distributed hash table proposals and $(ii)$
bread crumb proposals. We focus on the drawbacks related to publish/subscribe and content distribution issues
in both groups, and by the end of this chapter, present guidelines towards an effort to enhance the ICN design. 
The terminology used in the rest of this proposal is presented in Table~\ref{terminology}.  
Figure~\ref{tabproposals} summarizes some of the key features of the ICN proposals discussed in this chapter.


\vspace{0.1in}

\begin{table}[h!]
\center
\begin{tabular}{l|l}
\hline
\textbf{term} & \textbf{description} \\
\hline
\hline
content-data & data stored within the network \\
content-name &  unique name identifier for each content-data \\ 
lookup-request & request for content-data issued by a user \\
locator &  pointer towards content-data storage places \\
cache-router & router able to cache content-data \\
\hline
\end{tabular}
\caption{Table of notation}
\label{terminology}
\end{table}

 
\begin{figure}
\center
\includegraphics[scale=0.6]{proposal}
\label{tabproposals}
\caption{ICN proposals}
\end{figure}


\section{DONA}

DONA ~\cite{dona} is a content distribution solution which embraces ideas built on top of the 
TRIAD ~\cite{triad}, HIP ~\cite{hip} and SFS ~\cite{sfs} proposals. 
DONA adopts a route-by-name to the closest copy approach. Content names are flat. A set of 
\emph{resolution handlers} (RH), used to bind names to locations, are disposed in the network according to an hierarchical tree. Each domain 
has one logical RH, but might have  many physical instances. The use of flat names makes content identification by users
harder (e.g. $m$-bit identifier for each name). It is expected that users obtain the flat names for desired content-data 
through an external mechanism, such as a \emph{search engine}. To provide high-availability, the name resolution process should: 
$(i)$ forward lookup-requests to nearby copies of content-data, and $(ii)$ avoid failed or overloaded servers. 

\begin{figure}[h!]
\center
\includegraphics[scale=0.35]{fig2_1}
\caption{DONA Proposal}
\label{donasystem}
\end{figure}


Each RH stores information for all the contents published in its descendant RHs. 
Since DONA employs flat content-names, not considering name aggregation, a 
single \emph{logical RH} at the top level domain of the hierarchy imposes severe scalability problems. How
the different physical instances of RHs are interconnected, within a given hierarchical level, is left as an open issue.
Figure~\ref{donasystem} illustrates a lookup-request flowing through a set of resolution handlers disposed according to a structured
two level tree. A pointer location to the requested content-data is stored at the logical 
resolution handler in the second level. Content-data flows through a TCP/IP connection towards the user who submitted
the lookup-request. 






\section{Distributed Hash Table Systems}


A distributed hash table (DHT) is a hash table distributed among a dynamic set of participating
nodes. The hash space is subdivided into subsets partitions, with no intersection among them. 
Each partition is managed by one node.  Nodes are identified by a hash-prefix $id$ in the same range of the hashing key-space. 
DHT might be flat or hierarchical, comprising \emph{tiers}~\cite{canon}.  In DHT systems, both lookup-requests 
and content-data are routed according to a structured geometry topology. 


Hash-keys are obtained by applying a hash function to the content-ids.  In some cases, the hash-keys might be the content-ids themselves.
 Lookup-requests 
and content-data are forwarded by the network based on a key-based routing strategy. Once a message \emph{push($k$, content-data)} 
is sent to any node participating in the DHT, the message is forwarded from node to 
node through the network, until encountering the node in charge for hash key $k$. The content will be stored at this node. Each node has an $id$ is responsible for a a set of 
keys.  If a node with identification $id$ is responsible for key $k$, we also say that such a node has an $id$ that owns key $k$.
 Each node will have either a node $id$ that owns $k$, 
or a link to a node whose node $id$ is closer to $k$. 


To get content-data associated with the hash-key $k$, a message \emph{pop($k$)} 
is sent and routed through the network until reaching a node whose node $id$ owns $k$, which will then reply with the stored content-data. 
Each node maintains a set of links to other nodes, its neighbors. Trees, hypercubes and rings are examples of geometries. 
DHT-based systems can guarantee that any data object is located/delivered using at most $O(\log N)$ hops on average, where $N$ 
is the number of nodes in the system. 


An example of a DHT architecture, based on a ring geometry is presented in Figure~\ref{systemdhticn}. In this example, the hash key  
has 8 bits and the hash partition has 2 bits. A user sends a 
lookup-request for content-id 11011011, through the node of the ring identified by the hash-prefix $id$ 00. The lookup-request is 
then sent to the successor node identified by the hash-prefix $id$ 10, before reaching the node $id$ 11. This node is responsible for managing 
the partition where the corresponding content-data for content-id 11011011 is stored at.


\begin{figure}
\center
\includegraphics[scale=0.35]{fig2_2}
\caption{ICN with DHT, ring geometry, 8-bit hash key, 2-bit partition set}
\label{systemdhticn}
\end{figure}


Although DHT search overhead is small, the maintenance overhead might be large in case of node failures/repairs.  
Whenever a node fails or becomes repaired, requests must be sent along the network to reassure the node location in the network.
Neighborhood adjacencies must be reestablished and also, file-index databases related to a partition management must be resettled 
~\cite{dhtmanagement}. 



DHT structures might suffer from reliability flaws 
caused by malicious nodes participating in the DHT and not behaving according to how the protocols were originally set ~\cite{securitydht}.
In particular, DHTs are prone to Sybil attacks and Eclipse attacks. For instance, if a node responsible for managing
the content-data related to a partition misbehaves, all the information for that partition cannot be safeguarded.



PSIRP ~\cite{psirp}, MDHT/Netinf ~\cite{mdht} and MCACHE  ~\cite{multicache} are examples of ICN proposals relying on DHT. 
In the next subsections, publish/subscribe and content distribution issues are discussed in light of these proposals, . 
In DHT proposals, \emph{names} are non-human readable, but can be aggregated according to hash prefixes.  
Both PSIRP and MDHT/Netinf use name resolution systems, whereas MCACHE does not.


\subsection{Publish/Subscribe Issues in DHTs}

In MDHT/Netinf and PSIRP, publications and subscriptions are forwarded to a name resolution system (NRS),
organized according a multi-level hierarchical DHT structure. In PSIRP, the NRS is called \emph{rendezvous}.  
One big advantage of MDHT/Netinf and PSIRP, to DONA's proposal, is that different name resolution servers at the same hierarchical level 
can support name resolution instead of only one logical resolution server per hierarchical level. Each partition of the hash space in a level 
is assigned to a different node server. 


The highest levels of the hierarchy must be highly scalable due to the large number of globally available data objects. 
Failures or performance issues in the top levels might affect a large number of nodes in the network.  MDHT/Netinf claims also that for security reasons, published information should be managed on systems under the control of 
publisher trusted authorities. Interconnecting all nodes through an structure at top level hierarchies would hamper the deployment
of selected trusted authorities by the publishers. MDHT/Netinf proposes that hierarchical top levels name resolution servers should be
disposed in a Global Name Resolution (REX), managed by third parties. The REX would be connected with lower DHT hierarchical levels,
managed by local trusted authorities. The design of robust and scalable REX systems is an important open problem. 



\subsection{Content Distribution Issues in DHTs}


In MDHT/Netinf and PSIRP, all replicas are registered at the NRS. To reduce the overhead on the NRS, the replication policies should not
change cache content too frequently. Finding the right frequency at which to update cache content is an important open problem. 
Another issue regarding NRS is the delay between object placement and NRS synchronization ~\cite{netinf}.


In MCACHE, a rendezvous point is assigned to each published content-data.  This rendezvous point  becomes the root node of
a multicast tree, established along the routers. The same DHT tree structure for routing lookup-requests routes
content-data. The tree structure allows the building of the multicast trees for content-data delivery. 
A copy of each content-data, sent through a multicast tree, is stored at the closest
cache-router connected to the subscribers. 


Lookup-requests issued by the subscribers may be serviced by cached copies.  
Nonetheless, since 
caching of content is restricted to occur at cache-routers close to requesters,
  the limited storage space on such cache-routers might lead to frequent content replacements.
The maintenance of fresh pointers in the network in light of such frequent updates remains a challenge to be addressed.



\begin{comment}
These frequent replacements, in turn, might cause pointers to such contents, stored in other routers in the network, to rapidly stale. 
so to allow the discovery of such copies can point to copies that have already been replaced at those caches,
by newer copies, because of 
\end{comment}




\section {Bread crumb Systems}

Bread crumb (BC) systems are ICN approaches that performs content-id routing at each router without a fixed geometry topology. 
It is assumed that each router knows how to forward lookup requests, based on named-content tables. BC ~\cite{elisha2}, BC+ ~\cite{bc+}
and CCN ~\cite{ccn1} are examples of bread crumb systems. CCN relies on routing protocols to fill and set up named-content tables, 
whereas BC and BC+ do not specify how to fill those tables. Content-data, when sent to subscribers, may be stored as 
replicas in the cache-routers.  


Figure ~\ref{systemicnbread} shows a lookup-request being routed by its name according to named-content tables in routers. 
The names 
can be either human readable names or non-human readable names. When a user sends a lookup-request, the request carries
a content-id of the desired content. The routers forwards the request until a replica of
the content is reached and stores the request itself as a bread crumb, while the request is forwarded. 
The set of bread crumbs of a given request compose
a trail, represented by the green arrows in the figure. Once found, content-data follows the trail of bread crumbs
in the reverse path, until being delivered to the user that issued the content subscription. The path followed
by content-data is illustrated by the blue arrows.


\begin{figure}
\center
\includegraphics[scale=0.35]{fig2_3}
\caption{ICN with bread crumbs}
\label{systemicnbread}
\end{figure}



\subsection{Publish/Subscribe Issues in Bread crumb Systems}


\begin{comment}
The lookup requests are directed towards the nearest copy stored in the network, according to some metric 
at named-content tables. \eat{Lookup-requests carry content-ids composed of human readable names. Those } Content names are used
by the routers, to forward the requests according to information contained in routing-tables. As lookup-requests are forwarded, they are stored at 
the routers as a bread crumbs, which implies lookup-requests leaving a trail of bread crumbs from the subscriber up to 
the corresponding content-data storage location. Content-data is delivered in the reverse path of bread crumbs.

\end{comment}

Lookup requests are routed towards publicly known sources of  content-data. 
Nevertheless, while being forwarded, it may encounter a router wherein a bread crumb for that content-data
is already stored. In this case, following a previously established trail of bread crumbs for that content-data might be more efficient than 
retrieving data from  the publicly 
known source.  For this reason, the lookup request can then be alternatively routed towards an alternative subscriber,
through the bread crumb trail, in an attempt to find a replica of the file. 



According to BC and BC+, bread crumbs are guidance information for the location of content caches,
being designed with the purpose of reducing the load at storage locations (publishers).
An important issue comprising bread crumbs relies on the possibility that some lookup requests, 
when being forwarded towards cached replicas, cannot reach the intended content-data,
due to content-data not existing in the cache anymore. BC+ uses different techniques from BC, 
in order to enhance the probability of lookup requests finding replicas in caches. BC and BC+ leave a number of open 
issues, such as how the named-routing tables should be filled when content-data is published.

\eat{ (*obs to be shown difference between BC and BC+). }




\subsection{Content Distribution Issues in Bread Crumb Systems}


In Bread Crumb systems, content-data is sent to subscribers in the reverse path of bread crumb trails and  
 can be replicated as it is forwarded to users. Content-named-tables can be updated on the fly as content-data 
is sent to subscribers. In CCN, which adopts human readable names, an extension of IP routing protocols, such as OSPF and IS-IS, are proposed for intra-domain routing.  This extension determines how to populate named-content tables in the routers. For inter-domain routing ($i.e.,$ if a router in parc.com domain would like to obtain content 
with the prefix mit.edu), CCN proposes an extension of BGP, another IP network routing protocol, which can also determine how to populate the
named-content tables. 


In general, content-data replicas will not be stored in the domain they were originally published. 
To be encountered, all replicas should be announced by routing protocols in CCN referenced by their global names.  
Avoiding an  explosion of routing table sizes is an important challenge. Assuming published data-content are announced in routing tables only  through their prefixes, an estimate of the number of name prefixes that need to be supported is given by the number of second-level domains 
in the Domain Name System. This number was around 110 million in 2010 ~\cite{netinf}. 





\section{Summary of Open Problems and System Design Guidelines}

\label{sec:summaryopendesign}


Next, we introduce the main guidelines of our system design in light of the background introduced in this chapter.
  Our guidelines target two aspects of ICNs, namely the name resolution system and load balancing.

\begin{itemize}
\item \textbf{Name resolution system (NRS): }  
based on our discussion so far, it should be clear that the NRS  of an ICN 
should be distributed.  Gathering global knowledge about all content-data replicated in the network is 
very difficult, maybe infeasible (see~\textsection\ref{sec:motivation}).  
Also, trying to maintain distributed knowledge in the network, 
in the absence of a NRS, is difficult due to maintainability overheads  
and attacks that could disrupt pre-defined policies (see~\textsection\ref{envisioned}). Another important consideration concerns their
 published content being managed by trusted parties. \emph{A global single structured NRS fully 
connected would need to count only with trusted parties.  }
\item \textbf{Load balancing: }  
Bread crumbs proposals consider load balance policies for replicas of a given 
content over cache routers. 
Nevertheless, the cached replicas are not advertised to the neighborhood of the cache-router where it is stored.
  The information about the replicas is restricted only to the cache routers that the content-data traversed, when sent to users.
\emph{None of the proposals reviewed presents a deeper analysis on the load supported 
by the servers that store published content without disrupting service capacity for solving names 
in name resolution systems. }
\end{itemize}



In this thesis proposal, we devise a solution based on 1) \emph{a tiered architecture}, whereas
 separate \emph{storage areas} are dedicated to different published content, and 2) 
\emph{randomized algorithms} to search for published content across the tiers. 

\begin{enumerate}
\item \textbf{tiered architecture and storage areas: }  we propose randomized  algorithms to encounter popular 
contents in cache routers, in an effort to reduce the searching load in the storage areas. 
Note that we assume that the nodes in a storage areas are not connected according to any pre-established structure or topology. 
\item \textbf{random walks for search and load balance: } random walks have been proposed as a key algorithm
to  address 
topology search/maintenance in peer-to-peer (P2P) networks~\cite{nearoptimal}.
Nevertheless, to the best of our knowledge, previous work in the realm of Information Centric Networks  
has not considered the adoption of  randomized  algorithms for content discovery and replication. 
\end{enumerate}


Randomization is useful for solving complex problems in various applications ~\cite{mitzenmatcher}.  
Two principles at the heart of random algorithms are strongly linked to our investigation~\cite{motwani} are,
\begin{enumerate}
\item \textbf{foil the adversary:} while an adversary may be able to construct an input that foils a
deterministic algorithm, it is difficult to devise an algorithm that can defeat a
random algorithm. Foil the adversary is a key point to avoid nodes not behaving according to how the
protocols were originally set out. The absence of a \emph{deterministic}
policy for routing lookup requests and content-data avoids many outsiders exploring vulnerabilities
of deterministic predefined-defined policies;
\item \textbf{search/load balance:} in the absence of global knowledge, random search allows environment exploration
targeting opportunistic discovering of new resources. For problems involving choice between a number of known resources, 
such as communication links and source of contents, randomization can be used to spread the load among the resources. 
On the fly decisions allows load balance strategies to be applied, so to provide environment adaptability.
\end{enumerate}
These principles will be further discussed in the next chapter, where we introduce random walks. 





