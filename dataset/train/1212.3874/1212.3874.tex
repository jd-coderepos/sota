\documentclass[copyright,creativecommons]{eptcs}
\providecommand{\event}{ICE 2012} 

\usepackage{amsmath}
\usepackage{amssymb}
\usepackage{amsfonts}
\usepackage{enumerate}
\usepackage{times}
\usepackage{bm}
\usepackage{tikz}
\usepackage[all]{xy}
\usepackage{multirow}
\usepackage{algorithm}
\usepackage{algorithmic}
\usepackage{hyperref}
\usepackage{amsthm}
\usepackage{stmaryrd}
\usepackage{enumerate}
\usepackage{bm}

\newcommand{\tellp}[1]{\tell(#1)}
\newcommand{\askp}[2]{\ask \  #1 \  \rightarrow \ #2}
\newcommand{\skipp}{\Skip}
\newcommand{\defsymbol}{\stackrel{\textup{\texttt{def}}}  {=}}
\newcommand{\defp}[2]{#1 \defsymbol #2}
\newcommand{\todo}[1]
{\marginpar{\baselineskip0ex\rule{2,5cm}{0.5pt}\\begin{array}{c}
 \bigfrac{P \trans{a}Q}{P \newtrans{a} Q} \qquad
 \bigfrac{}{P \newtrans{\tau} P} \qquad
 \bigfrac{P \newtrans{\tau} P_1 \newtrans{a} Q_1 \newtrans{\tau} Q}{P \newtrans{a} Q} \\
\end{array}\begin{array}{c}
 \bigfrac{P \trans{a}Q}{P \newtrans{a} Q} \qquad
 \bigfrac{}{P \newtrans{true} P} \qquad
 \bigfrac{P \newtrans{a} Q \newtrans{b} R }{P \newtrans{a\sqcup b} R} \\
\end{array}\begin{array}{|c|}
\hline
\makebox{MR1} \quad \bigfrac{\G \trans{\A} \G'}{\G \newtrans{\A} \G'} \qquad
\makebox{MR2} \quad \bigfrac{}{\G \newtrans{\true} \G} \qquad
\makebox{MR3} \quad \bigfrac{\G \newtrans{\true} \G_1 \newtrans{\A} \G_2 \newtrans{\true} \G'}{\G \newtrans{\A} \G'} \\
\hline
\end{array}
P,Q::= \Stop \mid \tell(c) \mid \ask(c)\rightarrow P \mid P
\parallel Q  \mid P
 +   Q
\makebox{R1} \pairccp{\tellp{c}}{d} \trans{}
\pairccp{\Stop}{d \sqcup c} \quad
\makebox{R2} \bigfrac{c \sqsubseteq d}{\pairccp{\askp{(c)}{P}}{d} \trans{} \pairccp{P}{d}}
\quad
\makebox{R3} \bigfrac{\pairccp{P}{d} \trans{}
\pairccp{P'}{d'}}{\pairccp{P \parallel Q}{d} \trans{}
\pairccp{P'\parallel Q}{d'}} \quad
\makebox{R4} \bigfrac{\pairccp{P}{d} \trans{}
\pairccp{P'}{d'}}{\pairccp{P \, + \, Q}{d} \trans{}
\pairccp{P'}{d'}}

\makebox{LR1}\pairccp{\tellp{c}}{d} \trans{\true}
\pairccp{\Stop}{d \sqcup c} \quad
\makebox{LR2}\bigfrac{\alpha \in \min \{a\in \Con_0 \, | \, c
\sqsubseteq d \sqcup a \ \}} {\pairccp{\askp{(c)}{P}}{d}
\trans{\alpha} \pairccp{P}{d \sqcup \alpha}} \quad
\makebox{LR3}
\bigfrac{\pairccp{P}{d} \trans{\alpha} \pairccp{P'}{d'}}
{\pairccp{P\parallel Q}{d} \trans{\alpha} \pairccp{P'\parallel
Q}{d'}} \quad
\makebox{LR4}
\bigfrac{\pairccp{P}{d} \trans{\alpha} \pairccp{P'}{d'}}
{\pairccp{P + Q}{d} \trans{\alpha} \pairccp{P'}{d'}}
 \satstbisim = \symbis = \irrbis.\mbox{(a) } \pairccp{P+Q}{z<5} \trans{x<7} \pairccp{T}{z<5 \lub x<7} ~~~~~~~~~~~~
\mbox{(b) } \pairccp{P+Q}{z<5} \trans{x<5} \pairccp{T}{z<5 \lub x<5} \mbox{(e) } \pairccp{R+S}{\true} \trans{z<7} \pairccp{P}{z<7} ~~~~~~~~~~~~
\mbox{ (f) }\pairccp{R+S}{\true} \trans{z<5} \pairccp{P+Q}{z<5} \begin{array}{|c|}
\hline
\makebox{\rTau} \quad \bigfrac{}{\G \newtrans{} \G} \qquad
\makebox{\rLabel} \quad \bigfrac{\G \trans{\A} \G'}{\G \newtrans{\A} \G'} \qquad
\makebox{\rAdd} \quad \bigfrac{\G \newtrans{\A} \G' \newtrans{\B} \G''}
{\G \newtrans{\A \lub \B} \G''} \\
\hline
\end{array}
\label{eq:newSymToWeak:1}
\conf{Q}{d \lub \overbrace{\underbrace{\B \lub b'}_\A \lub b}^a} \reds \conf{Q'}{\overbrace{d' \lub b'}^{d''} \lub b}

Then, the transition  can be
rewritten as , and using
\eqref{eq:newSymToWeak:1}, . It is left to prove
that  which follows from .
\end{proof}
\end{lemma}


Using Lemma \ref{lem:weakToNewSym} and Lemma \ref{lem:newSymToWeak} we obtain
the following theorem.

\begin{theorem}
\label{th:weakSimEqNewSym}
 iff 
\end{theorem}

From the above results, we conclude that .
Therefore, given that using ccp-PR in combination with  (and )
we can decide , then we can use the same procedure to check
whether two configurations are in .

\section{Concluding Remarks} \label{sec:conclusions}
We showed that the transition relation given by Milner's saturation method is
not complete for ccp (in the sense of Definition \ref{def:completeness}). As
consequence we also showed that weak saturated barbed bisimilarity 
\cite{Aristizabal:11:FOSSACS}  cannot be computed using the ccp partition
refinement algorithm for (strong) bisimilarity ccp wrt to this transition
relation.
We then presented a new transition relation using another saturation mechanism
and showed that it is complete for ccp. We also showed that
the ccp partition refinement can be used to compute  using the new
transition relation.
To the best of our knowledge, this is the first approach to verifying weak
bisimilarity for ccp.  As future work,  we plan to investigate
other calculi where the nature of their transitions systems give rise to similar
situations regarding weak and strong bisimilarity,
in particular timed ccp (tcc) \cite{Saraswat:94:LICS},
non-deterministic timed ccp (ntcc) \cite{Palamidessi:01:CP},
universal temporal ccp (utcc) \cite{Olarte:08:SAC}
and Epistemic ccp (eccp) \cite{eccp-extended-version}.


\bibliographystyle{eptcs}
\bibliography{lab}
\end{document}